\chapter{Results}
\begin{chapquote}{Sir Terry Pratchett, \textit{A Hat Full of Sky}}
``It's still magic even if you know how it's done.''
\end{chapquote}

\section{Laser Current Driver}
For this project several commercial and publicly available laser current drivers were evaluated for their performance. The following devices were all tested for the requirements listed in \ref{lst:dgDrive_specs_environment} and \ref{lst:dgDrive_specs_electrical}.
\begin{itemize}
    \item Moglabs \device{DLC-202}
    \item Newport \device{TLB-6800-LN}
    \item SISYPH \device{SMC11 Puy Mary}
    \item Toptica \device{DCC 110}
    \item Vescent \device{D2-105}
    \item LQO \device{LQprO} \cite{datasheet_LQprO}
    \item A driver based on the work of \citeauthor{laser_driver_digital} \cite{laser_driver_digital}
\end{itemize}

As a disclaimer, Moglabs, Vescent and SISYPH provided demo units, free of charge to the author and without any obligations regarding this work. The opinions and measurements in this work are in no way biased by this service. All of these drivers claim low-noise in various comparative forms, but vary in features. The \device{DLC-202}, the \device{TLB-6800-LN} and the \device{D2-105} additionally include a peltier controller. The \device{DLC-202} and the \device{TLB-6800-LN} have a modulation source. The \device{TLB-6800-LN}, the \device{DCC 110} and the design by \citeauthor{laser_driver_digital} also feature a digital interface.

The drivers were all compared to our requirements. While not all drivers feature a remote accessible interface, their performance was assesed nonetheless to have a broader range of choices. A performance comparison can be found in section \ref{}. Unfortunately, none the drivers was able to properly drive the high compliance voltage required by the blue laser diode \device{PL 450B} of about \qtyrange[range-units = single]{6}{7}{\V} \cite{datasheet_osram_pl450b}. As it was discussed in section \ref{sec:compliance_voltage}, the compliance voltage of all laser drivers based on the design of \citeauthor{libbrecht_hall} \cite{libbrecht_hall} is limited to around \qtyrange[range-units = single]{2}{3}{\V} at full output (compare \ref{eqn:minimum_mosfet_vds} for details). Since the compliance voltage is rougly proportional to the reciprocal of the output current current, limiting the maximum output current to about \qtyrange[range-units = single]{30}{40}{\percent} increases the compliance voltage to the required level. Not only does this limit the choice of drivers, but also requiring a \qty{500}{\mA} driver for a \qty{150}{\mA} laser diode seems excessive and does not help with the noise requirements, because the output noise of those drivers scales roughly with $I_{max}$ as detailed in section \ref{sec:current_source_noise}, since the op-amp noise is the limiting factor. This lead to the decision to design a current source that meets all of our requirements, while surpassing all available alternative and tackling the compliance voltage limit. This design and its individual components are discussed in the following sections. First, the state of the art is presented, then the problems we encountered are outlined and finally our design, that resolves these issues is presented and the caveats and technical challenges are discussed.

\subsection{The State of the Art in Laser Current Drivers}
Prior to this work, all laser drivers for scientific purposes, were more or less strictly following the design proposed by \citeauthor{libbrecht_hall} \cite{libbrecht_hall}. This design was presented in 1993 and back then, blue laser diodes were were not available and only developed in 1996. See \cite{blue_laser_diodes_history} for an interesting historic summary. Finally, the efforts of Isamu Akasaki, Hiroshi Amano and Shuji Nakamura were rewarded with the Nobel Prize in Physics in 2014. The original laser driver design was therefore created for laser diodes requiring a low current and low compliance voltage compared to modern laser diodes. While the design remains useful for many low power near-infrared (NIR) laser diodes, these shortcommings were never addressed or even acknowleged by commercial alternatives. Sadly, the topic of the compliance voltage is usually not even mentioned in the datasheets -- the Moglabs \device{DLC-202} and SISYPH \device{SMC11} are notable exceptions, but it is unclear from the datasheet to which version and/or currents of the devices the numbers relate. The Newport \device{TLB-6800-LN} is a bit different to the rest of the drivers tested, because it comes with Newport laser heads and reads its configuration data from the laser head. Ours came with a \device{Vantage TLB-7100} and the laser head needs to be connected for it to work. Without some reverse engineering, these drivers can only be used with certain Newport laser heads. The \device{TLB-6800-LN} is included in the list of devices anyway to give an idea of its performance in existing systems.

\begin{table}[ht]
    \centering
    \begin{tabularx}{0.95\linewidth}{l>{\raggedright\arraybackslash}Xl>{\raggedright\arraybackslash}X}
        Laser driver& Output current& Compliance voltage & Additional features \\
        \midrule
        Moglabs \device{DLC-202} & \textbf{\qty[text-series-to-math, reset-text-series = false, reset-math-version = false]{100}{\mA}}, \qty{250}{\mA}, \qty{500}{\mA}& \qty{3.1}{\V} & TEC, PID, Piezo\\
        Newport \device{TLB-6800-LN} & & --& TEC, Piezo, Digital\\
        SISYPH \device{SMC11} & \qty{210}{\mA}, \textbf{\qty[text-series-to-math, reset-text-series = false, reset-math-version = false]{470}{\mA}}& \qty{5}{\V}&\\
        Toptica \device{DCC 110} & \textbf{\qty[text-series-to-math, reset-text-series = false, reset-math-version = false]{100}{\mA}}, \qty{500}{\mA}, \qty{3}{\A}, \qty{5}{\A} & --&\\
        Vescent \device{D2-105} & \qty{200}{\mA}, \textbf{\qty[text-series-to-math, reset-text-series = false, reset-math-version = false]{500}{\mA}} &  --& TEC\\
        LQO \device{LQprO} & \textbf{\qty[text-series-to-math, reset-text-series = false, reset-math-version = false]{140}{\mA}}, \qty{400}{\mA} & --&
    \end{tabularx}
    \caption{Overview of laser current drivers tested. Marked in bold is the version tested in this work. A dash denotes that no official information is available.}
    \label{tab:laser_current_drivers_tested}
\end{table}

The drivers shown in table \ref{tab:laser_current_drivers_tested} will now be discussed in a little more detail to familiarize with them.

Starting with the Moglabs \device{DLC-202}, which is fully integrated unit, that leaves little functions to be desired, it includes the current driver, a pizo driver, a temperature controller for a thermoelectric cooler (TEC) and finally a lock-in amplifier with a PID controller. It is the most integrated solution, that was tested and brings all features required to set up an ECDL locked to an atomic transition. Some of its features are accessible via pin headers. The current source can be remotely enabled and the PID controller can be manipulated to allow relocking the laser or deliberately taking it out lock remotely. There is no way directly adjust the output current over the whole range. It also features a broad range of protection mechanisms for the laser diode, for example disconneting the driver in case of a short or open condition. All relavant quantities can be adjusted and read back from the front panel display. The manual is fairly comprehensive and gives a lot of examples to set up a laser system.

The Vescent \device{D2-105} also features more than just the laser driver and includes a TEC controller as well. Unfortunately, adjusting this temperature controller prooved fairly cumbersome, because the driver has to be opened to adjust trimpots inside the driver. When the unit is placed in a rack, this might even be impossible to realize. The laser current is normally adjusted via a 10-turn potentiometer and can also be steered via an external input, but the stability of the current then depends on the external control voltage. It does not feature any protection features like open or short detection, so disconnecting and reconnecting the cable to the laser will most likely damage the laser diode. The display can be switched to show all relavant quantities. Moreover, the manual is not as comprehensive as the manual of the \device{DLC-202}, but covers all relavant settings of the driver.

The SISYPH \device{SMC11} does not include any additional features, but covers the most important protection features like an open detection and shuts down the driver accordingly. It is fully rackmountable, unlike the drivers discussed so far. The setpoint of the driver is adjusted via a recessed trimpot using a screwdriver, which has prooven troublesome to adjust in the lab when not directly in front of the unit. The current can be externally adjusted using an input connector, but again this limits the stability to that of the external source. The driver does not have any display and all setpoints must either be adjusted blindly or a volmeter must be attached to the monitoring connector limiting the usefulness in a lab environment. The user manual covers only the basic settings and gives no details regaring the layout of the pin headers or external connectors making it hard to understand without having the device at hand.

The Toptica \device{DCC 110} is also only a current source and is rack mounted. It comes with a separate display module, which connect via the backplane. The setpoint is adjusted via 10-turn potentiometer and can additionally be adjusted via the backplane with an external signal, again limiting the stability of the driver to the external source. The manual is fairly comprehensice and covers all essentials.

The Newport \device{TLB-6800-LN} is the only driver, that has a digital interface and supports Standard Commands for Programmable Instruments (SCPI) commands. It incorporates a TEC controller and a piezo controller. Unfortunately, it only works with a limited number of lasers, because it reads some parameters, like the maximum output current from the laser head at startup. The user manual covers all device functions, but gives little detail about the hardware, making this a closed system.

The results of the performance tests conducted prior to building our own solution will be presented on the following pages, including problems typically encountered and their solution. These tests include the stability, current noise and output impedance of the drivers. Not all drivers were put through the full test, if it was already clear that they could not perform in our setup for a laser system for the spectroscopy of highly charged ions driving a blue laser diode.

\clearpage
\subsection{Laser Driver: Design Concept}
In order to interpret the results in sections \ref{sec:results_current_noise}, \ref{sec:results_stability}, it it helpful to discuss in more detail the design concept of the current state of the art, which is based on the driver design presented by \citeauthor{libbrecht_hall} \cite{libbrecht_hall}. The design can be split into the four building blocks shown in figure \ref{fig:laser_driver_libbrecht_hall_concept}. A supply voltage input filter, a reference voltage to create the setpoint, a unidrectional current source and some form of bidirectional current source used for modulating the laser current at high frequency.

\begin{figure}[ht]
    \centering
    %\resizebox {0.8\textwidth} {!} {
        \import{figures/}{laser_driver_libbrecht_hall_concept.tex}
    %} % resizebox
    \caption{Building blocks of a laser driver based on \cite{libbrecht_hall}.}
    \label{fig:laser_driver_libbrecht_hall_concept}
\end{figure}

The original design is the most straightforward approach and it is possible to reproduce it even on prototype printed circuit boards (PCBs). \citeauthor{laser_driver_digital} \cite{laser_driver_digital} replaced the potentiometer with a DAC, but left the other parts untouched. \citeauthor{laser_driver_qcl_taubman} \cite{laser_driver_qcl_taubman,laser_driver_qcl_taubman_multiplexer} published some extensive modifications, which not only replaced the reference circuit with a DAC and an \device{LTZ1000} reference, but also added extensive filtering of the supply. The next sections will discuss these different elements separately and give some in insight into the different versions of the elements found in literature. The sections also details problems discoverd and the solution proposed in the design presented here.

\clearpage
\subsection{Supply Filtering}
The supply section of the design by \citeauthor{libbrecht_hall} was shown simplified in figure \ref{fig:laser_driver_libbrecht_hall_concept}. The original filteronly consists of a CLC filter or sometimes called \pi-filter shown in figure \ref{fig:laser_driver_libbrecht_hall_filter}. Do note, that due to the small input capacitance, the filter is basically just an LC-filter.

\begin{figure}[ht]
    \centering
    %\resizebox {0.8\textwidth} {!} {
        \import{figures/}{laser_driver_libbrecht_hall_supply_filter.tex}
    %} % resizebox
    \caption{Power supply filter of a laser driver based on \cite{libbrecht_hall}. The op-amps are supply by the filtered voltage and the current source is supplied by the \device{LM317}.}
    \label{fig:laser_driver_libbrecht_hall_filter}
\end{figure}

The LC-filter is best suited for a low impedance source like a power supply, because it has a high input impedance. From the transfer function
\begin{equation}
    H(s) = \frac{Z_{out}}{Z_{in}} = \frac{\frac{1}{sC}}{sL + \frac{1}{sC}} = \frac{1}{s^2LC +1} = \frac{\frac{1}{LC}}{s^2 + \frac{1}{LC}} = \frac{\frac{1}{LC}}{\left(s+i\frac{1}{\sqrt{LC}}\right)\left(s-i\frac{1}{\sqrt{LC}}\right)}\,, \label{eqn:transfer_function_lc_filter}
\end{equation}
one can deduce, that the passband gain at DC is \num{1} (obviously) and additionally, there are two complex poles in the imaginary plane at $s = \frac{\pm i}{\sqrt{LC}}$, putting the cutoff frequency of the 2\textsuperscript{nd} order filter at $f_c = \qty{2.3}{\kHz}$. Do note, due to the imaginary poles, there is some gain peaking at $f_c$. Normally, this is damped by the parasitic resistance of the inductor and capacitor. A more detailed analysis follows below, because the cutoff frequency is all that is of interest right now.

In this design, the op-amps are directly driven off the filtered supply rail. Using this information it is possible to estimate the effectiveness of the filter. Using the \qty{30}{\nA_{rms}} in a \qty{100}{\kHz} bandwidth current noise requirement from table \ref{lst:dgDrive_specs_electrical}, the voltage noise at the sense resistor (\qty{14}{\ohm}, (see section \ref{sec:component_selection}) must be no more than \qty{420}{\nV_{rms}}. Now, taking for example a low-noise switch-mode power supply like the Rohde \& Schwarz \device{HMP4040} used at CERN \cite{hmp4040_noise}, which does have fairly pronouced noise at the switching frequency of around \qty{170}{\kHz} and harmonics. The author measured these glitches to be about \qty{3}{\mV_{pp}}. The noise will have to go through the filter and the supply rejection ratio (PSRR) of the op-amp. The PSRR of the \device{LT1028} at \qty{170}{\kHz} is about \num{e-2} under ideal conditions \cite{datasheet_LT1028} and the filter adds another \num{e-2} when accounting for a \qty{0.5}{\ohm} series resistance of the output capacitor. The total filtering adds up to about \num{e-4}, which still leaves \qty{21}{\nA_{pp}} of ripple on the drive current in a very small bandwidth.

To have a better rejection of such switch-mode noise, the filter must be improved. The paper presented by \citeauthor{laser_driver_qcl_taubman} \cite{laser_driver_qcl_taubman} shows a brute-force approach. They applied extremely high values for the capacitor $C_{LC}$ of the LC filter of \qty{10}{\milli\farad} and then put a second filter based on a so-called capacitance multiplier behind it. This implementation is shown in a simplified form in figure \ref{fig:laser_driver_taubman_filter} and briefly discussed now. For a more detailed schematic and part names see \cite{laser_driver_qcl_taubman}.

\begin{figure}[ht]
    \centering
    %\resizebox {0.8\textwidth} {!} {
        \import{figures/}{laser_driver_taubman_filter.tex}
    %} % resizebox
    \caption{Power supply filter using a capacitance multiplier for a cutoff frequency of \qty{0.5}{\Hz}. This is a simplified schematic based on \cite{laser_driver_qcl_taubman}. Only the positive rail is shown.}
    \label{fig:laser_driver_taubman_filter}
\end{figure}

\citeauthor{laser_driver_qcl_taubman} built this filter for a driver with a driving capacity of \qty{2}{\A}, which limits the size of the inductor that can be used, in order to make up for that, he is forced to use giant capacitors. The second stage of the filter comprises a capacitance multiplier, which is formed by wrapping a feedback loop around the 2\textsuperscript{nd} order filter created by $R_1 C_1$ and $R_2 C_2$. This feedback loop removes the main current from the filter resistors, to allow larger values for $R_1$ and $R_2$, while maintaining a fairly low output impedance of the filter. The properties of this construction will be discussed now.

As a first note, the circuit presented by \citeauthor{laser_driver_qcl_taubman} misses a detail, which should included, when handling such capacitances. The circuit must include a reverse polarity protection, or rather a reverse current protection. If the input is shorted by accident, the \qty{10}{\milli\farad} of capacitance would immediately discharge via the parasitic body diode of the slow start-up transistor, likely vaporizing everything in its path. This could be implemented by adding another transistor to act as a reverse current protection.

% TODO: Add an appendix about the capacitance multiplier. See capacitance multipler/The Capacitance Multiplier Theory.pdf
The following explanation relies on some basic knowlegde about transistors. A good introduction to transistors is the Transistor Manual \cite{transistor_bible}, some even call the bible of transistors. Armed with some basic knowledge, the capacitance multiplier can be discussed and it must already be said, that the term capacitance multiplier is a bit misleading. It neither multiplies the capacitance, nor does is behave like a real 2\textsuperscript{nd} order filter. The only thing that is multiplied by the gain of the transitor, is the output capacitance seen by load, making the capacitor look more ideal. Unfortunately, it  highly depends on the properties of the transistor(s) and the gain of transistors drops with increasing output current (although it rises with temperature). Make sure to consult the datasheet, which typically gives a plot of the ac current gain $h_{fe}$ vs the collector curren $I_C$ to see at which load current, $h_{fe}$ starts dropping. Another issue is the bandwidth of the circuit, because $h_{fe}$ of any transistor rolls off with frequency. The high frequency resonse of this filter then becomes constant. This limits the suppression at around \qty{1}{\kHz} (depending on the output current of course) to around \numrange{500}{1000}. It is usefuly at low frequencies though, as it is shown in the publication of \citeauthor{laser_driver_qcl_taubman}. Another point is the maximum ripple voltage, that can be filtered. The muliplied capacitance does, of course, not store the same amount of energy as a real capacitor. This means the maximum peak-to-peak input ripple is limited to about one diode drop of \qty{0.68}{\V} from the Collector-Emitter diode. If more ripplle rejection is required an additional resistor from the base of $Q_2$ to ground like shown in figure \ref{fig:laser_driver_taubman_filter} can be applied. This reduces the output voltage further though. In this design, the current through $R_m$ is sufficient.

As a final remark regarding the capacitance multiplier is the output impedance. The transistor has an output impedance like a diode, so it increases with decreasing current. This means it will also drop about \qty{1.2}{\V} at \qty{500}{\mA} and about \qty{0.3}{\V} at \qty{1}{\mA}. This behaviour must be taken care of by the voltage regulator following the capacitance multiplier. The \qty{2}{\V} drop is also not a problem in this use case and even comes in handy. If just the supply rail of the laser diode current is fed through the capacitance multiplier, as it is the most sensitive, and the supply rail for the op-amps is not, then those extra \qty{2}{\V} will do not be a problem. In section \ref{sec:component_selection} it was already mentioned, that for example the \device{AD797} op-amp needs a supply that is \qty{3}{\V} above the diode supply. This means, that less voltage needs to be dropped by the linear regulator that follows the filter. To sum it up, the capacitance multiplier behaves like an ordinary $RC$ filter, but with a lower output impedance and only works at low ripple voltages, is limited in the high frequency domain.

% Note: 'Minimizing Input Filter Requirements In Military Power Supply Designs' has a more elaborate design
The power supply filters applied in this design use a passive LC filter for the negative and positive rail, then a capacitance multiplier on the diode supply. The negative rail is simply mirrored from the positive rail and pnp instead of npn transistors and vice versa are used. The combined filter is shown in figure \ref{fig:laser_driver_dgdrive_filter}. The diode supply and the analog rail, which is taken before the capacitance multiplier, are fed to low noise post-regulators, the \device{LT3045} and its negative counterpart, the \device{LT3094}. Both regulators have excellent power supply ripple rejection (PSRR) out to at least \qty{1}{\MHz} of more than \num{e3}. At low frequency the PSRR is even higher and more than \num{e5} can be expected. This allows a combined PSRR of better than \num{e6} from low to high frequencies, even beyond \qty{1}{\MHz}.

\begin{figure}[ht]
    \centering
    %\resizebox {0.8\textwidth} {!} {
        \import{figures/}{laser_driver_dgdrive_supply_filter.tex}
    %} % resizebox
    \caption{Power supply filter of the digital current driver.}
    \label{fig:laser_driver_dgdrive_filter}
\end{figure}

Regarding the filter circuit shown in figure \ref{fig:laser_driver_dgdrive_filter} a few explaining words on the choice of components are in order before proceeding to the measurement of the PSRR. Going back to equation \ref{eqn:transfer_function_lc_filter}, we saw, that the undamped second filter has excessive ringing at the cutoff frequency, because the filter poles are imaginary. To address this, there are several solutions. The most simple one is adding a damping element, either in parallel to the capacitor or in parallel the inductor. In this case a damping element in parallel to the capacitor was chosen, because using a a damping element parallel to the inductor will degrade the filter performance by making the blocking inductor lossy. Using the arangement shown in figure \ref{fig:laser_driver_dgdrive_filter}, the transfer function can be calculated.
\begin{align}
    H(s) &= \frac{Z_{out}}{Z_{in}} \nonumber\\
    Z_{out} &= \left(R_d + Z_{C_d}\right) || Z_{C_1} = \left(R_d + \frac{1}{s C_d}\right) || \frac{1}{s C_1} = \left(\left(R_d + \frac{1}{sC_d}\right)^{-1} + sC_1\right)^{-1} \nonumber\\
    &= \frac{s C_d R_d +1}{s^2 C_1 C_d R_d + s \left(C_1 C_d\right)}\\
    Z_{in} &= sL_1 + \left(R_d + Z_{C_d}\right) \nonumber\\
    H(s) &= \frac{\frac{s C_d R_d +1}{s^2 C_1 C_d R_d + s \left(C_1 C_d\right)}}{s L_1  + \frac{s C_d R_d +1}{s^2 C_1 C_d R_d + s \left(C_1 + C_d\right)}} = \frac{s C_d R_d +1}{s^3 L_1 C_1 C_d R_d + s^2 L_1 \left(C_1 + C_d\right) + s C_d R_d +1 }
\end{align}

% TODO: Add derivation, because the Middlebrook paper is hard to come by. This is already prepared in 'Input Filter Derivation.pdf' and Input Filter Theory.pdf', but needs to be checked again.
This is the transfer function of a 3\textsuperscript{rd} order filter. This type of filter was discussed by \citeauthor{input_filter_middlebrook} \cite{input_filter_middlebrook} (reprinted in \cite{input_filter_middlebrook_reprint1} and \cite{input_filter_middlebrook_reprint2}). \citeauthor{input_filter_middlebrook} derived, that there is an optimal value for the series resitance $R_d$ given a capacitance $C_d$ and the filter components $L_1$ and $C_1$. This optimal value has minimal gain peaking, hence a minimal quality factor $Q$ at the resonance freuency. The existance of such an optimal value can be easily understood from the fact, that if $R_d = \infty$ the resonance frequency is $\omega_0 = \frac{1}{\sqrt{L_1 C_1}}$ and in case $R_d = 0$ it is $\omega_1 = \frac{1}{\sqrt{L_1 \left(C_1 + C_d\right)}}$. In between $\omega_0$ and $\omega_1$, there is a lossy zone, where $R_d$ due to its lossy nature reduces $Q$, but at both ends $Q = \infty$, so there must be a minimum in between. By calculating, the minimum value of the transfer function at the point of resonance, \citeauthor{input_filter_middlebrook} found the following results:
\begin{align}
    R_0 &\coloneqq \sqrt{\frac{L_1}{C_1}}\\
    n &\coloneqq \frac{C_d}{C_1} \Rightarrow C_d = n C_1 \label{eqn:lc_filter_cd}\\
    Q_{optimal} &= \sqrt{\frac{(4+3 n) (2+n)}{2 n^2 (4+n)}}\\
    R_d &= R_0 \cdot Q_{optimal} \label{eqn:lc_filter_rd}
\end{align}

% TODO: Derive the 2x / 10 K rule. This is prepared in 'Capacitor life doubles theory.pdf'.
From these equations, it can be seen, that the damping capacitor $C_d$ needs to be fairly large, depending on $n$. A critically damped system with $Q = \num{0.5}$ would be preferred, but this would require $n \approx 6$ making the $C_d$ prohibitively large. For this filter $n=4$ was chosen, so making the filter slightly underdamped, so a slight gain peaking at the resonace can be expected. The following componets were chosen. First, a large, low resistance inductor $L_1$ capable of carrying at least \qty{1}{\A} was chosen. In this case a Coilcraft \device{MSS1210-125KEB}. High reliability capacitors were chosen to ensure a long lifetime of the device. Choosing capacitors rated with a lifetime of \qty{5000}{\hour} at \qty{105}{\celsius} gives a expected service life of more than \qty{10}{\year}, when assuming an Arrhenius law with a doubling of the lifetime every \qty{10}{\kelvin}. Apart from the reliability of the capacitors, there are no special requirements for them as there is little ripple current to be expected. The input power supply is supposed to be a filtered low noise supply and not the unfiltered output of a DC/DC regulator. So it is possible to maximize $L_1$ and choose a physically smaller $C_1$ since board space is limited. This results in the following design values, calculated from equations \ref{eqn:lc_filter_cd} and \ref{eqn:lc_filter_cd}, given the components values discussed above.
\begin{align}
    C_1 &= \qty{100}{\uF} \nonumber\\
    n &= 4 \nonumber\\
    Q_{optimal} &\approx \num{0.61} \nonumber\\
    R_d &= \num{0.61} \approx R_0 \approx \qty{2}{\ohm} \nonumber\\
    C_d &= \qty{400}{\uF} \approx \qty{390}{\uF} \nonumber\\
    f_c &\approx \qty{300}{\Hz} \nonumber
\end{align}

Do note, that $R_d$ does include the equivalent series resistance (ESR) of $C_d$, so the ESR of the capacitor must be subtracted from the final value of the damping resistor placed on the board. This may even absolve one from the need for a discrete resistor if the ESR of the capacitor is high enough. The transistors chose were a combination of a Toshiba \device{TTA004B}/\device{TTC004B} and Onsemi \device{BC817-40}/\device{BC807-40} for the positive/negative rail. The \device{TTA004B}/\device{TTC004B} are good up to about \qty{500}{\mA}. At this point the gains start dropping. A higher power transistor like the Onsemi \device{D45H8}/\device{D44H8} used by \citeauthor{laser_driver_qcl_taubman} is recommended for $Q_1$ and $Q_3$.

Finally, one last part of the capacitance multiplier should be explained. Highlighted in green in figure \ref{fig:laser_driver_dgdrive_filter} is a fast startup circuit. At startup, the capacitor $C_m$ is discharged and \qty{15}{\V} will be applied, it will then begin to charge with a current of \qty{1.5}{\mA} through the \qty{10}{\kilo\ohm} resistor. Because the $Q_2$ is an emitter follower, hence the emitter follows the voltage at the base (minus a diode drop for the base-emitter diode). As a sidenote, when using this kind of circuit, since $Q_2$ is an emitter follower, all output capacitors, that follow the capacitance multiplier will charge at the same rate as $C_m$, voltage-wise, this means, that for every \qty{10}{\uF} of output capacitance, a current of \qty{1.5}{\mA} will flow. While this not significant at this moment it become so, when looking at the fast startup circuit. Applying the input voltage of \qty{15}{\V} at startup over the LED, a \qty{625}{\nm} Würth Elektronik \device{150080RS75000}, it will start conducting, resulting in a \qtyrange[range-units = single]{1.8}{2}{\V} drop. The current flowing into $C_m$ is therefore dependent on the diode series resistor, which was chosen to be \qty{510}{\ohm}, a value not particularily important in this case. So at startup about \qty{25}{\mA} will flow into $C_m$, which means \qty{2.5}{\mA \per \uF} will flow through $Q_1$ and $Q_2$. Assuming roughly \qty{100}{\uF} of distributed bypassing capacitance around the board, this is around \qty{500}{\mA}. All these values are still well below the damage threshold of the transistors (\qty{2.5}{\A} and \qty{0.5}{\A}) and the LED (\qty{30}{\mA}), but these values must be kept in mind, when adding larger output capacitors. The fast startup circuit ensures an output voltage of \qty{13}{\V} within \qty{100}{\ms} instead of around \qty{0.5}{\second}, reducing the time to boot and leaving more time for self-checks without impacting the user experience.

After choosing the values above, the filter was simulated using LTSpice to assert the validity of the parameters chosen. The simulation was conducted with a load current of \qty{500}{\mA} running through the capacitance multiplier to simulate the worst case. As discussed above, the gain of the transistor $h_{fe}$ drops at higher currents as the transistor saturates. This particularily affects the high frequency behaviour above \qty{10}{\kHz}. The source file can be found in \external{source/spice/input\_filter\_dgdrive.asc}. The simulation additionally includes the series resistance and parasitic parallel capacitance of $L_1$, the latter will induce some ringing at the self resonance frequency of the inductor at \qty{1}{\MHz} and limit the useful attenuation beyond that to around \num{e3} due to the capacitive coupling of the conductor windings. At the \qty{170}{\kHz} discussed before, the damping is about the same figure of merit, \num{e3}.

The suppression is an order of magnitude better than the filter used by \citeauthor{libbrecht_hall} and it does not even include the high performance regulators that follow. The transfer function for both the damped LC filter and the LC filter with the capacitance multiplier in series is plotted in figure \ref{fig:laser_driver_input_filter}. The self resonance peak at \qty{1}{\MHz} can be clearly seen and is not damped, but from the output impedance shows shows that, there is enough capacitance present to compensate for this. The output impedance above \qty{1}{\MHz}, it is dominated by the local bypass capacitors and not accurately represented by the simulation. It can be expected to by lower than the simulated results, which do not include those capacitors.
\begin{figure}[ht]
    \centering
    %% Creator: Matplotlib, PGF backend
%%
%% To include the figure in your LaTeX document, write
%%   \input{<filename>.pgf}
%%
%% Make sure the required packages are loaded in your preamble
%%   \usepackage{pgf}
%%
%% Also ensure that all the required font packages are loaded; for instance,
%% the lmodern package is sometimes necessary when using math font.
%%   \usepackage{lmodern}
%%
%% Figures using additional raster images can only be included by \input if
%% they are in the same directory as the main LaTeX file. For loading figures
%% from other directories you can use the `import` package
%%   \usepackage{import}
%%
%% and then include the figures with
%%   \import{<path to file>}{<filename>.pgf}
%%
%% Matplotlib used the following preamble
%%   \def\mathdefault#1{#1}
%%   \everymath=\expandafter{\the\everymath\displaystyle}
%%   \usepackage{siunitx}
%%   \sisetup{per-mode = symbol}%
%%   \ifdefined\pdftexversion\else  % non-pdftex case.
%%     \usepackage{fontspec}
%%   \fi
%%   \makeatletter\@ifpackageloaded{underscore}{}{\usepackage[strings]{underscore}}\makeatother
%%
\begingroup%
\makeatletter%
\begin{pgfpicture}%
\pgfpathrectangle{\pgfpointorigin}{\pgfqpoint{5.431103in}{3.356606in}}%
\pgfusepath{use as bounding box, clip}%
\begin{pgfscope}%
\pgfsetbuttcap%
\pgfsetmiterjoin%
\definecolor{currentfill}{rgb}{1.000000,1.000000,1.000000}%
\pgfsetfillcolor{currentfill}%
\pgfsetlinewidth{0.000000pt}%
\definecolor{currentstroke}{rgb}{1.000000,1.000000,1.000000}%
\pgfsetstrokecolor{currentstroke}%
\pgfsetdash{}{0pt}%
\pgfpathmoveto{\pgfqpoint{0.000000in}{0.000000in}}%
\pgfpathlineto{\pgfqpoint{5.431103in}{0.000000in}}%
\pgfpathlineto{\pgfqpoint{5.431103in}{3.356606in}}%
\pgfpathlineto{\pgfqpoint{0.000000in}{3.356606in}}%
\pgfpathlineto{\pgfqpoint{0.000000in}{0.000000in}}%
\pgfpathclose%
\pgfusepath{fill}%
\end{pgfscope}%
\begin{pgfscope}%
\pgfsetbuttcap%
\pgfsetmiterjoin%
\definecolor{currentfill}{rgb}{1.000000,1.000000,1.000000}%
\pgfsetfillcolor{currentfill}%
\pgfsetlinewidth{0.000000pt}%
\definecolor{currentstroke}{rgb}{0.000000,0.000000,0.000000}%
\pgfsetstrokecolor{currentstroke}%
\pgfsetstrokeopacity{0.000000}%
\pgfsetdash{}{0pt}%
\pgfpathmoveto{\pgfqpoint{0.689445in}{0.524170in}}%
\pgfpathlineto{\pgfqpoint{4.796234in}{0.524170in}}%
\pgfpathlineto{\pgfqpoint{4.796234in}{3.206606in}}%
\pgfpathlineto{\pgfqpoint{0.689445in}{3.206606in}}%
\pgfpathlineto{\pgfqpoint{0.689445in}{0.524170in}}%
\pgfpathclose%
\pgfusepath{fill}%
\end{pgfscope}%
\begin{pgfscope}%
\pgfpathrectangle{\pgfqpoint{0.689445in}{0.524170in}}{\pgfqpoint{4.106789in}{2.682436in}}%
\pgfusepath{clip}%
\pgfsetrectcap%
\pgfsetroundjoin%
\pgfsetlinewidth{0.803000pt}%
\definecolor{currentstroke}{rgb}{0.450000,0.450000,0.450000}%
\pgfsetstrokecolor{currentstroke}%
\pgfsetdash{}{0pt}%
\pgfpathmoveto{\pgfqpoint{0.876117in}{0.524170in}}%
\pgfpathlineto{\pgfqpoint{0.876117in}{3.206606in}}%
\pgfusepath{stroke}%
\end{pgfscope}%
\begin{pgfscope}%
\pgfsetbuttcap%
\pgfsetroundjoin%
\definecolor{currentfill}{rgb}{0.000000,0.000000,0.000000}%
\pgfsetfillcolor{currentfill}%
\pgfsetlinewidth{0.803000pt}%
\definecolor{currentstroke}{rgb}{0.000000,0.000000,0.000000}%
\pgfsetstrokecolor{currentstroke}%
\pgfsetdash{}{0pt}%
\pgfsys@defobject{currentmarker}{\pgfqpoint{0.000000in}{-0.048611in}}{\pgfqpoint{0.000000in}{0.000000in}}{%
\pgfpathmoveto{\pgfqpoint{0.000000in}{0.000000in}}%
\pgfpathlineto{\pgfqpoint{0.000000in}{-0.048611in}}%
\pgfusepath{stroke,fill}%
}%
\begin{pgfscope}%
\pgfsys@transformshift{0.876117in}{0.524170in}%
\pgfsys@useobject{currentmarker}{}%
\end{pgfscope}%
\end{pgfscope}%
\begin{pgfscope}%
\definecolor{textcolor}{rgb}{0.000000,0.000000,0.000000}%
\pgfsetstrokecolor{textcolor}%
\pgfsetfillcolor{textcolor}%
\pgftext[x=0.876117in,y=0.426948in,,top]{\color{textcolor}{\rmfamily\fontsize{8.000000}{9.600000}\selectfont\catcode`\^=\active\def^{\ifmmode\sp\else\^{}\fi}\catcode`\%=\active\def%{\%}$\mathdefault{10^{-1}}$}}%
\end{pgfscope}%
\begin{pgfscope}%
\pgfpathrectangle{\pgfqpoint{0.689445in}{0.524170in}}{\pgfqpoint{4.106789in}{2.682436in}}%
\pgfusepath{clip}%
\pgfsetrectcap%
\pgfsetroundjoin%
\pgfsetlinewidth{0.803000pt}%
\definecolor{currentstroke}{rgb}{0.450000,0.450000,0.450000}%
\pgfsetstrokecolor{currentstroke}%
\pgfsetdash{}{0pt}%
\pgfpathmoveto{\pgfqpoint{1.342798in}{0.524170in}}%
\pgfpathlineto{\pgfqpoint{1.342798in}{3.206606in}}%
\pgfusepath{stroke}%
\end{pgfscope}%
\begin{pgfscope}%
\pgfsetbuttcap%
\pgfsetroundjoin%
\definecolor{currentfill}{rgb}{0.000000,0.000000,0.000000}%
\pgfsetfillcolor{currentfill}%
\pgfsetlinewidth{0.803000pt}%
\definecolor{currentstroke}{rgb}{0.000000,0.000000,0.000000}%
\pgfsetstrokecolor{currentstroke}%
\pgfsetdash{}{0pt}%
\pgfsys@defobject{currentmarker}{\pgfqpoint{0.000000in}{-0.048611in}}{\pgfqpoint{0.000000in}{0.000000in}}{%
\pgfpathmoveto{\pgfqpoint{0.000000in}{0.000000in}}%
\pgfpathlineto{\pgfqpoint{0.000000in}{-0.048611in}}%
\pgfusepath{stroke,fill}%
}%
\begin{pgfscope}%
\pgfsys@transformshift{1.342798in}{0.524170in}%
\pgfsys@useobject{currentmarker}{}%
\end{pgfscope}%
\end{pgfscope}%
\begin{pgfscope}%
\definecolor{textcolor}{rgb}{0.000000,0.000000,0.000000}%
\pgfsetstrokecolor{textcolor}%
\pgfsetfillcolor{textcolor}%
\pgftext[x=1.342798in,y=0.426948in,,top]{\color{textcolor}{\rmfamily\fontsize{8.000000}{9.600000}\selectfont\catcode`\^=\active\def^{\ifmmode\sp\else\^{}\fi}\catcode`\%=\active\def%{\%}$\mathdefault{10^{0}}$}}%
\end{pgfscope}%
\begin{pgfscope}%
\pgfpathrectangle{\pgfqpoint{0.689445in}{0.524170in}}{\pgfqpoint{4.106789in}{2.682436in}}%
\pgfusepath{clip}%
\pgfsetrectcap%
\pgfsetroundjoin%
\pgfsetlinewidth{0.803000pt}%
\definecolor{currentstroke}{rgb}{0.450000,0.450000,0.450000}%
\pgfsetstrokecolor{currentstroke}%
\pgfsetdash{}{0pt}%
\pgfpathmoveto{\pgfqpoint{1.809478in}{0.524170in}}%
\pgfpathlineto{\pgfqpoint{1.809478in}{3.206606in}}%
\pgfusepath{stroke}%
\end{pgfscope}%
\begin{pgfscope}%
\pgfsetbuttcap%
\pgfsetroundjoin%
\definecolor{currentfill}{rgb}{0.000000,0.000000,0.000000}%
\pgfsetfillcolor{currentfill}%
\pgfsetlinewidth{0.803000pt}%
\definecolor{currentstroke}{rgb}{0.000000,0.000000,0.000000}%
\pgfsetstrokecolor{currentstroke}%
\pgfsetdash{}{0pt}%
\pgfsys@defobject{currentmarker}{\pgfqpoint{0.000000in}{-0.048611in}}{\pgfqpoint{0.000000in}{0.000000in}}{%
\pgfpathmoveto{\pgfqpoint{0.000000in}{0.000000in}}%
\pgfpathlineto{\pgfqpoint{0.000000in}{-0.048611in}}%
\pgfusepath{stroke,fill}%
}%
\begin{pgfscope}%
\pgfsys@transformshift{1.809478in}{0.524170in}%
\pgfsys@useobject{currentmarker}{}%
\end{pgfscope}%
\end{pgfscope}%
\begin{pgfscope}%
\definecolor{textcolor}{rgb}{0.000000,0.000000,0.000000}%
\pgfsetstrokecolor{textcolor}%
\pgfsetfillcolor{textcolor}%
\pgftext[x=1.809478in,y=0.426948in,,top]{\color{textcolor}{\rmfamily\fontsize{8.000000}{9.600000}\selectfont\catcode`\^=\active\def^{\ifmmode\sp\else\^{}\fi}\catcode`\%=\active\def%{\%}$\mathdefault{10^{1}}$}}%
\end{pgfscope}%
\begin{pgfscope}%
\pgfpathrectangle{\pgfqpoint{0.689445in}{0.524170in}}{\pgfqpoint{4.106789in}{2.682436in}}%
\pgfusepath{clip}%
\pgfsetrectcap%
\pgfsetroundjoin%
\pgfsetlinewidth{0.803000pt}%
\definecolor{currentstroke}{rgb}{0.450000,0.450000,0.450000}%
\pgfsetstrokecolor{currentstroke}%
\pgfsetdash{}{0pt}%
\pgfpathmoveto{\pgfqpoint{2.276159in}{0.524170in}}%
\pgfpathlineto{\pgfqpoint{2.276159in}{3.206606in}}%
\pgfusepath{stroke}%
\end{pgfscope}%
\begin{pgfscope}%
\pgfsetbuttcap%
\pgfsetroundjoin%
\definecolor{currentfill}{rgb}{0.000000,0.000000,0.000000}%
\pgfsetfillcolor{currentfill}%
\pgfsetlinewidth{0.803000pt}%
\definecolor{currentstroke}{rgb}{0.000000,0.000000,0.000000}%
\pgfsetstrokecolor{currentstroke}%
\pgfsetdash{}{0pt}%
\pgfsys@defobject{currentmarker}{\pgfqpoint{0.000000in}{-0.048611in}}{\pgfqpoint{0.000000in}{0.000000in}}{%
\pgfpathmoveto{\pgfqpoint{0.000000in}{0.000000in}}%
\pgfpathlineto{\pgfqpoint{0.000000in}{-0.048611in}}%
\pgfusepath{stroke,fill}%
}%
\begin{pgfscope}%
\pgfsys@transformshift{2.276159in}{0.524170in}%
\pgfsys@useobject{currentmarker}{}%
\end{pgfscope}%
\end{pgfscope}%
\begin{pgfscope}%
\definecolor{textcolor}{rgb}{0.000000,0.000000,0.000000}%
\pgfsetstrokecolor{textcolor}%
\pgfsetfillcolor{textcolor}%
\pgftext[x=2.276159in,y=0.426948in,,top]{\color{textcolor}{\rmfamily\fontsize{8.000000}{9.600000}\selectfont\catcode`\^=\active\def^{\ifmmode\sp\else\^{}\fi}\catcode`\%=\active\def%{\%}$\mathdefault{10^{2}}$}}%
\end{pgfscope}%
\begin{pgfscope}%
\pgfpathrectangle{\pgfqpoint{0.689445in}{0.524170in}}{\pgfqpoint{4.106789in}{2.682436in}}%
\pgfusepath{clip}%
\pgfsetrectcap%
\pgfsetroundjoin%
\pgfsetlinewidth{0.803000pt}%
\definecolor{currentstroke}{rgb}{0.450000,0.450000,0.450000}%
\pgfsetstrokecolor{currentstroke}%
\pgfsetdash{}{0pt}%
\pgfpathmoveto{\pgfqpoint{2.742840in}{0.524170in}}%
\pgfpathlineto{\pgfqpoint{2.742840in}{3.206606in}}%
\pgfusepath{stroke}%
\end{pgfscope}%
\begin{pgfscope}%
\pgfsetbuttcap%
\pgfsetroundjoin%
\definecolor{currentfill}{rgb}{0.000000,0.000000,0.000000}%
\pgfsetfillcolor{currentfill}%
\pgfsetlinewidth{0.803000pt}%
\definecolor{currentstroke}{rgb}{0.000000,0.000000,0.000000}%
\pgfsetstrokecolor{currentstroke}%
\pgfsetdash{}{0pt}%
\pgfsys@defobject{currentmarker}{\pgfqpoint{0.000000in}{-0.048611in}}{\pgfqpoint{0.000000in}{0.000000in}}{%
\pgfpathmoveto{\pgfqpoint{0.000000in}{0.000000in}}%
\pgfpathlineto{\pgfqpoint{0.000000in}{-0.048611in}}%
\pgfusepath{stroke,fill}%
}%
\begin{pgfscope}%
\pgfsys@transformshift{2.742840in}{0.524170in}%
\pgfsys@useobject{currentmarker}{}%
\end{pgfscope}%
\end{pgfscope}%
\begin{pgfscope}%
\definecolor{textcolor}{rgb}{0.000000,0.000000,0.000000}%
\pgfsetstrokecolor{textcolor}%
\pgfsetfillcolor{textcolor}%
\pgftext[x=2.742840in,y=0.426948in,,top]{\color{textcolor}{\rmfamily\fontsize{8.000000}{9.600000}\selectfont\catcode`\^=\active\def^{\ifmmode\sp\else\^{}\fi}\catcode`\%=\active\def%{\%}$\mathdefault{10^{3}}$}}%
\end{pgfscope}%
\begin{pgfscope}%
\pgfpathrectangle{\pgfqpoint{0.689445in}{0.524170in}}{\pgfqpoint{4.106789in}{2.682436in}}%
\pgfusepath{clip}%
\pgfsetrectcap%
\pgfsetroundjoin%
\pgfsetlinewidth{0.803000pt}%
\definecolor{currentstroke}{rgb}{0.450000,0.450000,0.450000}%
\pgfsetstrokecolor{currentstroke}%
\pgfsetdash{}{0pt}%
\pgfpathmoveto{\pgfqpoint{3.209520in}{0.524170in}}%
\pgfpathlineto{\pgfqpoint{3.209520in}{3.206606in}}%
\pgfusepath{stroke}%
\end{pgfscope}%
\begin{pgfscope}%
\pgfsetbuttcap%
\pgfsetroundjoin%
\definecolor{currentfill}{rgb}{0.000000,0.000000,0.000000}%
\pgfsetfillcolor{currentfill}%
\pgfsetlinewidth{0.803000pt}%
\definecolor{currentstroke}{rgb}{0.000000,0.000000,0.000000}%
\pgfsetstrokecolor{currentstroke}%
\pgfsetdash{}{0pt}%
\pgfsys@defobject{currentmarker}{\pgfqpoint{0.000000in}{-0.048611in}}{\pgfqpoint{0.000000in}{0.000000in}}{%
\pgfpathmoveto{\pgfqpoint{0.000000in}{0.000000in}}%
\pgfpathlineto{\pgfqpoint{0.000000in}{-0.048611in}}%
\pgfusepath{stroke,fill}%
}%
\begin{pgfscope}%
\pgfsys@transformshift{3.209520in}{0.524170in}%
\pgfsys@useobject{currentmarker}{}%
\end{pgfscope}%
\end{pgfscope}%
\begin{pgfscope}%
\definecolor{textcolor}{rgb}{0.000000,0.000000,0.000000}%
\pgfsetstrokecolor{textcolor}%
\pgfsetfillcolor{textcolor}%
\pgftext[x=3.209520in,y=0.426948in,,top]{\color{textcolor}{\rmfamily\fontsize{8.000000}{9.600000}\selectfont\catcode`\^=\active\def^{\ifmmode\sp\else\^{}\fi}\catcode`\%=\active\def%{\%}$\mathdefault{10^{4}}$}}%
\end{pgfscope}%
\begin{pgfscope}%
\pgfpathrectangle{\pgfqpoint{0.689445in}{0.524170in}}{\pgfqpoint{4.106789in}{2.682436in}}%
\pgfusepath{clip}%
\pgfsetrectcap%
\pgfsetroundjoin%
\pgfsetlinewidth{0.803000pt}%
\definecolor{currentstroke}{rgb}{0.450000,0.450000,0.450000}%
\pgfsetstrokecolor{currentstroke}%
\pgfsetdash{}{0pt}%
\pgfpathmoveto{\pgfqpoint{3.676201in}{0.524170in}}%
\pgfpathlineto{\pgfqpoint{3.676201in}{3.206606in}}%
\pgfusepath{stroke}%
\end{pgfscope}%
\begin{pgfscope}%
\pgfsetbuttcap%
\pgfsetroundjoin%
\definecolor{currentfill}{rgb}{0.000000,0.000000,0.000000}%
\pgfsetfillcolor{currentfill}%
\pgfsetlinewidth{0.803000pt}%
\definecolor{currentstroke}{rgb}{0.000000,0.000000,0.000000}%
\pgfsetstrokecolor{currentstroke}%
\pgfsetdash{}{0pt}%
\pgfsys@defobject{currentmarker}{\pgfqpoint{0.000000in}{-0.048611in}}{\pgfqpoint{0.000000in}{0.000000in}}{%
\pgfpathmoveto{\pgfqpoint{0.000000in}{0.000000in}}%
\pgfpathlineto{\pgfqpoint{0.000000in}{-0.048611in}}%
\pgfusepath{stroke,fill}%
}%
\begin{pgfscope}%
\pgfsys@transformshift{3.676201in}{0.524170in}%
\pgfsys@useobject{currentmarker}{}%
\end{pgfscope}%
\end{pgfscope}%
\begin{pgfscope}%
\definecolor{textcolor}{rgb}{0.000000,0.000000,0.000000}%
\pgfsetstrokecolor{textcolor}%
\pgfsetfillcolor{textcolor}%
\pgftext[x=3.676201in,y=0.426948in,,top]{\color{textcolor}{\rmfamily\fontsize{8.000000}{9.600000}\selectfont\catcode`\^=\active\def^{\ifmmode\sp\else\^{}\fi}\catcode`\%=\active\def%{\%}$\mathdefault{10^{5}}$}}%
\end{pgfscope}%
\begin{pgfscope}%
\pgfpathrectangle{\pgfqpoint{0.689445in}{0.524170in}}{\pgfqpoint{4.106789in}{2.682436in}}%
\pgfusepath{clip}%
\pgfsetrectcap%
\pgfsetroundjoin%
\pgfsetlinewidth{0.803000pt}%
\definecolor{currentstroke}{rgb}{0.450000,0.450000,0.450000}%
\pgfsetstrokecolor{currentstroke}%
\pgfsetdash{}{0pt}%
\pgfpathmoveto{\pgfqpoint{4.142881in}{0.524170in}}%
\pgfpathlineto{\pgfqpoint{4.142881in}{3.206606in}}%
\pgfusepath{stroke}%
\end{pgfscope}%
\begin{pgfscope}%
\pgfsetbuttcap%
\pgfsetroundjoin%
\definecolor{currentfill}{rgb}{0.000000,0.000000,0.000000}%
\pgfsetfillcolor{currentfill}%
\pgfsetlinewidth{0.803000pt}%
\definecolor{currentstroke}{rgb}{0.000000,0.000000,0.000000}%
\pgfsetstrokecolor{currentstroke}%
\pgfsetdash{}{0pt}%
\pgfsys@defobject{currentmarker}{\pgfqpoint{0.000000in}{-0.048611in}}{\pgfqpoint{0.000000in}{0.000000in}}{%
\pgfpathmoveto{\pgfqpoint{0.000000in}{0.000000in}}%
\pgfpathlineto{\pgfqpoint{0.000000in}{-0.048611in}}%
\pgfusepath{stroke,fill}%
}%
\begin{pgfscope}%
\pgfsys@transformshift{4.142881in}{0.524170in}%
\pgfsys@useobject{currentmarker}{}%
\end{pgfscope}%
\end{pgfscope}%
\begin{pgfscope}%
\definecolor{textcolor}{rgb}{0.000000,0.000000,0.000000}%
\pgfsetstrokecolor{textcolor}%
\pgfsetfillcolor{textcolor}%
\pgftext[x=4.142881in,y=0.426948in,,top]{\color{textcolor}{\rmfamily\fontsize{8.000000}{9.600000}\selectfont\catcode`\^=\active\def^{\ifmmode\sp\else\^{}\fi}\catcode`\%=\active\def%{\%}$\mathdefault{10^{6}}$}}%
\end{pgfscope}%
\begin{pgfscope}%
\pgfpathrectangle{\pgfqpoint{0.689445in}{0.524170in}}{\pgfqpoint{4.106789in}{2.682436in}}%
\pgfusepath{clip}%
\pgfsetrectcap%
\pgfsetroundjoin%
\pgfsetlinewidth{0.803000pt}%
\definecolor{currentstroke}{rgb}{0.450000,0.450000,0.450000}%
\pgfsetstrokecolor{currentstroke}%
\pgfsetdash{}{0pt}%
\pgfpathmoveto{\pgfqpoint{4.609562in}{0.524170in}}%
\pgfpathlineto{\pgfqpoint{4.609562in}{3.206606in}}%
\pgfusepath{stroke}%
\end{pgfscope}%
\begin{pgfscope}%
\pgfsetbuttcap%
\pgfsetroundjoin%
\definecolor{currentfill}{rgb}{0.000000,0.000000,0.000000}%
\pgfsetfillcolor{currentfill}%
\pgfsetlinewidth{0.803000pt}%
\definecolor{currentstroke}{rgb}{0.000000,0.000000,0.000000}%
\pgfsetstrokecolor{currentstroke}%
\pgfsetdash{}{0pt}%
\pgfsys@defobject{currentmarker}{\pgfqpoint{0.000000in}{-0.048611in}}{\pgfqpoint{0.000000in}{0.000000in}}{%
\pgfpathmoveto{\pgfqpoint{0.000000in}{0.000000in}}%
\pgfpathlineto{\pgfqpoint{0.000000in}{-0.048611in}}%
\pgfusepath{stroke,fill}%
}%
\begin{pgfscope}%
\pgfsys@transformshift{4.609562in}{0.524170in}%
\pgfsys@useobject{currentmarker}{}%
\end{pgfscope}%
\end{pgfscope}%
\begin{pgfscope}%
\definecolor{textcolor}{rgb}{0.000000,0.000000,0.000000}%
\pgfsetstrokecolor{textcolor}%
\pgfsetfillcolor{textcolor}%
\pgftext[x=4.609562in,y=0.426948in,,top]{\color{textcolor}{\rmfamily\fontsize{8.000000}{9.600000}\selectfont\catcode`\^=\active\def^{\ifmmode\sp\else\^{}\fi}\catcode`\%=\active\def%{\%}$\mathdefault{10^{7}}$}}%
\end{pgfscope}%
\begin{pgfscope}%
\pgfsetbuttcap%
\pgfsetroundjoin%
\definecolor{currentfill}{rgb}{0.000000,0.000000,0.000000}%
\pgfsetfillcolor{currentfill}%
\pgfsetlinewidth{0.602250pt}%
\definecolor{currentstroke}{rgb}{0.000000,0.000000,0.000000}%
\pgfsetstrokecolor{currentstroke}%
\pgfsetdash{}{0pt}%
\pgfsys@defobject{currentmarker}{\pgfqpoint{0.000000in}{-0.027778in}}{\pgfqpoint{0.000000in}{0.000000in}}{%
\pgfpathmoveto{\pgfqpoint{0.000000in}{0.000000in}}%
\pgfpathlineto{\pgfqpoint{0.000000in}{-0.027778in}}%
\pgfusepath{stroke,fill}%
}%
\begin{pgfscope}%
\pgfsys@transformshift{0.690407in}{0.524170in}%
\pgfsys@useobject{currentmarker}{}%
\end{pgfscope}%
\end{pgfscope}%
\begin{pgfscope}%
\pgfsetbuttcap%
\pgfsetroundjoin%
\definecolor{currentfill}{rgb}{0.000000,0.000000,0.000000}%
\pgfsetfillcolor{currentfill}%
\pgfsetlinewidth{0.602250pt}%
\definecolor{currentstroke}{rgb}{0.000000,0.000000,0.000000}%
\pgfsetstrokecolor{currentstroke}%
\pgfsetdash{}{0pt}%
\pgfsys@defobject{currentmarker}{\pgfqpoint{0.000000in}{-0.027778in}}{\pgfqpoint{0.000000in}{0.000000in}}{%
\pgfpathmoveto{\pgfqpoint{0.000000in}{0.000000in}}%
\pgfpathlineto{\pgfqpoint{0.000000in}{-0.027778in}}%
\pgfusepath{stroke,fill}%
}%
\begin{pgfscope}%
\pgfsys@transformshift{0.735633in}{0.524170in}%
\pgfsys@useobject{currentmarker}{}%
\end{pgfscope}%
\end{pgfscope}%
\begin{pgfscope}%
\pgfsetbuttcap%
\pgfsetroundjoin%
\definecolor{currentfill}{rgb}{0.000000,0.000000,0.000000}%
\pgfsetfillcolor{currentfill}%
\pgfsetlinewidth{0.602250pt}%
\definecolor{currentstroke}{rgb}{0.000000,0.000000,0.000000}%
\pgfsetstrokecolor{currentstroke}%
\pgfsetdash{}{0pt}%
\pgfsys@defobject{currentmarker}{\pgfqpoint{0.000000in}{-0.027778in}}{\pgfqpoint{0.000000in}{0.000000in}}{%
\pgfpathmoveto{\pgfqpoint{0.000000in}{0.000000in}}%
\pgfpathlineto{\pgfqpoint{0.000000in}{-0.027778in}}%
\pgfusepath{stroke,fill}%
}%
\begin{pgfscope}%
\pgfsys@transformshift{0.772585in}{0.524170in}%
\pgfsys@useobject{currentmarker}{}%
\end{pgfscope}%
\end{pgfscope}%
\begin{pgfscope}%
\pgfsetbuttcap%
\pgfsetroundjoin%
\definecolor{currentfill}{rgb}{0.000000,0.000000,0.000000}%
\pgfsetfillcolor{currentfill}%
\pgfsetlinewidth{0.602250pt}%
\definecolor{currentstroke}{rgb}{0.000000,0.000000,0.000000}%
\pgfsetstrokecolor{currentstroke}%
\pgfsetdash{}{0pt}%
\pgfsys@defobject{currentmarker}{\pgfqpoint{0.000000in}{-0.027778in}}{\pgfqpoint{0.000000in}{0.000000in}}{%
\pgfpathmoveto{\pgfqpoint{0.000000in}{0.000000in}}%
\pgfpathlineto{\pgfqpoint{0.000000in}{-0.027778in}}%
\pgfusepath{stroke,fill}%
}%
\begin{pgfscope}%
\pgfsys@transformshift{0.803828in}{0.524170in}%
\pgfsys@useobject{currentmarker}{}%
\end{pgfscope}%
\end{pgfscope}%
\begin{pgfscope}%
\pgfsetbuttcap%
\pgfsetroundjoin%
\definecolor{currentfill}{rgb}{0.000000,0.000000,0.000000}%
\pgfsetfillcolor{currentfill}%
\pgfsetlinewidth{0.602250pt}%
\definecolor{currentstroke}{rgb}{0.000000,0.000000,0.000000}%
\pgfsetstrokecolor{currentstroke}%
\pgfsetdash{}{0pt}%
\pgfsys@defobject{currentmarker}{\pgfqpoint{0.000000in}{-0.027778in}}{\pgfqpoint{0.000000in}{0.000000in}}{%
\pgfpathmoveto{\pgfqpoint{0.000000in}{0.000000in}}%
\pgfpathlineto{\pgfqpoint{0.000000in}{-0.027778in}}%
\pgfusepath{stroke,fill}%
}%
\begin{pgfscope}%
\pgfsys@transformshift{0.830891in}{0.524170in}%
\pgfsys@useobject{currentmarker}{}%
\end{pgfscope}%
\end{pgfscope}%
\begin{pgfscope}%
\pgfsetbuttcap%
\pgfsetroundjoin%
\definecolor{currentfill}{rgb}{0.000000,0.000000,0.000000}%
\pgfsetfillcolor{currentfill}%
\pgfsetlinewidth{0.602250pt}%
\definecolor{currentstroke}{rgb}{0.000000,0.000000,0.000000}%
\pgfsetstrokecolor{currentstroke}%
\pgfsetdash{}{0pt}%
\pgfsys@defobject{currentmarker}{\pgfqpoint{0.000000in}{-0.027778in}}{\pgfqpoint{0.000000in}{0.000000in}}{%
\pgfpathmoveto{\pgfqpoint{0.000000in}{0.000000in}}%
\pgfpathlineto{\pgfqpoint{0.000000in}{-0.027778in}}%
\pgfusepath{stroke,fill}%
}%
\begin{pgfscope}%
\pgfsys@transformshift{0.854763in}{0.524170in}%
\pgfsys@useobject{currentmarker}{}%
\end{pgfscope}%
\end{pgfscope}%
\begin{pgfscope}%
\pgfsetbuttcap%
\pgfsetroundjoin%
\definecolor{currentfill}{rgb}{0.000000,0.000000,0.000000}%
\pgfsetfillcolor{currentfill}%
\pgfsetlinewidth{0.602250pt}%
\definecolor{currentstroke}{rgb}{0.000000,0.000000,0.000000}%
\pgfsetstrokecolor{currentstroke}%
\pgfsetdash{}{0pt}%
\pgfsys@defobject{currentmarker}{\pgfqpoint{0.000000in}{-0.027778in}}{\pgfqpoint{0.000000in}{0.000000in}}{%
\pgfpathmoveto{\pgfqpoint{0.000000in}{0.000000in}}%
\pgfpathlineto{\pgfqpoint{0.000000in}{-0.027778in}}%
\pgfusepath{stroke,fill}%
}%
\begin{pgfscope}%
\pgfsys@transformshift{1.016602in}{0.524170in}%
\pgfsys@useobject{currentmarker}{}%
\end{pgfscope}%
\end{pgfscope}%
\begin{pgfscope}%
\pgfsetbuttcap%
\pgfsetroundjoin%
\definecolor{currentfill}{rgb}{0.000000,0.000000,0.000000}%
\pgfsetfillcolor{currentfill}%
\pgfsetlinewidth{0.602250pt}%
\definecolor{currentstroke}{rgb}{0.000000,0.000000,0.000000}%
\pgfsetstrokecolor{currentstroke}%
\pgfsetdash{}{0pt}%
\pgfsys@defobject{currentmarker}{\pgfqpoint{0.000000in}{-0.027778in}}{\pgfqpoint{0.000000in}{0.000000in}}{%
\pgfpathmoveto{\pgfqpoint{0.000000in}{0.000000in}}%
\pgfpathlineto{\pgfqpoint{0.000000in}{-0.027778in}}%
\pgfusepath{stroke,fill}%
}%
\begin{pgfscope}%
\pgfsys@transformshift{1.098781in}{0.524170in}%
\pgfsys@useobject{currentmarker}{}%
\end{pgfscope}%
\end{pgfscope}%
\begin{pgfscope}%
\pgfsetbuttcap%
\pgfsetroundjoin%
\definecolor{currentfill}{rgb}{0.000000,0.000000,0.000000}%
\pgfsetfillcolor{currentfill}%
\pgfsetlinewidth{0.602250pt}%
\definecolor{currentstroke}{rgb}{0.000000,0.000000,0.000000}%
\pgfsetstrokecolor{currentstroke}%
\pgfsetdash{}{0pt}%
\pgfsys@defobject{currentmarker}{\pgfqpoint{0.000000in}{-0.027778in}}{\pgfqpoint{0.000000in}{0.000000in}}{%
\pgfpathmoveto{\pgfqpoint{0.000000in}{0.000000in}}%
\pgfpathlineto{\pgfqpoint{0.000000in}{-0.027778in}}%
\pgfusepath{stroke,fill}%
}%
\begin{pgfscope}%
\pgfsys@transformshift{1.157087in}{0.524170in}%
\pgfsys@useobject{currentmarker}{}%
\end{pgfscope}%
\end{pgfscope}%
\begin{pgfscope}%
\pgfsetbuttcap%
\pgfsetroundjoin%
\definecolor{currentfill}{rgb}{0.000000,0.000000,0.000000}%
\pgfsetfillcolor{currentfill}%
\pgfsetlinewidth{0.602250pt}%
\definecolor{currentstroke}{rgb}{0.000000,0.000000,0.000000}%
\pgfsetstrokecolor{currentstroke}%
\pgfsetdash{}{0pt}%
\pgfsys@defobject{currentmarker}{\pgfqpoint{0.000000in}{-0.027778in}}{\pgfqpoint{0.000000in}{0.000000in}}{%
\pgfpathmoveto{\pgfqpoint{0.000000in}{0.000000in}}%
\pgfpathlineto{\pgfqpoint{0.000000in}{-0.027778in}}%
\pgfusepath{stroke,fill}%
}%
\begin{pgfscope}%
\pgfsys@transformshift{1.202313in}{0.524170in}%
\pgfsys@useobject{currentmarker}{}%
\end{pgfscope}%
\end{pgfscope}%
\begin{pgfscope}%
\pgfsetbuttcap%
\pgfsetroundjoin%
\definecolor{currentfill}{rgb}{0.000000,0.000000,0.000000}%
\pgfsetfillcolor{currentfill}%
\pgfsetlinewidth{0.602250pt}%
\definecolor{currentstroke}{rgb}{0.000000,0.000000,0.000000}%
\pgfsetstrokecolor{currentstroke}%
\pgfsetdash{}{0pt}%
\pgfsys@defobject{currentmarker}{\pgfqpoint{0.000000in}{-0.027778in}}{\pgfqpoint{0.000000in}{0.000000in}}{%
\pgfpathmoveto{\pgfqpoint{0.000000in}{0.000000in}}%
\pgfpathlineto{\pgfqpoint{0.000000in}{-0.027778in}}%
\pgfusepath{stroke,fill}%
}%
\begin{pgfscope}%
\pgfsys@transformshift{1.239265in}{0.524170in}%
\pgfsys@useobject{currentmarker}{}%
\end{pgfscope}%
\end{pgfscope}%
\begin{pgfscope}%
\pgfsetbuttcap%
\pgfsetroundjoin%
\definecolor{currentfill}{rgb}{0.000000,0.000000,0.000000}%
\pgfsetfillcolor{currentfill}%
\pgfsetlinewidth{0.602250pt}%
\definecolor{currentstroke}{rgb}{0.000000,0.000000,0.000000}%
\pgfsetstrokecolor{currentstroke}%
\pgfsetdash{}{0pt}%
\pgfsys@defobject{currentmarker}{\pgfqpoint{0.000000in}{-0.027778in}}{\pgfqpoint{0.000000in}{0.000000in}}{%
\pgfpathmoveto{\pgfqpoint{0.000000in}{0.000000in}}%
\pgfpathlineto{\pgfqpoint{0.000000in}{-0.027778in}}%
\pgfusepath{stroke,fill}%
}%
\begin{pgfscope}%
\pgfsys@transformshift{1.270508in}{0.524170in}%
\pgfsys@useobject{currentmarker}{}%
\end{pgfscope}%
\end{pgfscope}%
\begin{pgfscope}%
\pgfsetbuttcap%
\pgfsetroundjoin%
\definecolor{currentfill}{rgb}{0.000000,0.000000,0.000000}%
\pgfsetfillcolor{currentfill}%
\pgfsetlinewidth{0.602250pt}%
\definecolor{currentstroke}{rgb}{0.000000,0.000000,0.000000}%
\pgfsetstrokecolor{currentstroke}%
\pgfsetdash{}{0pt}%
\pgfsys@defobject{currentmarker}{\pgfqpoint{0.000000in}{-0.027778in}}{\pgfqpoint{0.000000in}{0.000000in}}{%
\pgfpathmoveto{\pgfqpoint{0.000000in}{0.000000in}}%
\pgfpathlineto{\pgfqpoint{0.000000in}{-0.027778in}}%
\pgfusepath{stroke,fill}%
}%
\begin{pgfscope}%
\pgfsys@transformshift{1.297572in}{0.524170in}%
\pgfsys@useobject{currentmarker}{}%
\end{pgfscope}%
\end{pgfscope}%
\begin{pgfscope}%
\pgfsetbuttcap%
\pgfsetroundjoin%
\definecolor{currentfill}{rgb}{0.000000,0.000000,0.000000}%
\pgfsetfillcolor{currentfill}%
\pgfsetlinewidth{0.602250pt}%
\definecolor{currentstroke}{rgb}{0.000000,0.000000,0.000000}%
\pgfsetstrokecolor{currentstroke}%
\pgfsetdash{}{0pt}%
\pgfsys@defobject{currentmarker}{\pgfqpoint{0.000000in}{-0.027778in}}{\pgfqpoint{0.000000in}{0.000000in}}{%
\pgfpathmoveto{\pgfqpoint{0.000000in}{0.000000in}}%
\pgfpathlineto{\pgfqpoint{0.000000in}{-0.027778in}}%
\pgfusepath{stroke,fill}%
}%
\begin{pgfscope}%
\pgfsys@transformshift{1.321444in}{0.524170in}%
\pgfsys@useobject{currentmarker}{}%
\end{pgfscope}%
\end{pgfscope}%
\begin{pgfscope}%
\pgfsetbuttcap%
\pgfsetroundjoin%
\definecolor{currentfill}{rgb}{0.000000,0.000000,0.000000}%
\pgfsetfillcolor{currentfill}%
\pgfsetlinewidth{0.602250pt}%
\definecolor{currentstroke}{rgb}{0.000000,0.000000,0.000000}%
\pgfsetstrokecolor{currentstroke}%
\pgfsetdash{}{0pt}%
\pgfsys@defobject{currentmarker}{\pgfqpoint{0.000000in}{-0.027778in}}{\pgfqpoint{0.000000in}{0.000000in}}{%
\pgfpathmoveto{\pgfqpoint{0.000000in}{0.000000in}}%
\pgfpathlineto{\pgfqpoint{0.000000in}{-0.027778in}}%
\pgfusepath{stroke,fill}%
}%
\begin{pgfscope}%
\pgfsys@transformshift{1.483283in}{0.524170in}%
\pgfsys@useobject{currentmarker}{}%
\end{pgfscope}%
\end{pgfscope}%
\begin{pgfscope}%
\pgfsetbuttcap%
\pgfsetroundjoin%
\definecolor{currentfill}{rgb}{0.000000,0.000000,0.000000}%
\pgfsetfillcolor{currentfill}%
\pgfsetlinewidth{0.602250pt}%
\definecolor{currentstroke}{rgb}{0.000000,0.000000,0.000000}%
\pgfsetstrokecolor{currentstroke}%
\pgfsetdash{}{0pt}%
\pgfsys@defobject{currentmarker}{\pgfqpoint{0.000000in}{-0.027778in}}{\pgfqpoint{0.000000in}{0.000000in}}{%
\pgfpathmoveto{\pgfqpoint{0.000000in}{0.000000in}}%
\pgfpathlineto{\pgfqpoint{0.000000in}{-0.027778in}}%
\pgfusepath{stroke,fill}%
}%
\begin{pgfscope}%
\pgfsys@transformshift{1.565461in}{0.524170in}%
\pgfsys@useobject{currentmarker}{}%
\end{pgfscope}%
\end{pgfscope}%
\begin{pgfscope}%
\pgfsetbuttcap%
\pgfsetroundjoin%
\definecolor{currentfill}{rgb}{0.000000,0.000000,0.000000}%
\pgfsetfillcolor{currentfill}%
\pgfsetlinewidth{0.602250pt}%
\definecolor{currentstroke}{rgb}{0.000000,0.000000,0.000000}%
\pgfsetstrokecolor{currentstroke}%
\pgfsetdash{}{0pt}%
\pgfsys@defobject{currentmarker}{\pgfqpoint{0.000000in}{-0.027778in}}{\pgfqpoint{0.000000in}{0.000000in}}{%
\pgfpathmoveto{\pgfqpoint{0.000000in}{0.000000in}}%
\pgfpathlineto{\pgfqpoint{0.000000in}{-0.027778in}}%
\pgfusepath{stroke,fill}%
}%
\begin{pgfscope}%
\pgfsys@transformshift{1.623768in}{0.524170in}%
\pgfsys@useobject{currentmarker}{}%
\end{pgfscope}%
\end{pgfscope}%
\begin{pgfscope}%
\pgfsetbuttcap%
\pgfsetroundjoin%
\definecolor{currentfill}{rgb}{0.000000,0.000000,0.000000}%
\pgfsetfillcolor{currentfill}%
\pgfsetlinewidth{0.602250pt}%
\definecolor{currentstroke}{rgb}{0.000000,0.000000,0.000000}%
\pgfsetstrokecolor{currentstroke}%
\pgfsetdash{}{0pt}%
\pgfsys@defobject{currentmarker}{\pgfqpoint{0.000000in}{-0.027778in}}{\pgfqpoint{0.000000in}{0.000000in}}{%
\pgfpathmoveto{\pgfqpoint{0.000000in}{0.000000in}}%
\pgfpathlineto{\pgfqpoint{0.000000in}{-0.027778in}}%
\pgfusepath{stroke,fill}%
}%
\begin{pgfscope}%
\pgfsys@transformshift{1.668994in}{0.524170in}%
\pgfsys@useobject{currentmarker}{}%
\end{pgfscope}%
\end{pgfscope}%
\begin{pgfscope}%
\pgfsetbuttcap%
\pgfsetroundjoin%
\definecolor{currentfill}{rgb}{0.000000,0.000000,0.000000}%
\pgfsetfillcolor{currentfill}%
\pgfsetlinewidth{0.602250pt}%
\definecolor{currentstroke}{rgb}{0.000000,0.000000,0.000000}%
\pgfsetstrokecolor{currentstroke}%
\pgfsetdash{}{0pt}%
\pgfsys@defobject{currentmarker}{\pgfqpoint{0.000000in}{-0.027778in}}{\pgfqpoint{0.000000in}{0.000000in}}{%
\pgfpathmoveto{\pgfqpoint{0.000000in}{0.000000in}}%
\pgfpathlineto{\pgfqpoint{0.000000in}{-0.027778in}}%
\pgfusepath{stroke,fill}%
}%
\begin{pgfscope}%
\pgfsys@transformshift{1.705946in}{0.524170in}%
\pgfsys@useobject{currentmarker}{}%
\end{pgfscope}%
\end{pgfscope}%
\begin{pgfscope}%
\pgfsetbuttcap%
\pgfsetroundjoin%
\definecolor{currentfill}{rgb}{0.000000,0.000000,0.000000}%
\pgfsetfillcolor{currentfill}%
\pgfsetlinewidth{0.602250pt}%
\definecolor{currentstroke}{rgb}{0.000000,0.000000,0.000000}%
\pgfsetstrokecolor{currentstroke}%
\pgfsetdash{}{0pt}%
\pgfsys@defobject{currentmarker}{\pgfqpoint{0.000000in}{-0.027778in}}{\pgfqpoint{0.000000in}{0.000000in}}{%
\pgfpathmoveto{\pgfqpoint{0.000000in}{0.000000in}}%
\pgfpathlineto{\pgfqpoint{0.000000in}{-0.027778in}}%
\pgfusepath{stroke,fill}%
}%
\begin{pgfscope}%
\pgfsys@transformshift{1.737189in}{0.524170in}%
\pgfsys@useobject{currentmarker}{}%
\end{pgfscope}%
\end{pgfscope}%
\begin{pgfscope}%
\pgfsetbuttcap%
\pgfsetroundjoin%
\definecolor{currentfill}{rgb}{0.000000,0.000000,0.000000}%
\pgfsetfillcolor{currentfill}%
\pgfsetlinewidth{0.602250pt}%
\definecolor{currentstroke}{rgb}{0.000000,0.000000,0.000000}%
\pgfsetstrokecolor{currentstroke}%
\pgfsetdash{}{0pt}%
\pgfsys@defobject{currentmarker}{\pgfqpoint{0.000000in}{-0.027778in}}{\pgfqpoint{0.000000in}{0.000000in}}{%
\pgfpathmoveto{\pgfqpoint{0.000000in}{0.000000in}}%
\pgfpathlineto{\pgfqpoint{0.000000in}{-0.027778in}}%
\pgfusepath{stroke,fill}%
}%
\begin{pgfscope}%
\pgfsys@transformshift{1.764252in}{0.524170in}%
\pgfsys@useobject{currentmarker}{}%
\end{pgfscope}%
\end{pgfscope}%
\begin{pgfscope}%
\pgfsetbuttcap%
\pgfsetroundjoin%
\definecolor{currentfill}{rgb}{0.000000,0.000000,0.000000}%
\pgfsetfillcolor{currentfill}%
\pgfsetlinewidth{0.602250pt}%
\definecolor{currentstroke}{rgb}{0.000000,0.000000,0.000000}%
\pgfsetstrokecolor{currentstroke}%
\pgfsetdash{}{0pt}%
\pgfsys@defobject{currentmarker}{\pgfqpoint{0.000000in}{-0.027778in}}{\pgfqpoint{0.000000in}{0.000000in}}{%
\pgfpathmoveto{\pgfqpoint{0.000000in}{0.000000in}}%
\pgfpathlineto{\pgfqpoint{0.000000in}{-0.027778in}}%
\pgfusepath{stroke,fill}%
}%
\begin{pgfscope}%
\pgfsys@transformshift{1.788124in}{0.524170in}%
\pgfsys@useobject{currentmarker}{}%
\end{pgfscope}%
\end{pgfscope}%
\begin{pgfscope}%
\pgfsetbuttcap%
\pgfsetroundjoin%
\definecolor{currentfill}{rgb}{0.000000,0.000000,0.000000}%
\pgfsetfillcolor{currentfill}%
\pgfsetlinewidth{0.602250pt}%
\definecolor{currentstroke}{rgb}{0.000000,0.000000,0.000000}%
\pgfsetstrokecolor{currentstroke}%
\pgfsetdash{}{0pt}%
\pgfsys@defobject{currentmarker}{\pgfqpoint{0.000000in}{-0.027778in}}{\pgfqpoint{0.000000in}{0.000000in}}{%
\pgfpathmoveto{\pgfqpoint{0.000000in}{0.000000in}}%
\pgfpathlineto{\pgfqpoint{0.000000in}{-0.027778in}}%
\pgfusepath{stroke,fill}%
}%
\begin{pgfscope}%
\pgfsys@transformshift{1.949963in}{0.524170in}%
\pgfsys@useobject{currentmarker}{}%
\end{pgfscope}%
\end{pgfscope}%
\begin{pgfscope}%
\pgfsetbuttcap%
\pgfsetroundjoin%
\definecolor{currentfill}{rgb}{0.000000,0.000000,0.000000}%
\pgfsetfillcolor{currentfill}%
\pgfsetlinewidth{0.602250pt}%
\definecolor{currentstroke}{rgb}{0.000000,0.000000,0.000000}%
\pgfsetstrokecolor{currentstroke}%
\pgfsetdash{}{0pt}%
\pgfsys@defobject{currentmarker}{\pgfqpoint{0.000000in}{-0.027778in}}{\pgfqpoint{0.000000in}{0.000000in}}{%
\pgfpathmoveto{\pgfqpoint{0.000000in}{0.000000in}}%
\pgfpathlineto{\pgfqpoint{0.000000in}{-0.027778in}}%
\pgfusepath{stroke,fill}%
}%
\begin{pgfscope}%
\pgfsys@transformshift{2.032142in}{0.524170in}%
\pgfsys@useobject{currentmarker}{}%
\end{pgfscope}%
\end{pgfscope}%
\begin{pgfscope}%
\pgfsetbuttcap%
\pgfsetroundjoin%
\definecolor{currentfill}{rgb}{0.000000,0.000000,0.000000}%
\pgfsetfillcolor{currentfill}%
\pgfsetlinewidth{0.602250pt}%
\definecolor{currentstroke}{rgb}{0.000000,0.000000,0.000000}%
\pgfsetstrokecolor{currentstroke}%
\pgfsetdash{}{0pt}%
\pgfsys@defobject{currentmarker}{\pgfqpoint{0.000000in}{-0.027778in}}{\pgfqpoint{0.000000in}{0.000000in}}{%
\pgfpathmoveto{\pgfqpoint{0.000000in}{0.000000in}}%
\pgfpathlineto{\pgfqpoint{0.000000in}{-0.027778in}}%
\pgfusepath{stroke,fill}%
}%
\begin{pgfscope}%
\pgfsys@transformshift{2.090448in}{0.524170in}%
\pgfsys@useobject{currentmarker}{}%
\end{pgfscope}%
\end{pgfscope}%
\begin{pgfscope}%
\pgfsetbuttcap%
\pgfsetroundjoin%
\definecolor{currentfill}{rgb}{0.000000,0.000000,0.000000}%
\pgfsetfillcolor{currentfill}%
\pgfsetlinewidth{0.602250pt}%
\definecolor{currentstroke}{rgb}{0.000000,0.000000,0.000000}%
\pgfsetstrokecolor{currentstroke}%
\pgfsetdash{}{0pt}%
\pgfsys@defobject{currentmarker}{\pgfqpoint{0.000000in}{-0.027778in}}{\pgfqpoint{0.000000in}{0.000000in}}{%
\pgfpathmoveto{\pgfqpoint{0.000000in}{0.000000in}}%
\pgfpathlineto{\pgfqpoint{0.000000in}{-0.027778in}}%
\pgfusepath{stroke,fill}%
}%
\begin{pgfscope}%
\pgfsys@transformshift{2.135674in}{0.524170in}%
\pgfsys@useobject{currentmarker}{}%
\end{pgfscope}%
\end{pgfscope}%
\begin{pgfscope}%
\pgfsetbuttcap%
\pgfsetroundjoin%
\definecolor{currentfill}{rgb}{0.000000,0.000000,0.000000}%
\pgfsetfillcolor{currentfill}%
\pgfsetlinewidth{0.602250pt}%
\definecolor{currentstroke}{rgb}{0.000000,0.000000,0.000000}%
\pgfsetstrokecolor{currentstroke}%
\pgfsetdash{}{0pt}%
\pgfsys@defobject{currentmarker}{\pgfqpoint{0.000000in}{-0.027778in}}{\pgfqpoint{0.000000in}{0.000000in}}{%
\pgfpathmoveto{\pgfqpoint{0.000000in}{0.000000in}}%
\pgfpathlineto{\pgfqpoint{0.000000in}{-0.027778in}}%
\pgfusepath{stroke,fill}%
}%
\begin{pgfscope}%
\pgfsys@transformshift{2.172627in}{0.524170in}%
\pgfsys@useobject{currentmarker}{}%
\end{pgfscope}%
\end{pgfscope}%
\begin{pgfscope}%
\pgfsetbuttcap%
\pgfsetroundjoin%
\definecolor{currentfill}{rgb}{0.000000,0.000000,0.000000}%
\pgfsetfillcolor{currentfill}%
\pgfsetlinewidth{0.602250pt}%
\definecolor{currentstroke}{rgb}{0.000000,0.000000,0.000000}%
\pgfsetstrokecolor{currentstroke}%
\pgfsetdash{}{0pt}%
\pgfsys@defobject{currentmarker}{\pgfqpoint{0.000000in}{-0.027778in}}{\pgfqpoint{0.000000in}{0.000000in}}{%
\pgfpathmoveto{\pgfqpoint{0.000000in}{0.000000in}}%
\pgfpathlineto{\pgfqpoint{0.000000in}{-0.027778in}}%
\pgfusepath{stroke,fill}%
}%
\begin{pgfscope}%
\pgfsys@transformshift{2.203869in}{0.524170in}%
\pgfsys@useobject{currentmarker}{}%
\end{pgfscope}%
\end{pgfscope}%
\begin{pgfscope}%
\pgfsetbuttcap%
\pgfsetroundjoin%
\definecolor{currentfill}{rgb}{0.000000,0.000000,0.000000}%
\pgfsetfillcolor{currentfill}%
\pgfsetlinewidth{0.602250pt}%
\definecolor{currentstroke}{rgb}{0.000000,0.000000,0.000000}%
\pgfsetstrokecolor{currentstroke}%
\pgfsetdash{}{0pt}%
\pgfsys@defobject{currentmarker}{\pgfqpoint{0.000000in}{-0.027778in}}{\pgfqpoint{0.000000in}{0.000000in}}{%
\pgfpathmoveto{\pgfqpoint{0.000000in}{0.000000in}}%
\pgfpathlineto{\pgfqpoint{0.000000in}{-0.027778in}}%
\pgfusepath{stroke,fill}%
}%
\begin{pgfscope}%
\pgfsys@transformshift{2.230933in}{0.524170in}%
\pgfsys@useobject{currentmarker}{}%
\end{pgfscope}%
\end{pgfscope}%
\begin{pgfscope}%
\pgfsetbuttcap%
\pgfsetroundjoin%
\definecolor{currentfill}{rgb}{0.000000,0.000000,0.000000}%
\pgfsetfillcolor{currentfill}%
\pgfsetlinewidth{0.602250pt}%
\definecolor{currentstroke}{rgb}{0.000000,0.000000,0.000000}%
\pgfsetstrokecolor{currentstroke}%
\pgfsetdash{}{0pt}%
\pgfsys@defobject{currentmarker}{\pgfqpoint{0.000000in}{-0.027778in}}{\pgfqpoint{0.000000in}{0.000000in}}{%
\pgfpathmoveto{\pgfqpoint{0.000000in}{0.000000in}}%
\pgfpathlineto{\pgfqpoint{0.000000in}{-0.027778in}}%
\pgfusepath{stroke,fill}%
}%
\begin{pgfscope}%
\pgfsys@transformshift{2.254805in}{0.524170in}%
\pgfsys@useobject{currentmarker}{}%
\end{pgfscope}%
\end{pgfscope}%
\begin{pgfscope}%
\pgfsetbuttcap%
\pgfsetroundjoin%
\definecolor{currentfill}{rgb}{0.000000,0.000000,0.000000}%
\pgfsetfillcolor{currentfill}%
\pgfsetlinewidth{0.602250pt}%
\definecolor{currentstroke}{rgb}{0.000000,0.000000,0.000000}%
\pgfsetstrokecolor{currentstroke}%
\pgfsetdash{}{0pt}%
\pgfsys@defobject{currentmarker}{\pgfqpoint{0.000000in}{-0.027778in}}{\pgfqpoint{0.000000in}{0.000000in}}{%
\pgfpathmoveto{\pgfqpoint{0.000000in}{0.000000in}}%
\pgfpathlineto{\pgfqpoint{0.000000in}{-0.027778in}}%
\pgfusepath{stroke,fill}%
}%
\begin{pgfscope}%
\pgfsys@transformshift{2.416644in}{0.524170in}%
\pgfsys@useobject{currentmarker}{}%
\end{pgfscope}%
\end{pgfscope}%
\begin{pgfscope}%
\pgfsetbuttcap%
\pgfsetroundjoin%
\definecolor{currentfill}{rgb}{0.000000,0.000000,0.000000}%
\pgfsetfillcolor{currentfill}%
\pgfsetlinewidth{0.602250pt}%
\definecolor{currentstroke}{rgb}{0.000000,0.000000,0.000000}%
\pgfsetstrokecolor{currentstroke}%
\pgfsetdash{}{0pt}%
\pgfsys@defobject{currentmarker}{\pgfqpoint{0.000000in}{-0.027778in}}{\pgfqpoint{0.000000in}{0.000000in}}{%
\pgfpathmoveto{\pgfqpoint{0.000000in}{0.000000in}}%
\pgfpathlineto{\pgfqpoint{0.000000in}{-0.027778in}}%
\pgfusepath{stroke,fill}%
}%
\begin{pgfscope}%
\pgfsys@transformshift{2.498822in}{0.524170in}%
\pgfsys@useobject{currentmarker}{}%
\end{pgfscope}%
\end{pgfscope}%
\begin{pgfscope}%
\pgfsetbuttcap%
\pgfsetroundjoin%
\definecolor{currentfill}{rgb}{0.000000,0.000000,0.000000}%
\pgfsetfillcolor{currentfill}%
\pgfsetlinewidth{0.602250pt}%
\definecolor{currentstroke}{rgb}{0.000000,0.000000,0.000000}%
\pgfsetstrokecolor{currentstroke}%
\pgfsetdash{}{0pt}%
\pgfsys@defobject{currentmarker}{\pgfqpoint{0.000000in}{-0.027778in}}{\pgfqpoint{0.000000in}{0.000000in}}{%
\pgfpathmoveto{\pgfqpoint{0.000000in}{0.000000in}}%
\pgfpathlineto{\pgfqpoint{0.000000in}{-0.027778in}}%
\pgfusepath{stroke,fill}%
}%
\begin{pgfscope}%
\pgfsys@transformshift{2.557129in}{0.524170in}%
\pgfsys@useobject{currentmarker}{}%
\end{pgfscope}%
\end{pgfscope}%
\begin{pgfscope}%
\pgfsetbuttcap%
\pgfsetroundjoin%
\definecolor{currentfill}{rgb}{0.000000,0.000000,0.000000}%
\pgfsetfillcolor{currentfill}%
\pgfsetlinewidth{0.602250pt}%
\definecolor{currentstroke}{rgb}{0.000000,0.000000,0.000000}%
\pgfsetstrokecolor{currentstroke}%
\pgfsetdash{}{0pt}%
\pgfsys@defobject{currentmarker}{\pgfqpoint{0.000000in}{-0.027778in}}{\pgfqpoint{0.000000in}{0.000000in}}{%
\pgfpathmoveto{\pgfqpoint{0.000000in}{0.000000in}}%
\pgfpathlineto{\pgfqpoint{0.000000in}{-0.027778in}}%
\pgfusepath{stroke,fill}%
}%
\begin{pgfscope}%
\pgfsys@transformshift{2.602355in}{0.524170in}%
\pgfsys@useobject{currentmarker}{}%
\end{pgfscope}%
\end{pgfscope}%
\begin{pgfscope}%
\pgfsetbuttcap%
\pgfsetroundjoin%
\definecolor{currentfill}{rgb}{0.000000,0.000000,0.000000}%
\pgfsetfillcolor{currentfill}%
\pgfsetlinewidth{0.602250pt}%
\definecolor{currentstroke}{rgb}{0.000000,0.000000,0.000000}%
\pgfsetstrokecolor{currentstroke}%
\pgfsetdash{}{0pt}%
\pgfsys@defobject{currentmarker}{\pgfqpoint{0.000000in}{-0.027778in}}{\pgfqpoint{0.000000in}{0.000000in}}{%
\pgfpathmoveto{\pgfqpoint{0.000000in}{0.000000in}}%
\pgfpathlineto{\pgfqpoint{0.000000in}{-0.027778in}}%
\pgfusepath{stroke,fill}%
}%
\begin{pgfscope}%
\pgfsys@transformshift{2.639307in}{0.524170in}%
\pgfsys@useobject{currentmarker}{}%
\end{pgfscope}%
\end{pgfscope}%
\begin{pgfscope}%
\pgfsetbuttcap%
\pgfsetroundjoin%
\definecolor{currentfill}{rgb}{0.000000,0.000000,0.000000}%
\pgfsetfillcolor{currentfill}%
\pgfsetlinewidth{0.602250pt}%
\definecolor{currentstroke}{rgb}{0.000000,0.000000,0.000000}%
\pgfsetstrokecolor{currentstroke}%
\pgfsetdash{}{0pt}%
\pgfsys@defobject{currentmarker}{\pgfqpoint{0.000000in}{-0.027778in}}{\pgfqpoint{0.000000in}{0.000000in}}{%
\pgfpathmoveto{\pgfqpoint{0.000000in}{0.000000in}}%
\pgfpathlineto{\pgfqpoint{0.000000in}{-0.027778in}}%
\pgfusepath{stroke,fill}%
}%
\begin{pgfscope}%
\pgfsys@transformshift{2.670550in}{0.524170in}%
\pgfsys@useobject{currentmarker}{}%
\end{pgfscope}%
\end{pgfscope}%
\begin{pgfscope}%
\pgfsetbuttcap%
\pgfsetroundjoin%
\definecolor{currentfill}{rgb}{0.000000,0.000000,0.000000}%
\pgfsetfillcolor{currentfill}%
\pgfsetlinewidth{0.602250pt}%
\definecolor{currentstroke}{rgb}{0.000000,0.000000,0.000000}%
\pgfsetstrokecolor{currentstroke}%
\pgfsetdash{}{0pt}%
\pgfsys@defobject{currentmarker}{\pgfqpoint{0.000000in}{-0.027778in}}{\pgfqpoint{0.000000in}{0.000000in}}{%
\pgfpathmoveto{\pgfqpoint{0.000000in}{0.000000in}}%
\pgfpathlineto{\pgfqpoint{0.000000in}{-0.027778in}}%
\pgfusepath{stroke,fill}%
}%
\begin{pgfscope}%
\pgfsys@transformshift{2.697614in}{0.524170in}%
\pgfsys@useobject{currentmarker}{}%
\end{pgfscope}%
\end{pgfscope}%
\begin{pgfscope}%
\pgfsetbuttcap%
\pgfsetroundjoin%
\definecolor{currentfill}{rgb}{0.000000,0.000000,0.000000}%
\pgfsetfillcolor{currentfill}%
\pgfsetlinewidth{0.602250pt}%
\definecolor{currentstroke}{rgb}{0.000000,0.000000,0.000000}%
\pgfsetstrokecolor{currentstroke}%
\pgfsetdash{}{0pt}%
\pgfsys@defobject{currentmarker}{\pgfqpoint{0.000000in}{-0.027778in}}{\pgfqpoint{0.000000in}{0.000000in}}{%
\pgfpathmoveto{\pgfqpoint{0.000000in}{0.000000in}}%
\pgfpathlineto{\pgfqpoint{0.000000in}{-0.027778in}}%
\pgfusepath{stroke,fill}%
}%
\begin{pgfscope}%
\pgfsys@transformshift{2.721485in}{0.524170in}%
\pgfsys@useobject{currentmarker}{}%
\end{pgfscope}%
\end{pgfscope}%
\begin{pgfscope}%
\pgfsetbuttcap%
\pgfsetroundjoin%
\definecolor{currentfill}{rgb}{0.000000,0.000000,0.000000}%
\pgfsetfillcolor{currentfill}%
\pgfsetlinewidth{0.602250pt}%
\definecolor{currentstroke}{rgb}{0.000000,0.000000,0.000000}%
\pgfsetstrokecolor{currentstroke}%
\pgfsetdash{}{0pt}%
\pgfsys@defobject{currentmarker}{\pgfqpoint{0.000000in}{-0.027778in}}{\pgfqpoint{0.000000in}{0.000000in}}{%
\pgfpathmoveto{\pgfqpoint{0.000000in}{0.000000in}}%
\pgfpathlineto{\pgfqpoint{0.000000in}{-0.027778in}}%
\pgfusepath{stroke,fill}%
}%
\begin{pgfscope}%
\pgfsys@transformshift{2.883324in}{0.524170in}%
\pgfsys@useobject{currentmarker}{}%
\end{pgfscope}%
\end{pgfscope}%
\begin{pgfscope}%
\pgfsetbuttcap%
\pgfsetroundjoin%
\definecolor{currentfill}{rgb}{0.000000,0.000000,0.000000}%
\pgfsetfillcolor{currentfill}%
\pgfsetlinewidth{0.602250pt}%
\definecolor{currentstroke}{rgb}{0.000000,0.000000,0.000000}%
\pgfsetstrokecolor{currentstroke}%
\pgfsetdash{}{0pt}%
\pgfsys@defobject{currentmarker}{\pgfqpoint{0.000000in}{-0.027778in}}{\pgfqpoint{0.000000in}{0.000000in}}{%
\pgfpathmoveto{\pgfqpoint{0.000000in}{0.000000in}}%
\pgfpathlineto{\pgfqpoint{0.000000in}{-0.027778in}}%
\pgfusepath{stroke,fill}%
}%
\begin{pgfscope}%
\pgfsys@transformshift{2.965503in}{0.524170in}%
\pgfsys@useobject{currentmarker}{}%
\end{pgfscope}%
\end{pgfscope}%
\begin{pgfscope}%
\pgfsetbuttcap%
\pgfsetroundjoin%
\definecolor{currentfill}{rgb}{0.000000,0.000000,0.000000}%
\pgfsetfillcolor{currentfill}%
\pgfsetlinewidth{0.602250pt}%
\definecolor{currentstroke}{rgb}{0.000000,0.000000,0.000000}%
\pgfsetstrokecolor{currentstroke}%
\pgfsetdash{}{0pt}%
\pgfsys@defobject{currentmarker}{\pgfqpoint{0.000000in}{-0.027778in}}{\pgfqpoint{0.000000in}{0.000000in}}{%
\pgfpathmoveto{\pgfqpoint{0.000000in}{0.000000in}}%
\pgfpathlineto{\pgfqpoint{0.000000in}{-0.027778in}}%
\pgfusepath{stroke,fill}%
}%
\begin{pgfscope}%
\pgfsys@transformshift{3.023809in}{0.524170in}%
\pgfsys@useobject{currentmarker}{}%
\end{pgfscope}%
\end{pgfscope}%
\begin{pgfscope}%
\pgfsetbuttcap%
\pgfsetroundjoin%
\definecolor{currentfill}{rgb}{0.000000,0.000000,0.000000}%
\pgfsetfillcolor{currentfill}%
\pgfsetlinewidth{0.602250pt}%
\definecolor{currentstroke}{rgb}{0.000000,0.000000,0.000000}%
\pgfsetstrokecolor{currentstroke}%
\pgfsetdash{}{0pt}%
\pgfsys@defobject{currentmarker}{\pgfqpoint{0.000000in}{-0.027778in}}{\pgfqpoint{0.000000in}{0.000000in}}{%
\pgfpathmoveto{\pgfqpoint{0.000000in}{0.000000in}}%
\pgfpathlineto{\pgfqpoint{0.000000in}{-0.027778in}}%
\pgfusepath{stroke,fill}%
}%
\begin{pgfscope}%
\pgfsys@transformshift{3.069035in}{0.524170in}%
\pgfsys@useobject{currentmarker}{}%
\end{pgfscope}%
\end{pgfscope}%
\begin{pgfscope}%
\pgfsetbuttcap%
\pgfsetroundjoin%
\definecolor{currentfill}{rgb}{0.000000,0.000000,0.000000}%
\pgfsetfillcolor{currentfill}%
\pgfsetlinewidth{0.602250pt}%
\definecolor{currentstroke}{rgb}{0.000000,0.000000,0.000000}%
\pgfsetstrokecolor{currentstroke}%
\pgfsetdash{}{0pt}%
\pgfsys@defobject{currentmarker}{\pgfqpoint{0.000000in}{-0.027778in}}{\pgfqpoint{0.000000in}{0.000000in}}{%
\pgfpathmoveto{\pgfqpoint{0.000000in}{0.000000in}}%
\pgfpathlineto{\pgfqpoint{0.000000in}{-0.027778in}}%
\pgfusepath{stroke,fill}%
}%
\begin{pgfscope}%
\pgfsys@transformshift{3.105988in}{0.524170in}%
\pgfsys@useobject{currentmarker}{}%
\end{pgfscope}%
\end{pgfscope}%
\begin{pgfscope}%
\pgfsetbuttcap%
\pgfsetroundjoin%
\definecolor{currentfill}{rgb}{0.000000,0.000000,0.000000}%
\pgfsetfillcolor{currentfill}%
\pgfsetlinewidth{0.602250pt}%
\definecolor{currentstroke}{rgb}{0.000000,0.000000,0.000000}%
\pgfsetstrokecolor{currentstroke}%
\pgfsetdash{}{0pt}%
\pgfsys@defobject{currentmarker}{\pgfqpoint{0.000000in}{-0.027778in}}{\pgfqpoint{0.000000in}{0.000000in}}{%
\pgfpathmoveto{\pgfqpoint{0.000000in}{0.000000in}}%
\pgfpathlineto{\pgfqpoint{0.000000in}{-0.027778in}}%
\pgfusepath{stroke,fill}%
}%
\begin{pgfscope}%
\pgfsys@transformshift{3.137230in}{0.524170in}%
\pgfsys@useobject{currentmarker}{}%
\end{pgfscope}%
\end{pgfscope}%
\begin{pgfscope}%
\pgfsetbuttcap%
\pgfsetroundjoin%
\definecolor{currentfill}{rgb}{0.000000,0.000000,0.000000}%
\pgfsetfillcolor{currentfill}%
\pgfsetlinewidth{0.602250pt}%
\definecolor{currentstroke}{rgb}{0.000000,0.000000,0.000000}%
\pgfsetstrokecolor{currentstroke}%
\pgfsetdash{}{0pt}%
\pgfsys@defobject{currentmarker}{\pgfqpoint{0.000000in}{-0.027778in}}{\pgfqpoint{0.000000in}{0.000000in}}{%
\pgfpathmoveto{\pgfqpoint{0.000000in}{0.000000in}}%
\pgfpathlineto{\pgfqpoint{0.000000in}{-0.027778in}}%
\pgfusepath{stroke,fill}%
}%
\begin{pgfscope}%
\pgfsys@transformshift{3.164294in}{0.524170in}%
\pgfsys@useobject{currentmarker}{}%
\end{pgfscope}%
\end{pgfscope}%
\begin{pgfscope}%
\pgfsetbuttcap%
\pgfsetroundjoin%
\definecolor{currentfill}{rgb}{0.000000,0.000000,0.000000}%
\pgfsetfillcolor{currentfill}%
\pgfsetlinewidth{0.602250pt}%
\definecolor{currentstroke}{rgb}{0.000000,0.000000,0.000000}%
\pgfsetstrokecolor{currentstroke}%
\pgfsetdash{}{0pt}%
\pgfsys@defobject{currentmarker}{\pgfqpoint{0.000000in}{-0.027778in}}{\pgfqpoint{0.000000in}{0.000000in}}{%
\pgfpathmoveto{\pgfqpoint{0.000000in}{0.000000in}}%
\pgfpathlineto{\pgfqpoint{0.000000in}{-0.027778in}}%
\pgfusepath{stroke,fill}%
}%
\begin{pgfscope}%
\pgfsys@transformshift{3.188166in}{0.524170in}%
\pgfsys@useobject{currentmarker}{}%
\end{pgfscope}%
\end{pgfscope}%
\begin{pgfscope}%
\pgfsetbuttcap%
\pgfsetroundjoin%
\definecolor{currentfill}{rgb}{0.000000,0.000000,0.000000}%
\pgfsetfillcolor{currentfill}%
\pgfsetlinewidth{0.602250pt}%
\definecolor{currentstroke}{rgb}{0.000000,0.000000,0.000000}%
\pgfsetstrokecolor{currentstroke}%
\pgfsetdash{}{0pt}%
\pgfsys@defobject{currentmarker}{\pgfqpoint{0.000000in}{-0.027778in}}{\pgfqpoint{0.000000in}{0.000000in}}{%
\pgfpathmoveto{\pgfqpoint{0.000000in}{0.000000in}}%
\pgfpathlineto{\pgfqpoint{0.000000in}{-0.027778in}}%
\pgfusepath{stroke,fill}%
}%
\begin{pgfscope}%
\pgfsys@transformshift{3.350005in}{0.524170in}%
\pgfsys@useobject{currentmarker}{}%
\end{pgfscope}%
\end{pgfscope}%
\begin{pgfscope}%
\pgfsetbuttcap%
\pgfsetroundjoin%
\definecolor{currentfill}{rgb}{0.000000,0.000000,0.000000}%
\pgfsetfillcolor{currentfill}%
\pgfsetlinewidth{0.602250pt}%
\definecolor{currentstroke}{rgb}{0.000000,0.000000,0.000000}%
\pgfsetstrokecolor{currentstroke}%
\pgfsetdash{}{0pt}%
\pgfsys@defobject{currentmarker}{\pgfqpoint{0.000000in}{-0.027778in}}{\pgfqpoint{0.000000in}{0.000000in}}{%
\pgfpathmoveto{\pgfqpoint{0.000000in}{0.000000in}}%
\pgfpathlineto{\pgfqpoint{0.000000in}{-0.027778in}}%
\pgfusepath{stroke,fill}%
}%
\begin{pgfscope}%
\pgfsys@transformshift{3.432183in}{0.524170in}%
\pgfsys@useobject{currentmarker}{}%
\end{pgfscope}%
\end{pgfscope}%
\begin{pgfscope}%
\pgfsetbuttcap%
\pgfsetroundjoin%
\definecolor{currentfill}{rgb}{0.000000,0.000000,0.000000}%
\pgfsetfillcolor{currentfill}%
\pgfsetlinewidth{0.602250pt}%
\definecolor{currentstroke}{rgb}{0.000000,0.000000,0.000000}%
\pgfsetstrokecolor{currentstroke}%
\pgfsetdash{}{0pt}%
\pgfsys@defobject{currentmarker}{\pgfqpoint{0.000000in}{-0.027778in}}{\pgfqpoint{0.000000in}{0.000000in}}{%
\pgfpathmoveto{\pgfqpoint{0.000000in}{0.000000in}}%
\pgfpathlineto{\pgfqpoint{0.000000in}{-0.027778in}}%
\pgfusepath{stroke,fill}%
}%
\begin{pgfscope}%
\pgfsys@transformshift{3.490490in}{0.524170in}%
\pgfsys@useobject{currentmarker}{}%
\end{pgfscope}%
\end{pgfscope}%
\begin{pgfscope}%
\pgfsetbuttcap%
\pgfsetroundjoin%
\definecolor{currentfill}{rgb}{0.000000,0.000000,0.000000}%
\pgfsetfillcolor{currentfill}%
\pgfsetlinewidth{0.602250pt}%
\definecolor{currentstroke}{rgb}{0.000000,0.000000,0.000000}%
\pgfsetstrokecolor{currentstroke}%
\pgfsetdash{}{0pt}%
\pgfsys@defobject{currentmarker}{\pgfqpoint{0.000000in}{-0.027778in}}{\pgfqpoint{0.000000in}{0.000000in}}{%
\pgfpathmoveto{\pgfqpoint{0.000000in}{0.000000in}}%
\pgfpathlineto{\pgfqpoint{0.000000in}{-0.027778in}}%
\pgfusepath{stroke,fill}%
}%
\begin{pgfscope}%
\pgfsys@transformshift{3.535716in}{0.524170in}%
\pgfsys@useobject{currentmarker}{}%
\end{pgfscope}%
\end{pgfscope}%
\begin{pgfscope}%
\pgfsetbuttcap%
\pgfsetroundjoin%
\definecolor{currentfill}{rgb}{0.000000,0.000000,0.000000}%
\pgfsetfillcolor{currentfill}%
\pgfsetlinewidth{0.602250pt}%
\definecolor{currentstroke}{rgb}{0.000000,0.000000,0.000000}%
\pgfsetstrokecolor{currentstroke}%
\pgfsetdash{}{0pt}%
\pgfsys@defobject{currentmarker}{\pgfqpoint{0.000000in}{-0.027778in}}{\pgfqpoint{0.000000in}{0.000000in}}{%
\pgfpathmoveto{\pgfqpoint{0.000000in}{0.000000in}}%
\pgfpathlineto{\pgfqpoint{0.000000in}{-0.027778in}}%
\pgfusepath{stroke,fill}%
}%
\begin{pgfscope}%
\pgfsys@transformshift{3.572668in}{0.524170in}%
\pgfsys@useobject{currentmarker}{}%
\end{pgfscope}%
\end{pgfscope}%
\begin{pgfscope}%
\pgfsetbuttcap%
\pgfsetroundjoin%
\definecolor{currentfill}{rgb}{0.000000,0.000000,0.000000}%
\pgfsetfillcolor{currentfill}%
\pgfsetlinewidth{0.602250pt}%
\definecolor{currentstroke}{rgb}{0.000000,0.000000,0.000000}%
\pgfsetstrokecolor{currentstroke}%
\pgfsetdash{}{0pt}%
\pgfsys@defobject{currentmarker}{\pgfqpoint{0.000000in}{-0.027778in}}{\pgfqpoint{0.000000in}{0.000000in}}{%
\pgfpathmoveto{\pgfqpoint{0.000000in}{0.000000in}}%
\pgfpathlineto{\pgfqpoint{0.000000in}{-0.027778in}}%
\pgfusepath{stroke,fill}%
}%
\begin{pgfscope}%
\pgfsys@transformshift{3.603911in}{0.524170in}%
\pgfsys@useobject{currentmarker}{}%
\end{pgfscope}%
\end{pgfscope}%
\begin{pgfscope}%
\pgfsetbuttcap%
\pgfsetroundjoin%
\definecolor{currentfill}{rgb}{0.000000,0.000000,0.000000}%
\pgfsetfillcolor{currentfill}%
\pgfsetlinewidth{0.602250pt}%
\definecolor{currentstroke}{rgb}{0.000000,0.000000,0.000000}%
\pgfsetstrokecolor{currentstroke}%
\pgfsetdash{}{0pt}%
\pgfsys@defobject{currentmarker}{\pgfqpoint{0.000000in}{-0.027778in}}{\pgfqpoint{0.000000in}{0.000000in}}{%
\pgfpathmoveto{\pgfqpoint{0.000000in}{0.000000in}}%
\pgfpathlineto{\pgfqpoint{0.000000in}{-0.027778in}}%
\pgfusepath{stroke,fill}%
}%
\begin{pgfscope}%
\pgfsys@transformshift{3.630975in}{0.524170in}%
\pgfsys@useobject{currentmarker}{}%
\end{pgfscope}%
\end{pgfscope}%
\begin{pgfscope}%
\pgfsetbuttcap%
\pgfsetroundjoin%
\definecolor{currentfill}{rgb}{0.000000,0.000000,0.000000}%
\pgfsetfillcolor{currentfill}%
\pgfsetlinewidth{0.602250pt}%
\definecolor{currentstroke}{rgb}{0.000000,0.000000,0.000000}%
\pgfsetstrokecolor{currentstroke}%
\pgfsetdash{}{0pt}%
\pgfsys@defobject{currentmarker}{\pgfqpoint{0.000000in}{-0.027778in}}{\pgfqpoint{0.000000in}{0.000000in}}{%
\pgfpathmoveto{\pgfqpoint{0.000000in}{0.000000in}}%
\pgfpathlineto{\pgfqpoint{0.000000in}{-0.027778in}}%
\pgfusepath{stroke,fill}%
}%
\begin{pgfscope}%
\pgfsys@transformshift{3.654846in}{0.524170in}%
\pgfsys@useobject{currentmarker}{}%
\end{pgfscope}%
\end{pgfscope}%
\begin{pgfscope}%
\pgfsetbuttcap%
\pgfsetroundjoin%
\definecolor{currentfill}{rgb}{0.000000,0.000000,0.000000}%
\pgfsetfillcolor{currentfill}%
\pgfsetlinewidth{0.602250pt}%
\definecolor{currentstroke}{rgb}{0.000000,0.000000,0.000000}%
\pgfsetstrokecolor{currentstroke}%
\pgfsetdash{}{0pt}%
\pgfsys@defobject{currentmarker}{\pgfqpoint{0.000000in}{-0.027778in}}{\pgfqpoint{0.000000in}{0.000000in}}{%
\pgfpathmoveto{\pgfqpoint{0.000000in}{0.000000in}}%
\pgfpathlineto{\pgfqpoint{0.000000in}{-0.027778in}}%
\pgfusepath{stroke,fill}%
}%
\begin{pgfscope}%
\pgfsys@transformshift{3.816685in}{0.524170in}%
\pgfsys@useobject{currentmarker}{}%
\end{pgfscope}%
\end{pgfscope}%
\begin{pgfscope}%
\pgfsetbuttcap%
\pgfsetroundjoin%
\definecolor{currentfill}{rgb}{0.000000,0.000000,0.000000}%
\pgfsetfillcolor{currentfill}%
\pgfsetlinewidth{0.602250pt}%
\definecolor{currentstroke}{rgb}{0.000000,0.000000,0.000000}%
\pgfsetstrokecolor{currentstroke}%
\pgfsetdash{}{0pt}%
\pgfsys@defobject{currentmarker}{\pgfqpoint{0.000000in}{-0.027778in}}{\pgfqpoint{0.000000in}{0.000000in}}{%
\pgfpathmoveto{\pgfqpoint{0.000000in}{0.000000in}}%
\pgfpathlineto{\pgfqpoint{0.000000in}{-0.027778in}}%
\pgfusepath{stroke,fill}%
}%
\begin{pgfscope}%
\pgfsys@transformshift{3.898864in}{0.524170in}%
\pgfsys@useobject{currentmarker}{}%
\end{pgfscope}%
\end{pgfscope}%
\begin{pgfscope}%
\pgfsetbuttcap%
\pgfsetroundjoin%
\definecolor{currentfill}{rgb}{0.000000,0.000000,0.000000}%
\pgfsetfillcolor{currentfill}%
\pgfsetlinewidth{0.602250pt}%
\definecolor{currentstroke}{rgb}{0.000000,0.000000,0.000000}%
\pgfsetstrokecolor{currentstroke}%
\pgfsetdash{}{0pt}%
\pgfsys@defobject{currentmarker}{\pgfqpoint{0.000000in}{-0.027778in}}{\pgfqpoint{0.000000in}{0.000000in}}{%
\pgfpathmoveto{\pgfqpoint{0.000000in}{0.000000in}}%
\pgfpathlineto{\pgfqpoint{0.000000in}{-0.027778in}}%
\pgfusepath{stroke,fill}%
}%
\begin{pgfscope}%
\pgfsys@transformshift{3.957170in}{0.524170in}%
\pgfsys@useobject{currentmarker}{}%
\end{pgfscope}%
\end{pgfscope}%
\begin{pgfscope}%
\pgfsetbuttcap%
\pgfsetroundjoin%
\definecolor{currentfill}{rgb}{0.000000,0.000000,0.000000}%
\pgfsetfillcolor{currentfill}%
\pgfsetlinewidth{0.602250pt}%
\definecolor{currentstroke}{rgb}{0.000000,0.000000,0.000000}%
\pgfsetstrokecolor{currentstroke}%
\pgfsetdash{}{0pt}%
\pgfsys@defobject{currentmarker}{\pgfqpoint{0.000000in}{-0.027778in}}{\pgfqpoint{0.000000in}{0.000000in}}{%
\pgfpathmoveto{\pgfqpoint{0.000000in}{0.000000in}}%
\pgfpathlineto{\pgfqpoint{0.000000in}{-0.027778in}}%
\pgfusepath{stroke,fill}%
}%
\begin{pgfscope}%
\pgfsys@transformshift{4.002396in}{0.524170in}%
\pgfsys@useobject{currentmarker}{}%
\end{pgfscope}%
\end{pgfscope}%
\begin{pgfscope}%
\pgfsetbuttcap%
\pgfsetroundjoin%
\definecolor{currentfill}{rgb}{0.000000,0.000000,0.000000}%
\pgfsetfillcolor{currentfill}%
\pgfsetlinewidth{0.602250pt}%
\definecolor{currentstroke}{rgb}{0.000000,0.000000,0.000000}%
\pgfsetstrokecolor{currentstroke}%
\pgfsetdash{}{0pt}%
\pgfsys@defobject{currentmarker}{\pgfqpoint{0.000000in}{-0.027778in}}{\pgfqpoint{0.000000in}{0.000000in}}{%
\pgfpathmoveto{\pgfqpoint{0.000000in}{0.000000in}}%
\pgfpathlineto{\pgfqpoint{0.000000in}{-0.027778in}}%
\pgfusepath{stroke,fill}%
}%
\begin{pgfscope}%
\pgfsys@transformshift{4.039349in}{0.524170in}%
\pgfsys@useobject{currentmarker}{}%
\end{pgfscope}%
\end{pgfscope}%
\begin{pgfscope}%
\pgfsetbuttcap%
\pgfsetroundjoin%
\definecolor{currentfill}{rgb}{0.000000,0.000000,0.000000}%
\pgfsetfillcolor{currentfill}%
\pgfsetlinewidth{0.602250pt}%
\definecolor{currentstroke}{rgb}{0.000000,0.000000,0.000000}%
\pgfsetstrokecolor{currentstroke}%
\pgfsetdash{}{0pt}%
\pgfsys@defobject{currentmarker}{\pgfqpoint{0.000000in}{-0.027778in}}{\pgfqpoint{0.000000in}{0.000000in}}{%
\pgfpathmoveto{\pgfqpoint{0.000000in}{0.000000in}}%
\pgfpathlineto{\pgfqpoint{0.000000in}{-0.027778in}}%
\pgfusepath{stroke,fill}%
}%
\begin{pgfscope}%
\pgfsys@transformshift{4.070591in}{0.524170in}%
\pgfsys@useobject{currentmarker}{}%
\end{pgfscope}%
\end{pgfscope}%
\begin{pgfscope}%
\pgfsetbuttcap%
\pgfsetroundjoin%
\definecolor{currentfill}{rgb}{0.000000,0.000000,0.000000}%
\pgfsetfillcolor{currentfill}%
\pgfsetlinewidth{0.602250pt}%
\definecolor{currentstroke}{rgb}{0.000000,0.000000,0.000000}%
\pgfsetstrokecolor{currentstroke}%
\pgfsetdash{}{0pt}%
\pgfsys@defobject{currentmarker}{\pgfqpoint{0.000000in}{-0.027778in}}{\pgfqpoint{0.000000in}{0.000000in}}{%
\pgfpathmoveto{\pgfqpoint{0.000000in}{0.000000in}}%
\pgfpathlineto{\pgfqpoint{0.000000in}{-0.027778in}}%
\pgfusepath{stroke,fill}%
}%
\begin{pgfscope}%
\pgfsys@transformshift{4.097655in}{0.524170in}%
\pgfsys@useobject{currentmarker}{}%
\end{pgfscope}%
\end{pgfscope}%
\begin{pgfscope}%
\pgfsetbuttcap%
\pgfsetroundjoin%
\definecolor{currentfill}{rgb}{0.000000,0.000000,0.000000}%
\pgfsetfillcolor{currentfill}%
\pgfsetlinewidth{0.602250pt}%
\definecolor{currentstroke}{rgb}{0.000000,0.000000,0.000000}%
\pgfsetstrokecolor{currentstroke}%
\pgfsetdash{}{0pt}%
\pgfsys@defobject{currentmarker}{\pgfqpoint{0.000000in}{-0.027778in}}{\pgfqpoint{0.000000in}{0.000000in}}{%
\pgfpathmoveto{\pgfqpoint{0.000000in}{0.000000in}}%
\pgfpathlineto{\pgfqpoint{0.000000in}{-0.027778in}}%
\pgfusepath{stroke,fill}%
}%
\begin{pgfscope}%
\pgfsys@transformshift{4.121527in}{0.524170in}%
\pgfsys@useobject{currentmarker}{}%
\end{pgfscope}%
\end{pgfscope}%
\begin{pgfscope}%
\pgfsetbuttcap%
\pgfsetroundjoin%
\definecolor{currentfill}{rgb}{0.000000,0.000000,0.000000}%
\pgfsetfillcolor{currentfill}%
\pgfsetlinewidth{0.602250pt}%
\definecolor{currentstroke}{rgb}{0.000000,0.000000,0.000000}%
\pgfsetstrokecolor{currentstroke}%
\pgfsetdash{}{0pt}%
\pgfsys@defobject{currentmarker}{\pgfqpoint{0.000000in}{-0.027778in}}{\pgfqpoint{0.000000in}{0.000000in}}{%
\pgfpathmoveto{\pgfqpoint{0.000000in}{0.000000in}}%
\pgfpathlineto{\pgfqpoint{0.000000in}{-0.027778in}}%
\pgfusepath{stroke,fill}%
}%
\begin{pgfscope}%
\pgfsys@transformshift{4.283366in}{0.524170in}%
\pgfsys@useobject{currentmarker}{}%
\end{pgfscope}%
\end{pgfscope}%
\begin{pgfscope}%
\pgfsetbuttcap%
\pgfsetroundjoin%
\definecolor{currentfill}{rgb}{0.000000,0.000000,0.000000}%
\pgfsetfillcolor{currentfill}%
\pgfsetlinewidth{0.602250pt}%
\definecolor{currentstroke}{rgb}{0.000000,0.000000,0.000000}%
\pgfsetstrokecolor{currentstroke}%
\pgfsetdash{}{0pt}%
\pgfsys@defobject{currentmarker}{\pgfqpoint{0.000000in}{-0.027778in}}{\pgfqpoint{0.000000in}{0.000000in}}{%
\pgfpathmoveto{\pgfqpoint{0.000000in}{0.000000in}}%
\pgfpathlineto{\pgfqpoint{0.000000in}{-0.027778in}}%
\pgfusepath{stroke,fill}%
}%
\begin{pgfscope}%
\pgfsys@transformshift{4.365544in}{0.524170in}%
\pgfsys@useobject{currentmarker}{}%
\end{pgfscope}%
\end{pgfscope}%
\begin{pgfscope}%
\pgfsetbuttcap%
\pgfsetroundjoin%
\definecolor{currentfill}{rgb}{0.000000,0.000000,0.000000}%
\pgfsetfillcolor{currentfill}%
\pgfsetlinewidth{0.602250pt}%
\definecolor{currentstroke}{rgb}{0.000000,0.000000,0.000000}%
\pgfsetstrokecolor{currentstroke}%
\pgfsetdash{}{0pt}%
\pgfsys@defobject{currentmarker}{\pgfqpoint{0.000000in}{-0.027778in}}{\pgfqpoint{0.000000in}{0.000000in}}{%
\pgfpathmoveto{\pgfqpoint{0.000000in}{0.000000in}}%
\pgfpathlineto{\pgfqpoint{0.000000in}{-0.027778in}}%
\pgfusepath{stroke,fill}%
}%
\begin{pgfscope}%
\pgfsys@transformshift{4.423851in}{0.524170in}%
\pgfsys@useobject{currentmarker}{}%
\end{pgfscope}%
\end{pgfscope}%
\begin{pgfscope}%
\pgfsetbuttcap%
\pgfsetroundjoin%
\definecolor{currentfill}{rgb}{0.000000,0.000000,0.000000}%
\pgfsetfillcolor{currentfill}%
\pgfsetlinewidth{0.602250pt}%
\definecolor{currentstroke}{rgb}{0.000000,0.000000,0.000000}%
\pgfsetstrokecolor{currentstroke}%
\pgfsetdash{}{0pt}%
\pgfsys@defobject{currentmarker}{\pgfqpoint{0.000000in}{-0.027778in}}{\pgfqpoint{0.000000in}{0.000000in}}{%
\pgfpathmoveto{\pgfqpoint{0.000000in}{0.000000in}}%
\pgfpathlineto{\pgfqpoint{0.000000in}{-0.027778in}}%
\pgfusepath{stroke,fill}%
}%
\begin{pgfscope}%
\pgfsys@transformshift{4.469077in}{0.524170in}%
\pgfsys@useobject{currentmarker}{}%
\end{pgfscope}%
\end{pgfscope}%
\begin{pgfscope}%
\pgfsetbuttcap%
\pgfsetroundjoin%
\definecolor{currentfill}{rgb}{0.000000,0.000000,0.000000}%
\pgfsetfillcolor{currentfill}%
\pgfsetlinewidth{0.602250pt}%
\definecolor{currentstroke}{rgb}{0.000000,0.000000,0.000000}%
\pgfsetstrokecolor{currentstroke}%
\pgfsetdash{}{0pt}%
\pgfsys@defobject{currentmarker}{\pgfqpoint{0.000000in}{-0.027778in}}{\pgfqpoint{0.000000in}{0.000000in}}{%
\pgfpathmoveto{\pgfqpoint{0.000000in}{0.000000in}}%
\pgfpathlineto{\pgfqpoint{0.000000in}{-0.027778in}}%
\pgfusepath{stroke,fill}%
}%
\begin{pgfscope}%
\pgfsys@transformshift{4.506029in}{0.524170in}%
\pgfsys@useobject{currentmarker}{}%
\end{pgfscope}%
\end{pgfscope}%
\begin{pgfscope}%
\pgfsetbuttcap%
\pgfsetroundjoin%
\definecolor{currentfill}{rgb}{0.000000,0.000000,0.000000}%
\pgfsetfillcolor{currentfill}%
\pgfsetlinewidth{0.602250pt}%
\definecolor{currentstroke}{rgb}{0.000000,0.000000,0.000000}%
\pgfsetstrokecolor{currentstroke}%
\pgfsetdash{}{0pt}%
\pgfsys@defobject{currentmarker}{\pgfqpoint{0.000000in}{-0.027778in}}{\pgfqpoint{0.000000in}{0.000000in}}{%
\pgfpathmoveto{\pgfqpoint{0.000000in}{0.000000in}}%
\pgfpathlineto{\pgfqpoint{0.000000in}{-0.027778in}}%
\pgfusepath{stroke,fill}%
}%
\begin{pgfscope}%
\pgfsys@transformshift{4.537272in}{0.524170in}%
\pgfsys@useobject{currentmarker}{}%
\end{pgfscope}%
\end{pgfscope}%
\begin{pgfscope}%
\pgfsetbuttcap%
\pgfsetroundjoin%
\definecolor{currentfill}{rgb}{0.000000,0.000000,0.000000}%
\pgfsetfillcolor{currentfill}%
\pgfsetlinewidth{0.602250pt}%
\definecolor{currentstroke}{rgb}{0.000000,0.000000,0.000000}%
\pgfsetstrokecolor{currentstroke}%
\pgfsetdash{}{0pt}%
\pgfsys@defobject{currentmarker}{\pgfqpoint{0.000000in}{-0.027778in}}{\pgfqpoint{0.000000in}{0.000000in}}{%
\pgfpathmoveto{\pgfqpoint{0.000000in}{0.000000in}}%
\pgfpathlineto{\pgfqpoint{0.000000in}{-0.027778in}}%
\pgfusepath{stroke,fill}%
}%
\begin{pgfscope}%
\pgfsys@transformshift{4.564336in}{0.524170in}%
\pgfsys@useobject{currentmarker}{}%
\end{pgfscope}%
\end{pgfscope}%
\begin{pgfscope}%
\pgfsetbuttcap%
\pgfsetroundjoin%
\definecolor{currentfill}{rgb}{0.000000,0.000000,0.000000}%
\pgfsetfillcolor{currentfill}%
\pgfsetlinewidth{0.602250pt}%
\definecolor{currentstroke}{rgb}{0.000000,0.000000,0.000000}%
\pgfsetstrokecolor{currentstroke}%
\pgfsetdash{}{0pt}%
\pgfsys@defobject{currentmarker}{\pgfqpoint{0.000000in}{-0.027778in}}{\pgfqpoint{0.000000in}{0.000000in}}{%
\pgfpathmoveto{\pgfqpoint{0.000000in}{0.000000in}}%
\pgfpathlineto{\pgfqpoint{0.000000in}{-0.027778in}}%
\pgfusepath{stroke,fill}%
}%
\begin{pgfscope}%
\pgfsys@transformshift{4.588207in}{0.524170in}%
\pgfsys@useobject{currentmarker}{}%
\end{pgfscope}%
\end{pgfscope}%
\begin{pgfscope}%
\pgfsetbuttcap%
\pgfsetroundjoin%
\definecolor{currentfill}{rgb}{0.000000,0.000000,0.000000}%
\pgfsetfillcolor{currentfill}%
\pgfsetlinewidth{0.602250pt}%
\definecolor{currentstroke}{rgb}{0.000000,0.000000,0.000000}%
\pgfsetstrokecolor{currentstroke}%
\pgfsetdash{}{0pt}%
\pgfsys@defobject{currentmarker}{\pgfqpoint{0.000000in}{-0.027778in}}{\pgfqpoint{0.000000in}{0.000000in}}{%
\pgfpathmoveto{\pgfqpoint{0.000000in}{0.000000in}}%
\pgfpathlineto{\pgfqpoint{0.000000in}{-0.027778in}}%
\pgfusepath{stroke,fill}%
}%
\begin{pgfscope}%
\pgfsys@transformshift{4.750046in}{0.524170in}%
\pgfsys@useobject{currentmarker}{}%
\end{pgfscope}%
\end{pgfscope}%
\begin{pgfscope}%
\definecolor{textcolor}{rgb}{0.000000,0.000000,0.000000}%
\pgfsetstrokecolor{textcolor}%
\pgfsetfillcolor{textcolor}%
\pgftext[x=2.742840in,y=0.271531in,,top]{\color{textcolor}{\rmfamily\fontsize{10.000000}{12.000000}\selectfont\catcode`\^=\active\def^{\ifmmode\sp\else\^{}\fi}\catcode`\%=\active\def%{\%}Frequency in \unit{\Hz}}}%
\end{pgfscope}%
\begin{pgfscope}%
\pgfpathrectangle{\pgfqpoint{0.689445in}{0.524170in}}{\pgfqpoint{4.106789in}{2.682436in}}%
\pgfusepath{clip}%
\pgfsetrectcap%
\pgfsetroundjoin%
\pgfsetlinewidth{0.803000pt}%
\definecolor{currentstroke}{rgb}{0.450000,0.450000,0.450000}%
\pgfsetstrokecolor{currentstroke}%
\pgfsetdash{}{0pt}%
\pgfpathmoveto{\pgfqpoint{0.689445in}{0.744168in}}%
\pgfpathlineto{\pgfqpoint{4.796234in}{0.744168in}}%
\pgfusepath{stroke}%
\end{pgfscope}%
\begin{pgfscope}%
\pgfsetbuttcap%
\pgfsetroundjoin%
\definecolor{currentfill}{rgb}{0.000000,0.000000,0.000000}%
\pgfsetfillcolor{currentfill}%
\pgfsetlinewidth{0.803000pt}%
\definecolor{currentstroke}{rgb}{0.000000,0.000000,0.000000}%
\pgfsetstrokecolor{currentstroke}%
\pgfsetdash{}{0pt}%
\pgfsys@defobject{currentmarker}{\pgfqpoint{-0.048611in}{0.000000in}}{\pgfqpoint{-0.000000in}{0.000000in}}{%
\pgfpathmoveto{\pgfqpoint{-0.000000in}{0.000000in}}%
\pgfpathlineto{\pgfqpoint{-0.048611in}{0.000000in}}%
\pgfusepath{stroke,fill}%
}%
\begin{pgfscope}%
\pgfsys@transformshift{0.689445in}{0.744168in}%
\pgfsys@useobject{currentmarker}{}%
\end{pgfscope}%
\end{pgfscope}%
\begin{pgfscope}%
\definecolor{textcolor}{rgb}{0.000000,0.000000,0.000000}%
\pgfsetstrokecolor{textcolor}%
\pgfsetfillcolor{textcolor}%
\pgftext[x=0.336050in, y=0.705016in, left, base]{\color{textcolor}{\rmfamily\fontsize{8.000000}{9.600000}\selectfont\catcode`\^=\active\def^{\ifmmode\sp\else\^{}\fi}\catcode`\%=\active\def%{\%}$\mathdefault{10^{-8}}$}}%
\end{pgfscope}%
\begin{pgfscope}%
\pgfpathrectangle{\pgfqpoint{0.689445in}{0.524170in}}{\pgfqpoint{4.106789in}{2.682436in}}%
\pgfusepath{clip}%
\pgfsetrectcap%
\pgfsetroundjoin%
\pgfsetlinewidth{0.803000pt}%
\definecolor{currentstroke}{rgb}{0.450000,0.450000,0.450000}%
\pgfsetstrokecolor{currentstroke}%
\pgfsetdash{}{0pt}%
\pgfpathmoveto{\pgfqpoint{0.689445in}{1.037018in}}%
\pgfpathlineto{\pgfqpoint{4.796234in}{1.037018in}}%
\pgfusepath{stroke}%
\end{pgfscope}%
\begin{pgfscope}%
\pgfsetbuttcap%
\pgfsetroundjoin%
\definecolor{currentfill}{rgb}{0.000000,0.000000,0.000000}%
\pgfsetfillcolor{currentfill}%
\pgfsetlinewidth{0.803000pt}%
\definecolor{currentstroke}{rgb}{0.000000,0.000000,0.000000}%
\pgfsetstrokecolor{currentstroke}%
\pgfsetdash{}{0pt}%
\pgfsys@defobject{currentmarker}{\pgfqpoint{-0.048611in}{0.000000in}}{\pgfqpoint{-0.000000in}{0.000000in}}{%
\pgfpathmoveto{\pgfqpoint{-0.000000in}{0.000000in}}%
\pgfpathlineto{\pgfqpoint{-0.048611in}{0.000000in}}%
\pgfusepath{stroke,fill}%
}%
\begin{pgfscope}%
\pgfsys@transformshift{0.689445in}{1.037018in}%
\pgfsys@useobject{currentmarker}{}%
\end{pgfscope}%
\end{pgfscope}%
\begin{pgfscope}%
\definecolor{textcolor}{rgb}{0.000000,0.000000,0.000000}%
\pgfsetstrokecolor{textcolor}%
\pgfsetfillcolor{textcolor}%
\pgftext[x=0.336050in, y=0.997865in, left, base]{\color{textcolor}{\rmfamily\fontsize{8.000000}{9.600000}\selectfont\catcode`\^=\active\def^{\ifmmode\sp\else\^{}\fi}\catcode`\%=\active\def%{\%}$\mathdefault{10^{-7}}$}}%
\end{pgfscope}%
\begin{pgfscope}%
\pgfpathrectangle{\pgfqpoint{0.689445in}{0.524170in}}{\pgfqpoint{4.106789in}{2.682436in}}%
\pgfusepath{clip}%
\pgfsetrectcap%
\pgfsetroundjoin%
\pgfsetlinewidth{0.803000pt}%
\definecolor{currentstroke}{rgb}{0.450000,0.450000,0.450000}%
\pgfsetstrokecolor{currentstroke}%
\pgfsetdash{}{0pt}%
\pgfpathmoveto{\pgfqpoint{0.689445in}{1.329868in}}%
\pgfpathlineto{\pgfqpoint{4.796234in}{1.329868in}}%
\pgfusepath{stroke}%
\end{pgfscope}%
\begin{pgfscope}%
\pgfsetbuttcap%
\pgfsetroundjoin%
\definecolor{currentfill}{rgb}{0.000000,0.000000,0.000000}%
\pgfsetfillcolor{currentfill}%
\pgfsetlinewidth{0.803000pt}%
\definecolor{currentstroke}{rgb}{0.000000,0.000000,0.000000}%
\pgfsetstrokecolor{currentstroke}%
\pgfsetdash{}{0pt}%
\pgfsys@defobject{currentmarker}{\pgfqpoint{-0.048611in}{0.000000in}}{\pgfqpoint{-0.000000in}{0.000000in}}{%
\pgfpathmoveto{\pgfqpoint{-0.000000in}{0.000000in}}%
\pgfpathlineto{\pgfqpoint{-0.048611in}{0.000000in}}%
\pgfusepath{stroke,fill}%
}%
\begin{pgfscope}%
\pgfsys@transformshift{0.689445in}{1.329868in}%
\pgfsys@useobject{currentmarker}{}%
\end{pgfscope}%
\end{pgfscope}%
\begin{pgfscope}%
\definecolor{textcolor}{rgb}{0.000000,0.000000,0.000000}%
\pgfsetstrokecolor{textcolor}%
\pgfsetfillcolor{textcolor}%
\pgftext[x=0.336050in, y=1.290715in, left, base]{\color{textcolor}{\rmfamily\fontsize{8.000000}{9.600000}\selectfont\catcode`\^=\active\def^{\ifmmode\sp\else\^{}\fi}\catcode`\%=\active\def%{\%}$\mathdefault{10^{-6}}$}}%
\end{pgfscope}%
\begin{pgfscope}%
\pgfpathrectangle{\pgfqpoint{0.689445in}{0.524170in}}{\pgfqpoint{4.106789in}{2.682436in}}%
\pgfusepath{clip}%
\pgfsetrectcap%
\pgfsetroundjoin%
\pgfsetlinewidth{0.803000pt}%
\definecolor{currentstroke}{rgb}{0.450000,0.450000,0.450000}%
\pgfsetstrokecolor{currentstroke}%
\pgfsetdash{}{0pt}%
\pgfpathmoveto{\pgfqpoint{0.689445in}{1.622717in}}%
\pgfpathlineto{\pgfqpoint{4.796234in}{1.622717in}}%
\pgfusepath{stroke}%
\end{pgfscope}%
\begin{pgfscope}%
\pgfsetbuttcap%
\pgfsetroundjoin%
\definecolor{currentfill}{rgb}{0.000000,0.000000,0.000000}%
\pgfsetfillcolor{currentfill}%
\pgfsetlinewidth{0.803000pt}%
\definecolor{currentstroke}{rgb}{0.000000,0.000000,0.000000}%
\pgfsetstrokecolor{currentstroke}%
\pgfsetdash{}{0pt}%
\pgfsys@defobject{currentmarker}{\pgfqpoint{-0.048611in}{0.000000in}}{\pgfqpoint{-0.000000in}{0.000000in}}{%
\pgfpathmoveto{\pgfqpoint{-0.000000in}{0.000000in}}%
\pgfpathlineto{\pgfqpoint{-0.048611in}{0.000000in}}%
\pgfusepath{stroke,fill}%
}%
\begin{pgfscope}%
\pgfsys@transformshift{0.689445in}{1.622717in}%
\pgfsys@useobject{currentmarker}{}%
\end{pgfscope}%
\end{pgfscope}%
\begin{pgfscope}%
\definecolor{textcolor}{rgb}{0.000000,0.000000,0.000000}%
\pgfsetstrokecolor{textcolor}%
\pgfsetfillcolor{textcolor}%
\pgftext[x=0.336050in, y=1.583565in, left, base]{\color{textcolor}{\rmfamily\fontsize{8.000000}{9.600000}\selectfont\catcode`\^=\active\def^{\ifmmode\sp\else\^{}\fi}\catcode`\%=\active\def%{\%}$\mathdefault{10^{-5}}$}}%
\end{pgfscope}%
\begin{pgfscope}%
\pgfpathrectangle{\pgfqpoint{0.689445in}{0.524170in}}{\pgfqpoint{4.106789in}{2.682436in}}%
\pgfusepath{clip}%
\pgfsetrectcap%
\pgfsetroundjoin%
\pgfsetlinewidth{0.803000pt}%
\definecolor{currentstroke}{rgb}{0.450000,0.450000,0.450000}%
\pgfsetstrokecolor{currentstroke}%
\pgfsetdash{}{0pt}%
\pgfpathmoveto{\pgfqpoint{0.689445in}{1.915567in}}%
\pgfpathlineto{\pgfqpoint{4.796234in}{1.915567in}}%
\pgfusepath{stroke}%
\end{pgfscope}%
\begin{pgfscope}%
\pgfsetbuttcap%
\pgfsetroundjoin%
\definecolor{currentfill}{rgb}{0.000000,0.000000,0.000000}%
\pgfsetfillcolor{currentfill}%
\pgfsetlinewidth{0.803000pt}%
\definecolor{currentstroke}{rgb}{0.000000,0.000000,0.000000}%
\pgfsetstrokecolor{currentstroke}%
\pgfsetdash{}{0pt}%
\pgfsys@defobject{currentmarker}{\pgfqpoint{-0.048611in}{0.000000in}}{\pgfqpoint{-0.000000in}{0.000000in}}{%
\pgfpathmoveto{\pgfqpoint{-0.000000in}{0.000000in}}%
\pgfpathlineto{\pgfqpoint{-0.048611in}{0.000000in}}%
\pgfusepath{stroke,fill}%
}%
\begin{pgfscope}%
\pgfsys@transformshift{0.689445in}{1.915567in}%
\pgfsys@useobject{currentmarker}{}%
\end{pgfscope}%
\end{pgfscope}%
\begin{pgfscope}%
\definecolor{textcolor}{rgb}{0.000000,0.000000,0.000000}%
\pgfsetstrokecolor{textcolor}%
\pgfsetfillcolor{textcolor}%
\pgftext[x=0.336050in, y=1.876414in, left, base]{\color{textcolor}{\rmfamily\fontsize{8.000000}{9.600000}\selectfont\catcode`\^=\active\def^{\ifmmode\sp\else\^{}\fi}\catcode`\%=\active\def%{\%}$\mathdefault{10^{-4}}$}}%
\end{pgfscope}%
\begin{pgfscope}%
\pgfpathrectangle{\pgfqpoint{0.689445in}{0.524170in}}{\pgfqpoint{4.106789in}{2.682436in}}%
\pgfusepath{clip}%
\pgfsetrectcap%
\pgfsetroundjoin%
\pgfsetlinewidth{0.803000pt}%
\definecolor{currentstroke}{rgb}{0.450000,0.450000,0.450000}%
\pgfsetstrokecolor{currentstroke}%
\pgfsetdash{}{0pt}%
\pgfpathmoveto{\pgfqpoint{0.689445in}{2.208417in}}%
\pgfpathlineto{\pgfqpoint{4.796234in}{2.208417in}}%
\pgfusepath{stroke}%
\end{pgfscope}%
\begin{pgfscope}%
\pgfsetbuttcap%
\pgfsetroundjoin%
\definecolor{currentfill}{rgb}{0.000000,0.000000,0.000000}%
\pgfsetfillcolor{currentfill}%
\pgfsetlinewidth{0.803000pt}%
\definecolor{currentstroke}{rgb}{0.000000,0.000000,0.000000}%
\pgfsetstrokecolor{currentstroke}%
\pgfsetdash{}{0pt}%
\pgfsys@defobject{currentmarker}{\pgfqpoint{-0.048611in}{0.000000in}}{\pgfqpoint{-0.000000in}{0.000000in}}{%
\pgfpathmoveto{\pgfqpoint{-0.000000in}{0.000000in}}%
\pgfpathlineto{\pgfqpoint{-0.048611in}{0.000000in}}%
\pgfusepath{stroke,fill}%
}%
\begin{pgfscope}%
\pgfsys@transformshift{0.689445in}{2.208417in}%
\pgfsys@useobject{currentmarker}{}%
\end{pgfscope}%
\end{pgfscope}%
\begin{pgfscope}%
\definecolor{textcolor}{rgb}{0.000000,0.000000,0.000000}%
\pgfsetstrokecolor{textcolor}%
\pgfsetfillcolor{textcolor}%
\pgftext[x=0.336050in, y=2.169264in, left, base]{\color{textcolor}{\rmfamily\fontsize{8.000000}{9.600000}\selectfont\catcode`\^=\active\def^{\ifmmode\sp\else\^{}\fi}\catcode`\%=\active\def%{\%}$\mathdefault{10^{-3}}$}}%
\end{pgfscope}%
\begin{pgfscope}%
\pgfpathrectangle{\pgfqpoint{0.689445in}{0.524170in}}{\pgfqpoint{4.106789in}{2.682436in}}%
\pgfusepath{clip}%
\pgfsetrectcap%
\pgfsetroundjoin%
\pgfsetlinewidth{0.803000pt}%
\definecolor{currentstroke}{rgb}{0.450000,0.450000,0.450000}%
\pgfsetstrokecolor{currentstroke}%
\pgfsetdash{}{0pt}%
\pgfpathmoveto{\pgfqpoint{0.689445in}{2.501267in}}%
\pgfpathlineto{\pgfqpoint{4.796234in}{2.501267in}}%
\pgfusepath{stroke}%
\end{pgfscope}%
\begin{pgfscope}%
\pgfsetbuttcap%
\pgfsetroundjoin%
\definecolor{currentfill}{rgb}{0.000000,0.000000,0.000000}%
\pgfsetfillcolor{currentfill}%
\pgfsetlinewidth{0.803000pt}%
\definecolor{currentstroke}{rgb}{0.000000,0.000000,0.000000}%
\pgfsetstrokecolor{currentstroke}%
\pgfsetdash{}{0pt}%
\pgfsys@defobject{currentmarker}{\pgfqpoint{-0.048611in}{0.000000in}}{\pgfqpoint{-0.000000in}{0.000000in}}{%
\pgfpathmoveto{\pgfqpoint{-0.000000in}{0.000000in}}%
\pgfpathlineto{\pgfqpoint{-0.048611in}{0.000000in}}%
\pgfusepath{stroke,fill}%
}%
\begin{pgfscope}%
\pgfsys@transformshift{0.689445in}{2.501267in}%
\pgfsys@useobject{currentmarker}{}%
\end{pgfscope}%
\end{pgfscope}%
\begin{pgfscope}%
\definecolor{textcolor}{rgb}{0.000000,0.000000,0.000000}%
\pgfsetstrokecolor{textcolor}%
\pgfsetfillcolor{textcolor}%
\pgftext[x=0.336050in, y=2.462114in, left, base]{\color{textcolor}{\rmfamily\fontsize{8.000000}{9.600000}\selectfont\catcode`\^=\active\def^{\ifmmode\sp\else\^{}\fi}\catcode`\%=\active\def%{\%}$\mathdefault{10^{-2}}$}}%
\end{pgfscope}%
\begin{pgfscope}%
\pgfpathrectangle{\pgfqpoint{0.689445in}{0.524170in}}{\pgfqpoint{4.106789in}{2.682436in}}%
\pgfusepath{clip}%
\pgfsetrectcap%
\pgfsetroundjoin%
\pgfsetlinewidth{0.803000pt}%
\definecolor{currentstroke}{rgb}{0.450000,0.450000,0.450000}%
\pgfsetstrokecolor{currentstroke}%
\pgfsetdash{}{0pt}%
\pgfpathmoveto{\pgfqpoint{0.689445in}{2.794116in}}%
\pgfpathlineto{\pgfqpoint{4.796234in}{2.794116in}}%
\pgfusepath{stroke}%
\end{pgfscope}%
\begin{pgfscope}%
\pgfsetbuttcap%
\pgfsetroundjoin%
\definecolor{currentfill}{rgb}{0.000000,0.000000,0.000000}%
\pgfsetfillcolor{currentfill}%
\pgfsetlinewidth{0.803000pt}%
\definecolor{currentstroke}{rgb}{0.000000,0.000000,0.000000}%
\pgfsetstrokecolor{currentstroke}%
\pgfsetdash{}{0pt}%
\pgfsys@defobject{currentmarker}{\pgfqpoint{-0.048611in}{0.000000in}}{\pgfqpoint{-0.000000in}{0.000000in}}{%
\pgfpathmoveto{\pgfqpoint{-0.000000in}{0.000000in}}%
\pgfpathlineto{\pgfqpoint{-0.048611in}{0.000000in}}%
\pgfusepath{stroke,fill}%
}%
\begin{pgfscope}%
\pgfsys@transformshift{0.689445in}{2.794116in}%
\pgfsys@useobject{currentmarker}{}%
\end{pgfscope}%
\end{pgfscope}%
\begin{pgfscope}%
\definecolor{textcolor}{rgb}{0.000000,0.000000,0.000000}%
\pgfsetstrokecolor{textcolor}%
\pgfsetfillcolor{textcolor}%
\pgftext[x=0.336050in, y=2.754964in, left, base]{\color{textcolor}{\rmfamily\fontsize{8.000000}{9.600000}\selectfont\catcode`\^=\active\def^{\ifmmode\sp\else\^{}\fi}\catcode`\%=\active\def%{\%}$\mathdefault{10^{-1}}$}}%
\end{pgfscope}%
\begin{pgfscope}%
\pgfpathrectangle{\pgfqpoint{0.689445in}{0.524170in}}{\pgfqpoint{4.106789in}{2.682436in}}%
\pgfusepath{clip}%
\pgfsetrectcap%
\pgfsetroundjoin%
\pgfsetlinewidth{0.803000pt}%
\definecolor{currentstroke}{rgb}{0.450000,0.450000,0.450000}%
\pgfsetstrokecolor{currentstroke}%
\pgfsetdash{}{0pt}%
\pgfpathmoveto{\pgfqpoint{0.689445in}{3.086966in}}%
\pgfpathlineto{\pgfqpoint{4.796234in}{3.086966in}}%
\pgfusepath{stroke}%
\end{pgfscope}%
\begin{pgfscope}%
\pgfsetbuttcap%
\pgfsetroundjoin%
\definecolor{currentfill}{rgb}{0.000000,0.000000,0.000000}%
\pgfsetfillcolor{currentfill}%
\pgfsetlinewidth{0.803000pt}%
\definecolor{currentstroke}{rgb}{0.000000,0.000000,0.000000}%
\pgfsetstrokecolor{currentstroke}%
\pgfsetdash{}{0pt}%
\pgfsys@defobject{currentmarker}{\pgfqpoint{-0.048611in}{0.000000in}}{\pgfqpoint{-0.000000in}{0.000000in}}{%
\pgfpathmoveto{\pgfqpoint{-0.000000in}{0.000000in}}%
\pgfpathlineto{\pgfqpoint{-0.048611in}{0.000000in}}%
\pgfusepath{stroke,fill}%
}%
\begin{pgfscope}%
\pgfsys@transformshift{0.689445in}{3.086966in}%
\pgfsys@useobject{currentmarker}{}%
\end{pgfscope}%
\end{pgfscope}%
\begin{pgfscope}%
\definecolor{textcolor}{rgb}{0.000000,0.000000,0.000000}%
\pgfsetstrokecolor{textcolor}%
\pgfsetfillcolor{textcolor}%
\pgftext[x=0.416296in, y=3.047813in, left, base]{\color{textcolor}{\rmfamily\fontsize{8.000000}{9.600000}\selectfont\catcode`\^=\active\def^{\ifmmode\sp\else\^{}\fi}\catcode`\%=\active\def%{\%}$\mathdefault{10^{0}}$}}%
\end{pgfscope}%
\begin{pgfscope}%
\pgfsetbuttcap%
\pgfsetroundjoin%
\definecolor{currentfill}{rgb}{0.000000,0.000000,0.000000}%
\pgfsetfillcolor{currentfill}%
\pgfsetlinewidth{0.602250pt}%
\definecolor{currentstroke}{rgb}{0.000000,0.000000,0.000000}%
\pgfsetstrokecolor{currentstroke}%
\pgfsetdash{}{0pt}%
\pgfsys@defobject{currentmarker}{\pgfqpoint{-0.027778in}{0.000000in}}{\pgfqpoint{-0.000000in}{0.000000in}}{%
\pgfpathmoveto{\pgfqpoint{-0.000000in}{0.000000in}}%
\pgfpathlineto{\pgfqpoint{-0.027778in}{0.000000in}}%
\pgfusepath{stroke,fill}%
}%
\begin{pgfscope}%
\pgfsys@transformshift{0.689445in}{0.539475in}%
\pgfsys@useobject{currentmarker}{}%
\end{pgfscope}%
\end{pgfscope}%
\begin{pgfscope}%
\pgfsetbuttcap%
\pgfsetroundjoin%
\definecolor{currentfill}{rgb}{0.000000,0.000000,0.000000}%
\pgfsetfillcolor{currentfill}%
\pgfsetlinewidth{0.602250pt}%
\definecolor{currentstroke}{rgb}{0.000000,0.000000,0.000000}%
\pgfsetstrokecolor{currentstroke}%
\pgfsetdash{}{0pt}%
\pgfsys@defobject{currentmarker}{\pgfqpoint{-0.027778in}{0.000000in}}{\pgfqpoint{-0.000000in}{0.000000in}}{%
\pgfpathmoveto{\pgfqpoint{-0.000000in}{0.000000in}}%
\pgfpathlineto{\pgfqpoint{-0.027778in}{0.000000in}}%
\pgfusepath{stroke,fill}%
}%
\begin{pgfscope}%
\pgfsys@transformshift{0.689445in}{0.591043in}%
\pgfsys@useobject{currentmarker}{}%
\end{pgfscope}%
\end{pgfscope}%
\begin{pgfscope}%
\pgfsetbuttcap%
\pgfsetroundjoin%
\definecolor{currentfill}{rgb}{0.000000,0.000000,0.000000}%
\pgfsetfillcolor{currentfill}%
\pgfsetlinewidth{0.602250pt}%
\definecolor{currentstroke}{rgb}{0.000000,0.000000,0.000000}%
\pgfsetstrokecolor{currentstroke}%
\pgfsetdash{}{0pt}%
\pgfsys@defobject{currentmarker}{\pgfqpoint{-0.027778in}{0.000000in}}{\pgfqpoint{-0.000000in}{0.000000in}}{%
\pgfpathmoveto{\pgfqpoint{-0.000000in}{0.000000in}}%
\pgfpathlineto{\pgfqpoint{-0.027778in}{0.000000in}}%
\pgfusepath{stroke,fill}%
}%
\begin{pgfscope}%
\pgfsys@transformshift{0.689445in}{0.627632in}%
\pgfsys@useobject{currentmarker}{}%
\end{pgfscope}%
\end{pgfscope}%
\begin{pgfscope}%
\pgfsetbuttcap%
\pgfsetroundjoin%
\definecolor{currentfill}{rgb}{0.000000,0.000000,0.000000}%
\pgfsetfillcolor{currentfill}%
\pgfsetlinewidth{0.602250pt}%
\definecolor{currentstroke}{rgb}{0.000000,0.000000,0.000000}%
\pgfsetstrokecolor{currentstroke}%
\pgfsetdash{}{0pt}%
\pgfsys@defobject{currentmarker}{\pgfqpoint{-0.027778in}{0.000000in}}{\pgfqpoint{-0.000000in}{0.000000in}}{%
\pgfpathmoveto{\pgfqpoint{-0.000000in}{0.000000in}}%
\pgfpathlineto{\pgfqpoint{-0.027778in}{0.000000in}}%
\pgfusepath{stroke,fill}%
}%
\begin{pgfscope}%
\pgfsys@transformshift{0.689445in}{0.656012in}%
\pgfsys@useobject{currentmarker}{}%
\end{pgfscope}%
\end{pgfscope}%
\begin{pgfscope}%
\pgfsetbuttcap%
\pgfsetroundjoin%
\definecolor{currentfill}{rgb}{0.000000,0.000000,0.000000}%
\pgfsetfillcolor{currentfill}%
\pgfsetlinewidth{0.602250pt}%
\definecolor{currentstroke}{rgb}{0.000000,0.000000,0.000000}%
\pgfsetstrokecolor{currentstroke}%
\pgfsetdash{}{0pt}%
\pgfsys@defobject{currentmarker}{\pgfqpoint{-0.027778in}{0.000000in}}{\pgfqpoint{-0.000000in}{0.000000in}}{%
\pgfpathmoveto{\pgfqpoint{-0.000000in}{0.000000in}}%
\pgfpathlineto{\pgfqpoint{-0.027778in}{0.000000in}}%
\pgfusepath{stroke,fill}%
}%
\begin{pgfscope}%
\pgfsys@transformshift{0.689445in}{0.679200in}%
\pgfsys@useobject{currentmarker}{}%
\end{pgfscope}%
\end{pgfscope}%
\begin{pgfscope}%
\pgfsetbuttcap%
\pgfsetroundjoin%
\definecolor{currentfill}{rgb}{0.000000,0.000000,0.000000}%
\pgfsetfillcolor{currentfill}%
\pgfsetlinewidth{0.602250pt}%
\definecolor{currentstroke}{rgb}{0.000000,0.000000,0.000000}%
\pgfsetstrokecolor{currentstroke}%
\pgfsetdash{}{0pt}%
\pgfsys@defobject{currentmarker}{\pgfqpoint{-0.027778in}{0.000000in}}{\pgfqpoint{-0.000000in}{0.000000in}}{%
\pgfpathmoveto{\pgfqpoint{-0.000000in}{0.000000in}}%
\pgfpathlineto{\pgfqpoint{-0.027778in}{0.000000in}}%
\pgfusepath{stroke,fill}%
}%
\begin{pgfscope}%
\pgfsys@transformshift{0.689445in}{0.698805in}%
\pgfsys@useobject{currentmarker}{}%
\end{pgfscope}%
\end{pgfscope}%
\begin{pgfscope}%
\pgfsetbuttcap%
\pgfsetroundjoin%
\definecolor{currentfill}{rgb}{0.000000,0.000000,0.000000}%
\pgfsetfillcolor{currentfill}%
\pgfsetlinewidth{0.602250pt}%
\definecolor{currentstroke}{rgb}{0.000000,0.000000,0.000000}%
\pgfsetstrokecolor{currentstroke}%
\pgfsetdash{}{0pt}%
\pgfsys@defobject{currentmarker}{\pgfqpoint{-0.027778in}{0.000000in}}{\pgfqpoint{-0.000000in}{0.000000in}}{%
\pgfpathmoveto{\pgfqpoint{-0.000000in}{0.000000in}}%
\pgfpathlineto{\pgfqpoint{-0.027778in}{0.000000in}}%
\pgfusepath{stroke,fill}%
}%
\begin{pgfscope}%
\pgfsys@transformshift{0.689445in}{0.715788in}%
\pgfsys@useobject{currentmarker}{}%
\end{pgfscope}%
\end{pgfscope}%
\begin{pgfscope}%
\pgfsetbuttcap%
\pgfsetroundjoin%
\definecolor{currentfill}{rgb}{0.000000,0.000000,0.000000}%
\pgfsetfillcolor{currentfill}%
\pgfsetlinewidth{0.602250pt}%
\definecolor{currentstroke}{rgb}{0.000000,0.000000,0.000000}%
\pgfsetstrokecolor{currentstroke}%
\pgfsetdash{}{0pt}%
\pgfsys@defobject{currentmarker}{\pgfqpoint{-0.027778in}{0.000000in}}{\pgfqpoint{-0.000000in}{0.000000in}}{%
\pgfpathmoveto{\pgfqpoint{-0.000000in}{0.000000in}}%
\pgfpathlineto{\pgfqpoint{-0.027778in}{0.000000in}}%
\pgfusepath{stroke,fill}%
}%
\begin{pgfscope}%
\pgfsys@transformshift{0.689445in}{0.730768in}%
\pgfsys@useobject{currentmarker}{}%
\end{pgfscope}%
\end{pgfscope}%
\begin{pgfscope}%
\pgfsetbuttcap%
\pgfsetroundjoin%
\definecolor{currentfill}{rgb}{0.000000,0.000000,0.000000}%
\pgfsetfillcolor{currentfill}%
\pgfsetlinewidth{0.602250pt}%
\definecolor{currentstroke}{rgb}{0.000000,0.000000,0.000000}%
\pgfsetstrokecolor{currentstroke}%
\pgfsetdash{}{0pt}%
\pgfsys@defobject{currentmarker}{\pgfqpoint{-0.027778in}{0.000000in}}{\pgfqpoint{-0.000000in}{0.000000in}}{%
\pgfpathmoveto{\pgfqpoint{-0.000000in}{0.000000in}}%
\pgfpathlineto{\pgfqpoint{-0.027778in}{0.000000in}}%
\pgfusepath{stroke,fill}%
}%
\begin{pgfscope}%
\pgfsys@transformshift{0.689445in}{0.832325in}%
\pgfsys@useobject{currentmarker}{}%
\end{pgfscope}%
\end{pgfscope}%
\begin{pgfscope}%
\pgfsetbuttcap%
\pgfsetroundjoin%
\definecolor{currentfill}{rgb}{0.000000,0.000000,0.000000}%
\pgfsetfillcolor{currentfill}%
\pgfsetlinewidth{0.602250pt}%
\definecolor{currentstroke}{rgb}{0.000000,0.000000,0.000000}%
\pgfsetstrokecolor{currentstroke}%
\pgfsetdash{}{0pt}%
\pgfsys@defobject{currentmarker}{\pgfqpoint{-0.027778in}{0.000000in}}{\pgfqpoint{-0.000000in}{0.000000in}}{%
\pgfpathmoveto{\pgfqpoint{-0.000000in}{0.000000in}}%
\pgfpathlineto{\pgfqpoint{-0.027778in}{0.000000in}}%
\pgfusepath{stroke,fill}%
}%
\begin{pgfscope}%
\pgfsys@transformshift{0.689445in}{0.883893in}%
\pgfsys@useobject{currentmarker}{}%
\end{pgfscope}%
\end{pgfscope}%
\begin{pgfscope}%
\pgfsetbuttcap%
\pgfsetroundjoin%
\definecolor{currentfill}{rgb}{0.000000,0.000000,0.000000}%
\pgfsetfillcolor{currentfill}%
\pgfsetlinewidth{0.602250pt}%
\definecolor{currentstroke}{rgb}{0.000000,0.000000,0.000000}%
\pgfsetstrokecolor{currentstroke}%
\pgfsetdash{}{0pt}%
\pgfsys@defobject{currentmarker}{\pgfqpoint{-0.027778in}{0.000000in}}{\pgfqpoint{-0.000000in}{0.000000in}}{%
\pgfpathmoveto{\pgfqpoint{-0.000000in}{0.000000in}}%
\pgfpathlineto{\pgfqpoint{-0.027778in}{0.000000in}}%
\pgfusepath{stroke,fill}%
}%
\begin{pgfscope}%
\pgfsys@transformshift{0.689445in}{0.920481in}%
\pgfsys@useobject{currentmarker}{}%
\end{pgfscope}%
\end{pgfscope}%
\begin{pgfscope}%
\pgfsetbuttcap%
\pgfsetroundjoin%
\definecolor{currentfill}{rgb}{0.000000,0.000000,0.000000}%
\pgfsetfillcolor{currentfill}%
\pgfsetlinewidth{0.602250pt}%
\definecolor{currentstroke}{rgb}{0.000000,0.000000,0.000000}%
\pgfsetstrokecolor{currentstroke}%
\pgfsetdash{}{0pt}%
\pgfsys@defobject{currentmarker}{\pgfqpoint{-0.027778in}{0.000000in}}{\pgfqpoint{-0.000000in}{0.000000in}}{%
\pgfpathmoveto{\pgfqpoint{-0.000000in}{0.000000in}}%
\pgfpathlineto{\pgfqpoint{-0.027778in}{0.000000in}}%
\pgfusepath{stroke,fill}%
}%
\begin{pgfscope}%
\pgfsys@transformshift{0.689445in}{0.948861in}%
\pgfsys@useobject{currentmarker}{}%
\end{pgfscope}%
\end{pgfscope}%
\begin{pgfscope}%
\pgfsetbuttcap%
\pgfsetroundjoin%
\definecolor{currentfill}{rgb}{0.000000,0.000000,0.000000}%
\pgfsetfillcolor{currentfill}%
\pgfsetlinewidth{0.602250pt}%
\definecolor{currentstroke}{rgb}{0.000000,0.000000,0.000000}%
\pgfsetstrokecolor{currentstroke}%
\pgfsetdash{}{0pt}%
\pgfsys@defobject{currentmarker}{\pgfqpoint{-0.027778in}{0.000000in}}{\pgfqpoint{-0.000000in}{0.000000in}}{%
\pgfpathmoveto{\pgfqpoint{-0.000000in}{0.000000in}}%
\pgfpathlineto{\pgfqpoint{-0.027778in}{0.000000in}}%
\pgfusepath{stroke,fill}%
}%
\begin{pgfscope}%
\pgfsys@transformshift{0.689445in}{0.972050in}%
\pgfsys@useobject{currentmarker}{}%
\end{pgfscope}%
\end{pgfscope}%
\begin{pgfscope}%
\pgfsetbuttcap%
\pgfsetroundjoin%
\definecolor{currentfill}{rgb}{0.000000,0.000000,0.000000}%
\pgfsetfillcolor{currentfill}%
\pgfsetlinewidth{0.602250pt}%
\definecolor{currentstroke}{rgb}{0.000000,0.000000,0.000000}%
\pgfsetstrokecolor{currentstroke}%
\pgfsetdash{}{0pt}%
\pgfsys@defobject{currentmarker}{\pgfqpoint{-0.027778in}{0.000000in}}{\pgfqpoint{-0.000000in}{0.000000in}}{%
\pgfpathmoveto{\pgfqpoint{-0.000000in}{0.000000in}}%
\pgfpathlineto{\pgfqpoint{-0.027778in}{0.000000in}}%
\pgfusepath{stroke,fill}%
}%
\begin{pgfscope}%
\pgfsys@transformshift{0.689445in}{0.991655in}%
\pgfsys@useobject{currentmarker}{}%
\end{pgfscope}%
\end{pgfscope}%
\begin{pgfscope}%
\pgfsetbuttcap%
\pgfsetroundjoin%
\definecolor{currentfill}{rgb}{0.000000,0.000000,0.000000}%
\pgfsetfillcolor{currentfill}%
\pgfsetlinewidth{0.602250pt}%
\definecolor{currentstroke}{rgb}{0.000000,0.000000,0.000000}%
\pgfsetstrokecolor{currentstroke}%
\pgfsetdash{}{0pt}%
\pgfsys@defobject{currentmarker}{\pgfqpoint{-0.027778in}{0.000000in}}{\pgfqpoint{-0.000000in}{0.000000in}}{%
\pgfpathmoveto{\pgfqpoint{-0.000000in}{0.000000in}}%
\pgfpathlineto{\pgfqpoint{-0.027778in}{0.000000in}}%
\pgfusepath{stroke,fill}%
}%
\begin{pgfscope}%
\pgfsys@transformshift{0.689445in}{1.008638in}%
\pgfsys@useobject{currentmarker}{}%
\end{pgfscope}%
\end{pgfscope}%
\begin{pgfscope}%
\pgfsetbuttcap%
\pgfsetroundjoin%
\definecolor{currentfill}{rgb}{0.000000,0.000000,0.000000}%
\pgfsetfillcolor{currentfill}%
\pgfsetlinewidth{0.602250pt}%
\definecolor{currentstroke}{rgb}{0.000000,0.000000,0.000000}%
\pgfsetstrokecolor{currentstroke}%
\pgfsetdash{}{0pt}%
\pgfsys@defobject{currentmarker}{\pgfqpoint{-0.027778in}{0.000000in}}{\pgfqpoint{-0.000000in}{0.000000in}}{%
\pgfpathmoveto{\pgfqpoint{-0.000000in}{0.000000in}}%
\pgfpathlineto{\pgfqpoint{-0.027778in}{0.000000in}}%
\pgfusepath{stroke,fill}%
}%
\begin{pgfscope}%
\pgfsys@transformshift{0.689445in}{1.023618in}%
\pgfsys@useobject{currentmarker}{}%
\end{pgfscope}%
\end{pgfscope}%
\begin{pgfscope}%
\pgfsetbuttcap%
\pgfsetroundjoin%
\definecolor{currentfill}{rgb}{0.000000,0.000000,0.000000}%
\pgfsetfillcolor{currentfill}%
\pgfsetlinewidth{0.602250pt}%
\definecolor{currentstroke}{rgb}{0.000000,0.000000,0.000000}%
\pgfsetstrokecolor{currentstroke}%
\pgfsetdash{}{0pt}%
\pgfsys@defobject{currentmarker}{\pgfqpoint{-0.027778in}{0.000000in}}{\pgfqpoint{-0.000000in}{0.000000in}}{%
\pgfpathmoveto{\pgfqpoint{-0.000000in}{0.000000in}}%
\pgfpathlineto{\pgfqpoint{-0.027778in}{0.000000in}}%
\pgfusepath{stroke,fill}%
}%
\begin{pgfscope}%
\pgfsys@transformshift{0.689445in}{1.125175in}%
\pgfsys@useobject{currentmarker}{}%
\end{pgfscope}%
\end{pgfscope}%
\begin{pgfscope}%
\pgfsetbuttcap%
\pgfsetroundjoin%
\definecolor{currentfill}{rgb}{0.000000,0.000000,0.000000}%
\pgfsetfillcolor{currentfill}%
\pgfsetlinewidth{0.602250pt}%
\definecolor{currentstroke}{rgb}{0.000000,0.000000,0.000000}%
\pgfsetstrokecolor{currentstroke}%
\pgfsetdash{}{0pt}%
\pgfsys@defobject{currentmarker}{\pgfqpoint{-0.027778in}{0.000000in}}{\pgfqpoint{-0.000000in}{0.000000in}}{%
\pgfpathmoveto{\pgfqpoint{-0.000000in}{0.000000in}}%
\pgfpathlineto{\pgfqpoint{-0.027778in}{0.000000in}}%
\pgfusepath{stroke,fill}%
}%
\begin{pgfscope}%
\pgfsys@transformshift{0.689445in}{1.176743in}%
\pgfsys@useobject{currentmarker}{}%
\end{pgfscope}%
\end{pgfscope}%
\begin{pgfscope}%
\pgfsetbuttcap%
\pgfsetroundjoin%
\definecolor{currentfill}{rgb}{0.000000,0.000000,0.000000}%
\pgfsetfillcolor{currentfill}%
\pgfsetlinewidth{0.602250pt}%
\definecolor{currentstroke}{rgb}{0.000000,0.000000,0.000000}%
\pgfsetstrokecolor{currentstroke}%
\pgfsetdash{}{0pt}%
\pgfsys@defobject{currentmarker}{\pgfqpoint{-0.027778in}{0.000000in}}{\pgfqpoint{-0.000000in}{0.000000in}}{%
\pgfpathmoveto{\pgfqpoint{-0.000000in}{0.000000in}}%
\pgfpathlineto{\pgfqpoint{-0.027778in}{0.000000in}}%
\pgfusepath{stroke,fill}%
}%
\begin{pgfscope}%
\pgfsys@transformshift{0.689445in}{1.213331in}%
\pgfsys@useobject{currentmarker}{}%
\end{pgfscope}%
\end{pgfscope}%
\begin{pgfscope}%
\pgfsetbuttcap%
\pgfsetroundjoin%
\definecolor{currentfill}{rgb}{0.000000,0.000000,0.000000}%
\pgfsetfillcolor{currentfill}%
\pgfsetlinewidth{0.602250pt}%
\definecolor{currentstroke}{rgb}{0.000000,0.000000,0.000000}%
\pgfsetstrokecolor{currentstroke}%
\pgfsetdash{}{0pt}%
\pgfsys@defobject{currentmarker}{\pgfqpoint{-0.027778in}{0.000000in}}{\pgfqpoint{-0.000000in}{0.000000in}}{%
\pgfpathmoveto{\pgfqpoint{-0.000000in}{0.000000in}}%
\pgfpathlineto{\pgfqpoint{-0.027778in}{0.000000in}}%
\pgfusepath{stroke,fill}%
}%
\begin{pgfscope}%
\pgfsys@transformshift{0.689445in}{1.241711in}%
\pgfsys@useobject{currentmarker}{}%
\end{pgfscope}%
\end{pgfscope}%
\begin{pgfscope}%
\pgfsetbuttcap%
\pgfsetroundjoin%
\definecolor{currentfill}{rgb}{0.000000,0.000000,0.000000}%
\pgfsetfillcolor{currentfill}%
\pgfsetlinewidth{0.602250pt}%
\definecolor{currentstroke}{rgb}{0.000000,0.000000,0.000000}%
\pgfsetstrokecolor{currentstroke}%
\pgfsetdash{}{0pt}%
\pgfsys@defobject{currentmarker}{\pgfqpoint{-0.027778in}{0.000000in}}{\pgfqpoint{-0.000000in}{0.000000in}}{%
\pgfpathmoveto{\pgfqpoint{-0.000000in}{0.000000in}}%
\pgfpathlineto{\pgfqpoint{-0.027778in}{0.000000in}}%
\pgfusepath{stroke,fill}%
}%
\begin{pgfscope}%
\pgfsys@transformshift{0.689445in}{1.264899in}%
\pgfsys@useobject{currentmarker}{}%
\end{pgfscope}%
\end{pgfscope}%
\begin{pgfscope}%
\pgfsetbuttcap%
\pgfsetroundjoin%
\definecolor{currentfill}{rgb}{0.000000,0.000000,0.000000}%
\pgfsetfillcolor{currentfill}%
\pgfsetlinewidth{0.602250pt}%
\definecolor{currentstroke}{rgb}{0.000000,0.000000,0.000000}%
\pgfsetstrokecolor{currentstroke}%
\pgfsetdash{}{0pt}%
\pgfsys@defobject{currentmarker}{\pgfqpoint{-0.027778in}{0.000000in}}{\pgfqpoint{-0.000000in}{0.000000in}}{%
\pgfpathmoveto{\pgfqpoint{-0.000000in}{0.000000in}}%
\pgfpathlineto{\pgfqpoint{-0.027778in}{0.000000in}}%
\pgfusepath{stroke,fill}%
}%
\begin{pgfscope}%
\pgfsys@transformshift{0.689445in}{1.284505in}%
\pgfsys@useobject{currentmarker}{}%
\end{pgfscope}%
\end{pgfscope}%
\begin{pgfscope}%
\pgfsetbuttcap%
\pgfsetroundjoin%
\definecolor{currentfill}{rgb}{0.000000,0.000000,0.000000}%
\pgfsetfillcolor{currentfill}%
\pgfsetlinewidth{0.602250pt}%
\definecolor{currentstroke}{rgb}{0.000000,0.000000,0.000000}%
\pgfsetstrokecolor{currentstroke}%
\pgfsetdash{}{0pt}%
\pgfsys@defobject{currentmarker}{\pgfqpoint{-0.027778in}{0.000000in}}{\pgfqpoint{-0.000000in}{0.000000in}}{%
\pgfpathmoveto{\pgfqpoint{-0.000000in}{0.000000in}}%
\pgfpathlineto{\pgfqpoint{-0.027778in}{0.000000in}}%
\pgfusepath{stroke,fill}%
}%
\begin{pgfscope}%
\pgfsys@transformshift{0.689445in}{1.301488in}%
\pgfsys@useobject{currentmarker}{}%
\end{pgfscope}%
\end{pgfscope}%
\begin{pgfscope}%
\pgfsetbuttcap%
\pgfsetroundjoin%
\definecolor{currentfill}{rgb}{0.000000,0.000000,0.000000}%
\pgfsetfillcolor{currentfill}%
\pgfsetlinewidth{0.602250pt}%
\definecolor{currentstroke}{rgb}{0.000000,0.000000,0.000000}%
\pgfsetstrokecolor{currentstroke}%
\pgfsetdash{}{0pt}%
\pgfsys@defobject{currentmarker}{\pgfqpoint{-0.027778in}{0.000000in}}{\pgfqpoint{-0.000000in}{0.000000in}}{%
\pgfpathmoveto{\pgfqpoint{-0.000000in}{0.000000in}}%
\pgfpathlineto{\pgfqpoint{-0.027778in}{0.000000in}}%
\pgfusepath{stroke,fill}%
}%
\begin{pgfscope}%
\pgfsys@transformshift{0.689445in}{1.316468in}%
\pgfsys@useobject{currentmarker}{}%
\end{pgfscope}%
\end{pgfscope}%
\begin{pgfscope}%
\pgfsetbuttcap%
\pgfsetroundjoin%
\definecolor{currentfill}{rgb}{0.000000,0.000000,0.000000}%
\pgfsetfillcolor{currentfill}%
\pgfsetlinewidth{0.602250pt}%
\definecolor{currentstroke}{rgb}{0.000000,0.000000,0.000000}%
\pgfsetstrokecolor{currentstroke}%
\pgfsetdash{}{0pt}%
\pgfsys@defobject{currentmarker}{\pgfqpoint{-0.027778in}{0.000000in}}{\pgfqpoint{-0.000000in}{0.000000in}}{%
\pgfpathmoveto{\pgfqpoint{-0.000000in}{0.000000in}}%
\pgfpathlineto{\pgfqpoint{-0.027778in}{0.000000in}}%
\pgfusepath{stroke,fill}%
}%
\begin{pgfscope}%
\pgfsys@transformshift{0.689445in}{1.418024in}%
\pgfsys@useobject{currentmarker}{}%
\end{pgfscope}%
\end{pgfscope}%
\begin{pgfscope}%
\pgfsetbuttcap%
\pgfsetroundjoin%
\definecolor{currentfill}{rgb}{0.000000,0.000000,0.000000}%
\pgfsetfillcolor{currentfill}%
\pgfsetlinewidth{0.602250pt}%
\definecolor{currentstroke}{rgb}{0.000000,0.000000,0.000000}%
\pgfsetstrokecolor{currentstroke}%
\pgfsetdash{}{0pt}%
\pgfsys@defobject{currentmarker}{\pgfqpoint{-0.027778in}{0.000000in}}{\pgfqpoint{-0.000000in}{0.000000in}}{%
\pgfpathmoveto{\pgfqpoint{-0.000000in}{0.000000in}}%
\pgfpathlineto{\pgfqpoint{-0.027778in}{0.000000in}}%
\pgfusepath{stroke,fill}%
}%
\begin{pgfscope}%
\pgfsys@transformshift{0.689445in}{1.469593in}%
\pgfsys@useobject{currentmarker}{}%
\end{pgfscope}%
\end{pgfscope}%
\begin{pgfscope}%
\pgfsetbuttcap%
\pgfsetroundjoin%
\definecolor{currentfill}{rgb}{0.000000,0.000000,0.000000}%
\pgfsetfillcolor{currentfill}%
\pgfsetlinewidth{0.602250pt}%
\definecolor{currentstroke}{rgb}{0.000000,0.000000,0.000000}%
\pgfsetstrokecolor{currentstroke}%
\pgfsetdash{}{0pt}%
\pgfsys@defobject{currentmarker}{\pgfqpoint{-0.027778in}{0.000000in}}{\pgfqpoint{-0.000000in}{0.000000in}}{%
\pgfpathmoveto{\pgfqpoint{-0.000000in}{0.000000in}}%
\pgfpathlineto{\pgfqpoint{-0.027778in}{0.000000in}}%
\pgfusepath{stroke,fill}%
}%
\begin{pgfscope}%
\pgfsys@transformshift{0.689445in}{1.506181in}%
\pgfsys@useobject{currentmarker}{}%
\end{pgfscope}%
\end{pgfscope}%
\begin{pgfscope}%
\pgfsetbuttcap%
\pgfsetroundjoin%
\definecolor{currentfill}{rgb}{0.000000,0.000000,0.000000}%
\pgfsetfillcolor{currentfill}%
\pgfsetlinewidth{0.602250pt}%
\definecolor{currentstroke}{rgb}{0.000000,0.000000,0.000000}%
\pgfsetstrokecolor{currentstroke}%
\pgfsetdash{}{0pt}%
\pgfsys@defobject{currentmarker}{\pgfqpoint{-0.027778in}{0.000000in}}{\pgfqpoint{-0.000000in}{0.000000in}}{%
\pgfpathmoveto{\pgfqpoint{-0.000000in}{0.000000in}}%
\pgfpathlineto{\pgfqpoint{-0.027778in}{0.000000in}}%
\pgfusepath{stroke,fill}%
}%
\begin{pgfscope}%
\pgfsys@transformshift{0.689445in}{1.534561in}%
\pgfsys@useobject{currentmarker}{}%
\end{pgfscope}%
\end{pgfscope}%
\begin{pgfscope}%
\pgfsetbuttcap%
\pgfsetroundjoin%
\definecolor{currentfill}{rgb}{0.000000,0.000000,0.000000}%
\pgfsetfillcolor{currentfill}%
\pgfsetlinewidth{0.602250pt}%
\definecolor{currentstroke}{rgb}{0.000000,0.000000,0.000000}%
\pgfsetstrokecolor{currentstroke}%
\pgfsetdash{}{0pt}%
\pgfsys@defobject{currentmarker}{\pgfqpoint{-0.027778in}{0.000000in}}{\pgfqpoint{-0.000000in}{0.000000in}}{%
\pgfpathmoveto{\pgfqpoint{-0.000000in}{0.000000in}}%
\pgfpathlineto{\pgfqpoint{-0.027778in}{0.000000in}}%
\pgfusepath{stroke,fill}%
}%
\begin{pgfscope}%
\pgfsys@transformshift{0.689445in}{1.557749in}%
\pgfsys@useobject{currentmarker}{}%
\end{pgfscope}%
\end{pgfscope}%
\begin{pgfscope}%
\pgfsetbuttcap%
\pgfsetroundjoin%
\definecolor{currentfill}{rgb}{0.000000,0.000000,0.000000}%
\pgfsetfillcolor{currentfill}%
\pgfsetlinewidth{0.602250pt}%
\definecolor{currentstroke}{rgb}{0.000000,0.000000,0.000000}%
\pgfsetstrokecolor{currentstroke}%
\pgfsetdash{}{0pt}%
\pgfsys@defobject{currentmarker}{\pgfqpoint{-0.027778in}{0.000000in}}{\pgfqpoint{-0.000000in}{0.000000in}}{%
\pgfpathmoveto{\pgfqpoint{-0.000000in}{0.000000in}}%
\pgfpathlineto{\pgfqpoint{-0.027778in}{0.000000in}}%
\pgfusepath{stroke,fill}%
}%
\begin{pgfscope}%
\pgfsys@transformshift{0.689445in}{1.577354in}%
\pgfsys@useobject{currentmarker}{}%
\end{pgfscope}%
\end{pgfscope}%
\begin{pgfscope}%
\pgfsetbuttcap%
\pgfsetroundjoin%
\definecolor{currentfill}{rgb}{0.000000,0.000000,0.000000}%
\pgfsetfillcolor{currentfill}%
\pgfsetlinewidth{0.602250pt}%
\definecolor{currentstroke}{rgb}{0.000000,0.000000,0.000000}%
\pgfsetstrokecolor{currentstroke}%
\pgfsetdash{}{0pt}%
\pgfsys@defobject{currentmarker}{\pgfqpoint{-0.027778in}{0.000000in}}{\pgfqpoint{-0.000000in}{0.000000in}}{%
\pgfpathmoveto{\pgfqpoint{-0.000000in}{0.000000in}}%
\pgfpathlineto{\pgfqpoint{-0.027778in}{0.000000in}}%
\pgfusepath{stroke,fill}%
}%
\begin{pgfscope}%
\pgfsys@transformshift{0.689445in}{1.594337in}%
\pgfsys@useobject{currentmarker}{}%
\end{pgfscope}%
\end{pgfscope}%
\begin{pgfscope}%
\pgfsetbuttcap%
\pgfsetroundjoin%
\definecolor{currentfill}{rgb}{0.000000,0.000000,0.000000}%
\pgfsetfillcolor{currentfill}%
\pgfsetlinewidth{0.602250pt}%
\definecolor{currentstroke}{rgb}{0.000000,0.000000,0.000000}%
\pgfsetstrokecolor{currentstroke}%
\pgfsetdash{}{0pt}%
\pgfsys@defobject{currentmarker}{\pgfqpoint{-0.027778in}{0.000000in}}{\pgfqpoint{-0.000000in}{0.000000in}}{%
\pgfpathmoveto{\pgfqpoint{-0.000000in}{0.000000in}}%
\pgfpathlineto{\pgfqpoint{-0.027778in}{0.000000in}}%
\pgfusepath{stroke,fill}%
}%
\begin{pgfscope}%
\pgfsys@transformshift{0.689445in}{1.609317in}%
\pgfsys@useobject{currentmarker}{}%
\end{pgfscope}%
\end{pgfscope}%
\begin{pgfscope}%
\pgfsetbuttcap%
\pgfsetroundjoin%
\definecolor{currentfill}{rgb}{0.000000,0.000000,0.000000}%
\pgfsetfillcolor{currentfill}%
\pgfsetlinewidth{0.602250pt}%
\definecolor{currentstroke}{rgb}{0.000000,0.000000,0.000000}%
\pgfsetstrokecolor{currentstroke}%
\pgfsetdash{}{0pt}%
\pgfsys@defobject{currentmarker}{\pgfqpoint{-0.027778in}{0.000000in}}{\pgfqpoint{-0.000000in}{0.000000in}}{%
\pgfpathmoveto{\pgfqpoint{-0.000000in}{0.000000in}}%
\pgfpathlineto{\pgfqpoint{-0.027778in}{0.000000in}}%
\pgfusepath{stroke,fill}%
}%
\begin{pgfscope}%
\pgfsys@transformshift{0.689445in}{1.710874in}%
\pgfsys@useobject{currentmarker}{}%
\end{pgfscope}%
\end{pgfscope}%
\begin{pgfscope}%
\pgfsetbuttcap%
\pgfsetroundjoin%
\definecolor{currentfill}{rgb}{0.000000,0.000000,0.000000}%
\pgfsetfillcolor{currentfill}%
\pgfsetlinewidth{0.602250pt}%
\definecolor{currentstroke}{rgb}{0.000000,0.000000,0.000000}%
\pgfsetstrokecolor{currentstroke}%
\pgfsetdash{}{0pt}%
\pgfsys@defobject{currentmarker}{\pgfqpoint{-0.027778in}{0.000000in}}{\pgfqpoint{-0.000000in}{0.000000in}}{%
\pgfpathmoveto{\pgfqpoint{-0.000000in}{0.000000in}}%
\pgfpathlineto{\pgfqpoint{-0.027778in}{0.000000in}}%
\pgfusepath{stroke,fill}%
}%
\begin{pgfscope}%
\pgfsys@transformshift{0.689445in}{1.762442in}%
\pgfsys@useobject{currentmarker}{}%
\end{pgfscope}%
\end{pgfscope}%
\begin{pgfscope}%
\pgfsetbuttcap%
\pgfsetroundjoin%
\definecolor{currentfill}{rgb}{0.000000,0.000000,0.000000}%
\pgfsetfillcolor{currentfill}%
\pgfsetlinewidth{0.602250pt}%
\definecolor{currentstroke}{rgb}{0.000000,0.000000,0.000000}%
\pgfsetstrokecolor{currentstroke}%
\pgfsetdash{}{0pt}%
\pgfsys@defobject{currentmarker}{\pgfqpoint{-0.027778in}{0.000000in}}{\pgfqpoint{-0.000000in}{0.000000in}}{%
\pgfpathmoveto{\pgfqpoint{-0.000000in}{0.000000in}}%
\pgfpathlineto{\pgfqpoint{-0.027778in}{0.000000in}}%
\pgfusepath{stroke,fill}%
}%
\begin{pgfscope}%
\pgfsys@transformshift{0.689445in}{1.799031in}%
\pgfsys@useobject{currentmarker}{}%
\end{pgfscope}%
\end{pgfscope}%
\begin{pgfscope}%
\pgfsetbuttcap%
\pgfsetroundjoin%
\definecolor{currentfill}{rgb}{0.000000,0.000000,0.000000}%
\pgfsetfillcolor{currentfill}%
\pgfsetlinewidth{0.602250pt}%
\definecolor{currentstroke}{rgb}{0.000000,0.000000,0.000000}%
\pgfsetstrokecolor{currentstroke}%
\pgfsetdash{}{0pt}%
\pgfsys@defobject{currentmarker}{\pgfqpoint{-0.027778in}{0.000000in}}{\pgfqpoint{-0.000000in}{0.000000in}}{%
\pgfpathmoveto{\pgfqpoint{-0.000000in}{0.000000in}}%
\pgfpathlineto{\pgfqpoint{-0.027778in}{0.000000in}}%
\pgfusepath{stroke,fill}%
}%
\begin{pgfscope}%
\pgfsys@transformshift{0.689445in}{1.827411in}%
\pgfsys@useobject{currentmarker}{}%
\end{pgfscope}%
\end{pgfscope}%
\begin{pgfscope}%
\pgfsetbuttcap%
\pgfsetroundjoin%
\definecolor{currentfill}{rgb}{0.000000,0.000000,0.000000}%
\pgfsetfillcolor{currentfill}%
\pgfsetlinewidth{0.602250pt}%
\definecolor{currentstroke}{rgb}{0.000000,0.000000,0.000000}%
\pgfsetstrokecolor{currentstroke}%
\pgfsetdash{}{0pt}%
\pgfsys@defobject{currentmarker}{\pgfqpoint{-0.027778in}{0.000000in}}{\pgfqpoint{-0.000000in}{0.000000in}}{%
\pgfpathmoveto{\pgfqpoint{-0.000000in}{0.000000in}}%
\pgfpathlineto{\pgfqpoint{-0.027778in}{0.000000in}}%
\pgfusepath{stroke,fill}%
}%
\begin{pgfscope}%
\pgfsys@transformshift{0.689445in}{1.850599in}%
\pgfsys@useobject{currentmarker}{}%
\end{pgfscope}%
\end{pgfscope}%
\begin{pgfscope}%
\pgfsetbuttcap%
\pgfsetroundjoin%
\definecolor{currentfill}{rgb}{0.000000,0.000000,0.000000}%
\pgfsetfillcolor{currentfill}%
\pgfsetlinewidth{0.602250pt}%
\definecolor{currentstroke}{rgb}{0.000000,0.000000,0.000000}%
\pgfsetstrokecolor{currentstroke}%
\pgfsetdash{}{0pt}%
\pgfsys@defobject{currentmarker}{\pgfqpoint{-0.027778in}{0.000000in}}{\pgfqpoint{-0.000000in}{0.000000in}}{%
\pgfpathmoveto{\pgfqpoint{-0.000000in}{0.000000in}}%
\pgfpathlineto{\pgfqpoint{-0.027778in}{0.000000in}}%
\pgfusepath{stroke,fill}%
}%
\begin{pgfscope}%
\pgfsys@transformshift{0.689445in}{1.870204in}%
\pgfsys@useobject{currentmarker}{}%
\end{pgfscope}%
\end{pgfscope}%
\begin{pgfscope}%
\pgfsetbuttcap%
\pgfsetroundjoin%
\definecolor{currentfill}{rgb}{0.000000,0.000000,0.000000}%
\pgfsetfillcolor{currentfill}%
\pgfsetlinewidth{0.602250pt}%
\definecolor{currentstroke}{rgb}{0.000000,0.000000,0.000000}%
\pgfsetstrokecolor{currentstroke}%
\pgfsetdash{}{0pt}%
\pgfsys@defobject{currentmarker}{\pgfqpoint{-0.027778in}{0.000000in}}{\pgfqpoint{-0.000000in}{0.000000in}}{%
\pgfpathmoveto{\pgfqpoint{-0.000000in}{0.000000in}}%
\pgfpathlineto{\pgfqpoint{-0.027778in}{0.000000in}}%
\pgfusepath{stroke,fill}%
}%
\begin{pgfscope}%
\pgfsys@transformshift{0.689445in}{1.887187in}%
\pgfsys@useobject{currentmarker}{}%
\end{pgfscope}%
\end{pgfscope}%
\begin{pgfscope}%
\pgfsetbuttcap%
\pgfsetroundjoin%
\definecolor{currentfill}{rgb}{0.000000,0.000000,0.000000}%
\pgfsetfillcolor{currentfill}%
\pgfsetlinewidth{0.602250pt}%
\definecolor{currentstroke}{rgb}{0.000000,0.000000,0.000000}%
\pgfsetstrokecolor{currentstroke}%
\pgfsetdash{}{0pt}%
\pgfsys@defobject{currentmarker}{\pgfqpoint{-0.027778in}{0.000000in}}{\pgfqpoint{-0.000000in}{0.000000in}}{%
\pgfpathmoveto{\pgfqpoint{-0.000000in}{0.000000in}}%
\pgfpathlineto{\pgfqpoint{-0.027778in}{0.000000in}}%
\pgfusepath{stroke,fill}%
}%
\begin{pgfscope}%
\pgfsys@transformshift{0.689445in}{1.902167in}%
\pgfsys@useobject{currentmarker}{}%
\end{pgfscope}%
\end{pgfscope}%
\begin{pgfscope}%
\pgfsetbuttcap%
\pgfsetroundjoin%
\definecolor{currentfill}{rgb}{0.000000,0.000000,0.000000}%
\pgfsetfillcolor{currentfill}%
\pgfsetlinewidth{0.602250pt}%
\definecolor{currentstroke}{rgb}{0.000000,0.000000,0.000000}%
\pgfsetstrokecolor{currentstroke}%
\pgfsetdash{}{0pt}%
\pgfsys@defobject{currentmarker}{\pgfqpoint{-0.027778in}{0.000000in}}{\pgfqpoint{-0.000000in}{0.000000in}}{%
\pgfpathmoveto{\pgfqpoint{-0.000000in}{0.000000in}}%
\pgfpathlineto{\pgfqpoint{-0.027778in}{0.000000in}}%
\pgfusepath{stroke,fill}%
}%
\begin{pgfscope}%
\pgfsys@transformshift{0.689445in}{2.003724in}%
\pgfsys@useobject{currentmarker}{}%
\end{pgfscope}%
\end{pgfscope}%
\begin{pgfscope}%
\pgfsetbuttcap%
\pgfsetroundjoin%
\definecolor{currentfill}{rgb}{0.000000,0.000000,0.000000}%
\pgfsetfillcolor{currentfill}%
\pgfsetlinewidth{0.602250pt}%
\definecolor{currentstroke}{rgb}{0.000000,0.000000,0.000000}%
\pgfsetstrokecolor{currentstroke}%
\pgfsetdash{}{0pt}%
\pgfsys@defobject{currentmarker}{\pgfqpoint{-0.027778in}{0.000000in}}{\pgfqpoint{-0.000000in}{0.000000in}}{%
\pgfpathmoveto{\pgfqpoint{-0.000000in}{0.000000in}}%
\pgfpathlineto{\pgfqpoint{-0.027778in}{0.000000in}}%
\pgfusepath{stroke,fill}%
}%
\begin{pgfscope}%
\pgfsys@transformshift{0.689445in}{2.055292in}%
\pgfsys@useobject{currentmarker}{}%
\end{pgfscope}%
\end{pgfscope}%
\begin{pgfscope}%
\pgfsetbuttcap%
\pgfsetroundjoin%
\definecolor{currentfill}{rgb}{0.000000,0.000000,0.000000}%
\pgfsetfillcolor{currentfill}%
\pgfsetlinewidth{0.602250pt}%
\definecolor{currentstroke}{rgb}{0.000000,0.000000,0.000000}%
\pgfsetstrokecolor{currentstroke}%
\pgfsetdash{}{0pt}%
\pgfsys@defobject{currentmarker}{\pgfqpoint{-0.027778in}{0.000000in}}{\pgfqpoint{-0.000000in}{0.000000in}}{%
\pgfpathmoveto{\pgfqpoint{-0.000000in}{0.000000in}}%
\pgfpathlineto{\pgfqpoint{-0.027778in}{0.000000in}}%
\pgfusepath{stroke,fill}%
}%
\begin{pgfscope}%
\pgfsys@transformshift{0.689445in}{2.091880in}%
\pgfsys@useobject{currentmarker}{}%
\end{pgfscope}%
\end{pgfscope}%
\begin{pgfscope}%
\pgfsetbuttcap%
\pgfsetroundjoin%
\definecolor{currentfill}{rgb}{0.000000,0.000000,0.000000}%
\pgfsetfillcolor{currentfill}%
\pgfsetlinewidth{0.602250pt}%
\definecolor{currentstroke}{rgb}{0.000000,0.000000,0.000000}%
\pgfsetstrokecolor{currentstroke}%
\pgfsetdash{}{0pt}%
\pgfsys@defobject{currentmarker}{\pgfqpoint{-0.027778in}{0.000000in}}{\pgfqpoint{-0.000000in}{0.000000in}}{%
\pgfpathmoveto{\pgfqpoint{-0.000000in}{0.000000in}}%
\pgfpathlineto{\pgfqpoint{-0.027778in}{0.000000in}}%
\pgfusepath{stroke,fill}%
}%
\begin{pgfscope}%
\pgfsys@transformshift{0.689445in}{2.120260in}%
\pgfsys@useobject{currentmarker}{}%
\end{pgfscope}%
\end{pgfscope}%
\begin{pgfscope}%
\pgfsetbuttcap%
\pgfsetroundjoin%
\definecolor{currentfill}{rgb}{0.000000,0.000000,0.000000}%
\pgfsetfillcolor{currentfill}%
\pgfsetlinewidth{0.602250pt}%
\definecolor{currentstroke}{rgb}{0.000000,0.000000,0.000000}%
\pgfsetstrokecolor{currentstroke}%
\pgfsetdash{}{0pt}%
\pgfsys@defobject{currentmarker}{\pgfqpoint{-0.027778in}{0.000000in}}{\pgfqpoint{-0.000000in}{0.000000in}}{%
\pgfpathmoveto{\pgfqpoint{-0.000000in}{0.000000in}}%
\pgfpathlineto{\pgfqpoint{-0.027778in}{0.000000in}}%
\pgfusepath{stroke,fill}%
}%
\begin{pgfscope}%
\pgfsys@transformshift{0.689445in}{2.143449in}%
\pgfsys@useobject{currentmarker}{}%
\end{pgfscope}%
\end{pgfscope}%
\begin{pgfscope}%
\pgfsetbuttcap%
\pgfsetroundjoin%
\definecolor{currentfill}{rgb}{0.000000,0.000000,0.000000}%
\pgfsetfillcolor{currentfill}%
\pgfsetlinewidth{0.602250pt}%
\definecolor{currentstroke}{rgb}{0.000000,0.000000,0.000000}%
\pgfsetstrokecolor{currentstroke}%
\pgfsetdash{}{0pt}%
\pgfsys@defobject{currentmarker}{\pgfqpoint{-0.027778in}{0.000000in}}{\pgfqpoint{-0.000000in}{0.000000in}}{%
\pgfpathmoveto{\pgfqpoint{-0.000000in}{0.000000in}}%
\pgfpathlineto{\pgfqpoint{-0.027778in}{0.000000in}}%
\pgfusepath{stroke,fill}%
}%
\begin{pgfscope}%
\pgfsys@transformshift{0.689445in}{2.163054in}%
\pgfsys@useobject{currentmarker}{}%
\end{pgfscope}%
\end{pgfscope}%
\begin{pgfscope}%
\pgfsetbuttcap%
\pgfsetroundjoin%
\definecolor{currentfill}{rgb}{0.000000,0.000000,0.000000}%
\pgfsetfillcolor{currentfill}%
\pgfsetlinewidth{0.602250pt}%
\definecolor{currentstroke}{rgb}{0.000000,0.000000,0.000000}%
\pgfsetstrokecolor{currentstroke}%
\pgfsetdash{}{0pt}%
\pgfsys@defobject{currentmarker}{\pgfqpoint{-0.027778in}{0.000000in}}{\pgfqpoint{-0.000000in}{0.000000in}}{%
\pgfpathmoveto{\pgfqpoint{-0.000000in}{0.000000in}}%
\pgfpathlineto{\pgfqpoint{-0.027778in}{0.000000in}}%
\pgfusepath{stroke,fill}%
}%
\begin{pgfscope}%
\pgfsys@transformshift{0.689445in}{2.180037in}%
\pgfsys@useobject{currentmarker}{}%
\end{pgfscope}%
\end{pgfscope}%
\begin{pgfscope}%
\pgfsetbuttcap%
\pgfsetroundjoin%
\definecolor{currentfill}{rgb}{0.000000,0.000000,0.000000}%
\pgfsetfillcolor{currentfill}%
\pgfsetlinewidth{0.602250pt}%
\definecolor{currentstroke}{rgb}{0.000000,0.000000,0.000000}%
\pgfsetstrokecolor{currentstroke}%
\pgfsetdash{}{0pt}%
\pgfsys@defobject{currentmarker}{\pgfqpoint{-0.027778in}{0.000000in}}{\pgfqpoint{-0.000000in}{0.000000in}}{%
\pgfpathmoveto{\pgfqpoint{-0.000000in}{0.000000in}}%
\pgfpathlineto{\pgfqpoint{-0.027778in}{0.000000in}}%
\pgfusepath{stroke,fill}%
}%
\begin{pgfscope}%
\pgfsys@transformshift{0.689445in}{2.195017in}%
\pgfsys@useobject{currentmarker}{}%
\end{pgfscope}%
\end{pgfscope}%
\begin{pgfscope}%
\pgfsetbuttcap%
\pgfsetroundjoin%
\definecolor{currentfill}{rgb}{0.000000,0.000000,0.000000}%
\pgfsetfillcolor{currentfill}%
\pgfsetlinewidth{0.602250pt}%
\definecolor{currentstroke}{rgb}{0.000000,0.000000,0.000000}%
\pgfsetstrokecolor{currentstroke}%
\pgfsetdash{}{0pt}%
\pgfsys@defobject{currentmarker}{\pgfqpoint{-0.027778in}{0.000000in}}{\pgfqpoint{-0.000000in}{0.000000in}}{%
\pgfpathmoveto{\pgfqpoint{-0.000000in}{0.000000in}}%
\pgfpathlineto{\pgfqpoint{-0.027778in}{0.000000in}}%
\pgfusepath{stroke,fill}%
}%
\begin{pgfscope}%
\pgfsys@transformshift{0.689445in}{2.296573in}%
\pgfsys@useobject{currentmarker}{}%
\end{pgfscope}%
\end{pgfscope}%
\begin{pgfscope}%
\pgfsetbuttcap%
\pgfsetroundjoin%
\definecolor{currentfill}{rgb}{0.000000,0.000000,0.000000}%
\pgfsetfillcolor{currentfill}%
\pgfsetlinewidth{0.602250pt}%
\definecolor{currentstroke}{rgb}{0.000000,0.000000,0.000000}%
\pgfsetstrokecolor{currentstroke}%
\pgfsetdash{}{0pt}%
\pgfsys@defobject{currentmarker}{\pgfqpoint{-0.027778in}{0.000000in}}{\pgfqpoint{-0.000000in}{0.000000in}}{%
\pgfpathmoveto{\pgfqpoint{-0.000000in}{0.000000in}}%
\pgfpathlineto{\pgfqpoint{-0.027778in}{0.000000in}}%
\pgfusepath{stroke,fill}%
}%
\begin{pgfscope}%
\pgfsys@transformshift{0.689445in}{2.348142in}%
\pgfsys@useobject{currentmarker}{}%
\end{pgfscope}%
\end{pgfscope}%
\begin{pgfscope}%
\pgfsetbuttcap%
\pgfsetroundjoin%
\definecolor{currentfill}{rgb}{0.000000,0.000000,0.000000}%
\pgfsetfillcolor{currentfill}%
\pgfsetlinewidth{0.602250pt}%
\definecolor{currentstroke}{rgb}{0.000000,0.000000,0.000000}%
\pgfsetstrokecolor{currentstroke}%
\pgfsetdash{}{0pt}%
\pgfsys@defobject{currentmarker}{\pgfqpoint{-0.027778in}{0.000000in}}{\pgfqpoint{-0.000000in}{0.000000in}}{%
\pgfpathmoveto{\pgfqpoint{-0.000000in}{0.000000in}}%
\pgfpathlineto{\pgfqpoint{-0.027778in}{0.000000in}}%
\pgfusepath{stroke,fill}%
}%
\begin{pgfscope}%
\pgfsys@transformshift{0.689445in}{2.384730in}%
\pgfsys@useobject{currentmarker}{}%
\end{pgfscope}%
\end{pgfscope}%
\begin{pgfscope}%
\pgfsetbuttcap%
\pgfsetroundjoin%
\definecolor{currentfill}{rgb}{0.000000,0.000000,0.000000}%
\pgfsetfillcolor{currentfill}%
\pgfsetlinewidth{0.602250pt}%
\definecolor{currentstroke}{rgb}{0.000000,0.000000,0.000000}%
\pgfsetstrokecolor{currentstroke}%
\pgfsetdash{}{0pt}%
\pgfsys@defobject{currentmarker}{\pgfqpoint{-0.027778in}{0.000000in}}{\pgfqpoint{-0.000000in}{0.000000in}}{%
\pgfpathmoveto{\pgfqpoint{-0.000000in}{0.000000in}}%
\pgfpathlineto{\pgfqpoint{-0.027778in}{0.000000in}}%
\pgfusepath{stroke,fill}%
}%
\begin{pgfscope}%
\pgfsys@transformshift{0.689445in}{2.413110in}%
\pgfsys@useobject{currentmarker}{}%
\end{pgfscope}%
\end{pgfscope}%
\begin{pgfscope}%
\pgfsetbuttcap%
\pgfsetroundjoin%
\definecolor{currentfill}{rgb}{0.000000,0.000000,0.000000}%
\pgfsetfillcolor{currentfill}%
\pgfsetlinewidth{0.602250pt}%
\definecolor{currentstroke}{rgb}{0.000000,0.000000,0.000000}%
\pgfsetstrokecolor{currentstroke}%
\pgfsetdash{}{0pt}%
\pgfsys@defobject{currentmarker}{\pgfqpoint{-0.027778in}{0.000000in}}{\pgfqpoint{-0.000000in}{0.000000in}}{%
\pgfpathmoveto{\pgfqpoint{-0.000000in}{0.000000in}}%
\pgfpathlineto{\pgfqpoint{-0.027778in}{0.000000in}}%
\pgfusepath{stroke,fill}%
}%
\begin{pgfscope}%
\pgfsys@transformshift{0.689445in}{2.436298in}%
\pgfsys@useobject{currentmarker}{}%
\end{pgfscope}%
\end{pgfscope}%
\begin{pgfscope}%
\pgfsetbuttcap%
\pgfsetroundjoin%
\definecolor{currentfill}{rgb}{0.000000,0.000000,0.000000}%
\pgfsetfillcolor{currentfill}%
\pgfsetlinewidth{0.602250pt}%
\definecolor{currentstroke}{rgb}{0.000000,0.000000,0.000000}%
\pgfsetstrokecolor{currentstroke}%
\pgfsetdash{}{0pt}%
\pgfsys@defobject{currentmarker}{\pgfqpoint{-0.027778in}{0.000000in}}{\pgfqpoint{-0.000000in}{0.000000in}}{%
\pgfpathmoveto{\pgfqpoint{-0.000000in}{0.000000in}}%
\pgfpathlineto{\pgfqpoint{-0.027778in}{0.000000in}}%
\pgfusepath{stroke,fill}%
}%
\begin{pgfscope}%
\pgfsys@transformshift{0.689445in}{2.455904in}%
\pgfsys@useobject{currentmarker}{}%
\end{pgfscope}%
\end{pgfscope}%
\begin{pgfscope}%
\pgfsetbuttcap%
\pgfsetroundjoin%
\definecolor{currentfill}{rgb}{0.000000,0.000000,0.000000}%
\pgfsetfillcolor{currentfill}%
\pgfsetlinewidth{0.602250pt}%
\definecolor{currentstroke}{rgb}{0.000000,0.000000,0.000000}%
\pgfsetstrokecolor{currentstroke}%
\pgfsetdash{}{0pt}%
\pgfsys@defobject{currentmarker}{\pgfqpoint{-0.027778in}{0.000000in}}{\pgfqpoint{-0.000000in}{0.000000in}}{%
\pgfpathmoveto{\pgfqpoint{-0.000000in}{0.000000in}}%
\pgfpathlineto{\pgfqpoint{-0.027778in}{0.000000in}}%
\pgfusepath{stroke,fill}%
}%
\begin{pgfscope}%
\pgfsys@transformshift{0.689445in}{2.472887in}%
\pgfsys@useobject{currentmarker}{}%
\end{pgfscope}%
\end{pgfscope}%
\begin{pgfscope}%
\pgfsetbuttcap%
\pgfsetroundjoin%
\definecolor{currentfill}{rgb}{0.000000,0.000000,0.000000}%
\pgfsetfillcolor{currentfill}%
\pgfsetlinewidth{0.602250pt}%
\definecolor{currentstroke}{rgb}{0.000000,0.000000,0.000000}%
\pgfsetstrokecolor{currentstroke}%
\pgfsetdash{}{0pt}%
\pgfsys@defobject{currentmarker}{\pgfqpoint{-0.027778in}{0.000000in}}{\pgfqpoint{-0.000000in}{0.000000in}}{%
\pgfpathmoveto{\pgfqpoint{-0.000000in}{0.000000in}}%
\pgfpathlineto{\pgfqpoint{-0.027778in}{0.000000in}}%
\pgfusepath{stroke,fill}%
}%
\begin{pgfscope}%
\pgfsys@transformshift{0.689445in}{2.487867in}%
\pgfsys@useobject{currentmarker}{}%
\end{pgfscope}%
\end{pgfscope}%
\begin{pgfscope}%
\pgfsetbuttcap%
\pgfsetroundjoin%
\definecolor{currentfill}{rgb}{0.000000,0.000000,0.000000}%
\pgfsetfillcolor{currentfill}%
\pgfsetlinewidth{0.602250pt}%
\definecolor{currentstroke}{rgb}{0.000000,0.000000,0.000000}%
\pgfsetstrokecolor{currentstroke}%
\pgfsetdash{}{0pt}%
\pgfsys@defobject{currentmarker}{\pgfqpoint{-0.027778in}{0.000000in}}{\pgfqpoint{-0.000000in}{0.000000in}}{%
\pgfpathmoveto{\pgfqpoint{-0.000000in}{0.000000in}}%
\pgfpathlineto{\pgfqpoint{-0.027778in}{0.000000in}}%
\pgfusepath{stroke,fill}%
}%
\begin{pgfscope}%
\pgfsys@transformshift{0.689445in}{2.589423in}%
\pgfsys@useobject{currentmarker}{}%
\end{pgfscope}%
\end{pgfscope}%
\begin{pgfscope}%
\pgfsetbuttcap%
\pgfsetroundjoin%
\definecolor{currentfill}{rgb}{0.000000,0.000000,0.000000}%
\pgfsetfillcolor{currentfill}%
\pgfsetlinewidth{0.602250pt}%
\definecolor{currentstroke}{rgb}{0.000000,0.000000,0.000000}%
\pgfsetstrokecolor{currentstroke}%
\pgfsetdash{}{0pt}%
\pgfsys@defobject{currentmarker}{\pgfqpoint{-0.027778in}{0.000000in}}{\pgfqpoint{-0.000000in}{0.000000in}}{%
\pgfpathmoveto{\pgfqpoint{-0.000000in}{0.000000in}}%
\pgfpathlineto{\pgfqpoint{-0.027778in}{0.000000in}}%
\pgfusepath{stroke,fill}%
}%
\begin{pgfscope}%
\pgfsys@transformshift{0.689445in}{2.640992in}%
\pgfsys@useobject{currentmarker}{}%
\end{pgfscope}%
\end{pgfscope}%
\begin{pgfscope}%
\pgfsetbuttcap%
\pgfsetroundjoin%
\definecolor{currentfill}{rgb}{0.000000,0.000000,0.000000}%
\pgfsetfillcolor{currentfill}%
\pgfsetlinewidth{0.602250pt}%
\definecolor{currentstroke}{rgb}{0.000000,0.000000,0.000000}%
\pgfsetstrokecolor{currentstroke}%
\pgfsetdash{}{0pt}%
\pgfsys@defobject{currentmarker}{\pgfqpoint{-0.027778in}{0.000000in}}{\pgfqpoint{-0.000000in}{0.000000in}}{%
\pgfpathmoveto{\pgfqpoint{-0.000000in}{0.000000in}}%
\pgfpathlineto{\pgfqpoint{-0.027778in}{0.000000in}}%
\pgfusepath{stroke,fill}%
}%
\begin{pgfscope}%
\pgfsys@transformshift{0.689445in}{2.677580in}%
\pgfsys@useobject{currentmarker}{}%
\end{pgfscope}%
\end{pgfscope}%
\begin{pgfscope}%
\pgfsetbuttcap%
\pgfsetroundjoin%
\definecolor{currentfill}{rgb}{0.000000,0.000000,0.000000}%
\pgfsetfillcolor{currentfill}%
\pgfsetlinewidth{0.602250pt}%
\definecolor{currentstroke}{rgb}{0.000000,0.000000,0.000000}%
\pgfsetstrokecolor{currentstroke}%
\pgfsetdash{}{0pt}%
\pgfsys@defobject{currentmarker}{\pgfqpoint{-0.027778in}{0.000000in}}{\pgfqpoint{-0.000000in}{0.000000in}}{%
\pgfpathmoveto{\pgfqpoint{-0.000000in}{0.000000in}}%
\pgfpathlineto{\pgfqpoint{-0.027778in}{0.000000in}}%
\pgfusepath{stroke,fill}%
}%
\begin{pgfscope}%
\pgfsys@transformshift{0.689445in}{2.705960in}%
\pgfsys@useobject{currentmarker}{}%
\end{pgfscope}%
\end{pgfscope}%
\begin{pgfscope}%
\pgfsetbuttcap%
\pgfsetroundjoin%
\definecolor{currentfill}{rgb}{0.000000,0.000000,0.000000}%
\pgfsetfillcolor{currentfill}%
\pgfsetlinewidth{0.602250pt}%
\definecolor{currentstroke}{rgb}{0.000000,0.000000,0.000000}%
\pgfsetstrokecolor{currentstroke}%
\pgfsetdash{}{0pt}%
\pgfsys@defobject{currentmarker}{\pgfqpoint{-0.027778in}{0.000000in}}{\pgfqpoint{-0.000000in}{0.000000in}}{%
\pgfpathmoveto{\pgfqpoint{-0.000000in}{0.000000in}}%
\pgfpathlineto{\pgfqpoint{-0.027778in}{0.000000in}}%
\pgfusepath{stroke,fill}%
}%
\begin{pgfscope}%
\pgfsys@transformshift{0.689445in}{2.729148in}%
\pgfsys@useobject{currentmarker}{}%
\end{pgfscope}%
\end{pgfscope}%
\begin{pgfscope}%
\pgfsetbuttcap%
\pgfsetroundjoin%
\definecolor{currentfill}{rgb}{0.000000,0.000000,0.000000}%
\pgfsetfillcolor{currentfill}%
\pgfsetlinewidth{0.602250pt}%
\definecolor{currentstroke}{rgb}{0.000000,0.000000,0.000000}%
\pgfsetstrokecolor{currentstroke}%
\pgfsetdash{}{0pt}%
\pgfsys@defobject{currentmarker}{\pgfqpoint{-0.027778in}{0.000000in}}{\pgfqpoint{-0.000000in}{0.000000in}}{%
\pgfpathmoveto{\pgfqpoint{-0.000000in}{0.000000in}}%
\pgfpathlineto{\pgfqpoint{-0.027778in}{0.000000in}}%
\pgfusepath{stroke,fill}%
}%
\begin{pgfscope}%
\pgfsys@transformshift{0.689445in}{2.748753in}%
\pgfsys@useobject{currentmarker}{}%
\end{pgfscope}%
\end{pgfscope}%
\begin{pgfscope}%
\pgfsetbuttcap%
\pgfsetroundjoin%
\definecolor{currentfill}{rgb}{0.000000,0.000000,0.000000}%
\pgfsetfillcolor{currentfill}%
\pgfsetlinewidth{0.602250pt}%
\definecolor{currentstroke}{rgb}{0.000000,0.000000,0.000000}%
\pgfsetstrokecolor{currentstroke}%
\pgfsetdash{}{0pt}%
\pgfsys@defobject{currentmarker}{\pgfqpoint{-0.027778in}{0.000000in}}{\pgfqpoint{-0.000000in}{0.000000in}}{%
\pgfpathmoveto{\pgfqpoint{-0.000000in}{0.000000in}}%
\pgfpathlineto{\pgfqpoint{-0.027778in}{0.000000in}}%
\pgfusepath{stroke,fill}%
}%
\begin{pgfscope}%
\pgfsys@transformshift{0.689445in}{2.765736in}%
\pgfsys@useobject{currentmarker}{}%
\end{pgfscope}%
\end{pgfscope}%
\begin{pgfscope}%
\pgfsetbuttcap%
\pgfsetroundjoin%
\definecolor{currentfill}{rgb}{0.000000,0.000000,0.000000}%
\pgfsetfillcolor{currentfill}%
\pgfsetlinewidth{0.602250pt}%
\definecolor{currentstroke}{rgb}{0.000000,0.000000,0.000000}%
\pgfsetstrokecolor{currentstroke}%
\pgfsetdash{}{0pt}%
\pgfsys@defobject{currentmarker}{\pgfqpoint{-0.027778in}{0.000000in}}{\pgfqpoint{-0.000000in}{0.000000in}}{%
\pgfpathmoveto{\pgfqpoint{-0.000000in}{0.000000in}}%
\pgfpathlineto{\pgfqpoint{-0.027778in}{0.000000in}}%
\pgfusepath{stroke,fill}%
}%
\begin{pgfscope}%
\pgfsys@transformshift{0.689445in}{2.780716in}%
\pgfsys@useobject{currentmarker}{}%
\end{pgfscope}%
\end{pgfscope}%
\begin{pgfscope}%
\pgfsetbuttcap%
\pgfsetroundjoin%
\definecolor{currentfill}{rgb}{0.000000,0.000000,0.000000}%
\pgfsetfillcolor{currentfill}%
\pgfsetlinewidth{0.602250pt}%
\definecolor{currentstroke}{rgb}{0.000000,0.000000,0.000000}%
\pgfsetstrokecolor{currentstroke}%
\pgfsetdash{}{0pt}%
\pgfsys@defobject{currentmarker}{\pgfqpoint{-0.027778in}{0.000000in}}{\pgfqpoint{-0.000000in}{0.000000in}}{%
\pgfpathmoveto{\pgfqpoint{-0.000000in}{0.000000in}}%
\pgfpathlineto{\pgfqpoint{-0.027778in}{0.000000in}}%
\pgfusepath{stroke,fill}%
}%
\begin{pgfscope}%
\pgfsys@transformshift{0.689445in}{2.882273in}%
\pgfsys@useobject{currentmarker}{}%
\end{pgfscope}%
\end{pgfscope}%
\begin{pgfscope}%
\pgfsetbuttcap%
\pgfsetroundjoin%
\definecolor{currentfill}{rgb}{0.000000,0.000000,0.000000}%
\pgfsetfillcolor{currentfill}%
\pgfsetlinewidth{0.602250pt}%
\definecolor{currentstroke}{rgb}{0.000000,0.000000,0.000000}%
\pgfsetstrokecolor{currentstroke}%
\pgfsetdash{}{0pt}%
\pgfsys@defobject{currentmarker}{\pgfqpoint{-0.027778in}{0.000000in}}{\pgfqpoint{-0.000000in}{0.000000in}}{%
\pgfpathmoveto{\pgfqpoint{-0.000000in}{0.000000in}}%
\pgfpathlineto{\pgfqpoint{-0.027778in}{0.000000in}}%
\pgfusepath{stroke,fill}%
}%
\begin{pgfscope}%
\pgfsys@transformshift{0.689445in}{2.933841in}%
\pgfsys@useobject{currentmarker}{}%
\end{pgfscope}%
\end{pgfscope}%
\begin{pgfscope}%
\pgfsetbuttcap%
\pgfsetroundjoin%
\definecolor{currentfill}{rgb}{0.000000,0.000000,0.000000}%
\pgfsetfillcolor{currentfill}%
\pgfsetlinewidth{0.602250pt}%
\definecolor{currentstroke}{rgb}{0.000000,0.000000,0.000000}%
\pgfsetstrokecolor{currentstroke}%
\pgfsetdash{}{0pt}%
\pgfsys@defobject{currentmarker}{\pgfqpoint{-0.027778in}{0.000000in}}{\pgfqpoint{-0.000000in}{0.000000in}}{%
\pgfpathmoveto{\pgfqpoint{-0.000000in}{0.000000in}}%
\pgfpathlineto{\pgfqpoint{-0.027778in}{0.000000in}}%
\pgfusepath{stroke,fill}%
}%
\begin{pgfscope}%
\pgfsys@transformshift{0.689445in}{2.970430in}%
\pgfsys@useobject{currentmarker}{}%
\end{pgfscope}%
\end{pgfscope}%
\begin{pgfscope}%
\pgfsetbuttcap%
\pgfsetroundjoin%
\definecolor{currentfill}{rgb}{0.000000,0.000000,0.000000}%
\pgfsetfillcolor{currentfill}%
\pgfsetlinewidth{0.602250pt}%
\definecolor{currentstroke}{rgb}{0.000000,0.000000,0.000000}%
\pgfsetstrokecolor{currentstroke}%
\pgfsetdash{}{0pt}%
\pgfsys@defobject{currentmarker}{\pgfqpoint{-0.027778in}{0.000000in}}{\pgfqpoint{-0.000000in}{0.000000in}}{%
\pgfpathmoveto{\pgfqpoint{-0.000000in}{0.000000in}}%
\pgfpathlineto{\pgfqpoint{-0.027778in}{0.000000in}}%
\pgfusepath{stroke,fill}%
}%
\begin{pgfscope}%
\pgfsys@transformshift{0.689445in}{2.998810in}%
\pgfsys@useobject{currentmarker}{}%
\end{pgfscope}%
\end{pgfscope}%
\begin{pgfscope}%
\pgfsetbuttcap%
\pgfsetroundjoin%
\definecolor{currentfill}{rgb}{0.000000,0.000000,0.000000}%
\pgfsetfillcolor{currentfill}%
\pgfsetlinewidth{0.602250pt}%
\definecolor{currentstroke}{rgb}{0.000000,0.000000,0.000000}%
\pgfsetstrokecolor{currentstroke}%
\pgfsetdash{}{0pt}%
\pgfsys@defobject{currentmarker}{\pgfqpoint{-0.027778in}{0.000000in}}{\pgfqpoint{-0.000000in}{0.000000in}}{%
\pgfpathmoveto{\pgfqpoint{-0.000000in}{0.000000in}}%
\pgfpathlineto{\pgfqpoint{-0.027778in}{0.000000in}}%
\pgfusepath{stroke,fill}%
}%
\begin{pgfscope}%
\pgfsys@transformshift{0.689445in}{3.021998in}%
\pgfsys@useobject{currentmarker}{}%
\end{pgfscope}%
\end{pgfscope}%
\begin{pgfscope}%
\pgfsetbuttcap%
\pgfsetroundjoin%
\definecolor{currentfill}{rgb}{0.000000,0.000000,0.000000}%
\pgfsetfillcolor{currentfill}%
\pgfsetlinewidth{0.602250pt}%
\definecolor{currentstroke}{rgb}{0.000000,0.000000,0.000000}%
\pgfsetstrokecolor{currentstroke}%
\pgfsetdash{}{0pt}%
\pgfsys@defobject{currentmarker}{\pgfqpoint{-0.027778in}{0.000000in}}{\pgfqpoint{-0.000000in}{0.000000in}}{%
\pgfpathmoveto{\pgfqpoint{-0.000000in}{0.000000in}}%
\pgfpathlineto{\pgfqpoint{-0.027778in}{0.000000in}}%
\pgfusepath{stroke,fill}%
}%
\begin{pgfscope}%
\pgfsys@transformshift{0.689445in}{3.041603in}%
\pgfsys@useobject{currentmarker}{}%
\end{pgfscope}%
\end{pgfscope}%
\begin{pgfscope}%
\pgfsetbuttcap%
\pgfsetroundjoin%
\definecolor{currentfill}{rgb}{0.000000,0.000000,0.000000}%
\pgfsetfillcolor{currentfill}%
\pgfsetlinewidth{0.602250pt}%
\definecolor{currentstroke}{rgb}{0.000000,0.000000,0.000000}%
\pgfsetstrokecolor{currentstroke}%
\pgfsetdash{}{0pt}%
\pgfsys@defobject{currentmarker}{\pgfqpoint{-0.027778in}{0.000000in}}{\pgfqpoint{-0.000000in}{0.000000in}}{%
\pgfpathmoveto{\pgfqpoint{-0.000000in}{0.000000in}}%
\pgfpathlineto{\pgfqpoint{-0.027778in}{0.000000in}}%
\pgfusepath{stroke,fill}%
}%
\begin{pgfscope}%
\pgfsys@transformshift{0.689445in}{3.058586in}%
\pgfsys@useobject{currentmarker}{}%
\end{pgfscope}%
\end{pgfscope}%
\begin{pgfscope}%
\pgfsetbuttcap%
\pgfsetroundjoin%
\definecolor{currentfill}{rgb}{0.000000,0.000000,0.000000}%
\pgfsetfillcolor{currentfill}%
\pgfsetlinewidth{0.602250pt}%
\definecolor{currentstroke}{rgb}{0.000000,0.000000,0.000000}%
\pgfsetstrokecolor{currentstroke}%
\pgfsetdash{}{0pt}%
\pgfsys@defobject{currentmarker}{\pgfqpoint{-0.027778in}{0.000000in}}{\pgfqpoint{-0.000000in}{0.000000in}}{%
\pgfpathmoveto{\pgfqpoint{-0.000000in}{0.000000in}}%
\pgfpathlineto{\pgfqpoint{-0.027778in}{0.000000in}}%
\pgfusepath{stroke,fill}%
}%
\begin{pgfscope}%
\pgfsys@transformshift{0.689445in}{3.073566in}%
\pgfsys@useobject{currentmarker}{}%
\end{pgfscope}%
\end{pgfscope}%
\begin{pgfscope}%
\pgfsetbuttcap%
\pgfsetroundjoin%
\definecolor{currentfill}{rgb}{0.000000,0.000000,0.000000}%
\pgfsetfillcolor{currentfill}%
\pgfsetlinewidth{0.602250pt}%
\definecolor{currentstroke}{rgb}{0.000000,0.000000,0.000000}%
\pgfsetstrokecolor{currentstroke}%
\pgfsetdash{}{0pt}%
\pgfsys@defobject{currentmarker}{\pgfqpoint{-0.027778in}{0.000000in}}{\pgfqpoint{-0.000000in}{0.000000in}}{%
\pgfpathmoveto{\pgfqpoint{-0.000000in}{0.000000in}}%
\pgfpathlineto{\pgfqpoint{-0.027778in}{0.000000in}}%
\pgfusepath{stroke,fill}%
}%
\begin{pgfscope}%
\pgfsys@transformshift{0.689445in}{3.175123in}%
\pgfsys@useobject{currentmarker}{}%
\end{pgfscope}%
\end{pgfscope}%
\begin{pgfscope}%
\definecolor{textcolor}{rgb}{0.000000,0.000000,0.000000}%
\pgfsetstrokecolor{textcolor}%
\pgfsetfillcolor{textcolor}%
\pgftext[x=0.280494in,y=1.865388in,,bottom,rotate=90.000000]{\color{textcolor}{\rmfamily\fontsize{10.000000}{12.000000}\selectfont\catcode`\^=\active\def^{\ifmmode\sp\else\^{}\fi}\catcode`\%=\active\def%{\%}Magnitude in \unit{\V \per \V}}}%
\end{pgfscope}%
\begin{pgfscope}%
\pgfpathrectangle{\pgfqpoint{0.689445in}{0.524170in}}{\pgfqpoint{4.106789in}{2.682436in}}%
\pgfusepath{clip}%
\pgfsetrectcap%
\pgfsetroundjoin%
\pgfsetlinewidth{1.505625pt}%
\definecolor{currentstroke}{rgb}{0.003922,0.450980,0.698039}%
\pgfsetstrokecolor{currentstroke}%
\pgfsetstrokeopacity{0.700000}%
\pgfsetdash{}{0pt}%
\pgfpathmoveto{\pgfqpoint{0.876117in}{3.078747in}}%
\pgfpathlineto{\pgfqpoint{1.333464in}{3.079893in}}%
\pgfpathlineto{\pgfqpoint{1.907481in}{3.084008in}}%
\pgfpathlineto{\pgfqpoint{2.308827in}{3.083444in}}%
\pgfpathlineto{\pgfqpoint{2.355495in}{3.080551in}}%
\pgfpathlineto{\pgfqpoint{2.388162in}{3.076532in}}%
\pgfpathlineto{\pgfqpoint{2.416163in}{3.071178in}}%
\pgfpathlineto{\pgfqpoint{2.444164in}{3.063624in}}%
\pgfpathlineto{\pgfqpoint{2.472165in}{3.053579in}}%
\pgfpathlineto{\pgfqpoint{2.500166in}{3.040929in}}%
\pgfpathlineto{\pgfqpoint{2.528166in}{3.025730in}}%
\pgfpathlineto{\pgfqpoint{2.556167in}{3.008158in}}%
\pgfpathlineto{\pgfqpoint{2.588835in}{2.984955in}}%
\pgfpathlineto{\pgfqpoint{2.621503in}{2.959168in}}%
\pgfpathlineto{\pgfqpoint{2.658837in}{2.926924in}}%
\pgfpathlineto{\pgfqpoint{2.700838in}{2.887616in}}%
\pgfpathlineto{\pgfqpoint{2.747506in}{2.840917in}}%
\pgfpathlineto{\pgfqpoint{2.812842in}{2.772033in}}%
\pgfpathlineto{\pgfqpoint{2.943512in}{2.633199in}}%
\pgfpathlineto{\pgfqpoint{2.994847in}{2.582163in}}%
\pgfpathlineto{\pgfqpoint{3.041515in}{2.538718in}}%
\pgfpathlineto{\pgfqpoint{3.088183in}{2.498293in}}%
\pgfpathlineto{\pgfqpoint{3.134851in}{2.460704in}}%
\pgfpathlineto{\pgfqpoint{3.186186in}{2.422097in}}%
\pgfpathlineto{\pgfqpoint{3.246854in}{2.379223in}}%
\pgfpathlineto{\pgfqpoint{3.326190in}{2.326013in}}%
\pgfpathlineto{\pgfqpoint{3.447527in}{2.247676in}}%
\pgfpathlineto{\pgfqpoint{3.792871in}{2.026132in}}%
\pgfpathlineto{\pgfqpoint{3.862873in}{1.978147in}}%
\pgfpathlineto{\pgfqpoint{3.914208in}{1.940489in}}%
\pgfpathlineto{\pgfqpoint{3.956209in}{1.906870in}}%
\pgfpathlineto{\pgfqpoint{3.988876in}{1.877827in}}%
\pgfpathlineto{\pgfqpoint{4.016877in}{1.849735in}}%
\pgfpathlineto{\pgfqpoint{4.040211in}{1.822859in}}%
\pgfpathlineto{\pgfqpoint{4.058879in}{1.797938in}}%
\pgfpathlineto{\pgfqpoint{4.072879in}{1.776337in}}%
\pgfpathlineto{\pgfqpoint{4.086879in}{1.751135in}}%
\pgfpathlineto{\pgfqpoint{4.100880in}{1.720597in}}%
\pgfpathlineto{\pgfqpoint{4.110213in}{1.695786in}}%
\pgfpathlineto{\pgfqpoint{4.119547in}{1.665523in}}%
\pgfpathlineto{\pgfqpoint{4.128881in}{1.626734in}}%
\pgfpathlineto{\pgfqpoint{4.138214in}{1.573441in}}%
\pgfpathlineto{\pgfqpoint{4.152215in}{1.466558in}}%
\pgfpathlineto{\pgfqpoint{4.156881in}{1.476762in}}%
\pgfpathlineto{\pgfqpoint{4.170882in}{1.586219in}}%
\pgfpathlineto{\pgfqpoint{4.180216in}{1.635445in}}%
\pgfpathlineto{\pgfqpoint{4.189549in}{1.671812in}}%
\pgfpathlineto{\pgfqpoint{4.198883in}{1.700489in}}%
\pgfpathlineto{\pgfqpoint{4.212883in}{1.734629in}}%
\pgfpathlineto{\pgfqpoint{4.226884in}{1.762013in}}%
\pgfpathlineto{\pgfqpoint{4.240884in}{1.785007in}}%
\pgfpathlineto{\pgfqpoint{4.259551in}{1.811073in}}%
\pgfpathlineto{\pgfqpoint{4.278218in}{1.833527in}}%
\pgfpathlineto{\pgfqpoint{4.301552in}{1.858141in}}%
\pgfpathlineto{\pgfqpoint{4.329553in}{1.884250in}}%
\pgfpathlineto{\pgfqpoint{4.362221in}{1.911627in}}%
\pgfpathlineto{\pgfqpoint{4.404222in}{1.943820in}}%
\pgfpathlineto{\pgfqpoint{4.460224in}{1.983839in}}%
\pgfpathlineto{\pgfqpoint{4.548893in}{2.044173in}}%
\pgfpathlineto{\pgfqpoint{4.609562in}{2.084427in}}%
\pgfpathlineto{\pgfqpoint{4.609562in}{2.084427in}}%
\pgfusepath{stroke}%
\end{pgfscope}%
\begin{pgfscope}%
\pgfpathrectangle{\pgfqpoint{0.689445in}{0.524170in}}{\pgfqpoint{4.106789in}{2.682436in}}%
\pgfusepath{clip}%
\pgfsetrectcap%
\pgfsetroundjoin%
\pgfsetlinewidth{1.505625pt}%
\definecolor{currentstroke}{rgb}{0.870588,0.560784,0.019608}%
\pgfsetstrokecolor{currentstroke}%
\pgfsetstrokeopacity{0.700000}%
\pgfsetdash{}{0pt}%
\pgfpathmoveto{\pgfqpoint{0.876117in}{3.074211in}}%
\pgfpathlineto{\pgfqpoint{1.100124in}{3.072697in}}%
\pgfpathlineto{\pgfqpoint{1.198127in}{3.070016in}}%
\pgfpathlineto{\pgfqpoint{1.268129in}{3.065934in}}%
\pgfpathlineto{\pgfqpoint{1.319464in}{3.060910in}}%
\pgfpathlineto{\pgfqpoint{1.366132in}{3.054171in}}%
\pgfpathlineto{\pgfqpoint{1.408133in}{3.045842in}}%
\pgfpathlineto{\pgfqpoint{1.445468in}{3.036341in}}%
\pgfpathlineto{\pgfqpoint{1.482802in}{3.024715in}}%
\pgfpathlineto{\pgfqpoint{1.520137in}{3.010959in}}%
\pgfpathlineto{\pgfqpoint{1.562138in}{2.993104in}}%
\pgfpathlineto{\pgfqpoint{1.608806in}{2.970718in}}%
\pgfpathlineto{\pgfqpoint{1.664808in}{2.941102in}}%
\pgfpathlineto{\pgfqpoint{1.734810in}{2.901211in}}%
\pgfpathlineto{\pgfqpoint{1.828146in}{2.845247in}}%
\pgfpathlineto{\pgfqpoint{2.000817in}{2.738632in}}%
\pgfpathlineto{\pgfqpoint{2.248158in}{2.586385in}}%
\pgfpathlineto{\pgfqpoint{2.346161in}{2.525493in}}%
\pgfpathlineto{\pgfqpoint{2.392829in}{2.494045in}}%
\pgfpathlineto{\pgfqpoint{2.430164in}{2.466423in}}%
\pgfpathlineto{\pgfqpoint{2.467498in}{2.435955in}}%
\pgfpathlineto{\pgfqpoint{2.504832in}{2.402522in}}%
\pgfpathlineto{\pgfqpoint{2.546834in}{2.361857in}}%
\pgfpathlineto{\pgfqpoint{2.598169in}{2.309019in}}%
\pgfpathlineto{\pgfqpoint{2.668171in}{2.233514in}}%
\pgfpathlineto{\pgfqpoint{2.752173in}{2.139506in}}%
\pgfpathlineto{\pgfqpoint{2.929512in}{1.939587in}}%
\pgfpathlineto{\pgfqpoint{2.985513in}{1.880678in}}%
\pgfpathlineto{\pgfqpoint{3.032181in}{1.834702in}}%
\pgfpathlineto{\pgfqpoint{3.078849in}{1.791931in}}%
\pgfpathlineto{\pgfqpoint{3.125518in}{1.752310in}}%
\pgfpathlineto{\pgfqpoint{3.176852in}{1.711888in}}%
\pgfpathlineto{\pgfqpoint{3.232854in}{1.670676in}}%
\pgfpathlineto{\pgfqpoint{3.307523in}{1.618678in}}%
\pgfpathlineto{\pgfqpoint{3.512862in}{1.480073in}}%
\pgfpathlineto{\pgfqpoint{3.694868in}{1.354899in}}%
\pgfpathlineto{\pgfqpoint{3.778870in}{1.294828in}}%
\pgfpathlineto{\pgfqpoint{3.834872in}{1.252045in}}%
\pgfpathlineto{\pgfqpoint{3.876873in}{1.217400in}}%
\pgfpathlineto{\pgfqpoint{3.914208in}{1.183937in}}%
\pgfpathlineto{\pgfqpoint{3.946875in}{1.151900in}}%
\pgfpathlineto{\pgfqpoint{3.974876in}{1.121776in}}%
\pgfpathlineto{\pgfqpoint{4.002877in}{1.088432in}}%
\pgfpathlineto{\pgfqpoint{4.026211in}{1.057362in}}%
\pgfpathlineto{\pgfqpoint{4.049545in}{1.022111in}}%
\pgfpathlineto{\pgfqpoint{4.068212in}{0.989567in}}%
\pgfpathlineto{\pgfqpoint{4.082213in}{0.961447in}}%
\pgfpathlineto{\pgfqpoint{4.096213in}{0.928620in}}%
\pgfpathlineto{\pgfqpoint{4.110213in}{0.888515in}}%
\pgfpathlineto{\pgfqpoint{4.119547in}{0.855326in}}%
\pgfpathlineto{\pgfqpoint{4.128881in}{0.813605in}}%
\pgfpathlineto{\pgfqpoint{4.138214in}{0.757378in}}%
\pgfpathlineto{\pgfqpoint{4.152215in}{0.646099in}}%
\pgfpathlineto{\pgfqpoint{4.156881in}{0.654840in}}%
\pgfpathlineto{\pgfqpoint{4.170882in}{0.759917in}}%
\pgfpathlineto{\pgfqpoint{4.180216in}{0.806232in}}%
\pgfpathlineto{\pgfqpoint{4.189549in}{0.839694in}}%
\pgfpathlineto{\pgfqpoint{4.198883in}{0.865472in}}%
\pgfpathlineto{\pgfqpoint{4.208216in}{0.886253in}}%
\pgfpathlineto{\pgfqpoint{4.222217in}{0.911209in}}%
\pgfpathlineto{\pgfqpoint{4.236217in}{0.931132in}}%
\pgfpathlineto{\pgfqpoint{4.250218in}{0.947608in}}%
\pgfpathlineto{\pgfqpoint{4.268885in}{0.965799in}}%
\pgfpathlineto{\pgfqpoint{4.287552in}{0.980879in}}%
\pgfpathlineto{\pgfqpoint{4.310886in}{0.996573in}}%
\pgfpathlineto{\pgfqpoint{4.334220in}{1.009662in}}%
\pgfpathlineto{\pgfqpoint{4.362221in}{1.022783in}}%
\pgfpathlineto{\pgfqpoint{4.394889in}{1.035386in}}%
\pgfpathlineto{\pgfqpoint{4.432223in}{1.047112in}}%
\pgfpathlineto{\pgfqpoint{4.474224in}{1.057791in}}%
\pgfpathlineto{\pgfqpoint{4.520892in}{1.067397in}}%
\pgfpathlineto{\pgfqpoint{4.576894in}{1.076666in}}%
\pgfpathlineto{\pgfqpoint{4.609562in}{1.081223in}}%
\pgfpathlineto{\pgfqpoint{4.609562in}{1.081223in}}%
\pgfusepath{stroke}%
\end{pgfscope}%
\begin{pgfscope}%
\pgfsetrectcap%
\pgfsetmiterjoin%
\pgfsetlinewidth{0.803000pt}%
\definecolor{currentstroke}{rgb}{0.000000,0.000000,0.000000}%
\pgfsetstrokecolor{currentstroke}%
\pgfsetdash{}{0pt}%
\pgfpathmoveto{\pgfqpoint{0.689445in}{0.524170in}}%
\pgfpathlineto{\pgfqpoint{0.689445in}{3.206606in}}%
\pgfusepath{stroke}%
\end{pgfscope}%
\begin{pgfscope}%
\pgfsetrectcap%
\pgfsetmiterjoin%
\pgfsetlinewidth{0.803000pt}%
\definecolor{currentstroke}{rgb}{0.000000,0.000000,0.000000}%
\pgfsetstrokecolor{currentstroke}%
\pgfsetdash{}{0pt}%
\pgfpathmoveto{\pgfqpoint{4.796234in}{0.524170in}}%
\pgfpathlineto{\pgfqpoint{4.796234in}{3.206606in}}%
\pgfusepath{stroke}%
\end{pgfscope}%
\begin{pgfscope}%
\pgfsetrectcap%
\pgfsetmiterjoin%
\pgfsetlinewidth{0.803000pt}%
\definecolor{currentstroke}{rgb}{0.000000,0.000000,0.000000}%
\pgfsetstrokecolor{currentstroke}%
\pgfsetdash{}{0pt}%
\pgfpathmoveto{\pgfqpoint{0.689445in}{0.524170in}}%
\pgfpathlineto{\pgfqpoint{4.796234in}{0.524170in}}%
\pgfusepath{stroke}%
\end{pgfscope}%
\begin{pgfscope}%
\pgfsetrectcap%
\pgfsetmiterjoin%
\pgfsetlinewidth{0.803000pt}%
\definecolor{currentstroke}{rgb}{0.000000,0.000000,0.000000}%
\pgfsetstrokecolor{currentstroke}%
\pgfsetdash{}{0pt}%
\pgfpathmoveto{\pgfqpoint{0.689445in}{3.206606in}}%
\pgfpathlineto{\pgfqpoint{4.796234in}{3.206606in}}%
\pgfusepath{stroke}%
\end{pgfscope}%
\begin{pgfscope}%
\pgfsetbuttcap%
\pgfsetroundjoin%
\definecolor{currentfill}{rgb}{0.000000,0.000000,0.000000}%
\pgfsetfillcolor{currentfill}%
\pgfsetlinewidth{0.803000pt}%
\definecolor{currentstroke}{rgb}{0.000000,0.000000,0.000000}%
\pgfsetstrokecolor{currentstroke}%
\pgfsetdash{}{0pt}%
\pgfsys@defobject{currentmarker}{\pgfqpoint{0.000000in}{0.000000in}}{\pgfqpoint{0.048611in}{0.000000in}}{%
\pgfpathmoveto{\pgfqpoint{0.000000in}{0.000000in}}%
\pgfpathlineto{\pgfqpoint{0.048611in}{0.000000in}}%
\pgfusepath{stroke,fill}%
}%
\begin{pgfscope}%
\pgfsys@transformshift{4.796234in}{0.642354in}%
\pgfsys@useobject{currentmarker}{}%
\end{pgfscope}%
\end{pgfscope}%
\begin{pgfscope}%
\definecolor{textcolor}{rgb}{0.000000,0.000000,0.000000}%
\pgfsetstrokecolor{textcolor}%
\pgfsetfillcolor{textcolor}%
\pgftext[x=4.893456in, y=0.603798in, left, base]{\color{textcolor}{\rmfamily\fontsize{8.000000}{9.600000}\selectfont\catcode`\^=\active\def^{\ifmmode\sp\else\^{}\fi}\catcode`\%=\active\def%{\%}$\mathdefault{0.00}$}}%
\end{pgfscope}%
\begin{pgfscope}%
\pgfsetbuttcap%
\pgfsetroundjoin%
\definecolor{currentfill}{rgb}{0.000000,0.000000,0.000000}%
\pgfsetfillcolor{currentfill}%
\pgfsetlinewidth{0.803000pt}%
\definecolor{currentstroke}{rgb}{0.000000,0.000000,0.000000}%
\pgfsetstrokecolor{currentstroke}%
\pgfsetdash{}{0pt}%
\pgfsys@defobject{currentmarker}{\pgfqpoint{0.000000in}{0.000000in}}{\pgfqpoint{0.048611in}{0.000000in}}{%
\pgfpathmoveto{\pgfqpoint{0.000000in}{0.000000in}}%
\pgfpathlineto{\pgfqpoint{0.048611in}{0.000000in}}%
\pgfusepath{stroke,fill}%
}%
\begin{pgfscope}%
\pgfsys@transformshift{4.796234in}{0.946483in}%
\pgfsys@useobject{currentmarker}{}%
\end{pgfscope}%
\end{pgfscope}%
\begin{pgfscope}%
\definecolor{textcolor}{rgb}{0.000000,0.000000,0.000000}%
\pgfsetstrokecolor{textcolor}%
\pgfsetfillcolor{textcolor}%
\pgftext[x=4.893456in, y=0.907928in, left, base]{\color{textcolor}{\rmfamily\fontsize{8.000000}{9.600000}\selectfont\catcode`\^=\active\def^{\ifmmode\sp\else\^{}\fi}\catcode`\%=\active\def%{\%}$\mathdefault{0.25}$}}%
\end{pgfscope}%
\begin{pgfscope}%
\pgfsetbuttcap%
\pgfsetroundjoin%
\definecolor{currentfill}{rgb}{0.000000,0.000000,0.000000}%
\pgfsetfillcolor{currentfill}%
\pgfsetlinewidth{0.803000pt}%
\definecolor{currentstroke}{rgb}{0.000000,0.000000,0.000000}%
\pgfsetstrokecolor{currentstroke}%
\pgfsetdash{}{0pt}%
\pgfsys@defobject{currentmarker}{\pgfqpoint{0.000000in}{0.000000in}}{\pgfqpoint{0.048611in}{0.000000in}}{%
\pgfpathmoveto{\pgfqpoint{0.000000in}{0.000000in}}%
\pgfpathlineto{\pgfqpoint{0.048611in}{0.000000in}}%
\pgfusepath{stroke,fill}%
}%
\begin{pgfscope}%
\pgfsys@transformshift{4.796234in}{1.250613in}%
\pgfsys@useobject{currentmarker}{}%
\end{pgfscope}%
\end{pgfscope}%
\begin{pgfscope}%
\definecolor{textcolor}{rgb}{0.000000,0.000000,0.000000}%
\pgfsetstrokecolor{textcolor}%
\pgfsetfillcolor{textcolor}%
\pgftext[x=4.893456in, y=1.212057in, left, base]{\color{textcolor}{\rmfamily\fontsize{8.000000}{9.600000}\selectfont\catcode`\^=\active\def^{\ifmmode\sp\else\^{}\fi}\catcode`\%=\active\def%{\%}$\mathdefault{0.50}$}}%
\end{pgfscope}%
\begin{pgfscope}%
\pgfsetbuttcap%
\pgfsetroundjoin%
\definecolor{currentfill}{rgb}{0.000000,0.000000,0.000000}%
\pgfsetfillcolor{currentfill}%
\pgfsetlinewidth{0.803000pt}%
\definecolor{currentstroke}{rgb}{0.000000,0.000000,0.000000}%
\pgfsetstrokecolor{currentstroke}%
\pgfsetdash{}{0pt}%
\pgfsys@defobject{currentmarker}{\pgfqpoint{0.000000in}{0.000000in}}{\pgfqpoint{0.048611in}{0.000000in}}{%
\pgfpathmoveto{\pgfqpoint{0.000000in}{0.000000in}}%
\pgfpathlineto{\pgfqpoint{0.048611in}{0.000000in}}%
\pgfusepath{stroke,fill}%
}%
\begin{pgfscope}%
\pgfsys@transformshift{4.796234in}{1.554742in}%
\pgfsys@useobject{currentmarker}{}%
\end{pgfscope}%
\end{pgfscope}%
\begin{pgfscope}%
\definecolor{textcolor}{rgb}{0.000000,0.000000,0.000000}%
\pgfsetstrokecolor{textcolor}%
\pgfsetfillcolor{textcolor}%
\pgftext[x=4.893456in, y=1.516186in, left, base]{\color{textcolor}{\rmfamily\fontsize{8.000000}{9.600000}\selectfont\catcode`\^=\active\def^{\ifmmode\sp\else\^{}\fi}\catcode`\%=\active\def%{\%}$\mathdefault{0.75}$}}%
\end{pgfscope}%
\begin{pgfscope}%
\pgfsetbuttcap%
\pgfsetroundjoin%
\definecolor{currentfill}{rgb}{0.000000,0.000000,0.000000}%
\pgfsetfillcolor{currentfill}%
\pgfsetlinewidth{0.803000pt}%
\definecolor{currentstroke}{rgb}{0.000000,0.000000,0.000000}%
\pgfsetstrokecolor{currentstroke}%
\pgfsetdash{}{0pt}%
\pgfsys@defobject{currentmarker}{\pgfqpoint{0.000000in}{0.000000in}}{\pgfqpoint{0.048611in}{0.000000in}}{%
\pgfpathmoveto{\pgfqpoint{0.000000in}{0.000000in}}%
\pgfpathlineto{\pgfqpoint{0.048611in}{0.000000in}}%
\pgfusepath{stroke,fill}%
}%
\begin{pgfscope}%
\pgfsys@transformshift{4.796234in}{1.858871in}%
\pgfsys@useobject{currentmarker}{}%
\end{pgfscope}%
\end{pgfscope}%
\begin{pgfscope}%
\definecolor{textcolor}{rgb}{0.000000,0.000000,0.000000}%
\pgfsetstrokecolor{textcolor}%
\pgfsetfillcolor{textcolor}%
\pgftext[x=4.893456in, y=1.820316in, left, base]{\color{textcolor}{\rmfamily\fontsize{8.000000}{9.600000}\selectfont\catcode`\^=\active\def^{\ifmmode\sp\else\^{}\fi}\catcode`\%=\active\def%{\%}$\mathdefault{1.00}$}}%
\end{pgfscope}%
\begin{pgfscope}%
\pgfsetbuttcap%
\pgfsetroundjoin%
\definecolor{currentfill}{rgb}{0.000000,0.000000,0.000000}%
\pgfsetfillcolor{currentfill}%
\pgfsetlinewidth{0.803000pt}%
\definecolor{currentstroke}{rgb}{0.000000,0.000000,0.000000}%
\pgfsetstrokecolor{currentstroke}%
\pgfsetdash{}{0pt}%
\pgfsys@defobject{currentmarker}{\pgfqpoint{0.000000in}{0.000000in}}{\pgfqpoint{0.048611in}{0.000000in}}{%
\pgfpathmoveto{\pgfqpoint{0.000000in}{0.000000in}}%
\pgfpathlineto{\pgfqpoint{0.048611in}{0.000000in}}%
\pgfusepath{stroke,fill}%
}%
\begin{pgfscope}%
\pgfsys@transformshift{4.796234in}{2.163001in}%
\pgfsys@useobject{currentmarker}{}%
\end{pgfscope}%
\end{pgfscope}%
\begin{pgfscope}%
\definecolor{textcolor}{rgb}{0.000000,0.000000,0.000000}%
\pgfsetstrokecolor{textcolor}%
\pgfsetfillcolor{textcolor}%
\pgftext[x=4.893456in, y=2.124445in, left, base]{\color{textcolor}{\rmfamily\fontsize{8.000000}{9.600000}\selectfont\catcode`\^=\active\def^{\ifmmode\sp\else\^{}\fi}\catcode`\%=\active\def%{\%}$\mathdefault{1.25}$}}%
\end{pgfscope}%
\begin{pgfscope}%
\pgfsetbuttcap%
\pgfsetroundjoin%
\definecolor{currentfill}{rgb}{0.000000,0.000000,0.000000}%
\pgfsetfillcolor{currentfill}%
\pgfsetlinewidth{0.803000pt}%
\definecolor{currentstroke}{rgb}{0.000000,0.000000,0.000000}%
\pgfsetstrokecolor{currentstroke}%
\pgfsetdash{}{0pt}%
\pgfsys@defobject{currentmarker}{\pgfqpoint{0.000000in}{0.000000in}}{\pgfqpoint{0.048611in}{0.000000in}}{%
\pgfpathmoveto{\pgfqpoint{0.000000in}{0.000000in}}%
\pgfpathlineto{\pgfqpoint{0.048611in}{0.000000in}}%
\pgfusepath{stroke,fill}%
}%
\begin{pgfscope}%
\pgfsys@transformshift{4.796234in}{2.467130in}%
\pgfsys@useobject{currentmarker}{}%
\end{pgfscope}%
\end{pgfscope}%
\begin{pgfscope}%
\definecolor{textcolor}{rgb}{0.000000,0.000000,0.000000}%
\pgfsetstrokecolor{textcolor}%
\pgfsetfillcolor{textcolor}%
\pgftext[x=4.893456in, y=2.428575in, left, base]{\color{textcolor}{\rmfamily\fontsize{8.000000}{9.600000}\selectfont\catcode`\^=\active\def^{\ifmmode\sp\else\^{}\fi}\catcode`\%=\active\def%{\%}$\mathdefault{1.50}$}}%
\end{pgfscope}%
\begin{pgfscope}%
\pgfsetbuttcap%
\pgfsetroundjoin%
\definecolor{currentfill}{rgb}{0.000000,0.000000,0.000000}%
\pgfsetfillcolor{currentfill}%
\pgfsetlinewidth{0.803000pt}%
\definecolor{currentstroke}{rgb}{0.000000,0.000000,0.000000}%
\pgfsetstrokecolor{currentstroke}%
\pgfsetdash{}{0pt}%
\pgfsys@defobject{currentmarker}{\pgfqpoint{0.000000in}{0.000000in}}{\pgfqpoint{0.048611in}{0.000000in}}{%
\pgfpathmoveto{\pgfqpoint{0.000000in}{0.000000in}}%
\pgfpathlineto{\pgfqpoint{0.048611in}{0.000000in}}%
\pgfusepath{stroke,fill}%
}%
\begin{pgfscope}%
\pgfsys@transformshift{4.796234in}{2.771259in}%
\pgfsys@useobject{currentmarker}{}%
\end{pgfscope}%
\end{pgfscope}%
\begin{pgfscope}%
\definecolor{textcolor}{rgb}{0.000000,0.000000,0.000000}%
\pgfsetstrokecolor{textcolor}%
\pgfsetfillcolor{textcolor}%
\pgftext[x=4.893456in, y=2.732704in, left, base]{\color{textcolor}{\rmfamily\fontsize{8.000000}{9.600000}\selectfont\catcode`\^=\active\def^{\ifmmode\sp\else\^{}\fi}\catcode`\%=\active\def%{\%}$\mathdefault{1.75}$}}%
\end{pgfscope}%
\begin{pgfscope}%
\pgfsetbuttcap%
\pgfsetroundjoin%
\definecolor{currentfill}{rgb}{0.000000,0.000000,0.000000}%
\pgfsetfillcolor{currentfill}%
\pgfsetlinewidth{0.803000pt}%
\definecolor{currentstroke}{rgb}{0.000000,0.000000,0.000000}%
\pgfsetstrokecolor{currentstroke}%
\pgfsetdash{}{0pt}%
\pgfsys@defobject{currentmarker}{\pgfqpoint{0.000000in}{0.000000in}}{\pgfqpoint{0.048611in}{0.000000in}}{%
\pgfpathmoveto{\pgfqpoint{0.000000in}{0.000000in}}%
\pgfpathlineto{\pgfqpoint{0.048611in}{0.000000in}}%
\pgfusepath{stroke,fill}%
}%
\begin{pgfscope}%
\pgfsys@transformshift{4.796234in}{3.075389in}%
\pgfsys@useobject{currentmarker}{}%
\end{pgfscope}%
\end{pgfscope}%
\begin{pgfscope}%
\definecolor{textcolor}{rgb}{0.000000,0.000000,0.000000}%
\pgfsetstrokecolor{textcolor}%
\pgfsetfillcolor{textcolor}%
\pgftext[x=4.893456in, y=3.036833in, left, base]{\color{textcolor}{\rmfamily\fontsize{8.000000}{9.600000}\selectfont\catcode`\^=\active\def^{\ifmmode\sp\else\^{}\fi}\catcode`\%=\active\def%{\%}$\mathdefault{2.00}$}}%
\end{pgfscope}%
\begin{pgfscope}%
\definecolor{textcolor}{rgb}{0.000000,0.000000,0.000000}%
\pgfsetstrokecolor{textcolor}%
\pgfsetfillcolor{textcolor}%
\pgftext[x=5.158891in,y=1.865388in,,top,rotate=90.000000]{\color{textcolor}{\rmfamily\fontsize{10.000000}{12.000000}\selectfont\catcode`\^=\active\def^{\ifmmode\sp\else\^{}\fi}\catcode`\%=\active\def%{\%}Impedance in \unit{\ohm}}}%
\end{pgfscope}%
\begin{pgfscope}%
\pgfpathrectangle{\pgfqpoint{0.689445in}{0.524170in}}{\pgfqpoint{4.106789in}{2.682436in}}%
\pgfusepath{clip}%
\pgfsetbuttcap%
\pgfsetroundjoin%
\pgfsetlinewidth{1.505625pt}%
\definecolor{currentstroke}{rgb}{0.007843,0.619608,0.450980}%
\pgfsetstrokecolor{currentstroke}%
\pgfsetstrokeopacity{0.700000}%
\pgfsetdash{{5.550000pt}{2.400000pt}}{0.000000pt}%
\pgfpathmoveto{\pgfqpoint{0.876117in}{2.004251in}}%
\pgfpathlineto{\pgfqpoint{1.104791in}{2.005734in}}%
\pgfpathlineto{\pgfqpoint{1.207461in}{2.008434in}}%
\pgfpathlineto{\pgfqpoint{1.282130in}{2.012507in}}%
\pgfpathlineto{\pgfqpoint{1.347465in}{2.018307in}}%
\pgfpathlineto{\pgfqpoint{1.412800in}{2.026347in}}%
\pgfpathlineto{\pgfqpoint{1.604139in}{2.051623in}}%
\pgfpathlineto{\pgfqpoint{1.674141in}{2.057278in}}%
\pgfpathlineto{\pgfqpoint{1.772144in}{2.062573in}}%
\pgfpathlineto{\pgfqpoint{1.884147in}{2.068929in}}%
\pgfpathlineto{\pgfqpoint{1.940149in}{2.074354in}}%
\pgfpathlineto{\pgfqpoint{1.982150in}{2.080595in}}%
\pgfpathlineto{\pgfqpoint{2.014818in}{2.087443in}}%
\pgfpathlineto{\pgfqpoint{2.042819in}{2.095244in}}%
\pgfpathlineto{\pgfqpoint{2.070820in}{2.105398in}}%
\pgfpathlineto{\pgfqpoint{2.094154in}{2.116149in}}%
\pgfpathlineto{\pgfqpoint{2.117488in}{2.129495in}}%
\pgfpathlineto{\pgfqpoint{2.136155in}{2.142426in}}%
\pgfpathlineto{\pgfqpoint{2.154822in}{2.157722in}}%
\pgfpathlineto{\pgfqpoint{2.173489in}{2.175760in}}%
\pgfpathlineto{\pgfqpoint{2.192157in}{2.196946in}}%
\pgfpathlineto{\pgfqpoint{2.210824in}{2.221713in}}%
\pgfpathlineto{\pgfqpoint{2.229491in}{2.250492in}}%
\pgfpathlineto{\pgfqpoint{2.248158in}{2.283693in}}%
\pgfpathlineto{\pgfqpoint{2.266825in}{2.321657in}}%
\pgfpathlineto{\pgfqpoint{2.285493in}{2.364596in}}%
\pgfpathlineto{\pgfqpoint{2.304160in}{2.412516in}}%
\pgfpathlineto{\pgfqpoint{2.327494in}{2.478932in}}%
\pgfpathlineto{\pgfqpoint{2.355495in}{2.566068in}}%
\pgfpathlineto{\pgfqpoint{2.411496in}{2.742583in}}%
\pgfpathlineto{\pgfqpoint{2.430164in}{2.792902in}}%
\pgfpathlineto{\pgfqpoint{2.444164in}{2.824899in}}%
\pgfpathlineto{\pgfqpoint{2.458164in}{2.850774in}}%
\pgfpathlineto{\pgfqpoint{2.467498in}{2.864180in}}%
\pgfpathlineto{\pgfqpoint{2.476832in}{2.874271in}}%
\pgfpathlineto{\pgfqpoint{2.486165in}{2.880896in}}%
\pgfpathlineto{\pgfqpoint{2.495499in}{2.883963in}}%
\pgfpathlineto{\pgfqpoint{2.504832in}{2.883427in}}%
\pgfpathlineto{\pgfqpoint{2.514166in}{2.879296in}}%
\pgfpathlineto{\pgfqpoint{2.523500in}{2.871619in}}%
\pgfpathlineto{\pgfqpoint{2.532833in}{2.860482in}}%
\pgfpathlineto{\pgfqpoint{2.542167in}{2.846004in}}%
\pgfpathlineto{\pgfqpoint{2.551501in}{2.828325in}}%
\pgfpathlineto{\pgfqpoint{2.565501in}{2.796159in}}%
\pgfpathlineto{\pgfqpoint{2.579501in}{2.757743in}}%
\pgfpathlineto{\pgfqpoint{2.598169in}{2.697879in}}%
\pgfpathlineto{\pgfqpoint{2.616836in}{2.629526in}}%
\pgfpathlineto{\pgfqpoint{2.640170in}{2.534448in}}%
\pgfpathlineto{\pgfqpoint{2.668171in}{2.410045in}}%
\pgfpathlineto{\pgfqpoint{2.714839in}{2.190110in}}%
\pgfpathlineto{\pgfqpoint{2.766174in}{1.950498in}}%
\pgfpathlineto{\pgfqpoint{2.798841in}{1.808483in}}%
\pgfpathlineto{\pgfqpoint{2.826842in}{1.696530in}}%
\pgfpathlineto{\pgfqpoint{2.850176in}{1.611145in}}%
\pgfpathlineto{\pgfqpoint{2.873510in}{1.533368in}}%
\pgfpathlineto{\pgfqpoint{2.896844in}{1.463304in}}%
\pgfpathlineto{\pgfqpoint{2.920178in}{1.400843in}}%
\pgfpathlineto{\pgfqpoint{2.943512in}{1.345709in}}%
\pgfpathlineto{\pgfqpoint{2.962179in}{1.306613in}}%
\pgfpathlineto{\pgfqpoint{2.980847in}{1.271710in}}%
\pgfpathlineto{\pgfqpoint{2.999514in}{1.240731in}}%
\pgfpathlineto{\pgfqpoint{3.018181in}{1.213389in}}%
\pgfpathlineto{\pgfqpoint{3.036848in}{1.189392in}}%
\pgfpathlineto{\pgfqpoint{3.055515in}{1.168440in}}%
\pgfpathlineto{\pgfqpoint{3.074183in}{1.150242in}}%
\pgfpathlineto{\pgfqpoint{3.097517in}{1.130930in}}%
\pgfpathlineto{\pgfqpoint{3.120851in}{1.114944in}}%
\pgfpathlineto{\pgfqpoint{3.144185in}{1.101795in}}%
\pgfpathlineto{\pgfqpoint{3.167519in}{1.091039in}}%
\pgfpathlineto{\pgfqpoint{3.195520in}{1.080743in}}%
\pgfpathlineto{\pgfqpoint{3.223520in}{1.072748in}}%
\pgfpathlineto{\pgfqpoint{3.256188in}{1.065688in}}%
\pgfpathlineto{\pgfqpoint{3.293523in}{1.059867in}}%
\pgfpathlineto{\pgfqpoint{3.340191in}{1.054969in}}%
\pgfpathlineto{\pgfqpoint{3.400859in}{1.051134in}}%
\pgfpathlineto{\pgfqpoint{3.480195in}{1.048526in}}%
\pgfpathlineto{\pgfqpoint{3.610865in}{1.046745in}}%
\pgfpathlineto{\pgfqpoint{3.928208in}{1.042997in}}%
\pgfpathlineto{\pgfqpoint{4.128881in}{1.037386in}}%
\pgfpathlineto{\pgfqpoint{4.287552in}{1.031023in}}%
\pgfpathlineto{\pgfqpoint{4.394889in}{1.027352in}}%
\pgfpathlineto{\pgfqpoint{4.464891in}{1.027343in}}%
\pgfpathlineto{\pgfqpoint{4.539560in}{1.029820in}}%
\pgfpathlineto{\pgfqpoint{4.609562in}{1.033375in}}%
\pgfpathlineto{\pgfqpoint{4.609562in}{1.033375in}}%
\pgfusepath{stroke}%
\end{pgfscope}%
\begin{pgfscope}%
\pgfpathrectangle{\pgfqpoint{0.689445in}{0.524170in}}{\pgfqpoint{4.106789in}{2.682436in}}%
\pgfusepath{clip}%
\pgfsetbuttcap%
\pgfsetroundjoin%
\pgfsetlinewidth{1.505625pt}%
\definecolor{currentstroke}{rgb}{0.800000,0.470588,0.737255}%
\pgfsetstrokecolor{currentstroke}%
\pgfsetstrokeopacity{0.700000}%
\pgfsetdash{{5.550000pt}{2.400000pt}}{0.000000pt}%
\pgfpathmoveto{\pgfqpoint{0.876117in}{3.084677in}}%
\pgfpathlineto{\pgfqpoint{0.941453in}{3.080878in}}%
\pgfpathlineto{\pgfqpoint{0.992788in}{3.075634in}}%
\pgfpathlineto{\pgfqpoint{1.030122in}{3.069797in}}%
\pgfpathlineto{\pgfqpoint{1.062790in}{3.062650in}}%
\pgfpathlineto{\pgfqpoint{1.090790in}{3.054474in}}%
\pgfpathlineto{\pgfqpoint{1.118791in}{3.043821in}}%
\pgfpathlineto{\pgfqpoint{1.142125in}{3.032551in}}%
\pgfpathlineto{\pgfqpoint{1.165459in}{3.018586in}}%
\pgfpathlineto{\pgfqpoint{1.184127in}{3.005089in}}%
\pgfpathlineto{\pgfqpoint{1.202794in}{2.989163in}}%
\pgfpathlineto{\pgfqpoint{1.221461in}{2.970436in}}%
\pgfpathlineto{\pgfqpoint{1.240128in}{2.948505in}}%
\pgfpathlineto{\pgfqpoint{1.258795in}{2.922941in}}%
\pgfpathlineto{\pgfqpoint{1.277463in}{2.893304in}}%
\pgfpathlineto{\pgfqpoint{1.296130in}{2.859160in}}%
\pgfpathlineto{\pgfqpoint{1.314797in}{2.820098in}}%
\pgfpathlineto{\pgfqpoint{1.333464in}{2.775764in}}%
\pgfpathlineto{\pgfqpoint{1.352132in}{2.725889in}}%
\pgfpathlineto{\pgfqpoint{1.370799in}{2.670324in}}%
\pgfpathlineto{\pgfqpoint{1.394133in}{2.592880in}}%
\pgfpathlineto{\pgfqpoint{1.417467in}{2.506948in}}%
\pgfpathlineto{\pgfqpoint{1.445468in}{2.393859in}}%
\pgfpathlineto{\pgfqpoint{1.478135in}{2.251172in}}%
\pgfpathlineto{\pgfqpoint{1.534137in}{1.993372in}}%
\pgfpathlineto{\pgfqpoint{1.580805in}{1.781957in}}%
\pgfpathlineto{\pgfqpoint{1.613473in}{1.643102in}}%
\pgfpathlineto{\pgfqpoint{1.641474in}{1.532578in}}%
\pgfpathlineto{\pgfqpoint{1.669474in}{1.430949in}}%
\pgfpathlineto{\pgfqpoint{1.697475in}{1.338620in}}%
\pgfpathlineto{\pgfqpoint{1.720809in}{1.268772in}}%
\pgfpathlineto{\pgfqpoint{1.744143in}{1.205179in}}%
\pgfpathlineto{\pgfqpoint{1.767477in}{1.147552in}}%
\pgfpathlineto{\pgfqpoint{1.790811in}{1.095539in}}%
\pgfpathlineto{\pgfqpoint{1.814145in}{1.048750in}}%
\pgfpathlineto{\pgfqpoint{1.837479in}{1.006784in}}%
\pgfpathlineto{\pgfqpoint{1.860813in}{0.969244in}}%
\pgfpathlineto{\pgfqpoint{1.884147in}{0.935744in}}%
\pgfpathlineto{\pgfqpoint{1.907481in}{0.905920in}}%
\pgfpathlineto{\pgfqpoint{1.930815in}{0.879431in}}%
\pgfpathlineto{\pgfqpoint{1.954149in}{0.855964in}}%
\pgfpathlineto{\pgfqpoint{1.977483in}{0.835229in}}%
\pgfpathlineto{\pgfqpoint{2.005484in}{0.813585in}}%
\pgfpathlineto{\pgfqpoint{2.033485in}{0.795087in}}%
\pgfpathlineto{\pgfqpoint{2.061486in}{0.779367in}}%
\pgfpathlineto{\pgfqpoint{2.089487in}{0.766094in}}%
\pgfpathlineto{\pgfqpoint{2.122154in}{0.753304in}}%
\pgfpathlineto{\pgfqpoint{2.154822in}{0.743003in}}%
\pgfpathlineto{\pgfqpoint{2.192157in}{0.733775in}}%
\pgfpathlineto{\pgfqpoint{2.234158in}{0.725996in}}%
\pgfpathlineto{\pgfqpoint{2.280826in}{0.719780in}}%
\pgfpathlineto{\pgfqpoint{2.341494in}{0.714208in}}%
\pgfpathlineto{\pgfqpoint{2.439497in}{0.707926in}}%
\pgfpathlineto{\pgfqpoint{2.565501in}{0.701904in}}%
\pgfpathlineto{\pgfqpoint{2.686838in}{0.698608in}}%
\pgfpathlineto{\pgfqpoint{2.910845in}{0.695323in}}%
\pgfpathlineto{\pgfqpoint{3.265522in}{0.691542in}}%
\pgfpathlineto{\pgfqpoint{3.755536in}{0.688611in}}%
\pgfpathlineto{\pgfqpoint{3.853539in}{0.685317in}}%
\pgfpathlineto{\pgfqpoint{3.937542in}{0.680246in}}%
\pgfpathlineto{\pgfqpoint{4.035545in}{0.671915in}}%
\pgfpathlineto{\pgfqpoint{4.189549in}{0.658747in}}%
\pgfpathlineto{\pgfqpoint{4.282885in}{0.653161in}}%
\pgfpathlineto{\pgfqpoint{4.385555in}{0.649290in}}%
\pgfpathlineto{\pgfqpoint{4.511559in}{0.646899in}}%
\pgfpathlineto{\pgfqpoint{4.609562in}{0.646099in}}%
\pgfpathlineto{\pgfqpoint{4.609562in}{0.646099in}}%
\pgfusepath{stroke}%
\end{pgfscope}%
\begin{pgfscope}%
\pgfsetrectcap%
\pgfsetmiterjoin%
\pgfsetlinewidth{0.803000pt}%
\definecolor{currentstroke}{rgb}{0.000000,0.000000,0.000000}%
\pgfsetstrokecolor{currentstroke}%
\pgfsetdash{}{0pt}%
\pgfpathmoveto{\pgfqpoint{0.689445in}{0.524170in}}%
\pgfpathlineto{\pgfqpoint{0.689445in}{3.206606in}}%
\pgfusepath{stroke}%
\end{pgfscope}%
\begin{pgfscope}%
\pgfsetrectcap%
\pgfsetmiterjoin%
\pgfsetlinewidth{0.803000pt}%
\definecolor{currentstroke}{rgb}{0.000000,0.000000,0.000000}%
\pgfsetstrokecolor{currentstroke}%
\pgfsetdash{}{0pt}%
\pgfpathmoveto{\pgfqpoint{4.796234in}{0.524170in}}%
\pgfpathlineto{\pgfqpoint{4.796234in}{3.206606in}}%
\pgfusepath{stroke}%
\end{pgfscope}%
\begin{pgfscope}%
\pgfsetrectcap%
\pgfsetmiterjoin%
\pgfsetlinewidth{0.803000pt}%
\definecolor{currentstroke}{rgb}{0.000000,0.000000,0.000000}%
\pgfsetstrokecolor{currentstroke}%
\pgfsetdash{}{0pt}%
\pgfpathmoveto{\pgfqpoint{0.689445in}{0.524170in}}%
\pgfpathlineto{\pgfqpoint{4.796234in}{0.524170in}}%
\pgfusepath{stroke}%
\end{pgfscope}%
\begin{pgfscope}%
\pgfsetrectcap%
\pgfsetmiterjoin%
\pgfsetlinewidth{0.803000pt}%
\definecolor{currentstroke}{rgb}{0.000000,0.000000,0.000000}%
\pgfsetstrokecolor{currentstroke}%
\pgfsetdash{}{0pt}%
\pgfpathmoveto{\pgfqpoint{0.689445in}{3.206606in}}%
\pgfpathlineto{\pgfqpoint{4.796234in}{3.206606in}}%
\pgfusepath{stroke}%
\end{pgfscope}%
\begin{pgfscope}%
\pgfsetbuttcap%
\pgfsetmiterjoin%
\definecolor{currentfill}{rgb}{1.000000,1.000000,1.000000}%
\pgfsetfillcolor{currentfill}%
\pgfsetfillopacity{0.800000}%
\pgfsetlinewidth{1.003750pt}%
\definecolor{currentstroke}{rgb}{0.800000,0.800000,0.800000}%
\pgfsetstrokecolor{currentstroke}%
\pgfsetstrokeopacity{0.800000}%
\pgfsetdash{}{0pt}%
\pgfpathmoveto{\pgfqpoint{3.312456in}{2.494829in}}%
\pgfpathlineto{\pgfqpoint{4.718456in}{2.494829in}}%
\pgfpathquadraticcurveto{\pgfqpoint{4.740678in}{2.494829in}}{\pgfqpoint{4.740678in}{2.517051in}}%
\pgfpathlineto{\pgfqpoint{4.740678in}{3.128828in}}%
\pgfpathquadraticcurveto{\pgfqpoint{4.740678in}{3.151050in}}{\pgfqpoint{4.718456in}{3.151050in}}%
\pgfpathlineto{\pgfqpoint{3.312456in}{3.151050in}}%
\pgfpathquadraticcurveto{\pgfqpoint{3.290234in}{3.151050in}}{\pgfqpoint{3.290234in}{3.128828in}}%
\pgfpathlineto{\pgfqpoint{3.290234in}{2.517051in}}%
\pgfpathquadraticcurveto{\pgfqpoint{3.290234in}{2.494829in}}{\pgfqpoint{3.312456in}{2.494829in}}%
\pgfpathlineto{\pgfqpoint{3.312456in}{2.494829in}}%
\pgfpathclose%
\pgfusepath{stroke,fill}%
\end{pgfscope}%
\begin{pgfscope}%
\pgfsetrectcap%
\pgfsetroundjoin%
\pgfsetlinewidth{1.505625pt}%
\definecolor{currentstroke}{rgb}{0.003922,0.450980,0.698039}%
\pgfsetstrokecolor{currentstroke}%
\pgfsetstrokeopacity{0.700000}%
\pgfsetdash{}{0pt}%
\pgfpathmoveto{\pgfqpoint{3.334678in}{3.067273in}}%
\pgfpathlineto{\pgfqpoint{3.445789in}{3.067273in}}%
\pgfpathlineto{\pgfqpoint{3.556900in}{3.067273in}}%
\pgfusepath{stroke}%
\end{pgfscope}%
\begin{pgfscope}%
\definecolor{textcolor}{rgb}{0.000000,0.000000,0.000000}%
\pgfsetstrokecolor{textcolor}%
\pgfsetfillcolor{textcolor}%
\pgftext[x=3.645789in,y=3.028384in,left,base]{\color{textcolor}{\rmfamily\fontsize{8.000000}{9.600000}\selectfont\catcode`\^=\active\def^{\ifmmode\sp\else\^{}\fi}\catcode`\%=\active\def%{\%}Mag. LC only}}%
\end{pgfscope}%
\begin{pgfscope}%
\pgfsetrectcap%
\pgfsetroundjoin%
\pgfsetlinewidth{1.505625pt}%
\definecolor{currentstroke}{rgb}{0.870588,0.560784,0.019608}%
\pgfsetstrokecolor{currentstroke}%
\pgfsetstrokeopacity{0.700000}%
\pgfsetdash{}{0pt}%
\pgfpathmoveto{\pgfqpoint{3.334678in}{2.910717in}}%
\pgfpathlineto{\pgfqpoint{3.445789in}{2.910717in}}%
\pgfpathlineto{\pgfqpoint{3.556900in}{2.910717in}}%
\pgfusepath{stroke}%
\end{pgfscope}%
\begin{pgfscope}%
\definecolor{textcolor}{rgb}{0.000000,0.000000,0.000000}%
\pgfsetstrokecolor{textcolor}%
\pgfsetfillcolor{textcolor}%
\pgftext[x=3.645789in,y=2.871828in,left,base]{\color{textcolor}{\rmfamily\fontsize{8.000000}{9.600000}\selectfont\catcode`\^=\active\def^{\ifmmode\sp\else\^{}\fi}\catcode`\%=\active\def%{\%}Mag. LC + C Mult.}}%
\end{pgfscope}%
\begin{pgfscope}%
\pgfsetbuttcap%
\pgfsetroundjoin%
\pgfsetlinewidth{1.505625pt}%
\definecolor{currentstroke}{rgb}{0.007843,0.619608,0.450980}%
\pgfsetstrokecolor{currentstroke}%
\pgfsetstrokeopacity{0.700000}%
\pgfsetdash{{5.550000pt}{2.400000pt}}{0.000000pt}%
\pgfpathmoveto{\pgfqpoint{3.334678in}{2.754607in}}%
\pgfpathlineto{\pgfqpoint{3.445789in}{2.754607in}}%
\pgfpathlineto{\pgfqpoint{3.556900in}{2.754607in}}%
\pgfusepath{stroke}%
\end{pgfscope}%
\begin{pgfscope}%
\definecolor{textcolor}{rgb}{0.000000,0.000000,0.000000}%
\pgfsetstrokecolor{textcolor}%
\pgfsetfillcolor{textcolor}%
\pgftext[x=3.645789in,y=2.715718in,left,base]{\color{textcolor}{\rmfamily\fontsize{8.000000}{9.600000}\selectfont\catcode`\^=\active\def^{\ifmmode\sp\else\^{}\fi}\catcode`\%=\active\def%{\%}$Z_{out}$ LC filter}}%
\end{pgfscope}%
\begin{pgfscope}%
\pgfsetbuttcap%
\pgfsetroundjoin%
\pgfsetlinewidth{1.505625pt}%
\definecolor{currentstroke}{rgb}{0.800000,0.470588,0.737255}%
\pgfsetstrokecolor{currentstroke}%
\pgfsetstrokeopacity{0.700000}%
\pgfsetdash{{5.550000pt}{2.400000pt}}{0.000000pt}%
\pgfpathmoveto{\pgfqpoint{3.334678in}{2.599718in}}%
\pgfpathlineto{\pgfqpoint{3.445789in}{2.599718in}}%
\pgfpathlineto{\pgfqpoint{3.556900in}{2.599718in}}%
\pgfusepath{stroke}%
\end{pgfscope}%
\begin{pgfscope}%
\definecolor{textcolor}{rgb}{0.000000,0.000000,0.000000}%
\pgfsetstrokecolor{textcolor}%
\pgfsetfillcolor{textcolor}%
\pgftext[x=3.645789in,y=2.560829in,left,base]{\color{textcolor}{\rmfamily\fontsize{8.000000}{9.600000}\selectfont\catcode`\^=\active\def^{\ifmmode\sp\else\^{}\fi}\catcode`\%=\active\def%{\%}$Z_{out}$ C Mult.}}%
\end{pgfscope}%
\end{pgfpicture}%
\makeatother%
\endgroup%

    \caption{Simulated response of input filter used in the digital current driver. Both magnitude and output impedance of the stages are shown.}
    \label{fig:laser_driver_input_filter}
\end{figure}

At \qty{300}{\Hz}, the LC filter cutoff frequency, the output imedance of the LC filter shows some gain peaking. This is due to the underdamped response chosen and discussed above. This peaking increases the output impeance of \qty{1.2}{\ohm} in the passband, which is mostly the resistance of the inductor, to a total of \qty{1.8}{\ohm}. This can be easily compensated for by the regulators downstream.

The rejection ratio of the LC filter and the capacitance multiplier is better than \num{e3} at at \qty{100}{\kHz} an above, delivering the performance estimated above. This is expected to keep switch-mode noise away from the laser driver current.

The high rejection ratio of the filter is expected to make the the experimental validation rather challenging, because there are a number of complications that derive from the active nature of the circuit. The capacitance multiplier must be loaded, preferably at the maximum current to show the worst case and additionally, the ripple voltage must be low enough to not saturate the capacitance multiplier.

This requires a highly sensitive VNA, that has a low frequency range. This setup uses an Omicron Lab \device{Bode 100}, which can measure from \qty{1}{\Hz} to \qty{50}{\MHz} with an exceptionally low noise floor of about \qty{180}{\nV \Hz\tothe{-0.5}} \cite{datasheet_bode100}. Additionally a Stanford Research \device{SR560} was used as a pre-amplifier. To apply the ripple voltage to the power supply rails a Picotest \device{J2123A} negative line injector and a self-designed positive line injector was used. The positive line injector design is available open-source and be found in a Github repository at \cite{line_injector_github}. This injector is called \device{PB02}. During the measurement, it was found, that since the expected signal is extremely small, ground currents became an issue. There is an inherent ground loop issue built into the VNA. The outputs and inputs of the \device{Bode 100} are not isolated. The measrement is a 3-port measurement as shown in figure \ref{fig:laser_driver_supply_filter_measurement}. The \device{Bode 100} is driving the line injectors, measuring the signal going into the line injector and finally sampling the signal accross the output capacitor of the filter. The ground current now has two choices of flowing. One is through the low side of the measurement cables and their resistance, or through the ground plane of the VNA. The latter is the dreaded ground loop. This ground loop becomes more pronouced at higher frequencies, because the return path trough the cable is inductive and its impedance increases with frequency. Typically this problem would be addressed using a common-mode choke inserted into CH2. CH2 is the VNA input measuring the filter output. This common-mode choke prevents any current flowing through CH2, that has not flown through the cable.

%TODO: Add an image showing the ground loop, the Bode 100 and the 2-port shunt-thru measurement.
Unfortunately, the author did not have a suitable common mode choke at hand, so the only feasable solution to at least suppress the ground loop for low frequencies was to add transformers at the output and the CH2. This isolates the output and the battery powered \device{SR560} is driving the VNA via the transformer, isolating the input as well. The transformer used for isolating the VNA output, was an injection transformer named \device{PB01}. It is center tapped to create an anti-symmetrical output for the injection transformers. The center tap reduces the output amplitude by one half. The details regarding this device and its construction can be found in annex \ref{sec:injection_transformer}. The transformer used at the output of the \device{SR560} is a Picotest \device{J2123A}. Both transformers are unfortunately injection transformers and not dedicated isolation transformers as discussed in annex \ref{sec:injection_transformer}, yet the only transformers available at the time. The consequences of this subtle detail will become imminent in a moment.

\begin{figure}[ht]
    \centering
    \resizebox {0.9\textwidth} {!} {
        \import{figures/}{laser_driver_dgdrive_supply_filter_measurement_setup.tex}
    } % resizebox
    \caption{Power and grounding scheme for a low noise measurement of the line filter rejection ratio, minimizing the interfernce of circuit return currents.}
    \label{fig:laser_driver_supply_filter_measurement}
\end{figure}

The digital current driver is powered by the Rohde \& Schwarz \device{HMP4040} and there is a single point of ground connected to protective earth at the power supply. The power supply feeds into the line injectors, through which the current driver is powered. The output current is set to \qty{500}{\mA} accross a \qty{10}{\ohm} dummy resistor. The SR560 measures the ripple voltage after the LC-filter and drives the VNA input via the transformer. The measurement cable used is a short  twisted pair to reduce noise pickup.

The ouput of the VNA was set to \qty{-27}{\dB m} (\qty{10}{\mV_{rms}}), which must by multiplied by about $0.5 \cdot 0.975 = 0.485$ to give the ripple voltage on the positive supply, the latter term comes from the line injector \cite{line_injector_github}. Putting this into perspective, given a \qty{-60}{\dB} (\num{e-3}) suppression, results in a ripple voltage of only \qty{4.9}{\uV_{rms}}.

\begin{figure}[ht]
    \centering
    %% Creator: Matplotlib, PGF backend
%%
%% To include the figure in your LaTeX document, write
%%   \input{<filename>.pgf}
%%
%% Make sure the required packages are loaded in your preamble
%%   \usepackage{pgf}
%%
%% Also ensure that all the required font packages are loaded; for instance,
%% the lmodern package is sometimes necessary when using math font.
%%   \usepackage{lmodern}
%%
%% Figures using additional raster images can only be included by \input if
%% they are in the same directory as the main LaTeX file. For loading figures
%% from other directories you can use the `import` package
%%   \usepackage{import}
%%
%% and then include the figures with
%%   \import{<path to file>}{<filename>.pgf}
%%
%% Matplotlib used the following preamble
%%   \usepackage{siunitx}
%%   \usepackage{fontspec}
%%
\begingroup%
\makeatletter%
\begin{pgfpicture}%
\pgfpathrectangle{\pgfpointorigin}{\pgfqpoint{5.492126in}{3.394321in}}%
\pgfusepath{use as bounding box, clip}%
\begin{pgfscope}%
\pgfsetbuttcap%
\pgfsetmiterjoin%
\definecolor{currentfill}{rgb}{1.000000,1.000000,1.000000}%
\pgfsetfillcolor{currentfill}%
\pgfsetlinewidth{0.000000pt}%
\definecolor{currentstroke}{rgb}{1.000000,1.000000,1.000000}%
\pgfsetstrokecolor{currentstroke}%
\pgfsetdash{}{0pt}%
\pgfpathmoveto{\pgfqpoint{0.000000in}{0.000000in}}%
\pgfpathlineto{\pgfqpoint{5.492126in}{0.000000in}}%
\pgfpathlineto{\pgfqpoint{5.492126in}{3.394321in}}%
\pgfpathlineto{\pgfqpoint{0.000000in}{3.394321in}}%
\pgfpathlineto{\pgfqpoint{0.000000in}{0.000000in}}%
\pgfpathclose%
\pgfusepath{fill}%
\end{pgfscope}%
\begin{pgfscope}%
\pgfsetbuttcap%
\pgfsetmiterjoin%
\definecolor{currentfill}{rgb}{1.000000,1.000000,1.000000}%
\pgfsetfillcolor{currentfill}%
\pgfsetlinewidth{0.000000pt}%
\definecolor{currentstroke}{rgb}{0.000000,0.000000,0.000000}%
\pgfsetstrokecolor{currentstroke}%
\pgfsetstrokeopacity{0.000000}%
\pgfsetdash{}{0pt}%
\pgfpathmoveto{\pgfqpoint{0.693677in}{0.524170in}}%
\pgfpathlineto{\pgfqpoint{5.342126in}{0.524170in}}%
\pgfpathlineto{\pgfqpoint{5.342126in}{3.244321in}}%
\pgfpathlineto{\pgfqpoint{0.693677in}{3.244321in}}%
\pgfpathlineto{\pgfqpoint{0.693677in}{0.524170in}}%
\pgfpathclose%
\pgfusepath{fill}%
\end{pgfscope}%
\begin{pgfscope}%
\pgfpathrectangle{\pgfqpoint{0.693677in}{0.524170in}}{\pgfqpoint{4.648449in}{2.720151in}}%
\pgfusepath{clip}%
\pgfsetrectcap%
\pgfsetroundjoin%
\pgfsetlinewidth{0.803000pt}%
\definecolor{currentstroke}{rgb}{0.450000,0.450000,0.450000}%
\pgfsetstrokecolor{currentstroke}%
\pgfsetdash{}{0pt}%
\pgfpathmoveto{\pgfqpoint{0.904970in}{0.524170in}}%
\pgfpathlineto{\pgfqpoint{0.904970in}{3.244321in}}%
\pgfusepath{stroke}%
\end{pgfscope}%
\begin{pgfscope}%
\pgfsetbuttcap%
\pgfsetroundjoin%
\definecolor{currentfill}{rgb}{0.000000,0.000000,0.000000}%
\pgfsetfillcolor{currentfill}%
\pgfsetlinewidth{0.803000pt}%
\definecolor{currentstroke}{rgb}{0.000000,0.000000,0.000000}%
\pgfsetstrokecolor{currentstroke}%
\pgfsetdash{}{0pt}%
\pgfsys@defobject{currentmarker}{\pgfqpoint{0.000000in}{-0.048611in}}{\pgfqpoint{0.000000in}{0.000000in}}{%
\pgfpathmoveto{\pgfqpoint{0.000000in}{0.000000in}}%
\pgfpathlineto{\pgfqpoint{0.000000in}{-0.048611in}}%
\pgfusepath{stroke,fill}%
}%
\begin{pgfscope}%
\pgfsys@transformshift{0.904970in}{0.524170in}%
\pgfsys@useobject{currentmarker}{}%
\end{pgfscope}%
\end{pgfscope}%
\begin{pgfscope}%
\definecolor{textcolor}{rgb}{0.000000,0.000000,0.000000}%
\pgfsetstrokecolor{textcolor}%
\pgfsetfillcolor{textcolor}%
\pgftext[x=0.904970in,y=0.426948in,,top]{\color{textcolor}\rmfamily\fontsize{8.000000}{9.600000}\selectfont \(\displaystyle {10^{2}}\)}%
\end{pgfscope}%
\begin{pgfscope}%
\pgfpathrectangle{\pgfqpoint{0.693677in}{0.524170in}}{\pgfqpoint{4.648449in}{2.720151in}}%
\pgfusepath{clip}%
\pgfsetrectcap%
\pgfsetroundjoin%
\pgfsetlinewidth{0.803000pt}%
\definecolor{currentstroke}{rgb}{0.450000,0.450000,0.450000}%
\pgfsetstrokecolor{currentstroke}%
\pgfsetdash{}{0pt}%
\pgfpathmoveto{\pgfqpoint{1.961436in}{0.524170in}}%
\pgfpathlineto{\pgfqpoint{1.961436in}{3.244321in}}%
\pgfusepath{stroke}%
\end{pgfscope}%
\begin{pgfscope}%
\pgfsetbuttcap%
\pgfsetroundjoin%
\definecolor{currentfill}{rgb}{0.000000,0.000000,0.000000}%
\pgfsetfillcolor{currentfill}%
\pgfsetlinewidth{0.803000pt}%
\definecolor{currentstroke}{rgb}{0.000000,0.000000,0.000000}%
\pgfsetstrokecolor{currentstroke}%
\pgfsetdash{}{0pt}%
\pgfsys@defobject{currentmarker}{\pgfqpoint{0.000000in}{-0.048611in}}{\pgfqpoint{0.000000in}{0.000000in}}{%
\pgfpathmoveto{\pgfqpoint{0.000000in}{0.000000in}}%
\pgfpathlineto{\pgfqpoint{0.000000in}{-0.048611in}}%
\pgfusepath{stroke,fill}%
}%
\begin{pgfscope}%
\pgfsys@transformshift{1.961436in}{0.524170in}%
\pgfsys@useobject{currentmarker}{}%
\end{pgfscope}%
\end{pgfscope}%
\begin{pgfscope}%
\definecolor{textcolor}{rgb}{0.000000,0.000000,0.000000}%
\pgfsetstrokecolor{textcolor}%
\pgfsetfillcolor{textcolor}%
\pgftext[x=1.961436in,y=0.426948in,,top]{\color{textcolor}\rmfamily\fontsize{8.000000}{9.600000}\selectfont \(\displaystyle {10^{3}}\)}%
\end{pgfscope}%
\begin{pgfscope}%
\pgfpathrectangle{\pgfqpoint{0.693677in}{0.524170in}}{\pgfqpoint{4.648449in}{2.720151in}}%
\pgfusepath{clip}%
\pgfsetrectcap%
\pgfsetroundjoin%
\pgfsetlinewidth{0.803000pt}%
\definecolor{currentstroke}{rgb}{0.450000,0.450000,0.450000}%
\pgfsetstrokecolor{currentstroke}%
\pgfsetdash{}{0pt}%
\pgfpathmoveto{\pgfqpoint{3.017901in}{0.524170in}}%
\pgfpathlineto{\pgfqpoint{3.017901in}{3.244321in}}%
\pgfusepath{stroke}%
\end{pgfscope}%
\begin{pgfscope}%
\pgfsetbuttcap%
\pgfsetroundjoin%
\definecolor{currentfill}{rgb}{0.000000,0.000000,0.000000}%
\pgfsetfillcolor{currentfill}%
\pgfsetlinewidth{0.803000pt}%
\definecolor{currentstroke}{rgb}{0.000000,0.000000,0.000000}%
\pgfsetstrokecolor{currentstroke}%
\pgfsetdash{}{0pt}%
\pgfsys@defobject{currentmarker}{\pgfqpoint{0.000000in}{-0.048611in}}{\pgfqpoint{0.000000in}{0.000000in}}{%
\pgfpathmoveto{\pgfqpoint{0.000000in}{0.000000in}}%
\pgfpathlineto{\pgfqpoint{0.000000in}{-0.048611in}}%
\pgfusepath{stroke,fill}%
}%
\begin{pgfscope}%
\pgfsys@transformshift{3.017901in}{0.524170in}%
\pgfsys@useobject{currentmarker}{}%
\end{pgfscope}%
\end{pgfscope}%
\begin{pgfscope}%
\definecolor{textcolor}{rgb}{0.000000,0.000000,0.000000}%
\pgfsetstrokecolor{textcolor}%
\pgfsetfillcolor{textcolor}%
\pgftext[x=3.017901in,y=0.426948in,,top]{\color{textcolor}\rmfamily\fontsize{8.000000}{9.600000}\selectfont \(\displaystyle {10^{4}}\)}%
\end{pgfscope}%
\begin{pgfscope}%
\pgfpathrectangle{\pgfqpoint{0.693677in}{0.524170in}}{\pgfqpoint{4.648449in}{2.720151in}}%
\pgfusepath{clip}%
\pgfsetrectcap%
\pgfsetroundjoin%
\pgfsetlinewidth{0.803000pt}%
\definecolor{currentstroke}{rgb}{0.450000,0.450000,0.450000}%
\pgfsetstrokecolor{currentstroke}%
\pgfsetdash{}{0pt}%
\pgfpathmoveto{\pgfqpoint{4.074367in}{0.524170in}}%
\pgfpathlineto{\pgfqpoint{4.074367in}{3.244321in}}%
\pgfusepath{stroke}%
\end{pgfscope}%
\begin{pgfscope}%
\pgfsetbuttcap%
\pgfsetroundjoin%
\definecolor{currentfill}{rgb}{0.000000,0.000000,0.000000}%
\pgfsetfillcolor{currentfill}%
\pgfsetlinewidth{0.803000pt}%
\definecolor{currentstroke}{rgb}{0.000000,0.000000,0.000000}%
\pgfsetstrokecolor{currentstroke}%
\pgfsetdash{}{0pt}%
\pgfsys@defobject{currentmarker}{\pgfqpoint{0.000000in}{-0.048611in}}{\pgfqpoint{0.000000in}{0.000000in}}{%
\pgfpathmoveto{\pgfqpoint{0.000000in}{0.000000in}}%
\pgfpathlineto{\pgfqpoint{0.000000in}{-0.048611in}}%
\pgfusepath{stroke,fill}%
}%
\begin{pgfscope}%
\pgfsys@transformshift{4.074367in}{0.524170in}%
\pgfsys@useobject{currentmarker}{}%
\end{pgfscope}%
\end{pgfscope}%
\begin{pgfscope}%
\definecolor{textcolor}{rgb}{0.000000,0.000000,0.000000}%
\pgfsetstrokecolor{textcolor}%
\pgfsetfillcolor{textcolor}%
\pgftext[x=4.074367in,y=0.426948in,,top]{\color{textcolor}\rmfamily\fontsize{8.000000}{9.600000}\selectfont \(\displaystyle {10^{5}}\)}%
\end{pgfscope}%
\begin{pgfscope}%
\pgfpathrectangle{\pgfqpoint{0.693677in}{0.524170in}}{\pgfqpoint{4.648449in}{2.720151in}}%
\pgfusepath{clip}%
\pgfsetrectcap%
\pgfsetroundjoin%
\pgfsetlinewidth{0.803000pt}%
\definecolor{currentstroke}{rgb}{0.450000,0.450000,0.450000}%
\pgfsetstrokecolor{currentstroke}%
\pgfsetdash{}{0pt}%
\pgfpathmoveto{\pgfqpoint{5.130833in}{0.524170in}}%
\pgfpathlineto{\pgfqpoint{5.130833in}{3.244321in}}%
\pgfusepath{stroke}%
\end{pgfscope}%
\begin{pgfscope}%
\pgfsetbuttcap%
\pgfsetroundjoin%
\definecolor{currentfill}{rgb}{0.000000,0.000000,0.000000}%
\pgfsetfillcolor{currentfill}%
\pgfsetlinewidth{0.803000pt}%
\definecolor{currentstroke}{rgb}{0.000000,0.000000,0.000000}%
\pgfsetstrokecolor{currentstroke}%
\pgfsetdash{}{0pt}%
\pgfsys@defobject{currentmarker}{\pgfqpoint{0.000000in}{-0.048611in}}{\pgfqpoint{0.000000in}{0.000000in}}{%
\pgfpathmoveto{\pgfqpoint{0.000000in}{0.000000in}}%
\pgfpathlineto{\pgfqpoint{0.000000in}{-0.048611in}}%
\pgfusepath{stroke,fill}%
}%
\begin{pgfscope}%
\pgfsys@transformshift{5.130833in}{0.524170in}%
\pgfsys@useobject{currentmarker}{}%
\end{pgfscope}%
\end{pgfscope}%
\begin{pgfscope}%
\definecolor{textcolor}{rgb}{0.000000,0.000000,0.000000}%
\pgfsetstrokecolor{textcolor}%
\pgfsetfillcolor{textcolor}%
\pgftext[x=5.130833in,y=0.426948in,,top]{\color{textcolor}\rmfamily\fontsize{8.000000}{9.600000}\selectfont \(\displaystyle {10^{6}}\)}%
\end{pgfscope}%
\begin{pgfscope}%
\pgfpathrectangle{\pgfqpoint{0.693677in}{0.524170in}}{\pgfqpoint{4.648449in}{2.720151in}}%
\pgfusepath{clip}%
\pgfsetrectcap%
\pgfsetroundjoin%
\pgfsetlinewidth{0.803000pt}%
\definecolor{currentstroke}{rgb}{0.850000,0.850000,0.850000}%
\pgfsetstrokecolor{currentstroke}%
\pgfsetdash{}{0pt}%
\pgfpathmoveto{\pgfqpoint{0.741321in}{0.524170in}}%
\pgfpathlineto{\pgfqpoint{0.741321in}{3.244321in}}%
\pgfusepath{stroke}%
\end{pgfscope}%
\begin{pgfscope}%
\pgfsetbuttcap%
\pgfsetroundjoin%
\definecolor{currentfill}{rgb}{0.000000,0.000000,0.000000}%
\pgfsetfillcolor{currentfill}%
\pgfsetlinewidth{0.602250pt}%
\definecolor{currentstroke}{rgb}{0.000000,0.000000,0.000000}%
\pgfsetstrokecolor{currentstroke}%
\pgfsetdash{}{0pt}%
\pgfsys@defobject{currentmarker}{\pgfqpoint{0.000000in}{-0.027778in}}{\pgfqpoint{0.000000in}{0.000000in}}{%
\pgfpathmoveto{\pgfqpoint{0.000000in}{0.000000in}}%
\pgfpathlineto{\pgfqpoint{0.000000in}{-0.027778in}}%
\pgfusepath{stroke,fill}%
}%
\begin{pgfscope}%
\pgfsys@transformshift{0.741321in}{0.524170in}%
\pgfsys@useobject{currentmarker}{}%
\end{pgfscope}%
\end{pgfscope}%
\begin{pgfscope}%
\pgfpathrectangle{\pgfqpoint{0.693677in}{0.524170in}}{\pgfqpoint{4.648449in}{2.720151in}}%
\pgfusepath{clip}%
\pgfsetrectcap%
\pgfsetroundjoin%
\pgfsetlinewidth{0.803000pt}%
\definecolor{currentstroke}{rgb}{0.850000,0.850000,0.850000}%
\pgfsetstrokecolor{currentstroke}%
\pgfsetdash{}{0pt}%
\pgfpathmoveto{\pgfqpoint{0.802588in}{0.524170in}}%
\pgfpathlineto{\pgfqpoint{0.802588in}{3.244321in}}%
\pgfusepath{stroke}%
\end{pgfscope}%
\begin{pgfscope}%
\pgfsetbuttcap%
\pgfsetroundjoin%
\definecolor{currentfill}{rgb}{0.000000,0.000000,0.000000}%
\pgfsetfillcolor{currentfill}%
\pgfsetlinewidth{0.602250pt}%
\definecolor{currentstroke}{rgb}{0.000000,0.000000,0.000000}%
\pgfsetstrokecolor{currentstroke}%
\pgfsetdash{}{0pt}%
\pgfsys@defobject{currentmarker}{\pgfqpoint{0.000000in}{-0.027778in}}{\pgfqpoint{0.000000in}{0.000000in}}{%
\pgfpathmoveto{\pgfqpoint{0.000000in}{0.000000in}}%
\pgfpathlineto{\pgfqpoint{0.000000in}{-0.027778in}}%
\pgfusepath{stroke,fill}%
}%
\begin{pgfscope}%
\pgfsys@transformshift{0.802588in}{0.524170in}%
\pgfsys@useobject{currentmarker}{}%
\end{pgfscope}%
\end{pgfscope}%
\begin{pgfscope}%
\pgfpathrectangle{\pgfqpoint{0.693677in}{0.524170in}}{\pgfqpoint{4.648449in}{2.720151in}}%
\pgfusepath{clip}%
\pgfsetrectcap%
\pgfsetroundjoin%
\pgfsetlinewidth{0.803000pt}%
\definecolor{currentstroke}{rgb}{0.850000,0.850000,0.850000}%
\pgfsetstrokecolor{currentstroke}%
\pgfsetdash{}{0pt}%
\pgfpathmoveto{\pgfqpoint{0.856629in}{0.524170in}}%
\pgfpathlineto{\pgfqpoint{0.856629in}{3.244321in}}%
\pgfusepath{stroke}%
\end{pgfscope}%
\begin{pgfscope}%
\pgfsetbuttcap%
\pgfsetroundjoin%
\definecolor{currentfill}{rgb}{0.000000,0.000000,0.000000}%
\pgfsetfillcolor{currentfill}%
\pgfsetlinewidth{0.602250pt}%
\definecolor{currentstroke}{rgb}{0.000000,0.000000,0.000000}%
\pgfsetstrokecolor{currentstroke}%
\pgfsetdash{}{0pt}%
\pgfsys@defobject{currentmarker}{\pgfqpoint{0.000000in}{-0.027778in}}{\pgfqpoint{0.000000in}{0.000000in}}{%
\pgfpathmoveto{\pgfqpoint{0.000000in}{0.000000in}}%
\pgfpathlineto{\pgfqpoint{0.000000in}{-0.027778in}}%
\pgfusepath{stroke,fill}%
}%
\begin{pgfscope}%
\pgfsys@transformshift{0.856629in}{0.524170in}%
\pgfsys@useobject{currentmarker}{}%
\end{pgfscope}%
\end{pgfscope}%
\begin{pgfscope}%
\pgfpathrectangle{\pgfqpoint{0.693677in}{0.524170in}}{\pgfqpoint{4.648449in}{2.720151in}}%
\pgfusepath{clip}%
\pgfsetrectcap%
\pgfsetroundjoin%
\pgfsetlinewidth{0.803000pt}%
\definecolor{currentstroke}{rgb}{0.850000,0.850000,0.850000}%
\pgfsetstrokecolor{currentstroke}%
\pgfsetdash{}{0pt}%
\pgfpathmoveto{\pgfqpoint{1.222998in}{0.524170in}}%
\pgfpathlineto{\pgfqpoint{1.222998in}{3.244321in}}%
\pgfusepath{stroke}%
\end{pgfscope}%
\begin{pgfscope}%
\pgfsetbuttcap%
\pgfsetroundjoin%
\definecolor{currentfill}{rgb}{0.000000,0.000000,0.000000}%
\pgfsetfillcolor{currentfill}%
\pgfsetlinewidth{0.602250pt}%
\definecolor{currentstroke}{rgb}{0.000000,0.000000,0.000000}%
\pgfsetstrokecolor{currentstroke}%
\pgfsetdash{}{0pt}%
\pgfsys@defobject{currentmarker}{\pgfqpoint{0.000000in}{-0.027778in}}{\pgfqpoint{0.000000in}{0.000000in}}{%
\pgfpathmoveto{\pgfqpoint{0.000000in}{0.000000in}}%
\pgfpathlineto{\pgfqpoint{0.000000in}{-0.027778in}}%
\pgfusepath{stroke,fill}%
}%
\begin{pgfscope}%
\pgfsys@transformshift{1.222998in}{0.524170in}%
\pgfsys@useobject{currentmarker}{}%
\end{pgfscope}%
\end{pgfscope}%
\begin{pgfscope}%
\pgfpathrectangle{\pgfqpoint{0.693677in}{0.524170in}}{\pgfqpoint{4.648449in}{2.720151in}}%
\pgfusepath{clip}%
\pgfsetrectcap%
\pgfsetroundjoin%
\pgfsetlinewidth{0.803000pt}%
\definecolor{currentstroke}{rgb}{0.850000,0.850000,0.850000}%
\pgfsetstrokecolor{currentstroke}%
\pgfsetdash{}{0pt}%
\pgfpathmoveto{\pgfqpoint{1.409032in}{0.524170in}}%
\pgfpathlineto{\pgfqpoint{1.409032in}{3.244321in}}%
\pgfusepath{stroke}%
\end{pgfscope}%
\begin{pgfscope}%
\pgfsetbuttcap%
\pgfsetroundjoin%
\definecolor{currentfill}{rgb}{0.000000,0.000000,0.000000}%
\pgfsetfillcolor{currentfill}%
\pgfsetlinewidth{0.602250pt}%
\definecolor{currentstroke}{rgb}{0.000000,0.000000,0.000000}%
\pgfsetstrokecolor{currentstroke}%
\pgfsetdash{}{0pt}%
\pgfsys@defobject{currentmarker}{\pgfqpoint{0.000000in}{-0.027778in}}{\pgfqpoint{0.000000in}{0.000000in}}{%
\pgfpathmoveto{\pgfqpoint{0.000000in}{0.000000in}}%
\pgfpathlineto{\pgfqpoint{0.000000in}{-0.027778in}}%
\pgfusepath{stroke,fill}%
}%
\begin{pgfscope}%
\pgfsys@transformshift{1.409032in}{0.524170in}%
\pgfsys@useobject{currentmarker}{}%
\end{pgfscope}%
\end{pgfscope}%
\begin{pgfscope}%
\pgfpathrectangle{\pgfqpoint{0.693677in}{0.524170in}}{\pgfqpoint{4.648449in}{2.720151in}}%
\pgfusepath{clip}%
\pgfsetrectcap%
\pgfsetroundjoin%
\pgfsetlinewidth{0.803000pt}%
\definecolor{currentstroke}{rgb}{0.850000,0.850000,0.850000}%
\pgfsetstrokecolor{currentstroke}%
\pgfsetdash{}{0pt}%
\pgfpathmoveto{\pgfqpoint{1.541026in}{0.524170in}}%
\pgfpathlineto{\pgfqpoint{1.541026in}{3.244321in}}%
\pgfusepath{stroke}%
\end{pgfscope}%
\begin{pgfscope}%
\pgfsetbuttcap%
\pgfsetroundjoin%
\definecolor{currentfill}{rgb}{0.000000,0.000000,0.000000}%
\pgfsetfillcolor{currentfill}%
\pgfsetlinewidth{0.602250pt}%
\definecolor{currentstroke}{rgb}{0.000000,0.000000,0.000000}%
\pgfsetstrokecolor{currentstroke}%
\pgfsetdash{}{0pt}%
\pgfsys@defobject{currentmarker}{\pgfqpoint{0.000000in}{-0.027778in}}{\pgfqpoint{0.000000in}{0.000000in}}{%
\pgfpathmoveto{\pgfqpoint{0.000000in}{0.000000in}}%
\pgfpathlineto{\pgfqpoint{0.000000in}{-0.027778in}}%
\pgfusepath{stroke,fill}%
}%
\begin{pgfscope}%
\pgfsys@transformshift{1.541026in}{0.524170in}%
\pgfsys@useobject{currentmarker}{}%
\end{pgfscope}%
\end{pgfscope}%
\begin{pgfscope}%
\pgfpathrectangle{\pgfqpoint{0.693677in}{0.524170in}}{\pgfqpoint{4.648449in}{2.720151in}}%
\pgfusepath{clip}%
\pgfsetrectcap%
\pgfsetroundjoin%
\pgfsetlinewidth{0.803000pt}%
\definecolor{currentstroke}{rgb}{0.850000,0.850000,0.850000}%
\pgfsetstrokecolor{currentstroke}%
\pgfsetdash{}{0pt}%
\pgfpathmoveto{\pgfqpoint{1.643408in}{0.524170in}}%
\pgfpathlineto{\pgfqpoint{1.643408in}{3.244321in}}%
\pgfusepath{stroke}%
\end{pgfscope}%
\begin{pgfscope}%
\pgfsetbuttcap%
\pgfsetroundjoin%
\definecolor{currentfill}{rgb}{0.000000,0.000000,0.000000}%
\pgfsetfillcolor{currentfill}%
\pgfsetlinewidth{0.602250pt}%
\definecolor{currentstroke}{rgb}{0.000000,0.000000,0.000000}%
\pgfsetstrokecolor{currentstroke}%
\pgfsetdash{}{0pt}%
\pgfsys@defobject{currentmarker}{\pgfqpoint{0.000000in}{-0.027778in}}{\pgfqpoint{0.000000in}{0.000000in}}{%
\pgfpathmoveto{\pgfqpoint{0.000000in}{0.000000in}}%
\pgfpathlineto{\pgfqpoint{0.000000in}{-0.027778in}}%
\pgfusepath{stroke,fill}%
}%
\begin{pgfscope}%
\pgfsys@transformshift{1.643408in}{0.524170in}%
\pgfsys@useobject{currentmarker}{}%
\end{pgfscope}%
\end{pgfscope}%
\begin{pgfscope}%
\pgfpathrectangle{\pgfqpoint{0.693677in}{0.524170in}}{\pgfqpoint{4.648449in}{2.720151in}}%
\pgfusepath{clip}%
\pgfsetrectcap%
\pgfsetroundjoin%
\pgfsetlinewidth{0.803000pt}%
\definecolor{currentstroke}{rgb}{0.850000,0.850000,0.850000}%
\pgfsetstrokecolor{currentstroke}%
\pgfsetdash{}{0pt}%
\pgfpathmoveto{\pgfqpoint{1.727060in}{0.524170in}}%
\pgfpathlineto{\pgfqpoint{1.727060in}{3.244321in}}%
\pgfusepath{stroke}%
\end{pgfscope}%
\begin{pgfscope}%
\pgfsetbuttcap%
\pgfsetroundjoin%
\definecolor{currentfill}{rgb}{0.000000,0.000000,0.000000}%
\pgfsetfillcolor{currentfill}%
\pgfsetlinewidth{0.602250pt}%
\definecolor{currentstroke}{rgb}{0.000000,0.000000,0.000000}%
\pgfsetstrokecolor{currentstroke}%
\pgfsetdash{}{0pt}%
\pgfsys@defobject{currentmarker}{\pgfqpoint{0.000000in}{-0.027778in}}{\pgfqpoint{0.000000in}{0.000000in}}{%
\pgfpathmoveto{\pgfqpoint{0.000000in}{0.000000in}}%
\pgfpathlineto{\pgfqpoint{0.000000in}{-0.027778in}}%
\pgfusepath{stroke,fill}%
}%
\begin{pgfscope}%
\pgfsys@transformshift{1.727060in}{0.524170in}%
\pgfsys@useobject{currentmarker}{}%
\end{pgfscope}%
\end{pgfscope}%
\begin{pgfscope}%
\pgfpathrectangle{\pgfqpoint{0.693677in}{0.524170in}}{\pgfqpoint{4.648449in}{2.720151in}}%
\pgfusepath{clip}%
\pgfsetrectcap%
\pgfsetroundjoin%
\pgfsetlinewidth{0.803000pt}%
\definecolor{currentstroke}{rgb}{0.850000,0.850000,0.850000}%
\pgfsetstrokecolor{currentstroke}%
\pgfsetdash{}{0pt}%
\pgfpathmoveto{\pgfqpoint{1.797787in}{0.524170in}}%
\pgfpathlineto{\pgfqpoint{1.797787in}{3.244321in}}%
\pgfusepath{stroke}%
\end{pgfscope}%
\begin{pgfscope}%
\pgfsetbuttcap%
\pgfsetroundjoin%
\definecolor{currentfill}{rgb}{0.000000,0.000000,0.000000}%
\pgfsetfillcolor{currentfill}%
\pgfsetlinewidth{0.602250pt}%
\definecolor{currentstroke}{rgb}{0.000000,0.000000,0.000000}%
\pgfsetstrokecolor{currentstroke}%
\pgfsetdash{}{0pt}%
\pgfsys@defobject{currentmarker}{\pgfqpoint{0.000000in}{-0.027778in}}{\pgfqpoint{0.000000in}{0.000000in}}{%
\pgfpathmoveto{\pgfqpoint{0.000000in}{0.000000in}}%
\pgfpathlineto{\pgfqpoint{0.000000in}{-0.027778in}}%
\pgfusepath{stroke,fill}%
}%
\begin{pgfscope}%
\pgfsys@transformshift{1.797787in}{0.524170in}%
\pgfsys@useobject{currentmarker}{}%
\end{pgfscope}%
\end{pgfscope}%
\begin{pgfscope}%
\pgfpathrectangle{\pgfqpoint{0.693677in}{0.524170in}}{\pgfqpoint{4.648449in}{2.720151in}}%
\pgfusepath{clip}%
\pgfsetrectcap%
\pgfsetroundjoin%
\pgfsetlinewidth{0.803000pt}%
\definecolor{currentstroke}{rgb}{0.850000,0.850000,0.850000}%
\pgfsetstrokecolor{currentstroke}%
\pgfsetdash{}{0pt}%
\pgfpathmoveto{\pgfqpoint{1.859054in}{0.524170in}}%
\pgfpathlineto{\pgfqpoint{1.859054in}{3.244321in}}%
\pgfusepath{stroke}%
\end{pgfscope}%
\begin{pgfscope}%
\pgfsetbuttcap%
\pgfsetroundjoin%
\definecolor{currentfill}{rgb}{0.000000,0.000000,0.000000}%
\pgfsetfillcolor{currentfill}%
\pgfsetlinewidth{0.602250pt}%
\definecolor{currentstroke}{rgb}{0.000000,0.000000,0.000000}%
\pgfsetstrokecolor{currentstroke}%
\pgfsetdash{}{0pt}%
\pgfsys@defobject{currentmarker}{\pgfqpoint{0.000000in}{-0.027778in}}{\pgfqpoint{0.000000in}{0.000000in}}{%
\pgfpathmoveto{\pgfqpoint{0.000000in}{0.000000in}}%
\pgfpathlineto{\pgfqpoint{0.000000in}{-0.027778in}}%
\pgfusepath{stroke,fill}%
}%
\begin{pgfscope}%
\pgfsys@transformshift{1.859054in}{0.524170in}%
\pgfsys@useobject{currentmarker}{}%
\end{pgfscope}%
\end{pgfscope}%
\begin{pgfscope}%
\pgfpathrectangle{\pgfqpoint{0.693677in}{0.524170in}}{\pgfqpoint{4.648449in}{2.720151in}}%
\pgfusepath{clip}%
\pgfsetrectcap%
\pgfsetroundjoin%
\pgfsetlinewidth{0.803000pt}%
\definecolor{currentstroke}{rgb}{0.850000,0.850000,0.850000}%
\pgfsetstrokecolor{currentstroke}%
\pgfsetdash{}{0pt}%
\pgfpathmoveto{\pgfqpoint{1.913095in}{0.524170in}}%
\pgfpathlineto{\pgfqpoint{1.913095in}{3.244321in}}%
\pgfusepath{stroke}%
\end{pgfscope}%
\begin{pgfscope}%
\pgfsetbuttcap%
\pgfsetroundjoin%
\definecolor{currentfill}{rgb}{0.000000,0.000000,0.000000}%
\pgfsetfillcolor{currentfill}%
\pgfsetlinewidth{0.602250pt}%
\definecolor{currentstroke}{rgb}{0.000000,0.000000,0.000000}%
\pgfsetstrokecolor{currentstroke}%
\pgfsetdash{}{0pt}%
\pgfsys@defobject{currentmarker}{\pgfqpoint{0.000000in}{-0.027778in}}{\pgfqpoint{0.000000in}{0.000000in}}{%
\pgfpathmoveto{\pgfqpoint{0.000000in}{0.000000in}}%
\pgfpathlineto{\pgfqpoint{0.000000in}{-0.027778in}}%
\pgfusepath{stroke,fill}%
}%
\begin{pgfscope}%
\pgfsys@transformshift{1.913095in}{0.524170in}%
\pgfsys@useobject{currentmarker}{}%
\end{pgfscope}%
\end{pgfscope}%
\begin{pgfscope}%
\pgfpathrectangle{\pgfqpoint{0.693677in}{0.524170in}}{\pgfqpoint{4.648449in}{2.720151in}}%
\pgfusepath{clip}%
\pgfsetrectcap%
\pgfsetroundjoin%
\pgfsetlinewidth{0.803000pt}%
\definecolor{currentstroke}{rgb}{0.850000,0.850000,0.850000}%
\pgfsetstrokecolor{currentstroke}%
\pgfsetdash{}{0pt}%
\pgfpathmoveto{\pgfqpoint{2.279464in}{0.524170in}}%
\pgfpathlineto{\pgfqpoint{2.279464in}{3.244321in}}%
\pgfusepath{stroke}%
\end{pgfscope}%
\begin{pgfscope}%
\pgfsetbuttcap%
\pgfsetroundjoin%
\definecolor{currentfill}{rgb}{0.000000,0.000000,0.000000}%
\pgfsetfillcolor{currentfill}%
\pgfsetlinewidth{0.602250pt}%
\definecolor{currentstroke}{rgb}{0.000000,0.000000,0.000000}%
\pgfsetstrokecolor{currentstroke}%
\pgfsetdash{}{0pt}%
\pgfsys@defobject{currentmarker}{\pgfqpoint{0.000000in}{-0.027778in}}{\pgfqpoint{0.000000in}{0.000000in}}{%
\pgfpathmoveto{\pgfqpoint{0.000000in}{0.000000in}}%
\pgfpathlineto{\pgfqpoint{0.000000in}{-0.027778in}}%
\pgfusepath{stroke,fill}%
}%
\begin{pgfscope}%
\pgfsys@transformshift{2.279464in}{0.524170in}%
\pgfsys@useobject{currentmarker}{}%
\end{pgfscope}%
\end{pgfscope}%
\begin{pgfscope}%
\pgfpathrectangle{\pgfqpoint{0.693677in}{0.524170in}}{\pgfqpoint{4.648449in}{2.720151in}}%
\pgfusepath{clip}%
\pgfsetrectcap%
\pgfsetroundjoin%
\pgfsetlinewidth{0.803000pt}%
\definecolor{currentstroke}{rgb}{0.850000,0.850000,0.850000}%
\pgfsetstrokecolor{currentstroke}%
\pgfsetdash{}{0pt}%
\pgfpathmoveto{\pgfqpoint{2.465498in}{0.524170in}}%
\pgfpathlineto{\pgfqpoint{2.465498in}{3.244321in}}%
\pgfusepath{stroke}%
\end{pgfscope}%
\begin{pgfscope}%
\pgfsetbuttcap%
\pgfsetroundjoin%
\definecolor{currentfill}{rgb}{0.000000,0.000000,0.000000}%
\pgfsetfillcolor{currentfill}%
\pgfsetlinewidth{0.602250pt}%
\definecolor{currentstroke}{rgb}{0.000000,0.000000,0.000000}%
\pgfsetstrokecolor{currentstroke}%
\pgfsetdash{}{0pt}%
\pgfsys@defobject{currentmarker}{\pgfqpoint{0.000000in}{-0.027778in}}{\pgfqpoint{0.000000in}{0.000000in}}{%
\pgfpathmoveto{\pgfqpoint{0.000000in}{0.000000in}}%
\pgfpathlineto{\pgfqpoint{0.000000in}{-0.027778in}}%
\pgfusepath{stroke,fill}%
}%
\begin{pgfscope}%
\pgfsys@transformshift{2.465498in}{0.524170in}%
\pgfsys@useobject{currentmarker}{}%
\end{pgfscope}%
\end{pgfscope}%
\begin{pgfscope}%
\pgfpathrectangle{\pgfqpoint{0.693677in}{0.524170in}}{\pgfqpoint{4.648449in}{2.720151in}}%
\pgfusepath{clip}%
\pgfsetrectcap%
\pgfsetroundjoin%
\pgfsetlinewidth{0.803000pt}%
\definecolor{currentstroke}{rgb}{0.850000,0.850000,0.850000}%
\pgfsetstrokecolor{currentstroke}%
\pgfsetdash{}{0pt}%
\pgfpathmoveto{\pgfqpoint{2.597492in}{0.524170in}}%
\pgfpathlineto{\pgfqpoint{2.597492in}{3.244321in}}%
\pgfusepath{stroke}%
\end{pgfscope}%
\begin{pgfscope}%
\pgfsetbuttcap%
\pgfsetroundjoin%
\definecolor{currentfill}{rgb}{0.000000,0.000000,0.000000}%
\pgfsetfillcolor{currentfill}%
\pgfsetlinewidth{0.602250pt}%
\definecolor{currentstroke}{rgb}{0.000000,0.000000,0.000000}%
\pgfsetstrokecolor{currentstroke}%
\pgfsetdash{}{0pt}%
\pgfsys@defobject{currentmarker}{\pgfqpoint{0.000000in}{-0.027778in}}{\pgfqpoint{0.000000in}{0.000000in}}{%
\pgfpathmoveto{\pgfqpoint{0.000000in}{0.000000in}}%
\pgfpathlineto{\pgfqpoint{0.000000in}{-0.027778in}}%
\pgfusepath{stroke,fill}%
}%
\begin{pgfscope}%
\pgfsys@transformshift{2.597492in}{0.524170in}%
\pgfsys@useobject{currentmarker}{}%
\end{pgfscope}%
\end{pgfscope}%
\begin{pgfscope}%
\pgfpathrectangle{\pgfqpoint{0.693677in}{0.524170in}}{\pgfqpoint{4.648449in}{2.720151in}}%
\pgfusepath{clip}%
\pgfsetrectcap%
\pgfsetroundjoin%
\pgfsetlinewidth{0.803000pt}%
\definecolor{currentstroke}{rgb}{0.850000,0.850000,0.850000}%
\pgfsetstrokecolor{currentstroke}%
\pgfsetdash{}{0pt}%
\pgfpathmoveto{\pgfqpoint{2.699874in}{0.524170in}}%
\pgfpathlineto{\pgfqpoint{2.699874in}{3.244321in}}%
\pgfusepath{stroke}%
\end{pgfscope}%
\begin{pgfscope}%
\pgfsetbuttcap%
\pgfsetroundjoin%
\definecolor{currentfill}{rgb}{0.000000,0.000000,0.000000}%
\pgfsetfillcolor{currentfill}%
\pgfsetlinewidth{0.602250pt}%
\definecolor{currentstroke}{rgb}{0.000000,0.000000,0.000000}%
\pgfsetstrokecolor{currentstroke}%
\pgfsetdash{}{0pt}%
\pgfsys@defobject{currentmarker}{\pgfqpoint{0.000000in}{-0.027778in}}{\pgfqpoint{0.000000in}{0.000000in}}{%
\pgfpathmoveto{\pgfqpoint{0.000000in}{0.000000in}}%
\pgfpathlineto{\pgfqpoint{0.000000in}{-0.027778in}}%
\pgfusepath{stroke,fill}%
}%
\begin{pgfscope}%
\pgfsys@transformshift{2.699874in}{0.524170in}%
\pgfsys@useobject{currentmarker}{}%
\end{pgfscope}%
\end{pgfscope}%
\begin{pgfscope}%
\pgfpathrectangle{\pgfqpoint{0.693677in}{0.524170in}}{\pgfqpoint{4.648449in}{2.720151in}}%
\pgfusepath{clip}%
\pgfsetrectcap%
\pgfsetroundjoin%
\pgfsetlinewidth{0.803000pt}%
\definecolor{currentstroke}{rgb}{0.850000,0.850000,0.850000}%
\pgfsetstrokecolor{currentstroke}%
\pgfsetdash{}{0pt}%
\pgfpathmoveto{\pgfqpoint{2.783526in}{0.524170in}}%
\pgfpathlineto{\pgfqpoint{2.783526in}{3.244321in}}%
\pgfusepath{stroke}%
\end{pgfscope}%
\begin{pgfscope}%
\pgfsetbuttcap%
\pgfsetroundjoin%
\definecolor{currentfill}{rgb}{0.000000,0.000000,0.000000}%
\pgfsetfillcolor{currentfill}%
\pgfsetlinewidth{0.602250pt}%
\definecolor{currentstroke}{rgb}{0.000000,0.000000,0.000000}%
\pgfsetstrokecolor{currentstroke}%
\pgfsetdash{}{0pt}%
\pgfsys@defobject{currentmarker}{\pgfqpoint{0.000000in}{-0.027778in}}{\pgfqpoint{0.000000in}{0.000000in}}{%
\pgfpathmoveto{\pgfqpoint{0.000000in}{0.000000in}}%
\pgfpathlineto{\pgfqpoint{0.000000in}{-0.027778in}}%
\pgfusepath{stroke,fill}%
}%
\begin{pgfscope}%
\pgfsys@transformshift{2.783526in}{0.524170in}%
\pgfsys@useobject{currentmarker}{}%
\end{pgfscope}%
\end{pgfscope}%
\begin{pgfscope}%
\pgfpathrectangle{\pgfqpoint{0.693677in}{0.524170in}}{\pgfqpoint{4.648449in}{2.720151in}}%
\pgfusepath{clip}%
\pgfsetrectcap%
\pgfsetroundjoin%
\pgfsetlinewidth{0.803000pt}%
\definecolor{currentstroke}{rgb}{0.850000,0.850000,0.850000}%
\pgfsetstrokecolor{currentstroke}%
\pgfsetdash{}{0pt}%
\pgfpathmoveto{\pgfqpoint{2.854253in}{0.524170in}}%
\pgfpathlineto{\pgfqpoint{2.854253in}{3.244321in}}%
\pgfusepath{stroke}%
\end{pgfscope}%
\begin{pgfscope}%
\pgfsetbuttcap%
\pgfsetroundjoin%
\definecolor{currentfill}{rgb}{0.000000,0.000000,0.000000}%
\pgfsetfillcolor{currentfill}%
\pgfsetlinewidth{0.602250pt}%
\definecolor{currentstroke}{rgb}{0.000000,0.000000,0.000000}%
\pgfsetstrokecolor{currentstroke}%
\pgfsetdash{}{0pt}%
\pgfsys@defobject{currentmarker}{\pgfqpoint{0.000000in}{-0.027778in}}{\pgfqpoint{0.000000in}{0.000000in}}{%
\pgfpathmoveto{\pgfqpoint{0.000000in}{0.000000in}}%
\pgfpathlineto{\pgfqpoint{0.000000in}{-0.027778in}}%
\pgfusepath{stroke,fill}%
}%
\begin{pgfscope}%
\pgfsys@transformshift{2.854253in}{0.524170in}%
\pgfsys@useobject{currentmarker}{}%
\end{pgfscope}%
\end{pgfscope}%
\begin{pgfscope}%
\pgfpathrectangle{\pgfqpoint{0.693677in}{0.524170in}}{\pgfqpoint{4.648449in}{2.720151in}}%
\pgfusepath{clip}%
\pgfsetrectcap%
\pgfsetroundjoin%
\pgfsetlinewidth{0.803000pt}%
\definecolor{currentstroke}{rgb}{0.850000,0.850000,0.850000}%
\pgfsetstrokecolor{currentstroke}%
\pgfsetdash{}{0pt}%
\pgfpathmoveto{\pgfqpoint{2.915519in}{0.524170in}}%
\pgfpathlineto{\pgfqpoint{2.915519in}{3.244321in}}%
\pgfusepath{stroke}%
\end{pgfscope}%
\begin{pgfscope}%
\pgfsetbuttcap%
\pgfsetroundjoin%
\definecolor{currentfill}{rgb}{0.000000,0.000000,0.000000}%
\pgfsetfillcolor{currentfill}%
\pgfsetlinewidth{0.602250pt}%
\definecolor{currentstroke}{rgb}{0.000000,0.000000,0.000000}%
\pgfsetstrokecolor{currentstroke}%
\pgfsetdash{}{0pt}%
\pgfsys@defobject{currentmarker}{\pgfqpoint{0.000000in}{-0.027778in}}{\pgfqpoint{0.000000in}{0.000000in}}{%
\pgfpathmoveto{\pgfqpoint{0.000000in}{0.000000in}}%
\pgfpathlineto{\pgfqpoint{0.000000in}{-0.027778in}}%
\pgfusepath{stroke,fill}%
}%
\begin{pgfscope}%
\pgfsys@transformshift{2.915519in}{0.524170in}%
\pgfsys@useobject{currentmarker}{}%
\end{pgfscope}%
\end{pgfscope}%
\begin{pgfscope}%
\pgfpathrectangle{\pgfqpoint{0.693677in}{0.524170in}}{\pgfqpoint{4.648449in}{2.720151in}}%
\pgfusepath{clip}%
\pgfsetrectcap%
\pgfsetroundjoin%
\pgfsetlinewidth{0.803000pt}%
\definecolor{currentstroke}{rgb}{0.850000,0.850000,0.850000}%
\pgfsetstrokecolor{currentstroke}%
\pgfsetdash{}{0pt}%
\pgfpathmoveto{\pgfqpoint{2.969560in}{0.524170in}}%
\pgfpathlineto{\pgfqpoint{2.969560in}{3.244321in}}%
\pgfusepath{stroke}%
\end{pgfscope}%
\begin{pgfscope}%
\pgfsetbuttcap%
\pgfsetroundjoin%
\definecolor{currentfill}{rgb}{0.000000,0.000000,0.000000}%
\pgfsetfillcolor{currentfill}%
\pgfsetlinewidth{0.602250pt}%
\definecolor{currentstroke}{rgb}{0.000000,0.000000,0.000000}%
\pgfsetstrokecolor{currentstroke}%
\pgfsetdash{}{0pt}%
\pgfsys@defobject{currentmarker}{\pgfqpoint{0.000000in}{-0.027778in}}{\pgfqpoint{0.000000in}{0.000000in}}{%
\pgfpathmoveto{\pgfqpoint{0.000000in}{0.000000in}}%
\pgfpathlineto{\pgfqpoint{0.000000in}{-0.027778in}}%
\pgfusepath{stroke,fill}%
}%
\begin{pgfscope}%
\pgfsys@transformshift{2.969560in}{0.524170in}%
\pgfsys@useobject{currentmarker}{}%
\end{pgfscope}%
\end{pgfscope}%
\begin{pgfscope}%
\pgfpathrectangle{\pgfqpoint{0.693677in}{0.524170in}}{\pgfqpoint{4.648449in}{2.720151in}}%
\pgfusepath{clip}%
\pgfsetrectcap%
\pgfsetroundjoin%
\pgfsetlinewidth{0.803000pt}%
\definecolor{currentstroke}{rgb}{0.850000,0.850000,0.850000}%
\pgfsetstrokecolor{currentstroke}%
\pgfsetdash{}{0pt}%
\pgfpathmoveto{\pgfqpoint{3.335929in}{0.524170in}}%
\pgfpathlineto{\pgfqpoint{3.335929in}{3.244321in}}%
\pgfusepath{stroke}%
\end{pgfscope}%
\begin{pgfscope}%
\pgfsetbuttcap%
\pgfsetroundjoin%
\definecolor{currentfill}{rgb}{0.000000,0.000000,0.000000}%
\pgfsetfillcolor{currentfill}%
\pgfsetlinewidth{0.602250pt}%
\definecolor{currentstroke}{rgb}{0.000000,0.000000,0.000000}%
\pgfsetstrokecolor{currentstroke}%
\pgfsetdash{}{0pt}%
\pgfsys@defobject{currentmarker}{\pgfqpoint{0.000000in}{-0.027778in}}{\pgfqpoint{0.000000in}{0.000000in}}{%
\pgfpathmoveto{\pgfqpoint{0.000000in}{0.000000in}}%
\pgfpathlineto{\pgfqpoint{0.000000in}{-0.027778in}}%
\pgfusepath{stroke,fill}%
}%
\begin{pgfscope}%
\pgfsys@transformshift{3.335929in}{0.524170in}%
\pgfsys@useobject{currentmarker}{}%
\end{pgfscope}%
\end{pgfscope}%
\begin{pgfscope}%
\pgfpathrectangle{\pgfqpoint{0.693677in}{0.524170in}}{\pgfqpoint{4.648449in}{2.720151in}}%
\pgfusepath{clip}%
\pgfsetrectcap%
\pgfsetroundjoin%
\pgfsetlinewidth{0.803000pt}%
\definecolor{currentstroke}{rgb}{0.850000,0.850000,0.850000}%
\pgfsetstrokecolor{currentstroke}%
\pgfsetdash{}{0pt}%
\pgfpathmoveto{\pgfqpoint{3.521964in}{0.524170in}}%
\pgfpathlineto{\pgfqpoint{3.521964in}{3.244321in}}%
\pgfusepath{stroke}%
\end{pgfscope}%
\begin{pgfscope}%
\pgfsetbuttcap%
\pgfsetroundjoin%
\definecolor{currentfill}{rgb}{0.000000,0.000000,0.000000}%
\pgfsetfillcolor{currentfill}%
\pgfsetlinewidth{0.602250pt}%
\definecolor{currentstroke}{rgb}{0.000000,0.000000,0.000000}%
\pgfsetstrokecolor{currentstroke}%
\pgfsetdash{}{0pt}%
\pgfsys@defobject{currentmarker}{\pgfqpoint{0.000000in}{-0.027778in}}{\pgfqpoint{0.000000in}{0.000000in}}{%
\pgfpathmoveto{\pgfqpoint{0.000000in}{0.000000in}}%
\pgfpathlineto{\pgfqpoint{0.000000in}{-0.027778in}}%
\pgfusepath{stroke,fill}%
}%
\begin{pgfscope}%
\pgfsys@transformshift{3.521964in}{0.524170in}%
\pgfsys@useobject{currentmarker}{}%
\end{pgfscope}%
\end{pgfscope}%
\begin{pgfscope}%
\pgfpathrectangle{\pgfqpoint{0.693677in}{0.524170in}}{\pgfqpoint{4.648449in}{2.720151in}}%
\pgfusepath{clip}%
\pgfsetrectcap%
\pgfsetroundjoin%
\pgfsetlinewidth{0.803000pt}%
\definecolor{currentstroke}{rgb}{0.850000,0.850000,0.850000}%
\pgfsetstrokecolor{currentstroke}%
\pgfsetdash{}{0pt}%
\pgfpathmoveto{\pgfqpoint{3.653957in}{0.524170in}}%
\pgfpathlineto{\pgfqpoint{3.653957in}{3.244321in}}%
\pgfusepath{stroke}%
\end{pgfscope}%
\begin{pgfscope}%
\pgfsetbuttcap%
\pgfsetroundjoin%
\definecolor{currentfill}{rgb}{0.000000,0.000000,0.000000}%
\pgfsetfillcolor{currentfill}%
\pgfsetlinewidth{0.602250pt}%
\definecolor{currentstroke}{rgb}{0.000000,0.000000,0.000000}%
\pgfsetstrokecolor{currentstroke}%
\pgfsetdash{}{0pt}%
\pgfsys@defobject{currentmarker}{\pgfqpoint{0.000000in}{-0.027778in}}{\pgfqpoint{0.000000in}{0.000000in}}{%
\pgfpathmoveto{\pgfqpoint{0.000000in}{0.000000in}}%
\pgfpathlineto{\pgfqpoint{0.000000in}{-0.027778in}}%
\pgfusepath{stroke,fill}%
}%
\begin{pgfscope}%
\pgfsys@transformshift{3.653957in}{0.524170in}%
\pgfsys@useobject{currentmarker}{}%
\end{pgfscope}%
\end{pgfscope}%
\begin{pgfscope}%
\pgfpathrectangle{\pgfqpoint{0.693677in}{0.524170in}}{\pgfqpoint{4.648449in}{2.720151in}}%
\pgfusepath{clip}%
\pgfsetrectcap%
\pgfsetroundjoin%
\pgfsetlinewidth{0.803000pt}%
\definecolor{currentstroke}{rgb}{0.850000,0.850000,0.850000}%
\pgfsetstrokecolor{currentstroke}%
\pgfsetdash{}{0pt}%
\pgfpathmoveto{\pgfqpoint{3.756339in}{0.524170in}}%
\pgfpathlineto{\pgfqpoint{3.756339in}{3.244321in}}%
\pgfusepath{stroke}%
\end{pgfscope}%
\begin{pgfscope}%
\pgfsetbuttcap%
\pgfsetroundjoin%
\definecolor{currentfill}{rgb}{0.000000,0.000000,0.000000}%
\pgfsetfillcolor{currentfill}%
\pgfsetlinewidth{0.602250pt}%
\definecolor{currentstroke}{rgb}{0.000000,0.000000,0.000000}%
\pgfsetstrokecolor{currentstroke}%
\pgfsetdash{}{0pt}%
\pgfsys@defobject{currentmarker}{\pgfqpoint{0.000000in}{-0.027778in}}{\pgfqpoint{0.000000in}{0.000000in}}{%
\pgfpathmoveto{\pgfqpoint{0.000000in}{0.000000in}}%
\pgfpathlineto{\pgfqpoint{0.000000in}{-0.027778in}}%
\pgfusepath{stroke,fill}%
}%
\begin{pgfscope}%
\pgfsys@transformshift{3.756339in}{0.524170in}%
\pgfsys@useobject{currentmarker}{}%
\end{pgfscope}%
\end{pgfscope}%
\begin{pgfscope}%
\pgfpathrectangle{\pgfqpoint{0.693677in}{0.524170in}}{\pgfqpoint{4.648449in}{2.720151in}}%
\pgfusepath{clip}%
\pgfsetrectcap%
\pgfsetroundjoin%
\pgfsetlinewidth{0.803000pt}%
\definecolor{currentstroke}{rgb}{0.850000,0.850000,0.850000}%
\pgfsetstrokecolor{currentstroke}%
\pgfsetdash{}{0pt}%
\pgfpathmoveto{\pgfqpoint{3.839992in}{0.524170in}}%
\pgfpathlineto{\pgfqpoint{3.839992in}{3.244321in}}%
\pgfusepath{stroke}%
\end{pgfscope}%
\begin{pgfscope}%
\pgfsetbuttcap%
\pgfsetroundjoin%
\definecolor{currentfill}{rgb}{0.000000,0.000000,0.000000}%
\pgfsetfillcolor{currentfill}%
\pgfsetlinewidth{0.602250pt}%
\definecolor{currentstroke}{rgb}{0.000000,0.000000,0.000000}%
\pgfsetstrokecolor{currentstroke}%
\pgfsetdash{}{0pt}%
\pgfsys@defobject{currentmarker}{\pgfqpoint{0.000000in}{-0.027778in}}{\pgfqpoint{0.000000in}{0.000000in}}{%
\pgfpathmoveto{\pgfqpoint{0.000000in}{0.000000in}}%
\pgfpathlineto{\pgfqpoint{0.000000in}{-0.027778in}}%
\pgfusepath{stroke,fill}%
}%
\begin{pgfscope}%
\pgfsys@transformshift{3.839992in}{0.524170in}%
\pgfsys@useobject{currentmarker}{}%
\end{pgfscope}%
\end{pgfscope}%
\begin{pgfscope}%
\pgfpathrectangle{\pgfqpoint{0.693677in}{0.524170in}}{\pgfqpoint{4.648449in}{2.720151in}}%
\pgfusepath{clip}%
\pgfsetrectcap%
\pgfsetroundjoin%
\pgfsetlinewidth{0.803000pt}%
\definecolor{currentstroke}{rgb}{0.850000,0.850000,0.850000}%
\pgfsetstrokecolor{currentstroke}%
\pgfsetdash{}{0pt}%
\pgfpathmoveto{\pgfqpoint{3.910719in}{0.524170in}}%
\pgfpathlineto{\pgfqpoint{3.910719in}{3.244321in}}%
\pgfusepath{stroke}%
\end{pgfscope}%
\begin{pgfscope}%
\pgfsetbuttcap%
\pgfsetroundjoin%
\definecolor{currentfill}{rgb}{0.000000,0.000000,0.000000}%
\pgfsetfillcolor{currentfill}%
\pgfsetlinewidth{0.602250pt}%
\definecolor{currentstroke}{rgb}{0.000000,0.000000,0.000000}%
\pgfsetstrokecolor{currentstroke}%
\pgfsetdash{}{0pt}%
\pgfsys@defobject{currentmarker}{\pgfqpoint{0.000000in}{-0.027778in}}{\pgfqpoint{0.000000in}{0.000000in}}{%
\pgfpathmoveto{\pgfqpoint{0.000000in}{0.000000in}}%
\pgfpathlineto{\pgfqpoint{0.000000in}{-0.027778in}}%
\pgfusepath{stroke,fill}%
}%
\begin{pgfscope}%
\pgfsys@transformshift{3.910719in}{0.524170in}%
\pgfsys@useobject{currentmarker}{}%
\end{pgfscope}%
\end{pgfscope}%
\begin{pgfscope}%
\pgfpathrectangle{\pgfqpoint{0.693677in}{0.524170in}}{\pgfqpoint{4.648449in}{2.720151in}}%
\pgfusepath{clip}%
\pgfsetrectcap%
\pgfsetroundjoin%
\pgfsetlinewidth{0.803000pt}%
\definecolor{currentstroke}{rgb}{0.850000,0.850000,0.850000}%
\pgfsetstrokecolor{currentstroke}%
\pgfsetdash{}{0pt}%
\pgfpathmoveto{\pgfqpoint{3.971985in}{0.524170in}}%
\pgfpathlineto{\pgfqpoint{3.971985in}{3.244321in}}%
\pgfusepath{stroke}%
\end{pgfscope}%
\begin{pgfscope}%
\pgfsetbuttcap%
\pgfsetroundjoin%
\definecolor{currentfill}{rgb}{0.000000,0.000000,0.000000}%
\pgfsetfillcolor{currentfill}%
\pgfsetlinewidth{0.602250pt}%
\definecolor{currentstroke}{rgb}{0.000000,0.000000,0.000000}%
\pgfsetstrokecolor{currentstroke}%
\pgfsetdash{}{0pt}%
\pgfsys@defobject{currentmarker}{\pgfqpoint{0.000000in}{-0.027778in}}{\pgfqpoint{0.000000in}{0.000000in}}{%
\pgfpathmoveto{\pgfqpoint{0.000000in}{0.000000in}}%
\pgfpathlineto{\pgfqpoint{0.000000in}{-0.027778in}}%
\pgfusepath{stroke,fill}%
}%
\begin{pgfscope}%
\pgfsys@transformshift{3.971985in}{0.524170in}%
\pgfsys@useobject{currentmarker}{}%
\end{pgfscope}%
\end{pgfscope}%
\begin{pgfscope}%
\pgfpathrectangle{\pgfqpoint{0.693677in}{0.524170in}}{\pgfqpoint{4.648449in}{2.720151in}}%
\pgfusepath{clip}%
\pgfsetrectcap%
\pgfsetroundjoin%
\pgfsetlinewidth{0.803000pt}%
\definecolor{currentstroke}{rgb}{0.850000,0.850000,0.850000}%
\pgfsetstrokecolor{currentstroke}%
\pgfsetdash{}{0pt}%
\pgfpathmoveto{\pgfqpoint{4.026026in}{0.524170in}}%
\pgfpathlineto{\pgfqpoint{4.026026in}{3.244321in}}%
\pgfusepath{stroke}%
\end{pgfscope}%
\begin{pgfscope}%
\pgfsetbuttcap%
\pgfsetroundjoin%
\definecolor{currentfill}{rgb}{0.000000,0.000000,0.000000}%
\pgfsetfillcolor{currentfill}%
\pgfsetlinewidth{0.602250pt}%
\definecolor{currentstroke}{rgb}{0.000000,0.000000,0.000000}%
\pgfsetstrokecolor{currentstroke}%
\pgfsetdash{}{0pt}%
\pgfsys@defobject{currentmarker}{\pgfqpoint{0.000000in}{-0.027778in}}{\pgfqpoint{0.000000in}{0.000000in}}{%
\pgfpathmoveto{\pgfqpoint{0.000000in}{0.000000in}}%
\pgfpathlineto{\pgfqpoint{0.000000in}{-0.027778in}}%
\pgfusepath{stroke,fill}%
}%
\begin{pgfscope}%
\pgfsys@transformshift{4.026026in}{0.524170in}%
\pgfsys@useobject{currentmarker}{}%
\end{pgfscope}%
\end{pgfscope}%
\begin{pgfscope}%
\pgfpathrectangle{\pgfqpoint{0.693677in}{0.524170in}}{\pgfqpoint{4.648449in}{2.720151in}}%
\pgfusepath{clip}%
\pgfsetrectcap%
\pgfsetroundjoin%
\pgfsetlinewidth{0.803000pt}%
\definecolor{currentstroke}{rgb}{0.850000,0.850000,0.850000}%
\pgfsetstrokecolor{currentstroke}%
\pgfsetdash{}{0pt}%
\pgfpathmoveto{\pgfqpoint{4.392395in}{0.524170in}}%
\pgfpathlineto{\pgfqpoint{4.392395in}{3.244321in}}%
\pgfusepath{stroke}%
\end{pgfscope}%
\begin{pgfscope}%
\pgfsetbuttcap%
\pgfsetroundjoin%
\definecolor{currentfill}{rgb}{0.000000,0.000000,0.000000}%
\pgfsetfillcolor{currentfill}%
\pgfsetlinewidth{0.602250pt}%
\definecolor{currentstroke}{rgb}{0.000000,0.000000,0.000000}%
\pgfsetstrokecolor{currentstroke}%
\pgfsetdash{}{0pt}%
\pgfsys@defobject{currentmarker}{\pgfqpoint{0.000000in}{-0.027778in}}{\pgfqpoint{0.000000in}{0.000000in}}{%
\pgfpathmoveto{\pgfqpoint{0.000000in}{0.000000in}}%
\pgfpathlineto{\pgfqpoint{0.000000in}{-0.027778in}}%
\pgfusepath{stroke,fill}%
}%
\begin{pgfscope}%
\pgfsys@transformshift{4.392395in}{0.524170in}%
\pgfsys@useobject{currentmarker}{}%
\end{pgfscope}%
\end{pgfscope}%
\begin{pgfscope}%
\pgfpathrectangle{\pgfqpoint{0.693677in}{0.524170in}}{\pgfqpoint{4.648449in}{2.720151in}}%
\pgfusepath{clip}%
\pgfsetrectcap%
\pgfsetroundjoin%
\pgfsetlinewidth{0.803000pt}%
\definecolor{currentstroke}{rgb}{0.850000,0.850000,0.850000}%
\pgfsetstrokecolor{currentstroke}%
\pgfsetdash{}{0pt}%
\pgfpathmoveto{\pgfqpoint{4.578430in}{0.524170in}}%
\pgfpathlineto{\pgfqpoint{4.578430in}{3.244321in}}%
\pgfusepath{stroke}%
\end{pgfscope}%
\begin{pgfscope}%
\pgfsetbuttcap%
\pgfsetroundjoin%
\definecolor{currentfill}{rgb}{0.000000,0.000000,0.000000}%
\pgfsetfillcolor{currentfill}%
\pgfsetlinewidth{0.602250pt}%
\definecolor{currentstroke}{rgb}{0.000000,0.000000,0.000000}%
\pgfsetstrokecolor{currentstroke}%
\pgfsetdash{}{0pt}%
\pgfsys@defobject{currentmarker}{\pgfqpoint{0.000000in}{-0.027778in}}{\pgfqpoint{0.000000in}{0.000000in}}{%
\pgfpathmoveto{\pgfqpoint{0.000000in}{0.000000in}}%
\pgfpathlineto{\pgfqpoint{0.000000in}{-0.027778in}}%
\pgfusepath{stroke,fill}%
}%
\begin{pgfscope}%
\pgfsys@transformshift{4.578430in}{0.524170in}%
\pgfsys@useobject{currentmarker}{}%
\end{pgfscope}%
\end{pgfscope}%
\begin{pgfscope}%
\pgfpathrectangle{\pgfqpoint{0.693677in}{0.524170in}}{\pgfqpoint{4.648449in}{2.720151in}}%
\pgfusepath{clip}%
\pgfsetrectcap%
\pgfsetroundjoin%
\pgfsetlinewidth{0.803000pt}%
\definecolor{currentstroke}{rgb}{0.850000,0.850000,0.850000}%
\pgfsetstrokecolor{currentstroke}%
\pgfsetdash{}{0pt}%
\pgfpathmoveto{\pgfqpoint{4.710423in}{0.524170in}}%
\pgfpathlineto{\pgfqpoint{4.710423in}{3.244321in}}%
\pgfusepath{stroke}%
\end{pgfscope}%
\begin{pgfscope}%
\pgfsetbuttcap%
\pgfsetroundjoin%
\definecolor{currentfill}{rgb}{0.000000,0.000000,0.000000}%
\pgfsetfillcolor{currentfill}%
\pgfsetlinewidth{0.602250pt}%
\definecolor{currentstroke}{rgb}{0.000000,0.000000,0.000000}%
\pgfsetstrokecolor{currentstroke}%
\pgfsetdash{}{0pt}%
\pgfsys@defobject{currentmarker}{\pgfqpoint{0.000000in}{-0.027778in}}{\pgfqpoint{0.000000in}{0.000000in}}{%
\pgfpathmoveto{\pgfqpoint{0.000000in}{0.000000in}}%
\pgfpathlineto{\pgfqpoint{0.000000in}{-0.027778in}}%
\pgfusepath{stroke,fill}%
}%
\begin{pgfscope}%
\pgfsys@transformshift{4.710423in}{0.524170in}%
\pgfsys@useobject{currentmarker}{}%
\end{pgfscope}%
\end{pgfscope}%
\begin{pgfscope}%
\pgfpathrectangle{\pgfqpoint{0.693677in}{0.524170in}}{\pgfqpoint{4.648449in}{2.720151in}}%
\pgfusepath{clip}%
\pgfsetrectcap%
\pgfsetroundjoin%
\pgfsetlinewidth{0.803000pt}%
\definecolor{currentstroke}{rgb}{0.850000,0.850000,0.850000}%
\pgfsetstrokecolor{currentstroke}%
\pgfsetdash{}{0pt}%
\pgfpathmoveto{\pgfqpoint{4.812805in}{0.524170in}}%
\pgfpathlineto{\pgfqpoint{4.812805in}{3.244321in}}%
\pgfusepath{stroke}%
\end{pgfscope}%
\begin{pgfscope}%
\pgfsetbuttcap%
\pgfsetroundjoin%
\definecolor{currentfill}{rgb}{0.000000,0.000000,0.000000}%
\pgfsetfillcolor{currentfill}%
\pgfsetlinewidth{0.602250pt}%
\definecolor{currentstroke}{rgb}{0.000000,0.000000,0.000000}%
\pgfsetstrokecolor{currentstroke}%
\pgfsetdash{}{0pt}%
\pgfsys@defobject{currentmarker}{\pgfqpoint{0.000000in}{-0.027778in}}{\pgfqpoint{0.000000in}{0.000000in}}{%
\pgfpathmoveto{\pgfqpoint{0.000000in}{0.000000in}}%
\pgfpathlineto{\pgfqpoint{0.000000in}{-0.027778in}}%
\pgfusepath{stroke,fill}%
}%
\begin{pgfscope}%
\pgfsys@transformshift{4.812805in}{0.524170in}%
\pgfsys@useobject{currentmarker}{}%
\end{pgfscope}%
\end{pgfscope}%
\begin{pgfscope}%
\pgfpathrectangle{\pgfqpoint{0.693677in}{0.524170in}}{\pgfqpoint{4.648449in}{2.720151in}}%
\pgfusepath{clip}%
\pgfsetrectcap%
\pgfsetroundjoin%
\pgfsetlinewidth{0.803000pt}%
\definecolor{currentstroke}{rgb}{0.850000,0.850000,0.850000}%
\pgfsetstrokecolor{currentstroke}%
\pgfsetdash{}{0pt}%
\pgfpathmoveto{\pgfqpoint{4.896457in}{0.524170in}}%
\pgfpathlineto{\pgfqpoint{4.896457in}{3.244321in}}%
\pgfusepath{stroke}%
\end{pgfscope}%
\begin{pgfscope}%
\pgfsetbuttcap%
\pgfsetroundjoin%
\definecolor{currentfill}{rgb}{0.000000,0.000000,0.000000}%
\pgfsetfillcolor{currentfill}%
\pgfsetlinewidth{0.602250pt}%
\definecolor{currentstroke}{rgb}{0.000000,0.000000,0.000000}%
\pgfsetstrokecolor{currentstroke}%
\pgfsetdash{}{0pt}%
\pgfsys@defobject{currentmarker}{\pgfqpoint{0.000000in}{-0.027778in}}{\pgfqpoint{0.000000in}{0.000000in}}{%
\pgfpathmoveto{\pgfqpoint{0.000000in}{0.000000in}}%
\pgfpathlineto{\pgfqpoint{0.000000in}{-0.027778in}}%
\pgfusepath{stroke,fill}%
}%
\begin{pgfscope}%
\pgfsys@transformshift{4.896457in}{0.524170in}%
\pgfsys@useobject{currentmarker}{}%
\end{pgfscope}%
\end{pgfscope}%
\begin{pgfscope}%
\pgfpathrectangle{\pgfqpoint{0.693677in}{0.524170in}}{\pgfqpoint{4.648449in}{2.720151in}}%
\pgfusepath{clip}%
\pgfsetrectcap%
\pgfsetroundjoin%
\pgfsetlinewidth{0.803000pt}%
\definecolor{currentstroke}{rgb}{0.850000,0.850000,0.850000}%
\pgfsetstrokecolor{currentstroke}%
\pgfsetdash{}{0pt}%
\pgfpathmoveto{\pgfqpoint{4.967184in}{0.524170in}}%
\pgfpathlineto{\pgfqpoint{4.967184in}{3.244321in}}%
\pgfusepath{stroke}%
\end{pgfscope}%
\begin{pgfscope}%
\pgfsetbuttcap%
\pgfsetroundjoin%
\definecolor{currentfill}{rgb}{0.000000,0.000000,0.000000}%
\pgfsetfillcolor{currentfill}%
\pgfsetlinewidth{0.602250pt}%
\definecolor{currentstroke}{rgb}{0.000000,0.000000,0.000000}%
\pgfsetstrokecolor{currentstroke}%
\pgfsetdash{}{0pt}%
\pgfsys@defobject{currentmarker}{\pgfqpoint{0.000000in}{-0.027778in}}{\pgfqpoint{0.000000in}{0.000000in}}{%
\pgfpathmoveto{\pgfqpoint{0.000000in}{0.000000in}}%
\pgfpathlineto{\pgfqpoint{0.000000in}{-0.027778in}}%
\pgfusepath{stroke,fill}%
}%
\begin{pgfscope}%
\pgfsys@transformshift{4.967184in}{0.524170in}%
\pgfsys@useobject{currentmarker}{}%
\end{pgfscope}%
\end{pgfscope}%
\begin{pgfscope}%
\pgfpathrectangle{\pgfqpoint{0.693677in}{0.524170in}}{\pgfqpoint{4.648449in}{2.720151in}}%
\pgfusepath{clip}%
\pgfsetrectcap%
\pgfsetroundjoin%
\pgfsetlinewidth{0.803000pt}%
\definecolor{currentstroke}{rgb}{0.850000,0.850000,0.850000}%
\pgfsetstrokecolor{currentstroke}%
\pgfsetdash{}{0pt}%
\pgfpathmoveto{\pgfqpoint{5.028451in}{0.524170in}}%
\pgfpathlineto{\pgfqpoint{5.028451in}{3.244321in}}%
\pgfusepath{stroke}%
\end{pgfscope}%
\begin{pgfscope}%
\pgfsetbuttcap%
\pgfsetroundjoin%
\definecolor{currentfill}{rgb}{0.000000,0.000000,0.000000}%
\pgfsetfillcolor{currentfill}%
\pgfsetlinewidth{0.602250pt}%
\definecolor{currentstroke}{rgb}{0.000000,0.000000,0.000000}%
\pgfsetstrokecolor{currentstroke}%
\pgfsetdash{}{0pt}%
\pgfsys@defobject{currentmarker}{\pgfqpoint{0.000000in}{-0.027778in}}{\pgfqpoint{0.000000in}{0.000000in}}{%
\pgfpathmoveto{\pgfqpoint{0.000000in}{0.000000in}}%
\pgfpathlineto{\pgfqpoint{0.000000in}{-0.027778in}}%
\pgfusepath{stroke,fill}%
}%
\begin{pgfscope}%
\pgfsys@transformshift{5.028451in}{0.524170in}%
\pgfsys@useobject{currentmarker}{}%
\end{pgfscope}%
\end{pgfscope}%
\begin{pgfscope}%
\pgfpathrectangle{\pgfqpoint{0.693677in}{0.524170in}}{\pgfqpoint{4.648449in}{2.720151in}}%
\pgfusepath{clip}%
\pgfsetrectcap%
\pgfsetroundjoin%
\pgfsetlinewidth{0.803000pt}%
\definecolor{currentstroke}{rgb}{0.850000,0.850000,0.850000}%
\pgfsetstrokecolor{currentstroke}%
\pgfsetdash{}{0pt}%
\pgfpathmoveto{\pgfqpoint{5.082492in}{0.524170in}}%
\pgfpathlineto{\pgfqpoint{5.082492in}{3.244321in}}%
\pgfusepath{stroke}%
\end{pgfscope}%
\begin{pgfscope}%
\pgfsetbuttcap%
\pgfsetroundjoin%
\definecolor{currentfill}{rgb}{0.000000,0.000000,0.000000}%
\pgfsetfillcolor{currentfill}%
\pgfsetlinewidth{0.602250pt}%
\definecolor{currentstroke}{rgb}{0.000000,0.000000,0.000000}%
\pgfsetstrokecolor{currentstroke}%
\pgfsetdash{}{0pt}%
\pgfsys@defobject{currentmarker}{\pgfqpoint{0.000000in}{-0.027778in}}{\pgfqpoint{0.000000in}{0.000000in}}{%
\pgfpathmoveto{\pgfqpoint{0.000000in}{0.000000in}}%
\pgfpathlineto{\pgfqpoint{0.000000in}{-0.027778in}}%
\pgfusepath{stroke,fill}%
}%
\begin{pgfscope}%
\pgfsys@transformshift{5.082492in}{0.524170in}%
\pgfsys@useobject{currentmarker}{}%
\end{pgfscope}%
\end{pgfscope}%
\begin{pgfscope}%
\definecolor{textcolor}{rgb}{0.000000,0.000000,0.000000}%
\pgfsetstrokecolor{textcolor}%
\pgfsetfillcolor{textcolor}%
\pgftext[x=3.017901in,y=0.271531in,,top]{\color{textcolor}\rmfamily\fontsize{10.000000}{12.000000}\selectfont Frequency in \unit{\Hz}}%
\end{pgfscope}%
\begin{pgfscope}%
\pgfpathrectangle{\pgfqpoint{0.693677in}{0.524170in}}{\pgfqpoint{4.648449in}{2.720151in}}%
\pgfusepath{clip}%
\pgfsetrectcap%
\pgfsetroundjoin%
\pgfsetlinewidth{0.803000pt}%
\definecolor{currentstroke}{rgb}{0.450000,0.450000,0.450000}%
\pgfsetstrokecolor{currentstroke}%
\pgfsetdash{}{0pt}%
\pgfpathmoveto{\pgfqpoint{0.693677in}{0.781400in}}%
\pgfpathlineto{\pgfqpoint{5.342126in}{0.781400in}}%
\pgfusepath{stroke}%
\end{pgfscope}%
\begin{pgfscope}%
\pgfsetbuttcap%
\pgfsetroundjoin%
\definecolor{currentfill}{rgb}{0.000000,0.000000,0.000000}%
\pgfsetfillcolor{currentfill}%
\pgfsetlinewidth{0.803000pt}%
\definecolor{currentstroke}{rgb}{0.000000,0.000000,0.000000}%
\pgfsetstrokecolor{currentstroke}%
\pgfsetdash{}{0pt}%
\pgfsys@defobject{currentmarker}{\pgfqpoint{-0.048611in}{0.000000in}}{\pgfqpoint{-0.000000in}{0.000000in}}{%
\pgfpathmoveto{\pgfqpoint{-0.000000in}{0.000000in}}%
\pgfpathlineto{\pgfqpoint{-0.048611in}{0.000000in}}%
\pgfusepath{stroke,fill}%
}%
\begin{pgfscope}%
\pgfsys@transformshift{0.693677in}{0.781400in}%
\pgfsys@useobject{currentmarker}{}%
\end{pgfscope}%
\end{pgfscope}%
\begin{pgfscope}%
\definecolor{textcolor}{rgb}{0.000000,0.000000,0.000000}%
\pgfsetstrokecolor{textcolor}%
\pgfsetfillcolor{textcolor}%
\pgftext[x=0.327546in, y=0.742845in, left, base]{\color{textcolor}\rmfamily\fontsize{8.000000}{9.600000}\selectfont \(\displaystyle {\ensuremath{-}100}\)}%
\end{pgfscope}%
\begin{pgfscope}%
\pgfpathrectangle{\pgfqpoint{0.693677in}{0.524170in}}{\pgfqpoint{4.648449in}{2.720151in}}%
\pgfusepath{clip}%
\pgfsetrectcap%
\pgfsetroundjoin%
\pgfsetlinewidth{0.803000pt}%
\definecolor{currentstroke}{rgb}{0.450000,0.450000,0.450000}%
\pgfsetstrokecolor{currentstroke}%
\pgfsetdash{}{0pt}%
\pgfpathmoveto{\pgfqpoint{0.693677in}{1.245234in}}%
\pgfpathlineto{\pgfqpoint{5.342126in}{1.245234in}}%
\pgfusepath{stroke}%
\end{pgfscope}%
\begin{pgfscope}%
\pgfsetbuttcap%
\pgfsetroundjoin%
\definecolor{currentfill}{rgb}{0.000000,0.000000,0.000000}%
\pgfsetfillcolor{currentfill}%
\pgfsetlinewidth{0.803000pt}%
\definecolor{currentstroke}{rgb}{0.000000,0.000000,0.000000}%
\pgfsetstrokecolor{currentstroke}%
\pgfsetdash{}{0pt}%
\pgfsys@defobject{currentmarker}{\pgfqpoint{-0.048611in}{0.000000in}}{\pgfqpoint{-0.000000in}{0.000000in}}{%
\pgfpathmoveto{\pgfqpoint{-0.000000in}{0.000000in}}%
\pgfpathlineto{\pgfqpoint{-0.048611in}{0.000000in}}%
\pgfusepath{stroke,fill}%
}%
\begin{pgfscope}%
\pgfsys@transformshift{0.693677in}{1.245234in}%
\pgfsys@useobject{currentmarker}{}%
\end{pgfscope}%
\end{pgfscope}%
\begin{pgfscope}%
\definecolor{textcolor}{rgb}{0.000000,0.000000,0.000000}%
\pgfsetstrokecolor{textcolor}%
\pgfsetfillcolor{textcolor}%
\pgftext[x=0.386575in, y=1.206679in, left, base]{\color{textcolor}\rmfamily\fontsize{8.000000}{9.600000}\selectfont \(\displaystyle {\ensuremath{-}80}\)}%
\end{pgfscope}%
\begin{pgfscope}%
\pgfpathrectangle{\pgfqpoint{0.693677in}{0.524170in}}{\pgfqpoint{4.648449in}{2.720151in}}%
\pgfusepath{clip}%
\pgfsetrectcap%
\pgfsetroundjoin%
\pgfsetlinewidth{0.803000pt}%
\definecolor{currentstroke}{rgb}{0.450000,0.450000,0.450000}%
\pgfsetstrokecolor{currentstroke}%
\pgfsetdash{}{0pt}%
\pgfpathmoveto{\pgfqpoint{0.693677in}{1.709068in}}%
\pgfpathlineto{\pgfqpoint{5.342126in}{1.709068in}}%
\pgfusepath{stroke}%
\end{pgfscope}%
\begin{pgfscope}%
\pgfsetbuttcap%
\pgfsetroundjoin%
\definecolor{currentfill}{rgb}{0.000000,0.000000,0.000000}%
\pgfsetfillcolor{currentfill}%
\pgfsetlinewidth{0.803000pt}%
\definecolor{currentstroke}{rgb}{0.000000,0.000000,0.000000}%
\pgfsetstrokecolor{currentstroke}%
\pgfsetdash{}{0pt}%
\pgfsys@defobject{currentmarker}{\pgfqpoint{-0.048611in}{0.000000in}}{\pgfqpoint{-0.000000in}{0.000000in}}{%
\pgfpathmoveto{\pgfqpoint{-0.000000in}{0.000000in}}%
\pgfpathlineto{\pgfqpoint{-0.048611in}{0.000000in}}%
\pgfusepath{stroke,fill}%
}%
\begin{pgfscope}%
\pgfsys@transformshift{0.693677in}{1.709068in}%
\pgfsys@useobject{currentmarker}{}%
\end{pgfscope}%
\end{pgfscope}%
\begin{pgfscope}%
\definecolor{textcolor}{rgb}{0.000000,0.000000,0.000000}%
\pgfsetstrokecolor{textcolor}%
\pgfsetfillcolor{textcolor}%
\pgftext[x=0.386575in, y=1.670513in, left, base]{\color{textcolor}\rmfamily\fontsize{8.000000}{9.600000}\selectfont \(\displaystyle {\ensuremath{-}60}\)}%
\end{pgfscope}%
\begin{pgfscope}%
\pgfpathrectangle{\pgfqpoint{0.693677in}{0.524170in}}{\pgfqpoint{4.648449in}{2.720151in}}%
\pgfusepath{clip}%
\pgfsetrectcap%
\pgfsetroundjoin%
\pgfsetlinewidth{0.803000pt}%
\definecolor{currentstroke}{rgb}{0.450000,0.450000,0.450000}%
\pgfsetstrokecolor{currentstroke}%
\pgfsetdash{}{0pt}%
\pgfpathmoveto{\pgfqpoint{0.693677in}{2.172903in}}%
\pgfpathlineto{\pgfqpoint{5.342126in}{2.172903in}}%
\pgfusepath{stroke}%
\end{pgfscope}%
\begin{pgfscope}%
\pgfsetbuttcap%
\pgfsetroundjoin%
\definecolor{currentfill}{rgb}{0.000000,0.000000,0.000000}%
\pgfsetfillcolor{currentfill}%
\pgfsetlinewidth{0.803000pt}%
\definecolor{currentstroke}{rgb}{0.000000,0.000000,0.000000}%
\pgfsetstrokecolor{currentstroke}%
\pgfsetdash{}{0pt}%
\pgfsys@defobject{currentmarker}{\pgfqpoint{-0.048611in}{0.000000in}}{\pgfqpoint{-0.000000in}{0.000000in}}{%
\pgfpathmoveto{\pgfqpoint{-0.000000in}{0.000000in}}%
\pgfpathlineto{\pgfqpoint{-0.048611in}{0.000000in}}%
\pgfusepath{stroke,fill}%
}%
\begin{pgfscope}%
\pgfsys@transformshift{0.693677in}{2.172903in}%
\pgfsys@useobject{currentmarker}{}%
\end{pgfscope}%
\end{pgfscope}%
\begin{pgfscope}%
\definecolor{textcolor}{rgb}{0.000000,0.000000,0.000000}%
\pgfsetstrokecolor{textcolor}%
\pgfsetfillcolor{textcolor}%
\pgftext[x=0.386575in, y=2.134347in, left, base]{\color{textcolor}\rmfamily\fontsize{8.000000}{9.600000}\selectfont \(\displaystyle {\ensuremath{-}40}\)}%
\end{pgfscope}%
\begin{pgfscope}%
\pgfpathrectangle{\pgfqpoint{0.693677in}{0.524170in}}{\pgfqpoint{4.648449in}{2.720151in}}%
\pgfusepath{clip}%
\pgfsetrectcap%
\pgfsetroundjoin%
\pgfsetlinewidth{0.803000pt}%
\definecolor{currentstroke}{rgb}{0.450000,0.450000,0.450000}%
\pgfsetstrokecolor{currentstroke}%
\pgfsetdash{}{0pt}%
\pgfpathmoveto{\pgfqpoint{0.693677in}{2.636737in}}%
\pgfpathlineto{\pgfqpoint{5.342126in}{2.636737in}}%
\pgfusepath{stroke}%
\end{pgfscope}%
\begin{pgfscope}%
\pgfsetbuttcap%
\pgfsetroundjoin%
\definecolor{currentfill}{rgb}{0.000000,0.000000,0.000000}%
\pgfsetfillcolor{currentfill}%
\pgfsetlinewidth{0.803000pt}%
\definecolor{currentstroke}{rgb}{0.000000,0.000000,0.000000}%
\pgfsetstrokecolor{currentstroke}%
\pgfsetdash{}{0pt}%
\pgfsys@defobject{currentmarker}{\pgfqpoint{-0.048611in}{0.000000in}}{\pgfqpoint{-0.000000in}{0.000000in}}{%
\pgfpathmoveto{\pgfqpoint{-0.000000in}{0.000000in}}%
\pgfpathlineto{\pgfqpoint{-0.048611in}{0.000000in}}%
\pgfusepath{stroke,fill}%
}%
\begin{pgfscope}%
\pgfsys@transformshift{0.693677in}{2.636737in}%
\pgfsys@useobject{currentmarker}{}%
\end{pgfscope}%
\end{pgfscope}%
\begin{pgfscope}%
\definecolor{textcolor}{rgb}{0.000000,0.000000,0.000000}%
\pgfsetstrokecolor{textcolor}%
\pgfsetfillcolor{textcolor}%
\pgftext[x=0.386575in, y=2.598181in, left, base]{\color{textcolor}\rmfamily\fontsize{8.000000}{9.600000}\selectfont \(\displaystyle {\ensuremath{-}20}\)}%
\end{pgfscope}%
\begin{pgfscope}%
\pgfpathrectangle{\pgfqpoint{0.693677in}{0.524170in}}{\pgfqpoint{4.648449in}{2.720151in}}%
\pgfusepath{clip}%
\pgfsetrectcap%
\pgfsetroundjoin%
\pgfsetlinewidth{0.803000pt}%
\definecolor{currentstroke}{rgb}{0.450000,0.450000,0.450000}%
\pgfsetstrokecolor{currentstroke}%
\pgfsetdash{}{0pt}%
\pgfpathmoveto{\pgfqpoint{0.693677in}{3.100571in}}%
\pgfpathlineto{\pgfqpoint{5.342126in}{3.100571in}}%
\pgfusepath{stroke}%
\end{pgfscope}%
\begin{pgfscope}%
\pgfsetbuttcap%
\pgfsetroundjoin%
\definecolor{currentfill}{rgb}{0.000000,0.000000,0.000000}%
\pgfsetfillcolor{currentfill}%
\pgfsetlinewidth{0.803000pt}%
\definecolor{currentstroke}{rgb}{0.000000,0.000000,0.000000}%
\pgfsetstrokecolor{currentstroke}%
\pgfsetdash{}{0pt}%
\pgfsys@defobject{currentmarker}{\pgfqpoint{-0.048611in}{0.000000in}}{\pgfqpoint{-0.000000in}{0.000000in}}{%
\pgfpathmoveto{\pgfqpoint{-0.000000in}{0.000000in}}%
\pgfpathlineto{\pgfqpoint{-0.048611in}{0.000000in}}%
\pgfusepath{stroke,fill}%
}%
\begin{pgfscope}%
\pgfsys@transformshift{0.693677in}{3.100571in}%
\pgfsys@useobject{currentmarker}{}%
\end{pgfscope}%
\end{pgfscope}%
\begin{pgfscope}%
\definecolor{textcolor}{rgb}{0.000000,0.000000,0.000000}%
\pgfsetstrokecolor{textcolor}%
\pgfsetfillcolor{textcolor}%
\pgftext[x=0.537426in, y=3.062015in, left, base]{\color{textcolor}\rmfamily\fontsize{8.000000}{9.600000}\selectfont \(\displaystyle {0}\)}%
\end{pgfscope}%
\begin{pgfscope}%
\definecolor{textcolor}{rgb}{0.000000,0.000000,0.000000}%
\pgfsetstrokecolor{textcolor}%
\pgfsetfillcolor{textcolor}%
\pgftext[x=0.271991in,y=1.884245in,,bottom,rotate=90.000000]{\color{textcolor}\rmfamily\fontsize{10.000000}{12.000000}\selectfont Magnitude in \unit{\dB}}%
\end{pgfscope}%
\begin{pgfscope}%
\pgfpathrectangle{\pgfqpoint{0.693677in}{0.524170in}}{\pgfqpoint{4.648449in}{2.720151in}}%
\pgfusepath{clip}%
\pgfsetrectcap%
\pgfsetroundjoin%
\pgfsetlinewidth{1.003750pt}%
\definecolor{currentstroke}{rgb}{0.870588,0.560784,0.019608}%
\pgfsetstrokecolor{currentstroke}%
\pgfsetstrokeopacity{0.700000}%
\pgfsetdash{}{0pt}%
\pgfpathmoveto{\pgfqpoint{0.904970in}{3.120677in}}%
\pgfpathlineto{\pgfqpoint{0.915535in}{3.120656in}}%
\pgfpathlineto{\pgfqpoint{0.947229in}{3.117537in}}%
\pgfpathlineto{\pgfqpoint{1.000052in}{3.113483in}}%
\pgfpathlineto{\pgfqpoint{1.052875in}{3.107861in}}%
\pgfpathlineto{\pgfqpoint{1.063440in}{3.107386in}}%
\pgfpathlineto{\pgfqpoint{1.084569in}{3.104678in}}%
\pgfpathlineto{\pgfqpoint{1.116263in}{3.101410in}}%
\pgfpathlineto{\pgfqpoint{1.147957in}{3.096993in}}%
\pgfpathlineto{\pgfqpoint{1.158522in}{3.096287in}}%
\pgfpathlineto{\pgfqpoint{1.169086in}{3.093408in}}%
\pgfpathlineto{\pgfqpoint{1.179651in}{3.093561in}}%
\pgfpathlineto{\pgfqpoint{1.190216in}{3.090887in}}%
\pgfpathlineto{\pgfqpoint{1.221910in}{3.086819in}}%
\pgfpathlineto{\pgfqpoint{1.348686in}{3.062061in}}%
\pgfpathlineto{\pgfqpoint{1.443768in}{3.035567in}}%
\pgfpathlineto{\pgfqpoint{1.496591in}{3.017768in}}%
\pgfpathlineto{\pgfqpoint{1.517720in}{3.008849in}}%
\pgfpathlineto{\pgfqpoint{1.528285in}{3.005430in}}%
\pgfpathlineto{\pgfqpoint{1.538849in}{2.999697in}}%
\pgfpathlineto{\pgfqpoint{1.549414in}{2.996321in}}%
\pgfpathlineto{\pgfqpoint{1.612802in}{2.967748in}}%
\pgfpathlineto{\pgfqpoint{1.655061in}{2.945891in}}%
\pgfpathlineto{\pgfqpoint{1.665625in}{2.941195in}}%
\pgfpathlineto{\pgfqpoint{1.676190in}{2.933020in}}%
\pgfpathlineto{\pgfqpoint{1.686755in}{2.928549in}}%
\pgfpathlineto{\pgfqpoint{1.739578in}{2.897571in}}%
\pgfpathlineto{\pgfqpoint{1.750143in}{2.890145in}}%
\pgfpathlineto{\pgfqpoint{1.760707in}{2.884474in}}%
\pgfpathlineto{\pgfqpoint{1.792401in}{2.864097in}}%
\pgfpathlineto{\pgfqpoint{1.802966in}{2.858773in}}%
\pgfpathlineto{\pgfqpoint{1.824095in}{2.842470in}}%
\pgfpathlineto{\pgfqpoint{1.845225in}{2.829543in}}%
\pgfpathlineto{\pgfqpoint{1.876918in}{2.806918in}}%
\pgfpathlineto{\pgfqpoint{1.887483in}{2.797934in}}%
\pgfpathlineto{\pgfqpoint{1.898048in}{2.791533in}}%
\pgfpathlineto{\pgfqpoint{1.908612in}{2.783163in}}%
\pgfpathlineto{\pgfqpoint{1.919177in}{2.776920in}}%
\pgfpathlineto{\pgfqpoint{1.972000in}{2.735258in}}%
\pgfpathlineto{\pgfqpoint{1.982565in}{2.726268in}}%
\pgfpathlineto{\pgfqpoint{2.014259in}{2.705419in}}%
\pgfpathlineto{\pgfqpoint{2.035388in}{2.686129in}}%
\pgfpathlineto{\pgfqpoint{2.045953in}{2.676273in}}%
\pgfpathlineto{\pgfqpoint{2.056518in}{2.670479in}}%
\pgfpathlineto{\pgfqpoint{2.077647in}{2.652218in}}%
\pgfpathlineto{\pgfqpoint{2.088212in}{2.641751in}}%
\pgfpathlineto{\pgfqpoint{2.098776in}{2.636482in}}%
\pgfpathlineto{\pgfqpoint{2.109341in}{2.623879in}}%
\pgfpathlineto{\pgfqpoint{2.119906in}{2.618029in}}%
\pgfpathlineto{\pgfqpoint{2.130470in}{2.608270in}}%
\pgfpathlineto{\pgfqpoint{2.141035in}{2.595944in}}%
\pgfpathlineto{\pgfqpoint{2.151600in}{2.592321in}}%
\pgfpathlineto{\pgfqpoint{2.162164in}{2.582444in}}%
\pgfpathlineto{\pgfqpoint{2.172729in}{2.568000in}}%
\pgfpathlineto{\pgfqpoint{2.183294in}{2.568716in}}%
\pgfpathlineto{\pgfqpoint{2.193858in}{2.556279in}}%
\pgfpathlineto{\pgfqpoint{2.204423in}{2.547810in}}%
\pgfpathlineto{\pgfqpoint{2.214988in}{2.534152in}}%
\pgfpathlineto{\pgfqpoint{2.225552in}{2.527643in}}%
\pgfpathlineto{\pgfqpoint{2.246681in}{2.504778in}}%
\pgfpathlineto{\pgfqpoint{2.257246in}{2.499015in}}%
\pgfpathlineto{\pgfqpoint{2.267811in}{2.491484in}}%
\pgfpathlineto{\pgfqpoint{2.288940in}{2.474269in}}%
\pgfpathlineto{\pgfqpoint{2.310069in}{2.441132in}}%
\pgfpathlineto{\pgfqpoint{2.320634in}{2.445773in}}%
\pgfpathlineto{\pgfqpoint{2.331199in}{2.433034in}}%
\pgfpathlineto{\pgfqpoint{2.341763in}{2.429731in}}%
\pgfpathlineto{\pgfqpoint{2.352328in}{2.403020in}}%
\pgfpathlineto{\pgfqpoint{2.362893in}{2.403720in}}%
\pgfpathlineto{\pgfqpoint{2.384022in}{2.381855in}}%
\pgfpathlineto{\pgfqpoint{2.394587in}{2.377619in}}%
\pgfpathlineto{\pgfqpoint{2.405151in}{2.363285in}}%
\pgfpathlineto{\pgfqpoint{2.415716in}{2.365588in}}%
\pgfpathlineto{\pgfqpoint{2.426281in}{2.352161in}}%
\pgfpathlineto{\pgfqpoint{2.436845in}{2.329503in}}%
\pgfpathlineto{\pgfqpoint{2.457975in}{2.314999in}}%
\pgfpathlineto{\pgfqpoint{2.479104in}{2.285715in}}%
\pgfpathlineto{\pgfqpoint{2.489669in}{2.286257in}}%
\pgfpathlineto{\pgfqpoint{2.500233in}{2.280137in}}%
\pgfpathlineto{\pgfqpoint{2.510798in}{2.259023in}}%
\pgfpathlineto{\pgfqpoint{2.521363in}{2.264367in}}%
\pgfpathlineto{\pgfqpoint{2.531927in}{2.245524in}}%
\pgfpathlineto{\pgfqpoint{2.542492in}{2.215916in}}%
\pgfpathlineto{\pgfqpoint{2.553057in}{2.198828in}}%
\pgfpathlineto{\pgfqpoint{2.563621in}{2.198668in}}%
\pgfpathlineto{\pgfqpoint{2.574186in}{2.191826in}}%
\pgfpathlineto{\pgfqpoint{2.584751in}{2.178013in}}%
\pgfpathlineto{\pgfqpoint{2.595315in}{2.186925in}}%
\pgfpathlineto{\pgfqpoint{2.605880in}{2.140081in}}%
\pgfpathlineto{\pgfqpoint{2.616445in}{2.147966in}}%
\pgfpathlineto{\pgfqpoint{2.627009in}{2.128304in}}%
\pgfpathlineto{\pgfqpoint{2.637574in}{2.133135in}}%
\pgfpathlineto{\pgfqpoint{2.648138in}{2.118070in}}%
\pgfpathlineto{\pgfqpoint{2.658703in}{2.115470in}}%
\pgfpathlineto{\pgfqpoint{2.669268in}{2.109314in}}%
\pgfpathlineto{\pgfqpoint{2.679832in}{2.088652in}}%
\pgfpathlineto{\pgfqpoint{2.690397in}{2.107331in}}%
\pgfpathlineto{\pgfqpoint{2.700962in}{2.062899in}}%
\pgfpathlineto{\pgfqpoint{2.711526in}{2.055441in}}%
\pgfpathlineto{\pgfqpoint{2.722091in}{2.007396in}}%
\pgfpathlineto{\pgfqpoint{2.732656in}{2.024833in}}%
\pgfpathlineto{\pgfqpoint{2.743220in}{2.045604in}}%
\pgfpathlineto{\pgfqpoint{2.753785in}{2.008876in}}%
\pgfpathlineto{\pgfqpoint{2.764350in}{1.992144in}}%
\pgfpathlineto{\pgfqpoint{2.774914in}{1.972671in}}%
\pgfpathlineto{\pgfqpoint{2.785479in}{1.984625in}}%
\pgfpathlineto{\pgfqpoint{2.796044in}{1.964183in}}%
\pgfpathlineto{\pgfqpoint{2.806608in}{2.001620in}}%
\pgfpathlineto{\pgfqpoint{2.817173in}{1.956052in}}%
\pgfpathlineto{\pgfqpoint{2.827738in}{1.965686in}}%
\pgfpathlineto{\pgfqpoint{2.848867in}{1.906972in}}%
\pgfpathlineto{\pgfqpoint{2.859432in}{1.846321in}}%
\pgfpathlineto{\pgfqpoint{2.869996in}{1.880951in}}%
\pgfpathlineto{\pgfqpoint{2.880561in}{1.885313in}}%
\pgfpathlineto{\pgfqpoint{2.891126in}{1.936813in}}%
\pgfpathlineto{\pgfqpoint{2.901690in}{1.911360in}}%
\pgfpathlineto{\pgfqpoint{2.912255in}{1.893615in}}%
\pgfpathlineto{\pgfqpoint{2.922820in}{1.882636in}}%
\pgfpathlineto{\pgfqpoint{2.933384in}{1.835352in}}%
\pgfpathlineto{\pgfqpoint{2.943949in}{1.861211in}}%
\pgfpathlineto{\pgfqpoint{2.954514in}{1.852851in}}%
\pgfpathlineto{\pgfqpoint{2.965078in}{1.873159in}}%
\pgfpathlineto{\pgfqpoint{2.975643in}{1.862407in}}%
\pgfpathlineto{\pgfqpoint{2.986208in}{1.844682in}}%
\pgfpathlineto{\pgfqpoint{2.996772in}{1.841571in}}%
\pgfpathlineto{\pgfqpoint{3.007337in}{1.830936in}}%
\pgfpathlineto{\pgfqpoint{3.017901in}{1.777212in}}%
\pgfpathlineto{\pgfqpoint{3.028466in}{1.811838in}}%
\pgfpathlineto{\pgfqpoint{3.039031in}{1.788765in}}%
\pgfpathlineto{\pgfqpoint{3.049595in}{1.780449in}}%
\pgfpathlineto{\pgfqpoint{3.060160in}{1.876329in}}%
\pgfpathlineto{\pgfqpoint{3.070725in}{1.804059in}}%
\pgfpathlineto{\pgfqpoint{3.081289in}{1.838131in}}%
\pgfpathlineto{\pgfqpoint{3.091854in}{1.778412in}}%
\pgfpathlineto{\pgfqpoint{3.102419in}{1.869559in}}%
\pgfpathlineto{\pgfqpoint{3.112983in}{1.849037in}}%
\pgfpathlineto{\pgfqpoint{3.123548in}{1.823393in}}%
\pgfpathlineto{\pgfqpoint{3.134113in}{1.834202in}}%
\pgfpathlineto{\pgfqpoint{3.144677in}{1.852703in}}%
\pgfpathlineto{\pgfqpoint{3.155242in}{1.851132in}}%
\pgfpathlineto{\pgfqpoint{3.165807in}{1.783533in}}%
\pgfpathlineto{\pgfqpoint{3.186936in}{1.876397in}}%
\pgfpathlineto{\pgfqpoint{3.197501in}{1.852944in}}%
\pgfpathlineto{\pgfqpoint{3.208065in}{1.855976in}}%
\pgfpathlineto{\pgfqpoint{3.218630in}{1.883803in}}%
\pgfpathlineto{\pgfqpoint{3.229195in}{1.857113in}}%
\pgfpathlineto{\pgfqpoint{3.239759in}{1.892744in}}%
\pgfpathlineto{\pgfqpoint{3.250324in}{1.875570in}}%
\pgfpathlineto{\pgfqpoint{3.271453in}{1.868589in}}%
\pgfpathlineto{\pgfqpoint{3.282018in}{1.882682in}}%
\pgfpathlineto{\pgfqpoint{3.292583in}{1.900946in}}%
\pgfpathlineto{\pgfqpoint{3.303147in}{1.870531in}}%
\pgfpathlineto{\pgfqpoint{3.313712in}{1.884950in}}%
\pgfpathlineto{\pgfqpoint{3.324277in}{1.907464in}}%
\pgfpathlineto{\pgfqpoint{3.334841in}{1.882936in}}%
\pgfpathlineto{\pgfqpoint{3.345406in}{1.925153in}}%
\pgfpathlineto{\pgfqpoint{3.355971in}{1.920872in}}%
\pgfpathlineto{\pgfqpoint{3.366535in}{1.912698in}}%
\pgfpathlineto{\pgfqpoint{3.377100in}{1.933956in}}%
\pgfpathlineto{\pgfqpoint{3.387665in}{1.935881in}}%
\pgfpathlineto{\pgfqpoint{3.398229in}{1.943103in}}%
\pgfpathlineto{\pgfqpoint{3.408794in}{1.928744in}}%
\pgfpathlineto{\pgfqpoint{3.419358in}{1.945742in}}%
\pgfpathlineto{\pgfqpoint{3.429923in}{1.951948in}}%
\pgfpathlineto{\pgfqpoint{3.440488in}{1.959821in}}%
\pgfpathlineto{\pgfqpoint{3.451052in}{1.961117in}}%
\pgfpathlineto{\pgfqpoint{3.461617in}{1.966811in}}%
\pgfpathlineto{\pgfqpoint{3.472182in}{1.968203in}}%
\pgfpathlineto{\pgfqpoint{3.482746in}{1.972155in}}%
\pgfpathlineto{\pgfqpoint{3.493311in}{1.984139in}}%
\pgfpathlineto{\pgfqpoint{3.514440in}{1.977645in}}%
\pgfpathlineto{\pgfqpoint{3.525005in}{2.007945in}}%
\pgfpathlineto{\pgfqpoint{3.535570in}{1.998661in}}%
\pgfpathlineto{\pgfqpoint{3.546134in}{2.016249in}}%
\pgfpathlineto{\pgfqpoint{3.556699in}{2.016752in}}%
\pgfpathlineto{\pgfqpoint{3.567264in}{2.024597in}}%
\pgfpathlineto{\pgfqpoint{3.577828in}{2.026632in}}%
\pgfpathlineto{\pgfqpoint{3.588393in}{2.031138in}}%
\pgfpathlineto{\pgfqpoint{3.609522in}{2.046071in}}%
\pgfpathlineto{\pgfqpoint{3.620087in}{2.054080in}}%
\pgfpathlineto{\pgfqpoint{3.630652in}{2.055887in}}%
\pgfpathlineto{\pgfqpoint{3.641216in}{2.053458in}}%
\pgfpathlineto{\pgfqpoint{3.651781in}{2.072345in}}%
\pgfpathlineto{\pgfqpoint{3.662346in}{2.075205in}}%
\pgfpathlineto{\pgfqpoint{3.672910in}{2.075743in}}%
\pgfpathlineto{\pgfqpoint{3.683475in}{2.080479in}}%
\pgfpathlineto{\pgfqpoint{3.694040in}{2.087428in}}%
\pgfpathlineto{\pgfqpoint{3.704604in}{2.088869in}}%
\pgfpathlineto{\pgfqpoint{3.715169in}{2.098426in}}%
\pgfpathlineto{\pgfqpoint{3.725734in}{2.101749in}}%
\pgfpathlineto{\pgfqpoint{3.736298in}{2.103765in}}%
\pgfpathlineto{\pgfqpoint{3.746863in}{2.108262in}}%
\pgfpathlineto{\pgfqpoint{3.757428in}{2.111501in}}%
\pgfpathlineto{\pgfqpoint{3.778557in}{2.123223in}}%
\pgfpathlineto{\pgfqpoint{3.799686in}{2.129035in}}%
\pgfpathlineto{\pgfqpoint{3.810251in}{2.129876in}}%
\pgfpathlineto{\pgfqpoint{3.820815in}{2.140959in}}%
\pgfpathlineto{\pgfqpoint{3.831380in}{2.140648in}}%
\pgfpathlineto{\pgfqpoint{3.841945in}{2.150477in}}%
\pgfpathlineto{\pgfqpoint{3.873639in}{2.161679in}}%
\pgfpathlineto{\pgfqpoint{3.884203in}{2.168720in}}%
\pgfpathlineto{\pgfqpoint{3.894768in}{2.172991in}}%
\pgfpathlineto{\pgfqpoint{3.905333in}{2.179607in}}%
\pgfpathlineto{\pgfqpoint{3.915897in}{2.182446in}}%
\pgfpathlineto{\pgfqpoint{3.926462in}{2.182961in}}%
\pgfpathlineto{\pgfqpoint{3.937027in}{2.191050in}}%
\pgfpathlineto{\pgfqpoint{3.947591in}{2.196860in}}%
\pgfpathlineto{\pgfqpoint{3.979285in}{2.208960in}}%
\pgfpathlineto{\pgfqpoint{4.000415in}{2.219281in}}%
\pgfpathlineto{\pgfqpoint{4.010979in}{2.222876in}}%
\pgfpathlineto{\pgfqpoint{4.032109in}{2.232072in}}%
\pgfpathlineto{\pgfqpoint{4.042673in}{2.236413in}}%
\pgfpathlineto{\pgfqpoint{4.063803in}{2.246527in}}%
\pgfpathlineto{\pgfqpoint{4.180014in}{2.297898in}}%
\pgfpathlineto{\pgfqpoint{4.190578in}{2.301284in}}%
\pgfpathlineto{\pgfqpoint{4.253966in}{2.328767in}}%
\pgfpathlineto{\pgfqpoint{4.264531in}{2.332003in}}%
\pgfpathlineto{\pgfqpoint{4.296225in}{2.346478in}}%
\pgfpathlineto{\pgfqpoint{4.306790in}{2.349931in}}%
\pgfpathlineto{\pgfqpoint{4.338484in}{2.363083in}}%
\pgfpathlineto{\pgfqpoint{4.349048in}{2.366193in}}%
\pgfpathlineto{\pgfqpoint{4.359613in}{2.371939in}}%
\pgfpathlineto{\pgfqpoint{4.454695in}{2.407990in}}%
\pgfpathlineto{\pgfqpoint{4.486389in}{2.421759in}}%
\pgfpathlineto{\pgfqpoint{4.507518in}{2.427897in}}%
\pgfpathlineto{\pgfqpoint{4.528648in}{2.435941in}}%
\pgfpathlineto{\pgfqpoint{4.613165in}{2.467592in}}%
\pgfpathlineto{\pgfqpoint{4.655423in}{2.481090in}}%
\pgfpathlineto{\pgfqpoint{4.708247in}{2.498548in}}%
\pgfpathlineto{\pgfqpoint{4.718811in}{2.500811in}}%
\pgfpathlineto{\pgfqpoint{4.739941in}{2.508519in}}%
\pgfpathlineto{\pgfqpoint{4.782199in}{2.519996in}}%
\pgfpathlineto{\pgfqpoint{4.803329in}{2.526991in}}%
\pgfpathlineto{\pgfqpoint{4.813893in}{2.528376in}}%
\pgfpathlineto{\pgfqpoint{4.824458in}{2.532178in}}%
\pgfpathlineto{\pgfqpoint{4.845587in}{2.535712in}}%
\pgfpathlineto{\pgfqpoint{4.856152in}{2.538678in}}%
\pgfpathlineto{\pgfqpoint{4.866717in}{2.542861in}}%
\pgfpathlineto{\pgfqpoint{4.877281in}{2.545440in}}%
\pgfpathlineto{\pgfqpoint{4.898411in}{2.548738in}}%
\pgfpathlineto{\pgfqpoint{4.908975in}{2.557312in}}%
\pgfpathlineto{\pgfqpoint{4.919540in}{2.548477in}}%
\pgfpathlineto{\pgfqpoint{4.930105in}{2.546497in}}%
\pgfpathlineto{\pgfqpoint{4.940669in}{2.566168in}}%
\pgfpathlineto{\pgfqpoint{4.951234in}{2.557318in}}%
\pgfpathlineto{\pgfqpoint{4.961798in}{2.556874in}}%
\pgfpathlineto{\pgfqpoint{4.972363in}{2.560323in}}%
\pgfpathlineto{\pgfqpoint{4.993492in}{2.559547in}}%
\pgfpathlineto{\pgfqpoint{5.004057in}{2.561821in}}%
\pgfpathlineto{\pgfqpoint{5.014622in}{2.566781in}}%
\pgfpathlineto{\pgfqpoint{5.025186in}{2.568304in}}%
\pgfpathlineto{\pgfqpoint{5.046316in}{2.580338in}}%
\pgfpathlineto{\pgfqpoint{5.056880in}{2.587987in}}%
\pgfpathlineto{\pgfqpoint{5.067445in}{2.593086in}}%
\pgfpathlineto{\pgfqpoint{5.088574in}{2.605641in}}%
\pgfpathlineto{\pgfqpoint{5.120268in}{2.631717in}}%
\pgfpathlineto{\pgfqpoint{5.130833in}{2.641693in}}%
\pgfpathlineto{\pgfqpoint{5.130833in}{2.641693in}}%
\pgfusepath{stroke}%
\end{pgfscope}%
\begin{pgfscope}%
\pgfpathrectangle{\pgfqpoint{0.693677in}{0.524170in}}{\pgfqpoint{4.648449in}{2.720151in}}%
\pgfusepath{clip}%
\pgfsetrectcap%
\pgfsetroundjoin%
\pgfsetlinewidth{1.003750pt}%
\definecolor{currentstroke}{rgb}{0.003922,0.450980,0.698039}%
\pgfsetstrokecolor{currentstroke}%
\pgfsetstrokeopacity{0.700000}%
\pgfsetdash{}{0pt}%
\pgfpathmoveto{\pgfqpoint{0.915535in}{3.096194in}}%
\pgfpathlineto{\pgfqpoint{1.021181in}{3.093642in}}%
\pgfpathlineto{\pgfqpoint{1.095134in}{3.089694in}}%
\pgfpathlineto{\pgfqpoint{1.158522in}{3.084045in}}%
\pgfpathlineto{\pgfqpoint{1.211345in}{3.077203in}}%
\pgfpathlineto{\pgfqpoint{1.264168in}{3.068008in}}%
\pgfpathlineto{\pgfqpoint{1.316992in}{3.056156in}}%
\pgfpathlineto{\pgfqpoint{1.359250in}{3.044638in}}%
\pgfpathlineto{\pgfqpoint{1.412074in}{3.027654in}}%
\pgfpathlineto{\pgfqpoint{1.464897in}{3.007864in}}%
\pgfpathlineto{\pgfqpoint{1.517720in}{2.985424in}}%
\pgfpathlineto{\pgfqpoint{1.570543in}{2.960534in}}%
\pgfpathlineto{\pgfqpoint{1.623367in}{2.933402in}}%
\pgfpathlineto{\pgfqpoint{1.686755in}{2.898155in}}%
\pgfpathlineto{\pgfqpoint{1.750143in}{2.860258in}}%
\pgfpathlineto{\pgfqpoint{1.813531in}{2.819990in}}%
\pgfpathlineto{\pgfqpoint{1.887483in}{2.770395in}}%
\pgfpathlineto{\pgfqpoint{1.972000in}{2.710862in}}%
\pgfpathlineto{\pgfqpoint{2.077647in}{2.633365in}}%
\pgfpathlineto{\pgfqpoint{2.436845in}{2.366812in}}%
\pgfpathlineto{\pgfqpoint{2.521363in}{2.308169in}}%
\pgfpathlineto{\pgfqpoint{2.605880in}{2.252362in}}%
\pgfpathlineto{\pgfqpoint{2.679832in}{2.206040in}}%
\pgfpathlineto{\pgfqpoint{2.753785in}{2.162047in}}%
\pgfpathlineto{\pgfqpoint{2.838302in}{2.114426in}}%
\pgfpathlineto{\pgfqpoint{2.922820in}{2.069285in}}%
\pgfpathlineto{\pgfqpoint{3.017901in}{2.020921in}}%
\pgfpathlineto{\pgfqpoint{3.134113in}{1.964430in}}%
\pgfpathlineto{\pgfqpoint{3.271453in}{1.900186in}}%
\pgfpathlineto{\pgfqpoint{3.461617in}{1.813820in}}%
\pgfpathlineto{\pgfqpoint{3.799686in}{1.663185in}}%
\pgfpathlineto{\pgfqpoint{4.180014in}{1.493170in}}%
\pgfpathlineto{\pgfqpoint{4.359613in}{1.410455in}}%
\pgfpathlineto{\pgfqpoint{4.486389in}{1.349573in}}%
\pgfpathlineto{\pgfqpoint{4.581471in}{1.301445in}}%
\pgfpathlineto{\pgfqpoint{4.655423in}{1.261679in}}%
\pgfpathlineto{\pgfqpoint{4.718811in}{1.225183in}}%
\pgfpathlineto{\pgfqpoint{4.771635in}{1.192359in}}%
\pgfpathlineto{\pgfqpoint{4.813893in}{1.163938in}}%
\pgfpathlineto{\pgfqpoint{4.856152in}{1.132925in}}%
\pgfpathlineto{\pgfqpoint{4.887846in}{1.107426in}}%
\pgfpathlineto{\pgfqpoint{4.919540in}{1.079398in}}%
\pgfpathlineto{\pgfqpoint{4.951234in}{1.048039in}}%
\pgfpathlineto{\pgfqpoint{4.972363in}{1.024713in}}%
\pgfpathlineto{\pgfqpoint{4.993492in}{0.998873in}}%
\pgfpathlineto{\pgfqpoint{5.014622in}{0.969797in}}%
\pgfpathlineto{\pgfqpoint{5.035751in}{0.936428in}}%
\pgfpathlineto{\pgfqpoint{5.056880in}{0.897131in}}%
\pgfpathlineto{\pgfqpoint{5.078010in}{0.849199in}}%
\pgfpathlineto{\pgfqpoint{5.088574in}{0.820611in}}%
\pgfpathlineto{\pgfqpoint{5.099139in}{0.787762in}}%
\pgfpathlineto{\pgfqpoint{5.109704in}{0.749301in}}%
\pgfpathlineto{\pgfqpoint{5.120268in}{0.703353in}}%
\pgfpathlineto{\pgfqpoint{5.130833in}{0.647813in}}%
\pgfpathlineto{\pgfqpoint{5.130833in}{0.647813in}}%
\pgfusepath{stroke}%
\end{pgfscope}%
\begin{pgfscope}%
\pgfsetrectcap%
\pgfsetmiterjoin%
\pgfsetlinewidth{0.803000pt}%
\definecolor{currentstroke}{rgb}{0.000000,0.000000,0.000000}%
\pgfsetstrokecolor{currentstroke}%
\pgfsetdash{}{0pt}%
\pgfpathmoveto{\pgfqpoint{0.693677in}{0.524170in}}%
\pgfpathlineto{\pgfqpoint{0.693677in}{3.244321in}}%
\pgfusepath{stroke}%
\end{pgfscope}%
\begin{pgfscope}%
\pgfsetrectcap%
\pgfsetmiterjoin%
\pgfsetlinewidth{0.803000pt}%
\definecolor{currentstroke}{rgb}{0.000000,0.000000,0.000000}%
\pgfsetstrokecolor{currentstroke}%
\pgfsetdash{}{0pt}%
\pgfpathmoveto{\pgfqpoint{5.342126in}{0.524170in}}%
\pgfpathlineto{\pgfqpoint{5.342126in}{3.244321in}}%
\pgfusepath{stroke}%
\end{pgfscope}%
\begin{pgfscope}%
\pgfsetrectcap%
\pgfsetmiterjoin%
\pgfsetlinewidth{0.803000pt}%
\definecolor{currentstroke}{rgb}{0.000000,0.000000,0.000000}%
\pgfsetstrokecolor{currentstroke}%
\pgfsetdash{}{0pt}%
\pgfpathmoveto{\pgfqpoint{0.693677in}{0.524170in}}%
\pgfpathlineto{\pgfqpoint{5.342126in}{0.524170in}}%
\pgfusepath{stroke}%
\end{pgfscope}%
\begin{pgfscope}%
\pgfsetrectcap%
\pgfsetmiterjoin%
\pgfsetlinewidth{0.803000pt}%
\definecolor{currentstroke}{rgb}{0.000000,0.000000,0.000000}%
\pgfsetstrokecolor{currentstroke}%
\pgfsetdash{}{0pt}%
\pgfpathmoveto{\pgfqpoint{0.693677in}{3.244321in}}%
\pgfpathlineto{\pgfqpoint{5.342126in}{3.244321in}}%
\pgfusepath{stroke}%
\end{pgfscope}%
\begin{pgfscope}%
\pgfsetbuttcap%
\pgfsetmiterjoin%
\definecolor{currentfill}{rgb}{1.000000,1.000000,1.000000}%
\pgfsetfillcolor{currentfill}%
\pgfsetfillopacity{0.800000}%
\pgfsetlinewidth{1.003750pt}%
\definecolor{currentstroke}{rgb}{0.800000,0.800000,0.800000}%
\pgfsetstrokecolor{currentstroke}%
\pgfsetstrokeopacity{0.800000}%
\pgfsetdash{}{0pt}%
\pgfpathmoveto{\pgfqpoint{0.771455in}{0.579725in}}%
\pgfpathlineto{\pgfqpoint{1.680899in}{0.579725in}}%
\pgfpathquadraticcurveto{\pgfqpoint{1.703121in}{0.579725in}}{\pgfqpoint{1.703121in}{0.601948in}}%
\pgfpathlineto{\pgfqpoint{1.703121in}{0.900614in}}%
\pgfpathquadraticcurveto{\pgfqpoint{1.703121in}{0.922836in}}{\pgfqpoint{1.680899in}{0.922836in}}%
\pgfpathlineto{\pgfqpoint{0.771455in}{0.922836in}}%
\pgfpathquadraticcurveto{\pgfqpoint{0.749232in}{0.922836in}}{\pgfqpoint{0.749232in}{0.900614in}}%
\pgfpathlineto{\pgfqpoint{0.749232in}{0.601948in}}%
\pgfpathquadraticcurveto{\pgfqpoint{0.749232in}{0.579725in}}{\pgfqpoint{0.771455in}{0.579725in}}%
\pgfpathlineto{\pgfqpoint{0.771455in}{0.579725in}}%
\pgfpathclose%
\pgfusepath{stroke,fill}%
\end{pgfscope}%
\begin{pgfscope}%
\pgfsetrectcap%
\pgfsetroundjoin%
\pgfsetlinewidth{1.003750pt}%
\definecolor{currentstroke}{rgb}{0.870588,0.560784,0.019608}%
\pgfsetstrokecolor{currentstroke}%
\pgfsetstrokeopacity{0.700000}%
\pgfsetdash{}{0pt}%
\pgfpathmoveto{\pgfqpoint{0.793677in}{0.839503in}}%
\pgfpathlineto{\pgfqpoint{0.904788in}{0.839503in}}%
\pgfpathlineto{\pgfqpoint{1.015899in}{0.839503in}}%
\pgfusepath{stroke}%
\end{pgfscope}%
\begin{pgfscope}%
\definecolor{textcolor}{rgb}{0.000000,0.000000,0.000000}%
\pgfsetstrokecolor{textcolor}%
\pgfsetfillcolor{textcolor}%
\pgftext[x=1.104788in,y=0.800614in,left,base]{\color{textcolor}\rmfamily\fontsize{8.000000}{9.600000}\selectfont LC Filter}%
\end{pgfscope}%
\begin{pgfscope}%
\pgfsetrectcap%
\pgfsetroundjoin%
\pgfsetlinewidth{1.003750pt}%
\definecolor{currentstroke}{rgb}{0.003922,0.450980,0.698039}%
\pgfsetstrokecolor{currentstroke}%
\pgfsetstrokeopacity{0.700000}%
\pgfsetdash{}{0pt}%
\pgfpathmoveto{\pgfqpoint{0.793677in}{0.684614in}}%
\pgfpathlineto{\pgfqpoint{0.904788in}{0.684614in}}%
\pgfpathlineto{\pgfqpoint{1.015899in}{0.684614in}}%
\pgfusepath{stroke}%
\end{pgfscope}%
\begin{pgfscope}%
\definecolor{textcolor}{rgb}{0.000000,0.000000,0.000000}%
\pgfsetstrokecolor{textcolor}%
\pgfsetfillcolor{textcolor}%
\pgftext[x=1.104788in,y=0.645725in,left,base]{\color{textcolor}\rmfamily\fontsize{8.000000}{9.600000}\selectfont Simulation}%
\end{pgfscope}%
\end{pgfpicture}%
\makeatother%
\endgroup%

    \caption{Measured response of input filter used in the digital current driver. Above \qty{10}{\kHz} capacitive coupling through the transformer can be seen.}
    \label{fig:laser_driver_input_filter}
\end{figure}

Figure \ref{fig:laser_driver_input_filter} shows the measurement of the LC-filter output. At low frequencies, there is good agreement with the simulation and the filter rolls off with \qty{-40}{\dB \per decade}. At around \qty{10}{\kHz} the noise floor of the measurement is reached at \qty{-55}{\dB}. Then the ground loop shows itself again coupling through the transformers. The magnitude rising with \qty{20}{\dB \per decade} is an artifact, which can be significantly influence by changing the type of probing and the location of probing, so at this point it is clear that for any usable data above \qty{10}{\kHz} a common-mode choke is required. Additionally the noise floor of the measurement is reached at around the same frequency, requiring a lower noise amplifier. Due to the lack of another amplifier and the choke, the author left the measurement as is. It is still a good example to show the pitfalls of a 2- or 3- measurement. This topic will be revisted later in section \ref{sec:results_current_noise}, when measuring the current noise of the driver.

To conclude, the measurements show that the LC supply filter is correctly damped with the expected corner frequency of \qty{300}{\Hz} and will likely perform as intended, but above \qty{10}{\kHz} the filter performance cannot be accurately measured due to the limited setup.


\clearpage
\subsection{TODO: Need title}
The author had actually intended to test the stability of the laser drivers first, but the plans were foiled by a misbehaving lab. First tests revealed, that depending on the temperature of cooling water supplied to the overhead air conditioning unit there were temperature fluctuation of up to \qty{2}{\K} observable in the lab. While the effect is not suprising, the observed temperature fluctuations impose a limit on the observable accuracy, due to the temperature coeficient of the multimeter used to record the data. To illustrate the effect, a sample measurement\footnote{\qty{100}{\mA} range, \qty{10}{\plc}, AZERO ON, $f_s = \qty{0.5}{\Hz}$} is shown in figure \ref{fig:laser_driver_aircon}. The Keysight \device{34470A} used to record the data is specified at \qty{6}{\uA \per \K} for a current of \qty{50}{\mA} on the \qty{100}{\mA} range.

\begin{figure}[ht]
    \centering
    %% Creator: Matplotlib, PGF backend
%%
%% To include the figure in your LaTeX document, write
%%   \input{<filename>.pgf}
%%
%% Make sure the required packages are loaded in your preamble
%%   \usepackage{pgf}
%%
%% Also ensure that all the required font packages are loaded; for instance,
%% the lmodern package is sometimes necessary when using math font.
%%   \usepackage{lmodern}
%%
%% Figures using additional raster images can only be included by \input if
%% they are in the same directory as the main LaTeX file. For loading figures
%% from other directories you can use the `import` package
%%   \usepackage{import}
%%
%% and then include the figures with
%%   \import{<path to file>}{<filename>.pgf}
%%
%% Matplotlib used the following preamble
%%   \usepackage{siunitx}
%%   \usepackage{fontspec}
%%
\begingroup%
\makeatletter%
\begin{pgfpicture}%
\pgfpathrectangle{\pgfpointorigin}{\pgfqpoint{5.492126in}{3.394321in}}%
\pgfusepath{use as bounding box, clip}%
\begin{pgfscope}%
\pgfsetbuttcap%
\pgfsetmiterjoin%
\definecolor{currentfill}{rgb}{1.000000,1.000000,1.000000}%
\pgfsetfillcolor{currentfill}%
\pgfsetlinewidth{0.000000pt}%
\definecolor{currentstroke}{rgb}{1.000000,1.000000,1.000000}%
\pgfsetstrokecolor{currentstroke}%
\pgfsetdash{}{0pt}%
\pgfpathmoveto{\pgfqpoint{0.000000in}{0.000000in}}%
\pgfpathlineto{\pgfqpoint{5.492126in}{0.000000in}}%
\pgfpathlineto{\pgfqpoint{5.492126in}{3.394321in}}%
\pgfpathlineto{\pgfqpoint{0.000000in}{3.394321in}}%
\pgfpathlineto{\pgfqpoint{0.000000in}{0.000000in}}%
\pgfpathclose%
\pgfusepath{fill}%
\end{pgfscope}%
\begin{pgfscope}%
\pgfsetbuttcap%
\pgfsetmiterjoin%
\definecolor{currentfill}{rgb}{1.000000,1.000000,1.000000}%
\pgfsetfillcolor{currentfill}%
\pgfsetlinewidth{0.000000pt}%
\definecolor{currentstroke}{rgb}{0.000000,0.000000,0.000000}%
\pgfsetstrokecolor{currentstroke}%
\pgfsetstrokeopacity{0.000000}%
\pgfsetdash{}{0pt}%
\pgfpathmoveto{\pgfqpoint{0.667540in}{0.539544in}}%
\pgfpathlineto{\pgfqpoint{4.857257in}{0.539544in}}%
\pgfpathlineto{\pgfqpoint{4.857257in}{3.120077in}}%
\pgfpathlineto{\pgfqpoint{0.667540in}{3.120077in}}%
\pgfpathlineto{\pgfqpoint{0.667540in}{0.539544in}}%
\pgfpathclose%
\pgfusepath{fill}%
\end{pgfscope}%
\begin{pgfscope}%
\pgfpathrectangle{\pgfqpoint{0.667540in}{0.539544in}}{\pgfqpoint{4.189718in}{2.580533in}}%
\pgfusepath{clip}%
\pgfsetrectcap%
\pgfsetroundjoin%
\pgfsetlinewidth{0.803000pt}%
\definecolor{currentstroke}{rgb}{0.450000,0.450000,0.450000}%
\pgfsetstrokecolor{currentstroke}%
\pgfsetdash{}{0pt}%
\pgfpathmoveto{\pgfqpoint{0.857962in}{0.539544in}}%
\pgfpathlineto{\pgfqpoint{0.857962in}{3.120077in}}%
\pgfusepath{stroke}%
\end{pgfscope}%
\begin{pgfscope}%
\pgfsetbuttcap%
\pgfsetroundjoin%
\definecolor{currentfill}{rgb}{0.000000,0.000000,0.000000}%
\pgfsetfillcolor{currentfill}%
\pgfsetlinewidth{0.803000pt}%
\definecolor{currentstroke}{rgb}{0.000000,0.000000,0.000000}%
\pgfsetstrokecolor{currentstroke}%
\pgfsetdash{}{0pt}%
\pgfsys@defobject{currentmarker}{\pgfqpoint{0.000000in}{-0.048611in}}{\pgfqpoint{0.000000in}{0.000000in}}{%
\pgfpathmoveto{\pgfqpoint{0.000000in}{0.000000in}}%
\pgfpathlineto{\pgfqpoint{0.000000in}{-0.048611in}}%
\pgfusepath{stroke,fill}%
}%
\begin{pgfscope}%
\pgfsys@transformshift{0.857962in}{0.539544in}%
\pgfsys@useobject{currentmarker}{}%
\end{pgfscope}%
\end{pgfscope}%
\begin{pgfscope}%
\definecolor{textcolor}{rgb}{0.000000,0.000000,0.000000}%
\pgfsetstrokecolor{textcolor}%
\pgfsetfillcolor{textcolor}%
\pgftext[x=0.857962in,y=0.442322in,,top]{\color{textcolor}\rmfamily\fontsize{8.000000}{9.600000}\selectfont \(\displaystyle {00{:}00}\)}%
\end{pgfscope}%
\begin{pgfscope}%
\pgfpathrectangle{\pgfqpoint{0.667540in}{0.539544in}}{\pgfqpoint{4.189718in}{2.580533in}}%
\pgfusepath{clip}%
\pgfsetrectcap%
\pgfsetroundjoin%
\pgfsetlinewidth{0.803000pt}%
\definecolor{currentstroke}{rgb}{0.450000,0.450000,0.450000}%
\pgfsetstrokecolor{currentstroke}%
\pgfsetdash{}{0pt}%
\pgfpathmoveto{\pgfqpoint{1.334069in}{0.539544in}}%
\pgfpathlineto{\pgfqpoint{1.334069in}{3.120077in}}%
\pgfusepath{stroke}%
\end{pgfscope}%
\begin{pgfscope}%
\pgfsetbuttcap%
\pgfsetroundjoin%
\definecolor{currentfill}{rgb}{0.000000,0.000000,0.000000}%
\pgfsetfillcolor{currentfill}%
\pgfsetlinewidth{0.803000pt}%
\definecolor{currentstroke}{rgb}{0.000000,0.000000,0.000000}%
\pgfsetstrokecolor{currentstroke}%
\pgfsetdash{}{0pt}%
\pgfsys@defobject{currentmarker}{\pgfqpoint{0.000000in}{-0.048611in}}{\pgfqpoint{0.000000in}{0.000000in}}{%
\pgfpathmoveto{\pgfqpoint{0.000000in}{0.000000in}}%
\pgfpathlineto{\pgfqpoint{0.000000in}{-0.048611in}}%
\pgfusepath{stroke,fill}%
}%
\begin{pgfscope}%
\pgfsys@transformshift{1.334069in}{0.539544in}%
\pgfsys@useobject{currentmarker}{}%
\end{pgfscope}%
\end{pgfscope}%
\begin{pgfscope}%
\definecolor{textcolor}{rgb}{0.000000,0.000000,0.000000}%
\pgfsetstrokecolor{textcolor}%
\pgfsetfillcolor{textcolor}%
\pgftext[x=1.334069in,y=0.442322in,,top]{\color{textcolor}\rmfamily\fontsize{8.000000}{9.600000}\selectfont \(\displaystyle {03{:}00}\)}%
\end{pgfscope}%
\begin{pgfscope}%
\pgfpathrectangle{\pgfqpoint{0.667540in}{0.539544in}}{\pgfqpoint{4.189718in}{2.580533in}}%
\pgfusepath{clip}%
\pgfsetrectcap%
\pgfsetroundjoin%
\pgfsetlinewidth{0.803000pt}%
\definecolor{currentstroke}{rgb}{0.450000,0.450000,0.450000}%
\pgfsetstrokecolor{currentstroke}%
\pgfsetdash{}{0pt}%
\pgfpathmoveto{\pgfqpoint{1.810176in}{0.539544in}}%
\pgfpathlineto{\pgfqpoint{1.810176in}{3.120077in}}%
\pgfusepath{stroke}%
\end{pgfscope}%
\begin{pgfscope}%
\pgfsetbuttcap%
\pgfsetroundjoin%
\definecolor{currentfill}{rgb}{0.000000,0.000000,0.000000}%
\pgfsetfillcolor{currentfill}%
\pgfsetlinewidth{0.803000pt}%
\definecolor{currentstroke}{rgb}{0.000000,0.000000,0.000000}%
\pgfsetstrokecolor{currentstroke}%
\pgfsetdash{}{0pt}%
\pgfsys@defobject{currentmarker}{\pgfqpoint{0.000000in}{-0.048611in}}{\pgfqpoint{0.000000in}{0.000000in}}{%
\pgfpathmoveto{\pgfqpoint{0.000000in}{0.000000in}}%
\pgfpathlineto{\pgfqpoint{0.000000in}{-0.048611in}}%
\pgfusepath{stroke,fill}%
}%
\begin{pgfscope}%
\pgfsys@transformshift{1.810176in}{0.539544in}%
\pgfsys@useobject{currentmarker}{}%
\end{pgfscope}%
\end{pgfscope}%
\begin{pgfscope}%
\definecolor{textcolor}{rgb}{0.000000,0.000000,0.000000}%
\pgfsetstrokecolor{textcolor}%
\pgfsetfillcolor{textcolor}%
\pgftext[x=1.810176in,y=0.442322in,,top]{\color{textcolor}\rmfamily\fontsize{8.000000}{9.600000}\selectfont \(\displaystyle {06{:}00}\)}%
\end{pgfscope}%
\begin{pgfscope}%
\pgfpathrectangle{\pgfqpoint{0.667540in}{0.539544in}}{\pgfqpoint{4.189718in}{2.580533in}}%
\pgfusepath{clip}%
\pgfsetrectcap%
\pgfsetroundjoin%
\pgfsetlinewidth{0.803000pt}%
\definecolor{currentstroke}{rgb}{0.450000,0.450000,0.450000}%
\pgfsetstrokecolor{currentstroke}%
\pgfsetdash{}{0pt}%
\pgfpathmoveto{\pgfqpoint{2.286283in}{0.539544in}}%
\pgfpathlineto{\pgfqpoint{2.286283in}{3.120077in}}%
\pgfusepath{stroke}%
\end{pgfscope}%
\begin{pgfscope}%
\pgfsetbuttcap%
\pgfsetroundjoin%
\definecolor{currentfill}{rgb}{0.000000,0.000000,0.000000}%
\pgfsetfillcolor{currentfill}%
\pgfsetlinewidth{0.803000pt}%
\definecolor{currentstroke}{rgb}{0.000000,0.000000,0.000000}%
\pgfsetstrokecolor{currentstroke}%
\pgfsetdash{}{0pt}%
\pgfsys@defobject{currentmarker}{\pgfqpoint{0.000000in}{-0.048611in}}{\pgfqpoint{0.000000in}{0.000000in}}{%
\pgfpathmoveto{\pgfqpoint{0.000000in}{0.000000in}}%
\pgfpathlineto{\pgfqpoint{0.000000in}{-0.048611in}}%
\pgfusepath{stroke,fill}%
}%
\begin{pgfscope}%
\pgfsys@transformshift{2.286283in}{0.539544in}%
\pgfsys@useobject{currentmarker}{}%
\end{pgfscope}%
\end{pgfscope}%
\begin{pgfscope}%
\definecolor{textcolor}{rgb}{0.000000,0.000000,0.000000}%
\pgfsetstrokecolor{textcolor}%
\pgfsetfillcolor{textcolor}%
\pgftext[x=2.286283in,y=0.442322in,,top]{\color{textcolor}\rmfamily\fontsize{8.000000}{9.600000}\selectfont \(\displaystyle {09{:}00}\)}%
\end{pgfscope}%
\begin{pgfscope}%
\pgfpathrectangle{\pgfqpoint{0.667540in}{0.539544in}}{\pgfqpoint{4.189718in}{2.580533in}}%
\pgfusepath{clip}%
\pgfsetrectcap%
\pgfsetroundjoin%
\pgfsetlinewidth{0.803000pt}%
\definecolor{currentstroke}{rgb}{0.450000,0.450000,0.450000}%
\pgfsetstrokecolor{currentstroke}%
\pgfsetdash{}{0pt}%
\pgfpathmoveto{\pgfqpoint{2.762390in}{0.539544in}}%
\pgfpathlineto{\pgfqpoint{2.762390in}{3.120077in}}%
\pgfusepath{stroke}%
\end{pgfscope}%
\begin{pgfscope}%
\pgfsetbuttcap%
\pgfsetroundjoin%
\definecolor{currentfill}{rgb}{0.000000,0.000000,0.000000}%
\pgfsetfillcolor{currentfill}%
\pgfsetlinewidth{0.803000pt}%
\definecolor{currentstroke}{rgb}{0.000000,0.000000,0.000000}%
\pgfsetstrokecolor{currentstroke}%
\pgfsetdash{}{0pt}%
\pgfsys@defobject{currentmarker}{\pgfqpoint{0.000000in}{-0.048611in}}{\pgfqpoint{0.000000in}{0.000000in}}{%
\pgfpathmoveto{\pgfqpoint{0.000000in}{0.000000in}}%
\pgfpathlineto{\pgfqpoint{0.000000in}{-0.048611in}}%
\pgfusepath{stroke,fill}%
}%
\begin{pgfscope}%
\pgfsys@transformshift{2.762390in}{0.539544in}%
\pgfsys@useobject{currentmarker}{}%
\end{pgfscope}%
\end{pgfscope}%
\begin{pgfscope}%
\definecolor{textcolor}{rgb}{0.000000,0.000000,0.000000}%
\pgfsetstrokecolor{textcolor}%
\pgfsetfillcolor{textcolor}%
\pgftext[x=2.762390in,y=0.442322in,,top]{\color{textcolor}\rmfamily\fontsize{8.000000}{9.600000}\selectfont \(\displaystyle {12{:}00}\)}%
\end{pgfscope}%
\begin{pgfscope}%
\pgfpathrectangle{\pgfqpoint{0.667540in}{0.539544in}}{\pgfqpoint{4.189718in}{2.580533in}}%
\pgfusepath{clip}%
\pgfsetrectcap%
\pgfsetroundjoin%
\pgfsetlinewidth{0.803000pt}%
\definecolor{currentstroke}{rgb}{0.450000,0.450000,0.450000}%
\pgfsetstrokecolor{currentstroke}%
\pgfsetdash{}{0pt}%
\pgfpathmoveto{\pgfqpoint{3.238497in}{0.539544in}}%
\pgfpathlineto{\pgfqpoint{3.238497in}{3.120077in}}%
\pgfusepath{stroke}%
\end{pgfscope}%
\begin{pgfscope}%
\pgfsetbuttcap%
\pgfsetroundjoin%
\definecolor{currentfill}{rgb}{0.000000,0.000000,0.000000}%
\pgfsetfillcolor{currentfill}%
\pgfsetlinewidth{0.803000pt}%
\definecolor{currentstroke}{rgb}{0.000000,0.000000,0.000000}%
\pgfsetstrokecolor{currentstroke}%
\pgfsetdash{}{0pt}%
\pgfsys@defobject{currentmarker}{\pgfqpoint{0.000000in}{-0.048611in}}{\pgfqpoint{0.000000in}{0.000000in}}{%
\pgfpathmoveto{\pgfqpoint{0.000000in}{0.000000in}}%
\pgfpathlineto{\pgfqpoint{0.000000in}{-0.048611in}}%
\pgfusepath{stroke,fill}%
}%
\begin{pgfscope}%
\pgfsys@transformshift{3.238497in}{0.539544in}%
\pgfsys@useobject{currentmarker}{}%
\end{pgfscope}%
\end{pgfscope}%
\begin{pgfscope}%
\definecolor{textcolor}{rgb}{0.000000,0.000000,0.000000}%
\pgfsetstrokecolor{textcolor}%
\pgfsetfillcolor{textcolor}%
\pgftext[x=3.238497in,y=0.442322in,,top]{\color{textcolor}\rmfamily\fontsize{8.000000}{9.600000}\selectfont \(\displaystyle {15{:}00}\)}%
\end{pgfscope}%
\begin{pgfscope}%
\pgfpathrectangle{\pgfqpoint{0.667540in}{0.539544in}}{\pgfqpoint{4.189718in}{2.580533in}}%
\pgfusepath{clip}%
\pgfsetrectcap%
\pgfsetroundjoin%
\pgfsetlinewidth{0.803000pt}%
\definecolor{currentstroke}{rgb}{0.450000,0.450000,0.450000}%
\pgfsetstrokecolor{currentstroke}%
\pgfsetdash{}{0pt}%
\pgfpathmoveto{\pgfqpoint{3.714604in}{0.539544in}}%
\pgfpathlineto{\pgfqpoint{3.714604in}{3.120077in}}%
\pgfusepath{stroke}%
\end{pgfscope}%
\begin{pgfscope}%
\pgfsetbuttcap%
\pgfsetroundjoin%
\definecolor{currentfill}{rgb}{0.000000,0.000000,0.000000}%
\pgfsetfillcolor{currentfill}%
\pgfsetlinewidth{0.803000pt}%
\definecolor{currentstroke}{rgb}{0.000000,0.000000,0.000000}%
\pgfsetstrokecolor{currentstroke}%
\pgfsetdash{}{0pt}%
\pgfsys@defobject{currentmarker}{\pgfqpoint{0.000000in}{-0.048611in}}{\pgfqpoint{0.000000in}{0.000000in}}{%
\pgfpathmoveto{\pgfqpoint{0.000000in}{0.000000in}}%
\pgfpathlineto{\pgfqpoint{0.000000in}{-0.048611in}}%
\pgfusepath{stroke,fill}%
}%
\begin{pgfscope}%
\pgfsys@transformshift{3.714604in}{0.539544in}%
\pgfsys@useobject{currentmarker}{}%
\end{pgfscope}%
\end{pgfscope}%
\begin{pgfscope}%
\definecolor{textcolor}{rgb}{0.000000,0.000000,0.000000}%
\pgfsetstrokecolor{textcolor}%
\pgfsetfillcolor{textcolor}%
\pgftext[x=3.714604in,y=0.442322in,,top]{\color{textcolor}\rmfamily\fontsize{8.000000}{9.600000}\selectfont \(\displaystyle {18{:}00}\)}%
\end{pgfscope}%
\begin{pgfscope}%
\pgfpathrectangle{\pgfqpoint{0.667540in}{0.539544in}}{\pgfqpoint{4.189718in}{2.580533in}}%
\pgfusepath{clip}%
\pgfsetrectcap%
\pgfsetroundjoin%
\pgfsetlinewidth{0.803000pt}%
\definecolor{currentstroke}{rgb}{0.450000,0.450000,0.450000}%
\pgfsetstrokecolor{currentstroke}%
\pgfsetdash{}{0pt}%
\pgfpathmoveto{\pgfqpoint{4.190711in}{0.539544in}}%
\pgfpathlineto{\pgfqpoint{4.190711in}{3.120077in}}%
\pgfusepath{stroke}%
\end{pgfscope}%
\begin{pgfscope}%
\pgfsetbuttcap%
\pgfsetroundjoin%
\definecolor{currentfill}{rgb}{0.000000,0.000000,0.000000}%
\pgfsetfillcolor{currentfill}%
\pgfsetlinewidth{0.803000pt}%
\definecolor{currentstroke}{rgb}{0.000000,0.000000,0.000000}%
\pgfsetstrokecolor{currentstroke}%
\pgfsetdash{}{0pt}%
\pgfsys@defobject{currentmarker}{\pgfqpoint{0.000000in}{-0.048611in}}{\pgfqpoint{0.000000in}{0.000000in}}{%
\pgfpathmoveto{\pgfqpoint{0.000000in}{0.000000in}}%
\pgfpathlineto{\pgfqpoint{0.000000in}{-0.048611in}}%
\pgfusepath{stroke,fill}%
}%
\begin{pgfscope}%
\pgfsys@transformshift{4.190711in}{0.539544in}%
\pgfsys@useobject{currentmarker}{}%
\end{pgfscope}%
\end{pgfscope}%
\begin{pgfscope}%
\definecolor{textcolor}{rgb}{0.000000,0.000000,0.000000}%
\pgfsetstrokecolor{textcolor}%
\pgfsetfillcolor{textcolor}%
\pgftext[x=4.190711in,y=0.442322in,,top]{\color{textcolor}\rmfamily\fontsize{8.000000}{9.600000}\selectfont \(\displaystyle {21{:}00}\)}%
\end{pgfscope}%
\begin{pgfscope}%
\pgfpathrectangle{\pgfqpoint{0.667540in}{0.539544in}}{\pgfqpoint{4.189718in}{2.580533in}}%
\pgfusepath{clip}%
\pgfsetrectcap%
\pgfsetroundjoin%
\pgfsetlinewidth{0.803000pt}%
\definecolor{currentstroke}{rgb}{0.450000,0.450000,0.450000}%
\pgfsetstrokecolor{currentstroke}%
\pgfsetdash{}{0pt}%
\pgfpathmoveto{\pgfqpoint{4.666818in}{0.539544in}}%
\pgfpathlineto{\pgfqpoint{4.666818in}{3.120077in}}%
\pgfusepath{stroke}%
\end{pgfscope}%
\begin{pgfscope}%
\pgfsetbuttcap%
\pgfsetroundjoin%
\definecolor{currentfill}{rgb}{0.000000,0.000000,0.000000}%
\pgfsetfillcolor{currentfill}%
\pgfsetlinewidth{0.803000pt}%
\definecolor{currentstroke}{rgb}{0.000000,0.000000,0.000000}%
\pgfsetstrokecolor{currentstroke}%
\pgfsetdash{}{0pt}%
\pgfsys@defobject{currentmarker}{\pgfqpoint{0.000000in}{-0.048611in}}{\pgfqpoint{0.000000in}{0.000000in}}{%
\pgfpathmoveto{\pgfqpoint{0.000000in}{0.000000in}}%
\pgfpathlineto{\pgfqpoint{0.000000in}{-0.048611in}}%
\pgfusepath{stroke,fill}%
}%
\begin{pgfscope}%
\pgfsys@transformshift{4.666818in}{0.539544in}%
\pgfsys@useobject{currentmarker}{}%
\end{pgfscope}%
\end{pgfscope}%
\begin{pgfscope}%
\definecolor{textcolor}{rgb}{0.000000,0.000000,0.000000}%
\pgfsetstrokecolor{textcolor}%
\pgfsetfillcolor{textcolor}%
\pgftext[x=4.666818in,y=0.442322in,,top]{\color{textcolor}\rmfamily\fontsize{8.000000}{9.600000}\selectfont \(\displaystyle {00{:}00}\)}%
\end{pgfscope}%
\begin{pgfscope}%
\definecolor{textcolor}{rgb}{0.000000,0.000000,0.000000}%
\pgfsetstrokecolor{textcolor}%
\pgfsetfillcolor{textcolor}%
\pgftext[x=2.762399in,y=0.288100in,,top]{\color{textcolor}\rmfamily\fontsize{10.000000}{12.000000}\selectfont Time (UTC)}%
\end{pgfscope}%
\begin{pgfscope}%
\pgfpathrectangle{\pgfqpoint{0.667540in}{0.539544in}}{\pgfqpoint{4.189718in}{2.580533in}}%
\pgfusepath{clip}%
\pgfsetrectcap%
\pgfsetroundjoin%
\pgfsetlinewidth{0.803000pt}%
\definecolor{currentstroke}{rgb}{0.450000,0.450000,0.450000}%
\pgfsetstrokecolor{currentstroke}%
\pgfsetdash{}{0pt}%
\pgfpathmoveto{\pgfqpoint{0.667540in}{1.033864in}}%
\pgfpathlineto{\pgfqpoint{4.857257in}{1.033864in}}%
\pgfusepath{stroke}%
\end{pgfscope}%
\begin{pgfscope}%
\pgfsetbuttcap%
\pgfsetroundjoin%
\definecolor{currentfill}{rgb}{0.000000,0.000000,0.000000}%
\pgfsetfillcolor{currentfill}%
\pgfsetlinewidth{0.803000pt}%
\definecolor{currentstroke}{rgb}{0.000000,0.000000,0.000000}%
\pgfsetstrokecolor{currentstroke}%
\pgfsetdash{}{0pt}%
\pgfsys@defobject{currentmarker}{\pgfqpoint{-0.048611in}{0.000000in}}{\pgfqpoint{-0.000000in}{0.000000in}}{%
\pgfpathmoveto{\pgfqpoint{-0.000000in}{0.000000in}}%
\pgfpathlineto{\pgfqpoint{-0.048611in}{0.000000in}}%
\pgfusepath{stroke,fill}%
}%
\begin{pgfscope}%
\pgfsys@transformshift{0.667540in}{1.033864in}%
\pgfsys@useobject{currentmarker}{}%
\end{pgfscope}%
\end{pgfscope}%
\begin{pgfscope}%
\definecolor{textcolor}{rgb}{0.000000,0.000000,0.000000}%
\pgfsetstrokecolor{textcolor}%
\pgfsetfillcolor{textcolor}%
\pgftext[x=0.327644in, y=0.995308in, left, base]{\color{textcolor}\rmfamily\fontsize{8.000000}{9.600000}\selectfont \(\displaystyle {\ensuremath{-}1.0}\)}%
\end{pgfscope}%
\begin{pgfscope}%
\pgfpathrectangle{\pgfqpoint{0.667540in}{0.539544in}}{\pgfqpoint{4.189718in}{2.580533in}}%
\pgfusepath{clip}%
\pgfsetrectcap%
\pgfsetroundjoin%
\pgfsetlinewidth{0.803000pt}%
\definecolor{currentstroke}{rgb}{0.450000,0.450000,0.450000}%
\pgfsetstrokecolor{currentstroke}%
\pgfsetdash{}{0pt}%
\pgfpathmoveto{\pgfqpoint{0.667540in}{1.580193in}}%
\pgfpathlineto{\pgfqpoint{4.857257in}{1.580193in}}%
\pgfusepath{stroke}%
\end{pgfscope}%
\begin{pgfscope}%
\pgfsetbuttcap%
\pgfsetroundjoin%
\definecolor{currentfill}{rgb}{0.000000,0.000000,0.000000}%
\pgfsetfillcolor{currentfill}%
\pgfsetlinewidth{0.803000pt}%
\definecolor{currentstroke}{rgb}{0.000000,0.000000,0.000000}%
\pgfsetstrokecolor{currentstroke}%
\pgfsetdash{}{0pt}%
\pgfsys@defobject{currentmarker}{\pgfqpoint{-0.048611in}{0.000000in}}{\pgfqpoint{-0.000000in}{0.000000in}}{%
\pgfpathmoveto{\pgfqpoint{-0.000000in}{0.000000in}}%
\pgfpathlineto{\pgfqpoint{-0.048611in}{0.000000in}}%
\pgfusepath{stroke,fill}%
}%
\begin{pgfscope}%
\pgfsys@transformshift{0.667540in}{1.580193in}%
\pgfsys@useobject{currentmarker}{}%
\end{pgfscope}%
\end{pgfscope}%
\begin{pgfscope}%
\definecolor{textcolor}{rgb}{0.000000,0.000000,0.000000}%
\pgfsetstrokecolor{textcolor}%
\pgfsetfillcolor{textcolor}%
\pgftext[x=0.327644in, y=1.541638in, left, base]{\color{textcolor}\rmfamily\fontsize{8.000000}{9.600000}\selectfont \(\displaystyle {\ensuremath{-}0.5}\)}%
\end{pgfscope}%
\begin{pgfscope}%
\pgfpathrectangle{\pgfqpoint{0.667540in}{0.539544in}}{\pgfqpoint{4.189718in}{2.580533in}}%
\pgfusepath{clip}%
\pgfsetrectcap%
\pgfsetroundjoin%
\pgfsetlinewidth{0.803000pt}%
\definecolor{currentstroke}{rgb}{0.450000,0.450000,0.450000}%
\pgfsetstrokecolor{currentstroke}%
\pgfsetdash{}{0pt}%
\pgfpathmoveto{\pgfqpoint{0.667540in}{2.126523in}}%
\pgfpathlineto{\pgfqpoint{4.857257in}{2.126523in}}%
\pgfusepath{stroke}%
\end{pgfscope}%
\begin{pgfscope}%
\pgfsetbuttcap%
\pgfsetroundjoin%
\definecolor{currentfill}{rgb}{0.000000,0.000000,0.000000}%
\pgfsetfillcolor{currentfill}%
\pgfsetlinewidth{0.803000pt}%
\definecolor{currentstroke}{rgb}{0.000000,0.000000,0.000000}%
\pgfsetstrokecolor{currentstroke}%
\pgfsetdash{}{0pt}%
\pgfsys@defobject{currentmarker}{\pgfqpoint{-0.048611in}{0.000000in}}{\pgfqpoint{-0.000000in}{0.000000in}}{%
\pgfpathmoveto{\pgfqpoint{-0.000000in}{0.000000in}}%
\pgfpathlineto{\pgfqpoint{-0.048611in}{0.000000in}}%
\pgfusepath{stroke,fill}%
}%
\begin{pgfscope}%
\pgfsys@transformshift{0.667540in}{2.126523in}%
\pgfsys@useobject{currentmarker}{}%
\end{pgfscope}%
\end{pgfscope}%
\begin{pgfscope}%
\definecolor{textcolor}{rgb}{0.000000,0.000000,0.000000}%
\pgfsetstrokecolor{textcolor}%
\pgfsetfillcolor{textcolor}%
\pgftext[x=0.419467in, y=2.087967in, left, base]{\color{textcolor}\rmfamily\fontsize{8.000000}{9.600000}\selectfont \(\displaystyle {0.0}\)}%
\end{pgfscope}%
\begin{pgfscope}%
\pgfpathrectangle{\pgfqpoint{0.667540in}{0.539544in}}{\pgfqpoint{4.189718in}{2.580533in}}%
\pgfusepath{clip}%
\pgfsetrectcap%
\pgfsetroundjoin%
\pgfsetlinewidth{0.803000pt}%
\definecolor{currentstroke}{rgb}{0.450000,0.450000,0.450000}%
\pgfsetstrokecolor{currentstroke}%
\pgfsetdash{}{0pt}%
\pgfpathmoveto{\pgfqpoint{0.667540in}{2.672852in}}%
\pgfpathlineto{\pgfqpoint{4.857257in}{2.672852in}}%
\pgfusepath{stroke}%
\end{pgfscope}%
\begin{pgfscope}%
\pgfsetbuttcap%
\pgfsetroundjoin%
\definecolor{currentfill}{rgb}{0.000000,0.000000,0.000000}%
\pgfsetfillcolor{currentfill}%
\pgfsetlinewidth{0.803000pt}%
\definecolor{currentstroke}{rgb}{0.000000,0.000000,0.000000}%
\pgfsetstrokecolor{currentstroke}%
\pgfsetdash{}{0pt}%
\pgfsys@defobject{currentmarker}{\pgfqpoint{-0.048611in}{0.000000in}}{\pgfqpoint{-0.000000in}{0.000000in}}{%
\pgfpathmoveto{\pgfqpoint{-0.000000in}{0.000000in}}%
\pgfpathlineto{\pgfqpoint{-0.048611in}{0.000000in}}%
\pgfusepath{stroke,fill}%
}%
\begin{pgfscope}%
\pgfsys@transformshift{0.667540in}{2.672852in}%
\pgfsys@useobject{currentmarker}{}%
\end{pgfscope}%
\end{pgfscope}%
\begin{pgfscope}%
\definecolor{textcolor}{rgb}{0.000000,0.000000,0.000000}%
\pgfsetstrokecolor{textcolor}%
\pgfsetfillcolor{textcolor}%
\pgftext[x=0.419467in, y=2.634297in, left, base]{\color{textcolor}\rmfamily\fontsize{8.000000}{9.600000}\selectfont \(\displaystyle {0.5}\)}%
\end{pgfscope}%
\begin{pgfscope}%
\definecolor{textcolor}{rgb}{0.000000,0.000000,0.000000}%
\pgfsetstrokecolor{textcolor}%
\pgfsetfillcolor{textcolor}%
\pgftext[x=0.272089in,y=1.829811in,,bottom,rotate=90.000000]{\color{textcolor}\rmfamily\fontsize{10.000000}{12.000000}\selectfont Current deviation in \unit{\A}}%
\end{pgfscope}%
\begin{pgfscope}%
\definecolor{textcolor}{rgb}{0.000000,0.000000,0.000000}%
\pgfsetstrokecolor{textcolor}%
\pgfsetfillcolor{textcolor}%
\pgftext[x=0.667540in,y=3.161744in,left,base]{\color{textcolor}\rmfamily\fontsize{8.000000}{9.600000}\selectfont \(\displaystyle \times{10^{\ensuremath{-}6}}{}\)}%
\end{pgfscope}%
\begin{pgfscope}%
\pgfpathrectangle{\pgfqpoint{0.667540in}{0.539544in}}{\pgfqpoint{4.189718in}{2.580533in}}%
\pgfusepath{clip}%
\pgfsetrectcap%
\pgfsetroundjoin%
\pgfsetlinewidth{1.505625pt}%
\definecolor{currentstroke}{rgb}{0.003922,0.450980,0.698039}%
\pgfsetstrokecolor{currentstroke}%
\pgfsetstrokeopacity{0.700000}%
\pgfsetdash{}{0pt}%
\pgfpathmoveto{\pgfqpoint{0.857982in}{2.539384in}}%
\pgfpathlineto{\pgfqpoint{0.858048in}{2.547688in}}%
\pgfpathlineto{\pgfqpoint{0.858092in}{2.444650in}}%
\pgfpathlineto{\pgfqpoint{0.858996in}{2.232019in}}%
\pgfpathlineto{\pgfqpoint{0.858422in}{2.585931in}}%
\pgfpathlineto{\pgfqpoint{0.859216in}{2.386193in}}%
\pgfpathlineto{\pgfqpoint{0.859767in}{2.236717in}}%
\pgfpathlineto{\pgfqpoint{0.860318in}{2.530096in}}%
\pgfpathlineto{\pgfqpoint{0.860869in}{2.315935in}}%
\pgfpathlineto{\pgfqpoint{0.861067in}{2.575223in}}%
\pgfpathlineto{\pgfqpoint{0.861464in}{2.317792in}}%
\pgfpathlineto{\pgfqpoint{0.862059in}{2.656189in}}%
\pgfpathlineto{\pgfqpoint{0.861552in}{2.294956in}}%
\pgfpathlineto{\pgfqpoint{0.862588in}{2.536215in}}%
\pgfpathlineto{\pgfqpoint{0.863492in}{2.243710in}}%
\pgfpathlineto{\pgfqpoint{0.862787in}{2.621115in}}%
\pgfpathlineto{\pgfqpoint{0.863735in}{2.313750in}}%
\pgfpathlineto{\pgfqpoint{0.864638in}{2.568121in}}%
\pgfpathlineto{\pgfqpoint{0.864418in}{2.254418in}}%
\pgfpathlineto{\pgfqpoint{0.864837in}{2.411433in}}%
\pgfpathlineto{\pgfqpoint{0.865322in}{2.241853in}}%
\pgfpathlineto{\pgfqpoint{0.865718in}{2.504746in}}%
\pgfpathlineto{\pgfqpoint{0.865917in}{2.314514in}}%
\pgfpathlineto{\pgfqpoint{0.866358in}{2.567247in}}%
\pgfpathlineto{\pgfqpoint{0.866953in}{2.254527in}}%
\pgfpathlineto{\pgfqpoint{0.867019in}{2.258133in}}%
\pgfpathlineto{\pgfqpoint{0.867217in}{2.541788in}}%
\pgfpathlineto{\pgfqpoint{0.868143in}{2.483112in}}%
\pgfpathlineto{\pgfqpoint{0.868584in}{2.309488in}}%
\pgfpathlineto{\pgfqpoint{0.868540in}{2.527583in}}%
\pgfpathlineto{\pgfqpoint{0.869179in}{2.393514in}}%
\pgfpathlineto{\pgfqpoint{0.869465in}{2.579375in}}%
\pgfpathlineto{\pgfqpoint{0.869642in}{2.347950in}}%
\pgfpathlineto{\pgfqpoint{0.870303in}{2.501578in}}%
\pgfpathlineto{\pgfqpoint{0.870964in}{2.652037in}}%
\pgfpathlineto{\pgfqpoint{0.870788in}{2.350244in}}%
\pgfpathlineto{\pgfqpoint{0.871383in}{2.447491in}}%
\pgfpathlineto{\pgfqpoint{0.871956in}{2.227539in}}%
\pgfpathlineto{\pgfqpoint{0.872463in}{2.526272in}}%
\pgfpathlineto{\pgfqpoint{0.872507in}{2.321180in}}%
\pgfpathlineto{\pgfqpoint{0.873301in}{2.581014in}}%
\pgfpathlineto{\pgfqpoint{0.872750in}{2.286652in}}%
\pgfpathlineto{\pgfqpoint{0.873631in}{2.391547in}}%
\pgfpathlineto{\pgfqpoint{0.874491in}{2.236936in}}%
\pgfpathlineto{\pgfqpoint{0.874337in}{2.560035in}}%
\pgfpathlineto{\pgfqpoint{0.874734in}{2.300419in}}%
\pgfpathlineto{\pgfqpoint{0.875505in}{2.624611in}}%
\pgfpathlineto{\pgfqpoint{0.875880in}{2.522120in}}%
\pgfpathlineto{\pgfqpoint{0.876321in}{2.320415in}}%
\pgfpathlineto{\pgfqpoint{0.876100in}{2.574677in}}%
\pgfpathlineto{\pgfqpoint{0.877004in}{2.405314in}}%
\pgfpathlineto{\pgfqpoint{0.877996in}{2.669410in}}%
\pgfpathlineto{\pgfqpoint{0.877555in}{2.335603in}}%
\pgfpathlineto{\pgfqpoint{0.878150in}{2.488903in}}%
\pgfpathlineto{\pgfqpoint{0.879120in}{2.420393in}}%
\pgfpathlineto{\pgfqpoint{0.879076in}{2.594891in}}%
\pgfpathlineto{\pgfqpoint{0.879186in}{2.531298in}}%
\pgfpathlineto{\pgfqpoint{0.879781in}{2.709729in}}%
\pgfpathlineto{\pgfqpoint{0.879935in}{2.373737in}}%
\pgfpathlineto{\pgfqpoint{0.880200in}{2.533702in}}%
\pgfpathlineto{\pgfqpoint{0.880420in}{2.351556in}}%
\pgfpathlineto{\pgfqpoint{0.881258in}{2.580796in}}%
\pgfpathlineto{\pgfqpoint{0.881324in}{2.419956in}}%
\pgfpathlineto{\pgfqpoint{0.881368in}{2.460057in}}%
\pgfpathlineto{\pgfqpoint{0.881390in}{2.411870in}}%
\pgfpathlineto{\pgfqpoint{0.881456in}{2.541788in}}%
\pgfpathlineto{\pgfqpoint{0.882184in}{2.268623in}}%
\pgfpathlineto{\pgfqpoint{0.882470in}{2.399742in}}%
\pgfpathlineto{\pgfqpoint{0.883154in}{2.269278in}}%
\pgfpathlineto{\pgfqpoint{0.882647in}{2.622426in}}%
\pgfpathlineto{\pgfqpoint{0.883484in}{2.592924in}}%
\pgfpathlineto{\pgfqpoint{0.883506in}{2.613685in}}%
\pgfpathlineto{\pgfqpoint{0.883749in}{2.239995in}}%
\pgfpathlineto{\pgfqpoint{0.884388in}{2.447819in}}%
\pgfpathlineto{\pgfqpoint{0.885314in}{2.307412in}}%
\pgfpathlineto{\pgfqpoint{0.885380in}{2.605271in}}%
\pgfpathlineto{\pgfqpoint{0.885512in}{2.354178in}}%
\pgfpathlineto{\pgfqpoint{0.886504in}{2.547142in}}%
\pgfpathlineto{\pgfqpoint{0.886085in}{2.260537in}}%
\pgfpathlineto{\pgfqpoint{0.886658in}{2.456997in}}%
\pgfpathlineto{\pgfqpoint{0.886768in}{2.328938in}}%
\pgfpathlineto{\pgfqpoint{0.887430in}{2.674327in}}%
\pgfpathlineto{\pgfqpoint{0.887694in}{2.600026in}}%
\pgfpathlineto{\pgfqpoint{0.888796in}{2.371442in}}%
\pgfpathlineto{\pgfqpoint{0.888355in}{2.635866in}}%
\pgfpathlineto{\pgfqpoint{0.889017in}{2.464974in}}%
\pgfpathlineto{\pgfqpoint{0.889105in}{2.605380in}}%
\pgfpathlineto{\pgfqpoint{0.889722in}{2.348168in}}%
\pgfpathlineto{\pgfqpoint{0.890119in}{2.516001in}}%
\pgfpathlineto{\pgfqpoint{0.891199in}{2.425092in}}%
\pgfpathlineto{\pgfqpoint{0.891045in}{2.653020in}}%
\pgfpathlineto{\pgfqpoint{0.891221in}{2.493492in}}%
\pgfpathlineto{\pgfqpoint{0.891728in}{2.703829in}}%
\pgfpathlineto{\pgfqpoint{0.891419in}{2.442246in}}%
\pgfpathlineto{\pgfqpoint{0.892323in}{2.622535in}}%
\pgfpathlineto{\pgfqpoint{0.892742in}{2.439078in}}%
\pgfpathlineto{\pgfqpoint{0.893117in}{2.674546in}}%
\pgfpathlineto{\pgfqpoint{0.893447in}{2.462133in}}%
\pgfpathlineto{\pgfqpoint{0.893822in}{2.641766in}}%
\pgfpathlineto{\pgfqpoint{0.894175in}{2.366307in}}%
\pgfpathlineto{\pgfqpoint{0.894571in}{2.523540in}}%
\pgfpathlineto{\pgfqpoint{0.894748in}{2.357784in}}%
\pgfpathlineto{\pgfqpoint{0.895012in}{2.698803in}}%
\pgfpathlineto{\pgfqpoint{0.895673in}{2.566372in}}%
\pgfpathlineto{\pgfqpoint{0.896026in}{2.342159in}}%
\pgfpathlineto{\pgfqpoint{0.896709in}{2.624393in}}%
\pgfpathlineto{\pgfqpoint{0.896776in}{2.605490in}}%
\pgfpathlineto{\pgfqpoint{0.897767in}{2.351883in}}%
\pgfpathlineto{\pgfqpoint{0.897569in}{2.717596in}}%
\pgfpathlineto{\pgfqpoint{0.897922in}{2.466394in}}%
\pgfpathlineto{\pgfqpoint{0.898208in}{2.659248in}}%
\pgfpathlineto{\pgfqpoint{0.898120in}{2.403020in}}%
\pgfpathlineto{\pgfqpoint{0.899046in}{2.567356in}}%
\pgfpathlineto{\pgfqpoint{0.899795in}{2.417443in}}%
\pgfpathlineto{\pgfqpoint{0.899421in}{2.677387in}}%
\pgfpathlineto{\pgfqpoint{0.900148in}{2.536980in}}%
\pgfpathlineto{\pgfqpoint{0.900589in}{2.621770in}}%
\pgfpathlineto{\pgfqpoint{0.900897in}{2.388597in}}%
\pgfpathlineto{\pgfqpoint{0.901250in}{2.567137in}}%
\pgfpathlineto{\pgfqpoint{0.901470in}{2.352648in}}%
\pgfpathlineto{\pgfqpoint{0.902066in}{2.580905in}}%
\pgfpathlineto{\pgfqpoint{0.902374in}{2.406626in}}%
\pgfpathlineto{\pgfqpoint{0.902903in}{2.662636in}}%
\pgfpathlineto{\pgfqpoint{0.902440in}{2.370459in}}%
\pgfpathlineto{\pgfqpoint{0.903498in}{2.594017in}}%
\pgfpathlineto{\pgfqpoint{0.903829in}{2.328063in}}%
\pgfpathlineto{\pgfqpoint{0.903542in}{2.648759in}}%
\pgfpathlineto{\pgfqpoint{0.904556in}{2.496551in}}%
\pgfpathlineto{\pgfqpoint{0.904578in}{2.664930in}}%
\pgfpathlineto{\pgfqpoint{0.905174in}{2.398759in}}%
\pgfpathlineto{\pgfqpoint{0.905658in}{2.525835in}}%
\pgfpathlineto{\pgfqpoint{0.906408in}{2.609860in}}%
\pgfpathlineto{\pgfqpoint{0.906430in}{2.375485in}}%
\pgfpathlineto{\pgfqpoint{0.906738in}{2.606582in}}%
\pgfpathlineto{\pgfqpoint{0.907752in}{2.392203in}}%
\pgfpathlineto{\pgfqpoint{0.907201in}{2.685363in}}%
\pgfpathlineto{\pgfqpoint{0.907863in}{2.499283in}}%
\pgfpathlineto{\pgfqpoint{0.908877in}{2.676512in}}%
\pgfpathlineto{\pgfqpoint{0.908722in}{2.433724in}}%
\pgfpathlineto{\pgfqpoint{0.908987in}{2.544410in}}%
\pgfpathlineto{\pgfqpoint{0.909361in}{2.423671in}}%
\pgfpathlineto{\pgfqpoint{0.909163in}{2.706779in}}%
\pgfpathlineto{\pgfqpoint{0.910089in}{2.546049in}}%
\pgfpathlineto{\pgfqpoint{0.910640in}{2.762395in}}%
\pgfpathlineto{\pgfqpoint{0.911103in}{2.413728in}}%
\pgfpathlineto{\pgfqpoint{0.911169in}{2.478523in}}%
\pgfpathlineto{\pgfqpoint{0.911433in}{2.391001in}}%
\pgfpathlineto{\pgfqpoint{0.911852in}{2.650616in}}%
\pgfpathlineto{\pgfqpoint{0.911940in}{2.641766in}}%
\pgfpathlineto{\pgfqpoint{0.911962in}{2.697710in}}%
\pgfpathlineto{\pgfqpoint{0.912580in}{2.425310in}}%
\pgfpathlineto{\pgfqpoint{0.912998in}{2.528566in}}%
\pgfpathlineto{\pgfqpoint{0.913020in}{2.529222in}}%
\pgfpathlineto{\pgfqpoint{0.913087in}{2.379855in}}%
\pgfpathlineto{\pgfqpoint{0.913924in}{2.629091in}}%
\pgfpathlineto{\pgfqpoint{0.914145in}{2.471748in}}%
\pgfpathlineto{\pgfqpoint{0.914211in}{2.652474in}}%
\pgfpathlineto{\pgfqpoint{0.914497in}{2.377779in}}%
\pgfpathlineto{\pgfqpoint{0.915048in}{2.474480in}}%
\pgfpathlineto{\pgfqpoint{0.915070in}{2.316918in}}%
\pgfpathlineto{\pgfqpoint{0.915181in}{2.675966in}}%
\pgfpathlineto{\pgfqpoint{0.916150in}{2.325113in}}%
\pgfpathlineto{\pgfqpoint{0.916217in}{2.491853in}}%
\pgfpathlineto{\pgfqpoint{0.916746in}{2.262941in}}%
\pgfpathlineto{\pgfqpoint{0.917275in}{2.395918in}}%
\pgfpathlineto{\pgfqpoint{0.917297in}{2.297032in}}%
\pgfpathlineto{\pgfqpoint{0.918068in}{2.631276in}}%
\pgfpathlineto{\pgfqpoint{0.918333in}{2.528676in}}%
\pgfpathlineto{\pgfqpoint{0.918972in}{2.710822in}}%
\pgfpathlineto{\pgfqpoint{0.918840in}{2.405314in}}%
\pgfpathlineto{\pgfqpoint{0.919347in}{2.485297in}}%
\pgfpathlineto{\pgfqpoint{0.919523in}{2.391875in}}%
\pgfpathlineto{\pgfqpoint{0.919809in}{2.720765in}}%
\pgfpathlineto{\pgfqpoint{0.920052in}{2.567356in}}%
\pgfpathlineto{\pgfqpoint{0.920074in}{2.725682in}}%
\pgfpathlineto{\pgfqpoint{0.920779in}{2.349261in}}%
\pgfpathlineto{\pgfqpoint{0.921154in}{2.516329in}}%
\pgfpathlineto{\pgfqpoint{0.921463in}{2.390564in}}%
\pgfpathlineto{\pgfqpoint{0.921573in}{2.599699in}}%
\pgfpathlineto{\pgfqpoint{0.922234in}{2.539056in}}%
\pgfpathlineto{\pgfqpoint{0.922565in}{2.337133in}}%
\pgfpathlineto{\pgfqpoint{0.923160in}{2.655096in}}%
\pgfpathlineto{\pgfqpoint{0.923490in}{2.303042in}}%
\pgfpathlineto{\pgfqpoint{0.923821in}{2.666679in}}%
\pgfpathlineto{\pgfqpoint{0.924262in}{2.508899in}}%
\pgfpathlineto{\pgfqpoint{0.924593in}{2.669738in}}%
\pgfpathlineto{\pgfqpoint{0.924328in}{2.421486in}}%
\pgfpathlineto{\pgfqpoint{0.925386in}{2.635866in}}%
\pgfpathlineto{\pgfqpoint{0.925474in}{2.656189in}}%
\pgfpathlineto{\pgfqpoint{0.926510in}{2.398540in}}%
\pgfpathlineto{\pgfqpoint{0.927127in}{2.714756in}}%
\pgfpathlineto{\pgfqpoint{0.927304in}{2.344781in}}%
\pgfpathlineto{\pgfqpoint{0.927612in}{2.460166in}}%
\pgfpathlineto{\pgfqpoint{0.928119in}{2.374064in}}%
\pgfpathlineto{\pgfqpoint{0.927745in}{2.594563in}}%
\pgfpathlineto{\pgfqpoint{0.928207in}{2.525398in}}%
\pgfpathlineto{\pgfqpoint{0.928714in}{2.659686in}}%
\pgfpathlineto{\pgfqpoint{0.928406in}{2.338553in}}%
\pgfpathlineto{\pgfqpoint{0.929287in}{2.460384in}}%
\pgfpathlineto{\pgfqpoint{0.929354in}{2.356910in}}%
\pgfpathlineto{\pgfqpoint{0.929398in}{2.553479in}}%
\pgfpathlineto{\pgfqpoint{0.929816in}{2.540258in}}%
\pgfpathlineto{\pgfqpoint{0.930103in}{2.639690in}}%
\pgfpathlineto{\pgfqpoint{0.930500in}{2.269060in}}%
\pgfpathlineto{\pgfqpoint{0.930874in}{2.407828in}}%
\pgfpathlineto{\pgfqpoint{0.930897in}{2.366525in}}%
\pgfpathlineto{\pgfqpoint{0.931470in}{2.661434in}}%
\pgfpathlineto{\pgfqpoint{0.931910in}{2.525835in}}%
\pgfpathlineto{\pgfqpoint{0.932484in}{2.696071in}}%
\pgfpathlineto{\pgfqpoint{0.932175in}{2.393404in}}%
\pgfpathlineto{\pgfqpoint{0.933013in}{2.579375in}}%
\pgfpathlineto{\pgfqpoint{0.933035in}{2.392858in}}%
\pgfpathlineto{\pgfqpoint{0.933189in}{2.702081in}}%
\pgfpathlineto{\pgfqpoint{0.934115in}{2.639253in}}%
\pgfpathlineto{\pgfqpoint{0.934181in}{2.433942in}}%
\pgfpathlineto{\pgfqpoint{0.935084in}{2.701097in}}%
\pgfpathlineto{\pgfqpoint{0.935173in}{2.613575in}}%
\pgfpathlineto{\pgfqpoint{0.936054in}{2.730599in}}%
\pgfpathlineto{\pgfqpoint{0.935658in}{2.451643in}}%
\pgfpathlineto{\pgfqpoint{0.936275in}{2.594344in}}%
\pgfpathlineto{\pgfqpoint{0.937024in}{2.740542in}}%
\pgfpathlineto{\pgfqpoint{0.936627in}{2.490542in}}%
\pgfpathlineto{\pgfqpoint{0.937355in}{2.626141in}}%
\pgfpathlineto{\pgfqpoint{0.937752in}{2.418208in}}%
\pgfpathlineto{\pgfqpoint{0.937443in}{2.660341in}}%
\pgfpathlineto{\pgfqpoint{0.938479in}{2.473933in}}%
\pgfpathlineto{\pgfqpoint{0.939449in}{2.423016in}}%
\pgfpathlineto{\pgfqpoint{0.939647in}{2.679790in}}%
\pgfpathlineto{\pgfqpoint{0.939890in}{2.461259in}}%
\pgfpathlineto{\pgfqpoint{0.940132in}{2.770809in}}%
\pgfpathlineto{\pgfqpoint{0.940749in}{2.535013in}}%
\pgfpathlineto{\pgfqpoint{0.940859in}{2.749065in}}%
\pgfpathlineto{\pgfqpoint{0.941763in}{2.399742in}}%
\pgfpathlineto{\pgfqpoint{0.941851in}{2.531298in}}%
\pgfpathlineto{\pgfqpoint{0.942182in}{2.259663in}}%
\pgfpathlineto{\pgfqpoint{0.942909in}{2.645262in}}%
\pgfpathlineto{\pgfqpoint{0.942931in}{2.499392in}}%
\pgfpathlineto{\pgfqpoint{0.944011in}{2.645044in}}%
\pgfpathlineto{\pgfqpoint{0.943835in}{2.377779in}}%
\pgfpathlineto{\pgfqpoint{0.944034in}{2.551294in}}%
\pgfpathlineto{\pgfqpoint{0.944739in}{2.316481in}}%
\pgfpathlineto{\pgfqpoint{0.944364in}{2.553916in}}%
\pgfpathlineto{\pgfqpoint{0.945224in}{2.391875in}}%
\pgfpathlineto{\pgfqpoint{0.946238in}{2.611062in}}%
\pgfpathlineto{\pgfqpoint{0.946061in}{2.368929in}}%
\pgfpathlineto{\pgfqpoint{0.946348in}{2.504528in}}%
\pgfpathlineto{\pgfqpoint{0.946370in}{2.504965in}}%
\pgfpathlineto{\pgfqpoint{0.946877in}{2.401709in}}%
\pgfpathlineto{\pgfqpoint{0.947296in}{2.633352in}}%
\pgfpathlineto{\pgfqpoint{0.947450in}{2.573912in}}%
\pgfpathlineto{\pgfqpoint{0.948133in}{2.690826in}}%
\pgfpathlineto{\pgfqpoint{0.948023in}{2.445087in}}%
\pgfpathlineto{\pgfqpoint{0.948244in}{2.555227in}}%
\pgfpathlineto{\pgfqpoint{0.949213in}{2.382696in}}%
\pgfpathlineto{\pgfqpoint{0.948949in}{2.628545in}}%
\pgfpathlineto{\pgfqpoint{0.949346in}{2.409467in}}%
\pgfpathlineto{\pgfqpoint{0.949919in}{2.711696in}}%
\pgfpathlineto{\pgfqpoint{0.950117in}{2.370349in}}%
\pgfpathlineto{\pgfqpoint{0.950492in}{2.639799in}}%
\pgfpathlineto{\pgfqpoint{0.950867in}{2.431538in}}%
\pgfpathlineto{\pgfqpoint{0.951506in}{2.689078in}}%
\pgfpathlineto{\pgfqpoint{0.951594in}{2.495459in}}%
\pgfpathlineto{\pgfqpoint{0.951858in}{2.736609in}}%
\pgfpathlineto{\pgfqpoint{0.952211in}{2.338334in}}%
\pgfpathlineto{\pgfqpoint{0.952674in}{2.509773in}}%
\pgfpathlineto{\pgfqpoint{0.952718in}{2.362482in}}%
\pgfpathlineto{\pgfqpoint{0.953445in}{2.631276in}}%
\pgfpathlineto{\pgfqpoint{0.953776in}{2.537417in}}%
\pgfpathlineto{\pgfqpoint{0.954239in}{2.642203in}}%
\pgfpathlineto{\pgfqpoint{0.954437in}{2.392093in}}%
\pgfpathlineto{\pgfqpoint{0.954856in}{2.606473in}}%
\pgfpathlineto{\pgfqpoint{0.955297in}{2.397775in}}%
\pgfpathlineto{\pgfqpoint{0.955760in}{2.635756in}}%
\pgfpathlineto{\pgfqpoint{0.955958in}{2.562220in}}%
\pgfpathlineto{\pgfqpoint{0.956531in}{2.776928in}}%
\pgfpathlineto{\pgfqpoint{0.956575in}{2.499392in}}%
\pgfpathlineto{\pgfqpoint{0.957060in}{2.701316in}}%
\pgfpathlineto{\pgfqpoint{0.957964in}{2.506713in}}%
\pgfpathlineto{\pgfqpoint{0.957611in}{2.825770in}}%
\pgfpathlineto{\pgfqpoint{0.958162in}{2.740870in}}%
\pgfpathlineto{\pgfqpoint{0.958339in}{2.898978in}}%
\pgfpathlineto{\pgfqpoint{0.958493in}{2.501796in}}%
\pgfpathlineto{\pgfqpoint{0.959220in}{2.700551in}}%
\pgfpathlineto{\pgfqpoint{0.960146in}{2.487482in}}%
\pgfpathlineto{\pgfqpoint{0.959309in}{2.743055in}}%
\pgfpathlineto{\pgfqpoint{0.960345in}{2.559489in}}%
\pgfpathlineto{\pgfqpoint{0.960675in}{2.761631in}}%
\pgfpathlineto{\pgfqpoint{0.960565in}{2.413291in}}%
\pgfpathlineto{\pgfqpoint{0.961447in}{2.654441in}}%
\pgfpathlineto{\pgfqpoint{0.961601in}{2.405424in}}%
\pgfpathlineto{\pgfqpoint{0.962527in}{2.622863in}}%
\pgfpathlineto{\pgfqpoint{0.962769in}{2.758134in}}%
\pgfpathlineto{\pgfqpoint{0.962946in}{2.479397in}}%
\pgfpathlineto{\pgfqpoint{0.963607in}{2.655752in}}%
\pgfpathlineto{\pgfqpoint{0.964312in}{2.440061in}}%
\pgfpathlineto{\pgfqpoint{0.964489in}{2.701097in}}%
\pgfpathlineto{\pgfqpoint{0.964709in}{2.608440in}}%
\pgfpathlineto{\pgfqpoint{0.965524in}{2.538400in}}%
\pgfpathlineto{\pgfqpoint{0.965502in}{2.736609in}}%
\pgfpathlineto{\pgfqpoint{0.965613in}{2.681429in}}%
\pgfpathlineto{\pgfqpoint{0.965635in}{2.769170in}}%
\pgfpathlineto{\pgfqpoint{0.965767in}{2.372535in}}%
\pgfpathlineto{\pgfqpoint{0.966693in}{2.632478in}}%
\pgfpathlineto{\pgfqpoint{0.967023in}{2.436018in}}%
\pgfpathlineto{\pgfqpoint{0.967751in}{2.762286in}}%
\pgfpathlineto{\pgfqpoint{0.967773in}{2.623956in}}%
\pgfpathlineto{\pgfqpoint{0.968037in}{2.707763in}}%
\pgfpathlineto{\pgfqpoint{0.968302in}{2.390782in}}%
\pgfpathlineto{\pgfqpoint{0.968809in}{2.471857in}}%
\pgfpathlineto{\pgfqpoint{0.969382in}{2.422469in}}%
\pgfpathlineto{\pgfqpoint{0.969183in}{2.678916in}}%
\pgfpathlineto{\pgfqpoint{0.969624in}{2.491853in}}%
\pgfpathlineto{\pgfqpoint{0.970550in}{2.750158in}}%
\pgfpathlineto{\pgfqpoint{0.969933in}{2.394716in}}%
\pgfpathlineto{\pgfqpoint{0.970726in}{2.463444in}}%
\pgfpathlineto{\pgfqpoint{0.970947in}{2.334073in}}%
\pgfpathlineto{\pgfqpoint{0.971564in}{2.694651in}}%
\pgfpathlineto{\pgfqpoint{0.971806in}{2.488138in}}%
\pgfpathlineto{\pgfqpoint{0.972181in}{2.679244in}}%
\pgfpathlineto{\pgfqpoint{0.972490in}{2.296595in}}%
\pgfpathlineto{\pgfqpoint{0.972909in}{2.581888in}}%
\pgfpathlineto{\pgfqpoint{0.973989in}{2.363356in}}%
\pgfpathlineto{\pgfqpoint{0.973746in}{2.616963in}}%
\pgfpathlineto{\pgfqpoint{0.974033in}{2.529003in}}%
\pgfpathlineto{\pgfqpoint{0.974231in}{2.418426in}}%
\pgfpathlineto{\pgfqpoint{0.974936in}{2.620350in}}%
\pgfpathlineto{\pgfqpoint{0.975113in}{2.537308in}}%
\pgfpathlineto{\pgfqpoint{0.976281in}{2.694869in}}%
\pgfpathlineto{\pgfqpoint{0.975399in}{2.432959in}}%
\pgfpathlineto{\pgfqpoint{0.976303in}{2.644279in}}%
\pgfpathlineto{\pgfqpoint{0.977405in}{2.347731in}}%
\pgfpathlineto{\pgfqpoint{0.976744in}{2.670284in}}%
\pgfpathlineto{\pgfqpoint{0.977449in}{2.408702in}}%
\pgfpathlineto{\pgfqpoint{0.978022in}{2.631386in}}%
\pgfpathlineto{\pgfqpoint{0.977670in}{2.380730in}}%
\pgfpathlineto{\pgfqpoint{0.978551in}{2.423890in}}%
\pgfpathlineto{\pgfqpoint{0.978595in}{2.390454in}}%
\pgfpathlineto{\pgfqpoint{0.978617in}{2.544082in}}%
\pgfpathlineto{\pgfqpoint{0.978772in}{2.415258in}}%
\pgfpathlineto{\pgfqpoint{0.979213in}{2.648868in}}%
\pgfpathlineto{\pgfqpoint{0.979477in}{2.229615in}}%
\pgfpathlineto{\pgfqpoint{0.979874in}{2.527474in}}%
\pgfpathlineto{\pgfqpoint{0.979962in}{2.234860in}}%
\pgfpathlineto{\pgfqpoint{0.980954in}{2.523322in}}%
\pgfpathlineto{\pgfqpoint{0.981439in}{2.656080in}}%
\pgfpathlineto{\pgfqpoint{0.981152in}{2.377015in}}%
\pgfpathlineto{\pgfqpoint{0.982034in}{2.481473in}}%
\pgfpathlineto{\pgfqpoint{0.982695in}{2.380730in}}%
\pgfpathlineto{\pgfqpoint{0.982916in}{2.671049in}}%
\pgfpathlineto{\pgfqpoint{0.983092in}{2.538400in}}%
\pgfpathlineto{\pgfqpoint{0.983202in}{2.766001in}}%
\pgfpathlineto{\pgfqpoint{0.983797in}{2.398649in}}%
\pgfpathlineto{\pgfqpoint{0.984172in}{2.588663in}}%
\pgfpathlineto{\pgfqpoint{0.984348in}{2.660341in}}%
\pgfpathlineto{\pgfqpoint{0.985318in}{2.372753in}}%
\pgfpathlineto{\pgfqpoint{0.985891in}{2.654659in}}%
\pgfpathlineto{\pgfqpoint{0.985362in}{2.330904in}}%
\pgfpathlineto{\pgfqpoint{0.986464in}{2.486936in}}%
\pgfpathlineto{\pgfqpoint{0.987192in}{2.661980in}}%
\pgfpathlineto{\pgfqpoint{0.986905in}{2.424764in}}%
\pgfpathlineto{\pgfqpoint{0.987611in}{2.615105in}}%
\pgfpathlineto{\pgfqpoint{0.988095in}{2.424327in}}%
\pgfpathlineto{\pgfqpoint{0.988580in}{2.686346in}}%
\pgfpathlineto{\pgfqpoint{0.988713in}{2.613248in}}%
\pgfpathlineto{\pgfqpoint{0.988779in}{2.502998in}}%
\pgfpathlineto{\pgfqpoint{0.989087in}{2.763816in}}%
\pgfpathlineto{\pgfqpoint{0.989153in}{2.646464in}}%
\pgfpathlineto{\pgfqpoint{0.989176in}{2.803261in}}%
\pgfpathlineto{\pgfqpoint{0.989462in}{2.366525in}}%
\pgfpathlineto{\pgfqpoint{0.990256in}{2.641657in}}%
\pgfpathlineto{\pgfqpoint{0.990322in}{2.780206in}}%
\pgfpathlineto{\pgfqpoint{0.990763in}{2.468798in}}%
\pgfpathlineto{\pgfqpoint{0.991336in}{2.580031in}}%
\pgfpathlineto{\pgfqpoint{0.992438in}{2.412089in}}%
\pgfpathlineto{\pgfqpoint{0.991600in}{2.733440in}}%
\pgfpathlineto{\pgfqpoint{0.992460in}{2.507915in}}%
\pgfpathlineto{\pgfqpoint{0.992482in}{2.507260in}}%
\pgfpathlineto{\pgfqpoint{0.993297in}{2.674546in}}%
\pgfpathlineto{\pgfqpoint{0.992989in}{2.360515in}}%
\pgfpathlineto{\pgfqpoint{0.993584in}{2.537854in}}%
\pgfpathlineto{\pgfqpoint{0.994355in}{2.405752in}}%
\pgfpathlineto{\pgfqpoint{0.993870in}{2.709839in}}%
\pgfpathlineto{\pgfqpoint{0.994664in}{2.619476in}}%
\pgfpathlineto{\pgfqpoint{0.995237in}{2.745787in}}%
\pgfpathlineto{\pgfqpoint{0.995391in}{2.417880in}}%
\pgfpathlineto{\pgfqpoint{0.995766in}{2.617618in}}%
\pgfpathlineto{\pgfqpoint{0.996273in}{2.400507in}}%
\pgfpathlineto{\pgfqpoint{0.995810in}{2.648213in}}%
\pgfpathlineto{\pgfqpoint{0.996912in}{2.513816in}}%
\pgfpathlineto{\pgfqpoint{0.997177in}{2.308723in}}%
\pgfpathlineto{\pgfqpoint{0.997463in}{2.732566in}}%
\pgfpathlineto{\pgfqpoint{0.997992in}{2.491088in}}%
\pgfpathlineto{\pgfqpoint{0.998081in}{2.698366in}}%
\pgfpathlineto{\pgfqpoint{0.998301in}{2.417115in}}%
\pgfpathlineto{\pgfqpoint{0.999072in}{2.505402in}}%
\pgfpathlineto{\pgfqpoint{0.999205in}{2.369912in}}%
\pgfpathlineto{\pgfqpoint{0.999954in}{2.720110in}}%
\pgfpathlineto{\pgfqpoint{1.000174in}{2.491525in}}%
\pgfpathlineto{\pgfqpoint{1.000902in}{2.671705in}}%
\pgfpathlineto{\pgfqpoint{1.000417in}{2.328501in}}%
\pgfpathlineto{\pgfqpoint{1.001100in}{2.635319in}}%
\pgfpathlineto{\pgfqpoint{1.001122in}{2.426075in}}%
\pgfpathlineto{\pgfqpoint{1.001299in}{2.718580in}}%
\pgfpathlineto{\pgfqpoint{1.002202in}{2.517421in}}%
\pgfpathlineto{\pgfqpoint{1.002335in}{2.646027in}}%
\pgfpathlineto{\pgfqpoint{1.002974in}{2.382041in}}%
\pgfpathlineto{\pgfqpoint{1.003304in}{2.495568in}}%
\pgfpathlineto{\pgfqpoint{1.004010in}{2.380183in}}%
\pgfpathlineto{\pgfqpoint{1.004296in}{2.774961in}}%
\pgfpathlineto{\pgfqpoint{1.004407in}{2.469344in}}%
\pgfpathlineto{\pgfqpoint{1.005002in}{2.654987in}}%
\pgfpathlineto{\pgfqpoint{1.005046in}{2.440717in}}%
\pgfpathlineto{\pgfqpoint{1.005465in}{2.551512in}}%
\pgfpathlineto{\pgfqpoint{1.005773in}{2.401053in}}%
\pgfpathlineto{\pgfqpoint{1.006280in}{2.793208in}}%
\pgfpathlineto{\pgfqpoint{1.006567in}{2.482893in}}%
\pgfpathlineto{\pgfqpoint{1.006787in}{2.756058in}}%
\pgfpathlineto{\pgfqpoint{1.007316in}{2.463225in}}%
\pgfpathlineto{\pgfqpoint{1.007669in}{2.548781in}}%
\pgfpathlineto{\pgfqpoint{1.008396in}{2.785232in}}%
\pgfpathlineto{\pgfqpoint{1.007757in}{2.505402in}}%
\pgfpathlineto{\pgfqpoint{1.008793in}{2.606254in}}%
\pgfpathlineto{\pgfqpoint{1.008837in}{2.511193in}}%
\pgfpathlineto{\pgfqpoint{1.008969in}{2.758025in}}%
\pgfpathlineto{\pgfqpoint{1.009851in}{2.695197in}}%
\pgfpathlineto{\pgfqpoint{1.009873in}{2.806211in}}%
\pgfpathlineto{\pgfqpoint{1.010534in}{2.437876in}}%
\pgfpathlineto{\pgfqpoint{1.010931in}{2.554899in}}%
\pgfpathlineto{\pgfqpoint{1.011879in}{2.625704in}}%
\pgfpathlineto{\pgfqpoint{1.011328in}{2.347841in}}%
\pgfpathlineto{\pgfqpoint{1.011945in}{2.502015in}}%
\pgfpathlineto{\pgfqpoint{1.012143in}{2.345437in}}%
\pgfpathlineto{\pgfqpoint{1.012254in}{2.642203in}}%
\pgfpathlineto{\pgfqpoint{1.013025in}{2.503435in}}%
\pgfpathlineto{\pgfqpoint{1.013334in}{2.724043in}}%
\pgfpathlineto{\pgfqpoint{1.013179in}{2.358658in}}%
\pgfpathlineto{\pgfqpoint{1.014127in}{2.584401in}}%
\pgfpathlineto{\pgfqpoint{1.014149in}{2.454484in}}%
\pgfpathlineto{\pgfqpoint{1.015141in}{2.732129in}}%
\pgfpathlineto{\pgfqpoint{1.015229in}{2.620022in}}%
\pgfpathlineto{\pgfqpoint{1.015979in}{2.765564in}}%
\pgfpathlineto{\pgfqpoint{1.016155in}{2.490542in}}%
\pgfpathlineto{\pgfqpoint{1.016309in}{2.710057in}}%
\pgfpathlineto{\pgfqpoint{1.017279in}{2.475463in}}%
\pgfpathlineto{\pgfqpoint{1.016684in}{2.764690in}}%
\pgfpathlineto{\pgfqpoint{1.017433in}{2.584511in}}%
\pgfpathlineto{\pgfqpoint{1.017588in}{2.684270in}}%
\pgfpathlineto{\pgfqpoint{1.018183in}{2.434051in}}%
\pgfpathlineto{\pgfqpoint{1.018491in}{2.555992in}}%
\pgfpathlineto{\pgfqpoint{1.018866in}{2.331451in}}%
\pgfpathlineto{\pgfqpoint{1.018690in}{2.659358in}}%
\pgfpathlineto{\pgfqpoint{1.019593in}{2.560909in}}%
\pgfpathlineto{\pgfqpoint{1.020541in}{2.762068in}}%
\pgfpathlineto{\pgfqpoint{1.020343in}{2.350244in}}%
\pgfpathlineto{\pgfqpoint{1.020629in}{2.476447in}}%
\pgfpathlineto{\pgfqpoint{1.020652in}{2.351337in}}%
\pgfpathlineto{\pgfqpoint{1.021158in}{2.634336in}}%
\pgfpathlineto{\pgfqpoint{1.021732in}{2.379091in}}%
\pgfpathlineto{\pgfqpoint{1.022745in}{2.642968in}}%
\pgfpathlineto{\pgfqpoint{1.022900in}{2.582981in}}%
\pgfpathlineto{\pgfqpoint{1.023804in}{2.340083in}}%
\pgfpathlineto{\pgfqpoint{1.022944in}{2.736499in}}%
\pgfpathlineto{\pgfqpoint{1.024090in}{2.402036in}}%
\pgfpathlineto{\pgfqpoint{1.024134in}{2.711150in}}%
\pgfpathlineto{\pgfqpoint{1.024355in}{2.325878in}}%
\pgfpathlineto{\pgfqpoint{1.025192in}{2.552714in}}%
\pgfpathlineto{\pgfqpoint{1.026250in}{2.326862in}}%
\pgfpathlineto{\pgfqpoint{1.025479in}{2.610953in}}%
\pgfpathlineto{\pgfqpoint{1.026316in}{2.375813in}}%
\pgfpathlineto{\pgfqpoint{1.026867in}{2.571180in}}%
\pgfpathlineto{\pgfqpoint{1.026581in}{2.329812in}}%
\pgfpathlineto{\pgfqpoint{1.027485in}{2.564187in}}%
\pgfpathlineto{\pgfqpoint{1.028124in}{2.190061in}}%
\pgfpathlineto{\pgfqpoint{1.027903in}{2.579484in}}%
\pgfpathlineto{\pgfqpoint{1.028609in}{2.520481in}}%
\pgfpathlineto{\pgfqpoint{1.028895in}{2.315607in}}%
\pgfpathlineto{\pgfqpoint{1.029512in}{2.591722in}}%
\pgfpathlineto{\pgfqpoint{1.029689in}{2.446180in}}%
\pgfpathlineto{\pgfqpoint{1.030152in}{2.603523in}}%
\pgfpathlineto{\pgfqpoint{1.030725in}{2.343798in}}%
\pgfpathlineto{\pgfqpoint{1.030813in}{2.535341in}}%
\pgfpathlineto{\pgfqpoint{1.031915in}{2.289165in}}%
\pgfpathlineto{\pgfqpoint{1.031364in}{2.641766in}}%
\pgfpathlineto{\pgfqpoint{1.031959in}{2.464427in}}%
\pgfpathlineto{\pgfqpoint{1.032334in}{2.606473in}}%
\pgfpathlineto{\pgfqpoint{1.032554in}{2.322491in}}%
\pgfpathlineto{\pgfqpoint{1.032951in}{2.441372in}}%
\pgfpathlineto{\pgfqpoint{1.033502in}{2.217814in}}%
\pgfpathlineto{\pgfqpoint{1.033546in}{2.551294in}}%
\pgfpathlineto{\pgfqpoint{1.034053in}{2.427714in}}%
\pgfpathlineto{\pgfqpoint{1.034847in}{2.660887in}}%
\pgfpathlineto{\pgfqpoint{1.034273in}{2.271901in}}%
\pgfpathlineto{\pgfqpoint{1.035155in}{2.476993in}}%
\pgfpathlineto{\pgfqpoint{1.036103in}{2.330030in}}%
\pgfpathlineto{\pgfqpoint{1.036213in}{2.555227in}}%
\pgfpathlineto{\pgfqpoint{1.036235in}{2.442683in}}%
\pgfpathlineto{\pgfqpoint{1.036588in}{2.590739in}}%
\pgfpathlineto{\pgfqpoint{1.036478in}{2.378544in}}%
\pgfpathlineto{\pgfqpoint{1.037337in}{2.441700in}}%
\pgfpathlineto{\pgfqpoint{1.038483in}{2.598934in}}%
\pgfpathlineto{\pgfqpoint{1.037668in}{2.357237in}}%
\pgfpathlineto{\pgfqpoint{1.038528in}{2.503654in}}%
\pgfpathlineto{\pgfqpoint{1.039145in}{2.563094in}}%
\pgfpathlineto{\pgfqpoint{1.039740in}{2.344344in}}%
\pgfpathlineto{\pgfqpoint{1.040357in}{2.579375in}}%
\pgfpathlineto{\pgfqpoint{1.039872in}{2.200878in}}%
\pgfpathlineto{\pgfqpoint{1.040842in}{2.391328in}}%
\pgfpathlineto{\pgfqpoint{1.041151in}{2.257696in}}%
\pgfpathlineto{\pgfqpoint{1.041018in}{2.548671in}}%
\pgfpathlineto{\pgfqpoint{1.041834in}{2.509554in}}%
\pgfpathlineto{\pgfqpoint{1.042760in}{2.640455in}}%
\pgfpathlineto{\pgfqpoint{1.042164in}{2.291459in}}%
\pgfpathlineto{\pgfqpoint{1.042914in}{2.507697in}}%
\pgfpathlineto{\pgfqpoint{1.043906in}{2.365651in}}%
\pgfpathlineto{\pgfqpoint{1.043685in}{2.702408in}}%
\pgfpathlineto{\pgfqpoint{1.043994in}{2.522557in}}%
\pgfpathlineto{\pgfqpoint{1.044810in}{2.622863in}}%
\pgfpathlineto{\pgfqpoint{1.044104in}{2.412417in}}%
\pgfpathlineto{\pgfqpoint{1.045052in}{2.552386in}}%
\pgfpathlineto{\pgfqpoint{1.046088in}{2.332106in}}%
\pgfpathlineto{\pgfqpoint{1.045625in}{2.635647in}}%
\pgfpathlineto{\pgfqpoint{1.046110in}{2.606801in}}%
\pgfpathlineto{\pgfqpoint{1.047168in}{2.740870in}}%
\pgfpathlineto{\pgfqpoint{1.046661in}{2.427058in}}%
\pgfpathlineto{\pgfqpoint{1.047190in}{2.549436in}}%
\pgfpathlineto{\pgfqpoint{1.048182in}{2.353959in}}%
\pgfpathlineto{\pgfqpoint{1.047477in}{2.656735in}}%
\pgfpathlineto{\pgfqpoint{1.048358in}{2.497207in}}%
\pgfpathlineto{\pgfqpoint{1.048380in}{2.604725in}}%
\pgfpathlineto{\pgfqpoint{1.048887in}{2.314514in}}%
\pgfpathlineto{\pgfqpoint{1.049460in}{2.519607in}}%
\pgfpathlineto{\pgfqpoint{1.049615in}{2.584073in}}%
\pgfpathlineto{\pgfqpoint{1.049791in}{2.358112in}}%
\pgfpathlineto{\pgfqpoint{1.050320in}{2.446836in}}%
\pgfpathlineto{\pgfqpoint{1.050827in}{2.364012in}}%
\pgfpathlineto{\pgfqpoint{1.050607in}{2.667116in}}%
\pgfpathlineto{\pgfqpoint{1.051378in}{2.535232in}}%
\pgfpathlineto{\pgfqpoint{1.051444in}{2.605380in}}%
\pgfpathlineto{\pgfqpoint{1.051687in}{2.371551in}}%
\pgfpathlineto{\pgfqpoint{1.052392in}{2.401381in}}%
\pgfpathlineto{\pgfqpoint{1.053097in}{2.279549in}}%
\pgfpathlineto{\pgfqpoint{1.052458in}{2.543536in}}%
\pgfpathlineto{\pgfqpoint{1.053450in}{2.497863in}}%
\pgfpathlineto{\pgfqpoint{1.054354in}{2.685254in}}%
\pgfpathlineto{\pgfqpoint{1.054288in}{2.388488in}}%
\pgfpathlineto{\pgfqpoint{1.054552in}{2.555555in}}%
\pgfpathlineto{\pgfqpoint{1.055588in}{2.308177in}}%
\pgfpathlineto{\pgfqpoint{1.054927in}{2.587789in}}%
\pgfpathlineto{\pgfqpoint{1.055698in}{2.397884in}}%
\pgfpathlineto{\pgfqpoint{1.056426in}{2.546595in}}%
\pgfpathlineto{\pgfqpoint{1.055764in}{2.260756in}}%
\pgfpathlineto{\pgfqpoint{1.056822in}{2.483767in}}%
\pgfpathlineto{\pgfqpoint{1.057021in}{2.584511in}}%
\pgfpathlineto{\pgfqpoint{1.057704in}{2.307194in}}%
\pgfpathlineto{\pgfqpoint{1.057726in}{2.425529in}}%
\pgfpathlineto{\pgfqpoint{1.058299in}{2.239886in}}%
\pgfpathlineto{\pgfqpoint{1.057814in}{2.533156in}}%
\pgfpathlineto{\pgfqpoint{1.058828in}{2.463225in}}%
\pgfpathlineto{\pgfqpoint{1.059005in}{2.276708in}}%
\pgfpathlineto{\pgfqpoint{1.059644in}{2.696727in}}%
\pgfpathlineto{\pgfqpoint{1.059820in}{2.544410in}}%
\pgfpathlineto{\pgfqpoint{1.059842in}{2.656845in}}%
\pgfpathlineto{\pgfqpoint{1.060614in}{2.340848in}}%
\pgfpathlineto{\pgfqpoint{1.060900in}{2.517203in}}%
\pgfpathlineto{\pgfqpoint{1.060966in}{2.330577in}}%
\pgfpathlineto{\pgfqpoint{1.061275in}{2.644060in}}%
\pgfpathlineto{\pgfqpoint{1.062002in}{2.417771in}}%
\pgfpathlineto{\pgfqpoint{1.062862in}{2.765455in}}%
\pgfpathlineto{\pgfqpoint{1.062267in}{2.415148in}}%
\pgfpathlineto{\pgfqpoint{1.063126in}{2.595765in}}%
\pgfpathlineto{\pgfqpoint{1.064052in}{2.711368in}}%
\pgfpathlineto{\pgfqpoint{1.063611in}{2.366525in}}%
\pgfpathlineto{\pgfqpoint{1.064074in}{2.581670in}}%
\pgfpathlineto{\pgfqpoint{1.064647in}{2.468470in}}%
\pgfpathlineto{\pgfqpoint{1.065110in}{2.745896in}}%
\pgfpathlineto{\pgfqpoint{1.065154in}{2.639799in}}%
\pgfpathlineto{\pgfqpoint{1.065198in}{2.784139in}}%
\pgfpathlineto{\pgfqpoint{1.065948in}{2.505074in}}%
\pgfpathlineto{\pgfqpoint{1.066234in}{2.700879in}}%
\pgfpathlineto{\pgfqpoint{1.066874in}{2.488794in}}%
\pgfpathlineto{\pgfqpoint{1.066411in}{2.748519in}}%
\pgfpathlineto{\pgfqpoint{1.067336in}{2.571071in}}%
\pgfpathlineto{\pgfqpoint{1.067733in}{2.760647in}}%
\pgfpathlineto{\pgfqpoint{1.067843in}{2.499392in}}%
\pgfpathlineto{\pgfqpoint{1.068439in}{2.540804in}}%
\pgfpathlineto{\pgfqpoint{1.068901in}{2.426512in}}%
\pgfpathlineto{\pgfqpoint{1.068769in}{2.803042in}}%
\pgfpathlineto{\pgfqpoint{1.069519in}{2.623956in}}%
\pgfpathlineto{\pgfqpoint{1.069651in}{2.437002in}}%
\pgfpathlineto{\pgfqpoint{1.069981in}{2.734096in}}%
\pgfpathlineto{\pgfqpoint{1.070643in}{2.477102in}}%
\pgfpathlineto{\pgfqpoint{1.071524in}{2.708090in}}%
\pgfpathlineto{\pgfqpoint{1.071238in}{2.418099in}}%
\pgfpathlineto{\pgfqpoint{1.071767in}{2.583855in}}%
\pgfpathlineto{\pgfqpoint{1.072472in}{2.803917in}}%
\pgfpathlineto{\pgfqpoint{1.072384in}{2.403348in}}%
\pgfpathlineto{\pgfqpoint{1.072869in}{2.716832in}}%
\pgfpathlineto{\pgfqpoint{1.073067in}{2.496114in}}%
\pgfpathlineto{\pgfqpoint{1.073244in}{2.784358in}}%
\pgfpathlineto{\pgfqpoint{1.073971in}{2.669957in}}%
\pgfpathlineto{\pgfqpoint{1.074214in}{2.517640in}}%
\pgfpathlineto{\pgfqpoint{1.074522in}{2.771792in}}%
\pgfpathlineto{\pgfqpoint{1.075073in}{2.628873in}}%
\pgfpathlineto{\pgfqpoint{1.076087in}{2.762942in}}%
\pgfpathlineto{\pgfqpoint{1.075933in}{2.511849in}}%
\pgfpathlineto{\pgfqpoint{1.076175in}{2.669629in}}%
\pgfpathlineto{\pgfqpoint{1.076197in}{2.669629in}}%
\pgfpathlineto{\pgfqpoint{1.077057in}{2.474261in}}%
\pgfpathlineto{\pgfqpoint{1.076550in}{2.728305in}}%
\pgfpathlineto{\pgfqpoint{1.077211in}{2.616198in}}%
\pgfpathlineto{\pgfqpoint{1.077233in}{2.777911in}}%
\pgfpathlineto{\pgfqpoint{1.077432in}{2.409029in}}%
\pgfpathlineto{\pgfqpoint{1.078313in}{2.609423in}}%
\pgfpathlineto{\pgfqpoint{1.079195in}{2.804135in}}%
\pgfpathlineto{\pgfqpoint{1.079019in}{2.476884in}}%
\pgfpathlineto{\pgfqpoint{1.079371in}{2.531080in}}%
\pgfpathlineto{\pgfqpoint{1.079570in}{2.853305in}}%
\pgfpathlineto{\pgfqpoint{1.079460in}{2.471639in}}%
\pgfpathlineto{\pgfqpoint{1.080540in}{2.636630in}}%
\pgfpathlineto{\pgfqpoint{1.080628in}{2.512177in}}%
\pgfpathlineto{\pgfqpoint{1.081113in}{2.811565in}}%
\pgfpathlineto{\pgfqpoint{1.081642in}{2.600136in}}%
\pgfpathlineto{\pgfqpoint{1.082391in}{2.692137in}}%
\pgfpathlineto{\pgfqpoint{1.081950in}{2.427386in}}%
\pgfpathlineto{\pgfqpoint{1.082501in}{2.547251in}}%
\pgfpathlineto{\pgfqpoint{1.083493in}{2.392312in}}%
\pgfpathlineto{\pgfqpoint{1.083185in}{2.754091in}}%
\pgfpathlineto{\pgfqpoint{1.083559in}{2.509008in}}%
\pgfpathlineto{\pgfqpoint{1.083581in}{2.705686in}}%
\pgfpathlineto{\pgfqpoint{1.084287in}{2.382806in}}%
\pgfpathlineto{\pgfqpoint{1.084661in}{2.491197in}}%
\pgfpathlineto{\pgfqpoint{1.085830in}{2.731364in}}%
\pgfpathlineto{\pgfqpoint{1.084838in}{2.402146in}}%
\pgfpathlineto{\pgfqpoint{1.085852in}{2.672470in}}%
\pgfpathlineto{\pgfqpoint{1.086006in}{2.437002in}}%
\pgfpathlineto{\pgfqpoint{1.086248in}{2.726228in}}%
\pgfpathlineto{\pgfqpoint{1.086954in}{2.529768in}}%
\pgfpathlineto{\pgfqpoint{1.087968in}{2.722841in}}%
\pgfpathlineto{\pgfqpoint{1.087593in}{2.448693in}}%
\pgfpathlineto{\pgfqpoint{1.088078in}{2.608549in}}%
\pgfpathlineto{\pgfqpoint{1.089180in}{2.536871in}}%
\pgfpathlineto{\pgfqpoint{1.088783in}{2.806867in}}%
\pgfpathlineto{\pgfqpoint{1.089202in}{2.567247in}}%
\pgfpathlineto{\pgfqpoint{1.089709in}{2.470437in}}%
\pgfpathlineto{\pgfqpoint{1.089489in}{2.808396in}}%
\pgfpathlineto{\pgfqpoint{1.090128in}{2.607456in}}%
\pgfpathlineto{\pgfqpoint{1.091010in}{2.740761in}}%
\pgfpathlineto{\pgfqpoint{1.090921in}{2.502343in}}%
\pgfpathlineto{\pgfqpoint{1.091230in}{2.708746in}}%
\pgfpathlineto{\pgfqpoint{1.092178in}{2.476884in}}%
\pgfpathlineto{\pgfqpoint{1.091362in}{2.752234in}}%
\pgfpathlineto{\pgfqpoint{1.092398in}{2.529659in}}%
\pgfpathlineto{\pgfqpoint{1.093522in}{2.800311in}}%
\pgfpathlineto{\pgfqpoint{1.093037in}{2.503982in}}%
\pgfpathlineto{\pgfqpoint{1.093633in}{2.684270in}}%
\pgfpathlineto{\pgfqpoint{1.094140in}{2.523759in}}%
\pgfpathlineto{\pgfqpoint{1.094294in}{2.838117in}}%
\pgfpathlineto{\pgfqpoint{1.094713in}{2.687221in}}%
\pgfpathlineto{\pgfqpoint{1.094801in}{2.831124in}}%
\pgfpathlineto{\pgfqpoint{1.095462in}{2.558942in}}%
\pgfpathlineto{\pgfqpoint{1.095815in}{2.686565in}}%
\pgfpathlineto{\pgfqpoint{1.096851in}{2.430336in}}%
\pgfpathlineto{\pgfqpoint{1.096035in}{2.851775in}}%
\pgfpathlineto{\pgfqpoint{1.096939in}{2.588335in}}%
\pgfpathlineto{\pgfqpoint{1.097027in}{2.528457in}}%
\pgfpathlineto{\pgfqpoint{1.097358in}{2.776928in}}%
\pgfpathlineto{\pgfqpoint{1.097865in}{2.654441in}}%
\pgfpathlineto{\pgfqpoint{1.098504in}{2.814188in}}%
\pgfpathlineto{\pgfqpoint{1.098041in}{2.497426in}}%
\pgfpathlineto{\pgfqpoint{1.098967in}{2.645481in}}%
\pgfpathlineto{\pgfqpoint{1.099187in}{2.475572in}}%
\pgfpathlineto{\pgfqpoint{1.099716in}{2.838991in}}%
\pgfpathlineto{\pgfqpoint{1.100069in}{2.656189in}}%
\pgfpathlineto{\pgfqpoint{1.100752in}{2.831233in}}%
\pgfpathlineto{\pgfqpoint{1.100267in}{2.523868in}}%
\pgfpathlineto{\pgfqpoint{1.101193in}{2.776054in}}%
\pgfpathlineto{\pgfqpoint{1.102053in}{2.379418in}}%
\pgfpathlineto{\pgfqpoint{1.102361in}{2.409794in}}%
\pgfpathlineto{\pgfqpoint{1.103419in}{2.660887in}}%
\pgfpathlineto{\pgfqpoint{1.102427in}{2.385428in}}%
\pgfpathlineto{\pgfqpoint{1.103507in}{2.570852in}}%
\pgfpathlineto{\pgfqpoint{1.104455in}{2.401053in}}%
\pgfpathlineto{\pgfqpoint{1.104036in}{2.648213in}}%
\pgfpathlineto{\pgfqpoint{1.104631in}{2.464974in}}%
\pgfpathlineto{\pgfqpoint{1.104720in}{2.323256in}}%
\pgfpathlineto{\pgfqpoint{1.105425in}{2.687221in}}%
\pgfpathlineto{\pgfqpoint{1.105667in}{2.529331in}}%
\pgfpathlineto{\pgfqpoint{1.106593in}{2.708200in}}%
\pgfpathlineto{\pgfqpoint{1.105888in}{2.372972in}}%
\pgfpathlineto{\pgfqpoint{1.106792in}{2.606254in}}%
\pgfpathlineto{\pgfqpoint{1.107299in}{2.402583in}}%
\pgfpathlineto{\pgfqpoint{1.107056in}{2.693012in}}%
\pgfpathlineto{\pgfqpoint{1.107982in}{2.521683in}}%
\pgfpathlineto{\pgfqpoint{1.108026in}{2.727540in}}%
\pgfpathlineto{\pgfqpoint{1.108797in}{2.422360in}}%
\pgfpathlineto{\pgfqpoint{1.109106in}{2.639690in}}%
\pgfpathlineto{\pgfqpoint{1.109415in}{2.456014in}}%
\pgfpathlineto{\pgfqpoint{1.110010in}{2.722076in}}%
\pgfpathlineto{\pgfqpoint{1.110230in}{2.537308in}}%
\pgfpathlineto{\pgfqpoint{1.110274in}{2.687548in}}%
\pgfpathlineto{\pgfqpoint{1.110539in}{2.407391in}}%
\pgfpathlineto{\pgfqpoint{1.111332in}{2.526381in}}%
\pgfpathlineto{\pgfqpoint{1.112368in}{2.402801in}}%
\pgfpathlineto{\pgfqpoint{1.111773in}{2.721967in}}%
\pgfpathlineto{\pgfqpoint{1.112434in}{2.457762in}}%
\pgfpathlineto{\pgfqpoint{1.113448in}{2.713772in}}%
\pgfpathlineto{\pgfqpoint{1.113096in}{2.445197in}}%
\pgfpathlineto{\pgfqpoint{1.113581in}{2.614231in}}%
\pgfpathlineto{\pgfqpoint{1.114088in}{2.725245in}}%
\pgfpathlineto{\pgfqpoint{1.113977in}{2.527474in}}%
\pgfpathlineto{\pgfqpoint{1.114154in}{2.584401in}}%
\pgfpathlineto{\pgfqpoint{1.114330in}{2.740652in}}%
\pgfpathlineto{\pgfqpoint{1.115278in}{2.466285in}}%
\pgfpathlineto{\pgfqpoint{1.115785in}{2.695197in}}%
\pgfpathlineto{\pgfqpoint{1.116336in}{2.449676in}}%
\pgfpathlineto{\pgfqpoint{1.116402in}{2.553807in}}%
\pgfpathlineto{\pgfqpoint{1.117372in}{2.757479in}}%
\pgfpathlineto{\pgfqpoint{1.116622in}{2.387613in}}%
\pgfpathlineto{\pgfqpoint{1.117548in}{2.602430in}}%
\pgfpathlineto{\pgfqpoint{1.118452in}{2.404877in}}%
\pgfpathlineto{\pgfqpoint{1.117989in}{2.646137in}}%
\pgfpathlineto{\pgfqpoint{1.118672in}{2.414274in}}%
\pgfpathlineto{\pgfqpoint{1.119201in}{2.638925in}}%
\pgfpathlineto{\pgfqpoint{1.119664in}{2.337460in}}%
\pgfpathlineto{\pgfqpoint{1.119774in}{2.508680in}}%
\pgfpathlineto{\pgfqpoint{1.120149in}{2.392203in}}%
\pgfpathlineto{\pgfqpoint{1.120259in}{2.663291in}}%
\pgfpathlineto{\pgfqpoint{1.120832in}{2.491307in}}%
\pgfpathlineto{\pgfqpoint{1.121846in}{2.717487in}}%
\pgfpathlineto{\pgfqpoint{1.121427in}{2.409248in}}%
\pgfpathlineto{\pgfqpoint{1.121956in}{2.662089in}}%
\pgfpathlineto{\pgfqpoint{1.122001in}{2.636630in}}%
\pgfpathlineto{\pgfqpoint{1.122023in}{2.677277in}}%
\pgfpathlineto{\pgfqpoint{1.122596in}{2.460822in}}%
\pgfpathlineto{\pgfqpoint{1.123081in}{2.710385in}}%
\pgfpathlineto{\pgfqpoint{1.123125in}{2.603414in}}%
\pgfpathlineto{\pgfqpoint{1.123301in}{2.753217in}}%
\pgfpathlineto{\pgfqpoint{1.123455in}{2.471202in}}%
\pgfpathlineto{\pgfqpoint{1.124227in}{2.624283in}}%
\pgfpathlineto{\pgfqpoint{1.124513in}{2.505402in}}%
\pgfpathlineto{\pgfqpoint{1.125263in}{2.750813in}}%
\pgfpathlineto{\pgfqpoint{1.125307in}{2.689515in}}%
\pgfpathlineto{\pgfqpoint{1.125395in}{2.749721in}}%
\pgfpathlineto{\pgfqpoint{1.126563in}{2.305008in}}%
\pgfpathlineto{\pgfqpoint{1.127599in}{2.653239in}}%
\pgfpathlineto{\pgfqpoint{1.127687in}{2.639253in}}%
\pgfpathlineto{\pgfqpoint{1.128723in}{2.396355in}}%
\pgfpathlineto{\pgfqpoint{1.127842in}{2.724917in}}%
\pgfpathlineto{\pgfqpoint{1.128812in}{2.398868in}}%
\pgfpathlineto{\pgfqpoint{1.129274in}{2.683943in}}%
\pgfpathlineto{\pgfqpoint{1.129671in}{2.397338in}}%
\pgfpathlineto{\pgfqpoint{1.129914in}{2.476884in}}%
\pgfpathlineto{\pgfqpoint{1.130972in}{2.641329in}}%
\pgfpathlineto{\pgfqpoint{1.130465in}{2.337460in}}%
\pgfpathlineto{\pgfqpoint{1.131060in}{2.633352in}}%
\pgfpathlineto{\pgfqpoint{1.131545in}{2.402583in}}%
\pgfpathlineto{\pgfqpoint{1.131743in}{2.784904in}}%
\pgfpathlineto{\pgfqpoint{1.132162in}{2.496770in}}%
\pgfpathlineto{\pgfqpoint{1.132581in}{2.705249in}}%
\pgfpathlineto{\pgfqpoint{1.132294in}{2.438968in}}%
\pgfpathlineto{\pgfqpoint{1.133242in}{2.596202in}}%
\pgfpathlineto{\pgfqpoint{1.133352in}{2.409248in}}%
\pgfpathlineto{\pgfqpoint{1.134124in}{2.724043in}}%
\pgfpathlineto{\pgfqpoint{1.134344in}{2.607456in}}%
\pgfpathlineto{\pgfqpoint{1.135138in}{2.756714in}}%
\pgfpathlineto{\pgfqpoint{1.134697in}{2.508243in}}%
\pgfpathlineto{\pgfqpoint{1.135446in}{2.700988in}}%
\pgfpathlineto{\pgfqpoint{1.135490in}{2.500922in}}%
\pgfpathlineto{\pgfqpoint{1.136262in}{2.764362in}}%
\pgfpathlineto{\pgfqpoint{1.136548in}{2.671486in}}%
\pgfpathlineto{\pgfqpoint{1.137188in}{2.442137in}}%
\pgfpathlineto{\pgfqpoint{1.136879in}{2.704594in}}%
\pgfpathlineto{\pgfqpoint{1.137739in}{2.589100in}}%
\pgfpathlineto{\pgfqpoint{1.138091in}{2.656189in}}%
\pgfpathlineto{\pgfqpoint{1.138290in}{2.378654in}}%
\pgfpathlineto{\pgfqpoint{1.138686in}{2.544410in}}%
\pgfpathlineto{\pgfqpoint{1.138995in}{2.363029in}}%
\pgfpathlineto{\pgfqpoint{1.139744in}{2.666241in}}%
\pgfpathlineto{\pgfqpoint{1.139899in}{2.375048in}}%
\pgfpathlineto{\pgfqpoint{1.141045in}{2.435909in}}%
\pgfpathlineto{\pgfqpoint{1.141684in}{2.709729in}}%
\pgfpathlineto{\pgfqpoint{1.142147in}{2.563641in}}%
\pgfpathlineto{\pgfqpoint{1.142478in}{2.407828in}}%
\pgfpathlineto{\pgfqpoint{1.142830in}{2.760866in}}%
\pgfpathlineto{\pgfqpoint{1.143249in}{2.434379in}}%
\pgfpathlineto{\pgfqpoint{1.144307in}{2.769498in}}%
\pgfpathlineto{\pgfqpoint{1.144395in}{2.590629in}}%
\pgfpathlineto{\pgfqpoint{1.145123in}{2.490323in}}%
\pgfpathlineto{\pgfqpoint{1.144704in}{2.775398in}}%
\pgfpathlineto{\pgfqpoint{1.145453in}{2.609532in}}%
\pgfpathlineto{\pgfqpoint{1.146467in}{2.735844in}}%
\pgfpathlineto{\pgfqpoint{1.145740in}{2.459183in}}%
\pgfpathlineto{\pgfqpoint{1.146577in}{2.720000in}}%
\pgfpathlineto{\pgfqpoint{1.146996in}{2.514471in}}%
\pgfpathlineto{\pgfqpoint{1.147305in}{2.766657in}}%
\pgfpathlineto{\pgfqpoint{1.147702in}{2.639799in}}%
\pgfpathlineto{\pgfqpoint{1.147812in}{2.410778in}}%
\pgfpathlineto{\pgfqpoint{1.148164in}{2.743711in}}%
\pgfpathlineto{\pgfqpoint{1.148804in}{2.626250in}}%
\pgfpathlineto{\pgfqpoint{1.148826in}{2.726338in}}%
\pgfpathlineto{\pgfqpoint{1.149244in}{2.389799in}}%
\pgfpathlineto{\pgfqpoint{1.149862in}{2.490542in}}%
\pgfpathlineto{\pgfqpoint{1.149884in}{2.408155in}}%
\pgfpathlineto{\pgfqpoint{1.150876in}{2.785669in}}%
\pgfpathlineto{\pgfqpoint{1.150898in}{2.719891in}}%
\pgfpathlineto{\pgfqpoint{1.152044in}{2.435690in}}%
\pgfpathlineto{\pgfqpoint{1.151052in}{2.721967in}}%
\pgfpathlineto{\pgfqpoint{1.152154in}{2.533593in}}%
\pgfpathlineto{\pgfqpoint{1.152485in}{2.721967in}}%
\pgfpathlineto{\pgfqpoint{1.152992in}{2.438422in}}%
\pgfpathlineto{\pgfqpoint{1.153256in}{2.589100in}}%
\pgfpathlineto{\pgfqpoint{1.153917in}{2.362154in}}%
\pgfpathlineto{\pgfqpoint{1.153697in}{2.676512in}}%
\pgfpathlineto{\pgfqpoint{1.154292in}{2.526927in}}%
\pgfpathlineto{\pgfqpoint{1.154865in}{2.767859in}}%
\pgfpathlineto{\pgfqpoint{1.154579in}{2.479506in}}%
\pgfpathlineto{\pgfqpoint{1.155372in}{2.524087in}}%
\pgfpathlineto{\pgfqpoint{1.155394in}{2.435472in}}%
\pgfpathlineto{\pgfqpoint{1.156188in}{2.797798in}}%
\pgfpathlineto{\pgfqpoint{1.156430in}{2.678479in}}%
\pgfpathlineto{\pgfqpoint{1.156496in}{2.909686in}}%
\pgfpathlineto{\pgfqpoint{1.157224in}{2.532172in}}%
\pgfpathlineto{\pgfqpoint{1.157510in}{2.660560in}}%
\pgfpathlineto{\pgfqpoint{1.158171in}{2.508243in}}%
\pgfpathlineto{\pgfqpoint{1.158216in}{2.759445in}}%
\pgfpathlineto{\pgfqpoint{1.158634in}{2.536871in}}%
\pgfpathlineto{\pgfqpoint{1.158965in}{2.857129in}}%
\pgfpathlineto{\pgfqpoint{1.159759in}{2.705140in}}%
\pgfpathlineto{\pgfqpoint{1.160442in}{2.545503in}}%
\pgfpathlineto{\pgfqpoint{1.160310in}{2.937767in}}%
\pgfpathlineto{\pgfqpoint{1.160861in}{2.652255in}}%
\pgfpathlineto{\pgfqpoint{1.161742in}{2.834183in}}%
\pgfpathlineto{\pgfqpoint{1.161390in}{2.635101in}}%
\pgfpathlineto{\pgfqpoint{1.161985in}{2.699677in}}%
\pgfpathlineto{\pgfqpoint{1.162844in}{2.577408in}}%
\pgfpathlineto{\pgfqpoint{1.162227in}{2.891329in}}%
\pgfpathlineto{\pgfqpoint{1.163087in}{2.717815in}}%
\pgfpathlineto{\pgfqpoint{1.163616in}{2.856473in}}%
\pgfpathlineto{\pgfqpoint{1.163770in}{2.492181in}}%
\pgfpathlineto{\pgfqpoint{1.164167in}{2.678479in}}%
\pgfpathlineto{\pgfqpoint{1.164784in}{2.535013in}}%
\pgfpathlineto{\pgfqpoint{1.164564in}{2.805993in}}%
\pgfpathlineto{\pgfqpoint{1.165247in}{2.715520in}}%
\pgfpathlineto{\pgfqpoint{1.166173in}{2.796049in}}%
\pgfpathlineto{\pgfqpoint{1.165534in}{2.565061in}}%
\pgfpathlineto{\pgfqpoint{1.166283in}{2.736172in}}%
\pgfpathlineto{\pgfqpoint{1.166371in}{2.509554in}}%
\pgfpathlineto{\pgfqpoint{1.167187in}{2.845875in}}%
\pgfpathlineto{\pgfqpoint{1.167385in}{2.744804in}}%
\pgfpathlineto{\pgfqpoint{1.168465in}{2.508680in}}%
\pgfpathlineto{\pgfqpoint{1.167495in}{2.968908in}}%
\pgfpathlineto{\pgfqpoint{1.168553in}{2.575223in}}%
\pgfpathlineto{\pgfqpoint{1.169435in}{2.756167in}}%
\pgfpathlineto{\pgfqpoint{1.169259in}{2.501796in}}%
\pgfpathlineto{\pgfqpoint{1.169655in}{2.646137in}}%
\pgfpathlineto{\pgfqpoint{1.170052in}{2.534685in}}%
\pgfpathlineto{\pgfqpoint{1.169876in}{2.813641in}}%
\pgfpathlineto{\pgfqpoint{1.170603in}{2.690826in}}%
\pgfpathlineto{\pgfqpoint{1.170625in}{2.755075in}}%
\pgfpathlineto{\pgfqpoint{1.171022in}{2.410559in}}%
\pgfpathlineto{\pgfqpoint{1.171617in}{2.414056in}}%
\pgfpathlineto{\pgfqpoint{1.171948in}{2.401053in}}%
\pgfpathlineto{\pgfqpoint{1.171815in}{2.633243in}}%
\pgfpathlineto{\pgfqpoint{1.172102in}{2.499939in}}%
\pgfpathlineto{\pgfqpoint{1.172984in}{2.775617in}}%
\pgfpathlineto{\pgfqpoint{1.172344in}{2.404113in}}%
\pgfpathlineto{\pgfqpoint{1.173226in}{2.651272in}}%
\pgfpathlineto{\pgfqpoint{1.173623in}{2.725901in}}%
\pgfpathlineto{\pgfqpoint{1.174218in}{2.455030in}}%
\pgfpathlineto{\pgfqpoint{1.174262in}{2.651163in}}%
\pgfpathlineto{\pgfqpoint{1.175188in}{2.335166in}}%
\pgfpathlineto{\pgfqpoint{1.174394in}{2.726556in}}%
\pgfpathlineto{\pgfqpoint{1.175364in}{2.562002in}}%
\pgfpathlineto{\pgfqpoint{1.175805in}{2.700005in}}%
\pgfpathlineto{\pgfqpoint{1.176224in}{2.370786in}}%
\pgfpathlineto{\pgfqpoint{1.176466in}{2.606364in}}%
\pgfpathlineto{\pgfqpoint{1.177348in}{2.397884in}}%
\pgfpathlineto{\pgfqpoint{1.177304in}{2.620022in}}%
\pgfpathlineto{\pgfqpoint{1.177613in}{2.536324in}}%
\pgfpathlineto{\pgfqpoint{1.178340in}{2.733659in}}%
\pgfpathlineto{\pgfqpoint{1.177657in}{2.437766in}}%
\pgfpathlineto{\pgfqpoint{1.178715in}{2.650398in}}%
\pgfpathlineto{\pgfqpoint{1.179662in}{2.327080in}}%
\pgfpathlineto{\pgfqpoint{1.179089in}{2.707216in}}%
\pgfpathlineto{\pgfqpoint{1.179839in}{2.557850in}}%
\pgfpathlineto{\pgfqpoint{1.180390in}{2.738903in}}%
\pgfpathlineto{\pgfqpoint{1.180456in}{2.456888in}}%
\pgfpathlineto{\pgfqpoint{1.180941in}{2.609642in}}%
\pgfpathlineto{\pgfqpoint{1.181911in}{2.429025in}}%
\pgfpathlineto{\pgfqpoint{1.181360in}{2.751250in}}%
\pgfpathlineto{\pgfqpoint{1.181999in}{2.636412in}}%
\pgfpathlineto{\pgfqpoint{1.182881in}{2.522010in}}%
\pgfpathlineto{\pgfqpoint{1.183145in}{2.838445in}}%
\pgfpathlineto{\pgfqpoint{1.184071in}{2.530642in}}%
\pgfpathlineto{\pgfqpoint{1.184269in}{2.662417in}}%
\pgfpathlineto{\pgfqpoint{1.184534in}{2.521355in}}%
\pgfpathlineto{\pgfqpoint{1.184446in}{2.723825in}}%
\pgfpathlineto{\pgfqpoint{1.184732in}{2.698803in}}%
\pgfpathlineto{\pgfqpoint{1.184776in}{2.881714in}}%
\pgfpathlineto{\pgfqpoint{1.185768in}{2.536652in}}%
\pgfpathlineto{\pgfqpoint{1.185812in}{2.649852in}}%
\pgfpathlineto{\pgfqpoint{1.185900in}{2.502015in}}%
\pgfpathlineto{\pgfqpoint{1.186650in}{2.827518in}}%
\pgfpathlineto{\pgfqpoint{1.186936in}{2.601993in}}%
\pgfpathlineto{\pgfqpoint{1.187311in}{2.551512in}}%
\pgfpathlineto{\pgfqpoint{1.187487in}{2.842597in}}%
\pgfpathlineto{\pgfqpoint{1.187972in}{2.641766in}}%
\pgfpathlineto{\pgfqpoint{1.188149in}{2.873628in}}%
\pgfpathlineto{\pgfqpoint{1.188303in}{2.604943in}}%
\pgfpathlineto{\pgfqpoint{1.189096in}{2.773322in}}%
\pgfpathlineto{\pgfqpoint{1.189846in}{2.483986in}}%
\pgfpathlineto{\pgfqpoint{1.189229in}{2.802059in}}%
\pgfpathlineto{\pgfqpoint{1.190221in}{2.661871in}}%
\pgfpathlineto{\pgfqpoint{1.191036in}{2.978742in}}%
\pgfpathlineto{\pgfqpoint{1.191301in}{2.559598in}}%
\pgfpathlineto{\pgfqpoint{1.191323in}{2.775726in}}%
\pgfpathlineto{\pgfqpoint{1.191763in}{2.449349in}}%
\pgfpathlineto{\pgfqpoint{1.191411in}{2.915696in}}%
\pgfpathlineto{\pgfqpoint{1.192447in}{2.709074in}}%
\pgfpathlineto{\pgfqpoint{1.193064in}{2.920285in}}%
\pgfpathlineto{\pgfqpoint{1.192601in}{2.589537in}}%
\pgfpathlineto{\pgfqpoint{1.193549in}{2.772994in}}%
\pgfpathlineto{\pgfqpoint{1.193659in}{2.832981in}}%
\pgfpathlineto{\pgfqpoint{1.194695in}{2.550857in}}%
\pgfpathlineto{\pgfqpoint{1.195621in}{2.765455in}}%
\pgfpathlineto{\pgfqpoint{1.195378in}{2.516547in}}%
\pgfpathlineto{\pgfqpoint{1.195819in}{2.651491in}}%
\pgfpathlineto{\pgfqpoint{1.196216in}{2.460057in}}%
\pgfpathlineto{\pgfqpoint{1.196304in}{2.697601in}}%
\pgfpathlineto{\pgfqpoint{1.196921in}{2.600682in}}%
\pgfpathlineto{\pgfqpoint{1.197847in}{2.814734in}}%
\pgfpathlineto{\pgfqpoint{1.197098in}{2.531298in}}%
\pgfpathlineto{\pgfqpoint{1.198045in}{2.658484in}}%
\pgfpathlineto{\pgfqpoint{1.198332in}{2.517858in}}%
\pgfpathlineto{\pgfqpoint{1.199037in}{2.799546in}}%
\pgfpathlineto{\pgfqpoint{1.199148in}{2.669738in}}%
\pgfpathlineto{\pgfqpoint{1.199456in}{2.551512in}}%
\pgfpathlineto{\pgfqpoint{1.199588in}{2.826207in}}%
\pgfpathlineto{\pgfqpoint{1.200007in}{2.639799in}}%
\pgfpathlineto{\pgfqpoint{1.200492in}{2.532500in}}%
\pgfpathlineto{\pgfqpoint{1.201131in}{2.819214in}}%
\pgfpathlineto{\pgfqpoint{1.201462in}{2.517203in}}%
\pgfpathlineto{\pgfqpoint{1.201197in}{2.875049in}}%
\pgfpathlineto{\pgfqpoint{1.202255in}{2.762942in}}%
\pgfpathlineto{\pgfqpoint{1.202498in}{2.818777in}}%
\pgfpathlineto{\pgfqpoint{1.203115in}{2.612701in}}%
\pgfpathlineto{\pgfqpoint{1.203313in}{2.769388in}}%
\pgfpathlineto{\pgfqpoint{1.203556in}{2.653130in}}%
\pgfpathlineto{\pgfqpoint{1.203688in}{2.928480in}}%
\pgfpathlineto{\pgfqpoint{1.204394in}{2.828174in}}%
\pgfpathlineto{\pgfqpoint{1.204416in}{2.853086in}}%
\pgfpathlineto{\pgfqpoint{1.204967in}{2.638706in}}%
\pgfpathlineto{\pgfqpoint{1.205341in}{2.651928in}}%
\pgfpathlineto{\pgfqpoint{1.205363in}{2.652692in}}%
\pgfpathlineto{\pgfqpoint{1.206157in}{2.916679in}}%
\pgfpathlineto{\pgfqpoint{1.205694in}{2.537745in}}%
\pgfpathlineto{\pgfqpoint{1.206488in}{2.704157in}}%
\pgfpathlineto{\pgfqpoint{1.206686in}{2.853960in}}%
\pgfpathlineto{\pgfqpoint{1.207413in}{2.514253in}}%
\pgfpathlineto{\pgfqpoint{1.207524in}{2.646137in}}%
\pgfpathlineto{\pgfqpoint{1.207876in}{2.524742in}}%
\pgfpathlineto{\pgfqpoint{1.208449in}{2.738357in}}%
\pgfpathlineto{\pgfqpoint{1.208626in}{2.656735in}}%
\pgfpathlineto{\pgfqpoint{1.209750in}{2.828938in}}%
\pgfpathlineto{\pgfqpoint{1.209133in}{2.565826in}}%
\pgfpathlineto{\pgfqpoint{1.209772in}{2.803370in}}%
\pgfpathlineto{\pgfqpoint{1.210169in}{2.513378in}}%
\pgfpathlineto{\pgfqpoint{1.209926in}{2.867400in}}%
\pgfpathlineto{\pgfqpoint{1.210896in}{2.641220in}}%
\pgfpathlineto{\pgfqpoint{1.211050in}{2.774633in}}%
\pgfpathlineto{\pgfqpoint{1.211601in}{2.450878in}}%
\pgfpathlineto{\pgfqpoint{1.211954in}{2.607566in}}%
\pgfpathlineto{\pgfqpoint{1.212152in}{2.451971in}}%
\pgfpathlineto{\pgfqpoint{1.212792in}{2.791897in}}%
\pgfpathlineto{\pgfqpoint{1.213056in}{2.658484in}}%
\pgfpathlineto{\pgfqpoint{1.213453in}{2.541351in}}%
\pgfpathlineto{\pgfqpoint{1.213894in}{2.853086in}}%
\pgfpathlineto{\pgfqpoint{1.214114in}{2.729834in}}%
\pgfpathlineto{\pgfqpoint{1.214180in}{2.645809in}}%
\pgfpathlineto{\pgfqpoint{1.214224in}{2.799218in}}%
\pgfpathlineto{\pgfqpoint{1.214445in}{2.709839in}}%
\pgfpathlineto{\pgfqpoint{1.214886in}{2.544191in}}%
\pgfpathlineto{\pgfqpoint{1.215370in}{2.798235in}}%
\pgfpathlineto{\pgfqpoint{1.215569in}{2.573475in}}%
\pgfpathlineto{\pgfqpoint{1.216274in}{2.801076in}}%
\pgfpathlineto{\pgfqpoint{1.216142in}{2.513488in}}%
\pgfpathlineto{\pgfqpoint{1.216693in}{2.633025in}}%
\pgfpathlineto{\pgfqpoint{1.216715in}{2.469235in}}%
\pgfpathlineto{\pgfqpoint{1.217134in}{2.825442in}}%
\pgfpathlineto{\pgfqpoint{1.217795in}{2.606254in}}%
\pgfpathlineto{\pgfqpoint{1.218192in}{2.877562in}}%
\pgfpathlineto{\pgfqpoint{1.218302in}{2.514690in}}%
\pgfpathlineto{\pgfqpoint{1.218875in}{2.573584in}}%
\pgfpathlineto{\pgfqpoint{1.219206in}{2.435472in}}%
\pgfpathlineto{\pgfqpoint{1.219492in}{2.705577in}}%
\pgfpathlineto{\pgfqpoint{1.219845in}{2.579812in}}%
\pgfpathlineto{\pgfqpoint{1.220220in}{2.907610in}}%
\pgfpathlineto{\pgfqpoint{1.220286in}{2.421377in}}%
\pgfpathlineto{\pgfqpoint{1.220969in}{2.771027in}}%
\pgfpathlineto{\pgfqpoint{1.221454in}{2.477758in}}%
\pgfpathlineto{\pgfqpoint{1.221256in}{2.822055in}}%
\pgfpathlineto{\pgfqpoint{1.222093in}{2.668536in}}%
\pgfpathlineto{\pgfqpoint{1.222424in}{2.809926in}}%
\pgfpathlineto{\pgfqpoint{1.222666in}{2.577408in}}%
\pgfpathlineto{\pgfqpoint{1.223217in}{2.767094in}}%
\pgfpathlineto{\pgfqpoint{1.224209in}{2.593798in}}%
\pgfpathlineto{\pgfqpoint{1.223945in}{2.887942in}}%
\pgfpathlineto{\pgfqpoint{1.224297in}{2.779222in}}%
\pgfpathlineto{\pgfqpoint{1.224959in}{2.852758in}}%
\pgfpathlineto{\pgfqpoint{1.224474in}{2.557631in}}%
\pgfpathlineto{\pgfqpoint{1.225355in}{2.739340in}}%
\pgfpathlineto{\pgfqpoint{1.226303in}{2.541132in}}%
\pgfpathlineto{\pgfqpoint{1.226149in}{2.821399in}}%
\pgfpathlineto{\pgfqpoint{1.226458in}{2.704485in}}%
\pgfpathlineto{\pgfqpoint{1.227185in}{2.806211in}}%
\pgfpathlineto{\pgfqpoint{1.227009in}{2.470000in}}%
\pgfpathlineto{\pgfqpoint{1.227494in}{2.636084in}}%
\pgfpathlineto{\pgfqpoint{1.227956in}{2.475791in}}%
\pgfpathlineto{\pgfqpoint{1.228265in}{2.815280in}}%
\pgfpathlineto{\pgfqpoint{1.228596in}{2.621442in}}%
\pgfpathlineto{\pgfqpoint{1.229367in}{2.424764in}}%
\pgfpathlineto{\pgfqpoint{1.228816in}{2.817684in}}%
\pgfpathlineto{\pgfqpoint{1.229676in}{2.640018in}}%
\pgfpathlineto{\pgfqpoint{1.229698in}{2.639471in}}%
\pgfpathlineto{\pgfqpoint{1.229940in}{2.783921in}}%
\pgfpathlineto{\pgfqpoint{1.230293in}{2.506058in}}%
\pgfpathlineto{\pgfqpoint{1.230778in}{2.665367in}}%
\pgfpathlineto{\pgfqpoint{1.230844in}{2.467268in}}%
\pgfpathlineto{\pgfqpoint{1.231351in}{2.797688in}}%
\pgfpathlineto{\pgfqpoint{1.231880in}{2.653567in}}%
\pgfpathlineto{\pgfqpoint{1.232519in}{2.532937in}}%
\pgfpathlineto{\pgfqpoint{1.232299in}{2.797470in}}%
\pgfpathlineto{\pgfqpoint{1.232982in}{2.649196in}}%
\pgfpathlineto{\pgfqpoint{1.233335in}{2.886303in}}%
\pgfpathlineto{\pgfqpoint{1.233136in}{2.573038in}}%
\pgfpathlineto{\pgfqpoint{1.234084in}{2.661871in}}%
\pgfpathlineto{\pgfqpoint{1.234172in}{2.652583in}}%
\pgfpathlineto{\pgfqpoint{1.234194in}{2.752999in}}%
\pgfpathlineto{\pgfqpoint{1.235032in}{2.872426in}}%
\pgfpathlineto{\pgfqpoint{1.234547in}{2.630512in}}%
\pgfpathlineto{\pgfqpoint{1.235318in}{2.844782in}}%
\pgfpathlineto{\pgfqpoint{1.236332in}{2.576862in}}%
\pgfpathlineto{\pgfqpoint{1.236487in}{2.660887in}}%
\pgfpathlineto{\pgfqpoint{1.237016in}{2.838007in}}%
\pgfpathlineto{\pgfqpoint{1.236663in}{2.600354in}}%
\pgfpathlineto{\pgfqpoint{1.237589in}{2.722404in}}%
\pgfpathlineto{\pgfqpoint{1.238074in}{2.545721in}}%
\pgfpathlineto{\pgfqpoint{1.237699in}{2.812439in}}%
\pgfpathlineto{\pgfqpoint{1.238713in}{2.672907in}}%
\pgfpathlineto{\pgfqpoint{1.238867in}{2.802824in}}%
\pgfpathlineto{\pgfqpoint{1.238757in}{2.569541in}}%
\pgfpathlineto{\pgfqpoint{1.239837in}{2.729834in}}%
\pgfpathlineto{\pgfqpoint{1.240344in}{2.848169in}}%
\pgfpathlineto{\pgfqpoint{1.240432in}{2.643842in}}%
\pgfpathlineto{\pgfqpoint{1.240719in}{2.730053in}}%
\pgfpathlineto{\pgfqpoint{1.240741in}{2.504309in}}%
\pgfpathlineto{\pgfqpoint{1.241755in}{2.854288in}}%
\pgfpathlineto{\pgfqpoint{1.241799in}{2.655096in}}%
\pgfpathlineto{\pgfqpoint{1.242879in}{2.846421in}}%
\pgfpathlineto{\pgfqpoint{1.242085in}{2.577080in}}%
\pgfpathlineto{\pgfqpoint{1.242901in}{2.748956in}}%
\pgfpathlineto{\pgfqpoint{1.243342in}{2.640673in}}%
\pgfpathlineto{\pgfqpoint{1.243165in}{2.842706in}}%
\pgfpathlineto{\pgfqpoint{1.243849in}{2.684380in}}%
\pgfpathlineto{\pgfqpoint{1.244047in}{2.902037in}}%
\pgfpathlineto{\pgfqpoint{1.244708in}{2.577299in}}%
\pgfpathlineto{\pgfqpoint{1.244951in}{2.690389in}}%
\pgfpathlineto{\pgfqpoint{1.245105in}{2.544956in}}%
\pgfpathlineto{\pgfqpoint{1.246075in}{2.793536in}}%
\pgfpathlineto{\pgfqpoint{1.246472in}{2.595000in}}%
\pgfpathlineto{\pgfqpoint{1.246736in}{2.899415in}}%
\pgfpathlineto{\pgfqpoint{1.247199in}{2.663619in}}%
\pgfpathlineto{\pgfqpoint{1.247904in}{2.947492in}}%
\pgfpathlineto{\pgfqpoint{1.248169in}{2.546158in}}%
\pgfpathlineto{\pgfqpoint{1.248367in}{2.890674in}}%
\pgfpathlineto{\pgfqpoint{1.248411in}{2.901382in}}%
\pgfpathlineto{\pgfqpoint{1.249558in}{2.494257in}}%
\pgfpathlineto{\pgfqpoint{1.250572in}{2.840411in}}%
\pgfpathlineto{\pgfqpoint{1.250682in}{2.712679in}}%
\pgfpathlineto{\pgfqpoint{1.250968in}{2.497863in}}%
\pgfpathlineto{\pgfqpoint{1.251674in}{2.880621in}}%
\pgfpathlineto{\pgfqpoint{1.251762in}{2.738138in}}%
\pgfpathlineto{\pgfqpoint{1.252225in}{2.904660in}}%
\pgfpathlineto{\pgfqpoint{1.252688in}{2.572382in}}%
\pgfpathlineto{\pgfqpoint{1.252864in}{2.711368in}}%
\pgfpathlineto{\pgfqpoint{1.253128in}{2.511849in}}%
\pgfpathlineto{\pgfqpoint{1.253657in}{2.827736in}}%
\pgfpathlineto{\pgfqpoint{1.253966in}{2.686783in}}%
\pgfpathlineto{\pgfqpoint{1.254076in}{2.821071in}}%
\pgfpathlineto{\pgfqpoint{1.254561in}{2.533811in}}%
\pgfpathlineto{\pgfqpoint{1.254715in}{2.604834in}}%
\pgfpathlineto{\pgfqpoint{1.254737in}{2.499939in}}%
\pgfpathlineto{\pgfqpoint{1.255707in}{2.815499in}}%
\pgfpathlineto{\pgfqpoint{1.255751in}{2.746443in}}%
\pgfpathlineto{\pgfqpoint{1.256677in}{2.917007in}}%
\pgfpathlineto{\pgfqpoint{1.255928in}{2.588881in}}%
\pgfpathlineto{\pgfqpoint{1.256831in}{2.713335in}}%
\pgfpathlineto{\pgfqpoint{1.257471in}{2.513816in}}%
\pgfpathlineto{\pgfqpoint{1.257316in}{2.789056in}}%
\pgfpathlineto{\pgfqpoint{1.257934in}{2.735516in}}%
\pgfpathlineto{\pgfqpoint{1.258154in}{2.838445in}}%
\pgfpathlineto{\pgfqpoint{1.258264in}{2.482893in}}%
\pgfpathlineto{\pgfqpoint{1.258992in}{2.586587in}}%
\pgfpathlineto{\pgfqpoint{1.260138in}{2.837243in}}%
\pgfpathlineto{\pgfqpoint{1.259366in}{2.467268in}}%
\pgfpathlineto{\pgfqpoint{1.260226in}{2.707763in}}%
\pgfpathlineto{\pgfqpoint{1.260645in}{2.592268in}}%
\pgfpathlineto{\pgfqpoint{1.260424in}{2.811347in}}%
\pgfpathlineto{\pgfqpoint{1.261350in}{2.650398in}}%
\pgfpathlineto{\pgfqpoint{1.262166in}{2.800529in}}%
\pgfpathlineto{\pgfqpoint{1.262386in}{2.480489in}}%
\pgfpathlineto{\pgfqpoint{1.262408in}{2.541460in}}%
\pgfpathlineto{\pgfqpoint{1.263246in}{2.452190in}}%
\pgfpathlineto{\pgfqpoint{1.262805in}{2.756167in}}%
\pgfpathlineto{\pgfqpoint{1.263510in}{2.462788in}}%
\pgfpathlineto{\pgfqpoint{1.264105in}{2.722623in}}%
\pgfpathlineto{\pgfqpoint{1.264656in}{2.549545in}}%
\pgfpathlineto{\pgfqpoint{1.265318in}{2.841613in}}%
\pgfpathlineto{\pgfqpoint{1.264833in}{2.479397in}}%
\pgfpathlineto{\pgfqpoint{1.265847in}{2.676075in}}%
\pgfpathlineto{\pgfqpoint{1.266684in}{2.548562in}}%
\pgfpathlineto{\pgfqpoint{1.266199in}{2.834948in}}%
\pgfpathlineto{\pgfqpoint{1.266949in}{2.686346in}}%
\pgfpathlineto{\pgfqpoint{1.268029in}{2.834292in}}%
\pgfpathlineto{\pgfqpoint{1.267279in}{2.540039in}}%
\pgfpathlineto{\pgfqpoint{1.268073in}{2.765783in}}%
\pgfpathlineto{\pgfqpoint{1.268800in}{2.478850in}}%
\pgfpathlineto{\pgfqpoint{1.268271in}{2.801731in}}%
\pgfpathlineto{\pgfqpoint{1.269197in}{2.604615in}}%
\pgfpathlineto{\pgfqpoint{1.269506in}{2.850136in}}%
\pgfpathlineto{\pgfqpoint{1.270035in}{2.478086in}}%
\pgfpathlineto{\pgfqpoint{1.270255in}{2.565389in}}%
\pgfpathlineto{\pgfqpoint{1.270850in}{2.440607in}}%
\pgfpathlineto{\pgfqpoint{1.271313in}{2.732238in}}%
\pgfpathlineto{\pgfqpoint{1.271335in}{2.659904in}}%
\pgfpathlineto{\pgfqpoint{1.271357in}{2.661543in}}%
\pgfpathlineto{\pgfqpoint{1.271379in}{2.651600in}}%
\pgfpathlineto{\pgfqpoint{1.271864in}{2.474698in}}%
\pgfpathlineto{\pgfqpoint{1.272085in}{2.762395in}}%
\pgfpathlineto{\pgfqpoint{1.272481in}{2.682194in}}%
\pgfpathlineto{\pgfqpoint{1.272966in}{2.525616in}}%
\pgfpathlineto{\pgfqpoint{1.273143in}{2.810363in}}%
\pgfpathlineto{\pgfqpoint{1.273583in}{2.582107in}}%
\pgfpathlineto{\pgfqpoint{1.274333in}{2.462570in}}%
\pgfpathlineto{\pgfqpoint{1.274730in}{2.814515in}}%
\pgfpathlineto{\pgfqpoint{1.275303in}{2.535341in}}%
\pgfpathlineto{\pgfqpoint{1.275082in}{2.879529in}}%
\pgfpathlineto{\pgfqpoint{1.275832in}{2.657937in}}%
\pgfpathlineto{\pgfqpoint{1.276140in}{2.849371in}}%
\pgfpathlineto{\pgfqpoint{1.276405in}{2.581560in}}%
\pgfpathlineto{\pgfqpoint{1.276912in}{2.706124in}}%
\pgfpathlineto{\pgfqpoint{1.276934in}{2.598387in}}%
\pgfpathlineto{\pgfqpoint{1.277529in}{2.834839in}}%
\pgfpathlineto{\pgfqpoint{1.278014in}{2.699458in}}%
\pgfpathlineto{\pgfqpoint{1.278719in}{2.594344in}}%
\pgfpathlineto{\pgfqpoint{1.278499in}{2.864013in}}%
\pgfpathlineto{\pgfqpoint{1.279116in}{2.671923in}}%
\pgfpathlineto{\pgfqpoint{1.279138in}{2.809052in}}%
\pgfpathlineto{\pgfqpoint{1.279689in}{2.443448in}}%
\pgfpathlineto{\pgfqpoint{1.280218in}{2.766001in}}%
\pgfpathlineto{\pgfqpoint{1.280571in}{2.489777in}}%
\pgfpathlineto{\pgfqpoint{1.280945in}{2.845328in}}%
\pgfpathlineto{\pgfqpoint{1.281320in}{2.756386in}}%
\pgfpathlineto{\pgfqpoint{1.282400in}{2.556429in}}%
\pgfpathlineto{\pgfqpoint{1.281408in}{2.806757in}}%
\pgfpathlineto{\pgfqpoint{1.282444in}{2.729616in}}%
\pgfpathlineto{\pgfqpoint{1.282466in}{2.818012in}}%
\pgfpathlineto{\pgfqpoint{1.283017in}{2.576534in}}%
\pgfpathlineto{\pgfqpoint{1.283524in}{2.609860in}}%
\pgfpathlineto{\pgfqpoint{1.284516in}{2.912090in}}%
\pgfpathlineto{\pgfqpoint{1.284626in}{2.677605in}}%
\pgfpathlineto{\pgfqpoint{1.284670in}{2.599152in}}%
\pgfpathlineto{\pgfqpoint{1.285199in}{2.896574in}}%
\pgfpathlineto{\pgfqpoint{1.285684in}{2.790586in}}%
\pgfpathlineto{\pgfqpoint{1.285773in}{2.863466in}}%
\pgfpathlineto{\pgfqpoint{1.285728in}{2.766329in}}%
\pgfpathlineto{\pgfqpoint{1.285905in}{2.775726in}}%
\pgfpathlineto{\pgfqpoint{1.286831in}{2.370677in}}%
\pgfpathlineto{\pgfqpoint{1.286037in}{2.845656in}}%
\pgfpathlineto{\pgfqpoint{1.287073in}{2.598715in}}%
\pgfpathlineto{\pgfqpoint{1.288065in}{2.257041in}}%
\pgfpathlineto{\pgfqpoint{1.288241in}{2.511630in}}%
\pgfpathlineto{\pgfqpoint{1.289167in}{2.592815in}}%
\pgfpathlineto{\pgfqpoint{1.288858in}{2.298343in}}%
\pgfpathlineto{\pgfqpoint{1.289277in}{2.319650in}}%
\pgfpathlineto{\pgfqpoint{1.290049in}{2.124173in}}%
\pgfpathlineto{\pgfqpoint{1.289432in}{2.462023in}}%
\pgfpathlineto{\pgfqpoint{1.290423in}{2.213006in}}%
\pgfpathlineto{\pgfqpoint{1.291217in}{2.367946in}}%
\pgfpathlineto{\pgfqpoint{1.291459in}{2.063203in}}%
\pgfpathlineto{\pgfqpoint{1.291526in}{2.213771in}}%
\pgfpathlineto{\pgfqpoint{1.291988in}{2.439405in}}%
\pgfpathlineto{\pgfqpoint{1.291702in}{2.128762in}}%
\pgfpathlineto{\pgfqpoint{1.292694in}{2.355926in}}%
\pgfpathlineto{\pgfqpoint{1.293487in}{2.050200in}}%
\pgfpathlineto{\pgfqpoint{1.293135in}{2.427058in}}%
\pgfpathlineto{\pgfqpoint{1.293840in}{2.204156in}}%
\pgfpathlineto{\pgfqpoint{1.294149in}{2.435144in}}%
\pgfpathlineto{\pgfqpoint{1.294545in}{2.106691in}}%
\pgfpathlineto{\pgfqpoint{1.294678in}{2.161433in}}%
\pgfpathlineto{\pgfqpoint{1.295251in}{1.937110in}}%
\pgfpathlineto{\pgfqpoint{1.294964in}{2.351665in}}%
\pgfpathlineto{\pgfqpoint{1.295802in}{2.025288in}}%
\pgfpathlineto{\pgfqpoint{1.296066in}{2.404987in}}%
\pgfpathlineto{\pgfqpoint{1.296926in}{2.202080in}}%
\pgfpathlineto{\pgfqpoint{1.297080in}{1.940825in}}%
\pgfpathlineto{\pgfqpoint{1.297301in}{2.228304in}}%
\pgfpathlineto{\pgfqpoint{1.298050in}{2.091831in}}%
\pgfpathlineto{\pgfqpoint{1.299020in}{1.921266in}}%
\pgfpathlineto{\pgfqpoint{1.299174in}{2.181319in}}%
\pgfpathlineto{\pgfqpoint{1.299328in}{1.803915in}}%
\pgfpathlineto{\pgfqpoint{1.300320in}{1.930226in}}%
\pgfpathlineto{\pgfqpoint{1.300386in}{2.108439in}}%
\pgfpathlineto{\pgfqpoint{1.300497in}{1.864885in}}%
\pgfpathlineto{\pgfqpoint{1.301444in}{2.002888in}}%
\pgfpathlineto{\pgfqpoint{1.301466in}{2.004199in}}%
\pgfpathlineto{\pgfqpoint{1.301511in}{1.973496in}}%
\pgfpathlineto{\pgfqpoint{1.301973in}{2.205467in}}%
\pgfpathlineto{\pgfqpoint{1.302062in}{1.855488in}}%
\pgfpathlineto{\pgfqpoint{1.302613in}{2.035559in}}%
\pgfpathlineto{\pgfqpoint{1.303516in}{1.864230in}}%
\pgfpathlineto{\pgfqpoint{1.303649in}{2.165039in}}%
\pgfpathlineto{\pgfqpoint{1.303715in}{1.921376in}}%
\pgfpathlineto{\pgfqpoint{1.303957in}{2.133679in}}%
\pgfpathlineto{\pgfqpoint{1.304773in}{1.785449in}}%
\pgfpathlineto{\pgfqpoint{1.304817in}{1.884990in}}%
\pgfpathlineto{\pgfqpoint{1.305522in}{1.708963in}}%
\pgfpathlineto{\pgfqpoint{1.305302in}{2.069868in}}%
\pgfpathlineto{\pgfqpoint{1.305919in}{1.898211in}}%
\pgfpathlineto{\pgfqpoint{1.306206in}{1.678041in}}%
\pgfpathlineto{\pgfqpoint{1.306492in}{1.982892in}}%
\pgfpathlineto{\pgfqpoint{1.306977in}{1.873954in}}%
\pgfpathlineto{\pgfqpoint{1.307374in}{2.007805in}}%
\pgfpathlineto{\pgfqpoint{1.307572in}{1.706996in}}%
\pgfpathlineto{\pgfqpoint{1.308057in}{1.805882in}}%
\pgfpathlineto{\pgfqpoint{1.308961in}{1.681318in}}%
\pgfpathlineto{\pgfqpoint{1.308277in}{1.995567in}}%
\pgfpathlineto{\pgfqpoint{1.309159in}{1.779658in}}%
\pgfpathlineto{\pgfqpoint{1.309291in}{1.673451in}}%
\pgfpathlineto{\pgfqpoint{1.310283in}{1.992289in}}%
\pgfpathlineto{\pgfqpoint{1.311187in}{1.745239in}}%
\pgfpathlineto{\pgfqpoint{1.310878in}{2.059488in}}%
\pgfpathlineto{\pgfqpoint{1.311385in}{2.000921in}}%
\pgfpathlineto{\pgfqpoint{1.311407in}{2.037416in}}%
\pgfpathlineto{\pgfqpoint{1.312091in}{1.681756in}}%
\pgfpathlineto{\pgfqpoint{1.312311in}{1.704374in}}%
\pgfpathlineto{\pgfqpoint{1.312377in}{1.590409in}}%
\pgfpathlineto{\pgfqpoint{1.312664in}{1.858548in}}%
\pgfpathlineto{\pgfqpoint{1.313237in}{1.752560in}}%
\pgfpathlineto{\pgfqpoint{1.313259in}{1.869693in}}%
\pgfpathlineto{\pgfqpoint{1.314052in}{1.528018in}}%
\pgfpathlineto{\pgfqpoint{1.314317in}{1.595108in}}%
\pgfpathlineto{\pgfqpoint{1.314956in}{1.798670in}}%
\pgfpathlineto{\pgfqpoint{1.314802in}{1.543097in}}%
\pgfpathlineto{\pgfqpoint{1.315441in}{1.701751in}}%
\pgfpathlineto{\pgfqpoint{1.315529in}{1.624391in}}%
\pgfpathlineto{\pgfqpoint{1.316014in}{1.931975in}}%
\pgfpathlineto{\pgfqpoint{1.316521in}{1.746222in}}%
\pgfpathlineto{\pgfqpoint{1.317557in}{1.930226in}}%
\pgfpathlineto{\pgfqpoint{1.316720in}{1.642311in}}%
\pgfpathlineto{\pgfqpoint{1.317601in}{1.884553in}}%
\pgfpathlineto{\pgfqpoint{1.317645in}{1.648102in}}%
\pgfpathlineto{\pgfqpoint{1.318218in}{1.957980in}}%
\pgfpathlineto{\pgfqpoint{1.318703in}{1.816917in}}%
\pgfpathlineto{\pgfqpoint{1.319409in}{1.976336in}}%
\pgfpathlineto{\pgfqpoint{1.319673in}{1.677822in}}%
\pgfpathlineto{\pgfqpoint{1.319783in}{1.776926in}}%
\pgfpathlineto{\pgfqpoint{1.320445in}{1.627232in}}%
\pgfpathlineto{\pgfqpoint{1.320048in}{1.955685in}}%
\pgfpathlineto{\pgfqpoint{1.320908in}{1.733111in}}%
\pgfpathlineto{\pgfqpoint{1.321194in}{1.938640in}}%
\pgfpathlineto{\pgfqpoint{1.320952in}{1.629964in}}%
\pgfpathlineto{\pgfqpoint{1.322010in}{1.780204in}}%
\pgfpathlineto{\pgfqpoint{1.322230in}{1.921485in}}%
\pgfpathlineto{\pgfqpoint{1.322935in}{1.696069in}}%
\pgfpathlineto{\pgfqpoint{1.323134in}{1.813202in}}%
\pgfpathlineto{\pgfqpoint{1.323178in}{1.856253in}}%
\pgfpathlineto{\pgfqpoint{1.323310in}{1.641873in}}%
\pgfpathlineto{\pgfqpoint{1.323354in}{1.651598in}}%
\pgfpathlineto{\pgfqpoint{1.323398in}{1.578609in}}%
\pgfpathlineto{\pgfqpoint{1.324214in}{1.886301in}}%
\pgfpathlineto{\pgfqpoint{1.324302in}{1.799107in}}%
\pgfpathlineto{\pgfqpoint{1.325029in}{1.891765in}}%
\pgfpathlineto{\pgfqpoint{1.324611in}{1.502887in}}%
\pgfpathlineto{\pgfqpoint{1.325382in}{1.761957in}}%
\pgfpathlineto{\pgfqpoint{1.326220in}{1.539601in}}%
\pgfpathlineto{\pgfqpoint{1.325492in}{1.837459in}}%
\pgfpathlineto{\pgfqpoint{1.326484in}{1.720873in}}%
\pgfpathlineto{\pgfqpoint{1.326683in}{1.668097in}}%
\pgfpathlineto{\pgfqpoint{1.326859in}{1.878981in}}%
\pgfpathlineto{\pgfqpoint{1.326881in}{1.968797in}}%
\pgfpathlineto{\pgfqpoint{1.327410in}{1.618600in}}%
\pgfpathlineto{\pgfqpoint{1.327939in}{1.754964in}}%
\pgfpathlineto{\pgfqpoint{1.328622in}{1.594234in}}%
\pgfpathlineto{\pgfqpoint{1.328490in}{1.929024in}}%
\pgfpathlineto{\pgfqpoint{1.328975in}{1.817682in}}%
\pgfpathlineto{\pgfqpoint{1.329372in}{1.625921in}}%
\pgfpathlineto{\pgfqpoint{1.329217in}{1.825877in}}%
\pgfpathlineto{\pgfqpoint{1.329438in}{1.775724in}}%
\pgfpathlineto{\pgfqpoint{1.329879in}{1.519386in}}%
\pgfpathlineto{\pgfqpoint{1.330231in}{1.878216in}}%
\pgfpathlineto{\pgfqpoint{1.330562in}{1.624609in}}%
\pgfpathlineto{\pgfqpoint{1.331400in}{1.925528in}}%
\pgfpathlineto{\pgfqpoint{1.330628in}{1.583088in}}%
\pgfpathlineto{\pgfqpoint{1.331752in}{1.739994in}}%
\pgfpathlineto{\pgfqpoint{1.332083in}{1.547358in}}%
\pgfpathlineto{\pgfqpoint{1.332347in}{1.945523in}}%
\pgfpathlineto{\pgfqpoint{1.332854in}{1.740868in}}%
\pgfpathlineto{\pgfqpoint{1.333956in}{1.553587in}}%
\pgfpathlineto{\pgfqpoint{1.333317in}{1.827953in}}%
\pgfpathlineto{\pgfqpoint{1.333978in}{1.695851in}}%
\pgfpathlineto{\pgfqpoint{1.334133in}{1.807302in}}%
\pgfpathlineto{\pgfqpoint{1.334331in}{1.526052in}}%
\pgfpathlineto{\pgfqpoint{1.335058in}{1.663836in}}%
\pgfpathlineto{\pgfqpoint{1.335720in}{1.401379in}}%
\pgfpathlineto{\pgfqpoint{1.335125in}{1.778237in}}%
\pgfpathlineto{\pgfqpoint{1.336205in}{1.449784in}}%
\pgfpathlineto{\pgfqpoint{1.336734in}{1.821944in}}%
\pgfpathlineto{\pgfqpoint{1.337329in}{1.595217in}}%
\pgfpathlineto{\pgfqpoint{1.338343in}{1.792223in}}%
\pgfpathlineto{\pgfqpoint{1.337990in}{1.542769in}}%
\pgfpathlineto{\pgfqpoint{1.338387in}{1.752014in}}%
\pgfpathlineto{\pgfqpoint{1.339070in}{1.426183in}}%
\pgfpathlineto{\pgfqpoint{1.338519in}{1.813858in}}%
\pgfpathlineto{\pgfqpoint{1.339511in}{1.543097in}}%
\pgfpathlineto{\pgfqpoint{1.339996in}{1.734859in}}%
\pgfpathlineto{\pgfqpoint{1.340084in}{1.422030in}}%
\pgfpathlineto{\pgfqpoint{1.340635in}{1.601773in}}%
\pgfpathlineto{\pgfqpoint{1.341164in}{1.380072in}}%
\pgfpathlineto{\pgfqpoint{1.341517in}{1.671266in}}%
\pgfpathlineto{\pgfqpoint{1.341737in}{1.547796in}}%
\pgfpathlineto{\pgfqpoint{1.342398in}{1.719671in}}%
\pgfpathlineto{\pgfqpoint{1.341936in}{1.382367in}}%
\pgfpathlineto{\pgfqpoint{1.342817in}{1.583307in}}%
\pgfpathlineto{\pgfqpoint{1.343148in}{1.367288in}}%
\pgfpathlineto{\pgfqpoint{1.343787in}{1.740431in}}%
\pgfpathlineto{\pgfqpoint{1.343897in}{1.534465in}}%
\pgfpathlineto{\pgfqpoint{1.344008in}{1.805991in}}%
\pgfpathlineto{\pgfqpoint{1.344801in}{1.482782in}}%
\pgfpathlineto{\pgfqpoint{1.345021in}{1.609749in}}%
\pgfpathlineto{\pgfqpoint{1.345220in}{1.755292in}}%
\pgfpathlineto{\pgfqpoint{1.345837in}{1.419080in}}%
\pgfpathlineto{\pgfqpoint{1.346102in}{1.588224in}}%
\pgfpathlineto{\pgfqpoint{1.346124in}{1.418862in}}%
\pgfpathlineto{\pgfqpoint{1.346278in}{1.693119in}}%
\pgfpathlineto{\pgfqpoint{1.347204in}{1.505291in}}%
\pgfpathlineto{\pgfqpoint{1.348328in}{1.702079in}}%
\pgfpathlineto{\pgfqpoint{1.347402in}{1.367288in}}%
\pgfpathlineto{\pgfqpoint{1.348350in}{1.641218in}}%
\pgfpathlineto{\pgfqpoint{1.349143in}{1.352210in}}%
\pgfpathlineto{\pgfqpoint{1.348945in}{1.658700in}}%
\pgfpathlineto{\pgfqpoint{1.349474in}{1.551074in}}%
\pgfpathlineto{\pgfqpoint{1.349496in}{1.555116in}}%
\pgfpathlineto{\pgfqpoint{1.349606in}{1.416676in}}%
\pgfpathlineto{\pgfqpoint{1.349628in}{1.504526in}}%
\pgfpathlineto{\pgfqpoint{1.349650in}{1.293534in}}%
\pgfpathlineto{\pgfqpoint{1.350664in}{1.683613in}}%
\pgfpathlineto{\pgfqpoint{1.350730in}{1.547249in}}%
\pgfpathlineto{\pgfqpoint{1.350995in}{1.331121in}}%
\pgfpathlineto{\pgfqpoint{1.350818in}{1.616961in}}%
\pgfpathlineto{\pgfqpoint{1.351854in}{1.459181in}}%
\pgfpathlineto{\pgfqpoint{1.352472in}{1.598604in}}%
\pgfpathlineto{\pgfqpoint{1.352714in}{1.294080in}}%
\pgfpathlineto{\pgfqpoint{1.352957in}{1.441917in}}%
\pgfpathlineto{\pgfqpoint{1.353772in}{1.341174in}}%
\pgfpathlineto{\pgfqpoint{1.353111in}{1.567136in}}%
\pgfpathlineto{\pgfqpoint{1.354015in}{1.537743in}}%
\pgfpathlineto{\pgfqpoint{1.354125in}{1.648539in}}%
\pgfpathlineto{\pgfqpoint{1.354786in}{1.333853in}}%
\pgfpathlineto{\pgfqpoint{1.354852in}{1.370566in}}%
\pgfpathlineto{\pgfqpoint{1.355029in}{1.320522in}}%
\pgfpathlineto{\pgfqpoint{1.355558in}{1.610296in}}%
\pgfpathlineto{\pgfqpoint{1.355866in}{1.377013in}}%
\pgfpathlineto{\pgfqpoint{1.356483in}{1.654658in}}%
\pgfpathlineto{\pgfqpoint{1.356660in}{1.250046in}}%
\pgfpathlineto{\pgfqpoint{1.356968in}{1.452953in}}%
\pgfpathlineto{\pgfqpoint{1.357145in}{1.319211in}}%
\pgfpathlineto{\pgfqpoint{1.357321in}{1.559596in}}%
\pgfpathlineto{\pgfqpoint{1.358048in}{1.470435in}}%
\pgfpathlineto{\pgfqpoint{1.358291in}{1.562000in}}%
\pgfpathlineto{\pgfqpoint{1.358732in}{1.252122in}}%
\pgfpathlineto{\pgfqpoint{1.359106in}{1.361606in}}%
\pgfpathlineto{\pgfqpoint{1.360120in}{1.242616in}}%
\pgfpathlineto{\pgfqpoint{1.359657in}{1.564295in}}%
\pgfpathlineto{\pgfqpoint{1.360186in}{1.377122in}}%
\pgfpathlineto{\pgfqpoint{1.360936in}{1.588770in}}%
\pgfpathlineto{\pgfqpoint{1.360473in}{1.297467in}}%
\pgfpathlineto{\pgfqpoint{1.361310in}{1.516108in}}%
\pgfpathlineto{\pgfqpoint{1.362148in}{1.277690in}}%
\pgfpathlineto{\pgfqpoint{1.361994in}{1.707542in}}%
\pgfpathlineto{\pgfqpoint{1.362435in}{1.363355in}}%
\pgfpathlineto{\pgfqpoint{1.363162in}{1.661651in}}%
\pgfpathlineto{\pgfqpoint{1.362655in}{1.222620in}}%
\pgfpathlineto{\pgfqpoint{1.363581in}{1.532280in}}%
\pgfpathlineto{\pgfqpoint{1.364374in}{1.243818in}}%
\pgfpathlineto{\pgfqpoint{1.364617in}{1.601336in}}%
\pgfpathlineto{\pgfqpoint{1.364771in}{1.361716in}}%
\pgfpathlineto{\pgfqpoint{1.365278in}{1.624609in}}%
\pgfpathlineto{\pgfqpoint{1.365851in}{1.268293in}}%
\pgfpathlineto{\pgfqpoint{1.365895in}{1.478739in}}%
\pgfpathlineto{\pgfqpoint{1.365939in}{1.622206in}}%
\pgfpathlineto{\pgfqpoint{1.366116in}{1.235295in}}%
\pgfpathlineto{\pgfqpoint{1.366821in}{1.323363in}}%
\pgfpathlineto{\pgfqpoint{1.367196in}{1.195959in}}%
\pgfpathlineto{\pgfqpoint{1.367637in}{1.447052in}}%
\pgfpathlineto{\pgfqpoint{1.367857in}{1.350133in}}%
\pgfpathlineto{\pgfqpoint{1.368805in}{1.495239in}}%
\pgfpathlineto{\pgfqpoint{1.368452in}{1.112371in}}%
\pgfpathlineto{\pgfqpoint{1.368959in}{1.316698in}}%
\pgfpathlineto{\pgfqpoint{1.369113in}{1.220544in}}%
\pgfpathlineto{\pgfqpoint{1.369312in}{1.474806in}}%
\pgfpathlineto{\pgfqpoint{1.370039in}{1.281842in}}%
\pgfpathlineto{\pgfqpoint{1.370436in}{1.436672in}}%
\pgfpathlineto{\pgfqpoint{1.370700in}{1.157061in}}%
\pgfpathlineto{\pgfqpoint{1.371141in}{1.322380in}}%
\pgfpathlineto{\pgfqpoint{1.371780in}{1.248516in}}%
\pgfpathlineto{\pgfqpoint{1.371957in}{1.562109in}}%
\pgfpathlineto{\pgfqpoint{1.372221in}{1.350243in}}%
\pgfpathlineto{\pgfqpoint{1.373147in}{1.479723in}}%
\pgfpathlineto{\pgfqpoint{1.372860in}{1.226444in}}%
\pgfpathlineto{\pgfqpoint{1.373323in}{1.333525in}}%
\pgfpathlineto{\pgfqpoint{1.374029in}{1.143839in}}%
\pgfpathlineto{\pgfqpoint{1.373632in}{1.419517in}}%
\pgfpathlineto{\pgfqpoint{1.374381in}{1.213333in}}%
\pgfpathlineto{\pgfqpoint{1.374448in}{1.467594in}}%
\pgfpathlineto{\pgfqpoint{1.374756in}{1.178914in}}%
\pgfpathlineto{\pgfqpoint{1.375483in}{1.269386in}}%
\pgfpathlineto{\pgfqpoint{1.376057in}{1.108984in}}%
\pgfpathlineto{\pgfqpoint{1.375924in}{1.472402in}}%
\pgfpathlineto{\pgfqpoint{1.376564in}{1.238901in}}%
\pgfpathlineto{\pgfqpoint{1.377357in}{1.550636in}}%
\pgfpathlineto{\pgfqpoint{1.377710in}{1.391764in}}%
\pgfpathlineto{\pgfqpoint{1.378239in}{1.661541in}}%
\pgfpathlineto{\pgfqpoint{1.378437in}{1.305334in}}%
\pgfpathlineto{\pgfqpoint{1.378790in}{1.418534in}}%
\pgfpathlineto{\pgfqpoint{1.379054in}{1.297467in}}%
\pgfpathlineto{\pgfqpoint{1.379253in}{1.601554in}}%
\pgfpathlineto{\pgfqpoint{1.379870in}{1.490759in}}%
\pgfpathlineto{\pgfqpoint{1.380928in}{1.605488in}}%
\pgfpathlineto{\pgfqpoint{1.380178in}{1.273210in}}%
\pgfpathlineto{\pgfqpoint{1.380972in}{1.550855in}}%
\pgfpathlineto{\pgfqpoint{1.381854in}{1.321287in}}%
\pgfpathlineto{\pgfqpoint{1.381501in}{1.691699in}}%
\pgfpathlineto{\pgfqpoint{1.382096in}{1.422577in}}%
\pgfpathlineto{\pgfqpoint{1.382383in}{1.616852in}}%
\pgfpathlineto{\pgfqpoint{1.382581in}{1.316043in}}%
\pgfpathlineto{\pgfqpoint{1.383176in}{1.568993in}}%
\pgfpathlineto{\pgfqpoint{1.383705in}{1.242179in}}%
\pgfpathlineto{\pgfqpoint{1.384300in}{1.324237in}}%
\pgfpathlineto{\pgfqpoint{1.385248in}{1.432192in}}%
\pgfpathlineto{\pgfqpoint{1.385292in}{1.189185in}}%
\pgfpathlineto{\pgfqpoint{1.385336in}{1.354832in}}%
\pgfpathlineto{\pgfqpoint{1.386130in}{1.579483in}}%
\pgfpathlineto{\pgfqpoint{1.386416in}{1.128542in}}%
\pgfpathlineto{\pgfqpoint{1.386945in}{1.496987in}}%
\pgfpathlineto{\pgfqpoint{1.387540in}{1.437765in}}%
\pgfpathlineto{\pgfqpoint{1.388510in}{1.145697in}}%
\pgfpathlineto{\pgfqpoint{1.387717in}{1.438857in}}%
\pgfpathlineto{\pgfqpoint{1.388665in}{1.343468in}}%
\pgfpathlineto{\pgfqpoint{1.388819in}{1.421047in}}%
\pgfpathlineto{\pgfqpoint{1.389524in}{1.214534in}}%
\pgfpathlineto{\pgfqpoint{1.389789in}{1.390234in}}%
\pgfpathlineto{\pgfqpoint{1.390009in}{1.137174in}}%
\pgfpathlineto{\pgfqpoint{1.390648in}{1.397555in}}%
\pgfpathlineto{\pgfqpoint{1.390913in}{1.185579in}}%
\pgfpathlineto{\pgfqpoint{1.390957in}{1.391217in}}%
\pgfpathlineto{\pgfqpoint{1.391023in}{1.086256in}}%
\pgfpathlineto{\pgfqpoint{1.391993in}{1.148429in}}%
\pgfpathlineto{\pgfqpoint{1.392147in}{1.015015in}}%
\pgfpathlineto{\pgfqpoint{1.392390in}{1.390562in}}%
\pgfpathlineto{\pgfqpoint{1.393007in}{1.350680in}}%
\pgfpathlineto{\pgfqpoint{1.393426in}{1.467157in}}%
\pgfpathlineto{\pgfqpoint{1.393712in}{1.212568in}}%
\pgfpathlineto{\pgfqpoint{1.394285in}{1.094670in}}%
\pgfpathlineto{\pgfqpoint{1.394638in}{1.386082in}}%
\pgfpathlineto{\pgfqpoint{1.394770in}{1.290146in}}%
\pgfpathlineto{\pgfqpoint{1.395586in}{1.464098in}}%
\pgfpathlineto{\pgfqpoint{1.394947in}{1.088988in}}%
\pgfpathlineto{\pgfqpoint{1.395850in}{1.277035in}}%
\pgfpathlineto{\pgfqpoint{1.396578in}{1.087021in}}%
\pgfpathlineto{\pgfqpoint{1.396644in}{1.344998in}}%
\pgfpathlineto{\pgfqpoint{1.396952in}{1.179788in}}%
\pgfpathlineto{\pgfqpoint{1.397019in}{1.364447in}}%
\pgfpathlineto{\pgfqpoint{1.397592in}{1.018839in}}%
\pgfpathlineto{\pgfqpoint{1.398054in}{1.184049in}}%
\pgfpathlineto{\pgfqpoint{1.398826in}{0.958634in}}%
\pgfpathlineto{\pgfqpoint{1.398980in}{1.309377in}}%
\pgfpathlineto{\pgfqpoint{1.399179in}{1.093796in}}%
\pgfpathlineto{\pgfqpoint{1.399972in}{1.349696in}}%
\pgfpathlineto{\pgfqpoint{1.399840in}{1.052930in}}%
\pgfpathlineto{\pgfqpoint{1.400369in}{1.222511in}}%
\pgfpathlineto{\pgfqpoint{1.400986in}{1.050526in}}%
\pgfpathlineto{\pgfqpoint{1.400810in}{1.321943in}}%
\pgfpathlineto{\pgfqpoint{1.401537in}{1.071505in}}%
\pgfpathlineto{\pgfqpoint{1.401669in}{1.317244in}}%
\pgfpathlineto{\pgfqpoint{1.401824in}{0.932628in}}%
\pgfpathlineto{\pgfqpoint{1.402639in}{1.097183in}}%
\pgfpathlineto{\pgfqpoint{1.402661in}{1.021462in}}%
\pgfpathlineto{\pgfqpoint{1.403565in}{1.361388in}}%
\pgfpathlineto{\pgfqpoint{1.403719in}{1.049871in}}%
\pgfpathlineto{\pgfqpoint{1.404336in}{1.513049in}}%
\pgfpathlineto{\pgfqpoint{1.404865in}{1.365649in}}%
\pgfpathlineto{\pgfqpoint{1.405262in}{1.163944in}}%
\pgfpathlineto{\pgfqpoint{1.405946in}{1.440278in}}%
\pgfpathlineto{\pgfqpoint{1.405990in}{1.313529in}}%
\pgfpathlineto{\pgfqpoint{1.406276in}{1.561672in}}%
\pgfpathlineto{\pgfqpoint{1.406783in}{1.202843in}}%
\pgfpathlineto{\pgfqpoint{1.407048in}{1.237371in}}%
\pgfpathlineto{\pgfqpoint{1.407202in}{1.146353in}}%
\pgfpathlineto{\pgfqpoint{1.407687in}{1.440278in}}%
\pgfpathlineto{\pgfqpoint{1.408040in}{1.254526in}}%
\pgfpathlineto{\pgfqpoint{1.408238in}{1.401379in}}%
\pgfpathlineto{\pgfqpoint{1.408723in}{1.085273in}}%
\pgfpathlineto{\pgfqpoint{1.409164in}{1.380619in}}%
\pgfpathlineto{\pgfqpoint{1.410067in}{1.113136in}}%
\pgfpathlineto{\pgfqpoint{1.409362in}{1.486060in}}%
\pgfpathlineto{\pgfqpoint{1.410420in}{1.117069in}}%
\pgfpathlineto{\pgfqpoint{1.411192in}{1.348276in}}%
\pgfpathlineto{\pgfqpoint{1.411478in}{1.111825in}}%
\pgfpathlineto{\pgfqpoint{1.411544in}{1.237808in}}%
\pgfpathlineto{\pgfqpoint{1.412558in}{1.074237in}}%
\pgfpathlineto{\pgfqpoint{1.411853in}{1.399740in}}%
\pgfpathlineto{\pgfqpoint{1.412668in}{1.142200in}}%
\pgfpathlineto{\pgfqpoint{1.412779in}{1.323800in}}%
\pgfpathlineto{\pgfqpoint{1.412911in}{1.040146in}}%
\pgfpathlineto{\pgfqpoint{1.413792in}{1.201860in}}%
\pgfpathlineto{\pgfqpoint{1.414432in}{0.977865in}}%
\pgfpathlineto{\pgfqpoint{1.414873in}{1.183721in}}%
\pgfpathlineto{\pgfqpoint{1.415798in}{1.003651in}}%
\pgfpathlineto{\pgfqpoint{1.415975in}{1.290693in}}%
\pgfpathlineto{\pgfqpoint{1.416129in}{1.124499in}}%
\pgfpathlineto{\pgfqpoint{1.416261in}{1.324237in}}%
\pgfpathlineto{\pgfqpoint{1.417077in}{1.271243in}}%
\pgfpathlineto{\pgfqpoint{1.418267in}{1.005181in}}%
\pgfpathlineto{\pgfqpoint{1.417760in}{1.370129in}}%
\pgfpathlineto{\pgfqpoint{1.418333in}{1.019276in}}%
\pgfpathlineto{\pgfqpoint{1.418620in}{0.981033in}}%
\pgfpathlineto{\pgfqpoint{1.419501in}{1.318119in}}%
\pgfpathlineto{\pgfqpoint{1.419876in}{1.326423in}}%
\pgfpathlineto{\pgfqpoint{1.420625in}{1.009770in}}%
\pgfpathlineto{\pgfqpoint{1.421683in}{1.366305in}}%
\pgfpathlineto{\pgfqpoint{1.421750in}{1.263813in}}%
\pgfpathlineto{\pgfqpoint{1.422279in}{0.957323in}}%
\pgfpathlineto{\pgfqpoint{1.422345in}{1.323254in}}%
\pgfpathlineto{\pgfqpoint{1.422874in}{1.135098in}}%
\pgfpathlineto{\pgfqpoint{1.423844in}{1.290474in}}%
\pgfpathlineto{\pgfqpoint{1.423006in}{1.022773in}}%
\pgfpathlineto{\pgfqpoint{1.423998in}{1.280531in}}%
\pgfpathlineto{\pgfqpoint{1.424769in}{1.002886in}}%
\pgfpathlineto{\pgfqpoint{1.424196in}{1.335383in}}%
\pgfpathlineto{\pgfqpoint{1.425144in}{1.173669in}}%
\pgfpathlineto{\pgfqpoint{1.425849in}{1.238682in}}%
\pgfpathlineto{\pgfqpoint{1.426026in}{0.932410in}}%
\pgfpathlineto{\pgfqpoint{1.426202in}{1.126903in}}%
\pgfpathlineto{\pgfqpoint{1.426907in}{0.896789in}}%
\pgfpathlineto{\pgfqpoint{1.426775in}{1.263486in}}%
\pgfpathlineto{\pgfqpoint{1.427326in}{1.084617in}}%
\pgfpathlineto{\pgfqpoint{1.428142in}{1.274521in}}%
\pgfpathlineto{\pgfqpoint{1.427899in}{1.035011in}}%
\pgfpathlineto{\pgfqpoint{1.428428in}{1.146462in}}%
\pgfpathlineto{\pgfqpoint{1.428979in}{0.971636in}}%
\pgfpathlineto{\pgfqpoint{1.429156in}{1.291676in}}%
\pgfpathlineto{\pgfqpoint{1.429530in}{1.124827in}}%
\pgfpathlineto{\pgfqpoint{1.429949in}{1.248516in}}%
\pgfpathlineto{\pgfqpoint{1.430236in}{1.026706in}}%
\pgfpathlineto{\pgfqpoint{1.430655in}{1.157388in}}%
\pgfpathlineto{\pgfqpoint{1.430875in}{1.346965in}}%
\pgfpathlineto{\pgfqpoint{1.430765in}{1.023428in}}%
\pgfpathlineto{\pgfqpoint{1.431713in}{1.190277in}}%
\pgfpathlineto{\pgfqpoint{1.432771in}{1.057847in}}%
\pgfpathlineto{\pgfqpoint{1.432286in}{1.425090in}}%
\pgfpathlineto{\pgfqpoint{1.432837in}{1.077406in}}%
\pgfpathlineto{\pgfqpoint{1.432881in}{1.358984in}}%
\pgfpathlineto{\pgfqpoint{1.433652in}{1.009989in}}%
\pgfpathlineto{\pgfqpoint{1.433961in}{1.181318in}}%
\pgfpathlineto{\pgfqpoint{1.434446in}{1.371440in}}%
\pgfpathlineto{\pgfqpoint{1.434159in}{1.019495in}}%
\pgfpathlineto{\pgfqpoint{1.435063in}{1.285339in}}%
\pgfpathlineto{\pgfqpoint{1.435790in}{0.987261in}}%
\pgfpathlineto{\pgfqpoint{1.436209in}{1.059377in}}%
\pgfpathlineto{\pgfqpoint{1.436981in}{1.265780in}}%
\pgfpathlineto{\pgfqpoint{1.436363in}{0.882038in}}%
\pgfpathlineto{\pgfqpoint{1.437355in}{1.250046in}}%
\pgfpathlineto{\pgfqpoint{1.437620in}{1.255837in}}%
\pgfpathlineto{\pgfqpoint{1.438502in}{0.986059in}}%
\pgfpathlineto{\pgfqpoint{1.439538in}{1.254089in}}%
\pgfpathlineto{\pgfqpoint{1.439626in}{1.087349in}}%
\pgfpathlineto{\pgfqpoint{1.440155in}{0.918752in}}%
\pgfpathlineto{\pgfqpoint{1.440221in}{1.205356in}}%
\pgfpathlineto{\pgfqpoint{1.440728in}{1.061781in}}%
\pgfpathlineto{\pgfqpoint{1.441609in}{1.243271in}}%
\pgfpathlineto{\pgfqpoint{1.441786in}{0.980268in}}%
\pgfpathlineto{\pgfqpoint{1.441808in}{1.100352in}}%
\pgfpathlineto{\pgfqpoint{1.442337in}{0.993817in}}%
\pgfpathlineto{\pgfqpoint{1.442138in}{1.259443in}}%
\pgfpathlineto{\pgfqpoint{1.442910in}{1.019713in}}%
\pgfpathlineto{\pgfqpoint{1.442954in}{1.290584in}}%
\pgfpathlineto{\pgfqpoint{1.443219in}{0.934158in}}%
\pgfpathlineto{\pgfqpoint{1.444012in}{1.091283in}}%
\pgfpathlineto{\pgfqpoint{1.444563in}{0.960928in}}%
\pgfpathlineto{\pgfqpoint{1.444277in}{1.271899in}}%
\pgfpathlineto{\pgfqpoint{1.445092in}{1.041457in}}%
\pgfpathlineto{\pgfqpoint{1.445621in}{1.249390in}}%
\pgfpathlineto{\pgfqpoint{1.445996in}{0.920937in}}%
\pgfpathlineto{\pgfqpoint{1.446172in}{1.025614in}}%
\pgfpathlineto{\pgfqpoint{1.446745in}{0.965299in}}%
\pgfpathlineto{\pgfqpoint{1.446371in}{1.279766in}}%
\pgfpathlineto{\pgfqpoint{1.447274in}{1.043533in}}%
\pgfpathlineto{\pgfqpoint{1.448266in}{1.233437in}}%
\pgfpathlineto{\pgfqpoint{1.447980in}{0.940277in}}%
\pgfpathlineto{\pgfqpoint{1.448376in}{1.027580in}}%
\pgfpathlineto{\pgfqpoint{1.449368in}{1.324675in}}%
\pgfpathlineto{\pgfqpoint{1.449258in}{0.939621in}}%
\pgfpathlineto{\pgfqpoint{1.449545in}{1.198691in}}%
\pgfpathlineto{\pgfqpoint{1.450492in}{0.969560in}}%
\pgfpathlineto{\pgfqpoint{1.449721in}{1.256165in}}%
\pgfpathlineto{\pgfqpoint{1.450691in}{1.097183in}}%
\pgfpathlineto{\pgfqpoint{1.451705in}{1.341829in}}%
\pgfpathlineto{\pgfqpoint{1.451286in}{0.976990in}}%
\pgfpathlineto{\pgfqpoint{1.451815in}{1.185688in}}%
\pgfpathlineto{\pgfqpoint{1.452212in}{0.962349in}}%
\pgfpathlineto{\pgfqpoint{1.452719in}{1.329264in}}%
\pgfpathlineto{\pgfqpoint{1.452917in}{1.262611in}}%
\pgfpathlineto{\pgfqpoint{1.453402in}{1.065933in}}%
\pgfpathlineto{\pgfqpoint{1.453292in}{1.393949in}}%
\pgfpathlineto{\pgfqpoint{1.454019in}{1.221527in}}%
\pgfpathlineto{\pgfqpoint{1.454107in}{1.281514in}}%
\pgfpathlineto{\pgfqpoint{1.454724in}{0.993927in}}%
\pgfpathlineto{\pgfqpoint{1.455011in}{1.013485in}}%
\pgfpathlineto{\pgfqpoint{1.455430in}{0.931645in}}%
\pgfpathlineto{\pgfqpoint{1.455386in}{1.194976in}}%
\pgfpathlineto{\pgfqpoint{1.456025in}{1.025942in}}%
\pgfpathlineto{\pgfqpoint{1.456069in}{1.201641in}}%
\pgfpathlineto{\pgfqpoint{1.456796in}{0.900176in}}%
\pgfpathlineto{\pgfqpoint{1.457127in}{1.025832in}}%
\pgfpathlineto{\pgfqpoint{1.458163in}{0.793970in}}%
\pgfpathlineto{\pgfqpoint{1.457436in}{1.164491in}}%
\pgfpathlineto{\pgfqpoint{1.458251in}{0.957869in}}%
\pgfpathlineto{\pgfqpoint{1.458339in}{1.140889in}}%
\pgfpathlineto{\pgfqpoint{1.458472in}{0.867834in}}%
\pgfpathlineto{\pgfqpoint{1.459353in}{1.003214in}}%
\pgfpathlineto{\pgfqpoint{1.460367in}{0.852536in}}%
\pgfpathlineto{\pgfqpoint{1.459860in}{1.132694in}}%
\pgfpathlineto{\pgfqpoint{1.460455in}{0.949128in}}%
\pgfpathlineto{\pgfqpoint{1.460852in}{1.130837in}}%
\pgfpathlineto{\pgfqpoint{1.461006in}{0.817571in}}%
\pgfpathlineto{\pgfqpoint{1.461535in}{0.923013in}}%
\pgfpathlineto{\pgfqpoint{1.462373in}{0.904438in}}%
\pgfpathlineto{\pgfqpoint{1.461646in}{1.111825in}}%
\pgfpathlineto{\pgfqpoint{1.462395in}{1.028345in}}%
\pgfpathlineto{\pgfqpoint{1.463321in}{0.825220in}}%
\pgfpathlineto{\pgfqpoint{1.463497in}{1.201204in}}%
\pgfpathlineto{\pgfqpoint{1.463718in}{0.861278in}}%
\pgfpathlineto{\pgfqpoint{1.464621in}{0.934486in}}%
\pgfpathlineto{\pgfqpoint{1.464864in}{1.242725in}}%
\pgfpathlineto{\pgfqpoint{1.465745in}{1.114228in}}%
\pgfpathlineto{\pgfqpoint{1.466803in}{0.794188in}}%
\pgfpathlineto{\pgfqpoint{1.466583in}{1.208088in}}%
\pgfpathlineto{\pgfqpoint{1.466936in}{0.979176in}}%
\pgfpathlineto{\pgfqpoint{1.467377in}{0.835928in}}%
\pgfpathlineto{\pgfqpoint{1.467068in}{1.050636in}}%
\pgfpathlineto{\pgfqpoint{1.467509in}{1.006274in}}%
\pgfpathlineto{\pgfqpoint{1.468567in}{1.138048in}}%
\pgfpathlineto{\pgfqpoint{1.467795in}{0.792003in}}%
\pgfpathlineto{\pgfqpoint{1.468633in}{1.058066in}}%
\pgfpathlineto{\pgfqpoint{1.468699in}{0.880072in}}%
\pgfpathlineto{\pgfqpoint{1.469272in}{1.156405in}}%
\pgfpathlineto{\pgfqpoint{1.469757in}{0.936453in}}%
\pgfpathlineto{\pgfqpoint{1.470022in}{1.164928in}}%
\pgfpathlineto{\pgfqpoint{1.470727in}{0.882585in}}%
\pgfpathlineto{\pgfqpoint{1.470881in}{1.005072in}}%
\pgfpathlineto{\pgfqpoint{1.471653in}{0.843249in}}%
\pgfpathlineto{\pgfqpoint{1.470991in}{1.103411in}}%
\pgfpathlineto{\pgfqpoint{1.472027in}{0.929787in}}%
\pgfpathlineto{\pgfqpoint{1.472490in}{1.111387in}}%
\pgfpathlineto{\pgfqpoint{1.473107in}{0.810578in}}%
\pgfpathlineto{\pgfqpoint{1.473130in}{0.975461in}}%
\pgfpathlineto{\pgfqpoint{1.473570in}{0.794626in}}%
\pgfpathlineto{\pgfqpoint{1.473416in}{1.118162in}}%
\pgfpathlineto{\pgfqpoint{1.474232in}{0.942244in}}%
\pgfpathlineto{\pgfqpoint{1.474298in}{1.181099in}}%
\pgfpathlineto{\pgfqpoint{1.475113in}{0.861824in}}%
\pgfpathlineto{\pgfqpoint{1.475356in}{1.005946in}}%
\pgfpathlineto{\pgfqpoint{1.475730in}{1.198145in}}%
\pgfpathlineto{\pgfqpoint{1.476370in}{0.876356in}}%
\pgfpathlineto{\pgfqpoint{1.476678in}{0.830356in}}%
\pgfpathlineto{\pgfqpoint{1.476965in}{1.107782in}}%
\pgfpathlineto{\pgfqpoint{1.477185in}{0.880509in}}%
\pgfpathlineto{\pgfqpoint{1.477207in}{1.119582in}}%
\pgfpathlineto{\pgfqpoint{1.478177in}{0.824018in}}%
\pgfpathlineto{\pgfqpoint{1.478287in}{0.973385in}}%
\pgfpathlineto{\pgfqpoint{1.478464in}{1.103520in}}%
\pgfpathlineto{\pgfqpoint{1.478486in}{0.859857in}}%
\pgfpathlineto{\pgfqpoint{1.479411in}{1.087458in}}%
\pgfpathlineto{\pgfqpoint{1.479456in}{0.907607in}}%
\pgfpathlineto{\pgfqpoint{1.480249in}{1.192026in}}%
\pgfpathlineto{\pgfqpoint{1.480536in}{0.996112in}}%
\pgfpathlineto{\pgfqpoint{1.480778in}{0.844997in}}%
\pgfpathlineto{\pgfqpoint{1.480624in}{1.125701in}}%
\pgfpathlineto{\pgfqpoint{1.481021in}{1.001138in}}%
\pgfpathlineto{\pgfqpoint{1.481043in}{1.130072in}}%
\pgfpathlineto{\pgfqpoint{1.481814in}{0.844014in}}%
\pgfpathlineto{\pgfqpoint{1.482101in}{0.992834in}}%
\pgfpathlineto{\pgfqpoint{1.482982in}{0.762938in}}%
\pgfpathlineto{\pgfqpoint{1.483048in}{1.081121in}}%
\pgfpathlineto{\pgfqpoint{1.483203in}{0.939621in}}%
\pgfpathlineto{\pgfqpoint{1.483864in}{1.019386in}}%
\pgfpathlineto{\pgfqpoint{1.483732in}{0.733983in}}%
\pgfpathlineto{\pgfqpoint{1.483952in}{0.816151in}}%
\pgfpathlineto{\pgfqpoint{1.483974in}{0.754088in}}%
\pgfpathlineto{\pgfqpoint{1.484018in}{0.984858in}}%
\pgfpathlineto{\pgfqpoint{1.485054in}{0.785775in}}%
\pgfpathlineto{\pgfqpoint{1.485209in}{0.658589in}}%
\pgfpathlineto{\pgfqpoint{1.486178in}{0.984530in}}%
\pgfpathlineto{\pgfqpoint{1.486972in}{0.706885in}}%
\pgfpathlineto{\pgfqpoint{1.487192in}{1.019386in}}%
\pgfpathlineto{\pgfqpoint{1.487303in}{0.732453in}}%
\pgfpathlineto{\pgfqpoint{1.487964in}{1.031951in}}%
\pgfpathlineto{\pgfqpoint{1.488427in}{0.895587in}}%
\pgfpathlineto{\pgfqpoint{1.488735in}{0.777143in}}%
\pgfpathlineto{\pgfqpoint{1.489595in}{1.094451in}}%
\pgfpathlineto{\pgfqpoint{1.489837in}{0.875482in}}%
\pgfpathlineto{\pgfqpoint{1.489926in}{1.121658in}}%
\pgfpathlineto{\pgfqpoint{1.490719in}{1.013485in}}%
\pgfpathlineto{\pgfqpoint{1.491336in}{0.882585in}}%
\pgfpathlineto{\pgfqpoint{1.490763in}{1.162852in}}%
\pgfpathlineto{\pgfqpoint{1.491777in}{1.006492in}}%
\pgfpathlineto{\pgfqpoint{1.492703in}{1.180990in}}%
\pgfpathlineto{\pgfqpoint{1.492130in}{0.961038in}}%
\pgfpathlineto{\pgfqpoint{1.492857in}{1.016435in}}%
\pgfpathlineto{\pgfqpoint{1.493937in}{0.886955in}}%
\pgfpathlineto{\pgfqpoint{1.493122in}{1.250374in}}%
\pgfpathlineto{\pgfqpoint{1.493959in}{0.991523in}}%
\pgfpathlineto{\pgfqpoint{1.494202in}{0.884005in}}%
\pgfpathlineto{\pgfqpoint{1.494378in}{1.124390in}}%
\pgfpathlineto{\pgfqpoint{1.494731in}{1.080574in}}%
\pgfpathlineto{\pgfqpoint{1.494951in}{1.193228in}}%
\pgfpathlineto{\pgfqpoint{1.495149in}{0.903017in}}%
\pgfpathlineto{\pgfqpoint{1.495855in}{1.129526in}}%
\pgfpathlineto{\pgfqpoint{1.496803in}{0.892856in}}%
\pgfpathlineto{\pgfqpoint{1.496935in}{1.193883in}}%
\pgfpathlineto{\pgfqpoint{1.496957in}{1.173232in}}%
\pgfpathlineto{\pgfqpoint{1.497530in}{0.890889in}}%
\pgfpathlineto{\pgfqpoint{1.497045in}{1.229067in}}%
\pgfpathlineto{\pgfqpoint{1.498081in}{0.956011in}}%
\pgfpathlineto{\pgfqpoint{1.498985in}{1.265671in}}%
\pgfpathlineto{\pgfqpoint{1.499183in}{1.060797in}}%
\pgfpathlineto{\pgfqpoint{1.499999in}{0.949674in}}%
\pgfpathlineto{\pgfqpoint{1.499888in}{1.262830in}}%
\pgfpathlineto{\pgfqpoint{1.500263in}{1.044845in}}%
\pgfpathlineto{\pgfqpoint{1.500748in}{1.295719in}}%
\pgfpathlineto{\pgfqpoint{1.500969in}{0.951204in}}%
\pgfpathlineto{\pgfqpoint{1.501409in}{1.283372in}}%
\pgfpathlineto{\pgfqpoint{1.501520in}{0.950002in}}%
\pgfpathlineto{\pgfqpoint{1.502247in}{1.331777in}}%
\pgfpathlineto{\pgfqpoint{1.502489in}{1.247096in}}%
\pgfpathlineto{\pgfqpoint{1.502511in}{1.354504in}}%
\pgfpathlineto{\pgfqpoint{1.503437in}{1.016217in}}%
\pgfpathlineto{\pgfqpoint{1.503569in}{1.117179in}}%
\pgfpathlineto{\pgfqpoint{1.504253in}{1.010863in}}%
\pgfpathlineto{\pgfqpoint{1.504099in}{1.367616in}}%
\pgfpathlineto{\pgfqpoint{1.504628in}{1.151160in}}%
\pgfpathlineto{\pgfqpoint{1.505686in}{1.386082in}}%
\pgfpathlineto{\pgfqpoint{1.505024in}{1.019932in}}%
\pgfpathlineto{\pgfqpoint{1.505752in}{1.315168in}}%
\pgfpathlineto{\pgfqpoint{1.506016in}{1.357345in}}%
\pgfpathlineto{\pgfqpoint{1.505818in}{1.013376in}}%
\pgfpathlineto{\pgfqpoint{1.506501in}{1.240212in}}%
\pgfpathlineto{\pgfqpoint{1.506744in}{0.984967in}}%
\pgfpathlineto{\pgfqpoint{1.506920in}{1.365212in}}%
\pgfpathlineto{\pgfqpoint{1.507625in}{1.186890in}}%
\pgfpathlineto{\pgfqpoint{1.507669in}{1.185907in}}%
\pgfpathlineto{\pgfqpoint{1.507691in}{1.241523in}}%
\pgfpathlineto{\pgfqpoint{1.508176in}{1.367725in}}%
\pgfpathlineto{\pgfqpoint{1.508419in}{0.997642in}}%
\pgfpathlineto{\pgfqpoint{1.508683in}{1.088442in}}%
\pgfpathlineto{\pgfqpoint{1.509719in}{0.929787in}}%
\pgfpathlineto{\pgfqpoint{1.509411in}{1.366742in}}%
\pgfpathlineto{\pgfqpoint{1.509829in}{0.946614in}}%
\pgfpathlineto{\pgfqpoint{1.510028in}{1.249718in}}%
\pgfpathlineto{\pgfqpoint{1.510954in}{1.103411in}}%
\pgfpathlineto{\pgfqpoint{1.511813in}{1.301510in}}%
\pgfpathlineto{\pgfqpoint{1.511372in}{0.961256in}}%
\pgfpathlineto{\pgfqpoint{1.512056in}{1.121549in}}%
\pgfpathlineto{\pgfqpoint{1.512298in}{0.937545in}}%
\pgfpathlineto{\pgfqpoint{1.512673in}{1.251139in}}%
\pgfpathlineto{\pgfqpoint{1.513158in}{1.142419in}}%
\pgfpathlineto{\pgfqpoint{1.513687in}{0.810469in}}%
\pgfpathlineto{\pgfqpoint{1.513577in}{1.215299in}}%
\pgfpathlineto{\pgfqpoint{1.514326in}{0.953280in}}%
\pgfpathlineto{\pgfqpoint{1.514833in}{1.201204in}}%
\pgfpathlineto{\pgfqpoint{1.514613in}{0.922030in}}%
\pgfpathlineto{\pgfqpoint{1.515428in}{0.956995in}}%
\pgfpathlineto{\pgfqpoint{1.515759in}{1.276925in}}%
\pgfpathlineto{\pgfqpoint{1.515516in}{0.881601in}}%
\pgfpathlineto{\pgfqpoint{1.516552in}{1.095107in}}%
\pgfpathlineto{\pgfqpoint{1.517610in}{0.954045in}}%
\pgfpathlineto{\pgfqpoint{1.517169in}{1.259224in}}%
\pgfpathlineto{\pgfqpoint{1.517654in}{1.030858in}}%
\pgfpathlineto{\pgfqpoint{1.517919in}{0.886409in}}%
\pgfpathlineto{\pgfqpoint{1.517765in}{1.189294in}}%
\pgfpathlineto{\pgfqpoint{1.518514in}{1.096527in}}%
\pgfpathlineto{\pgfqpoint{1.519175in}{1.191698in}}%
\pgfpathlineto{\pgfqpoint{1.518756in}{0.895806in}}%
\pgfpathlineto{\pgfqpoint{1.519594in}{1.076532in}}%
\pgfpathlineto{\pgfqpoint{1.520365in}{0.833961in}}%
\pgfpathlineto{\pgfqpoint{1.519638in}{1.202734in}}%
\pgfpathlineto{\pgfqpoint{1.520718in}{0.991195in}}%
\pgfpathlineto{\pgfqpoint{1.521335in}{1.139906in}}%
\pgfpathlineto{\pgfqpoint{1.521225in}{0.798778in}}%
\pgfpathlineto{\pgfqpoint{1.521754in}{0.975024in}}%
\pgfpathlineto{\pgfqpoint{1.522790in}{0.832104in}}%
\pgfpathlineto{\pgfqpoint{1.522459in}{1.103630in}}%
\pgfpathlineto{\pgfqpoint{1.522856in}{0.998516in}}%
\pgfpathlineto{\pgfqpoint{1.523782in}{0.746330in}}%
\pgfpathlineto{\pgfqpoint{1.523518in}{1.094123in}}%
\pgfpathlineto{\pgfqpoint{1.523980in}{0.906405in}}%
\pgfpathlineto{\pgfqpoint{1.524487in}{1.172467in}}%
\pgfpathlineto{\pgfqpoint{1.524091in}{0.760425in}}%
\pgfpathlineto{\pgfqpoint{1.525082in}{0.953280in}}%
\pgfpathlineto{\pgfqpoint{1.525281in}{0.756273in}}%
\pgfpathlineto{\pgfqpoint{1.525854in}{1.093140in}}%
\pgfpathlineto{\pgfqpoint{1.526207in}{0.910120in}}%
\pgfpathlineto{\pgfqpoint{1.527353in}{1.234421in}}%
\pgfpathlineto{\pgfqpoint{1.527066in}{0.891107in}}%
\pgfpathlineto{\pgfqpoint{1.527375in}{1.108656in}}%
\pgfpathlineto{\pgfqpoint{1.527683in}{0.881164in}}%
\pgfpathlineto{\pgfqpoint{1.527992in}{1.226444in}}%
\pgfpathlineto{\pgfqpoint{1.528499in}{1.003870in}}%
\pgfpathlineto{\pgfqpoint{1.528918in}{1.134880in}}%
\pgfpathlineto{\pgfqpoint{1.528653in}{0.886081in}}%
\pgfpathlineto{\pgfqpoint{1.529579in}{1.025942in}}%
\pgfpathlineto{\pgfqpoint{1.529910in}{0.884442in}}%
\pgfpathlineto{\pgfqpoint{1.530196in}{1.232673in}}%
\pgfpathlineto{\pgfqpoint{1.530659in}{1.076750in}}%
\pgfpathlineto{\pgfqpoint{1.530681in}{1.156405in}}%
\pgfpathlineto{\pgfqpoint{1.530946in}{0.853301in}}%
\pgfpathlineto{\pgfqpoint{1.531739in}{1.035557in}}%
\pgfpathlineto{\pgfqpoint{1.532268in}{0.940386in}}%
\pgfpathlineto{\pgfqpoint{1.531960in}{1.256165in}}%
\pgfpathlineto{\pgfqpoint{1.532819in}{1.140124in}}%
\pgfpathlineto{\pgfqpoint{1.532951in}{1.240212in}}%
\pgfpathlineto{\pgfqpoint{1.533084in}{1.069539in}}%
\pgfpathlineto{\pgfqpoint{1.533458in}{1.071287in}}%
\pgfpathlineto{\pgfqpoint{1.534120in}{0.936999in}}%
\pgfpathlineto{\pgfqpoint{1.534318in}{1.179023in}}%
\pgfpathlineto{\pgfqpoint{1.534538in}{1.094779in}}%
\pgfpathlineto{\pgfqpoint{1.535310in}{1.154329in}}%
\pgfpathlineto{\pgfqpoint{1.535200in}{0.871986in}}%
\pgfpathlineto{\pgfqpoint{1.535442in}{0.996768in}}%
\pgfpathlineto{\pgfqpoint{1.535508in}{0.855596in}}%
\pgfpathlineto{\pgfqpoint{1.536302in}{1.110623in}}%
\pgfpathlineto{\pgfqpoint{1.536522in}{0.995893in}}%
\pgfpathlineto{\pgfqpoint{1.536941in}{1.131383in}}%
\pgfpathlineto{\pgfqpoint{1.537448in}{0.834617in}}%
\pgfpathlineto{\pgfqpoint{1.537580in}{0.951641in}}%
\pgfpathlineto{\pgfqpoint{1.538638in}{0.848057in}}%
\pgfpathlineto{\pgfqpoint{1.537845in}{1.141436in}}%
\pgfpathlineto{\pgfqpoint{1.538660in}{1.016763in}}%
\pgfpathlineto{\pgfqpoint{1.538881in}{1.155749in}}%
\pgfpathlineto{\pgfqpoint{1.539410in}{0.752012in}}%
\pgfpathlineto{\pgfqpoint{1.539718in}{0.871986in}}%
\pgfpathlineto{\pgfqpoint{1.540291in}{1.027690in}}%
\pgfpathlineto{\pgfqpoint{1.540578in}{0.702187in}}%
\pgfpathlineto{\pgfqpoint{1.540820in}{0.888813in}}%
\pgfpathlineto{\pgfqpoint{1.541305in}{0.812436in}}%
\pgfpathlineto{\pgfqpoint{1.541129in}{1.102318in}}%
\pgfpathlineto{\pgfqpoint{1.541878in}{0.956011in}}%
\pgfpathlineto{\pgfqpoint{1.542385in}{0.871549in}}%
\pgfpathlineto{\pgfqpoint{1.542077in}{1.163726in}}%
\pgfpathlineto{\pgfqpoint{1.542782in}{1.107563in}}%
\pgfpathlineto{\pgfqpoint{1.543355in}{1.231580in}}%
\pgfpathlineto{\pgfqpoint{1.542959in}{0.853738in}}%
\pgfpathlineto{\pgfqpoint{1.543862in}{1.124936in}}%
\pgfpathlineto{\pgfqpoint{1.544590in}{0.850133in}}%
\pgfpathlineto{\pgfqpoint{1.544083in}{1.178914in}}%
\pgfpathlineto{\pgfqpoint{1.545030in}{1.000701in}}%
\pgfpathlineto{\pgfqpoint{1.545493in}{1.184705in}}%
\pgfpathlineto{\pgfqpoint{1.545339in}{0.968358in}}%
\pgfpathlineto{\pgfqpoint{1.545846in}{1.035557in}}%
\pgfpathlineto{\pgfqpoint{1.545868in}{0.897554in}}%
\pgfpathlineto{\pgfqpoint{1.546750in}{1.259006in}}%
\pgfpathlineto{\pgfqpoint{1.546948in}{1.007803in}}%
\pgfpathlineto{\pgfqpoint{1.547213in}{0.930771in}}%
\pgfpathlineto{\pgfqpoint{1.548116in}{1.316917in}}%
\pgfpathlineto{\pgfqpoint{1.548227in}{1.140234in}}%
\pgfpathlineto{\pgfqpoint{1.548381in}{1.407061in}}%
\pgfpathlineto{\pgfqpoint{1.549152in}{1.344452in}}%
\pgfpathlineto{\pgfqpoint{1.549858in}{1.580794in}}%
\pgfpathlineto{\pgfqpoint{1.549593in}{1.185470in}}%
\pgfpathlineto{\pgfqpoint{1.550299in}{1.499391in}}%
\pgfpathlineto{\pgfqpoint{1.550497in}{1.329154in}}%
\pgfpathlineto{\pgfqpoint{1.551334in}{1.640999in}}%
\pgfpathlineto{\pgfqpoint{1.551423in}{1.412415in}}%
\pgfpathlineto{\pgfqpoint{1.551996in}{1.327297in}}%
\pgfpathlineto{\pgfqpoint{1.551731in}{1.593141in}}%
\pgfpathlineto{\pgfqpoint{1.552348in}{1.519714in}}%
\pgfpathlineto{\pgfqpoint{1.552459in}{1.568228in}}%
\pgfpathlineto{\pgfqpoint{1.553032in}{1.319976in}}%
\pgfpathlineto{\pgfqpoint{1.553318in}{1.444539in}}%
\pgfpathlineto{\pgfqpoint{1.553627in}{1.266217in}}%
\pgfpathlineto{\pgfqpoint{1.553759in}{1.622970in}}%
\pgfpathlineto{\pgfqpoint{1.554420in}{1.302931in}}%
\pgfpathlineto{\pgfqpoint{1.555104in}{1.582214in}}%
\pgfpathlineto{\pgfqpoint{1.555545in}{1.546266in}}%
\pgfpathlineto{\pgfqpoint{1.555765in}{1.417551in}}%
\pgfpathlineto{\pgfqpoint{1.556074in}{1.647555in}}%
\pgfpathlineto{\pgfqpoint{1.556647in}{1.524303in}}%
\pgfpathlineto{\pgfqpoint{1.557242in}{1.327297in}}%
\pgfpathlineto{\pgfqpoint{1.557771in}{1.586148in}}%
\pgfpathlineto{\pgfqpoint{1.558035in}{1.407389in}}%
\pgfpathlineto{\pgfqpoint{1.558763in}{1.696506in}}%
\pgfpathlineto{\pgfqpoint{1.558851in}{1.578062in}}%
\pgfpathlineto{\pgfqpoint{1.559314in}{1.762722in}}%
\pgfpathlineto{\pgfqpoint{1.559071in}{1.488573in}}%
\pgfpathlineto{\pgfqpoint{1.559975in}{1.714208in}}%
\pgfpathlineto{\pgfqpoint{1.560636in}{1.521900in}}%
\pgfpathlineto{\pgfqpoint{1.560548in}{1.871113in}}%
\pgfpathlineto{\pgfqpoint{1.561055in}{1.660121in}}%
\pgfpathlineto{\pgfqpoint{1.562025in}{1.845436in}}%
\pgfpathlineto{\pgfqpoint{1.561408in}{1.513158in}}%
\pgfpathlineto{\pgfqpoint{1.562157in}{1.756712in}}%
\pgfpathlineto{\pgfqpoint{1.562532in}{1.565060in}}%
\pgfpathlineto{\pgfqpoint{1.563149in}{1.857674in}}%
\pgfpathlineto{\pgfqpoint{1.563281in}{1.581122in}}%
\pgfpathlineto{\pgfqpoint{1.564185in}{1.793972in}}%
\pgfpathlineto{\pgfqpoint{1.563766in}{1.553040in}}%
\pgfpathlineto{\pgfqpoint{1.564405in}{1.697927in}}%
\pgfpathlineto{\pgfqpoint{1.565265in}{1.439732in}}%
\pgfpathlineto{\pgfqpoint{1.565089in}{1.712896in}}%
\pgfpathlineto{\pgfqpoint{1.565552in}{1.579701in}}%
\pgfpathlineto{\pgfqpoint{1.566323in}{1.381384in}}%
\pgfpathlineto{\pgfqpoint{1.566059in}{1.716939in}}%
\pgfpathlineto{\pgfqpoint{1.566499in}{1.625265in}}%
\pgfpathlineto{\pgfqpoint{1.567050in}{1.754199in}}%
\pgfpathlineto{\pgfqpoint{1.567469in}{1.408919in}}%
\pgfpathlineto{\pgfqpoint{1.567535in}{1.564404in}}%
\pgfpathlineto{\pgfqpoint{1.568329in}{1.409574in}}%
\pgfpathlineto{\pgfqpoint{1.567844in}{1.644387in}}%
\pgfpathlineto{\pgfqpoint{1.568615in}{1.639251in}}%
\pgfpathlineto{\pgfqpoint{1.569189in}{1.342048in}}%
\pgfpathlineto{\pgfqpoint{1.569497in}{1.708744in}}%
\pgfpathlineto{\pgfqpoint{1.569695in}{1.624609in}}%
\pgfpathlineto{\pgfqpoint{1.570798in}{1.761957in}}%
\pgfpathlineto{\pgfqpoint{1.570136in}{1.438420in}}%
\pgfpathlineto{\pgfqpoint{1.570842in}{1.727756in}}%
\pgfpathlineto{\pgfqpoint{1.570974in}{1.578499in}}%
\pgfpathlineto{\pgfqpoint{1.571701in}{1.805008in}}%
\pgfpathlineto{\pgfqpoint{1.571922in}{1.674763in}}%
\pgfpathlineto{\pgfqpoint{1.571966in}{1.867398in}}%
\pgfpathlineto{\pgfqpoint{1.572803in}{1.584181in}}%
\pgfpathlineto{\pgfqpoint{1.573002in}{1.689295in}}%
\pgfpathlineto{\pgfqpoint{1.573663in}{1.536213in}}%
\pgfpathlineto{\pgfqpoint{1.573178in}{1.791021in}}%
\pgfpathlineto{\pgfqpoint{1.574082in}{1.730488in}}%
\pgfpathlineto{\pgfqpoint{1.575118in}{1.884881in}}%
\pgfpathlineto{\pgfqpoint{1.574346in}{1.510973in}}%
\pgfpathlineto{\pgfqpoint{1.575162in}{1.806974in}}%
\pgfpathlineto{\pgfqpoint{1.575294in}{1.669081in}}%
\pgfpathlineto{\pgfqpoint{1.575911in}{1.983111in}}%
\pgfpathlineto{\pgfqpoint{1.576242in}{1.850899in}}%
\pgfpathlineto{\pgfqpoint{1.576506in}{1.982237in}}%
\pgfpathlineto{\pgfqpoint{1.577278in}{1.772228in}}%
\pgfpathlineto{\pgfqpoint{1.577366in}{1.975899in}}%
\pgfpathlineto{\pgfqpoint{1.578380in}{1.682411in}}%
\pgfpathlineto{\pgfqpoint{1.578512in}{1.870130in}}%
\pgfpathlineto{\pgfqpoint{1.578689in}{1.912088in}}%
\pgfpathlineto{\pgfqpoint{1.579041in}{1.687984in}}%
\pgfpathlineto{\pgfqpoint{1.579350in}{1.747315in}}%
\pgfpathlineto{\pgfqpoint{1.580143in}{1.612372in}}%
\pgfpathlineto{\pgfqpoint{1.579658in}{1.913399in}}%
\pgfpathlineto{\pgfqpoint{1.580408in}{1.717267in}}%
\pgfpathlineto{\pgfqpoint{1.580474in}{1.969890in}}%
\pgfpathlineto{\pgfqpoint{1.580761in}{1.687874in}}%
\pgfpathlineto{\pgfqpoint{1.581510in}{1.829265in}}%
\pgfpathlineto{\pgfqpoint{1.582590in}{1.586148in}}%
\pgfpathlineto{\pgfqpoint{1.581708in}{1.892857in}}%
\pgfpathlineto{\pgfqpoint{1.582656in}{1.702516in}}%
\pgfpathlineto{\pgfqpoint{1.583295in}{2.040257in}}%
\pgfpathlineto{\pgfqpoint{1.582877in}{1.693884in}}%
\pgfpathlineto{\pgfqpoint{1.583780in}{1.968360in}}%
\pgfpathlineto{\pgfqpoint{1.584243in}{1.667223in}}%
\pgfpathlineto{\pgfqpoint{1.583846in}{1.993163in}}%
\pgfpathlineto{\pgfqpoint{1.584904in}{1.788727in}}%
\pgfpathlineto{\pgfqpoint{1.584949in}{1.902800in}}%
\pgfpathlineto{\pgfqpoint{1.585500in}{1.586366in}}%
\pgfpathlineto{\pgfqpoint{1.585852in}{1.694103in}}%
\pgfpathlineto{\pgfqpoint{1.585874in}{1.641764in}}%
\pgfpathlineto{\pgfqpoint{1.586602in}{1.907062in}}%
\pgfpathlineto{\pgfqpoint{1.586910in}{1.732892in}}%
\pgfpathlineto{\pgfqpoint{1.587792in}{2.046376in}}%
\pgfpathlineto{\pgfqpoint{1.588101in}{1.997206in}}%
\pgfpathlineto{\pgfqpoint{1.588145in}{2.101774in}}%
\pgfpathlineto{\pgfqpoint{1.588938in}{1.768622in}}%
\pgfpathlineto{\pgfqpoint{1.589092in}{2.007696in}}%
\pgfpathlineto{\pgfqpoint{1.590040in}{1.749610in}}%
\pgfpathlineto{\pgfqpoint{1.589269in}{2.084838in}}%
\pgfpathlineto{\pgfqpoint{1.590217in}{1.878216in}}%
\pgfpathlineto{\pgfqpoint{1.590746in}{1.807193in}}%
\pgfpathlineto{\pgfqpoint{1.591319in}{2.047359in}}%
\pgfpathlineto{\pgfqpoint{1.591826in}{1.839863in}}%
\pgfpathlineto{\pgfqpoint{1.591892in}{2.234423in}}%
\pgfpathlineto{\pgfqpoint{1.592443in}{1.913946in}}%
\pgfpathlineto{\pgfqpoint{1.593369in}{2.168644in}}%
\pgfpathlineto{\pgfqpoint{1.592509in}{1.868600in}}%
\pgfpathlineto{\pgfqpoint{1.593545in}{1.981035in}}%
\pgfpathlineto{\pgfqpoint{1.594603in}{1.891655in}}%
\pgfpathlineto{\pgfqpoint{1.594074in}{2.167552in}}%
\pgfpathlineto{\pgfqpoint{1.594669in}{1.932412in}}%
\pgfpathlineto{\pgfqpoint{1.595264in}{2.155314in}}%
\pgfpathlineto{\pgfqpoint{1.595683in}{1.833635in}}%
\pgfpathlineto{\pgfqpoint{1.595815in}{2.051730in}}%
\pgfpathlineto{\pgfqpoint{1.596609in}{1.839973in}}%
\pgfpathlineto{\pgfqpoint{1.596300in}{2.149086in}}%
\pgfpathlineto{\pgfqpoint{1.596983in}{1.921813in}}%
\pgfpathlineto{\pgfqpoint{1.598108in}{2.254855in}}%
\pgfpathlineto{\pgfqpoint{1.597424in}{1.909575in}}%
\pgfpathlineto{\pgfqpoint{1.598196in}{2.065825in}}%
\pgfpathlineto{\pgfqpoint{1.599276in}{1.807193in}}%
\pgfpathlineto{\pgfqpoint{1.598394in}{2.147228in}}%
\pgfpathlineto{\pgfqpoint{1.599320in}{1.916568in}}%
\pgfpathlineto{\pgfqpoint{1.599474in}{2.068557in}}%
\pgfpathlineto{\pgfqpoint{1.599849in}{1.752997in}}%
\pgfpathlineto{\pgfqpoint{1.600422in}{1.987809in}}%
\pgfpathlineto{\pgfqpoint{1.601436in}{1.853412in}}%
\pgfpathlineto{\pgfqpoint{1.601017in}{2.175419in}}%
\pgfpathlineto{\pgfqpoint{1.601524in}{1.925528in}}%
\pgfpathlineto{\pgfqpoint{1.602097in}{2.097622in}}%
\pgfpathlineto{\pgfqpoint{1.602538in}{1.818010in}}%
\pgfpathlineto{\pgfqpoint{1.602626in}{1.943229in}}%
\pgfpathlineto{\pgfqpoint{1.602736in}{1.762394in}}%
\pgfpathlineto{\pgfqpoint{1.603287in}{2.057630in}}%
\pgfpathlineto{\pgfqpoint{1.603728in}{1.909356in}}%
\pgfpathlineto{\pgfqpoint{1.604852in}{2.193011in}}%
\pgfpathlineto{\pgfqpoint{1.604037in}{1.901599in}}%
\pgfpathlineto{\pgfqpoint{1.605029in}{2.118054in}}%
\pgfpathlineto{\pgfqpoint{1.605139in}{1.961476in}}%
\pgfpathlineto{\pgfqpoint{1.605536in}{2.203391in}}%
\pgfpathlineto{\pgfqpoint{1.605866in}{2.102102in}}%
\pgfpathlineto{\pgfqpoint{1.605888in}{2.230489in}}%
\pgfpathlineto{\pgfqpoint{1.606748in}{1.797687in}}%
\pgfpathlineto{\pgfqpoint{1.606946in}{1.923998in}}%
\pgfpathlineto{\pgfqpoint{1.607960in}{2.072272in}}%
\pgfpathlineto{\pgfqpoint{1.607542in}{1.817573in}}%
\pgfpathlineto{\pgfqpoint{1.608049in}{1.898648in}}%
\pgfpathlineto{\pgfqpoint{1.608886in}{2.202080in}}%
\pgfpathlineto{\pgfqpoint{1.608137in}{1.875375in}}%
\pgfpathlineto{\pgfqpoint{1.609239in}{2.114886in}}%
\pgfpathlineto{\pgfqpoint{1.610054in}{1.861826in}}%
\pgfpathlineto{\pgfqpoint{1.609900in}{2.135318in}}%
\pgfpathlineto{\pgfqpoint{1.610385in}{1.953937in}}%
\pgfpathlineto{\pgfqpoint{1.610473in}{2.063749in}}%
\pgfpathlineto{\pgfqpoint{1.611223in}{1.851008in}}%
\pgfpathlineto{\pgfqpoint{1.611465in}{1.933176in}}%
\pgfpathlineto{\pgfqpoint{1.611575in}{1.862591in}}%
\pgfpathlineto{\pgfqpoint{1.612170in}{2.092486in}}%
\pgfpathlineto{\pgfqpoint{1.612435in}{2.011739in}}%
\pgfpathlineto{\pgfqpoint{1.613030in}{2.168207in}}%
\pgfpathlineto{\pgfqpoint{1.612677in}{1.873299in}}%
\pgfpathlineto{\pgfqpoint{1.613559in}{2.103522in}}%
\pgfpathlineto{\pgfqpoint{1.613691in}{1.946944in}}%
\pgfpathlineto{\pgfqpoint{1.613824in}{2.180336in}}%
\pgfpathlineto{\pgfqpoint{1.614661in}{2.069650in}}%
\pgfpathlineto{\pgfqpoint{1.614705in}{2.179353in}}%
\pgfpathlineto{\pgfqpoint{1.615146in}{1.878434in}}%
\pgfpathlineto{\pgfqpoint{1.615719in}{2.033483in}}%
\pgfpathlineto{\pgfqpoint{1.616072in}{1.857564in}}%
\pgfpathlineto{\pgfqpoint{1.616358in}{2.125594in}}%
\pgfpathlineto{\pgfqpoint{1.616799in}{2.044628in}}%
\pgfpathlineto{\pgfqpoint{1.616976in}{2.196179in}}%
\pgfpathlineto{\pgfqpoint{1.617020in}{1.941044in}}%
\pgfpathlineto{\pgfqpoint{1.617923in}{2.090082in}}%
\pgfpathlineto{\pgfqpoint{1.618959in}{1.972512in}}%
\pgfpathlineto{\pgfqpoint{1.618474in}{2.265345in}}%
\pgfpathlineto{\pgfqpoint{1.619025in}{2.027582in}}%
\pgfpathlineto{\pgfqpoint{1.619202in}{2.230708in}}%
\pgfpathlineto{\pgfqpoint{1.619907in}{1.940388in}}%
\pgfpathlineto{\pgfqpoint{1.620150in}{2.130838in}}%
\pgfpathlineto{\pgfqpoint{1.620260in}{1.938968in}}%
\pgfpathlineto{\pgfqpoint{1.620943in}{2.208964in}}%
\pgfpathlineto{\pgfqpoint{1.621274in}{2.039820in}}%
\pgfpathlineto{\pgfqpoint{1.621648in}{2.244256in}}%
\pgfpathlineto{\pgfqpoint{1.621803in}{1.976883in}}%
\pgfpathlineto{\pgfqpoint{1.622398in}{2.204156in}}%
\pgfpathlineto{\pgfqpoint{1.622552in}{1.978850in}}%
\pgfpathlineto{\pgfqpoint{1.623390in}{2.266656in}}%
\pgfpathlineto{\pgfqpoint{1.623522in}{2.040038in}}%
\pgfpathlineto{\pgfqpoint{1.624602in}{2.343688in}}%
\pgfpathlineto{\pgfqpoint{1.623566in}{1.978959in}}%
\pgfpathlineto{\pgfqpoint{1.624690in}{2.201315in}}%
\pgfpathlineto{\pgfqpoint{1.625462in}{2.023430in}}%
\pgfpathlineto{\pgfqpoint{1.625131in}{2.316918in}}%
\pgfpathlineto{\pgfqpoint{1.625814in}{2.077626in}}%
\pgfpathlineto{\pgfqpoint{1.626564in}{2.232019in}}%
\pgfpathlineto{\pgfqpoint{1.626432in}{1.912307in}}%
\pgfpathlineto{\pgfqpoint{1.626894in}{1.993054in}}%
\pgfpathlineto{\pgfqpoint{1.627490in}{1.943229in}}%
\pgfpathlineto{\pgfqpoint{1.627137in}{2.134991in}}%
\pgfpathlineto{\pgfqpoint{1.627512in}{2.037416in}}%
\pgfpathlineto{\pgfqpoint{1.628305in}{2.151818in}}%
\pgfpathlineto{\pgfqpoint{1.627688in}{1.871113in}}%
\pgfpathlineto{\pgfqpoint{1.628592in}{1.930445in}}%
\pgfpathlineto{\pgfqpoint{1.628636in}{1.873189in}}%
\pgfpathlineto{\pgfqpoint{1.629231in}{2.208636in}}%
\pgfpathlineto{\pgfqpoint{1.629606in}{2.118382in}}%
\pgfpathlineto{\pgfqpoint{1.630708in}{1.932739in}}%
\pgfpathlineto{\pgfqpoint{1.630465in}{2.312220in}}%
\pgfpathlineto{\pgfqpoint{1.630752in}{2.049435in}}%
\pgfpathlineto{\pgfqpoint{1.630906in}{2.197272in}}%
\pgfpathlineto{\pgfqpoint{1.631501in}{1.969453in}}%
\pgfpathlineto{\pgfqpoint{1.631832in}{2.139908in}}%
\pgfpathlineto{\pgfqpoint{1.631920in}{1.979942in}}%
\pgfpathlineto{\pgfqpoint{1.632758in}{2.258133in}}%
\pgfpathlineto{\pgfqpoint{1.632912in}{2.147556in}}%
\pgfpathlineto{\pgfqpoint{1.632934in}{2.359423in}}%
\pgfpathlineto{\pgfqpoint{1.633860in}{2.092923in}}%
\pgfpathlineto{\pgfqpoint{1.634014in}{2.203391in}}%
\pgfpathlineto{\pgfqpoint{1.634301in}{2.031625in}}%
\pgfpathlineto{\pgfqpoint{1.634234in}{2.310799in}}%
\pgfpathlineto{\pgfqpoint{1.635094in}{2.256057in}}%
\pgfpathlineto{\pgfqpoint{1.635623in}{2.350572in}}%
\pgfpathlineto{\pgfqpoint{1.635998in}{2.086258in}}%
\pgfpathlineto{\pgfqpoint{1.636174in}{2.279003in}}%
\pgfpathlineto{\pgfqpoint{1.636857in}{2.043316in}}%
\pgfpathlineto{\pgfqpoint{1.636725in}{2.394825in}}%
\pgfpathlineto{\pgfqpoint{1.637320in}{2.065060in}}%
\pgfpathlineto{\pgfqpoint{1.638048in}{2.321617in}}%
\pgfpathlineto{\pgfqpoint{1.637761in}{2.033045in}}%
\pgfpathlineto{\pgfqpoint{1.638466in}{2.250157in}}%
\pgfpathlineto{\pgfqpoint{1.639304in}{2.007149in}}%
\pgfpathlineto{\pgfqpoint{1.638709in}{2.272119in}}%
\pgfpathlineto{\pgfqpoint{1.639613in}{2.099807in}}%
\pgfpathlineto{\pgfqpoint{1.639921in}{2.304899in}}%
\pgfpathlineto{\pgfqpoint{1.640340in}{2.008679in}}%
\pgfpathlineto{\pgfqpoint{1.640671in}{2.044409in}}%
\pgfpathlineto{\pgfqpoint{1.640979in}{1.919518in}}%
\pgfpathlineto{\pgfqpoint{1.641398in}{2.221311in}}%
\pgfpathlineto{\pgfqpoint{1.641795in}{1.963115in}}%
\pgfpathlineto{\pgfqpoint{1.642214in}{2.287963in}}%
\pgfpathlineto{\pgfqpoint{1.641905in}{1.901817in}}%
\pgfpathlineto{\pgfqpoint{1.643029in}{2.175091in}}%
\pgfpathlineto{\pgfqpoint{1.643955in}{2.057740in}}%
\pgfpathlineto{\pgfqpoint{1.644021in}{2.353632in}}%
\pgfpathlineto{\pgfqpoint{1.644153in}{2.058504in}}%
\pgfpathlineto{\pgfqpoint{1.644352in}{2.318120in}}%
\pgfpathlineto{\pgfqpoint{1.644660in}{2.019169in}}%
\pgfpathlineto{\pgfqpoint{1.645277in}{2.314514in}}%
\pgfpathlineto{\pgfqpoint{1.646115in}{2.088225in}}%
\pgfpathlineto{\pgfqpoint{1.645476in}{2.316481in}}%
\pgfpathlineto{\pgfqpoint{1.646402in}{2.167115in}}%
\pgfpathlineto{\pgfqpoint{1.647305in}{2.431210in}}%
\pgfpathlineto{\pgfqpoint{1.646468in}{2.122971in}}%
\pgfpathlineto{\pgfqpoint{1.647636in}{2.214645in}}%
\pgfpathlineto{\pgfqpoint{1.647658in}{2.082980in}}%
\pgfpathlineto{\pgfqpoint{1.648385in}{2.352976in}}%
\pgfpathlineto{\pgfqpoint{1.648716in}{2.233111in}}%
\pgfpathlineto{\pgfqpoint{1.649774in}{2.514034in}}%
\pgfpathlineto{\pgfqpoint{1.649465in}{2.081778in}}%
\pgfpathlineto{\pgfqpoint{1.649862in}{2.404113in}}%
\pgfpathlineto{\pgfqpoint{1.650083in}{2.158811in}}%
\pgfpathlineto{\pgfqpoint{1.650171in}{2.442356in}}%
\pgfpathlineto{\pgfqpoint{1.650964in}{2.415804in}}%
\pgfpathlineto{\pgfqpoint{1.651493in}{2.226118in}}%
\pgfpathlineto{\pgfqpoint{1.651361in}{2.471311in}}%
\pgfpathlineto{\pgfqpoint{1.652066in}{2.360188in}}%
\pgfpathlineto{\pgfqpoint{1.652860in}{2.592159in}}%
\pgfpathlineto{\pgfqpoint{1.652794in}{2.312001in}}%
\pgfpathlineto{\pgfqpoint{1.653191in}{2.505184in}}%
\pgfpathlineto{\pgfqpoint{1.654006in}{2.268841in}}%
\pgfpathlineto{\pgfqpoint{1.654204in}{2.538400in}}%
\pgfpathlineto{\pgfqpoint{1.654315in}{2.404877in}}%
\pgfpathlineto{\pgfqpoint{1.654645in}{2.274960in}}%
\pgfpathlineto{\pgfqpoint{1.654800in}{2.493274in}}%
\pgfpathlineto{\pgfqpoint{1.655417in}{2.372972in}}%
\pgfpathlineto{\pgfqpoint{1.655858in}{2.528785in}}%
\pgfpathlineto{\pgfqpoint{1.655725in}{2.188968in}}%
\pgfpathlineto{\pgfqpoint{1.656475in}{2.446289in}}%
\pgfpathlineto{\pgfqpoint{1.656651in}{2.555555in}}%
\pgfpathlineto{\pgfqpoint{1.657599in}{2.250485in}}%
\pgfpathlineto{\pgfqpoint{1.658062in}{2.485188in}}%
\pgfpathlineto{\pgfqpoint{1.658348in}{2.150616in}}%
\pgfpathlineto{\pgfqpoint{1.658745in}{2.430118in}}%
\pgfpathlineto{\pgfqpoint{1.659847in}{2.211695in}}%
\pgfpathlineto{\pgfqpoint{1.658921in}{2.498955in}}%
\pgfpathlineto{\pgfqpoint{1.659869in}{2.340957in}}%
\pgfpathlineto{\pgfqpoint{1.660553in}{2.111061in}}%
\pgfpathlineto{\pgfqpoint{1.660707in}{2.403238in}}%
\pgfpathlineto{\pgfqpoint{1.660993in}{2.303806in}}%
\pgfpathlineto{\pgfqpoint{1.661192in}{2.488903in}}%
\pgfpathlineto{\pgfqpoint{1.661500in}{2.124392in}}%
\pgfpathlineto{\pgfqpoint{1.662029in}{2.300747in}}%
\pgfpathlineto{\pgfqpoint{1.662757in}{2.056756in}}%
\pgfpathlineto{\pgfqpoint{1.662580in}{2.450988in}}%
\pgfpathlineto{\pgfqpoint{1.663131in}{2.212897in}}%
\pgfpathlineto{\pgfqpoint{1.664057in}{2.468033in}}%
\pgfpathlineto{\pgfqpoint{1.663396in}{2.160122in}}%
\pgfpathlineto{\pgfqpoint{1.664256in}{2.312657in}}%
\pgfpathlineto{\pgfqpoint{1.664278in}{2.314296in}}%
\pgfpathlineto{\pgfqpoint{1.664300in}{2.297141in}}%
\pgfpathlineto{\pgfqpoint{1.665137in}{2.079265in}}%
\pgfpathlineto{\pgfqpoint{1.665225in}{2.432631in}}%
\pgfpathlineto{\pgfqpoint{1.665402in}{2.247097in}}%
\pgfpathlineto{\pgfqpoint{1.666283in}{2.441919in}}%
\pgfpathlineto{\pgfqpoint{1.665622in}{2.101337in}}%
\pgfpathlineto{\pgfqpoint{1.666416in}{2.173780in}}%
\pgfpathlineto{\pgfqpoint{1.667121in}{2.121879in}}%
\pgfpathlineto{\pgfqpoint{1.666746in}{2.405861in}}%
\pgfpathlineto{\pgfqpoint{1.667386in}{2.269606in}}%
\pgfpathlineto{\pgfqpoint{1.667408in}{2.403020in}}%
\pgfpathlineto{\pgfqpoint{1.668267in}{2.038727in}}%
\pgfpathlineto{\pgfqpoint{1.668488in}{2.223496in}}%
\pgfpathlineto{\pgfqpoint{1.669215in}{2.130620in}}%
\pgfpathlineto{\pgfqpoint{1.669634in}{2.365432in}}%
\pgfpathlineto{\pgfqpoint{1.669678in}{2.196944in}}%
\pgfpathlineto{\pgfqpoint{1.670604in}{2.494694in}}%
\pgfpathlineto{\pgfqpoint{1.670736in}{2.353195in}}%
\pgfpathlineto{\pgfqpoint{1.670780in}{2.357128in}}%
\pgfpathlineto{\pgfqpoint{1.670802in}{2.351993in}}%
\pgfpathlineto{\pgfqpoint{1.671067in}{2.113684in}}%
\pgfpathlineto{\pgfqpoint{1.671662in}{2.469016in}}%
\pgfpathlineto{\pgfqpoint{1.671926in}{2.254965in}}%
\pgfpathlineto{\pgfqpoint{1.672389in}{2.538619in}}%
\pgfpathlineto{\pgfqpoint{1.673050in}{2.391219in}}%
\pgfpathlineto{\pgfqpoint{1.673359in}{2.179462in}}%
\pgfpathlineto{\pgfqpoint{1.673447in}{2.465848in}}%
\pgfpathlineto{\pgfqpoint{1.674175in}{2.278566in}}%
\pgfpathlineto{\pgfqpoint{1.674637in}{2.393077in}}%
\pgfpathlineto{\pgfqpoint{1.675188in}{2.196179in}}%
\pgfpathlineto{\pgfqpoint{1.675299in}{2.318885in}}%
\pgfpathlineto{\pgfqpoint{1.675585in}{2.097949in}}%
\pgfpathlineto{\pgfqpoint{1.675982in}{2.361608in}}%
\pgfpathlineto{\pgfqpoint{1.676313in}{2.289711in}}%
\pgfpathlineto{\pgfqpoint{1.676335in}{2.496114in}}%
\pgfpathlineto{\pgfqpoint{1.676467in}{2.112919in}}%
\pgfpathlineto{\pgfqpoint{1.677415in}{2.220109in}}%
\pgfpathlineto{\pgfqpoint{1.678407in}{2.404877in}}%
\pgfpathlineto{\pgfqpoint{1.677966in}{2.144169in}}%
\pgfpathlineto{\pgfqpoint{1.678495in}{2.253435in}}%
\pgfpathlineto{\pgfqpoint{1.678869in}{2.172032in}}%
\pgfpathlineto{\pgfqpoint{1.679465in}{2.474698in}}%
\pgfpathlineto{\pgfqpoint{1.679553in}{2.413728in}}%
\pgfpathlineto{\pgfqpoint{1.680456in}{2.522229in}}%
\pgfpathlineto{\pgfqpoint{1.680192in}{2.254309in}}%
\pgfpathlineto{\pgfqpoint{1.680633in}{2.352539in}}%
\pgfpathlineto{\pgfqpoint{1.681492in}{2.236608in}}%
\pgfpathlineto{\pgfqpoint{1.681382in}{2.542225in}}%
\pgfpathlineto{\pgfqpoint{1.681735in}{2.337133in}}%
\pgfpathlineto{\pgfqpoint{1.682264in}{2.472622in}}%
\pgfpathlineto{\pgfqpoint{1.682440in}{2.271136in}}%
\pgfpathlineto{\pgfqpoint{1.682815in}{2.387941in}}%
\pgfpathlineto{\pgfqpoint{1.683851in}{2.246005in}}%
\pgfpathlineto{\pgfqpoint{1.683366in}{2.526163in}}%
\pgfpathlineto{\pgfqpoint{1.683895in}{2.323584in}}%
\pgfpathlineto{\pgfqpoint{1.684578in}{2.497644in}}%
\pgfpathlineto{\pgfqpoint{1.684799in}{2.202408in}}%
\pgfpathlineto{\pgfqpoint{1.684997in}{2.330904in}}%
\pgfpathlineto{\pgfqpoint{1.685394in}{2.447382in}}%
\pgfpathlineto{\pgfqpoint{1.685680in}{2.119147in}}%
\pgfpathlineto{\pgfqpoint{1.686055in}{2.329265in}}%
\pgfpathlineto{\pgfqpoint{1.686805in}{2.238465in}}%
\pgfpathlineto{\pgfqpoint{1.686959in}{2.478632in}}%
\pgfpathlineto{\pgfqpoint{1.687135in}{2.400397in}}%
\pgfpathlineto{\pgfqpoint{1.687422in}{2.505730in}}%
\pgfpathlineto{\pgfqpoint{1.688105in}{2.215957in}}%
\pgfpathlineto{\pgfqpoint{1.688237in}{2.409248in}}%
\pgfpathlineto{\pgfqpoint{1.688700in}{2.575332in}}%
\pgfpathlineto{\pgfqpoint{1.688899in}{2.292771in}}%
\pgfpathlineto{\pgfqpoint{1.689361in}{2.435144in}}%
\pgfpathlineto{\pgfqpoint{1.690464in}{2.208854in}}%
\pgfpathlineto{\pgfqpoint{1.689626in}{2.519170in}}%
\pgfpathlineto{\pgfqpoint{1.690530in}{2.232237in}}%
\pgfpathlineto{\pgfqpoint{1.690971in}{2.547032in}}%
\pgfpathlineto{\pgfqpoint{1.691654in}{2.409139in}}%
\pgfpathlineto{\pgfqpoint{1.691698in}{2.173561in}}%
\pgfpathlineto{\pgfqpoint{1.692668in}{2.450441in}}%
\pgfpathlineto{\pgfqpoint{1.692756in}{2.415039in}}%
\pgfpathlineto{\pgfqpoint{1.693461in}{2.491962in}}%
\pgfpathlineto{\pgfqpoint{1.692800in}{2.267530in}}%
\pgfpathlineto{\pgfqpoint{1.693792in}{2.368710in}}%
\pgfpathlineto{\pgfqpoint{1.693836in}{2.234532in}}%
\pgfpathlineto{\pgfqpoint{1.694343in}{2.531517in}}%
\pgfpathlineto{\pgfqpoint{1.694607in}{2.446289in}}%
\pgfpathlineto{\pgfqpoint{1.694960in}{2.515017in}}%
\pgfpathlineto{\pgfqpoint{1.695313in}{2.215520in}}%
\pgfpathlineto{\pgfqpoint{1.695555in}{2.225572in}}%
\pgfpathlineto{\pgfqpoint{1.696283in}{2.188640in}}%
\pgfpathlineto{\pgfqpoint{1.695754in}{2.439952in}}%
\pgfpathlineto{\pgfqpoint{1.696393in}{2.409794in}}%
\pgfpathlineto{\pgfqpoint{1.697473in}{2.544301in}}%
\pgfpathlineto{\pgfqpoint{1.696856in}{2.204265in}}%
\pgfpathlineto{\pgfqpoint{1.697495in}{2.500594in}}%
\pgfpathlineto{\pgfqpoint{1.698244in}{2.239667in}}%
\pgfpathlineto{\pgfqpoint{1.697539in}{2.561565in}}%
\pgfpathlineto{\pgfqpoint{1.698619in}{2.318994in}}%
\pgfpathlineto{\pgfqpoint{1.699677in}{2.577080in}}%
\pgfpathlineto{\pgfqpoint{1.698994in}{2.220327in}}%
\pgfpathlineto{\pgfqpoint{1.699765in}{2.460166in}}%
\pgfpathlineto{\pgfqpoint{1.700603in}{2.637942in}}%
\pgfpathlineto{\pgfqpoint{1.700360in}{2.316372in}}%
\pgfpathlineto{\pgfqpoint{1.700889in}{2.562002in}}%
\pgfpathlineto{\pgfqpoint{1.701881in}{2.309051in}}%
\pgfpathlineto{\pgfqpoint{1.701330in}{2.566809in}}%
\pgfpathlineto{\pgfqpoint{1.702036in}{2.438531in}}%
\pgfpathlineto{\pgfqpoint{1.702829in}{2.283811in}}%
\pgfpathlineto{\pgfqpoint{1.702521in}{2.475354in}}%
\pgfpathlineto{\pgfqpoint{1.703270in}{2.329265in}}%
\pgfpathlineto{\pgfqpoint{1.703446in}{2.445197in}}%
\pgfpathlineto{\pgfqpoint{1.703821in}{2.116197in}}%
\pgfpathlineto{\pgfqpoint{1.704306in}{2.188422in}}%
\pgfpathlineto{\pgfqpoint{1.704482in}{2.072600in}}%
\pgfpathlineto{\pgfqpoint{1.705320in}{2.337351in}}%
\pgfpathlineto{\pgfqpoint{1.705386in}{2.240869in}}%
\pgfpathlineto{\pgfqpoint{1.705915in}{2.152473in}}%
\pgfpathlineto{\pgfqpoint{1.705717in}{2.402036in}}%
\pgfpathlineto{\pgfqpoint{1.706003in}{2.372862in}}%
\pgfpathlineto{\pgfqpoint{1.706664in}{2.563204in}}%
\pgfpathlineto{\pgfqpoint{1.706179in}{2.278020in}}%
\pgfpathlineto{\pgfqpoint{1.707127in}{2.432631in}}%
\pgfpathlineto{\pgfqpoint{1.707414in}{2.202517in}}%
\pgfpathlineto{\pgfqpoint{1.707568in}{2.564406in}}%
\pgfpathlineto{\pgfqpoint{1.708229in}{2.416350in}}%
\pgfpathlineto{\pgfqpoint{1.708736in}{2.558396in}}%
\pgfpathlineto{\pgfqpoint{1.708384in}{2.332653in}}%
\pgfpathlineto{\pgfqpoint{1.709309in}{2.384117in}}%
\pgfpathlineto{\pgfqpoint{1.710015in}{2.613029in}}%
\pgfpathlineto{\pgfqpoint{1.709376in}{2.293426in}}%
\pgfpathlineto{\pgfqpoint{1.710390in}{2.407172in}}%
\pgfpathlineto{\pgfqpoint{1.711205in}{2.240104in}}%
\pgfpathlineto{\pgfqpoint{1.710808in}{2.493383in}}%
\pgfpathlineto{\pgfqpoint{1.711514in}{2.297797in}}%
\pgfpathlineto{\pgfqpoint{1.711646in}{2.572928in}}%
\pgfpathlineto{\pgfqpoint{1.712087in}{2.227102in}}%
\pgfpathlineto{\pgfqpoint{1.712638in}{2.470765in}}%
\pgfpathlineto{\pgfqpoint{1.713365in}{2.135428in}}%
\pgfpathlineto{\pgfqpoint{1.712990in}{2.486936in}}%
\pgfpathlineto{\pgfqpoint{1.713806in}{2.302167in}}%
\pgfpathlineto{\pgfqpoint{1.714489in}{2.497535in}}%
\pgfpathlineto{\pgfqpoint{1.714644in}{2.243929in}}%
\pgfpathlineto{\pgfqpoint{1.714842in}{2.351119in}}%
\pgfpathlineto{\pgfqpoint{1.714864in}{2.236826in}}%
\pgfpathlineto{\pgfqpoint{1.715591in}{2.604397in}}%
\pgfpathlineto{\pgfqpoint{1.715922in}{2.453829in}}%
\pgfpathlineto{\pgfqpoint{1.716142in}{2.527146in}}%
\pgfpathlineto{\pgfqpoint{1.716297in}{2.280314in}}%
\pgfpathlineto{\pgfqpoint{1.716583in}{2.336805in}}%
\pgfpathlineto{\pgfqpoint{1.716848in}{2.121114in}}%
\pgfpathlineto{\pgfqpoint{1.716870in}{2.405205in}}%
\pgfpathlineto{\pgfqpoint{1.717685in}{2.340520in}}%
\pgfpathlineto{\pgfqpoint{1.718721in}{2.546595in}}%
\pgfpathlineto{\pgfqpoint{1.718281in}{2.264034in}}%
\pgfpathlineto{\pgfqpoint{1.718832in}{2.441481in}}%
\pgfpathlineto{\pgfqpoint{1.719978in}{2.208636in}}%
\pgfpathlineto{\pgfqpoint{1.719581in}{2.566919in}}%
\pgfpathlineto{\pgfqpoint{1.720000in}{2.332653in}}%
\pgfpathlineto{\pgfqpoint{1.720705in}{2.470765in}}%
\pgfpathlineto{\pgfqpoint{1.720639in}{2.250703in}}%
\pgfpathlineto{\pgfqpoint{1.720948in}{2.346529in}}%
\pgfpathlineto{\pgfqpoint{1.721190in}{2.170721in}}%
\pgfpathlineto{\pgfqpoint{1.721962in}{2.402474in}}%
\pgfpathlineto{\pgfqpoint{1.722050in}{2.310581in}}%
\pgfpathlineto{\pgfqpoint{1.722535in}{2.189405in}}%
\pgfpathlineto{\pgfqpoint{1.723284in}{2.509226in}}%
\pgfpathlineto{\pgfqpoint{1.723571in}{2.218251in}}%
\pgfpathlineto{\pgfqpoint{1.724408in}{2.326752in}}%
\pgfpathlineto{\pgfqpoint{1.724893in}{2.506713in}}%
\pgfpathlineto{\pgfqpoint{1.724452in}{2.198037in}}%
\pgfpathlineto{\pgfqpoint{1.725554in}{2.423234in}}%
\pgfpathlineto{\pgfqpoint{1.726150in}{2.279003in}}%
\pgfpathlineto{\pgfqpoint{1.725841in}{2.544410in}}%
\pgfpathlineto{\pgfqpoint{1.726679in}{2.353850in}}%
\pgfpathlineto{\pgfqpoint{1.727516in}{2.595765in}}%
\pgfpathlineto{\pgfqpoint{1.726921in}{2.268732in}}%
\pgfpathlineto{\pgfqpoint{1.727781in}{2.442028in}}%
\pgfpathlineto{\pgfqpoint{1.728728in}{2.182849in}}%
\pgfpathlineto{\pgfqpoint{1.728310in}{2.463007in}}%
\pgfpathlineto{\pgfqpoint{1.728883in}{2.335056in}}%
\pgfpathlineto{\pgfqpoint{1.729279in}{2.539384in}}%
\pgfpathlineto{\pgfqpoint{1.729588in}{2.295502in}}%
\pgfpathlineto{\pgfqpoint{1.730007in}{2.476774in}}%
\pgfpathlineto{\pgfqpoint{1.731065in}{2.165913in}}%
\pgfpathlineto{\pgfqpoint{1.731131in}{2.326862in}}%
\pgfpathlineto{\pgfqpoint{1.731814in}{2.193120in}}%
\pgfpathlineto{\pgfqpoint{1.731726in}{2.450988in}}%
\pgfpathlineto{\pgfqpoint{1.731969in}{2.361499in}}%
\pgfpathlineto{\pgfqpoint{1.732806in}{2.552714in}}%
\pgfpathlineto{\pgfqpoint{1.732431in}{2.265236in}}%
\pgfpathlineto{\pgfqpoint{1.733027in}{2.401599in}}%
\pgfpathlineto{\pgfqpoint{1.733049in}{2.220874in}}%
\pgfpathlineto{\pgfqpoint{1.733467in}{2.618711in}}%
\pgfpathlineto{\pgfqpoint{1.734129in}{2.405642in}}%
\pgfpathlineto{\pgfqpoint{1.734834in}{2.581233in}}%
\pgfpathlineto{\pgfqpoint{1.734371in}{2.348387in}}%
\pgfpathlineto{\pgfqpoint{1.735275in}{2.525944in}}%
\pgfpathlineto{\pgfqpoint{1.736443in}{2.339536in}}%
\pgfpathlineto{\pgfqpoint{1.735848in}{2.604615in}}%
\pgfpathlineto{\pgfqpoint{1.736487in}{2.449567in}}%
\pgfpathlineto{\pgfqpoint{1.736906in}{2.565826in}}%
\pgfpathlineto{\pgfqpoint{1.737148in}{2.313094in}}%
\pgfpathlineto{\pgfqpoint{1.737479in}{2.444650in}}%
\pgfpathlineto{\pgfqpoint{1.737501in}{2.271682in}}%
\pgfpathlineto{\pgfqpoint{1.738295in}{2.578829in}}%
\pgfpathlineto{\pgfqpoint{1.738581in}{2.381167in}}%
\pgfpathlineto{\pgfqpoint{1.738647in}{2.575114in}}%
\pgfpathlineto{\pgfqpoint{1.739617in}{2.264471in}}%
\pgfpathlineto{\pgfqpoint{1.739683in}{2.416569in}}%
\pgfpathlineto{\pgfqpoint{1.740036in}{2.208308in}}%
\pgfpathlineto{\pgfqpoint{1.740256in}{2.492290in}}%
\pgfpathlineto{\pgfqpoint{1.740653in}{2.399414in}}%
\pgfpathlineto{\pgfqpoint{1.741425in}{2.537963in}}%
\pgfpathlineto{\pgfqpoint{1.740741in}{2.274414in}}%
\pgfpathlineto{\pgfqpoint{1.741733in}{2.393077in}}%
\pgfpathlineto{\pgfqpoint{1.742174in}{2.284139in}}%
\pgfpathlineto{\pgfqpoint{1.742086in}{2.567793in}}%
\pgfpathlineto{\pgfqpoint{1.742857in}{2.291787in}}%
\pgfpathlineto{\pgfqpoint{1.743430in}{2.672360in}}%
\pgfpathlineto{\pgfqpoint{1.743959in}{2.426184in}}%
\pgfpathlineto{\pgfqpoint{1.744797in}{2.319104in}}%
\pgfpathlineto{\pgfqpoint{1.744687in}{2.600354in}}%
\pgfpathlineto{\pgfqpoint{1.745062in}{2.416569in}}%
\pgfpathlineto{\pgfqpoint{1.746164in}{2.674218in}}%
\pgfpathlineto{\pgfqpoint{1.745172in}{2.339209in}}%
\pgfpathlineto{\pgfqpoint{1.746208in}{2.512941in}}%
\pgfpathlineto{\pgfqpoint{1.747266in}{2.300419in}}%
\pgfpathlineto{\pgfqpoint{1.747067in}{2.613248in}}%
\pgfpathlineto{\pgfqpoint{1.747332in}{2.442028in}}%
\pgfpathlineto{\pgfqpoint{1.747971in}{2.163618in}}%
\pgfpathlineto{\pgfqpoint{1.747552in}{2.535232in}}%
\pgfpathlineto{\pgfqpoint{1.748544in}{2.347404in}}%
\pgfpathlineto{\pgfqpoint{1.748809in}{2.415148in}}%
\pgfpathlineto{\pgfqpoint{1.748897in}{2.167880in}}%
\pgfpathlineto{\pgfqpoint{1.749602in}{2.320743in}}%
\pgfpathlineto{\pgfqpoint{1.749646in}{2.187329in}}%
\pgfpathlineto{\pgfqpoint{1.749955in}{2.463335in}}%
\pgfpathlineto{\pgfqpoint{1.750660in}{2.335712in}}%
\pgfpathlineto{\pgfqpoint{1.751432in}{2.641657in}}%
\pgfpathlineto{\pgfqpoint{1.751255in}{2.302823in}}%
\pgfpathlineto{\pgfqpoint{1.751806in}{2.554572in}}%
\pgfpathlineto{\pgfqpoint{1.751828in}{2.614996in}}%
\pgfpathlineto{\pgfqpoint{1.752225in}{2.345437in}}%
\pgfpathlineto{\pgfqpoint{1.752864in}{2.443667in}}%
\pgfpathlineto{\pgfqpoint{1.753570in}{2.296486in}}%
\pgfpathlineto{\pgfqpoint{1.753041in}{2.645481in}}%
\pgfpathlineto{\pgfqpoint{1.753944in}{2.475026in}}%
\pgfpathlineto{\pgfqpoint{1.754738in}{2.643186in}}%
\pgfpathlineto{\pgfqpoint{1.754958in}{2.293535in}}%
\pgfpathlineto{\pgfqpoint{1.755025in}{2.536652in}}%
\pgfpathlineto{\pgfqpoint{1.755598in}{2.340083in}}%
\pgfpathlineto{\pgfqpoint{1.755950in}{2.673999in}}%
\pgfpathlineto{\pgfqpoint{1.756127in}{2.484532in}}%
\pgfpathlineto{\pgfqpoint{1.757251in}{2.668208in}}%
\pgfpathlineto{\pgfqpoint{1.756656in}{2.380183in}}%
\pgfpathlineto{\pgfqpoint{1.757273in}{2.564078in}}%
\pgfpathlineto{\pgfqpoint{1.757537in}{2.368383in}}%
\pgfpathlineto{\pgfqpoint{1.758353in}{2.665477in}}%
\pgfpathlineto{\pgfqpoint{1.758375in}{2.504419in}}%
\pgfpathlineto{\pgfqpoint{1.759389in}{2.642312in}}%
\pgfpathlineto{\pgfqpoint{1.758639in}{2.317683in}}%
\pgfpathlineto{\pgfqpoint{1.759455in}{2.473169in}}%
\pgfpathlineto{\pgfqpoint{1.760226in}{2.368820in}}%
\pgfpathlineto{\pgfqpoint{1.759653in}{2.606145in}}%
\pgfpathlineto{\pgfqpoint{1.760359in}{2.483658in}}%
\pgfpathlineto{\pgfqpoint{1.760866in}{2.568448in}}%
\pgfpathlineto{\pgfqpoint{1.761351in}{2.329375in}}%
\pgfpathlineto{\pgfqpoint{1.761417in}{2.496879in}}%
\pgfpathlineto{\pgfqpoint{1.761439in}{2.301403in}}%
\pgfpathlineto{\pgfqpoint{1.762276in}{2.597513in}}%
\pgfpathlineto{\pgfqpoint{1.762519in}{2.473824in}}%
\pgfpathlineto{\pgfqpoint{1.763489in}{2.773322in}}%
\pgfpathlineto{\pgfqpoint{1.762629in}{2.401490in}}%
\pgfpathlineto{\pgfqpoint{1.763643in}{2.511958in}}%
\pgfpathlineto{\pgfqpoint{1.763665in}{2.512067in}}%
\pgfpathlineto{\pgfqpoint{1.763753in}{2.670394in}}%
\pgfpathlineto{\pgfqpoint{1.764414in}{2.379746in}}%
\pgfpathlineto{\pgfqpoint{1.764745in}{2.474371in}}%
\pgfpathlineto{\pgfqpoint{1.764789in}{2.355271in}}%
\pgfpathlineto{\pgfqpoint{1.765671in}{2.696836in}}%
\pgfpathlineto{\pgfqpoint{1.765759in}{2.592268in}}%
\pgfpathlineto{\pgfqpoint{1.765781in}{2.652583in}}%
\pgfpathlineto{\pgfqpoint{1.766023in}{2.381167in}}%
\pgfpathlineto{\pgfqpoint{1.766795in}{2.512832in}}%
\pgfpathlineto{\pgfqpoint{1.767787in}{2.251031in}}%
\pgfpathlineto{\pgfqpoint{1.767126in}{2.602539in}}%
\pgfpathlineto{\pgfqpoint{1.767897in}{2.505839in}}%
\pgfpathlineto{\pgfqpoint{1.768228in}{2.302495in}}%
\pgfpathlineto{\pgfqpoint{1.768382in}{2.530533in}}%
\pgfpathlineto{\pgfqpoint{1.769021in}{2.404550in}}%
\pgfpathlineto{\pgfqpoint{1.769043in}{2.535232in}}%
\pgfpathlineto{\pgfqpoint{1.769969in}{2.255620in}}%
\pgfpathlineto{\pgfqpoint{1.770123in}{2.513488in}}%
\pgfpathlineto{\pgfqpoint{1.770211in}{2.252233in}}%
\pgfpathlineto{\pgfqpoint{1.771049in}{2.587024in}}%
\pgfpathlineto{\pgfqpoint{1.771292in}{2.420393in}}%
\pgfpathlineto{\pgfqpoint{1.771644in}{2.631823in}}%
\pgfpathlineto{\pgfqpoint{1.771953in}{2.325878in}}%
\pgfpathlineto{\pgfqpoint{1.772394in}{2.426403in}}%
\pgfpathlineto{\pgfqpoint{1.772724in}{2.258133in}}%
\pgfpathlineto{\pgfqpoint{1.772526in}{2.537308in}}%
\pgfpathlineto{\pgfqpoint{1.773474in}{2.513378in}}%
\pgfpathlineto{\pgfqpoint{1.774488in}{2.391438in}}%
\pgfpathlineto{\pgfqpoint{1.773981in}{2.679572in}}%
\pgfpathlineto{\pgfqpoint{1.774576in}{2.465083in}}%
\pgfpathlineto{\pgfqpoint{1.775193in}{2.606473in}}%
\pgfpathlineto{\pgfqpoint{1.775039in}{2.328173in}}%
\pgfpathlineto{\pgfqpoint{1.775612in}{2.345109in}}%
\pgfpathlineto{\pgfqpoint{1.776163in}{2.324021in}}%
\pgfpathlineto{\pgfqpoint{1.775898in}{2.528348in}}%
\pgfpathlineto{\pgfqpoint{1.776273in}{2.372862in}}%
\pgfpathlineto{\pgfqpoint{1.776295in}{2.622207in}}%
\pgfpathlineto{\pgfqpoint{1.776604in}{2.350244in}}%
\pgfpathlineto{\pgfqpoint{1.777375in}{2.425856in}}%
\pgfpathlineto{\pgfqpoint{1.777463in}{2.302932in}}%
\pgfpathlineto{\pgfqpoint{1.778301in}{2.662636in}}%
\pgfpathlineto{\pgfqpoint{1.778477in}{2.399633in}}%
\pgfpathlineto{\pgfqpoint{1.778565in}{2.690171in}}%
\pgfpathlineto{\pgfqpoint{1.779623in}{2.580468in}}%
\pgfpathlineto{\pgfqpoint{1.780395in}{2.364449in}}%
\pgfpathlineto{\pgfqpoint{1.779800in}{2.639253in}}%
\pgfpathlineto{\pgfqpoint{1.780725in}{2.437876in}}%
\pgfpathlineto{\pgfqpoint{1.781210in}{2.659467in}}%
\pgfpathlineto{\pgfqpoint{1.780902in}{2.309051in}}%
\pgfpathlineto{\pgfqpoint{1.781828in}{2.484314in}}%
\pgfpathlineto{\pgfqpoint{1.782246in}{2.257368in}}%
\pgfpathlineto{\pgfqpoint{1.781894in}{2.582107in}}%
\pgfpathlineto{\pgfqpoint{1.782842in}{2.480708in}}%
\pgfpathlineto{\pgfqpoint{1.782864in}{2.617837in}}%
\pgfpathlineto{\pgfqpoint{1.783525in}{2.365105in}}%
\pgfpathlineto{\pgfqpoint{1.783922in}{2.428042in}}%
\pgfpathlineto{\pgfqpoint{1.784715in}{2.279331in}}%
\pgfpathlineto{\pgfqpoint{1.784274in}{2.666132in}}%
\pgfpathlineto{\pgfqpoint{1.784803in}{2.501578in}}%
\pgfpathlineto{\pgfqpoint{1.784847in}{2.553151in}}%
\pgfpathlineto{\pgfqpoint{1.785487in}{2.307631in}}%
\pgfpathlineto{\pgfqpoint{1.785839in}{2.435253in}}%
\pgfpathlineto{\pgfqpoint{1.786104in}{2.345327in}}%
\pgfpathlineto{\pgfqpoint{1.786611in}{2.627015in}}%
\pgfpathlineto{\pgfqpoint{1.786941in}{2.434051in}}%
\pgfpathlineto{\pgfqpoint{1.787206in}{2.689734in}}%
\pgfpathlineto{\pgfqpoint{1.787007in}{2.381057in}}%
\pgfpathlineto{\pgfqpoint{1.788088in}{2.608768in}}%
\pgfpathlineto{\pgfqpoint{1.788925in}{2.381385in}}%
\pgfpathlineto{\pgfqpoint{1.788617in}{2.619694in}}%
\pgfpathlineto{\pgfqpoint{1.789278in}{2.451425in}}%
\pgfpathlineto{\pgfqpoint{1.789454in}{2.756386in}}%
\pgfpathlineto{\pgfqpoint{1.790358in}{2.367836in}}%
\pgfpathlineto{\pgfqpoint{1.790380in}{2.541569in}}%
\pgfpathlineto{\pgfqpoint{1.790777in}{2.446617in}}%
\pgfpathlineto{\pgfqpoint{1.790865in}{2.699458in}}%
\pgfpathlineto{\pgfqpoint{1.791460in}{2.614559in}}%
\pgfpathlineto{\pgfqpoint{1.792342in}{2.310799in}}%
\pgfpathlineto{\pgfqpoint{1.792628in}{2.375594in}}%
\pgfpathlineto{\pgfqpoint{1.793730in}{2.646137in}}%
\pgfpathlineto{\pgfqpoint{1.793664in}{2.319650in}}%
\pgfpathlineto{\pgfqpoint{1.793752in}{2.456997in}}%
\pgfpathlineto{\pgfqpoint{1.794568in}{2.406079in}}%
\pgfpathlineto{\pgfqpoint{1.793929in}{2.620131in}}%
\pgfpathlineto{\pgfqpoint{1.794722in}{2.439842in}}%
\pgfpathlineto{\pgfqpoint{1.795075in}{2.645262in}}%
\pgfpathlineto{\pgfqpoint{1.795758in}{2.379091in}}%
\pgfpathlineto{\pgfqpoint{1.795846in}{2.534467in}}%
\pgfpathlineto{\pgfqpoint{1.796838in}{2.464427in}}%
\pgfpathlineto{\pgfqpoint{1.796331in}{2.753326in}}%
\pgfpathlineto{\pgfqpoint{1.796904in}{2.545612in}}%
\pgfpathlineto{\pgfqpoint{1.797830in}{2.397994in}}%
\pgfpathlineto{\pgfqpoint{1.798028in}{2.672142in}}%
\pgfpathlineto{\pgfqpoint{1.798602in}{2.760538in}}%
\pgfpathlineto{\pgfqpoint{1.798359in}{2.498737in}}%
\pgfpathlineto{\pgfqpoint{1.799064in}{2.605271in}}%
\pgfpathlineto{\pgfqpoint{1.799968in}{2.504637in}}%
\pgfpathlineto{\pgfqpoint{1.799593in}{2.691045in}}%
\pgfpathlineto{\pgfqpoint{1.800189in}{2.546267in}}%
\pgfpathlineto{\pgfqpoint{1.800541in}{2.694978in}}%
\pgfpathlineto{\pgfqpoint{1.801004in}{2.437985in}}%
\pgfpathlineto{\pgfqpoint{1.801291in}{2.601338in}}%
\pgfpathlineto{\pgfqpoint{1.802415in}{2.417224in}}%
\pgfpathlineto{\pgfqpoint{1.802106in}{2.808615in}}%
\pgfpathlineto{\pgfqpoint{1.802437in}{2.494366in}}%
\pgfpathlineto{\pgfqpoint{1.802812in}{2.671705in}}%
\pgfpathlineto{\pgfqpoint{1.803186in}{2.326752in}}%
\pgfpathlineto{\pgfqpoint{1.803539in}{2.481910in}}%
\pgfpathlineto{\pgfqpoint{1.803759in}{2.394606in}}%
\pgfpathlineto{\pgfqpoint{1.804244in}{2.677933in}}%
\pgfpathlineto{\pgfqpoint{1.804619in}{2.431320in}}%
\pgfpathlineto{\pgfqpoint{1.804906in}{2.691810in}}%
\pgfpathlineto{\pgfqpoint{1.805765in}{2.567356in}}%
\pgfpathlineto{\pgfqpoint{1.806581in}{2.434816in}}%
\pgfpathlineto{\pgfqpoint{1.806757in}{2.722186in}}%
\pgfpathlineto{\pgfqpoint{1.806889in}{2.534685in}}%
\pgfpathlineto{\pgfqpoint{1.807462in}{2.403238in}}%
\pgfpathlineto{\pgfqpoint{1.807859in}{2.586150in}}%
\pgfpathlineto{\pgfqpoint{1.807969in}{2.546267in}}%
\pgfpathlineto{\pgfqpoint{1.808388in}{2.698038in}}%
\pgfpathlineto{\pgfqpoint{1.808895in}{2.389689in}}%
\pgfpathlineto{\pgfqpoint{1.809049in}{2.472076in}}%
\pgfpathlineto{\pgfqpoint{1.809072in}{2.473606in}}%
\pgfpathlineto{\pgfqpoint{1.809865in}{2.261739in}}%
\pgfpathlineto{\pgfqpoint{1.809292in}{2.505621in}}%
\pgfpathlineto{\pgfqpoint{1.810174in}{2.420502in}}%
\pgfpathlineto{\pgfqpoint{1.810526in}{2.485734in}}%
\pgfpathlineto{\pgfqpoint{1.810350in}{2.271245in}}%
\pgfpathlineto{\pgfqpoint{1.810636in}{2.311237in}}%
\pgfpathlineto{\pgfqpoint{1.810659in}{2.290367in}}%
\pgfpathlineto{\pgfqpoint{1.811055in}{2.627780in}}%
\pgfpathlineto{\pgfqpoint{1.811430in}{2.450551in}}%
\pgfpathlineto{\pgfqpoint{1.812488in}{2.613466in}}%
\pgfpathlineto{\pgfqpoint{1.812224in}{2.308614in}}%
\pgfpathlineto{\pgfqpoint{1.812532in}{2.525398in}}%
\pgfpathlineto{\pgfqpoint{1.812753in}{2.262613in}}%
\pgfpathlineto{\pgfqpoint{1.812841in}{2.649524in}}%
\pgfpathlineto{\pgfqpoint{1.813634in}{2.394825in}}%
\pgfpathlineto{\pgfqpoint{1.814648in}{2.657609in}}%
\pgfpathlineto{\pgfqpoint{1.813766in}{2.317683in}}%
\pgfpathlineto{\pgfqpoint{1.814758in}{2.550420in}}%
\pgfpathlineto{\pgfqpoint{1.815750in}{2.343361in}}%
\pgfpathlineto{\pgfqpoint{1.815353in}{2.588226in}}%
\pgfpathlineto{\pgfqpoint{1.815860in}{2.518295in}}%
\pgfpathlineto{\pgfqpoint{1.816169in}{2.660778in}}%
\pgfpathlineto{\pgfqpoint{1.816742in}{2.382369in}}%
\pgfpathlineto{\pgfqpoint{1.816940in}{2.497863in}}%
\pgfpathlineto{\pgfqpoint{1.817734in}{2.408483in}}%
\pgfpathlineto{\pgfqpoint{1.817558in}{2.684489in}}%
\pgfpathlineto{\pgfqpoint{1.817999in}{2.559379in}}%
\pgfpathlineto{\pgfqpoint{1.818021in}{2.649961in}}%
\pgfpathlineto{\pgfqpoint{1.818395in}{2.412635in}}%
\pgfpathlineto{\pgfqpoint{1.819079in}{2.482565in}}%
\pgfpathlineto{\pgfqpoint{1.819630in}{2.384882in}}%
\pgfpathlineto{\pgfqpoint{1.819321in}{2.709729in}}%
\pgfpathlineto{\pgfqpoint{1.820137in}{2.544847in}}%
\pgfpathlineto{\pgfqpoint{1.820313in}{2.659248in}}%
\pgfpathlineto{\pgfqpoint{1.820644in}{2.389799in}}%
\pgfpathlineto{\pgfqpoint{1.821239in}{2.566482in}}%
\pgfpathlineto{\pgfqpoint{1.821657in}{2.376687in}}%
\pgfpathlineto{\pgfqpoint{1.822363in}{2.472185in}}%
\pgfpathlineto{\pgfqpoint{1.823002in}{2.739450in}}%
\pgfpathlineto{\pgfqpoint{1.822539in}{2.345109in}}%
\pgfpathlineto{\pgfqpoint{1.823509in}{2.649196in}}%
\pgfpathlineto{\pgfqpoint{1.824192in}{2.379746in}}%
\pgfpathlineto{\pgfqpoint{1.823884in}{2.736062in}}%
\pgfpathlineto{\pgfqpoint{1.824633in}{2.527583in}}%
\pgfpathlineto{\pgfqpoint{1.825515in}{2.650179in}}%
\pgfpathlineto{\pgfqpoint{1.825118in}{2.325223in}}%
\pgfpathlineto{\pgfqpoint{1.825735in}{2.508899in}}%
\pgfpathlineto{\pgfqpoint{1.826264in}{2.706670in}}%
\pgfpathlineto{\pgfqpoint{1.826397in}{2.438203in}}%
\pgfpathlineto{\pgfqpoint{1.826815in}{2.543536in}}%
\pgfpathlineto{\pgfqpoint{1.827741in}{2.361280in}}%
\pgfpathlineto{\pgfqpoint{1.826970in}{2.672470in}}%
\pgfpathlineto{\pgfqpoint{1.827939in}{2.465411in}}%
\pgfpathlineto{\pgfqpoint{1.828490in}{2.668208in}}%
\pgfpathlineto{\pgfqpoint{1.828226in}{2.340083in}}%
\pgfpathlineto{\pgfqpoint{1.828997in}{2.440717in}}%
\pgfpathlineto{\pgfqpoint{1.829549in}{2.345000in}}%
\pgfpathlineto{\pgfqpoint{1.829306in}{2.583746in}}%
\pgfpathlineto{\pgfqpoint{1.830011in}{2.538400in}}%
\pgfpathlineto{\pgfqpoint{1.830937in}{2.695306in}}%
\pgfpathlineto{\pgfqpoint{1.830232in}{2.347404in}}%
\pgfpathlineto{\pgfqpoint{1.831136in}{2.581342in}}%
\pgfpathlineto{\pgfqpoint{1.831356in}{2.643077in}}%
\pgfpathlineto{\pgfqpoint{1.831819in}{2.386084in}}%
\pgfpathlineto{\pgfqpoint{1.831885in}{2.444650in}}%
\pgfpathlineto{\pgfqpoint{1.832767in}{2.350463in}}%
\pgfpathlineto{\pgfqpoint{1.832194in}{2.630839in}}%
\pgfpathlineto{\pgfqpoint{1.832987in}{2.464427in}}%
\pgfpathlineto{\pgfqpoint{1.833031in}{2.352867in}}%
\pgfpathlineto{\pgfqpoint{1.833274in}{2.618055in}}%
\pgfpathlineto{\pgfqpoint{1.834067in}{2.397557in}}%
\pgfpathlineto{\pgfqpoint{1.835147in}{2.601447in}}%
\pgfpathlineto{\pgfqpoint{1.834376in}{2.335056in}}%
\pgfpathlineto{\pgfqpoint{1.835191in}{2.575988in}}%
\pgfpathlineto{\pgfqpoint{1.836337in}{2.202408in}}%
\pgfpathlineto{\pgfqpoint{1.836911in}{2.612592in}}%
\pgfpathlineto{\pgfqpoint{1.837462in}{2.412963in}}%
\pgfpathlineto{\pgfqpoint{1.838542in}{2.621224in}}%
\pgfpathlineto{\pgfqpoint{1.838101in}{2.363138in}}%
\pgfpathlineto{\pgfqpoint{1.838652in}{2.497426in}}%
\pgfpathlineto{\pgfqpoint{1.838740in}{2.381822in}}%
\pgfpathlineto{\pgfqpoint{1.839467in}{2.707763in}}%
\pgfpathlineto{\pgfqpoint{1.839710in}{2.542443in}}%
\pgfpathlineto{\pgfqpoint{1.839842in}{2.708200in}}%
\pgfpathlineto{\pgfqpoint{1.840481in}{2.404222in}}%
\pgfpathlineto{\pgfqpoint{1.840834in}{2.585166in}}%
\pgfpathlineto{\pgfqpoint{1.840856in}{2.584511in}}%
\pgfpathlineto{\pgfqpoint{1.841738in}{2.378107in}}%
\pgfpathlineto{\pgfqpoint{1.841187in}{2.709402in}}%
\pgfpathlineto{\pgfqpoint{1.841958in}{2.632369in}}%
\pgfpathlineto{\pgfqpoint{1.842994in}{2.414930in}}%
\pgfpathlineto{\pgfqpoint{1.842068in}{2.687985in}}%
\pgfpathlineto{\pgfqpoint{1.843060in}{2.528020in}}%
\pgfpathlineto{\pgfqpoint{1.843655in}{2.614012in}}%
\pgfpathlineto{\pgfqpoint{1.843854in}{2.369912in}}%
\pgfpathlineto{\pgfqpoint{1.844140in}{2.477867in}}%
\pgfpathlineto{\pgfqpoint{1.844625in}{2.408265in}}%
\pgfpathlineto{\pgfqpoint{1.844228in}{2.595109in}}%
\pgfpathlineto{\pgfqpoint{1.844669in}{2.519060in}}%
\pgfpathlineto{\pgfqpoint{1.845551in}{2.686674in}}%
\pgfpathlineto{\pgfqpoint{1.845705in}{2.354834in}}%
\pgfpathlineto{\pgfqpoint{1.845771in}{2.509554in}}%
\pgfpathlineto{\pgfqpoint{1.846741in}{2.727758in}}%
\pgfpathlineto{\pgfqpoint{1.846080in}{2.419628in}}%
\pgfpathlineto{\pgfqpoint{1.846785in}{2.570962in}}%
\pgfpathlineto{\pgfqpoint{1.846807in}{2.388050in}}%
\pgfpathlineto{\pgfqpoint{1.847491in}{2.724371in}}%
\pgfpathlineto{\pgfqpoint{1.847887in}{2.607566in}}%
\pgfpathlineto{\pgfqpoint{1.848284in}{2.425419in}}%
\pgfpathlineto{\pgfqpoint{1.847976in}{2.695743in}}%
\pgfpathlineto{\pgfqpoint{1.848968in}{2.603086in}}%
\pgfpathlineto{\pgfqpoint{1.849519in}{2.774961in}}%
\pgfpathlineto{\pgfqpoint{1.849166in}{2.520262in}}%
\pgfpathlineto{\pgfqpoint{1.850026in}{2.526927in}}%
\pgfpathlineto{\pgfqpoint{1.851017in}{2.386630in}}%
\pgfpathlineto{\pgfqpoint{1.850466in}{2.698475in}}%
\pgfpathlineto{\pgfqpoint{1.851128in}{2.517421in}}%
\pgfpathlineto{\pgfqpoint{1.851436in}{2.709074in}}%
\pgfpathlineto{\pgfqpoint{1.852009in}{2.427933in}}%
\pgfpathlineto{\pgfqpoint{1.852406in}{2.677714in}}%
\pgfpathlineto{\pgfqpoint{1.853486in}{2.445852in}}%
\pgfpathlineto{\pgfqpoint{1.852516in}{2.732457in}}%
\pgfpathlineto{\pgfqpoint{1.853530in}{2.573693in}}%
\pgfpathlineto{\pgfqpoint{1.853596in}{2.720110in}}%
\pgfpathlineto{\pgfqpoint{1.854169in}{2.335275in}}%
\pgfpathlineto{\pgfqpoint{1.854632in}{2.580249in}}%
\pgfpathlineto{\pgfqpoint{1.855294in}{2.359860in}}%
\pgfpathlineto{\pgfqpoint{1.854963in}{2.734314in}}%
\pgfpathlineto{\pgfqpoint{1.855778in}{2.391765in}}%
\pgfpathlineto{\pgfqpoint{1.856462in}{2.622426in}}%
\pgfpathlineto{\pgfqpoint{1.855867in}{2.344781in}}%
\pgfpathlineto{\pgfqpoint{1.856881in}{2.520371in}}%
\pgfpathlineto{\pgfqpoint{1.856947in}{2.260319in}}%
\pgfpathlineto{\pgfqpoint{1.857828in}{2.667443in}}%
\pgfpathlineto{\pgfqpoint{1.857983in}{2.487373in}}%
\pgfpathlineto{\pgfqpoint{1.858534in}{2.686783in}}%
\pgfpathlineto{\pgfqpoint{1.858049in}{2.464427in}}%
\pgfpathlineto{\pgfqpoint{1.859085in}{2.535341in}}%
\pgfpathlineto{\pgfqpoint{1.859349in}{2.362591in}}%
\pgfpathlineto{\pgfqpoint{1.859261in}{2.720984in}}%
\pgfpathlineto{\pgfqpoint{1.860187in}{2.544956in}}%
\pgfpathlineto{\pgfqpoint{1.861223in}{2.361499in}}%
\pgfpathlineto{\pgfqpoint{1.860363in}{2.714428in}}%
\pgfpathlineto{\pgfqpoint{1.861333in}{2.410887in}}%
\pgfpathlineto{\pgfqpoint{1.861928in}{2.710603in}}%
\pgfpathlineto{\pgfqpoint{1.862215in}{2.387067in}}%
\pgfpathlineto{\pgfqpoint{1.862479in}{2.561455in}}%
\pgfpathlineto{\pgfqpoint{1.862678in}{2.437220in}}%
\pgfpathlineto{\pgfqpoint{1.863251in}{2.683068in}}%
\pgfpathlineto{\pgfqpoint{1.863537in}{2.624393in}}%
\pgfpathlineto{\pgfqpoint{1.863890in}{2.697710in}}%
\pgfpathlineto{\pgfqpoint{1.863692in}{2.442465in}}%
\pgfpathlineto{\pgfqpoint{1.864463in}{2.515345in}}%
\pgfpathlineto{\pgfqpoint{1.865389in}{2.418863in}}%
\pgfpathlineto{\pgfqpoint{1.864728in}{2.739231in}}%
\pgfpathlineto{\pgfqpoint{1.865499in}{2.519607in}}%
\pgfpathlineto{\pgfqpoint{1.865764in}{2.791132in}}%
\pgfpathlineto{\pgfqpoint{1.866182in}{2.430118in}}%
\pgfpathlineto{\pgfqpoint{1.866623in}{2.709729in}}%
\pgfpathlineto{\pgfqpoint{1.867725in}{2.447710in}}%
\pgfpathlineto{\pgfqpoint{1.866667in}{2.802278in}}%
\pgfpathlineto{\pgfqpoint{1.867769in}{2.505839in}}%
\pgfpathlineto{\pgfqpoint{1.868453in}{2.733768in}}%
\pgfpathlineto{\pgfqpoint{1.868254in}{2.378107in}}%
\pgfpathlineto{\pgfqpoint{1.868827in}{2.513051in}}%
\pgfpathlineto{\pgfqpoint{1.868916in}{2.303697in}}%
\pgfpathlineto{\pgfqpoint{1.869819in}{2.688641in}}%
\pgfpathlineto{\pgfqpoint{1.869929in}{2.566263in}}%
\pgfpathlineto{\pgfqpoint{1.870855in}{2.784467in}}%
\pgfpathlineto{\pgfqpoint{1.870194in}{2.370240in}}%
\pgfpathlineto{\pgfqpoint{1.871120in}{2.716941in}}%
\pgfpathlineto{\pgfqpoint{1.871891in}{2.465411in}}%
\pgfpathlineto{\pgfqpoint{1.871428in}{2.752780in}}%
\pgfpathlineto{\pgfqpoint{1.872266in}{2.659358in}}%
\pgfpathlineto{\pgfqpoint{1.872707in}{2.805337in}}%
\pgfpathlineto{\pgfqpoint{1.872398in}{2.419191in}}%
\pgfpathlineto{\pgfqpoint{1.873302in}{2.655752in}}%
\pgfpathlineto{\pgfqpoint{1.873919in}{2.484204in}}%
\pgfpathlineto{\pgfqpoint{1.874272in}{2.741089in}}%
\pgfpathlineto{\pgfqpoint{1.874404in}{2.600791in}}%
\pgfpathlineto{\pgfqpoint{1.874492in}{2.789166in}}%
\pgfpathlineto{\pgfqpoint{1.874691in}{2.501687in}}%
\pgfpathlineto{\pgfqpoint{1.875462in}{2.648540in}}%
\pgfpathlineto{\pgfqpoint{1.875837in}{2.468033in}}%
\pgfpathlineto{\pgfqpoint{1.875528in}{2.695525in}}%
\pgfpathlineto{\pgfqpoint{1.876586in}{2.579703in}}%
\pgfpathlineto{\pgfqpoint{1.877203in}{2.786325in}}%
\pgfpathlineto{\pgfqpoint{1.876674in}{2.492509in}}%
\pgfpathlineto{\pgfqpoint{1.877710in}{2.707763in}}%
\pgfpathlineto{\pgfqpoint{1.878349in}{2.470874in}}%
\pgfpathlineto{\pgfqpoint{1.878834in}{2.575114in}}%
\pgfpathlineto{\pgfqpoint{1.878923in}{2.755949in}}%
\pgfpathlineto{\pgfqpoint{1.879870in}{2.353195in}}%
\pgfpathlineto{\pgfqpoint{1.879937in}{2.530315in}}%
\pgfpathlineto{\pgfqpoint{1.880003in}{2.409576in}}%
\pgfpathlineto{\pgfqpoint{1.880620in}{2.715302in}}%
\pgfpathlineto{\pgfqpoint{1.880995in}{2.601338in}}%
\pgfpathlineto{\pgfqpoint{1.881435in}{2.394169in}}%
\pgfpathlineto{\pgfqpoint{1.882119in}{2.720328in}}%
\pgfpathlineto{\pgfqpoint{1.883199in}{2.363356in}}%
\pgfpathlineto{\pgfqpoint{1.883287in}{2.381057in}}%
\pgfpathlineto{\pgfqpoint{1.883463in}{2.724480in}}%
\pgfpathlineto{\pgfqpoint{1.884389in}{2.459729in}}%
\pgfpathlineto{\pgfqpoint{1.885293in}{2.355271in}}%
\pgfpathlineto{\pgfqpoint{1.884940in}{2.589646in}}%
\pgfpathlineto{\pgfqpoint{1.885425in}{2.413728in}}%
\pgfpathlineto{\pgfqpoint{1.886086in}{2.682741in}}%
\pgfpathlineto{\pgfqpoint{1.885778in}{2.395371in}}%
\pgfpathlineto{\pgfqpoint{1.886527in}{2.624611in}}%
\pgfpathlineto{\pgfqpoint{1.886747in}{2.404659in}}%
\pgfpathlineto{\pgfqpoint{1.887232in}{2.631276in}}%
\pgfpathlineto{\pgfqpoint{1.887629in}{2.597513in}}%
\pgfpathlineto{\pgfqpoint{1.888555in}{2.425966in}}%
\pgfpathlineto{\pgfqpoint{1.887761in}{2.693995in}}%
\pgfpathlineto{\pgfqpoint{1.888775in}{2.432085in}}%
\pgfpathlineto{\pgfqpoint{1.888864in}{2.728523in}}%
\pgfpathlineto{\pgfqpoint{1.889877in}{2.543317in}}%
\pgfpathlineto{\pgfqpoint{1.890032in}{2.399305in}}%
\pgfpathlineto{\pgfqpoint{1.890010in}{2.680337in}}%
\pgfpathlineto{\pgfqpoint{1.890891in}{2.506167in}}%
\pgfpathlineto{\pgfqpoint{1.891663in}{2.698038in}}%
\pgfpathlineto{\pgfqpoint{1.891068in}{2.330795in}}%
\pgfpathlineto{\pgfqpoint{1.891993in}{2.546595in}}%
\pgfpathlineto{\pgfqpoint{1.892258in}{2.460494in}}%
\pgfpathlineto{\pgfqpoint{1.892809in}{2.756386in}}%
\pgfpathlineto{\pgfqpoint{1.892875in}{2.687985in}}%
\pgfpathlineto{\pgfqpoint{1.892941in}{2.752671in}}%
\pgfpathlineto{\pgfqpoint{1.893404in}{2.371770in}}%
\pgfpathlineto{\pgfqpoint{1.893603in}{2.375485in}}%
\pgfpathlineto{\pgfqpoint{1.893625in}{2.332871in}}%
\pgfpathlineto{\pgfqpoint{1.893691in}{2.693558in}}%
\pgfpathlineto{\pgfqpoint{1.894616in}{2.510210in}}%
\pgfpathlineto{\pgfqpoint{1.895278in}{2.656735in}}%
\pgfpathlineto{\pgfqpoint{1.894815in}{2.353304in}}%
\pgfpathlineto{\pgfqpoint{1.895719in}{2.518842in}}%
\pgfpathlineto{\pgfqpoint{1.895895in}{2.636303in}}%
\pgfpathlineto{\pgfqpoint{1.896071in}{2.316372in}}%
\pgfpathlineto{\pgfqpoint{1.896777in}{2.447491in}}%
\pgfpathlineto{\pgfqpoint{1.896799in}{2.433396in}}%
\pgfpathlineto{\pgfqpoint{1.897107in}{2.644170in}}%
\pgfpathlineto{\pgfqpoint{1.897658in}{2.452080in}}%
\pgfpathlineto{\pgfqpoint{1.898518in}{2.787308in}}%
\pgfpathlineto{\pgfqpoint{1.898628in}{2.400616in}}%
\pgfpathlineto{\pgfqpoint{1.898782in}{2.600791in}}%
\pgfpathlineto{\pgfqpoint{1.899620in}{2.451425in}}%
\pgfpathlineto{\pgfqpoint{1.899245in}{2.739450in}}%
\pgfpathlineto{\pgfqpoint{1.899885in}{2.515017in}}%
\pgfpathlineto{\pgfqpoint{1.899929in}{2.655315in}}%
\pgfpathlineto{\pgfqpoint{1.900502in}{2.430883in}}%
\pgfpathlineto{\pgfqpoint{1.900987in}{2.588881in}}%
\pgfpathlineto{\pgfqpoint{1.901890in}{2.416678in}}%
\pgfpathlineto{\pgfqpoint{1.901295in}{2.727540in}}%
\pgfpathlineto{\pgfqpoint{1.902089in}{2.519497in}}%
\pgfpathlineto{\pgfqpoint{1.902574in}{2.665695in}}%
\pgfpathlineto{\pgfqpoint{1.903081in}{2.346092in}}%
\pgfpathlineto{\pgfqpoint{1.903169in}{2.469672in}}%
\pgfpathlineto{\pgfqpoint{1.903191in}{2.421486in}}%
\pgfpathlineto{\pgfqpoint{1.903676in}{2.756714in}}%
\pgfpathlineto{\pgfqpoint{1.904183in}{2.678042in}}%
\pgfpathlineto{\pgfqpoint{1.904205in}{2.736390in}}%
\pgfpathlineto{\pgfqpoint{1.904690in}{2.455249in}}%
\pgfpathlineto{\pgfqpoint{1.905263in}{2.603741in}}%
\pgfpathlineto{\pgfqpoint{1.905417in}{2.621224in}}%
\pgfpathlineto{\pgfqpoint{1.905527in}{2.417552in}}%
\pgfpathlineto{\pgfqpoint{1.905549in}{2.513051in}}%
\pgfpathlineto{\pgfqpoint{1.905748in}{2.647011in}}%
\pgfpathlineto{\pgfqpoint{1.906651in}{2.348496in}}%
\pgfpathlineto{\pgfqpoint{1.907643in}{2.697710in}}%
\pgfpathlineto{\pgfqpoint{1.907776in}{2.513816in}}%
\pgfpathlineto{\pgfqpoint{1.908767in}{2.373190in}}%
\pgfpathlineto{\pgfqpoint{1.908194in}{2.770590in}}%
\pgfpathlineto{\pgfqpoint{1.908900in}{2.427386in}}%
\pgfpathlineto{\pgfqpoint{1.909407in}{2.612701in}}%
\pgfpathlineto{\pgfqpoint{1.909318in}{2.347731in}}%
\pgfpathlineto{\pgfqpoint{1.910024in}{2.559489in}}%
\pgfpathlineto{\pgfqpoint{1.910178in}{2.358986in}}%
\pgfpathlineto{\pgfqpoint{1.910972in}{2.721749in}}%
\pgfpathlineto{\pgfqpoint{1.911104in}{2.613685in}}%
\pgfpathlineto{\pgfqpoint{1.911941in}{2.751906in}}%
\pgfpathlineto{\pgfqpoint{1.911479in}{2.339646in}}%
\pgfpathlineto{\pgfqpoint{1.912184in}{2.620896in}}%
\pgfpathlineto{\pgfqpoint{1.913022in}{2.337023in}}%
\pgfpathlineto{\pgfqpoint{1.912559in}{2.649305in}}%
\pgfpathlineto{\pgfqpoint{1.913352in}{2.436346in}}%
\pgfpathlineto{\pgfqpoint{1.914432in}{2.756167in}}%
\pgfpathlineto{\pgfqpoint{1.914190in}{2.364449in}}%
\pgfpathlineto{\pgfqpoint{1.914498in}{2.612155in}}%
\pgfpathlineto{\pgfqpoint{1.914653in}{2.692028in}}%
\pgfpathlineto{\pgfqpoint{1.914851in}{2.472622in}}%
\pgfpathlineto{\pgfqpoint{1.915578in}{2.590629in}}%
\pgfpathlineto{\pgfqpoint{1.915931in}{2.477321in}}%
\pgfpathlineto{\pgfqpoint{1.915645in}{2.730271in}}%
\pgfpathlineto{\pgfqpoint{1.916725in}{2.478304in}}%
\pgfpathlineto{\pgfqpoint{1.917430in}{2.792771in}}%
\pgfpathlineto{\pgfqpoint{1.917849in}{2.704157in}}%
\pgfpathlineto{\pgfqpoint{1.918444in}{2.439624in}}%
\pgfpathlineto{\pgfqpoint{1.917893in}{2.725682in}}%
\pgfpathlineto{\pgfqpoint{1.918995in}{2.469672in}}%
\pgfpathlineto{\pgfqpoint{1.919171in}{2.753873in}}%
\pgfpathlineto{\pgfqpoint{1.919833in}{2.451315in}}%
\pgfpathlineto{\pgfqpoint{1.920097in}{2.607675in}}%
\pgfpathlineto{\pgfqpoint{1.921045in}{2.478741in}}%
\pgfpathlineto{\pgfqpoint{1.921177in}{2.778021in}}%
\pgfpathlineto{\pgfqpoint{1.921221in}{2.535013in}}%
\pgfpathlineto{\pgfqpoint{1.921794in}{2.463007in}}%
\pgfpathlineto{\pgfqpoint{1.921706in}{2.790695in}}%
\pgfpathlineto{\pgfqpoint{1.922169in}{2.617290in}}%
\pgfpathlineto{\pgfqpoint{1.922962in}{2.735735in}}%
\pgfpathlineto{\pgfqpoint{1.922544in}{2.438313in}}%
\pgfpathlineto{\pgfqpoint{1.923249in}{2.608003in}}%
\pgfpathlineto{\pgfqpoint{1.923293in}{2.499829in}}%
\pgfpathlineto{\pgfqpoint{1.923491in}{2.805446in}}%
\pgfpathlineto{\pgfqpoint{1.924307in}{2.608331in}}%
\pgfpathlineto{\pgfqpoint{1.924726in}{2.801076in}}%
\pgfpathlineto{\pgfqpoint{1.924858in}{2.465520in}}%
\pgfpathlineto{\pgfqpoint{1.925409in}{2.649305in}}%
\pgfpathlineto{\pgfqpoint{1.925652in}{2.427933in}}%
\pgfpathlineto{\pgfqpoint{1.926467in}{2.727540in}}%
\pgfpathlineto{\pgfqpoint{1.926511in}{2.627343in}}%
\pgfpathlineto{\pgfqpoint{1.926533in}{2.626141in}}%
\pgfpathlineto{\pgfqpoint{1.926577in}{2.672142in}}%
\pgfpathlineto{\pgfqpoint{1.926599in}{2.739340in}}%
\pgfpathlineto{\pgfqpoint{1.927613in}{2.488903in}}%
\pgfpathlineto{\pgfqpoint{1.927657in}{2.591941in}}%
\pgfpathlineto{\pgfqpoint{1.928010in}{2.793646in}}%
\pgfpathlineto{\pgfqpoint{1.928341in}{2.485734in}}%
\pgfpathlineto{\pgfqpoint{1.928804in}{2.677387in}}%
\pgfpathlineto{\pgfqpoint{1.929641in}{2.467487in}}%
\pgfpathlineto{\pgfqpoint{1.929200in}{2.763816in}}%
\pgfpathlineto{\pgfqpoint{1.929928in}{2.562002in}}%
\pgfpathlineto{\pgfqpoint{1.929950in}{2.722186in}}%
\pgfpathlineto{\pgfqpoint{1.930457in}{2.383571in}}%
\pgfpathlineto{\pgfqpoint{1.931030in}{2.584948in}}%
\pgfpathlineto{\pgfqpoint{1.931250in}{2.392421in}}%
\pgfpathlineto{\pgfqpoint{1.931713in}{2.682304in}}%
\pgfpathlineto{\pgfqpoint{1.932176in}{2.525725in}}%
\pgfpathlineto{\pgfqpoint{1.932661in}{2.412307in}}%
\pgfpathlineto{\pgfqpoint{1.932507in}{2.736499in}}%
\pgfpathlineto{\pgfqpoint{1.933212in}{2.535669in}}%
\pgfpathlineto{\pgfqpoint{1.934314in}{2.762395in}}%
\pgfpathlineto{\pgfqpoint{1.933388in}{2.507915in}}%
\pgfpathlineto{\pgfqpoint{1.934358in}{2.682850in}}%
\pgfpathlineto{\pgfqpoint{1.935284in}{2.514799in}}%
\pgfpathlineto{\pgfqpoint{1.934512in}{2.803042in}}%
\pgfpathlineto{\pgfqpoint{1.935526in}{2.539821in}}%
\pgfpathlineto{\pgfqpoint{1.936099in}{2.727867in}}%
\pgfpathlineto{\pgfqpoint{1.936452in}{2.357456in}}%
\pgfpathlineto{\pgfqpoint{1.936629in}{2.628326in}}%
\pgfpathlineto{\pgfqpoint{1.936761in}{2.483221in}}%
\pgfpathlineto{\pgfqpoint{1.937047in}{2.738794in}}%
\pgfpathlineto{\pgfqpoint{1.937731in}{2.609532in}}%
\pgfpathlineto{\pgfqpoint{1.938348in}{2.862483in}}%
\pgfpathlineto{\pgfqpoint{1.938216in}{2.552496in}}%
\pgfpathlineto{\pgfqpoint{1.938855in}{2.698256in}}%
\pgfpathlineto{\pgfqpoint{1.939670in}{2.735298in}}%
\pgfpathlineto{\pgfqpoint{1.939009in}{2.496770in}}%
\pgfpathlineto{\pgfqpoint{1.939891in}{2.710385in}}%
\pgfpathlineto{\pgfqpoint{1.940155in}{2.733659in}}%
\pgfpathlineto{\pgfqpoint{1.941059in}{2.464427in}}%
\pgfpathlineto{\pgfqpoint{1.941941in}{2.741307in}}%
\pgfpathlineto{\pgfqpoint{1.941103in}{2.410122in}}%
\pgfpathlineto{\pgfqpoint{1.942161in}{2.599371in}}%
\pgfpathlineto{\pgfqpoint{1.943197in}{2.495350in}}%
\pgfpathlineto{\pgfqpoint{1.942624in}{2.799218in}}%
\pgfpathlineto{\pgfqpoint{1.943263in}{2.617618in}}%
\pgfpathlineto{\pgfqpoint{1.943528in}{2.777911in}}%
\pgfpathlineto{\pgfqpoint{1.943748in}{2.496442in}}%
\pgfpathlineto{\pgfqpoint{1.944387in}{2.660450in}}%
\pgfpathlineto{\pgfqpoint{1.944475in}{2.555664in}}%
\pgfpathlineto{\pgfqpoint{1.945181in}{2.777365in}}%
\pgfpathlineto{\pgfqpoint{1.945313in}{2.627343in}}%
\pgfpathlineto{\pgfqpoint{1.945357in}{2.823475in}}%
\pgfpathlineto{\pgfqpoint{1.945864in}{2.468798in}}%
\pgfpathlineto{\pgfqpoint{1.946393in}{2.616198in}}%
\pgfpathlineto{\pgfqpoint{1.946415in}{2.456669in}}%
\pgfpathlineto{\pgfqpoint{1.947385in}{2.707981in}}%
\pgfpathlineto{\pgfqpoint{1.947473in}{2.627780in}}%
\pgfpathlineto{\pgfqpoint{1.948267in}{2.740761in}}%
\pgfpathlineto{\pgfqpoint{1.948090in}{2.497753in}}%
\pgfpathlineto{\pgfqpoint{1.948597in}{2.720874in}}%
\pgfpathlineto{\pgfqpoint{1.948796in}{2.485625in}}%
\pgfpathlineto{\pgfqpoint{1.949501in}{2.782828in}}%
\pgfpathlineto{\pgfqpoint{1.949766in}{2.643077in}}%
\pgfpathlineto{\pgfqpoint{1.950515in}{2.834620in}}%
\pgfpathlineto{\pgfqpoint{1.950691in}{2.552277in}}%
\pgfpathlineto{\pgfqpoint{1.950868in}{2.628873in}}%
\pgfpathlineto{\pgfqpoint{1.951793in}{2.544629in}}%
\pgfpathlineto{\pgfqpoint{1.951220in}{2.792990in}}%
\pgfpathlineto{\pgfqpoint{1.951926in}{2.649196in}}%
\pgfpathlineto{\pgfqpoint{1.952829in}{2.829703in}}%
\pgfpathlineto{\pgfqpoint{1.952477in}{2.476010in}}%
\pgfpathlineto{\pgfqpoint{1.953028in}{2.635538in}}%
\pgfpathlineto{\pgfqpoint{1.953380in}{2.581888in}}%
\pgfpathlineto{\pgfqpoint{1.954196in}{2.828829in}}%
\pgfpathlineto{\pgfqpoint{1.955210in}{2.590848in}}%
\pgfpathlineto{\pgfqpoint{1.954769in}{2.901710in}}%
\pgfpathlineto{\pgfqpoint{1.955320in}{2.729397in}}%
\pgfpathlineto{\pgfqpoint{1.955607in}{2.528457in}}%
\pgfpathlineto{\pgfqpoint{1.956025in}{2.824896in}}%
\pgfpathlineto{\pgfqpoint{1.956444in}{2.640673in}}%
\pgfpathlineto{\pgfqpoint{1.957128in}{2.814625in}}%
\pgfpathlineto{\pgfqpoint{1.956510in}{2.575441in}}%
\pgfpathlineto{\pgfqpoint{1.957546in}{2.662963in}}%
\pgfpathlineto{\pgfqpoint{1.958450in}{2.511849in}}%
\pgfpathlineto{\pgfqpoint{1.958031in}{2.815171in}}%
\pgfpathlineto{\pgfqpoint{1.958516in}{2.662308in}}%
\pgfpathlineto{\pgfqpoint{1.958913in}{2.829703in}}%
\pgfpathlineto{\pgfqpoint{1.958781in}{2.564843in}}%
\pgfpathlineto{\pgfqpoint{1.959596in}{2.648213in}}%
\pgfpathlineto{\pgfqpoint{1.959971in}{2.513488in}}%
\pgfpathlineto{\pgfqpoint{1.960566in}{2.782937in}}%
\pgfpathlineto{\pgfqpoint{1.960720in}{2.570634in}}%
\pgfpathlineto{\pgfqpoint{1.961734in}{2.409904in}}%
\pgfpathlineto{\pgfqpoint{1.961867in}{2.820088in}}%
\pgfpathlineto{\pgfqpoint{1.962263in}{2.552823in}}%
\pgfpathlineto{\pgfqpoint{1.962792in}{2.855381in}}%
\pgfpathlineto{\pgfqpoint{1.962969in}{2.730490in}}%
\pgfpathlineto{\pgfqpoint{1.963542in}{2.787854in}}%
\pgfpathlineto{\pgfqpoint{1.963145in}{2.609423in}}%
\pgfpathlineto{\pgfqpoint{1.963806in}{2.636303in}}%
\pgfpathlineto{\pgfqpoint{1.963872in}{2.484314in}}%
\pgfpathlineto{\pgfqpoint{1.964027in}{2.876797in}}%
\pgfpathlineto{\pgfqpoint{1.964864in}{2.678151in}}%
\pgfpathlineto{\pgfqpoint{1.965944in}{2.849699in}}%
\pgfpathlineto{\pgfqpoint{1.965019in}{2.592706in}}%
\pgfpathlineto{\pgfqpoint{1.965966in}{2.801950in}}%
\pgfpathlineto{\pgfqpoint{1.966165in}{2.640018in}}%
\pgfpathlineto{\pgfqpoint{1.966275in}{2.891220in}}%
\pgfpathlineto{\pgfqpoint{1.967068in}{2.685909in}}%
\pgfpathlineto{\pgfqpoint{1.967752in}{2.867509in}}%
\pgfpathlineto{\pgfqpoint{1.967906in}{2.586150in}}%
\pgfpathlineto{\pgfqpoint{1.968171in}{2.674218in}}%
\pgfpathlineto{\pgfqpoint{1.969008in}{2.577736in}}%
\pgfpathlineto{\pgfqpoint{1.968810in}{2.872208in}}%
\pgfpathlineto{\pgfqpoint{1.969118in}{2.653020in}}%
\pgfpathlineto{\pgfqpoint{1.969736in}{2.872645in}}%
\pgfpathlineto{\pgfqpoint{1.970243in}{2.733221in}}%
\pgfpathlineto{\pgfqpoint{1.970507in}{2.838663in}}%
\pgfpathlineto{\pgfqpoint{1.970397in}{2.532500in}}%
\pgfpathlineto{\pgfqpoint{1.971301in}{2.727430in}}%
\pgfpathlineto{\pgfqpoint{1.971433in}{2.589646in}}%
\pgfpathlineto{\pgfqpoint{1.971631in}{2.855381in}}%
\pgfpathlineto{\pgfqpoint{1.972403in}{2.715520in}}%
\pgfpathlineto{\pgfqpoint{1.972667in}{2.816701in}}%
\pgfpathlineto{\pgfqpoint{1.973373in}{2.562220in}}%
\pgfpathlineto{\pgfqpoint{1.973527in}{2.757260in}}%
\pgfpathlineto{\pgfqpoint{1.974210in}{2.504528in}}%
\pgfpathlineto{\pgfqpoint{1.973615in}{2.826098in}}%
\pgfpathlineto{\pgfqpoint{1.974629in}{2.694978in}}%
\pgfpathlineto{\pgfqpoint{1.975400in}{2.876906in}}%
\pgfpathlineto{\pgfqpoint{1.974849in}{2.557085in}}%
\pgfpathlineto{\pgfqpoint{1.975753in}{2.741963in}}%
\pgfpathlineto{\pgfqpoint{1.976458in}{2.474480in}}%
\pgfpathlineto{\pgfqpoint{1.975885in}{2.759008in}}%
\pgfpathlineto{\pgfqpoint{1.976855in}{2.654550in}}%
\pgfpathlineto{\pgfqpoint{1.977031in}{2.522448in}}%
\pgfpathlineto{\pgfqpoint{1.977979in}{2.762068in}}%
\pgfpathlineto{\pgfqpoint{1.978178in}{2.552277in}}%
\pgfpathlineto{\pgfqpoint{1.978817in}{2.824786in}}%
\pgfpathlineto{\pgfqpoint{1.979103in}{2.646137in}}%
\pgfpathlineto{\pgfqpoint{1.979919in}{2.910014in}}%
\pgfpathlineto{\pgfqpoint{1.979258in}{2.562876in}}%
\pgfpathlineto{\pgfqpoint{1.980294in}{2.708200in}}%
\pgfpathlineto{\pgfqpoint{1.980889in}{2.602649in}}%
\pgfpathlineto{\pgfqpoint{1.981197in}{2.855381in}}%
\pgfpathlineto{\pgfqpoint{1.981396in}{2.716176in}}%
\pgfpathlineto{\pgfqpoint{1.981594in}{2.832107in}}%
\pgfpathlineto{\pgfqpoint{1.981925in}{2.587024in}}%
\pgfpathlineto{\pgfqpoint{1.982520in}{2.779004in}}%
\pgfpathlineto{\pgfqpoint{1.983556in}{2.557631in}}%
\pgfpathlineto{\pgfqpoint{1.983247in}{2.895700in}}%
\pgfpathlineto{\pgfqpoint{1.983644in}{2.693230in}}%
\pgfpathlineto{\pgfqpoint{1.983666in}{2.756277in}}%
\pgfpathlineto{\pgfqpoint{1.984239in}{2.481801in}}%
\pgfpathlineto{\pgfqpoint{1.984702in}{2.578392in}}%
\pgfpathlineto{\pgfqpoint{1.984834in}{2.734533in}}%
\pgfpathlineto{\pgfqpoint{1.985848in}{2.390673in}}%
\pgfpathlineto{\pgfqpoint{1.987061in}{2.740761in}}%
\pgfpathlineto{\pgfqpoint{1.986201in}{2.352539in}}%
\pgfpathlineto{\pgfqpoint{1.987105in}{2.714646in}}%
\pgfpathlineto{\pgfqpoint{1.988075in}{2.460166in}}%
\pgfpathlineto{\pgfqpoint{1.987700in}{2.737483in}}%
\pgfpathlineto{\pgfqpoint{1.988207in}{2.676294in}}%
\pgfpathlineto{\pgfqpoint{1.988670in}{2.799655in}}%
\pgfpathlineto{\pgfqpoint{1.988471in}{2.503545in}}%
\pgfpathlineto{\pgfqpoint{1.989243in}{2.752234in}}%
\pgfpathlineto{\pgfqpoint{1.989662in}{2.822601in}}%
\pgfpathlineto{\pgfqpoint{1.990389in}{2.423999in}}%
\pgfpathlineto{\pgfqpoint{1.991204in}{2.850573in}}%
\pgfpathlineto{\pgfqpoint{1.991557in}{2.704703in}}%
\pgfpathlineto{\pgfqpoint{1.991822in}{2.483112in}}%
\pgfpathlineto{\pgfqpoint{1.992196in}{2.825442in}}%
\pgfpathlineto{\pgfqpoint{1.992703in}{2.552933in}}%
\pgfpathlineto{\pgfqpoint{1.993188in}{2.738248in}}%
\pgfpathlineto{\pgfqpoint{1.993607in}{2.446508in}}%
\pgfpathlineto{\pgfqpoint{1.993827in}{2.609095in}}%
\pgfpathlineto{\pgfqpoint{1.994511in}{2.388597in}}%
\pgfpathlineto{\pgfqpoint{1.994092in}{2.682413in}}%
\pgfpathlineto{\pgfqpoint{1.994709in}{2.532172in}}%
\pgfpathlineto{\pgfqpoint{1.995767in}{2.751360in}}%
\pgfpathlineto{\pgfqpoint{1.995282in}{2.495131in}}%
\pgfpathlineto{\pgfqpoint{1.995833in}{2.741963in}}%
\pgfpathlineto{\pgfqpoint{1.996891in}{2.422906in}}%
\pgfpathlineto{\pgfqpoint{1.996010in}{2.810472in}}%
\pgfpathlineto{\pgfqpoint{1.996979in}{2.616853in}}%
\pgfpathlineto{\pgfqpoint{1.997993in}{2.733659in}}%
\pgfpathlineto{\pgfqpoint{1.997442in}{2.452736in}}%
\pgfpathlineto{\pgfqpoint{1.998104in}{2.666241in}}%
\pgfpathlineto{\pgfqpoint{1.998853in}{2.467596in}}%
\pgfpathlineto{\pgfqpoint{1.998985in}{2.716067in}}%
\pgfpathlineto{\pgfqpoint{1.999228in}{2.539493in}}%
\pgfpathlineto{\pgfqpoint{1.999977in}{2.735079in}}%
\pgfpathlineto{\pgfqpoint{2.000308in}{2.435363in}}%
\pgfpathlineto{\pgfqpoint{2.000484in}{2.727103in}}%
\pgfpathlineto{\pgfqpoint{2.001123in}{2.414165in}}%
\pgfpathlineto{\pgfqpoint{2.001498in}{2.564078in}}%
\pgfpathlineto{\pgfqpoint{2.002336in}{2.400944in}}%
\pgfpathlineto{\pgfqpoint{2.001851in}{2.722623in}}%
\pgfpathlineto{\pgfqpoint{2.002556in}{2.585822in}}%
\pgfpathlineto{\pgfqpoint{2.002578in}{2.747645in}}%
\pgfpathlineto{\pgfqpoint{2.003482in}{2.405752in}}%
\pgfpathlineto{\pgfqpoint{2.003658in}{2.575769in}}%
\pgfpathlineto{\pgfqpoint{2.004562in}{2.720110in}}%
\pgfpathlineto{\pgfqpoint{2.003812in}{2.448912in}}%
\pgfpathlineto{\pgfqpoint{2.004760in}{2.634445in}}%
\pgfpathlineto{\pgfqpoint{2.004848in}{2.483549in}}%
\pgfpathlineto{\pgfqpoint{2.005796in}{2.713226in}}%
\pgfpathlineto{\pgfqpoint{2.005818in}{2.644825in}}%
\pgfpathlineto{\pgfqpoint{2.005862in}{2.809489in}}%
\pgfpathlineto{\pgfqpoint{2.006325in}{2.428807in}}%
\pgfpathlineto{\pgfqpoint{2.006854in}{2.554790in}}%
\pgfpathlineto{\pgfqpoint{2.006876in}{2.459838in}}%
\pgfpathlineto{\pgfqpoint{2.007736in}{2.783702in}}%
\pgfpathlineto{\pgfqpoint{2.007934in}{2.593689in}}%
\pgfpathlineto{\pgfqpoint{2.007978in}{2.758571in}}%
\pgfpathlineto{\pgfqpoint{2.008728in}{2.472622in}}%
\pgfpathlineto{\pgfqpoint{2.009014in}{2.579484in}}%
\pgfpathlineto{\pgfqpoint{2.009036in}{2.494366in}}%
\pgfpathlineto{\pgfqpoint{2.009213in}{2.814297in}}%
\pgfpathlineto{\pgfqpoint{2.010094in}{2.693667in}}%
\pgfpathlineto{\pgfqpoint{2.011152in}{2.582762in}}%
\pgfpathlineto{\pgfqpoint{2.010579in}{2.824021in}}%
\pgfpathlineto{\pgfqpoint{2.011175in}{2.595437in}}%
\pgfpathlineto{\pgfqpoint{2.011197in}{2.776054in}}%
\pgfpathlineto{\pgfqpoint{2.011902in}{2.397557in}}%
\pgfpathlineto{\pgfqpoint{2.012255in}{2.567684in}}%
\pgfpathlineto{\pgfqpoint{2.012872in}{2.385974in}}%
\pgfpathlineto{\pgfqpoint{2.012431in}{2.688532in}}%
\pgfpathlineto{\pgfqpoint{2.013357in}{2.532391in}}%
\pgfpathlineto{\pgfqpoint{2.014437in}{2.776709in}}%
\pgfpathlineto{\pgfqpoint{2.013401in}{2.419847in}}%
\pgfpathlineto{\pgfqpoint{2.014503in}{2.602539in}}%
\pgfpathlineto{\pgfqpoint{2.015649in}{2.802715in}}%
\pgfpathlineto{\pgfqpoint{2.015208in}{2.549545in}}%
\pgfpathlineto{\pgfqpoint{2.015671in}{2.785232in}}%
\pgfpathlineto{\pgfqpoint{2.016619in}{2.524851in}}%
\pgfpathlineto{\pgfqpoint{2.016376in}{2.796924in}}%
\pgfpathlineto{\pgfqpoint{2.016817in}{2.580358in}}%
\pgfpathlineto{\pgfqpoint{2.017941in}{2.790258in}}%
\pgfpathlineto{\pgfqpoint{2.017523in}{2.484860in}}%
\pgfpathlineto{\pgfqpoint{2.017985in}{2.673125in}}%
\pgfpathlineto{\pgfqpoint{2.018845in}{2.447163in}}%
\pgfpathlineto{\pgfqpoint{2.018515in}{2.733221in}}%
\pgfpathlineto{\pgfqpoint{2.019110in}{2.550529in}}%
\pgfpathlineto{\pgfqpoint{2.019793in}{2.667116in}}%
\pgfpathlineto{\pgfqpoint{2.019154in}{2.381494in}}%
\pgfpathlineto{\pgfqpoint{2.020190in}{2.570962in}}%
\pgfpathlineto{\pgfqpoint{2.020300in}{2.497426in}}%
\pgfpathlineto{\pgfqpoint{2.021204in}{2.772011in}}%
\pgfpathlineto{\pgfqpoint{2.021270in}{2.615105in}}%
\pgfpathlineto{\pgfqpoint{2.021380in}{2.751141in}}%
\pgfpathlineto{\pgfqpoint{2.022262in}{2.404768in}}%
\pgfpathlineto{\pgfqpoint{2.022350in}{2.484860in}}%
\pgfpathlineto{\pgfqpoint{2.023077in}{2.397666in}}%
\pgfpathlineto{\pgfqpoint{2.022702in}{2.717706in}}%
\pgfpathlineto{\pgfqpoint{2.023452in}{2.452736in}}%
\pgfpathlineto{\pgfqpoint{2.023893in}{2.678042in}}%
\pgfpathlineto{\pgfqpoint{2.023496in}{2.393295in}}%
\pgfpathlineto{\pgfqpoint{2.024708in}{2.579594in}}%
\pgfpathlineto{\pgfqpoint{2.025678in}{2.387067in}}%
\pgfpathlineto{\pgfqpoint{2.025061in}{2.726119in}}%
\pgfpathlineto{\pgfqpoint{2.025832in}{2.450988in}}%
\pgfpathlineto{\pgfqpoint{2.026229in}{2.784904in}}%
\pgfpathlineto{\pgfqpoint{2.026780in}{2.410996in}}%
\pgfpathlineto{\pgfqpoint{2.026957in}{2.490105in}}%
\pgfpathlineto{\pgfqpoint{2.027574in}{2.307412in}}%
\pgfpathlineto{\pgfqpoint{2.027728in}{2.670175in}}%
\pgfpathlineto{\pgfqpoint{2.028059in}{2.501687in}}%
\pgfpathlineto{\pgfqpoint{2.029029in}{2.853851in}}%
\pgfpathlineto{\pgfqpoint{2.028213in}{2.417334in}}%
\pgfpathlineto{\pgfqpoint{2.029183in}{2.591941in}}%
\pgfpathlineto{\pgfqpoint{2.030109in}{2.758025in}}%
\pgfpathlineto{\pgfqpoint{2.029315in}{2.495568in}}%
\pgfpathlineto{\pgfqpoint{2.030351in}{2.699131in}}%
\pgfpathlineto{\pgfqpoint{2.030439in}{2.518623in}}%
\pgfpathlineto{\pgfqpoint{2.030902in}{2.815280in}}%
\pgfpathlineto{\pgfqpoint{2.031321in}{2.734642in}}%
\pgfpathlineto{\pgfqpoint{2.032136in}{2.831233in}}%
\pgfpathlineto{\pgfqpoint{2.031475in}{2.620787in}}%
\pgfpathlineto{\pgfqpoint{2.032379in}{2.670394in}}%
\pgfpathlineto{\pgfqpoint{2.033261in}{2.508680in}}%
\pgfpathlineto{\pgfqpoint{2.033437in}{2.776382in}}%
\pgfpathlineto{\pgfqpoint{2.033459in}{2.725136in}}%
\pgfpathlineto{\pgfqpoint{2.034385in}{2.937986in}}%
\pgfpathlineto{\pgfqpoint{2.033856in}{2.575223in}}%
\pgfpathlineto{\pgfqpoint{2.034517in}{2.683615in}}%
\pgfpathlineto{\pgfqpoint{2.034583in}{2.550529in}}%
\pgfpathlineto{\pgfqpoint{2.035024in}{2.824568in}}%
\pgfpathlineto{\pgfqpoint{2.035575in}{2.761740in}}%
\pgfpathlineto{\pgfqpoint{2.035840in}{2.910123in}}%
\pgfpathlineto{\pgfqpoint{2.036324in}{2.601338in}}%
\pgfpathlineto{\pgfqpoint{2.036677in}{2.777802in}}%
\pgfpathlineto{\pgfqpoint{2.037581in}{2.537308in}}%
\pgfpathlineto{\pgfqpoint{2.037008in}{2.850792in}}%
\pgfpathlineto{\pgfqpoint{2.037779in}{2.595219in}}%
\pgfpathlineto{\pgfqpoint{2.037978in}{2.850682in}}%
\pgfpathlineto{\pgfqpoint{2.038485in}{2.584073in}}%
\pgfpathlineto{\pgfqpoint{2.038903in}{2.734096in}}%
\pgfpathlineto{\pgfqpoint{2.039146in}{2.360188in}}%
\pgfpathlineto{\pgfqpoint{2.039278in}{2.786215in}}%
\pgfpathlineto{\pgfqpoint{2.040027in}{2.537745in}}%
\pgfpathlineto{\pgfqpoint{2.041063in}{2.737811in}}%
\pgfpathlineto{\pgfqpoint{2.040094in}{2.510756in}}%
\pgfpathlineto{\pgfqpoint{2.041152in}{2.675311in}}%
\pgfpathlineto{\pgfqpoint{2.041703in}{2.479506in}}%
\pgfpathlineto{\pgfqpoint{2.042166in}{2.738903in}}%
\pgfpathlineto{\pgfqpoint{2.042254in}{2.646792in}}%
\pgfpathlineto{\pgfqpoint{2.042584in}{2.849153in}}%
\pgfpathlineto{\pgfqpoint{2.042871in}{2.521246in}}%
\pgfpathlineto{\pgfqpoint{2.043334in}{2.565608in}}%
\pgfpathlineto{\pgfqpoint{2.043863in}{2.390782in}}%
\pgfpathlineto{\pgfqpoint{2.043510in}{2.687548in}}%
\pgfpathlineto{\pgfqpoint{2.044348in}{2.585057in}}%
\pgfpathlineto{\pgfqpoint{2.044678in}{2.748847in}}%
\pgfpathlineto{\pgfqpoint{2.045163in}{2.391001in}}%
\pgfpathlineto{\pgfqpoint{2.045450in}{2.560144in}}%
\pgfpathlineto{\pgfqpoint{2.045494in}{2.478413in}}%
\pgfpathlineto{\pgfqpoint{2.046376in}{2.771465in}}%
\pgfpathlineto{\pgfqpoint{2.046530in}{2.636084in}}%
\pgfpathlineto{\pgfqpoint{2.047301in}{2.774087in}}%
\pgfpathlineto{\pgfqpoint{2.046927in}{2.428151in}}%
\pgfpathlineto{\pgfqpoint{2.047412in}{2.601556in}}%
\pgfpathlineto{\pgfqpoint{2.047852in}{2.472622in}}%
\pgfpathlineto{\pgfqpoint{2.047632in}{2.708309in}}%
\pgfpathlineto{\pgfqpoint{2.048514in}{2.570852in}}%
\pgfpathlineto{\pgfqpoint{2.048800in}{2.756604in}}%
\pgfpathlineto{\pgfqpoint{2.049417in}{2.531954in}}%
\pgfpathlineto{\pgfqpoint{2.049594in}{2.608877in}}%
\pgfpathlineto{\pgfqpoint{2.049682in}{2.430227in}}%
\pgfpathlineto{\pgfqpoint{2.050608in}{2.798890in}}%
\pgfpathlineto{\pgfqpoint{2.050652in}{2.635756in}}%
\pgfpathlineto{\pgfqpoint{2.050762in}{2.826098in}}%
\pgfpathlineto{\pgfqpoint{2.051379in}{2.527255in}}%
\pgfpathlineto{\pgfqpoint{2.051754in}{2.734533in}}%
\pgfpathlineto{\pgfqpoint{2.052834in}{2.532828in}}%
\pgfpathlineto{\pgfqpoint{2.052283in}{2.851884in}}%
\pgfpathlineto{\pgfqpoint{2.052900in}{2.537963in}}%
\pgfpathlineto{\pgfqpoint{2.053253in}{2.767859in}}%
\pgfpathlineto{\pgfqpoint{2.053142in}{2.487264in}}%
\pgfpathlineto{\pgfqpoint{2.054046in}{2.632369in}}%
\pgfpathlineto{\pgfqpoint{2.054068in}{2.389252in}}%
\pgfpathlineto{\pgfqpoint{2.054663in}{2.762614in}}%
\pgfpathlineto{\pgfqpoint{2.055148in}{2.549764in}}%
\pgfpathlineto{\pgfqpoint{2.055611in}{2.485188in}}%
\pgfpathlineto{\pgfqpoint{2.056030in}{2.756495in}}%
\pgfpathlineto{\pgfqpoint{2.056228in}{2.609970in}}%
\pgfpathlineto{\pgfqpoint{2.056339in}{2.528239in}}%
\pgfpathlineto{\pgfqpoint{2.057507in}{2.831124in}}%
\pgfpathlineto{\pgfqpoint{2.058344in}{2.525398in}}%
\pgfpathlineto{\pgfqpoint{2.058609in}{2.722623in}}%
\pgfpathlineto{\pgfqpoint{2.058631in}{2.797798in}}%
\pgfpathlineto{\pgfqpoint{2.059380in}{2.498081in}}%
\pgfpathlineto{\pgfqpoint{2.059667in}{2.593252in}}%
\pgfpathlineto{\pgfqpoint{2.060549in}{2.458855in}}%
\pgfpathlineto{\pgfqpoint{2.060681in}{2.724699in}}%
\pgfpathlineto{\pgfqpoint{2.060747in}{2.583090in}}%
\pgfpathlineto{\pgfqpoint{2.061783in}{2.849699in}}%
\pgfpathlineto{\pgfqpoint{2.061232in}{2.503763in}}%
\pgfpathlineto{\pgfqpoint{2.061871in}{2.696399in}}%
\pgfpathlineto{\pgfqpoint{2.062378in}{2.515782in}}%
\pgfpathlineto{\pgfqpoint{2.062819in}{2.758680in}}%
\pgfpathlineto{\pgfqpoint{2.062973in}{2.640783in}}%
\pgfpathlineto{\pgfqpoint{2.063877in}{2.838991in}}%
\pgfpathlineto{\pgfqpoint{2.063348in}{2.572164in}}%
\pgfpathlineto{\pgfqpoint{2.064075in}{2.611718in}}%
\pgfpathlineto{\pgfqpoint{2.064494in}{2.817684in}}%
\pgfpathlineto{\pgfqpoint{2.064913in}{2.557959in}}%
\pgfpathlineto{\pgfqpoint{2.065199in}{2.674983in}}%
\pgfpathlineto{\pgfqpoint{2.065310in}{2.479069in}}%
\pgfpathlineto{\pgfqpoint{2.065662in}{2.837898in}}%
\pgfpathlineto{\pgfqpoint{2.066368in}{2.588772in}}%
\pgfpathlineto{\pgfqpoint{2.067382in}{2.458308in}}%
\pgfpathlineto{\pgfqpoint{2.067051in}{2.731910in}}%
\pgfpathlineto{\pgfqpoint{2.067470in}{2.552277in}}%
\pgfpathlineto{\pgfqpoint{2.068572in}{2.805118in}}%
\pgfpathlineto{\pgfqpoint{2.068462in}{2.522229in}}%
\pgfpathlineto{\pgfqpoint{2.068616in}{2.643623in}}%
\pgfpathlineto{\pgfqpoint{2.069167in}{2.459729in}}%
\pgfpathlineto{\pgfqpoint{2.069520in}{2.819760in}}%
\pgfpathlineto{\pgfqpoint{2.069674in}{2.659795in}}%
\pgfpathlineto{\pgfqpoint{2.069718in}{2.802387in}}%
\pgfpathlineto{\pgfqpoint{2.070534in}{2.502015in}}%
\pgfpathlineto{\pgfqpoint{2.070776in}{2.658484in}}%
\pgfpathlineto{\pgfqpoint{2.071195in}{2.534685in}}%
\pgfpathlineto{\pgfqpoint{2.070952in}{2.897557in}}%
\pgfpathlineto{\pgfqpoint{2.071856in}{2.682085in}}%
\pgfpathlineto{\pgfqpoint{2.072032in}{2.875376in}}%
\pgfpathlineto{\pgfqpoint{2.072539in}{2.555883in}}%
\pgfpathlineto{\pgfqpoint{2.072980in}{2.778785in}}%
\pgfpathlineto{\pgfqpoint{2.073950in}{2.504091in}}%
\pgfpathlineto{\pgfqpoint{2.074104in}{2.679244in}}%
\pgfpathlineto{\pgfqpoint{2.074854in}{2.498300in}}%
\pgfpathlineto{\pgfqpoint{2.074655in}{2.830359in}}%
\pgfpathlineto{\pgfqpoint{2.075140in}{2.658811in}}%
\pgfpathlineto{\pgfqpoint{2.075912in}{2.862592in}}%
\pgfpathlineto{\pgfqpoint{2.075515in}{2.542771in}}%
\pgfpathlineto{\pgfqpoint{2.076220in}{2.634882in}}%
\pgfpathlineto{\pgfqpoint{2.077168in}{2.523103in}}%
\pgfpathlineto{\pgfqpoint{2.076639in}{2.795066in}}%
\pgfpathlineto{\pgfqpoint{2.077278in}{2.674218in}}%
\pgfpathlineto{\pgfqpoint{2.077389in}{2.853633in}}%
\pgfpathlineto{\pgfqpoint{2.078314in}{2.491962in}}%
\pgfpathlineto{\pgfqpoint{2.078359in}{2.647448in}}%
\pgfpathlineto{\pgfqpoint{2.078645in}{2.531735in}}%
\pgfpathlineto{\pgfqpoint{2.078998in}{2.812767in}}%
\pgfpathlineto{\pgfqpoint{2.079439in}{2.712133in}}%
\pgfpathlineto{\pgfqpoint{2.079505in}{2.799437in}}%
\pgfpathlineto{\pgfqpoint{2.080629in}{2.520590in}}%
\pgfpathlineto{\pgfqpoint{2.080959in}{2.784904in}}%
\pgfpathlineto{\pgfqpoint{2.080783in}{2.514253in}}%
\pgfpathlineto{\pgfqpoint{2.081753in}{2.662417in}}%
\pgfpathlineto{\pgfqpoint{2.082414in}{2.586477in}}%
\pgfpathlineto{\pgfqpoint{2.082062in}{2.782828in}}%
\pgfpathlineto{\pgfqpoint{2.082569in}{2.753873in}}%
\pgfpathlineto{\pgfqpoint{2.082591in}{2.810691in}}%
\pgfpathlineto{\pgfqpoint{2.083516in}{2.520044in}}%
\pgfpathlineto{\pgfqpoint{2.083605in}{2.592050in}}%
\pgfpathlineto{\pgfqpoint{2.084729in}{2.752780in}}%
\pgfpathlineto{\pgfqpoint{2.083891in}{2.541241in}}%
\pgfpathlineto{\pgfqpoint{2.084773in}{2.750376in}}%
\pgfpathlineto{\pgfqpoint{2.085015in}{2.600245in}}%
\pgfpathlineto{\pgfqpoint{2.085588in}{2.882916in}}%
\pgfpathlineto{\pgfqpoint{2.085875in}{2.742728in}}%
\pgfpathlineto{\pgfqpoint{2.086051in}{2.577080in}}%
\pgfpathlineto{\pgfqpoint{2.086183in}{2.828392in}}%
\pgfpathlineto{\pgfqpoint{2.086470in}{2.796596in}}%
\pgfpathlineto{\pgfqpoint{2.087396in}{2.920503in}}%
\pgfpathlineto{\pgfqpoint{2.086999in}{2.640127in}}%
\pgfpathlineto{\pgfqpoint{2.087550in}{2.769716in}}%
\pgfpathlineto{\pgfqpoint{2.088630in}{2.565280in}}%
\pgfpathlineto{\pgfqpoint{2.088388in}{2.975792in}}%
\pgfpathlineto{\pgfqpoint{2.088674in}{2.613685in}}%
\pgfpathlineto{\pgfqpoint{2.088806in}{2.859751in}}%
\pgfpathlineto{\pgfqpoint{2.089291in}{2.489449in}}%
\pgfpathlineto{\pgfqpoint{2.089798in}{2.711478in}}%
\pgfpathlineto{\pgfqpoint{2.089931in}{2.595546in}}%
\pgfpathlineto{\pgfqpoint{2.090922in}{2.969454in}}%
\pgfpathlineto{\pgfqpoint{2.092025in}{2.495787in}}%
\pgfpathlineto{\pgfqpoint{2.092069in}{2.556866in}}%
\pgfpathlineto{\pgfqpoint{2.092686in}{2.863139in}}%
\pgfpathlineto{\pgfqpoint{2.093215in}{2.652583in}}%
\pgfpathlineto{\pgfqpoint{2.093303in}{2.537089in}}%
\pgfpathlineto{\pgfqpoint{2.094008in}{2.867837in}}%
\pgfpathlineto{\pgfqpoint{2.094758in}{2.987593in}}%
\pgfpathlineto{\pgfqpoint{2.094096in}{2.609751in}}%
\pgfpathlineto{\pgfqpoint{2.094956in}{2.744913in}}%
\pgfpathlineto{\pgfqpoint{2.095397in}{2.584729in}}%
\pgfpathlineto{\pgfqpoint{2.095221in}{2.854616in}}%
\pgfpathlineto{\pgfqpoint{2.096036in}{2.821290in}}%
\pgfpathlineto{\pgfqpoint{2.096190in}{2.831998in}}%
\pgfpathlineto{\pgfqpoint{2.096235in}{2.630512in}}%
\pgfpathlineto{\pgfqpoint{2.096389in}{2.747863in}}%
\pgfpathlineto{\pgfqpoint{2.097204in}{2.455140in}}%
\pgfpathlineto{\pgfqpoint{2.096742in}{2.780315in}}%
\pgfpathlineto{\pgfqpoint{2.097513in}{2.603632in}}%
\pgfpathlineto{\pgfqpoint{2.098549in}{2.707544in}}%
\pgfpathlineto{\pgfqpoint{2.097888in}{2.488794in}}%
\pgfpathlineto{\pgfqpoint{2.098615in}{2.659139in}}%
\pgfpathlineto{\pgfqpoint{2.099541in}{2.429025in}}%
\pgfpathlineto{\pgfqpoint{2.099387in}{2.788838in}}%
\pgfpathlineto{\pgfqpoint{2.099739in}{2.601775in}}%
\pgfpathlineto{\pgfqpoint{2.100158in}{2.734533in}}%
\pgfpathlineto{\pgfqpoint{2.100709in}{2.382806in}}%
\pgfpathlineto{\pgfqpoint{2.100753in}{2.496114in}}%
\pgfpathlineto{\pgfqpoint{2.100819in}{2.562220in}}%
\pgfpathlineto{\pgfqpoint{2.101965in}{2.238902in}}%
\pgfpathlineto{\pgfqpoint{2.102803in}{2.523103in}}%
\pgfpathlineto{\pgfqpoint{2.102891in}{2.163946in}}%
\pgfpathlineto{\pgfqpoint{2.103068in}{2.321835in}}%
\pgfpathlineto{\pgfqpoint{2.103310in}{2.461368in}}%
\pgfpathlineto{\pgfqpoint{2.104170in}{2.177495in}}%
\pgfpathlineto{\pgfqpoint{2.104941in}{2.423343in}}%
\pgfpathlineto{\pgfqpoint{2.104368in}{2.119256in}}%
\pgfpathlineto{\pgfqpoint{2.105294in}{2.256931in}}%
\pgfpathlineto{\pgfqpoint{2.105404in}{2.435690in}}%
\pgfpathlineto{\pgfqpoint{2.105845in}{2.128653in}}%
\pgfpathlineto{\pgfqpoint{2.106396in}{2.305992in}}%
\pgfpathlineto{\pgfqpoint{2.106969in}{2.043972in}}%
\pgfpathlineto{\pgfqpoint{2.106484in}{2.335056in}}%
\pgfpathlineto{\pgfqpoint{2.107520in}{2.189405in}}%
\pgfpathlineto{\pgfqpoint{2.107542in}{2.285340in}}%
\pgfpathlineto{\pgfqpoint{2.108049in}{1.976774in}}%
\pgfpathlineto{\pgfqpoint{2.108600in}{2.143076in}}%
\pgfpathlineto{\pgfqpoint{2.108688in}{1.906078in}}%
\pgfpathlineto{\pgfqpoint{2.109372in}{2.334073in}}%
\pgfpathlineto{\pgfqpoint{2.109702in}{2.215629in}}%
\pgfpathlineto{\pgfqpoint{2.109768in}{1.992617in}}%
\pgfpathlineto{\pgfqpoint{2.109879in}{2.250375in}}%
\pgfpathlineto{\pgfqpoint{2.110870in}{2.139361in}}%
\pgfpathlineto{\pgfqpoint{2.111179in}{2.311783in}}%
\pgfpathlineto{\pgfqpoint{2.111752in}{2.005948in}}%
\pgfpathlineto{\pgfqpoint{2.111906in}{2.025397in}}%
\pgfpathlineto{\pgfqpoint{2.112524in}{2.148430in}}%
\pgfpathlineto{\pgfqpoint{2.112149in}{1.874173in}}%
\pgfpathlineto{\pgfqpoint{2.112986in}{2.052276in}}%
\pgfpathlineto{\pgfqpoint{2.113449in}{1.900615in}}%
\pgfpathlineto{\pgfqpoint{2.113648in}{2.141656in}}%
\pgfpathlineto{\pgfqpoint{2.114089in}{2.047032in}}%
\pgfpathlineto{\pgfqpoint{2.114155in}{2.137176in}}%
\pgfpathlineto{\pgfqpoint{2.114838in}{1.739120in}}%
\pgfpathlineto{\pgfqpoint{2.115169in}{1.957324in}}%
\pgfpathlineto{\pgfqpoint{2.116227in}{1.896682in}}%
\pgfpathlineto{\pgfqpoint{2.115742in}{2.179025in}}%
\pgfpathlineto{\pgfqpoint{2.116271in}{1.947490in}}%
\pgfpathlineto{\pgfqpoint{2.116359in}{2.141328in}}%
\pgfpathlineto{\pgfqpoint{2.117329in}{1.841939in}}%
\pgfpathlineto{\pgfqpoint{2.117373in}{1.942573in}}%
\pgfpathlineto{\pgfqpoint{2.118144in}{2.035668in}}%
\pgfpathlineto{\pgfqpoint{2.117946in}{1.738465in}}%
\pgfpathlineto{\pgfqpoint{2.118409in}{1.962023in}}%
\pgfpathlineto{\pgfqpoint{2.118497in}{1.745785in}}%
\pgfpathlineto{\pgfqpoint{2.119313in}{2.225026in}}%
\pgfpathlineto{\pgfqpoint{2.119489in}{1.958526in}}%
\pgfpathlineto{\pgfqpoint{2.119797in}{2.184379in}}%
\pgfpathlineto{\pgfqpoint{2.120216in}{1.816480in}}%
\pgfpathlineto{\pgfqpoint{2.120569in}{1.943666in}}%
\pgfpathlineto{\pgfqpoint{2.121318in}{1.803806in}}%
\pgfpathlineto{\pgfqpoint{2.120878in}{2.031953in}}%
\pgfpathlineto{\pgfqpoint{2.121693in}{1.879308in}}%
\pgfpathlineto{\pgfqpoint{2.121715in}{1.878653in}}%
\pgfpathlineto{\pgfqpoint{2.121737in}{1.899741in}}%
\pgfpathlineto{\pgfqpoint{2.122090in}{2.168972in}}%
\pgfpathlineto{\pgfqpoint{2.122266in}{1.818775in}}%
\pgfpathlineto{\pgfqpoint{2.122795in}{1.963115in}}%
\pgfpathlineto{\pgfqpoint{2.123677in}{1.834400in}}%
\pgfpathlineto{\pgfqpoint{2.123060in}{2.088662in}}%
\pgfpathlineto{\pgfqpoint{2.123875in}{1.906516in}}%
\pgfpathlineto{\pgfqpoint{2.124294in}{2.180882in}}%
\pgfpathlineto{\pgfqpoint{2.124955in}{1.876030in}}%
\pgfpathlineto{\pgfqpoint{2.124999in}{2.075222in}}%
\pgfpathlineto{\pgfqpoint{2.125881in}{1.834291in}}%
\pgfpathlineto{\pgfqpoint{2.125352in}{2.103522in}}%
\pgfpathlineto{\pgfqpoint{2.126124in}{1.958854in}}%
\pgfpathlineto{\pgfqpoint{2.127027in}{2.043644in}}%
\pgfpathlineto{\pgfqpoint{2.126630in}{1.727319in}}%
\pgfpathlineto{\pgfqpoint{2.127137in}{1.867508in}}%
\pgfpathlineto{\pgfqpoint{2.127755in}{1.625811in}}%
\pgfpathlineto{\pgfqpoint{2.127336in}{2.018076in}}%
\pgfpathlineto{\pgfqpoint{2.128262in}{1.771026in}}%
\pgfpathlineto{\pgfqpoint{2.128989in}{1.942683in}}%
\pgfpathlineto{\pgfqpoint{2.128835in}{1.621004in}}%
\pgfpathlineto{\pgfqpoint{2.129342in}{1.855488in}}%
\pgfpathlineto{\pgfqpoint{2.130267in}{1.615977in}}%
\pgfpathlineto{\pgfqpoint{2.129386in}{1.918316in}}%
\pgfpathlineto{\pgfqpoint{2.130488in}{1.650615in}}%
\pgfpathlineto{\pgfqpoint{2.130510in}{1.574129in}}%
\pgfpathlineto{\pgfqpoint{2.131347in}{1.884662in}}%
\pgfpathlineto{\pgfqpoint{2.131590in}{1.667332in}}%
\pgfpathlineto{\pgfqpoint{2.131722in}{1.914383in}}%
\pgfpathlineto{\pgfqpoint{2.132582in}{1.550746in}}%
\pgfpathlineto{\pgfqpoint{2.132736in}{1.828500in}}%
\pgfpathlineto{\pgfqpoint{2.133023in}{1.580138in}}%
\pgfpathlineto{\pgfqpoint{2.133287in}{1.860078in}}%
\pgfpathlineto{\pgfqpoint{2.133838in}{1.738683in}}%
\pgfpathlineto{\pgfqpoint{2.134213in}{1.835383in}}%
\pgfpathlineto{\pgfqpoint{2.134720in}{1.523538in}}%
\pgfpathlineto{\pgfqpoint{2.134896in}{1.682520in}}%
\pgfpathlineto{\pgfqpoint{2.135205in}{1.541021in}}%
\pgfpathlineto{\pgfqpoint{2.135337in}{1.796703in}}%
\pgfpathlineto{\pgfqpoint{2.136020in}{1.652800in}}%
\pgfpathlineto{\pgfqpoint{2.136439in}{1.503106in}}%
\pgfpathlineto{\pgfqpoint{2.136682in}{1.893404in}}%
\pgfpathlineto{\pgfqpoint{2.137078in}{1.580685in}}%
\pgfpathlineto{\pgfqpoint{2.137872in}{1.884116in}}%
\pgfpathlineto{\pgfqpoint{2.137497in}{1.529876in}}%
\pgfpathlineto{\pgfqpoint{2.138180in}{1.589207in}}%
\pgfpathlineto{\pgfqpoint{2.139150in}{1.851118in}}%
\pgfpathlineto{\pgfqpoint{2.138225in}{1.563202in}}%
\pgfpathlineto{\pgfqpoint{2.139349in}{1.751030in}}%
\pgfpathlineto{\pgfqpoint{2.140098in}{1.497752in}}%
\pgfpathlineto{\pgfqpoint{2.139393in}{1.838880in}}%
\pgfpathlineto{\pgfqpoint{2.140495in}{1.644605in}}%
\pgfpathlineto{\pgfqpoint{2.140627in}{1.783701in}}%
\pgfpathlineto{\pgfqpoint{2.141046in}{1.529002in}}%
\pgfpathlineto{\pgfqpoint{2.141597in}{1.619365in}}%
\pgfpathlineto{\pgfqpoint{2.141817in}{1.854505in}}%
\pgfpathlineto{\pgfqpoint{2.142060in}{1.562874in}}%
\pgfpathlineto{\pgfqpoint{2.142545in}{1.582761in}}%
\pgfpathlineto{\pgfqpoint{2.142721in}{1.527035in}}%
\pgfpathlineto{\pgfqpoint{2.142986in}{1.836258in}}%
\pgfpathlineto{\pgfqpoint{2.143581in}{1.745021in}}%
\pgfpathlineto{\pgfqpoint{2.143801in}{1.548997in}}%
\pgfpathlineto{\pgfqpoint{2.144352in}{1.889252in}}%
\pgfpathlineto{\pgfqpoint{2.144705in}{1.623735in}}%
\pgfpathlineto{\pgfqpoint{2.145322in}{1.842814in}}%
\pgfpathlineto{\pgfqpoint{2.145454in}{1.589317in}}%
\pgfpathlineto{\pgfqpoint{2.145807in}{1.606362in}}%
\pgfpathlineto{\pgfqpoint{2.146534in}{1.880729in}}%
\pgfpathlineto{\pgfqpoint{2.145873in}{1.546375in}}%
\pgfpathlineto{\pgfqpoint{2.146909in}{1.681756in}}%
\pgfpathlineto{\pgfqpoint{2.147901in}{1.507804in}}%
\pgfpathlineto{\pgfqpoint{2.147659in}{1.838552in}}%
\pgfpathlineto{\pgfqpoint{2.147989in}{1.566371in}}%
\pgfpathlineto{\pgfqpoint{2.148232in}{1.805226in}}%
\pgfpathlineto{\pgfqpoint{2.148386in}{1.486388in}}%
\pgfpathlineto{\pgfqpoint{2.149091in}{1.662197in}}%
\pgfpathlineto{\pgfqpoint{2.150061in}{1.449128in}}%
\pgfpathlineto{\pgfqpoint{2.149664in}{1.827079in}}%
\pgfpathlineto{\pgfqpoint{2.150193in}{1.638923in}}%
\pgfpathlineto{\pgfqpoint{2.150722in}{1.915038in}}%
\pgfpathlineto{\pgfqpoint{2.150326in}{1.536323in}}%
\pgfpathlineto{\pgfqpoint{2.151362in}{1.701642in}}%
\pgfpathlineto{\pgfqpoint{2.151406in}{1.661760in}}%
\pgfpathlineto{\pgfqpoint{2.151472in}{1.702516in}}%
\pgfpathlineto{\pgfqpoint{2.151494in}{1.550746in}}%
\pgfpathlineto{\pgfqpoint{2.152155in}{1.830139in}}%
\pgfpathlineto{\pgfqpoint{2.152552in}{1.732455in}}%
\pgfpathlineto{\pgfqpoint{2.152794in}{1.925200in}}%
\pgfpathlineto{\pgfqpoint{2.152949in}{1.509443in}}%
\pgfpathlineto{\pgfqpoint{2.153654in}{1.738574in}}%
\pgfpathlineto{\pgfqpoint{2.154734in}{1.506056in}}%
\pgfpathlineto{\pgfqpoint{2.154073in}{1.950768in}}%
\pgfpathlineto{\pgfqpoint{2.154822in}{1.540038in}}%
\pgfpathlineto{\pgfqpoint{2.155704in}{1.859968in}}%
\pgfpathlineto{\pgfqpoint{2.155946in}{1.856909in}}%
\pgfpathlineto{\pgfqpoint{2.156409in}{1.601227in}}%
\pgfpathlineto{\pgfqpoint{2.157026in}{1.926620in}}%
\pgfpathlineto{\pgfqpoint{2.157070in}{1.733657in}}%
\pgfpathlineto{\pgfqpoint{2.157225in}{1.882259in}}%
\pgfpathlineto{\pgfqpoint{2.157445in}{1.622752in}}%
\pgfpathlineto{\pgfqpoint{2.157622in}{1.676729in}}%
\pgfpathlineto{\pgfqpoint{2.158327in}{1.577516in}}%
\pgfpathlineto{\pgfqpoint{2.158062in}{1.919409in}}%
\pgfpathlineto{\pgfqpoint{2.158724in}{1.674544in}}%
\pgfpathlineto{\pgfqpoint{2.159186in}{1.845654in}}%
\pgfpathlineto{\pgfqpoint{2.159319in}{1.537087in}}%
\pgfpathlineto{\pgfqpoint{2.159826in}{1.734203in}}%
\pgfpathlineto{\pgfqpoint{2.160796in}{1.534028in}}%
\pgfpathlineto{\pgfqpoint{2.160333in}{1.881603in}}%
\pgfpathlineto{\pgfqpoint{2.160928in}{1.731034in}}%
\pgfpathlineto{\pgfqpoint{2.160950in}{1.804570in}}%
\pgfpathlineto{\pgfqpoint{2.161258in}{1.487044in}}%
\pgfpathlineto{\pgfqpoint{2.161986in}{1.677822in}}%
\pgfpathlineto{\pgfqpoint{2.162493in}{1.483110in}}%
\pgfpathlineto{\pgfqpoint{2.162559in}{1.777363in}}%
\pgfpathlineto{\pgfqpoint{2.163110in}{1.562546in}}%
\pgfpathlineto{\pgfqpoint{2.163639in}{1.773867in}}%
\pgfpathlineto{\pgfqpoint{2.163374in}{1.468687in}}%
\pgfpathlineto{\pgfqpoint{2.164234in}{1.628215in}}%
\pgfpathlineto{\pgfqpoint{2.165138in}{1.551401in}}%
\pgfpathlineto{\pgfqpoint{2.164851in}{1.960493in}}%
\pgfpathlineto{\pgfqpoint{2.165248in}{1.812328in}}%
\pgfpathlineto{\pgfqpoint{2.165270in}{1.910449in}}%
\pgfpathlineto{\pgfqpoint{2.166284in}{1.519605in}}%
\pgfpathlineto{\pgfqpoint{2.166350in}{1.781406in}}%
\pgfpathlineto{\pgfqpoint{2.167055in}{1.535011in}}%
\pgfpathlineto{\pgfqpoint{2.166482in}{1.798124in}}%
\pgfpathlineto{\pgfqpoint{2.167474in}{1.687984in}}%
\pgfpathlineto{\pgfqpoint{2.167761in}{1.469343in}}%
\pgfpathlineto{\pgfqpoint{2.168466in}{1.798998in}}%
\pgfpathlineto{\pgfqpoint{2.168665in}{1.558067in}}%
\pgfpathlineto{\pgfqpoint{2.169436in}{1.767857in}}%
\pgfpathlineto{\pgfqpoint{2.169590in}{1.473932in}}%
\pgfpathlineto{\pgfqpoint{2.169723in}{1.675746in}}%
\pgfpathlineto{\pgfqpoint{2.170406in}{1.446506in}}%
\pgfpathlineto{\pgfqpoint{2.169921in}{1.754308in}}%
\pgfpathlineto{\pgfqpoint{2.170825in}{1.653128in}}%
\pgfpathlineto{\pgfqpoint{2.170847in}{1.675637in}}%
\pgfpathlineto{\pgfqpoint{2.171398in}{1.384771in}}%
\pgfpathlineto{\pgfqpoint{2.171772in}{1.635536in}}%
\pgfpathlineto{\pgfqpoint{2.172191in}{1.398757in}}%
\pgfpathlineto{\pgfqpoint{2.172257in}{1.728194in}}%
\pgfpathlineto{\pgfqpoint{2.172897in}{1.462240in}}%
\pgfpathlineto{\pgfqpoint{2.173800in}{1.820633in}}%
\pgfpathlineto{\pgfqpoint{2.173051in}{1.415256in}}%
\pgfpathlineto{\pgfqpoint{2.174065in}{1.706996in}}%
\pgfpathlineto{\pgfqpoint{2.174704in}{1.411650in}}%
\pgfpathlineto{\pgfqpoint{2.175211in}{1.454920in}}%
\pgfpathlineto{\pgfqpoint{2.175630in}{1.692464in}}%
\pgfpathlineto{\pgfqpoint{2.176137in}{1.394933in}}%
\pgfpathlineto{\pgfqpoint{2.176313in}{1.512830in}}%
\pgfpathlineto{\pgfqpoint{2.176401in}{1.413398in}}%
\pgfpathlineto{\pgfqpoint{2.177063in}{1.703172in}}%
\pgfpathlineto{\pgfqpoint{2.177239in}{1.666021in}}%
\pgfpathlineto{\pgfqpoint{2.177503in}{1.805445in}}%
\pgfpathlineto{\pgfqpoint{2.177371in}{1.459509in}}%
\pgfpathlineto{\pgfqpoint{2.178363in}{1.757258in}}%
\pgfpathlineto{\pgfqpoint{2.178495in}{1.417441in}}%
\pgfpathlineto{\pgfqpoint{2.178738in}{1.782171in}}%
\pgfpathlineto{\pgfqpoint{2.179509in}{1.532826in}}%
\pgfpathlineto{\pgfqpoint{2.180545in}{1.858657in}}%
\pgfpathlineto{\pgfqpoint{2.180347in}{1.496659in}}%
\pgfpathlineto{\pgfqpoint{2.180633in}{1.626686in}}%
\pgfpathlineto{\pgfqpoint{2.180986in}{1.773102in}}%
\pgfpathlineto{\pgfqpoint{2.180854in}{1.489229in}}%
\pgfpathlineto{\pgfqpoint{2.181559in}{1.595654in}}%
\pgfpathlineto{\pgfqpoint{2.182463in}{1.394605in}}%
\pgfpathlineto{\pgfqpoint{2.182088in}{1.729286in}}%
\pgfpathlineto{\pgfqpoint{2.182683in}{1.494255in}}%
\pgfpathlineto{\pgfqpoint{2.183190in}{1.366523in}}%
\pgfpathlineto{\pgfqpoint{2.182749in}{1.663508in}}%
\pgfpathlineto{\pgfqpoint{2.183653in}{1.498298in}}%
\pgfpathlineto{\pgfqpoint{2.183697in}{1.730379in}}%
\pgfpathlineto{\pgfqpoint{2.184380in}{1.341939in}}%
\pgfpathlineto{\pgfqpoint{2.184755in}{1.464972in}}%
\pgfpathlineto{\pgfqpoint{2.185328in}{1.410120in}}%
\pgfpathlineto{\pgfqpoint{2.185505in}{1.744802in}}%
\pgfpathlineto{\pgfqpoint{2.185791in}{1.519168in}}%
\pgfpathlineto{\pgfqpoint{2.185990in}{1.647337in}}%
\pgfpathlineto{\pgfqpoint{2.186474in}{1.367288in}}%
\pgfpathlineto{\pgfqpoint{2.186915in}{1.580357in}}%
\pgfpathlineto{\pgfqpoint{2.187753in}{1.656843in}}%
\pgfpathlineto{\pgfqpoint{2.186981in}{1.378324in}}%
\pgfpathlineto{\pgfqpoint{2.187929in}{1.541240in}}%
\pgfpathlineto{\pgfqpoint{2.188304in}{1.292441in}}%
\pgfpathlineto{\pgfqpoint{2.188150in}{1.589317in}}%
\pgfpathlineto{\pgfqpoint{2.189097in}{1.386847in}}%
\pgfpathlineto{\pgfqpoint{2.190222in}{1.638049in}}%
\pgfpathlineto{\pgfqpoint{2.189803in}{1.269277in}}%
\pgfpathlineto{\pgfqpoint{2.190244in}{1.508241in}}%
\pgfpathlineto{\pgfqpoint{2.190266in}{1.510427in}}%
\pgfpathlineto{\pgfqpoint{2.190288in}{1.463661in}}%
\pgfpathlineto{\pgfqpoint{2.190310in}{1.470545in}}%
\pgfpathlineto{\pgfqpoint{2.191236in}{1.270806in}}%
\pgfpathlineto{\pgfqpoint{2.190883in}{1.563202in}}%
\pgfpathlineto{\pgfqpoint{2.191434in}{1.316480in}}%
\pgfpathlineto{\pgfqpoint{2.192426in}{1.544627in}}%
\pgfpathlineto{\pgfqpoint{2.191588in}{1.228411in}}%
\pgfpathlineto{\pgfqpoint{2.192536in}{1.504089in}}%
\pgfpathlineto{\pgfqpoint{2.193594in}{1.265234in}}%
\pgfpathlineto{\pgfqpoint{2.193396in}{1.602756in}}%
\pgfpathlineto{\pgfqpoint{2.193660in}{1.388923in}}%
\pgfpathlineto{\pgfqpoint{2.193726in}{1.549435in}}%
\pgfpathlineto{\pgfqpoint{2.194388in}{1.241305in}}%
\pgfpathlineto{\pgfqpoint{2.194740in}{1.381930in}}%
\pgfpathlineto{\pgfqpoint{2.195335in}{1.230815in}}%
\pgfpathlineto{\pgfqpoint{2.195512in}{1.613137in}}%
\pgfpathlineto{\pgfqpoint{2.195864in}{1.298014in}}%
\pgfpathlineto{\pgfqpoint{2.196966in}{1.437983in}}%
\pgfpathlineto{\pgfqpoint{2.196526in}{1.172467in}}%
\pgfpathlineto{\pgfqpoint{2.197033in}{1.407389in}}%
\pgfpathlineto{\pgfqpoint{2.197518in}{1.135754in}}%
\pgfpathlineto{\pgfqpoint{2.197782in}{1.567026in}}%
\pgfpathlineto{\pgfqpoint{2.198135in}{1.287961in}}%
\pgfpathlineto{\pgfqpoint{2.199083in}{1.563748in}}%
\pgfpathlineto{\pgfqpoint{2.198950in}{1.210929in}}%
\pgfpathlineto{\pgfqpoint{2.199259in}{1.451314in}}%
\pgfpathlineto{\pgfqpoint{2.200008in}{1.248298in}}%
\pgfpathlineto{\pgfqpoint{2.199656in}{1.471528in}}%
\pgfpathlineto{\pgfqpoint{2.200361in}{1.426292in}}%
\pgfpathlineto{\pgfqpoint{2.200427in}{1.548233in}}%
\pgfpathlineto{\pgfqpoint{2.200978in}{1.191370in}}%
\pgfpathlineto{\pgfqpoint{2.201397in}{1.411213in}}%
\pgfpathlineto{\pgfqpoint{2.201507in}{1.233437in}}%
\pgfpathlineto{\pgfqpoint{2.202190in}{1.536541in}}%
\pgfpathlineto{\pgfqpoint{2.202521in}{1.299980in}}%
\pgfpathlineto{\pgfqpoint{2.203381in}{1.585929in}}%
\pgfpathlineto{\pgfqpoint{2.202808in}{1.248625in}}%
\pgfpathlineto{\pgfqpoint{2.203645in}{1.463552in}}%
\pgfpathlineto{\pgfqpoint{2.203689in}{1.304460in}}%
\pgfpathlineto{\pgfqpoint{2.204042in}{1.693228in}}%
\pgfpathlineto{\pgfqpoint{2.204747in}{1.356471in}}%
\pgfpathlineto{\pgfqpoint{2.204880in}{1.667879in}}%
\pgfpathlineto{\pgfqpoint{2.205012in}{1.355706in}}%
\pgfpathlineto{\pgfqpoint{2.205871in}{1.548888in}}%
\pgfpathlineto{\pgfqpoint{2.206555in}{1.265562in}}%
\pgfpathlineto{\pgfqpoint{2.206996in}{1.431209in}}%
\pgfpathlineto{\pgfqpoint{2.207018in}{1.531515in}}%
\pgfpathlineto{\pgfqpoint{2.207326in}{1.184049in}}%
\pgfpathlineto{\pgfqpoint{2.208054in}{1.368709in}}%
\pgfpathlineto{\pgfqpoint{2.208715in}{1.149631in}}%
\pgfpathlineto{\pgfqpoint{2.208230in}{1.488355in}}%
\pgfpathlineto{\pgfqpoint{2.209156in}{1.281733in}}%
\pgfpathlineto{\pgfqpoint{2.209398in}{1.465518in}}%
\pgfpathlineto{\pgfqpoint{2.210103in}{1.136191in}}%
\pgfpathlineto{\pgfqpoint{2.210236in}{1.271899in}}%
\pgfpathlineto{\pgfqpoint{2.210258in}{1.128979in}}%
\pgfpathlineto{\pgfqpoint{2.211228in}{1.460055in}}%
\pgfpathlineto{\pgfqpoint{2.211338in}{1.246877in}}%
\pgfpathlineto{\pgfqpoint{2.211779in}{1.348057in}}%
\pgfpathlineto{\pgfqpoint{2.212286in}{1.089316in}}%
\pgfpathlineto{\pgfqpoint{2.212396in}{1.209399in}}%
\pgfpathlineto{\pgfqpoint{2.213211in}{1.074346in}}%
\pgfpathlineto{\pgfqpoint{2.212859in}{1.332542in}}%
\pgfpathlineto{\pgfqpoint{2.213498in}{1.222074in}}%
\pgfpathlineto{\pgfqpoint{2.213939in}{1.088769in}}%
\pgfpathlineto{\pgfqpoint{2.214556in}{1.308940in}}%
\pgfpathlineto{\pgfqpoint{2.214578in}{1.185470in}}%
\pgfpathlineto{\pgfqpoint{2.214622in}{1.360732in}}%
\pgfpathlineto{\pgfqpoint{2.215548in}{0.975133in}}%
\pgfpathlineto{\pgfqpoint{2.215658in}{1.083415in}}%
\pgfpathlineto{\pgfqpoint{2.215702in}{1.087349in}}%
\pgfpathlineto{\pgfqpoint{2.216011in}{1.289382in}}%
\pgfpathlineto{\pgfqpoint{2.215856in}{1.007039in}}%
\pgfpathlineto{\pgfqpoint{2.216826in}{1.197708in}}%
\pgfpathlineto{\pgfqpoint{2.217796in}{1.037633in}}%
\pgfpathlineto{\pgfqpoint{2.217686in}{1.230597in}}%
\pgfpathlineto{\pgfqpoint{2.217928in}{1.204701in}}%
\pgfpathlineto{\pgfqpoint{2.217950in}{1.205902in}}%
\pgfpathlineto{\pgfqpoint{2.218369in}{0.954045in}}%
\pgfpathlineto{\pgfqpoint{2.218391in}{1.233328in}}%
\pgfpathlineto{\pgfqpoint{2.219075in}{1.081667in}}%
\pgfpathlineto{\pgfqpoint{2.219163in}{1.364666in}}%
\pgfpathlineto{\pgfqpoint{2.219119in}{1.041129in}}%
\pgfpathlineto{\pgfqpoint{2.220199in}{1.113245in}}%
\pgfpathlineto{\pgfqpoint{2.220750in}{1.000045in}}%
\pgfpathlineto{\pgfqpoint{2.220926in}{1.351117in}}%
\pgfpathlineto{\pgfqpoint{2.221235in}{1.224587in}}%
\pgfpathlineto{\pgfqpoint{2.221874in}{1.317900in}}%
\pgfpathlineto{\pgfqpoint{2.221323in}{1.052930in}}%
\pgfpathlineto{\pgfqpoint{2.222138in}{1.161759in}}%
\pgfpathlineto{\pgfqpoint{2.222160in}{1.027034in}}%
\pgfpathlineto{\pgfqpoint{2.223064in}{1.302275in}}%
\pgfpathlineto{\pgfqpoint{2.223218in}{1.272773in}}%
\pgfpathlineto{\pgfqpoint{2.223747in}{1.139469in}}%
\pgfpathlineto{\pgfqpoint{2.224321in}{1.398101in}}%
\pgfpathlineto{\pgfqpoint{2.224629in}{1.184159in}}%
\pgfpathlineto{\pgfqpoint{2.225202in}{1.527581in}}%
\pgfpathlineto{\pgfqpoint{2.225423in}{1.351335in}}%
\pgfpathlineto{\pgfqpoint{2.225709in}{1.408700in}}%
\pgfpathlineto{\pgfqpoint{2.225555in}{1.139797in}}%
\pgfpathlineto{\pgfqpoint{2.225974in}{1.319867in}}%
\pgfpathlineto{\pgfqpoint{2.226591in}{1.154547in}}%
\pgfpathlineto{\pgfqpoint{2.226128in}{1.387175in}}%
\pgfpathlineto{\pgfqpoint{2.227098in}{1.224915in}}%
\pgfpathlineto{\pgfqpoint{2.227208in}{1.424653in}}%
\pgfpathlineto{\pgfqpoint{2.227340in}{1.071615in}}%
\pgfpathlineto{\pgfqpoint{2.228178in}{1.218250in}}%
\pgfpathlineto{\pgfqpoint{2.228839in}{1.036540in}}%
\pgfpathlineto{\pgfqpoint{2.228487in}{1.312000in}}%
\pgfpathlineto{\pgfqpoint{2.229302in}{1.158153in}}%
\pgfpathlineto{\pgfqpoint{2.230051in}{1.007257in}}%
\pgfpathlineto{\pgfqpoint{2.230206in}{1.377996in}}%
\pgfpathlineto{\pgfqpoint{2.230404in}{1.143184in}}%
\pgfpathlineto{\pgfqpoint{2.230999in}{1.339753in}}%
\pgfpathlineto{\pgfqpoint{2.230625in}{0.954154in}}%
\pgfpathlineto{\pgfqpoint{2.231484in}{1.104941in}}%
\pgfpathlineto{\pgfqpoint{2.231881in}{0.972620in}}%
\pgfpathlineto{\pgfqpoint{2.232322in}{1.326095in}}%
\pgfpathlineto{\pgfqpoint{2.232520in}{1.199347in}}%
\pgfpathlineto{\pgfqpoint{2.232564in}{1.292004in}}%
\pgfpathlineto{\pgfqpoint{2.232983in}{0.991413in}}%
\pgfpathlineto{\pgfqpoint{2.233534in}{1.176291in}}%
\pgfpathlineto{\pgfqpoint{2.233865in}{1.018839in}}%
\pgfpathlineto{\pgfqpoint{2.234328in}{1.337896in}}%
\pgfpathlineto{\pgfqpoint{2.234592in}{1.204591in}}%
\pgfpathlineto{\pgfqpoint{2.235320in}{1.375155in}}%
\pgfpathlineto{\pgfqpoint{2.234746in}{1.063529in}}%
\pgfpathlineto{\pgfqpoint{2.235694in}{1.265234in}}%
\pgfpathlineto{\pgfqpoint{2.235804in}{1.099150in}}%
\pgfpathlineto{\pgfqpoint{2.236620in}{1.496222in}}%
\pgfpathlineto{\pgfqpoint{2.236686in}{1.418971in}}%
\pgfpathlineto{\pgfqpoint{2.236708in}{1.506712in}}%
\pgfpathlineto{\pgfqpoint{2.237612in}{1.112371in}}%
\pgfpathlineto{\pgfqpoint{2.237744in}{1.274412in}}%
\pgfpathlineto{\pgfqpoint{2.238119in}{1.145697in}}%
\pgfpathlineto{\pgfqpoint{2.238934in}{1.447380in}}%
\pgfpathlineto{\pgfqpoint{2.239243in}{1.242834in}}%
\pgfpathlineto{\pgfqpoint{2.239221in}{1.508351in}}%
\pgfpathlineto{\pgfqpoint{2.240037in}{1.303258in}}%
\pgfpathlineto{\pgfqpoint{2.240389in}{1.440933in}}%
\pgfpathlineto{\pgfqpoint{2.240566in}{1.122314in}}%
\pgfpathlineto{\pgfqpoint{2.241117in}{1.255946in}}%
\pgfpathlineto{\pgfqpoint{2.241778in}{1.165693in}}%
\pgfpathlineto{\pgfqpoint{2.241205in}{1.447599in}}%
\pgfpathlineto{\pgfqpoint{2.242175in}{1.367070in}}%
\pgfpathlineto{\pgfqpoint{2.243122in}{1.480597in}}%
\pgfpathlineto{\pgfqpoint{2.242241in}{1.108656in}}%
\pgfpathlineto{\pgfqpoint{2.243211in}{1.326751in}}%
\pgfpathlineto{\pgfqpoint{2.243563in}{1.195522in}}%
\pgfpathlineto{\pgfqpoint{2.243916in}{1.399740in}}%
\pgfpathlineto{\pgfqpoint{2.244357in}{1.278783in}}%
\pgfpathlineto{\pgfqpoint{2.244555in}{1.109530in}}%
\pgfpathlineto{\pgfqpoint{2.245040in}{1.297140in}}%
\pgfpathlineto{\pgfqpoint{2.245216in}{1.252013in}}%
\pgfpathlineto{\pgfqpoint{2.245238in}{1.436454in}}%
\pgfpathlineto{\pgfqpoint{2.245789in}{1.097948in}}%
\pgfpathlineto{\pgfqpoint{2.246318in}{1.400177in}}%
\pgfpathlineto{\pgfqpoint{2.246914in}{1.100570in}}%
\pgfpathlineto{\pgfqpoint{2.247443in}{1.312000in}}%
\pgfpathlineto{\pgfqpoint{2.248258in}{1.501139in}}%
\pgfpathlineto{\pgfqpoint{2.247729in}{1.145697in}}%
\pgfpathlineto{\pgfqpoint{2.248479in}{1.288070in}}%
\pgfpathlineto{\pgfqpoint{2.249515in}{1.059814in}}%
\pgfpathlineto{\pgfqpoint{2.248589in}{1.420719in}}%
\pgfpathlineto{\pgfqpoint{2.249581in}{1.248407in}}%
\pgfpathlineto{\pgfqpoint{2.250132in}{1.414491in}}%
\pgfpathlineto{\pgfqpoint{2.249647in}{1.144276in}}%
\pgfpathlineto{\pgfqpoint{2.250749in}{1.340300in}}%
\pgfpathlineto{\pgfqpoint{2.251035in}{1.188092in}}%
\pgfpathlineto{\pgfqpoint{2.250859in}{1.545501in}}%
\pgfpathlineto{\pgfqpoint{2.251873in}{1.283809in}}%
\pgfpathlineto{\pgfqpoint{2.252821in}{1.417878in}}%
\pgfpathlineto{\pgfqpoint{2.252424in}{1.112043in}}%
\pgfpathlineto{\pgfqpoint{2.252909in}{1.225789in}}%
\pgfpathlineto{\pgfqpoint{2.253526in}{1.048450in}}%
\pgfpathlineto{\pgfqpoint{2.253107in}{1.374937in}}%
\pgfpathlineto{\pgfqpoint{2.253769in}{1.264032in}}%
\pgfpathlineto{\pgfqpoint{2.253989in}{1.293752in}}%
\pgfpathlineto{\pgfqpoint{2.254364in}{1.030749in}}%
\pgfpathlineto{\pgfqpoint{2.254761in}{1.136191in}}%
\pgfpathlineto{\pgfqpoint{2.254805in}{0.986278in}}%
\pgfpathlineto{\pgfqpoint{2.255708in}{1.265671in}}%
\pgfpathlineto{\pgfqpoint{2.255841in}{1.112152in}}%
\pgfpathlineto{\pgfqpoint{2.256061in}{1.092266in}}%
\pgfpathlineto{\pgfqpoint{2.256987in}{1.367835in}}%
\pgfpathlineto{\pgfqpoint{2.257450in}{1.066588in}}%
\pgfpathlineto{\pgfqpoint{2.258133in}{1.133459in}}%
\pgfpathlineto{\pgfqpoint{2.259191in}{1.256165in}}%
\pgfpathlineto{\pgfqpoint{2.259103in}{1.011300in}}%
\pgfpathlineto{\pgfqpoint{2.259235in}{1.110732in}}%
\pgfpathlineto{\pgfqpoint{2.259301in}{0.966829in}}%
\pgfpathlineto{\pgfqpoint{2.259522in}{1.306864in}}%
\pgfpathlineto{\pgfqpoint{2.260359in}{1.065933in}}%
\pgfpathlineto{\pgfqpoint{2.260381in}{1.065605in}}%
\pgfpathlineto{\pgfqpoint{2.260536in}{1.266982in}}%
\pgfpathlineto{\pgfqpoint{2.261395in}{0.930006in}}%
\pgfpathlineto{\pgfqpoint{2.261461in}{1.050308in}}%
\pgfpathlineto{\pgfqpoint{2.261748in}{0.983219in}}%
\pgfpathlineto{\pgfqpoint{2.261572in}{1.259989in}}%
\pgfpathlineto{\pgfqpoint{2.262475in}{1.153673in}}%
\pgfpathlineto{\pgfqpoint{2.262519in}{1.288617in}}%
\pgfpathlineto{\pgfqpoint{2.262630in}{1.009115in}}%
\pgfpathlineto{\pgfqpoint{2.263599in}{1.197270in}}%
\pgfpathlineto{\pgfqpoint{2.264040in}{1.433394in}}%
\pgfpathlineto{\pgfqpoint{2.263754in}{1.026925in}}%
\pgfpathlineto{\pgfqpoint{2.264679in}{1.233765in}}%
\pgfpathlineto{\pgfqpoint{2.265407in}{1.056208in}}%
\pgfpathlineto{\pgfqpoint{2.265561in}{1.313092in}}%
\pgfpathlineto{\pgfqpoint{2.265804in}{1.135644in}}%
\pgfpathlineto{\pgfqpoint{2.266156in}{1.305007in}}%
\pgfpathlineto{\pgfqpoint{2.266266in}{0.972620in}}%
\pgfpathlineto{\pgfqpoint{2.266884in}{1.159683in}}%
\pgfpathlineto{\pgfqpoint{2.266950in}{0.926182in}}%
\pgfpathlineto{\pgfqpoint{2.267611in}{1.325876in}}%
\pgfpathlineto{\pgfqpoint{2.267964in}{1.116523in}}%
\pgfpathlineto{\pgfqpoint{2.268427in}{1.375702in}}%
\pgfpathlineto{\pgfqpoint{2.268581in}{0.994364in}}%
\pgfpathlineto{\pgfqpoint{2.269088in}{1.284902in}}%
\pgfpathlineto{\pgfqpoint{2.269970in}{1.385426in}}%
\pgfpathlineto{\pgfqpoint{2.269661in}{1.040474in}}%
\pgfpathlineto{\pgfqpoint{2.270058in}{1.229395in}}%
\pgfpathlineto{\pgfqpoint{2.271094in}{1.002668in}}%
\pgfpathlineto{\pgfqpoint{2.270785in}{1.261082in}}%
\pgfpathlineto{\pgfqpoint{2.271182in}{1.174106in}}%
\pgfpathlineto{\pgfqpoint{2.271689in}{1.341611in}}%
\pgfpathlineto{\pgfqpoint{2.271314in}{1.011518in}}%
\pgfpathlineto{\pgfqpoint{2.272130in}{1.100789in}}%
\pgfpathlineto{\pgfqpoint{2.273166in}{1.005290in}}%
\pgfpathlineto{\pgfqpoint{2.272857in}{1.308503in}}%
\pgfpathlineto{\pgfqpoint{2.273210in}{1.197380in}}%
\pgfpathlineto{\pgfqpoint{2.273871in}{1.007476in}}%
\pgfpathlineto{\pgfqpoint{2.273562in}{1.296812in}}%
\pgfpathlineto{\pgfqpoint{2.274312in}{1.128542in}}%
\pgfpathlineto{\pgfqpoint{2.274510in}{1.335820in}}%
\pgfpathlineto{\pgfqpoint{2.274841in}{0.955793in}}%
\pgfpathlineto{\pgfqpoint{2.275370in}{1.150395in}}%
\pgfpathlineto{\pgfqpoint{2.275392in}{0.960163in}}%
\pgfpathlineto{\pgfqpoint{2.275700in}{1.290037in}}%
\pgfpathlineto{\pgfqpoint{2.276472in}{1.002996in}}%
\pgfpathlineto{\pgfqpoint{2.277486in}{1.228521in}}%
\pgfpathlineto{\pgfqpoint{2.276692in}{0.979285in}}%
\pgfpathlineto{\pgfqpoint{2.277596in}{1.212895in}}%
\pgfpathlineto{\pgfqpoint{2.277993in}{1.008022in}}%
\pgfpathlineto{\pgfqpoint{2.278346in}{1.337349in}}%
\pgfpathlineto{\pgfqpoint{2.278742in}{1.048887in}}%
\pgfpathlineto{\pgfqpoint{2.279227in}{0.982126in}}%
\pgfpathlineto{\pgfqpoint{2.279117in}{1.294408in}}%
\pgfpathlineto{\pgfqpoint{2.279602in}{1.162852in}}%
\pgfpathlineto{\pgfqpoint{2.279800in}{1.205575in}}%
\pgfpathlineto{\pgfqpoint{2.280329in}{0.958415in}}%
\pgfpathlineto{\pgfqpoint{2.280616in}{1.090955in}}%
\pgfpathlineto{\pgfqpoint{2.280946in}{0.917440in}}%
\pgfpathlineto{\pgfqpoint{2.280814in}{1.252668in}}%
\pgfpathlineto{\pgfqpoint{2.281718in}{1.098385in}}%
\pgfpathlineto{\pgfqpoint{2.282688in}{1.276488in}}%
\pgfpathlineto{\pgfqpoint{2.282027in}{1.005072in}}%
\pgfpathlineto{\pgfqpoint{2.282820in}{1.099368in}}%
\pgfpathlineto{\pgfqpoint{2.283040in}{1.255072in}}%
\pgfpathlineto{\pgfqpoint{2.283724in}{0.969997in}}%
\pgfpathlineto{\pgfqpoint{2.283966in}{1.195631in}}%
\pgfpathlineto{\pgfqpoint{2.284143in}{1.058175in}}%
\pgfpathlineto{\pgfqpoint{2.284826in}{1.293534in}}%
\pgfpathlineto{\pgfqpoint{2.285090in}{1.088114in}}%
\pgfpathlineto{\pgfqpoint{2.285950in}{1.286759in}}%
\pgfpathlineto{\pgfqpoint{2.285179in}{1.026706in}}%
\pgfpathlineto{\pgfqpoint{2.286192in}{1.182847in}}%
\pgfpathlineto{\pgfqpoint{2.286391in}{0.965408in}}%
\pgfpathlineto{\pgfqpoint{2.286744in}{1.339098in}}%
\pgfpathlineto{\pgfqpoint{2.287295in}{1.123407in}}%
\pgfpathlineto{\pgfqpoint{2.287581in}{1.243599in}}%
\pgfpathlineto{\pgfqpoint{2.287713in}{0.997642in}}%
\pgfpathlineto{\pgfqpoint{2.288397in}{1.154766in}}%
\pgfpathlineto{\pgfqpoint{2.288551in}{0.989447in}}%
\pgfpathlineto{\pgfqpoint{2.288771in}{1.319758in}}%
\pgfpathlineto{\pgfqpoint{2.289499in}{1.077843in}}%
\pgfpathlineto{\pgfqpoint{2.290380in}{1.241742in}}%
\pgfpathlineto{\pgfqpoint{2.289697in}{0.956230in}}%
\pgfpathlineto{\pgfqpoint{2.290645in}{1.181427in}}%
\pgfpathlineto{\pgfqpoint{2.290843in}{1.298123in}}%
\pgfpathlineto{\pgfqpoint{2.291394in}{1.009770in}}%
\pgfpathlineto{\pgfqpoint{2.291769in}{1.263049in}}%
\pgfpathlineto{\pgfqpoint{2.292078in}{1.033262in}}%
\pgfpathlineto{\pgfqpoint{2.292651in}{1.284683in}}%
\pgfpathlineto{\pgfqpoint{2.292871in}{1.211584in}}%
\pgfpathlineto{\pgfqpoint{2.293510in}{1.297467in}}%
\pgfpathlineto{\pgfqpoint{2.293400in}{1.024740in}}%
\pgfpathlineto{\pgfqpoint{2.293907in}{1.131602in}}%
\pgfpathlineto{\pgfqpoint{2.294061in}{1.034027in}}%
\pgfpathlineto{\pgfqpoint{2.294679in}{1.316480in}}%
\pgfpathlineto{\pgfqpoint{2.294943in}{1.276270in}}%
\pgfpathlineto{\pgfqpoint{2.295913in}{1.368490in}}%
\pgfpathlineto{\pgfqpoint{2.295582in}{1.092047in}}%
\pgfpathlineto{\pgfqpoint{2.295935in}{1.315059in}}%
\pgfpathlineto{\pgfqpoint{2.296684in}{1.126685in}}%
\pgfpathlineto{\pgfqpoint{2.296795in}{1.407826in}}%
\pgfpathlineto{\pgfqpoint{2.297037in}{1.213770in}}%
\pgfpathlineto{\pgfqpoint{2.298095in}{1.296265in}}%
\pgfpathlineto{\pgfqpoint{2.297544in}{0.984093in}}%
\pgfpathlineto{\pgfqpoint{2.298139in}{1.272773in}}%
\pgfpathlineto{\pgfqpoint{2.298227in}{1.055006in}}%
\pgfpathlineto{\pgfqpoint{2.298734in}{1.376248in}}%
\pgfpathlineto{\pgfqpoint{2.299285in}{1.082213in}}%
\pgfpathlineto{\pgfqpoint{2.300189in}{1.372205in}}%
\pgfpathlineto{\pgfqpoint{2.299792in}{1.055880in}}%
\pgfpathlineto{\pgfqpoint{2.300387in}{1.106033in}}%
\pgfpathlineto{\pgfqpoint{2.300696in}{1.346528in}}%
\pgfpathlineto{\pgfqpoint{2.300564in}{1.006492in}}%
\pgfpathlineto{\pgfqpoint{2.301534in}{1.240977in}}%
\pgfpathlineto{\pgfqpoint{2.301732in}{0.988463in}}%
\pgfpathlineto{\pgfqpoint{2.302129in}{1.307301in}}%
\pgfpathlineto{\pgfqpoint{2.302658in}{1.139687in}}%
\pgfpathlineto{\pgfqpoint{2.303143in}{1.276051in}}%
\pgfpathlineto{\pgfqpoint{2.303010in}{0.935797in}}%
\pgfpathlineto{\pgfqpoint{2.303650in}{1.082760in}}%
\pgfpathlineto{\pgfqpoint{2.304311in}{0.869254in}}%
\pgfpathlineto{\pgfqpoint{2.304509in}{1.208743in}}%
\pgfpathlineto{\pgfqpoint{2.304796in}{0.928149in}}%
\pgfpathlineto{\pgfqpoint{2.305104in}{0.897226in}}%
\pgfpathlineto{\pgfqpoint{2.304928in}{1.080574in}}%
\pgfpathlineto{\pgfqpoint{2.305215in}{1.061999in}}%
\pgfpathlineto{\pgfqpoint{2.306295in}{1.196943in}}%
\pgfpathlineto{\pgfqpoint{2.306052in}{0.899958in}}%
\pgfpathlineto{\pgfqpoint{2.306317in}{1.084836in}}%
\pgfpathlineto{\pgfqpoint{2.306647in}{0.945740in}}%
\pgfpathlineto{\pgfqpoint{2.306361in}{1.244255in}}%
\pgfpathlineto{\pgfqpoint{2.307353in}{1.009770in}}%
\pgfpathlineto{\pgfqpoint{2.307926in}{1.298451in}}%
\pgfpathlineto{\pgfqpoint{2.308367in}{0.978629in}}%
\pgfpathlineto{\pgfqpoint{2.308477in}{1.143839in}}%
\pgfpathlineto{\pgfqpoint{2.309028in}{0.938420in}}%
\pgfpathlineto{\pgfqpoint{2.309182in}{1.234749in}}%
\pgfpathlineto{\pgfqpoint{2.309535in}{1.054132in}}%
\pgfpathlineto{\pgfqpoint{2.309711in}{1.209071in}}%
\pgfpathlineto{\pgfqpoint{2.309866in}{0.914490in}}%
\pgfpathlineto{\pgfqpoint{2.310615in}{1.150505in}}%
\pgfpathlineto{\pgfqpoint{2.310879in}{0.837021in}}%
\pgfpathlineto{\pgfqpoint{2.311364in}{1.187983in}}%
\pgfpathlineto{\pgfqpoint{2.311717in}{1.029547in}}%
\pgfpathlineto{\pgfqpoint{2.312819in}{1.215409in}}%
\pgfpathlineto{\pgfqpoint{2.312378in}{0.873953in}}%
\pgfpathlineto{\pgfqpoint{2.312863in}{1.090845in}}%
\pgfpathlineto{\pgfqpoint{2.313304in}{0.856033in}}%
\pgfpathlineto{\pgfqpoint{2.313370in}{1.126029in}}%
\pgfpathlineto{\pgfqpoint{2.314098in}{0.994364in}}%
\pgfpathlineto{\pgfqpoint{2.314340in}{1.212240in}}%
\pgfpathlineto{\pgfqpoint{2.314957in}{0.850460in}}%
\pgfpathlineto{\pgfqpoint{2.315332in}{1.145915in}}%
\pgfpathlineto{\pgfqpoint{2.316412in}{0.851116in}}%
\pgfpathlineto{\pgfqpoint{2.316346in}{1.290802in}}%
\pgfpathlineto{\pgfqpoint{2.316456in}{1.044845in}}%
\pgfpathlineto{\pgfqpoint{2.316478in}{0.944320in}}%
\pgfpathlineto{\pgfqpoint{2.317206in}{1.247751in}}%
\pgfpathlineto{\pgfqpoint{2.317536in}{1.034464in}}%
\pgfpathlineto{\pgfqpoint{2.318374in}{1.287852in}}%
\pgfpathlineto{\pgfqpoint{2.318660in}{1.189950in}}%
\pgfpathlineto{\pgfqpoint{2.318903in}{1.367288in}}%
\pgfpathlineto{\pgfqpoint{2.319057in}{1.050526in}}%
\pgfpathlineto{\pgfqpoint{2.319674in}{1.186672in}}%
\pgfpathlineto{\pgfqpoint{2.320247in}{1.059923in}}%
\pgfpathlineto{\pgfqpoint{2.320402in}{1.366742in}}%
\pgfpathlineto{\pgfqpoint{2.320776in}{1.161322in}}%
\pgfpathlineto{\pgfqpoint{2.321526in}{1.323036in}}%
\pgfpathlineto{\pgfqpoint{2.321834in}{1.017419in}}%
\pgfpathlineto{\pgfqpoint{2.321856in}{1.087349in}}%
\pgfpathlineto{\pgfqpoint{2.322385in}{1.051728in}}%
\pgfpathlineto{\pgfqpoint{2.322253in}{1.284792in}}%
\pgfpathlineto{\pgfqpoint{2.322804in}{1.209945in}}%
\pgfpathlineto{\pgfqpoint{2.323620in}{1.322271in}}%
\pgfpathlineto{\pgfqpoint{2.323113in}{1.000920in}}%
\pgfpathlineto{\pgfqpoint{2.323884in}{1.294408in}}%
\pgfpathlineto{\pgfqpoint{2.324237in}{1.024849in}}%
\pgfpathlineto{\pgfqpoint{2.324964in}{1.348713in}}%
\pgfpathlineto{\pgfqpoint{2.325008in}{1.169845in}}%
\pgfpathlineto{\pgfqpoint{2.325185in}{1.268184in}}%
\pgfpathlineto{\pgfqpoint{2.325141in}{1.032497in}}%
\pgfpathlineto{\pgfqpoint{2.326133in}{1.185142in}}%
\pgfpathlineto{\pgfqpoint{2.326728in}{1.047248in}}%
\pgfpathlineto{\pgfqpoint{2.326397in}{1.279329in}}%
\pgfpathlineto{\pgfqpoint{2.327257in}{1.149521in}}%
\pgfpathlineto{\pgfqpoint{2.327653in}{1.260645in}}%
\pgfpathlineto{\pgfqpoint{2.328160in}{0.940496in}}%
\pgfpathlineto{\pgfqpoint{2.328293in}{1.104722in}}%
\pgfpathlineto{\pgfqpoint{2.328954in}{0.927930in}}%
\pgfpathlineto{\pgfqpoint{2.328866in}{1.194430in}}%
\pgfpathlineto{\pgfqpoint{2.329395in}{1.078717in}}%
\pgfpathlineto{\pgfqpoint{2.329858in}{0.911322in}}%
\pgfpathlineto{\pgfqpoint{2.330254in}{1.134661in}}%
\pgfpathlineto{\pgfqpoint{2.330563in}{0.969232in}}%
\pgfpathlineto{\pgfqpoint{2.331511in}{1.167222in}}%
\pgfpathlineto{\pgfqpoint{2.331026in}{0.850023in}}%
\pgfpathlineto{\pgfqpoint{2.331665in}{0.992725in}}%
\pgfpathlineto{\pgfqpoint{2.331731in}{0.864009in}}%
\pgfpathlineto{\pgfqpoint{2.331974in}{1.090190in}}%
\pgfpathlineto{\pgfqpoint{2.332811in}{0.908481in}}%
\pgfpathlineto{\pgfqpoint{2.333847in}{1.093686in}}%
\pgfpathlineto{\pgfqpoint{2.333561in}{0.858874in}}%
\pgfpathlineto{\pgfqpoint{2.334046in}{0.960273in}}%
\pgfpathlineto{\pgfqpoint{2.334663in}{1.195522in}}%
\pgfpathlineto{\pgfqpoint{2.334707in}{0.904219in}}%
\pgfpathlineto{\pgfqpoint{2.334839in}{0.947489in}}%
\pgfpathlineto{\pgfqpoint{2.334883in}{0.873953in}}%
\pgfpathlineto{\pgfqpoint{2.335060in}{1.155312in}}%
\pgfpathlineto{\pgfqpoint{2.335831in}{1.123079in}}%
\pgfpathlineto{\pgfqpoint{2.336757in}{1.286431in}}%
\pgfpathlineto{\pgfqpoint{2.336514in}{0.991851in}}%
\pgfpathlineto{\pgfqpoint{2.336933in}{1.168206in}}%
\pgfpathlineto{\pgfqpoint{2.337661in}{0.913398in}}%
\pgfpathlineto{\pgfqpoint{2.337220in}{1.177930in}}%
\pgfpathlineto{\pgfqpoint{2.338057in}{1.087021in}}%
\pgfpathlineto{\pgfqpoint{2.339093in}{0.894495in}}%
\pgfpathlineto{\pgfqpoint{2.338167in}{1.225024in}}%
\pgfpathlineto{\pgfqpoint{2.339181in}{1.009224in}}%
\pgfpathlineto{\pgfqpoint{2.339358in}{1.106798in}}%
\pgfpathlineto{\pgfqpoint{2.339270in}{0.848712in}}%
\pgfpathlineto{\pgfqpoint{2.340328in}{1.055116in}}%
\pgfpathlineto{\pgfqpoint{2.341430in}{1.228411in}}%
\pgfpathlineto{\pgfqpoint{2.340394in}{0.799324in}}%
\pgfpathlineto{\pgfqpoint{2.341474in}{1.180771in}}%
\pgfpathlineto{\pgfqpoint{2.342069in}{0.929897in}}%
\pgfpathlineto{\pgfqpoint{2.341540in}{1.226226in}}%
\pgfpathlineto{\pgfqpoint{2.342576in}{1.077187in}}%
\pgfpathlineto{\pgfqpoint{2.342620in}{1.207541in}}%
\pgfpathlineto{\pgfqpoint{2.342730in}{0.929569in}}%
\pgfpathlineto{\pgfqpoint{2.343656in}{1.064075in}}%
\pgfpathlineto{\pgfqpoint{2.344494in}{0.870565in}}%
\pgfpathlineto{\pgfqpoint{2.344361in}{1.220216in}}%
\pgfpathlineto{\pgfqpoint{2.344736in}{1.090081in}}%
\pgfpathlineto{\pgfqpoint{2.345045in}{1.101991in}}%
\pgfpathlineto{\pgfqpoint{2.344846in}{0.853957in}}%
\pgfpathlineto{\pgfqpoint{2.345111in}{0.930006in}}%
\pgfpathlineto{\pgfqpoint{2.345199in}{0.853083in}}%
\pgfpathlineto{\pgfqpoint{2.345684in}{1.178805in}}%
\pgfpathlineto{\pgfqpoint{2.346169in}{0.978848in}}%
\pgfpathlineto{\pgfqpoint{2.346852in}{1.187546in}}%
\pgfpathlineto{\pgfqpoint{2.347072in}{0.858437in}}%
\pgfpathlineto{\pgfqpoint{2.347271in}{0.986606in}}%
\pgfpathlineto{\pgfqpoint{2.347513in}{1.189403in}}%
\pgfpathlineto{\pgfqpoint{2.347425in}{0.913725in}}%
\pgfpathlineto{\pgfqpoint{2.348351in}{1.048450in}}%
\pgfpathlineto{\pgfqpoint{2.348880in}{0.835272in}}%
\pgfpathlineto{\pgfqpoint{2.349056in}{1.136082in}}%
\pgfpathlineto{\pgfqpoint{2.349475in}{0.866413in}}%
\pgfpathlineto{\pgfqpoint{2.350445in}{1.192900in}}%
\pgfpathlineto{\pgfqpoint{2.349916in}{0.772663in}}%
\pgfpathlineto{\pgfqpoint{2.350599in}{1.045172in}}%
\pgfpathlineto{\pgfqpoint{2.351040in}{1.262065in}}%
\pgfpathlineto{\pgfqpoint{2.351106in}{0.950985in}}%
\pgfpathlineto{\pgfqpoint{2.351767in}{1.190715in}}%
\pgfpathlineto{\pgfqpoint{2.352847in}{0.885316in}}%
\pgfpathlineto{\pgfqpoint{2.352473in}{1.222948in}}%
\pgfpathlineto{\pgfqpoint{2.352936in}{0.901925in}}%
\pgfpathlineto{\pgfqpoint{2.353134in}{1.165693in}}%
\pgfpathlineto{\pgfqpoint{2.353905in}{0.811999in}}%
\pgfpathlineto{\pgfqpoint{2.354082in}{1.001903in}}%
\pgfpathlineto{\pgfqpoint{2.354919in}{0.851444in}}%
\pgfpathlineto{\pgfqpoint{2.354214in}{1.116523in}}%
\pgfpathlineto{\pgfqpoint{2.355030in}{1.060470in}}%
\pgfpathlineto{\pgfqpoint{2.355052in}{1.202078in}}%
\pgfpathlineto{\pgfqpoint{2.355250in}{0.835928in}}%
\pgfpathlineto{\pgfqpoint{2.356110in}{0.866632in}}%
\pgfpathlineto{\pgfqpoint{2.356462in}{1.090845in}}%
\pgfpathlineto{\pgfqpoint{2.356330in}{0.851553in}}%
\pgfpathlineto{\pgfqpoint{2.357256in}{0.950111in}}%
\pgfpathlineto{\pgfqpoint{2.358204in}{0.840954in}}%
\pgfpathlineto{\pgfqpoint{2.357388in}{1.174325in}}%
\pgfpathlineto{\pgfqpoint{2.358358in}{0.970762in}}%
\pgfpathlineto{\pgfqpoint{2.359350in}{1.119364in}}%
\pgfpathlineto{\pgfqpoint{2.359041in}{0.865539in}}%
\pgfpathlineto{\pgfqpoint{2.359416in}{0.943118in}}%
\pgfpathlineto{\pgfqpoint{2.360055in}{0.831120in}}%
\pgfpathlineto{\pgfqpoint{2.359835in}{1.235404in}}%
\pgfpathlineto{\pgfqpoint{2.360452in}{0.918752in}}%
\pgfpathlineto{\pgfqpoint{2.360937in}{1.206667in}}%
\pgfpathlineto{\pgfqpoint{2.361554in}{1.023538in}}%
\pgfpathlineto{\pgfqpoint{2.361620in}{0.833743in}}%
\pgfpathlineto{\pgfqpoint{2.362127in}{1.176728in}}%
\pgfpathlineto{\pgfqpoint{2.362700in}{0.857016in}}%
\pgfpathlineto{\pgfqpoint{2.363229in}{1.061234in}}%
\pgfpathlineto{\pgfqpoint{2.363802in}{0.922467in}}%
\pgfpathlineto{\pgfqpoint{2.363824in}{0.856470in}}%
\pgfpathlineto{\pgfqpoint{2.364596in}{1.295391in}}%
\pgfpathlineto{\pgfqpoint{2.364816in}{1.196833in}}%
\pgfpathlineto{\pgfqpoint{2.364904in}{0.915911in}}%
\pgfpathlineto{\pgfqpoint{2.365720in}{1.303149in}}%
\pgfpathlineto{\pgfqpoint{2.366029in}{1.076750in}}%
\pgfpathlineto{\pgfqpoint{2.366932in}{1.362371in}}%
\pgfpathlineto{\pgfqpoint{2.366271in}{0.936999in}}%
\pgfpathlineto{\pgfqpoint{2.367131in}{1.117288in}}%
\pgfpathlineto{\pgfqpoint{2.367219in}{1.085382in}}%
\pgfpathlineto{\pgfqpoint{2.367285in}{1.259443in}}%
\pgfpathlineto{\pgfqpoint{2.368299in}{1.460492in}}%
\pgfpathlineto{\pgfqpoint{2.367792in}{1.090299in}}%
\pgfpathlineto{\pgfqpoint{2.368431in}{1.373189in}}%
\pgfpathlineto{\pgfqpoint{2.368982in}{1.174543in}}%
\pgfpathlineto{\pgfqpoint{2.369445in}{1.456449in}}%
\pgfpathlineto{\pgfqpoint{2.369599in}{1.203062in}}%
\pgfpathlineto{\pgfqpoint{2.370283in}{1.535886in}}%
\pgfpathlineto{\pgfqpoint{2.370723in}{1.336257in}}%
\pgfpathlineto{\pgfqpoint{2.370966in}{1.210055in}}%
\pgfpathlineto{\pgfqpoint{2.371715in}{1.510317in}}%
\pgfpathlineto{\pgfqpoint{2.371737in}{1.493600in}}%
\pgfpathlineto{\pgfqpoint{2.372421in}{1.593687in}}%
\pgfpathlineto{\pgfqpoint{2.371892in}{1.299325in}}%
\pgfpathlineto{\pgfqpoint{2.372729in}{1.419736in}}%
\pgfpathlineto{\pgfqpoint{2.373457in}{1.298232in}}%
\pgfpathlineto{\pgfqpoint{2.373236in}{1.622970in}}%
\pgfpathlineto{\pgfqpoint{2.373831in}{1.378980in}}%
\pgfpathlineto{\pgfqpoint{2.375022in}{1.678259in}}%
\pgfpathlineto{\pgfqpoint{2.373964in}{1.368381in}}%
\pgfpathlineto{\pgfqpoint{2.375044in}{1.662088in}}%
\pgfpathlineto{\pgfqpoint{2.375088in}{1.666458in}}%
\pgfpathlineto{\pgfqpoint{2.375176in}{1.528128in}}%
\pgfpathlineto{\pgfqpoint{2.375198in}{1.511082in}}%
\pgfpathlineto{\pgfqpoint{2.375881in}{1.733329in}}%
\pgfpathlineto{\pgfqpoint{2.376190in}{1.545829in}}%
\pgfpathlineto{\pgfqpoint{2.377182in}{1.872971in}}%
\pgfpathlineto{\pgfqpoint{2.376234in}{1.447926in}}%
\pgfpathlineto{\pgfqpoint{2.377336in}{1.669955in}}%
\pgfpathlineto{\pgfqpoint{2.378108in}{1.610296in}}%
\pgfpathlineto{\pgfqpoint{2.378041in}{1.874173in}}%
\pgfpathlineto{\pgfqpoint{2.378372in}{1.791131in}}%
\pgfpathlineto{\pgfqpoint{2.379055in}{1.843141in}}%
\pgfpathlineto{\pgfqpoint{2.378967in}{1.604942in}}%
\pgfpathlineto{\pgfqpoint{2.379121in}{1.700003in}}%
\pgfpathlineto{\pgfqpoint{2.379518in}{1.854068in}}%
\pgfpathlineto{\pgfqpoint{2.380268in}{1.471746in}}%
\pgfpathlineto{\pgfqpoint{2.381039in}{1.885755in}}%
\pgfpathlineto{\pgfqpoint{2.381392in}{1.648429in}}%
\pgfpathlineto{\pgfqpoint{2.382097in}{1.858439in}}%
\pgfpathlineto{\pgfqpoint{2.381458in}{1.466611in}}%
\pgfpathlineto{\pgfqpoint{2.382472in}{1.737481in}}%
\pgfpathlineto{\pgfqpoint{2.382494in}{1.584837in}}%
\pgfpathlineto{\pgfqpoint{2.382869in}{1.826861in}}%
\pgfpathlineto{\pgfqpoint{2.383574in}{1.688093in}}%
\pgfpathlineto{\pgfqpoint{2.384169in}{1.871988in}}%
\pgfpathlineto{\pgfqpoint{2.383662in}{1.602866in}}%
\pgfpathlineto{\pgfqpoint{2.384742in}{1.782171in}}%
\pgfpathlineto{\pgfqpoint{2.385183in}{1.917879in}}%
\pgfpathlineto{\pgfqpoint{2.385844in}{1.611170in}}%
\pgfpathlineto{\pgfqpoint{2.386990in}{1.930882in}}%
\pgfpathlineto{\pgfqpoint{2.387652in}{1.695195in}}%
\pgfpathlineto{\pgfqpoint{2.387718in}{1.995567in}}%
\pgfpathlineto{\pgfqpoint{2.388115in}{1.845545in}}%
\pgfpathlineto{\pgfqpoint{2.389107in}{2.026599in}}%
\pgfpathlineto{\pgfqpoint{2.388181in}{1.749937in}}%
\pgfpathlineto{\pgfqpoint{2.389283in}{1.973277in}}%
\pgfpathlineto{\pgfqpoint{2.390076in}{2.097075in}}%
\pgfpathlineto{\pgfqpoint{2.389613in}{1.852429in}}%
\pgfpathlineto{\pgfqpoint{2.390142in}{1.958526in}}%
\pgfpathlineto{\pgfqpoint{2.390231in}{1.824566in}}%
\pgfpathlineto{\pgfqpoint{2.390980in}{2.141328in}}%
\pgfpathlineto{\pgfqpoint{2.391223in}{2.095655in}}%
\pgfpathlineto{\pgfqpoint{2.392369in}{1.819758in}}%
\pgfpathlineto{\pgfqpoint{2.391333in}{2.154768in}}%
\pgfpathlineto{\pgfqpoint{2.392435in}{1.886192in}}%
\pgfpathlineto{\pgfqpoint{2.393405in}{2.168972in}}%
\pgfpathlineto{\pgfqpoint{2.393052in}{1.731472in}}%
\pgfpathlineto{\pgfqpoint{2.393537in}{1.943775in}}%
\pgfpathlineto{\pgfqpoint{2.394066in}{1.732236in}}%
\pgfpathlineto{\pgfqpoint{2.393890in}{1.999610in}}%
\pgfpathlineto{\pgfqpoint{2.394639in}{1.940716in}}%
\pgfpathlineto{\pgfqpoint{2.395080in}{1.795501in}}%
\pgfpathlineto{\pgfqpoint{2.395609in}{2.097185in}}%
\pgfpathlineto{\pgfqpoint{2.395675in}{2.026052in}}%
\pgfpathlineto{\pgfqpoint{2.396116in}{2.148212in}}%
\pgfpathlineto{\pgfqpoint{2.396446in}{1.943338in}}%
\pgfpathlineto{\pgfqpoint{2.396777in}{2.106691in}}%
\pgfpathlineto{\pgfqpoint{2.397438in}{1.914601in}}%
\pgfpathlineto{\pgfqpoint{2.396865in}{2.207106in}}%
\pgfpathlineto{\pgfqpoint{2.397901in}{2.011192in}}%
\pgfpathlineto{\pgfqpoint{2.398276in}{1.890235in}}%
\pgfpathlineto{\pgfqpoint{2.399003in}{2.186892in}}%
\pgfpathlineto{\pgfqpoint{2.400017in}{1.841939in}}%
\pgfpathlineto{\pgfqpoint{2.400150in}{1.893841in}}%
\pgfpathlineto{\pgfqpoint{2.400458in}{1.858876in}}%
\pgfpathlineto{\pgfqpoint{2.401296in}{2.084728in}}%
\pgfpathlineto{\pgfqpoint{2.401692in}{1.797905in}}%
\pgfpathlineto{\pgfqpoint{2.402045in}{2.177167in}}%
\pgfpathlineto{\pgfqpoint{2.402464in}{1.992071in}}%
\pgfpathlineto{\pgfqpoint{2.403456in}{2.115432in}}%
\pgfpathlineto{\pgfqpoint{2.403280in}{1.846856in}}%
\pgfpathlineto{\pgfqpoint{2.403566in}{2.081669in}}%
\pgfpathlineto{\pgfqpoint{2.403610in}{1.857127in}}%
\pgfpathlineto{\pgfqpoint{2.404029in}{2.138487in}}%
\pgfpathlineto{\pgfqpoint{2.404690in}{1.978522in}}%
\pgfpathlineto{\pgfqpoint{2.405550in}{1.872425in}}%
\pgfpathlineto{\pgfqpoint{2.405175in}{2.159794in}}%
\pgfpathlineto{\pgfqpoint{2.405704in}{2.060143in}}%
\pgfpathlineto{\pgfqpoint{2.406233in}{2.185471in}}%
\pgfpathlineto{\pgfqpoint{2.405880in}{1.939623in}}%
\pgfpathlineto{\pgfqpoint{2.406762in}{1.963115in}}%
\pgfpathlineto{\pgfqpoint{2.406828in}{1.941808in}}%
\pgfpathlineto{\pgfqpoint{2.406961in}{1.992726in}}%
\pgfpathlineto{\pgfqpoint{2.407512in}{2.164602in}}%
\pgfpathlineto{\pgfqpoint{2.407401in}{1.818119in}}%
\pgfpathlineto{\pgfqpoint{2.407864in}{2.004855in}}%
\pgfpathlineto{\pgfqpoint{2.407886in}{1.813312in}}%
\pgfpathlineto{\pgfqpoint{2.408702in}{2.144497in}}%
\pgfpathlineto{\pgfqpoint{2.408966in}{1.889142in}}%
\pgfpathlineto{\pgfqpoint{2.409495in}{2.127014in}}%
\pgfpathlineto{\pgfqpoint{2.409672in}{1.835165in}}%
\pgfpathlineto{\pgfqpoint{2.410157in}{2.084619in}}%
\pgfpathlineto{\pgfqpoint{2.410531in}{1.906516in}}%
\pgfpathlineto{\pgfqpoint{2.410223in}{2.203610in}}%
\pgfpathlineto{\pgfqpoint{2.411259in}{2.086258in}}%
\pgfpathlineto{\pgfqpoint{2.411545in}{2.200113in}}%
\pgfpathlineto{\pgfqpoint{2.411413in}{1.915148in}}%
\pgfpathlineto{\pgfqpoint{2.412339in}{1.992617in}}%
\pgfpathlineto{\pgfqpoint{2.413353in}{1.899413in}}%
\pgfpathlineto{\pgfqpoint{2.412537in}{2.171704in}}%
\pgfpathlineto{\pgfqpoint{2.413419in}{2.008351in}}%
\pgfpathlineto{\pgfqpoint{2.414080in}{2.139798in}}%
\pgfpathlineto{\pgfqpoint{2.413705in}{1.947381in}}%
\pgfpathlineto{\pgfqpoint{2.414543in}{2.058941in}}%
\pgfpathlineto{\pgfqpoint{2.415314in}{2.131713in}}%
\pgfpathlineto{\pgfqpoint{2.415006in}{1.875703in}}%
\pgfpathlineto{\pgfqpoint{2.415645in}{2.053478in}}%
\pgfpathlineto{\pgfqpoint{2.416483in}{2.179243in}}%
\pgfpathlineto{\pgfqpoint{2.415888in}{1.914601in}}%
\pgfpathlineto{\pgfqpoint{2.416659in}{2.130401in}}%
\pgfpathlineto{\pgfqpoint{2.416968in}{1.926183in}}%
\pgfpathlineto{\pgfqpoint{2.417453in}{2.274414in}}%
\pgfpathlineto{\pgfqpoint{2.417761in}{2.127123in}}%
\pgfpathlineto{\pgfqpoint{2.418511in}{1.986935in}}%
\pgfpathlineto{\pgfqpoint{2.418731in}{2.358549in}}%
\pgfpathlineto{\pgfqpoint{2.418885in}{2.029549in}}%
\pgfpathlineto{\pgfqpoint{2.419811in}{2.262285in}}%
\pgfpathlineto{\pgfqpoint{2.419502in}{1.993163in}}%
\pgfpathlineto{\pgfqpoint{2.420031in}{2.162089in}}%
\pgfpathlineto{\pgfqpoint{2.420891in}{1.965082in}}%
\pgfpathlineto{\pgfqpoint{2.420560in}{2.275507in}}%
\pgfpathlineto{\pgfqpoint{2.421111in}{2.159794in}}%
\pgfpathlineto{\pgfqpoint{2.421310in}{2.213553in}}%
\pgfpathlineto{\pgfqpoint{2.421663in}{1.954702in}}%
\pgfpathlineto{\pgfqpoint{2.422125in}{2.089973in}}%
\pgfpathlineto{\pgfqpoint{2.422147in}{1.958089in}}%
\pgfpathlineto{\pgfqpoint{2.422588in}{2.286105in}}%
\pgfpathlineto{\pgfqpoint{2.423205in}{2.161761in}}%
\pgfpathlineto{\pgfqpoint{2.423977in}{2.068229in}}%
\pgfpathlineto{\pgfqpoint{2.423779in}{2.345655in}}%
\pgfpathlineto{\pgfqpoint{2.424197in}{2.180773in}}%
\pgfpathlineto{\pgfqpoint{2.425277in}{2.275397in}}%
\pgfpathlineto{\pgfqpoint{2.424726in}{2.031188in}}%
\pgfpathlineto{\pgfqpoint{2.425299in}{2.205467in}}%
\pgfpathlineto{\pgfqpoint{2.425652in}{2.329156in}}%
\pgfpathlineto{\pgfqpoint{2.425520in}{2.103959in}}%
\pgfpathlineto{\pgfqpoint{2.425696in}{2.194978in}}%
\pgfpathlineto{\pgfqpoint{2.426027in}{2.047250in}}%
\pgfpathlineto{\pgfqpoint{2.426688in}{2.350244in}}%
\pgfpathlineto{\pgfqpoint{2.426798in}{2.124392in}}%
\pgfpathlineto{\pgfqpoint{2.427856in}{2.333964in}}%
\pgfpathlineto{\pgfqpoint{2.427570in}{1.993054in}}%
\pgfpathlineto{\pgfqpoint{2.427944in}{2.209291in}}%
\pgfpathlineto{\pgfqpoint{2.428231in}{2.047359in}}%
\pgfpathlineto{\pgfqpoint{2.428914in}{2.234969in}}%
\pgfpathlineto{\pgfqpoint{2.428958in}{2.210384in}}%
\pgfpathlineto{\pgfqpoint{2.428980in}{2.316809in}}%
\pgfpathlineto{\pgfqpoint{2.429884in}{2.016328in}}%
\pgfpathlineto{\pgfqpoint{2.430061in}{2.264034in}}%
\pgfpathlineto{\pgfqpoint{2.430722in}{2.083089in}}%
\pgfpathlineto{\pgfqpoint{2.430413in}{2.330249in}}%
\pgfpathlineto{\pgfqpoint{2.431207in}{2.132915in}}%
\pgfpathlineto{\pgfqpoint{2.432132in}{2.268186in}}%
\pgfpathlineto{\pgfqpoint{2.432265in}{1.985733in}}%
\pgfpathlineto{\pgfqpoint{2.432287in}{2.103194in}}%
\pgfpathlineto{\pgfqpoint{2.432419in}{2.064951in}}%
\pgfpathlineto{\pgfqpoint{2.432661in}{2.392312in}}%
\pgfpathlineto{\pgfqpoint{2.433146in}{2.200332in}}%
\pgfpathlineto{\pgfqpoint{2.434271in}{2.393951in}}%
\pgfpathlineto{\pgfqpoint{2.433455in}{2.097075in}}%
\pgfpathlineto{\pgfqpoint{2.434293in}{2.315716in}}%
\pgfpathlineto{\pgfqpoint{2.434822in}{2.412417in}}%
\pgfpathlineto{\pgfqpoint{2.435439in}{2.045720in}}%
\pgfpathlineto{\pgfqpoint{2.435461in}{2.296923in}}%
\pgfpathlineto{\pgfqpoint{2.436475in}{2.012176in}}%
\pgfpathlineto{\pgfqpoint{2.436563in}{2.226992in}}%
\pgfpathlineto{\pgfqpoint{2.436982in}{2.400070in}}%
\pgfpathlineto{\pgfqpoint{2.437158in}{2.100790in}}%
\pgfpathlineto{\pgfqpoint{2.437665in}{2.236826in}}%
\pgfpathlineto{\pgfqpoint{2.438635in}{2.056101in}}%
\pgfpathlineto{\pgfqpoint{2.437841in}{2.273868in}}%
\pgfpathlineto{\pgfqpoint{2.438789in}{2.105598in}}%
\pgfpathlineto{\pgfqpoint{2.439010in}{2.366307in}}%
\pgfpathlineto{\pgfqpoint{2.439891in}{2.068885in}}%
\pgfpathlineto{\pgfqpoint{2.439935in}{2.365651in}}%
\pgfpathlineto{\pgfqpoint{2.440795in}{2.043644in}}%
\pgfpathlineto{\pgfqpoint{2.441037in}{2.384445in}}%
\pgfpathlineto{\pgfqpoint{2.441059in}{2.228631in}}%
\pgfpathlineto{\pgfqpoint{2.441897in}{2.350900in}}%
\pgfpathlineto{\pgfqpoint{2.441500in}{2.069103in}}%
\pgfpathlineto{\pgfqpoint{2.442162in}{2.304025in}}%
\pgfpathlineto{\pgfqpoint{2.442580in}{2.140891in}}%
\pgfpathlineto{\pgfqpoint{2.443021in}{2.470546in}}%
\pgfpathlineto{\pgfqpoint{2.443242in}{2.228194in}}%
\pgfpathlineto{\pgfqpoint{2.444388in}{2.501468in}}%
\pgfpathlineto{\pgfqpoint{2.444476in}{2.175637in}}%
\pgfpathlineto{\pgfqpoint{2.445512in}{2.392093in}}%
\pgfpathlineto{\pgfqpoint{2.446261in}{2.571399in}}%
\pgfpathlineto{\pgfqpoint{2.446328in}{2.239886in}}%
\pgfpathlineto{\pgfqpoint{2.446614in}{2.485953in}}%
\pgfpathlineto{\pgfqpoint{2.446923in}{2.229287in}}%
\pgfpathlineto{\pgfqpoint{2.447760in}{2.313859in}}%
\pgfpathlineto{\pgfqpoint{2.447782in}{2.315279in}}%
\pgfpathlineto{\pgfqpoint{2.447804in}{2.259444in}}%
\pgfpathlineto{\pgfqpoint{2.448444in}{2.130620in}}%
\pgfpathlineto{\pgfqpoint{2.448091in}{2.461914in}}%
\pgfpathlineto{\pgfqpoint{2.448884in}{2.287089in}}%
\pgfpathlineto{\pgfqpoint{2.449061in}{2.502015in}}%
\pgfpathlineto{\pgfqpoint{2.449744in}{2.199785in}}%
\pgfpathlineto{\pgfqpoint{2.449986in}{2.374174in}}%
\pgfpathlineto{\pgfqpoint{2.451045in}{2.210493in}}%
\pgfpathlineto{\pgfqpoint{2.450956in}{2.495240in}}%
\pgfpathlineto{\pgfqpoint{2.451111in}{2.268732in}}%
\pgfpathlineto{\pgfqpoint{2.451992in}{2.460822in}}%
\pgfpathlineto{\pgfqpoint{2.451199in}{2.180773in}}%
\pgfpathlineto{\pgfqpoint{2.452235in}{2.391001in}}%
\pgfpathlineto{\pgfqpoint{2.453227in}{2.122206in}}%
\pgfpathlineto{\pgfqpoint{2.452720in}{2.511739in}}%
\pgfpathlineto{\pgfqpoint{2.453381in}{2.210603in}}%
\pgfpathlineto{\pgfqpoint{2.453910in}{2.374720in}}%
\pgfpathlineto{\pgfqpoint{2.454417in}{2.006494in}}%
\pgfpathlineto{\pgfqpoint{2.454483in}{2.290039in}}%
\pgfpathlineto{\pgfqpoint{2.455387in}{2.074348in}}%
\pgfpathlineto{\pgfqpoint{2.455519in}{2.362264in}}%
\pgfpathlineto{\pgfqpoint{2.455585in}{2.281188in}}%
\pgfpathlineto{\pgfqpoint{2.456687in}{2.408374in}}%
\pgfpathlineto{\pgfqpoint{2.455916in}{2.107456in}}%
\pgfpathlineto{\pgfqpoint{2.456709in}{2.333527in}}%
\pgfpathlineto{\pgfqpoint{2.457525in}{2.106909in}}%
\pgfpathlineto{\pgfqpoint{2.457657in}{2.480599in}}%
\pgfpathlineto{\pgfqpoint{2.457789in}{2.229287in}}%
\pgfpathlineto{\pgfqpoint{2.458847in}{2.482565in}}%
\pgfpathlineto{\pgfqpoint{2.457988in}{2.166787in}}%
\pgfpathlineto{\pgfqpoint{2.458891in}{2.302714in}}%
\pgfpathlineto{\pgfqpoint{2.459707in}{2.106909in}}%
\pgfpathlineto{\pgfqpoint{2.459531in}{2.449567in}}%
\pgfpathlineto{\pgfqpoint{2.459949in}{2.345546in}}%
\pgfpathlineto{\pgfqpoint{2.460390in}{2.448365in}}%
\pgfpathlineto{\pgfqpoint{2.460148in}{2.147556in}}%
\pgfpathlineto{\pgfqpoint{2.460853in}{2.242399in}}%
\pgfpathlineto{\pgfqpoint{2.461470in}{2.139798in}}%
\pgfpathlineto{\pgfqpoint{2.461382in}{2.417006in}}%
\pgfpathlineto{\pgfqpoint{2.461955in}{2.241416in}}%
\pgfpathlineto{\pgfqpoint{2.462925in}{2.456779in}}%
\pgfpathlineto{\pgfqpoint{2.462396in}{2.070742in}}%
\pgfpathlineto{\pgfqpoint{2.463057in}{2.230270in}}%
\pgfpathlineto{\pgfqpoint{2.464270in}{2.432740in}}%
\pgfpathlineto{\pgfqpoint{2.463741in}{2.189405in}}%
\pgfpathlineto{\pgfqpoint{2.464292in}{2.347513in}}%
\pgfpathlineto{\pgfqpoint{2.465438in}{2.171376in}}%
\pgfpathlineto{\pgfqpoint{2.465041in}{2.431648in}}%
\pgfpathlineto{\pgfqpoint{2.465460in}{2.213006in}}%
\pgfpathlineto{\pgfqpoint{2.465835in}{2.468689in}}%
\pgfpathlineto{\pgfqpoint{2.466474in}{2.114121in}}%
\pgfpathlineto{\pgfqpoint{2.466584in}{2.406189in}}%
\pgfpathlineto{\pgfqpoint{2.466827in}{2.015344in}}%
\pgfpathlineto{\pgfqpoint{2.467708in}{2.267639in}}%
\pgfpathlineto{\pgfqpoint{2.467818in}{2.072709in}}%
\pgfpathlineto{\pgfqpoint{2.468524in}{2.397666in}}%
\pgfpathlineto{\pgfqpoint{2.468832in}{2.240541in}}%
\pgfpathlineto{\pgfqpoint{2.468965in}{2.430992in}}%
\pgfpathlineto{\pgfqpoint{2.469075in}{2.166678in}}%
\pgfpathlineto{\pgfqpoint{2.469957in}{2.273649in}}%
\pgfpathlineto{\pgfqpoint{2.471037in}{2.114339in}}%
\pgfpathlineto{\pgfqpoint{2.470089in}{2.440607in}}%
\pgfpathlineto{\pgfqpoint{2.471059in}{2.218251in}}%
\pgfpathlineto{\pgfqpoint{2.471477in}{2.178806in}}%
\pgfpathlineto{\pgfqpoint{2.472205in}{2.464427in}}%
\pgfpathlineto{\pgfqpoint{2.472976in}{2.203391in}}%
\pgfpathlineto{\pgfqpoint{2.472359in}{2.515782in}}%
\pgfpathlineto{\pgfqpoint{2.473329in}{2.318776in}}%
\pgfpathlineto{\pgfqpoint{2.474365in}{2.552823in}}%
\pgfpathlineto{\pgfqpoint{2.474145in}{2.205358in}}%
\pgfpathlineto{\pgfqpoint{2.474541in}{2.504965in}}%
\pgfpathlineto{\pgfqpoint{2.475004in}{2.279659in}}%
\pgfpathlineto{\pgfqpoint{2.475621in}{2.543864in}}%
\pgfpathlineto{\pgfqpoint{2.475643in}{2.508024in}}%
\pgfpathlineto{\pgfqpoint{2.475665in}{2.519934in}}%
\pgfpathlineto{\pgfqpoint{2.476128in}{2.183395in}}%
\pgfpathlineto{\pgfqpoint{2.476327in}{2.155970in}}%
\pgfpathlineto{\pgfqpoint{2.476437in}{2.379200in}}%
\pgfpathlineto{\pgfqpoint{2.477120in}{2.307849in}}%
\pgfpathlineto{\pgfqpoint{2.477980in}{2.468361in}}%
\pgfpathlineto{\pgfqpoint{2.477186in}{2.192355in}}%
\pgfpathlineto{\pgfqpoint{2.478222in}{2.303916in}}%
\pgfpathlineto{\pgfqpoint{2.478994in}{2.098496in}}%
\pgfpathlineto{\pgfqpoint{2.478377in}{2.390454in}}%
\pgfpathlineto{\pgfqpoint{2.479302in}{2.314733in}}%
\pgfpathlineto{\pgfqpoint{2.480382in}{2.402801in}}%
\pgfpathlineto{\pgfqpoint{2.479920in}{2.119912in}}%
\pgfpathlineto{\pgfqpoint{2.480404in}{2.387395in}}%
\pgfpathlineto{\pgfqpoint{2.480537in}{2.091393in}}%
\pgfpathlineto{\pgfqpoint{2.481507in}{2.239777in}}%
\pgfpathlineto{\pgfqpoint{2.482432in}{2.430336in}}%
\pgfpathlineto{\pgfqpoint{2.481749in}{2.123736in}}%
\pgfpathlineto{\pgfqpoint{2.482609in}{2.379200in}}%
\pgfpathlineto{\pgfqpoint{2.483380in}{2.153566in}}%
\pgfpathlineto{\pgfqpoint{2.483049in}{2.415476in}}%
\pgfpathlineto{\pgfqpoint{2.483733in}{2.280861in}}%
\pgfpathlineto{\pgfqpoint{2.484637in}{2.190170in}}%
\pgfpathlineto{\pgfqpoint{2.484328in}{2.451425in}}%
\pgfpathlineto{\pgfqpoint{2.484747in}{2.267749in}}%
\pgfpathlineto{\pgfqpoint{2.485232in}{2.452190in}}%
\pgfpathlineto{\pgfqpoint{2.485827in}{2.211477in}}%
\pgfpathlineto{\pgfqpoint{2.486444in}{2.362264in}}%
\pgfpathlineto{\pgfqpoint{2.486069in}{2.056210in}}%
\pgfpathlineto{\pgfqpoint{2.486973in}{2.252779in}}%
\pgfpathlineto{\pgfqpoint{2.487348in}{2.170393in}}%
\pgfpathlineto{\pgfqpoint{2.487634in}{2.481473in}}%
\pgfpathlineto{\pgfqpoint{2.488053in}{2.282718in}}%
\pgfpathlineto{\pgfqpoint{2.488758in}{2.399960in}}%
\pgfpathlineto{\pgfqpoint{2.488340in}{2.171267in}}%
\pgfpathlineto{\pgfqpoint{2.488847in}{2.191263in}}%
\pgfpathlineto{\pgfqpoint{2.488869in}{2.098059in}}%
\pgfpathlineto{\pgfqpoint{2.489640in}{2.314733in}}%
\pgfpathlineto{\pgfqpoint{2.489905in}{2.282063in}}%
\pgfpathlineto{\pgfqpoint{2.490345in}{2.440170in}}%
\pgfpathlineto{\pgfqpoint{2.490015in}{2.161761in}}%
\pgfpathlineto{\pgfqpoint{2.491007in}{2.365105in}}%
\pgfpathlineto{\pgfqpoint{2.491139in}{2.185034in}}%
\pgfpathlineto{\pgfqpoint{2.491822in}{2.473933in}}%
\pgfpathlineto{\pgfqpoint{2.492087in}{2.393295in}}%
\pgfpathlineto{\pgfqpoint{2.492109in}{2.465192in}}%
\pgfpathlineto{\pgfqpoint{2.493101in}{2.187547in}}%
\pgfpathlineto{\pgfqpoint{2.493167in}{2.401162in}}%
\pgfpathlineto{\pgfqpoint{2.494026in}{2.139361in}}%
\pgfpathlineto{\pgfqpoint{2.493387in}{2.427495in}}%
\pgfpathlineto{\pgfqpoint{2.494313in}{2.188203in}}%
\pgfpathlineto{\pgfqpoint{2.494379in}{2.138706in}}%
\pgfpathlineto{\pgfqpoint{2.494974in}{2.402801in}}%
\pgfpathlineto{\pgfqpoint{2.495261in}{2.396027in}}%
\pgfpathlineto{\pgfqpoint{2.496231in}{2.074129in}}%
\pgfpathlineto{\pgfqpoint{2.496539in}{2.199130in}}%
\pgfpathlineto{\pgfqpoint{2.497443in}{2.491088in}}%
\pgfpathlineto{\pgfqpoint{2.496671in}{2.101446in}}%
\pgfpathlineto{\pgfqpoint{2.497685in}{2.375594in}}%
\pgfpathlineto{\pgfqpoint{2.498258in}{2.168863in}}%
\pgfpathlineto{\pgfqpoint{2.498038in}{2.470874in}}%
\pgfpathlineto{\pgfqpoint{2.498655in}{2.406079in}}%
\pgfpathlineto{\pgfqpoint{2.499228in}{2.271792in}}%
\pgfpathlineto{\pgfqpoint{2.499757in}{2.512395in}}%
\pgfpathlineto{\pgfqpoint{2.500463in}{2.538073in}}%
\pgfpathlineto{\pgfqpoint{2.500881in}{2.239558in}}%
\pgfpathlineto{\pgfqpoint{2.501234in}{2.464646in}}%
\pgfpathlineto{\pgfqpoint{2.501587in}{2.153129in}}%
\pgfpathlineto{\pgfqpoint{2.502006in}{2.376687in}}%
\pgfpathlineto{\pgfqpoint{2.502226in}{2.103741in}}%
\pgfpathlineto{\pgfqpoint{2.502975in}{2.486827in}}%
\pgfpathlineto{\pgfqpoint{2.503130in}{2.292443in}}%
\pgfpathlineto{\pgfqpoint{2.503504in}{2.451534in}}%
\pgfpathlineto{\pgfqpoint{2.503394in}{2.140563in}}%
\pgfpathlineto{\pgfqpoint{2.504254in}{2.424108in}}%
\pgfpathlineto{\pgfqpoint{2.504871in}{2.186455in}}%
\pgfpathlineto{\pgfqpoint{2.505334in}{2.482238in}}%
\pgfpathlineto{\pgfqpoint{2.505400in}{2.252342in}}%
\pgfpathlineto{\pgfqpoint{2.506149in}{2.531735in}}%
\pgfpathlineto{\pgfqpoint{2.506238in}{2.223059in}}%
\pgfpathlineto{\pgfqpoint{2.506524in}{2.352976in}}%
\pgfpathlineto{\pgfqpoint{2.507163in}{2.558396in}}%
\pgfpathlineto{\pgfqpoint{2.507406in}{2.176402in}}%
\pgfpathlineto{\pgfqpoint{2.507692in}{2.457544in}}%
\pgfpathlineto{\pgfqpoint{2.508486in}{2.172250in}}%
\pgfpathlineto{\pgfqpoint{2.508023in}{2.504965in}}%
\pgfpathlineto{\pgfqpoint{2.508817in}{2.388378in}}%
\pgfpathlineto{\pgfqpoint{2.508839in}{2.388488in}}%
\pgfpathlineto{\pgfqpoint{2.509676in}{2.508789in}}%
\pgfpathlineto{\pgfqpoint{2.509324in}{2.192137in}}%
\pgfpathlineto{\pgfqpoint{2.509985in}{2.476228in}}%
\pgfpathlineto{\pgfqpoint{2.511087in}{2.290476in}}%
\pgfpathlineto{\pgfqpoint{2.510271in}{2.578610in}}%
\pgfpathlineto{\pgfqpoint{2.511153in}{2.369585in}}%
\pgfpathlineto{\pgfqpoint{2.512233in}{2.231691in}}%
\pgfpathlineto{\pgfqpoint{2.511748in}{2.494257in}}%
\pgfpathlineto{\pgfqpoint{2.512299in}{2.321945in}}%
\pgfpathlineto{\pgfqpoint{2.513269in}{2.451534in}}%
\pgfpathlineto{\pgfqpoint{2.512872in}{2.191481in}}%
\pgfpathlineto{\pgfqpoint{2.513423in}{2.394606in}}%
\pgfpathlineto{\pgfqpoint{2.513864in}{2.158483in}}%
\pgfpathlineto{\pgfqpoint{2.513974in}{2.481473in}}%
\pgfpathlineto{\pgfqpoint{2.514547in}{2.291569in}}%
\pgfpathlineto{\pgfqpoint{2.514768in}{2.541023in}}%
\pgfpathlineto{\pgfqpoint{2.515319in}{2.243164in}}%
\pgfpathlineto{\pgfqpoint{2.515650in}{2.393077in}}%
\pgfpathlineto{\pgfqpoint{2.516774in}{2.186455in}}%
\pgfpathlineto{\pgfqpoint{2.516553in}{2.548016in}}%
\pgfpathlineto{\pgfqpoint{2.516840in}{2.307849in}}%
\pgfpathlineto{\pgfqpoint{2.517699in}{2.717050in}}%
\pgfpathlineto{\pgfqpoint{2.517082in}{2.301184in}}%
\pgfpathlineto{\pgfqpoint{2.517986in}{2.470109in}}%
\pgfpathlineto{\pgfqpoint{2.518934in}{2.251905in}}%
\pgfpathlineto{\pgfqpoint{2.518339in}{2.622972in}}%
\pgfpathlineto{\pgfqpoint{2.519198in}{2.361390in}}%
\pgfpathlineto{\pgfqpoint{2.519639in}{2.454484in}}%
\pgfpathlineto{\pgfqpoint{2.519419in}{2.134007in}}%
\pgfpathlineto{\pgfqpoint{2.520278in}{2.288400in}}%
\pgfpathlineto{\pgfqpoint{2.520807in}{2.446289in}}%
\pgfpathlineto{\pgfqpoint{2.520543in}{2.229943in}}%
\pgfpathlineto{\pgfqpoint{2.521336in}{2.294628in}}%
\pgfpathlineto{\pgfqpoint{2.521358in}{2.252889in}}%
\pgfpathlineto{\pgfqpoint{2.521689in}{2.538400in}}%
\pgfpathlineto{\pgfqpoint{2.522350in}{2.337679in}}%
\pgfpathlineto{\pgfqpoint{2.523320in}{2.580796in}}%
\pgfpathlineto{\pgfqpoint{2.522703in}{2.328828in}}%
\pgfpathlineto{\pgfqpoint{2.523474in}{2.460275in}}%
\pgfpathlineto{\pgfqpoint{2.524510in}{2.197709in}}%
\pgfpathlineto{\pgfqpoint{2.523805in}{2.580577in}}%
\pgfpathlineto{\pgfqpoint{2.524599in}{2.274742in}}%
\pgfpathlineto{\pgfqpoint{2.525392in}{2.525398in}}%
\pgfpathlineto{\pgfqpoint{2.524841in}{2.240323in}}%
\pgfpathlineto{\pgfqpoint{2.525701in}{2.442137in}}%
\pgfpathlineto{\pgfqpoint{2.526626in}{2.199567in}}%
\pgfpathlineto{\pgfqpoint{2.526009in}{2.450441in}}%
\pgfpathlineto{\pgfqpoint{2.526825in}{2.326424in}}%
\pgfpathlineto{\pgfqpoint{2.527685in}{2.506932in}}%
\pgfpathlineto{\pgfqpoint{2.527001in}{2.233111in}}%
\pgfpathlineto{\pgfqpoint{2.527905in}{2.289820in}}%
\pgfpathlineto{\pgfqpoint{2.527927in}{2.222622in}}%
\pgfpathlineto{\pgfqpoint{2.528787in}{2.612701in}}%
\pgfpathlineto{\pgfqpoint{2.528963in}{2.450660in}}%
\pgfpathlineto{\pgfqpoint{2.529624in}{2.552605in}}%
\pgfpathlineto{\pgfqpoint{2.529249in}{2.262395in}}%
\pgfpathlineto{\pgfqpoint{2.529933in}{2.375485in}}%
\pgfpathlineto{\pgfqpoint{2.529955in}{2.285450in}}%
\pgfpathlineto{\pgfqpoint{2.530528in}{2.562876in}}%
\pgfpathlineto{\pgfqpoint{2.531035in}{2.300747in}}%
\pgfpathlineto{\pgfqpoint{2.531917in}{2.500594in}}%
\pgfpathlineto{\pgfqpoint{2.531476in}{2.279440in}}%
\pgfpathlineto{\pgfqpoint{2.532159in}{2.446945in}}%
\pgfpathlineto{\pgfqpoint{2.532864in}{2.261958in}}%
\pgfpathlineto{\pgfqpoint{2.532291in}{2.540476in}}%
\pgfpathlineto{\pgfqpoint{2.533305in}{2.369366in}}%
\pgfpathlineto{\pgfqpoint{2.533592in}{2.590957in}}%
\pgfpathlineto{\pgfqpoint{2.534385in}{2.310144in}}%
\pgfpathlineto{\pgfqpoint{2.534848in}{2.276927in}}%
\pgfpathlineto{\pgfqpoint{2.534782in}{2.490979in}}%
\pgfpathlineto{\pgfqpoint{2.535377in}{2.372316in}}%
\pgfpathlineto{\pgfqpoint{2.535708in}{2.503763in}}%
\pgfpathlineto{\pgfqpoint{2.535487in}{2.195742in}}%
\pgfpathlineto{\pgfqpoint{2.536435in}{2.290476in}}%
\pgfpathlineto{\pgfqpoint{2.536876in}{2.200441in}}%
\pgfpathlineto{\pgfqpoint{2.537141in}{2.459292in}}%
\pgfpathlineto{\pgfqpoint{2.537449in}{2.425638in}}%
\pgfpathlineto{\pgfqpoint{2.537625in}{2.489121in}}%
\pgfpathlineto{\pgfqpoint{2.538221in}{2.197491in}}%
\pgfpathlineto{\pgfqpoint{2.538485in}{2.341831in}}%
\pgfpathlineto{\pgfqpoint{2.538728in}{2.227102in}}%
\pgfpathlineto{\pgfqpoint{2.538860in}{2.466940in}}%
\pgfpathlineto{\pgfqpoint{2.539146in}{2.394825in}}%
\pgfpathlineto{\pgfqpoint{2.539168in}{2.478632in}}%
\pgfpathlineto{\pgfqpoint{2.539389in}{2.213553in}}%
\pgfpathlineto{\pgfqpoint{2.540248in}{2.418208in}}%
\pgfpathlineto{\pgfqpoint{2.540381in}{2.202626in}}%
\pgfpathlineto{\pgfqpoint{2.541108in}{2.514690in}}%
\pgfpathlineto{\pgfqpoint{2.541351in}{2.381057in}}%
\pgfpathlineto{\pgfqpoint{2.542122in}{2.603086in}}%
\pgfpathlineto{\pgfqpoint{2.541593in}{2.273649in}}%
\pgfpathlineto{\pgfqpoint{2.542431in}{2.501796in}}%
\pgfpathlineto{\pgfqpoint{2.542453in}{2.322819in}}%
\pgfpathlineto{\pgfqpoint{2.543070in}{2.668536in}}%
\pgfpathlineto{\pgfqpoint{2.543555in}{2.378654in}}%
\pgfpathlineto{\pgfqpoint{2.543621in}{2.308068in}}%
\pgfpathlineto{\pgfqpoint{2.543863in}{2.581014in}}%
\pgfpathlineto{\pgfqpoint{2.544613in}{2.470655in}}%
\pgfpathlineto{\pgfqpoint{2.544899in}{2.578719in}}%
\pgfpathlineto{\pgfqpoint{2.545142in}{2.315279in}}%
\pgfpathlineto{\pgfqpoint{2.545693in}{2.482238in}}%
\pgfpathlineto{\pgfqpoint{2.546134in}{2.212569in}}%
\pgfpathlineto{\pgfqpoint{2.546641in}{2.558177in}}%
\pgfpathlineto{\pgfqpoint{2.546795in}{2.425419in}}%
\pgfpathlineto{\pgfqpoint{2.547875in}{2.644825in}}%
\pgfpathlineto{\pgfqpoint{2.547544in}{2.378872in}}%
\pgfpathlineto{\pgfqpoint{2.547897in}{2.495787in}}%
\pgfpathlineto{\pgfqpoint{2.548889in}{2.339646in}}%
\pgfpathlineto{\pgfqpoint{2.548360in}{2.614449in}}%
\pgfpathlineto{\pgfqpoint{2.549021in}{2.410341in}}%
\pgfpathlineto{\pgfqpoint{2.550035in}{2.664384in}}%
\pgfpathlineto{\pgfqpoint{2.549087in}{2.340629in}}%
\pgfpathlineto{\pgfqpoint{2.550145in}{2.462788in}}%
\pgfpathlineto{\pgfqpoint{2.550608in}{2.670066in}}%
\pgfpathlineto{\pgfqpoint{2.551159in}{2.337351in}}%
\pgfpathlineto{\pgfqpoint{2.551203in}{2.438531in}}%
\pgfpathlineto{\pgfqpoint{2.551776in}{2.258024in}}%
\pgfpathlineto{\pgfqpoint{2.551997in}{2.529441in}}%
\pgfpathlineto{\pgfqpoint{2.552283in}{2.445852in}}%
\pgfpathlineto{\pgfqpoint{2.553253in}{2.661762in}}%
\pgfpathlineto{\pgfqpoint{2.552327in}{2.395262in}}%
\pgfpathlineto{\pgfqpoint{2.553452in}{2.565717in}}%
\pgfpathlineto{\pgfqpoint{2.553716in}{2.661762in}}%
\pgfpathlineto{\pgfqpoint{2.554576in}{2.319322in}}%
\pgfpathlineto{\pgfqpoint{2.555502in}{2.699131in}}%
\pgfpathlineto{\pgfqpoint{2.555700in}{2.456997in}}%
\pgfpathlineto{\pgfqpoint{2.555986in}{2.684598in}}%
\pgfpathlineto{\pgfqpoint{2.556692in}{2.354178in}}%
\pgfpathlineto{\pgfqpoint{2.556802in}{2.541132in}}%
\pgfpathlineto{\pgfqpoint{2.557750in}{2.310472in}}%
\pgfpathlineto{\pgfqpoint{2.557419in}{2.588772in}}%
\pgfpathlineto{\pgfqpoint{2.557926in}{2.434051in}}%
\pgfpathlineto{\pgfqpoint{2.558455in}{2.309925in}}%
\pgfpathlineto{\pgfqpoint{2.557992in}{2.559161in}}%
\pgfpathlineto{\pgfqpoint{2.558587in}{2.478413in}}%
\pgfpathlineto{\pgfqpoint{2.558720in}{2.672470in}}%
\pgfpathlineto{\pgfqpoint{2.559249in}{2.317683in}}%
\pgfpathlineto{\pgfqpoint{2.559667in}{2.519497in}}%
\pgfpathlineto{\pgfqpoint{2.559910in}{2.203500in}}%
\pgfpathlineto{\pgfqpoint{2.560285in}{2.624720in}}%
\pgfpathlineto{\pgfqpoint{2.560770in}{2.443339in}}%
\pgfpathlineto{\pgfqpoint{2.561012in}{2.525835in}}%
\pgfpathlineto{\pgfqpoint{2.561761in}{2.182740in}}%
\pgfpathlineto{\pgfqpoint{2.561783in}{2.256057in}}%
\pgfpathlineto{\pgfqpoint{2.561806in}{2.239995in}}%
\pgfpathlineto{\pgfqpoint{2.562092in}{2.518186in}}%
\pgfpathlineto{\pgfqpoint{2.562775in}{2.262285in}}%
\pgfpathlineto{\pgfqpoint{2.563569in}{2.465848in}}%
\pgfpathlineto{\pgfqpoint{2.563833in}{2.169628in}}%
\pgfpathlineto{\pgfqpoint{2.563900in}{2.393951in}}%
\pgfpathlineto{\pgfqpoint{2.564539in}{2.287854in}}%
\pgfpathlineto{\pgfqpoint{2.564803in}{2.622535in}}%
\pgfpathlineto{\pgfqpoint{2.564980in}{2.344344in}}%
\pgfpathlineto{\pgfqpoint{2.565046in}{2.584511in}}%
\pgfpathlineto{\pgfqpoint{2.565156in}{2.275834in}}%
\pgfpathlineto{\pgfqpoint{2.566104in}{2.474152in}}%
\pgfpathlineto{\pgfqpoint{2.566699in}{2.271901in}}%
\pgfpathlineto{\pgfqpoint{2.566192in}{2.632151in}}%
\pgfpathlineto{\pgfqpoint{2.566963in}{2.463335in}}%
\pgfpathlineto{\pgfqpoint{2.566985in}{2.609423in}}%
\pgfpathlineto{\pgfqpoint{2.567052in}{2.296049in}}%
\pgfpathlineto{\pgfqpoint{2.568065in}{2.412963in}}%
\pgfpathlineto{\pgfqpoint{2.568242in}{2.226228in}}%
\pgfpathlineto{\pgfqpoint{2.568308in}{2.533374in}}%
\pgfpathlineto{\pgfqpoint{2.569035in}{2.396792in}}%
\pgfpathlineto{\pgfqpoint{2.569939in}{2.566045in}}%
\pgfpathlineto{\pgfqpoint{2.569873in}{2.302823in}}%
\pgfpathlineto{\pgfqpoint{2.570159in}{2.543099in}}%
\pgfpathlineto{\pgfqpoint{2.570556in}{2.295721in}}%
\pgfpathlineto{\pgfqpoint{2.570248in}{2.585166in}}%
\pgfpathlineto{\pgfqpoint{2.571284in}{2.383571in}}%
\pgfpathlineto{\pgfqpoint{2.571306in}{2.383571in}}%
\pgfpathlineto{\pgfqpoint{2.571328in}{2.292989in}}%
\pgfpathlineto{\pgfqpoint{2.571526in}{2.659576in}}%
\pgfpathlineto{\pgfqpoint{2.572364in}{2.518186in}}%
\pgfpathlineto{\pgfqpoint{2.573378in}{2.663401in}}%
\pgfpathlineto{\pgfqpoint{2.572628in}{2.321180in}}%
\pgfpathlineto{\pgfqpoint{2.573444in}{2.621442in}}%
\pgfpathlineto{\pgfqpoint{2.574215in}{2.348168in}}%
\pgfpathlineto{\pgfqpoint{2.574017in}{2.640018in}}%
\pgfpathlineto{\pgfqpoint{2.574568in}{2.450332in}}%
\pgfpathlineto{\pgfqpoint{2.575405in}{2.521246in}}%
\pgfpathlineto{\pgfqpoint{2.575295in}{2.233548in}}%
\pgfpathlineto{\pgfqpoint{2.575582in}{2.344781in}}%
\pgfpathlineto{\pgfqpoint{2.575802in}{2.237373in}}%
\pgfpathlineto{\pgfqpoint{2.575868in}{2.530861in}}%
\pgfpathlineto{\pgfqpoint{2.576331in}{2.272338in}}%
\pgfpathlineto{\pgfqpoint{2.576772in}{2.571399in}}%
\pgfpathlineto{\pgfqpoint{2.576419in}{2.238465in}}%
\pgfpathlineto{\pgfqpoint{2.577433in}{2.320524in}}%
\pgfpathlineto{\pgfqpoint{2.578073in}{2.677059in}}%
\pgfpathlineto{\pgfqpoint{2.578602in}{2.576753in}}%
\pgfpathlineto{\pgfqpoint{2.579549in}{2.333527in}}%
\pgfpathlineto{\pgfqpoint{2.578734in}{2.686783in}}%
\pgfpathlineto{\pgfqpoint{2.579726in}{2.488357in}}%
\pgfpathlineto{\pgfqpoint{2.579770in}{2.651491in}}%
\pgfpathlineto{\pgfqpoint{2.580541in}{2.360515in}}%
\pgfpathlineto{\pgfqpoint{2.580806in}{2.420502in}}%
\pgfpathlineto{\pgfqpoint{2.581665in}{2.314187in}}%
\pgfpathlineto{\pgfqpoint{2.581379in}{2.668208in}}%
\pgfpathlineto{\pgfqpoint{2.581842in}{2.456560in}}%
\pgfpathlineto{\pgfqpoint{2.582944in}{2.718908in}}%
\pgfpathlineto{\pgfqpoint{2.582613in}{2.423453in}}%
\pgfpathlineto{\pgfqpoint{2.582966in}{2.606801in}}%
\pgfpathlineto{\pgfqpoint{2.583605in}{2.455468in}}%
\pgfpathlineto{\pgfqpoint{2.583098in}{2.717924in}}%
\pgfpathlineto{\pgfqpoint{2.584068in}{2.537198in}}%
\pgfpathlineto{\pgfqpoint{2.584751in}{2.741963in}}%
\pgfpathlineto{\pgfqpoint{2.584156in}{2.407172in}}%
\pgfpathlineto{\pgfqpoint{2.585192in}{2.682522in}}%
\pgfpathlineto{\pgfqpoint{2.585435in}{2.440717in}}%
\pgfpathlineto{\pgfqpoint{2.585302in}{2.778130in}}%
\pgfpathlineto{\pgfqpoint{2.586316in}{2.523212in}}%
\pgfpathlineto{\pgfqpoint{2.586625in}{2.688532in}}%
\pgfpathlineto{\pgfqpoint{2.587220in}{2.397447in}}%
\pgfpathlineto{\pgfqpoint{2.587418in}{2.526381in}}%
\pgfpathlineto{\pgfqpoint{2.587440in}{2.526490in}}%
\pgfpathlineto{\pgfqpoint{2.588190in}{2.406844in}}%
\pgfpathlineto{\pgfqpoint{2.588300in}{2.648103in}}%
\pgfpathlineto{\pgfqpoint{2.588542in}{2.554681in}}%
\pgfpathlineto{\pgfqpoint{2.588895in}{2.611390in}}%
\pgfpathlineto{\pgfqpoint{2.588763in}{2.251140in}}%
\pgfpathlineto{\pgfqpoint{2.589512in}{2.509882in}}%
\pgfpathlineto{\pgfqpoint{2.589733in}{2.333417in}}%
\pgfpathlineto{\pgfqpoint{2.590328in}{2.638488in}}%
\pgfpathlineto{\pgfqpoint{2.590592in}{2.559598in}}%
\pgfpathlineto{\pgfqpoint{2.591143in}{2.674218in}}%
\pgfpathlineto{\pgfqpoint{2.590636in}{2.433177in}}%
\pgfpathlineto{\pgfqpoint{2.591606in}{2.580249in}}%
\pgfpathlineto{\pgfqpoint{2.592422in}{2.310035in}}%
\pgfpathlineto{\pgfqpoint{2.591739in}{2.663291in}}%
\pgfpathlineto{\pgfqpoint{2.592730in}{2.427823in}}%
\pgfpathlineto{\pgfqpoint{2.592752in}{2.427386in}}%
\pgfpathlineto{\pgfqpoint{2.593083in}{2.630074in}}%
\pgfpathlineto{\pgfqpoint{2.593700in}{2.342596in}}%
\pgfpathlineto{\pgfqpoint{2.593855in}{2.392640in}}%
\pgfpathlineto{\pgfqpoint{2.594339in}{2.259007in}}%
\pgfpathlineto{\pgfqpoint{2.594802in}{2.603195in}}%
\pgfpathlineto{\pgfqpoint{2.594913in}{2.526272in}}%
\pgfpathlineto{\pgfqpoint{2.595243in}{2.368710in}}%
\pgfpathlineto{\pgfqpoint{2.595552in}{2.618383in}}%
\pgfpathlineto{\pgfqpoint{2.595574in}{2.595328in}}%
\pgfpathlineto{\pgfqpoint{2.595596in}{2.616963in}}%
\pgfpathlineto{\pgfqpoint{2.596257in}{2.287854in}}%
\pgfpathlineto{\pgfqpoint{2.596456in}{2.387067in}}%
\pgfpathlineto{\pgfqpoint{2.596478in}{2.292333in}}%
\pgfpathlineto{\pgfqpoint{2.597249in}{2.744694in}}%
\pgfpathlineto{\pgfqpoint{2.597558in}{2.425419in}}%
\pgfpathlineto{\pgfqpoint{2.597624in}{2.352320in}}%
\pgfpathlineto{\pgfqpoint{2.597800in}{2.613685in}}%
\pgfpathlineto{\pgfqpoint{2.598594in}{2.439187in}}%
\pgfpathlineto{\pgfqpoint{2.599343in}{2.588116in}}%
\pgfpathlineto{\pgfqpoint{2.598836in}{2.310253in}}%
\pgfpathlineto{\pgfqpoint{2.599696in}{2.470765in}}%
\pgfpathlineto{\pgfqpoint{2.600357in}{2.594454in}}%
\pgfpathlineto{\pgfqpoint{2.600666in}{2.303151in}}%
\pgfpathlineto{\pgfqpoint{2.600688in}{2.437876in}}%
\pgfpathlineto{\pgfqpoint{2.600710in}{2.299545in}}%
\pgfpathlineto{\pgfqpoint{2.601724in}{2.618274in}}%
\pgfpathlineto{\pgfqpoint{2.601768in}{2.520918in}}%
\pgfpathlineto{\pgfqpoint{2.601834in}{2.628108in}}%
\pgfpathlineto{\pgfqpoint{2.602054in}{2.325332in}}%
\pgfpathlineto{\pgfqpoint{2.602848in}{2.440389in}}%
\pgfpathlineto{\pgfqpoint{2.604016in}{2.655096in}}%
\pgfpathlineto{\pgfqpoint{2.602980in}{2.335384in}}%
\pgfpathlineto{\pgfqpoint{2.604060in}{2.600245in}}%
\pgfpathlineto{\pgfqpoint{2.604898in}{2.373409in}}%
\pgfpathlineto{\pgfqpoint{2.604677in}{2.655096in}}%
\pgfpathlineto{\pgfqpoint{2.605162in}{2.650507in}}%
\pgfpathlineto{\pgfqpoint{2.605779in}{2.360406in}}%
\pgfpathlineto{\pgfqpoint{2.606286in}{2.419082in}}%
\pgfpathlineto{\pgfqpoint{2.606308in}{2.590411in}}%
\pgfpathlineto{\pgfqpoint{2.606992in}{2.277036in}}%
\pgfpathlineto{\pgfqpoint{2.607388in}{2.411652in}}%
\pgfpathlineto{\pgfqpoint{2.608292in}{2.671486in}}%
\pgfpathlineto{\pgfqpoint{2.607697in}{2.348496in}}%
\pgfpathlineto{\pgfqpoint{2.608535in}{2.558396in}}%
\pgfpathlineto{\pgfqpoint{2.608887in}{2.414930in}}%
\pgfpathlineto{\pgfqpoint{2.609240in}{2.709183in}}%
\pgfpathlineto{\pgfqpoint{2.609659in}{2.449895in}}%
\pgfpathlineto{\pgfqpoint{2.610188in}{2.661215in}}%
\pgfpathlineto{\pgfqpoint{2.610011in}{2.330795in}}%
\pgfpathlineto{\pgfqpoint{2.610805in}{2.512177in}}%
\pgfpathlineto{\pgfqpoint{2.610827in}{2.448365in}}%
\pgfpathlineto{\pgfqpoint{2.611202in}{2.763597in}}%
\pgfpathlineto{\pgfqpoint{2.611885in}{2.503872in}}%
\pgfpathlineto{\pgfqpoint{2.612392in}{2.649961in}}%
\pgfpathlineto{\pgfqpoint{2.612061in}{2.386302in}}%
\pgfpathlineto{\pgfqpoint{2.612987in}{2.491197in}}%
\pgfpathlineto{\pgfqpoint{2.613406in}{2.644388in}}%
\pgfpathlineto{\pgfqpoint{2.613913in}{2.391219in}}%
\pgfpathlineto{\pgfqpoint{2.613935in}{2.345109in}}%
\pgfpathlineto{\pgfqpoint{2.614376in}{2.647448in}}%
\pgfpathlineto{\pgfqpoint{2.614993in}{2.413619in}}%
\pgfpathlineto{\pgfqpoint{2.615103in}{2.663838in}}%
\pgfpathlineto{\pgfqpoint{2.615588in}{2.288618in}}%
\pgfpathlineto{\pgfqpoint{2.615919in}{2.383024in}}%
\pgfpathlineto{\pgfqpoint{2.616514in}{2.304462in}}%
\pgfpathlineto{\pgfqpoint{2.616448in}{2.551731in}}%
\pgfpathlineto{\pgfqpoint{2.617021in}{2.365651in}}%
\pgfpathlineto{\pgfqpoint{2.618079in}{2.665695in}}%
\pgfpathlineto{\pgfqpoint{2.617197in}{2.332325in}}%
\pgfpathlineto{\pgfqpoint{2.618145in}{2.593361in}}%
\pgfpathlineto{\pgfqpoint{2.618740in}{2.420830in}}%
\pgfpathlineto{\pgfqpoint{2.618387in}{2.750376in}}%
\pgfpathlineto{\pgfqpoint{2.619269in}{2.512614in}}%
\pgfpathlineto{\pgfqpoint{2.619996in}{2.726119in}}%
\pgfpathlineto{\pgfqpoint{2.619754in}{2.388597in}}%
\pgfpathlineto{\pgfqpoint{2.620393in}{2.593689in}}%
\pgfpathlineto{\pgfqpoint{2.620856in}{2.336477in}}%
\pgfpathlineto{\pgfqpoint{2.620547in}{2.689734in}}%
\pgfpathlineto{\pgfqpoint{2.621517in}{2.375485in}}%
\pgfpathlineto{\pgfqpoint{2.622267in}{2.651600in}}%
\pgfpathlineto{\pgfqpoint{2.622068in}{2.345655in}}%
\pgfpathlineto{\pgfqpoint{2.622641in}{2.483767in}}%
\pgfpathlineto{\pgfqpoint{2.623192in}{2.780534in}}%
\pgfpathlineto{\pgfqpoint{2.623501in}{2.434379in}}%
\pgfpathlineto{\pgfqpoint{2.623721in}{2.488357in}}%
\pgfpathlineto{\pgfqpoint{2.623876in}{2.425856in}}%
\pgfpathlineto{\pgfqpoint{2.624515in}{2.752015in}}%
\pgfpathlineto{\pgfqpoint{2.624735in}{2.564187in}}%
\pgfpathlineto{\pgfqpoint{2.625771in}{2.759992in}}%
\pgfpathlineto{\pgfqpoint{2.625815in}{2.443339in}}%
\pgfpathlineto{\pgfqpoint{2.625837in}{2.584401in}}%
\pgfpathlineto{\pgfqpoint{2.625882in}{2.717815in}}%
\pgfpathlineto{\pgfqpoint{2.626344in}{2.384991in}}%
\pgfpathlineto{\pgfqpoint{2.626873in}{2.559926in}}%
\pgfpathlineto{\pgfqpoint{2.627094in}{2.409139in}}%
\pgfpathlineto{\pgfqpoint{2.627402in}{2.682304in}}%
\pgfpathlineto{\pgfqpoint{2.627998in}{2.461805in}}%
\pgfpathlineto{\pgfqpoint{2.628967in}{2.747645in}}%
\pgfpathlineto{\pgfqpoint{2.628394in}{2.383571in}}%
\pgfpathlineto{\pgfqpoint{2.629100in}{2.588335in}}%
\pgfpathlineto{\pgfqpoint{2.629937in}{2.413291in}}%
\pgfpathlineto{\pgfqpoint{2.629452in}{2.733986in}}%
\pgfpathlineto{\pgfqpoint{2.630224in}{2.456014in}}%
\pgfpathlineto{\pgfqpoint{2.630400in}{2.433287in}}%
\pgfpathlineto{\pgfqpoint{2.630466in}{2.639144in}}%
\pgfpathlineto{\pgfqpoint{2.630709in}{2.574458in}}%
\pgfpathlineto{\pgfqpoint{2.631282in}{2.659795in}}%
\pgfpathlineto{\pgfqpoint{2.631524in}{2.350681in}}%
\pgfpathlineto{\pgfqpoint{2.631767in}{2.452517in}}%
\pgfpathlineto{\pgfqpoint{2.632097in}{2.807850in}}%
\pgfpathlineto{\pgfqpoint{2.632208in}{2.349480in}}%
\pgfpathlineto{\pgfqpoint{2.632671in}{2.565389in}}%
\pgfpathlineto{\pgfqpoint{2.632737in}{2.295065in}}%
\pgfpathlineto{\pgfqpoint{2.633464in}{2.623191in}}%
\pgfpathlineto{\pgfqpoint{2.633773in}{2.445415in}}%
\pgfpathlineto{\pgfqpoint{2.634831in}{2.648322in}}%
\pgfpathlineto{\pgfqpoint{2.633817in}{2.343251in}}%
\pgfpathlineto{\pgfqpoint{2.634897in}{2.588772in}}%
\pgfpathlineto{\pgfqpoint{2.635161in}{2.366744in}}%
\pgfpathlineto{\pgfqpoint{2.635756in}{2.648540in}}%
\pgfpathlineto{\pgfqpoint{2.635999in}{2.624611in}}%
\pgfpathlineto{\pgfqpoint{2.636175in}{2.412307in}}%
\pgfpathlineto{\pgfqpoint{2.636792in}{2.694651in}}%
\pgfpathlineto{\pgfqpoint{2.637101in}{2.463225in}}%
\pgfpathlineto{\pgfqpoint{2.637233in}{2.681757in}}%
\pgfpathlineto{\pgfqpoint{2.637586in}{2.412417in}}%
\pgfpathlineto{\pgfqpoint{2.638225in}{2.671268in}}%
\pgfpathlineto{\pgfqpoint{2.639305in}{2.399633in}}%
\pgfpathlineto{\pgfqpoint{2.638952in}{2.709292in}}%
\pgfpathlineto{\pgfqpoint{2.639371in}{2.564952in}}%
\pgfpathlineto{\pgfqpoint{2.639393in}{2.680774in}}%
\pgfpathlineto{\pgfqpoint{2.640429in}{2.386302in}}%
\pgfpathlineto{\pgfqpoint{2.640473in}{2.520481in}}%
\pgfpathlineto{\pgfqpoint{2.640716in}{2.462133in}}%
\pgfpathlineto{\pgfqpoint{2.640980in}{2.759992in}}%
\pgfpathlineto{\pgfqpoint{2.641509in}{2.530642in}}%
\pgfpathlineto{\pgfqpoint{2.641575in}{2.791897in}}%
\pgfpathlineto{\pgfqpoint{2.641642in}{2.498518in}}%
\pgfpathlineto{\pgfqpoint{2.642611in}{2.551184in}}%
\pgfpathlineto{\pgfqpoint{2.643030in}{2.455249in}}%
\pgfpathlineto{\pgfqpoint{2.643625in}{2.704375in}}%
\pgfpathlineto{\pgfqpoint{2.643692in}{2.582325in}}%
\pgfpathlineto{\pgfqpoint{2.643824in}{2.753217in}}%
\pgfpathlineto{\pgfqpoint{2.644705in}{2.510647in}}%
\pgfpathlineto{\pgfqpoint{2.644772in}{2.678261in}}%
\pgfpathlineto{\pgfqpoint{2.645190in}{2.476228in}}%
\pgfpathlineto{\pgfqpoint{2.645367in}{2.745241in}}%
\pgfpathlineto{\pgfqpoint{2.645896in}{2.521573in}}%
\pgfpathlineto{\pgfqpoint{2.647152in}{2.847077in}}%
\pgfpathlineto{\pgfqpoint{2.647593in}{2.417224in}}%
\pgfpathlineto{\pgfqpoint{2.648364in}{2.695853in}}%
\pgfpathlineto{\pgfqpoint{2.649665in}{2.457981in}}%
\pgfpathlineto{\pgfqpoint{2.649026in}{2.725027in}}%
\pgfpathlineto{\pgfqpoint{2.649731in}{2.520699in}}%
\pgfpathlineto{\pgfqpoint{2.650304in}{2.717706in}}%
\pgfpathlineto{\pgfqpoint{2.650745in}{2.442683in}}%
\pgfpathlineto{\pgfqpoint{2.650833in}{2.558177in}}%
\pgfpathlineto{\pgfqpoint{2.650899in}{2.511412in}}%
\pgfpathlineto{\pgfqpoint{2.651164in}{2.667225in}}%
\pgfpathlineto{\pgfqpoint{2.651186in}{2.656626in}}%
\pgfpathlineto{\pgfqpoint{2.652023in}{2.743055in}}%
\pgfpathlineto{\pgfqpoint{2.652222in}{2.507587in}}%
\pgfpathlineto{\pgfqpoint{2.652266in}{2.595219in}}%
\pgfpathlineto{\pgfqpoint{2.652773in}{2.436237in}}%
\pgfpathlineto{\pgfqpoint{2.653214in}{2.701971in}}%
\pgfpathlineto{\pgfqpoint{2.653346in}{2.558942in}}%
\pgfpathlineto{\pgfqpoint{2.653699in}{2.683287in}}%
\pgfpathlineto{\pgfqpoint{2.654426in}{2.475026in}}%
\pgfpathlineto{\pgfqpoint{2.654448in}{2.583855in}}%
\pgfpathlineto{\pgfqpoint{2.654470in}{2.461368in}}%
\pgfpathlineto{\pgfqpoint{2.655109in}{2.759555in}}%
\pgfpathlineto{\pgfqpoint{2.655550in}{2.589318in}}%
\pgfpathlineto{\pgfqpoint{2.656344in}{2.475572in}}%
\pgfpathlineto{\pgfqpoint{2.656145in}{2.731473in}}%
\pgfpathlineto{\pgfqpoint{2.656630in}{2.623081in}}%
\pgfpathlineto{\pgfqpoint{2.657512in}{2.442465in}}%
\pgfpathlineto{\pgfqpoint{2.657358in}{2.725136in}}%
\pgfpathlineto{\pgfqpoint{2.657754in}{2.464427in}}%
\pgfpathlineto{\pgfqpoint{2.658019in}{2.766657in}}%
\pgfpathlineto{\pgfqpoint{2.658856in}{2.447163in}}%
\pgfpathlineto{\pgfqpoint{2.658878in}{2.595328in}}%
\pgfpathlineto{\pgfqpoint{2.659143in}{2.670175in}}%
\pgfpathlineto{\pgfqpoint{2.659297in}{2.385319in}}%
\pgfpathlineto{\pgfqpoint{2.659826in}{2.551512in}}%
\pgfpathlineto{\pgfqpoint{2.660069in}{2.659467in}}%
\pgfpathlineto{\pgfqpoint{2.660928in}{2.342924in}}%
\pgfpathlineto{\pgfqpoint{2.661986in}{2.762505in}}%
\pgfpathlineto{\pgfqpoint{2.661237in}{2.310799in}}%
\pgfpathlineto{\pgfqpoint{2.662052in}{2.656735in}}%
\pgfpathlineto{\pgfqpoint{2.662581in}{2.460712in}}%
\pgfpathlineto{\pgfqpoint{2.662670in}{2.749830in}}%
\pgfpathlineto{\pgfqpoint{2.663155in}{2.581560in}}%
\pgfpathlineto{\pgfqpoint{2.663265in}{2.764909in}}%
\pgfpathlineto{\pgfqpoint{2.664169in}{2.463772in}}%
\pgfpathlineto{\pgfqpoint{2.664257in}{2.628217in}}%
\pgfpathlineto{\pgfqpoint{2.665072in}{2.451971in}}%
\pgfpathlineto{\pgfqpoint{2.664918in}{2.792444in}}%
\pgfpathlineto{\pgfqpoint{2.665403in}{2.523649in}}%
\pgfpathlineto{\pgfqpoint{2.665711in}{2.476337in}}%
\pgfpathlineto{\pgfqpoint{2.665601in}{2.782828in}}%
\pgfpathlineto{\pgfqpoint{2.666152in}{2.564733in}}%
\pgfpathlineto{\pgfqpoint{2.666637in}{2.840084in}}%
\pgfpathlineto{\pgfqpoint{2.666836in}{2.522775in}}%
\pgfpathlineto{\pgfqpoint{2.667254in}{2.656408in}}%
\pgfpathlineto{\pgfqpoint{2.667850in}{2.499611in}}%
\pgfpathlineto{\pgfqpoint{2.668092in}{2.824240in}}%
\pgfpathlineto{\pgfqpoint{2.668334in}{2.666679in}}%
\pgfpathlineto{\pgfqpoint{2.668665in}{2.782282in}}%
\pgfpathlineto{\pgfqpoint{2.668401in}{2.461259in}}%
\pgfpathlineto{\pgfqpoint{2.669238in}{2.551403in}}%
\pgfpathlineto{\pgfqpoint{2.670186in}{2.452080in}}%
\pgfpathlineto{\pgfqpoint{2.669944in}{2.798453in}}%
\pgfpathlineto{\pgfqpoint{2.670296in}{2.704047in}}%
\pgfpathlineto{\pgfqpoint{2.670803in}{2.531954in}}%
\pgfpathlineto{\pgfqpoint{2.670362in}{2.803370in}}%
\pgfpathlineto{\pgfqpoint{2.671376in}{2.746443in}}%
\pgfpathlineto{\pgfqpoint{2.671486in}{2.539165in}}%
\pgfpathlineto{\pgfqpoint{2.672324in}{2.857457in}}%
\pgfpathlineto{\pgfqpoint{2.672456in}{2.740652in}}%
\pgfpathlineto{\pgfqpoint{2.672853in}{2.905971in}}%
\pgfpathlineto{\pgfqpoint{2.673404in}{2.633352in}}%
\pgfpathlineto{\pgfqpoint{2.673536in}{2.693121in}}%
\pgfpathlineto{\pgfqpoint{2.673845in}{2.521464in}}%
\pgfpathlineto{\pgfqpoint{2.674132in}{2.813204in}}%
\pgfpathlineto{\pgfqpoint{2.674440in}{2.693449in}}%
\pgfpathlineto{\pgfqpoint{2.674661in}{2.818995in}}%
\pgfpathlineto{\pgfqpoint{2.674881in}{2.479943in}}%
\pgfpathlineto{\pgfqpoint{2.675542in}{2.741963in}}%
\pgfpathlineto{\pgfqpoint{2.675719in}{2.570197in}}%
\pgfpathlineto{\pgfqpoint{2.676248in}{2.936128in}}%
\pgfpathlineto{\pgfqpoint{2.676688in}{2.649305in}}%
\pgfpathlineto{\pgfqpoint{2.677107in}{2.861172in}}%
\pgfpathlineto{\pgfqpoint{2.677570in}{2.523103in}}%
\pgfpathlineto{\pgfqpoint{2.677790in}{2.686674in}}%
\pgfpathlineto{\pgfqpoint{2.677901in}{2.432194in}}%
\pgfpathlineto{\pgfqpoint{2.678187in}{2.840084in}}%
\pgfpathlineto{\pgfqpoint{2.678848in}{2.664821in}}%
\pgfpathlineto{\pgfqpoint{2.679686in}{2.829813in}}%
\pgfpathlineto{\pgfqpoint{2.679311in}{2.550638in}}%
\pgfpathlineto{\pgfqpoint{2.679929in}{2.676512in}}%
\pgfpathlineto{\pgfqpoint{2.680259in}{2.405424in}}%
\pgfpathlineto{\pgfqpoint{2.680810in}{2.778239in}}%
\pgfpathlineto{\pgfqpoint{2.681031in}{2.622754in}}%
\pgfpathlineto{\pgfqpoint{2.681053in}{2.622426in}}%
\pgfpathlineto{\pgfqpoint{2.681427in}{2.799000in}}%
\pgfpathlineto{\pgfqpoint{2.681207in}{2.531626in}}%
\pgfpathlineto{\pgfqpoint{2.682177in}{2.723060in}}%
\pgfpathlineto{\pgfqpoint{2.682596in}{2.473715in}}%
\pgfpathlineto{\pgfqpoint{2.682794in}{2.768951in}}%
\pgfpathlineto{\pgfqpoint{2.683301in}{2.575441in}}%
\pgfpathlineto{\pgfqpoint{2.683433in}{2.805228in}}%
\pgfpathlineto{\pgfqpoint{2.683698in}{2.454266in}}%
\pgfpathlineto{\pgfqpoint{2.684425in}{2.727540in}}%
\pgfpathlineto{\pgfqpoint{2.684601in}{2.571289in}}%
\pgfpathlineto{\pgfqpoint{2.685483in}{2.841613in}}%
\pgfpathlineto{\pgfqpoint{2.685527in}{2.744913in}}%
\pgfpathlineto{\pgfqpoint{2.686034in}{2.546923in}}%
\pgfpathlineto{\pgfqpoint{2.685946in}{2.799000in}}%
\pgfpathlineto{\pgfqpoint{2.686629in}{2.684270in}}%
\pgfpathlineto{\pgfqpoint{2.686651in}{2.856255in}}%
\pgfpathlineto{\pgfqpoint{2.687687in}{2.490214in}}%
\pgfpathlineto{\pgfqpoint{2.687709in}{2.539274in}}%
\pgfpathlineto{\pgfqpoint{2.688944in}{2.784904in}}%
\pgfpathlineto{\pgfqpoint{2.688481in}{2.533156in}}%
\pgfpathlineto{\pgfqpoint{2.689054in}{2.695962in}}%
\pgfpathlineto{\pgfqpoint{2.689825in}{2.557303in}}%
\pgfpathlineto{\pgfqpoint{2.689230in}{2.766657in}}%
\pgfpathlineto{\pgfqpoint{2.690112in}{2.760647in}}%
\pgfpathlineto{\pgfqpoint{2.690310in}{2.854616in}}%
\pgfpathlineto{\pgfqpoint{2.690575in}{2.568339in}}%
\pgfpathlineto{\pgfqpoint{2.691082in}{2.640564in}}%
\pgfpathlineto{\pgfqpoint{2.691390in}{2.464318in}}%
\pgfpathlineto{\pgfqpoint{2.691192in}{2.823475in}}%
\pgfpathlineto{\pgfqpoint{2.692184in}{2.676185in}}%
\pgfpathlineto{\pgfqpoint{2.692845in}{2.791351in}}%
\pgfpathlineto{\pgfqpoint{2.693264in}{2.530533in}}%
\pgfpathlineto{\pgfqpoint{2.693286in}{2.647776in}}%
\pgfpathlineto{\pgfqpoint{2.694057in}{2.508680in}}%
\pgfpathlineto{\pgfqpoint{2.694212in}{2.841504in}}%
\pgfpathlineto{\pgfqpoint{2.694366in}{2.602212in}}%
\pgfpathlineto{\pgfqpoint{2.695093in}{2.878217in}}%
\pgfpathlineto{\pgfqpoint{2.694520in}{2.597295in}}%
\pgfpathlineto{\pgfqpoint{2.695446in}{2.713772in}}%
\pgfpathlineto{\pgfqpoint{2.695468in}{2.508461in}}%
\pgfpathlineto{\pgfqpoint{2.695556in}{2.820634in}}%
\pgfpathlineto{\pgfqpoint{2.696526in}{2.727758in}}%
\pgfpathlineto{\pgfqpoint{2.697364in}{2.845765in}}%
\pgfpathlineto{\pgfqpoint{2.697408in}{2.568448in}}%
\pgfpathlineto{\pgfqpoint{2.697584in}{2.579921in}}%
\pgfpathlineto{\pgfqpoint{2.697849in}{2.400725in}}%
\pgfpathlineto{\pgfqpoint{2.698267in}{2.758680in}}%
\pgfpathlineto{\pgfqpoint{2.698488in}{2.560909in}}%
\pgfpathlineto{\pgfqpoint{2.699127in}{2.798672in}}%
\pgfpathlineto{\pgfqpoint{2.699524in}{2.477867in}}%
\pgfpathlineto{\pgfqpoint{2.699590in}{2.524524in}}%
\pgfpathlineto{\pgfqpoint{2.700516in}{2.781189in}}%
\pgfpathlineto{\pgfqpoint{2.700428in}{2.465192in}}%
\pgfpathlineto{\pgfqpoint{2.700890in}{2.736062in}}%
\pgfpathlineto{\pgfqpoint{2.701552in}{2.480926in}}%
\pgfpathlineto{\pgfqpoint{2.701045in}{2.740324in}}%
\pgfpathlineto{\pgfqpoint{2.702015in}{2.629419in}}%
\pgfpathlineto{\pgfqpoint{2.702191in}{2.753326in}}%
\pgfpathlineto{\pgfqpoint{2.702301in}{2.431648in}}%
\pgfpathlineto{\pgfqpoint{2.703139in}{2.698475in}}%
\pgfpathlineto{\pgfqpoint{2.704131in}{2.524851in}}%
\pgfpathlineto{\pgfqpoint{2.703403in}{2.868711in}}%
\pgfpathlineto{\pgfqpoint{2.704219in}{2.627561in}}%
\pgfpathlineto{\pgfqpoint{2.704616in}{2.905206in}}%
\pgfpathlineto{\pgfqpoint{2.705034in}{2.575441in}}%
\pgfpathlineto{\pgfqpoint{2.705343in}{2.852540in}}%
\pgfpathlineto{\pgfqpoint{2.706269in}{2.567793in}}%
\pgfpathlineto{\pgfqpoint{2.706467in}{2.650179in}}%
\pgfpathlineto{\pgfqpoint{2.706599in}{2.832326in}}%
\pgfpathlineto{\pgfqpoint{2.707547in}{2.550638in}}%
\pgfpathlineto{\pgfqpoint{2.707790in}{2.756604in}}%
\pgfpathlineto{\pgfqpoint{2.708385in}{2.545393in}}%
\pgfpathlineto{\pgfqpoint{2.708671in}{2.688095in}}%
\pgfpathlineto{\pgfqpoint{2.708759in}{2.763707in}}%
\pgfpathlineto{\pgfqpoint{2.709002in}{2.534467in}}%
\pgfpathlineto{\pgfqpoint{2.709773in}{2.697054in}}%
\pgfpathlineto{\pgfqpoint{2.710677in}{2.532937in}}%
\pgfpathlineto{\pgfqpoint{2.710280in}{2.772120in}}%
\pgfpathlineto{\pgfqpoint{2.710920in}{2.612483in}}%
\pgfpathlineto{\pgfqpoint{2.711118in}{2.826316in}}%
\pgfpathlineto{\pgfqpoint{2.711360in}{2.565935in}}%
\pgfpathlineto{\pgfqpoint{2.712066in}{2.718471in}}%
\pgfpathlineto{\pgfqpoint{2.712617in}{2.513706in}}%
\pgfpathlineto{\pgfqpoint{2.712859in}{2.789603in}}%
\pgfpathlineto{\pgfqpoint{2.713212in}{2.635428in}}%
\pgfpathlineto{\pgfqpoint{2.713454in}{2.875158in}}%
\pgfpathlineto{\pgfqpoint{2.713675in}{2.500267in}}%
\pgfpathlineto{\pgfqpoint{2.714314in}{2.683943in}}%
\pgfpathlineto{\pgfqpoint{2.715284in}{2.423671in}}%
\pgfpathlineto{\pgfqpoint{2.714468in}{2.778458in}}%
\pgfpathlineto{\pgfqpoint{2.715570in}{2.558396in}}%
\pgfpathlineto{\pgfqpoint{2.716408in}{2.397338in}}%
\pgfpathlineto{\pgfqpoint{2.715659in}{2.661325in}}%
\pgfpathlineto{\pgfqpoint{2.716496in}{2.538400in}}%
\pgfpathlineto{\pgfqpoint{2.717047in}{2.692684in}}%
\pgfpathlineto{\pgfqpoint{2.717532in}{2.424655in}}%
\pgfpathlineto{\pgfqpoint{2.717598in}{2.530752in}}%
\pgfpathlineto{\pgfqpoint{2.717885in}{2.647229in}}%
\pgfpathlineto{\pgfqpoint{2.717642in}{2.493929in}}%
\pgfpathlineto{\pgfqpoint{2.717907in}{2.587133in}}%
\pgfpathlineto{\pgfqpoint{2.718965in}{2.799437in}}%
\pgfpathlineto{\pgfqpoint{2.718282in}{2.463772in}}%
\pgfpathlineto{\pgfqpoint{2.719031in}{2.730599in}}%
\pgfpathlineto{\pgfqpoint{2.719979in}{2.474480in}}%
\pgfpathlineto{\pgfqpoint{2.719384in}{2.757479in}}%
\pgfpathlineto{\pgfqpoint{2.720133in}{2.626469in}}%
\pgfpathlineto{\pgfqpoint{2.721059in}{2.781080in}}%
\pgfpathlineto{\pgfqpoint{2.720993in}{2.514034in}}%
\pgfpathlineto{\pgfqpoint{2.721213in}{2.633789in}}%
\pgfpathlineto{\pgfqpoint{2.721786in}{2.751469in}}%
\pgfpathlineto{\pgfqpoint{2.722315in}{2.499720in}}%
\pgfpathlineto{\pgfqpoint{2.722492in}{2.771792in}}%
\pgfpathlineto{\pgfqpoint{2.722844in}{2.483003in}}%
\pgfpathlineto{\pgfqpoint{2.723439in}{2.747535in}}%
\pgfpathlineto{\pgfqpoint{2.723616in}{2.562220in}}%
\pgfpathlineto{\pgfqpoint{2.723814in}{2.818012in}}%
\pgfpathlineto{\pgfqpoint{2.724542in}{2.748191in}}%
\pgfpathlineto{\pgfqpoint{2.724564in}{2.818449in}}%
\pgfpathlineto{\pgfqpoint{2.725555in}{2.526818in}}%
\pgfpathlineto{\pgfqpoint{2.725622in}{2.657609in}}%
\pgfpathlineto{\pgfqpoint{2.725952in}{2.896137in}}%
\pgfpathlineto{\pgfqpoint{2.726129in}{2.541132in}}%
\pgfpathlineto{\pgfqpoint{2.726636in}{2.627780in}}%
\pgfpathlineto{\pgfqpoint{2.726680in}{2.559379in}}%
\pgfpathlineto{\pgfqpoint{2.726900in}{2.798672in}}%
\pgfpathlineto{\pgfqpoint{2.727694in}{2.685800in}}%
\pgfpathlineto{\pgfqpoint{2.728112in}{2.848934in}}%
\pgfpathlineto{\pgfqpoint{2.728421in}{2.619366in}}%
\pgfpathlineto{\pgfqpoint{2.728730in}{2.724808in}}%
\pgfpathlineto{\pgfqpoint{2.728774in}{2.561565in}}%
\pgfpathlineto{\pgfqpoint{2.729236in}{2.814734in}}%
\pgfpathlineto{\pgfqpoint{2.729854in}{2.670940in}}%
\pgfpathlineto{\pgfqpoint{2.730096in}{2.897776in}}%
\pgfpathlineto{\pgfqpoint{2.730471in}{2.591831in}}%
\pgfpathlineto{\pgfqpoint{2.730956in}{2.706014in}}%
\pgfpathlineto{\pgfqpoint{2.731837in}{2.554025in}}%
\pgfpathlineto{\pgfqpoint{2.731463in}{2.871880in}}%
\pgfpathlineto{\pgfqpoint{2.732036in}{2.759227in}}%
\pgfpathlineto{\pgfqpoint{2.732212in}{2.593907in}}%
\pgfpathlineto{\pgfqpoint{2.732411in}{2.870350in}}%
\pgfpathlineto{\pgfqpoint{2.733160in}{2.698256in}}%
\pgfpathlineto{\pgfqpoint{2.733402in}{2.842815in}}%
\pgfpathlineto{\pgfqpoint{2.733711in}{2.551840in}}%
\pgfpathlineto{\pgfqpoint{2.734240in}{2.643842in}}%
\pgfpathlineto{\pgfqpoint{2.735122in}{2.567684in}}%
\pgfpathlineto{\pgfqpoint{2.734527in}{2.900617in}}%
\pgfpathlineto{\pgfqpoint{2.735254in}{2.687985in}}%
\pgfpathlineto{\pgfqpoint{2.735474in}{2.794083in}}%
\pgfpathlineto{\pgfqpoint{2.736158in}{2.469344in}}%
\pgfpathlineto{\pgfqpoint{2.736224in}{2.601338in}}%
\pgfpathlineto{\pgfqpoint{2.736312in}{2.424436in}}%
\pgfpathlineto{\pgfqpoint{2.736731in}{2.815827in}}%
\pgfpathlineto{\pgfqpoint{2.737326in}{2.590629in}}%
\pgfpathlineto{\pgfqpoint{2.737723in}{2.831014in}}%
\pgfpathlineto{\pgfqpoint{2.738296in}{2.504856in}}%
\pgfpathlineto{\pgfqpoint{2.738494in}{2.726993in}}%
\pgfpathlineto{\pgfqpoint{2.738979in}{2.517093in}}%
\pgfpathlineto{\pgfqpoint{2.738648in}{2.757479in}}%
\pgfpathlineto{\pgfqpoint{2.739618in}{2.654441in}}%
\pgfpathlineto{\pgfqpoint{2.739993in}{2.462461in}}%
\pgfpathlineto{\pgfqpoint{2.740500in}{2.754747in}}%
\pgfpathlineto{\pgfqpoint{2.740742in}{2.585712in}}%
\pgfpathlineto{\pgfqpoint{2.741514in}{2.783156in}}%
\pgfpathlineto{\pgfqpoint{2.740853in}{2.503435in}}%
\pgfpathlineto{\pgfqpoint{2.741867in}{2.605380in}}%
\pgfpathlineto{\pgfqpoint{2.742153in}{2.718908in}}%
\pgfpathlineto{\pgfqpoint{2.742285in}{2.440717in}}%
\pgfpathlineto{\pgfqpoint{2.742418in}{2.542006in}}%
\pgfpathlineto{\pgfqpoint{2.742748in}{2.408811in}}%
\pgfpathlineto{\pgfqpoint{2.742638in}{2.605817in}}%
\pgfpathlineto{\pgfqpoint{2.743520in}{2.440170in}}%
\pgfpathlineto{\pgfqpoint{2.743938in}{2.726884in}}%
\pgfpathlineto{\pgfqpoint{2.744644in}{2.578610in}}%
\pgfpathlineto{\pgfqpoint{2.744666in}{2.495240in}}%
\pgfpathlineto{\pgfqpoint{2.745658in}{2.861937in}}%
\pgfpathlineto{\pgfqpoint{2.745724in}{2.550310in}}%
\pgfpathlineto{\pgfqpoint{2.746341in}{2.791351in}}%
\pgfpathlineto{\pgfqpoint{2.746121in}{2.444541in}}%
\pgfpathlineto{\pgfqpoint{2.746848in}{2.690171in}}%
\pgfpathlineto{\pgfqpoint{2.747223in}{2.557631in}}%
\pgfpathlineto{\pgfqpoint{2.747620in}{2.840084in}}%
\pgfpathlineto{\pgfqpoint{2.747972in}{2.640564in}}%
\pgfpathlineto{\pgfqpoint{2.748545in}{2.863904in}}%
\pgfpathlineto{\pgfqpoint{2.748854in}{2.561893in}}%
\pgfpathlineto{\pgfqpoint{2.748986in}{2.736281in}}%
\pgfpathlineto{\pgfqpoint{2.749846in}{2.547360in}}%
\pgfpathlineto{\pgfqpoint{2.749339in}{2.858768in}}%
\pgfpathlineto{\pgfqpoint{2.750088in}{2.680446in}}%
\pgfpathlineto{\pgfqpoint{2.750838in}{2.892640in}}%
\pgfpathlineto{\pgfqpoint{2.750419in}{2.495350in}}%
\pgfpathlineto{\pgfqpoint{2.751212in}{2.752343in}}%
\pgfpathlineto{\pgfqpoint{2.751940in}{2.838663in}}%
\pgfpathlineto{\pgfqpoint{2.752336in}{2.517858in}}%
\pgfpathlineto{\pgfqpoint{2.753064in}{2.768296in}}%
\pgfpathlineto{\pgfqpoint{2.753461in}{2.637832in}}%
\pgfpathlineto{\pgfqpoint{2.754298in}{2.486718in}}%
\pgfpathlineto{\pgfqpoint{2.753637in}{2.750267in}}%
\pgfpathlineto{\pgfqpoint{2.754585in}{2.582325in}}%
\pgfpathlineto{\pgfqpoint{2.755312in}{2.781845in}}%
\pgfpathlineto{\pgfqpoint{2.754651in}{2.458636in}}%
\pgfpathlineto{\pgfqpoint{2.755753in}{2.691482in}}%
\pgfpathlineto{\pgfqpoint{2.755907in}{2.495240in}}%
\pgfpathlineto{\pgfqpoint{2.756194in}{2.789603in}}%
\pgfpathlineto{\pgfqpoint{2.756833in}{2.680774in}}%
\pgfpathlineto{\pgfqpoint{2.757296in}{2.817793in}}%
\pgfpathlineto{\pgfqpoint{2.757825in}{2.522448in}}%
\pgfpathlineto{\pgfqpoint{2.757891in}{2.648868in}}%
\pgfpathlineto{\pgfqpoint{2.758156in}{2.452627in}}%
\pgfpathlineto{\pgfqpoint{2.758751in}{2.741963in}}%
\pgfpathlineto{\pgfqpoint{2.758993in}{2.617727in}}%
\pgfpathlineto{\pgfqpoint{2.759280in}{2.590083in}}%
\pgfpathlineto{\pgfqpoint{2.759676in}{2.799983in}}%
\pgfpathlineto{\pgfqpoint{2.759765in}{2.765892in}}%
\pgfpathlineto{\pgfqpoint{2.759787in}{2.825879in}}%
\pgfpathlineto{\pgfqpoint{2.760712in}{2.475026in}}%
\pgfpathlineto{\pgfqpoint{2.760845in}{2.713444in}}%
\pgfpathlineto{\pgfqpoint{2.761131in}{2.423343in}}%
\pgfpathlineto{\pgfqpoint{2.761660in}{2.773650in}}%
\pgfpathlineto{\pgfqpoint{2.761991in}{2.636740in}}%
\pgfpathlineto{\pgfqpoint{2.762718in}{2.805118in}}%
\pgfpathlineto{\pgfqpoint{2.762189in}{2.542880in}}%
\pgfpathlineto{\pgfqpoint{2.763093in}{2.607019in}}%
\pgfpathlineto{\pgfqpoint{2.763666in}{2.506167in}}%
\pgfpathlineto{\pgfqpoint{2.763798in}{2.863029in}}%
\pgfpathlineto{\pgfqpoint{2.764129in}{2.750704in}}%
\pgfpathlineto{\pgfqpoint{2.764702in}{2.478523in}}%
\pgfpathlineto{\pgfqpoint{2.765319in}{2.658374in}}%
\pgfpathlineto{\pgfqpoint{2.766311in}{2.830905in}}%
\pgfpathlineto{\pgfqpoint{2.765363in}{2.580249in}}%
\pgfpathlineto{\pgfqpoint{2.766421in}{2.757697in}}%
\pgfpathlineto{\pgfqpoint{2.767193in}{2.543536in}}%
\pgfpathlineto{\pgfqpoint{2.766686in}{2.820962in}}%
\pgfpathlineto{\pgfqpoint{2.767523in}{2.696290in}}%
\pgfpathlineto{\pgfqpoint{2.768097in}{2.904878in}}%
\pgfpathlineto{\pgfqpoint{2.768251in}{2.519279in}}%
\pgfpathlineto{\pgfqpoint{2.768626in}{2.675638in}}%
\pgfpathlineto{\pgfqpoint{2.769088in}{2.437002in}}%
\pgfpathlineto{\pgfqpoint{2.768824in}{2.783484in}}%
\pgfpathlineto{\pgfqpoint{2.769772in}{2.555009in}}%
\pgfpathlineto{\pgfqpoint{2.770257in}{2.724480in}}%
\pgfpathlineto{\pgfqpoint{2.770764in}{2.362591in}}%
\pgfpathlineto{\pgfqpoint{2.770874in}{2.537308in}}%
\pgfpathlineto{\pgfqpoint{2.771160in}{2.747754in}}%
\pgfpathlineto{\pgfqpoint{2.771381in}{2.507697in}}%
\pgfpathlineto{\pgfqpoint{2.771866in}{2.540476in}}%
\pgfpathlineto{\pgfqpoint{2.771954in}{2.429571in}}%
\pgfpathlineto{\pgfqpoint{2.772351in}{2.750595in}}%
\pgfpathlineto{\pgfqpoint{2.772946in}{2.606692in}}%
\pgfpathlineto{\pgfqpoint{2.773497in}{2.837680in}}%
\pgfpathlineto{\pgfqpoint{2.773541in}{2.504528in}}%
\pgfpathlineto{\pgfqpoint{2.774092in}{2.645372in}}%
\pgfpathlineto{\pgfqpoint{2.775194in}{2.564624in}}%
\pgfpathlineto{\pgfqpoint{2.774863in}{2.796705in}}%
\pgfpathlineto{\pgfqpoint{2.775216in}{2.610079in}}%
\pgfpathlineto{\pgfqpoint{2.775899in}{2.834839in}}%
\pgfpathlineto{\pgfqpoint{2.775348in}{2.510865in}}%
\pgfpathlineto{\pgfqpoint{2.776384in}{2.692247in}}%
\pgfpathlineto{\pgfqpoint{2.776935in}{2.449786in}}%
\pgfpathlineto{\pgfqpoint{2.777156in}{2.751250in}}%
\pgfpathlineto{\pgfqpoint{2.777508in}{2.557085in}}%
\pgfpathlineto{\pgfqpoint{2.777751in}{2.806867in}}%
\pgfpathlineto{\pgfqpoint{2.777619in}{2.530752in}}%
\pgfpathlineto{\pgfqpoint{2.778500in}{2.666788in}}%
\pgfpathlineto{\pgfqpoint{2.778522in}{2.483658in}}%
\pgfpathlineto{\pgfqpoint{2.778831in}{2.781080in}}%
\pgfpathlineto{\pgfqpoint{2.779602in}{2.596858in}}%
\pgfpathlineto{\pgfqpoint{2.780242in}{2.756932in}}%
\pgfpathlineto{\pgfqpoint{2.780660in}{2.511849in}}%
\pgfpathlineto{\pgfqpoint{2.780727in}{2.826862in}}%
\pgfpathlineto{\pgfqpoint{2.781256in}{2.409794in}}%
\pgfpathlineto{\pgfqpoint{2.781785in}{2.693121in}}%
\pgfpathlineto{\pgfqpoint{2.782512in}{2.489996in}}%
\pgfpathlineto{\pgfqpoint{2.782710in}{2.754856in}}%
\pgfpathlineto{\pgfqpoint{2.782887in}{2.659904in}}%
\pgfpathlineto{\pgfqpoint{2.782975in}{2.698256in}}%
\pgfpathlineto{\pgfqpoint{2.782997in}{2.628217in}}%
\pgfpathlineto{\pgfqpoint{2.783173in}{2.531735in}}%
\pgfpathlineto{\pgfqpoint{2.783504in}{2.836696in}}%
\pgfpathlineto{\pgfqpoint{2.784077in}{2.670394in}}%
\pgfpathlineto{\pgfqpoint{2.784518in}{2.903567in}}%
\pgfpathlineto{\pgfqpoint{2.784760in}{2.489558in}}%
\pgfpathlineto{\pgfqpoint{2.785113in}{2.757260in}}%
\pgfpathlineto{\pgfqpoint{2.785598in}{2.554572in}}%
\pgfpathlineto{\pgfqpoint{2.785510in}{2.873847in}}%
\pgfpathlineto{\pgfqpoint{2.786237in}{2.630074in}}%
\pgfpathlineto{\pgfqpoint{2.786700in}{2.830359in}}%
\pgfpathlineto{\pgfqpoint{2.786435in}{2.577080in}}%
\pgfpathlineto{\pgfqpoint{2.787339in}{2.757479in}}%
\pgfpathlineto{\pgfqpoint{2.787890in}{2.550857in}}%
\pgfpathlineto{\pgfqpoint{2.788067in}{2.883134in}}%
\pgfpathlineto{\pgfqpoint{2.788441in}{2.774305in}}%
\pgfpathlineto{\pgfqpoint{2.789103in}{2.553370in}}%
\pgfpathlineto{\pgfqpoint{2.788750in}{2.835167in}}%
\pgfpathlineto{\pgfqpoint{2.789654in}{2.630074in}}%
\pgfpathlineto{\pgfqpoint{2.790271in}{2.834074in}}%
\pgfpathlineto{\pgfqpoint{2.789698in}{2.566700in}}%
\pgfpathlineto{\pgfqpoint{2.790778in}{2.765018in}}%
\pgfpathlineto{\pgfqpoint{2.791174in}{2.502780in}}%
\pgfpathlineto{\pgfqpoint{2.791373in}{2.914494in}}%
\pgfpathlineto{\pgfqpoint{2.791902in}{2.652146in}}%
\pgfpathlineto{\pgfqpoint{2.791968in}{2.763270in}}%
\pgfpathlineto{\pgfqpoint{2.792012in}{2.542443in}}%
\pgfpathlineto{\pgfqpoint{2.792673in}{2.586477in}}%
\pgfpathlineto{\pgfqpoint{2.792695in}{2.515017in}}%
\pgfpathlineto{\pgfqpoint{2.793158in}{2.804463in}}%
\pgfpathlineto{\pgfqpoint{2.793687in}{2.802168in}}%
\pgfpathlineto{\pgfqpoint{2.793709in}{2.835822in}}%
\pgfpathlineto{\pgfqpoint{2.794481in}{2.517312in}}%
\pgfpathlineto{\pgfqpoint{2.794591in}{2.656189in}}%
\pgfpathlineto{\pgfqpoint{2.794613in}{2.514253in}}%
\pgfpathlineto{\pgfqpoint{2.795517in}{2.869804in}}%
\pgfpathlineto{\pgfqpoint{2.795671in}{2.763379in}}%
\pgfpathlineto{\pgfqpoint{2.796531in}{2.553370in}}%
\pgfpathlineto{\pgfqpoint{2.796002in}{2.824459in}}%
\pgfpathlineto{\pgfqpoint{2.796751in}{2.734096in}}%
\pgfpathlineto{\pgfqpoint{2.796773in}{2.782610in}}%
\pgfpathlineto{\pgfqpoint{2.797170in}{2.496114in}}%
\pgfpathlineto{\pgfqpoint{2.797809in}{2.645699in}}%
\pgfpathlineto{\pgfqpoint{2.798052in}{2.700770in}}%
\pgfpathlineto{\pgfqpoint{2.798338in}{2.538619in}}%
\pgfpathlineto{\pgfqpoint{2.798669in}{2.651709in}}%
\pgfpathlineto{\pgfqpoint{2.798999in}{2.465192in}}%
\pgfpathlineto{\pgfqpoint{2.799727in}{2.776163in}}%
\pgfpathlineto{\pgfqpoint{2.799749in}{2.623409in}}%
\pgfpathlineto{\pgfqpoint{2.800013in}{2.762942in}}%
\pgfpathlineto{\pgfqpoint{2.800057in}{2.521464in}}%
\pgfpathlineto{\pgfqpoint{2.800851in}{2.597841in}}%
\pgfpathlineto{\pgfqpoint{2.801424in}{2.832107in}}%
\pgfpathlineto{\pgfqpoint{2.801512in}{2.532391in}}%
\pgfpathlineto{\pgfqpoint{2.801975in}{2.671705in}}%
\pgfpathlineto{\pgfqpoint{2.802019in}{2.560035in}}%
\pgfpathlineto{\pgfqpoint{2.802504in}{2.851229in}}%
\pgfpathlineto{\pgfqpoint{2.802526in}{2.894935in}}%
\pgfpathlineto{\pgfqpoint{2.802702in}{2.540476in}}%
\pgfpathlineto{\pgfqpoint{2.803540in}{2.814843in}}%
\pgfpathlineto{\pgfqpoint{2.804135in}{2.595546in}}%
\pgfpathlineto{\pgfqpoint{2.804245in}{2.846530in}}%
\pgfpathlineto{\pgfqpoint{2.804664in}{2.602539in}}%
\pgfpathlineto{\pgfqpoint{2.805744in}{2.567684in}}%
\pgfpathlineto{\pgfqpoint{2.805810in}{2.839537in}}%
\pgfpathlineto{\pgfqpoint{2.806736in}{2.555555in}}%
\pgfpathlineto{\pgfqpoint{2.806802in}{2.868711in}}%
\pgfpathlineto{\pgfqpoint{2.807001in}{2.573147in}}%
\pgfpathlineto{\pgfqpoint{2.807662in}{2.848934in}}%
\pgfpathlineto{\pgfqpoint{2.807067in}{2.566919in}}%
\pgfpathlineto{\pgfqpoint{2.808125in}{2.665477in}}%
\pgfpathlineto{\pgfqpoint{2.809117in}{2.838007in}}%
\pgfpathlineto{\pgfqpoint{2.808918in}{2.563313in}}%
\pgfpathlineto{\pgfqpoint{2.809227in}{2.728851in}}%
\pgfpathlineto{\pgfqpoint{2.809403in}{2.600682in}}%
\pgfpathlineto{\pgfqpoint{2.809844in}{2.904223in}}%
\pgfpathlineto{\pgfqpoint{2.810329in}{2.689515in}}%
\pgfpathlineto{\pgfqpoint{2.810373in}{2.864778in}}%
\pgfpathlineto{\pgfqpoint{2.810571in}{2.625376in}}%
\pgfpathlineto{\pgfqpoint{2.811431in}{2.683396in}}%
\pgfpathlineto{\pgfqpoint{2.811585in}{2.567137in}}%
\pgfpathlineto{\pgfqpoint{2.811894in}{2.904878in}}%
\pgfpathlineto{\pgfqpoint{2.812423in}{2.764253in}}%
\pgfpathlineto{\pgfqpoint{2.812577in}{2.846530in}}%
\pgfpathlineto{\pgfqpoint{2.812974in}{2.569323in}}%
\pgfpathlineto{\pgfqpoint{2.813503in}{2.738357in}}%
\pgfpathlineto{\pgfqpoint{2.814010in}{2.556429in}}%
\pgfpathlineto{\pgfqpoint{2.813569in}{2.926950in}}%
\pgfpathlineto{\pgfqpoint{2.814495in}{2.749939in}}%
\pgfpathlineto{\pgfqpoint{2.814517in}{2.924655in}}%
\pgfpathlineto{\pgfqpoint{2.815156in}{2.591176in}}%
\pgfpathlineto{\pgfqpoint{2.815597in}{2.777256in}}%
\pgfpathlineto{\pgfqpoint{2.816214in}{2.863357in}}%
\pgfpathlineto{\pgfqpoint{2.816479in}{2.585166in}}%
\pgfpathlineto{\pgfqpoint{2.816655in}{2.813641in}}%
\pgfpathlineto{\pgfqpoint{2.817426in}{2.575223in}}%
\pgfpathlineto{\pgfqpoint{2.816964in}{2.865980in}}%
\pgfpathlineto{\pgfqpoint{2.817779in}{2.727321in}}%
\pgfpathlineto{\pgfqpoint{2.818088in}{2.890783in}}%
\pgfpathlineto{\pgfqpoint{2.818374in}{2.571617in}}%
\pgfpathlineto{\pgfqpoint{2.818881in}{2.757916in}}%
\pgfpathlineto{\pgfqpoint{2.819741in}{2.536980in}}%
\pgfpathlineto{\pgfqpoint{2.819102in}{2.900508in}}%
\pgfpathlineto{\pgfqpoint{2.819961in}{2.695525in}}%
\pgfpathlineto{\pgfqpoint{2.820182in}{2.893624in}}%
\pgfpathlineto{\pgfqpoint{2.820799in}{2.585822in}}%
\pgfpathlineto{\pgfqpoint{2.821063in}{2.769607in}}%
\pgfpathlineto{\pgfqpoint{2.822143in}{2.579921in}}%
\pgfpathlineto{\pgfqpoint{2.821526in}{2.856473in}}%
\pgfpathlineto{\pgfqpoint{2.822166in}{2.693886in}}%
\pgfpathlineto{\pgfqpoint{2.823091in}{2.903786in}}%
\pgfpathlineto{\pgfqpoint{2.822959in}{2.608658in}}%
\pgfpathlineto{\pgfqpoint{2.823268in}{2.744585in}}%
\pgfpathlineto{\pgfqpoint{2.824171in}{2.551075in}}%
\pgfpathlineto{\pgfqpoint{2.823510in}{2.894389in}}%
\pgfpathlineto{\pgfqpoint{2.824392in}{2.623518in}}%
\pgfpathlineto{\pgfqpoint{2.824524in}{2.610297in}}%
\pgfpathlineto{\pgfqpoint{2.824678in}{2.818995in}}%
\pgfpathlineto{\pgfqpoint{2.825494in}{2.912308in}}%
\pgfpathlineto{\pgfqpoint{2.825295in}{2.519170in}}%
\pgfpathlineto{\pgfqpoint{2.825692in}{2.664493in}}%
\pgfpathlineto{\pgfqpoint{2.826376in}{2.907173in}}%
\pgfpathlineto{\pgfqpoint{2.825979in}{2.623081in}}%
\pgfpathlineto{\pgfqpoint{2.826750in}{2.682085in}}%
\pgfpathlineto{\pgfqpoint{2.827500in}{2.591613in}}%
\pgfpathlineto{\pgfqpoint{2.827698in}{2.864996in}}%
\pgfpathlineto{\pgfqpoint{2.827852in}{2.690826in}}%
\pgfpathlineto{\pgfqpoint{2.828029in}{2.594017in}}%
\pgfpathlineto{\pgfqpoint{2.828315in}{2.881714in}}%
\pgfpathlineto{\pgfqpoint{2.828756in}{2.845984in}}%
\pgfpathlineto{\pgfqpoint{2.828822in}{2.963882in}}%
\pgfpathlineto{\pgfqpoint{2.828976in}{2.639471in}}%
\pgfpathlineto{\pgfqpoint{2.829814in}{2.771683in}}%
\pgfpathlineto{\pgfqpoint{2.830541in}{2.534904in}}%
\pgfpathlineto{\pgfqpoint{2.830365in}{2.869804in}}%
\pgfpathlineto{\pgfqpoint{2.830894in}{2.818886in}}%
\pgfpathlineto{\pgfqpoint{2.830960in}{2.885757in}}%
\pgfpathlineto{\pgfqpoint{2.831004in}{2.797798in}}%
\pgfpathlineto{\pgfqpoint{2.831599in}{2.512395in}}%
\pgfpathlineto{\pgfqpoint{2.832084in}{2.883899in}}%
\pgfpathlineto{\pgfqpoint{2.832128in}{2.728086in}}%
\pgfpathlineto{\pgfqpoint{2.832966in}{2.869585in}}%
\pgfpathlineto{\pgfqpoint{2.833120in}{2.629637in}}%
\pgfpathlineto{\pgfqpoint{2.833253in}{2.531080in}}%
\pgfpathlineto{\pgfqpoint{2.834112in}{2.819869in}}%
\pgfpathlineto{\pgfqpoint{2.834134in}{2.910779in}}%
\pgfpathlineto{\pgfqpoint{2.834443in}{2.579921in}}%
\pgfpathlineto{\pgfqpoint{2.835192in}{2.728414in}}%
\pgfpathlineto{\pgfqpoint{2.835413in}{2.535122in}}%
\pgfpathlineto{\pgfqpoint{2.836272in}{2.813532in}}%
\pgfpathlineto{\pgfqpoint{2.836294in}{2.785997in}}%
\pgfpathlineto{\pgfqpoint{2.837242in}{2.595874in}}%
\pgfpathlineto{\pgfqpoint{2.836868in}{2.842597in}}%
\pgfpathlineto{\pgfqpoint{2.837419in}{2.683068in}}%
\pgfpathlineto{\pgfqpoint{2.837485in}{2.858549in}}%
\pgfpathlineto{\pgfqpoint{2.837617in}{2.614996in}}%
\pgfpathlineto{\pgfqpoint{2.837926in}{2.652365in}}%
\pgfpathlineto{\pgfqpoint{2.837948in}{2.507806in}}%
\pgfpathlineto{\pgfqpoint{2.838234in}{2.829813in}}%
\pgfpathlineto{\pgfqpoint{2.839050in}{2.560144in}}%
\pgfpathlineto{\pgfqpoint{2.839182in}{2.500813in}}%
\pgfpathlineto{\pgfqpoint{2.839557in}{2.731473in}}%
\pgfpathlineto{\pgfqpoint{2.839645in}{2.715848in}}%
\pgfpathlineto{\pgfqpoint{2.840108in}{2.866198in}}%
\pgfpathlineto{\pgfqpoint{2.840284in}{2.614668in}}%
\pgfpathlineto{\pgfqpoint{2.840769in}{2.767422in}}%
\pgfpathlineto{\pgfqpoint{2.841386in}{2.822929in}}%
\pgfpathlineto{\pgfqpoint{2.840857in}{2.557850in}}%
\pgfpathlineto{\pgfqpoint{2.841827in}{2.818449in}}%
\pgfpathlineto{\pgfqpoint{2.842510in}{2.567465in}}%
\pgfpathlineto{\pgfqpoint{2.842180in}{2.836587in}}%
\pgfpathlineto{\pgfqpoint{2.842951in}{2.713444in}}%
\pgfpathlineto{\pgfqpoint{2.843127in}{2.863685in}}%
\pgfpathlineto{\pgfqpoint{2.843326in}{2.568339in}}%
\pgfpathlineto{\pgfqpoint{2.843921in}{2.640564in}}%
\pgfpathlineto{\pgfqpoint{2.844869in}{2.495131in}}%
\pgfpathlineto{\pgfqpoint{2.844384in}{2.752452in}}%
\pgfpathlineto{\pgfqpoint{2.845023in}{2.611827in}}%
\pgfpathlineto{\pgfqpoint{2.845772in}{2.727212in}}%
\pgfpathlineto{\pgfqpoint{2.846037in}{2.457871in}}%
\pgfpathlineto{\pgfqpoint{2.846147in}{2.659795in}}%
\pgfpathlineto{\pgfqpoint{2.846588in}{2.501359in}}%
\pgfpathlineto{\pgfqpoint{2.846301in}{2.727977in}}%
\pgfpathlineto{\pgfqpoint{2.847271in}{2.617072in}}%
\pgfpathlineto{\pgfqpoint{2.847844in}{2.799109in}}%
\pgfpathlineto{\pgfqpoint{2.848285in}{2.547032in}}%
\pgfpathlineto{\pgfqpoint{2.848418in}{2.661871in}}%
\pgfpathlineto{\pgfqpoint{2.848682in}{2.824786in}}%
\pgfpathlineto{\pgfqpoint{2.848506in}{2.514253in}}%
\pgfpathlineto{\pgfqpoint{2.849299in}{2.650616in}}%
\pgfpathlineto{\pgfqpoint{2.849343in}{2.505402in}}%
\pgfpathlineto{\pgfqpoint{2.849586in}{2.803261in}}%
\pgfpathlineto{\pgfqpoint{2.850401in}{2.653785in}}%
\pgfpathlineto{\pgfqpoint{2.850600in}{2.512614in}}%
\pgfpathlineto{\pgfqpoint{2.850556in}{2.758243in}}%
\pgfpathlineto{\pgfqpoint{2.851459in}{2.596639in}}%
\pgfpathlineto{\pgfqpoint{2.852143in}{2.852103in}}%
\pgfpathlineto{\pgfqpoint{2.851988in}{2.526490in}}%
\pgfpathlineto{\pgfqpoint{2.852561in}{2.661215in}}%
\pgfpathlineto{\pgfqpoint{2.852583in}{2.621989in}}%
\pgfpathlineto{\pgfqpoint{2.853201in}{2.878982in}}%
\pgfpathlineto{\pgfqpoint{2.853509in}{2.796705in}}%
\pgfpathlineto{\pgfqpoint{2.853531in}{2.948912in}}%
\pgfpathlineto{\pgfqpoint{2.854193in}{2.653130in}}%
\pgfpathlineto{\pgfqpoint{2.854611in}{2.761412in}}%
\pgfpathlineto{\pgfqpoint{2.855515in}{2.906626in}}%
\pgfpathlineto{\pgfqpoint{2.855295in}{2.624174in}}%
\pgfpathlineto{\pgfqpoint{2.855647in}{2.723169in}}%
\pgfpathlineto{\pgfqpoint{2.856132in}{2.923672in}}%
\pgfpathlineto{\pgfqpoint{2.856771in}{2.539274in}}%
\pgfpathlineto{\pgfqpoint{2.857852in}{2.807850in}}%
\pgfpathlineto{\pgfqpoint{2.857896in}{2.644825in}}%
\pgfpathlineto{\pgfqpoint{2.857918in}{2.642531in}}%
\pgfpathlineto{\pgfqpoint{2.857940in}{2.701534in}}%
\pgfpathlineto{\pgfqpoint{2.858755in}{2.887942in}}%
\pgfpathlineto{\pgfqpoint{2.858799in}{2.633243in}}%
\pgfpathlineto{\pgfqpoint{2.859064in}{2.842160in}}%
\pgfpathlineto{\pgfqpoint{2.859769in}{2.558942in}}%
\pgfpathlineto{\pgfqpoint{2.859879in}{2.877015in}}%
\pgfpathlineto{\pgfqpoint{2.860188in}{2.702627in}}%
\pgfpathlineto{\pgfqpoint{2.860959in}{2.595765in}}%
\pgfpathlineto{\pgfqpoint{2.860827in}{2.870897in}}%
\pgfpathlineto{\pgfqpoint{2.861268in}{2.688750in}}%
\pgfpathlineto{\pgfqpoint{2.861488in}{2.894389in}}%
\pgfpathlineto{\pgfqpoint{2.862150in}{2.571508in}}%
\pgfpathlineto{\pgfqpoint{2.862392in}{2.789275in}}%
\pgfpathlineto{\pgfqpoint{2.862921in}{2.604069in}}%
\pgfpathlineto{\pgfqpoint{2.862745in}{2.890127in}}%
\pgfpathlineto{\pgfqpoint{2.863516in}{2.723715in}}%
\pgfpathlineto{\pgfqpoint{2.864376in}{2.856910in}}%
\pgfpathlineto{\pgfqpoint{2.864178in}{2.589318in}}%
\pgfpathlineto{\pgfqpoint{2.864618in}{2.763925in}}%
\pgfpathlineto{\pgfqpoint{2.865147in}{2.877452in}}%
\pgfpathlineto{\pgfqpoint{2.865720in}{2.603851in}}%
\pgfpathlineto{\pgfqpoint{2.866161in}{2.887068in}}%
\pgfpathlineto{\pgfqpoint{2.865963in}{2.548344in}}%
\pgfpathlineto{\pgfqpoint{2.866845in}{2.835276in}}%
\pgfpathlineto{\pgfqpoint{2.867947in}{2.548344in}}%
\pgfpathlineto{\pgfqpoint{2.867241in}{2.942903in}}%
\pgfpathlineto{\pgfqpoint{2.867969in}{2.665586in}}%
\pgfpathlineto{\pgfqpoint{2.868013in}{2.566700in}}%
\pgfpathlineto{\pgfqpoint{2.868476in}{2.815499in}}%
\pgfpathlineto{\pgfqpoint{2.868784in}{2.678042in}}%
\pgfpathlineto{\pgfqpoint{2.868895in}{2.856910in}}%
\pgfpathlineto{\pgfqpoint{2.869468in}{2.523540in}}%
\pgfpathlineto{\pgfqpoint{2.869864in}{2.623846in}}%
\pgfpathlineto{\pgfqpoint{2.870437in}{2.583636in}}%
\pgfpathlineto{\pgfqpoint{2.869997in}{2.841832in}}%
\pgfpathlineto{\pgfqpoint{2.870834in}{2.647448in}}%
\pgfpathlineto{\pgfqpoint{2.871518in}{2.808396in}}%
\pgfpathlineto{\pgfqpoint{2.871760in}{2.564296in}}%
\pgfpathlineto{\pgfqpoint{2.871914in}{2.719673in}}%
\pgfpathlineto{\pgfqpoint{2.871936in}{2.589974in}}%
\pgfpathlineto{\pgfqpoint{2.872355in}{2.861063in}}%
\pgfpathlineto{\pgfqpoint{2.873016in}{2.744804in}}%
\pgfpathlineto{\pgfqpoint{2.874030in}{2.559707in}}%
\pgfpathlineto{\pgfqpoint{2.873060in}{2.802278in}}%
\pgfpathlineto{\pgfqpoint{2.874163in}{2.630074in}}%
\pgfpathlineto{\pgfqpoint{2.875000in}{2.915368in}}%
\pgfpathlineto{\pgfqpoint{2.874317in}{2.545066in}}%
\pgfpathlineto{\pgfqpoint{2.875265in}{2.604725in}}%
\pgfpathlineto{\pgfqpoint{2.875661in}{2.835931in}}%
\pgfpathlineto{\pgfqpoint{2.876014in}{2.522120in}}%
\pgfpathlineto{\pgfqpoint{2.876477in}{2.780097in}}%
\pgfpathlineto{\pgfqpoint{2.877072in}{2.646137in}}%
\pgfpathlineto{\pgfqpoint{2.877469in}{2.888598in}}%
\pgfpathlineto{\pgfqpoint{2.877601in}{2.688095in}}%
\pgfpathlineto{\pgfqpoint{2.877711in}{2.871006in}}%
\pgfpathlineto{\pgfqpoint{2.878461in}{2.556976in}}%
\pgfpathlineto{\pgfqpoint{2.878681in}{2.678261in}}%
\pgfpathlineto{\pgfqpoint{2.879673in}{2.571836in}}%
\pgfpathlineto{\pgfqpoint{2.879034in}{2.848606in}}%
\pgfpathlineto{\pgfqpoint{2.879739in}{2.602430in}}%
\pgfpathlineto{\pgfqpoint{2.880070in}{2.852758in}}%
\pgfpathlineto{\pgfqpoint{2.880136in}{2.564843in}}%
\pgfpathlineto{\pgfqpoint{2.880863in}{2.766438in}}%
\pgfpathlineto{\pgfqpoint{2.881084in}{2.468033in}}%
\pgfpathlineto{\pgfqpoint{2.881348in}{2.815936in}}%
\pgfpathlineto{\pgfqpoint{2.881965in}{2.786434in}}%
\pgfpathlineto{\pgfqpoint{2.882472in}{2.497098in}}%
\pgfpathlineto{\pgfqpoint{2.882583in}{2.849043in}}%
\pgfpathlineto{\pgfqpoint{2.883112in}{2.661325in}}%
\pgfpathlineto{\pgfqpoint{2.883685in}{2.825988in}}%
\pgfpathlineto{\pgfqpoint{2.883222in}{2.506822in}}%
\pgfpathlineto{\pgfqpoint{2.884104in}{2.509008in}}%
\pgfpathlineto{\pgfqpoint{2.884677in}{2.952955in}}%
\pgfpathlineto{\pgfqpoint{2.884258in}{2.494257in}}%
\pgfpathlineto{\pgfqpoint{2.885228in}{2.601775in}}%
\pgfpathlineto{\pgfqpoint{2.885294in}{2.473387in}}%
\pgfpathlineto{\pgfqpoint{2.885867in}{2.873300in}}%
\pgfpathlineto{\pgfqpoint{2.886308in}{2.660778in}}%
\pgfpathlineto{\pgfqpoint{2.887057in}{2.771465in}}%
\pgfpathlineto{\pgfqpoint{2.886660in}{2.525835in}}%
\pgfpathlineto{\pgfqpoint{2.887366in}{2.650179in}}%
\pgfpathlineto{\pgfqpoint{2.887807in}{2.510647in}}%
\pgfpathlineto{\pgfqpoint{2.888115in}{2.837461in}}%
\pgfpathlineto{\pgfqpoint{2.888468in}{2.673999in}}%
\pgfpathlineto{\pgfqpoint{2.889460in}{2.517968in}}%
\pgfpathlineto{\pgfqpoint{2.888821in}{2.803698in}}%
\pgfpathlineto{\pgfqpoint{2.889548in}{2.644170in}}%
\pgfpathlineto{\pgfqpoint{2.890055in}{2.775835in}}%
\pgfpathlineto{\pgfqpoint{2.890319in}{2.429462in}}%
\pgfpathlineto{\pgfqpoint{2.890672in}{2.764362in}}%
\pgfpathlineto{\pgfqpoint{2.891003in}{2.534467in}}%
\pgfpathlineto{\pgfqpoint{2.891554in}{2.776928in}}%
\pgfpathlineto{\pgfqpoint{2.891774in}{2.735625in}}%
\pgfpathlineto{\pgfqpoint{2.892303in}{2.809271in}}%
\pgfpathlineto{\pgfqpoint{2.892127in}{2.520044in}}%
\pgfpathlineto{\pgfqpoint{2.892435in}{2.642531in}}%
\pgfpathlineto{\pgfqpoint{2.893119in}{2.526709in}}%
\pgfpathlineto{\pgfqpoint{2.893295in}{2.871224in}}%
\pgfpathlineto{\pgfqpoint{2.893449in}{2.698912in}}%
\pgfpathlineto{\pgfqpoint{2.894199in}{2.864231in}}%
\pgfpathlineto{\pgfqpoint{2.893670in}{2.584948in}}%
\pgfpathlineto{\pgfqpoint{2.894551in}{2.731036in}}%
\pgfpathlineto{\pgfqpoint{2.894860in}{2.557850in}}%
\pgfpathlineto{\pgfqpoint{2.895455in}{2.864778in}}%
\pgfpathlineto{\pgfqpoint{2.895521in}{2.786543in}}%
\pgfpathlineto{\pgfqpoint{2.896425in}{3.002780in}}%
\pgfpathlineto{\pgfqpoint{2.895742in}{2.627671in}}%
\pgfpathlineto{\pgfqpoint{2.896645in}{2.916570in}}%
\pgfpathlineto{\pgfqpoint{2.897086in}{2.640673in}}%
\pgfpathlineto{\pgfqpoint{2.896910in}{2.918755in}}%
\pgfpathlineto{\pgfqpoint{2.897858in}{2.787854in}}%
\pgfpathlineto{\pgfqpoint{2.898078in}{2.626250in}}%
\pgfpathlineto{\pgfqpoint{2.898982in}{2.913073in}}%
\pgfpathlineto{\pgfqpoint{2.900040in}{2.598387in}}%
\pgfpathlineto{\pgfqpoint{2.899621in}{2.968580in}}%
\pgfpathlineto{\pgfqpoint{2.900128in}{2.788401in}}%
\pgfpathlineto{\pgfqpoint{2.901142in}{2.905862in}}%
\pgfpathlineto{\pgfqpoint{2.900525in}{2.586040in}}%
\pgfpathlineto{\pgfqpoint{2.901186in}{2.806211in}}%
\pgfpathlineto{\pgfqpoint{2.901429in}{2.854397in}}%
\pgfpathlineto{\pgfqpoint{2.902332in}{2.596530in}}%
\pgfpathlineto{\pgfqpoint{2.902883in}{2.923563in}}%
\pgfpathlineto{\pgfqpoint{2.903478in}{2.763379in}}%
\pgfpathlineto{\pgfqpoint{2.904074in}{2.615214in}}%
\pgfpathlineto{\pgfqpoint{2.903897in}{2.886412in}}%
\pgfpathlineto{\pgfqpoint{2.904558in}{2.724917in}}%
\pgfpathlineto{\pgfqpoint{2.904581in}{2.836915in}}%
\pgfpathlineto{\pgfqpoint{2.904999in}{2.575988in}}%
\pgfpathlineto{\pgfqpoint{2.905661in}{2.721639in}}%
\pgfpathlineto{\pgfqpoint{2.906168in}{2.488138in}}%
\pgfpathlineto{\pgfqpoint{2.905969in}{2.882588in}}%
\pgfpathlineto{\pgfqpoint{2.906785in}{2.516766in}}%
\pgfpathlineto{\pgfqpoint{2.907931in}{2.842160in}}%
\pgfpathlineto{\pgfqpoint{2.907975in}{2.720874in}}%
\pgfpathlineto{\pgfqpoint{2.908702in}{2.501250in}}%
\pgfpathlineto{\pgfqpoint{2.908813in}{2.808069in}}%
\pgfpathlineto{\pgfqpoint{2.909121in}{2.540367in}}%
\pgfpathlineto{\pgfqpoint{2.910113in}{2.805009in}}%
\pgfpathlineto{\pgfqpoint{2.909320in}{2.458308in}}%
\pgfpathlineto{\pgfqpoint{2.910245in}{2.631167in}}%
\pgfpathlineto{\pgfqpoint{2.911017in}{2.403566in}}%
\pgfpathlineto{\pgfqpoint{2.911303in}{2.739777in}}%
\pgfpathlineto{\pgfqpoint{2.911458in}{2.631167in}}%
\pgfpathlineto{\pgfqpoint{2.911392in}{2.780534in}}%
\pgfpathlineto{\pgfqpoint{2.911502in}{2.728960in}}%
\pgfpathlineto{\pgfqpoint{2.911612in}{2.432303in}}%
\pgfpathlineto{\pgfqpoint{2.911722in}{2.829157in}}%
\pgfpathlineto{\pgfqpoint{2.912604in}{2.682522in}}%
\pgfpathlineto{\pgfqpoint{2.913287in}{2.864996in}}%
\pgfpathlineto{\pgfqpoint{2.912868in}{2.552277in}}%
\pgfpathlineto{\pgfqpoint{2.913684in}{2.635975in}}%
\pgfpathlineto{\pgfqpoint{2.913706in}{2.592050in}}%
\pgfpathlineto{\pgfqpoint{2.913926in}{2.778348in}}%
\pgfpathlineto{\pgfqpoint{2.914786in}{2.621880in}}%
\pgfpathlineto{\pgfqpoint{2.915249in}{2.875704in}}%
\pgfpathlineto{\pgfqpoint{2.915095in}{2.530861in}}%
\pgfpathlineto{\pgfqpoint{2.915910in}{2.713444in}}%
\pgfpathlineto{\pgfqpoint{2.916395in}{2.461368in}}%
\pgfpathlineto{\pgfqpoint{2.916042in}{2.809489in}}%
\pgfpathlineto{\pgfqpoint{2.916682in}{2.612373in}}%
\pgfpathlineto{\pgfqpoint{2.916968in}{2.830687in}}%
\pgfpathlineto{\pgfqpoint{2.917034in}{2.493055in}}%
\pgfpathlineto{\pgfqpoint{2.917806in}{2.736609in}}%
\pgfpathlineto{\pgfqpoint{2.917850in}{2.546049in}}%
\pgfpathlineto{\pgfqpoint{2.918136in}{2.888816in}}%
\pgfpathlineto{\pgfqpoint{2.918886in}{2.669738in}}%
\pgfpathlineto{\pgfqpoint{2.919723in}{2.844563in}}%
\pgfpathlineto{\pgfqpoint{2.919459in}{2.556101in}}%
\pgfpathlineto{\pgfqpoint{2.919988in}{2.679026in}}%
\pgfpathlineto{\pgfqpoint{2.920561in}{2.856910in}}%
\pgfpathlineto{\pgfqpoint{2.920825in}{2.631167in}}%
\pgfpathlineto{\pgfqpoint{2.921068in}{2.749830in}}%
\pgfpathlineto{\pgfqpoint{2.921354in}{2.573912in}}%
\pgfpathlineto{\pgfqpoint{2.921443in}{2.942575in}}%
\pgfpathlineto{\pgfqpoint{2.922170in}{2.766001in}}%
\pgfpathlineto{\pgfqpoint{2.922368in}{2.838445in}}%
\pgfpathlineto{\pgfqpoint{2.923052in}{2.613248in}}%
\pgfpathlineto{\pgfqpoint{2.923184in}{2.738466in}}%
\pgfpathlineto{\pgfqpoint{2.923316in}{2.504309in}}%
\pgfpathlineto{\pgfqpoint{2.923867in}{2.851010in}}%
\pgfpathlineto{\pgfqpoint{2.924308in}{2.653020in}}%
\pgfpathlineto{\pgfqpoint{2.924330in}{2.828283in}}%
\pgfpathlineto{\pgfqpoint{2.925234in}{2.540913in}}%
\pgfpathlineto{\pgfqpoint{2.925410in}{2.697819in}}%
\pgfpathlineto{\pgfqpoint{2.925587in}{2.490870in}}%
\pgfpathlineto{\pgfqpoint{2.926049in}{2.739996in}}%
\pgfpathlineto{\pgfqpoint{2.926534in}{2.611827in}}%
\pgfpathlineto{\pgfqpoint{2.927041in}{2.820525in}}%
\pgfpathlineto{\pgfqpoint{2.927372in}{2.472185in}}%
\pgfpathlineto{\pgfqpoint{2.927658in}{2.659248in}}%
\pgfpathlineto{\pgfqpoint{2.928033in}{2.847404in}}%
\pgfpathlineto{\pgfqpoint{2.927747in}{2.528566in}}%
\pgfpathlineto{\pgfqpoint{2.928408in}{2.615542in}}%
\pgfpathlineto{\pgfqpoint{2.928540in}{2.529768in}}%
\pgfpathlineto{\pgfqpoint{2.928739in}{2.854397in}}%
\pgfpathlineto{\pgfqpoint{2.928981in}{2.693558in}}%
\pgfpathlineto{\pgfqpoint{2.929290in}{2.864013in}}%
\pgfpathlineto{\pgfqpoint{2.929730in}{2.515017in}}%
\pgfpathlineto{\pgfqpoint{2.930083in}{2.686565in}}%
\pgfpathlineto{\pgfqpoint{2.930612in}{2.831014in}}%
\pgfpathlineto{\pgfqpoint{2.930149in}{2.537635in}}%
\pgfpathlineto{\pgfqpoint{2.931185in}{2.667990in}}%
\pgfpathlineto{\pgfqpoint{2.931935in}{2.499174in}}%
\pgfpathlineto{\pgfqpoint{2.932001in}{2.842597in}}%
\pgfpathlineto{\pgfqpoint{2.932243in}{2.651272in}}%
\pgfpathlineto{\pgfqpoint{2.933081in}{2.532063in}}%
\pgfpathlineto{\pgfqpoint{2.933389in}{2.925420in}}%
\pgfpathlineto{\pgfqpoint{2.934249in}{2.555227in}}%
\pgfpathlineto{\pgfqpoint{2.934536in}{2.689078in}}%
\pgfpathlineto{\pgfqpoint{2.934690in}{2.832981in}}%
\pgfpathlineto{\pgfqpoint{2.935395in}{2.554899in}}%
\pgfpathlineto{\pgfqpoint{2.935572in}{2.582872in}}%
\pgfpathlineto{\pgfqpoint{2.935836in}{2.499611in}}%
\pgfpathlineto{\pgfqpoint{2.936321in}{2.802059in}}%
\pgfpathlineto{\pgfqpoint{2.936519in}{2.629747in}}%
\pgfpathlineto{\pgfqpoint{2.937423in}{2.823803in}}%
\pgfpathlineto{\pgfqpoint{2.936938in}{2.558505in}}%
\pgfpathlineto{\pgfqpoint{2.937621in}{2.728851in}}%
\pgfpathlineto{\pgfqpoint{2.938591in}{2.521246in}}%
\pgfpathlineto{\pgfqpoint{2.938040in}{2.895263in}}%
\pgfpathlineto{\pgfqpoint{2.938702in}{2.766985in}}%
\pgfpathlineto{\pgfqpoint{2.939561in}{2.889472in}}%
\pgfpathlineto{\pgfqpoint{2.939032in}{2.588116in}}%
\pgfpathlineto{\pgfqpoint{2.939738in}{2.738576in}}%
\pgfpathlineto{\pgfqpoint{2.940090in}{2.675201in}}%
\pgfpathlineto{\pgfqpoint{2.940531in}{2.920285in}}%
\pgfpathlineto{\pgfqpoint{2.940751in}{2.892422in}}%
\pgfpathlineto{\pgfqpoint{2.940796in}{2.986063in}}%
\pgfpathlineto{\pgfqpoint{2.941325in}{2.617946in}}%
\pgfpathlineto{\pgfqpoint{2.941699in}{2.713007in}}%
\pgfpathlineto{\pgfqpoint{2.941986in}{2.621770in}}%
\pgfpathlineto{\pgfqpoint{2.941743in}{2.850792in}}%
\pgfpathlineto{\pgfqpoint{2.942625in}{2.689078in}}%
\pgfpathlineto{\pgfqpoint{2.942779in}{2.911434in}}%
\pgfpathlineto{\pgfqpoint{2.943661in}{2.559926in}}%
\pgfpathlineto{\pgfqpoint{2.943727in}{2.693230in}}%
\pgfpathlineto{\pgfqpoint{2.943815in}{2.519279in}}%
\pgfpathlineto{\pgfqpoint{2.944631in}{2.863904in}}%
\pgfpathlineto{\pgfqpoint{2.944829in}{2.663619in}}%
\pgfpathlineto{\pgfqpoint{2.945446in}{2.889690in}}%
\pgfpathlineto{\pgfqpoint{2.945336in}{2.629310in}}%
\pgfpathlineto{\pgfqpoint{2.945953in}{2.817356in}}%
\pgfpathlineto{\pgfqpoint{2.946747in}{2.619913in}}%
\pgfpathlineto{\pgfqpoint{2.946945in}{2.978305in}}%
\pgfpathlineto{\pgfqpoint{2.947100in}{2.739340in}}%
\pgfpathlineto{\pgfqpoint{2.947254in}{2.863904in}}%
\pgfpathlineto{\pgfqpoint{2.947386in}{2.637942in}}%
\pgfpathlineto{\pgfqpoint{2.948180in}{2.792007in}}%
\pgfpathlineto{\pgfqpoint{2.948797in}{2.904441in}}%
\pgfpathlineto{\pgfqpoint{2.949282in}{2.611062in}}%
\pgfpathlineto{\pgfqpoint{2.949414in}{2.980709in}}%
\pgfpathlineto{\pgfqpoint{2.949678in}{2.592378in}}%
\pgfpathlineto{\pgfqpoint{2.950406in}{2.889144in}}%
\pgfpathlineto{\pgfqpoint{2.951574in}{2.593470in}}%
\pgfpathlineto{\pgfqpoint{2.952169in}{2.813860in}}%
\pgfpathlineto{\pgfqpoint{2.952059in}{2.514471in}}%
\pgfpathlineto{\pgfqpoint{2.952654in}{2.556429in}}%
\pgfpathlineto{\pgfqpoint{2.952897in}{2.474480in}}%
\pgfpathlineto{\pgfqpoint{2.953007in}{2.806102in}}%
\pgfpathlineto{\pgfqpoint{2.953734in}{2.578610in}}%
\pgfpathlineto{\pgfqpoint{2.954329in}{2.790586in}}%
\pgfpathlineto{\pgfqpoint{2.954263in}{2.535013in}}%
\pgfpathlineto{\pgfqpoint{2.954924in}{2.765783in}}%
\pgfpathlineto{\pgfqpoint{2.955696in}{2.537198in}}%
\pgfpathlineto{\pgfqpoint{2.955960in}{2.775070in}}%
\pgfpathlineto{\pgfqpoint{2.956027in}{2.653457in}}%
\pgfpathlineto{\pgfqpoint{2.956291in}{2.803807in}}%
\pgfpathlineto{\pgfqpoint{2.956401in}{2.474371in}}%
\pgfpathlineto{\pgfqpoint{2.957129in}{2.802933in}}%
\pgfpathlineto{\pgfqpoint{2.957327in}{2.566372in}}%
\pgfpathlineto{\pgfqpoint{2.957547in}{2.830905in}}%
\pgfpathlineto{\pgfqpoint{2.958253in}{2.665914in}}%
\pgfpathlineto{\pgfqpoint{2.959245in}{2.836369in}}%
\pgfpathlineto{\pgfqpoint{2.958848in}{2.568995in}}%
\pgfpathlineto{\pgfqpoint{2.959399in}{2.754201in}}%
\pgfpathlineto{\pgfqpoint{2.959972in}{2.493164in}}%
\pgfpathlineto{\pgfqpoint{2.959443in}{2.853195in}}%
\pgfpathlineto{\pgfqpoint{2.960523in}{2.689078in}}%
\pgfpathlineto{\pgfqpoint{2.961184in}{2.858003in}}%
\pgfpathlineto{\pgfqpoint{2.961118in}{2.497644in}}%
\pgfpathlineto{\pgfqpoint{2.961713in}{2.771027in}}%
\pgfpathlineto{\pgfqpoint{2.962309in}{2.568339in}}%
\pgfpathlineto{\pgfqpoint{2.962661in}{2.851556in}}%
\pgfpathlineto{\pgfqpoint{2.962815in}{2.645699in}}%
\pgfpathlineto{\pgfqpoint{2.962904in}{2.852977in}}%
\pgfpathlineto{\pgfqpoint{2.963256in}{2.590083in}}%
\pgfpathlineto{\pgfqpoint{2.963918in}{2.732129in}}%
\pgfpathlineto{\pgfqpoint{2.964491in}{2.512395in}}%
\pgfpathlineto{\pgfqpoint{2.964843in}{2.832544in}}%
\pgfpathlineto{\pgfqpoint{2.965020in}{2.627343in}}%
\pgfpathlineto{\pgfqpoint{2.965130in}{2.822820in}}%
\pgfpathlineto{\pgfqpoint{2.965438in}{2.546158in}}%
\pgfpathlineto{\pgfqpoint{2.966122in}{2.692575in}}%
\pgfpathlineto{\pgfqpoint{2.966188in}{2.580686in}}%
\pgfpathlineto{\pgfqpoint{2.966430in}{2.850027in}}%
\pgfpathlineto{\pgfqpoint{2.967180in}{2.714428in}}%
\pgfpathlineto{\pgfqpoint{2.968106in}{2.847623in}}%
\pgfpathlineto{\pgfqpoint{2.967819in}{2.553151in}}%
\pgfpathlineto{\pgfqpoint{2.968304in}{2.823912in}}%
\pgfpathlineto{\pgfqpoint{2.968855in}{2.518514in}}%
\pgfpathlineto{\pgfqpoint{2.968701in}{2.921268in}}%
\pgfpathlineto{\pgfqpoint{2.969428in}{2.709183in}}%
\pgfpathlineto{\pgfqpoint{2.969847in}{2.853305in}}%
\pgfpathlineto{\pgfqpoint{2.969626in}{2.596639in}}%
\pgfpathlineto{\pgfqpoint{2.970552in}{2.832763in}}%
\pgfpathlineto{\pgfqpoint{2.971125in}{2.559379in}}%
\pgfpathlineto{\pgfqpoint{2.971654in}{2.869258in}}%
\pgfpathlineto{\pgfqpoint{2.972580in}{2.575769in}}%
\pgfpathlineto{\pgfqpoint{2.972778in}{2.780752in}}%
\pgfpathlineto{\pgfqpoint{2.973197in}{2.897776in}}%
\pgfpathlineto{\pgfqpoint{2.973572in}{2.604615in}}%
\pgfpathlineto{\pgfqpoint{2.973859in}{2.857020in}}%
\pgfpathlineto{\pgfqpoint{2.974101in}{2.613685in}}%
\pgfpathlineto{\pgfqpoint{2.974542in}{2.909358in}}%
\pgfpathlineto{\pgfqpoint{2.974961in}{2.715630in}}%
\pgfpathlineto{\pgfqpoint{2.975379in}{2.962680in}}%
\pgfpathlineto{\pgfqpoint{2.975027in}{2.658593in}}%
\pgfpathlineto{\pgfqpoint{2.976063in}{2.808287in}}%
\pgfpathlineto{\pgfqpoint{2.977121in}{2.621442in}}%
\pgfpathlineto{\pgfqpoint{2.976393in}{2.901928in}}%
\pgfpathlineto{\pgfqpoint{2.977187in}{2.728632in}}%
\pgfpathlineto{\pgfqpoint{2.977980in}{2.855272in}}%
\pgfpathlineto{\pgfqpoint{2.978157in}{2.538947in}}%
\pgfpathlineto{\pgfqpoint{2.978245in}{2.643186in}}%
\pgfpathlineto{\pgfqpoint{2.978267in}{2.582653in}}%
\pgfpathlineto{\pgfqpoint{2.978994in}{2.857457in}}%
\pgfpathlineto{\pgfqpoint{2.979325in}{2.718034in}}%
\pgfpathlineto{\pgfqpoint{2.980185in}{2.891548in}}%
\pgfpathlineto{\pgfqpoint{2.979986in}{2.557850in}}%
\pgfpathlineto{\pgfqpoint{2.980449in}{2.843362in}}%
\pgfpathlineto{\pgfqpoint{2.980714in}{2.599589in}}%
\pgfpathlineto{\pgfqpoint{2.980581in}{2.903239in}}%
\pgfpathlineto{\pgfqpoint{2.981551in}{2.828720in}}%
\pgfpathlineto{\pgfqpoint{2.982499in}{2.604506in}}%
\pgfpathlineto{\pgfqpoint{2.981683in}{2.943777in}}%
\pgfpathlineto{\pgfqpoint{2.982675in}{2.739559in}}%
\pgfpathlineto{\pgfqpoint{2.982918in}{2.819760in}}%
\pgfpathlineto{\pgfqpoint{2.983094in}{2.611171in}}%
\pgfpathlineto{\pgfqpoint{2.983733in}{2.707325in}}%
\pgfpathlineto{\pgfqpoint{2.984615in}{2.577408in}}%
\pgfpathlineto{\pgfqpoint{2.984395in}{2.916023in}}%
\pgfpathlineto{\pgfqpoint{2.984813in}{2.755075in}}%
\pgfpathlineto{\pgfqpoint{2.985827in}{2.860079in}}%
\pgfpathlineto{\pgfqpoint{2.985254in}{2.605599in}}%
\pgfpathlineto{\pgfqpoint{2.985915in}{2.790586in}}%
\pgfpathlineto{\pgfqpoint{2.986863in}{2.627015in}}%
\pgfpathlineto{\pgfqpoint{2.986092in}{2.932632in}}%
\pgfpathlineto{\pgfqpoint{2.987062in}{2.643186in}}%
\pgfpathlineto{\pgfqpoint{2.987569in}{2.888051in}}%
\pgfpathlineto{\pgfqpoint{2.987635in}{2.607784in}}%
\pgfpathlineto{\pgfqpoint{2.988208in}{2.756932in}}%
\pgfpathlineto{\pgfqpoint{2.989090in}{2.527146in}}%
\pgfpathlineto{\pgfqpoint{2.988494in}{2.802605in}}%
\pgfpathlineto{\pgfqpoint{2.989310in}{2.737920in}}%
\pgfpathlineto{\pgfqpoint{2.990214in}{2.863576in}}%
\pgfpathlineto{\pgfqpoint{2.989354in}{2.639253in}}%
\pgfpathlineto{\pgfqpoint{2.990434in}{2.807085in}}%
\pgfpathlineto{\pgfqpoint{2.990985in}{2.600245in}}%
\pgfpathlineto{\pgfqpoint{2.990655in}{2.931867in}}%
\pgfpathlineto{\pgfqpoint{2.991426in}{2.809598in}}%
\pgfpathlineto{\pgfqpoint{2.991448in}{2.921268in}}%
\pgfpathlineto{\pgfqpoint{2.991514in}{2.627452in}}%
\pgfpathlineto{\pgfqpoint{2.992528in}{2.840630in}}%
\pgfpathlineto{\pgfqpoint{2.993652in}{2.517640in}}%
\pgfpathlineto{\pgfqpoint{2.992793in}{2.873082in}}%
\pgfpathlineto{\pgfqpoint{2.993696in}{2.599371in}}%
\pgfpathlineto{\pgfqpoint{2.993718in}{2.761412in}}%
\pgfpathlineto{\pgfqpoint{2.994688in}{2.400616in}}%
\pgfpathlineto{\pgfqpoint{2.994776in}{2.625704in}}%
\pgfpathlineto{\pgfqpoint{2.995856in}{2.440061in}}%
\pgfpathlineto{\pgfqpoint{2.995063in}{2.718034in}}%
\pgfpathlineto{\pgfqpoint{2.995900in}{2.458308in}}%
\pgfpathlineto{\pgfqpoint{2.997025in}{2.262832in}}%
\pgfpathlineto{\pgfqpoint{2.996760in}{2.547469in}}%
\pgfpathlineto{\pgfqpoint{2.997113in}{2.265236in}}%
\pgfpathlineto{\pgfqpoint{2.998017in}{2.484204in}}%
\pgfpathlineto{\pgfqpoint{2.997245in}{2.134554in}}%
\pgfpathlineto{\pgfqpoint{2.998215in}{2.221638in}}%
\pgfpathlineto{\pgfqpoint{2.998325in}{2.488247in}}%
\pgfpathlineto{\pgfqpoint{2.999207in}{2.199567in}}%
\pgfpathlineto{\pgfqpoint{2.999361in}{2.332434in}}%
\pgfpathlineto{\pgfqpoint{2.999405in}{2.145480in}}%
\pgfpathlineto{\pgfqpoint{2.999934in}{2.409576in}}%
\pgfpathlineto{\pgfqpoint{3.000463in}{2.366962in}}%
\pgfpathlineto{\pgfqpoint{3.001014in}{2.060362in}}%
\pgfpathlineto{\pgfqpoint{3.000860in}{2.459401in}}%
\pgfpathlineto{\pgfqpoint{3.001587in}{2.284466in}}%
\pgfpathlineto{\pgfqpoint{3.001720in}{2.050747in}}%
\pgfpathlineto{\pgfqpoint{3.002094in}{2.317683in}}%
\pgfpathlineto{\pgfqpoint{3.002711in}{2.191044in}}%
\pgfpathlineto{\pgfqpoint{3.003307in}{2.366744in}}%
\pgfpathlineto{\pgfqpoint{3.002976in}{1.999173in}}%
\pgfpathlineto{\pgfqpoint{3.003792in}{2.167770in}}%
\pgfpathlineto{\pgfqpoint{3.004387in}{2.054899in}}%
\pgfpathlineto{\pgfqpoint{3.004056in}{2.332653in}}%
\pgfpathlineto{\pgfqpoint{3.004916in}{2.074457in}}%
\pgfpathlineto{\pgfqpoint{3.005665in}{2.313203in}}%
\pgfpathlineto{\pgfqpoint{3.006018in}{2.199130in}}%
\pgfpathlineto{\pgfqpoint{3.006326in}{2.327736in}}%
\pgfpathlineto{\pgfqpoint{3.007186in}{1.966065in}}%
\pgfpathlineto{\pgfqpoint{3.007450in}{2.230817in}}%
\pgfpathlineto{\pgfqpoint{3.007539in}{1.888596in}}%
\pgfpathlineto{\pgfqpoint{3.008310in}{2.062001in}}%
\pgfpathlineto{\pgfqpoint{3.009412in}{1.839208in}}%
\pgfpathlineto{\pgfqpoint{3.008663in}{2.174217in}}%
\pgfpathlineto{\pgfqpoint{3.009434in}{1.982783in}}%
\pgfpathlineto{\pgfqpoint{3.009567in}{2.153675in}}%
\pgfpathlineto{\pgfqpoint{3.010162in}{1.818010in}}%
\pgfpathlineto{\pgfqpoint{3.010536in}{1.998517in}}%
\pgfpathlineto{\pgfqpoint{3.010558in}{1.998955in}}%
\pgfpathlineto{\pgfqpoint{3.010779in}{2.081887in}}%
\pgfpathlineto{\pgfqpoint{3.011418in}{1.784466in}}%
\pgfpathlineto{\pgfqpoint{3.011550in}{1.903565in}}%
\pgfpathlineto{\pgfqpoint{3.012234in}{1.791568in}}%
\pgfpathlineto{\pgfqpoint{3.012300in}{2.080795in}}%
\pgfpathlineto{\pgfqpoint{3.012498in}{1.853849in}}%
\pgfpathlineto{\pgfqpoint{3.013203in}{2.079593in}}%
\pgfpathlineto{\pgfqpoint{3.013137in}{1.793972in}}%
\pgfpathlineto{\pgfqpoint{3.013622in}{1.952626in}}%
\pgfpathlineto{\pgfqpoint{3.014306in}{2.109859in}}%
\pgfpathlineto{\pgfqpoint{3.014658in}{1.824129in}}%
\pgfpathlineto{\pgfqpoint{3.014702in}{1.898976in}}%
\pgfpathlineto{\pgfqpoint{3.015099in}{1.712459in}}%
\pgfpathlineto{\pgfqpoint{3.014790in}{2.063312in}}%
\pgfpathlineto{\pgfqpoint{3.015804in}{1.862481in}}%
\pgfpathlineto{\pgfqpoint{3.015915in}{2.034357in}}%
\pgfpathlineto{\pgfqpoint{3.016157in}{1.674544in}}%
\pgfpathlineto{\pgfqpoint{3.016884in}{1.823583in}}%
\pgfpathlineto{\pgfqpoint{3.017039in}{1.705576in}}%
\pgfpathlineto{\pgfqpoint{3.017237in}{2.001686in}}%
\pgfpathlineto{\pgfqpoint{3.017987in}{1.824785in}}%
\pgfpathlineto{\pgfqpoint{3.018538in}{1.716721in}}%
\pgfpathlineto{\pgfqpoint{3.018692in}{1.962350in}}%
\pgfpathlineto{\pgfqpoint{3.018912in}{1.885755in}}%
\pgfpathlineto{\pgfqpoint{3.019574in}{2.071616in}}%
\pgfpathlineto{\pgfqpoint{3.019111in}{1.792770in}}%
\pgfpathlineto{\pgfqpoint{3.020014in}{1.882477in}}%
\pgfpathlineto{\pgfqpoint{3.020852in}{2.103303in}}%
\pgfpathlineto{\pgfqpoint{3.021183in}{2.057958in}}%
\pgfpathlineto{\pgfqpoint{3.022307in}{1.798124in}}%
\pgfpathlineto{\pgfqpoint{3.021513in}{2.128544in}}%
\pgfpathlineto{\pgfqpoint{3.022351in}{1.899413in}}%
\pgfpathlineto{\pgfqpoint{3.023122in}{2.136193in}}%
\pgfpathlineto{\pgfqpoint{3.023277in}{1.758460in}}%
\pgfpathlineto{\pgfqpoint{3.023475in}{2.012066in}}%
\pgfpathlineto{\pgfqpoint{3.024599in}{1.788071in}}%
\pgfpathlineto{\pgfqpoint{3.024048in}{2.048780in}}%
\pgfpathlineto{\pgfqpoint{3.024621in}{1.837897in}}%
\pgfpathlineto{\pgfqpoint{3.025018in}{1.953063in}}%
\pgfpathlineto{\pgfqpoint{3.025349in}{1.673561in}}%
\pgfpathlineto{\pgfqpoint{3.025657in}{1.771900in}}%
\pgfpathlineto{\pgfqpoint{3.026605in}{1.585602in}}%
\pgfpathlineto{\pgfqpoint{3.026010in}{1.932084in}}%
\pgfpathlineto{\pgfqpoint{3.026759in}{1.639688in}}%
\pgfpathlineto{\pgfqpoint{3.026980in}{1.824894in}}%
\pgfpathlineto{\pgfqpoint{3.027773in}{1.562546in}}%
\pgfpathlineto{\pgfqpoint{3.027883in}{1.701751in}}%
\pgfpathlineto{\pgfqpoint{3.028434in}{1.515016in}}%
\pgfpathlineto{\pgfqpoint{3.028589in}{1.832105in}}%
\pgfpathlineto{\pgfqpoint{3.028897in}{1.690606in}}%
\pgfpathlineto{\pgfqpoint{3.028919in}{1.862591in}}%
\pgfpathlineto{\pgfqpoint{3.029118in}{1.548779in}}%
\pgfpathlineto{\pgfqpoint{3.029999in}{1.641436in}}%
\pgfpathlineto{\pgfqpoint{3.030044in}{1.603412in}}%
\pgfpathlineto{\pgfqpoint{3.030815in}{1.877779in}}%
\pgfpathlineto{\pgfqpoint{3.030991in}{1.715082in}}%
\pgfpathlineto{\pgfqpoint{3.031741in}{1.855379in}}%
\pgfpathlineto{\pgfqpoint{3.031366in}{1.572162in}}%
\pgfpathlineto{\pgfqpoint{3.032071in}{1.765672in}}%
\pgfpathlineto{\pgfqpoint{3.032093in}{1.622643in}}%
\pgfpathlineto{\pgfqpoint{3.033019in}{1.923015in}}%
\pgfpathlineto{\pgfqpoint{3.033174in}{1.695523in}}%
\pgfpathlineto{\pgfqpoint{3.033284in}{1.969562in}}%
\pgfpathlineto{\pgfqpoint{3.033879in}{1.670392in}}%
\pgfpathlineto{\pgfqpoint{3.034298in}{1.796048in}}%
\pgfpathlineto{\pgfqpoint{3.035356in}{2.023539in}}%
\pgfpathlineto{\pgfqpoint{3.034937in}{1.638268in}}%
\pgfpathlineto{\pgfqpoint{3.035422in}{1.829265in}}%
\pgfpathlineto{\pgfqpoint{3.035797in}{1.667332in}}%
\pgfpathlineto{\pgfqpoint{3.035929in}{1.958308in}}%
\pgfpathlineto{\pgfqpoint{3.036568in}{1.686563in}}%
\pgfpathlineto{\pgfqpoint{3.037428in}{1.903128in}}%
\pgfpathlineto{\pgfqpoint{3.036744in}{1.575221in}}%
\pgfpathlineto{\pgfqpoint{3.037670in}{1.745785in}}%
\pgfpathlineto{\pgfqpoint{3.037846in}{1.666677in}}%
\pgfpathlineto{\pgfqpoint{3.038419in}{1.973058in}}%
\pgfpathlineto{\pgfqpoint{3.038706in}{1.764361in}}%
\pgfpathlineto{\pgfqpoint{3.039037in}{2.009772in}}%
\pgfpathlineto{\pgfqpoint{3.038904in}{1.737590in}}%
\pgfpathlineto{\pgfqpoint{3.039808in}{1.825440in}}%
\pgfpathlineto{\pgfqpoint{3.040403in}{1.599151in}}%
\pgfpathlineto{\pgfqpoint{3.039984in}{1.954920in}}%
\pgfpathlineto{\pgfqpoint{3.040910in}{1.735951in}}%
\pgfpathlineto{\pgfqpoint{3.041395in}{1.991634in}}%
\pgfpathlineto{\pgfqpoint{3.041946in}{1.550746in}}%
\pgfpathlineto{\pgfqpoint{3.041990in}{1.672140in}}%
\pgfpathlineto{\pgfqpoint{3.042255in}{1.598823in}}%
\pgfpathlineto{\pgfqpoint{3.042762in}{1.882914in}}%
\pgfpathlineto{\pgfqpoint{3.043070in}{1.683722in}}%
\pgfpathlineto{\pgfqpoint{3.044128in}{1.912197in}}%
\pgfpathlineto{\pgfqpoint{3.043908in}{1.589972in}}%
\pgfpathlineto{\pgfqpoint{3.044172in}{1.756603in}}%
\pgfpathlineto{\pgfqpoint{3.044591in}{1.447380in}}%
\pgfpathlineto{\pgfqpoint{3.044415in}{1.817792in}}%
\pgfpathlineto{\pgfqpoint{3.045297in}{1.566371in}}%
\pgfpathlineto{\pgfqpoint{3.046333in}{1.345544in}}%
\pgfpathlineto{\pgfqpoint{3.046200in}{1.662962in}}%
\pgfpathlineto{\pgfqpoint{3.046619in}{1.415802in}}%
\pgfpathlineto{\pgfqpoint{3.046773in}{1.668097in}}%
\pgfpathlineto{\pgfqpoint{3.047236in}{1.383350in}}%
\pgfpathlineto{\pgfqpoint{3.047765in}{1.575549in}}%
\pgfpathlineto{\pgfqpoint{3.048537in}{1.379963in}}%
\pgfpathlineto{\pgfqpoint{3.047986in}{1.603193in}}%
\pgfpathlineto{\pgfqpoint{3.048911in}{1.478739in}}%
\pgfpathlineto{\pgfqpoint{3.049110in}{1.698473in}}%
\pgfpathlineto{\pgfqpoint{3.049573in}{1.385536in}}%
\pgfpathlineto{\pgfqpoint{3.050036in}{1.553150in}}%
\pgfpathlineto{\pgfqpoint{3.050741in}{1.397446in}}%
\pgfpathlineto{\pgfqpoint{3.050521in}{1.727210in}}%
\pgfpathlineto{\pgfqpoint{3.051138in}{1.509662in}}%
\pgfpathlineto{\pgfqpoint{3.051865in}{1.690497in}}%
\pgfpathlineto{\pgfqpoint{3.052108in}{1.388814in}}%
\pgfpathlineto{\pgfqpoint{3.052262in}{1.566589in}}%
\pgfpathlineto{\pgfqpoint{3.053210in}{1.326751in}}%
\pgfpathlineto{\pgfqpoint{3.052923in}{1.665693in}}%
\pgfpathlineto{\pgfqpoint{3.053408in}{1.391982in}}%
\pgfpathlineto{\pgfqpoint{3.054466in}{1.572817in}}%
\pgfpathlineto{\pgfqpoint{3.054246in}{1.250264in}}%
\pgfpathlineto{\pgfqpoint{3.054532in}{1.463333in}}%
\pgfpathlineto{\pgfqpoint{3.054642in}{1.247533in}}%
\pgfpathlineto{\pgfqpoint{3.054576in}{1.573692in}}%
\pgfpathlineto{\pgfqpoint{3.055678in}{1.411541in}}%
\pgfpathlineto{\pgfqpoint{3.056097in}{1.558285in}}%
\pgfpathlineto{\pgfqpoint{3.056362in}{1.255946in}}%
\pgfpathlineto{\pgfqpoint{3.056626in}{1.359312in}}%
\pgfpathlineto{\pgfqpoint{3.056758in}{1.250483in}}%
\pgfpathlineto{\pgfqpoint{3.057332in}{1.530969in}}%
\pgfpathlineto{\pgfqpoint{3.057684in}{1.289382in}}%
\pgfpathlineto{\pgfqpoint{3.057883in}{1.590956in}}%
\pgfpathlineto{\pgfqpoint{3.058412in}{1.213114in}}%
\pgfpathlineto{\pgfqpoint{3.058808in}{1.422795in}}%
\pgfpathlineto{\pgfqpoint{3.058963in}{1.481690in}}%
\pgfpathlineto{\pgfqpoint{3.058897in}{1.324237in}}%
\pgfpathlineto{\pgfqpoint{3.059007in}{1.400724in}}%
\pgfpathlineto{\pgfqpoint{3.059029in}{1.229613in}}%
\pgfpathlineto{\pgfqpoint{3.059536in}{1.560361in}}%
\pgfpathlineto{\pgfqpoint{3.060087in}{1.454920in}}%
\pgfpathlineto{\pgfqpoint{3.060197in}{1.528455in}}%
\pgfpathlineto{\pgfqpoint{3.060660in}{1.264250in}}%
\pgfpathlineto{\pgfqpoint{3.061123in}{1.432520in}}%
\pgfpathlineto{\pgfqpoint{3.061145in}{1.254526in}}%
\pgfpathlineto{\pgfqpoint{3.061806in}{1.548779in}}%
\pgfpathlineto{\pgfqpoint{3.062225in}{1.260098in}}%
\pgfpathlineto{\pgfqpoint{3.063305in}{1.598167in}}%
\pgfpathlineto{\pgfqpoint{3.063349in}{1.565060in}}%
\pgfpathlineto{\pgfqpoint{3.064363in}{1.299216in}}%
\pgfpathlineto{\pgfqpoint{3.064451in}{1.334399in}}%
\pgfpathlineto{\pgfqpoint{3.064539in}{1.538399in}}%
\pgfpathlineto{\pgfqpoint{3.065465in}{1.236825in}}%
\pgfpathlineto{\pgfqpoint{3.065531in}{1.386847in}}%
\pgfpathlineto{\pgfqpoint{3.066016in}{1.176619in}}%
\pgfpathlineto{\pgfqpoint{3.066214in}{1.490540in}}%
\pgfpathlineto{\pgfqpoint{3.066633in}{1.328171in}}%
\pgfpathlineto{\pgfqpoint{3.067537in}{1.558176in}}%
\pgfpathlineto{\pgfqpoint{3.067361in}{1.259334in}}%
\pgfpathlineto{\pgfqpoint{3.067757in}{1.387612in}}%
\pgfpathlineto{\pgfqpoint{3.067934in}{1.202297in}}%
\pgfpathlineto{\pgfqpoint{3.068330in}{1.493927in}}%
\pgfpathlineto{\pgfqpoint{3.068793in}{1.444211in}}%
\pgfpathlineto{\pgfqpoint{3.068815in}{1.492179in}}%
\pgfpathlineto{\pgfqpoint{3.069411in}{1.193118in}}%
\pgfpathlineto{\pgfqpoint{3.069829in}{1.455794in}}%
\pgfpathlineto{\pgfqpoint{3.070050in}{1.178586in}}%
\pgfpathlineto{\pgfqpoint{3.070821in}{1.485951in}}%
\pgfpathlineto{\pgfqpoint{3.070953in}{1.325221in}}%
\pgfpathlineto{\pgfqpoint{3.071372in}{1.576314in}}%
\pgfpathlineto{\pgfqpoint{3.070998in}{1.297904in}}%
\pgfpathlineto{\pgfqpoint{3.072078in}{1.415147in}}%
\pgfpathlineto{\pgfqpoint{3.072474in}{1.579701in}}%
\pgfpathlineto{\pgfqpoint{3.072541in}{1.311672in}}%
\pgfpathlineto{\pgfqpoint{3.072849in}{1.383460in}}%
\pgfpathlineto{\pgfqpoint{3.073025in}{1.197052in}}%
\pgfpathlineto{\pgfqpoint{3.073907in}{1.534465in}}%
\pgfpathlineto{\pgfqpoint{3.073929in}{1.462459in}}%
\pgfpathlineto{\pgfqpoint{3.075075in}{1.214207in}}%
\pgfpathlineto{\pgfqpoint{3.073973in}{1.524522in}}%
\pgfpathlineto{\pgfqpoint{3.075141in}{1.320632in}}%
\pgfpathlineto{\pgfqpoint{3.075847in}{1.483438in}}%
\pgfpathlineto{\pgfqpoint{3.075538in}{1.167004in}}%
\pgfpathlineto{\pgfqpoint{3.076199in}{1.272555in}}%
\pgfpathlineto{\pgfqpoint{3.076420in}{1.187327in}}%
\pgfpathlineto{\pgfqpoint{3.076839in}{1.473385in}}%
\pgfpathlineto{\pgfqpoint{3.076993in}{1.387721in}}%
\pgfpathlineto{\pgfqpoint{3.077742in}{1.497861in}}%
\pgfpathlineto{\pgfqpoint{3.077654in}{1.169517in}}%
\pgfpathlineto{\pgfqpoint{3.077985in}{1.276707in}}%
\pgfpathlineto{\pgfqpoint{3.078624in}{1.120784in}}%
\pgfpathlineto{\pgfqpoint{3.078249in}{1.374281in}}%
\pgfpathlineto{\pgfqpoint{3.079109in}{1.189185in}}%
\pgfpathlineto{\pgfqpoint{3.079307in}{1.485295in}}%
\pgfpathlineto{\pgfqpoint{3.079660in}{1.147336in}}%
\pgfpathlineto{\pgfqpoint{3.080233in}{1.264141in}}%
\pgfpathlineto{\pgfqpoint{3.080850in}{1.408044in}}%
\pgfpathlineto{\pgfqpoint{3.080938in}{1.149412in}}%
\pgfpathlineto{\pgfqpoint{3.081291in}{1.256711in}}%
\pgfpathlineto{\pgfqpoint{3.082195in}{1.091720in}}%
\pgfpathlineto{\pgfqpoint{3.081622in}{1.442572in}}%
\pgfpathlineto{\pgfqpoint{3.082349in}{1.224259in}}%
\pgfpathlineto{\pgfqpoint{3.082371in}{1.374609in}}%
\pgfpathlineto{\pgfqpoint{3.083032in}{1.079810in}}%
\pgfpathlineto{\pgfqpoint{3.083473in}{1.321834in}}%
\pgfpathlineto{\pgfqpoint{3.084421in}{0.998079in}}%
\pgfpathlineto{\pgfqpoint{3.084642in}{1.248188in}}%
\pgfpathlineto{\pgfqpoint{3.085766in}{1.524522in}}%
\pgfpathlineto{\pgfqpoint{3.084840in}{1.123297in}}%
\pgfpathlineto{\pgfqpoint{3.085832in}{1.444758in}}%
\pgfpathlineto{\pgfqpoint{3.086251in}{1.204701in}}%
\pgfpathlineto{\pgfqpoint{3.085898in}{1.469889in}}%
\pgfpathlineto{\pgfqpoint{3.086934in}{1.406078in}}%
\pgfpathlineto{\pgfqpoint{3.087132in}{1.497752in}}%
\pgfpathlineto{\pgfqpoint{3.087375in}{1.205684in}}%
\pgfpathlineto{\pgfqpoint{3.087617in}{1.298888in}}%
\pgfpathlineto{\pgfqpoint{3.088587in}{1.093468in}}%
\pgfpathlineto{\pgfqpoint{3.088146in}{1.391982in}}%
\pgfpathlineto{\pgfqpoint{3.088807in}{1.158809in}}%
\pgfpathlineto{\pgfqpoint{3.089932in}{1.437000in}}%
\pgfpathlineto{\pgfqpoint{3.089226in}{1.089206in}}%
\pgfpathlineto{\pgfqpoint{3.089954in}{1.419299in}}%
\pgfpathlineto{\pgfqpoint{3.091012in}{1.120675in}}%
\pgfpathlineto{\pgfqpoint{3.090240in}{1.424107in}}%
\pgfpathlineto{\pgfqpoint{3.091100in}{1.262611in}}%
\pgfpathlineto{\pgfqpoint{3.092114in}{1.379198in}}%
\pgfpathlineto{\pgfqpoint{3.091210in}{1.052493in}}%
\pgfpathlineto{\pgfqpoint{3.092136in}{1.233437in}}%
\pgfpathlineto{\pgfqpoint{3.092489in}{1.145588in}}%
\pgfpathlineto{\pgfqpoint{3.092224in}{1.356580in}}%
\pgfpathlineto{\pgfqpoint{3.093238in}{1.243271in}}%
\pgfpathlineto{\pgfqpoint{3.093855in}{1.377231in}}%
\pgfpathlineto{\pgfqpoint{3.093591in}{1.144058in}}%
\pgfpathlineto{\pgfqpoint{3.093987in}{1.256274in}}%
\pgfpathlineto{\pgfqpoint{3.094781in}{1.006164in}}%
\pgfpathlineto{\pgfqpoint{3.094053in}{1.400942in}}%
\pgfpathlineto{\pgfqpoint{3.095134in}{1.031842in}}%
\pgfpathlineto{\pgfqpoint{3.096258in}{1.355378in}}%
\pgfpathlineto{\pgfqpoint{3.096500in}{1.099368in}}%
\pgfpathlineto{\pgfqpoint{3.096654in}{1.028673in}}%
\pgfpathlineto{\pgfqpoint{3.097360in}{1.308066in}}%
\pgfpathlineto{\pgfqpoint{3.097514in}{1.177166in}}%
\pgfpathlineto{\pgfqpoint{3.098286in}{1.306209in}}%
\pgfpathlineto{\pgfqpoint{3.097977in}{1.018184in}}%
\pgfpathlineto{\pgfqpoint{3.098616in}{1.280859in}}%
\pgfpathlineto{\pgfqpoint{3.099057in}{1.123188in}}%
\pgfpathlineto{\pgfqpoint{3.099167in}{1.366633in}}%
\pgfpathlineto{\pgfqpoint{3.099718in}{1.167113in}}%
\pgfpathlineto{\pgfqpoint{3.099806in}{1.424871in}}%
\pgfpathlineto{\pgfqpoint{3.099939in}{1.070631in}}%
\pgfpathlineto{\pgfqpoint{3.100820in}{1.179242in}}%
\pgfpathlineto{\pgfqpoint{3.101283in}{0.983656in}}%
\pgfpathlineto{\pgfqpoint{3.101790in}{1.304570in}}%
\pgfpathlineto{\pgfqpoint{3.101945in}{1.111387in}}%
\pgfpathlineto{\pgfqpoint{3.102143in}{1.260645in}}%
\pgfpathlineto{\pgfqpoint{3.102870in}{1.025723in}}%
\pgfpathlineto{\pgfqpoint{3.103069in}{1.163289in}}%
\pgfpathlineto{\pgfqpoint{3.103113in}{1.181536in}}%
\pgfpathlineto{\pgfqpoint{3.103201in}{1.120129in}}%
\pgfpathlineto{\pgfqpoint{3.103223in}{1.166785in}}%
\pgfpathlineto{\pgfqpoint{3.103245in}{1.047358in}}%
\pgfpathlineto{\pgfqpoint{3.103509in}{1.324237in}}%
\pgfpathlineto{\pgfqpoint{3.104325in}{1.123079in}}%
\pgfpathlineto{\pgfqpoint{3.105097in}{1.365321in}}%
\pgfpathlineto{\pgfqpoint{3.105185in}{0.876466in}}%
\pgfpathlineto{\pgfqpoint{3.105449in}{1.204482in}}%
\pgfpathlineto{\pgfqpoint{3.105890in}{1.046702in}}%
\pgfpathlineto{\pgfqpoint{3.105802in}{1.334618in}}%
\pgfpathlineto{\pgfqpoint{3.106529in}{1.174762in}}%
\pgfpathlineto{\pgfqpoint{3.106838in}{1.449893in}}%
\pgfpathlineto{\pgfqpoint{3.106904in}{1.101881in}}%
\pgfpathlineto{\pgfqpoint{3.107631in}{1.243818in}}%
\pgfpathlineto{\pgfqpoint{3.108535in}{1.050636in}}%
\pgfpathlineto{\pgfqpoint{3.107874in}{1.363573in}}%
\pgfpathlineto{\pgfqpoint{3.108755in}{1.176619in}}%
\pgfpathlineto{\pgfqpoint{3.108800in}{1.364557in}}%
\pgfpathlineto{\pgfqpoint{3.109086in}{1.015780in}}%
\pgfpathlineto{\pgfqpoint{3.109858in}{1.166239in}}%
\pgfpathlineto{\pgfqpoint{3.110938in}{1.073581in}}%
\pgfpathlineto{\pgfqpoint{3.110320in}{1.347620in}}%
\pgfpathlineto{\pgfqpoint{3.110960in}{1.110623in}}%
\pgfpathlineto{\pgfqpoint{3.111136in}{1.313092in}}%
\pgfpathlineto{\pgfqpoint{3.111930in}{0.964206in}}%
\pgfpathlineto{\pgfqpoint{3.112040in}{1.063638in}}%
\pgfpathlineto{\pgfqpoint{3.112260in}{1.274631in}}%
\pgfpathlineto{\pgfqpoint{3.112348in}{0.995893in}}%
\pgfpathlineto{\pgfqpoint{3.113164in}{1.119801in}}%
\pgfpathlineto{\pgfqpoint{3.114178in}{0.936562in}}%
\pgfpathlineto{\pgfqpoint{3.113715in}{1.251685in}}%
\pgfpathlineto{\pgfqpoint{3.114266in}{1.133241in}}%
\pgfpathlineto{\pgfqpoint{3.114927in}{1.301619in}}%
\pgfpathlineto{\pgfqpoint{3.114376in}{0.981142in}}%
\pgfpathlineto{\pgfqpoint{3.115346in}{1.184705in}}%
\pgfpathlineto{\pgfqpoint{3.116007in}{1.028564in}}%
\pgfpathlineto{\pgfqpoint{3.116272in}{1.334181in}}%
\pgfpathlineto{\pgfqpoint{3.116448in}{1.199347in}}%
\pgfpathlineto{\pgfqpoint{3.116492in}{1.121549in}}%
\pgfpathlineto{\pgfqpoint{3.117021in}{1.441152in}}%
\pgfpathlineto{\pgfqpoint{3.117528in}{1.281405in}}%
\pgfpathlineto{\pgfqpoint{3.117572in}{1.412852in}}%
\pgfpathlineto{\pgfqpoint{3.117947in}{1.132694in}}%
\pgfpathlineto{\pgfqpoint{3.118476in}{1.259443in}}%
\pgfpathlineto{\pgfqpoint{3.118983in}{1.094123in}}%
\pgfpathlineto{\pgfqpoint{3.119270in}{1.424981in}}%
\pgfpathlineto{\pgfqpoint{3.119578in}{1.204591in}}%
\pgfpathlineto{\pgfqpoint{3.119909in}{1.449238in}}%
\pgfpathlineto{\pgfqpoint{3.119975in}{1.107782in}}%
\pgfpathlineto{\pgfqpoint{3.120680in}{1.273866in}}%
\pgfpathlineto{\pgfqpoint{3.120724in}{1.093796in}}%
\pgfpathlineto{\pgfqpoint{3.120967in}{1.349806in}}%
\pgfpathlineto{\pgfqpoint{3.121826in}{1.200439in}}%
\pgfpathlineto{\pgfqpoint{3.122620in}{1.344452in}}%
\pgfpathlineto{\pgfqpoint{3.122377in}{1.084727in}}%
\pgfpathlineto{\pgfqpoint{3.122884in}{1.235732in}}%
\pgfpathlineto{\pgfqpoint{3.123480in}{1.025177in}}%
\pgfpathlineto{\pgfqpoint{3.123149in}{1.394605in}}%
\pgfpathlineto{\pgfqpoint{3.123987in}{1.136737in}}%
\pgfpathlineto{\pgfqpoint{3.124956in}{1.403892in}}%
\pgfpathlineto{\pgfqpoint{3.124339in}{1.023319in}}%
\pgfpathlineto{\pgfqpoint{3.125133in}{1.332105in}}%
\pgfpathlineto{\pgfqpoint{3.125816in}{1.101663in}}%
\pgfpathlineto{\pgfqpoint{3.125463in}{1.380946in}}%
\pgfpathlineto{\pgfqpoint{3.126213in}{1.339972in}}%
\pgfpathlineto{\pgfqpoint{3.126786in}{1.447926in}}%
\pgfpathlineto{\pgfqpoint{3.126896in}{1.125592in}}%
\pgfpathlineto{\pgfqpoint{3.127271in}{1.322708in}}%
\pgfpathlineto{\pgfqpoint{3.128351in}{1.160557in}}%
\pgfpathlineto{\pgfqpoint{3.127866in}{1.416458in}}%
\pgfpathlineto{\pgfqpoint{3.128373in}{1.218031in}}%
\pgfpathlineto{\pgfqpoint{3.129144in}{1.480597in}}%
\pgfpathlineto{\pgfqpoint{3.128461in}{1.063529in}}%
\pgfpathlineto{\pgfqpoint{3.129497in}{1.312000in}}%
\pgfpathlineto{\pgfqpoint{3.130070in}{1.058175in}}%
\pgfpathlineto{\pgfqpoint{3.129784in}{1.431209in}}%
\pgfpathlineto{\pgfqpoint{3.130709in}{1.197489in}}%
\pgfpathlineto{\pgfqpoint{3.131437in}{1.334836in}}%
\pgfpathlineto{\pgfqpoint{3.131635in}{1.104831in}}%
\pgfpathlineto{\pgfqpoint{3.131789in}{1.300964in}}%
\pgfpathlineto{\pgfqpoint{3.132627in}{1.041129in}}%
\pgfpathlineto{\pgfqpoint{3.132010in}{1.353849in}}%
\pgfpathlineto{\pgfqpoint{3.132914in}{1.157498in}}%
\pgfpathlineto{\pgfqpoint{3.133883in}{1.359093in}}%
\pgfpathlineto{\pgfqpoint{3.133398in}{1.047248in}}%
\pgfpathlineto{\pgfqpoint{3.133994in}{1.150505in}}%
\pgfpathlineto{\pgfqpoint{3.134897in}{0.976335in}}%
\pgfpathlineto{\pgfqpoint{3.134302in}{1.315824in}}%
\pgfpathlineto{\pgfqpoint{3.135074in}{1.188966in}}%
\pgfpathlineto{\pgfqpoint{3.135316in}{0.945413in}}%
\pgfpathlineto{\pgfqpoint{3.135735in}{1.249718in}}%
\pgfpathlineto{\pgfqpoint{3.136110in}{1.212568in}}%
\pgfpathlineto{\pgfqpoint{3.136132in}{1.281077in}}%
\pgfpathlineto{\pgfqpoint{3.136440in}{1.013376in}}%
\pgfpathlineto{\pgfqpoint{3.137190in}{1.127559in}}%
\pgfpathlineto{\pgfqpoint{3.138160in}{1.003651in}}%
\pgfpathlineto{\pgfqpoint{3.137785in}{1.255400in}}%
\pgfpathlineto{\pgfqpoint{3.138314in}{1.003979in}}%
\pgfpathlineto{\pgfqpoint{3.138358in}{0.844669in}}%
\pgfpathlineto{\pgfqpoint{3.138380in}{1.144386in}}%
\pgfpathlineto{\pgfqpoint{3.138490in}{1.076641in}}%
\pgfpathlineto{\pgfqpoint{3.139548in}{1.252231in}}%
\pgfpathlineto{\pgfqpoint{3.138909in}{0.970434in}}%
\pgfpathlineto{\pgfqpoint{3.139592in}{1.119145in}}%
\pgfpathlineto{\pgfqpoint{3.140231in}{0.886409in}}%
\pgfpathlineto{\pgfqpoint{3.140408in}{1.205247in}}%
\pgfpathlineto{\pgfqpoint{3.140738in}{1.021571in}}%
\pgfpathlineto{\pgfqpoint{3.141730in}{1.322380in}}%
\pgfpathlineto{\pgfqpoint{3.141047in}{0.913835in}}%
\pgfpathlineto{\pgfqpoint{3.141973in}{1.263813in}}%
\pgfpathlineto{\pgfqpoint{3.142193in}{1.065277in}}%
\pgfpathlineto{\pgfqpoint{3.142656in}{1.430990in}}%
\pgfpathlineto{\pgfqpoint{3.143097in}{1.079045in}}%
\pgfpathlineto{\pgfqpoint{3.143185in}{1.364229in}}%
\pgfpathlineto{\pgfqpoint{3.144221in}{1.241523in}}%
\pgfpathlineto{\pgfqpoint{3.144309in}{1.029984in}}%
\pgfpathlineto{\pgfqpoint{3.144552in}{1.392529in}}%
\pgfpathlineto{\pgfqpoint{3.145411in}{1.090627in}}%
\pgfpathlineto{\pgfqpoint{3.145962in}{1.292441in}}%
\pgfpathlineto{\pgfqpoint{3.146161in}{0.994473in}}%
\pgfpathlineto{\pgfqpoint{3.146491in}{1.045719in}}%
\pgfpathlineto{\pgfqpoint{3.146535in}{0.914600in}}%
\pgfpathlineto{\pgfqpoint{3.147351in}{1.292878in}}%
\pgfpathlineto{\pgfqpoint{3.147527in}{1.225461in}}%
\pgfpathlineto{\pgfqpoint{3.147748in}{1.267310in}}%
\pgfpathlineto{\pgfqpoint{3.148100in}{1.082541in}}%
\pgfpathlineto{\pgfqpoint{3.148122in}{1.136300in}}%
\pgfpathlineto{\pgfqpoint{3.148563in}{0.930989in}}%
\pgfpathlineto{\pgfqpoint{3.148806in}{1.189185in}}%
\pgfpathlineto{\pgfqpoint{3.149225in}{1.063092in}}%
\pgfpathlineto{\pgfqpoint{3.150128in}{1.284355in}}%
\pgfpathlineto{\pgfqpoint{3.149665in}{0.993271in}}%
\pgfpathlineto{\pgfqpoint{3.150371in}{1.116523in}}%
\pgfpathlineto{\pgfqpoint{3.150988in}{0.832869in}}%
\pgfpathlineto{\pgfqpoint{3.150635in}{1.284246in}}%
\pgfpathlineto{\pgfqpoint{3.151495in}{1.047685in}}%
\pgfpathlineto{\pgfqpoint{3.152046in}{1.186453in}}%
\pgfpathlineto{\pgfqpoint{3.151649in}{0.896680in}}%
\pgfpathlineto{\pgfqpoint{3.152178in}{0.985732in}}%
\pgfpathlineto{\pgfqpoint{3.152200in}{0.918533in}}%
\pgfpathlineto{\pgfqpoint{3.152839in}{1.251466in}}%
\pgfpathlineto{\pgfqpoint{3.153258in}{1.076532in}}%
\pgfpathlineto{\pgfqpoint{3.153501in}{0.927930in}}%
\pgfpathlineto{\pgfqpoint{3.154206in}{1.218468in}}%
\pgfpathlineto{\pgfqpoint{3.154360in}{1.018184in}}%
\pgfpathlineto{\pgfqpoint{3.154603in}{1.273320in}}%
\pgfpathlineto{\pgfqpoint{3.154978in}{0.946505in}}%
\pgfpathlineto{\pgfqpoint{3.155529in}{1.183612in}}%
\pgfpathlineto{\pgfqpoint{3.155551in}{1.183831in}}%
\pgfpathlineto{\pgfqpoint{3.155661in}{1.299762in}}%
\pgfpathlineto{\pgfqpoint{3.156697in}{0.975679in}}%
\pgfpathlineto{\pgfqpoint{3.157799in}{1.232891in}}%
\pgfpathlineto{\pgfqpoint{3.157248in}{0.914272in}}%
\pgfpathlineto{\pgfqpoint{3.157887in}{1.177930in}}%
\pgfpathlineto{\pgfqpoint{3.158967in}{0.904875in}}%
\pgfpathlineto{\pgfqpoint{3.158703in}{1.181645in}}%
\pgfpathlineto{\pgfqpoint{3.159011in}{0.995347in}}%
\pgfpathlineto{\pgfqpoint{3.159650in}{1.123407in}}%
\pgfpathlineto{\pgfqpoint{3.159188in}{0.827733in}}%
\pgfpathlineto{\pgfqpoint{3.160135in}{1.054023in}}%
\pgfpathlineto{\pgfqpoint{3.160929in}{0.826859in}}%
\pgfpathlineto{\pgfqpoint{3.161127in}{1.232673in}}%
\pgfpathlineto{\pgfqpoint{3.161282in}{0.915692in}}%
\pgfpathlineto{\pgfqpoint{3.162406in}{1.187655in}}%
\pgfpathlineto{\pgfqpoint{3.161568in}{0.859311in}}%
\pgfpathlineto{\pgfqpoint{3.162450in}{1.022445in}}%
\pgfpathlineto{\pgfqpoint{3.163331in}{0.817134in}}%
\pgfpathlineto{\pgfqpoint{3.163376in}{1.145151in}}%
\pgfpathlineto{\pgfqpoint{3.163552in}{0.984311in}}%
\pgfpathlineto{\pgfqpoint{3.163949in}{1.164163in}}%
\pgfpathlineto{\pgfqpoint{3.164147in}{0.885535in}}%
\pgfpathlineto{\pgfqpoint{3.164434in}{1.006492in}}%
\pgfpathlineto{\pgfqpoint{3.164456in}{0.813419in}}%
\pgfpathlineto{\pgfqpoint{3.164963in}{1.190387in}}%
\pgfpathlineto{\pgfqpoint{3.165558in}{0.882694in}}%
\pgfpathlineto{\pgfqpoint{3.165756in}{1.137939in}}%
\pgfpathlineto{\pgfqpoint{3.166109in}{0.834835in}}%
\pgfpathlineto{\pgfqpoint{3.166682in}{0.967266in}}%
\pgfpathlineto{\pgfqpoint{3.167012in}{0.862261in}}%
\pgfpathlineto{\pgfqpoint{3.167233in}{1.060251in}}%
\pgfpathlineto{\pgfqpoint{3.167497in}{0.954045in}}%
\pgfpathlineto{\pgfqpoint{3.168159in}{1.119364in}}%
\pgfpathlineto{\pgfqpoint{3.167960in}{0.831776in}}%
\pgfpathlineto{\pgfqpoint{3.168599in}{1.037524in}}%
\pgfpathlineto{\pgfqpoint{3.169415in}{1.200002in}}%
\pgfpathlineto{\pgfqpoint{3.169724in}{0.786758in}}%
\pgfpathlineto{\pgfqpoint{3.170253in}{1.216064in}}%
\pgfpathlineto{\pgfqpoint{3.170848in}{1.123735in}}%
\pgfpathlineto{\pgfqpoint{3.171178in}{0.921155in}}%
\pgfpathlineto{\pgfqpoint{3.171487in}{1.264250in}}%
\pgfpathlineto{\pgfqpoint{3.171972in}{1.027799in}}%
\pgfpathlineto{\pgfqpoint{3.172104in}{1.236715in}}%
\pgfpathlineto{\pgfqpoint{3.172986in}{0.854285in}}%
\pgfpathlineto{\pgfqpoint{3.173096in}{1.059268in}}%
\pgfpathlineto{\pgfqpoint{3.173559in}{0.810797in}}%
\pgfpathlineto{\pgfqpoint{3.173294in}{1.120129in}}%
\pgfpathlineto{\pgfqpoint{3.174242in}{0.876029in}}%
\pgfpathlineto{\pgfqpoint{3.174264in}{0.832650in}}%
\pgfpathlineto{\pgfqpoint{3.174705in}{1.163289in}}%
\pgfpathlineto{\pgfqpoint{3.175256in}{0.984967in}}%
\pgfpathlineto{\pgfqpoint{3.175410in}{1.268184in}}%
\pgfpathlineto{\pgfqpoint{3.176116in}{0.889468in}}%
\pgfpathlineto{\pgfqpoint{3.176336in}{1.012720in}}%
\pgfpathlineto{\pgfqpoint{3.176535in}{0.899084in}}%
\pgfpathlineto{\pgfqpoint{3.176402in}{1.241742in}}%
\pgfpathlineto{\pgfqpoint{3.177460in}{0.913616in}}%
\pgfpathlineto{\pgfqpoint{3.178166in}{1.298779in}}%
\pgfpathlineto{\pgfqpoint{3.177549in}{0.898428in}}%
\pgfpathlineto{\pgfqpoint{3.178629in}{1.242616in}}%
\pgfpathlineto{\pgfqpoint{3.178761in}{0.922685in}}%
\pgfpathlineto{\pgfqpoint{3.179775in}{1.093031in}}%
\pgfpathlineto{\pgfqpoint{3.180767in}{0.918424in}}%
\pgfpathlineto{\pgfqpoint{3.180436in}{1.213114in}}%
\pgfpathlineto{\pgfqpoint{3.180855in}{1.029766in}}%
\pgfpathlineto{\pgfqpoint{3.181759in}{1.263376in}}%
\pgfpathlineto{\pgfqpoint{3.180899in}{0.912961in}}%
\pgfpathlineto{\pgfqpoint{3.181935in}{1.005400in}}%
\pgfpathlineto{\pgfqpoint{3.182155in}{0.974587in}}%
\pgfpathlineto{\pgfqpoint{3.182508in}{1.281077in}}%
\pgfpathlineto{\pgfqpoint{3.182905in}{1.121003in}}%
\pgfpathlineto{\pgfqpoint{3.183279in}{1.262611in}}%
\pgfpathlineto{\pgfqpoint{3.183919in}{0.940168in}}%
\pgfpathlineto{\pgfqpoint{3.183963in}{1.117616in}}%
\pgfpathlineto{\pgfqpoint{3.184646in}{0.933503in}}%
\pgfpathlineto{\pgfqpoint{3.184337in}{1.232563in}}%
\pgfpathlineto{\pgfqpoint{3.185087in}{1.045500in}}%
\pgfpathlineto{\pgfqpoint{3.185880in}{0.929241in}}%
\pgfpathlineto{\pgfqpoint{3.186057in}{1.134115in}}%
\pgfpathlineto{\pgfqpoint{3.186189in}{1.016217in}}%
\pgfpathlineto{\pgfqpoint{3.186608in}{1.159464in}}%
\pgfpathlineto{\pgfqpoint{3.187093in}{0.903127in}}%
\pgfpathlineto{\pgfqpoint{3.187269in}{1.008459in}}%
\pgfpathlineto{\pgfqpoint{3.187600in}{0.894058in}}%
\pgfpathlineto{\pgfqpoint{3.188063in}{1.166130in}}%
\pgfpathlineto{\pgfqpoint{3.188371in}{1.000592in}}%
\pgfpathlineto{\pgfqpoint{3.188393in}{0.998953in}}%
\pgfpathlineto{\pgfqpoint{3.188437in}{1.057738in}}%
\pgfpathlineto{\pgfqpoint{3.188834in}{1.152909in}}%
\pgfpathlineto{\pgfqpoint{3.188658in}{0.873188in}}%
\pgfpathlineto{\pgfqpoint{3.189319in}{1.130728in}}%
\pgfpathlineto{\pgfqpoint{3.189716in}{0.840954in}}%
\pgfpathlineto{\pgfqpoint{3.190024in}{1.196833in}}%
\pgfpathlineto{\pgfqpoint{3.190421in}{0.961475in}}%
\pgfpathlineto{\pgfqpoint{3.191082in}{1.165037in}}%
\pgfpathlineto{\pgfqpoint{3.191391in}{0.876247in}}%
\pgfpathlineto{\pgfqpoint{3.191545in}{1.049871in}}%
\pgfpathlineto{\pgfqpoint{3.191567in}{1.048887in}}%
\pgfpathlineto{\pgfqpoint{3.192251in}{0.806863in}}%
\pgfpathlineto{\pgfqpoint{3.192691in}{0.913725in}}%
\pgfpathlineto{\pgfqpoint{3.193397in}{0.792331in}}%
\pgfpathlineto{\pgfqpoint{3.193066in}{1.112699in}}%
\pgfpathlineto{\pgfqpoint{3.193727in}{0.979613in}}%
\pgfpathlineto{\pgfqpoint{3.193749in}{1.091610in}}%
\pgfpathlineto{\pgfqpoint{3.194653in}{0.773865in}}%
\pgfpathlineto{\pgfqpoint{3.194829in}{0.941697in}}%
\pgfpathlineto{\pgfqpoint{3.195006in}{0.750701in}}%
\pgfpathlineto{\pgfqpoint{3.195601in}{1.057192in}}%
\pgfpathlineto{\pgfqpoint{3.195910in}{0.909355in}}%
\pgfpathlineto{\pgfqpoint{3.196262in}{1.047795in}}%
\pgfpathlineto{\pgfqpoint{3.196527in}{0.785010in}}%
\pgfpathlineto{\pgfqpoint{3.196990in}{0.871221in}}%
\pgfpathlineto{\pgfqpoint{3.197056in}{0.785775in}}%
\pgfpathlineto{\pgfqpoint{3.197298in}{1.052384in}}%
\pgfpathlineto{\pgfqpoint{3.198026in}{0.930115in}}%
\pgfpathlineto{\pgfqpoint{3.198070in}{1.060360in}}%
\pgfpathlineto{\pgfqpoint{3.198687in}{0.773865in}}%
\pgfpathlineto{\pgfqpoint{3.199128in}{0.939184in}}%
\pgfpathlineto{\pgfqpoint{3.200186in}{0.759114in}}%
\pgfpathlineto{\pgfqpoint{3.199546in}{1.104176in}}%
\pgfpathlineto{\pgfqpoint{3.200274in}{0.900067in}}%
\pgfpathlineto{\pgfqpoint{3.201288in}{1.146680in}}%
\pgfpathlineto{\pgfqpoint{3.200825in}{0.807410in}}%
\pgfpathlineto{\pgfqpoint{3.201398in}{1.105924in}}%
\pgfpathlineto{\pgfqpoint{3.201993in}{0.944101in}}%
\pgfpathlineto{\pgfqpoint{3.201795in}{1.211257in}}%
\pgfpathlineto{\pgfqpoint{3.202500in}{1.104394in}}%
\pgfpathlineto{\pgfqpoint{3.203624in}{0.907060in}}%
\pgfpathlineto{\pgfqpoint{3.203404in}{1.169298in}}%
\pgfpathlineto{\pgfqpoint{3.203690in}{0.954045in}}%
\pgfpathlineto{\pgfqpoint{3.204594in}{1.190824in}}%
\pgfpathlineto{\pgfqpoint{3.204263in}{0.883568in}}%
\pgfpathlineto{\pgfqpoint{3.204837in}{1.142528in}}%
\pgfpathlineto{\pgfqpoint{3.205850in}{1.256820in}}%
\pgfpathlineto{\pgfqpoint{3.205454in}{0.979066in}}%
\pgfpathlineto{\pgfqpoint{3.205917in}{1.105596in}}%
\pgfpathlineto{\pgfqpoint{3.206688in}{0.908371in}}%
\pgfpathlineto{\pgfqpoint{3.206622in}{1.227100in}}%
\pgfpathlineto{\pgfqpoint{3.207019in}{0.957541in}}%
\pgfpathlineto{\pgfqpoint{3.207195in}{1.136300in}}%
\pgfpathlineto{\pgfqpoint{3.207856in}{0.788397in}}%
\pgfpathlineto{\pgfqpoint{3.208099in}{1.071178in}}%
\pgfpathlineto{\pgfqpoint{3.209113in}{0.797357in}}%
\pgfpathlineto{\pgfqpoint{3.208716in}{1.146134in}}%
\pgfpathlineto{\pgfqpoint{3.209223in}{0.830574in}}%
\pgfpathlineto{\pgfqpoint{3.209421in}{1.082541in}}%
\pgfpathlineto{\pgfqpoint{3.209664in}{0.776487in}}%
\pgfpathlineto{\pgfqpoint{3.210369in}{1.002012in}}%
\pgfpathlineto{\pgfqpoint{3.210722in}{0.832650in}}%
\pgfpathlineto{\pgfqpoint{3.211096in}{1.149521in}}%
\pgfpathlineto{\pgfqpoint{3.211471in}{0.995019in}}%
\pgfpathlineto{\pgfqpoint{3.211912in}{1.104504in}}%
\pgfpathlineto{\pgfqpoint{3.211648in}{0.805115in}}%
\pgfpathlineto{\pgfqpoint{3.212529in}{1.068337in}}%
\pgfpathlineto{\pgfqpoint{3.212948in}{0.778891in}}%
\pgfpathlineto{\pgfqpoint{3.213257in}{1.123625in}}%
\pgfpathlineto{\pgfqpoint{3.213653in}{0.838550in}}%
\pgfpathlineto{\pgfqpoint{3.214513in}{1.069539in}}%
\pgfpathlineto{\pgfqpoint{3.213786in}{0.690932in}}%
\pgfpathlineto{\pgfqpoint{3.214755in}{0.890015in}}%
\pgfpathlineto{\pgfqpoint{3.215020in}{1.044517in}}%
\pgfpathlineto{\pgfqpoint{3.215880in}{0.754416in}}%
\pgfpathlineto{\pgfqpoint{3.216034in}{1.017200in}}%
\pgfpathlineto{\pgfqpoint{3.216232in}{0.745565in}}%
\pgfpathlineto{\pgfqpoint{3.216982in}{0.970653in}}%
\pgfpathlineto{\pgfqpoint{3.217621in}{0.738354in}}%
\pgfpathlineto{\pgfqpoint{3.217753in}{1.052602in}}%
\pgfpathlineto{\pgfqpoint{3.218106in}{0.841828in}}%
\pgfpathlineto{\pgfqpoint{3.219164in}{1.134770in}}%
\pgfpathlineto{\pgfqpoint{3.218392in}{0.775723in}}%
\pgfpathlineto{\pgfqpoint{3.219230in}{0.950329in}}%
\pgfpathlineto{\pgfqpoint{3.219891in}{0.897773in}}%
\pgfpathlineto{\pgfqpoint{3.219340in}{1.088442in}}%
\pgfpathlineto{\pgfqpoint{3.219957in}{1.017528in}}%
\pgfpathlineto{\pgfqpoint{3.219979in}{1.213223in}}%
\pgfpathlineto{\pgfqpoint{3.220090in}{0.780202in}}%
\pgfpathlineto{\pgfqpoint{3.221059in}{1.088551in}}%
\pgfpathlineto{\pgfqpoint{3.221170in}{0.793314in}}%
\pgfpathlineto{\pgfqpoint{3.222184in}{1.010863in}}%
\pgfpathlineto{\pgfqpoint{3.222757in}{1.188092in}}%
\pgfpathlineto{\pgfqpoint{3.222272in}{0.828279in}}%
\pgfpathlineto{\pgfqpoint{3.223286in}{1.104176in}}%
\pgfpathlineto{\pgfqpoint{3.224013in}{0.836365in}}%
\pgfpathlineto{\pgfqpoint{3.223462in}{1.170173in}}%
\pgfpathlineto{\pgfqpoint{3.224388in}{1.081230in}}%
\pgfpathlineto{\pgfqpoint{3.224432in}{1.179788in}}%
\pgfpathlineto{\pgfqpoint{3.224917in}{0.925089in}}%
\pgfpathlineto{\pgfqpoint{3.225115in}{0.933065in}}%
\pgfpathlineto{\pgfqpoint{3.225137in}{0.791894in}}%
\pgfpathlineto{\pgfqpoint{3.225424in}{1.158918in}}%
\pgfpathlineto{\pgfqpoint{3.226217in}{0.893511in}}%
\pgfpathlineto{\pgfqpoint{3.227011in}{0.751466in}}%
\pgfpathlineto{\pgfqpoint{3.226724in}{1.077624in}}%
\pgfpathlineto{\pgfqpoint{3.227275in}{0.864119in}}%
\pgfpathlineto{\pgfqpoint{3.227804in}{1.069866in}}%
\pgfpathlineto{\pgfqpoint{3.227937in}{0.732672in}}%
\pgfpathlineto{\pgfqpoint{3.228355in}{0.798778in}}%
\pgfpathlineto{\pgfqpoint{3.228796in}{0.736605in}}%
\pgfpathlineto{\pgfqpoint{3.228973in}{1.023319in}}%
\pgfpathlineto{\pgfqpoint{3.229435in}{0.775395in}}%
\pgfpathlineto{\pgfqpoint{3.230537in}{1.061344in}}%
\pgfpathlineto{\pgfqpoint{3.229502in}{0.698253in}}%
\pgfpathlineto{\pgfqpoint{3.230582in}{0.893183in}}%
\pgfpathlineto{\pgfqpoint{3.231507in}{1.038726in}}%
\pgfpathlineto{\pgfqpoint{3.231067in}{0.822161in}}%
\pgfpathlineto{\pgfqpoint{3.231662in}{0.969779in}}%
\pgfpathlineto{\pgfqpoint{3.231684in}{0.819210in}}%
\pgfpathlineto{\pgfqpoint{3.231794in}{1.153892in}}%
\pgfpathlineto{\pgfqpoint{3.232764in}{1.031077in}}%
\pgfpathlineto{\pgfqpoint{3.232984in}{1.185688in}}%
\pgfpathlineto{\pgfqpoint{3.233337in}{0.821833in}}%
\pgfpathlineto{\pgfqpoint{3.233756in}{0.971964in}}%
\pgfpathlineto{\pgfqpoint{3.233778in}{0.811671in}}%
\pgfpathlineto{\pgfqpoint{3.233998in}{1.167113in}}%
\pgfpathlineto{\pgfqpoint{3.234858in}{0.957869in}}%
\pgfpathlineto{\pgfqpoint{3.235166in}{1.124390in}}%
\pgfpathlineto{\pgfqpoint{3.235277in}{0.847620in}}%
\pgfpathlineto{\pgfqpoint{3.235938in}{0.898865in}}%
\pgfpathlineto{\pgfqpoint{3.236753in}{1.099477in}}%
\pgfpathlineto{\pgfqpoint{3.236709in}{0.842703in}}%
\pgfpathlineto{\pgfqpoint{3.236864in}{0.970325in}}%
\pgfpathlineto{\pgfqpoint{3.236886in}{0.763813in}}%
\pgfpathlineto{\pgfqpoint{3.237657in}{1.110841in}}%
\pgfpathlineto{\pgfqpoint{3.237966in}{0.986715in}}%
\pgfpathlineto{\pgfqpoint{3.238054in}{0.785447in}}%
\pgfpathlineto{\pgfqpoint{3.238296in}{1.108000in}}%
\pgfpathlineto{\pgfqpoint{3.239046in}{0.964643in}}%
\pgfpathlineto{\pgfqpoint{3.239905in}{1.127231in}}%
\pgfpathlineto{\pgfqpoint{3.240060in}{0.838550in}}%
\pgfpathlineto{\pgfqpoint{3.240148in}{0.958415in}}%
\pgfpathlineto{\pgfqpoint{3.240236in}{1.105706in}}%
\pgfpathlineto{\pgfqpoint{3.240809in}{0.788070in}}%
\pgfpathlineto{\pgfqpoint{3.241162in}{0.890015in}}%
\pgfpathlineto{\pgfqpoint{3.242198in}{0.751575in}}%
\pgfpathlineto{\pgfqpoint{3.242021in}{0.981142in}}%
\pgfpathlineto{\pgfqpoint{3.242242in}{0.917331in}}%
\pgfpathlineto{\pgfqpoint{3.242462in}{1.050745in}}%
\pgfpathlineto{\pgfqpoint{3.243146in}{0.753214in}}%
\pgfpathlineto{\pgfqpoint{3.243278in}{0.832213in}}%
\pgfpathlineto{\pgfqpoint{3.243300in}{0.728738in}}%
\pgfpathlineto{\pgfqpoint{3.243851in}{1.058940in}}%
\pgfpathlineto{\pgfqpoint{3.244358in}{0.924652in}}%
\pgfpathlineto{\pgfqpoint{3.244622in}{1.009115in}}%
\pgfpathlineto{\pgfqpoint{3.244997in}{0.675526in}}%
\pgfpathlineto{\pgfqpoint{3.245284in}{0.950876in}}%
\pgfpathlineto{\pgfqpoint{3.246143in}{0.709180in}}%
\pgfpathlineto{\pgfqpoint{3.245526in}{1.003105in}}%
\pgfpathlineto{\pgfqpoint{3.246386in}{0.974040in}}%
\pgfpathlineto{\pgfqpoint{3.247422in}{0.706885in}}%
\pgfpathlineto{\pgfqpoint{3.247378in}{1.015233in}}%
\pgfpathlineto{\pgfqpoint{3.247554in}{0.819757in}}%
\pgfpathlineto{\pgfqpoint{3.247973in}{0.999171in}}%
\pgfpathlineto{\pgfqpoint{3.247598in}{0.758021in}}%
\pgfpathlineto{\pgfqpoint{3.248347in}{0.882148in}}%
\pgfpathlineto{\pgfqpoint{3.248369in}{0.676181in}}%
\pgfpathlineto{\pgfqpoint{3.248766in}{0.974696in}}%
\pgfpathlineto{\pgfqpoint{3.249450in}{0.757475in}}%
\pgfpathlineto{\pgfqpoint{3.249494in}{1.036868in}}%
\pgfpathlineto{\pgfqpoint{3.250574in}{1.022336in}}%
\pgfpathlineto{\pgfqpoint{3.250926in}{1.102974in}}%
\pgfpathlineto{\pgfqpoint{3.251676in}{0.762064in}}%
\pgfpathlineto{\pgfqpoint{3.252271in}{1.092922in}}%
\pgfpathlineto{\pgfqpoint{3.252822in}{1.069320in}}%
\pgfpathlineto{\pgfqpoint{3.253505in}{0.792550in}}%
\pgfpathlineto{\pgfqpoint{3.253990in}{0.826094in}}%
\pgfpathlineto{\pgfqpoint{3.254762in}{1.088551in}}%
\pgfpathlineto{\pgfqpoint{3.254078in}{0.723384in}}%
\pgfpathlineto{\pgfqpoint{3.255092in}{0.887611in}}%
\pgfpathlineto{\pgfqpoint{3.255357in}{0.656841in}}%
\pgfpathlineto{\pgfqpoint{3.256062in}{1.069866in}}%
\pgfpathlineto{\pgfqpoint{3.256150in}{1.004853in}}%
\pgfpathlineto{\pgfqpoint{3.256238in}{0.815495in}}%
\pgfpathlineto{\pgfqpoint{3.256194in}{1.076532in}}%
\pgfpathlineto{\pgfqpoint{3.257032in}{0.963660in}}%
\pgfpathlineto{\pgfqpoint{3.257759in}{1.194867in}}%
\pgfpathlineto{\pgfqpoint{3.257958in}{0.874717in}}%
\pgfpathlineto{\pgfqpoint{3.258134in}{1.058831in}}%
\pgfpathlineto{\pgfqpoint{3.258795in}{0.808393in}}%
\pgfpathlineto{\pgfqpoint{3.259082in}{1.148866in}}%
\pgfpathlineto{\pgfqpoint{3.259236in}{1.044407in}}%
\pgfpathlineto{\pgfqpoint{3.259435in}{1.266545in}}%
\pgfpathlineto{\pgfqpoint{3.260162in}{1.019823in}}%
\pgfpathlineto{\pgfqpoint{3.260360in}{1.202624in}}%
\pgfpathlineto{\pgfqpoint{3.261330in}{1.349041in}}%
\pgfpathlineto{\pgfqpoint{3.260603in}{0.988245in}}%
\pgfpathlineto{\pgfqpoint{3.261462in}{1.229067in}}%
\pgfpathlineto{\pgfqpoint{3.261484in}{1.137065in}}%
\pgfpathlineto{\pgfqpoint{3.262102in}{1.413508in}}%
\pgfpathlineto{\pgfqpoint{3.262542in}{1.268293in}}%
\pgfpathlineto{\pgfqpoint{3.262851in}{1.258787in}}%
\pgfpathlineto{\pgfqpoint{3.263667in}{1.602428in}}%
\pgfpathlineto{\pgfqpoint{3.264592in}{1.330684in}}%
\pgfpathlineto{\pgfqpoint{3.263777in}{1.665584in}}%
\pgfpathlineto{\pgfqpoint{3.264769in}{1.476554in}}%
\pgfpathlineto{\pgfqpoint{3.264835in}{1.734422in}}%
\pgfpathlineto{\pgfqpoint{3.265694in}{1.365431in}}%
\pgfpathlineto{\pgfqpoint{3.265849in}{1.462459in}}%
\pgfpathlineto{\pgfqpoint{3.265871in}{1.358001in}}%
\pgfpathlineto{\pgfqpoint{3.266642in}{1.662962in}}%
\pgfpathlineto{\pgfqpoint{3.266907in}{1.520698in}}%
\pgfpathlineto{\pgfqpoint{3.268009in}{1.698473in}}%
\pgfpathlineto{\pgfqpoint{3.267061in}{1.436235in}}%
\pgfpathlineto{\pgfqpoint{3.268031in}{1.622752in}}%
\pgfpathlineto{\pgfqpoint{3.268582in}{1.423779in}}%
\pgfpathlineto{\pgfqpoint{3.268295in}{1.710930in}}%
\pgfpathlineto{\pgfqpoint{3.269199in}{1.559268in}}%
\pgfpathlineto{\pgfqpoint{3.269530in}{1.440169in}}%
\pgfpathlineto{\pgfqpoint{3.269596in}{1.746004in}}%
\pgfpathlineto{\pgfqpoint{3.269618in}{1.599041in}}%
\pgfpathlineto{\pgfqpoint{3.269640in}{1.757914in}}%
\pgfpathlineto{\pgfqpoint{3.270544in}{1.502559in}}%
\pgfpathlineto{\pgfqpoint{3.270720in}{1.621659in}}%
\pgfpathlineto{\pgfqpoint{3.271778in}{1.802385in}}%
\pgfpathlineto{\pgfqpoint{3.271227in}{1.472730in}}%
\pgfpathlineto{\pgfqpoint{3.271844in}{1.734640in}}%
\pgfpathlineto{\pgfqpoint{3.272329in}{1.499063in}}%
\pgfpathlineto{\pgfqpoint{3.272836in}{1.830576in}}%
\pgfpathlineto{\pgfqpoint{3.272946in}{1.681428in}}%
\pgfpathlineto{\pgfqpoint{3.273541in}{1.790147in}}%
\pgfpathlineto{\pgfqpoint{3.273343in}{1.508351in}}%
\pgfpathlineto{\pgfqpoint{3.273740in}{1.578936in}}%
\pgfpathlineto{\pgfqpoint{3.273872in}{1.728631in}}%
\pgfpathlineto{\pgfqpoint{3.273938in}{1.498735in}}%
\pgfpathlineto{\pgfqpoint{3.274269in}{1.567026in}}%
\pgfpathlineto{\pgfqpoint{3.274798in}{1.327515in}}%
\pgfpathlineto{\pgfqpoint{3.275217in}{1.633897in}}%
\pgfpathlineto{\pgfqpoint{3.275327in}{1.618491in}}%
\pgfpathlineto{\pgfqpoint{3.275834in}{1.381165in}}%
\pgfpathlineto{\pgfqpoint{3.276385in}{1.683832in}}%
\pgfpathlineto{\pgfqpoint{3.276539in}{1.497424in}}%
\pgfpathlineto{\pgfqpoint{3.277024in}{1.800200in}}%
\pgfpathlineto{\pgfqpoint{3.277487in}{1.613464in}}%
\pgfpathlineto{\pgfqpoint{3.278545in}{1.825440in}}%
\pgfpathlineto{\pgfqpoint{3.278347in}{1.555007in}}%
\pgfpathlineto{\pgfqpoint{3.278655in}{1.764798in}}%
\pgfpathlineto{\pgfqpoint{3.279735in}{1.538945in}}%
\pgfpathlineto{\pgfqpoint{3.279162in}{1.831122in}}%
\pgfpathlineto{\pgfqpoint{3.279801in}{1.667332in}}%
\pgfpathlineto{\pgfqpoint{3.280837in}{1.885974in}}%
\pgfpathlineto{\pgfqpoint{3.280463in}{1.551292in}}%
\pgfpathlineto{\pgfqpoint{3.280948in}{1.776489in}}%
\pgfpathlineto{\pgfqpoint{3.281014in}{1.662525in}}%
\pgfpathlineto{\pgfqpoint{3.281256in}{1.939951in}}%
\pgfpathlineto{\pgfqpoint{3.282072in}{1.705248in}}%
\pgfpathlineto{\pgfqpoint{3.282975in}{1.844890in}}%
\pgfpathlineto{\pgfqpoint{3.282535in}{1.390016in}}%
\pgfpathlineto{\pgfqpoint{3.283152in}{1.746987in}}%
\pgfpathlineto{\pgfqpoint{3.283328in}{1.621659in}}%
\pgfpathlineto{\pgfqpoint{3.283879in}{1.945961in}}%
\pgfpathlineto{\pgfqpoint{3.284232in}{1.800528in}}%
\pgfpathlineto{\pgfqpoint{3.284276in}{1.947272in}}%
\pgfpathlineto{\pgfqpoint{3.285202in}{1.657389in}}%
\pgfpathlineto{\pgfqpoint{3.285290in}{1.684050in}}%
\pgfpathlineto{\pgfqpoint{3.286370in}{1.544080in}}%
\pgfpathlineto{\pgfqpoint{3.286017in}{1.897884in}}%
\pgfpathlineto{\pgfqpoint{3.286414in}{1.613683in}}%
\pgfpathlineto{\pgfqpoint{3.286656in}{1.826533in}}%
\pgfpathlineto{\pgfqpoint{3.287450in}{1.527909in}}%
\pgfpathlineto{\pgfqpoint{3.287494in}{1.623298in}}%
\pgfpathlineto{\pgfqpoint{3.288243in}{1.492835in}}%
\pgfpathlineto{\pgfqpoint{3.287869in}{1.801402in}}%
\pgfpathlineto{\pgfqpoint{3.288508in}{1.584290in}}%
\pgfpathlineto{\pgfqpoint{3.289456in}{1.857892in}}%
\pgfpathlineto{\pgfqpoint{3.289037in}{1.538180in}}%
\pgfpathlineto{\pgfqpoint{3.289632in}{1.822599in}}%
\pgfpathlineto{\pgfqpoint{3.289963in}{1.997425in}}%
\pgfpathlineto{\pgfqpoint{3.290315in}{1.690169in}}%
\pgfpathlineto{\pgfqpoint{3.290734in}{1.832761in}}%
\pgfpathlineto{\pgfqpoint{3.290800in}{1.704374in}}%
\pgfpathlineto{\pgfqpoint{3.291484in}{2.033264in}}%
\pgfpathlineto{\pgfqpoint{3.291506in}{2.088990in}}%
\pgfpathlineto{\pgfqpoint{3.292057in}{1.758351in}}%
\pgfpathlineto{\pgfqpoint{3.292498in}{1.864339in}}%
\pgfpathlineto{\pgfqpoint{3.292806in}{1.733438in}}%
\pgfpathlineto{\pgfqpoint{3.292850in}{2.029440in}}%
\pgfpathlineto{\pgfqpoint{3.293622in}{1.817792in}}%
\pgfpathlineto{\pgfqpoint{3.294349in}{2.078500in}}%
\pgfpathlineto{\pgfqpoint{3.294525in}{1.775069in}}%
\pgfpathlineto{\pgfqpoint{3.295032in}{2.024304in}}%
\pgfpathlineto{\pgfqpoint{3.295231in}{1.677057in}}%
\pgfpathlineto{\pgfqpoint{3.296134in}{1.852647in}}%
\pgfpathlineto{\pgfqpoint{3.296994in}{2.055991in}}%
\pgfpathlineto{\pgfqpoint{3.296421in}{1.709837in}}%
\pgfpathlineto{\pgfqpoint{3.297237in}{1.951096in}}%
\pgfpathlineto{\pgfqpoint{3.297810in}{1.820414in}}%
\pgfpathlineto{\pgfqpoint{3.297677in}{2.085275in}}%
\pgfpathlineto{\pgfqpoint{3.298339in}{1.955030in}}%
\pgfpathlineto{\pgfqpoint{3.299397in}{2.223824in}}%
\pgfpathlineto{\pgfqpoint{3.298757in}{1.866852in}}%
\pgfpathlineto{\pgfqpoint{3.299441in}{2.154440in}}%
\pgfpathlineto{\pgfqpoint{3.299573in}{1.813858in}}%
\pgfpathlineto{\pgfqpoint{3.300168in}{2.160012in}}%
\pgfpathlineto{\pgfqpoint{3.300565in}{1.981909in}}%
\pgfpathlineto{\pgfqpoint{3.301336in}{2.080685in}}%
\pgfpathlineto{\pgfqpoint{3.301491in}{1.844452in}}%
\pgfpathlineto{\pgfqpoint{3.301513in}{1.946507in}}%
\pgfpathlineto{\pgfqpoint{3.301535in}{1.856581in}}%
\pgfpathlineto{\pgfqpoint{3.302350in}{2.216612in}}%
\pgfpathlineto{\pgfqpoint{3.302571in}{2.140782in}}%
\pgfpathlineto{\pgfqpoint{3.302659in}{2.018404in}}%
\pgfpathlineto{\pgfqpoint{3.302615in}{2.172797in}}%
\pgfpathlineto{\pgfqpoint{3.302747in}{2.087897in}}%
\pgfpathlineto{\pgfqpoint{3.303034in}{1.853412in}}%
\pgfpathlineto{\pgfqpoint{3.303254in}{2.194431in}}%
\pgfpathlineto{\pgfqpoint{3.303827in}{2.141874in}}%
\pgfpathlineto{\pgfqpoint{3.304114in}{2.274086in}}%
\pgfpathlineto{\pgfqpoint{3.304444in}{1.899850in}}%
\pgfpathlineto{\pgfqpoint{3.304907in}{2.073365in}}%
\pgfpathlineto{\pgfqpoint{3.305084in}{2.245021in}}%
\pgfpathlineto{\pgfqpoint{3.305613in}{1.942355in}}%
\pgfpathlineto{\pgfqpoint{3.306075in}{2.205686in}}%
\pgfpathlineto{\pgfqpoint{3.306164in}{2.272993in}}%
\pgfpathlineto{\pgfqpoint{3.307244in}{1.925091in}}%
\pgfpathlineto{\pgfqpoint{3.307442in}{2.215082in}}%
\pgfpathlineto{\pgfqpoint{3.308302in}{1.880729in}}%
\pgfpathlineto{\pgfqpoint{3.308346in}{2.029003in}}%
\pgfpathlineto{\pgfqpoint{3.309271in}{1.775834in}}%
\pgfpathlineto{\pgfqpoint{3.308654in}{2.107565in}}%
\pgfpathlineto{\pgfqpoint{3.309470in}{1.873954in}}%
\pgfpathlineto{\pgfqpoint{3.310374in}{2.162198in}}%
\pgfpathlineto{\pgfqpoint{3.309536in}{1.833635in}}%
\pgfpathlineto{\pgfqpoint{3.310616in}{2.014580in}}%
\pgfpathlineto{\pgfqpoint{3.311652in}{2.274632in}}%
\pgfpathlineto{\pgfqpoint{3.310947in}{1.849151in}}%
\pgfpathlineto{\pgfqpoint{3.312115in}{2.161761in}}%
\pgfpathlineto{\pgfqpoint{3.312798in}{2.261521in}}%
\pgfpathlineto{\pgfqpoint{3.313239in}{1.965519in}}%
\pgfpathlineto{\pgfqpoint{3.313570in}{2.240541in}}%
\pgfpathlineto{\pgfqpoint{3.313371in}{1.849588in}}%
\pgfpathlineto{\pgfqpoint{3.314363in}{2.071289in}}%
\pgfpathlineto{\pgfqpoint{3.314892in}{2.189514in}}%
\pgfpathlineto{\pgfqpoint{3.314495in}{1.969234in}}%
\pgfpathlineto{\pgfqpoint{3.314980in}{2.105380in}}%
\pgfpathlineto{\pgfqpoint{3.315531in}{1.783264in}}%
\pgfpathlineto{\pgfqpoint{3.315399in}{2.170502in}}%
\pgfpathlineto{\pgfqpoint{3.316105in}{1.996332in}}%
\pgfpathlineto{\pgfqpoint{3.316369in}{2.055882in}}%
\pgfpathlineto{\pgfqpoint{3.316457in}{1.825987in}}%
\pgfpathlineto{\pgfqpoint{3.317140in}{1.952189in}}%
\pgfpathlineto{\pgfqpoint{3.317207in}{1.829592in}}%
\pgfpathlineto{\pgfqpoint{3.317978in}{2.211914in}}%
\pgfpathlineto{\pgfqpoint{3.318221in}{2.053369in}}%
\pgfpathlineto{\pgfqpoint{3.318243in}{2.054899in}}%
\pgfpathlineto{\pgfqpoint{3.319212in}{1.882477in}}%
\pgfpathlineto{\pgfqpoint{3.318992in}{2.252342in}}%
\pgfpathlineto{\pgfqpoint{3.319345in}{2.044628in}}%
\pgfpathlineto{\pgfqpoint{3.319852in}{2.260209in}}%
\pgfpathlineto{\pgfqpoint{3.319786in}{2.012831in}}%
\pgfpathlineto{\pgfqpoint{3.320491in}{2.205358in}}%
\pgfpathlineto{\pgfqpoint{3.320667in}{2.030642in}}%
\pgfpathlineto{\pgfqpoint{3.321549in}{2.333199in}}%
\pgfpathlineto{\pgfqpoint{3.321571in}{2.302932in}}%
\pgfpathlineto{\pgfqpoint{3.322210in}{2.009226in}}%
\pgfpathlineto{\pgfqpoint{3.323004in}{2.091831in}}%
\pgfpathlineto{\pgfqpoint{3.323555in}{2.281516in}}%
\pgfpathlineto{\pgfqpoint{3.323290in}{1.980161in}}%
\pgfpathlineto{\pgfqpoint{3.324106in}{2.135646in}}%
\pgfpathlineto{\pgfqpoint{3.325120in}{1.957980in}}%
\pgfpathlineto{\pgfqpoint{3.324260in}{2.278566in}}%
\pgfpathlineto{\pgfqpoint{3.325208in}{2.165039in}}%
\pgfpathlineto{\pgfqpoint{3.326067in}{1.972840in}}%
\pgfpathlineto{\pgfqpoint{3.325781in}{2.237591in}}%
\pgfpathlineto{\pgfqpoint{3.326398in}{1.993163in}}%
\pgfpathlineto{\pgfqpoint{3.327478in}{2.216721in}}%
\pgfpathlineto{\pgfqpoint{3.327324in}{1.940279in}}%
\pgfpathlineto{\pgfqpoint{3.327544in}{2.087132in}}%
\pgfpathlineto{\pgfqpoint{3.327566in}{2.127123in}}%
\pgfpathlineto{\pgfqpoint{3.327721in}{1.869365in}}%
\pgfpathlineto{\pgfqpoint{3.328602in}{2.015344in}}%
\pgfpathlineto{\pgfqpoint{3.329550in}{1.925637in}}%
\pgfpathlineto{\pgfqpoint{3.329175in}{2.138706in}}%
\pgfpathlineto{\pgfqpoint{3.329682in}{1.984094in}}%
\pgfpathlineto{\pgfqpoint{3.330300in}{2.095546in}}%
\pgfpathlineto{\pgfqpoint{3.330630in}{1.857346in}}%
\pgfpathlineto{\pgfqpoint{3.330762in}{1.950550in}}%
\pgfpathlineto{\pgfqpoint{3.331115in}{1.838334in}}%
\pgfpathlineto{\pgfqpoint{3.331512in}{2.187438in}}%
\pgfpathlineto{\pgfqpoint{3.331732in}{2.071616in}}%
\pgfpathlineto{\pgfqpoint{3.332702in}{2.348059in}}%
\pgfpathlineto{\pgfqpoint{3.331820in}{1.966175in}}%
\pgfpathlineto{\pgfqpoint{3.332878in}{2.194650in}}%
\pgfpathlineto{\pgfqpoint{3.332901in}{2.091940in}}%
\pgfpathlineto{\pgfqpoint{3.333099in}{2.417662in}}%
\pgfpathlineto{\pgfqpoint{3.333959in}{2.317574in}}%
\pgfpathlineto{\pgfqpoint{3.334862in}{2.063312in}}%
\pgfpathlineto{\pgfqpoint{3.334730in}{2.322491in}}%
\pgfpathlineto{\pgfqpoint{3.335171in}{2.092377in}}%
\pgfpathlineto{\pgfqpoint{3.335193in}{2.277364in}}%
\pgfpathlineto{\pgfqpoint{3.335656in}{2.002888in}}%
\pgfpathlineto{\pgfqpoint{3.336273in}{2.139908in}}%
\pgfpathlineto{\pgfqpoint{3.336868in}{2.416678in}}%
\pgfpathlineto{\pgfqpoint{3.337088in}{2.092814in}}%
\pgfpathlineto{\pgfqpoint{3.337551in}{2.328173in}}%
\pgfpathlineto{\pgfqpoint{3.338323in}{1.982346in}}%
\pgfpathlineto{\pgfqpoint{3.338676in}{2.129090in}}%
\pgfpathlineto{\pgfqpoint{3.338874in}{2.330795in}}%
\pgfpathlineto{\pgfqpoint{3.339359in}{2.030860in}}%
\pgfpathlineto{\pgfqpoint{3.339734in}{2.183723in}}%
\pgfpathlineto{\pgfqpoint{3.340703in}{1.971529in}}%
\pgfpathlineto{\pgfqpoint{3.340174in}{2.266875in}}%
\pgfpathlineto{\pgfqpoint{3.340836in}{2.086586in}}%
\pgfpathlineto{\pgfqpoint{3.341916in}{2.412854in}}%
\pgfpathlineto{\pgfqpoint{3.341012in}{2.063312in}}%
\pgfpathlineto{\pgfqpoint{3.341982in}{2.302277in}}%
\pgfpathlineto{\pgfqpoint{3.342511in}{2.066918in}}%
\pgfpathlineto{\pgfqpoint{3.343018in}{2.346529in}}%
\pgfpathlineto{\pgfqpoint{3.343062in}{2.313313in}}%
\pgfpathlineto{\pgfqpoint{3.343701in}{2.385865in}}%
\pgfpathlineto{\pgfqpoint{3.343833in}{1.984422in}}%
\pgfpathlineto{\pgfqpoint{3.344098in}{2.201861in}}%
\pgfpathlineto{\pgfqpoint{3.344120in}{2.009007in}}%
\pgfpathlineto{\pgfqpoint{3.344340in}{2.350681in}}%
\pgfpathlineto{\pgfqpoint{3.345200in}{2.184379in}}%
\pgfpathlineto{\pgfqpoint{3.346015in}{2.403020in}}%
\pgfpathlineto{\pgfqpoint{3.345486in}{2.092377in}}%
\pgfpathlineto{\pgfqpoint{3.346324in}{2.312875in}}%
\pgfpathlineto{\pgfqpoint{3.347360in}{2.079702in}}%
\pgfpathlineto{\pgfqpoint{3.346721in}{2.410996in}}%
\pgfpathlineto{\pgfqpoint{3.347426in}{2.119912in}}%
\pgfpathlineto{\pgfqpoint{3.348154in}{2.429462in}}%
\pgfpathlineto{\pgfqpoint{3.348550in}{2.360406in}}%
\pgfpathlineto{\pgfqpoint{3.349630in}{1.996114in}}%
\pgfpathlineto{\pgfqpoint{3.349145in}{2.421923in}}%
\pgfpathlineto{\pgfqpoint{3.350071in}{2.098059in}}%
\pgfpathlineto{\pgfqpoint{3.350093in}{2.305336in}}%
\pgfpathlineto{\pgfqpoint{3.350534in}{2.017748in}}%
\pgfpathlineto{\pgfqpoint{3.351173in}{2.118382in}}%
\pgfpathlineto{\pgfqpoint{3.352143in}{2.011848in}}%
\pgfpathlineto{\pgfqpoint{3.351790in}{2.291678in}}%
\pgfpathlineto{\pgfqpoint{3.352253in}{2.152036in}}%
\pgfpathlineto{\pgfqpoint{3.352804in}{2.289930in}}%
\pgfpathlineto{\pgfqpoint{3.352959in}{1.975572in}}%
\pgfpathlineto{\pgfqpoint{3.353355in}{2.202408in}}%
\pgfpathlineto{\pgfqpoint{3.353642in}{2.020589in}}%
\pgfpathlineto{\pgfqpoint{3.354171in}{2.306320in}}%
\pgfpathlineto{\pgfqpoint{3.354458in}{2.149632in}}%
\pgfpathlineto{\pgfqpoint{3.355185in}{2.321180in}}%
\pgfpathlineto{\pgfqpoint{3.354634in}{2.057084in}}%
\pgfpathlineto{\pgfqpoint{3.355560in}{2.164274in}}%
\pgfpathlineto{\pgfqpoint{3.356574in}{2.045283in}}%
\pgfpathlineto{\pgfqpoint{3.356684in}{2.381932in}}%
\pgfpathlineto{\pgfqpoint{3.357411in}{2.117180in}}%
\pgfpathlineto{\pgfqpoint{3.357059in}{2.420502in}}%
\pgfpathlineto{\pgfqpoint{3.357808in}{2.246988in}}%
\pgfpathlineto{\pgfqpoint{3.357940in}{2.464100in}}%
\pgfpathlineto{\pgfqpoint{3.358469in}{2.150397in}}%
\pgfpathlineto{\pgfqpoint{3.358866in}{2.185908in}}%
\pgfpathlineto{\pgfqpoint{3.358976in}{2.119147in}}%
\pgfpathlineto{\pgfqpoint{3.359064in}{2.339427in}}%
\pgfpathlineto{\pgfqpoint{3.359946in}{2.205576in}}%
\pgfpathlineto{\pgfqpoint{3.360585in}{2.459073in}}%
\pgfpathlineto{\pgfqpoint{3.360012in}{2.183395in}}%
\pgfpathlineto{\pgfqpoint{3.361114in}{2.340192in}}%
\pgfpathlineto{\pgfqpoint{3.361247in}{2.410122in}}%
\pgfpathlineto{\pgfqpoint{3.361599in}{2.085056in}}%
\pgfpathlineto{\pgfqpoint{3.361996in}{2.288946in}}%
\pgfpathlineto{\pgfqpoint{3.363032in}{2.120021in}}%
\pgfpathlineto{\pgfqpoint{3.362150in}{2.425419in}}%
\pgfpathlineto{\pgfqpoint{3.363098in}{2.240541in}}%
\pgfpathlineto{\pgfqpoint{3.363142in}{2.373846in}}%
\pgfpathlineto{\pgfqpoint{3.363561in}{2.050637in}}%
\pgfpathlineto{\pgfqpoint{3.364112in}{2.222731in}}%
\pgfpathlineto{\pgfqpoint{3.364134in}{2.097075in}}%
\pgfpathlineto{\pgfqpoint{3.365016in}{2.331341in}}%
\pgfpathlineto{\pgfqpoint{3.365214in}{2.150397in}}%
\pgfpathlineto{\pgfqpoint{3.365699in}{2.358549in}}%
\pgfpathlineto{\pgfqpoint{3.365346in}{2.073802in}}%
\pgfpathlineto{\pgfqpoint{3.366338in}{2.310472in}}%
\pgfpathlineto{\pgfqpoint{3.366735in}{1.987591in}}%
\pgfpathlineto{\pgfqpoint{3.367462in}{2.195852in}}%
\pgfpathlineto{\pgfqpoint{3.368035in}{2.404113in}}%
\pgfpathlineto{\pgfqpoint{3.368168in}{2.104942in}}%
\pgfpathlineto{\pgfqpoint{3.368498in}{2.170721in}}%
\pgfpathlineto{\pgfqpoint{3.369027in}{2.114886in}}%
\pgfpathlineto{\pgfqpoint{3.369380in}{2.373409in}}%
\pgfpathlineto{\pgfqpoint{3.369468in}{2.246770in}}%
\pgfpathlineto{\pgfqpoint{3.369667in}{2.420393in}}%
\pgfpathlineto{\pgfqpoint{3.370460in}{2.201424in}}%
\pgfpathlineto{\pgfqpoint{3.370592in}{2.373409in}}%
\pgfpathlineto{\pgfqpoint{3.371209in}{2.166678in}}%
\pgfpathlineto{\pgfqpoint{3.370879in}{2.512614in}}%
\pgfpathlineto{\pgfqpoint{3.371716in}{2.276490in}}%
\pgfpathlineto{\pgfqpoint{3.372245in}{2.471530in}}%
\pgfpathlineto{\pgfqpoint{3.372598in}{2.203719in}}%
\pgfpathlineto{\pgfqpoint{3.372819in}{2.253981in}}%
\pgfpathlineto{\pgfqpoint{3.372885in}{2.131822in}}%
\pgfpathlineto{\pgfqpoint{3.373678in}{2.450878in}}%
\pgfpathlineto{\pgfqpoint{3.373899in}{2.253653in}}%
\pgfpathlineto{\pgfqpoint{3.374582in}{2.476993in}}%
\pgfpathlineto{\pgfqpoint{3.373943in}{2.217814in}}%
\pgfpathlineto{\pgfqpoint{3.375001in}{2.341722in}}%
\pgfpathlineto{\pgfqpoint{3.375971in}{2.120240in}}%
\pgfpathlineto{\pgfqpoint{3.375772in}{2.423999in}}%
\pgfpathlineto{\pgfqpoint{3.376125in}{2.293535in}}%
\pgfpathlineto{\pgfqpoint{3.376169in}{2.164602in}}%
\pgfpathlineto{\pgfqpoint{3.376235in}{2.345000in}}%
\pgfpathlineto{\pgfqpoint{3.376367in}{2.198146in}}%
\pgfpathlineto{\pgfqpoint{3.376389in}{2.068448in}}%
\pgfpathlineto{\pgfqpoint{3.376566in}{2.389252in}}%
\pgfpathlineto{\pgfqpoint{3.377469in}{2.145152in}}%
\pgfpathlineto{\pgfqpoint{3.377513in}{2.053587in}}%
\pgfpathlineto{\pgfqpoint{3.378373in}{2.307303in}}%
\pgfpathlineto{\pgfqpoint{3.378417in}{2.267967in}}%
\pgfpathlineto{\pgfqpoint{3.378968in}{2.463553in}}%
\pgfpathlineto{\pgfqpoint{3.379101in}{2.094671in}}%
\pgfpathlineto{\pgfqpoint{3.379541in}{2.342268in}}%
\pgfpathlineto{\pgfqpoint{3.379850in}{2.488247in}}%
\pgfpathlineto{\pgfqpoint{3.380181in}{2.170283in}}%
\pgfpathlineto{\pgfqpoint{3.380577in}{2.333964in}}%
\pgfpathlineto{\pgfqpoint{3.381106in}{2.099807in}}%
\pgfpathlineto{\pgfqpoint{3.380842in}{2.409139in}}%
\pgfpathlineto{\pgfqpoint{3.381679in}{2.284139in}}%
\pgfpathlineto{\pgfqpoint{3.381944in}{2.399305in}}%
\pgfpathlineto{\pgfqpoint{3.382363in}{2.088771in}}%
\pgfpathlineto{\pgfqpoint{3.382605in}{2.253216in}}%
\pgfpathlineto{\pgfqpoint{3.382737in}{2.111498in}}%
\pgfpathlineto{\pgfqpoint{3.383663in}{2.392858in}}%
\pgfpathlineto{\pgfqpoint{3.383707in}{2.266328in}}%
\pgfpathlineto{\pgfqpoint{3.384699in}{2.460384in}}%
\pgfpathlineto{\pgfqpoint{3.384523in}{2.234532in}}%
\pgfpathlineto{\pgfqpoint{3.384831in}{2.291459in}}%
\pgfpathlineto{\pgfqpoint{3.385162in}{2.190825in}}%
\pgfpathlineto{\pgfqpoint{3.385713in}{2.444213in}}%
\pgfpathlineto{\pgfqpoint{3.385845in}{2.263597in}}%
\pgfpathlineto{\pgfqpoint{3.386573in}{2.494257in}}%
\pgfpathlineto{\pgfqpoint{3.386947in}{2.285450in}}%
\pgfpathlineto{\pgfqpoint{3.387080in}{2.460166in}}%
\pgfpathlineto{\pgfqpoint{3.387543in}{2.245568in}}%
\pgfpathlineto{\pgfqpoint{3.387653in}{2.300310in}}%
\pgfpathlineto{\pgfqpoint{3.388667in}{2.082980in}}%
\pgfpathlineto{\pgfqpoint{3.387807in}{2.317574in}}%
\pgfpathlineto{\pgfqpoint{3.388777in}{2.180336in}}%
\pgfpathlineto{\pgfqpoint{3.389835in}{2.417224in}}%
\pgfpathlineto{\pgfqpoint{3.388843in}{2.027254in}}%
\pgfpathlineto{\pgfqpoint{3.389901in}{2.275944in}}%
\pgfpathlineto{\pgfqpoint{3.389989in}{2.047796in}}%
\pgfpathlineto{\pgfqpoint{3.390805in}{2.473278in}}%
\pgfpathlineto{\pgfqpoint{3.391025in}{2.174217in}}%
\pgfpathlineto{\pgfqpoint{3.391797in}{2.496005in}}%
\pgfpathlineto{\pgfqpoint{3.391202in}{2.156188in}}%
\pgfpathlineto{\pgfqpoint{3.392149in}{2.405424in}}%
\pgfpathlineto{\pgfqpoint{3.392260in}{2.123955in}}%
\pgfpathlineto{\pgfqpoint{3.392899in}{2.460494in}}%
\pgfpathlineto{\pgfqpoint{3.393296in}{2.190825in}}%
\pgfpathlineto{\pgfqpoint{3.394354in}{2.440061in}}%
\pgfpathlineto{\pgfqpoint{3.393538in}{2.180336in}}%
\pgfpathlineto{\pgfqpoint{3.394420in}{2.314624in}}%
\pgfpathlineto{\pgfqpoint{3.394971in}{2.444541in}}%
\pgfpathlineto{\pgfqpoint{3.394464in}{2.156844in}}%
\pgfpathlineto{\pgfqpoint{3.395478in}{2.338116in}}%
\pgfpathlineto{\pgfqpoint{3.395786in}{2.215520in}}%
\pgfpathlineto{\pgfqpoint{3.396403in}{2.584511in}}%
\pgfpathlineto{\pgfqpoint{3.396558in}{2.443339in}}%
\pgfpathlineto{\pgfqpoint{3.396844in}{2.216066in}}%
\pgfpathlineto{\pgfqpoint{3.397373in}{2.524524in}}%
\pgfpathlineto{\pgfqpoint{3.397638in}{2.428260in}}%
\pgfpathlineto{\pgfqpoint{3.398718in}{2.584620in}}%
\pgfpathlineto{\pgfqpoint{3.398542in}{2.330795in}}%
\pgfpathlineto{\pgfqpoint{3.398740in}{2.412417in}}%
\pgfpathlineto{\pgfqpoint{3.399291in}{2.214099in}}%
\pgfpathlineto{\pgfqpoint{3.398872in}{2.569213in}}%
\pgfpathlineto{\pgfqpoint{3.399886in}{2.291787in}}%
\pgfpathlineto{\pgfqpoint{3.400040in}{2.531407in}}%
\pgfpathlineto{\pgfqpoint{3.400415in}{2.193229in}}%
\pgfpathlineto{\pgfqpoint{3.401010in}{2.400397in}}%
\pgfpathlineto{\pgfqpoint{3.402046in}{2.485734in}}%
\pgfpathlineto{\pgfqpoint{3.401870in}{2.277910in}}%
\pgfpathlineto{\pgfqpoint{3.402090in}{2.392421in}}%
\pgfpathlineto{\pgfqpoint{3.402575in}{2.234204in}}%
\pgfpathlineto{\pgfqpoint{3.402641in}{2.502233in}}%
\pgfpathlineto{\pgfqpoint{3.403192in}{2.382041in}}%
\pgfpathlineto{\pgfqpoint{3.403281in}{2.495459in}}%
\pgfpathlineto{\pgfqpoint{3.403964in}{2.219562in}}%
\pgfpathlineto{\pgfqpoint{3.404250in}{2.277583in}}%
\pgfpathlineto{\pgfqpoint{3.404405in}{2.215082in}}%
\pgfpathlineto{\pgfqpoint{3.404779in}{2.470765in}}%
\pgfpathlineto{\pgfqpoint{3.405286in}{2.356800in}}%
\pgfpathlineto{\pgfqpoint{3.405793in}{2.530424in}}%
\pgfpathlineto{\pgfqpoint{3.406036in}{2.187329in}}%
\pgfpathlineto{\pgfqpoint{3.406366in}{2.451425in}}%
\pgfpathlineto{\pgfqpoint{3.406719in}{2.194322in}}%
\pgfpathlineto{\pgfqpoint{3.406631in}{2.504637in}}%
\pgfpathlineto{\pgfqpoint{3.407469in}{2.434270in}}%
\pgfpathlineto{\pgfqpoint{3.408593in}{2.183505in}}%
\pgfpathlineto{\pgfqpoint{3.407535in}{2.454047in}}%
\pgfpathlineto{\pgfqpoint{3.408615in}{2.229724in}}%
\pgfpathlineto{\pgfqpoint{3.408879in}{2.443339in}}%
\pgfpathlineto{\pgfqpoint{3.409474in}{2.221092in}}%
\pgfpathlineto{\pgfqpoint{3.409739in}{2.348168in}}%
\pgfpathlineto{\pgfqpoint{3.410643in}{2.548781in}}%
\pgfpathlineto{\pgfqpoint{3.409783in}{2.229724in}}%
\pgfpathlineto{\pgfqpoint{3.410863in}{2.453064in}}%
\pgfpathlineto{\pgfqpoint{3.411083in}{2.200659in}}%
\pgfpathlineto{\pgfqpoint{3.410951in}{2.586587in}}%
\pgfpathlineto{\pgfqpoint{3.412009in}{2.272775in}}%
\pgfpathlineto{\pgfqpoint{3.412296in}{2.440389in}}%
\pgfpathlineto{\pgfqpoint{3.412957in}{2.176949in}}%
\pgfpathlineto{\pgfqpoint{3.413089in}{2.317355in}}%
\pgfpathlineto{\pgfqpoint{3.413795in}{2.493383in}}%
\pgfpathlineto{\pgfqpoint{3.414235in}{2.180773in}}%
\pgfpathlineto{\pgfqpoint{3.414257in}{2.157718in}}%
\pgfpathlineto{\pgfqpoint{3.414919in}{2.441372in}}%
\pgfpathlineto{\pgfqpoint{3.415183in}{2.333527in}}%
\pgfpathlineto{\pgfqpoint{3.415800in}{2.468798in}}%
\pgfpathlineto{\pgfqpoint{3.416043in}{2.164274in}}%
\pgfpathlineto{\pgfqpoint{3.416241in}{2.283264in}}%
\pgfpathlineto{\pgfqpoint{3.416748in}{2.083854in}}%
\pgfpathlineto{\pgfqpoint{3.416329in}{2.404331in}}%
\pgfpathlineto{\pgfqpoint{3.417365in}{2.144934in}}%
\pgfpathlineto{\pgfqpoint{3.418357in}{2.488794in}}%
\pgfpathlineto{\pgfqpoint{3.417762in}{2.128544in}}%
\pgfpathlineto{\pgfqpoint{3.418534in}{2.413619in}}%
\pgfpathlineto{\pgfqpoint{3.419459in}{2.234969in}}%
\pgfpathlineto{\pgfqpoint{3.418754in}{2.538728in}}%
\pgfpathlineto{\pgfqpoint{3.419614in}{2.458308in}}%
\pgfpathlineto{\pgfqpoint{3.419636in}{2.469126in}}%
\pgfpathlineto{\pgfqpoint{3.420055in}{2.202626in}}%
\pgfpathlineto{\pgfqpoint{3.420231in}{2.364777in}}%
\pgfpathlineto{\pgfqpoint{3.420363in}{2.181975in}}%
\pgfpathlineto{\pgfqpoint{3.420782in}{2.462570in}}%
\pgfpathlineto{\pgfqpoint{3.421311in}{2.417771in}}%
\pgfpathlineto{\pgfqpoint{3.421333in}{2.427933in}}%
\pgfpathlineto{\pgfqpoint{3.421840in}{2.186892in}}%
\pgfpathlineto{\pgfqpoint{3.422038in}{2.248409in}}%
\pgfpathlineto{\pgfqpoint{3.422589in}{2.204811in}}%
\pgfpathlineto{\pgfqpoint{3.422655in}{2.436018in}}%
\pgfpathlineto{\pgfqpoint{3.422964in}{2.226446in}}%
\pgfpathlineto{\pgfqpoint{3.423978in}{2.494585in}}%
\pgfpathlineto{\pgfqpoint{3.423890in}{2.170174in}}%
\pgfpathlineto{\pgfqpoint{3.424088in}{2.318011in}}%
\pgfpathlineto{\pgfqpoint{3.424198in}{2.499611in}}%
\pgfpathlineto{\pgfqpoint{3.424529in}{2.218798in}}%
\pgfpathlineto{\pgfqpoint{3.425190in}{2.327189in}}%
\pgfpathlineto{\pgfqpoint{3.425543in}{2.539930in}}%
\pgfpathlineto{\pgfqpoint{3.425477in}{2.269715in}}%
\pgfpathlineto{\pgfqpoint{3.426292in}{2.319541in}}%
\pgfpathlineto{\pgfqpoint{3.427218in}{2.238684in}}%
\pgfpathlineto{\pgfqpoint{3.427086in}{2.550420in}}%
\pgfpathlineto{\pgfqpoint{3.427350in}{2.402801in}}%
\pgfpathlineto{\pgfqpoint{3.427813in}{2.552823in}}%
\pgfpathlineto{\pgfqpoint{3.427615in}{2.268623in}}%
\pgfpathlineto{\pgfqpoint{3.428453in}{2.433833in}}%
\pgfpathlineto{\pgfqpoint{3.428717in}{2.227648in}}%
\pgfpathlineto{\pgfqpoint{3.429444in}{2.515564in}}%
\pgfpathlineto{\pgfqpoint{3.429511in}{2.397884in}}%
\pgfpathlineto{\pgfqpoint{3.430458in}{2.630074in}}%
\pgfpathlineto{\pgfqpoint{3.429907in}{2.365869in}}%
\pgfpathlineto{\pgfqpoint{3.430569in}{2.399305in}}%
\pgfpathlineto{\pgfqpoint{3.430591in}{2.269934in}}%
\pgfpathlineto{\pgfqpoint{3.431186in}{2.678479in}}%
\pgfpathlineto{\pgfqpoint{3.431649in}{2.369803in}}%
\pgfpathlineto{\pgfqpoint{3.432200in}{2.622098in}}%
\pgfpathlineto{\pgfqpoint{3.432773in}{2.462461in}}%
\pgfpathlineto{\pgfqpoint{3.432795in}{2.342268in}}%
\pgfpathlineto{\pgfqpoint{3.433346in}{2.726993in}}%
\pgfpathlineto{\pgfqpoint{3.433875in}{2.434161in}}%
\pgfpathlineto{\pgfqpoint{3.433897in}{2.432849in}}%
\pgfpathlineto{\pgfqpoint{3.433919in}{2.476774in}}%
\pgfpathlineto{\pgfqpoint{3.433941in}{2.438968in}}%
\pgfpathlineto{\pgfqpoint{3.434889in}{2.563313in}}%
\pgfpathlineto{\pgfqpoint{3.434514in}{2.292333in}}%
\pgfpathlineto{\pgfqpoint{3.435065in}{2.497426in}}%
\pgfpathlineto{\pgfqpoint{3.435682in}{2.542990in}}%
\pgfpathlineto{\pgfqpoint{3.435594in}{2.331232in}}%
\pgfpathlineto{\pgfqpoint{3.435837in}{2.405970in}}%
\pgfpathlineto{\pgfqpoint{3.435881in}{2.321945in}}%
\pgfpathlineto{\pgfqpoint{3.436718in}{2.643842in}}%
\pgfpathlineto{\pgfqpoint{3.436873in}{2.460275in}}%
\pgfpathlineto{\pgfqpoint{3.437798in}{2.665586in}}%
\pgfpathlineto{\pgfqpoint{3.436917in}{2.337023in}}%
\pgfpathlineto{\pgfqpoint{3.438041in}{2.579484in}}%
\pgfpathlineto{\pgfqpoint{3.438614in}{2.382150in}}%
\pgfpathlineto{\pgfqpoint{3.438151in}{2.689624in}}%
\pgfpathlineto{\pgfqpoint{3.439143in}{2.453501in}}%
\pgfpathlineto{\pgfqpoint{3.440201in}{2.331341in}}%
\pgfpathlineto{\pgfqpoint{3.440267in}{2.632697in}}%
\pgfpathlineto{\pgfqpoint{3.440510in}{2.416350in}}%
\pgfpathlineto{\pgfqpoint{3.441281in}{2.649415in}}%
\pgfpathlineto{\pgfqpoint{3.441391in}{2.489012in}}%
\pgfpathlineto{\pgfqpoint{3.442273in}{2.393732in}}%
\pgfpathlineto{\pgfqpoint{3.441545in}{2.626032in}}%
\pgfpathlineto{\pgfqpoint{3.442317in}{2.427933in}}%
\pgfpathlineto{\pgfqpoint{3.443353in}{2.632588in}}%
\pgfpathlineto{\pgfqpoint{3.443133in}{2.259882in}}%
\pgfpathlineto{\pgfqpoint{3.443419in}{2.387613in}}%
\pgfpathlineto{\pgfqpoint{3.444345in}{2.556320in}}%
\pgfpathlineto{\pgfqpoint{3.443551in}{2.312111in}}%
\pgfpathlineto{\pgfqpoint{3.444543in}{2.502343in}}%
\pgfpathlineto{\pgfqpoint{3.445337in}{2.295611in}}%
\pgfpathlineto{\pgfqpoint{3.444918in}{2.532063in}}%
\pgfpathlineto{\pgfqpoint{3.445667in}{2.450441in}}%
\pgfpathlineto{\pgfqpoint{3.446351in}{2.587242in}}%
\pgfpathlineto{\pgfqpoint{3.446527in}{2.369257in}}%
\pgfpathlineto{\pgfqpoint{3.446747in}{2.459183in}}%
\pgfpathlineto{\pgfqpoint{3.446990in}{2.334947in}}%
\pgfpathlineto{\pgfqpoint{3.446858in}{2.595000in}}%
\pgfpathlineto{\pgfqpoint{3.447849in}{2.416023in}}%
\pgfpathlineto{\pgfqpoint{3.447982in}{2.346529in}}%
\pgfpathlineto{\pgfqpoint{3.448401in}{2.603523in}}%
\pgfpathlineto{\pgfqpoint{3.448687in}{2.460931in}}%
\pgfpathlineto{\pgfqpoint{3.449084in}{2.659576in}}%
\pgfpathlineto{\pgfqpoint{3.449525in}{2.409357in}}%
\pgfpathlineto{\pgfqpoint{3.449789in}{2.519934in}}%
\pgfpathlineto{\pgfqpoint{3.450781in}{2.653020in}}%
\pgfpathlineto{\pgfqpoint{3.450362in}{2.429462in}}%
\pgfpathlineto{\pgfqpoint{3.450869in}{2.600136in}}%
\pgfpathlineto{\pgfqpoint{3.451993in}{2.299654in}}%
\pgfpathlineto{\pgfqpoint{3.451663in}{2.746443in}}%
\pgfpathlineto{\pgfqpoint{3.452015in}{2.368273in}}%
\pgfpathlineto{\pgfqpoint{3.453073in}{2.624502in}}%
\pgfpathlineto{\pgfqpoint{3.452390in}{2.348933in}}%
\pgfpathlineto{\pgfqpoint{3.453294in}{2.461368in}}%
\pgfpathlineto{\pgfqpoint{3.453470in}{2.314733in}}%
\pgfpathlineto{\pgfqpoint{3.453867in}{2.620131in}}%
\pgfpathlineto{\pgfqpoint{3.454396in}{2.458745in}}%
\pgfpathlineto{\pgfqpoint{3.454616in}{2.714974in}}%
\pgfpathlineto{\pgfqpoint{3.455167in}{2.312548in}}%
\pgfpathlineto{\pgfqpoint{3.455432in}{2.389689in}}%
\pgfpathlineto{\pgfqpoint{3.455454in}{2.242945in}}%
\pgfpathlineto{\pgfqpoint{3.455807in}{2.670721in}}%
\pgfpathlineto{\pgfqpoint{3.456534in}{2.412307in}}%
\pgfpathlineto{\pgfqpoint{3.457019in}{2.353413in}}%
\pgfpathlineto{\pgfqpoint{3.457328in}{2.607566in}}%
\pgfpathlineto{\pgfqpoint{3.457548in}{2.468470in}}%
\pgfpathlineto{\pgfqpoint{3.457570in}{2.641766in}}%
\pgfpathlineto{\pgfqpoint{3.458011in}{2.356036in}}%
\pgfpathlineto{\pgfqpoint{3.458650in}{2.448693in}}%
\pgfpathlineto{\pgfqpoint{3.459003in}{2.620241in}}%
\pgfpathlineto{\pgfqpoint{3.459091in}{2.362591in}}%
\pgfpathlineto{\pgfqpoint{3.459752in}{2.485297in}}%
\pgfpathlineto{\pgfqpoint{3.459862in}{2.357019in}}%
\pgfpathlineto{\pgfqpoint{3.460413in}{2.665695in}}%
\pgfpathlineto{\pgfqpoint{3.460810in}{2.512177in}}%
\pgfpathlineto{\pgfqpoint{3.461493in}{2.676403in}}%
\pgfpathlineto{\pgfqpoint{3.461538in}{2.356036in}}%
\pgfpathlineto{\pgfqpoint{3.461912in}{2.609532in}}%
\pgfpathlineto{\pgfqpoint{3.462926in}{2.677387in}}%
\pgfpathlineto{\pgfqpoint{3.463036in}{2.410996in}}%
\pgfpathlineto{\pgfqpoint{3.463477in}{2.627015in}}%
\pgfpathlineto{\pgfqpoint{3.464116in}{2.378654in}}%
\pgfpathlineto{\pgfqpoint{3.464139in}{2.443120in}}%
\pgfpathlineto{\pgfqpoint{3.465373in}{2.697382in}}%
\pgfpathlineto{\pgfqpoint{3.464623in}{2.409467in}}%
\pgfpathlineto{\pgfqpoint{3.465439in}{2.575660in}}%
\pgfpathlineto{\pgfqpoint{3.465549in}{2.439515in}}%
\pgfpathlineto{\pgfqpoint{3.465792in}{2.728305in}}%
\pgfpathlineto{\pgfqpoint{3.466519in}{2.633462in}}%
\pgfpathlineto{\pgfqpoint{3.467268in}{2.761521in}}%
\pgfpathlineto{\pgfqpoint{3.466629in}{2.452408in}}%
\pgfpathlineto{\pgfqpoint{3.467511in}{2.595000in}}%
\pgfpathlineto{\pgfqpoint{3.467533in}{2.529659in}}%
\pgfpathlineto{\pgfqpoint{3.467687in}{2.764472in}}%
\pgfpathlineto{\pgfqpoint{3.468613in}{2.609970in}}%
\pgfpathlineto{\pgfqpoint{3.469076in}{2.428151in}}%
\pgfpathlineto{\pgfqpoint{3.468811in}{2.742400in}}%
\pgfpathlineto{\pgfqpoint{3.469759in}{2.524851in}}%
\pgfpathlineto{\pgfqpoint{3.469936in}{2.720984in}}%
\pgfpathlineto{\pgfqpoint{3.470332in}{2.362591in}}%
\pgfpathlineto{\pgfqpoint{3.470817in}{2.510210in}}%
\pgfpathlineto{\pgfqpoint{3.470972in}{2.459183in}}%
\pgfpathlineto{\pgfqpoint{3.470883in}{2.547797in}}%
\pgfpathlineto{\pgfqpoint{3.470994in}{2.516219in}}%
\pgfpathlineto{\pgfqpoint{3.471721in}{2.831889in}}%
\pgfpathlineto{\pgfqpoint{3.471170in}{2.450332in}}%
\pgfpathlineto{\pgfqpoint{3.472118in}{2.601556in}}%
\pgfpathlineto{\pgfqpoint{3.472801in}{2.726884in}}%
\pgfpathlineto{\pgfqpoint{3.472294in}{2.506822in}}%
\pgfpathlineto{\pgfqpoint{3.473198in}{2.553370in}}%
\pgfpathlineto{\pgfqpoint{3.473617in}{2.391656in}}%
\pgfpathlineto{\pgfqpoint{3.474079in}{2.680555in}}%
\pgfpathlineto{\pgfqpoint{3.474234in}{2.575769in}}%
\pgfpathlineto{\pgfqpoint{3.474256in}{2.709074in}}%
\pgfpathlineto{\pgfqpoint{3.475027in}{2.410778in}}%
\pgfpathlineto{\pgfqpoint{3.475336in}{2.581233in}}%
\pgfpathlineto{\pgfqpoint{3.475424in}{2.403129in}}%
\pgfpathlineto{\pgfqpoint{3.475843in}{2.652692in}}%
\pgfpathlineto{\pgfqpoint{3.476416in}{2.528785in}}%
\pgfpathlineto{\pgfqpoint{3.476504in}{2.677605in}}%
\pgfpathlineto{\pgfqpoint{3.477474in}{2.364340in}}%
\pgfpathlineto{\pgfqpoint{3.477540in}{2.619694in}}%
\pgfpathlineto{\pgfqpoint{3.477650in}{2.387613in}}%
\pgfpathlineto{\pgfqpoint{3.478179in}{2.657609in}}%
\pgfpathlineto{\pgfqpoint{3.478642in}{2.530533in}}%
\pgfpathlineto{\pgfqpoint{3.479744in}{2.706233in}}%
\pgfpathlineto{\pgfqpoint{3.479193in}{2.278347in}}%
\pgfpathlineto{\pgfqpoint{3.479788in}{2.634882in}}%
\pgfpathlineto{\pgfqpoint{3.480670in}{2.392421in}}%
\pgfpathlineto{\pgfqpoint{3.480406in}{2.651928in}}%
\pgfpathlineto{\pgfqpoint{3.480890in}{2.565170in}}%
\pgfpathlineto{\pgfqpoint{3.480912in}{2.602430in}}%
\pgfpathlineto{\pgfqpoint{3.481596in}{2.279112in}}%
\pgfpathlineto{\pgfqpoint{3.481970in}{2.526709in}}%
\pgfpathlineto{\pgfqpoint{3.481993in}{2.529331in}}%
\pgfpathlineto{\pgfqpoint{3.482125in}{2.413946in}}%
\pgfpathlineto{\pgfqpoint{3.482279in}{2.523212in}}%
\pgfpathlineto{\pgfqpoint{3.482522in}{2.278129in}}%
\pgfpathlineto{\pgfqpoint{3.483293in}{2.623409in}}%
\pgfpathlineto{\pgfqpoint{3.483403in}{2.417771in}}%
\pgfpathlineto{\pgfqpoint{3.483932in}{2.656845in}}%
\pgfpathlineto{\pgfqpoint{3.483778in}{2.339318in}}%
\pgfpathlineto{\pgfqpoint{3.484527in}{2.508134in}}%
\pgfpathlineto{\pgfqpoint{3.485145in}{2.586259in}}%
\pgfpathlineto{\pgfqpoint{3.485519in}{2.357893in}}%
\pgfpathlineto{\pgfqpoint{3.485652in}{2.549436in}}%
\pgfpathlineto{\pgfqpoint{3.486136in}{2.388597in}}%
\pgfpathlineto{\pgfqpoint{3.486423in}{2.638051in}}%
\pgfpathlineto{\pgfqpoint{3.486754in}{2.487045in}}%
\pgfpathlineto{\pgfqpoint{3.487812in}{2.686019in}}%
\pgfpathlineto{\pgfqpoint{3.487327in}{2.361499in}}%
\pgfpathlineto{\pgfqpoint{3.487856in}{2.589755in}}%
\pgfpathlineto{\pgfqpoint{3.488914in}{2.408046in}}%
\pgfpathlineto{\pgfqpoint{3.488186in}{2.707216in}}%
\pgfpathlineto{\pgfqpoint{3.488958in}{2.504856in}}%
\pgfpathlineto{\pgfqpoint{3.489994in}{2.672470in}}%
\pgfpathlineto{\pgfqpoint{3.489244in}{2.455905in}}%
\pgfpathlineto{\pgfqpoint{3.490038in}{2.556757in}}%
\pgfpathlineto{\pgfqpoint{3.490721in}{2.266328in}}%
\pgfpathlineto{\pgfqpoint{3.490611in}{2.661980in}}%
\pgfpathlineto{\pgfqpoint{3.491140in}{2.541132in}}%
\pgfpathlineto{\pgfqpoint{3.491956in}{2.325987in}}%
\pgfpathlineto{\pgfqpoint{3.491603in}{2.597513in}}%
\pgfpathlineto{\pgfqpoint{3.492242in}{2.514471in}}%
\pgfpathlineto{\pgfqpoint{3.492705in}{2.344344in}}%
\pgfpathlineto{\pgfqpoint{3.492881in}{2.575114in}}%
\pgfpathlineto{\pgfqpoint{3.493256in}{2.488684in}}%
\pgfpathlineto{\pgfqpoint{3.494336in}{2.704375in}}%
\pgfpathlineto{\pgfqpoint{3.493300in}{2.369366in}}%
\pgfpathlineto{\pgfqpoint{3.494402in}{2.593361in}}%
\pgfpathlineto{\pgfqpoint{3.494424in}{2.344016in}}%
\pgfpathlineto{\pgfqpoint{3.494997in}{2.697273in}}%
\pgfpathlineto{\pgfqpoint{3.495504in}{2.548562in}}%
\pgfpathlineto{\pgfqpoint{3.495592in}{2.740542in}}%
\pgfpathlineto{\pgfqpoint{3.496364in}{2.446726in}}%
\pgfpathlineto{\pgfqpoint{3.496628in}{2.639471in}}%
\pgfpathlineto{\pgfqpoint{3.496937in}{2.428042in}}%
\pgfpathlineto{\pgfqpoint{3.497686in}{2.680009in}}%
\pgfpathlineto{\pgfqpoint{3.497775in}{2.568339in}}%
\pgfpathlineto{\pgfqpoint{3.498700in}{2.687985in}}%
\pgfpathlineto{\pgfqpoint{3.498304in}{2.450113in}}%
\pgfpathlineto{\pgfqpoint{3.498811in}{2.546486in}}%
\pgfpathlineto{\pgfqpoint{3.499648in}{2.396027in}}%
\pgfpathlineto{\pgfqpoint{3.499494in}{2.604288in}}%
\pgfpathlineto{\pgfqpoint{3.499891in}{2.568230in}}%
\pgfpathlineto{\pgfqpoint{3.500927in}{2.647776in}}%
\pgfpathlineto{\pgfqpoint{3.500794in}{2.371988in}}%
\pgfpathlineto{\pgfqpoint{3.500971in}{2.574349in}}%
\pgfpathlineto{\pgfqpoint{3.501169in}{2.404659in}}%
\pgfpathlineto{\pgfqpoint{3.501698in}{2.695962in}}%
\pgfpathlineto{\pgfqpoint{3.502095in}{2.505074in}}%
\pgfpathlineto{\pgfqpoint{3.502183in}{2.593252in}}%
\pgfpathlineto{\pgfqpoint{3.502800in}{2.399960in}}%
\pgfpathlineto{\pgfqpoint{3.503021in}{2.424873in}}%
\pgfpathlineto{\pgfqpoint{3.503748in}{2.315279in}}%
\pgfpathlineto{\pgfqpoint{3.503792in}{2.627780in}}%
\pgfpathlineto{\pgfqpoint{3.504101in}{2.484860in}}%
\pgfpathlineto{\pgfqpoint{3.504762in}{2.403785in}}%
\pgfpathlineto{\pgfqpoint{3.505048in}{2.634991in}}%
\pgfpathlineto{\pgfqpoint{3.505247in}{2.646246in}}%
\pgfpathlineto{\pgfqpoint{3.505225in}{2.489340in}}%
\pgfpathlineto{\pgfqpoint{3.505357in}{2.512395in}}%
\pgfpathlineto{\pgfqpoint{3.506349in}{2.273758in}}%
\pgfpathlineto{\pgfqpoint{3.505533in}{2.633789in}}%
\pgfpathlineto{\pgfqpoint{3.506481in}{2.412635in}}%
\pgfpathlineto{\pgfqpoint{3.507473in}{2.679353in}}%
\pgfpathlineto{\pgfqpoint{3.506635in}{2.270480in}}%
\pgfpathlineto{\pgfqpoint{3.507583in}{2.508243in}}%
\pgfpathlineto{\pgfqpoint{3.507649in}{2.386193in}}%
\pgfpathlineto{\pgfqpoint{3.507804in}{2.662417in}}%
\pgfpathlineto{\pgfqpoint{3.508663in}{2.519934in}}%
\pgfpathlineto{\pgfqpoint{3.509545in}{2.747098in}}%
\pgfpathlineto{\pgfqpoint{3.509082in}{2.413619in}}%
\pgfpathlineto{\pgfqpoint{3.509787in}{2.546377in}}%
\pgfpathlineto{\pgfqpoint{3.509810in}{2.547142in}}%
\pgfpathlineto{\pgfqpoint{3.510361in}{2.443339in}}%
\pgfpathlineto{\pgfqpoint{3.510846in}{2.754310in}}%
\pgfpathlineto{\pgfqpoint{3.511110in}{2.433396in}}%
\pgfpathlineto{\pgfqpoint{3.510890in}{2.860188in}}%
\pgfpathlineto{\pgfqpoint{3.512036in}{2.644279in}}%
\pgfpathlineto{\pgfqpoint{3.512939in}{2.760429in}}%
\pgfpathlineto{\pgfqpoint{3.512675in}{2.409248in}}%
\pgfpathlineto{\pgfqpoint{3.513050in}{2.581560in}}%
\pgfpathlineto{\pgfqpoint{3.513579in}{2.455140in}}%
\pgfpathlineto{\pgfqpoint{3.514020in}{2.773213in}}%
\pgfpathlineto{\pgfqpoint{3.514130in}{2.605490in}}%
\pgfpathlineto{\pgfqpoint{3.514328in}{2.728195in}}%
\pgfpathlineto{\pgfqpoint{3.514350in}{2.474480in}}%
\pgfpathlineto{\pgfqpoint{3.515232in}{2.605271in}}%
\pgfpathlineto{\pgfqpoint{3.516114in}{2.421377in}}%
\pgfpathlineto{\pgfqpoint{3.515849in}{2.758243in}}%
\pgfpathlineto{\pgfqpoint{3.516246in}{2.694214in}}%
\pgfpathlineto{\pgfqpoint{3.516532in}{2.809380in}}%
\pgfpathlineto{\pgfqpoint{3.516621in}{2.517203in}}%
\pgfpathlineto{\pgfqpoint{3.517326in}{2.623191in}}%
\pgfpathlineto{\pgfqpoint{3.517348in}{2.621005in}}%
\pgfpathlineto{\pgfqpoint{3.517370in}{2.713007in}}%
\pgfpathlineto{\pgfqpoint{3.517965in}{2.831670in}}%
\pgfpathlineto{\pgfqpoint{3.518384in}{2.553370in}}%
\pgfpathlineto{\pgfqpoint{3.518913in}{2.782500in}}%
\pgfpathlineto{\pgfqpoint{3.519177in}{2.471857in}}%
\pgfpathlineto{\pgfqpoint{3.519552in}{2.657719in}}%
\pgfpathlineto{\pgfqpoint{3.520368in}{2.472404in}}%
\pgfpathlineto{\pgfqpoint{3.519839in}{2.802824in}}%
\pgfpathlineto{\pgfqpoint{3.520676in}{2.584620in}}%
\pgfpathlineto{\pgfqpoint{3.521602in}{2.795831in}}%
\pgfpathlineto{\pgfqpoint{3.521227in}{2.511739in}}%
\pgfpathlineto{\pgfqpoint{3.521756in}{2.604178in}}%
\pgfpathlineto{\pgfqpoint{3.522836in}{2.392421in}}%
\pgfpathlineto{\pgfqpoint{3.521999in}{2.759773in}}%
\pgfpathlineto{\pgfqpoint{3.522880in}{2.516875in}}%
\pgfpathlineto{\pgfqpoint{3.523520in}{2.402911in}}%
\pgfpathlineto{\pgfqpoint{3.523299in}{2.710385in}}%
\pgfpathlineto{\pgfqpoint{3.523696in}{2.667990in}}%
\pgfpathlineto{\pgfqpoint{3.524181in}{2.732784in}}%
\pgfpathlineto{\pgfqpoint{3.523938in}{2.433396in}}%
\pgfpathlineto{\pgfqpoint{3.524512in}{2.601119in}}%
\pgfpathlineto{\pgfqpoint{3.525217in}{2.387613in}}%
\pgfpathlineto{\pgfqpoint{3.524798in}{2.698147in}}%
\pgfpathlineto{\pgfqpoint{3.525636in}{2.480708in}}%
\pgfpathlineto{\pgfqpoint{3.526231in}{2.339427in}}%
\pgfpathlineto{\pgfqpoint{3.526517in}{2.668099in}}%
\pgfpathlineto{\pgfqpoint{3.526694in}{2.539821in}}%
\pgfpathlineto{\pgfqpoint{3.526914in}{2.671705in}}%
\pgfpathlineto{\pgfqpoint{3.527223in}{2.359204in}}%
\pgfpathlineto{\pgfqpoint{3.527796in}{2.517858in}}%
\pgfpathlineto{\pgfqpoint{3.528016in}{2.613248in}}%
\pgfpathlineto{\pgfqpoint{3.528523in}{2.379091in}}%
\pgfpathlineto{\pgfqpoint{3.528876in}{2.487373in}}%
\pgfpathlineto{\pgfqpoint{3.529912in}{2.400944in}}%
\pgfpathlineto{\pgfqpoint{3.529471in}{2.660997in}}%
\pgfpathlineto{\pgfqpoint{3.530000in}{2.445306in}}%
\pgfpathlineto{\pgfqpoint{3.530441in}{2.704703in}}%
\pgfpathlineto{\pgfqpoint{3.530948in}{2.255074in}}%
\pgfpathlineto{\pgfqpoint{3.531124in}{2.529441in}}%
\pgfpathlineto{\pgfqpoint{3.532050in}{2.437111in}}%
\pgfpathlineto{\pgfqpoint{3.531741in}{2.663838in}}%
\pgfpathlineto{\pgfqpoint{3.532226in}{2.527583in}}%
\pgfpathlineto{\pgfqpoint{3.532469in}{2.730490in}}%
\pgfpathlineto{\pgfqpoint{3.532623in}{2.428042in}}%
\pgfpathlineto{\pgfqpoint{3.533350in}{2.614777in}}%
\pgfpathlineto{\pgfqpoint{3.533879in}{2.368820in}}%
\pgfpathlineto{\pgfqpoint{3.534188in}{2.785341in}}%
\pgfpathlineto{\pgfqpoint{3.534452in}{2.552277in}}%
\pgfpathlineto{\pgfqpoint{3.535488in}{2.811347in}}%
\pgfpathlineto{\pgfqpoint{3.534651in}{2.468798in}}%
\pgfpathlineto{\pgfqpoint{3.535577in}{2.688750in}}%
\pgfpathlineto{\pgfqpoint{3.536260in}{2.511412in}}%
\pgfpathlineto{\pgfqpoint{3.536084in}{2.781189in}}%
\pgfpathlineto{\pgfqpoint{3.536679in}{2.694104in}}%
\pgfpathlineto{\pgfqpoint{3.536723in}{2.816919in}}%
\pgfpathlineto{\pgfqpoint{3.536877in}{2.452080in}}%
\pgfpathlineto{\pgfqpoint{3.537560in}{2.660887in}}%
\pgfpathlineto{\pgfqpoint{3.538574in}{2.397447in}}%
\pgfpathlineto{\pgfqpoint{3.537825in}{2.728960in}}%
\pgfpathlineto{\pgfqpoint{3.538662in}{2.499392in}}%
\pgfpathlineto{\pgfqpoint{3.538795in}{2.710494in}}%
\pgfpathlineto{\pgfqpoint{3.539610in}{2.408155in}}%
\pgfpathlineto{\pgfqpoint{3.539765in}{2.538400in}}%
\pgfpathlineto{\pgfqpoint{3.540448in}{2.428916in}}%
\pgfpathlineto{\pgfqpoint{3.540250in}{2.703829in}}%
\pgfpathlineto{\pgfqpoint{3.540845in}{2.600573in}}%
\pgfpathlineto{\pgfqpoint{3.541484in}{2.383352in}}%
\pgfpathlineto{\pgfqpoint{3.541704in}{2.733221in}}%
\pgfpathlineto{\pgfqpoint{3.541859in}{2.573912in}}%
\pgfpathlineto{\pgfqpoint{3.541881in}{2.768405in}}%
\pgfpathlineto{\pgfqpoint{3.542035in}{2.334619in}}%
\pgfpathlineto{\pgfqpoint{3.542961in}{2.760429in}}%
\pgfpathlineto{\pgfqpoint{3.543997in}{2.440170in}}%
\pgfpathlineto{\pgfqpoint{3.544107in}{2.583855in}}%
\pgfpathlineto{\pgfqpoint{3.544129in}{2.586914in}}%
\pgfpathlineto{\pgfqpoint{3.544151in}{2.456123in}}%
\pgfpathlineto{\pgfqpoint{3.544173in}{2.415695in}}%
\pgfpathlineto{\pgfqpoint{3.544592in}{2.706779in}}%
\pgfpathlineto{\pgfqpoint{3.545143in}{2.614449in}}%
\pgfpathlineto{\pgfqpoint{3.545209in}{2.721202in}}%
\pgfpathlineto{\pgfqpoint{3.546091in}{2.467487in}}%
\pgfpathlineto{\pgfqpoint{3.546223in}{2.546595in}}%
\pgfpathlineto{\pgfqpoint{3.546686in}{2.740542in}}%
\pgfpathlineto{\pgfqpoint{3.547127in}{2.464209in}}%
\pgfpathlineto{\pgfqpoint{3.547281in}{2.589318in}}%
\pgfpathlineto{\pgfqpoint{3.547567in}{2.372535in}}%
\pgfpathlineto{\pgfqpoint{3.547810in}{2.685909in}}%
\pgfpathlineto{\pgfqpoint{3.548361in}{2.664930in}}%
\pgfpathlineto{\pgfqpoint{3.549221in}{2.453938in}}%
\pgfpathlineto{\pgfqpoint{3.548912in}{2.769826in}}%
\pgfpathlineto{\pgfqpoint{3.549485in}{2.569869in}}%
\pgfpathlineto{\pgfqpoint{3.549992in}{2.692575in}}%
\pgfpathlineto{\pgfqpoint{3.549573in}{2.463116in}}%
\pgfpathlineto{\pgfqpoint{3.550565in}{2.614122in}}%
\pgfpathlineto{\pgfqpoint{3.551138in}{2.386849in}}%
\pgfpathlineto{\pgfqpoint{3.550852in}{2.654113in}}%
\pgfpathlineto{\pgfqpoint{3.551689in}{2.511084in}}%
\pgfpathlineto{\pgfqpoint{3.552439in}{2.703501in}}%
\pgfpathlineto{\pgfqpoint{3.552306in}{2.383571in}}%
\pgfpathlineto{\pgfqpoint{3.552813in}{2.584948in}}%
\pgfpathlineto{\pgfqpoint{3.552924in}{2.404768in}}%
\pgfpathlineto{\pgfqpoint{3.553122in}{2.714100in}}%
\pgfpathlineto{\pgfqpoint{3.553871in}{2.613138in}}%
\pgfpathlineto{\pgfqpoint{3.554224in}{2.680118in}}%
\pgfpathlineto{\pgfqpoint{3.553938in}{2.391110in}}%
\pgfpathlineto{\pgfqpoint{3.554952in}{2.636630in}}%
\pgfpathlineto{\pgfqpoint{3.556054in}{2.466394in}}%
\pgfpathlineto{\pgfqpoint{3.555106in}{2.702846in}}%
\pgfpathlineto{\pgfqpoint{3.556076in}{2.574458in}}%
\pgfpathlineto{\pgfqpoint{3.556164in}{2.434161in}}%
\pgfpathlineto{\pgfqpoint{3.556693in}{2.773978in}}%
\pgfpathlineto{\pgfqpoint{3.556759in}{2.686674in}}%
\pgfpathlineto{\pgfqpoint{3.556781in}{2.891329in}}%
\pgfpathlineto{\pgfqpoint{3.557332in}{2.455468in}}%
\pgfpathlineto{\pgfqpoint{3.557839in}{2.599699in}}%
\pgfpathlineto{\pgfqpoint{3.557905in}{2.435909in}}%
\pgfpathlineto{\pgfqpoint{3.558699in}{2.730162in}}%
\pgfpathlineto{\pgfqpoint{3.558853in}{2.560254in}}%
\pgfpathlineto{\pgfqpoint{3.559823in}{2.704703in}}%
\pgfpathlineto{\pgfqpoint{3.559514in}{2.396573in}}%
\pgfpathlineto{\pgfqpoint{3.559955in}{2.528894in}}%
\pgfpathlineto{\pgfqpoint{3.560793in}{2.346857in}}%
\pgfpathlineto{\pgfqpoint{3.560109in}{2.630293in}}%
\pgfpathlineto{\pgfqpoint{3.561079in}{2.433942in}}%
\pgfpathlineto{\pgfqpoint{3.562049in}{2.683068in}}%
\pgfpathlineto{\pgfqpoint{3.562071in}{2.423890in}}%
\pgfpathlineto{\pgfqpoint{3.562181in}{2.467596in}}%
\pgfpathlineto{\pgfqpoint{3.562931in}{2.360625in}}%
\pgfpathlineto{\pgfqpoint{3.562798in}{2.628435in}}%
\pgfpathlineto{\pgfqpoint{3.563217in}{2.599371in}}%
\pgfpathlineto{\pgfqpoint{3.563592in}{2.682631in}}%
\pgfpathlineto{\pgfqpoint{3.563945in}{2.420830in}}%
\pgfpathlineto{\pgfqpoint{3.564143in}{2.616198in}}%
\pgfpathlineto{\pgfqpoint{3.564826in}{2.364777in}}%
\pgfpathlineto{\pgfqpoint{3.564694in}{2.703938in}}%
\pgfpathlineto{\pgfqpoint{3.565245in}{2.514362in}}%
\pgfpathlineto{\pgfqpoint{3.565488in}{2.697273in}}%
\pgfpathlineto{\pgfqpoint{3.565355in}{2.345655in}}%
\pgfpathlineto{\pgfqpoint{3.566369in}{2.589755in}}%
\pgfpathlineto{\pgfqpoint{3.566568in}{2.417771in}}%
\pgfpathlineto{\pgfqpoint{3.566766in}{2.682413in}}%
\pgfpathlineto{\pgfqpoint{3.567515in}{2.460822in}}%
\pgfpathlineto{\pgfqpoint{3.567626in}{2.663073in}}%
\pgfpathlineto{\pgfqpoint{3.567824in}{2.390345in}}%
\pgfpathlineto{\pgfqpoint{3.568618in}{2.455686in}}%
\pgfpathlineto{\pgfqpoint{3.569125in}{2.768951in}}%
\pgfpathlineto{\pgfqpoint{3.569742in}{2.569978in}}%
\pgfpathlineto{\pgfqpoint{3.570381in}{2.411215in}}%
\pgfpathlineto{\pgfqpoint{3.570535in}{2.728632in}}%
\pgfpathlineto{\pgfqpoint{3.570800in}{2.635319in}}%
\pgfpathlineto{\pgfqpoint{3.571086in}{2.481036in}}%
\pgfpathlineto{\pgfqpoint{3.571196in}{2.730053in}}%
\pgfpathlineto{\pgfqpoint{3.571703in}{2.563532in}}%
\pgfpathlineto{\pgfqpoint{3.572541in}{2.741526in}}%
\pgfpathlineto{\pgfqpoint{3.572321in}{2.439405in}}%
\pgfpathlineto{\pgfqpoint{3.572806in}{2.693012in}}%
\pgfpathlineto{\pgfqpoint{3.573114in}{2.427495in}}%
\pgfpathlineto{\pgfqpoint{3.572960in}{2.721311in}}%
\pgfpathlineto{\pgfqpoint{3.573930in}{2.584620in}}%
\pgfpathlineto{\pgfqpoint{3.574106in}{2.381494in}}%
\pgfpathlineto{\pgfqpoint{3.574569in}{2.708418in}}%
\pgfpathlineto{\pgfqpoint{3.575010in}{2.545284in}}%
\pgfpathlineto{\pgfqpoint{3.575274in}{2.722404in}}%
\pgfpathlineto{\pgfqpoint{3.575781in}{2.445852in}}%
\pgfpathlineto{\pgfqpoint{3.576134in}{2.624939in}}%
\pgfpathlineto{\pgfqpoint{3.576266in}{2.384226in}}%
\pgfpathlineto{\pgfqpoint{3.576641in}{2.707107in}}%
\pgfpathlineto{\pgfqpoint{3.577324in}{2.457325in}}%
\pgfpathlineto{\pgfqpoint{3.578316in}{2.649961in}}%
\pgfpathlineto{\pgfqpoint{3.577919in}{2.389799in}}%
\pgfpathlineto{\pgfqpoint{3.578448in}{2.529768in}}%
\pgfpathlineto{\pgfqpoint{3.579065in}{2.407500in}}%
\pgfpathlineto{\pgfqpoint{3.578647in}{2.701753in}}%
\pgfpathlineto{\pgfqpoint{3.579462in}{2.532828in}}%
\pgfpathlineto{\pgfqpoint{3.579528in}{2.674109in}}%
\pgfpathlineto{\pgfqpoint{3.579925in}{2.365214in}}%
\pgfpathlineto{\pgfqpoint{3.580564in}{2.541351in}}%
\pgfpathlineto{\pgfqpoint{3.581446in}{2.694541in}}%
\pgfpathlineto{\pgfqpoint{3.581181in}{2.380402in}}%
\pgfpathlineto{\pgfqpoint{3.581490in}{2.584729in}}%
\pgfpathlineto{\pgfqpoint{3.582151in}{2.337570in}}%
\pgfpathlineto{\pgfqpoint{3.582438in}{2.628326in}}%
\pgfpathlineto{\pgfqpoint{3.582592in}{2.487155in}}%
\pgfpathlineto{\pgfqpoint{3.582724in}{2.695525in}}%
\pgfpathlineto{\pgfqpoint{3.583518in}{2.328501in}}%
\pgfpathlineto{\pgfqpoint{3.583650in}{2.364668in}}%
\pgfpathlineto{\pgfqpoint{3.583716in}{2.307959in}}%
\pgfpathlineto{\pgfqpoint{3.583760in}{2.512286in}}%
\pgfpathlineto{\pgfqpoint{3.583782in}{2.404768in}}%
\pgfpathlineto{\pgfqpoint{3.584863in}{2.643951in}}%
\pgfpathlineto{\pgfqpoint{3.584047in}{2.343251in}}%
\pgfpathlineto{\pgfqpoint{3.584907in}{2.495350in}}%
\pgfpathlineto{\pgfqpoint{3.585590in}{2.759118in}}%
\pgfpathlineto{\pgfqpoint{3.585854in}{2.447710in}}%
\pgfpathlineto{\pgfqpoint{3.586053in}{2.671486in}}%
\pgfpathlineto{\pgfqpoint{3.586934in}{2.375813in}}%
\pgfpathlineto{\pgfqpoint{3.586450in}{2.748082in}}%
\pgfpathlineto{\pgfqpoint{3.587177in}{2.590192in}}%
\pgfpathlineto{\pgfqpoint{3.587508in}{2.735735in}}%
\pgfpathlineto{\pgfqpoint{3.587728in}{2.414056in}}%
\pgfpathlineto{\pgfqpoint{3.588191in}{2.508899in}}%
\pgfpathlineto{\pgfqpoint{3.588279in}{2.435363in}}%
\pgfpathlineto{\pgfqpoint{3.589205in}{2.706779in}}%
\pgfpathlineto{\pgfqpoint{3.589227in}{2.542225in}}%
\pgfpathlineto{\pgfqpoint{3.589271in}{2.769061in}}%
\pgfpathlineto{\pgfqpoint{3.590109in}{2.372425in}}%
\pgfpathlineto{\pgfqpoint{3.590329in}{2.556538in}}%
\pgfpathlineto{\pgfqpoint{3.590351in}{2.420939in}}%
\pgfpathlineto{\pgfqpoint{3.590417in}{2.706561in}}%
\pgfpathlineto{\pgfqpoint{3.591453in}{2.470874in}}%
\pgfpathlineto{\pgfqpoint{3.591828in}{2.697601in}}%
\pgfpathlineto{\pgfqpoint{3.591806in}{2.405861in}}%
\pgfpathlineto{\pgfqpoint{3.592533in}{2.496005in}}%
\pgfpathlineto{\pgfqpoint{3.593062in}{2.330577in}}%
\pgfpathlineto{\pgfqpoint{3.593437in}{2.612701in}}%
\pgfpathlineto{\pgfqpoint{3.593635in}{2.469672in}}%
\pgfpathlineto{\pgfqpoint{3.593812in}{2.647229in}}%
\pgfpathlineto{\pgfqpoint{3.594296in}{2.393951in}}%
\pgfpathlineto{\pgfqpoint{3.594759in}{2.592268in}}%
\pgfpathlineto{\pgfqpoint{3.595112in}{2.450878in}}%
\pgfpathlineto{\pgfqpoint{3.594980in}{2.654441in}}%
\pgfpathlineto{\pgfqpoint{3.595839in}{2.621880in}}%
\pgfpathlineto{\pgfqpoint{3.595861in}{2.747972in}}%
\pgfpathlineto{\pgfqpoint{3.596853in}{2.392749in}}%
\pgfpathlineto{\pgfqpoint{3.596919in}{2.474261in}}%
\pgfpathlineto{\pgfqpoint{3.597096in}{2.678370in}}%
\pgfpathlineto{\pgfqpoint{3.597360in}{2.440170in}}%
\pgfpathlineto{\pgfqpoint{3.598044in}{2.580468in}}%
\pgfpathlineto{\pgfqpoint{3.598088in}{2.437329in}}%
\pgfpathlineto{\pgfqpoint{3.598683in}{2.811019in}}%
\pgfpathlineto{\pgfqpoint{3.599146in}{2.603632in}}%
\pgfpathlineto{\pgfqpoint{3.599234in}{2.392858in}}%
\pgfpathlineto{\pgfqpoint{3.599961in}{2.631276in}}%
\pgfpathlineto{\pgfqpoint{3.600226in}{2.583309in}}%
\pgfpathlineto{\pgfqpoint{3.600975in}{2.692684in}}%
\pgfpathlineto{\pgfqpoint{3.600578in}{2.381167in}}%
\pgfpathlineto{\pgfqpoint{3.601306in}{2.554244in}}%
\pgfpathlineto{\pgfqpoint{3.601636in}{2.396355in}}%
\pgfpathlineto{\pgfqpoint{3.601879in}{2.681648in}}%
\pgfpathlineto{\pgfqpoint{3.602408in}{2.534030in}}%
\pgfpathlineto{\pgfqpoint{3.603135in}{2.714428in}}%
\pgfpathlineto{\pgfqpoint{3.602937in}{2.442356in}}%
\pgfpathlineto{\pgfqpoint{3.603510in}{2.589537in}}%
\pgfpathlineto{\pgfqpoint{3.603973in}{2.360515in}}%
\pgfpathlineto{\pgfqpoint{3.604237in}{2.636849in}}%
\pgfpathlineto{\pgfqpoint{3.604678in}{2.430992in}}%
\pgfpathlineto{\pgfqpoint{3.604899in}{2.685472in}}%
\pgfpathlineto{\pgfqpoint{3.605670in}{2.419628in}}%
\pgfpathlineto{\pgfqpoint{3.605824in}{2.598060in}}%
\pgfpathlineto{\pgfqpoint{3.605869in}{2.362264in}}%
\pgfpathlineto{\pgfqpoint{3.605935in}{2.699677in}}%
\pgfpathlineto{\pgfqpoint{3.606927in}{2.507150in}}%
\pgfpathlineto{\pgfqpoint{3.607566in}{2.674436in}}%
\pgfpathlineto{\pgfqpoint{3.607213in}{2.426512in}}%
\pgfpathlineto{\pgfqpoint{3.608051in}{2.519279in}}%
\pgfpathlineto{\pgfqpoint{3.609021in}{2.361936in}}%
\pgfpathlineto{\pgfqpoint{3.608866in}{2.618055in}}%
\pgfpathlineto{\pgfqpoint{3.609153in}{2.509008in}}%
\pgfpathlineto{\pgfqpoint{3.610012in}{2.679135in}}%
\pgfpathlineto{\pgfqpoint{3.609285in}{2.414602in}}%
\pgfpathlineto{\pgfqpoint{3.610255in}{2.450878in}}%
\pgfpathlineto{\pgfqpoint{3.611115in}{2.693558in}}%
\pgfpathlineto{\pgfqpoint{3.610674in}{2.376796in}}%
\pgfpathlineto{\pgfqpoint{3.611379in}{2.518405in}}%
\pgfpathlineto{\pgfqpoint{3.612173in}{2.823912in}}%
\pgfpathlineto{\pgfqpoint{3.611798in}{2.462133in}}%
\pgfpathlineto{\pgfqpoint{3.612481in}{2.518186in}}%
\pgfpathlineto{\pgfqpoint{3.613363in}{2.730599in}}%
\pgfpathlineto{\pgfqpoint{3.612569in}{2.458636in}}%
\pgfpathlineto{\pgfqpoint{3.613583in}{2.558942in}}%
\pgfpathlineto{\pgfqpoint{3.614597in}{2.436127in}}%
\pgfpathlineto{\pgfqpoint{3.614267in}{2.710166in}}%
\pgfpathlineto{\pgfqpoint{3.614707in}{2.473387in}}%
\pgfpathlineto{\pgfqpoint{3.615765in}{2.755075in}}%
\pgfpathlineto{\pgfqpoint{3.615479in}{2.400835in}}%
\pgfpathlineto{\pgfqpoint{3.615832in}{2.634664in}}%
\pgfpathlineto{\pgfqpoint{3.616118in}{2.483003in}}%
\pgfpathlineto{\pgfqpoint{3.616867in}{2.796705in}}%
\pgfpathlineto{\pgfqpoint{3.616890in}{2.705249in}}%
\pgfpathlineto{\pgfqpoint{3.616912in}{2.807959in}}%
\pgfpathlineto{\pgfqpoint{3.617066in}{2.499611in}}%
\pgfpathlineto{\pgfqpoint{3.617970in}{2.597185in}}%
\pgfpathlineto{\pgfqpoint{3.618653in}{2.772994in}}%
\pgfpathlineto{\pgfqpoint{3.618366in}{2.485625in}}%
\pgfpathlineto{\pgfqpoint{3.619094in}{2.621770in}}%
\pgfpathlineto{\pgfqpoint{3.619468in}{2.789712in}}%
\pgfpathlineto{\pgfqpoint{3.619380in}{2.496879in}}%
\pgfpathlineto{\pgfqpoint{3.619843in}{2.696617in}}%
\pgfpathlineto{\pgfqpoint{3.620879in}{2.462788in}}%
\pgfpathlineto{\pgfqpoint{3.620945in}{2.629310in}}%
\pgfpathlineto{\pgfqpoint{3.621210in}{2.805883in}}%
\pgfpathlineto{\pgfqpoint{3.621078in}{2.429681in}}%
\pgfpathlineto{\pgfqpoint{3.622025in}{2.591394in}}%
\pgfpathlineto{\pgfqpoint{3.622620in}{2.437111in}}%
\pgfpathlineto{\pgfqpoint{3.622819in}{2.772120in}}%
\pgfpathlineto{\pgfqpoint{3.623171in}{2.491635in}}%
\pgfpathlineto{\pgfqpoint{3.623745in}{2.779004in}}%
\pgfpathlineto{\pgfqpoint{3.623458in}{2.414711in}}%
\pgfpathlineto{\pgfqpoint{3.624296in}{2.537963in}}%
\pgfpathlineto{\pgfqpoint{3.624803in}{2.425419in}}%
\pgfpathlineto{\pgfqpoint{3.625155in}{2.691591in}}%
\pgfpathlineto{\pgfqpoint{3.625398in}{2.451534in}}%
\pgfpathlineto{\pgfqpoint{3.625530in}{2.609970in}}%
\pgfpathlineto{\pgfqpoint{3.625728in}{2.315170in}}%
\pgfpathlineto{\pgfqpoint{3.626500in}{2.419191in}}%
\pgfpathlineto{\pgfqpoint{3.626720in}{2.689515in}}%
\pgfpathlineto{\pgfqpoint{3.627404in}{2.337570in}}%
\pgfpathlineto{\pgfqpoint{3.627602in}{2.505621in}}%
\pgfpathlineto{\pgfqpoint{3.627646in}{2.402255in}}%
\pgfpathlineto{\pgfqpoint{3.628175in}{2.737374in}}%
\pgfpathlineto{\pgfqpoint{3.628682in}{2.594891in}}%
\pgfpathlineto{\pgfqpoint{3.629608in}{2.702190in}}%
\pgfpathlineto{\pgfqpoint{3.628726in}{2.356473in}}%
\pgfpathlineto{\pgfqpoint{3.629674in}{2.554899in}}%
\pgfpathlineto{\pgfqpoint{3.629916in}{2.749393in}}%
\pgfpathlineto{\pgfqpoint{3.630798in}{2.387286in}}%
\pgfpathlineto{\pgfqpoint{3.631746in}{2.738466in}}%
\pgfpathlineto{\pgfqpoint{3.631922in}{2.570306in}}%
\pgfpathlineto{\pgfqpoint{3.632848in}{2.457871in}}%
\pgfpathlineto{\pgfqpoint{3.632319in}{2.679026in}}%
\pgfpathlineto{\pgfqpoint{3.633024in}{2.559052in}}%
\pgfpathlineto{\pgfqpoint{3.633090in}{2.511302in}}%
\pgfpathlineto{\pgfqpoint{3.633068in}{2.633680in}}%
\pgfpathlineto{\pgfqpoint{3.633134in}{2.583636in}}%
\pgfpathlineto{\pgfqpoint{3.633157in}{2.717378in}}%
\pgfpathlineto{\pgfqpoint{3.634016in}{2.448474in}}%
\pgfpathlineto{\pgfqpoint{3.634215in}{2.526600in}}%
\pgfpathlineto{\pgfqpoint{3.634303in}{2.426075in}}%
\pgfpathlineto{\pgfqpoint{3.634655in}{2.695853in}}%
\pgfpathlineto{\pgfqpoint{3.635118in}{2.660778in}}%
\pgfpathlineto{\pgfqpoint{3.635140in}{2.695197in}}%
\pgfpathlineto{\pgfqpoint{3.635868in}{2.388269in}}%
\pgfpathlineto{\pgfqpoint{3.636132in}{2.544519in}}%
\pgfpathlineto{\pgfqpoint{3.636286in}{2.408374in}}%
\pgfpathlineto{\pgfqpoint{3.636882in}{2.677168in}}%
\pgfpathlineto{\pgfqpoint{3.637278in}{2.470328in}}%
\pgfpathlineto{\pgfqpoint{3.637587in}{2.414274in}}%
\pgfpathlineto{\pgfqpoint{3.638447in}{2.751687in}}%
\pgfpathlineto{\pgfqpoint{3.639240in}{2.299108in}}%
\pgfpathlineto{\pgfqpoint{3.639615in}{2.468470in}}%
\pgfpathlineto{\pgfqpoint{3.640254in}{2.641875in}}%
\pgfpathlineto{\pgfqpoint{3.640673in}{2.411324in}}%
\pgfpathlineto{\pgfqpoint{3.640717in}{2.441591in}}%
\pgfpathlineto{\pgfqpoint{3.640739in}{2.398977in}}%
\pgfpathlineto{\pgfqpoint{3.641246in}{2.668099in}}%
\pgfpathlineto{\pgfqpoint{3.641775in}{2.542115in}}%
\pgfpathlineto{\pgfqpoint{3.642348in}{2.751906in}}%
\pgfpathlineto{\pgfqpoint{3.642084in}{2.416678in}}%
\pgfpathlineto{\pgfqpoint{3.642921in}{2.725027in}}%
\pgfpathlineto{\pgfqpoint{3.643891in}{2.268186in}}%
\pgfpathlineto{\pgfqpoint{3.644067in}{2.443448in}}%
\pgfpathlineto{\pgfqpoint{3.644927in}{2.742291in}}%
\pgfpathlineto{\pgfqpoint{3.644508in}{2.364886in}}%
\pgfpathlineto{\pgfqpoint{3.645213in}{2.604069in}}%
\pgfpathlineto{\pgfqpoint{3.645280in}{2.493055in}}%
\pgfpathlineto{\pgfqpoint{3.645412in}{2.778348in}}%
\pgfpathlineto{\pgfqpoint{3.646271in}{2.549108in}}%
\pgfpathlineto{\pgfqpoint{3.646712in}{2.784686in}}%
\pgfpathlineto{\pgfqpoint{3.647374in}{2.700551in}}%
\pgfpathlineto{\pgfqpoint{3.647947in}{2.446726in}}%
\pgfpathlineto{\pgfqpoint{3.648101in}{2.804791in}}%
\pgfpathlineto{\pgfqpoint{3.648498in}{2.630074in}}%
\pgfpathlineto{\pgfqpoint{3.649468in}{2.866744in}}%
\pgfpathlineto{\pgfqpoint{3.648652in}{2.562111in}}%
\pgfpathlineto{\pgfqpoint{3.649622in}{2.682850in}}%
\pgfpathlineto{\pgfqpoint{3.650151in}{2.484969in}}%
\pgfpathlineto{\pgfqpoint{3.650239in}{2.775398in}}%
\pgfpathlineto{\pgfqpoint{3.650702in}{2.659248in}}%
\pgfpathlineto{\pgfqpoint{3.650724in}{2.781517in}}%
\pgfpathlineto{\pgfqpoint{3.651694in}{2.527583in}}%
\pgfpathlineto{\pgfqpoint{3.651782in}{2.655315in}}%
\pgfpathlineto{\pgfqpoint{3.652576in}{2.470655in}}%
\pgfpathlineto{\pgfqpoint{3.652135in}{2.759336in}}%
\pgfpathlineto{\pgfqpoint{3.652906in}{2.554899in}}%
\pgfpathlineto{\pgfqpoint{3.653589in}{2.407391in}}%
\pgfpathlineto{\pgfqpoint{3.653788in}{2.711587in}}%
\pgfpathlineto{\pgfqpoint{3.653898in}{2.591285in}}%
\pgfpathlineto{\pgfqpoint{3.654449in}{2.737374in}}%
\pgfpathlineto{\pgfqpoint{3.654118in}{2.467924in}}%
\pgfpathlineto{\pgfqpoint{3.655000in}{2.624174in}}%
\pgfpathlineto{\pgfqpoint{3.655970in}{2.497863in}}%
\pgfpathlineto{\pgfqpoint{3.655683in}{2.739231in}}%
\pgfpathlineto{\pgfqpoint{3.656102in}{2.629200in}}%
\pgfpathlineto{\pgfqpoint{3.656477in}{2.444869in}}%
\pgfpathlineto{\pgfqpoint{3.656146in}{2.722732in}}%
\pgfpathlineto{\pgfqpoint{3.657138in}{2.636303in}}%
\pgfpathlineto{\pgfqpoint{3.658218in}{2.738903in}}%
\pgfpathlineto{\pgfqpoint{3.657557in}{2.449458in}}%
\pgfpathlineto{\pgfqpoint{3.658262in}{2.703501in}}%
\pgfpathlineto{\pgfqpoint{3.659100in}{2.417771in}}%
\pgfpathlineto{\pgfqpoint{3.658461in}{2.743383in}}%
\pgfpathlineto{\pgfqpoint{3.659409in}{2.454593in}}%
\pgfpathlineto{\pgfqpoint{3.659871in}{2.710822in}}%
\pgfpathlineto{\pgfqpoint{3.660533in}{2.654441in}}%
\pgfpathlineto{\pgfqpoint{3.661106in}{2.464974in}}%
\pgfpathlineto{\pgfqpoint{3.661128in}{2.727103in}}%
\pgfpathlineto{\pgfqpoint{3.661613in}{2.659248in}}%
\pgfpathlineto{\pgfqpoint{3.661965in}{2.859861in}}%
\pgfpathlineto{\pgfqpoint{3.662142in}{2.454921in}}%
\pgfpathlineto{\pgfqpoint{3.662715in}{2.651709in}}%
\pgfpathlineto{\pgfqpoint{3.662913in}{2.485625in}}%
\pgfpathlineto{\pgfqpoint{3.663156in}{2.891001in}}%
\pgfpathlineto{\pgfqpoint{3.663641in}{2.752999in}}%
\pgfpathlineto{\pgfqpoint{3.663663in}{2.853523in}}%
\pgfpathlineto{\pgfqpoint{3.664236in}{2.566045in}}%
\pgfpathlineto{\pgfqpoint{3.664721in}{2.749939in}}%
\pgfpathlineto{\pgfqpoint{3.665007in}{2.608221in}}%
\pgfpathlineto{\pgfqpoint{3.665779in}{2.885320in}}%
\pgfpathlineto{\pgfqpoint{3.665823in}{2.690826in}}%
\pgfpathlineto{\pgfqpoint{3.665889in}{2.875704in}}%
\pgfpathlineto{\pgfqpoint{3.666462in}{2.587789in}}%
\pgfpathlineto{\pgfqpoint{3.666925in}{2.666460in}}%
\pgfpathlineto{\pgfqpoint{3.666991in}{2.856364in}}%
\pgfpathlineto{\pgfqpoint{3.667784in}{2.504856in}}%
\pgfpathlineto{\pgfqpoint{3.667917in}{2.623300in}}%
\pgfpathlineto{\pgfqpoint{3.668137in}{2.491853in}}%
\pgfpathlineto{\pgfqpoint{3.668820in}{2.713444in}}%
\pgfpathlineto{\pgfqpoint{3.668997in}{2.642749in}}%
\pgfpathlineto{\pgfqpoint{3.669261in}{2.749284in}}%
\pgfpathlineto{\pgfqpoint{3.669724in}{2.496879in}}%
\pgfpathlineto{\pgfqpoint{3.670099in}{2.628654in}}%
\pgfpathlineto{\pgfqpoint{3.670826in}{2.513160in}}%
\pgfpathlineto{\pgfqpoint{3.670297in}{2.756714in}}%
\pgfpathlineto{\pgfqpoint{3.671069in}{2.605490in}}%
\pgfpathlineto{\pgfqpoint{3.671906in}{2.784249in}}%
\pgfpathlineto{\pgfqpoint{3.672061in}{2.449458in}}%
\pgfpathlineto{\pgfqpoint{3.672171in}{2.736062in}}%
\pgfpathlineto{\pgfqpoint{3.673097in}{2.408155in}}%
\pgfpathlineto{\pgfqpoint{3.672325in}{2.757588in}}%
\pgfpathlineto{\pgfqpoint{3.673295in}{2.634773in}}%
\pgfpathlineto{\pgfqpoint{3.673824in}{2.528239in}}%
\pgfpathlineto{\pgfqpoint{3.674044in}{2.740652in}}%
\pgfpathlineto{\pgfqpoint{3.674375in}{2.588226in}}%
\pgfpathlineto{\pgfqpoint{3.675169in}{2.827518in}}%
\pgfpathlineto{\pgfqpoint{3.675411in}{2.522448in}}%
\pgfpathlineto{\pgfqpoint{3.675477in}{2.626250in}}%
\pgfpathlineto{\pgfqpoint{3.676359in}{2.781845in}}%
\pgfpathlineto{\pgfqpoint{3.676006in}{2.541023in}}%
\pgfpathlineto{\pgfqpoint{3.676381in}{2.622972in}}%
\pgfpathlineto{\pgfqpoint{3.676844in}{2.561237in}}%
\pgfpathlineto{\pgfqpoint{3.677218in}{2.802605in}}%
\pgfpathlineto{\pgfqpoint{3.677461in}{2.673562in}}%
\pgfpathlineto{\pgfqpoint{3.677681in}{2.778021in}}%
\pgfpathlineto{\pgfqpoint{3.678166in}{2.455140in}}%
\pgfpathlineto{\pgfqpoint{3.678497in}{2.702081in}}%
\pgfpathlineto{\pgfqpoint{3.678717in}{2.460275in}}%
\pgfpathlineto{\pgfqpoint{3.678938in}{2.741853in}}%
\pgfpathlineto{\pgfqpoint{3.679599in}{2.626250in}}%
\pgfpathlineto{\pgfqpoint{3.680172in}{2.475682in}}%
\pgfpathlineto{\pgfqpoint{3.680745in}{2.753982in}}%
\pgfpathlineto{\pgfqpoint{3.681583in}{2.409794in}}%
\pgfpathlineto{\pgfqpoint{3.681957in}{2.566372in}}%
\pgfpathlineto{\pgfqpoint{3.682156in}{2.718252in}}%
\pgfpathlineto{\pgfqpoint{3.682376in}{2.430227in}}%
\pgfpathlineto{\pgfqpoint{3.683038in}{2.574021in}}%
\pgfpathlineto{\pgfqpoint{3.683500in}{2.360078in}}%
\pgfpathlineto{\pgfqpoint{3.683941in}{2.667006in}}%
\pgfpathlineto{\pgfqpoint{3.684162in}{2.514253in}}%
\pgfpathlineto{\pgfqpoint{3.684404in}{2.669410in}}%
\pgfpathlineto{\pgfqpoint{3.684889in}{2.418208in}}%
\pgfpathlineto{\pgfqpoint{3.685308in}{2.629200in}}%
\pgfpathlineto{\pgfqpoint{3.686322in}{2.377124in}}%
\pgfpathlineto{\pgfqpoint{3.685859in}{2.666569in}}%
\pgfpathlineto{\pgfqpoint{3.686388in}{2.446398in}}%
\pgfpathlineto{\pgfqpoint{3.686697in}{2.733659in}}%
\pgfpathlineto{\pgfqpoint{3.687402in}{2.412307in}}%
\pgfpathlineto{\pgfqpoint{3.687490in}{2.493492in}}%
\pgfpathlineto{\pgfqpoint{3.687512in}{2.356691in}}%
\pgfpathlineto{\pgfqpoint{3.688129in}{2.731692in}}%
\pgfpathlineto{\pgfqpoint{3.688592in}{2.491853in}}%
\pgfpathlineto{\pgfqpoint{3.688791in}{2.737701in}}%
\pgfpathlineto{\pgfqpoint{3.689297in}{2.482675in}}%
\pgfpathlineto{\pgfqpoint{3.689893in}{2.585494in}}%
\pgfpathlineto{\pgfqpoint{3.690267in}{2.523103in}}%
\pgfpathlineto{\pgfqpoint{3.689959in}{2.729288in}}%
\pgfpathlineto{\pgfqpoint{3.690951in}{2.647229in}}%
\pgfpathlineto{\pgfqpoint{3.691061in}{2.759227in}}%
\pgfpathlineto{\pgfqpoint{3.691303in}{2.472185in}}%
\pgfpathlineto{\pgfqpoint{3.692075in}{2.729616in}}%
\pgfpathlineto{\pgfqpoint{3.692494in}{2.465083in}}%
\pgfpathlineto{\pgfqpoint{3.693221in}{2.565280in}}%
\pgfpathlineto{\pgfqpoint{3.693463in}{2.743383in}}%
\pgfpathlineto{\pgfqpoint{3.693287in}{2.409248in}}%
\pgfpathlineto{\pgfqpoint{3.694411in}{2.686565in}}%
\pgfpathlineto{\pgfqpoint{3.694654in}{2.845438in}}%
\pgfpathlineto{\pgfqpoint{3.695557in}{2.429025in}}%
\pgfpathlineto{\pgfqpoint{3.696351in}{2.734314in}}%
\pgfpathlineto{\pgfqpoint{3.695668in}{2.389252in}}%
\pgfpathlineto{\pgfqpoint{3.696704in}{2.671814in}}%
\pgfpathlineto{\pgfqpoint{3.697255in}{2.492509in}}%
\pgfpathlineto{\pgfqpoint{3.697188in}{2.726338in}}%
\pgfpathlineto{\pgfqpoint{3.697806in}{2.656408in}}%
\pgfpathlineto{\pgfqpoint{3.698599in}{2.757369in}}%
\pgfpathlineto{\pgfqpoint{3.698269in}{2.479725in}}%
\pgfpathlineto{\pgfqpoint{3.698908in}{2.703064in}}%
\pgfpathlineto{\pgfqpoint{3.699282in}{2.482893in}}%
\pgfpathlineto{\pgfqpoint{3.699216in}{2.817903in}}%
\pgfpathlineto{\pgfqpoint{3.700054in}{2.622972in}}%
\pgfpathlineto{\pgfqpoint{3.700142in}{2.773104in}}%
\pgfpathlineto{\pgfqpoint{3.700230in}{2.550420in}}%
\pgfpathlineto{\pgfqpoint{3.701112in}{2.606801in}}%
\pgfpathlineto{\pgfqpoint{3.701244in}{2.457762in}}%
\pgfpathlineto{\pgfqpoint{3.701531in}{2.759992in}}%
\pgfpathlineto{\pgfqpoint{3.702192in}{2.525507in}}%
\pgfpathlineto{\pgfqpoint{3.702589in}{2.504965in}}%
\pgfpathlineto{\pgfqpoint{3.703338in}{2.845656in}}%
\pgfpathlineto{\pgfqpoint{3.703713in}{2.539056in}}%
\pgfpathlineto{\pgfqpoint{3.704484in}{2.634445in}}%
\pgfpathlineto{\pgfqpoint{3.704595in}{2.831779in}}%
\pgfpathlineto{\pgfqpoint{3.704947in}{2.480380in}}%
\pgfpathlineto{\pgfqpoint{3.705586in}{2.588116in}}%
\pgfpathlineto{\pgfqpoint{3.706314in}{2.440170in}}%
\pgfpathlineto{\pgfqpoint{3.706138in}{2.746552in}}%
\pgfpathlineto{\pgfqpoint{3.706534in}{2.625595in}}%
\pgfpathlineto{\pgfqpoint{3.707262in}{2.766985in}}%
\pgfpathlineto{\pgfqpoint{3.706843in}{2.534467in}}%
\pgfpathlineto{\pgfqpoint{3.707658in}{2.728414in}}%
\pgfpathlineto{\pgfqpoint{3.708011in}{2.525616in}}%
\pgfpathlineto{\pgfqpoint{3.707725in}{2.832544in}}%
\pgfpathlineto{\pgfqpoint{3.708716in}{2.690608in}}%
\pgfpathlineto{\pgfqpoint{3.708827in}{2.881058in}}%
\pgfpathlineto{\pgfqpoint{3.709268in}{2.514908in}}%
\pgfpathlineto{\pgfqpoint{3.709797in}{2.700332in}}%
\pgfpathlineto{\pgfqpoint{3.710458in}{2.502889in}}%
\pgfpathlineto{\pgfqpoint{3.710171in}{2.825988in}}%
\pgfpathlineto{\pgfqpoint{3.710899in}{2.541023in}}%
\pgfpathlineto{\pgfqpoint{3.711163in}{2.662854in}}%
\pgfpathlineto{\pgfqpoint{3.711428in}{2.255292in}}%
\pgfpathlineto{\pgfqpoint{3.711979in}{2.450878in}}%
\pgfpathlineto{\pgfqpoint{3.712001in}{2.424764in}}%
\pgfpathlineto{\pgfqpoint{3.712508in}{2.658484in}}%
\pgfpathlineto{\pgfqpoint{3.712926in}{2.634664in}}%
\pgfpathlineto{\pgfqpoint{3.713852in}{2.740870in}}%
\pgfpathlineto{\pgfqpoint{3.713169in}{2.371661in}}%
\pgfpathlineto{\pgfqpoint{3.713984in}{2.634554in}}%
\pgfpathlineto{\pgfqpoint{3.714271in}{2.411215in}}%
\pgfpathlineto{\pgfqpoint{3.714668in}{2.708200in}}%
\pgfpathlineto{\pgfqpoint{3.715087in}{2.617400in}}%
\pgfpathlineto{\pgfqpoint{3.715726in}{2.718471in}}%
\pgfpathlineto{\pgfqpoint{3.715858in}{2.463881in}}%
\pgfpathlineto{\pgfqpoint{3.716189in}{2.711478in}}%
\pgfpathlineto{\pgfqpoint{3.717026in}{2.479725in}}%
\pgfpathlineto{\pgfqpoint{3.716630in}{2.770809in}}%
\pgfpathlineto{\pgfqpoint{3.717335in}{2.549655in}}%
\pgfpathlineto{\pgfqpoint{3.717820in}{2.802933in}}%
\pgfpathlineto{\pgfqpoint{3.717930in}{2.504419in}}%
\pgfpathlineto{\pgfqpoint{3.718459in}{2.606364in}}%
\pgfpathlineto{\pgfqpoint{3.718481in}{2.605490in}}%
\pgfpathlineto{\pgfqpoint{3.718966in}{2.811565in}}%
\pgfpathlineto{\pgfqpoint{3.719164in}{2.456560in}}%
\pgfpathlineto{\pgfqpoint{3.719583in}{2.554353in}}%
\pgfpathlineto{\pgfqpoint{3.720090in}{2.777474in}}%
\pgfpathlineto{\pgfqpoint{3.720421in}{2.489886in}}%
\pgfpathlineto{\pgfqpoint{3.720685in}{2.559489in}}%
\pgfpathlineto{\pgfqpoint{3.721280in}{2.739668in}}%
\pgfpathlineto{\pgfqpoint{3.721589in}{2.482456in}}%
\pgfpathlineto{\pgfqpoint{3.721809in}{2.631167in}}%
\pgfpathlineto{\pgfqpoint{3.722515in}{2.415476in}}%
\pgfpathlineto{\pgfqpoint{3.722228in}{2.698475in}}%
\pgfpathlineto{\pgfqpoint{3.722912in}{2.572273in}}%
\pgfpathlineto{\pgfqpoint{3.722934in}{2.674436in}}%
\pgfpathlineto{\pgfqpoint{3.723705in}{2.457544in}}%
\pgfpathlineto{\pgfqpoint{3.724014in}{2.613029in}}%
\pgfpathlineto{\pgfqpoint{3.724454in}{2.532063in}}%
\pgfpathlineto{\pgfqpoint{3.724653in}{2.742618in}}%
\pgfpathlineto{\pgfqpoint{3.724873in}{2.702846in}}%
\pgfpathlineto{\pgfqpoint{3.724895in}{2.821508in}}%
\pgfpathlineto{\pgfqpoint{3.725799in}{2.498628in}}%
\pgfpathlineto{\pgfqpoint{3.725975in}{2.687439in}}%
\pgfpathlineto{\pgfqpoint{3.726835in}{2.774961in}}%
\pgfpathlineto{\pgfqpoint{3.726284in}{2.516547in}}%
\pgfpathlineto{\pgfqpoint{3.726989in}{2.666023in}}%
\pgfpathlineto{\pgfqpoint{3.727188in}{2.524087in}}%
\pgfpathlineto{\pgfqpoint{3.727099in}{2.824896in}}%
\pgfpathlineto{\pgfqpoint{3.728069in}{2.724808in}}%
\pgfpathlineto{\pgfqpoint{3.728113in}{2.850573in}}%
\pgfpathlineto{\pgfqpoint{3.728312in}{2.534904in}}%
\pgfpathlineto{\pgfqpoint{3.729061in}{2.694214in}}%
\pgfpathlineto{\pgfqpoint{3.729480in}{2.524742in}}%
\pgfpathlineto{\pgfqpoint{3.729811in}{2.832216in}}%
\pgfpathlineto{\pgfqpoint{3.730141in}{2.668973in}}%
\pgfpathlineto{\pgfqpoint{3.730979in}{2.824677in}}%
\pgfpathlineto{\pgfqpoint{3.730648in}{2.496661in}}%
\pgfpathlineto{\pgfqpoint{3.731221in}{2.762942in}}%
\pgfpathlineto{\pgfqpoint{3.731993in}{2.470000in}}%
\pgfpathlineto{\pgfqpoint{3.731816in}{2.858768in}}%
\pgfpathlineto{\pgfqpoint{3.732323in}{2.692575in}}%
\pgfpathlineto{\pgfqpoint{3.732456in}{2.857457in}}%
\pgfpathlineto{\pgfqpoint{3.732654in}{2.562111in}}%
\pgfpathlineto{\pgfqpoint{3.733448in}{2.839756in}}%
\pgfpathlineto{\pgfqpoint{3.734021in}{2.485734in}}%
\pgfpathlineto{\pgfqpoint{3.734594in}{2.692137in}}%
\pgfpathlineto{\pgfqpoint{3.734924in}{2.850573in}}%
\pgfpathlineto{\pgfqpoint{3.735013in}{2.579266in}}%
\pgfpathlineto{\pgfqpoint{3.735652in}{2.630293in}}%
\pgfpathlineto{\pgfqpoint{3.736071in}{2.463007in}}%
\pgfpathlineto{\pgfqpoint{3.736269in}{2.773322in}}%
\pgfpathlineto{\pgfqpoint{3.736732in}{2.765564in}}%
\pgfpathlineto{\pgfqpoint{3.737591in}{2.421814in}}%
\pgfpathlineto{\pgfqpoint{3.736952in}{2.782063in}}%
\pgfpathlineto{\pgfqpoint{3.737988in}{2.451315in}}%
\pgfpathlineto{\pgfqpoint{3.738804in}{2.733331in}}%
\pgfpathlineto{\pgfqpoint{3.738914in}{2.416678in}}%
\pgfpathlineto{\pgfqpoint{3.739112in}{2.623628in}}%
\pgfpathlineto{\pgfqpoint{3.739906in}{2.794847in}}%
\pgfpathlineto{\pgfqpoint{3.739796in}{2.396355in}}%
\pgfpathlineto{\pgfqpoint{3.740148in}{2.568776in}}%
\pgfpathlineto{\pgfqpoint{3.740259in}{2.461259in}}%
\pgfpathlineto{\pgfqpoint{3.740788in}{2.809161in}}%
\pgfpathlineto{\pgfqpoint{3.741228in}{2.664602in}}%
\pgfpathlineto{\pgfqpoint{3.741625in}{2.610953in}}%
\pgfpathlineto{\pgfqpoint{3.741956in}{2.832981in}}%
\pgfpathlineto{\pgfqpoint{3.742242in}{2.782391in}}%
\pgfpathlineto{\pgfqpoint{3.742264in}{2.783702in}}%
\pgfpathlineto{\pgfqpoint{3.742727in}{2.534795in}}%
\pgfpathlineto{\pgfqpoint{3.742904in}{2.848606in}}%
\pgfpathlineto{\pgfqpoint{3.743389in}{2.664930in}}%
\pgfpathlineto{\pgfqpoint{3.744336in}{2.868274in}}%
\pgfpathlineto{\pgfqpoint{3.743719in}{2.570525in}}%
\pgfpathlineto{\pgfqpoint{3.744491in}{2.715302in}}%
\pgfpathlineto{\pgfqpoint{3.745284in}{2.522557in}}%
\pgfpathlineto{\pgfqpoint{3.744601in}{2.810363in}}%
\pgfpathlineto{\pgfqpoint{3.745615in}{2.672142in}}%
\pgfpathlineto{\pgfqpoint{3.745681in}{2.787745in}}%
\pgfpathlineto{\pgfqpoint{3.745835in}{2.421158in}}%
\pgfpathlineto{\pgfqpoint{3.746651in}{2.640673in}}%
\pgfpathlineto{\pgfqpoint{3.746937in}{2.538182in}}%
\pgfpathlineto{\pgfqpoint{3.747510in}{2.739013in}}%
\pgfpathlineto{\pgfqpoint{3.747731in}{2.627124in}}%
\pgfpathlineto{\pgfqpoint{3.748789in}{2.799764in}}%
\pgfpathlineto{\pgfqpoint{3.748480in}{2.496114in}}%
\pgfpathlineto{\pgfqpoint{3.748855in}{2.664821in}}%
\pgfpathlineto{\pgfqpoint{3.749362in}{2.508571in}}%
\pgfpathlineto{\pgfqpoint{3.749737in}{2.792662in}}%
\pgfpathlineto{\pgfqpoint{3.749935in}{2.666679in}}%
\pgfpathlineto{\pgfqpoint{3.750244in}{2.924655in}}%
\pgfpathlineto{\pgfqpoint{3.750133in}{2.569760in}}%
\pgfpathlineto{\pgfqpoint{3.751015in}{2.728851in}}%
\pgfpathlineto{\pgfqpoint{3.751258in}{2.567356in}}%
\pgfpathlineto{\pgfqpoint{3.751390in}{2.843908in}}%
\pgfpathlineto{\pgfqpoint{3.752117in}{2.714428in}}%
\pgfpathlineto{\pgfqpoint{3.753131in}{2.540804in}}%
\pgfpathlineto{\pgfqpoint{3.752382in}{2.854834in}}%
\pgfpathlineto{\pgfqpoint{3.753263in}{2.590302in}}%
\pgfpathlineto{\pgfqpoint{3.753528in}{2.772011in}}%
\pgfpathlineto{\pgfqpoint{3.753903in}{2.515236in}}%
\pgfpathlineto{\pgfqpoint{3.754365in}{2.586805in}}%
\pgfpathlineto{\pgfqpoint{3.755445in}{2.820743in}}%
\pgfpathlineto{\pgfqpoint{3.755335in}{2.525507in}}%
\pgfpathlineto{\pgfqpoint{3.755534in}{2.669192in}}%
\pgfpathlineto{\pgfqpoint{3.756526in}{2.779441in}}%
\pgfpathlineto{\pgfqpoint{3.756261in}{2.516110in}}%
\pgfpathlineto{\pgfqpoint{3.756570in}{2.682850in}}%
\pgfpathlineto{\pgfqpoint{3.756680in}{2.534139in}}%
\pgfpathlineto{\pgfqpoint{3.756922in}{2.821727in}}%
\pgfpathlineto{\pgfqpoint{3.757672in}{2.576206in}}%
\pgfpathlineto{\pgfqpoint{3.758113in}{2.797579in}}%
\pgfpathlineto{\pgfqpoint{3.758642in}{2.558615in}}%
\pgfpathlineto{\pgfqpoint{3.758840in}{2.721311in}}%
\pgfpathlineto{\pgfqpoint{3.759281in}{2.552168in}}%
\pgfpathlineto{\pgfqpoint{3.759678in}{2.824240in}}%
\pgfpathlineto{\pgfqpoint{3.759942in}{2.731692in}}%
\pgfpathlineto{\pgfqpoint{3.760118in}{2.855818in}}%
\pgfpathlineto{\pgfqpoint{3.760736in}{2.587242in}}%
\pgfpathlineto{\pgfqpoint{3.761022in}{2.693558in}}%
\pgfpathlineto{\pgfqpoint{3.761198in}{2.536324in}}%
\pgfpathlineto{\pgfqpoint{3.761816in}{2.883899in}}%
\pgfpathlineto{\pgfqpoint{3.762146in}{2.663947in}}%
\pgfpathlineto{\pgfqpoint{3.762631in}{2.851556in}}%
\pgfpathlineto{\pgfqpoint{3.762719in}{2.531080in}}%
\pgfpathlineto{\pgfqpoint{3.763270in}{2.824131in}}%
\pgfpathlineto{\pgfqpoint{3.763755in}{2.484204in}}%
\pgfpathlineto{\pgfqpoint{3.763667in}{2.825114in}}%
\pgfpathlineto{\pgfqpoint{3.764395in}{2.696399in}}%
\pgfpathlineto{\pgfqpoint{3.765100in}{2.494038in}}%
\pgfpathlineto{\pgfqpoint{3.764857in}{2.768842in}}%
\pgfpathlineto{\pgfqpoint{3.765585in}{2.600573in}}%
\pgfpathlineto{\pgfqpoint{3.766026in}{2.791242in}}%
\pgfpathlineto{\pgfqpoint{3.765915in}{2.522994in}}%
\pgfpathlineto{\pgfqpoint{3.766709in}{2.675638in}}%
\pgfpathlineto{\pgfqpoint{3.767789in}{2.403785in}}%
\pgfpathlineto{\pgfqpoint{3.766797in}{2.740870in}}%
\pgfpathlineto{\pgfqpoint{3.767877in}{2.490433in}}%
\pgfpathlineto{\pgfqpoint{3.768847in}{2.659576in}}%
\pgfpathlineto{\pgfqpoint{3.768516in}{2.373737in}}%
\pgfpathlineto{\pgfqpoint{3.769001in}{2.553698in}}%
\pgfpathlineto{\pgfqpoint{3.769618in}{2.784904in}}%
\pgfpathlineto{\pgfqpoint{3.769045in}{2.460275in}}%
\pgfpathlineto{\pgfqpoint{3.770081in}{2.761740in}}%
\pgfpathlineto{\pgfqpoint{3.770103in}{2.539056in}}%
\pgfpathlineto{\pgfqpoint{3.771007in}{2.835167in}}%
\pgfpathlineto{\pgfqpoint{3.771183in}{2.671814in}}%
\pgfpathlineto{\pgfqpoint{3.771823in}{2.818340in}}%
\pgfpathlineto{\pgfqpoint{3.771338in}{2.510647in}}%
\pgfpathlineto{\pgfqpoint{3.772043in}{2.669192in}}%
\pgfpathlineto{\pgfqpoint{3.772903in}{2.484642in}}%
\pgfpathlineto{\pgfqpoint{3.772131in}{2.787854in}}%
\pgfpathlineto{\pgfqpoint{3.773145in}{2.684489in}}%
\pgfpathlineto{\pgfqpoint{3.773762in}{2.392858in}}%
\pgfpathlineto{\pgfqpoint{3.773630in}{2.695088in}}%
\pgfpathlineto{\pgfqpoint{3.774335in}{2.489886in}}%
\pgfpathlineto{\pgfqpoint{3.774909in}{2.789603in}}%
\pgfpathlineto{\pgfqpoint{3.775173in}{2.477976in}}%
\pgfpathlineto{\pgfqpoint{3.775526in}{2.708200in}}%
\pgfpathlineto{\pgfqpoint{3.775834in}{2.489340in}}%
\pgfpathlineto{\pgfqpoint{3.775658in}{2.849480in}}%
\pgfpathlineto{\pgfqpoint{3.776628in}{2.628763in}}%
\pgfpathlineto{\pgfqpoint{3.777333in}{2.794629in}}%
\pgfpathlineto{\pgfqpoint{3.777465in}{2.562876in}}%
\pgfpathlineto{\pgfqpoint{3.777730in}{2.593470in}}%
\pgfpathlineto{\pgfqpoint{3.778039in}{2.794847in}}%
\pgfpathlineto{\pgfqpoint{3.778656in}{2.465411in}}%
\pgfpathlineto{\pgfqpoint{3.778876in}{2.730490in}}%
\pgfpathlineto{\pgfqpoint{3.779515in}{2.544956in}}%
\pgfpathlineto{\pgfqpoint{3.779626in}{2.778130in}}%
\pgfpathlineto{\pgfqpoint{3.780022in}{2.600136in}}%
\pgfpathlineto{\pgfqpoint{3.780044in}{2.541788in}}%
\pgfpathlineto{\pgfqpoint{3.780816in}{2.800092in}}%
\pgfpathlineto{\pgfqpoint{3.781058in}{2.705249in}}%
\pgfpathlineto{\pgfqpoint{3.781146in}{2.809598in}}%
\pgfpathlineto{\pgfqpoint{3.782028in}{2.609860in}}%
\pgfpathlineto{\pgfqpoint{3.782072in}{2.611718in}}%
\pgfpathlineto{\pgfqpoint{3.782094in}{2.496661in}}%
\pgfpathlineto{\pgfqpoint{3.782293in}{2.775070in}}%
\pgfpathlineto{\pgfqpoint{3.783152in}{2.660450in}}%
\pgfpathlineto{\pgfqpoint{3.783307in}{2.763597in}}%
\pgfpathlineto{\pgfqpoint{3.784078in}{2.462788in}}%
\pgfpathlineto{\pgfqpoint{3.784232in}{2.661543in}}%
\pgfpathlineto{\pgfqpoint{3.784497in}{2.509882in}}%
\pgfpathlineto{\pgfqpoint{3.784872in}{2.774852in}}%
\pgfpathlineto{\pgfqpoint{3.785356in}{2.609423in}}%
\pgfpathlineto{\pgfqpoint{3.786282in}{2.811565in}}%
\pgfpathlineto{\pgfqpoint{3.785885in}{2.489558in}}%
\pgfpathlineto{\pgfqpoint{3.786459in}{2.721967in}}%
\pgfpathlineto{\pgfqpoint{3.786789in}{2.563641in}}%
\pgfpathlineto{\pgfqpoint{3.786988in}{2.878436in}}%
\pgfpathlineto{\pgfqpoint{3.787561in}{2.602539in}}%
\pgfpathlineto{\pgfqpoint{3.788068in}{2.824459in}}%
\pgfpathlineto{\pgfqpoint{3.787737in}{2.518077in}}%
\pgfpathlineto{\pgfqpoint{3.788685in}{2.746006in}}%
\pgfpathlineto{\pgfqpoint{3.789015in}{2.503654in}}%
\pgfpathlineto{\pgfqpoint{3.789324in}{2.782937in}}%
\pgfpathlineto{\pgfqpoint{3.789787in}{2.665586in}}%
\pgfpathlineto{\pgfqpoint{3.790095in}{2.364449in}}%
\pgfpathlineto{\pgfqpoint{3.790889in}{2.720765in}}%
\pgfpathlineto{\pgfqpoint{3.790955in}{2.508789in}}%
\pgfpathlineto{\pgfqpoint{3.791396in}{2.758571in}}%
\pgfpathlineto{\pgfqpoint{3.792013in}{2.578064in}}%
\pgfpathlineto{\pgfqpoint{3.792079in}{2.474261in}}%
\pgfpathlineto{\pgfqpoint{3.792498in}{2.774087in}}%
\pgfpathlineto{\pgfqpoint{3.792652in}{2.653457in}}%
\pgfpathlineto{\pgfqpoint{3.793600in}{2.849043in}}%
\pgfpathlineto{\pgfqpoint{3.792785in}{2.574021in}}%
\pgfpathlineto{\pgfqpoint{3.793777in}{2.840521in}}%
\pgfpathlineto{\pgfqpoint{3.794239in}{2.582981in}}%
\pgfpathlineto{\pgfqpoint{3.794746in}{2.959074in}}%
\pgfpathlineto{\pgfqpoint{3.794879in}{2.718689in}}%
\pgfpathlineto{\pgfqpoint{3.795430in}{2.885320in}}%
\pgfpathlineto{\pgfqpoint{3.795319in}{2.631167in}}%
\pgfpathlineto{\pgfqpoint{3.795937in}{2.661652in}}%
\pgfpathlineto{\pgfqpoint{3.796289in}{2.764909in}}%
\pgfpathlineto{\pgfqpoint{3.796355in}{2.611936in}}%
\pgfpathlineto{\pgfqpoint{3.796664in}{2.491307in}}%
\pgfpathlineto{\pgfqpoint{3.796929in}{2.816701in}}%
\pgfpathlineto{\pgfqpoint{3.797435in}{2.574240in}}%
\pgfpathlineto{\pgfqpoint{3.798493in}{2.804135in}}%
\pgfpathlineto{\pgfqpoint{3.797942in}{2.496005in}}%
\pgfpathlineto{\pgfqpoint{3.798560in}{2.767312in}}%
\pgfpathlineto{\pgfqpoint{3.799375in}{2.462023in}}%
\pgfpathlineto{\pgfqpoint{3.799243in}{2.796596in}}%
\pgfpathlineto{\pgfqpoint{3.799684in}{2.651163in}}%
\pgfpathlineto{\pgfqpoint{3.799948in}{2.828064in}}%
\pgfpathlineto{\pgfqpoint{3.800654in}{2.510100in}}%
\pgfpathlineto{\pgfqpoint{3.800764in}{2.637614in}}%
\pgfpathlineto{\pgfqpoint{3.801050in}{2.473169in}}%
\pgfpathlineto{\pgfqpoint{3.801469in}{2.894279in}}%
\pgfpathlineto{\pgfqpoint{3.801844in}{2.619803in}}%
\pgfpathlineto{\pgfqpoint{3.801866in}{2.703610in}}%
\pgfpathlineto{\pgfqpoint{3.802219in}{2.417224in}}%
\pgfpathlineto{\pgfqpoint{3.802924in}{2.588116in}}%
\pgfpathlineto{\pgfqpoint{3.803497in}{2.486718in}}%
\pgfpathlineto{\pgfqpoint{3.803122in}{2.738685in}}%
\pgfpathlineto{\pgfqpoint{3.803673in}{2.573147in}}%
\pgfpathlineto{\pgfqpoint{3.803982in}{2.763270in}}%
\pgfpathlineto{\pgfqpoint{3.804401in}{2.502015in}}%
\pgfpathlineto{\pgfqpoint{3.804775in}{2.718798in}}%
\pgfpathlineto{\pgfqpoint{3.805327in}{2.464427in}}%
\pgfpathlineto{\pgfqpoint{3.805723in}{2.816154in}}%
\pgfpathlineto{\pgfqpoint{3.805833in}{2.683396in}}%
\pgfpathlineto{\pgfqpoint{3.806407in}{2.928917in}}%
\pgfpathlineto{\pgfqpoint{3.806120in}{2.629637in}}%
\pgfpathlineto{\pgfqpoint{3.806936in}{2.772557in}}%
\pgfpathlineto{\pgfqpoint{3.808060in}{2.554353in}}%
\pgfpathlineto{\pgfqpoint{3.807817in}{2.834183in}}%
\pgfpathlineto{\pgfqpoint{3.808082in}{2.648213in}}%
\pgfpathlineto{\pgfqpoint{3.808258in}{2.793318in}}%
\pgfpathlineto{\pgfqpoint{3.808677in}{2.401599in}}%
\pgfpathlineto{\pgfqpoint{3.809184in}{2.634664in}}%
\pgfpathlineto{\pgfqpoint{3.809867in}{2.410996in}}%
\pgfpathlineto{\pgfqpoint{3.809559in}{2.761303in}}%
\pgfpathlineto{\pgfqpoint{3.810220in}{2.618164in}}%
\pgfpathlineto{\pgfqpoint{3.810462in}{2.839756in}}%
\pgfpathlineto{\pgfqpoint{3.810947in}{2.565826in}}%
\pgfpathlineto{\pgfqpoint{3.811322in}{2.710603in}}%
\pgfpathlineto{\pgfqpoint{3.811851in}{2.407063in}}%
\pgfpathlineto{\pgfqpoint{3.811719in}{2.798781in}}%
\pgfpathlineto{\pgfqpoint{3.812468in}{2.544847in}}%
\pgfpathlineto{\pgfqpoint{3.812843in}{2.391328in}}%
\pgfpathlineto{\pgfqpoint{3.813328in}{2.664821in}}%
\pgfpathlineto{\pgfqpoint{3.813592in}{2.477648in}}%
\pgfpathlineto{\pgfqpoint{3.814672in}{2.711478in}}%
\pgfpathlineto{\pgfqpoint{3.813680in}{2.429244in}}%
\pgfpathlineto{\pgfqpoint{3.814716in}{2.644607in}}%
\pgfpathlineto{\pgfqpoint{3.815003in}{2.461477in}}%
\pgfpathlineto{\pgfqpoint{3.815576in}{2.710931in}}%
\pgfpathlineto{\pgfqpoint{3.815818in}{2.615105in}}%
\pgfpathlineto{\pgfqpoint{3.816348in}{2.514253in}}%
\pgfpathlineto{\pgfqpoint{3.816193in}{2.863904in}}%
\pgfpathlineto{\pgfqpoint{3.816899in}{2.652037in}}%
\pgfpathlineto{\pgfqpoint{3.817273in}{2.789930in}}%
\pgfpathlineto{\pgfqpoint{3.817163in}{2.543973in}}%
\pgfpathlineto{\pgfqpoint{3.817494in}{2.605380in}}%
\pgfpathlineto{\pgfqpoint{3.817538in}{2.496114in}}%
\pgfpathlineto{\pgfqpoint{3.817957in}{2.745787in}}%
\pgfpathlineto{\pgfqpoint{3.818508in}{2.700770in}}%
\pgfpathlineto{\pgfqpoint{3.819015in}{2.830250in}}%
\pgfpathlineto{\pgfqpoint{3.819279in}{2.513051in}}%
\pgfpathlineto{\pgfqpoint{3.819610in}{2.720656in}}%
\pgfpathlineto{\pgfqpoint{3.820006in}{2.502343in}}%
\pgfpathlineto{\pgfqpoint{3.820139in}{2.783156in}}%
\pgfpathlineto{\pgfqpoint{3.820734in}{2.670612in}}%
\pgfpathlineto{\pgfqpoint{3.821285in}{2.884664in}}%
\pgfpathlineto{\pgfqpoint{3.821549in}{2.534248in}}%
\pgfpathlineto{\pgfqpoint{3.821704in}{2.633243in}}%
\pgfpathlineto{\pgfqpoint{3.822784in}{2.465520in}}%
\pgfpathlineto{\pgfqpoint{3.822123in}{2.745896in}}%
\pgfpathlineto{\pgfqpoint{3.822806in}{2.557303in}}%
\pgfpathlineto{\pgfqpoint{3.822894in}{2.823038in}}%
\pgfpathlineto{\pgfqpoint{3.823158in}{2.514908in}}%
\pgfpathlineto{\pgfqpoint{3.823930in}{2.713881in}}%
\pgfpathlineto{\pgfqpoint{3.824261in}{2.803370in}}%
\pgfpathlineto{\pgfqpoint{3.825032in}{2.514253in}}%
\pgfpathlineto{\pgfqpoint{3.826090in}{2.866635in}}%
\pgfpathlineto{\pgfqpoint{3.826156in}{2.701534in}}%
\pgfpathlineto{\pgfqpoint{3.826178in}{2.539930in}}%
\pgfpathlineto{\pgfqpoint{3.826333in}{2.820306in}}%
\pgfpathlineto{\pgfqpoint{3.827258in}{2.656626in}}%
\pgfpathlineto{\pgfqpoint{3.828228in}{2.895809in}}%
\pgfpathlineto{\pgfqpoint{3.827677in}{2.469781in}}%
\pgfpathlineto{\pgfqpoint{3.828382in}{2.715302in}}%
\pgfpathlineto{\pgfqpoint{3.829286in}{2.479943in}}%
\pgfpathlineto{\pgfqpoint{3.828493in}{2.759555in}}%
\pgfpathlineto{\pgfqpoint{3.829529in}{2.527364in}}%
\pgfpathlineto{\pgfqpoint{3.830278in}{2.757369in}}%
\pgfpathlineto{\pgfqpoint{3.829859in}{2.492399in}}%
\pgfpathlineto{\pgfqpoint{3.830697in}{2.579594in}}%
\pgfpathlineto{\pgfqpoint{3.831623in}{2.472404in}}%
\pgfpathlineto{\pgfqpoint{3.831579in}{2.778348in}}%
\pgfpathlineto{\pgfqpoint{3.831733in}{2.650944in}}%
\pgfpathlineto{\pgfqpoint{3.832703in}{2.805556in}}%
\pgfpathlineto{\pgfqpoint{3.832262in}{2.514034in}}%
\pgfpathlineto{\pgfqpoint{3.832747in}{2.687439in}}%
\pgfpathlineto{\pgfqpoint{3.832769in}{2.460712in}}%
\pgfpathlineto{\pgfqpoint{3.833033in}{2.780206in}}%
\pgfpathlineto{\pgfqpoint{3.833849in}{2.611062in}}%
\pgfpathlineto{\pgfqpoint{3.834202in}{2.811674in}}%
\pgfpathlineto{\pgfqpoint{3.834753in}{2.440607in}}%
\pgfpathlineto{\pgfqpoint{3.834929in}{2.590083in}}%
\pgfpathlineto{\pgfqpoint{3.835370in}{2.500267in}}%
\pgfpathlineto{\pgfqpoint{3.835017in}{2.738685in}}%
\pgfpathlineto{\pgfqpoint{3.836031in}{2.500704in}}%
\pgfpathlineto{\pgfqpoint{3.837111in}{2.803479in}}%
\pgfpathlineto{\pgfqpoint{3.836273in}{2.490105in}}%
\pgfpathlineto{\pgfqpoint{3.837155in}{2.604506in}}%
\pgfpathlineto{\pgfqpoint{3.837971in}{2.507478in}}%
\pgfpathlineto{\pgfqpoint{3.837728in}{2.837243in}}%
\pgfpathlineto{\pgfqpoint{3.838081in}{2.638160in}}%
\pgfpathlineto{\pgfqpoint{3.838764in}{2.881277in}}%
\pgfpathlineto{\pgfqpoint{3.838191in}{2.554681in}}%
\pgfpathlineto{\pgfqpoint{3.839183in}{2.635866in}}%
\pgfpathlineto{\pgfqpoint{3.839403in}{2.494803in}}%
\pgfpathlineto{\pgfqpoint{3.839293in}{2.835167in}}%
\pgfpathlineto{\pgfqpoint{3.840241in}{2.581451in}}%
\pgfpathlineto{\pgfqpoint{3.840329in}{2.836587in}}%
\pgfpathlineto{\pgfqpoint{3.841365in}{2.680992in}}%
\pgfpathlineto{\pgfqpoint{3.841674in}{2.838991in}}%
\pgfpathlineto{\pgfqpoint{3.842115in}{2.576316in}}%
\pgfpathlineto{\pgfqpoint{3.842489in}{2.740214in}}%
\pgfpathlineto{\pgfqpoint{3.843393in}{2.453173in}}%
\pgfpathlineto{\pgfqpoint{3.843173in}{2.844454in}}%
\pgfpathlineto{\pgfqpoint{3.843503in}{2.670612in}}%
\pgfpathlineto{\pgfqpoint{3.844054in}{2.825223in}}%
\pgfpathlineto{\pgfqpoint{3.843900in}{2.533046in}}%
\pgfpathlineto{\pgfqpoint{3.844583in}{2.606473in}}%
\pgfpathlineto{\pgfqpoint{3.845223in}{2.541897in}}%
\pgfpathlineto{\pgfqpoint{3.845355in}{2.817684in}}%
\pgfpathlineto{\pgfqpoint{3.845377in}{2.806539in}}%
\pgfpathlineto{\pgfqpoint{3.845399in}{2.888925in}}%
\pgfpathlineto{\pgfqpoint{3.845950in}{2.582325in}}%
\pgfpathlineto{\pgfqpoint{3.846435in}{2.664165in}}%
\pgfpathlineto{\pgfqpoint{3.847317in}{2.856146in}}%
\pgfpathlineto{\pgfqpoint{3.846589in}{2.536215in}}%
\pgfpathlineto{\pgfqpoint{3.847537in}{2.764909in}}%
\pgfpathlineto{\pgfqpoint{3.847735in}{2.608877in}}%
\pgfpathlineto{\pgfqpoint{3.848154in}{2.964537in}}%
\pgfpathlineto{\pgfqpoint{3.848639in}{2.698366in}}%
\pgfpathlineto{\pgfqpoint{3.848992in}{2.923672in}}%
\pgfpathlineto{\pgfqpoint{3.848727in}{2.619913in}}%
\pgfpathlineto{\pgfqpoint{3.849763in}{2.770044in}}%
\pgfpathlineto{\pgfqpoint{3.849851in}{2.614122in}}%
\pgfpathlineto{\pgfqpoint{3.850843in}{2.861827in}}%
\pgfpathlineto{\pgfqpoint{3.850865in}{2.741853in}}%
\pgfpathlineto{\pgfqpoint{3.851218in}{2.844673in}}%
\pgfpathlineto{\pgfqpoint{3.851438in}{2.617837in}}%
\pgfpathlineto{\pgfqpoint{3.851967in}{2.799437in}}%
\pgfpathlineto{\pgfqpoint{3.852078in}{2.612920in}}%
\pgfpathlineto{\pgfqpoint{3.852607in}{2.902584in}}%
\pgfpathlineto{\pgfqpoint{3.853069in}{2.820962in}}%
\pgfpathlineto{\pgfqpoint{3.853092in}{2.964210in}}%
\pgfpathlineto{\pgfqpoint{3.853643in}{2.613794in}}%
\pgfpathlineto{\pgfqpoint{3.854150in}{2.755621in}}%
\pgfpathlineto{\pgfqpoint{3.854701in}{2.955031in}}%
\pgfpathlineto{\pgfqpoint{3.854436in}{2.655315in}}%
\pgfpathlineto{\pgfqpoint{3.854987in}{2.777365in}}%
\pgfpathlineto{\pgfqpoint{3.855274in}{2.623300in}}%
\pgfpathlineto{\pgfqpoint{3.855781in}{2.886303in}}%
\pgfpathlineto{\pgfqpoint{3.856111in}{2.687330in}}%
\pgfpathlineto{\pgfqpoint{3.856155in}{2.591067in}}%
\pgfpathlineto{\pgfqpoint{3.856795in}{2.945744in}}%
\pgfpathlineto{\pgfqpoint{3.857147in}{2.715083in}}%
\pgfpathlineto{\pgfqpoint{3.857742in}{2.941592in}}%
\pgfpathlineto{\pgfqpoint{3.858183in}{2.650398in}}%
\pgfpathlineto{\pgfqpoint{3.858249in}{2.717378in}}%
\pgfpathlineto{\pgfqpoint{3.858933in}{2.893187in}}%
\pgfpathlineto{\pgfqpoint{3.859241in}{2.644935in}}%
\pgfpathlineto{\pgfqpoint{3.859373in}{2.780206in}}%
\pgfpathlineto{\pgfqpoint{3.859880in}{2.609860in}}%
\pgfpathlineto{\pgfqpoint{3.859572in}{2.996771in}}%
\pgfpathlineto{\pgfqpoint{3.860476in}{2.778785in}}%
\pgfpathlineto{\pgfqpoint{3.860784in}{2.643842in}}%
\pgfpathlineto{\pgfqpoint{3.860828in}{2.874393in}}%
\pgfpathlineto{\pgfqpoint{3.861578in}{2.746989in}}%
\pgfpathlineto{\pgfqpoint{3.862415in}{2.902147in}}%
\pgfpathlineto{\pgfqpoint{3.862592in}{2.621115in}}%
\pgfpathlineto{\pgfqpoint{3.862636in}{2.722186in}}%
\pgfpathlineto{\pgfqpoint{3.862944in}{2.861500in}}%
\pgfpathlineto{\pgfqpoint{3.863738in}{2.609642in}}%
\pgfpathlineto{\pgfqpoint{3.864333in}{2.829266in}}%
\pgfpathlineto{\pgfqpoint{3.864201in}{2.584183in}}%
\pgfpathlineto{\pgfqpoint{3.864840in}{2.689952in}}%
\pgfpathlineto{\pgfqpoint{3.865435in}{2.568448in}}%
\pgfpathlineto{\pgfqpoint{3.864994in}{2.825660in}}%
\pgfpathlineto{\pgfqpoint{3.865920in}{2.747863in}}%
\pgfpathlineto{\pgfqpoint{3.865964in}{2.840084in}}%
\pgfpathlineto{\pgfqpoint{3.866052in}{2.509336in}}%
\pgfpathlineto{\pgfqpoint{3.866537in}{2.636849in}}%
\pgfpathlineto{\pgfqpoint{3.866559in}{2.434161in}}%
\pgfpathlineto{\pgfqpoint{3.867287in}{2.822492in}}%
\pgfpathlineto{\pgfqpoint{3.867639in}{2.662636in}}%
\pgfpathlineto{\pgfqpoint{3.868278in}{2.535887in}}%
\pgfpathlineto{\pgfqpoint{3.867882in}{2.796049in}}%
\pgfpathlineto{\pgfqpoint{3.868741in}{2.673344in}}%
\pgfpathlineto{\pgfqpoint{3.869028in}{2.895372in}}%
\pgfpathlineto{\pgfqpoint{3.869469in}{2.561455in}}%
\pgfpathlineto{\pgfqpoint{3.869865in}{2.775726in}}%
\pgfpathlineto{\pgfqpoint{3.870086in}{2.793973in}}%
\pgfpathlineto{\pgfqpoint{3.869910in}{2.593470in}}%
\pgfpathlineto{\pgfqpoint{3.870328in}{2.702081in}}%
\pgfpathlineto{\pgfqpoint{3.870394in}{2.599589in}}%
\pgfpathlineto{\pgfqpoint{3.871232in}{2.866307in}}%
\pgfpathlineto{\pgfqpoint{3.871430in}{2.666569in}}%
\pgfpathlineto{\pgfqpoint{3.871783in}{2.835604in}}%
\pgfpathlineto{\pgfqpoint{3.872466in}{2.630293in}}%
\pgfpathlineto{\pgfqpoint{3.872555in}{2.760538in}}%
\pgfpathlineto{\pgfqpoint{3.873106in}{2.590083in}}%
\pgfpathlineto{\pgfqpoint{3.873591in}{2.823366in}}%
\pgfpathlineto{\pgfqpoint{3.873701in}{2.640564in}}%
\pgfpathlineto{\pgfqpoint{3.874803in}{2.879747in}}%
\pgfpathlineto{\pgfqpoint{3.874296in}{2.557959in}}%
\pgfpathlineto{\pgfqpoint{3.874847in}{2.742728in}}%
\pgfpathlineto{\pgfqpoint{3.875486in}{2.567247in}}%
\pgfpathlineto{\pgfqpoint{3.875618in}{2.890237in}}%
\pgfpathlineto{\pgfqpoint{3.875927in}{2.773978in}}%
\pgfpathlineto{\pgfqpoint{3.876478in}{2.870678in}}%
\pgfpathlineto{\pgfqpoint{3.876588in}{2.589209in}}%
\pgfpathlineto{\pgfqpoint{3.876963in}{2.714537in}}%
\pgfpathlineto{\pgfqpoint{3.877756in}{2.857348in}}%
\pgfpathlineto{\pgfqpoint{3.878043in}{2.599589in}}%
\pgfpathlineto{\pgfqpoint{3.878770in}{2.552933in}}%
\pgfpathlineto{\pgfqpoint{3.879167in}{2.913510in}}%
\pgfpathlineto{\pgfqpoint{3.879366in}{2.578282in}}%
\pgfpathlineto{\pgfqpoint{3.880291in}{2.732347in}}%
\pgfpathlineto{\pgfqpoint{3.880666in}{2.853195in}}%
\pgfpathlineto{\pgfqpoint{3.881019in}{2.619694in}}%
\pgfpathlineto{\pgfqpoint{3.881393in}{2.738794in}}%
\pgfpathlineto{\pgfqpoint{3.881746in}{2.864450in}}%
\pgfpathlineto{\pgfqpoint{3.881438in}{2.667225in}}%
\pgfpathlineto{\pgfqpoint{3.881834in}{2.748082in}}%
\pgfpathlineto{\pgfqpoint{3.881856in}{2.548781in}}%
\pgfpathlineto{\pgfqpoint{3.881989in}{2.870787in}}%
\pgfpathlineto{\pgfqpoint{3.882914in}{2.806320in}}%
\pgfpathlineto{\pgfqpoint{3.883465in}{2.887833in}}%
\pgfpathlineto{\pgfqpoint{3.883113in}{2.582107in}}%
\pgfpathlineto{\pgfqpoint{3.883994in}{2.768951in}}%
\pgfpathlineto{\pgfqpoint{3.885052in}{2.613794in}}%
\pgfpathlineto{\pgfqpoint{3.884590in}{2.883353in}}%
\pgfpathlineto{\pgfqpoint{3.885229in}{2.626250in}}%
\pgfpathlineto{\pgfqpoint{3.886199in}{2.872754in}}%
\pgfpathlineto{\pgfqpoint{3.885890in}{2.605817in}}%
\pgfpathlineto{\pgfqpoint{3.886331in}{2.653348in}}%
\pgfpathlineto{\pgfqpoint{3.887345in}{2.838226in}}%
\pgfpathlineto{\pgfqpoint{3.886551in}{2.574458in}}%
\pgfpathlineto{\pgfqpoint{3.887477in}{2.732566in}}%
\pgfpathlineto{\pgfqpoint{3.888204in}{2.552386in}}%
\pgfpathlineto{\pgfqpoint{3.887852in}{2.918755in}}%
\pgfpathlineto{\pgfqpoint{3.888557in}{2.718252in}}%
\pgfpathlineto{\pgfqpoint{3.888623in}{2.897994in}}%
\pgfpathlineto{\pgfqpoint{3.889571in}{2.567356in}}%
\pgfpathlineto{\pgfqpoint{3.889637in}{2.690389in}}%
\pgfpathlineto{\pgfqpoint{3.889858in}{2.625922in}}%
\pgfpathlineto{\pgfqpoint{3.890409in}{2.887068in}}%
\pgfpathlineto{\pgfqpoint{3.890673in}{2.718580in}}%
\pgfpathlineto{\pgfqpoint{3.891004in}{2.903239in}}%
\pgfpathlineto{\pgfqpoint{3.891202in}{2.652474in}}%
\pgfpathlineto{\pgfqpoint{3.891753in}{2.709074in}}%
\pgfpathlineto{\pgfqpoint{3.892304in}{2.654987in}}%
\pgfpathlineto{\pgfqpoint{3.892569in}{2.909686in}}%
\pgfpathlineto{\pgfqpoint{3.892833in}{2.730271in}}%
\pgfpathlineto{\pgfqpoint{3.893186in}{2.932960in}}%
\pgfpathlineto{\pgfqpoint{3.893054in}{2.586259in}}%
\pgfpathlineto{\pgfqpoint{3.893935in}{2.783593in}}%
\pgfpathlineto{\pgfqpoint{3.894993in}{2.565826in}}%
\pgfpathlineto{\pgfqpoint{3.894046in}{2.831670in}}%
\pgfpathlineto{\pgfqpoint{3.895059in}{2.618055in}}%
\pgfpathlineto{\pgfqpoint{3.895104in}{2.752234in}}%
\pgfpathlineto{\pgfqpoint{3.895985in}{2.818449in}}%
\pgfpathlineto{\pgfqpoint{3.895280in}{2.547469in}}%
\pgfpathlineto{\pgfqpoint{3.896184in}{2.676950in}}%
\pgfpathlineto{\pgfqpoint{3.896514in}{2.502889in}}%
\pgfpathlineto{\pgfqpoint{3.897087in}{2.846749in}}%
\pgfpathlineto{\pgfqpoint{3.897242in}{2.671814in}}%
\pgfpathlineto{\pgfqpoint{3.897286in}{2.877562in}}%
\pgfpathlineto{\pgfqpoint{3.898167in}{2.528785in}}%
\pgfpathlineto{\pgfqpoint{3.898322in}{2.628326in}}%
\pgfpathlineto{\pgfqpoint{3.899225in}{2.523431in}}%
\pgfpathlineto{\pgfqpoint{3.898652in}{2.780971in}}%
\pgfpathlineto{\pgfqpoint{3.899424in}{2.649852in}}%
\pgfpathlineto{\pgfqpoint{3.900195in}{2.851666in}}%
\pgfpathlineto{\pgfqpoint{3.899578in}{2.529441in}}%
\pgfpathlineto{\pgfqpoint{3.900526in}{2.712679in}}%
\pgfpathlineto{\pgfqpoint{3.901319in}{2.445306in}}%
\pgfpathlineto{\pgfqpoint{3.901121in}{2.792881in}}%
\pgfpathlineto{\pgfqpoint{3.901628in}{2.643405in}}%
\pgfpathlineto{\pgfqpoint{3.902091in}{2.828611in}}%
\pgfpathlineto{\pgfqpoint{3.901738in}{2.479506in}}%
\pgfpathlineto{\pgfqpoint{3.902752in}{2.739996in}}%
\pgfpathlineto{\pgfqpoint{3.902862in}{2.567028in}}%
\pgfpathlineto{\pgfqpoint{3.903193in}{2.870787in}}%
\pgfpathlineto{\pgfqpoint{3.903854in}{2.704703in}}%
\pgfpathlineto{\pgfqpoint{3.904075in}{2.956452in}}%
\pgfpathlineto{\pgfqpoint{3.904317in}{2.568448in}}%
\pgfpathlineto{\pgfqpoint{3.904934in}{2.777911in}}%
\pgfpathlineto{\pgfqpoint{3.905044in}{2.640564in}}%
\pgfpathlineto{\pgfqpoint{3.905992in}{2.957544in}}%
\pgfpathlineto{\pgfqpoint{3.906014in}{2.897776in}}%
\pgfpathlineto{\pgfqpoint{3.906654in}{2.570962in}}%
\pgfpathlineto{\pgfqpoint{3.906279in}{2.916679in}}%
\pgfpathlineto{\pgfqpoint{3.907337in}{2.674546in}}%
\pgfpathlineto{\pgfqpoint{3.907535in}{2.815608in}}%
\pgfpathlineto{\pgfqpoint{3.907756in}{2.586150in}}%
\pgfpathlineto{\pgfqpoint{3.908042in}{2.632915in}}%
\pgfpathlineto{\pgfqpoint{3.908064in}{2.584948in}}%
\pgfpathlineto{\pgfqpoint{3.908351in}{2.927278in}}%
\pgfpathlineto{\pgfqpoint{3.909100in}{2.685800in}}%
\pgfpathlineto{\pgfqpoint{3.909188in}{2.901819in}}%
\pgfpathlineto{\pgfqpoint{3.910004in}{2.525616in}}%
\pgfpathlineto{\pgfqpoint{3.910180in}{2.640455in}}%
\pgfpathlineto{\pgfqpoint{3.910335in}{2.875376in}}%
\pgfpathlineto{\pgfqpoint{3.910577in}{2.601556in}}%
\pgfpathlineto{\pgfqpoint{3.910842in}{2.690826in}}%
\pgfpathlineto{\pgfqpoint{3.910864in}{2.490433in}}%
\pgfpathlineto{\pgfqpoint{3.911128in}{2.877562in}}%
\pgfpathlineto{\pgfqpoint{3.911944in}{2.666023in}}%
\pgfpathlineto{\pgfqpoint{3.912759in}{2.809598in}}%
\pgfpathlineto{\pgfqpoint{3.912340in}{2.557631in}}%
\pgfpathlineto{\pgfqpoint{3.913046in}{2.717487in}}%
\pgfpathlineto{\pgfqpoint{3.913465in}{2.581451in}}%
\pgfpathlineto{\pgfqpoint{3.913597in}{2.865324in}}%
\pgfpathlineto{\pgfqpoint{3.914170in}{2.649524in}}%
\pgfpathlineto{\pgfqpoint{3.914853in}{2.840193in}}%
\pgfpathlineto{\pgfqpoint{3.915184in}{2.523431in}}%
\pgfpathlineto{\pgfqpoint{3.915250in}{2.610079in}}%
\pgfpathlineto{\pgfqpoint{3.915603in}{2.431320in}}%
\pgfpathlineto{\pgfqpoint{3.916286in}{2.807959in}}%
\pgfpathlineto{\pgfqpoint{3.916308in}{2.755402in}}%
\pgfpathlineto{\pgfqpoint{3.917212in}{2.512395in}}%
\pgfpathlineto{\pgfqpoint{3.916815in}{2.815827in}}%
\pgfpathlineto{\pgfqpoint{3.917388in}{2.749065in}}%
\pgfpathlineto{\pgfqpoint{3.917697in}{2.844891in}}%
\pgfpathlineto{\pgfqpoint{3.918358in}{2.520918in}}%
\pgfpathlineto{\pgfqpoint{3.918380in}{2.522338in}}%
\pgfpathlineto{\pgfqpoint{3.919041in}{2.838117in}}%
\pgfpathlineto{\pgfqpoint{3.918975in}{2.431866in}}%
\pgfpathlineto{\pgfqpoint{3.919504in}{2.602321in}}%
\pgfpathlineto{\pgfqpoint{3.920430in}{2.478850in}}%
\pgfpathlineto{\pgfqpoint{3.920143in}{2.777256in}}%
\pgfpathlineto{\pgfqpoint{3.920496in}{2.627015in}}%
\pgfpathlineto{\pgfqpoint{3.921466in}{2.881277in}}%
\pgfpathlineto{\pgfqpoint{3.920606in}{2.488684in}}%
\pgfpathlineto{\pgfqpoint{3.921620in}{2.705140in}}%
\pgfpathlineto{\pgfqpoint{3.922700in}{2.917335in}}%
\pgfpathlineto{\pgfqpoint{3.922369in}{2.607675in}}%
\pgfpathlineto{\pgfqpoint{3.922744in}{2.827627in}}%
\pgfpathlineto{\pgfqpoint{3.923163in}{2.873628in}}%
\pgfpathlineto{\pgfqpoint{3.923890in}{2.512286in}}%
\pgfpathlineto{\pgfqpoint{3.924508in}{2.783375in}}%
\pgfpathlineto{\pgfqpoint{3.925037in}{2.686237in}}%
\pgfpathlineto{\pgfqpoint{3.925257in}{2.535232in}}%
\pgfpathlineto{\pgfqpoint{3.925455in}{2.816373in}}%
\pgfpathlineto{\pgfqpoint{3.926139in}{2.710385in}}%
\pgfpathlineto{\pgfqpoint{3.927219in}{2.516875in}}%
\pgfpathlineto{\pgfqpoint{3.926249in}{2.834511in}}%
\pgfpathlineto{\pgfqpoint{3.927263in}{2.624393in}}%
\pgfpathlineto{\pgfqpoint{3.927439in}{2.887833in}}%
\pgfpathlineto{\pgfqpoint{3.927770in}{2.577080in}}%
\pgfpathlineto{\pgfqpoint{3.928321in}{2.648540in}}%
\pgfpathlineto{\pgfqpoint{3.928343in}{2.499720in}}%
\pgfpathlineto{\pgfqpoint{3.928585in}{2.876032in}}%
\pgfpathlineto{\pgfqpoint{3.929423in}{2.638706in}}%
\pgfpathlineto{\pgfqpoint{3.929467in}{2.606254in}}%
\pgfpathlineto{\pgfqpoint{3.929511in}{2.661980in}}%
\pgfpathlineto{\pgfqpoint{3.929665in}{2.916351in}}%
\pgfpathlineto{\pgfqpoint{3.930018in}{2.572164in}}%
\pgfpathlineto{\pgfqpoint{3.930613in}{2.737701in}}%
\pgfpathlineto{\pgfqpoint{3.931142in}{2.597950in}}%
\pgfpathlineto{\pgfqpoint{3.931319in}{2.840084in}}%
\pgfpathlineto{\pgfqpoint{3.931649in}{2.758025in}}%
\pgfpathlineto{\pgfqpoint{3.932046in}{2.887505in}}%
\pgfpathlineto{\pgfqpoint{3.931958in}{2.623956in}}%
\pgfpathlineto{\pgfqpoint{3.932773in}{2.852103in}}%
\pgfpathlineto{\pgfqpoint{3.933699in}{2.600682in}}%
\pgfpathlineto{\pgfqpoint{3.933192in}{2.953829in}}%
\pgfpathlineto{\pgfqpoint{3.933942in}{2.746006in}}%
\pgfpathlineto{\pgfqpoint{3.934250in}{2.927824in}}%
\pgfpathlineto{\pgfqpoint{3.934867in}{2.572491in}}%
\pgfpathlineto{\pgfqpoint{3.935022in}{2.599152in}}%
\pgfpathlineto{\pgfqpoint{3.935749in}{2.918646in}}%
\pgfpathlineto{\pgfqpoint{3.936124in}{2.625048in}}%
\pgfpathlineto{\pgfqpoint{3.936344in}{2.593470in}}%
\pgfpathlineto{\pgfqpoint{3.936565in}{2.815608in}}%
\pgfpathlineto{\pgfqpoint{3.936741in}{2.677277in}}%
\pgfpathlineto{\pgfqpoint{3.936763in}{2.868274in}}%
\pgfpathlineto{\pgfqpoint{3.937049in}{2.599699in}}%
\pgfpathlineto{\pgfqpoint{3.937865in}{2.817138in}}%
\pgfpathlineto{\pgfqpoint{3.938879in}{2.651272in}}%
\pgfpathlineto{\pgfqpoint{3.938284in}{2.893405in}}%
\pgfpathlineto{\pgfqpoint{3.938989in}{2.718471in}}%
\pgfpathlineto{\pgfqpoint{3.939055in}{2.656735in}}%
\pgfpathlineto{\pgfqpoint{3.939099in}{2.812549in}}%
\pgfpathlineto{\pgfqpoint{3.939364in}{2.935691in}}%
\pgfpathlineto{\pgfqpoint{3.939827in}{2.648322in}}%
\pgfpathlineto{\pgfqpoint{3.940179in}{2.824021in}}%
\pgfpathlineto{\pgfqpoint{3.940907in}{2.632806in}}%
\pgfpathlineto{\pgfqpoint{3.940598in}{2.946727in}}%
\pgfpathlineto{\pgfqpoint{3.941304in}{2.722404in}}%
\pgfpathlineto{\pgfqpoint{3.941502in}{2.966504in}}%
\pgfpathlineto{\pgfqpoint{3.941722in}{2.631932in}}%
\pgfpathlineto{\pgfqpoint{3.942428in}{2.870459in}}%
\pgfpathlineto{\pgfqpoint{3.943530in}{2.492618in}}%
\pgfpathlineto{\pgfqpoint{3.942516in}{2.917444in}}%
\pgfpathlineto{\pgfqpoint{3.943574in}{2.602758in}}%
\pgfpathlineto{\pgfqpoint{3.943905in}{2.832544in}}%
\pgfpathlineto{\pgfqpoint{3.944588in}{2.567247in}}%
\pgfpathlineto{\pgfqpoint{3.944676in}{2.611827in}}%
\pgfpathlineto{\pgfqpoint{3.944742in}{2.522775in}}%
\pgfpathlineto{\pgfqpoint{3.945183in}{2.842706in}}%
\pgfpathlineto{\pgfqpoint{3.945756in}{2.562220in}}%
\pgfpathlineto{\pgfqpoint{3.946858in}{2.887614in}}%
\pgfpathlineto{\pgfqpoint{3.945910in}{2.442137in}}%
\pgfpathlineto{\pgfqpoint{3.946880in}{2.731910in}}%
\pgfpathlineto{\pgfqpoint{3.947431in}{2.573475in}}%
\pgfpathlineto{\pgfqpoint{3.947696in}{2.878436in}}%
\pgfpathlineto{\pgfqpoint{3.947982in}{2.712461in}}%
\pgfpathlineto{\pgfqpoint{3.948026in}{2.727977in}}%
\pgfpathlineto{\pgfqpoint{3.948048in}{2.895591in}}%
\pgfpathlineto{\pgfqpoint{3.948511in}{2.593798in}}%
\pgfpathlineto{\pgfqpoint{3.949106in}{2.723169in}}%
\pgfpathlineto{\pgfqpoint{3.950010in}{2.378872in}}%
\pgfpathlineto{\pgfqpoint{3.949239in}{2.750813in}}%
\pgfpathlineto{\pgfqpoint{3.950275in}{2.495022in}}%
\pgfpathlineto{\pgfqpoint{3.950429in}{2.603414in}}%
\pgfpathlineto{\pgfqpoint{3.951002in}{2.292661in}}%
\pgfpathlineto{\pgfqpoint{3.951333in}{2.394279in}}%
\pgfpathlineto{\pgfqpoint{3.952038in}{2.232128in}}%
\pgfpathlineto{\pgfqpoint{3.952236in}{2.530315in}}%
\pgfpathlineto{\pgfqpoint{3.952435in}{2.439952in}}%
\pgfpathlineto{\pgfqpoint{3.953338in}{2.578719in}}%
\pgfpathlineto{\pgfqpoint{3.952898in}{2.223168in}}%
\pgfpathlineto{\pgfqpoint{3.953515in}{2.415367in}}%
\pgfpathlineto{\pgfqpoint{3.954198in}{2.587789in}}%
\pgfpathlineto{\pgfqpoint{3.954617in}{2.090410in}}%
\pgfpathlineto{\pgfqpoint{3.954837in}{2.475354in}}%
\pgfpathlineto{\pgfqpoint{3.955741in}{2.335712in}}%
\pgfpathlineto{\pgfqpoint{3.956292in}{2.518514in}}%
\pgfpathlineto{\pgfqpoint{3.956094in}{2.194103in}}%
\pgfpathlineto{\pgfqpoint{3.956821in}{2.413619in}}%
\pgfpathlineto{\pgfqpoint{3.957438in}{2.136520in}}%
\pgfpathlineto{\pgfqpoint{3.956887in}{2.529878in}}%
\pgfpathlineto{\pgfqpoint{3.957923in}{2.274851in}}%
\pgfpathlineto{\pgfqpoint{3.958827in}{2.419956in}}%
\pgfpathlineto{\pgfqpoint{3.958011in}{2.134881in}}%
\pgfpathlineto{\pgfqpoint{3.959025in}{2.293972in}}%
\pgfpathlineto{\pgfqpoint{3.960039in}{1.989558in}}%
\pgfpathlineto{\pgfqpoint{3.959620in}{2.365214in}}%
\pgfpathlineto{\pgfqpoint{3.960149in}{2.223277in}}%
\pgfpathlineto{\pgfqpoint{3.961185in}{1.983767in}}%
\pgfpathlineto{\pgfqpoint{3.960877in}{2.328173in}}%
\pgfpathlineto{\pgfqpoint{3.961274in}{2.134444in}}%
\pgfpathlineto{\pgfqpoint{3.961428in}{1.919627in}}%
\pgfpathlineto{\pgfqpoint{3.962420in}{2.306975in}}%
\pgfpathlineto{\pgfqpoint{3.963544in}{1.952189in}}%
\pgfpathlineto{\pgfqpoint{3.963588in}{1.970655in}}%
\pgfpathlineto{\pgfqpoint{3.963610in}{1.929243in}}%
\pgfpathlineto{\pgfqpoint{3.964227in}{2.173999in}}%
\pgfpathlineto{\pgfqpoint{3.964646in}{2.039164in}}%
\pgfpathlineto{\pgfqpoint{3.965373in}{2.159248in}}%
\pgfpathlineto{\pgfqpoint{3.965109in}{1.843797in}}%
\pgfpathlineto{\pgfqpoint{3.965748in}{2.062657in}}%
\pgfpathlineto{\pgfqpoint{3.966542in}{1.847730in}}%
\pgfpathlineto{\pgfqpoint{3.966189in}{2.295721in}}%
\pgfpathlineto{\pgfqpoint{3.966850in}{2.052823in}}%
\pgfpathlineto{\pgfqpoint{3.967732in}{1.872971in}}%
\pgfpathlineto{\pgfqpoint{3.967511in}{2.163728in}}%
\pgfpathlineto{\pgfqpoint{3.967908in}{1.952626in}}%
\pgfpathlineto{\pgfqpoint{3.968459in}{2.169956in}}%
\pgfpathlineto{\pgfqpoint{3.968570in}{1.890453in}}%
\pgfpathlineto{\pgfqpoint{3.969010in}{1.970108in}}%
\pgfpathlineto{\pgfqpoint{3.969738in}{2.199785in}}%
\pgfpathlineto{\pgfqpoint{3.969429in}{1.841284in}}%
\pgfpathlineto{\pgfqpoint{3.970046in}{2.003981in}}%
\pgfpathlineto{\pgfqpoint{3.970090in}{1.865322in}}%
\pgfpathlineto{\pgfqpoint{3.970972in}{2.205686in}}%
\pgfpathlineto{\pgfqpoint{3.971148in}{1.990432in}}%
\pgfpathlineto{\pgfqpoint{3.971347in}{2.200659in}}%
\pgfpathlineto{\pgfqpoint{3.971479in}{1.869037in}}%
\pgfpathlineto{\pgfqpoint{3.972251in}{2.043644in}}%
\pgfpathlineto{\pgfqpoint{3.972824in}{2.129964in}}%
\pgfpathlineto{\pgfqpoint{3.973397in}{1.872643in}}%
\pgfpathlineto{\pgfqpoint{3.973639in}{2.190716in}}%
\pgfpathlineto{\pgfqpoint{3.974477in}{1.831887in}}%
\pgfpathlineto{\pgfqpoint{3.974521in}{1.972294in}}%
\pgfpathlineto{\pgfqpoint{3.975403in}{2.147665in}}%
\pgfpathlineto{\pgfqpoint{3.975226in}{1.767529in}}%
\pgfpathlineto{\pgfqpoint{3.975667in}{2.127779in}}%
\pgfpathlineto{\pgfqpoint{3.976791in}{1.889252in}}%
\pgfpathlineto{\pgfqpoint{3.975998in}{2.222731in}}%
\pgfpathlineto{\pgfqpoint{3.976813in}{1.930663in}}%
\pgfpathlineto{\pgfqpoint{3.977012in}{2.177604in}}%
\pgfpathlineto{\pgfqpoint{3.977122in}{1.841612in}}%
\pgfpathlineto{\pgfqpoint{3.977915in}{2.081122in}}%
\pgfpathlineto{\pgfqpoint{3.978400in}{1.853740in}}%
\pgfpathlineto{\pgfqpoint{3.979039in}{1.934597in}}%
\pgfpathlineto{\pgfqpoint{3.979789in}{1.695195in}}%
\pgfpathlineto{\pgfqpoint{3.979128in}{1.968032in}}%
\pgfpathlineto{\pgfqpoint{3.980252in}{1.742617in}}%
\pgfpathlineto{\pgfqpoint{3.981089in}{1.860296in}}%
\pgfpathlineto{\pgfqpoint{3.980825in}{1.614776in}}%
\pgfpathlineto{\pgfqpoint{3.981376in}{1.834728in}}%
\pgfpathlineto{\pgfqpoint{3.982412in}{1.686345in}}%
\pgfpathlineto{\pgfqpoint{3.981751in}{1.974807in}}%
\pgfpathlineto{\pgfqpoint{3.982456in}{1.733329in}}%
\pgfpathlineto{\pgfqpoint{3.983227in}{1.943775in}}%
\pgfpathlineto{\pgfqpoint{3.983338in}{1.651707in}}%
\pgfpathlineto{\pgfqpoint{3.983580in}{1.858001in}}%
\pgfpathlineto{\pgfqpoint{3.984550in}{2.084619in}}%
\pgfpathlineto{\pgfqpoint{3.983690in}{1.724151in}}%
\pgfpathlineto{\pgfqpoint{3.984682in}{1.928369in}}%
\pgfpathlineto{\pgfqpoint{3.985608in}{1.671157in}}%
\pgfpathlineto{\pgfqpoint{3.985321in}{2.040476in}}%
\pgfpathlineto{\pgfqpoint{3.985740in}{1.990541in}}%
\pgfpathlineto{\pgfqpoint{3.985762in}{2.070414in}}%
\pgfpathlineto{\pgfqpoint{3.986710in}{1.627560in}}%
\pgfpathlineto{\pgfqpoint{3.986798in}{1.889361in}}%
\pgfpathlineto{\pgfqpoint{3.986953in}{1.588880in}}%
\pgfpathlineto{\pgfqpoint{3.987922in}{1.752232in}}%
\pgfpathlineto{\pgfqpoint{3.988672in}{1.938968in}}%
\pgfpathlineto{\pgfqpoint{3.988143in}{1.637940in}}%
\pgfpathlineto{\pgfqpoint{3.989002in}{1.782499in}}%
\pgfpathlineto{\pgfqpoint{3.989796in}{1.598386in}}%
\pgfpathlineto{\pgfqpoint{3.989862in}{1.855051in}}%
\pgfpathlineto{\pgfqpoint{3.990105in}{1.656952in}}%
\pgfpathlineto{\pgfqpoint{3.990237in}{1.855598in}}%
\pgfpathlineto{\pgfqpoint{3.990700in}{1.549544in}}%
\pgfpathlineto{\pgfqpoint{3.991229in}{1.756166in}}%
\pgfpathlineto{\pgfqpoint{3.991824in}{1.548233in}}%
\pgfpathlineto{\pgfqpoint{3.991780in}{1.783919in}}%
\pgfpathlineto{\pgfqpoint{3.992088in}{1.750921in}}%
\pgfpathlineto{\pgfqpoint{3.992441in}{1.896572in}}%
\pgfpathlineto{\pgfqpoint{3.992794in}{1.562109in}}%
\pgfpathlineto{\pgfqpoint{3.993190in}{1.723058in}}%
\pgfpathlineto{\pgfqpoint{3.994094in}{1.443228in}}%
\pgfpathlineto{\pgfqpoint{3.993940in}{1.778456in}}%
\pgfpathlineto{\pgfqpoint{3.994315in}{1.642966in}}%
\pgfpathlineto{\pgfqpoint{3.994403in}{1.532389in}}%
\pgfpathlineto{\pgfqpoint{3.994557in}{1.789055in}}%
\pgfpathlineto{\pgfqpoint{3.995593in}{1.890453in}}%
\pgfpathlineto{\pgfqpoint{3.995064in}{1.566480in}}%
\pgfpathlineto{\pgfqpoint{3.995637in}{1.885755in}}%
\pgfpathlineto{\pgfqpoint{3.995703in}{1.636629in}}%
\pgfpathlineto{\pgfqpoint{3.996409in}{1.916459in}}%
\pgfpathlineto{\pgfqpoint{3.996739in}{1.780532in}}%
\pgfpathlineto{\pgfqpoint{3.996871in}{1.965628in}}%
\pgfpathlineto{\pgfqpoint{3.997621in}{1.635973in}}%
\pgfpathlineto{\pgfqpoint{3.997819in}{1.783045in}}%
\pgfpathlineto{\pgfqpoint{3.998238in}{1.587787in}}%
\pgfpathlineto{\pgfqpoint{3.998150in}{1.871550in}}%
\pgfpathlineto{\pgfqpoint{3.998899in}{1.700986in}}%
\pgfpathlineto{\pgfqpoint{3.999957in}{1.858657in}}%
\pgfpathlineto{\pgfqpoint{3.999715in}{1.604832in}}%
\pgfpathlineto{\pgfqpoint{4.000023in}{1.786214in}}%
\pgfpathlineto{\pgfqpoint{4.000508in}{1.566589in}}%
\pgfpathlineto{\pgfqpoint{4.000949in}{1.794955in}}%
\pgfpathlineto{\pgfqpoint{4.001126in}{1.783373in}}%
\pgfpathlineto{\pgfqpoint{4.001170in}{1.882586in}}%
\pgfpathlineto{\pgfqpoint{4.001456in}{1.651598in}}%
\pgfpathlineto{\pgfqpoint{4.002338in}{1.537197in}}%
\pgfpathlineto{\pgfqpoint{4.002404in}{1.806428in}}%
\pgfpathlineto{\pgfqpoint{4.002580in}{1.605379in}}%
\pgfpathlineto{\pgfqpoint{4.003572in}{1.473604in}}%
\pgfpathlineto{\pgfqpoint{4.003704in}{1.779002in}}%
\pgfpathlineto{\pgfqpoint{4.004189in}{1.807958in}}%
\pgfpathlineto{\pgfqpoint{4.004895in}{1.540256in}}%
\pgfpathlineto{\pgfqpoint{4.005975in}{1.811891in}}%
\pgfpathlineto{\pgfqpoint{4.005556in}{1.456886in}}%
\pgfpathlineto{\pgfqpoint{4.006019in}{1.702188in}}%
\pgfpathlineto{\pgfqpoint{4.006041in}{1.576970in}}%
\pgfpathlineto{\pgfqpoint{4.006790in}{1.825331in}}%
\pgfpathlineto{\pgfqpoint{4.007099in}{1.757258in}}%
\pgfpathlineto{\pgfqpoint{4.007474in}{1.925200in}}%
\pgfpathlineto{\pgfqpoint{4.007716in}{1.599041in}}%
\pgfpathlineto{\pgfqpoint{4.008091in}{1.701423in}}%
\pgfpathlineto{\pgfqpoint{4.008972in}{1.562546in}}%
\pgfpathlineto{\pgfqpoint{4.008245in}{1.888050in}}%
\pgfpathlineto{\pgfqpoint{4.009127in}{1.636519in}}%
\pgfpathlineto{\pgfqpoint{4.009876in}{1.787525in}}%
\pgfpathlineto{\pgfqpoint{4.010030in}{1.483984in}}%
\pgfpathlineto{\pgfqpoint{4.010207in}{1.530532in}}%
\pgfpathlineto{\pgfqpoint{4.010339in}{1.798889in}}%
\pgfpathlineto{\pgfqpoint{4.010978in}{1.432520in}}%
\pgfpathlineto{\pgfqpoint{4.011353in}{1.619365in}}%
\pgfpathlineto{\pgfqpoint{4.011485in}{1.580466in}}%
\pgfpathlineto{\pgfqpoint{4.011838in}{1.869037in}}%
\pgfpathlineto{\pgfqpoint{4.012345in}{1.656515in}}%
\pgfpathlineto{\pgfqpoint{4.012852in}{1.836695in}}%
\pgfpathlineto{\pgfqpoint{4.013116in}{1.556974in}}%
\pgfpathlineto{\pgfqpoint{4.013469in}{1.773648in}}%
\pgfpathlineto{\pgfqpoint{4.013888in}{1.572490in}}%
\pgfpathlineto{\pgfqpoint{4.014020in}{1.878762in}}%
\pgfpathlineto{\pgfqpoint{4.014593in}{1.717376in}}%
\pgfpathlineto{\pgfqpoint{4.015100in}{1.887066in}}%
\pgfpathlineto{\pgfqpoint{4.015299in}{1.499281in}}%
\pgfpathlineto{\pgfqpoint{4.015541in}{1.693666in}}%
\pgfpathlineto{\pgfqpoint{4.016533in}{1.497752in}}%
\pgfpathlineto{\pgfqpoint{4.015629in}{1.756384in}}%
\pgfpathlineto{\pgfqpoint{4.016643in}{1.589426in}}%
\pgfpathlineto{\pgfqpoint{4.017106in}{1.484094in}}%
\pgfpathlineto{\pgfqpoint{4.017789in}{1.807521in}}%
\pgfpathlineto{\pgfqpoint{4.018054in}{1.479941in}}%
\pgfpathlineto{\pgfqpoint{4.018891in}{1.647118in}}%
\pgfpathlineto{\pgfqpoint{4.019619in}{1.872206in}}%
\pgfpathlineto{\pgfqpoint{4.019090in}{1.534356in}}%
\pgfpathlineto{\pgfqpoint{4.020016in}{1.748845in}}%
\pgfpathlineto{\pgfqpoint{4.020545in}{1.540912in}}%
\pgfpathlineto{\pgfqpoint{4.020126in}{1.848058in}}%
\pgfpathlineto{\pgfqpoint{4.021140in}{1.611825in}}%
\pgfpathlineto{\pgfqpoint{4.021426in}{1.795938in}}%
\pgfpathlineto{\pgfqpoint{4.021757in}{1.590519in}}%
\pgfpathlineto{\pgfqpoint{4.022264in}{1.723386in}}%
\pgfpathlineto{\pgfqpoint{4.023256in}{1.543643in}}%
\pgfpathlineto{\pgfqpoint{4.022572in}{1.807193in}}%
\pgfpathlineto{\pgfqpoint{4.023366in}{1.712459in}}%
\pgfpathlineto{\pgfqpoint{4.023410in}{1.757695in}}%
\pgfpathlineto{\pgfqpoint{4.023917in}{1.542551in}}%
\pgfpathlineto{\pgfqpoint{4.024027in}{1.555663in}}%
\pgfpathlineto{\pgfqpoint{4.024159in}{1.405204in}}%
\pgfpathlineto{\pgfqpoint{4.024490in}{1.704155in}}%
\pgfpathlineto{\pgfqpoint{4.025129in}{1.503106in}}%
\pgfpathlineto{\pgfqpoint{4.025460in}{1.737809in}}%
\pgfpathlineto{\pgfqpoint{4.025636in}{1.408481in}}%
\pgfpathlineto{\pgfqpoint{4.026275in}{1.663727in}}%
\pgfpathlineto{\pgfqpoint{4.026297in}{1.662634in}}%
\pgfpathlineto{\pgfqpoint{4.027422in}{1.341829in}}%
\pgfpathlineto{\pgfqpoint{4.026606in}{1.666349in}}%
\pgfpathlineto{\pgfqpoint{4.027444in}{1.476882in}}%
\pgfpathlineto{\pgfqpoint{4.027510in}{1.709946in}}%
\pgfpathlineto{\pgfqpoint{4.027620in}{1.332979in}}%
\pgfpathlineto{\pgfqpoint{4.028546in}{1.462240in}}%
\pgfpathlineto{\pgfqpoint{4.028722in}{1.346965in}}%
\pgfpathlineto{\pgfqpoint{4.029097in}{1.775287in}}%
\pgfpathlineto{\pgfqpoint{4.029604in}{1.518731in}}%
\pgfpathlineto{\pgfqpoint{4.029824in}{1.664929in}}%
\pgfpathlineto{\pgfqpoint{4.030045in}{1.387065in}}%
\pgfpathlineto{\pgfqpoint{4.030684in}{1.552603in}}%
\pgfpathlineto{\pgfqpoint{4.031499in}{1.387612in}}%
\pgfpathlineto{\pgfqpoint{4.031081in}{1.660449in}}%
\pgfpathlineto{\pgfqpoint{4.031786in}{1.520370in}}%
\pgfpathlineto{\pgfqpoint{4.032161in}{1.392856in}}%
\pgfpathlineto{\pgfqpoint{4.031874in}{1.624063in}}%
\pgfpathlineto{\pgfqpoint{4.032183in}{1.508897in}}%
\pgfpathlineto{\pgfqpoint{4.032646in}{1.292660in}}%
\pgfpathlineto{\pgfqpoint{4.033307in}{1.716939in}}%
\pgfpathlineto{\pgfqpoint{4.034365in}{1.313857in}}%
\pgfpathlineto{\pgfqpoint{4.034431in}{1.479286in}}%
\pgfpathlineto{\pgfqpoint{4.034762in}{1.714972in}}%
\pgfpathlineto{\pgfqpoint{4.034497in}{1.398757in}}%
\pgfpathlineto{\pgfqpoint{4.035599in}{1.714208in}}%
\pgfpathlineto{\pgfqpoint{4.035908in}{1.463661in}}%
\pgfpathlineto{\pgfqpoint{4.036591in}{1.729723in}}%
\pgfpathlineto{\pgfqpoint{4.036723in}{1.529657in}}%
\pgfpathlineto{\pgfqpoint{4.037517in}{1.742726in}}%
\pgfpathlineto{\pgfqpoint{4.037451in}{1.449128in}}%
\pgfpathlineto{\pgfqpoint{4.037825in}{1.534793in}}%
\pgfpathlineto{\pgfqpoint{4.038310in}{1.689513in}}%
\pgfpathlineto{\pgfqpoint{4.038487in}{1.435142in}}%
\pgfpathlineto{\pgfqpoint{4.038905in}{1.602428in}}%
\pgfpathlineto{\pgfqpoint{4.039875in}{1.315933in}}%
\pgfpathlineto{\pgfqpoint{4.039324in}{1.624828in}}%
\pgfpathlineto{\pgfqpoint{4.040030in}{1.316917in}}%
\pgfpathlineto{\pgfqpoint{4.040140in}{1.590519in}}%
\pgfpathlineto{\pgfqpoint{4.041066in}{1.266654in}}%
\pgfpathlineto{\pgfqpoint{4.041132in}{1.305990in}}%
\pgfpathlineto{\pgfqpoint{4.041793in}{1.597293in}}%
\pgfpathlineto{\pgfqpoint{4.042278in}{1.492288in}}%
\pgfpathlineto{\pgfqpoint{4.042344in}{1.244036in}}%
\pgfpathlineto{\pgfqpoint{4.043116in}{1.652254in}}%
\pgfpathlineto{\pgfqpoint{4.043402in}{1.377122in}}%
\pgfpathlineto{\pgfqpoint{4.043424in}{1.258241in}}%
\pgfpathlineto{\pgfqpoint{4.044482in}{1.536104in}}%
\pgfpathlineto{\pgfqpoint{4.044636in}{1.231580in}}%
\pgfpathlineto{\pgfqpoint{4.044813in}{1.558176in}}%
\pgfpathlineto{\pgfqpoint{4.045606in}{1.402690in}}%
\pgfpathlineto{\pgfqpoint{4.045893in}{1.569758in}}%
\pgfpathlineto{\pgfqpoint{4.046510in}{1.298997in}}%
\pgfpathlineto{\pgfqpoint{4.046708in}{1.489010in}}%
\pgfpathlineto{\pgfqpoint{4.046774in}{1.265015in}}%
\pgfpathlineto{\pgfqpoint{4.047568in}{1.678478in}}%
\pgfpathlineto{\pgfqpoint{4.047810in}{1.401270in}}%
\pgfpathlineto{\pgfqpoint{4.047987in}{1.662634in}}%
\pgfpathlineto{\pgfqpoint{4.048802in}{1.293534in}}%
\pgfpathlineto{\pgfqpoint{4.048935in}{1.475571in}}%
\pgfpathlineto{\pgfqpoint{4.049508in}{1.585164in}}%
\pgfpathlineto{\pgfqpoint{4.049816in}{1.315824in}}%
\pgfpathlineto{\pgfqpoint{4.049838in}{1.408481in}}%
\pgfpathlineto{\pgfqpoint{4.050852in}{1.214862in}}%
\pgfpathlineto{\pgfqpoint{4.050632in}{1.565824in}}%
\pgfpathlineto{\pgfqpoint{4.050962in}{1.313857in}}%
\pgfpathlineto{\pgfqpoint{4.051712in}{1.107454in}}%
\pgfpathlineto{\pgfqpoint{4.051227in}{1.487590in}}%
\pgfpathlineto{\pgfqpoint{4.051932in}{1.381930in}}%
\pgfpathlineto{\pgfqpoint{4.052329in}{1.485186in}}%
\pgfpathlineto{\pgfqpoint{4.052483in}{1.153892in}}%
\pgfpathlineto{\pgfqpoint{4.053012in}{1.338661in}}%
\pgfpathlineto{\pgfqpoint{4.053211in}{1.205356in}}%
\pgfpathlineto{\pgfqpoint{4.053343in}{1.407607in}}%
\pgfpathlineto{\pgfqpoint{4.053828in}{1.316807in}}%
\pgfpathlineto{\pgfqpoint{4.053894in}{1.493163in}}%
\pgfpathlineto{\pgfqpoint{4.054798in}{1.198363in}}%
\pgfpathlineto{\pgfqpoint{4.054908in}{1.373516in}}%
\pgfpathlineto{\pgfqpoint{4.055635in}{1.221855in}}%
\pgfpathlineto{\pgfqpoint{4.054952in}{1.499281in}}%
\pgfpathlineto{\pgfqpoint{4.056010in}{1.365977in}}%
\pgfpathlineto{\pgfqpoint{4.056627in}{1.475899in}}%
\pgfpathlineto{\pgfqpoint{4.056649in}{1.216829in}}%
\pgfpathlineto{\pgfqpoint{4.057112in}{1.420610in}}%
\pgfpathlineto{\pgfqpoint{4.057377in}{1.183175in}}%
\pgfpathlineto{\pgfqpoint{4.057333in}{1.463114in}}%
\pgfpathlineto{\pgfqpoint{4.058236in}{1.278237in}}%
\pgfpathlineto{\pgfqpoint{4.059140in}{1.457870in}}%
\pgfpathlineto{\pgfqpoint{4.058942in}{1.140561in}}%
\pgfpathlineto{\pgfqpoint{4.059383in}{1.442572in}}%
\pgfpathlineto{\pgfqpoint{4.060396in}{1.490212in}}%
\pgfpathlineto{\pgfqpoint{4.060507in}{1.212131in}}%
\pgfpathlineto{\pgfqpoint{4.061080in}{1.542988in}}%
\pgfpathlineto{\pgfqpoint{4.061609in}{1.326095in}}%
\pgfpathlineto{\pgfqpoint{4.061631in}{1.189950in}}%
\pgfpathlineto{\pgfqpoint{4.061829in}{1.453936in}}%
\pgfpathlineto{\pgfqpoint{4.062711in}{1.315059in}}%
\pgfpathlineto{\pgfqpoint{4.063570in}{1.165037in}}%
\pgfpathlineto{\pgfqpoint{4.063152in}{1.478193in}}%
\pgfpathlineto{\pgfqpoint{4.063747in}{1.380400in}}%
\pgfpathlineto{\pgfqpoint{4.063769in}{1.465628in}}%
\pgfpathlineto{\pgfqpoint{4.064518in}{1.070850in}}%
\pgfpathlineto{\pgfqpoint{4.064805in}{1.243162in}}%
\pgfpathlineto{\pgfqpoint{4.065180in}{1.122205in}}%
\pgfpathlineto{\pgfqpoint{4.065554in}{1.414054in}}%
\pgfpathlineto{\pgfqpoint{4.065797in}{1.258896in}}%
\pgfpathlineto{\pgfqpoint{4.066436in}{1.361497in}}%
\pgfpathlineto{\pgfqpoint{4.066260in}{1.103302in}}%
\pgfpathlineto{\pgfqpoint{4.066921in}{1.311016in}}%
\pgfpathlineto{\pgfqpoint{4.067847in}{1.421593in}}%
\pgfpathlineto{\pgfqpoint{4.067009in}{1.115867in}}%
\pgfpathlineto{\pgfqpoint{4.067957in}{1.335383in}}%
\pgfpathlineto{\pgfqpoint{4.068905in}{1.131820in}}%
\pgfpathlineto{\pgfqpoint{4.068486in}{1.494692in}}%
\pgfpathlineto{\pgfqpoint{4.069081in}{1.209180in}}%
\pgfpathlineto{\pgfqpoint{4.069279in}{1.396572in}}%
\pgfpathlineto{\pgfqpoint{4.069610in}{1.090299in}}%
\pgfpathlineto{\pgfqpoint{4.070161in}{1.157825in}}%
\pgfpathlineto{\pgfqpoint{4.070910in}{1.338224in}}%
\pgfpathlineto{\pgfqpoint{4.070668in}{1.048887in}}%
\pgfpathlineto{\pgfqpoint{4.070955in}{1.318446in}}%
\pgfpathlineto{\pgfqpoint{4.071065in}{1.071833in}}%
\pgfpathlineto{\pgfqpoint{4.072057in}{1.194430in}}%
\pgfpathlineto{\pgfqpoint{4.072079in}{1.265234in}}%
\pgfpathlineto{\pgfqpoint{4.072762in}{0.999936in}}%
\pgfpathlineto{\pgfqpoint{4.073115in}{1.247861in}}%
\pgfpathlineto{\pgfqpoint{4.073776in}{0.996658in}}%
\pgfpathlineto{\pgfqpoint{4.074239in}{1.134115in}}%
\pgfpathlineto{\pgfqpoint{4.074614in}{0.999171in}}%
\pgfpathlineto{\pgfqpoint{4.075275in}{1.240321in}}%
\pgfpathlineto{\pgfqpoint{4.075319in}{1.191152in}}%
\pgfpathlineto{\pgfqpoint{4.075385in}{1.349150in}}%
\pgfpathlineto{\pgfqpoint{4.075826in}{0.998406in}}%
\pgfpathlineto{\pgfqpoint{4.076421in}{1.220872in}}%
\pgfpathlineto{\pgfqpoint{4.076994in}{1.416895in}}%
\pgfpathlineto{\pgfqpoint{4.076597in}{1.095653in}}%
\pgfpathlineto{\pgfqpoint{4.077479in}{1.277690in}}%
\pgfpathlineto{\pgfqpoint{4.078537in}{1.164163in}}%
\pgfpathlineto{\pgfqpoint{4.078449in}{1.433940in}}%
\pgfpathlineto{\pgfqpoint{4.078581in}{1.278674in}}%
\pgfpathlineto{\pgfqpoint{4.078912in}{1.013267in}}%
\pgfpathlineto{\pgfqpoint{4.079154in}{1.371550in}}%
\pgfpathlineto{\pgfqpoint{4.079749in}{1.109858in}}%
\pgfpathlineto{\pgfqpoint{4.080543in}{1.364557in}}%
\pgfpathlineto{\pgfqpoint{4.079815in}{1.087349in}}%
\pgfpathlineto{\pgfqpoint{4.080829in}{1.150614in}}%
\pgfpathlineto{\pgfqpoint{4.080895in}{1.067681in}}%
\pgfpathlineto{\pgfqpoint{4.081358in}{1.388595in}}%
\pgfpathlineto{\pgfqpoint{4.081887in}{1.220872in}}%
\pgfpathlineto{\pgfqpoint{4.081931in}{1.396899in}}%
\pgfpathlineto{\pgfqpoint{4.082879in}{1.038398in}}%
\pgfpathlineto{\pgfqpoint{4.082967in}{1.234749in}}%
\pgfpathlineto{\pgfqpoint{4.083276in}{1.007694in}}%
\pgfpathlineto{\pgfqpoint{4.083629in}{1.320632in}}%
\pgfpathlineto{\pgfqpoint{4.084070in}{1.112152in}}%
\pgfpathlineto{\pgfqpoint{4.084841in}{1.473058in}}%
\pgfpathlineto{\pgfqpoint{4.084951in}{1.071942in}}%
\pgfpathlineto{\pgfqpoint{4.085194in}{1.170937in}}%
\pgfpathlineto{\pgfqpoint{4.085789in}{1.039163in}}%
\pgfpathlineto{\pgfqpoint{4.085370in}{1.328499in}}%
\pgfpathlineto{\pgfqpoint{4.086119in}{1.149303in}}%
\pgfpathlineto{\pgfqpoint{4.086847in}{1.438202in}}%
\pgfpathlineto{\pgfqpoint{4.086164in}{1.088223in}}%
\pgfpathlineto{\pgfqpoint{4.087222in}{1.304023in}}%
\pgfpathlineto{\pgfqpoint{4.087905in}{1.131165in}}%
\pgfpathlineto{\pgfqpoint{4.088191in}{1.441808in}}%
\pgfpathlineto{\pgfqpoint{4.088346in}{1.134880in}}%
\pgfpathlineto{\pgfqpoint{4.089161in}{1.331012in}}%
\pgfpathlineto{\pgfqpoint{4.089382in}{1.048560in}}%
\pgfpathlineto{\pgfqpoint{4.089448in}{1.161868in}}%
\pgfpathlineto{\pgfqpoint{4.090307in}{1.316152in}}%
\pgfpathlineto{\pgfqpoint{4.089712in}{1.071615in}}%
\pgfpathlineto{\pgfqpoint{4.090594in}{1.249172in}}%
\pgfpathlineto{\pgfqpoint{4.091167in}{1.117943in}}%
\pgfpathlineto{\pgfqpoint{4.090991in}{1.407170in}}%
\pgfpathlineto{\pgfqpoint{4.091696in}{1.221527in}}%
\pgfpathlineto{\pgfqpoint{4.091806in}{1.407280in}}%
\pgfpathlineto{\pgfqpoint{4.092357in}{1.091501in}}%
\pgfpathlineto{\pgfqpoint{4.092754in}{1.238901in}}%
\pgfpathlineto{\pgfqpoint{4.093415in}{1.042441in}}%
\pgfpathlineto{\pgfqpoint{4.093658in}{1.379417in}}%
\pgfpathlineto{\pgfqpoint{4.093856in}{1.197052in}}%
\pgfpathlineto{\pgfqpoint{4.094055in}{1.054788in}}%
\pgfpathlineto{\pgfqpoint{4.094694in}{1.435907in}}%
\pgfpathlineto{\pgfqpoint{4.094826in}{1.370457in}}%
\pgfpathlineto{\pgfqpoint{4.094848in}{1.426510in}}%
\pgfpathlineto{\pgfqpoint{4.095135in}{1.158153in}}%
\pgfpathlineto{\pgfqpoint{4.095884in}{1.323145in}}%
\pgfpathlineto{\pgfqpoint{4.096369in}{1.130181in}}%
\pgfpathlineto{\pgfqpoint{4.096567in}{1.433285in}}%
\pgfpathlineto{\pgfqpoint{4.097008in}{1.257804in}}%
\pgfpathlineto{\pgfqpoint{4.097096in}{1.495894in}}%
\pgfpathlineto{\pgfqpoint{4.097846in}{1.177275in}}%
\pgfpathlineto{\pgfqpoint{4.098132in}{1.280859in}}%
\pgfpathlineto{\pgfqpoint{4.098595in}{1.126466in}}%
\pgfpathlineto{\pgfqpoint{4.098243in}{1.351226in}}%
\pgfpathlineto{\pgfqpoint{4.099212in}{1.280313in}}%
\pgfpathlineto{\pgfqpoint{4.099808in}{1.441589in}}%
\pgfpathlineto{\pgfqpoint{4.099455in}{1.067135in}}%
\pgfpathlineto{\pgfqpoint{4.100314in}{1.336694in}}%
\pgfpathlineto{\pgfqpoint{4.101593in}{1.032279in}}%
\pgfpathlineto{\pgfqpoint{4.100425in}{1.392747in}}%
\pgfpathlineto{\pgfqpoint{4.101637in}{1.172686in}}%
\pgfpathlineto{\pgfqpoint{4.102431in}{1.286869in}}%
\pgfpathlineto{\pgfqpoint{4.102144in}{1.061453in}}%
\pgfpathlineto{\pgfqpoint{4.102739in}{1.198145in}}%
\pgfpathlineto{\pgfqpoint{4.102805in}{1.035448in}}%
\pgfpathlineto{\pgfqpoint{4.102893in}{1.347402in}}%
\pgfpathlineto{\pgfqpoint{4.103841in}{1.158372in}}%
\pgfpathlineto{\pgfqpoint{4.104877in}{1.413836in}}%
\pgfpathlineto{\pgfqpoint{4.104591in}{1.043752in}}%
\pgfpathlineto{\pgfqpoint{4.104987in}{1.351226in}}%
\pgfpathlineto{\pgfqpoint{4.105671in}{1.472730in}}%
\pgfpathlineto{\pgfqpoint{4.106134in}{1.069648in}}%
\pgfpathlineto{\pgfqpoint{4.106552in}{1.408481in}}%
\pgfpathlineto{\pgfqpoint{4.107302in}{1.198909in}}%
\pgfpathlineto{\pgfqpoint{4.107434in}{1.070631in}}%
\pgfpathlineto{\pgfqpoint{4.107610in}{1.388923in}}%
\pgfpathlineto{\pgfqpoint{4.108382in}{1.244036in}}%
\pgfpathlineto{\pgfqpoint{4.108801in}{1.474478in}}%
\pgfpathlineto{\pgfqpoint{4.108558in}{1.162305in}}%
\pgfpathlineto{\pgfqpoint{4.109528in}{1.324237in}}%
\pgfpathlineto{\pgfqpoint{4.109550in}{1.324237in}}%
\pgfpathlineto{\pgfqpoint{4.110366in}{1.100352in}}%
\pgfpathlineto{\pgfqpoint{4.110189in}{1.470326in}}%
\pgfpathlineto{\pgfqpoint{4.110652in}{1.279220in}}%
\pgfpathlineto{\pgfqpoint{4.110740in}{1.425199in}}%
\pgfpathlineto{\pgfqpoint{4.110983in}{1.108874in}}%
\pgfpathlineto{\pgfqpoint{4.111688in}{1.182083in}}%
\pgfpathlineto{\pgfqpoint{4.112195in}{1.057301in}}%
\pgfpathlineto{\pgfqpoint{4.111975in}{1.324237in}}%
\pgfpathlineto{\pgfqpoint{4.112790in}{1.109639in}}%
\pgfpathlineto{\pgfqpoint{4.113628in}{1.322271in}}%
\pgfpathlineto{\pgfqpoint{4.112967in}{0.997095in}}%
\pgfpathlineto{\pgfqpoint{4.113892in}{1.203826in}}%
\pgfpathlineto{\pgfqpoint{4.114972in}{0.965845in}}%
\pgfpathlineto{\pgfqpoint{4.114135in}{1.288726in}}%
\pgfpathlineto{\pgfqpoint{4.115061in}{0.997860in}}%
\pgfpathlineto{\pgfqpoint{4.115303in}{1.326095in}}%
\pgfpathlineto{\pgfqpoint{4.116008in}{0.897554in}}%
\pgfpathlineto{\pgfqpoint{4.116207in}{1.105924in}}%
\pgfpathlineto{\pgfqpoint{4.116515in}{0.900286in}}%
\pgfpathlineto{\pgfqpoint{4.117287in}{1.227428in}}%
\pgfpathlineto{\pgfqpoint{4.117309in}{1.096527in}}%
\pgfpathlineto{\pgfqpoint{4.118014in}{1.206340in}}%
\pgfpathlineto{\pgfqpoint{4.117463in}{0.877449in}}%
\pgfpathlineto{\pgfqpoint{4.118323in}{1.121877in}}%
\pgfpathlineto{\pgfqpoint{4.118433in}{0.914381in}}%
\pgfpathlineto{\pgfqpoint{4.119381in}{1.173341in}}%
\pgfpathlineto{\pgfqpoint{4.119425in}{1.006274in}}%
\pgfpathlineto{\pgfqpoint{4.119469in}{1.280531in}}%
\pgfpathlineto{\pgfqpoint{4.120064in}{0.885753in}}%
\pgfpathlineto{\pgfqpoint{4.120527in}{1.092703in}}%
\pgfpathlineto{\pgfqpoint{4.121034in}{1.241851in}}%
\pgfpathlineto{\pgfqpoint{4.120703in}{0.884005in}}%
\pgfpathlineto{\pgfqpoint{4.121276in}{1.045391in}}%
\pgfpathlineto{\pgfqpoint{4.122246in}{0.883677in}}%
\pgfpathlineto{\pgfqpoint{4.121541in}{1.154329in}}%
\pgfpathlineto{\pgfqpoint{4.122379in}{1.059923in}}%
\pgfpathlineto{\pgfqpoint{4.122599in}{0.865976in}}%
\pgfpathlineto{\pgfqpoint{4.122511in}{1.159246in}}%
\pgfpathlineto{\pgfqpoint{4.123503in}{0.944975in}}%
\pgfpathlineto{\pgfqpoint{4.123525in}{1.084071in}}%
\pgfpathlineto{\pgfqpoint{4.124561in}{0.817681in}}%
\pgfpathlineto{\pgfqpoint{4.124583in}{0.927165in}}%
\pgfpathlineto{\pgfqpoint{4.124649in}{0.858437in}}%
\pgfpathlineto{\pgfqpoint{4.125310in}{1.173778in}}%
\pgfpathlineto{\pgfqpoint{4.125619in}{1.006820in}}%
\pgfpathlineto{\pgfqpoint{4.126721in}{1.237808in}}%
\pgfpathlineto{\pgfqpoint{4.126170in}{0.906405in}}%
\pgfpathlineto{\pgfqpoint{4.126743in}{1.231580in}}%
\pgfpathlineto{\pgfqpoint{4.126941in}{1.095325in}}%
\pgfpathlineto{\pgfqpoint{4.127360in}{1.484421in}}%
\pgfpathlineto{\pgfqpoint{4.127779in}{1.315059in}}%
\pgfpathlineto{\pgfqpoint{4.128264in}{1.398975in}}%
\pgfpathlineto{\pgfqpoint{4.127889in}{1.135098in}}%
\pgfpathlineto{\pgfqpoint{4.128771in}{1.200330in}}%
\pgfpathlineto{\pgfqpoint{4.129498in}{1.026925in}}%
\pgfpathlineto{\pgfqpoint{4.129057in}{1.331012in}}%
\pgfpathlineto{\pgfqpoint{4.129873in}{1.174434in}}%
\pgfpathlineto{\pgfqpoint{4.130071in}{1.330575in}}%
\pgfpathlineto{\pgfqpoint{4.130358in}{1.030203in}}%
\pgfpathlineto{\pgfqpoint{4.130975in}{1.184049in}}%
\pgfpathlineto{\pgfqpoint{4.131724in}{1.001138in}}%
\pgfpathlineto{\pgfqpoint{4.131548in}{1.348167in}}%
\pgfpathlineto{\pgfqpoint{4.132077in}{1.185251in}}%
\pgfpathlineto{\pgfqpoint{4.133091in}{0.937873in}}%
\pgfpathlineto{\pgfqpoint{4.132452in}{1.311890in}}%
\pgfpathlineto{\pgfqpoint{4.133201in}{1.105378in}}%
\pgfpathlineto{\pgfqpoint{4.134149in}{1.380291in}}%
\pgfpathlineto{\pgfqpoint{4.133444in}{1.037305in}}%
\pgfpathlineto{\pgfqpoint{4.134347in}{1.230597in}}%
\pgfpathlineto{\pgfqpoint{4.135207in}{1.055880in}}%
\pgfpathlineto{\pgfqpoint{4.135251in}{1.273975in}}%
\pgfpathlineto{\pgfqpoint{4.135471in}{1.157388in}}%
\pgfpathlineto{\pgfqpoint{4.135846in}{0.936016in}}%
\pgfpathlineto{\pgfqpoint{4.135912in}{1.222948in}}%
\pgfpathlineto{\pgfqpoint{4.136618in}{1.025942in}}%
\pgfpathlineto{\pgfqpoint{4.137345in}{1.162961in}}%
\pgfpathlineto{\pgfqpoint{4.137279in}{0.888376in}}%
\pgfpathlineto{\pgfqpoint{4.137698in}{1.035994in}}%
\pgfpathlineto{\pgfqpoint{4.138535in}{0.845653in}}%
\pgfpathlineto{\pgfqpoint{4.138447in}{1.128324in}}%
\pgfpathlineto{\pgfqpoint{4.138756in}{1.010972in}}%
\pgfpathlineto{\pgfqpoint{4.139814in}{1.306209in}}%
\pgfpathlineto{\pgfqpoint{4.139086in}{0.835382in}}%
\pgfpathlineto{\pgfqpoint{4.139858in}{1.101991in}}%
\pgfpathlineto{\pgfqpoint{4.140122in}{0.907279in}}%
\pgfpathlineto{\pgfqpoint{4.140475in}{1.206777in}}%
\pgfpathlineto{\pgfqpoint{4.140982in}{1.032279in}}%
\pgfpathlineto{\pgfqpoint{4.141996in}{1.224696in}}%
\pgfpathlineto{\pgfqpoint{4.141313in}{0.924324in}}%
\pgfpathlineto{\pgfqpoint{4.142084in}{1.068992in}}%
\pgfpathlineto{\pgfqpoint{4.142569in}{0.946942in}}%
\pgfpathlineto{\pgfqpoint{4.142988in}{1.288070in}}%
\pgfpathlineto{\pgfqpoint{4.143142in}{1.141326in}}%
\pgfpathlineto{\pgfqpoint{4.144178in}{1.304460in}}%
\pgfpathlineto{\pgfqpoint{4.143473in}{1.029219in}}%
\pgfpathlineto{\pgfqpoint{4.144266in}{1.211584in}}%
\pgfpathlineto{\pgfqpoint{4.144817in}{1.000045in}}%
\pgfpathlineto{\pgfqpoint{4.144685in}{1.304242in}}%
\pgfpathlineto{\pgfqpoint{4.145390in}{1.165365in}}%
\pgfpathlineto{\pgfqpoint{4.146360in}{1.282061in}}%
\pgfpathlineto{\pgfqpoint{4.145501in}{1.021243in}}%
\pgfpathlineto{\pgfqpoint{4.146492in}{1.134880in}}%
\pgfpathlineto{\pgfqpoint{4.146779in}{0.961802in}}%
\pgfpathlineto{\pgfqpoint{4.146669in}{1.284902in}}%
\pgfpathlineto{\pgfqpoint{4.147617in}{1.042550in}}%
\pgfpathlineto{\pgfqpoint{4.147969in}{1.200111in}}%
\pgfpathlineto{\pgfqpoint{4.147749in}{0.917878in}}%
\pgfpathlineto{\pgfqpoint{4.148741in}{1.145915in}}%
\pgfpathlineto{\pgfqpoint{4.149248in}{0.939949in}}%
\pgfpathlineto{\pgfqpoint{4.149821in}{1.233656in}}%
\pgfpathlineto{\pgfqpoint{4.149865in}{1.070303in}}%
\pgfpathlineto{\pgfqpoint{4.150328in}{1.216392in}}%
\pgfpathlineto{\pgfqpoint{4.150460in}{0.985622in}}%
\pgfpathlineto{\pgfqpoint{4.150989in}{1.127668in}}%
\pgfpathlineto{\pgfqpoint{4.151628in}{0.914709in}}%
\pgfpathlineto{\pgfqpoint{4.152025in}{1.235841in}}%
\pgfpathlineto{\pgfqpoint{4.152113in}{1.059923in}}%
\pgfpathlineto{\pgfqpoint{4.152201in}{1.172576in}}%
\pgfpathlineto{\pgfqpoint{4.153105in}{0.906951in}}%
\pgfpathlineto{\pgfqpoint{4.153193in}{1.072926in}}%
\pgfpathlineto{\pgfqpoint{4.153546in}{0.864446in}}%
\pgfpathlineto{\pgfqpoint{4.154229in}{1.179788in}}%
\pgfpathlineto{\pgfqpoint{4.154295in}{1.072489in}}%
\pgfpathlineto{\pgfqpoint{4.155221in}{1.204919in}}%
\pgfpathlineto{\pgfqpoint{4.154802in}{0.921046in}}%
\pgfpathlineto{\pgfqpoint{4.155397in}{1.088223in}}%
\pgfpathlineto{\pgfqpoint{4.155794in}{1.319976in}}%
\pgfpathlineto{\pgfqpoint{4.155574in}{0.997314in}}%
\pgfpathlineto{\pgfqpoint{4.156367in}{1.094233in}}%
\pgfpathlineto{\pgfqpoint{4.156918in}{0.940933in}}%
\pgfpathlineto{\pgfqpoint{4.156830in}{1.278018in}}%
\pgfpathlineto{\pgfqpoint{4.157469in}{1.119801in}}%
\pgfpathlineto{\pgfqpoint{4.158087in}{1.197708in}}%
\pgfpathlineto{\pgfqpoint{4.157756in}{0.933284in}}%
\pgfpathlineto{\pgfqpoint{4.158549in}{1.086475in}}%
\pgfpathlineto{\pgfqpoint{4.159321in}{1.300199in}}%
\pgfpathlineto{\pgfqpoint{4.159100in}{1.002996in}}%
\pgfpathlineto{\pgfqpoint{4.159607in}{1.140999in}}%
\pgfpathlineto{\pgfqpoint{4.160026in}{0.991632in}}%
\pgfpathlineto{\pgfqpoint{4.160643in}{1.355488in}}%
\pgfpathlineto{\pgfqpoint{4.160665in}{1.289600in}}%
\pgfpathlineto{\pgfqpoint{4.161591in}{1.463114in}}%
\pgfpathlineto{\pgfqpoint{4.161128in}{1.070303in}}%
\pgfpathlineto{\pgfqpoint{4.161723in}{1.251248in}}%
\pgfpathlineto{\pgfqpoint{4.161856in}{1.085710in}}%
\pgfpathlineto{\pgfqpoint{4.161768in}{1.370894in}}%
\pgfpathlineto{\pgfqpoint{4.162804in}{1.154985in}}%
\pgfpathlineto{\pgfqpoint{4.163531in}{1.485623in}}%
\pgfpathlineto{\pgfqpoint{4.162914in}{1.146462in}}%
\pgfpathlineto{\pgfqpoint{4.163928in}{1.300199in}}%
\pgfpathlineto{\pgfqpoint{4.164898in}{1.084180in}}%
\pgfpathlineto{\pgfqpoint{4.164038in}{1.459399in}}%
\pgfpathlineto{\pgfqpoint{4.165074in}{1.145915in}}%
\pgfpathlineto{\pgfqpoint{4.165779in}{1.328608in}}%
\pgfpathlineto{\pgfqpoint{4.165625in}{1.055116in}}%
\pgfpathlineto{\pgfqpoint{4.166176in}{1.206012in}}%
\pgfpathlineto{\pgfqpoint{4.166485in}{0.981798in}}%
\pgfpathlineto{\pgfqpoint{4.166639in}{1.279985in}}%
\pgfpathlineto{\pgfqpoint{4.167278in}{1.096199in}}%
\pgfpathlineto{\pgfqpoint{4.168314in}{1.275177in}}%
\pgfpathlineto{\pgfqpoint{4.167454in}{0.981470in}}%
\pgfpathlineto{\pgfqpoint{4.168380in}{1.173450in}}%
\pgfpathlineto{\pgfqpoint{4.168887in}{0.990212in}}%
\pgfpathlineto{\pgfqpoint{4.168490in}{1.354067in}}%
\pgfpathlineto{\pgfqpoint{4.169460in}{1.226226in}}%
\pgfpathlineto{\pgfqpoint{4.169725in}{1.397009in}}%
\pgfpathlineto{\pgfqpoint{4.170320in}{1.094779in}}%
\pgfpathlineto{\pgfqpoint{4.170430in}{1.224587in}}%
\pgfpathlineto{\pgfqpoint{4.170717in}{1.060032in}}%
\pgfpathlineto{\pgfqpoint{4.171113in}{1.284137in}}%
\pgfpathlineto{\pgfqpoint{4.171554in}{1.128761in}}%
\pgfpathlineto{\pgfqpoint{4.172524in}{1.367835in}}%
\pgfpathlineto{\pgfqpoint{4.171664in}{1.079263in}}%
\pgfpathlineto{\pgfqpoint{4.172700in}{1.224259in}}%
\pgfpathlineto{\pgfqpoint{4.172987in}{1.299980in}}%
\pgfpathlineto{\pgfqpoint{4.172965in}{1.177384in}}%
\pgfpathlineto{\pgfqpoint{4.173053in}{1.293862in}}%
\pgfpathlineto{\pgfqpoint{4.173119in}{1.423451in}}%
\pgfpathlineto{\pgfqpoint{4.173538in}{1.043752in}}%
\pgfpathlineto{\pgfqpoint{4.174133in}{1.153236in}}%
\pgfpathlineto{\pgfqpoint{4.175213in}{0.881929in}}%
\pgfpathlineto{\pgfqpoint{4.174309in}{1.257148in}}%
\pgfpathlineto{\pgfqpoint{4.175279in}{0.937218in}}%
\pgfpathlineto{\pgfqpoint{4.176227in}{1.230378in}}%
\pgfpathlineto{\pgfqpoint{4.175345in}{0.910557in}}%
\pgfpathlineto{\pgfqpoint{4.176403in}{1.075548in}}%
\pgfpathlineto{\pgfqpoint{4.176932in}{1.180006in}}%
\pgfpathlineto{\pgfqpoint{4.177065in}{1.000155in}}%
\pgfpathlineto{\pgfqpoint{4.177417in}{0.909573in}}%
\pgfpathlineto{\pgfqpoint{4.177241in}{1.201532in}}%
\pgfpathlineto{\pgfqpoint{4.178101in}{1.088005in}}%
\pgfpathlineto{\pgfqpoint{4.178167in}{1.203608in}}%
\pgfpathlineto{\pgfqpoint{4.178475in}{0.871330in}}%
\pgfpathlineto{\pgfqpoint{4.179181in}{1.161978in}}%
\pgfpathlineto{\pgfqpoint{4.180040in}{0.928258in}}%
\pgfpathlineto{\pgfqpoint{4.179820in}{1.228630in}}%
\pgfpathlineto{\pgfqpoint{4.180305in}{1.046374in}}%
\pgfpathlineto{\pgfqpoint{4.180768in}{1.183066in}}%
\pgfpathlineto{\pgfqpoint{4.180966in}{0.912633in}}%
\pgfpathlineto{\pgfqpoint{4.181407in}{1.026160in}}%
\pgfpathlineto{\pgfqpoint{4.181517in}{1.141982in}}%
\pgfpathlineto{\pgfqpoint{4.182311in}{0.852099in}}%
\pgfpathlineto{\pgfqpoint{4.182487in}{0.968468in}}%
\pgfpathlineto{\pgfqpoint{4.182531in}{0.829263in}}%
\pgfpathlineto{\pgfqpoint{4.182641in}{1.144495in}}%
\pgfpathlineto{\pgfqpoint{4.183567in}{1.014687in}}%
\pgfpathlineto{\pgfqpoint{4.184493in}{1.195085in}}%
\pgfpathlineto{\pgfqpoint{4.183854in}{0.893730in}}%
\pgfpathlineto{\pgfqpoint{4.184603in}{0.919954in}}%
\pgfpathlineto{\pgfqpoint{4.185463in}{0.795718in}}%
\pgfpathlineto{\pgfqpoint{4.184735in}{1.153673in}}%
\pgfpathlineto{\pgfqpoint{4.185661in}{0.945413in}}%
\pgfpathlineto{\pgfqpoint{4.185881in}{1.091064in}}%
\pgfpathlineto{\pgfqpoint{4.186344in}{0.821833in}}%
\pgfpathlineto{\pgfqpoint{4.186763in}{0.994691in}}%
\pgfpathlineto{\pgfqpoint{4.187402in}{0.839752in}}%
\pgfpathlineto{\pgfqpoint{4.187491in}{1.113682in}}%
\pgfpathlineto{\pgfqpoint{4.187887in}{0.954154in}}%
\pgfpathlineto{\pgfqpoint{4.188813in}{0.856470in}}%
\pgfpathlineto{\pgfqpoint{4.188042in}{1.122642in}}%
\pgfpathlineto{\pgfqpoint{4.188967in}{0.961802in}}%
\pgfpathlineto{\pgfqpoint{4.189011in}{1.094670in}}%
\pgfpathlineto{\pgfqpoint{4.189364in}{0.801182in}}%
\pgfpathlineto{\pgfqpoint{4.190047in}{0.962130in}}%
\pgfpathlineto{\pgfqpoint{4.190422in}{0.849696in}}%
\pgfpathlineto{\pgfqpoint{4.190819in}{1.129744in}}%
\pgfpathlineto{\pgfqpoint{4.191127in}{1.007039in}}%
\pgfpathlineto{\pgfqpoint{4.191767in}{1.201423in}}%
\pgfpathlineto{\pgfqpoint{4.191590in}{0.962130in}}%
\pgfpathlineto{\pgfqpoint{4.192296in}{1.158372in}}%
\pgfpathlineto{\pgfqpoint{4.193221in}{0.912086in}}%
\pgfpathlineto{\pgfqpoint{4.192825in}{1.227756in}}%
\pgfpathlineto{\pgfqpoint{4.193442in}{0.949237in}}%
\pgfpathlineto{\pgfqpoint{4.194522in}{1.133241in}}%
\pgfpathlineto{\pgfqpoint{4.194081in}{0.795718in}}%
\pgfpathlineto{\pgfqpoint{4.194566in}{1.028345in}}%
\pgfpathlineto{\pgfqpoint{4.195448in}{0.890452in}}%
\pgfpathlineto{\pgfqpoint{4.195139in}{1.200002in}}%
\pgfpathlineto{\pgfqpoint{4.195558in}{1.122860in}}%
\pgfpathlineto{\pgfqpoint{4.195580in}{1.273320in}}%
\pgfpathlineto{\pgfqpoint{4.195911in}{0.885644in}}%
\pgfpathlineto{\pgfqpoint{4.196638in}{1.016545in}}%
\pgfpathlineto{\pgfqpoint{4.197035in}{0.846855in}}%
\pgfpathlineto{\pgfqpoint{4.196770in}{1.129635in}}%
\pgfpathlineto{\pgfqpoint{4.197696in}{1.095325in}}%
\pgfpathlineto{\pgfqpoint{4.197718in}{1.113245in}}%
\pgfpathlineto{\pgfqpoint{4.197872in}{0.878214in}}%
\pgfpathlineto{\pgfqpoint{4.198556in}{0.929460in}}%
\pgfpathlineto{\pgfqpoint{4.198798in}{0.779328in}}%
\pgfpathlineto{\pgfqpoint{4.199085in}{1.099805in}}%
\pgfpathlineto{\pgfqpoint{4.199636in}{0.970981in}}%
\pgfpathlineto{\pgfqpoint{4.199702in}{1.137393in}}%
\pgfpathlineto{\pgfqpoint{4.200694in}{0.926400in}}%
\pgfpathlineto{\pgfqpoint{4.201223in}{0.857016in}}%
\pgfpathlineto{\pgfqpoint{4.201509in}{1.213770in}}%
\pgfpathlineto{\pgfqpoint{4.201752in}{1.020260in}}%
\pgfpathlineto{\pgfqpoint{4.202413in}{1.180225in}}%
\pgfpathlineto{\pgfqpoint{4.202104in}{0.893730in}}%
\pgfpathlineto{\pgfqpoint{4.202876in}{1.109639in}}%
\pgfpathlineto{\pgfqpoint{4.203515in}{1.272336in}}%
\pgfpathlineto{\pgfqpoint{4.203846in}{1.003761in}}%
\pgfpathlineto{\pgfqpoint{4.203934in}{1.125264in}}%
\pgfpathlineto{\pgfqpoint{4.204771in}{0.916020in}}%
\pgfpathlineto{\pgfqpoint{4.204176in}{1.234858in}}%
\pgfpathlineto{\pgfqpoint{4.205058in}{0.938529in}}%
\pgfpathlineto{\pgfqpoint{4.205080in}{1.186781in}}%
\pgfpathlineto{\pgfqpoint{4.205389in}{0.916239in}}%
\pgfpathlineto{\pgfqpoint{4.206182in}{1.139250in}}%
\pgfpathlineto{\pgfqpoint{4.206292in}{0.962786in}}%
\pgfpathlineto{\pgfqpoint{4.207174in}{1.430990in}}%
\pgfpathlineto{\pgfqpoint{4.207240in}{1.285448in}}%
\pgfpathlineto{\pgfqpoint{4.208276in}{1.421812in}}%
\pgfpathlineto{\pgfqpoint{4.207703in}{1.078171in}}%
\pgfpathlineto{\pgfqpoint{4.208320in}{1.337896in}}%
\pgfpathlineto{\pgfqpoint{4.208607in}{1.153018in}}%
\pgfpathlineto{\pgfqpoint{4.208849in}{1.422030in}}%
\pgfpathlineto{\pgfqpoint{4.209444in}{1.261737in}}%
\pgfpathlineto{\pgfqpoint{4.210436in}{1.515234in}}%
\pgfpathlineto{\pgfqpoint{4.210304in}{1.253215in}}%
\pgfpathlineto{\pgfqpoint{4.210591in}{1.495239in}}%
\pgfpathlineto{\pgfqpoint{4.211053in}{1.239666in}}%
\pgfpathlineto{\pgfqpoint{4.211428in}{1.541458in}}%
\pgfpathlineto{\pgfqpoint{4.211715in}{1.344779in}}%
\pgfpathlineto{\pgfqpoint{4.212244in}{1.518622in}}%
\pgfpathlineto{\pgfqpoint{4.211847in}{1.187546in}}%
\pgfpathlineto{\pgfqpoint{4.212773in}{1.292878in}}%
\pgfpathlineto{\pgfqpoint{4.212817in}{1.255291in}}%
\pgfpathlineto{\pgfqpoint{4.213434in}{1.526926in}}%
\pgfpathlineto{\pgfqpoint{4.213765in}{1.439294in}}%
\pgfpathlineto{\pgfqpoint{4.214205in}{1.592813in}}%
\pgfpathlineto{\pgfqpoint{4.214294in}{1.290584in}}%
\pgfpathlineto{\pgfqpoint{4.214889in}{1.448691in}}%
\pgfpathlineto{\pgfqpoint{4.214911in}{1.333744in}}%
\pgfpathlineto{\pgfqpoint{4.215704in}{1.686017in}}%
\pgfpathlineto{\pgfqpoint{4.215969in}{1.573910in}}%
\pgfpathlineto{\pgfqpoint{4.216035in}{1.553805in}}%
\pgfpathlineto{\pgfqpoint{4.216895in}{1.797140in}}%
\pgfpathlineto{\pgfqpoint{4.216145in}{1.458197in}}%
\pgfpathlineto{\pgfqpoint{4.217137in}{1.633788in}}%
\pgfpathlineto{\pgfqpoint{4.217357in}{1.431427in}}%
\pgfpathlineto{\pgfqpoint{4.217820in}{1.754308in}}%
\pgfpathlineto{\pgfqpoint{4.218239in}{1.677385in}}%
\pgfpathlineto{\pgfqpoint{4.218570in}{1.486279in}}%
\pgfpathlineto{\pgfqpoint{4.219077in}{1.801183in}}%
\pgfpathlineto{\pgfqpoint{4.219319in}{1.723495in}}%
\pgfpathlineto{\pgfqpoint{4.219363in}{1.591283in}}%
\pgfpathlineto{\pgfqpoint{4.219650in}{1.879308in}}%
\pgfpathlineto{\pgfqpoint{4.220377in}{1.612590in}}%
\pgfpathlineto{\pgfqpoint{4.220421in}{1.938312in}}%
\pgfpathlineto{\pgfqpoint{4.221501in}{1.869256in}}%
\pgfpathlineto{\pgfqpoint{4.221810in}{1.575549in}}%
\pgfpathlineto{\pgfqpoint{4.222537in}{1.913399in}}%
\pgfpathlineto{\pgfqpoint{4.222625in}{1.744911in}}%
\pgfpathlineto{\pgfqpoint{4.223309in}{1.899085in}}%
\pgfpathlineto{\pgfqpoint{4.222978in}{1.608001in}}%
\pgfpathlineto{\pgfqpoint{4.223728in}{1.738465in}}%
\pgfpathlineto{\pgfqpoint{4.224168in}{1.915694in}}%
\pgfpathlineto{\pgfqpoint{4.224477in}{1.567354in}}%
\pgfpathlineto{\pgfqpoint{4.224852in}{1.816371in}}%
\pgfpathlineto{\pgfqpoint{4.225160in}{1.695851in}}%
\pgfpathlineto{\pgfqpoint{4.225888in}{1.907280in}}%
\pgfpathlineto{\pgfqpoint{4.225932in}{1.860078in}}%
\pgfpathlineto{\pgfqpoint{4.226836in}{1.931319in}}%
\pgfpathlineto{\pgfqpoint{4.226351in}{1.670064in}}%
\pgfpathlineto{\pgfqpoint{4.226968in}{1.825877in}}%
\pgfpathlineto{\pgfqpoint{4.227144in}{1.788945in}}%
\pgfpathlineto{\pgfqpoint{4.227210in}{2.038290in}}%
\pgfpathlineto{\pgfqpoint{4.227695in}{1.913727in}}%
\pgfpathlineto{\pgfqpoint{4.228026in}{2.100025in}}%
\pgfpathlineto{\pgfqpoint{4.228643in}{1.747424in}}%
\pgfpathlineto{\pgfqpoint{4.228797in}{1.892857in}}%
\pgfpathlineto{\pgfqpoint{4.229282in}{1.981690in}}%
\pgfpathlineto{\pgfqpoint{4.229811in}{1.756166in}}%
\pgfpathlineto{\pgfqpoint{4.229833in}{1.835602in}}%
\pgfpathlineto{\pgfqpoint{4.230539in}{1.648320in}}%
\pgfpathlineto{\pgfqpoint{4.229899in}{1.893404in}}%
\pgfpathlineto{\pgfqpoint{4.230935in}{1.801292in}}%
\pgfpathlineto{\pgfqpoint{4.231839in}{1.573582in}}%
\pgfpathlineto{\pgfqpoint{4.231046in}{2.074457in}}%
\pgfpathlineto{\pgfqpoint{4.232126in}{1.706122in}}%
\pgfpathlineto{\pgfqpoint{4.232368in}{1.874610in}}%
\pgfpathlineto{\pgfqpoint{4.232721in}{1.614994in}}%
\pgfpathlineto{\pgfqpoint{4.233250in}{1.756603in}}%
\pgfpathlineto{\pgfqpoint{4.234021in}{1.632367in}}%
\pgfpathlineto{\pgfqpoint{4.233360in}{1.862372in}}%
\pgfpathlineto{\pgfqpoint{4.234308in}{1.755401in}}%
\pgfpathlineto{\pgfqpoint{4.234594in}{1.910995in}}%
\pgfpathlineto{\pgfqpoint{4.235256in}{1.640234in}}%
\pgfpathlineto{\pgfqpoint{4.235388in}{1.683941in}}%
\pgfpathlineto{\pgfqpoint{4.235961in}{1.548997in}}%
\pgfpathlineto{\pgfqpoint{4.235608in}{1.847184in}}%
\pgfpathlineto{\pgfqpoint{4.236115in}{1.733766in}}%
\pgfpathlineto{\pgfqpoint{4.236512in}{1.885646in}}%
\pgfpathlineto{\pgfqpoint{4.237151in}{1.607236in}}%
\pgfpathlineto{\pgfqpoint{4.237217in}{1.880947in}}%
\pgfpathlineto{\pgfqpoint{4.238253in}{1.651926in}}%
\pgfpathlineto{\pgfqpoint{4.237724in}{1.950659in}}%
\pgfpathlineto{\pgfqpoint{4.238341in}{1.783264in}}%
\pgfpathlineto{\pgfqpoint{4.239223in}{1.531624in}}%
\pgfpathlineto{\pgfqpoint{4.238716in}{1.808832in}}%
\pgfpathlineto{\pgfqpoint{4.239333in}{1.748845in}}%
\pgfpathlineto{\pgfqpoint{4.239355in}{1.841939in}}%
\pgfpathlineto{\pgfqpoint{4.240259in}{1.503871in}}%
\pgfpathlineto{\pgfqpoint{4.240413in}{1.640999in}}%
\pgfpathlineto{\pgfqpoint{4.240986in}{1.539601in}}%
\pgfpathlineto{\pgfqpoint{4.240590in}{1.841612in}}%
\pgfpathlineto{\pgfqpoint{4.241405in}{1.671922in}}%
\pgfpathlineto{\pgfqpoint{4.242221in}{1.925856in}}%
\pgfpathlineto{\pgfqpoint{4.241648in}{1.658045in}}%
\pgfpathlineto{\pgfqpoint{4.242507in}{1.700112in}}%
\pgfpathlineto{\pgfqpoint{4.243058in}{1.602756in}}%
\pgfpathlineto{\pgfqpoint{4.242860in}{1.859203in}}%
\pgfpathlineto{\pgfqpoint{4.243521in}{1.788836in}}%
\pgfpathlineto{\pgfqpoint{4.243543in}{1.902254in}}%
\pgfpathlineto{\pgfqpoint{4.243698in}{1.548451in}}%
\pgfpathlineto{\pgfqpoint{4.244623in}{1.884662in}}%
\pgfpathlineto{\pgfqpoint{4.245461in}{1.622206in}}%
\pgfpathlineto{\pgfqpoint{4.245703in}{1.955139in}}%
\pgfpathlineto{\pgfqpoint{4.245748in}{1.794627in}}%
\pgfpathlineto{\pgfqpoint{4.246343in}{2.034684in}}%
\pgfpathlineto{\pgfqpoint{4.246453in}{1.678259in}}%
\pgfpathlineto{\pgfqpoint{4.246850in}{1.743054in}}%
\pgfpathlineto{\pgfqpoint{4.246872in}{1.675527in}}%
\pgfpathlineto{\pgfqpoint{4.247114in}{1.985733in}}%
\pgfpathlineto{\pgfqpoint{4.247908in}{1.850025in}}%
\pgfpathlineto{\pgfqpoint{4.248282in}{1.992399in}}%
\pgfpathlineto{\pgfqpoint{4.247996in}{1.744802in}}%
\pgfpathlineto{\pgfqpoint{4.248900in}{1.771681in}}%
\pgfpathlineto{\pgfqpoint{4.249252in}{1.584181in}}%
\pgfpathlineto{\pgfqpoint{4.249803in}{1.892857in}}%
\pgfpathlineto{\pgfqpoint{4.250002in}{1.950987in}}%
\pgfpathlineto{\pgfqpoint{4.250090in}{1.692136in}}%
\pgfpathlineto{\pgfqpoint{4.250839in}{1.755729in}}%
\pgfpathlineto{\pgfqpoint{4.251016in}{1.675200in}}%
\pgfpathlineto{\pgfqpoint{4.251324in}{1.901708in}}%
\pgfpathlineto{\pgfqpoint{4.251831in}{1.869911in}}%
\pgfpathlineto{\pgfqpoint{4.253110in}{2.065060in}}%
\pgfpathlineto{\pgfqpoint{4.252052in}{1.797031in}}%
\pgfpathlineto{\pgfqpoint{4.253198in}{1.987263in}}%
\pgfpathlineto{\pgfqpoint{4.253925in}{1.844015in}}%
\pgfpathlineto{\pgfqpoint{4.254101in}{2.140891in}}%
\pgfpathlineto{\pgfqpoint{4.254300in}{1.998190in}}%
\pgfpathlineto{\pgfqpoint{4.254653in}{2.133898in}}%
\pgfpathlineto{\pgfqpoint{4.254542in}{1.886520in}}%
\pgfpathlineto{\pgfqpoint{4.254785in}{1.937219in}}%
\pgfpathlineto{\pgfqpoint{4.255600in}{1.828937in}}%
\pgfpathlineto{\pgfqpoint{4.255358in}{2.142311in}}%
\pgfpathlineto{\pgfqpoint{4.255821in}{1.884662in}}%
\pgfpathlineto{\pgfqpoint{4.255843in}{2.009335in}}%
\pgfpathlineto{\pgfqpoint{4.256504in}{1.714535in}}%
\pgfpathlineto{\pgfqpoint{4.256923in}{1.840628in}}%
\pgfpathlineto{\pgfqpoint{4.257738in}{1.693119in}}%
\pgfpathlineto{\pgfqpoint{4.257121in}{1.948474in}}%
\pgfpathlineto{\pgfqpoint{4.257871in}{1.839208in}}%
\pgfpathlineto{\pgfqpoint{4.258686in}{1.946507in}}%
\pgfpathlineto{\pgfqpoint{4.258157in}{1.656515in}}%
\pgfpathlineto{\pgfqpoint{4.258951in}{1.788618in}}%
\pgfpathlineto{\pgfqpoint{4.259369in}{1.961258in}}%
\pgfpathlineto{\pgfqpoint{4.259524in}{1.688858in}}%
\pgfpathlineto{\pgfqpoint{4.260097in}{1.891983in}}%
\pgfpathlineto{\pgfqpoint{4.261111in}{1.664382in}}%
\pgfpathlineto{\pgfqpoint{4.260383in}{1.900506in}}%
\pgfpathlineto{\pgfqpoint{4.261243in}{1.699020in}}%
\pgfpathlineto{\pgfqpoint{4.262125in}{1.912416in}}%
\pgfpathlineto{\pgfqpoint{4.261552in}{1.645261in}}%
\pgfpathlineto{\pgfqpoint{4.262389in}{1.835820in}}%
\pgfpathlineto{\pgfqpoint{4.262984in}{1.576642in}}%
\pgfpathlineto{\pgfqpoint{4.262588in}{1.882368in}}%
\pgfpathlineto{\pgfqpoint{4.263580in}{1.700986in}}%
\pgfpathlineto{\pgfqpoint{4.263646in}{1.776817in}}%
\pgfpathlineto{\pgfqpoint{4.263690in}{1.739666in}}%
\pgfpathlineto{\pgfqpoint{4.264461in}{1.967486in}}%
\pgfpathlineto{\pgfqpoint{4.263932in}{1.611716in}}%
\pgfpathlineto{\pgfqpoint{4.264836in}{1.939951in}}%
\pgfpathlineto{\pgfqpoint{4.265740in}{1.692136in}}%
\pgfpathlineto{\pgfqpoint{4.264990in}{1.942464in}}%
\pgfpathlineto{\pgfqpoint{4.265982in}{1.714098in}}%
\pgfpathlineto{\pgfqpoint{4.266136in}{1.874501in}}%
\pgfpathlineto{\pgfqpoint{4.266665in}{1.576423in}}%
\pgfpathlineto{\pgfqpoint{4.267084in}{1.802276in}}%
\pgfpathlineto{\pgfqpoint{4.267437in}{1.603193in}}%
\pgfpathlineto{\pgfqpoint{4.267988in}{1.902910in}}%
\pgfpathlineto{\pgfqpoint{4.268164in}{1.822490in}}%
\pgfpathlineto{\pgfqpoint{4.269002in}{2.001905in}}%
\pgfpathlineto{\pgfqpoint{4.268297in}{1.776489in}}%
\pgfpathlineto{\pgfqpoint{4.269288in}{1.977757in}}%
\pgfpathlineto{\pgfqpoint{4.270236in}{1.798998in}}%
\pgfpathlineto{\pgfqpoint{4.269663in}{2.132259in}}%
\pgfpathlineto{\pgfqpoint{4.270390in}{1.934488in}}%
\pgfpathlineto{\pgfqpoint{4.270721in}{2.066372in}}%
\pgfpathlineto{\pgfqpoint{4.271184in}{1.817573in}}%
\pgfpathlineto{\pgfqpoint{4.271360in}{1.907280in}}%
\pgfpathlineto{\pgfqpoint{4.271625in}{1.746332in}}%
\pgfpathlineto{\pgfqpoint{4.271581in}{2.032281in}}%
\pgfpathlineto{\pgfqpoint{4.272440in}{1.967814in}}%
\pgfpathlineto{\pgfqpoint{4.273454in}{2.103522in}}%
\pgfpathlineto{\pgfqpoint{4.273124in}{1.821834in}}%
\pgfpathlineto{\pgfqpoint{4.273498in}{2.048343in}}%
\pgfpathlineto{\pgfqpoint{4.273741in}{1.845764in}}%
\pgfpathlineto{\pgfqpoint{4.274578in}{2.116634in}}%
\pgfpathlineto{\pgfqpoint{4.274601in}{2.069103in}}%
\pgfpathlineto{\pgfqpoint{4.274645in}{2.195196in}}%
\pgfpathlineto{\pgfqpoint{4.275416in}{1.905095in}}%
\pgfpathlineto{\pgfqpoint{4.275725in}{2.122097in}}%
\pgfpathlineto{\pgfqpoint{4.276672in}{2.011520in}}%
\pgfpathlineto{\pgfqpoint{4.276011in}{2.278457in}}%
\pgfpathlineto{\pgfqpoint{4.276805in}{2.085602in}}%
\pgfpathlineto{\pgfqpoint{4.276937in}{2.270480in}}%
\pgfpathlineto{\pgfqpoint{4.277598in}{1.851883in}}%
\pgfpathlineto{\pgfqpoint{4.277929in}{2.124720in}}%
\pgfpathlineto{\pgfqpoint{4.278326in}{1.869474in}}%
\pgfpathlineto{\pgfqpoint{4.279031in}{2.136193in}}%
\pgfpathlineto{\pgfqpoint{4.279053in}{2.060690in}}%
\pgfpathlineto{\pgfqpoint{4.279824in}{2.180336in}}%
\pgfpathlineto{\pgfqpoint{4.279582in}{1.961476in}}%
\pgfpathlineto{\pgfqpoint{4.279979in}{2.039929in}}%
\pgfpathlineto{\pgfqpoint{4.280177in}{1.902145in}}%
\pgfpathlineto{\pgfqpoint{4.280971in}{2.209291in}}%
\pgfpathlineto{\pgfqpoint{4.281059in}{2.006275in}}%
\pgfpathlineto{\pgfqpoint{4.282029in}{2.195742in}}%
\pgfpathlineto{\pgfqpoint{4.281764in}{1.953718in}}%
\pgfpathlineto{\pgfqpoint{4.282183in}{2.095546in}}%
\pgfpathlineto{\pgfqpoint{4.282205in}{2.165257in}}%
\pgfpathlineto{\pgfqpoint{4.282844in}{1.882259in}}%
\pgfpathlineto{\pgfqpoint{4.283241in}{2.003434in}}%
\pgfpathlineto{\pgfqpoint{4.283990in}{1.830685in}}%
\pgfpathlineto{\pgfqpoint{4.284123in}{2.096966in}}%
\pgfpathlineto{\pgfqpoint{4.284365in}{1.949020in}}%
\pgfpathlineto{\pgfqpoint{4.284806in}{2.156407in}}%
\pgfpathlineto{\pgfqpoint{4.284475in}{1.821288in}}%
\pgfpathlineto{\pgfqpoint{4.285511in}{2.054571in}}%
\pgfpathlineto{\pgfqpoint{4.286084in}{1.921594in}}%
\pgfpathlineto{\pgfqpoint{4.285710in}{2.246770in}}%
\pgfpathlineto{\pgfqpoint{4.286437in}{2.094344in}}%
\pgfpathlineto{\pgfqpoint{4.287209in}{2.187985in}}%
\pgfpathlineto{\pgfqpoint{4.287429in}{1.844452in}}%
\pgfpathlineto{\pgfqpoint{4.287495in}{2.029549in}}%
\pgfpathlineto{\pgfqpoint{4.288046in}{1.842814in}}%
\pgfpathlineto{\pgfqpoint{4.288112in}{2.114339in}}%
\pgfpathlineto{\pgfqpoint{4.288597in}{1.944540in}}%
\pgfpathlineto{\pgfqpoint{4.289325in}{2.072272in}}%
\pgfpathlineto{\pgfqpoint{4.288884in}{1.815388in}}%
\pgfpathlineto{\pgfqpoint{4.289677in}{2.025288in}}%
\pgfpathlineto{\pgfqpoint{4.289699in}{1.901489in}}%
\pgfpathlineto{\pgfqpoint{4.290162in}{2.280970in}}%
\pgfpathlineto{\pgfqpoint{4.290757in}{2.130401in}}%
\pgfpathlineto{\pgfqpoint{4.291705in}{1.878434in}}%
\pgfpathlineto{\pgfqpoint{4.291617in}{2.226337in}}%
\pgfpathlineto{\pgfqpoint{4.291970in}{1.981581in}}%
\pgfpathlineto{\pgfqpoint{4.292609in}{2.245895in}}%
\pgfpathlineto{\pgfqpoint{4.292807in}{1.901926in}}%
\pgfpathlineto{\pgfqpoint{4.293028in}{2.098059in}}%
\pgfpathlineto{\pgfqpoint{4.293513in}{1.947709in}}%
\pgfpathlineto{\pgfqpoint{4.294042in}{2.252779in}}%
\pgfpathlineto{\pgfqpoint{4.294108in}{2.126686in}}%
\pgfpathlineto{\pgfqpoint{4.294791in}{2.048452in}}%
\pgfpathlineto{\pgfqpoint{4.295254in}{2.309379in}}%
\pgfpathlineto{\pgfqpoint{4.296069in}{1.980052in}}%
\pgfpathlineto{\pgfqpoint{4.296334in}{2.398868in}}%
\pgfpathlineto{\pgfqpoint{4.296378in}{2.133461in}}%
\pgfpathlineto{\pgfqpoint{4.297502in}{2.426184in}}%
\pgfpathlineto{\pgfqpoint{4.296554in}{2.061236in}}%
\pgfpathlineto{\pgfqpoint{4.297590in}{2.304462in}}%
\pgfpathlineto{\pgfqpoint{4.297656in}{2.118928in}}%
\pgfpathlineto{\pgfqpoint{4.297855in}{2.519716in}}%
\pgfpathlineto{\pgfqpoint{4.298692in}{2.249173in}}%
\pgfpathlineto{\pgfqpoint{4.299640in}{2.538400in}}%
\pgfpathlineto{\pgfqpoint{4.299199in}{2.188203in}}%
\pgfpathlineto{\pgfqpoint{4.299817in}{2.317792in}}%
\pgfpathlineto{\pgfqpoint{4.300235in}{2.195633in}}%
\pgfpathlineto{\pgfqpoint{4.300566in}{2.510647in}}%
\pgfpathlineto{\pgfqpoint{4.300897in}{2.284029in}}%
\pgfpathlineto{\pgfqpoint{4.301426in}{2.562220in}}%
\pgfpathlineto{\pgfqpoint{4.301933in}{2.253107in}}%
\pgfpathlineto{\pgfqpoint{4.302021in}{2.473387in}}%
\pgfpathlineto{\pgfqpoint{4.302638in}{2.164165in}}%
\pgfpathlineto{\pgfqpoint{4.302307in}{2.492072in}}%
\pgfpathlineto{\pgfqpoint{4.303123in}{2.330358in}}%
\pgfpathlineto{\pgfqpoint{4.303784in}{2.491307in}}%
\pgfpathlineto{\pgfqpoint{4.303718in}{2.232237in}}%
\pgfpathlineto{\pgfqpoint{4.304203in}{2.415039in}}%
\pgfpathlineto{\pgfqpoint{4.304401in}{2.206450in}}%
\pgfpathlineto{\pgfqpoint{4.304864in}{2.520044in}}%
\pgfpathlineto{\pgfqpoint{4.305283in}{2.440280in}}%
\pgfpathlineto{\pgfqpoint{4.305592in}{2.573147in}}%
\pgfpathlineto{\pgfqpoint{4.306076in}{2.326862in}}%
\pgfpathlineto{\pgfqpoint{4.306385in}{2.453938in}}%
\pgfpathlineto{\pgfqpoint{4.306826in}{2.596967in}}%
\pgfpathlineto{\pgfqpoint{4.306628in}{2.296049in}}%
\pgfpathlineto{\pgfqpoint{4.307223in}{2.414930in}}%
\pgfpathlineto{\pgfqpoint{4.308082in}{2.206450in}}%
\pgfpathlineto{\pgfqpoint{4.307553in}{2.563859in}}%
\pgfpathlineto{\pgfqpoint{4.308347in}{2.363029in}}%
\pgfpathlineto{\pgfqpoint{4.308854in}{2.553698in}}%
\pgfpathlineto{\pgfqpoint{4.309096in}{2.242399in}}%
\pgfpathlineto{\pgfqpoint{4.309427in}{2.518405in}}%
\pgfpathlineto{\pgfqpoint{4.310507in}{2.207543in}}%
\pgfpathlineto{\pgfqpoint{4.310154in}{2.535996in}}%
\pgfpathlineto{\pgfqpoint{4.310551in}{2.226118in}}%
\pgfpathlineto{\pgfqpoint{4.311058in}{2.455249in}}%
\pgfpathlineto{\pgfqpoint{4.310749in}{2.127779in}}%
\pgfpathlineto{\pgfqpoint{4.311697in}{2.282390in}}%
\pgfpathlineto{\pgfqpoint{4.312226in}{2.499174in}}%
\pgfpathlineto{\pgfqpoint{4.311984in}{2.228959in}}%
\pgfpathlineto{\pgfqpoint{4.312887in}{2.323802in}}%
\pgfpathlineto{\pgfqpoint{4.312909in}{2.243492in}}%
\pgfpathlineto{\pgfqpoint{4.313394in}{2.506604in}}%
\pgfpathlineto{\pgfqpoint{4.314012in}{2.254200in}}%
\pgfpathlineto{\pgfqpoint{4.314761in}{2.511739in}}%
\pgfpathlineto{\pgfqpoint{4.314166in}{2.213771in}}%
\pgfpathlineto{\pgfqpoint{4.315136in}{2.372098in}}%
\pgfpathlineto{\pgfqpoint{4.315995in}{2.217705in}}%
\pgfpathlineto{\pgfqpoint{4.316128in}{2.536652in}}%
\pgfpathlineto{\pgfqpoint{4.316260in}{2.251250in}}%
\pgfpathlineto{\pgfqpoint{4.317318in}{2.567793in}}%
\pgfpathlineto{\pgfqpoint{4.316745in}{2.142202in}}%
\pgfpathlineto{\pgfqpoint{4.317384in}{2.400725in}}%
\pgfpathlineto{\pgfqpoint{4.317582in}{2.516875in}}%
\pgfpathlineto{\pgfqpoint{4.318552in}{2.163837in}}%
\pgfpathlineto{\pgfqpoint{4.319280in}{2.643951in}}%
\pgfpathlineto{\pgfqpoint{4.319698in}{2.511193in}}%
\pgfpathlineto{\pgfqpoint{4.319765in}{2.314514in}}%
\pgfpathlineto{\pgfqpoint{4.319743in}{2.572273in}}%
\pgfpathlineto{\pgfqpoint{4.320602in}{2.526490in}}%
\pgfpathlineto{\pgfqpoint{4.320668in}{2.640236in}}%
\pgfpathlineto{\pgfqpoint{4.321550in}{2.251140in}}%
\pgfpathlineto{\pgfqpoint{4.321682in}{2.466285in}}%
\pgfpathlineto{\pgfqpoint{4.321770in}{2.242180in}}%
\pgfpathlineto{\pgfqpoint{4.322454in}{2.530861in}}%
\pgfpathlineto{\pgfqpoint{4.322850in}{2.354615in}}%
\pgfpathlineto{\pgfqpoint{4.323864in}{2.539493in}}%
\pgfpathlineto{\pgfqpoint{4.323335in}{2.215738in}}%
\pgfpathlineto{\pgfqpoint{4.323930in}{2.450769in}}%
\pgfpathlineto{\pgfqpoint{4.324261in}{2.136848in}}%
\pgfpathlineto{\pgfqpoint{4.325033in}{2.418099in}}%
\pgfpathlineto{\pgfqpoint{4.325187in}{2.060034in}}%
\pgfpathlineto{\pgfqpoint{4.326223in}{2.183614in}}%
\pgfpathlineto{\pgfqpoint{4.327215in}{2.119366in}}%
\pgfpathlineto{\pgfqpoint{4.327391in}{2.448365in}}%
\pgfpathlineto{\pgfqpoint{4.327788in}{2.135428in}}%
\pgfpathlineto{\pgfqpoint{4.328515in}{2.307849in}}%
\pgfpathlineto{\pgfqpoint{4.329463in}{2.668536in}}%
\pgfpathlineto{\pgfqpoint{4.328912in}{2.238465in}}%
\pgfpathlineto{\pgfqpoint{4.329683in}{2.499939in}}%
\pgfpathlineto{\pgfqpoint{4.330477in}{2.198583in}}%
\pgfpathlineto{\pgfqpoint{4.329860in}{2.570087in}}%
\pgfpathlineto{\pgfqpoint{4.330808in}{2.369257in}}%
\pgfpathlineto{\pgfqpoint{4.331160in}{2.505730in}}%
\pgfpathlineto{\pgfqpoint{4.331844in}{2.245349in}}%
\pgfpathlineto{\pgfqpoint{4.331888in}{2.478523in}}%
\pgfpathlineto{\pgfqpoint{4.332857in}{2.228194in}}%
\pgfpathlineto{\pgfqpoint{4.332703in}{2.546595in}}%
\pgfpathlineto{\pgfqpoint{4.333012in}{2.320196in}}%
\pgfpathlineto{\pgfqpoint{4.333651in}{2.567902in}}%
\pgfpathlineto{\pgfqpoint{4.333739in}{2.286870in}}%
\pgfpathlineto{\pgfqpoint{4.334136in}{2.400616in}}%
\pgfpathlineto{\pgfqpoint{4.334422in}{2.270480in}}%
\pgfpathlineto{\pgfqpoint{4.334974in}{2.544301in}}%
\pgfpathlineto{\pgfqpoint{4.335216in}{2.399851in}}%
\pgfpathlineto{\pgfqpoint{4.335965in}{2.649524in}}%
\pgfpathlineto{\pgfqpoint{4.335436in}{2.303479in}}%
\pgfpathlineto{\pgfqpoint{4.336340in}{2.569869in}}%
\pgfpathlineto{\pgfqpoint{4.336494in}{2.314405in}}%
\pgfpathlineto{\pgfqpoint{4.336406in}{2.655643in}}%
\pgfpathlineto{\pgfqpoint{4.337464in}{2.416678in}}%
\pgfpathlineto{\pgfqpoint{4.337993in}{2.642312in}}%
\pgfpathlineto{\pgfqpoint{4.338412in}{2.351337in}}%
\pgfpathlineto{\pgfqpoint{4.338588in}{2.513378in}}%
\pgfpathlineto{\pgfqpoint{4.338875in}{2.609642in}}%
\pgfpathlineto{\pgfqpoint{4.339624in}{2.348059in}}%
\pgfpathlineto{\pgfqpoint{4.339691in}{2.521573in}}%
\pgfpathlineto{\pgfqpoint{4.340330in}{2.254090in}}%
\pgfpathlineto{\pgfqpoint{4.340021in}{2.545721in}}%
\pgfpathlineto{\pgfqpoint{4.340815in}{2.419082in}}%
\pgfpathlineto{\pgfqpoint{4.341851in}{2.284466in}}%
\pgfpathlineto{\pgfqpoint{4.341498in}{2.631823in}}%
\pgfpathlineto{\pgfqpoint{4.341917in}{2.363793in}}%
\pgfpathlineto{\pgfqpoint{4.343019in}{2.515017in}}%
\pgfpathlineto{\pgfqpoint{4.342688in}{2.211586in}}%
\pgfpathlineto{\pgfqpoint{4.343041in}{2.506495in}}%
\pgfpathlineto{\pgfqpoint{4.343173in}{2.281188in}}%
\pgfpathlineto{\pgfqpoint{4.344033in}{2.554244in}}%
\pgfpathlineto{\pgfqpoint{4.344165in}{2.417662in}}%
\pgfpathlineto{\pgfqpoint{4.344981in}{2.340738in}}%
\pgfpathlineto{\pgfqpoint{4.344760in}{2.600791in}}%
\pgfpathlineto{\pgfqpoint{4.345135in}{2.492509in}}%
\pgfpathlineto{\pgfqpoint{4.345157in}{2.550857in}}%
\pgfpathlineto{\pgfqpoint{4.346039in}{2.293863in}}%
\pgfpathlineto{\pgfqpoint{4.346193in}{2.402474in}}%
\pgfpathlineto{\pgfqpoint{4.346876in}{2.207762in}}%
\pgfpathlineto{\pgfqpoint{4.347295in}{2.321398in}}%
\pgfpathlineto{\pgfqpoint{4.347802in}{2.620241in}}%
\pgfpathlineto{\pgfqpoint{4.348089in}{2.241088in}}%
\pgfpathlineto{\pgfqpoint{4.348529in}{2.472294in}}%
\pgfpathlineto{\pgfqpoint{4.348992in}{2.371333in}}%
\pgfpathlineto{\pgfqpoint{4.349411in}{2.592815in}}%
\pgfpathlineto{\pgfqpoint{4.349433in}{2.602867in}}%
\pgfpathlineto{\pgfqpoint{4.349918in}{2.342377in}}%
\pgfpathlineto{\pgfqpoint{4.350249in}{2.478086in}}%
\pgfpathlineto{\pgfqpoint{4.350601in}{2.364340in}}%
\pgfpathlineto{\pgfqpoint{4.350734in}{2.597622in}}%
\pgfpathlineto{\pgfqpoint{4.351351in}{2.420939in}}%
\pgfpathlineto{\pgfqpoint{4.351836in}{2.684270in}}%
\pgfpathlineto{\pgfqpoint{4.352387in}{2.313859in}}%
\pgfpathlineto{\pgfqpoint{4.352475in}{2.460384in}}%
\pgfpathlineto{\pgfqpoint{4.352563in}{2.567793in}}%
\pgfpathlineto{\pgfqpoint{4.352519in}{2.397666in}}%
\pgfpathlineto{\pgfqpoint{4.353136in}{2.483767in}}%
\pgfpathlineto{\pgfqpoint{4.354084in}{2.307849in}}%
\pgfpathlineto{\pgfqpoint{4.353357in}{2.586040in}}%
\pgfpathlineto{\pgfqpoint{4.354238in}{2.496442in}}%
\pgfpathlineto{\pgfqpoint{4.355010in}{2.344890in}}%
\pgfpathlineto{\pgfqpoint{4.354723in}{2.626250in}}%
\pgfpathlineto{\pgfqpoint{4.355120in}{2.433942in}}%
\pgfpathlineto{\pgfqpoint{4.355693in}{2.669629in}}%
\pgfpathlineto{\pgfqpoint{4.355230in}{2.404222in}}%
\pgfpathlineto{\pgfqpoint{4.356222in}{2.428042in}}%
\pgfpathlineto{\pgfqpoint{4.356288in}{2.642968in}}%
\pgfpathlineto{\pgfqpoint{4.357082in}{2.314405in}}%
\pgfpathlineto{\pgfqpoint{4.357346in}{2.499283in}}%
\pgfpathlineto{\pgfqpoint{4.357390in}{2.593689in}}%
\pgfpathlineto{\pgfqpoint{4.357655in}{2.307631in}}%
\pgfpathlineto{\pgfqpoint{4.358448in}{2.518514in}}%
\pgfpathlineto{\pgfqpoint{4.358669in}{2.257478in}}%
\pgfpathlineto{\pgfqpoint{4.359396in}{2.574567in}}%
\pgfpathlineto{\pgfqpoint{4.359572in}{2.451097in}}%
\pgfpathlineto{\pgfqpoint{4.360256in}{2.244694in}}%
\pgfpathlineto{\pgfqpoint{4.359903in}{2.507478in}}%
\pgfpathlineto{\pgfqpoint{4.360829in}{2.268404in}}%
\pgfpathlineto{\pgfqpoint{4.360961in}{2.513706in}}%
\pgfpathlineto{\pgfqpoint{4.361380in}{2.226883in}}%
\pgfpathlineto{\pgfqpoint{4.361931in}{2.326315in}}%
\pgfpathlineto{\pgfqpoint{4.362416in}{2.205795in}}%
\pgfpathlineto{\pgfqpoint{4.362768in}{2.493274in}}%
\pgfpathlineto{\pgfqpoint{4.362967in}{2.434051in}}%
\pgfpathlineto{\pgfqpoint{4.363496in}{2.600791in}}%
\pgfpathlineto{\pgfqpoint{4.363826in}{2.353632in}}%
\pgfpathlineto{\pgfqpoint{4.364091in}{2.518186in}}%
\pgfpathlineto{\pgfqpoint{4.364201in}{2.597513in}}%
\pgfpathlineto{\pgfqpoint{4.364378in}{2.346966in}}%
\pgfpathlineto{\pgfqpoint{4.365061in}{2.462351in}}%
\pgfpathlineto{\pgfqpoint{4.366009in}{2.242508in}}%
\pgfpathlineto{\pgfqpoint{4.365854in}{2.553807in}}%
\pgfpathlineto{\pgfqpoint{4.366185in}{2.339318in}}%
\pgfpathlineto{\pgfqpoint{4.367111in}{2.236717in}}%
\pgfpathlineto{\pgfqpoint{4.366494in}{2.502015in}}%
\pgfpathlineto{\pgfqpoint{4.367265in}{2.366744in}}%
\pgfpathlineto{\pgfqpoint{4.368103in}{2.234860in}}%
\pgfpathlineto{\pgfqpoint{4.368389in}{2.491962in}}%
\pgfpathlineto{\pgfqpoint{4.368742in}{2.338334in}}%
\pgfpathlineto{\pgfqpoint{4.368918in}{2.593361in}}%
\pgfpathlineto{\pgfqpoint{4.369469in}{2.506167in}}%
\pgfpathlineto{\pgfqpoint{4.369535in}{2.667662in}}%
\pgfpathlineto{\pgfqpoint{4.369800in}{2.356910in}}%
\pgfpathlineto{\pgfqpoint{4.370615in}{2.661871in}}%
\pgfpathlineto{\pgfqpoint{4.371277in}{2.383461in}}%
\pgfpathlineto{\pgfqpoint{4.371034in}{2.668099in}}%
\pgfpathlineto{\pgfqpoint{4.371740in}{2.446945in}}%
\pgfpathlineto{\pgfqpoint{4.372092in}{2.591067in}}%
\pgfpathlineto{\pgfqpoint{4.372533in}{2.297797in}}%
\pgfpathlineto{\pgfqpoint{4.372820in}{2.422360in}}%
\pgfpathlineto{\pgfqpoint{4.373569in}{2.242508in}}%
\pgfpathlineto{\pgfqpoint{4.373635in}{2.539384in}}%
\pgfpathlineto{\pgfqpoint{4.373966in}{2.278238in}}%
\pgfpathlineto{\pgfqpoint{4.375112in}{2.514471in}}%
\pgfpathlineto{\pgfqpoint{4.374671in}{2.185908in}}%
\pgfpathlineto{\pgfqpoint{4.375134in}{2.455140in}}%
\pgfpathlineto{\pgfqpoint{4.375332in}{2.329156in}}%
\pgfpathlineto{\pgfqpoint{4.375399in}{2.606254in}}%
\pgfpathlineto{\pgfqpoint{4.376214in}{2.419956in}}%
\pgfpathlineto{\pgfqpoint{4.376236in}{2.563204in}}%
\pgfpathlineto{\pgfqpoint{4.376787in}{2.321180in}}%
\pgfpathlineto{\pgfqpoint{4.377294in}{2.435253in}}%
\pgfpathlineto{\pgfqpoint{4.377316in}{2.352976in}}%
\pgfpathlineto{\pgfqpoint{4.378088in}{2.641001in}}%
\pgfpathlineto{\pgfqpoint{4.378374in}{2.493929in}}%
\pgfpathlineto{\pgfqpoint{4.378573in}{2.735516in}}%
\pgfpathlineto{\pgfqpoint{4.378947in}{2.409248in}}%
\pgfpathlineto{\pgfqpoint{4.379454in}{2.507587in}}%
\pgfpathlineto{\pgfqpoint{4.379763in}{2.370786in}}%
\pgfpathlineto{\pgfqpoint{4.379917in}{2.661215in}}%
\pgfpathlineto{\pgfqpoint{4.380578in}{2.403566in}}%
\pgfpathlineto{\pgfqpoint{4.380777in}{2.676512in}}%
\pgfpathlineto{\pgfqpoint{4.381725in}{2.509336in}}%
\pgfpathlineto{\pgfqpoint{4.382276in}{2.463881in}}%
\pgfpathlineto{\pgfqpoint{4.382342in}{2.691810in}}%
\pgfpathlineto{\pgfqpoint{4.382364in}{2.742509in}}%
\pgfpathlineto{\pgfqpoint{4.383069in}{2.405424in}}%
\pgfpathlineto{\pgfqpoint{4.383356in}{2.528785in}}%
\pgfpathlineto{\pgfqpoint{4.384061in}{2.383898in}}%
\pgfpathlineto{\pgfqpoint{4.383863in}{2.618274in}}%
\pgfpathlineto{\pgfqpoint{4.384480in}{2.479615in}}%
\pgfpathlineto{\pgfqpoint{4.385251in}{2.635319in}}%
\pgfpathlineto{\pgfqpoint{4.384965in}{2.348059in}}%
\pgfpathlineto{\pgfqpoint{4.385339in}{2.579594in}}%
\pgfpathlineto{\pgfqpoint{4.385780in}{2.357019in}}%
\pgfpathlineto{\pgfqpoint{4.386309in}{2.602212in}}%
\pgfpathlineto{\pgfqpoint{4.386442in}{2.498300in}}%
\pgfpathlineto{\pgfqpoint{4.386662in}{2.637067in}}%
\pgfpathlineto{\pgfqpoint{4.386860in}{2.426949in}}%
\pgfpathlineto{\pgfqpoint{4.387191in}{2.477102in}}%
\pgfpathlineto{\pgfqpoint{4.387323in}{2.322054in}}%
\pgfpathlineto{\pgfqpoint{4.388073in}{2.629310in}}%
\pgfpathlineto{\pgfqpoint{4.388293in}{2.423890in}}%
\pgfpathlineto{\pgfqpoint{4.389461in}{2.662526in}}%
\pgfpathlineto{\pgfqpoint{4.388690in}{2.362810in}}%
\pgfpathlineto{\pgfqpoint{4.389505in}{2.525398in}}%
\pgfpathlineto{\pgfqpoint{4.389726in}{2.336149in}}%
\pgfpathlineto{\pgfqpoint{4.390034in}{2.625704in}}%
\pgfpathlineto{\pgfqpoint{4.390608in}{2.480052in}}%
\pgfpathlineto{\pgfqpoint{4.391026in}{2.657500in}}%
\pgfpathlineto{\pgfqpoint{4.391621in}{2.376796in}}%
\pgfpathlineto{\pgfqpoint{4.391688in}{2.381822in}}%
\pgfpathlineto{\pgfqpoint{4.391710in}{2.351774in}}%
\pgfpathlineto{\pgfqpoint{4.392393in}{2.621224in}}%
\pgfpathlineto{\pgfqpoint{4.392591in}{2.518842in}}%
\pgfpathlineto{\pgfqpoint{4.392613in}{2.689515in}}%
\pgfpathlineto{\pgfqpoint{4.393142in}{2.323802in}}%
\pgfpathlineto{\pgfqpoint{4.393715in}{2.628326in}}%
\pgfpathlineto{\pgfqpoint{4.393914in}{2.400179in}}%
\pgfpathlineto{\pgfqpoint{4.394134in}{2.695853in}}%
\pgfpathlineto{\pgfqpoint{4.394818in}{2.598934in}}%
\pgfpathlineto{\pgfqpoint{4.395831in}{2.455030in}}%
\pgfpathlineto{\pgfqpoint{4.395721in}{2.743492in}}%
\pgfpathlineto{\pgfqpoint{4.395876in}{2.633571in}}%
\pgfpathlineto{\pgfqpoint{4.396052in}{2.787090in}}%
\pgfpathlineto{\pgfqpoint{4.396647in}{2.514908in}}%
\pgfpathlineto{\pgfqpoint{4.396934in}{2.587461in}}%
\pgfpathlineto{\pgfqpoint{4.397881in}{2.356691in}}%
\pgfpathlineto{\pgfqpoint{4.397220in}{2.675638in}}%
\pgfpathlineto{\pgfqpoint{4.398080in}{2.509882in}}%
\pgfpathlineto{\pgfqpoint{4.398895in}{2.690826in}}%
\pgfpathlineto{\pgfqpoint{4.398432in}{2.357565in}}%
\pgfpathlineto{\pgfqpoint{4.399204in}{2.550638in}}%
\pgfpathlineto{\pgfqpoint{4.400064in}{2.357565in}}%
\pgfpathlineto{\pgfqpoint{4.400218in}{2.650944in}}%
\pgfpathlineto{\pgfqpoint{4.400372in}{2.428479in}}%
\pgfpathlineto{\pgfqpoint{4.400593in}{2.411652in}}%
\pgfpathlineto{\pgfqpoint{4.401540in}{2.705249in}}%
\pgfpathlineto{\pgfqpoint{4.402025in}{2.472185in}}%
\pgfpathlineto{\pgfqpoint{4.402378in}{2.728195in}}%
\pgfpathlineto{\pgfqpoint{4.402664in}{2.547469in}}%
\pgfpathlineto{\pgfqpoint{4.403722in}{2.723825in}}%
\pgfpathlineto{\pgfqpoint{4.403546in}{2.469016in}}%
\pgfpathlineto{\pgfqpoint{4.403767in}{2.641766in}}%
\pgfpathlineto{\pgfqpoint{4.404274in}{2.447710in}}%
\pgfpathlineto{\pgfqpoint{4.404163in}{2.729179in}}%
\pgfpathlineto{\pgfqpoint{4.404891in}{2.562985in}}%
\pgfpathlineto{\pgfqpoint{4.405486in}{2.765892in}}%
\pgfpathlineto{\pgfqpoint{4.405243in}{2.471857in}}%
\pgfpathlineto{\pgfqpoint{4.406015in}{2.695743in}}%
\pgfpathlineto{\pgfqpoint{4.406786in}{2.447819in}}%
\pgfpathlineto{\pgfqpoint{4.407051in}{2.723278in}}%
\pgfpathlineto{\pgfqpoint{4.407161in}{2.503545in}}%
\pgfpathlineto{\pgfqpoint{4.408285in}{2.826316in}}%
\pgfpathlineto{\pgfqpoint{4.408109in}{2.470109in}}%
\pgfpathlineto{\pgfqpoint{4.408329in}{2.793755in}}%
\pgfpathlineto{\pgfqpoint{4.408616in}{2.501250in}}%
\pgfpathlineto{\pgfqpoint{4.409145in}{2.822164in}}%
\pgfpathlineto{\pgfqpoint{4.409475in}{2.555774in}}%
\pgfpathlineto{\pgfqpoint{4.409828in}{2.527146in}}%
\pgfpathlineto{\pgfqpoint{4.410622in}{2.877015in}}%
\pgfpathlineto{\pgfqpoint{4.411636in}{2.551731in}}%
\pgfpathlineto{\pgfqpoint{4.411724in}{2.778130in}}%
\pgfpathlineto{\pgfqpoint{4.411746in}{2.803042in}}%
\pgfpathlineto{\pgfqpoint{4.411922in}{2.546158in}}%
\pgfpathlineto{\pgfqpoint{4.412716in}{2.794301in}}%
\pgfpathlineto{\pgfqpoint{4.412804in}{2.479178in}}%
\pgfpathlineto{\pgfqpoint{4.413201in}{2.810363in}}%
\pgfpathlineto{\pgfqpoint{4.413862in}{2.570634in}}%
\pgfpathlineto{\pgfqpoint{4.413994in}{2.809271in}}%
\pgfpathlineto{\pgfqpoint{4.414699in}{2.444104in}}%
\pgfpathlineto{\pgfqpoint{4.414986in}{2.705140in}}%
\pgfpathlineto{\pgfqpoint{4.415096in}{2.817466in}}%
\pgfpathlineto{\pgfqpoint{4.415273in}{2.466285in}}%
\pgfpathlineto{\pgfqpoint{4.415802in}{2.611499in}}%
\pgfpathlineto{\pgfqpoint{4.415846in}{2.487264in}}%
\pgfpathlineto{\pgfqpoint{4.416485in}{2.806757in}}%
\pgfpathlineto{\pgfqpoint{4.416926in}{2.542443in}}%
\pgfpathlineto{\pgfqpoint{4.417499in}{2.482893in}}%
\pgfpathlineto{\pgfqpoint{4.417058in}{2.768951in}}%
\pgfpathlineto{\pgfqpoint{4.417741in}{2.607893in}}%
\pgfpathlineto{\pgfqpoint{4.418138in}{2.782500in}}%
\pgfpathlineto{\pgfqpoint{4.418270in}{2.485406in}}%
\pgfpathlineto{\pgfqpoint{4.418887in}{2.700005in}}%
\pgfpathlineto{\pgfqpoint{4.419328in}{2.439187in}}%
\pgfpathlineto{\pgfqpoint{4.419130in}{2.731036in}}%
\pgfpathlineto{\pgfqpoint{4.420034in}{2.534685in}}%
\pgfpathlineto{\pgfqpoint{4.420981in}{2.764909in}}%
\pgfpathlineto{\pgfqpoint{4.420717in}{2.525179in}}%
\pgfpathlineto{\pgfqpoint{4.421158in}{2.647776in}}%
\pgfpathlineto{\pgfqpoint{4.421687in}{2.542771in}}%
\pgfpathlineto{\pgfqpoint{4.421290in}{2.787854in}}%
\pgfpathlineto{\pgfqpoint{4.422238in}{2.609532in}}%
\pgfpathlineto{\pgfqpoint{4.423296in}{2.843252in}}%
\pgfpathlineto{\pgfqpoint{4.422789in}{2.504091in}}%
\pgfpathlineto{\pgfqpoint{4.423318in}{2.584948in}}%
\pgfpathlineto{\pgfqpoint{4.423913in}{2.445743in}}%
\pgfpathlineto{\pgfqpoint{4.423472in}{2.747754in}}%
\pgfpathlineto{\pgfqpoint{4.424398in}{2.624393in}}%
\pgfpathlineto{\pgfqpoint{4.424706in}{2.681757in}}%
\pgfpathlineto{\pgfqpoint{4.425235in}{2.403675in}}%
\pgfpathlineto{\pgfqpoint{4.425368in}{2.485953in}}%
\pgfpathlineto{\pgfqpoint{4.425654in}{2.287089in}}%
\pgfpathlineto{\pgfqpoint{4.426139in}{2.612701in}}%
\pgfpathlineto{\pgfqpoint{4.426448in}{2.580031in}}%
\pgfpathlineto{\pgfqpoint{4.427175in}{2.709074in}}%
\pgfpathlineto{\pgfqpoint{4.427087in}{2.432631in}}%
\pgfpathlineto{\pgfqpoint{4.427550in}{2.608221in}}%
\pgfpathlineto{\pgfqpoint{4.428365in}{2.463772in}}%
\pgfpathlineto{\pgfqpoint{4.428189in}{2.747317in}}%
\pgfpathlineto{\pgfqpoint{4.428630in}{2.598497in}}%
\pgfpathlineto{\pgfqpoint{4.428961in}{2.726556in}}%
\pgfpathlineto{\pgfqpoint{4.429446in}{2.403238in}}%
\pgfpathlineto{\pgfqpoint{4.429710in}{2.641220in}}%
\pgfpathlineto{\pgfqpoint{4.429908in}{2.457544in}}%
\pgfpathlineto{\pgfqpoint{4.430085in}{2.796159in}}%
\pgfpathlineto{\pgfqpoint{4.430812in}{2.582981in}}%
\pgfpathlineto{\pgfqpoint{4.431319in}{2.793427in}}%
\pgfpathlineto{\pgfqpoint{4.431672in}{2.477976in}}%
\pgfpathlineto{\pgfqpoint{4.431914in}{2.593033in}}%
\pgfpathlineto{\pgfqpoint{4.432245in}{2.446398in}}%
\pgfpathlineto{\pgfqpoint{4.432024in}{2.770700in}}%
\pgfpathlineto{\pgfqpoint{4.432642in}{2.630293in}}%
\pgfpathlineto{\pgfqpoint{4.432664in}{2.764472in}}%
\pgfpathlineto{\pgfqpoint{4.432818in}{2.488794in}}%
\pgfpathlineto{\pgfqpoint{4.433744in}{2.585494in}}%
\pgfpathlineto{\pgfqpoint{4.434603in}{2.471311in}}%
\pgfpathlineto{\pgfqpoint{4.434780in}{2.728195in}}%
\pgfpathlineto{\pgfqpoint{4.434846in}{2.579484in}}%
\pgfpathlineto{\pgfqpoint{4.435397in}{2.800092in}}%
\pgfpathlineto{\pgfqpoint{4.435485in}{2.465301in}}%
\pgfpathlineto{\pgfqpoint{4.435816in}{2.512395in}}%
\pgfpathlineto{\pgfqpoint{4.435838in}{2.447382in}}%
\pgfpathlineto{\pgfqpoint{4.436234in}{2.780097in}}%
\pgfpathlineto{\pgfqpoint{4.436896in}{2.551622in}}%
\pgfpathlineto{\pgfqpoint{4.437292in}{2.732238in}}%
\pgfpathlineto{\pgfqpoint{4.437072in}{2.505184in}}%
\pgfpathlineto{\pgfqpoint{4.437954in}{2.552277in}}%
\pgfpathlineto{\pgfqpoint{4.439056in}{2.373409in}}%
\pgfpathlineto{\pgfqpoint{4.438769in}{2.677168in}}%
\pgfpathlineto{\pgfqpoint{4.439078in}{2.468361in}}%
\pgfpathlineto{\pgfqpoint{4.439937in}{2.344235in}}%
\pgfpathlineto{\pgfqpoint{4.439673in}{2.684817in}}%
\pgfpathlineto{\pgfqpoint{4.440158in}{2.547360in}}%
\pgfpathlineto{\pgfqpoint{4.440511in}{2.277801in}}%
\pgfpathlineto{\pgfqpoint{4.441216in}{2.598715in}}%
\pgfpathlineto{\pgfqpoint{4.441238in}{2.550420in}}%
\pgfpathlineto{\pgfqpoint{4.441436in}{2.304899in}}%
\pgfpathlineto{\pgfqpoint{4.442318in}{2.611062in}}%
\pgfpathlineto{\pgfqpoint{4.443398in}{2.339318in}}%
\pgfpathlineto{\pgfqpoint{4.443134in}{2.710931in}}%
\pgfpathlineto{\pgfqpoint{4.443420in}{2.529441in}}%
\pgfpathlineto{\pgfqpoint{4.444081in}{2.429134in}}%
\pgfpathlineto{\pgfqpoint{4.443861in}{2.687985in}}%
\pgfpathlineto{\pgfqpoint{4.444412in}{2.578719in}}%
\pgfpathlineto{\pgfqpoint{4.444765in}{2.740542in}}%
\pgfpathlineto{\pgfqpoint{4.445007in}{2.386084in}}%
\pgfpathlineto{\pgfqpoint{4.445492in}{2.495131in}}%
\pgfpathlineto{\pgfqpoint{4.445558in}{2.381167in}}%
\pgfpathlineto{\pgfqpoint{4.445757in}{2.781080in}}%
\pgfpathlineto{\pgfqpoint{4.446550in}{2.602539in}}%
\pgfpathlineto{\pgfqpoint{4.446704in}{2.516110in}}%
\pgfpathlineto{\pgfqpoint{4.446837in}{2.735516in}}%
\pgfpathlineto{\pgfqpoint{4.447079in}{2.570634in}}%
\pgfpathlineto{\pgfqpoint{4.447586in}{2.749721in}}%
\pgfpathlineto{\pgfqpoint{4.448093in}{2.505074in}}%
\pgfpathlineto{\pgfqpoint{4.448181in}{2.640018in}}%
\pgfpathlineto{\pgfqpoint{4.448534in}{2.774305in}}%
\pgfpathlineto{\pgfqpoint{4.448335in}{2.529550in}}%
\pgfpathlineto{\pgfqpoint{4.449041in}{2.626359in}}%
\pgfpathlineto{\pgfqpoint{4.449812in}{2.469344in}}%
\pgfpathlineto{\pgfqpoint{4.449945in}{2.763379in}}%
\pgfpathlineto{\pgfqpoint{4.450099in}{2.564733in}}%
\pgfpathlineto{\pgfqpoint{4.450121in}{2.836259in}}%
\pgfpathlineto{\pgfqpoint{4.450870in}{2.293754in}}%
\pgfpathlineto{\pgfqpoint{4.451201in}{2.492946in}}%
\pgfpathlineto{\pgfqpoint{4.451796in}{2.306429in}}%
\pgfpathlineto{\pgfqpoint{4.451620in}{2.567684in}}%
\pgfpathlineto{\pgfqpoint{4.452237in}{2.549108in}}%
\pgfpathlineto{\pgfqpoint{4.452942in}{2.733112in}}%
\pgfpathlineto{\pgfqpoint{4.452678in}{2.496224in}}%
\pgfpathlineto{\pgfqpoint{4.453383in}{2.619694in}}%
\pgfpathlineto{\pgfqpoint{4.453846in}{2.455030in}}%
\pgfpathlineto{\pgfqpoint{4.454177in}{2.790477in}}%
\pgfpathlineto{\pgfqpoint{4.454441in}{2.592268in}}%
\pgfpathlineto{\pgfqpoint{4.454573in}{2.708200in}}%
\pgfpathlineto{\pgfqpoint{4.455169in}{2.371114in}}%
\pgfpathlineto{\pgfqpoint{4.455543in}{2.578829in}}%
\pgfpathlineto{\pgfqpoint{4.455808in}{2.717815in}}%
\pgfpathlineto{\pgfqpoint{4.455874in}{2.434161in}}%
\pgfpathlineto{\pgfqpoint{4.456050in}{2.498628in}}%
\pgfpathlineto{\pgfqpoint{4.456998in}{2.380074in}}%
\pgfpathlineto{\pgfqpoint{4.456778in}{2.635538in}}%
\pgfpathlineto{\pgfqpoint{4.457152in}{2.508024in}}%
\pgfpathlineto{\pgfqpoint{4.457262in}{2.483330in}}%
\pgfpathlineto{\pgfqpoint{4.457615in}{2.699895in}}%
\pgfpathlineto{\pgfqpoint{4.458034in}{2.545284in}}%
\pgfpathlineto{\pgfqpoint{4.458100in}{2.673781in}}%
\pgfpathlineto{\pgfqpoint{4.458321in}{2.416678in}}%
\pgfpathlineto{\pgfqpoint{4.459114in}{2.467596in}}%
\pgfpathlineto{\pgfqpoint{4.459356in}{2.706561in}}%
\pgfpathlineto{\pgfqpoint{4.459180in}{2.419519in}}%
\pgfpathlineto{\pgfqpoint{4.460238in}{2.583309in}}%
\pgfpathlineto{\pgfqpoint{4.461098in}{2.459838in}}%
\pgfpathlineto{\pgfqpoint{4.460966in}{2.634991in}}%
\pgfpathlineto{\pgfqpoint{4.461362in}{2.464318in}}%
\pgfpathlineto{\pgfqpoint{4.462178in}{2.735407in}}%
\pgfpathlineto{\pgfqpoint{4.462486in}{2.636849in}}%
\pgfpathlineto{\pgfqpoint{4.463082in}{2.531080in}}%
\pgfpathlineto{\pgfqpoint{4.463214in}{2.693886in}}%
\pgfpathlineto{\pgfqpoint{4.463589in}{2.570852in}}%
\pgfpathlineto{\pgfqpoint{4.464007in}{2.761412in}}%
\pgfpathlineto{\pgfqpoint{4.464669in}{2.485953in}}%
\pgfpathlineto{\pgfqpoint{4.465220in}{2.811347in}}%
\pgfpathlineto{\pgfqpoint{4.465903in}{2.691482in}}%
\pgfpathlineto{\pgfqpoint{4.466520in}{2.507697in}}%
\pgfpathlineto{\pgfqpoint{4.466234in}{2.836150in}}%
\pgfpathlineto{\pgfqpoint{4.467027in}{2.553588in}}%
\pgfpathlineto{\pgfqpoint{4.467181in}{2.760866in}}%
\pgfpathlineto{\pgfqpoint{4.467093in}{2.480708in}}%
\pgfpathlineto{\pgfqpoint{4.468107in}{2.581560in}}%
\pgfpathlineto{\pgfqpoint{4.468129in}{2.391110in}}%
\pgfpathlineto{\pgfqpoint{4.468901in}{2.727321in}}%
\pgfpathlineto{\pgfqpoint{4.469231in}{2.434816in}}%
\pgfpathlineto{\pgfqpoint{4.469253in}{2.386958in}}%
\pgfpathlineto{\pgfqpoint{4.469650in}{2.685582in}}%
\pgfpathlineto{\pgfqpoint{4.470267in}{2.542552in}}%
\pgfpathlineto{\pgfqpoint{4.470642in}{2.672797in}}%
\pgfpathlineto{\pgfqpoint{4.470554in}{2.454921in}}%
\pgfpathlineto{\pgfqpoint{4.471369in}{2.645372in}}%
\pgfpathlineto{\pgfqpoint{4.472185in}{2.448037in}}%
\pgfpathlineto{\pgfqpoint{4.471766in}{2.669192in}}%
\pgfpathlineto{\pgfqpoint{4.472471in}{2.658484in}}%
\pgfpathlineto{\pgfqpoint{4.472560in}{2.807413in}}%
\pgfpathlineto{\pgfqpoint{4.472582in}{2.569104in}}%
\pgfpathlineto{\pgfqpoint{4.472692in}{2.691045in}}%
\pgfpathlineto{\pgfqpoint{4.473507in}{2.475354in}}%
\pgfpathlineto{\pgfqpoint{4.473000in}{2.829048in}}%
\pgfpathlineto{\pgfqpoint{4.473816in}{2.576643in}}%
\pgfpathlineto{\pgfqpoint{4.473838in}{2.578173in}}%
\pgfpathlineto{\pgfqpoint{4.473860in}{2.550857in}}%
\pgfpathlineto{\pgfqpoint{4.474940in}{2.796596in}}%
\pgfpathlineto{\pgfqpoint{4.474411in}{2.461368in}}%
\pgfpathlineto{\pgfqpoint{4.474962in}{2.631713in}}%
\pgfpathlineto{\pgfqpoint{4.475094in}{2.493711in}}%
\pgfpathlineto{\pgfqpoint{4.475601in}{2.792553in}}%
\pgfpathlineto{\pgfqpoint{4.476064in}{2.634882in}}%
\pgfpathlineto{\pgfqpoint{4.476219in}{2.778021in}}%
\pgfpathlineto{\pgfqpoint{4.476439in}{2.473278in}}%
\pgfpathlineto{\pgfqpoint{4.477166in}{2.730708in}}%
\pgfpathlineto{\pgfqpoint{4.477872in}{2.435363in}}%
\pgfpathlineto{\pgfqpoint{4.477541in}{2.790368in}}%
\pgfpathlineto{\pgfqpoint{4.478291in}{2.518623in}}%
\pgfpathlineto{\pgfqpoint{4.478401in}{2.343907in}}%
\pgfpathlineto{\pgfqpoint{4.478709in}{2.696945in}}%
\pgfpathlineto{\pgfqpoint{4.479349in}{2.630074in}}%
\pgfpathlineto{\pgfqpoint{4.479613in}{2.453719in}}%
\pgfpathlineto{\pgfqpoint{4.479988in}{2.762177in}}%
\pgfpathlineto{\pgfqpoint{4.480473in}{2.531080in}}%
\pgfpathlineto{\pgfqpoint{4.480869in}{2.786215in}}%
\pgfpathlineto{\pgfqpoint{4.481376in}{2.478304in}}%
\pgfpathlineto{\pgfqpoint{4.481597in}{2.629310in}}%
\pgfpathlineto{\pgfqpoint{4.482258in}{2.566809in}}%
\pgfpathlineto{\pgfqpoint{4.482523in}{2.809926in}}%
\pgfpathlineto{\pgfqpoint{4.482633in}{2.702081in}}%
\pgfpathlineto{\pgfqpoint{4.482677in}{2.819214in}}%
\pgfpathlineto{\pgfqpoint{4.483404in}{2.387941in}}%
\pgfpathlineto{\pgfqpoint{4.483691in}{2.589865in}}%
\pgfpathlineto{\pgfqpoint{4.483713in}{2.430992in}}%
\pgfpathlineto{\pgfqpoint{4.484374in}{2.731255in}}%
\pgfpathlineto{\pgfqpoint{4.484793in}{2.586696in}}%
\pgfpathlineto{\pgfqpoint{4.485035in}{2.488357in}}%
\pgfpathlineto{\pgfqpoint{4.485785in}{2.726556in}}%
\pgfpathlineto{\pgfqpoint{4.485851in}{2.620022in}}%
\pgfpathlineto{\pgfqpoint{4.486909in}{2.871771in}}%
\pgfpathlineto{\pgfqpoint{4.486336in}{2.482675in}}%
\pgfpathlineto{\pgfqpoint{4.486953in}{2.628217in}}%
\pgfpathlineto{\pgfqpoint{4.487284in}{2.864122in}}%
\pgfpathlineto{\pgfqpoint{4.487857in}{2.501796in}}%
\pgfpathlineto{\pgfqpoint{4.488055in}{2.649087in}}%
\pgfpathlineto{\pgfqpoint{4.489157in}{2.449676in}}%
\pgfpathlineto{\pgfqpoint{4.488276in}{2.723169in}}%
\pgfpathlineto{\pgfqpoint{4.489201in}{2.603086in}}%
\pgfpathlineto{\pgfqpoint{4.490215in}{2.710057in}}%
\pgfpathlineto{\pgfqpoint{4.489620in}{2.434051in}}%
\pgfpathlineto{\pgfqpoint{4.490259in}{2.555009in}}%
\pgfpathlineto{\pgfqpoint{4.490766in}{2.369803in}}%
\pgfpathlineto{\pgfqpoint{4.491273in}{2.667116in}}%
\pgfpathlineto{\pgfqpoint{4.491361in}{2.574349in}}%
\pgfpathlineto{\pgfqpoint{4.491516in}{2.742837in}}%
\pgfpathlineto{\pgfqpoint{4.491428in}{2.484642in}}%
\pgfpathlineto{\pgfqpoint{4.492442in}{2.501578in}}%
\pgfpathlineto{\pgfqpoint{4.492750in}{2.466940in}}%
\pgfpathlineto{\pgfqpoint{4.492816in}{2.666679in}}%
\pgfpathlineto{\pgfqpoint{4.493323in}{2.527364in}}%
\pgfpathlineto{\pgfqpoint{4.493786in}{2.461914in}}%
\pgfpathlineto{\pgfqpoint{4.494469in}{2.847732in}}%
\pgfpathlineto{\pgfqpoint{4.494888in}{2.570087in}}%
\pgfpathlineto{\pgfqpoint{4.494822in}{2.858549in}}%
\pgfpathlineto{\pgfqpoint{4.495616in}{2.690826in}}%
\pgfpathlineto{\pgfqpoint{4.496431in}{2.793973in}}%
\pgfpathlineto{\pgfqpoint{4.496145in}{2.532172in}}%
\pgfpathlineto{\pgfqpoint{4.496629in}{2.628326in}}%
\pgfpathlineto{\pgfqpoint{4.497688in}{2.471420in}}%
\pgfpathlineto{\pgfqpoint{4.497048in}{2.831124in}}%
\pgfpathlineto{\pgfqpoint{4.497710in}{2.630839in}}%
\pgfpathlineto{\pgfqpoint{4.497776in}{2.747098in}}%
\pgfpathlineto{\pgfqpoint{4.498106in}{2.461259in}}%
\pgfpathlineto{\pgfqpoint{4.498790in}{2.626032in}}%
\pgfpathlineto{\pgfqpoint{4.499848in}{2.474371in}}%
\pgfpathlineto{\pgfqpoint{4.498944in}{2.765564in}}%
\pgfpathlineto{\pgfqpoint{4.499936in}{2.508134in}}%
\pgfpathlineto{\pgfqpoint{4.501082in}{2.720765in}}%
\pgfpathlineto{\pgfqpoint{4.500465in}{2.481145in}}%
\pgfpathlineto{\pgfqpoint{4.501126in}{2.681648in}}%
\pgfpathlineto{\pgfqpoint{4.501523in}{2.448912in}}%
\pgfpathlineto{\pgfqpoint{4.502052in}{2.791569in}}%
\pgfpathlineto{\pgfqpoint{4.502228in}{2.669629in}}%
\pgfpathlineto{\pgfqpoint{4.502867in}{2.840848in}}%
\pgfpathlineto{\pgfqpoint{4.502713in}{2.537635in}}%
\pgfpathlineto{\pgfqpoint{4.503286in}{2.806539in}}%
\pgfpathlineto{\pgfqpoint{4.503507in}{2.522120in}}%
\pgfpathlineto{\pgfqpoint{4.504410in}{2.677387in}}%
\pgfpathlineto{\pgfqpoint{4.505226in}{2.754310in}}%
\pgfpathlineto{\pgfqpoint{4.505072in}{2.432085in}}%
\pgfpathlineto{\pgfqpoint{4.505270in}{2.571180in}}%
\pgfpathlineto{\pgfqpoint{4.506196in}{2.423343in}}%
\pgfpathlineto{\pgfqpoint{4.505843in}{2.684270in}}%
\pgfpathlineto{\pgfqpoint{4.506328in}{2.636958in}}%
\pgfpathlineto{\pgfqpoint{4.506901in}{2.472404in}}%
\pgfpathlineto{\pgfqpoint{4.507474in}{2.830687in}}%
\pgfpathlineto{\pgfqpoint{4.508554in}{2.476337in}}%
\pgfpathlineto{\pgfqpoint{4.508025in}{2.885757in}}%
\pgfpathlineto{\pgfqpoint{4.508620in}{2.548344in}}%
\pgfpathlineto{\pgfqpoint{4.508819in}{2.765892in}}%
\pgfpathlineto{\pgfqpoint{4.509502in}{2.496879in}}%
\pgfpathlineto{\pgfqpoint{4.509767in}{2.630839in}}%
\pgfpathlineto{\pgfqpoint{4.509943in}{2.465301in}}%
\pgfpathlineto{\pgfqpoint{4.509833in}{2.708746in}}%
\pgfpathlineto{\pgfqpoint{4.510913in}{2.536652in}}%
\pgfpathlineto{\pgfqpoint{4.511750in}{2.728851in}}%
\pgfpathlineto{\pgfqpoint{4.511111in}{2.465192in}}%
\pgfpathlineto{\pgfqpoint{4.512059in}{2.607784in}}%
\pgfpathlineto{\pgfqpoint{4.512544in}{2.385974in}}%
\pgfpathlineto{\pgfqpoint{4.512720in}{2.681648in}}%
\pgfpathlineto{\pgfqpoint{4.513161in}{2.640018in}}%
\pgfpathlineto{\pgfqpoint{4.513183in}{2.661652in}}%
\pgfpathlineto{\pgfqpoint{4.513492in}{2.378107in}}%
\pgfpathlineto{\pgfqpoint{4.514043in}{2.576534in}}%
\pgfpathlineto{\pgfqpoint{4.514087in}{2.430227in}}%
\pgfpathlineto{\pgfqpoint{4.514880in}{2.731364in}}%
\pgfpathlineto{\pgfqpoint{4.515145in}{2.589537in}}%
\pgfpathlineto{\pgfqpoint{4.515828in}{2.791132in}}%
\pgfpathlineto{\pgfqpoint{4.516093in}{2.436783in}}%
\pgfpathlineto{\pgfqpoint{4.516225in}{2.538073in}}%
\pgfpathlineto{\pgfqpoint{4.517261in}{2.426731in}}%
\pgfpathlineto{\pgfqpoint{4.517129in}{2.791242in}}%
\pgfpathlineto{\pgfqpoint{4.517305in}{2.553698in}}%
\pgfpathlineto{\pgfqpoint{4.518407in}{2.766001in}}%
\pgfpathlineto{\pgfqpoint{4.517525in}{2.467377in}}%
\pgfpathlineto{\pgfqpoint{4.518451in}{2.737155in}}%
\pgfpathlineto{\pgfqpoint{4.519333in}{2.439405in}}%
\pgfpathlineto{\pgfqpoint{4.519575in}{2.566154in}}%
\pgfpathlineto{\pgfqpoint{4.520281in}{2.779878in}}%
\pgfpathlineto{\pgfqpoint{4.520391in}{2.487373in}}%
\pgfpathlineto{\pgfqpoint{4.520699in}{2.639253in}}%
\pgfpathlineto{\pgfqpoint{4.521647in}{2.514690in}}%
\pgfpathlineto{\pgfqpoint{4.520788in}{2.766657in}}%
\pgfpathlineto{\pgfqpoint{4.521757in}{2.644279in}}%
\pgfpathlineto{\pgfqpoint{4.522793in}{2.766657in}}%
\pgfpathlineto{\pgfqpoint{4.522242in}{2.483767in}}%
\pgfpathlineto{\pgfqpoint{4.522859in}{2.741416in}}%
\pgfpathlineto{\pgfqpoint{4.523300in}{2.503982in}}%
\pgfpathlineto{\pgfqpoint{4.522992in}{2.825114in}}%
\pgfpathlineto{\pgfqpoint{4.523962in}{2.660013in}}%
\pgfpathlineto{\pgfqpoint{4.524402in}{2.745241in}}%
\pgfpathlineto{\pgfqpoint{4.524601in}{2.489121in}}%
\pgfpathlineto{\pgfqpoint{4.525020in}{2.656626in}}%
\pgfpathlineto{\pgfqpoint{4.526188in}{2.444869in}}%
\pgfpathlineto{\pgfqpoint{4.525350in}{2.831452in}}%
\pgfpathlineto{\pgfqpoint{4.526210in}{2.461914in}}%
\pgfpathlineto{\pgfqpoint{4.526695in}{2.712133in}}%
\pgfpathlineto{\pgfqpoint{4.527268in}{2.451862in}}%
\pgfpathlineto{\pgfqpoint{4.527378in}{2.656080in}}%
\pgfpathlineto{\pgfqpoint{4.527400in}{2.658374in}}%
\pgfpathlineto{\pgfqpoint{4.527422in}{2.590739in}}%
\pgfpathlineto{\pgfqpoint{4.527907in}{2.449239in}}%
\pgfpathlineto{\pgfqpoint{4.527797in}{2.722404in}}%
\pgfpathlineto{\pgfqpoint{4.528524in}{2.582762in}}%
\pgfpathlineto{\pgfqpoint{4.529604in}{2.768077in}}%
\pgfpathlineto{\pgfqpoint{4.529075in}{2.421595in}}%
\pgfpathlineto{\pgfqpoint{4.529648in}{2.690826in}}%
\pgfpathlineto{\pgfqpoint{4.529781in}{2.468470in}}%
\pgfpathlineto{\pgfqpoint{4.529957in}{2.801076in}}%
\pgfpathlineto{\pgfqpoint{4.530773in}{2.533811in}}%
\pgfpathlineto{\pgfqpoint{4.531346in}{2.757369in}}%
\pgfpathlineto{\pgfqpoint{4.531500in}{2.477758in}}%
\pgfpathlineto{\pgfqpoint{4.531875in}{2.612483in}}%
\pgfpathlineto{\pgfqpoint{4.532867in}{2.794629in}}%
\pgfpathlineto{\pgfqpoint{4.532227in}{2.539056in}}%
\pgfpathlineto{\pgfqpoint{4.532977in}{2.658156in}}%
\pgfpathlineto{\pgfqpoint{4.533418in}{2.520044in}}%
\pgfpathlineto{\pgfqpoint{4.533682in}{2.744804in}}%
\pgfpathlineto{\pgfqpoint{4.534101in}{2.539821in}}%
\pgfpathlineto{\pgfqpoint{4.534299in}{2.882697in}}%
\pgfpathlineto{\pgfqpoint{4.534365in}{2.535232in}}%
\pgfpathlineto{\pgfqpoint{4.535225in}{2.634336in}}%
\pgfpathlineto{\pgfqpoint{4.535710in}{2.489996in}}%
\pgfpathlineto{\pgfqpoint{4.535490in}{2.792881in}}%
\pgfpathlineto{\pgfqpoint{4.536327in}{2.652692in}}%
\pgfpathlineto{\pgfqpoint{4.537517in}{2.376468in}}%
\pgfpathlineto{\pgfqpoint{4.536680in}{2.762833in}}%
\pgfpathlineto{\pgfqpoint{4.537584in}{2.446726in}}%
\pgfpathlineto{\pgfqpoint{4.538619in}{2.739668in}}%
\pgfpathlineto{\pgfqpoint{4.537650in}{2.432085in}}%
\pgfpathlineto{\pgfqpoint{4.538708in}{2.543427in}}%
\pgfpathlineto{\pgfqpoint{4.538730in}{2.496551in}}%
\pgfpathlineto{\pgfqpoint{4.538818in}{2.815499in}}%
\pgfpathlineto{\pgfqpoint{4.539766in}{2.596202in}}%
\pgfpathlineto{\pgfqpoint{4.540713in}{2.802824in}}%
\pgfpathlineto{\pgfqpoint{4.540008in}{2.554353in}}%
\pgfpathlineto{\pgfqpoint{4.540824in}{2.621005in}}%
\pgfpathlineto{\pgfqpoint{4.541485in}{2.439624in}}%
\pgfpathlineto{\pgfqpoint{4.541198in}{2.812221in}}%
\pgfpathlineto{\pgfqpoint{4.541904in}{2.797907in}}%
\pgfpathlineto{\pgfqpoint{4.542477in}{2.489996in}}%
\pgfpathlineto{\pgfqpoint{4.541948in}{2.834402in}}%
\pgfpathlineto{\pgfqpoint{4.543028in}{2.605599in}}%
\pgfpathlineto{\pgfqpoint{4.543336in}{2.759555in}}%
\pgfpathlineto{\pgfqpoint{4.543623in}{2.474480in}}%
\pgfpathlineto{\pgfqpoint{4.544174in}{2.715630in}}%
\pgfpathlineto{\pgfqpoint{4.544703in}{2.523540in}}%
\pgfpathlineto{\pgfqpoint{4.544549in}{2.766875in}}%
\pgfpathlineto{\pgfqpoint{4.545342in}{2.595546in}}%
\pgfpathlineto{\pgfqpoint{4.546158in}{2.754965in}}%
\pgfpathlineto{\pgfqpoint{4.545982in}{2.522338in}}%
\pgfpathlineto{\pgfqpoint{4.546466in}{2.650616in}}%
\pgfpathlineto{\pgfqpoint{4.546797in}{2.501796in}}%
\pgfpathlineto{\pgfqpoint{4.546555in}{2.738794in}}%
\pgfpathlineto{\pgfqpoint{4.547524in}{2.694651in}}%
\pgfpathlineto{\pgfqpoint{4.547546in}{2.789712in}}%
\pgfpathlineto{\pgfqpoint{4.548362in}{2.498081in}}%
\pgfpathlineto{\pgfqpoint{4.548605in}{2.670394in}}%
\pgfpathlineto{\pgfqpoint{4.549618in}{2.504746in}}%
\pgfpathlineto{\pgfqpoint{4.549354in}{2.809052in}}%
\pgfpathlineto{\pgfqpoint{4.549707in}{2.567793in}}%
\pgfpathlineto{\pgfqpoint{4.550412in}{2.896246in}}%
\pgfpathlineto{\pgfqpoint{4.550214in}{2.549545in}}%
\pgfpathlineto{\pgfqpoint{4.550831in}{2.785778in}}%
\pgfpathlineto{\pgfqpoint{4.550985in}{2.554681in}}%
\pgfpathlineto{\pgfqpoint{4.551580in}{2.815936in}}%
\pgfpathlineto{\pgfqpoint{4.551955in}{2.669410in}}%
\pgfpathlineto{\pgfqpoint{4.551999in}{2.779878in}}%
\pgfpathlineto{\pgfqpoint{4.552770in}{2.423671in}}%
\pgfpathlineto{\pgfqpoint{4.553057in}{2.654659in}}%
\pgfpathlineto{\pgfqpoint{4.553344in}{2.430992in}}%
\pgfpathlineto{\pgfqpoint{4.553895in}{2.737811in}}%
\pgfpathlineto{\pgfqpoint{4.554181in}{2.559161in}}%
\pgfpathlineto{\pgfqpoint{4.554446in}{2.823694in}}%
\pgfpathlineto{\pgfqpoint{4.554953in}{2.420939in}}%
\pgfpathlineto{\pgfqpoint{4.555305in}{2.705249in}}%
\pgfpathlineto{\pgfqpoint{4.555415in}{2.425638in}}%
\pgfpathlineto{\pgfqpoint{4.555548in}{2.802387in}}%
\pgfpathlineto{\pgfqpoint{4.556496in}{2.638488in}}%
\pgfpathlineto{\pgfqpoint{4.556518in}{2.771792in}}%
\pgfpathlineto{\pgfqpoint{4.557245in}{2.548344in}}%
\pgfpathlineto{\pgfqpoint{4.557598in}{2.661980in}}%
\pgfpathlineto{\pgfqpoint{4.558501in}{2.467924in}}%
\pgfpathlineto{\pgfqpoint{4.557928in}{2.847295in}}%
\pgfpathlineto{\pgfqpoint{4.558678in}{2.636740in}}%
\pgfpathlineto{\pgfqpoint{4.559185in}{2.813532in}}%
\pgfpathlineto{\pgfqpoint{4.558876in}{2.453391in}}%
\pgfpathlineto{\pgfqpoint{4.559780in}{2.765236in}}%
\pgfpathlineto{\pgfqpoint{4.560309in}{2.582872in}}%
\pgfpathlineto{\pgfqpoint{4.560838in}{2.881386in}}%
\pgfpathlineto{\pgfqpoint{4.560882in}{2.774961in}}%
\pgfpathlineto{\pgfqpoint{4.561279in}{2.557850in}}%
\pgfpathlineto{\pgfqpoint{4.561896in}{2.879201in}}%
\pgfpathlineto{\pgfqpoint{4.562028in}{2.639908in}}%
\pgfpathlineto{\pgfqpoint{4.562535in}{2.836259in}}%
\pgfpathlineto{\pgfqpoint{4.562910in}{2.540149in}}%
\pgfpathlineto{\pgfqpoint{4.563108in}{2.626469in}}%
\pgfpathlineto{\pgfqpoint{4.563990in}{2.468252in}}%
\pgfpathlineto{\pgfqpoint{4.563703in}{2.774087in}}%
\pgfpathlineto{\pgfqpoint{4.564166in}{2.651600in}}%
\pgfpathlineto{\pgfqpoint{4.565048in}{2.848716in}}%
\pgfpathlineto{\pgfqpoint{4.564497in}{2.517968in}}%
\pgfpathlineto{\pgfqpoint{4.565268in}{2.718034in}}%
\pgfpathlineto{\pgfqpoint{4.565819in}{2.604397in}}%
\pgfpathlineto{\pgfqpoint{4.566150in}{2.924437in}}%
\pgfpathlineto{\pgfqpoint{4.566326in}{2.734314in}}%
\pgfpathlineto{\pgfqpoint{4.566789in}{2.639471in}}%
\pgfpathlineto{\pgfqpoint{4.567450in}{2.904878in}}%
\pgfpathlineto{\pgfqpoint{4.568575in}{2.659248in}}%
\pgfpathlineto{\pgfqpoint{4.567671in}{2.974371in}}%
\pgfpathlineto{\pgfqpoint{4.568597in}{2.712461in}}%
\pgfpathlineto{\pgfqpoint{4.568663in}{2.875595in}}%
\pgfpathlineto{\pgfqpoint{4.569478in}{2.583636in}}%
\pgfpathlineto{\pgfqpoint{4.569633in}{2.662963in}}%
\pgfpathlineto{\pgfqpoint{4.569677in}{2.551840in}}%
\pgfpathlineto{\pgfqpoint{4.570250in}{2.886084in}}%
\pgfpathlineto{\pgfqpoint{4.570735in}{2.669082in}}%
\pgfpathlineto{\pgfqpoint{4.571440in}{2.862702in}}%
\pgfpathlineto{\pgfqpoint{4.571153in}{2.631604in}}%
\pgfpathlineto{\pgfqpoint{4.571837in}{2.757588in}}%
\pgfpathlineto{\pgfqpoint{4.572917in}{2.627015in}}%
\pgfpathlineto{\pgfqpoint{4.572740in}{2.918209in}}%
\pgfpathlineto{\pgfqpoint{4.572939in}{2.701097in}}%
\pgfpathlineto{\pgfqpoint{4.572961in}{2.868056in}}%
\pgfpathlineto{\pgfqpoint{4.573115in}{2.522010in}}%
\pgfpathlineto{\pgfqpoint{4.574041in}{2.656298in}}%
\pgfpathlineto{\pgfqpoint{4.575033in}{2.811565in}}%
\pgfpathlineto{\pgfqpoint{4.574614in}{2.484314in}}%
\pgfpathlineto{\pgfqpoint{4.575165in}{2.762068in}}%
\pgfpathlineto{\pgfqpoint{4.575826in}{2.551294in}}%
\pgfpathlineto{\pgfqpoint{4.575430in}{2.936784in}}%
\pgfpathlineto{\pgfqpoint{4.576289in}{2.628545in}}%
\pgfpathlineto{\pgfqpoint{4.576995in}{2.760210in}}%
\pgfpathlineto{\pgfqpoint{4.577281in}{2.485734in}}%
\pgfpathlineto{\pgfqpoint{4.577369in}{2.630074in}}%
\pgfpathlineto{\pgfqpoint{4.577480in}{2.526381in}}%
\pgfpathlineto{\pgfqpoint{4.578273in}{2.766875in}}%
\pgfpathlineto{\pgfqpoint{4.578427in}{2.632915in}}%
\pgfpathlineto{\pgfqpoint{4.578802in}{2.791788in}}%
\pgfpathlineto{\pgfqpoint{4.578560in}{2.518405in}}%
\pgfpathlineto{\pgfqpoint{4.579507in}{2.640127in}}%
\pgfpathlineto{\pgfqpoint{4.579838in}{2.569323in}}%
\pgfpathlineto{\pgfqpoint{4.580125in}{2.849371in}}%
\pgfpathlineto{\pgfqpoint{4.580609in}{2.637067in}}%
\pgfpathlineto{\pgfqpoint{4.581094in}{2.843252in}}%
\pgfpathlineto{\pgfqpoint{4.581271in}{2.516984in}}%
\pgfpathlineto{\pgfqpoint{4.581579in}{2.666460in}}%
\pgfpathlineto{\pgfqpoint{4.581800in}{2.496333in}}%
\pgfpathlineto{\pgfqpoint{4.582505in}{2.730381in}}%
\pgfpathlineto{\pgfqpoint{4.582659in}{2.599371in}}%
\pgfpathlineto{\pgfqpoint{4.582681in}{2.826972in}}%
\pgfpathlineto{\pgfqpoint{4.583343in}{2.513816in}}%
\pgfpathlineto{\pgfqpoint{4.583761in}{2.686346in}}%
\pgfpathlineto{\pgfqpoint{4.584511in}{2.502670in}}%
\pgfpathlineto{\pgfqpoint{4.584775in}{2.797142in}}%
\pgfpathlineto{\pgfqpoint{4.584864in}{2.680009in}}%
\pgfpathlineto{\pgfqpoint{4.585547in}{2.915368in}}%
\pgfpathlineto{\pgfqpoint{4.585216in}{2.531517in}}%
\pgfpathlineto{\pgfqpoint{4.586010in}{2.756932in}}%
\pgfpathlineto{\pgfqpoint{4.586737in}{2.592815in}}%
\pgfpathlineto{\pgfqpoint{4.587024in}{2.840739in}}%
\pgfpathlineto{\pgfqpoint{4.587134in}{2.712679in}}%
\pgfpathlineto{\pgfqpoint{4.588236in}{2.426949in}}%
\pgfpathlineto{\pgfqpoint{4.587509in}{2.807959in}}%
\pgfpathlineto{\pgfqpoint{4.588368in}{2.585494in}}%
\pgfpathlineto{\pgfqpoint{4.589316in}{2.753545in}}%
\pgfpathlineto{\pgfqpoint{4.589426in}{2.502670in}}%
\pgfpathlineto{\pgfqpoint{4.589448in}{2.534248in}}%
\pgfpathlineto{\pgfqpoint{4.590418in}{2.481801in}}%
\pgfpathlineto{\pgfqpoint{4.590462in}{2.683287in}}%
\pgfpathlineto{\pgfqpoint{4.590506in}{2.603195in}}%
\pgfpathlineto{\pgfqpoint{4.590595in}{2.801185in}}%
\pgfpathlineto{\pgfqpoint{4.591256in}{2.521355in}}%
\pgfpathlineto{\pgfqpoint{4.591608in}{2.711478in}}%
\pgfpathlineto{\pgfqpoint{4.592093in}{2.535669in}}%
\pgfpathlineto{\pgfqpoint{4.592688in}{2.796486in}}%
\pgfpathlineto{\pgfqpoint{4.592711in}{2.662745in}}%
\pgfpathlineto{\pgfqpoint{4.593438in}{2.777365in}}%
\pgfpathlineto{\pgfqpoint{4.592843in}{2.444104in}}%
\pgfpathlineto{\pgfqpoint{4.593769in}{2.657063in}}%
\pgfpathlineto{\pgfqpoint{4.594827in}{2.437111in}}%
\pgfpathlineto{\pgfqpoint{4.593813in}{2.719017in}}%
\pgfpathlineto{\pgfqpoint{4.594937in}{2.508899in}}%
\pgfpathlineto{\pgfqpoint{4.595752in}{2.722076in}}%
\pgfpathlineto{\pgfqpoint{4.595025in}{2.389252in}}%
\pgfpathlineto{\pgfqpoint{4.595995in}{2.466940in}}%
\pgfpathlineto{\pgfqpoint{4.596083in}{2.357784in}}%
\pgfpathlineto{\pgfqpoint{4.596700in}{2.763707in}}%
\pgfpathlineto{\pgfqpoint{4.596921in}{2.605162in}}%
\pgfpathlineto{\pgfqpoint{4.598045in}{2.895700in}}%
\pgfpathlineto{\pgfqpoint{4.597185in}{2.524305in}}%
\pgfpathlineto{\pgfqpoint{4.598067in}{2.838663in}}%
\pgfpathlineto{\pgfqpoint{4.599103in}{2.366853in}}%
\pgfpathlineto{\pgfqpoint{4.599213in}{2.522775in}}%
\pgfpathlineto{\pgfqpoint{4.599522in}{2.685363in}}%
\pgfpathlineto{\pgfqpoint{4.599632in}{2.395371in}}%
\pgfpathlineto{\pgfqpoint{4.600315in}{2.582544in}}%
\pgfpathlineto{\pgfqpoint{4.601086in}{2.348168in}}%
\pgfpathlineto{\pgfqpoint{4.600646in}{2.698912in}}%
\pgfpathlineto{\pgfqpoint{4.601417in}{2.479725in}}%
\pgfpathlineto{\pgfqpoint{4.601638in}{2.368710in}}%
\pgfpathlineto{\pgfqpoint{4.602607in}{2.763270in}}%
\pgfpathlineto{\pgfqpoint{4.603158in}{2.448474in}}%
\pgfpathlineto{\pgfqpoint{4.603732in}{2.569978in}}%
\pgfpathlineto{\pgfqpoint{4.603996in}{2.830905in}}%
\pgfpathlineto{\pgfqpoint{4.604701in}{2.480926in}}%
\pgfpathlineto{\pgfqpoint{4.604834in}{2.548125in}}%
\pgfpathlineto{\pgfqpoint{4.604900in}{2.730490in}}%
\pgfpathlineto{\pgfqpoint{4.605539in}{2.467377in}}%
\pgfpathlineto{\pgfqpoint{4.605958in}{2.597404in}}%
\pgfpathlineto{\pgfqpoint{4.606861in}{2.689078in}}%
\pgfpathlineto{\pgfqpoint{4.606134in}{2.449021in}}%
\pgfpathlineto{\pgfqpoint{4.606994in}{2.522775in}}%
\pgfpathlineto{\pgfqpoint{4.607060in}{2.449458in}}%
\pgfpathlineto{\pgfqpoint{4.607214in}{2.705031in}}%
\pgfpathlineto{\pgfqpoint{4.608008in}{2.609860in}}%
\pgfpathlineto{\pgfqpoint{4.608471in}{2.824021in}}%
\pgfpathlineto{\pgfqpoint{4.608845in}{2.500157in}}%
\pgfpathlineto{\pgfqpoint{4.609132in}{2.686565in}}%
\pgfpathlineto{\pgfqpoint{4.609969in}{2.412307in}}%
\pgfpathlineto{\pgfqpoint{4.609749in}{2.844782in}}%
\pgfpathlineto{\pgfqpoint{4.610256in}{2.564843in}}%
\pgfpathlineto{\pgfqpoint{4.610300in}{2.799218in}}%
\pgfpathlineto{\pgfqpoint{4.610719in}{2.441481in}}%
\pgfpathlineto{\pgfqpoint{4.611358in}{2.569978in}}%
\pgfpathlineto{\pgfqpoint{4.612107in}{2.839537in}}%
\pgfpathlineto{\pgfqpoint{4.612130in}{2.567247in}}%
\pgfpathlineto{\pgfqpoint{4.612526in}{2.779222in}}%
\pgfpathlineto{\pgfqpoint{4.612967in}{2.578719in}}%
\pgfpathlineto{\pgfqpoint{4.613011in}{2.829048in}}%
\pgfpathlineto{\pgfqpoint{4.613628in}{2.718908in}}%
\pgfpathlineto{\pgfqpoint{4.613805in}{2.826535in}}%
\pgfpathlineto{\pgfqpoint{4.614201in}{2.569978in}}%
\pgfpathlineto{\pgfqpoint{4.614708in}{2.740214in}}%
\pgfpathlineto{\pgfqpoint{4.615370in}{2.558833in}}%
\pgfpathlineto{\pgfqpoint{4.615259in}{2.811019in}}%
\pgfpathlineto{\pgfqpoint{4.615811in}{2.631604in}}%
\pgfpathlineto{\pgfqpoint{4.616053in}{2.859751in}}%
\pgfpathlineto{\pgfqpoint{4.616362in}{2.528020in}}%
\pgfpathlineto{\pgfqpoint{4.616913in}{2.632806in}}%
\pgfpathlineto{\pgfqpoint{4.617265in}{2.516329in}}%
\pgfpathlineto{\pgfqpoint{4.618059in}{2.828392in}}%
\pgfpathlineto{\pgfqpoint{4.618742in}{2.567684in}}%
\pgfpathlineto{\pgfqpoint{4.618610in}{2.832544in}}%
\pgfpathlineto{\pgfqpoint{4.619161in}{2.669192in}}%
\pgfpathlineto{\pgfqpoint{4.620241in}{2.842487in}}%
\pgfpathlineto{\pgfqpoint{4.620087in}{2.540367in}}%
\pgfpathlineto{\pgfqpoint{4.620263in}{2.720765in}}%
\pgfpathlineto{\pgfqpoint{4.620572in}{2.604178in}}%
\pgfpathlineto{\pgfqpoint{4.620660in}{2.890455in}}%
\pgfpathlineto{\pgfqpoint{4.621233in}{2.724917in}}%
\pgfpathlineto{\pgfqpoint{4.621255in}{2.853305in}}%
\pgfpathlineto{\pgfqpoint{4.621784in}{2.511958in}}%
\pgfpathlineto{\pgfqpoint{4.622313in}{2.757588in}}%
\pgfpathlineto{\pgfqpoint{4.623349in}{2.474371in}}%
\pgfpathlineto{\pgfqpoint{4.623195in}{2.846312in}}%
\pgfpathlineto{\pgfqpoint{4.623415in}{2.657609in}}%
\pgfpathlineto{\pgfqpoint{4.624231in}{2.817247in}}%
\pgfpathlineto{\pgfqpoint{4.623966in}{2.526272in}}%
\pgfpathlineto{\pgfqpoint{4.624495in}{2.704485in}}%
\pgfpathlineto{\pgfqpoint{4.624914in}{2.433833in}}%
\pgfpathlineto{\pgfqpoint{4.625311in}{2.752780in}}%
\pgfpathlineto{\pgfqpoint{4.625619in}{2.608003in}}%
\pgfpathlineto{\pgfqpoint{4.626391in}{2.519934in}}%
\pgfpathlineto{\pgfqpoint{4.626721in}{2.776163in}}%
\pgfpathlineto{\pgfqpoint{4.626809in}{2.523868in}}%
\pgfpathlineto{\pgfqpoint{4.627294in}{2.798562in}}%
\pgfpathlineto{\pgfqpoint{4.627823in}{2.646027in}}%
\pgfpathlineto{\pgfqpoint{4.628507in}{2.938095in}}%
\pgfpathlineto{\pgfqpoint{4.628286in}{2.518077in}}%
\pgfpathlineto{\pgfqpoint{4.628926in}{2.812221in}}%
\pgfpathlineto{\pgfqpoint{4.629719in}{2.493929in}}%
\pgfpathlineto{\pgfqpoint{4.629190in}{2.890564in}}%
\pgfpathlineto{\pgfqpoint{4.630050in}{2.681757in}}%
\pgfpathlineto{\pgfqpoint{4.630380in}{2.812658in}}%
\pgfpathlineto{\pgfqpoint{4.630711in}{2.604943in}}%
\pgfpathlineto{\pgfqpoint{4.631042in}{2.639799in}}%
\pgfpathlineto{\pgfqpoint{4.631350in}{2.582872in}}%
\pgfpathlineto{\pgfqpoint{4.631923in}{2.853960in}}%
\pgfpathlineto{\pgfqpoint{4.632078in}{2.731036in}}%
\pgfpathlineto{\pgfqpoint{4.633003in}{2.885429in}}%
\pgfpathlineto{\pgfqpoint{4.632695in}{2.535122in}}%
\pgfpathlineto{\pgfqpoint{4.633180in}{2.729834in}}%
\pgfpathlineto{\pgfqpoint{4.633907in}{2.563969in}}%
\pgfpathlineto{\pgfqpoint{4.633819in}{2.854725in}}%
\pgfpathlineto{\pgfqpoint{4.634282in}{2.612483in}}%
\pgfpathlineto{\pgfqpoint{4.634789in}{2.846421in}}%
\pgfpathlineto{\pgfqpoint{4.635163in}{2.499720in}}%
\pgfpathlineto{\pgfqpoint{4.635406in}{2.701534in}}%
\pgfpathlineto{\pgfqpoint{4.635759in}{2.536543in}}%
\pgfpathlineto{\pgfqpoint{4.636067in}{2.930119in}}%
\pgfpathlineto{\pgfqpoint{4.636464in}{2.715739in}}%
\pgfpathlineto{\pgfqpoint{4.636552in}{2.851556in}}%
\pgfpathlineto{\pgfqpoint{4.636817in}{2.595328in}}%
\pgfpathlineto{\pgfqpoint{4.637566in}{2.704375in}}%
\pgfpathlineto{\pgfqpoint{4.638470in}{2.882807in}}%
\pgfpathlineto{\pgfqpoint{4.637786in}{2.625048in}}%
\pgfpathlineto{\pgfqpoint{4.638712in}{2.741526in}}%
\pgfpathlineto{\pgfqpoint{4.639395in}{2.558396in}}%
\pgfpathlineto{\pgfqpoint{4.638999in}{2.833965in}}%
\pgfpathlineto{\pgfqpoint{4.639770in}{2.593798in}}%
\pgfpathlineto{\pgfqpoint{4.639792in}{2.852758in}}%
\pgfpathlineto{\pgfqpoint{4.640255in}{2.500485in}}%
\pgfpathlineto{\pgfqpoint{4.640872in}{2.659467in}}%
\pgfpathlineto{\pgfqpoint{4.641027in}{2.520918in}}%
\pgfpathlineto{\pgfqpoint{4.641644in}{2.781408in}}%
\pgfpathlineto{\pgfqpoint{4.641952in}{2.550857in}}%
\pgfpathlineto{\pgfqpoint{4.642636in}{2.822055in}}%
\pgfpathlineto{\pgfqpoint{4.642437in}{2.478086in}}%
\pgfpathlineto{\pgfqpoint{4.643076in}{2.635647in}}%
\pgfpathlineto{\pgfqpoint{4.643407in}{2.520481in}}%
\pgfpathlineto{\pgfqpoint{4.643165in}{2.773759in}}%
\pgfpathlineto{\pgfqpoint{4.644135in}{2.690499in}}%
\pgfpathlineto{\pgfqpoint{4.645237in}{2.892859in}}%
\pgfpathlineto{\pgfqpoint{4.644641in}{2.599917in}}%
\pgfpathlineto{\pgfqpoint{4.645303in}{2.805118in}}%
\pgfpathlineto{\pgfqpoint{4.645457in}{2.630839in}}%
\pgfpathlineto{\pgfqpoint{4.646030in}{2.888270in}}%
\pgfpathlineto{\pgfqpoint{4.646405in}{2.754310in}}%
\pgfpathlineto{\pgfqpoint{4.646934in}{2.878873in}}%
\pgfpathlineto{\pgfqpoint{4.646735in}{2.569323in}}%
\pgfpathlineto{\pgfqpoint{4.647397in}{2.724262in}}%
\pgfpathlineto{\pgfqpoint{4.648367in}{2.546923in}}%
\pgfpathlineto{\pgfqpoint{4.647441in}{2.846530in}}%
\pgfpathlineto{\pgfqpoint{4.648477in}{2.743165in}}%
\pgfpathlineto{\pgfqpoint{4.648719in}{2.840739in}}%
\pgfpathlineto{\pgfqpoint{4.648565in}{2.585822in}}%
\pgfpathlineto{\pgfqpoint{4.649469in}{2.700988in}}%
\pgfpathlineto{\pgfqpoint{4.649491in}{2.523868in}}%
\pgfpathlineto{\pgfqpoint{4.650218in}{2.910779in}}%
\pgfpathlineto{\pgfqpoint{4.650571in}{2.762286in}}%
\pgfpathlineto{\pgfqpoint{4.651298in}{2.584729in}}%
\pgfpathlineto{\pgfqpoint{4.650747in}{2.846421in}}%
\pgfpathlineto{\pgfqpoint{4.651717in}{2.641110in}}%
\pgfpathlineto{\pgfqpoint{4.652775in}{2.836259in}}%
\pgfpathlineto{\pgfqpoint{4.652400in}{2.553261in}}%
\pgfpathlineto{\pgfqpoint{4.652841in}{2.731692in}}%
\pgfpathlineto{\pgfqpoint{4.653767in}{2.886303in}}%
\pgfpathlineto{\pgfqpoint{4.653657in}{2.538947in}}%
\pgfpathlineto{\pgfqpoint{4.653877in}{2.705140in}}%
\pgfpathlineto{\pgfqpoint{4.654582in}{2.541897in}}%
\pgfpathlineto{\pgfqpoint{4.653987in}{2.794301in}}%
\pgfpathlineto{\pgfqpoint{4.654935in}{2.672470in}}%
\pgfpathlineto{\pgfqpoint{4.655420in}{2.775398in}}%
\pgfpathlineto{\pgfqpoint{4.655442in}{2.470655in}}%
\pgfpathlineto{\pgfqpoint{4.656015in}{2.671486in}}%
\pgfpathlineto{\pgfqpoint{4.656632in}{2.577080in}}%
\pgfpathlineto{\pgfqpoint{4.656919in}{2.858877in}}%
\pgfpathlineto{\pgfqpoint{4.656963in}{2.783265in}}%
\pgfpathlineto{\pgfqpoint{4.657161in}{2.864341in}}%
\pgfpathlineto{\pgfqpoint{4.657690in}{2.590192in}}%
\pgfpathlineto{\pgfqpoint{4.657977in}{2.697164in}}%
\pgfpathlineto{\pgfqpoint{4.657999in}{2.532063in}}%
\pgfpathlineto{\pgfqpoint{4.658925in}{2.872426in}}%
\pgfpathlineto{\pgfqpoint{4.659079in}{2.631604in}}%
\pgfpathlineto{\pgfqpoint{4.659850in}{2.826535in}}%
\pgfpathlineto{\pgfqpoint{4.659608in}{2.496770in}}%
\pgfpathlineto{\pgfqpoint{4.660203in}{2.732675in}}%
\pgfpathlineto{\pgfqpoint{4.661129in}{2.552496in}}%
\pgfpathlineto{\pgfqpoint{4.661217in}{2.842160in}}%
\pgfpathlineto{\pgfqpoint{4.661283in}{2.787417in}}%
\pgfpathlineto{\pgfqpoint{4.661834in}{2.942247in}}%
\pgfpathlineto{\pgfqpoint{4.661724in}{2.544410in}}%
\pgfpathlineto{\pgfqpoint{4.662297in}{2.789275in}}%
\pgfpathlineto{\pgfqpoint{4.662870in}{2.606801in}}%
\pgfpathlineto{\pgfqpoint{4.663223in}{2.946290in}}%
\pgfpathlineto{\pgfqpoint{4.663421in}{2.662089in}}%
\pgfpathlineto{\pgfqpoint{4.664325in}{2.837898in}}%
\pgfpathlineto{\pgfqpoint{4.664259in}{2.625376in}}%
\pgfpathlineto{\pgfqpoint{4.664523in}{2.676512in}}%
\pgfpathlineto{\pgfqpoint{4.664678in}{2.501687in}}%
\pgfpathlineto{\pgfqpoint{4.664920in}{2.814625in}}%
\pgfpathlineto{\pgfqpoint{4.665670in}{2.535778in}}%
\pgfpathlineto{\pgfqpoint{4.666331in}{2.789712in}}%
\pgfpathlineto{\pgfqpoint{4.666199in}{2.498628in}}%
\pgfpathlineto{\pgfqpoint{4.666816in}{2.770372in}}%
\pgfusepath{stroke}%
\end{pgfscope}%
\begin{pgfscope}%
\pgfsetrectcap%
\pgfsetmiterjoin%
\pgfsetlinewidth{0.803000pt}%
\definecolor{currentstroke}{rgb}{0.000000,0.000000,0.000000}%
\pgfsetstrokecolor{currentstroke}%
\pgfsetdash{}{0pt}%
\pgfpathmoveto{\pgfqpoint{0.667540in}{0.539544in}}%
\pgfpathlineto{\pgfqpoint{0.667540in}{3.120077in}}%
\pgfusepath{stroke}%
\end{pgfscope}%
\begin{pgfscope}%
\pgfsetrectcap%
\pgfsetmiterjoin%
\pgfsetlinewidth{0.803000pt}%
\definecolor{currentstroke}{rgb}{0.000000,0.000000,0.000000}%
\pgfsetstrokecolor{currentstroke}%
\pgfsetdash{}{0pt}%
\pgfpathmoveto{\pgfqpoint{4.857257in}{0.539544in}}%
\pgfpathlineto{\pgfqpoint{4.857257in}{3.120077in}}%
\pgfusepath{stroke}%
\end{pgfscope}%
\begin{pgfscope}%
\pgfsetrectcap%
\pgfsetmiterjoin%
\pgfsetlinewidth{0.803000pt}%
\definecolor{currentstroke}{rgb}{0.000000,0.000000,0.000000}%
\pgfsetstrokecolor{currentstroke}%
\pgfsetdash{}{0pt}%
\pgfpathmoveto{\pgfqpoint{0.667540in}{0.539544in}}%
\pgfpathlineto{\pgfqpoint{4.857257in}{0.539544in}}%
\pgfusepath{stroke}%
\end{pgfscope}%
\begin{pgfscope}%
\pgfsetrectcap%
\pgfsetmiterjoin%
\pgfsetlinewidth{0.803000pt}%
\definecolor{currentstroke}{rgb}{0.000000,0.000000,0.000000}%
\pgfsetstrokecolor{currentstroke}%
\pgfsetdash{}{0pt}%
\pgfpathmoveto{\pgfqpoint{0.667540in}{3.120077in}}%
\pgfpathlineto{\pgfqpoint{4.857257in}{3.120077in}}%
\pgfusepath{stroke}%
\end{pgfscope}%
\begin{pgfscope}%
\pgfpathrectangle{\pgfqpoint{0.667540in}{0.539544in}}{\pgfqpoint{4.189718in}{2.580533in}}%
\pgfusepath{clip}%
\pgfsetrectcap%
\pgfsetroundjoin%
\pgfsetlinewidth{0.803000pt}%
\definecolor{currentstroke}{rgb}{0.450000,0.450000,0.450000}%
\pgfsetstrokecolor{currentstroke}%
\pgfsetdash{}{0pt}%
\pgfpathmoveto{\pgfqpoint{0.667540in}{0.796098in}}%
\pgfpathlineto{\pgfqpoint{4.857257in}{0.796098in}}%
\pgfusepath{stroke}%
\end{pgfscope}%
\begin{pgfscope}%
\pgfsetbuttcap%
\pgfsetroundjoin%
\definecolor{currentfill}{rgb}{0.000000,0.000000,0.000000}%
\pgfsetfillcolor{currentfill}%
\pgfsetlinewidth{0.803000pt}%
\definecolor{currentstroke}{rgb}{0.000000,0.000000,0.000000}%
\pgfsetstrokecolor{currentstroke}%
\pgfsetdash{}{0pt}%
\pgfsys@defobject{currentmarker}{\pgfqpoint{0.000000in}{0.000000in}}{\pgfqpoint{0.048611in}{0.000000in}}{%
\pgfpathmoveto{\pgfqpoint{0.000000in}{0.000000in}}%
\pgfpathlineto{\pgfqpoint{0.048611in}{0.000000in}}%
\pgfusepath{stroke,fill}%
}%
\begin{pgfscope}%
\pgfsys@transformshift{4.857257in}{0.796098in}%
\pgfsys@useobject{currentmarker}{}%
\end{pgfscope}%
\end{pgfscope}%
\begin{pgfscope}%
\definecolor{textcolor}{rgb}{0.000000,0.000000,0.000000}%
\pgfsetstrokecolor{textcolor}%
\pgfsetfillcolor{textcolor}%
\pgftext[x=4.954480in, y=0.757542in, left, base]{\color{textcolor}\rmfamily\fontsize{8.000000}{9.600000}\selectfont \(\displaystyle {20.0}\)}%
\end{pgfscope}%
\begin{pgfscope}%
\pgfpathrectangle{\pgfqpoint{0.667540in}{0.539544in}}{\pgfqpoint{4.189718in}{2.580533in}}%
\pgfusepath{clip}%
\pgfsetrectcap%
\pgfsetroundjoin%
\pgfsetlinewidth{0.803000pt}%
\definecolor{currentstroke}{rgb}{0.450000,0.450000,0.450000}%
\pgfsetstrokecolor{currentstroke}%
\pgfsetdash{}{0pt}%
\pgfpathmoveto{\pgfqpoint{0.667540in}{1.331700in}}%
\pgfpathlineto{\pgfqpoint{4.857257in}{1.331700in}}%
\pgfusepath{stroke}%
\end{pgfscope}%
\begin{pgfscope}%
\pgfsetbuttcap%
\pgfsetroundjoin%
\definecolor{currentfill}{rgb}{0.000000,0.000000,0.000000}%
\pgfsetfillcolor{currentfill}%
\pgfsetlinewidth{0.803000pt}%
\definecolor{currentstroke}{rgb}{0.000000,0.000000,0.000000}%
\pgfsetstrokecolor{currentstroke}%
\pgfsetdash{}{0pt}%
\pgfsys@defobject{currentmarker}{\pgfqpoint{0.000000in}{0.000000in}}{\pgfqpoint{0.048611in}{0.000000in}}{%
\pgfpathmoveto{\pgfqpoint{0.000000in}{0.000000in}}%
\pgfpathlineto{\pgfqpoint{0.048611in}{0.000000in}}%
\pgfusepath{stroke,fill}%
}%
\begin{pgfscope}%
\pgfsys@transformshift{4.857257in}{1.331700in}%
\pgfsys@useobject{currentmarker}{}%
\end{pgfscope}%
\end{pgfscope}%
\begin{pgfscope}%
\definecolor{textcolor}{rgb}{0.000000,0.000000,0.000000}%
\pgfsetstrokecolor{textcolor}%
\pgfsetfillcolor{textcolor}%
\pgftext[x=4.954480in, y=1.293145in, left, base]{\color{textcolor}\rmfamily\fontsize{8.000000}{9.600000}\selectfont \(\displaystyle {20.5}\)}%
\end{pgfscope}%
\begin{pgfscope}%
\pgfpathrectangle{\pgfqpoint{0.667540in}{0.539544in}}{\pgfqpoint{4.189718in}{2.580533in}}%
\pgfusepath{clip}%
\pgfsetrectcap%
\pgfsetroundjoin%
\pgfsetlinewidth{0.803000pt}%
\definecolor{currentstroke}{rgb}{0.450000,0.450000,0.450000}%
\pgfsetstrokecolor{currentstroke}%
\pgfsetdash{}{0pt}%
\pgfpathmoveto{\pgfqpoint{0.667540in}{1.867303in}}%
\pgfpathlineto{\pgfqpoint{4.857257in}{1.867303in}}%
\pgfusepath{stroke}%
\end{pgfscope}%
\begin{pgfscope}%
\pgfsetbuttcap%
\pgfsetroundjoin%
\definecolor{currentfill}{rgb}{0.000000,0.000000,0.000000}%
\pgfsetfillcolor{currentfill}%
\pgfsetlinewidth{0.803000pt}%
\definecolor{currentstroke}{rgb}{0.000000,0.000000,0.000000}%
\pgfsetstrokecolor{currentstroke}%
\pgfsetdash{}{0pt}%
\pgfsys@defobject{currentmarker}{\pgfqpoint{0.000000in}{0.000000in}}{\pgfqpoint{0.048611in}{0.000000in}}{%
\pgfpathmoveto{\pgfqpoint{0.000000in}{0.000000in}}%
\pgfpathlineto{\pgfqpoint{0.048611in}{0.000000in}}%
\pgfusepath{stroke,fill}%
}%
\begin{pgfscope}%
\pgfsys@transformshift{4.857257in}{1.867303in}%
\pgfsys@useobject{currentmarker}{}%
\end{pgfscope}%
\end{pgfscope}%
\begin{pgfscope}%
\definecolor{textcolor}{rgb}{0.000000,0.000000,0.000000}%
\pgfsetstrokecolor{textcolor}%
\pgfsetfillcolor{textcolor}%
\pgftext[x=4.954480in, y=1.828747in, left, base]{\color{textcolor}\rmfamily\fontsize{8.000000}{9.600000}\selectfont \(\displaystyle {21.0}\)}%
\end{pgfscope}%
\begin{pgfscope}%
\pgfpathrectangle{\pgfqpoint{0.667540in}{0.539544in}}{\pgfqpoint{4.189718in}{2.580533in}}%
\pgfusepath{clip}%
\pgfsetrectcap%
\pgfsetroundjoin%
\pgfsetlinewidth{0.803000pt}%
\definecolor{currentstroke}{rgb}{0.450000,0.450000,0.450000}%
\pgfsetstrokecolor{currentstroke}%
\pgfsetdash{}{0pt}%
\pgfpathmoveto{\pgfqpoint{0.667540in}{2.402906in}}%
\pgfpathlineto{\pgfqpoint{4.857257in}{2.402906in}}%
\pgfusepath{stroke}%
\end{pgfscope}%
\begin{pgfscope}%
\pgfsetbuttcap%
\pgfsetroundjoin%
\definecolor{currentfill}{rgb}{0.000000,0.000000,0.000000}%
\pgfsetfillcolor{currentfill}%
\pgfsetlinewidth{0.803000pt}%
\definecolor{currentstroke}{rgb}{0.000000,0.000000,0.000000}%
\pgfsetstrokecolor{currentstroke}%
\pgfsetdash{}{0pt}%
\pgfsys@defobject{currentmarker}{\pgfqpoint{0.000000in}{0.000000in}}{\pgfqpoint{0.048611in}{0.000000in}}{%
\pgfpathmoveto{\pgfqpoint{0.000000in}{0.000000in}}%
\pgfpathlineto{\pgfqpoint{0.048611in}{0.000000in}}%
\pgfusepath{stroke,fill}%
}%
\begin{pgfscope}%
\pgfsys@transformshift{4.857257in}{2.402906in}%
\pgfsys@useobject{currentmarker}{}%
\end{pgfscope}%
\end{pgfscope}%
\begin{pgfscope}%
\definecolor{textcolor}{rgb}{0.000000,0.000000,0.000000}%
\pgfsetstrokecolor{textcolor}%
\pgfsetfillcolor{textcolor}%
\pgftext[x=4.954480in, y=2.364350in, left, base]{\color{textcolor}\rmfamily\fontsize{8.000000}{9.600000}\selectfont \(\displaystyle {21.5}\)}%
\end{pgfscope}%
\begin{pgfscope}%
\pgfpathrectangle{\pgfqpoint{0.667540in}{0.539544in}}{\pgfqpoint{4.189718in}{2.580533in}}%
\pgfusepath{clip}%
\pgfsetrectcap%
\pgfsetroundjoin%
\pgfsetlinewidth{0.803000pt}%
\definecolor{currentstroke}{rgb}{0.450000,0.450000,0.450000}%
\pgfsetstrokecolor{currentstroke}%
\pgfsetdash{}{0pt}%
\pgfpathmoveto{\pgfqpoint{0.667540in}{2.938508in}}%
\pgfpathlineto{\pgfqpoint{4.857257in}{2.938508in}}%
\pgfusepath{stroke}%
\end{pgfscope}%
\begin{pgfscope}%
\pgfsetbuttcap%
\pgfsetroundjoin%
\definecolor{currentfill}{rgb}{0.000000,0.000000,0.000000}%
\pgfsetfillcolor{currentfill}%
\pgfsetlinewidth{0.803000pt}%
\definecolor{currentstroke}{rgb}{0.000000,0.000000,0.000000}%
\pgfsetstrokecolor{currentstroke}%
\pgfsetdash{}{0pt}%
\pgfsys@defobject{currentmarker}{\pgfqpoint{0.000000in}{0.000000in}}{\pgfqpoint{0.048611in}{0.000000in}}{%
\pgfpathmoveto{\pgfqpoint{0.000000in}{0.000000in}}%
\pgfpathlineto{\pgfqpoint{0.048611in}{0.000000in}}%
\pgfusepath{stroke,fill}%
}%
\begin{pgfscope}%
\pgfsys@transformshift{4.857257in}{2.938508in}%
\pgfsys@useobject{currentmarker}{}%
\end{pgfscope}%
\end{pgfscope}%
\begin{pgfscope}%
\definecolor{textcolor}{rgb}{0.000000,0.000000,0.000000}%
\pgfsetstrokecolor{textcolor}%
\pgfsetfillcolor{textcolor}%
\pgftext[x=4.954480in, y=2.899953in, left, base]{\color{textcolor}\rmfamily\fontsize{8.000000}{9.600000}\selectfont \(\displaystyle {22.0}\)}%
\end{pgfscope}%
\begin{pgfscope}%
\definecolor{textcolor}{rgb}{0.000000,0.000000,0.000000}%
\pgfsetstrokecolor{textcolor}%
\pgfsetfillcolor{textcolor}%
\pgftext[x=5.219915in,y=1.829811in,,top,rotate=90.000000]{\color{textcolor}\rmfamily\fontsize{10.000000}{12.000000}\selectfont Temperature in \unit{\celsius}}%
\end{pgfscope}%
\begin{pgfscope}%
\pgfpathrectangle{\pgfqpoint{0.667540in}{0.539544in}}{\pgfqpoint{4.189718in}{2.580533in}}%
\pgfusepath{clip}%
\pgfsetrectcap%
\pgfsetroundjoin%
\pgfsetlinewidth{1.505625pt}%
\definecolor{currentstroke}{rgb}{0.909804,0.000000,0.043137}%
\pgfsetstrokecolor{currentstroke}%
\pgfsetstrokeopacity{0.700000}%
\pgfsetdash{}{0pt}%
\pgfpathmoveto{\pgfqpoint{0.858094in}{2.670707in}}%
\pgfpathlineto{\pgfqpoint{0.860740in}{2.595723in}}%
\pgfpathlineto{\pgfqpoint{0.863385in}{2.670707in}}%
\pgfpathlineto{\pgfqpoint{0.866030in}{2.595723in}}%
\pgfpathlineto{\pgfqpoint{0.868719in}{2.670707in}}%
\pgfpathlineto{\pgfqpoint{0.871364in}{2.595723in}}%
\pgfpathlineto{\pgfqpoint{0.874009in}{2.670707in}}%
\pgfpathlineto{\pgfqpoint{0.876742in}{2.595723in}}%
\pgfpathlineto{\pgfqpoint{0.879387in}{2.670707in}}%
\pgfpathlineto{\pgfqpoint{0.912626in}{2.734979in}}%
\pgfpathlineto{\pgfqpoint{0.915271in}{2.670707in}}%
\pgfpathlineto{\pgfqpoint{0.917961in}{2.734979in}}%
\pgfpathlineto{\pgfqpoint{0.920606in}{2.670707in}}%
\pgfpathlineto{\pgfqpoint{0.923251in}{2.734979in}}%
\pgfpathlineto{\pgfqpoint{0.925984in}{2.670707in}}%
\pgfpathlineto{\pgfqpoint{0.928629in}{2.734979in}}%
\pgfpathlineto{\pgfqpoint{0.952346in}{2.799252in}}%
\pgfpathlineto{\pgfqpoint{0.954991in}{2.734979in}}%
\pgfpathlineto{\pgfqpoint{0.957680in}{2.799252in}}%
\pgfpathlineto{\pgfqpoint{0.960325in}{2.734979in}}%
\pgfpathlineto{\pgfqpoint{0.963103in}{2.799252in}}%
\pgfpathlineto{\pgfqpoint{0.965748in}{2.734979in}}%
\pgfpathlineto{\pgfqpoint{0.968393in}{2.799252in}}%
\pgfpathlineto{\pgfqpoint{0.971038in}{2.734979in}}%
\pgfpathlineto{\pgfqpoint{0.973683in}{2.799252in}}%
\pgfpathlineto{\pgfqpoint{0.976328in}{2.734979in}}%
\pgfpathlineto{\pgfqpoint{0.978973in}{2.799252in}}%
\pgfpathlineto{\pgfqpoint{0.998061in}{2.863524in}}%
\pgfpathlineto{\pgfqpoint{1.000706in}{2.799252in}}%
\pgfpathlineto{\pgfqpoint{1.003439in}{2.863524in}}%
\pgfpathlineto{\pgfqpoint{1.006084in}{2.799252in}}%
\pgfpathlineto{\pgfqpoint{1.008729in}{2.863524in}}%
\pgfpathlineto{\pgfqpoint{1.011374in}{2.799252in}}%
\pgfpathlineto{\pgfqpoint{1.014196in}{2.863524in}}%
\pgfpathlineto{\pgfqpoint{1.016841in}{2.799252in}}%
\pgfpathlineto{\pgfqpoint{1.019530in}{2.863524in}}%
\pgfpathlineto{\pgfqpoint{1.022175in}{2.799252in}}%
\pgfpathlineto{\pgfqpoint{1.024864in}{2.863524in}}%
\pgfpathlineto{\pgfqpoint{1.027509in}{2.799252in}}%
\pgfpathlineto{\pgfqpoint{1.030154in}{2.863524in}}%
\pgfpathlineto{\pgfqpoint{1.032932in}{2.799252in}}%
\pgfpathlineto{\pgfqpoint{1.035577in}{2.863524in}}%
\pgfpathlineto{\pgfqpoint{1.039236in}{2.799252in}}%
\pgfpathlineto{\pgfqpoint{1.041881in}{2.863524in}}%
\pgfpathlineto{\pgfqpoint{1.098661in}{2.938508in}}%
\pgfpathlineto{\pgfqpoint{1.101306in}{2.863524in}}%
\pgfpathlineto{\pgfqpoint{1.104083in}{2.938508in}}%
\pgfpathlineto{\pgfqpoint{1.106728in}{2.863524in}}%
\pgfpathlineto{\pgfqpoint{1.109373in}{2.938508in}}%
\pgfpathlineto{\pgfqpoint{1.112018in}{2.863524in}}%
\pgfpathlineto{\pgfqpoint{1.122687in}{2.863524in}}%
\pgfpathlineto{\pgfqpoint{1.125332in}{2.938508in}}%
\pgfpathlineto{\pgfqpoint{1.128109in}{2.863524in}}%
\pgfpathlineto{\pgfqpoint{1.130754in}{2.938508in}}%
\pgfpathlineto{\pgfqpoint{1.134325in}{2.863524in}}%
\pgfpathlineto{\pgfqpoint{1.136970in}{2.938508in}}%
\pgfpathlineto{\pgfqpoint{1.143186in}{2.863524in}}%
\pgfpathlineto{\pgfqpoint{1.145831in}{2.938508in}}%
\pgfpathlineto{\pgfqpoint{1.195469in}{3.002780in}}%
\pgfpathlineto{\pgfqpoint{1.198114in}{2.938508in}}%
\pgfpathlineto{\pgfqpoint{1.206490in}{3.002780in}}%
\pgfpathlineto{\pgfqpoint{1.209135in}{2.938508in}}%
\pgfpathlineto{\pgfqpoint{1.212794in}{3.002780in}}%
\pgfpathlineto{\pgfqpoint{1.215439in}{2.938508in}}%
\pgfpathlineto{\pgfqpoint{1.218084in}{3.002780in}}%
\pgfpathlineto{\pgfqpoint{1.220729in}{2.938508in}}%
\pgfpathlineto{\pgfqpoint{1.223374in}{3.002780in}}%
\pgfpathlineto{\pgfqpoint{1.226063in}{2.938508in}}%
\pgfpathlineto{\pgfqpoint{1.228709in}{3.002780in}}%
\pgfpathlineto{\pgfqpoint{1.231354in}{2.938508in}}%
\pgfpathlineto{\pgfqpoint{1.233999in}{3.002780in}}%
\pgfpathlineto{\pgfqpoint{1.236864in}{2.938508in}}%
\pgfpathlineto{\pgfqpoint{1.239509in}{3.002780in}}%
\pgfpathlineto{\pgfqpoint{1.242815in}{2.938508in}}%
\pgfpathlineto{\pgfqpoint{1.245460in}{3.002780in}}%
\pgfpathlineto{\pgfqpoint{1.250001in}{2.938508in}}%
\pgfpathlineto{\pgfqpoint{1.252646in}{3.002780in}}%
\pgfpathlineto{\pgfqpoint{1.255467in}{2.938508in}}%
\pgfpathlineto{\pgfqpoint{1.258113in}{3.002780in}}%
\pgfpathlineto{\pgfqpoint{1.286943in}{2.938508in}}%
\pgfpathlineto{\pgfqpoint{1.294879in}{2.734979in}}%
\pgfpathlineto{\pgfqpoint{1.305459in}{2.467178in}}%
\pgfpathlineto{\pgfqpoint{1.308104in}{2.327921in}}%
\pgfpathlineto{\pgfqpoint{1.333540in}{1.728046in}}%
\pgfpathlineto{\pgfqpoint{1.346325in}{1.524517in}}%
\pgfpathlineto{\pgfqpoint{1.351747in}{1.460245in}}%
\pgfpathlineto{\pgfqpoint{1.354656in}{1.524517in}}%
\pgfpathlineto{\pgfqpoint{1.359947in}{1.395973in}}%
\pgfpathlineto{\pgfqpoint{1.365325in}{1.331700in}}%
\pgfpathlineto{\pgfqpoint{1.368102in}{1.395973in}}%
\pgfpathlineto{\pgfqpoint{1.373392in}{1.256716in}}%
\pgfpathlineto{\pgfqpoint{1.380578in}{1.192444in}}%
\pgfpathlineto{\pgfqpoint{1.383223in}{1.256716in}}%
\pgfpathlineto{\pgfqpoint{1.385868in}{1.192444in}}%
\pgfpathlineto{\pgfqpoint{1.394597in}{1.128171in}}%
\pgfpathlineto{\pgfqpoint{1.397242in}{1.192444in}}%
\pgfpathlineto{\pgfqpoint{1.399975in}{1.128171in}}%
\pgfpathlineto{\pgfqpoint{1.402620in}{1.192444in}}%
\pgfpathlineto{\pgfqpoint{1.405265in}{1.128171in}}%
\pgfpathlineto{\pgfqpoint{1.407954in}{1.192444in}}%
\pgfpathlineto{\pgfqpoint{1.410599in}{1.128171in}}%
\pgfpathlineto{\pgfqpoint{1.413376in}{1.192444in}}%
\pgfpathlineto{\pgfqpoint{1.416021in}{1.128171in}}%
\pgfpathlineto{\pgfqpoint{1.420342in}{1.063899in}}%
\pgfpathlineto{\pgfqpoint{1.422987in}{1.128171in}}%
\pgfpathlineto{\pgfqpoint{1.425632in}{1.063899in}}%
\pgfpathlineto{\pgfqpoint{1.430437in}{0.988915in}}%
\pgfpathlineto{\pgfqpoint{1.433214in}{1.063899in}}%
\pgfpathlineto{\pgfqpoint{1.438769in}{0.924643in}}%
\pgfpathlineto{\pgfqpoint{1.441678in}{0.988915in}}%
\pgfpathlineto{\pgfqpoint{1.447012in}{0.860370in}}%
\pgfpathlineto{\pgfqpoint{1.449657in}{0.924643in}}%
\pgfpathlineto{\pgfqpoint{1.452302in}{0.860370in}}%
\pgfpathlineto{\pgfqpoint{1.461340in}{0.796098in}}%
\pgfpathlineto{\pgfqpoint{1.463985in}{0.860370in}}%
\pgfpathlineto{\pgfqpoint{1.466630in}{0.796098in}}%
\pgfpathlineto{\pgfqpoint{1.470685in}{0.860370in}}%
\pgfpathlineto{\pgfqpoint{1.473331in}{0.796098in}}%
\pgfpathlineto{\pgfqpoint{1.476064in}{0.860370in}}%
\pgfpathlineto{\pgfqpoint{1.478709in}{0.796098in}}%
\pgfpathlineto{\pgfqpoint{1.509479in}{0.721114in}}%
\pgfpathlineto{\pgfqpoint{1.512124in}{0.796098in}}%
\pgfpathlineto{\pgfqpoint{1.514814in}{0.721114in}}%
\pgfpathlineto{\pgfqpoint{1.517459in}{0.796098in}}%
\pgfpathlineto{\pgfqpoint{1.520104in}{0.721114in}}%
\pgfpathlineto{\pgfqpoint{1.522793in}{0.796098in}}%
\pgfpathlineto{\pgfqpoint{1.525438in}{0.721114in}}%
\pgfpathlineto{\pgfqpoint{1.544702in}{0.796098in}}%
\pgfpathlineto{\pgfqpoint{1.548802in}{0.860370in}}%
\pgfpathlineto{\pgfqpoint{1.564673in}{1.256716in}}%
\pgfpathlineto{\pgfqpoint{1.567318in}{1.395973in}}%
\pgfpathlineto{\pgfqpoint{1.584510in}{1.792319in}}%
\pgfpathlineto{\pgfqpoint{1.599587in}{1.995848in}}%
\pgfpathlineto{\pgfqpoint{1.606288in}{2.060120in}}%
\pgfpathlineto{\pgfqpoint{1.612636in}{2.135104in}}%
\pgfpathlineto{\pgfqpoint{1.619954in}{2.199377in}}%
\pgfpathlineto{\pgfqpoint{1.622643in}{2.135104in}}%
\pgfpathlineto{\pgfqpoint{1.628242in}{2.263649in}}%
\pgfpathlineto{\pgfqpoint{1.631063in}{2.199377in}}%
\pgfpathlineto{\pgfqpoint{1.633708in}{2.263649in}}%
\pgfpathlineto{\pgfqpoint{1.640806in}{2.327921in}}%
\pgfpathlineto{\pgfqpoint{1.643451in}{2.263649in}}%
\pgfpathlineto{\pgfqpoint{1.646096in}{2.327921in}}%
\pgfpathlineto{\pgfqpoint{1.655574in}{2.402906in}}%
\pgfpathlineto{\pgfqpoint{1.658219in}{2.327921in}}%
\pgfpathlineto{\pgfqpoint{1.660864in}{2.402906in}}%
\pgfpathlineto{\pgfqpoint{1.670430in}{2.467178in}}%
\pgfpathlineto{\pgfqpoint{1.673075in}{2.402906in}}%
\pgfpathlineto{\pgfqpoint{1.675720in}{2.467178in}}%
\pgfpathlineto{\pgfqpoint{1.678674in}{2.402906in}}%
\pgfpathlineto{\pgfqpoint{1.681319in}{2.467178in}}%
\pgfpathlineto{\pgfqpoint{1.692252in}{2.531450in}}%
\pgfpathlineto{\pgfqpoint{1.694897in}{2.467178in}}%
\pgfpathlineto{\pgfqpoint{1.697542in}{2.531450in}}%
\pgfpathlineto{\pgfqpoint{1.700275in}{2.467178in}}%
\pgfpathlineto{\pgfqpoint{1.702920in}{2.531450in}}%
\pgfpathlineto{\pgfqpoint{1.717379in}{2.595723in}}%
\pgfpathlineto{\pgfqpoint{1.720025in}{2.531450in}}%
\pgfpathlineto{\pgfqpoint{1.722670in}{2.595723in}}%
\pgfpathlineto{\pgfqpoint{1.725359in}{2.531450in}}%
\pgfpathlineto{\pgfqpoint{1.728004in}{2.595723in}}%
\pgfpathlineto{\pgfqpoint{1.745020in}{2.670707in}}%
\pgfpathlineto{\pgfqpoint{1.747665in}{2.595723in}}%
\pgfpathlineto{\pgfqpoint{1.750354in}{2.670707in}}%
\pgfpathlineto{\pgfqpoint{1.752999in}{2.595723in}}%
\pgfpathlineto{\pgfqpoint{1.755644in}{2.670707in}}%
\pgfpathlineto{\pgfqpoint{1.758378in}{2.595723in}}%
\pgfpathlineto{\pgfqpoint{1.761023in}{2.670707in}}%
\pgfpathlineto{\pgfqpoint{1.764020in}{2.595723in}}%
\pgfpathlineto{\pgfqpoint{1.766665in}{2.670707in}}%
\pgfpathlineto{\pgfqpoint{1.777290in}{2.734979in}}%
\pgfpathlineto{\pgfqpoint{1.779935in}{2.670707in}}%
\pgfpathlineto{\pgfqpoint{1.782580in}{2.734979in}}%
\pgfpathlineto{\pgfqpoint{1.785225in}{2.670707in}}%
\pgfpathlineto{\pgfqpoint{1.788002in}{2.734979in}}%
\pgfpathlineto{\pgfqpoint{1.790647in}{2.670707in}}%
\pgfpathlineto{\pgfqpoint{1.793292in}{2.734979in}}%
\pgfpathlineto{\pgfqpoint{1.796069in}{2.670707in}}%
\pgfpathlineto{\pgfqpoint{1.798758in}{2.734979in}}%
\pgfpathlineto{\pgfqpoint{1.801668in}{2.670707in}}%
\pgfpathlineto{\pgfqpoint{1.804313in}{2.734979in}}%
\pgfpathlineto{\pgfqpoint{1.814849in}{2.799252in}}%
\pgfpathlineto{\pgfqpoint{1.817494in}{2.734979in}}%
\pgfpathlineto{\pgfqpoint{1.820183in}{2.799252in}}%
\pgfpathlineto{\pgfqpoint{1.822828in}{2.734979in}}%
\pgfpathlineto{\pgfqpoint{1.825473in}{2.799252in}}%
\pgfpathlineto{\pgfqpoint{1.828118in}{2.734979in}}%
\pgfpathlineto{\pgfqpoint{1.830763in}{2.799252in}}%
\pgfpathlineto{\pgfqpoint{1.833409in}{2.734979in}}%
\pgfpathlineto{\pgfqpoint{1.836054in}{2.799252in}}%
\pgfpathlineto{\pgfqpoint{1.841211in}{2.734979in}}%
\pgfpathlineto{\pgfqpoint{1.843856in}{2.799252in}}%
\pgfpathlineto{\pgfqpoint{1.869028in}{2.863524in}}%
\pgfpathlineto{\pgfqpoint{1.871673in}{2.799252in}}%
\pgfpathlineto{\pgfqpoint{1.874407in}{2.863524in}}%
\pgfpathlineto{\pgfqpoint{1.877052in}{2.799252in}}%
\pgfpathlineto{\pgfqpoint{1.879785in}{2.863524in}}%
\pgfpathlineto{\pgfqpoint{1.882430in}{2.799252in}}%
\pgfpathlineto{\pgfqpoint{1.885163in}{2.863524in}}%
\pgfpathlineto{\pgfqpoint{1.887808in}{2.799252in}}%
\pgfpathlineto{\pgfqpoint{1.890453in}{2.863524in}}%
\pgfpathlineto{\pgfqpoint{1.893186in}{2.799252in}}%
\pgfpathlineto{\pgfqpoint{1.895831in}{2.863524in}}%
\pgfpathlineto{\pgfqpoint{1.899843in}{2.799252in}}%
\pgfpathlineto{\pgfqpoint{1.902488in}{2.863524in}}%
\pgfpathlineto{\pgfqpoint{1.906103in}{2.799252in}}%
\pgfpathlineto{\pgfqpoint{1.908748in}{2.863524in}}%
\pgfpathlineto{\pgfqpoint{1.911613in}{2.799252in}}%
\pgfpathlineto{\pgfqpoint{1.914259in}{2.863524in}}%
\pgfpathlineto{\pgfqpoint{1.959268in}{2.938508in}}%
\pgfpathlineto{\pgfqpoint{1.961913in}{2.863524in}}%
\pgfpathlineto{\pgfqpoint{1.964691in}{2.938508in}}%
\pgfpathlineto{\pgfqpoint{1.967336in}{2.863524in}}%
\pgfpathlineto{\pgfqpoint{1.970377in}{2.938508in}}%
\pgfpathlineto{\pgfqpoint{1.973022in}{2.863524in}}%
\pgfpathlineto{\pgfqpoint{1.975668in}{2.938508in}}%
\pgfpathlineto{\pgfqpoint{1.978313in}{2.863524in}}%
\pgfpathlineto{\pgfqpoint{1.980958in}{2.938508in}}%
\pgfpathlineto{\pgfqpoint{1.983603in}{2.863524in}}%
\pgfpathlineto{\pgfqpoint{1.986248in}{2.938508in}}%
\pgfpathlineto{\pgfqpoint{1.989642in}{2.863524in}}%
\pgfpathlineto{\pgfqpoint{1.992287in}{2.938508in}}%
\pgfpathlineto{\pgfqpoint{2.039942in}{3.002780in}}%
\pgfpathlineto{\pgfqpoint{2.042587in}{2.938508in}}%
\pgfpathlineto{\pgfqpoint{2.045585in}{3.002780in}}%
\pgfpathlineto{\pgfqpoint{2.048230in}{2.938508in}}%
\pgfpathlineto{\pgfqpoint{2.050963in}{3.002780in}}%
\pgfpathlineto{\pgfqpoint{2.053608in}{2.938508in}}%
\pgfpathlineto{\pgfqpoint{2.056253in}{3.002780in}}%
\pgfpathlineto{\pgfqpoint{2.058898in}{2.938508in}}%
\pgfpathlineto{\pgfqpoint{2.061543in}{3.002780in}}%
\pgfpathlineto{\pgfqpoint{2.064320in}{2.938508in}}%
\pgfpathlineto{\pgfqpoint{2.066965in}{3.002780in}}%
\pgfpathlineto{\pgfqpoint{2.070977in}{2.938508in}}%
\pgfpathlineto{\pgfqpoint{2.073622in}{3.002780in}}%
\pgfpathlineto{\pgfqpoint{2.077017in}{2.938508in}}%
\pgfpathlineto{\pgfqpoint{2.079662in}{3.002780in}}%
\pgfpathlineto{\pgfqpoint{2.099940in}{2.938508in}}%
\pgfpathlineto{\pgfqpoint{2.107875in}{2.734979in}}%
\pgfpathlineto{\pgfqpoint{2.110520in}{2.595723in}}%
\pgfpathlineto{\pgfqpoint{2.136045in}{1.995848in}}%
\pgfpathlineto{\pgfqpoint{2.139924in}{1.931575in}}%
\pgfpathlineto{\pgfqpoint{2.142614in}{1.995848in}}%
\pgfpathlineto{\pgfqpoint{2.148918in}{1.867303in}}%
\pgfpathlineto{\pgfqpoint{2.151563in}{1.931575in}}%
\pgfpathlineto{\pgfqpoint{2.159586in}{1.728046in}}%
\pgfpathlineto{\pgfqpoint{2.169461in}{1.599502in}}%
\pgfpathlineto{\pgfqpoint{2.173208in}{1.524517in}}%
\pgfpathlineto{\pgfqpoint{2.182950in}{1.395973in}}%
\pgfpathlineto{\pgfqpoint{2.188946in}{1.331700in}}%
\pgfpathlineto{\pgfqpoint{2.194500in}{1.256716in}}%
\pgfpathlineto{\pgfqpoint{2.203846in}{1.192444in}}%
\pgfpathlineto{\pgfqpoint{2.207241in}{1.256716in}}%
\pgfpathlineto{\pgfqpoint{2.212619in}{1.128171in}}%
\pgfpathlineto{\pgfqpoint{2.219011in}{1.063899in}}%
\pgfpathlineto{\pgfqpoint{2.221656in}{1.128171in}}%
\pgfpathlineto{\pgfqpoint{2.224566in}{1.063899in}}%
\pgfpathlineto{\pgfqpoint{2.227211in}{1.128171in}}%
\pgfpathlineto{\pgfqpoint{2.235763in}{1.063899in}}%
\pgfpathlineto{\pgfqpoint{2.238408in}{1.128171in}}%
\pgfpathlineto{\pgfqpoint{2.249385in}{1.063899in}}%
\pgfpathlineto{\pgfqpoint{2.252030in}{1.128171in}}%
\pgfpathlineto{\pgfqpoint{2.254675in}{1.063899in}}%
\pgfpathlineto{\pgfqpoint{2.257408in}{1.128171in}}%
\pgfpathlineto{\pgfqpoint{2.262787in}{0.988915in}}%
\pgfpathlineto{\pgfqpoint{2.265432in}{1.063899in}}%
\pgfpathlineto{\pgfqpoint{2.268077in}{0.988915in}}%
\pgfpathlineto{\pgfqpoint{2.310574in}{0.924643in}}%
\pgfpathlineto{\pgfqpoint{2.313219in}{0.988915in}}%
\pgfpathlineto{\pgfqpoint{2.315864in}{0.924643in}}%
\pgfpathlineto{\pgfqpoint{2.333056in}{0.796098in}}%
\pgfpathlineto{\pgfqpoint{2.335745in}{0.860370in}}%
\pgfpathlineto{\pgfqpoint{2.342138in}{0.721114in}}%
\pgfpathlineto{\pgfqpoint{2.345312in}{0.796098in}}%
\pgfpathlineto{\pgfqpoint{2.347957in}{0.721114in}}%
\pgfpathlineto{\pgfqpoint{2.352541in}{0.656841in}}%
\pgfpathlineto{\pgfqpoint{2.355187in}{0.721114in}}%
\pgfpathlineto{\pgfqpoint{2.357832in}{0.656841in}}%
\pgfpathlineto{\pgfqpoint{2.376523in}{1.128171in}}%
\pgfpathlineto{\pgfqpoint{2.381813in}{1.331700in}}%
\pgfpathlineto{\pgfqpoint{2.396185in}{1.663774in}}%
\pgfpathlineto{\pgfqpoint{2.407162in}{1.867303in}}%
\pgfpathlineto{\pgfqpoint{2.417036in}{1.995848in}}%
\pgfpathlineto{\pgfqpoint{2.423429in}{2.060120in}}%
\pgfpathlineto{\pgfqpoint{2.426162in}{1.995848in}}%
\pgfpathlineto{\pgfqpoint{2.431672in}{2.135104in}}%
\pgfpathlineto{\pgfqpoint{2.434670in}{2.060120in}}%
\pgfpathlineto{\pgfqpoint{2.437315in}{2.135104in}}%
\pgfpathlineto{\pgfqpoint{2.440754in}{2.199377in}}%
\pgfpathlineto{\pgfqpoint{2.443531in}{2.135104in}}%
\pgfpathlineto{\pgfqpoint{2.449394in}{2.263649in}}%
\pgfpathlineto{\pgfqpoint{2.452083in}{2.199377in}}%
\pgfpathlineto{\pgfqpoint{2.457902in}{2.327921in}}%
\pgfpathlineto{\pgfqpoint{2.460547in}{2.263649in}}%
\pgfpathlineto{\pgfqpoint{2.463192in}{2.327921in}}%
\pgfpathlineto{\pgfqpoint{2.469981in}{2.402906in}}%
\pgfpathlineto{\pgfqpoint{2.472626in}{2.327921in}}%
\pgfpathlineto{\pgfqpoint{2.475271in}{2.402906in}}%
\pgfpathlineto{\pgfqpoint{2.483735in}{2.467178in}}%
\pgfpathlineto{\pgfqpoint{2.486380in}{2.402906in}}%
\pgfpathlineto{\pgfqpoint{2.489026in}{2.467178in}}%
\pgfpathlineto{\pgfqpoint{2.492067in}{2.402906in}}%
\pgfpathlineto{\pgfqpoint{2.494712in}{2.467178in}}%
\pgfpathlineto{\pgfqpoint{2.508290in}{2.531450in}}%
\pgfpathlineto{\pgfqpoint{2.510935in}{2.467178in}}%
\pgfpathlineto{\pgfqpoint{2.513580in}{2.531450in}}%
\pgfpathlineto{\pgfqpoint{2.516225in}{2.467178in}}%
\pgfpathlineto{\pgfqpoint{2.518870in}{2.531450in}}%
\pgfpathlineto{\pgfqpoint{2.530464in}{2.595723in}}%
\pgfpathlineto{\pgfqpoint{2.533109in}{2.531450in}}%
\pgfpathlineto{\pgfqpoint{2.535799in}{2.595723in}}%
\pgfpathlineto{\pgfqpoint{2.538444in}{2.531450in}}%
\pgfpathlineto{\pgfqpoint{2.541089in}{2.595723in}}%
\pgfpathlineto{\pgfqpoint{2.561279in}{2.670707in}}%
\pgfpathlineto{\pgfqpoint{2.563924in}{2.595723in}}%
\pgfpathlineto{\pgfqpoint{2.566569in}{2.670707in}}%
\pgfpathlineto{\pgfqpoint{2.569214in}{2.595723in}}%
\pgfpathlineto{\pgfqpoint{2.571859in}{2.670707in}}%
\pgfpathlineto{\pgfqpoint{2.575033in}{2.595723in}}%
\pgfpathlineto{\pgfqpoint{2.577678in}{2.670707in}}%
\pgfpathlineto{\pgfqpoint{2.582528in}{2.595723in}}%
\pgfpathlineto{\pgfqpoint{2.585173in}{2.670707in}}%
\pgfpathlineto{\pgfqpoint{2.597648in}{2.734979in}}%
\pgfpathlineto{\pgfqpoint{2.600293in}{2.670707in}}%
\pgfpathlineto{\pgfqpoint{2.602939in}{2.734979in}}%
\pgfpathlineto{\pgfqpoint{2.605584in}{2.670707in}}%
\pgfpathlineto{\pgfqpoint{2.608229in}{2.734979in}}%
\pgfpathlineto{\pgfqpoint{2.610918in}{2.670707in}}%
\pgfpathlineto{\pgfqpoint{2.613563in}{2.734979in}}%
\pgfpathlineto{\pgfqpoint{2.632739in}{2.799252in}}%
\pgfpathlineto{\pgfqpoint{2.635384in}{2.734979in}}%
\pgfpathlineto{\pgfqpoint{2.641953in}{2.799252in}}%
\pgfpathlineto{\pgfqpoint{2.644598in}{2.734979in}}%
\pgfpathlineto{\pgfqpoint{2.647596in}{2.799252in}}%
\pgfpathlineto{\pgfqpoint{2.650241in}{2.734979in}}%
\pgfpathlineto{\pgfqpoint{2.653150in}{2.799252in}}%
\pgfpathlineto{\pgfqpoint{2.655795in}{2.734979in}}%
\pgfpathlineto{\pgfqpoint{2.658661in}{2.799252in}}%
\pgfpathlineto{\pgfqpoint{2.661306in}{2.734979in}}%
\pgfpathlineto{\pgfqpoint{2.663995in}{2.799252in}}%
\pgfpathlineto{\pgfqpoint{2.666640in}{2.734979in}}%
\pgfpathlineto{\pgfqpoint{2.669285in}{2.799252in}}%
\pgfpathlineto{\pgfqpoint{2.671974in}{2.734979in}}%
\pgfpathlineto{\pgfqpoint{2.674619in}{2.799252in}}%
\pgfpathlineto{\pgfqpoint{2.677308in}{2.734979in}}%
\pgfpathlineto{\pgfqpoint{2.679953in}{2.799252in}}%
\pgfpathlineto{\pgfqpoint{2.684406in}{2.734979in}}%
\pgfpathlineto{\pgfqpoint{2.687051in}{2.799252in}}%
\pgfpathlineto{\pgfqpoint{2.700276in}{2.734979in}}%
\pgfpathlineto{\pgfqpoint{2.702921in}{2.799252in}}%
\pgfpathlineto{\pgfqpoint{2.745286in}{2.863524in}}%
\pgfpathlineto{\pgfqpoint{2.747931in}{2.799252in}}%
\pgfpathlineto{\pgfqpoint{2.750664in}{2.863524in}}%
\pgfpathlineto{\pgfqpoint{2.753309in}{2.799252in}}%
\pgfpathlineto{\pgfqpoint{2.755954in}{2.863524in}}%
\pgfpathlineto{\pgfqpoint{2.758643in}{2.799252in}}%
\pgfpathlineto{\pgfqpoint{2.761288in}{2.863524in}}%
\pgfpathlineto{\pgfqpoint{2.763977in}{2.799252in}}%
\pgfpathlineto{\pgfqpoint{2.766622in}{2.863524in}}%
\pgfpathlineto{\pgfqpoint{2.769311in}{2.799252in}}%
\pgfpathlineto{\pgfqpoint{2.771957in}{2.863524in}}%
\pgfpathlineto{\pgfqpoint{2.774602in}{2.799252in}}%
\pgfpathlineto{\pgfqpoint{2.777247in}{2.863524in}}%
\pgfpathlineto{\pgfqpoint{2.780024in}{2.799252in}}%
\pgfpathlineto{\pgfqpoint{2.782669in}{2.863524in}}%
\pgfpathlineto{\pgfqpoint{2.857171in}{2.938508in}}%
\pgfpathlineto{\pgfqpoint{2.859816in}{2.863524in}}%
\pgfpathlineto{\pgfqpoint{2.862549in}{2.938508in}}%
\pgfpathlineto{\pgfqpoint{2.865194in}{2.863524in}}%
\pgfpathlineto{\pgfqpoint{2.867883in}{2.938508in}}%
\pgfpathlineto{\pgfqpoint{2.870528in}{2.863524in}}%
\pgfpathlineto{\pgfqpoint{2.873261in}{2.938508in}}%
\pgfpathlineto{\pgfqpoint{2.875907in}{2.863524in}}%
\pgfpathlineto{\pgfqpoint{2.878684in}{2.938508in}}%
\pgfpathlineto{\pgfqpoint{2.881329in}{2.863524in}}%
\pgfpathlineto{\pgfqpoint{2.884018in}{2.938508in}}%
\pgfpathlineto{\pgfqpoint{2.886663in}{2.863524in}}%
\pgfpathlineto{\pgfqpoint{2.889352in}{2.938508in}}%
\pgfpathlineto{\pgfqpoint{2.891997in}{2.863524in}}%
\pgfpathlineto{\pgfqpoint{2.894642in}{2.938508in}}%
\pgfpathlineto{\pgfqpoint{2.897287in}{2.863524in}}%
\pgfpathlineto{\pgfqpoint{2.900020in}{2.938508in}}%
\pgfpathlineto{\pgfqpoint{2.902666in}{2.863524in}}%
\pgfpathlineto{\pgfqpoint{2.905311in}{2.938508in}}%
\pgfpathlineto{\pgfqpoint{2.907956in}{2.863524in}}%
\pgfpathlineto{\pgfqpoint{2.910601in}{2.938508in}}%
\pgfpathlineto{\pgfqpoint{2.913290in}{2.863524in}}%
\pgfpathlineto{\pgfqpoint{2.915935in}{2.938508in}}%
\pgfpathlineto{\pgfqpoint{2.918624in}{2.863524in}}%
\pgfpathlineto{\pgfqpoint{2.921313in}{2.938508in}}%
\pgfpathlineto{\pgfqpoint{2.923958in}{2.863524in}}%
\pgfpathlineto{\pgfqpoint{2.926647in}{2.938508in}}%
\pgfpathlineto{\pgfqpoint{2.929292in}{2.863524in}}%
\pgfpathlineto{\pgfqpoint{2.931937in}{2.938508in}}%
\pgfpathlineto{\pgfqpoint{2.934582in}{2.863524in}}%
\pgfpathlineto{\pgfqpoint{2.937227in}{2.938508in}}%
\pgfpathlineto{\pgfqpoint{2.941063in}{2.863524in}}%
\pgfpathlineto{\pgfqpoint{2.943708in}{2.938508in}}%
\pgfpathlineto{\pgfqpoint{2.995683in}{2.863524in}}%
\pgfpathlineto{\pgfqpoint{3.004235in}{2.670707in}}%
\pgfpathlineto{\pgfqpoint{3.012214in}{2.467178in}}%
\pgfpathlineto{\pgfqpoint{3.040252in}{1.792319in}}%
\pgfpathlineto{\pgfqpoint{3.044396in}{1.728046in}}%
\pgfpathlineto{\pgfqpoint{3.047129in}{1.792319in}}%
\pgfpathlineto{\pgfqpoint{3.052419in}{1.663774in}}%
\pgfpathlineto{\pgfqpoint{3.058370in}{1.599502in}}%
\pgfpathlineto{\pgfqpoint{3.061147in}{1.663774in}}%
\pgfpathlineto{\pgfqpoint{3.069127in}{1.460245in}}%
\pgfpathlineto{\pgfqpoint{3.080500in}{1.331700in}}%
\pgfpathlineto{\pgfqpoint{3.085394in}{1.256716in}}%
\pgfpathlineto{\pgfqpoint{3.088039in}{1.331700in}}%
\pgfpathlineto{\pgfqpoint{3.090684in}{1.256716in}}%
\pgfpathlineto{\pgfqpoint{3.095004in}{1.192444in}}%
\pgfpathlineto{\pgfqpoint{3.097649in}{1.256716in}}%
\pgfpathlineto{\pgfqpoint{3.100294in}{1.192444in}}%
\pgfpathlineto{\pgfqpoint{3.108670in}{1.128171in}}%
\pgfpathlineto{\pgfqpoint{3.111359in}{1.192444in}}%
\pgfpathlineto{\pgfqpoint{3.114004in}{1.128171in}}%
\pgfpathlineto{\pgfqpoint{3.119867in}{1.063899in}}%
\pgfpathlineto{\pgfqpoint{3.122512in}{1.128171in}}%
\pgfpathlineto{\pgfqpoint{3.125157in}{1.063899in}}%
\pgfpathlineto{\pgfqpoint{3.132696in}{0.988915in}}%
\pgfpathlineto{\pgfqpoint{3.135385in}{1.063899in}}%
\pgfpathlineto{\pgfqpoint{3.138030in}{0.988915in}}%
\pgfpathlineto{\pgfqpoint{3.142438in}{0.924643in}}%
\pgfpathlineto{\pgfqpoint{3.145127in}{0.988915in}}%
\pgfpathlineto{\pgfqpoint{3.147772in}{0.924643in}}%
\pgfpathlineto{\pgfqpoint{3.154209in}{0.860370in}}%
\pgfpathlineto{\pgfqpoint{3.156854in}{0.924643in}}%
\pgfpathlineto{\pgfqpoint{3.159499in}{0.860370in}}%
\pgfpathlineto{\pgfqpoint{3.162144in}{0.924643in}}%
\pgfpathlineto{\pgfqpoint{3.164789in}{0.860370in}}%
\pgfpathlineto{\pgfqpoint{3.167434in}{0.924643in}}%
\pgfpathlineto{\pgfqpoint{3.170079in}{0.860370in}}%
\pgfpathlineto{\pgfqpoint{3.172812in}{0.924643in}}%
\pgfpathlineto{\pgfqpoint{3.175457in}{0.860370in}}%
\pgfpathlineto{\pgfqpoint{3.178146in}{0.924643in}}%
\pgfpathlineto{\pgfqpoint{3.180791in}{0.860370in}}%
\pgfpathlineto{\pgfqpoint{3.183436in}{0.924643in}}%
\pgfpathlineto{\pgfqpoint{3.186081in}{0.860370in}}%
\pgfpathlineto{\pgfqpoint{3.217028in}{0.796098in}}%
\pgfpathlineto{\pgfqpoint{3.219673in}{0.860370in}}%
\pgfpathlineto{\pgfqpoint{3.222318in}{0.796098in}}%
\pgfpathlineto{\pgfqpoint{3.234794in}{0.721114in}}%
\pgfpathlineto{\pgfqpoint{3.237439in}{0.796098in}}%
\pgfpathlineto{\pgfqpoint{3.240084in}{0.721114in}}%
\pgfpathlineto{\pgfqpoint{3.242773in}{0.796098in}}%
\pgfpathlineto{\pgfqpoint{3.245418in}{0.721114in}}%
\pgfpathlineto{\pgfqpoint{3.248989in}{0.796098in}}%
\pgfpathlineto{\pgfqpoint{3.251634in}{0.721114in}}%
\pgfpathlineto{\pgfqpoint{3.258423in}{0.860370in}}%
\pgfpathlineto{\pgfqpoint{3.269003in}{1.128171in}}%
\pgfpathlineto{\pgfqpoint{3.271648in}{1.256716in}}%
\pgfpathlineto{\pgfqpoint{3.277379in}{1.395973in}}%
\pgfpathlineto{\pgfqpoint{3.298981in}{1.792319in}}%
\pgfpathlineto{\pgfqpoint{3.315732in}{1.995848in}}%
\pgfpathlineto{\pgfqpoint{3.318510in}{1.931575in}}%
\pgfpathlineto{\pgfqpoint{3.323844in}{2.060120in}}%
\pgfpathlineto{\pgfqpoint{3.331515in}{2.135104in}}%
\pgfpathlineto{\pgfqpoint{3.334160in}{2.060120in}}%
\pgfpathlineto{\pgfqpoint{3.336805in}{2.135104in}}%
\pgfpathlineto{\pgfqpoint{3.341962in}{2.199377in}}%
\pgfpathlineto{\pgfqpoint{3.344607in}{2.135104in}}%
\pgfpathlineto{\pgfqpoint{3.350382in}{2.263649in}}%
\pgfpathlineto{\pgfqpoint{3.353027in}{2.199377in}}%
\pgfpathlineto{\pgfqpoint{3.355673in}{2.263649in}}%
\pgfpathlineto{\pgfqpoint{3.358979in}{2.199377in}}%
\pgfpathlineto{\pgfqpoint{3.361624in}{2.263649in}}%
\pgfpathlineto{\pgfqpoint{3.365900in}{2.327921in}}%
\pgfpathlineto{\pgfqpoint{3.368545in}{2.263649in}}%
\pgfpathlineto{\pgfqpoint{3.371190in}{2.327921in}}%
\pgfpathlineto{\pgfqpoint{3.380624in}{2.402906in}}%
\pgfpathlineto{\pgfqpoint{3.383269in}{2.327921in}}%
\pgfpathlineto{\pgfqpoint{3.386002in}{2.402906in}}%
\pgfpathlineto{\pgfqpoint{3.389000in}{2.327921in}}%
\pgfpathlineto{\pgfqpoint{3.391645in}{2.402906in}}%
\pgfpathlineto{\pgfqpoint{3.398919in}{2.467178in}}%
\pgfpathlineto{\pgfqpoint{3.401564in}{2.402906in}}%
\pgfpathlineto{\pgfqpoint{3.404209in}{2.467178in}}%
\pgfpathlineto{\pgfqpoint{3.406942in}{2.402906in}}%
\pgfpathlineto{\pgfqpoint{3.409587in}{2.467178in}}%
\pgfpathlineto{\pgfqpoint{3.412585in}{2.402906in}}%
\pgfpathlineto{\pgfqpoint{3.415230in}{2.467178in}}%
\pgfpathlineto{\pgfqpoint{3.426956in}{2.531450in}}%
\pgfpathlineto{\pgfqpoint{3.429601in}{2.467178in}}%
\pgfpathlineto{\pgfqpoint{3.432246in}{2.531450in}}%
\pgfpathlineto{\pgfqpoint{3.435200in}{2.467178in}}%
\pgfpathlineto{\pgfqpoint{3.437845in}{2.531450in}}%
\pgfpathlineto{\pgfqpoint{3.452349in}{2.595723in}}%
\pgfpathlineto{\pgfqpoint{3.454994in}{2.531450in}}%
\pgfpathlineto{\pgfqpoint{3.457859in}{2.595723in}}%
\pgfpathlineto{\pgfqpoint{3.460504in}{2.531450in}}%
\pgfpathlineto{\pgfqpoint{3.463193in}{2.595723in}}%
\pgfpathlineto{\pgfqpoint{3.465882in}{2.531450in}}%
\pgfpathlineto{\pgfqpoint{3.468528in}{2.595723in}}%
\pgfpathlineto{\pgfqpoint{3.471437in}{2.531450in}}%
\pgfpathlineto{\pgfqpoint{3.474082in}{2.595723in}}%
\pgfpathlineto{\pgfqpoint{3.492641in}{2.670707in}}%
\pgfpathlineto{\pgfqpoint{3.495287in}{2.595723in}}%
\pgfpathlineto{\pgfqpoint{3.498020in}{2.670707in}}%
\pgfpathlineto{\pgfqpoint{3.500665in}{2.595723in}}%
\pgfpathlineto{\pgfqpoint{3.503310in}{2.670707in}}%
\pgfpathlineto{\pgfqpoint{3.505955in}{2.595723in}}%
\pgfpathlineto{\pgfqpoint{3.508600in}{2.670707in}}%
\pgfpathlineto{\pgfqpoint{3.511421in}{2.595723in}}%
\pgfpathlineto{\pgfqpoint{3.514066in}{2.670707in}}%
\pgfpathlineto{\pgfqpoint{3.518342in}{2.595723in}}%
\pgfpathlineto{\pgfqpoint{3.520987in}{2.670707in}}%
\pgfpathlineto{\pgfqpoint{3.535094in}{2.734979in}}%
\pgfpathlineto{\pgfqpoint{3.537739in}{2.670707in}}%
\pgfpathlineto{\pgfqpoint{3.540473in}{2.734979in}}%
\pgfpathlineto{\pgfqpoint{3.543118in}{2.670707in}}%
\pgfpathlineto{\pgfqpoint{3.545763in}{2.734979in}}%
\pgfpathlineto{\pgfqpoint{3.548408in}{2.670707in}}%
\pgfpathlineto{\pgfqpoint{3.551053in}{2.734979in}}%
\pgfpathlineto{\pgfqpoint{3.553698in}{2.670707in}}%
\pgfpathlineto{\pgfqpoint{3.556387in}{2.734979in}}%
\pgfpathlineto{\pgfqpoint{3.559032in}{2.670707in}}%
\pgfpathlineto{\pgfqpoint{3.561721in}{2.734979in}}%
\pgfpathlineto{\pgfqpoint{3.564366in}{2.670707in}}%
\pgfpathlineto{\pgfqpoint{3.567011in}{2.734979in}}%
\pgfpathlineto{\pgfqpoint{3.569744in}{2.670707in}}%
\pgfpathlineto{\pgfqpoint{3.572389in}{2.734979in}}%
\pgfpathlineto{\pgfqpoint{3.580369in}{2.670707in}}%
\pgfpathlineto{\pgfqpoint{3.583014in}{2.734979in}}%
\pgfpathlineto{\pgfqpoint{3.601793in}{2.799252in}}%
\pgfpathlineto{\pgfqpoint{3.604438in}{2.734979in}}%
\pgfpathlineto{\pgfqpoint{3.607172in}{2.799252in}}%
\pgfpathlineto{\pgfqpoint{3.609817in}{2.734979in}}%
\pgfpathlineto{\pgfqpoint{3.612682in}{2.799252in}}%
\pgfpathlineto{\pgfqpoint{3.615327in}{2.734979in}}%
\pgfpathlineto{\pgfqpoint{3.618016in}{2.799252in}}%
\pgfpathlineto{\pgfqpoint{3.620705in}{2.734979in}}%
\pgfpathlineto{\pgfqpoint{3.623350in}{2.799252in}}%
\pgfpathlineto{\pgfqpoint{3.626216in}{2.734979in}}%
\pgfpathlineto{\pgfqpoint{3.628861in}{2.799252in}}%
\pgfpathlineto{\pgfqpoint{3.632123in}{2.734979in}}%
\pgfpathlineto{\pgfqpoint{3.634768in}{2.799252in}}%
\pgfpathlineto{\pgfqpoint{3.639397in}{2.734979in}}%
\pgfpathlineto{\pgfqpoint{3.642042in}{2.799252in}}%
\pgfpathlineto{\pgfqpoint{3.653195in}{2.734979in}}%
\pgfpathlineto{\pgfqpoint{3.655840in}{2.799252in}}%
\pgfpathlineto{\pgfqpoint{3.675061in}{2.863524in}}%
\pgfpathlineto{\pgfqpoint{3.677706in}{2.799252in}}%
\pgfpathlineto{\pgfqpoint{3.680616in}{2.863524in}}%
\pgfpathlineto{\pgfqpoint{3.683261in}{2.799252in}}%
\pgfpathlineto{\pgfqpoint{3.685906in}{2.863524in}}%
\pgfpathlineto{\pgfqpoint{3.688551in}{2.799252in}}%
\pgfpathlineto{\pgfqpoint{3.691196in}{2.863524in}}%
\pgfpathlineto{\pgfqpoint{3.693841in}{2.799252in}}%
\pgfpathlineto{\pgfqpoint{3.696574in}{2.863524in}}%
\pgfpathlineto{\pgfqpoint{3.699219in}{2.799252in}}%
\pgfpathlineto{\pgfqpoint{3.701864in}{2.863524in}}%
\pgfpathlineto{\pgfqpoint{3.704509in}{2.799252in}}%
\pgfpathlineto{\pgfqpoint{3.707154in}{2.863524in}}%
\pgfpathlineto{\pgfqpoint{3.709799in}{2.799252in}}%
\pgfpathlineto{\pgfqpoint{3.712444in}{2.863524in}}%
\pgfpathlineto{\pgfqpoint{3.715177in}{2.799252in}}%
\pgfpathlineto{\pgfqpoint{3.717822in}{2.863524in}}%
\pgfpathlineto{\pgfqpoint{3.720467in}{2.799252in}}%
\pgfpathlineto{\pgfqpoint{3.723113in}{2.863524in}}%
\pgfpathlineto{\pgfqpoint{3.726066in}{2.799252in}}%
\pgfpathlineto{\pgfqpoint{3.728711in}{2.863524in}}%
\pgfpathlineto{\pgfqpoint{3.731797in}{2.799252in}}%
\pgfpathlineto{\pgfqpoint{3.734442in}{2.863524in}}%
\pgfpathlineto{\pgfqpoint{3.737131in}{2.799252in}}%
\pgfpathlineto{\pgfqpoint{3.739776in}{2.863524in}}%
\pgfpathlineto{\pgfqpoint{3.742686in}{2.799252in}}%
\pgfpathlineto{\pgfqpoint{3.745331in}{2.863524in}}%
\pgfpathlineto{\pgfqpoint{3.748108in}{2.799252in}}%
\pgfpathlineto{\pgfqpoint{3.750753in}{2.863524in}}%
\pgfpathlineto{\pgfqpoint{3.782846in}{2.938508in}}%
\pgfpathlineto{\pgfqpoint{3.785491in}{2.863524in}}%
\pgfpathlineto{\pgfqpoint{3.788225in}{2.938508in}}%
\pgfpathlineto{\pgfqpoint{3.790870in}{2.863524in}}%
\pgfpathlineto{\pgfqpoint{3.793603in}{2.938508in}}%
\pgfpathlineto{\pgfqpoint{3.796248in}{2.863524in}}%
\pgfpathlineto{\pgfqpoint{3.798893in}{2.938508in}}%
\pgfpathlineto{\pgfqpoint{3.801538in}{2.863524in}}%
\pgfpathlineto{\pgfqpoint{3.804183in}{2.938508in}}%
\pgfpathlineto{\pgfqpoint{3.806960in}{2.863524in}}%
\pgfpathlineto{\pgfqpoint{3.809605in}{2.938508in}}%
\pgfpathlineto{\pgfqpoint{3.812647in}{2.863524in}}%
\pgfpathlineto{\pgfqpoint{3.815292in}{2.938508in}}%
\pgfpathlineto{\pgfqpoint{3.818202in}{2.863524in}}%
\pgfpathlineto{\pgfqpoint{3.820847in}{2.938508in}}%
\pgfpathlineto{\pgfqpoint{3.857613in}{3.002780in}}%
\pgfpathlineto{\pgfqpoint{3.860258in}{2.938508in}}%
\pgfpathlineto{\pgfqpoint{3.863256in}{3.002780in}}%
\pgfpathlineto{\pgfqpoint{3.865901in}{2.938508in}}%
\pgfpathlineto{\pgfqpoint{3.869692in}{3.002780in}}%
\pgfpathlineto{\pgfqpoint{3.872337in}{2.938508in}}%
\pgfpathlineto{\pgfqpoint{3.874982in}{3.002780in}}%
\pgfpathlineto{\pgfqpoint{3.877627in}{2.938508in}}%
\pgfpathlineto{\pgfqpoint{3.880272in}{3.002780in}}%
\pgfpathlineto{\pgfqpoint{3.882917in}{2.938508in}}%
\pgfpathlineto{\pgfqpoint{3.885562in}{3.002780in}}%
\pgfpathlineto{\pgfqpoint{3.888295in}{2.938508in}}%
\pgfpathlineto{\pgfqpoint{3.890940in}{3.002780in}}%
\pgfpathlineto{\pgfqpoint{3.907075in}{2.938508in}}%
\pgfpathlineto{\pgfqpoint{3.909720in}{3.002780in}}%
\pgfpathlineto{\pgfqpoint{3.949881in}{2.938508in}}%
\pgfpathlineto{\pgfqpoint{3.960725in}{2.670707in}}%
\pgfpathlineto{\pgfqpoint{3.963370in}{2.531450in}}%
\pgfpathlineto{\pgfqpoint{3.971746in}{2.327921in}}%
\pgfpathlineto{\pgfqpoint{3.979329in}{2.199377in}}%
\pgfpathlineto{\pgfqpoint{3.985544in}{2.060120in}}%
\pgfpathlineto{\pgfqpoint{3.991011in}{1.995848in}}%
\pgfpathlineto{\pgfqpoint{3.994405in}{1.931575in}}%
\pgfpathlineto{\pgfqpoint{3.997403in}{1.995848in}}%
\pgfpathlineto{\pgfqpoint{4.005559in}{1.792319in}}%
\pgfpathlineto{\pgfqpoint{4.013538in}{1.728046in}}%
\pgfpathlineto{\pgfqpoint{4.016227in}{1.792319in}}%
\pgfpathlineto{\pgfqpoint{4.021693in}{1.663774in}}%
\pgfpathlineto{\pgfqpoint{4.024471in}{1.728046in}}%
\pgfpathlineto{\pgfqpoint{4.027116in}{1.599502in}}%
\pgfpathlineto{\pgfqpoint{4.038445in}{1.395973in}}%
\pgfpathlineto{\pgfqpoint{4.051318in}{1.256716in}}%
\pgfpathlineto{\pgfqpoint{4.054227in}{1.331700in}}%
\pgfpathlineto{\pgfqpoint{4.059561in}{1.192444in}}%
\pgfpathlineto{\pgfqpoint{4.065336in}{1.128171in}}%
\pgfpathlineto{\pgfqpoint{4.068026in}{1.192444in}}%
\pgfpathlineto{\pgfqpoint{4.073316in}{1.063899in}}%
\pgfpathlineto{\pgfqpoint{4.076534in}{1.128171in}}%
\pgfpathlineto{\pgfqpoint{4.079179in}{1.063899in}}%
\pgfpathlineto{\pgfqpoint{4.099546in}{0.988915in}}%
\pgfpathlineto{\pgfqpoint{4.102191in}{1.063899in}}%
\pgfpathlineto{\pgfqpoint{4.129655in}{0.988915in}}%
\pgfpathlineto{\pgfqpoint{4.132300in}{1.063899in}}%
\pgfpathlineto{\pgfqpoint{4.134945in}{0.988915in}}%
\pgfpathlineto{\pgfqpoint{4.138560in}{1.063899in}}%
\pgfpathlineto{\pgfqpoint{4.141205in}{0.988915in}}%
\pgfpathlineto{\pgfqpoint{4.148303in}{0.924643in}}%
\pgfpathlineto{\pgfqpoint{4.150948in}{0.988915in}}%
\pgfpathlineto{\pgfqpoint{4.153593in}{0.924643in}}%
\pgfpathlineto{\pgfqpoint{4.158221in}{0.988915in}}%
\pgfpathlineto{\pgfqpoint{4.160866in}{0.924643in}}%
\pgfpathlineto{\pgfqpoint{4.175855in}{0.860370in}}%
\pgfpathlineto{\pgfqpoint{4.179029in}{0.924643in}}%
\pgfpathlineto{\pgfqpoint{4.181674in}{0.860370in}}%
\pgfpathlineto{\pgfqpoint{4.191681in}{0.796098in}}%
\pgfpathlineto{\pgfqpoint{4.194326in}{0.860370in}}%
\pgfpathlineto{\pgfqpoint{4.196971in}{0.796098in}}%
\pgfpathlineto{\pgfqpoint{4.199749in}{0.860370in}}%
\pgfpathlineto{\pgfqpoint{4.202394in}{0.796098in}}%
\pgfpathlineto{\pgfqpoint{4.231577in}{1.524517in}}%
\pgfpathlineto{\pgfqpoint{4.237088in}{1.663774in}}%
\pgfpathlineto{\pgfqpoint{4.243877in}{1.792319in}}%
\pgfpathlineto{\pgfqpoint{4.252429in}{1.931575in}}%
\pgfpathlineto{\pgfqpoint{4.269842in}{2.135104in}}%
\pgfpathlineto{\pgfqpoint{4.272487in}{2.060120in}}%
\pgfpathlineto{\pgfqpoint{4.275132in}{2.135104in}}%
\pgfpathlineto{\pgfqpoint{4.278923in}{2.199377in}}%
\pgfpathlineto{\pgfqpoint{4.281613in}{2.135104in}}%
\pgfpathlineto{\pgfqpoint{4.287784in}{2.263649in}}%
\pgfpathlineto{\pgfqpoint{4.290473in}{2.199377in}}%
\pgfpathlineto{\pgfqpoint{4.293118in}{2.263649in}}%
\pgfpathlineto{\pgfqpoint{4.299158in}{2.327921in}}%
\pgfpathlineto{\pgfqpoint{4.301803in}{2.263649in}}%
\pgfpathlineto{\pgfqpoint{4.304448in}{2.327921in}}%
\pgfpathlineto{\pgfqpoint{4.313926in}{2.402906in}}%
\pgfpathlineto{\pgfqpoint{4.316571in}{2.327921in}}%
\pgfpathlineto{\pgfqpoint{4.319260in}{2.402906in}}%
\pgfpathlineto{\pgfqpoint{4.321949in}{2.327921in}}%
\pgfpathlineto{\pgfqpoint{4.324594in}{2.402906in}}%
\pgfpathlineto{\pgfqpoint{4.329752in}{2.467178in}}%
\pgfpathlineto{\pgfqpoint{4.332397in}{2.402906in}}%
\pgfpathlineto{\pgfqpoint{4.335042in}{2.467178in}}%
\pgfpathlineto{\pgfqpoint{4.351794in}{2.531450in}}%
\pgfpathlineto{\pgfqpoint{4.354439in}{2.467178in}}%
\pgfpathlineto{\pgfqpoint{4.357084in}{2.531450in}}%
\pgfpathlineto{\pgfqpoint{4.359862in}{2.467178in}}%
\pgfpathlineto{\pgfqpoint{4.362507in}{2.531450in}}%
\pgfpathlineto{\pgfqpoint{4.374850in}{2.595723in}}%
\pgfpathlineto{\pgfqpoint{4.377495in}{2.531450in}}%
\pgfpathlineto{\pgfqpoint{4.380140in}{2.595723in}}%
\pgfpathlineto{\pgfqpoint{4.382785in}{2.531450in}}%
\pgfpathlineto{\pgfqpoint{4.385430in}{2.595723in}}%
\pgfpathlineto{\pgfqpoint{4.388252in}{2.531450in}}%
\pgfpathlineto{\pgfqpoint{4.390897in}{2.595723in}}%
\pgfpathlineto{\pgfqpoint{4.409192in}{2.670707in}}%
\pgfpathlineto{\pgfqpoint{4.411837in}{2.595723in}}%
\pgfpathlineto{\pgfqpoint{4.414526in}{2.670707in}}%
\pgfpathlineto{\pgfqpoint{4.417171in}{2.595723in}}%
\pgfpathlineto{\pgfqpoint{4.419816in}{2.670707in}}%
\pgfpathlineto{\pgfqpoint{4.422461in}{2.595723in}}%
\pgfpathlineto{\pgfqpoint{4.425106in}{2.670707in}}%
\pgfpathlineto{\pgfqpoint{4.428192in}{2.595723in}}%
\pgfpathlineto{\pgfqpoint{4.430837in}{2.670707in}}%
\pgfpathlineto{\pgfqpoint{4.434628in}{2.595723in}}%
\pgfpathlineto{\pgfqpoint{4.437273in}{2.670707in}}%
\pgfpathlineto{\pgfqpoint{4.452835in}{2.734979in}}%
\pgfpathlineto{\pgfqpoint{4.455480in}{2.670707in}}%
\pgfpathlineto{\pgfqpoint{4.458257in}{2.734979in}}%
\pgfpathlineto{\pgfqpoint{4.460902in}{2.670707in}}%
\pgfpathlineto{\pgfqpoint{4.463547in}{2.734979in}}%
\pgfpathlineto{\pgfqpoint{4.466280in}{2.670707in}}%
\pgfpathlineto{\pgfqpoint{4.468925in}{2.734979in}}%
\pgfpathlineto{\pgfqpoint{4.475362in}{2.670707in}}%
\pgfpathlineto{\pgfqpoint{4.478007in}{2.734979in}}%
\pgfpathlineto{\pgfqpoint{4.489248in}{2.799252in}}%
\pgfpathlineto{\pgfqpoint{4.491893in}{2.734979in}}%
\pgfpathlineto{\pgfqpoint{4.495596in}{2.799252in}}%
\pgfpathlineto{\pgfqpoint{4.498241in}{2.734979in}}%
\pgfpathlineto{\pgfqpoint{4.500974in}{2.799252in}}%
\pgfpathlineto{\pgfqpoint{4.503619in}{2.734979in}}%
\pgfpathlineto{\pgfqpoint{4.506264in}{2.799252in}}%
\pgfpathlineto{\pgfqpoint{4.508910in}{2.734979in}}%
\pgfpathlineto{\pgfqpoint{4.511599in}{2.799252in}}%
\pgfpathlineto{\pgfqpoint{4.514332in}{2.734979in}}%
\pgfpathlineto{\pgfqpoint{4.516977in}{2.799252in}}%
\pgfpathlineto{\pgfqpoint{4.519754in}{2.734979in}}%
\pgfpathlineto{\pgfqpoint{4.522399in}{2.799252in}}%
\pgfpathlineto{\pgfqpoint{4.525088in}{2.734979in}}%
\pgfpathlineto{\pgfqpoint{4.527733in}{2.799252in}}%
\pgfpathlineto{\pgfqpoint{4.531525in}{2.734979in}}%
\pgfpathlineto{\pgfqpoint{4.534170in}{2.799252in}}%
\pgfpathlineto{\pgfqpoint{4.557270in}{2.863524in}}%
\pgfpathlineto{\pgfqpoint{4.559915in}{2.799252in}}%
\pgfpathlineto{\pgfqpoint{4.563662in}{2.863524in}}%
\pgfpathlineto{\pgfqpoint{4.566351in}{2.799252in}}%
\pgfpathlineto{\pgfqpoint{4.568996in}{2.863524in}}%
\pgfpathlineto{\pgfqpoint{4.571641in}{2.799252in}}%
\pgfpathlineto{\pgfqpoint{4.574286in}{2.863524in}}%
\pgfpathlineto{\pgfqpoint{4.577063in}{2.799252in}}%
\pgfpathlineto{\pgfqpoint{4.579708in}{2.863524in}}%
\pgfpathlineto{\pgfqpoint{4.582618in}{2.799252in}}%
\pgfpathlineto{\pgfqpoint{4.585263in}{2.863524in}}%
\pgfpathlineto{\pgfqpoint{4.590333in}{2.799252in}}%
\pgfpathlineto{\pgfqpoint{4.592978in}{2.863524in}}%
\pgfpathlineto{\pgfqpoint{4.600252in}{2.799252in}}%
\pgfpathlineto{\pgfqpoint{4.602897in}{2.863524in}}%
\pgfpathlineto{\pgfqpoint{4.607393in}{2.799252in}}%
\pgfpathlineto{\pgfqpoint{4.610038in}{2.863524in}}%
\pgfpathlineto{\pgfqpoint{4.649097in}{2.938508in}}%
\pgfpathlineto{\pgfqpoint{4.651742in}{2.863524in}}%
\pgfpathlineto{\pgfqpoint{4.654387in}{2.938508in}}%
\pgfpathlineto{\pgfqpoint{4.657032in}{2.863524in}}%
\pgfpathlineto{\pgfqpoint{4.659677in}{2.938508in}}%
\pgfpathlineto{\pgfqpoint{4.662322in}{2.863524in}}%
\pgfpathlineto{\pgfqpoint{4.664967in}{2.938508in}}%
\pgfpathlineto{\pgfqpoint{4.664967in}{2.938508in}}%
\pgfusepath{stroke}%
\end{pgfscope}%
\begin{pgfscope}%
\pgfsetrectcap%
\pgfsetmiterjoin%
\pgfsetlinewidth{0.803000pt}%
\definecolor{currentstroke}{rgb}{0.000000,0.000000,0.000000}%
\pgfsetstrokecolor{currentstroke}%
\pgfsetdash{}{0pt}%
\pgfpathmoveto{\pgfqpoint{0.667540in}{0.539544in}}%
\pgfpathlineto{\pgfqpoint{0.667540in}{3.120077in}}%
\pgfusepath{stroke}%
\end{pgfscope}%
\begin{pgfscope}%
\pgfsetrectcap%
\pgfsetmiterjoin%
\pgfsetlinewidth{0.803000pt}%
\definecolor{currentstroke}{rgb}{0.000000,0.000000,0.000000}%
\pgfsetstrokecolor{currentstroke}%
\pgfsetdash{}{0pt}%
\pgfpathmoveto{\pgfqpoint{4.857257in}{0.539544in}}%
\pgfpathlineto{\pgfqpoint{4.857257in}{3.120077in}}%
\pgfusepath{stroke}%
\end{pgfscope}%
\begin{pgfscope}%
\pgfsetrectcap%
\pgfsetmiterjoin%
\pgfsetlinewidth{0.803000pt}%
\definecolor{currentstroke}{rgb}{0.000000,0.000000,0.000000}%
\pgfsetstrokecolor{currentstroke}%
\pgfsetdash{}{0pt}%
\pgfpathmoveto{\pgfqpoint{0.667540in}{0.539544in}}%
\pgfpathlineto{\pgfqpoint{4.857257in}{0.539544in}}%
\pgfusepath{stroke}%
\end{pgfscope}%
\begin{pgfscope}%
\pgfsetrectcap%
\pgfsetmiterjoin%
\pgfsetlinewidth{0.803000pt}%
\definecolor{currentstroke}{rgb}{0.000000,0.000000,0.000000}%
\pgfsetstrokecolor{currentstroke}%
\pgfsetdash{}{0pt}%
\pgfpathmoveto{\pgfqpoint{0.667540in}{3.120077in}}%
\pgfpathlineto{\pgfqpoint{4.857257in}{3.120077in}}%
\pgfusepath{stroke}%
\end{pgfscope}%
\begin{pgfscope}%
\pgfsetbuttcap%
\pgfsetmiterjoin%
\definecolor{currentfill}{rgb}{1.000000,1.000000,1.000000}%
\pgfsetfillcolor{currentfill}%
\pgfsetfillopacity{0.800000}%
\pgfsetlinewidth{1.003750pt}%
\definecolor{currentstroke}{rgb}{0.800000,0.800000,0.800000}%
\pgfsetstrokecolor{currentstroke}%
\pgfsetstrokeopacity{0.800000}%
\pgfsetdash{}{0pt}%
\pgfpathmoveto{\pgfqpoint{0.745318in}{2.719411in}}%
\pgfpathlineto{\pgfqpoint{2.244873in}{2.719411in}}%
\pgfpathquadraticcurveto{\pgfqpoint{2.267096in}{2.719411in}}{\pgfqpoint{2.267096in}{2.741633in}}%
\pgfpathlineto{\pgfqpoint{2.267096in}{3.042300in}}%
\pgfpathquadraticcurveto{\pgfqpoint{2.267096in}{3.064522in}}{\pgfqpoint{2.244873in}{3.064522in}}%
\pgfpathlineto{\pgfqpoint{0.745318in}{3.064522in}}%
\pgfpathquadraticcurveto{\pgfqpoint{0.723095in}{3.064522in}}{\pgfqpoint{0.723095in}{3.042300in}}%
\pgfpathlineto{\pgfqpoint{0.723095in}{2.741633in}}%
\pgfpathquadraticcurveto{\pgfqpoint{0.723095in}{2.719411in}}{\pgfqpoint{0.745318in}{2.719411in}}%
\pgfpathlineto{\pgfqpoint{0.745318in}{2.719411in}}%
\pgfpathclose%
\pgfusepath{stroke,fill}%
\end{pgfscope}%
\begin{pgfscope}%
\pgfsetrectcap%
\pgfsetroundjoin%
\pgfsetlinewidth{1.505625pt}%
\definecolor{currentstroke}{rgb}{0.003922,0.450980,0.698039}%
\pgfsetstrokecolor{currentstroke}%
\pgfsetstrokeopacity{0.700000}%
\pgfsetdash{}{0pt}%
\pgfpathmoveto{\pgfqpoint{0.767540in}{2.980744in}}%
\pgfpathlineto{\pgfqpoint{0.878651in}{2.980744in}}%
\pgfpathlineto{\pgfqpoint{0.989762in}{2.980744in}}%
\pgfusepath{stroke}%
\end{pgfscope}%
\begin{pgfscope}%
\definecolor{textcolor}{rgb}{0.000000,0.000000,0.000000}%
\pgfsetstrokecolor{textcolor}%
\pgfsetfillcolor{textcolor}%
\pgftext[x=1.078651in,y=2.941855in,left,base]{\color{textcolor}\rmfamily\fontsize{8.000000}{9.600000}\selectfont Output current}%
\end{pgfscope}%
\begin{pgfscope}%
\pgfsetrectcap%
\pgfsetroundjoin%
\pgfsetlinewidth{1.505625pt}%
\definecolor{currentstroke}{rgb}{0.909804,0.000000,0.043137}%
\pgfsetstrokecolor{currentstroke}%
\pgfsetstrokeopacity{0.700000}%
\pgfsetdash{}{0pt}%
\pgfpathmoveto{\pgfqpoint{0.767540in}{2.824300in}}%
\pgfpathlineto{\pgfqpoint{0.878651in}{2.824300in}}%
\pgfpathlineto{\pgfqpoint{0.989762in}{2.824300in}}%
\pgfusepath{stroke}%
\end{pgfscope}%
\begin{pgfscope}%
\definecolor{textcolor}{rgb}{0.000000,0.000000,0.000000}%
\pgfsetstrokecolor{textcolor}%
\pgfsetfillcolor{textcolor}%
\pgftext[x=1.078651in,y=2.785411in,left,base]{\color{textcolor}\rmfamily\fontsize{8.000000}{9.600000}\selectfont Ambient Temperature}%
\end{pgfscope}%
\end{pgfpicture}%
\makeatother%
\endgroup%

    \caption{Measurering a \qty{50}{\mA} current using a Keysight \device{34470A} with changing ambient temperature. The current source is based on the design of \citeauthor{laser_driver_digital}.}
    \label{fig:laser_driver_aircon}
\end{figure}

Admittedly, the example in figure \ref{fig:laser_driver_aircon} does not reflect the correct way to measure current with high stability, but serves as an excellent example to highlight the problem. The room temperature heavily depends on the inner workings of the air conditioning system used in the building and the problem is not always present. Nonetheless, longer measurements over several days necessitated the development of a PID controller module to replace the stock temperaure controller of the air conditioning unit in the lab. This project is described in section \ref{sec:lab_temp_control}.

In the meantime, measurements, that only required short-term stability were conducted. Measurering the current noise of the drivers is one such measurement.

\clearpage
\subsection{Test Results: Current Noise}
\label{sec:results_current_noise}
The spectral current noise density is a quantity, that is both seemingly trivial to measure and also easy to understand and graphically compare, therefore many devices are emblazoned by such graphs. The upside is, that these numbers can used for reference. Defining the bandwidth of such a measurement is a matter of debate, the available measurement devices and depends of the future use-case of the current driver. We chose an upper frequency of \qty{1}{\MHz} for two reason, first, to limit the number of amplifiers required. As the noise power rises with the bandwidth (in the best case as $\sqrt{\Delta f}$ for white noise) and impedance matching comes into play, a higher power amplifier is required, a trait that does not bode well with low noise, low frequency frontends. So for frequencies above a few \unit{\MHz} different amplifiers are called for. More amplifiers make the whole measurement more intricate, because the amplifiers are the most critical parts in the whole chain.

The second reason does not root in the lazyness of the researcher, but has a physical origin. Cables used in the lab like RG-58 or RG-223 have a capacitance of about \qty{100}{\pF \per \m}. With a cable length of around \qty{3}{\m} resulting in \qty{300}{\pF}, one finds, that at \qty{10}{\MHz} the impedance seen by the laser diode approaches \qty{50}{\ohm}, not unsurprising, given that the cable impedance is about \qty{50}{\ohm} and a signal at \qty{10}{\MHz} has a wavelength of $\lambda \approx \qty{2}{\m}$. This is aproaching the quarter-wave rule beyond which one should treat the cable as a transmission line. It is therefore reasonable to limit the noise measurement to \qty{1}{\MHz}, beyond which a design specific implementation including the laser head is called for anyway.

\clearpage
\subsection{Test Results: Stability}
\label{sec:results_stability}
When remotely controlling a laser system, stability of the laser driver is of immediate concern, because uninterrupted operation of the system is a key requirement and if the laser cannot be locked again remotely, it is time consuming, if possible at all, to go to the remote laser lab and readjust the laser current driver. The development of the past years have also shown a greater demand of remote working rendering readjustment unfeasable. To assure the specifications given in \ref{lst:dgDrive_specs_environment}, the current drivers were first tested for \qty{24}{\hour} to ensure. This test was refined several times over the course of this work, to reflect the need for a better signal so noise ratio. The first tests were done by feeding the output current of the laser driver into a (calibrated) Keysight \device{34470A} and measuring the output over \qty{24}{\hour}. The \device{34470A} was warmed up for \qty{8}{\hour} and so was the laser driver. The ambient temperature and humidity were recorded by the lab monitoring system descibed in section \ref{}.


\subsection{Zener Diode Selection}
\label{sec:zener_diode_selection}
Early tests of the LM399 Zener diode as a reference have confirmed, what the data sheet \cite{datasheet_LM399} already suggest in the 'Low Frequency Noise Voltage' plot. There are random bi-stable voltage step changes. This phenomenon is called burst noise or popcorn noise.

\begin{figure}[ht]
    \centering
    %% Creator: Matplotlib, PGF backend
%%
%% To include the figure in your LaTeX document, write
%%   \input{<filename>.pgf}
%%
%% Make sure the required packages are loaded in your preamble
%%   \usepackage{pgf}
%%
%% Also ensure that all the required font packages are loaded; for instance,
%% the lmodern package is sometimes necessary when using math font.
%%   \usepackage{lmodern}
%%
%% Figures using additional raster images can only be included by \input if
%% they are in the same directory as the main LaTeX file. For loading figures
%% from other directories you can use the `import` package
%%   \usepackage{import}
%%
%% and then include the figures with
%%   \import{<path to file>}{<filename>.pgf}
%%
%% Matplotlib used the following preamble
%%   \usepackage{fontspec}
%%
\begingroup%
\makeatletter%
\begin{pgfpicture}%
\pgfpathrectangle{\pgfpointorigin}{\pgfqpoint{5.200000in}{3.210000in}}%
\pgfusepath{use as bounding box, clip}%
\begin{pgfscope}%
\pgfsetbuttcap%
\pgfsetmiterjoin%
\definecolor{currentfill}{rgb}{1.000000,1.000000,1.000000}%
\pgfsetfillcolor{currentfill}%
\pgfsetlinewidth{0.000000pt}%
\definecolor{currentstroke}{rgb}{1.000000,1.000000,1.000000}%
\pgfsetstrokecolor{currentstroke}%
\pgfsetdash{}{0pt}%
\pgfpathmoveto{\pgfqpoint{0.000000in}{0.000000in}}%
\pgfpathlineto{\pgfqpoint{5.200000in}{0.000000in}}%
\pgfpathlineto{\pgfqpoint{5.200000in}{3.210000in}}%
\pgfpathlineto{\pgfqpoint{0.000000in}{3.210000in}}%
\pgfpathlineto{\pgfqpoint{0.000000in}{0.000000in}}%
\pgfpathclose%
\pgfusepath{fill}%
\end{pgfscope}%
\begin{pgfscope}%
\pgfsetbuttcap%
\pgfsetmiterjoin%
\definecolor{currentfill}{rgb}{1.000000,1.000000,1.000000}%
\pgfsetfillcolor{currentfill}%
\pgfsetlinewidth{0.000000pt}%
\definecolor{currentstroke}{rgb}{0.000000,0.000000,0.000000}%
\pgfsetstrokecolor{currentstroke}%
\pgfsetstrokeopacity{0.000000}%
\pgfsetdash{}{0pt}%
\pgfpathmoveto{\pgfqpoint{0.483776in}{2.351653in}}%
\pgfpathlineto{\pgfqpoint{5.050249in}{2.351653in}}%
\pgfpathlineto{\pgfqpoint{5.050249in}{2.936535in}}%
\pgfpathlineto{\pgfqpoint{0.483776in}{2.936535in}}%
\pgfpathlineto{\pgfqpoint{0.483776in}{2.351653in}}%
\pgfpathclose%
\pgfusepath{fill}%
\end{pgfscope}%
\begin{pgfscope}%
\pgfsetbuttcap%
\pgfsetroundjoin%
\definecolor{currentfill}{rgb}{0.000000,0.000000,0.000000}%
\pgfsetfillcolor{currentfill}%
\pgfsetlinewidth{0.803000pt}%
\definecolor{currentstroke}{rgb}{0.000000,0.000000,0.000000}%
\pgfsetstrokecolor{currentstroke}%
\pgfsetdash{}{0pt}%
\pgfsys@defobject{currentmarker}{\pgfqpoint{0.000000in}{-0.048611in}}{\pgfqpoint{0.000000in}{0.000000in}}{%
\pgfpathmoveto{\pgfqpoint{0.000000in}{0.000000in}}%
\pgfpathlineto{\pgfqpoint{0.000000in}{-0.048611in}}%
\pgfusepath{stroke,fill}%
}%
\begin{pgfscope}%
\pgfsys@transformshift{0.691021in}{2.351653in}%
\pgfsys@useobject{currentmarker}{}%
\end{pgfscope}%
\end{pgfscope}%
\begin{pgfscope}%
\pgfsetbuttcap%
\pgfsetroundjoin%
\definecolor{currentfill}{rgb}{0.000000,0.000000,0.000000}%
\pgfsetfillcolor{currentfill}%
\pgfsetlinewidth{0.803000pt}%
\definecolor{currentstroke}{rgb}{0.000000,0.000000,0.000000}%
\pgfsetstrokecolor{currentstroke}%
\pgfsetdash{}{0pt}%
\pgfsys@defobject{currentmarker}{\pgfqpoint{0.000000in}{-0.048611in}}{\pgfqpoint{0.000000in}{0.000000in}}{%
\pgfpathmoveto{\pgfqpoint{0.000000in}{0.000000in}}%
\pgfpathlineto{\pgfqpoint{0.000000in}{-0.048611in}}%
\pgfusepath{stroke,fill}%
}%
\begin{pgfscope}%
\pgfsys@transformshift{1.210067in}{2.351653in}%
\pgfsys@useobject{currentmarker}{}%
\end{pgfscope}%
\end{pgfscope}%
\begin{pgfscope}%
\pgfsetbuttcap%
\pgfsetroundjoin%
\definecolor{currentfill}{rgb}{0.000000,0.000000,0.000000}%
\pgfsetfillcolor{currentfill}%
\pgfsetlinewidth{0.803000pt}%
\definecolor{currentstroke}{rgb}{0.000000,0.000000,0.000000}%
\pgfsetstrokecolor{currentstroke}%
\pgfsetdash{}{0pt}%
\pgfsys@defobject{currentmarker}{\pgfqpoint{0.000000in}{-0.048611in}}{\pgfqpoint{0.000000in}{0.000000in}}{%
\pgfpathmoveto{\pgfqpoint{0.000000in}{0.000000in}}%
\pgfpathlineto{\pgfqpoint{0.000000in}{-0.048611in}}%
\pgfusepath{stroke,fill}%
}%
\begin{pgfscope}%
\pgfsys@transformshift{1.729114in}{2.351653in}%
\pgfsys@useobject{currentmarker}{}%
\end{pgfscope}%
\end{pgfscope}%
\begin{pgfscope}%
\pgfsetbuttcap%
\pgfsetroundjoin%
\definecolor{currentfill}{rgb}{0.000000,0.000000,0.000000}%
\pgfsetfillcolor{currentfill}%
\pgfsetlinewidth{0.803000pt}%
\definecolor{currentstroke}{rgb}{0.000000,0.000000,0.000000}%
\pgfsetstrokecolor{currentstroke}%
\pgfsetdash{}{0pt}%
\pgfsys@defobject{currentmarker}{\pgfqpoint{0.000000in}{-0.048611in}}{\pgfqpoint{0.000000in}{0.000000in}}{%
\pgfpathmoveto{\pgfqpoint{0.000000in}{0.000000in}}%
\pgfpathlineto{\pgfqpoint{0.000000in}{-0.048611in}}%
\pgfusepath{stroke,fill}%
}%
\begin{pgfscope}%
\pgfsys@transformshift{2.248160in}{2.351653in}%
\pgfsys@useobject{currentmarker}{}%
\end{pgfscope}%
\end{pgfscope}%
\begin{pgfscope}%
\pgfsetbuttcap%
\pgfsetroundjoin%
\definecolor{currentfill}{rgb}{0.000000,0.000000,0.000000}%
\pgfsetfillcolor{currentfill}%
\pgfsetlinewidth{0.803000pt}%
\definecolor{currentstroke}{rgb}{0.000000,0.000000,0.000000}%
\pgfsetstrokecolor{currentstroke}%
\pgfsetdash{}{0pt}%
\pgfsys@defobject{currentmarker}{\pgfqpoint{0.000000in}{-0.048611in}}{\pgfqpoint{0.000000in}{0.000000in}}{%
\pgfpathmoveto{\pgfqpoint{0.000000in}{0.000000in}}%
\pgfpathlineto{\pgfqpoint{0.000000in}{-0.048611in}}%
\pgfusepath{stroke,fill}%
}%
\begin{pgfscope}%
\pgfsys@transformshift{2.767206in}{2.351653in}%
\pgfsys@useobject{currentmarker}{}%
\end{pgfscope}%
\end{pgfscope}%
\begin{pgfscope}%
\pgfsetbuttcap%
\pgfsetroundjoin%
\definecolor{currentfill}{rgb}{0.000000,0.000000,0.000000}%
\pgfsetfillcolor{currentfill}%
\pgfsetlinewidth{0.803000pt}%
\definecolor{currentstroke}{rgb}{0.000000,0.000000,0.000000}%
\pgfsetstrokecolor{currentstroke}%
\pgfsetdash{}{0pt}%
\pgfsys@defobject{currentmarker}{\pgfqpoint{0.000000in}{-0.048611in}}{\pgfqpoint{0.000000in}{0.000000in}}{%
\pgfpathmoveto{\pgfqpoint{0.000000in}{0.000000in}}%
\pgfpathlineto{\pgfqpoint{0.000000in}{-0.048611in}}%
\pgfusepath{stroke,fill}%
}%
\begin{pgfscope}%
\pgfsys@transformshift{3.286252in}{2.351653in}%
\pgfsys@useobject{currentmarker}{}%
\end{pgfscope}%
\end{pgfscope}%
\begin{pgfscope}%
\pgfsetbuttcap%
\pgfsetroundjoin%
\definecolor{currentfill}{rgb}{0.000000,0.000000,0.000000}%
\pgfsetfillcolor{currentfill}%
\pgfsetlinewidth{0.803000pt}%
\definecolor{currentstroke}{rgb}{0.000000,0.000000,0.000000}%
\pgfsetstrokecolor{currentstroke}%
\pgfsetdash{}{0pt}%
\pgfsys@defobject{currentmarker}{\pgfqpoint{0.000000in}{-0.048611in}}{\pgfqpoint{0.000000in}{0.000000in}}{%
\pgfpathmoveto{\pgfqpoint{0.000000in}{0.000000in}}%
\pgfpathlineto{\pgfqpoint{0.000000in}{-0.048611in}}%
\pgfusepath{stroke,fill}%
}%
\begin{pgfscope}%
\pgfsys@transformshift{3.805298in}{2.351653in}%
\pgfsys@useobject{currentmarker}{}%
\end{pgfscope}%
\end{pgfscope}%
\begin{pgfscope}%
\pgfsetbuttcap%
\pgfsetroundjoin%
\definecolor{currentfill}{rgb}{0.000000,0.000000,0.000000}%
\pgfsetfillcolor{currentfill}%
\pgfsetlinewidth{0.803000pt}%
\definecolor{currentstroke}{rgb}{0.000000,0.000000,0.000000}%
\pgfsetstrokecolor{currentstroke}%
\pgfsetdash{}{0pt}%
\pgfsys@defobject{currentmarker}{\pgfqpoint{0.000000in}{-0.048611in}}{\pgfqpoint{0.000000in}{0.000000in}}{%
\pgfpathmoveto{\pgfqpoint{0.000000in}{0.000000in}}%
\pgfpathlineto{\pgfqpoint{0.000000in}{-0.048611in}}%
\pgfusepath{stroke,fill}%
}%
\begin{pgfscope}%
\pgfsys@transformshift{4.324344in}{2.351653in}%
\pgfsys@useobject{currentmarker}{}%
\end{pgfscope}%
\end{pgfscope}%
\begin{pgfscope}%
\pgfsetbuttcap%
\pgfsetroundjoin%
\definecolor{currentfill}{rgb}{0.000000,0.000000,0.000000}%
\pgfsetfillcolor{currentfill}%
\pgfsetlinewidth{0.803000pt}%
\definecolor{currentstroke}{rgb}{0.000000,0.000000,0.000000}%
\pgfsetstrokecolor{currentstroke}%
\pgfsetdash{}{0pt}%
\pgfsys@defobject{currentmarker}{\pgfqpoint{0.000000in}{-0.048611in}}{\pgfqpoint{0.000000in}{0.000000in}}{%
\pgfpathmoveto{\pgfqpoint{0.000000in}{0.000000in}}%
\pgfpathlineto{\pgfqpoint{0.000000in}{-0.048611in}}%
\pgfusepath{stroke,fill}%
}%
\begin{pgfscope}%
\pgfsys@transformshift{4.843390in}{2.351653in}%
\pgfsys@useobject{currentmarker}{}%
\end{pgfscope}%
\end{pgfscope}%
\begin{pgfscope}%
\pgfsetbuttcap%
\pgfsetroundjoin%
\definecolor{currentfill}{rgb}{0.000000,0.000000,0.000000}%
\pgfsetfillcolor{currentfill}%
\pgfsetlinewidth{0.803000pt}%
\definecolor{currentstroke}{rgb}{0.000000,0.000000,0.000000}%
\pgfsetstrokecolor{currentstroke}%
\pgfsetdash{}{0pt}%
\pgfsys@defobject{currentmarker}{\pgfqpoint{-0.048611in}{0.000000in}}{\pgfqpoint{-0.000000in}{0.000000in}}{%
\pgfpathmoveto{\pgfqpoint{-0.000000in}{0.000000in}}%
\pgfpathlineto{\pgfqpoint{-0.048611in}{0.000000in}}%
\pgfusepath{stroke,fill}%
}%
\begin{pgfscope}%
\pgfsys@transformshift{0.483776in}{2.532831in}%
\pgfsys@useobject{currentmarker}{}%
\end{pgfscope}%
\end{pgfscope}%
\begin{pgfscope}%
\definecolor{textcolor}{rgb}{0.000000,0.000000,0.000000}%
\pgfsetstrokecolor{textcolor}%
\pgfsetfillcolor{textcolor}%
\pgftext[x=0.327525in, y=2.494275in, left, base]{\color{textcolor}\rmfamily\fontsize{8.000000}{9.600000}\selectfont \(\displaystyle {0}\)}%
\end{pgfscope}%
\begin{pgfscope}%
\pgfsetbuttcap%
\pgfsetroundjoin%
\definecolor{currentfill}{rgb}{0.000000,0.000000,0.000000}%
\pgfsetfillcolor{currentfill}%
\pgfsetlinewidth{0.803000pt}%
\definecolor{currentstroke}{rgb}{0.000000,0.000000,0.000000}%
\pgfsetstrokecolor{currentstroke}%
\pgfsetdash{}{0pt}%
\pgfsys@defobject{currentmarker}{\pgfqpoint{-0.048611in}{0.000000in}}{\pgfqpoint{-0.000000in}{0.000000in}}{%
\pgfpathmoveto{\pgfqpoint{-0.000000in}{0.000000in}}%
\pgfpathlineto{\pgfqpoint{-0.048611in}{0.000000in}}%
\pgfusepath{stroke,fill}%
}%
\begin{pgfscope}%
\pgfsys@transformshift{0.483776in}{2.740006in}%
\pgfsys@useobject{currentmarker}{}%
\end{pgfscope}%
\end{pgfscope}%
\begin{pgfscope}%
\definecolor{textcolor}{rgb}{0.000000,0.000000,0.000000}%
\pgfsetstrokecolor{textcolor}%
\pgfsetfillcolor{textcolor}%
\pgftext[x=0.327525in, y=2.701450in, left, base]{\color{textcolor}\rmfamily\fontsize{8.000000}{9.600000}\selectfont \(\displaystyle {5}\)}%
\end{pgfscope}%
\begin{pgfscope}%
\definecolor{textcolor}{rgb}{0.000000,0.000000,0.000000}%
\pgfsetstrokecolor{textcolor}%
\pgfsetfillcolor{textcolor}%
\pgftext[x=0.483776in,y=2.978201in,left,base]{\color{textcolor}\rmfamily\fontsize{8.000000}{9.600000}\selectfont \(\displaystyle \times{10^{\ensuremath{-}6}}{}\)}%
\end{pgfscope}%
\begin{pgfscope}%
\pgfpathrectangle{\pgfqpoint{0.483776in}{2.351653in}}{\pgfqpoint{4.566474in}{0.584881in}}%
\pgfusepath{clip}%
\pgfsetrectcap%
\pgfsetroundjoin%
\pgfsetlinewidth{0.501875pt}%
\definecolor{currentstroke}{rgb}{0.121569,0.466667,0.705882}%
\pgfsetstrokecolor{currentstroke}%
\pgfsetstrokeopacity{0.700000}%
\pgfsetdash{}{0pt}%
\pgfpathmoveto{\pgfqpoint{0.691343in}{2.530219in}}%
\pgfpathlineto{\pgfqpoint{0.692205in}{2.507777in}}%
\pgfpathlineto{\pgfqpoint{0.693071in}{2.590474in}}%
\pgfpathlineto{\pgfqpoint{0.694800in}{2.506363in}}%
\pgfpathlineto{\pgfqpoint{0.695666in}{2.545836in}}%
\pgfpathlineto{\pgfqpoint{0.696532in}{2.498186in}}%
\pgfpathlineto{\pgfqpoint{0.697397in}{2.501383in}}%
\pgfpathlineto{\pgfqpoint{0.699128in}{2.538888in}}%
\pgfpathlineto{\pgfqpoint{0.699993in}{2.536184in}}%
\pgfpathlineto{\pgfqpoint{0.700859in}{2.601850in}}%
\pgfpathlineto{\pgfqpoint{0.701725in}{2.538336in}}%
\pgfpathlineto{\pgfqpoint{0.702589in}{2.553523in}}%
\pgfpathlineto{\pgfqpoint{0.703453in}{2.510576in}}%
\pgfpathlineto{\pgfqpoint{0.706051in}{2.564068in}}%
\pgfpathlineto{\pgfqpoint{0.707780in}{2.536184in}}%
\pgfpathlineto{\pgfqpoint{0.708646in}{2.529913in}}%
\pgfpathlineto{\pgfqpoint{0.709512in}{2.574367in}}%
\pgfpathlineto{\pgfqpoint{0.710377in}{2.521305in}}%
\pgfpathlineto{\pgfqpoint{0.712105in}{2.585309in}}%
\pgfpathlineto{\pgfqpoint{0.712971in}{2.517924in}}%
\pgfpathlineto{\pgfqpoint{0.713837in}{2.551432in}}%
\pgfpathlineto{\pgfqpoint{0.714702in}{2.479803in}}%
\pgfpathlineto{\pgfqpoint{0.716430in}{2.549464in}}%
\pgfpathlineto{\pgfqpoint{0.719030in}{2.475436in}}%
\pgfpathlineto{\pgfqpoint{0.719895in}{2.469535in}}%
\pgfpathlineto{\pgfqpoint{0.720762in}{2.400426in}}%
\pgfpathlineto{\pgfqpoint{0.721627in}{2.506486in}}%
\pgfpathlineto{\pgfqpoint{0.722492in}{2.468950in}}%
\pgfpathlineto{\pgfqpoint{0.723356in}{2.490807in}}%
\pgfpathlineto{\pgfqpoint{0.724219in}{2.547435in}}%
\pgfpathlineto{\pgfqpoint{0.725084in}{2.460004in}}%
\pgfpathlineto{\pgfqpoint{0.725947in}{2.531019in}}%
\pgfpathlineto{\pgfqpoint{0.726812in}{2.496280in}}%
\pgfpathlineto{\pgfqpoint{0.728541in}{2.519030in}}%
\pgfpathlineto{\pgfqpoint{0.729407in}{2.497203in}}%
\pgfpathlineto{\pgfqpoint{0.730271in}{2.584758in}}%
\pgfpathlineto{\pgfqpoint{0.732865in}{2.466092in}}%
\pgfpathlineto{\pgfqpoint{0.733731in}{2.535384in}}%
\pgfpathlineto{\pgfqpoint{0.734597in}{2.484044in}}%
\pgfpathlineto{\pgfqpoint{0.735460in}{2.484475in}}%
\pgfpathlineto{\pgfqpoint{0.736325in}{2.528682in}}%
\pgfpathlineto{\pgfqpoint{0.737190in}{2.491331in}}%
\pgfpathlineto{\pgfqpoint{0.738056in}{2.494989in}}%
\pgfpathlineto{\pgfqpoint{0.738920in}{2.504888in}}%
\pgfpathlineto{\pgfqpoint{0.740649in}{2.546144in}}%
\pgfpathlineto{\pgfqpoint{0.742381in}{2.482078in}}%
\pgfpathlineto{\pgfqpoint{0.743246in}{2.512636in}}%
\pgfpathlineto{\pgfqpoint{0.744112in}{2.503659in}}%
\pgfpathlineto{\pgfqpoint{0.744977in}{2.529852in}}%
\pgfpathlineto{\pgfqpoint{0.745840in}{2.522658in}}%
\pgfpathlineto{\pgfqpoint{0.746705in}{2.526285in}}%
\pgfpathlineto{\pgfqpoint{0.747569in}{2.494989in}}%
\pgfpathlineto{\pgfqpoint{0.748435in}{2.558934in}}%
\pgfpathlineto{\pgfqpoint{0.749299in}{2.542886in}}%
\pgfpathlineto{\pgfqpoint{0.750163in}{2.552294in}}%
\pgfpathlineto{\pgfqpoint{0.751028in}{2.540704in}}%
\pgfpathlineto{\pgfqpoint{0.752757in}{2.494006in}}%
\pgfpathlineto{\pgfqpoint{0.753622in}{2.550357in}}%
\pgfpathlineto{\pgfqpoint{0.754488in}{2.440761in}}%
\pgfpathlineto{\pgfqpoint{0.755354in}{2.493268in}}%
\pgfpathlineto{\pgfqpoint{0.756219in}{2.471287in}}%
\pgfpathlineto{\pgfqpoint{0.757950in}{2.535078in}}%
\pgfpathlineto{\pgfqpoint{0.758816in}{2.534524in}}%
\pgfpathlineto{\pgfqpoint{0.759679in}{2.531450in}}%
\pgfpathlineto{\pgfqpoint{0.760542in}{2.495420in}}%
\pgfpathlineto{\pgfqpoint{0.761407in}{2.534339in}}%
\pgfpathlineto{\pgfqpoint{0.762269in}{2.509685in}}%
\pgfpathlineto{\pgfqpoint{0.763135in}{2.514727in}}%
\pgfpathlineto{\pgfqpoint{0.764000in}{2.521429in}}%
\pgfpathlineto{\pgfqpoint{0.764865in}{2.551617in}}%
\pgfpathlineto{\pgfqpoint{0.765730in}{2.507256in}}%
\pgfpathlineto{\pgfqpoint{0.766596in}{2.566988in}}%
\pgfpathlineto{\pgfqpoint{0.767462in}{2.500277in}}%
\pgfpathlineto{\pgfqpoint{0.769191in}{2.567172in}}%
\pgfpathlineto{\pgfqpoint{0.770922in}{2.489763in}}%
\pgfpathlineto{\pgfqpoint{0.771788in}{2.603141in}}%
\pgfpathlineto{\pgfqpoint{0.772652in}{2.517216in}}%
\pgfpathlineto{\pgfqpoint{0.773518in}{2.550939in}}%
\pgfpathlineto{\pgfqpoint{0.774383in}{2.502920in}}%
\pgfpathlineto{\pgfqpoint{0.775249in}{2.504857in}}%
\pgfpathlineto{\pgfqpoint{0.776979in}{2.541778in}}%
\pgfpathlineto{\pgfqpoint{0.779575in}{2.501568in}}%
\pgfpathlineto{\pgfqpoint{0.780439in}{2.532310in}}%
\pgfpathlineto{\pgfqpoint{0.781305in}{2.524870in}}%
\pgfpathlineto{\pgfqpoint{0.782170in}{2.511128in}}%
\pgfpathlineto{\pgfqpoint{0.783036in}{2.593610in}}%
\pgfpathlineto{\pgfqpoint{0.784766in}{2.501321in}}%
\pgfpathlineto{\pgfqpoint{0.785630in}{2.528990in}}%
\pgfpathlineto{\pgfqpoint{0.787360in}{2.487180in}}%
\pgfpathlineto{\pgfqpoint{0.788225in}{2.466951in}}%
\pgfpathlineto{\pgfqpoint{0.789089in}{2.410386in}}%
\pgfpathlineto{\pgfqpoint{0.789954in}{2.538704in}}%
\pgfpathlineto{\pgfqpoint{0.790816in}{2.520444in}}%
\pgfpathlineto{\pgfqpoint{0.791680in}{2.480601in}}%
\pgfpathlineto{\pgfqpoint{0.792546in}{2.576702in}}%
\pgfpathlineto{\pgfqpoint{0.793412in}{2.571906in}}%
\pgfpathlineto{\pgfqpoint{0.794274in}{2.527576in}}%
\pgfpathlineto{\pgfqpoint{0.795138in}{2.567480in}}%
\pgfpathlineto{\pgfqpoint{0.796004in}{2.509254in}}%
\pgfpathlineto{\pgfqpoint{0.796868in}{2.534216in}}%
\pgfpathlineto{\pgfqpoint{0.798599in}{2.477897in}}%
\pgfpathlineto{\pgfqpoint{0.799464in}{2.543440in}}%
\pgfpathlineto{\pgfqpoint{0.800327in}{2.537967in}}%
\pgfpathlineto{\pgfqpoint{0.801192in}{2.472855in}}%
\pgfpathlineto{\pgfqpoint{0.802058in}{2.568894in}}%
\pgfpathlineto{\pgfqpoint{0.803787in}{2.451429in}}%
\pgfpathlineto{\pgfqpoint{0.804652in}{2.464616in}}%
\pgfpathlineto{\pgfqpoint{0.805515in}{2.514727in}}%
\pgfpathlineto{\pgfqpoint{0.806378in}{2.440576in}}%
\pgfpathlineto{\pgfqpoint{0.808107in}{2.534401in}}%
\pgfpathlineto{\pgfqpoint{0.808973in}{2.535938in}}%
\pgfpathlineto{\pgfqpoint{0.809837in}{2.465722in}}%
\pgfpathlineto{\pgfqpoint{0.812431in}{2.522350in}}%
\pgfpathlineto{\pgfqpoint{0.813297in}{2.496218in}}%
\pgfpathlineto{\pgfqpoint{0.814161in}{2.511713in}}%
\pgfpathlineto{\pgfqpoint{0.815026in}{2.557949in}}%
\pgfpathlineto{\pgfqpoint{0.817620in}{2.473284in}}%
\pgfpathlineto{\pgfqpoint{0.818484in}{2.579837in}}%
\pgfpathlineto{\pgfqpoint{0.819349in}{2.551309in}}%
\pgfpathlineto{\pgfqpoint{0.820213in}{2.504272in}}%
\pgfpathlineto{\pgfqpoint{0.821078in}{2.576763in}}%
\pgfpathlineto{\pgfqpoint{0.821942in}{2.542455in}}%
\pgfpathlineto{\pgfqpoint{0.822807in}{2.555642in}}%
\pgfpathlineto{\pgfqpoint{0.823671in}{2.672556in}}%
\pgfpathlineto{\pgfqpoint{0.824536in}{2.516693in}}%
\pgfpathlineto{\pgfqpoint{0.825401in}{2.582297in}}%
\pgfpathlineto{\pgfqpoint{0.826268in}{2.535630in}}%
\pgfpathlineto{\pgfqpoint{0.827133in}{2.539596in}}%
\pgfpathlineto{\pgfqpoint{0.827999in}{2.555244in}}%
\pgfpathlineto{\pgfqpoint{0.828864in}{2.551924in}}%
\pgfpathlineto{\pgfqpoint{0.829730in}{2.505565in}}%
\pgfpathlineto{\pgfqpoint{0.831462in}{2.639784in}}%
\pgfpathlineto{\pgfqpoint{0.832328in}{2.563391in}}%
\pgfpathlineto{\pgfqpoint{0.833193in}{2.597730in}}%
\pgfpathlineto{\pgfqpoint{0.835786in}{2.510053in}}%
\pgfpathlineto{\pgfqpoint{0.836650in}{2.530650in}}%
\pgfpathlineto{\pgfqpoint{0.837516in}{2.501814in}}%
\pgfpathlineto{\pgfqpoint{0.838382in}{2.518968in}}%
\pgfpathlineto{\pgfqpoint{0.840114in}{2.487303in}}%
\pgfpathlineto{\pgfqpoint{0.840980in}{2.522042in}}%
\pgfpathlineto{\pgfqpoint{0.841845in}{2.510145in}}%
\pgfpathlineto{\pgfqpoint{0.842710in}{2.472239in}}%
\pgfpathlineto{\pgfqpoint{0.843575in}{2.501014in}}%
\pgfpathlineto{\pgfqpoint{0.844440in}{2.458220in}}%
\pgfpathlineto{\pgfqpoint{0.845305in}{2.550509in}}%
\pgfpathlineto{\pgfqpoint{0.847033in}{2.473961in}}%
\pgfpathlineto{\pgfqpoint{0.847897in}{2.532125in}}%
\pgfpathlineto{\pgfqpoint{0.849627in}{2.491485in}}%
\pgfpathlineto{\pgfqpoint{0.850491in}{2.489763in}}%
\pgfpathlineto{\pgfqpoint{0.851354in}{2.469073in}}%
\pgfpathlineto{\pgfqpoint{0.853081in}{2.535568in}}%
\pgfpathlineto{\pgfqpoint{0.853945in}{2.515033in}}%
\pgfpathlineto{\pgfqpoint{0.854810in}{2.520690in}}%
\pgfpathlineto{\pgfqpoint{0.855672in}{2.572858in}}%
\pgfpathlineto{\pgfqpoint{0.856537in}{2.558072in}}%
\pgfpathlineto{\pgfqpoint{0.857402in}{2.494250in}}%
\pgfpathlineto{\pgfqpoint{0.858267in}{2.561330in}}%
\pgfpathlineto{\pgfqpoint{0.859131in}{2.549218in}}%
\pgfpathlineto{\pgfqpoint{0.859997in}{2.479187in}}%
\pgfpathlineto{\pgfqpoint{0.860864in}{2.482630in}}%
\pgfpathlineto{\pgfqpoint{0.862597in}{2.580022in}}%
\pgfpathlineto{\pgfqpoint{0.863463in}{2.521796in}}%
\pgfpathlineto{\pgfqpoint{0.864329in}{2.544176in}}%
\pgfpathlineto{\pgfqpoint{0.866062in}{2.483798in}}%
\pgfpathlineto{\pgfqpoint{0.866929in}{2.519643in}}%
\pgfpathlineto{\pgfqpoint{0.867794in}{2.512511in}}%
\pgfpathlineto{\pgfqpoint{0.868659in}{2.511035in}}%
\pgfpathlineto{\pgfqpoint{0.869523in}{2.503841in}}%
\pgfpathlineto{\pgfqpoint{0.870388in}{2.487210in}}%
\pgfpathlineto{\pgfqpoint{0.872119in}{2.555920in}}%
\pgfpathlineto{\pgfqpoint{0.873848in}{2.522902in}}%
\pgfpathlineto{\pgfqpoint{0.874712in}{2.512141in}}%
\pgfpathlineto{\pgfqpoint{0.875577in}{2.538581in}}%
\pgfpathlineto{\pgfqpoint{0.876442in}{2.496955in}}%
\pgfpathlineto{\pgfqpoint{0.877309in}{2.512388in}}%
\pgfpathlineto{\pgfqpoint{0.878174in}{2.500275in}}%
\pgfpathlineto{\pgfqpoint{0.879035in}{2.516323in}}%
\pgfpathlineto{\pgfqpoint{0.879903in}{2.487333in}}%
\pgfpathlineto{\pgfqpoint{0.880769in}{2.493696in}}%
\pgfpathlineto{\pgfqpoint{0.881633in}{2.531694in}}%
\pgfpathlineto{\pgfqpoint{0.882498in}{2.495787in}}%
\pgfpathlineto{\pgfqpoint{0.884229in}{2.560499in}}%
\pgfpathlineto{\pgfqpoint{0.885096in}{2.504395in}}%
\pgfpathlineto{\pgfqpoint{0.886826in}{2.583126in}}%
\pgfpathlineto{\pgfqpoint{0.887691in}{2.422498in}}%
\pgfpathlineto{\pgfqpoint{0.889418in}{2.543961in}}%
\pgfpathlineto{\pgfqpoint{0.890283in}{2.565880in}}%
\pgfpathlineto{\pgfqpoint{0.891148in}{2.557395in}}%
\pgfpathlineto{\pgfqpoint{0.892011in}{2.564035in}}%
\pgfpathlineto{\pgfqpoint{0.893741in}{2.529726in}}%
\pgfpathlineto{\pgfqpoint{0.894605in}{2.551553in}}%
\pgfpathlineto{\pgfqpoint{0.896337in}{2.531725in}}%
\pgfpathlineto{\pgfqpoint{0.897202in}{2.561146in}}%
\pgfpathlineto{\pgfqpoint{0.898931in}{2.473561in}}%
\pgfpathlineto{\pgfqpoint{0.899795in}{2.509252in}}%
\pgfpathlineto{\pgfqpoint{0.900660in}{2.595085in}}%
\pgfpathlineto{\pgfqpoint{0.901525in}{2.502735in}}%
\pgfpathlineto{\pgfqpoint{0.902390in}{2.559547in}}%
\pgfpathlineto{\pgfqpoint{0.903254in}{2.503872in}}%
\pgfpathlineto{\pgfqpoint{0.904119in}{2.575288in}}%
\pgfpathlineto{\pgfqpoint{0.905849in}{2.545405in}}%
\pgfpathlineto{\pgfqpoint{0.907579in}{2.600250in}}%
\pgfpathlineto{\pgfqpoint{0.908445in}{2.501075in}}%
\pgfpathlineto{\pgfqpoint{0.910175in}{2.834938in}}%
\pgfpathlineto{\pgfqpoint{0.911039in}{2.755746in}}%
\pgfpathlineto{\pgfqpoint{0.911905in}{2.798847in}}%
\pgfpathlineto{\pgfqpoint{0.912770in}{2.696475in}}%
\pgfpathlineto{\pgfqpoint{0.913635in}{2.736931in}}%
\pgfpathlineto{\pgfqpoint{0.914501in}{2.692170in}}%
\pgfpathlineto{\pgfqpoint{0.916229in}{2.784951in}}%
\pgfpathlineto{\pgfqpoint{0.917094in}{2.781508in}}%
\pgfpathlineto{\pgfqpoint{0.917959in}{2.753809in}}%
\pgfpathlineto{\pgfqpoint{0.918824in}{2.549341in}}%
\pgfpathlineto{\pgfqpoint{0.919687in}{2.820183in}}%
\pgfpathlineto{\pgfqpoint{0.920552in}{2.780525in}}%
\pgfpathlineto{\pgfqpoint{0.921417in}{2.732813in}}%
\pgfpathlineto{\pgfqpoint{0.922280in}{2.833032in}}%
\pgfpathlineto{\pgfqpoint{0.924010in}{2.523517in}}%
\pgfpathlineto{\pgfqpoint{0.925740in}{2.485827in}}%
\pgfpathlineto{\pgfqpoint{0.927464in}{2.528467in}}%
\pgfpathlineto{\pgfqpoint{0.929193in}{2.484229in}}%
\pgfpathlineto{\pgfqpoint{0.930058in}{2.556166in}}%
\pgfpathlineto{\pgfqpoint{0.931787in}{2.469717in}}%
\pgfpathlineto{\pgfqpoint{0.932652in}{2.481645in}}%
\pgfpathlineto{\pgfqpoint{0.934381in}{2.541685in}}%
\pgfpathlineto{\pgfqpoint{0.935246in}{2.506055in}}%
\pgfpathlineto{\pgfqpoint{0.936109in}{2.512634in}}%
\pgfpathlineto{\pgfqpoint{0.936974in}{2.520934in}}%
\pgfpathlineto{\pgfqpoint{0.937840in}{2.507223in}}%
\pgfpathlineto{\pgfqpoint{0.938706in}{2.528251in}}%
\pgfpathlineto{\pgfqpoint{0.939570in}{2.454285in}}%
\pgfpathlineto{\pgfqpoint{0.941301in}{2.497938in}}%
\pgfpathlineto{\pgfqpoint{0.943032in}{2.479862in}}%
\pgfpathlineto{\pgfqpoint{0.943897in}{2.565664in}}%
\pgfpathlineto{\pgfqpoint{0.944763in}{2.555612in}}%
\pgfpathlineto{\pgfqpoint{0.945628in}{2.568953in}}%
\pgfpathlineto{\pgfqpoint{0.946490in}{2.480109in}}%
\pgfpathlineto{\pgfqpoint{0.947355in}{2.483921in}}%
\pgfpathlineto{\pgfqpoint{0.948218in}{2.525177in}}%
\pgfpathlineto{\pgfqpoint{0.951674in}{2.469410in}}%
\pgfpathlineto{\pgfqpoint{0.954269in}{2.527820in}}%
\pgfpathlineto{\pgfqpoint{0.955133in}{2.489270in}}%
\pgfpathlineto{\pgfqpoint{0.958594in}{2.554198in}}%
\pgfpathlineto{\pgfqpoint{0.959459in}{2.458344in}}%
\pgfpathlineto{\pgfqpoint{0.960325in}{2.558103in}}%
\pgfpathlineto{\pgfqpoint{0.961191in}{2.541041in}}%
\pgfpathlineto{\pgfqpoint{0.962055in}{2.594040in}}%
\pgfpathlineto{\pgfqpoint{0.963783in}{2.480847in}}%
\pgfpathlineto{\pgfqpoint{0.964647in}{2.514786in}}%
\pgfpathlineto{\pgfqpoint{0.965512in}{2.468581in}}%
\pgfpathlineto{\pgfqpoint{0.966375in}{2.564712in}}%
\pgfpathlineto{\pgfqpoint{0.967241in}{2.549279in}}%
\pgfpathlineto{\pgfqpoint{0.968105in}{2.561577in}}%
\pgfpathlineto{\pgfqpoint{0.968970in}{2.459267in}}%
\pgfpathlineto{\pgfqpoint{0.969836in}{2.547373in}}%
\pgfpathlineto{\pgfqpoint{0.970701in}{2.468858in}}%
\pgfpathlineto{\pgfqpoint{0.971567in}{2.494743in}}%
\pgfpathlineto{\pgfqpoint{0.972432in}{2.494681in}}%
\pgfpathlineto{\pgfqpoint{0.973297in}{2.508392in}}%
\pgfpathlineto{\pgfqpoint{0.974162in}{2.473899in}}%
\pgfpathlineto{\pgfqpoint{0.975027in}{2.506517in}}%
\pgfpathlineto{\pgfqpoint{0.975889in}{2.595454in}}%
\pgfpathlineto{\pgfqpoint{0.977616in}{2.480724in}}%
\pgfpathlineto{\pgfqpoint{0.978481in}{2.523517in}}%
\pgfpathlineto{\pgfqpoint{0.979346in}{2.518476in}}%
\pgfpathlineto{\pgfqpoint{0.980211in}{2.480539in}}%
\pgfpathlineto{\pgfqpoint{0.981076in}{2.562929in}}%
\pgfpathlineto{\pgfqpoint{0.981941in}{2.547404in}}%
\pgfpathlineto{\pgfqpoint{0.982806in}{2.574980in}}%
\pgfpathlineto{\pgfqpoint{0.984535in}{2.477404in}}%
\pgfpathlineto{\pgfqpoint{0.985400in}{2.521980in}}%
\pgfpathlineto{\pgfqpoint{0.986264in}{2.462156in}}%
\pgfpathlineto{\pgfqpoint{0.987127in}{2.527022in}}%
\pgfpathlineto{\pgfqpoint{0.987991in}{2.429569in}}%
\pgfpathlineto{\pgfqpoint{0.988856in}{2.514879in}}%
\pgfpathlineto{\pgfqpoint{0.989722in}{2.511158in}}%
\pgfpathlineto{\pgfqpoint{0.990585in}{2.512942in}}%
\pgfpathlineto{\pgfqpoint{0.991449in}{2.500860in}}%
\pgfpathlineto{\pgfqpoint{0.992314in}{2.526714in}}%
\pgfpathlineto{\pgfqpoint{0.993178in}{2.514602in}}%
\pgfpathlineto{\pgfqpoint{0.994043in}{2.534309in}}%
\pgfpathlineto{\pgfqpoint{0.994907in}{2.517062in}}%
\pgfpathlineto{\pgfqpoint{0.995771in}{2.468919in}}%
\pgfpathlineto{\pgfqpoint{0.997501in}{2.512880in}}%
\pgfpathlineto{\pgfqpoint{0.998367in}{2.528467in}}%
\pgfpathlineto{\pgfqpoint{1.000096in}{2.502920in}}%
\pgfpathlineto{\pgfqpoint{1.000961in}{2.560530in}}%
\pgfpathlineto{\pgfqpoint{1.001826in}{2.543684in}}%
\pgfpathlineto{\pgfqpoint{1.002691in}{2.549464in}}%
\pgfpathlineto{\pgfqpoint{1.004419in}{2.516323in}}%
\pgfpathlineto{\pgfqpoint{1.005284in}{2.516508in}}%
\pgfpathlineto{\pgfqpoint{1.006149in}{2.506548in}}%
\pgfpathlineto{\pgfqpoint{1.007014in}{2.552538in}}%
\pgfpathlineto{\pgfqpoint{1.007877in}{2.514355in}}%
\pgfpathlineto{\pgfqpoint{1.009607in}{2.556841in}}%
\pgfpathlineto{\pgfqpoint{1.012203in}{2.436700in}}%
\pgfpathlineto{\pgfqpoint{1.013933in}{2.590597in}}%
\pgfpathlineto{\pgfqpoint{1.014798in}{2.534278in}}%
\pgfpathlineto{\pgfqpoint{1.015662in}{2.474453in}}%
\pgfpathlineto{\pgfqpoint{1.017391in}{2.553277in}}%
\pgfpathlineto{\pgfqpoint{1.018255in}{2.585556in}}%
\pgfpathlineto{\pgfqpoint{1.019120in}{2.565328in}}%
\pgfpathlineto{\pgfqpoint{1.019986in}{2.489671in}}%
\pgfpathlineto{\pgfqpoint{1.020851in}{2.530711in}}%
\pgfpathlineto{\pgfqpoint{1.021716in}{2.492344in}}%
\pgfpathlineto{\pgfqpoint{1.022581in}{2.496311in}}%
\pgfpathlineto{\pgfqpoint{1.023445in}{2.487487in}}%
\pgfpathlineto{\pgfqpoint{1.025173in}{2.552353in}}%
\pgfpathlineto{\pgfqpoint{1.026903in}{2.506917in}}%
\pgfpathlineto{\pgfqpoint{1.027767in}{2.494620in}}%
\pgfpathlineto{\pgfqpoint{1.028629in}{2.617773in}}%
\pgfpathlineto{\pgfqpoint{1.030357in}{2.505072in}}%
\pgfpathlineto{\pgfqpoint{1.031221in}{2.541410in}}%
\pgfpathlineto{\pgfqpoint{1.032087in}{2.529636in}}%
\pgfpathlineto{\pgfqpoint{1.033816in}{2.555060in}}%
\pgfpathlineto{\pgfqpoint{1.035545in}{2.463631in}}%
\pgfpathlineto{\pgfqpoint{1.037275in}{2.479372in}}%
\pgfpathlineto{\pgfqpoint{1.038139in}{2.472239in}}%
\pgfpathlineto{\pgfqpoint{1.040734in}{2.553767in}}%
\pgfpathlineto{\pgfqpoint{1.041600in}{2.555858in}}%
\pgfpathlineto{\pgfqpoint{1.042466in}{2.565143in}}%
\pgfpathlineto{\pgfqpoint{1.043331in}{2.563360in}}%
\pgfpathlineto{\pgfqpoint{1.044195in}{2.552661in}}%
\pgfpathlineto{\pgfqpoint{1.045060in}{2.594777in}}%
\pgfpathlineto{\pgfqpoint{1.045927in}{2.473561in}}%
\pgfpathlineto{\pgfqpoint{1.046792in}{2.524993in}}%
\pgfpathlineto{\pgfqpoint{1.047657in}{2.520259in}}%
\pgfpathlineto{\pgfqpoint{1.048523in}{2.540395in}}%
\pgfpathlineto{\pgfqpoint{1.050256in}{2.485827in}}%
\pgfpathlineto{\pgfqpoint{1.051122in}{2.507777in}}%
\pgfpathlineto{\pgfqpoint{1.051987in}{2.445187in}}%
\pgfpathlineto{\pgfqpoint{1.054584in}{2.557025in}}%
\pgfpathlineto{\pgfqpoint{1.055449in}{2.498707in}}%
\pgfpathlineto{\pgfqpoint{1.057179in}{2.536921in}}%
\pgfpathlineto{\pgfqpoint{1.059773in}{2.497386in}}%
\pgfpathlineto{\pgfqpoint{1.060637in}{2.446323in}}%
\pgfpathlineto{\pgfqpoint{1.061503in}{2.553952in}}%
\pgfpathlineto{\pgfqpoint{1.062369in}{2.518414in}}%
\pgfpathlineto{\pgfqpoint{1.063234in}{2.519643in}}%
\pgfpathlineto{\pgfqpoint{1.064099in}{2.507900in}}%
\pgfpathlineto{\pgfqpoint{1.064962in}{2.556595in}}%
\pgfpathlineto{\pgfqpoint{1.068422in}{2.469410in}}%
\pgfpathlineto{\pgfqpoint{1.069287in}{2.471808in}}%
\pgfpathlineto{\pgfqpoint{1.071017in}{2.570183in}}%
\pgfpathlineto{\pgfqpoint{1.073608in}{2.452563in}}%
\pgfpathlineto{\pgfqpoint{1.074470in}{2.531263in}}%
\pgfpathlineto{\pgfqpoint{1.075335in}{2.509252in}}%
\pgfpathlineto{\pgfqpoint{1.076199in}{2.456499in}}%
\pgfpathlineto{\pgfqpoint{1.077064in}{2.521734in}}%
\pgfpathlineto{\pgfqpoint{1.077927in}{2.519766in}}%
\pgfpathlineto{\pgfqpoint{1.078792in}{2.559301in}}%
\pgfpathlineto{\pgfqpoint{1.080521in}{2.451149in}}%
\pgfpathlineto{\pgfqpoint{1.081384in}{2.500706in}}%
\pgfpathlineto{\pgfqpoint{1.082248in}{2.474759in}}%
\pgfpathlineto{\pgfqpoint{1.083981in}{2.563789in}}%
\pgfpathlineto{\pgfqpoint{1.084846in}{2.555612in}}%
\pgfpathlineto{\pgfqpoint{1.085711in}{2.565880in}}%
\pgfpathlineto{\pgfqpoint{1.086576in}{2.458957in}}%
\pgfpathlineto{\pgfqpoint{1.088306in}{2.596252in}}%
\pgfpathlineto{\pgfqpoint{1.090037in}{2.520626in}}%
\pgfpathlineto{\pgfqpoint{1.090903in}{2.523885in}}%
\pgfpathlineto{\pgfqpoint{1.092632in}{2.492160in}}%
\pgfpathlineto{\pgfqpoint{1.093498in}{2.452625in}}%
\pgfpathlineto{\pgfqpoint{1.094362in}{2.517121in}}%
\pgfpathlineto{\pgfqpoint{1.095225in}{2.508760in}}%
\pgfpathlineto{\pgfqpoint{1.096090in}{2.539133in}}%
\pgfpathlineto{\pgfqpoint{1.097822in}{2.469040in}}%
\pgfpathlineto{\pgfqpoint{1.099551in}{2.585125in}}%
\pgfpathlineto{\pgfqpoint{1.101281in}{2.538273in}}%
\pgfpathlineto{\pgfqpoint{1.102145in}{2.544728in}}%
\pgfpathlineto{\pgfqpoint{1.103871in}{2.487118in}}%
\pgfpathlineto{\pgfqpoint{1.104736in}{2.521119in}}%
\pgfpathlineto{\pgfqpoint{1.106466in}{2.483182in}}%
\pgfpathlineto{\pgfqpoint{1.107328in}{2.523271in}}%
\pgfpathlineto{\pgfqpoint{1.108192in}{2.516446in}}%
\pgfpathlineto{\pgfqpoint{1.109055in}{2.521611in}}%
\pgfpathlineto{\pgfqpoint{1.109918in}{2.573933in}}%
\pgfpathlineto{\pgfqpoint{1.110782in}{2.491913in}}%
\pgfpathlineto{\pgfqpoint{1.111647in}{2.517306in}}%
\pgfpathlineto{\pgfqpoint{1.113375in}{2.447583in}}%
\pgfpathlineto{\pgfqpoint{1.115967in}{2.557518in}}%
\pgfpathlineto{\pgfqpoint{1.117696in}{2.504734in}}%
\pgfpathlineto{\pgfqpoint{1.118561in}{2.575780in}}%
\pgfpathlineto{\pgfqpoint{1.120289in}{2.467321in}}%
\pgfpathlineto{\pgfqpoint{1.122016in}{2.537629in}}%
\pgfpathlineto{\pgfqpoint{1.122881in}{2.487610in}}%
\pgfpathlineto{\pgfqpoint{1.123746in}{2.492898in}}%
\pgfpathlineto{\pgfqpoint{1.124612in}{2.501229in}}%
\pgfpathlineto{\pgfqpoint{1.126342in}{2.547681in}}%
\pgfpathlineto{\pgfqpoint{1.127208in}{2.494066in}}%
\pgfpathlineto{\pgfqpoint{1.128072in}{2.512757in}}%
\pgfpathlineto{\pgfqpoint{1.128936in}{2.510912in}}%
\pgfpathlineto{\pgfqpoint{1.129800in}{2.459880in}}%
\pgfpathlineto{\pgfqpoint{1.130665in}{2.506055in}}%
\pgfpathlineto{\pgfqpoint{1.131530in}{2.488501in}}%
\pgfpathlineto{\pgfqpoint{1.133261in}{2.538396in}}%
\pgfpathlineto{\pgfqpoint{1.134126in}{2.495264in}}%
\pgfpathlineto{\pgfqpoint{1.134991in}{2.518045in}}%
\pgfpathlineto{\pgfqpoint{1.135856in}{2.459388in}}%
\pgfpathlineto{\pgfqpoint{1.136722in}{2.518014in}}%
\pgfpathlineto{\pgfqpoint{1.137586in}{2.437869in}}%
\pgfpathlineto{\pgfqpoint{1.138451in}{2.535137in}}%
\pgfpathlineto{\pgfqpoint{1.139316in}{2.480478in}}%
\pgfpathlineto{\pgfqpoint{1.140181in}{2.499415in}}%
\pgfpathlineto{\pgfqpoint{1.141044in}{2.486841in}}%
\pgfpathlineto{\pgfqpoint{1.142776in}{2.569507in}}%
\pgfpathlineto{\pgfqpoint{1.144506in}{2.463385in}}%
\pgfpathlineto{\pgfqpoint{1.145371in}{2.495295in}}%
\pgfpathlineto{\pgfqpoint{1.146235in}{2.469625in}}%
\pgfpathlineto{\pgfqpoint{1.147966in}{2.566434in}}%
\pgfpathlineto{\pgfqpoint{1.148829in}{2.487364in}}%
\pgfpathlineto{\pgfqpoint{1.149692in}{2.508085in}}%
\pgfpathlineto{\pgfqpoint{1.150558in}{2.502489in}}%
\pgfpathlineto{\pgfqpoint{1.152290in}{2.558932in}}%
\pgfpathlineto{\pgfqpoint{1.154885in}{2.468427in}}%
\pgfpathlineto{\pgfqpoint{1.155749in}{2.533508in}}%
\pgfpathlineto{\pgfqpoint{1.156614in}{2.523517in}}%
\pgfpathlineto{\pgfqpoint{1.158343in}{2.492590in}}%
\pgfpathlineto{\pgfqpoint{1.159208in}{2.547312in}}%
\pgfpathlineto{\pgfqpoint{1.160073in}{2.530065in}}%
\pgfpathlineto{\pgfqpoint{1.160937in}{2.482384in}}%
\pgfpathlineto{\pgfqpoint{1.161802in}{2.611931in}}%
\pgfpathlineto{\pgfqpoint{1.162667in}{2.610702in}}%
\pgfpathlineto{\pgfqpoint{1.164394in}{2.600065in}}%
\pgfpathlineto{\pgfqpoint{1.165259in}{2.497694in}}%
\pgfpathlineto{\pgfqpoint{1.166124in}{2.528190in}}%
\pgfpathlineto{\pgfqpoint{1.166990in}{2.603877in}}%
\pgfpathlineto{\pgfqpoint{1.167854in}{2.583219in}}%
\pgfpathlineto{\pgfqpoint{1.171315in}{2.496095in}}%
\pgfpathlineto{\pgfqpoint{1.172180in}{2.515431in}}%
\pgfpathlineto{\pgfqpoint{1.173045in}{2.574488in}}%
\pgfpathlineto{\pgfqpoint{1.173912in}{2.559178in}}%
\pgfpathlineto{\pgfqpoint{1.174778in}{2.559178in}}%
\pgfpathlineto{\pgfqpoint{1.177374in}{2.461664in}}%
\pgfpathlineto{\pgfqpoint{1.178239in}{2.509806in}}%
\pgfpathlineto{\pgfqpoint{1.179103in}{2.441097in}}%
\pgfpathlineto{\pgfqpoint{1.180834in}{2.542886in}}%
\pgfpathlineto{\pgfqpoint{1.181699in}{2.588876in}}%
\pgfpathlineto{\pgfqpoint{1.183430in}{2.523271in}}%
\pgfpathlineto{\pgfqpoint{1.184295in}{2.576702in}}%
\pgfpathlineto{\pgfqpoint{1.186026in}{2.486135in}}%
\pgfpathlineto{\pgfqpoint{1.186891in}{2.590782in}}%
\pgfpathlineto{\pgfqpoint{1.187758in}{2.588937in}}%
\pgfpathlineto{\pgfqpoint{1.189487in}{2.513496in}}%
\pgfpathlineto{\pgfqpoint{1.190352in}{2.593117in}}%
\pgfpathlineto{\pgfqpoint{1.191218in}{2.492744in}}%
\pgfpathlineto{\pgfqpoint{1.193813in}{2.581743in}}%
\pgfpathlineto{\pgfqpoint{1.194678in}{2.539442in}}%
\pgfpathlineto{\pgfqpoint{1.195541in}{2.600681in}}%
\pgfpathlineto{\pgfqpoint{1.196406in}{2.472362in}}%
\pgfpathlineto{\pgfqpoint{1.197272in}{2.600927in}}%
\pgfpathlineto{\pgfqpoint{1.199001in}{2.544299in}}%
\pgfpathlineto{\pgfqpoint{1.199865in}{2.545898in}}%
\pgfpathlineto{\pgfqpoint{1.200729in}{2.532772in}}%
\pgfpathlineto{\pgfqpoint{1.201595in}{2.496280in}}%
\pgfpathlineto{\pgfqpoint{1.203324in}{2.535661in}}%
\pgfpathlineto{\pgfqpoint{1.204189in}{2.471441in}}%
\pgfpathlineto{\pgfqpoint{1.205054in}{2.528069in}}%
\pgfpathlineto{\pgfqpoint{1.205919in}{2.463018in}}%
\pgfpathlineto{\pgfqpoint{1.206785in}{2.517370in}}%
\pgfpathlineto{\pgfqpoint{1.207651in}{2.503505in}}%
\pgfpathlineto{\pgfqpoint{1.208516in}{2.554752in}}%
\pgfpathlineto{\pgfqpoint{1.210246in}{2.513219in}}%
\pgfpathlineto{\pgfqpoint{1.211111in}{2.493206in}}%
\pgfpathlineto{\pgfqpoint{1.212842in}{2.554044in}}%
\pgfpathlineto{\pgfqpoint{1.214572in}{2.510114in}}%
\pgfpathlineto{\pgfqpoint{1.216303in}{2.592011in}}%
\pgfpathlineto{\pgfqpoint{1.217169in}{2.535445in}}%
\pgfpathlineto{\pgfqpoint{1.218035in}{2.588322in}}%
\pgfpathlineto{\pgfqpoint{1.218902in}{2.547558in}}%
\pgfpathlineto{\pgfqpoint{1.219768in}{2.608888in}}%
\pgfpathlineto{\pgfqpoint{1.220634in}{2.502674in}}%
\pgfpathlineto{\pgfqpoint{1.221500in}{2.524010in}}%
\pgfpathlineto{\pgfqpoint{1.222366in}{2.494281in}}%
\pgfpathlineto{\pgfqpoint{1.224098in}{2.561330in}}%
\pgfpathlineto{\pgfqpoint{1.224965in}{2.537413in}}%
\pgfpathlineto{\pgfqpoint{1.225831in}{2.549464in}}%
\pgfpathlineto{\pgfqpoint{1.226697in}{2.474515in}}%
\pgfpathlineto{\pgfqpoint{1.227562in}{2.481401in}}%
\pgfpathlineto{\pgfqpoint{1.228427in}{2.565205in}}%
\pgfpathlineto{\pgfqpoint{1.229292in}{2.541102in}}%
\pgfpathlineto{\pgfqpoint{1.230156in}{2.479864in}}%
\pgfpathlineto{\pgfqpoint{1.231020in}{2.523825in}}%
\pgfpathlineto{\pgfqpoint{1.231885in}{2.497817in}}%
\pgfpathlineto{\pgfqpoint{1.233617in}{2.549772in}}%
\pgfpathlineto{\pgfqpoint{1.235347in}{2.460004in}}%
\pgfpathlineto{\pgfqpoint{1.237076in}{2.572951in}}%
\pgfpathlineto{\pgfqpoint{1.237942in}{2.579098in}}%
\pgfpathlineto{\pgfqpoint{1.238807in}{2.480355in}}%
\pgfpathlineto{\pgfqpoint{1.240537in}{2.557580in}}%
\pgfpathlineto{\pgfqpoint{1.241402in}{2.549064in}}%
\pgfpathlineto{\pgfqpoint{1.243133in}{2.489086in}}%
\pgfpathlineto{\pgfqpoint{1.243999in}{2.544115in}}%
\pgfpathlineto{\pgfqpoint{1.244864in}{2.501137in}}%
\pgfpathlineto{\pgfqpoint{1.245728in}{2.508239in}}%
\pgfpathlineto{\pgfqpoint{1.246593in}{2.553092in}}%
\pgfpathlineto{\pgfqpoint{1.247458in}{2.468981in}}%
\pgfpathlineto{\pgfqpoint{1.248323in}{2.473376in}}%
\pgfpathlineto{\pgfqpoint{1.249187in}{2.476358in}}%
\pgfpathlineto{\pgfqpoint{1.250053in}{2.558686in}}%
\pgfpathlineto{\pgfqpoint{1.250916in}{2.495326in}}%
\pgfpathlineto{\pgfqpoint{1.251781in}{2.508760in}}%
\pgfpathlineto{\pgfqpoint{1.252647in}{2.540179in}}%
\pgfpathlineto{\pgfqpoint{1.253512in}{2.526776in}}%
\pgfpathlineto{\pgfqpoint{1.254377in}{2.550878in}}%
\pgfpathlineto{\pgfqpoint{1.255243in}{2.489517in}}%
\pgfpathlineto{\pgfqpoint{1.256109in}{2.575103in}}%
\pgfpathlineto{\pgfqpoint{1.257840in}{2.519797in}}%
\pgfpathlineto{\pgfqpoint{1.258706in}{2.577746in}}%
\pgfpathlineto{\pgfqpoint{1.259571in}{2.474944in}}%
\pgfpathlineto{\pgfqpoint{1.260436in}{2.496095in}}%
\pgfpathlineto{\pgfqpoint{1.262165in}{2.559178in}}%
\pgfpathlineto{\pgfqpoint{1.263030in}{2.549033in}}%
\pgfpathlineto{\pgfqpoint{1.265625in}{2.428707in}}%
\pgfpathlineto{\pgfqpoint{1.267353in}{2.603416in}}%
\pgfpathlineto{\pgfqpoint{1.268216in}{2.590351in}}%
\pgfpathlineto{\pgfqpoint{1.269081in}{2.517983in}}%
\pgfpathlineto{\pgfqpoint{1.269946in}{2.536767in}}%
\pgfpathlineto{\pgfqpoint{1.270811in}{2.558809in}}%
\pgfpathlineto{\pgfqpoint{1.274272in}{2.496280in}}%
\pgfpathlineto{\pgfqpoint{1.275137in}{2.497694in}}%
\pgfpathlineto{\pgfqpoint{1.276003in}{2.529113in}}%
\pgfpathlineto{\pgfqpoint{1.276869in}{2.496341in}}%
\pgfpathlineto{\pgfqpoint{1.278600in}{2.620108in}}%
\pgfpathlineto{\pgfqpoint{1.280331in}{2.511035in}}%
\pgfpathlineto{\pgfqpoint{1.281196in}{2.514786in}}%
\pgfpathlineto{\pgfqpoint{1.282062in}{2.492098in}}%
\pgfpathlineto{\pgfqpoint{1.283793in}{2.574734in}}%
\pgfpathlineto{\pgfqpoint{1.284658in}{2.573320in}}%
\pgfpathlineto{\pgfqpoint{1.285521in}{2.592871in}}%
\pgfpathlineto{\pgfqpoint{1.287250in}{2.515033in}}%
\pgfpathlineto{\pgfqpoint{1.288980in}{2.538611in}}%
\pgfpathlineto{\pgfqpoint{1.290711in}{2.564404in}}%
\pgfpathlineto{\pgfqpoint{1.292440in}{2.523333in}}%
\pgfpathlineto{\pgfqpoint{1.293306in}{2.579775in}}%
\pgfpathlineto{\pgfqpoint{1.295037in}{2.525362in}}%
\pgfpathlineto{\pgfqpoint{1.295904in}{2.533354in}}%
\pgfpathlineto{\pgfqpoint{1.296768in}{2.555612in}}%
\pgfpathlineto{\pgfqpoint{1.297633in}{2.509868in}}%
\pgfpathlineto{\pgfqpoint{1.298498in}{2.390250in}}%
\pgfpathlineto{\pgfqpoint{1.300229in}{2.534830in}}%
\pgfpathlineto{\pgfqpoint{1.301092in}{2.457851in}}%
\pgfpathlineto{\pgfqpoint{1.301958in}{2.461356in}}%
\pgfpathlineto{\pgfqpoint{1.302823in}{2.454562in}}%
\pgfpathlineto{\pgfqpoint{1.303687in}{2.514355in}}%
\pgfpathlineto{\pgfqpoint{1.304552in}{2.512880in}}%
\pgfpathlineto{\pgfqpoint{1.305417in}{2.529788in}}%
\pgfpathlineto{\pgfqpoint{1.306282in}{2.518168in}}%
\pgfpathlineto{\pgfqpoint{1.307147in}{2.530096in}}%
\pgfpathlineto{\pgfqpoint{1.308013in}{2.465261in}}%
\pgfpathlineto{\pgfqpoint{1.308879in}{2.566064in}}%
\pgfpathlineto{\pgfqpoint{1.309745in}{2.500583in}}%
\pgfpathlineto{\pgfqpoint{1.310610in}{2.543930in}}%
\pgfpathlineto{\pgfqpoint{1.311476in}{2.524192in}}%
\pgfpathlineto{\pgfqpoint{1.312342in}{2.533016in}}%
\pgfpathlineto{\pgfqpoint{1.313208in}{2.486872in}}%
\pgfpathlineto{\pgfqpoint{1.314074in}{2.542945in}}%
\pgfpathlineto{\pgfqpoint{1.314939in}{2.530188in}}%
\pgfpathlineto{\pgfqpoint{1.315803in}{2.527266in}}%
\pgfpathlineto{\pgfqpoint{1.316665in}{2.527943in}}%
\pgfpathlineto{\pgfqpoint{1.317529in}{2.505655in}}%
\pgfpathlineto{\pgfqpoint{1.318395in}{2.424220in}}%
\pgfpathlineto{\pgfqpoint{1.319260in}{2.523086in}}%
\pgfpathlineto{\pgfqpoint{1.320124in}{2.516015in}}%
\pgfpathlineto{\pgfqpoint{1.320989in}{2.513863in}}%
\pgfpathlineto{\pgfqpoint{1.321854in}{2.544851in}}%
\pgfpathlineto{\pgfqpoint{1.322720in}{2.466213in}}%
\pgfpathlineto{\pgfqpoint{1.323583in}{2.516139in}}%
\pgfpathlineto{\pgfqpoint{1.324448in}{2.476450in}}%
\pgfpathlineto{\pgfqpoint{1.325313in}{2.547004in}}%
\pgfpathlineto{\pgfqpoint{1.327043in}{2.459234in}}%
\pgfpathlineto{\pgfqpoint{1.329636in}{2.539563in}}%
\pgfpathlineto{\pgfqpoint{1.330500in}{2.495541in}}%
\pgfpathlineto{\pgfqpoint{1.331365in}{2.500213in}}%
\pgfpathlineto{\pgfqpoint{1.332230in}{2.566986in}}%
\pgfpathlineto{\pgfqpoint{1.333954in}{2.479862in}}%
\pgfpathlineto{\pgfqpoint{1.336550in}{2.549924in}}%
\pgfpathlineto{\pgfqpoint{1.337413in}{2.541778in}}%
\pgfpathlineto{\pgfqpoint{1.338278in}{2.584448in}}%
\pgfpathlineto{\pgfqpoint{1.339144in}{2.495541in}}%
\pgfpathlineto{\pgfqpoint{1.340007in}{2.551984in}}%
\pgfpathlineto{\pgfqpoint{1.341734in}{2.497324in}}%
\pgfpathlineto{\pgfqpoint{1.342599in}{2.526283in}}%
\pgfpathlineto{\pgfqpoint{1.343462in}{2.516385in}}%
\pgfpathlineto{\pgfqpoint{1.344329in}{2.526960in}}%
\pgfpathlineto{\pgfqpoint{1.345195in}{2.458097in}}%
\pgfpathlineto{\pgfqpoint{1.346925in}{2.544238in}}%
\pgfpathlineto{\pgfqpoint{1.347786in}{2.547927in}}%
\pgfpathlineto{\pgfqpoint{1.348652in}{2.473838in}}%
\pgfpathlineto{\pgfqpoint{1.349515in}{2.560900in}}%
\pgfpathlineto{\pgfqpoint{1.350380in}{2.489024in}}%
\pgfpathlineto{\pgfqpoint{1.351245in}{2.504395in}}%
\pgfpathlineto{\pgfqpoint{1.352107in}{2.505994in}}%
\pgfpathlineto{\pgfqpoint{1.353837in}{2.451088in}}%
\pgfpathlineto{\pgfqpoint{1.354702in}{2.470518in}}%
\pgfpathlineto{\pgfqpoint{1.355566in}{2.519643in}}%
\pgfpathlineto{\pgfqpoint{1.356431in}{2.515279in}}%
\pgfpathlineto{\pgfqpoint{1.357296in}{2.496218in}}%
\pgfpathlineto{\pgfqpoint{1.358162in}{2.511282in}}%
\pgfpathlineto{\pgfqpoint{1.359028in}{2.508331in}}%
\pgfpathlineto{\pgfqpoint{1.359893in}{2.448322in}}%
\pgfpathlineto{\pgfqpoint{1.360758in}{2.508146in}}%
\pgfpathlineto{\pgfqpoint{1.361622in}{2.501445in}}%
\pgfpathlineto{\pgfqpoint{1.362488in}{2.516169in}}%
\pgfpathlineto{\pgfqpoint{1.363353in}{2.571167in}}%
\pgfpathlineto{\pgfqpoint{1.364217in}{2.531817in}}%
\pgfpathlineto{\pgfqpoint{1.365947in}{2.565695in}}%
\pgfpathlineto{\pgfqpoint{1.367678in}{2.532800in}}%
\pgfpathlineto{\pgfqpoint{1.368543in}{2.521796in}}%
\pgfpathlineto{\pgfqpoint{1.369408in}{2.476912in}}%
\pgfpathlineto{\pgfqpoint{1.370273in}{2.503841in}}%
\pgfpathlineto{\pgfqpoint{1.371138in}{2.437562in}}%
\pgfpathlineto{\pgfqpoint{1.372866in}{2.504826in}}%
\pgfpathlineto{\pgfqpoint{1.373729in}{2.458282in}}%
\pgfpathlineto{\pgfqpoint{1.374594in}{2.502181in}}%
\pgfpathlineto{\pgfqpoint{1.375459in}{2.433564in}}%
\pgfpathlineto{\pgfqpoint{1.376322in}{2.510974in}}%
\pgfpathlineto{\pgfqpoint{1.377187in}{2.485766in}}%
\pgfpathlineto{\pgfqpoint{1.378052in}{2.514355in}}%
\pgfpathlineto{\pgfqpoint{1.379781in}{2.456684in}}%
\pgfpathlineto{\pgfqpoint{1.380646in}{2.523456in}}%
\pgfpathlineto{\pgfqpoint{1.381509in}{2.473838in}}%
\pgfpathlineto{\pgfqpoint{1.382373in}{2.523517in}}%
\pgfpathlineto{\pgfqpoint{1.383238in}{2.497263in}}%
\pgfpathlineto{\pgfqpoint{1.384104in}{2.500644in}}%
\pgfpathlineto{\pgfqpoint{1.384968in}{2.542270in}}%
\pgfpathlineto{\pgfqpoint{1.386696in}{2.466951in}}%
\pgfpathlineto{\pgfqpoint{1.387562in}{2.531941in}}%
\pgfpathlineto{\pgfqpoint{1.388427in}{2.443986in}}%
\pgfpathlineto{\pgfqpoint{1.389291in}{2.557395in}}%
\pgfpathlineto{\pgfqpoint{1.391021in}{2.413398in}}%
\pgfpathlineto{\pgfqpoint{1.391886in}{2.555365in}}%
\pgfpathlineto{\pgfqpoint{1.393613in}{2.480447in}}%
\pgfpathlineto{\pgfqpoint{1.394476in}{2.490623in}}%
\pgfpathlineto{\pgfqpoint{1.395342in}{2.549095in}}%
\pgfpathlineto{\pgfqpoint{1.396208in}{2.440820in}}%
\pgfpathlineto{\pgfqpoint{1.397938in}{2.504549in}}%
\pgfpathlineto{\pgfqpoint{1.398803in}{2.506178in}}%
\pgfpathlineto{\pgfqpoint{1.399668in}{2.575840in}}%
\pgfpathlineto{\pgfqpoint{1.400534in}{2.484996in}}%
\pgfpathlineto{\pgfqpoint{1.401400in}{2.519520in}}%
\pgfpathlineto{\pgfqpoint{1.402265in}{2.492529in}}%
\pgfpathlineto{\pgfqpoint{1.403997in}{2.552292in}}%
\pgfpathlineto{\pgfqpoint{1.404863in}{2.549524in}}%
\pgfpathlineto{\pgfqpoint{1.405729in}{2.520934in}}%
\pgfpathlineto{\pgfqpoint{1.406594in}{2.584386in}}%
\pgfpathlineto{\pgfqpoint{1.408324in}{2.495911in}}%
\pgfpathlineto{\pgfqpoint{1.409190in}{2.571229in}}%
\pgfpathlineto{\pgfqpoint{1.410055in}{2.483798in}}%
\pgfpathlineto{\pgfqpoint{1.410921in}{2.584755in}}%
\pgfpathlineto{\pgfqpoint{1.411785in}{2.509437in}}%
\pgfpathlineto{\pgfqpoint{1.412651in}{2.555673in}}%
\pgfpathlineto{\pgfqpoint{1.413515in}{2.524808in}}%
\pgfpathlineto{\pgfqpoint{1.414379in}{2.569569in}}%
\pgfpathlineto{\pgfqpoint{1.416108in}{2.494127in}}%
\pgfpathlineto{\pgfqpoint{1.417838in}{2.592565in}}%
\pgfpathlineto{\pgfqpoint{1.420430in}{2.485519in}}%
\pgfpathlineto{\pgfqpoint{1.422159in}{2.513894in}}%
\pgfpathlineto{\pgfqpoint{1.423024in}{2.522840in}}%
\pgfpathlineto{\pgfqpoint{1.423884in}{2.509622in}}%
\pgfpathlineto{\pgfqpoint{1.424749in}{2.541408in}}%
\pgfpathlineto{\pgfqpoint{1.425614in}{2.494250in}}%
\pgfpathlineto{\pgfqpoint{1.428206in}{2.572828in}}%
\pgfpathlineto{\pgfqpoint{1.429937in}{2.484290in}}%
\pgfpathlineto{\pgfqpoint{1.430801in}{2.498984in}}%
\pgfpathlineto{\pgfqpoint{1.431667in}{2.473068in}}%
\pgfpathlineto{\pgfqpoint{1.433398in}{2.559363in}}%
\pgfpathlineto{\pgfqpoint{1.435130in}{2.445738in}}%
\pgfpathlineto{\pgfqpoint{1.436861in}{2.532923in}}%
\pgfpathlineto{\pgfqpoint{1.437726in}{2.485827in}}%
\pgfpathlineto{\pgfqpoint{1.438592in}{2.524531in}}%
\pgfpathlineto{\pgfqpoint{1.439455in}{2.508944in}}%
\pgfpathlineto{\pgfqpoint{1.441185in}{2.562190in}}%
\pgfpathlineto{\pgfqpoint{1.442915in}{2.521919in}}%
\pgfpathlineto{\pgfqpoint{1.443778in}{2.534031in}}%
\pgfpathlineto{\pgfqpoint{1.444643in}{2.471993in}}%
\pgfpathlineto{\pgfqpoint{1.445508in}{2.539258in}}%
\pgfpathlineto{\pgfqpoint{1.446371in}{2.518291in}}%
\pgfpathlineto{\pgfqpoint{1.447237in}{2.535815in}}%
\pgfpathlineto{\pgfqpoint{1.448967in}{2.472362in}}%
\pgfpathlineto{\pgfqpoint{1.449831in}{2.456991in}}%
\pgfpathlineto{\pgfqpoint{1.452425in}{2.482754in}}%
\pgfpathlineto{\pgfqpoint{1.454154in}{2.525054in}}%
\pgfpathlineto{\pgfqpoint{1.455884in}{2.479126in}}%
\pgfpathlineto{\pgfqpoint{1.457614in}{2.545898in}}%
\pgfpathlineto{\pgfqpoint{1.460209in}{2.482938in}}%
\pgfpathlineto{\pgfqpoint{1.461074in}{2.455208in}}%
\pgfpathlineto{\pgfqpoint{1.461941in}{2.465599in}}%
\pgfpathlineto{\pgfqpoint{1.462806in}{2.448076in}}%
\pgfpathlineto{\pgfqpoint{1.463672in}{2.556658in}}%
\pgfpathlineto{\pgfqpoint{1.465402in}{2.512205in}}%
\pgfpathlineto{\pgfqpoint{1.466268in}{2.491300in}}%
\pgfpathlineto{\pgfqpoint{1.467134in}{2.506517in}}%
\pgfpathlineto{\pgfqpoint{1.467999in}{2.468796in}}%
\pgfpathlineto{\pgfqpoint{1.468864in}{2.540548in}}%
\pgfpathlineto{\pgfqpoint{1.469729in}{2.523671in}}%
\pgfpathlineto{\pgfqpoint{1.471459in}{2.480049in}}%
\pgfpathlineto{\pgfqpoint{1.472326in}{2.510853in}}%
\pgfpathlineto{\pgfqpoint{1.473192in}{2.456193in}}%
\pgfpathlineto{\pgfqpoint{1.474056in}{2.485889in}}%
\pgfpathlineto{\pgfqpoint{1.474922in}{2.474269in}}%
\pgfpathlineto{\pgfqpoint{1.475788in}{2.514111in}}%
\pgfpathlineto{\pgfqpoint{1.476654in}{2.474084in}}%
\pgfpathlineto{\pgfqpoint{1.479253in}{2.593948in}}%
\pgfpathlineto{\pgfqpoint{1.480117in}{2.539442in}}%
\pgfpathlineto{\pgfqpoint{1.480982in}{2.580453in}}%
\pgfpathlineto{\pgfqpoint{1.482711in}{2.496465in}}%
\pgfpathlineto{\pgfqpoint{1.483575in}{2.508177in}}%
\pgfpathlineto{\pgfqpoint{1.484441in}{2.472362in}}%
\pgfpathlineto{\pgfqpoint{1.485307in}{2.481217in}}%
\pgfpathlineto{\pgfqpoint{1.486174in}{2.477835in}}%
\pgfpathlineto{\pgfqpoint{1.487040in}{2.458898in}}%
\pgfpathlineto{\pgfqpoint{1.487905in}{2.484414in}}%
\pgfpathlineto{\pgfqpoint{1.488770in}{2.463539in}}%
\pgfpathlineto{\pgfqpoint{1.490500in}{2.534953in}}%
\pgfpathlineto{\pgfqpoint{1.493097in}{2.472239in}}%
\pgfpathlineto{\pgfqpoint{1.494827in}{2.556474in}}%
\pgfpathlineto{\pgfqpoint{1.495693in}{2.460650in}}%
\pgfpathlineto{\pgfqpoint{1.496558in}{2.578916in}}%
\pgfpathlineto{\pgfqpoint{1.497424in}{2.561392in}}%
\pgfpathlineto{\pgfqpoint{1.498288in}{2.559486in}}%
\pgfpathlineto{\pgfqpoint{1.499153in}{2.591213in}}%
\pgfpathlineto{\pgfqpoint{1.500017in}{2.470764in}}%
\pgfpathlineto{\pgfqpoint{1.500882in}{2.548604in}}%
\pgfpathlineto{\pgfqpoint{1.501745in}{2.520751in}}%
\pgfpathlineto{\pgfqpoint{1.502609in}{2.524410in}}%
\pgfpathlineto{\pgfqpoint{1.503473in}{2.562991in}}%
\pgfpathlineto{\pgfqpoint{1.504339in}{2.553092in}}%
\pgfpathlineto{\pgfqpoint{1.505204in}{2.498309in}}%
\pgfpathlineto{\pgfqpoint{1.506068in}{2.556843in}}%
\pgfpathlineto{\pgfqpoint{1.506933in}{2.487980in}}%
\pgfpathlineto{\pgfqpoint{1.507798in}{2.543070in}}%
\pgfpathlineto{\pgfqpoint{1.508662in}{2.502368in}}%
\pgfpathlineto{\pgfqpoint{1.509524in}{2.596316in}}%
\pgfpathlineto{\pgfqpoint{1.510389in}{2.557274in}}%
\pgfpathlineto{\pgfqpoint{1.511254in}{2.595149in}}%
\pgfpathlineto{\pgfqpoint{1.512984in}{2.526470in}}%
\pgfpathlineto{\pgfqpoint{1.513848in}{2.562685in}}%
\pgfpathlineto{\pgfqpoint{1.515579in}{2.478758in}}%
\pgfpathlineto{\pgfqpoint{1.516443in}{2.510730in}}%
\pgfpathlineto{\pgfqpoint{1.517308in}{2.489978in}}%
\pgfpathlineto{\pgfqpoint{1.519038in}{2.544484in}}%
\pgfpathlineto{\pgfqpoint{1.520767in}{2.504888in}}%
\pgfpathlineto{\pgfqpoint{1.521632in}{2.560684in}}%
\pgfpathlineto{\pgfqpoint{1.523360in}{2.465415in}}%
\pgfpathlineto{\pgfqpoint{1.525091in}{2.574857in}}%
\pgfpathlineto{\pgfqpoint{1.525956in}{2.500277in}}%
\pgfpathlineto{\pgfqpoint{1.526821in}{2.501506in}}%
\pgfpathlineto{\pgfqpoint{1.527685in}{2.506794in}}%
\pgfpathlineto{\pgfqpoint{1.528549in}{2.544546in}}%
\pgfpathlineto{\pgfqpoint{1.529411in}{2.449613in}}%
\pgfpathlineto{\pgfqpoint{1.530275in}{2.581435in}}%
\pgfpathlineto{\pgfqpoint{1.531141in}{2.550909in}}%
\pgfpathlineto{\pgfqpoint{1.532005in}{2.530157in}}%
\pgfpathlineto{\pgfqpoint{1.533734in}{2.571044in}}%
\pgfpathlineto{\pgfqpoint{1.535463in}{2.535630in}}%
\pgfpathlineto{\pgfqpoint{1.536326in}{2.576148in}}%
\pgfpathlineto{\pgfqpoint{1.538921in}{2.482446in}}%
\pgfpathlineto{\pgfqpoint{1.539786in}{2.525547in}}%
\pgfpathlineto{\pgfqpoint{1.541516in}{2.492221in}}%
\pgfpathlineto{\pgfqpoint{1.542382in}{2.558993in}}%
\pgfpathlineto{\pgfqpoint{1.544111in}{2.482138in}}%
\pgfpathlineto{\pgfqpoint{1.544975in}{2.546634in}}%
\pgfpathlineto{\pgfqpoint{1.545841in}{2.526376in}}%
\pgfpathlineto{\pgfqpoint{1.546705in}{2.464737in}}%
\pgfpathlineto{\pgfqpoint{1.548435in}{2.522963in}}%
\pgfpathlineto{\pgfqpoint{1.549300in}{2.480293in}}%
\pgfpathlineto{\pgfqpoint{1.550163in}{2.524993in}}%
\pgfpathlineto{\pgfqpoint{1.551028in}{2.502612in}}%
\pgfpathlineto{\pgfqpoint{1.551893in}{2.580022in}}%
\pgfpathlineto{\pgfqpoint{1.552758in}{2.480940in}}%
\pgfpathlineto{\pgfqpoint{1.553624in}{2.537536in}}%
\pgfpathlineto{\pgfqpoint{1.554488in}{2.523948in}}%
\pgfpathlineto{\pgfqpoint{1.555353in}{2.496311in}}%
\pgfpathlineto{\pgfqpoint{1.556218in}{2.564651in}}%
\pgfpathlineto{\pgfqpoint{1.557945in}{2.447768in}}%
\pgfpathlineto{\pgfqpoint{1.558810in}{2.471808in}}%
\pgfpathlineto{\pgfqpoint{1.559676in}{2.526899in}}%
\pgfpathlineto{\pgfqpoint{1.560542in}{2.510912in}}%
\pgfpathlineto{\pgfqpoint{1.562273in}{2.451488in}}%
\pgfpathlineto{\pgfqpoint{1.564872in}{2.531325in}}%
\pgfpathlineto{\pgfqpoint{1.565738in}{2.475313in}}%
\pgfpathlineto{\pgfqpoint{1.566603in}{2.499169in}}%
\pgfpathlineto{\pgfqpoint{1.567467in}{2.453056in}}%
\pgfpathlineto{\pgfqpoint{1.568332in}{2.461541in}}%
\pgfpathlineto{\pgfqpoint{1.570061in}{2.556595in}}%
\pgfpathlineto{\pgfqpoint{1.572656in}{2.504211in}}%
\pgfpathlineto{\pgfqpoint{1.574383in}{2.516139in}}%
\pgfpathlineto{\pgfqpoint{1.575249in}{2.557826in}}%
\pgfpathlineto{\pgfqpoint{1.576978in}{2.518845in}}%
\pgfpathlineto{\pgfqpoint{1.577842in}{2.571291in}}%
\pgfpathlineto{\pgfqpoint{1.578706in}{2.508023in}}%
\pgfpathlineto{\pgfqpoint{1.579570in}{2.521919in}}%
\pgfpathlineto{\pgfqpoint{1.580437in}{2.571167in}}%
\pgfpathlineto{\pgfqpoint{1.581303in}{2.518014in}}%
\pgfpathlineto{\pgfqpoint{1.583036in}{2.545713in}}%
\pgfpathlineto{\pgfqpoint{1.583903in}{2.472116in}}%
\pgfpathlineto{\pgfqpoint{1.585634in}{2.504703in}}%
\pgfpathlineto{\pgfqpoint{1.586500in}{2.456745in}}%
\pgfpathlineto{\pgfqpoint{1.589098in}{2.578115in}}%
\pgfpathlineto{\pgfqpoint{1.592558in}{2.513373in}}%
\pgfpathlineto{\pgfqpoint{1.593423in}{2.437746in}}%
\pgfpathlineto{\pgfqpoint{1.595155in}{2.552476in}}%
\pgfpathlineto{\pgfqpoint{1.596021in}{2.514540in}}%
\pgfpathlineto{\pgfqpoint{1.596886in}{2.545959in}}%
\pgfpathlineto{\pgfqpoint{1.597751in}{2.462033in}}%
\pgfpathlineto{\pgfqpoint{1.598615in}{2.550201in}}%
\pgfpathlineto{\pgfqpoint{1.599480in}{2.533477in}}%
\pgfpathlineto{\pgfqpoint{1.600345in}{2.569015in}}%
\pgfpathlineto{\pgfqpoint{1.601207in}{2.548849in}}%
\pgfpathlineto{\pgfqpoint{1.602937in}{2.490407in}}%
\pgfpathlineto{\pgfqpoint{1.603803in}{2.548541in}}%
\pgfpathlineto{\pgfqpoint{1.604669in}{2.500275in}}%
\pgfpathlineto{\pgfqpoint{1.606401in}{2.543684in}}%
\pgfpathlineto{\pgfqpoint{1.607265in}{2.521242in}}%
\pgfpathlineto{\pgfqpoint{1.608995in}{2.567909in}}%
\pgfpathlineto{\pgfqpoint{1.610725in}{2.447583in}}%
\pgfpathlineto{\pgfqpoint{1.611590in}{2.499107in}}%
\pgfpathlineto{\pgfqpoint{1.612456in}{2.452779in}}%
\pgfpathlineto{\pgfqpoint{1.615051in}{2.592625in}}%
\pgfpathlineto{\pgfqpoint{1.615917in}{2.516015in}}%
\pgfpathlineto{\pgfqpoint{1.618510in}{2.560407in}}%
\pgfpathlineto{\pgfqpoint{1.619375in}{2.481368in}}%
\pgfpathlineto{\pgfqpoint{1.621104in}{2.557210in}}%
\pgfpathlineto{\pgfqpoint{1.621969in}{2.490161in}}%
\pgfpathlineto{\pgfqpoint{1.622835in}{2.536490in}}%
\pgfpathlineto{\pgfqpoint{1.623700in}{2.536305in}}%
\pgfpathlineto{\pgfqpoint{1.624564in}{2.527759in}}%
\pgfpathlineto{\pgfqpoint{1.625428in}{2.554136in}}%
\pgfpathlineto{\pgfqpoint{1.626293in}{2.545375in}}%
\pgfpathlineto{\pgfqpoint{1.627159in}{2.509868in}}%
\pgfpathlineto{\pgfqpoint{1.628022in}{2.532002in}}%
\pgfpathlineto{\pgfqpoint{1.628886in}{2.490376in}}%
\pgfpathlineto{\pgfqpoint{1.630617in}{2.552661in}}%
\pgfpathlineto{\pgfqpoint{1.631481in}{2.547342in}}%
\pgfpathlineto{\pgfqpoint{1.632344in}{2.523210in}}%
\pgfpathlineto{\pgfqpoint{1.633210in}{2.562991in}}%
\pgfpathlineto{\pgfqpoint{1.634073in}{2.518168in}}%
\pgfpathlineto{\pgfqpoint{1.634936in}{2.527022in}}%
\pgfpathlineto{\pgfqpoint{1.635803in}{2.512880in}}%
\pgfpathlineto{\pgfqpoint{1.637533in}{2.601235in}}%
\pgfpathlineto{\pgfqpoint{1.638397in}{2.523733in}}%
\pgfpathlineto{\pgfqpoint{1.639263in}{2.545528in}}%
\pgfpathlineto{\pgfqpoint{1.640129in}{2.581620in}}%
\pgfpathlineto{\pgfqpoint{1.643590in}{2.477466in}}%
\pgfpathlineto{\pgfqpoint{1.644456in}{2.554567in}}%
\pgfpathlineto{\pgfqpoint{1.645321in}{2.502612in}}%
\pgfpathlineto{\pgfqpoint{1.646187in}{2.505994in}}%
\pgfpathlineto{\pgfqpoint{1.647053in}{2.579591in}}%
\pgfpathlineto{\pgfqpoint{1.647918in}{2.570860in}}%
\pgfpathlineto{\pgfqpoint{1.648784in}{2.467382in}}%
\pgfpathlineto{\pgfqpoint{1.649645in}{2.526345in}}%
\pgfpathlineto{\pgfqpoint{1.650509in}{2.522411in}}%
\pgfpathlineto{\pgfqpoint{1.651372in}{2.532125in}}%
\pgfpathlineto{\pgfqpoint{1.652236in}{2.555920in}}%
\pgfpathlineto{\pgfqpoint{1.653101in}{2.516508in}}%
\pgfpathlineto{\pgfqpoint{1.653965in}{2.589612in}}%
\pgfpathlineto{\pgfqpoint{1.654830in}{2.512757in}}%
\pgfpathlineto{\pgfqpoint{1.655694in}{2.524931in}}%
\pgfpathlineto{\pgfqpoint{1.656558in}{2.545159in}}%
\pgfpathlineto{\pgfqpoint{1.657422in}{2.538858in}}%
\pgfpathlineto{\pgfqpoint{1.658286in}{2.523456in}}%
\pgfpathlineto{\pgfqpoint{1.659151in}{2.566372in}}%
\pgfpathlineto{\pgfqpoint{1.660879in}{2.469902in}}%
\pgfpathlineto{\pgfqpoint{1.661745in}{2.543253in}}%
\pgfpathlineto{\pgfqpoint{1.662610in}{2.497571in}}%
\pgfpathlineto{\pgfqpoint{1.664342in}{2.546881in}}%
\pgfpathlineto{\pgfqpoint{1.665207in}{2.473530in}}%
\pgfpathlineto{\pgfqpoint{1.666935in}{2.508791in}}%
\pgfpathlineto{\pgfqpoint{1.667800in}{2.519397in}}%
\pgfpathlineto{\pgfqpoint{1.668666in}{2.543745in}}%
\pgfpathlineto{\pgfqpoint{1.669531in}{2.467873in}}%
\pgfpathlineto{\pgfqpoint{1.670393in}{2.500891in}}%
\pgfpathlineto{\pgfqpoint{1.671259in}{2.443157in}}%
\pgfpathlineto{\pgfqpoint{1.673852in}{2.542270in}}%
\pgfpathlineto{\pgfqpoint{1.674716in}{2.449551in}}%
\pgfpathlineto{\pgfqpoint{1.676445in}{2.591703in}}%
\pgfpathlineto{\pgfqpoint{1.678175in}{2.498677in}}%
\pgfpathlineto{\pgfqpoint{1.679039in}{2.475067in}}%
\pgfpathlineto{\pgfqpoint{1.679905in}{2.483859in}}%
\pgfpathlineto{\pgfqpoint{1.680770in}{2.524931in}}%
\pgfpathlineto{\pgfqpoint{1.681635in}{2.524685in}}%
\pgfpathlineto{\pgfqpoint{1.683364in}{2.482046in}}%
\pgfpathlineto{\pgfqpoint{1.684229in}{2.514909in}}%
\pgfpathlineto{\pgfqpoint{1.685094in}{2.607382in}}%
\pgfpathlineto{\pgfqpoint{1.686825in}{2.507715in}}%
\pgfpathlineto{\pgfqpoint{1.687689in}{2.520013in}}%
\pgfpathlineto{\pgfqpoint{1.688554in}{2.416441in}}%
\pgfpathlineto{\pgfqpoint{1.689420in}{2.540056in}}%
\pgfpathlineto{\pgfqpoint{1.690285in}{2.524562in}}%
\pgfpathlineto{\pgfqpoint{1.692015in}{2.456376in}}%
\pgfpathlineto{\pgfqpoint{1.692880in}{2.473653in}}%
\pgfpathlineto{\pgfqpoint{1.694607in}{2.549893in}}%
\pgfpathlineto{\pgfqpoint{1.695472in}{2.466490in}}%
\pgfpathlineto{\pgfqpoint{1.697201in}{2.527820in}}%
\pgfpathlineto{\pgfqpoint{1.698064in}{2.509406in}}%
\pgfpathlineto{\pgfqpoint{1.698929in}{2.563850in}}%
\pgfpathlineto{\pgfqpoint{1.700660in}{2.499354in}}%
\pgfpathlineto{\pgfqpoint{1.702389in}{2.446139in}}%
\pgfpathlineto{\pgfqpoint{1.703252in}{2.463631in}}%
\pgfpathlineto{\pgfqpoint{1.704117in}{2.459573in}}%
\pgfpathlineto{\pgfqpoint{1.704981in}{2.473530in}}%
\pgfpathlineto{\pgfqpoint{1.705846in}{2.449182in}}%
\pgfpathlineto{\pgfqpoint{1.708439in}{2.545036in}}%
\pgfpathlineto{\pgfqpoint{1.710169in}{2.497324in}}%
\pgfpathlineto{\pgfqpoint{1.711035in}{2.526283in}}%
\pgfpathlineto{\pgfqpoint{1.711900in}{2.520965in}}%
\pgfpathlineto{\pgfqpoint{1.712763in}{2.464553in}}%
\pgfpathlineto{\pgfqpoint{1.713627in}{2.533293in}}%
\pgfpathlineto{\pgfqpoint{1.714491in}{2.483983in}}%
\pgfpathlineto{\pgfqpoint{1.715355in}{2.519766in}}%
\pgfpathlineto{\pgfqpoint{1.716220in}{2.509622in}}%
\pgfpathlineto{\pgfqpoint{1.717084in}{2.447614in}}%
\pgfpathlineto{\pgfqpoint{1.717949in}{2.518106in}}%
\pgfpathlineto{\pgfqpoint{1.718815in}{2.506363in}}%
\pgfpathlineto{\pgfqpoint{1.721409in}{2.477158in}}%
\pgfpathlineto{\pgfqpoint{1.722274in}{2.529175in}}%
\pgfpathlineto{\pgfqpoint{1.723138in}{2.509806in}}%
\pgfpathlineto{\pgfqpoint{1.725732in}{2.559178in}}%
\pgfpathlineto{\pgfqpoint{1.727461in}{2.496649in}}%
\pgfpathlineto{\pgfqpoint{1.728326in}{2.579775in}}%
\pgfpathlineto{\pgfqpoint{1.729191in}{2.561084in}}%
\pgfpathlineto{\pgfqpoint{1.730056in}{2.524685in}}%
\pgfpathlineto{\pgfqpoint{1.730922in}{2.548572in}}%
\pgfpathlineto{\pgfqpoint{1.731786in}{2.529788in}}%
\pgfpathlineto{\pgfqpoint{1.732650in}{2.572704in}}%
\pgfpathlineto{\pgfqpoint{1.733516in}{2.560653in}}%
\pgfpathlineto{\pgfqpoint{1.736112in}{2.616390in}}%
\pgfpathlineto{\pgfqpoint{1.738706in}{2.513865in}}%
\pgfpathlineto{\pgfqpoint{1.739570in}{2.553154in}}%
\pgfpathlineto{\pgfqpoint{1.742164in}{2.498617in}}%
\pgfpathlineto{\pgfqpoint{1.743026in}{2.518229in}}%
\pgfpathlineto{\pgfqpoint{1.743891in}{2.454716in}}%
\pgfpathlineto{\pgfqpoint{1.744756in}{2.525793in}}%
\pgfpathlineto{\pgfqpoint{1.745620in}{2.463385in}}%
\pgfpathlineto{\pgfqpoint{1.746486in}{2.465599in}}%
\pgfpathlineto{\pgfqpoint{1.747352in}{2.541901in}}%
\pgfpathlineto{\pgfqpoint{1.748217in}{2.493021in}}%
\pgfpathlineto{\pgfqpoint{1.750814in}{2.554013in}}%
\pgfpathlineto{\pgfqpoint{1.751679in}{2.497878in}}%
\pgfpathlineto{\pgfqpoint{1.752545in}{2.511682in}}%
\pgfpathlineto{\pgfqpoint{1.753409in}{2.501137in}}%
\pgfpathlineto{\pgfqpoint{1.755138in}{2.568248in}}%
\pgfpathlineto{\pgfqpoint{1.756004in}{2.524870in}}%
\pgfpathlineto{\pgfqpoint{1.756870in}{2.595331in}}%
\pgfpathlineto{\pgfqpoint{1.757735in}{2.543376in}}%
\pgfpathlineto{\pgfqpoint{1.758602in}{2.561146in}}%
\pgfpathlineto{\pgfqpoint{1.759468in}{2.461263in}}%
\pgfpathlineto{\pgfqpoint{1.760334in}{2.542270in}}%
\pgfpathlineto{\pgfqpoint{1.761200in}{2.535014in}}%
\pgfpathlineto{\pgfqpoint{1.762931in}{2.571537in}}%
\pgfpathlineto{\pgfqpoint{1.763797in}{2.567971in}}%
\pgfpathlineto{\pgfqpoint{1.764663in}{2.472209in}}%
\pgfpathlineto{\pgfqpoint{1.765529in}{2.490500in}}%
\pgfpathlineto{\pgfqpoint{1.766394in}{2.449305in}}%
\pgfpathlineto{\pgfqpoint{1.768123in}{2.529665in}}%
\pgfpathlineto{\pgfqpoint{1.768988in}{2.514817in}}%
\pgfpathlineto{\pgfqpoint{1.771582in}{2.563483in}}%
\pgfpathlineto{\pgfqpoint{1.772446in}{2.537782in}}%
\pgfpathlineto{\pgfqpoint{1.773309in}{2.573751in}}%
\pgfpathlineto{\pgfqpoint{1.775039in}{2.521488in}}%
\pgfpathlineto{\pgfqpoint{1.775903in}{2.547127in}}%
\pgfpathlineto{\pgfqpoint{1.776768in}{2.535630in}}%
\pgfpathlineto{\pgfqpoint{1.777633in}{2.552292in}}%
\pgfpathlineto{\pgfqpoint{1.778496in}{2.549187in}}%
\pgfpathlineto{\pgfqpoint{1.781086in}{2.499631in}}%
\pgfpathlineto{\pgfqpoint{1.781951in}{2.544053in}}%
\pgfpathlineto{\pgfqpoint{1.782816in}{2.533231in}}%
\pgfpathlineto{\pgfqpoint{1.783681in}{2.476450in}}%
\pgfpathlineto{\pgfqpoint{1.784545in}{2.485827in}}%
\pgfpathlineto{\pgfqpoint{1.785410in}{2.498984in}}%
\pgfpathlineto{\pgfqpoint{1.786275in}{2.448137in}}%
\pgfpathlineto{\pgfqpoint{1.787138in}{2.509437in}}%
\pgfpathlineto{\pgfqpoint{1.788003in}{2.475313in}}%
\pgfpathlineto{\pgfqpoint{1.788868in}{2.550509in}}%
\pgfpathlineto{\pgfqpoint{1.789732in}{2.486443in}}%
\pgfpathlineto{\pgfqpoint{1.792326in}{2.530281in}}%
\pgfpathlineto{\pgfqpoint{1.793189in}{2.525054in}}%
\pgfpathlineto{\pgfqpoint{1.794054in}{2.558010in}}%
\pgfpathlineto{\pgfqpoint{1.794918in}{2.438116in}}%
\pgfpathlineto{\pgfqpoint{1.798377in}{2.578423in}}%
\pgfpathlineto{\pgfqpoint{1.800107in}{2.517922in}}%
\pgfpathlineto{\pgfqpoint{1.800973in}{2.558932in}}%
\pgfpathlineto{\pgfqpoint{1.802704in}{2.529726in}}%
\pgfpathlineto{\pgfqpoint{1.803570in}{2.550201in}}%
\pgfpathlineto{\pgfqpoint{1.804436in}{2.509314in}}%
\pgfpathlineto{\pgfqpoint{1.805299in}{2.516446in}}%
\pgfpathlineto{\pgfqpoint{1.807031in}{2.538765in}}%
\pgfpathlineto{\pgfqpoint{1.807897in}{2.593486in}}%
\pgfpathlineto{\pgfqpoint{1.809628in}{2.546881in}}%
\pgfpathlineto{\pgfqpoint{1.810493in}{2.561269in}}%
\pgfpathlineto{\pgfqpoint{1.811358in}{2.497447in}}%
\pgfpathlineto{\pgfqpoint{1.813089in}{2.555981in}}%
\pgfpathlineto{\pgfqpoint{1.813954in}{2.530465in}}%
\pgfpathlineto{\pgfqpoint{1.814820in}{2.539073in}}%
\pgfpathlineto{\pgfqpoint{1.815683in}{2.481093in}}%
\pgfpathlineto{\pgfqpoint{1.816549in}{2.505196in}}%
\pgfpathlineto{\pgfqpoint{1.817413in}{2.448630in}}%
\pgfpathlineto{\pgfqpoint{1.818276in}{2.530835in}}%
\pgfpathlineto{\pgfqpoint{1.819140in}{2.515433in}}%
\pgfpathlineto{\pgfqpoint{1.820870in}{2.492221in}}%
\pgfpathlineto{\pgfqpoint{1.821736in}{2.506209in}}%
\pgfpathlineto{\pgfqpoint{1.822603in}{2.570000in}}%
\pgfpathlineto{\pgfqpoint{1.824334in}{2.476727in}}%
\pgfpathlineto{\pgfqpoint{1.825198in}{2.502735in}}%
\pgfpathlineto{\pgfqpoint{1.827794in}{2.567786in}}%
\pgfpathlineto{\pgfqpoint{1.829522in}{2.543622in}}%
\pgfpathlineto{\pgfqpoint{1.830388in}{2.547250in}}%
\pgfpathlineto{\pgfqpoint{1.831253in}{2.598774in}}%
\pgfpathlineto{\pgfqpoint{1.832979in}{2.550632in}}%
\pgfpathlineto{\pgfqpoint{1.833845in}{2.550878in}}%
\pgfpathlineto{\pgfqpoint{1.835577in}{2.484721in}}%
\pgfpathlineto{\pgfqpoint{1.836442in}{2.543193in}}%
\pgfpathlineto{\pgfqpoint{1.837307in}{2.486012in}}%
\pgfpathlineto{\pgfqpoint{1.838171in}{2.510514in}}%
\pgfpathlineto{\pgfqpoint{1.839035in}{2.570800in}}%
\pgfpathlineto{\pgfqpoint{1.840763in}{2.530681in}}%
\pgfpathlineto{\pgfqpoint{1.841628in}{2.560101in}}%
\pgfpathlineto{\pgfqpoint{1.843359in}{2.469289in}}%
\pgfpathlineto{\pgfqpoint{1.844224in}{2.589060in}}%
\pgfpathlineto{\pgfqpoint{1.845955in}{2.507777in}}%
\pgfpathlineto{\pgfqpoint{1.847684in}{2.547866in}}%
\pgfpathlineto{\pgfqpoint{1.848546in}{2.508454in}}%
\pgfpathlineto{\pgfqpoint{1.849412in}{2.554814in}}%
\pgfpathlineto{\pgfqpoint{1.851141in}{2.485889in}}%
\pgfpathlineto{\pgfqpoint{1.852007in}{2.499354in}}%
\pgfpathlineto{\pgfqpoint{1.852870in}{2.477343in}}%
\pgfpathlineto{\pgfqpoint{1.853735in}{2.533785in}}%
\pgfpathlineto{\pgfqpoint{1.854601in}{2.483675in}}%
\pgfpathlineto{\pgfqpoint{1.856331in}{2.568709in}}%
\pgfpathlineto{\pgfqpoint{1.858062in}{2.471870in}}%
\pgfpathlineto{\pgfqpoint{1.858925in}{2.485581in}}%
\pgfpathlineto{\pgfqpoint{1.859790in}{2.547250in}}%
\pgfpathlineto{\pgfqpoint{1.861521in}{2.488901in}}%
\pgfpathlineto{\pgfqpoint{1.864120in}{2.573074in}}%
\pgfpathlineto{\pgfqpoint{1.865845in}{2.553952in}}%
\pgfpathlineto{\pgfqpoint{1.866708in}{2.563789in}}%
\pgfpathlineto{\pgfqpoint{1.867573in}{2.524746in}}%
\pgfpathlineto{\pgfqpoint{1.868440in}{2.526222in}}%
\pgfpathlineto{\pgfqpoint{1.869304in}{2.567201in}}%
\pgfpathlineto{\pgfqpoint{1.870168in}{2.533108in}}%
\pgfpathlineto{\pgfqpoint{1.871032in}{2.546942in}}%
\pgfpathlineto{\pgfqpoint{1.871897in}{2.469964in}}%
\pgfpathlineto{\pgfqpoint{1.872762in}{2.577500in}}%
\pgfpathlineto{\pgfqpoint{1.873626in}{2.457851in}}%
\pgfpathlineto{\pgfqpoint{1.874493in}{2.517552in}}%
\pgfpathlineto{\pgfqpoint{1.876224in}{2.411522in}}%
\pgfpathlineto{\pgfqpoint{1.877953in}{2.467873in}}%
\pgfpathlineto{\pgfqpoint{1.878816in}{2.454839in}}%
\pgfpathlineto{\pgfqpoint{1.879681in}{2.437500in}}%
\pgfpathlineto{\pgfqpoint{1.880547in}{2.555612in}}%
\pgfpathlineto{\pgfqpoint{1.881412in}{2.481892in}}%
\pgfpathlineto{\pgfqpoint{1.882278in}{2.492283in}}%
\pgfpathlineto{\pgfqpoint{1.883143in}{2.490469in}}%
\pgfpathlineto{\pgfqpoint{1.884008in}{2.479985in}}%
\pgfpathlineto{\pgfqpoint{1.886600in}{2.607382in}}%
\pgfpathlineto{\pgfqpoint{1.888327in}{2.556073in}}%
\pgfpathlineto{\pgfqpoint{1.889190in}{2.573997in}}%
\pgfpathlineto{\pgfqpoint{1.890054in}{2.630686in}}%
\pgfpathlineto{\pgfqpoint{1.890918in}{2.542886in}}%
\pgfpathlineto{\pgfqpoint{1.891784in}{2.558380in}}%
\pgfpathlineto{\pgfqpoint{1.892650in}{2.552846in}}%
\pgfpathlineto{\pgfqpoint{1.893515in}{2.509745in}}%
\pgfpathlineto{\pgfqpoint{1.894381in}{2.537044in}}%
\pgfpathlineto{\pgfqpoint{1.895246in}{2.471316in}}%
\pgfpathlineto{\pgfqpoint{1.896112in}{2.488655in}}%
\pgfpathlineto{\pgfqpoint{1.896975in}{2.454593in}}%
\pgfpathlineto{\pgfqpoint{1.897841in}{2.558932in}}%
\pgfpathlineto{\pgfqpoint{1.898707in}{2.497940in}}%
\pgfpathlineto{\pgfqpoint{1.899572in}{2.535507in}}%
\pgfpathlineto{\pgfqpoint{1.900437in}{2.507838in}}%
\pgfpathlineto{\pgfqpoint{1.901303in}{2.569815in}}%
\pgfpathlineto{\pgfqpoint{1.903034in}{2.469779in}}%
\pgfpathlineto{\pgfqpoint{1.903899in}{2.507592in}}%
\pgfpathlineto{\pgfqpoint{1.904764in}{2.496034in}}%
\pgfpathlineto{\pgfqpoint{1.906494in}{2.579652in}}%
\pgfpathlineto{\pgfqpoint{1.907360in}{2.453179in}}%
\pgfpathlineto{\pgfqpoint{1.908225in}{2.587706in}}%
\pgfpathlineto{\pgfqpoint{1.909953in}{2.515708in}}%
\pgfpathlineto{\pgfqpoint{1.910818in}{2.562498in}}%
\pgfpathlineto{\pgfqpoint{1.911684in}{2.495541in}}%
\pgfpathlineto{\pgfqpoint{1.912550in}{2.571966in}}%
\pgfpathlineto{\pgfqpoint{1.915146in}{2.508944in}}%
\pgfpathlineto{\pgfqpoint{1.916011in}{2.524993in}}%
\pgfpathlineto{\pgfqpoint{1.916877in}{2.478541in}}%
\pgfpathlineto{\pgfqpoint{1.918607in}{2.558993in}}%
\pgfpathlineto{\pgfqpoint{1.920335in}{2.533354in}}%
\pgfpathlineto{\pgfqpoint{1.921200in}{2.478695in}}%
\pgfpathlineto{\pgfqpoint{1.922066in}{2.543684in}}%
\pgfpathlineto{\pgfqpoint{1.923796in}{2.479924in}}%
\pgfpathlineto{\pgfqpoint{1.924661in}{2.536120in}}%
\pgfpathlineto{\pgfqpoint{1.925527in}{2.483367in}}%
\pgfpathlineto{\pgfqpoint{1.927259in}{2.521488in}}%
\pgfpathlineto{\pgfqpoint{1.928122in}{2.519705in}}%
\pgfpathlineto{\pgfqpoint{1.928985in}{2.510604in}}%
\pgfpathlineto{\pgfqpoint{1.930715in}{2.524469in}}%
\pgfpathlineto{\pgfqpoint{1.931580in}{2.596499in}}%
\pgfpathlineto{\pgfqpoint{1.933312in}{2.499107in}}%
\pgfpathlineto{\pgfqpoint{1.934179in}{2.502858in}}%
\pgfpathlineto{\pgfqpoint{1.936775in}{2.465107in}}%
\pgfpathlineto{\pgfqpoint{1.937642in}{2.576148in}}%
\pgfpathlineto{\pgfqpoint{1.938507in}{2.495849in}}%
\pgfpathlineto{\pgfqpoint{1.939373in}{2.539073in}}%
\pgfpathlineto{\pgfqpoint{1.941967in}{2.491546in}}%
\pgfpathlineto{\pgfqpoint{1.942833in}{2.475990in}}%
\pgfpathlineto{\pgfqpoint{1.943696in}{2.561946in}}%
\pgfpathlineto{\pgfqpoint{1.944561in}{2.502060in}}%
\pgfpathlineto{\pgfqpoint{1.945425in}{2.508823in}}%
\pgfpathlineto{\pgfqpoint{1.946290in}{2.498617in}}%
\pgfpathlineto{\pgfqpoint{1.947154in}{2.509070in}}%
\pgfpathlineto{\pgfqpoint{1.948018in}{2.489824in}}%
\pgfpathlineto{\pgfqpoint{1.948883in}{2.515833in}}%
\pgfpathlineto{\pgfqpoint{1.949748in}{2.515310in}}%
\pgfpathlineto{\pgfqpoint{1.953207in}{2.469104in}}%
\pgfpathlineto{\pgfqpoint{1.954072in}{2.548112in}}%
\pgfpathlineto{\pgfqpoint{1.954936in}{2.475713in}}%
\pgfpathlineto{\pgfqpoint{1.956666in}{2.599267in}}%
\pgfpathlineto{\pgfqpoint{1.957531in}{2.548112in}}%
\pgfpathlineto{\pgfqpoint{1.958396in}{2.548420in}}%
\pgfpathlineto{\pgfqpoint{1.959262in}{2.527668in}}%
\pgfpathlineto{\pgfqpoint{1.960126in}{2.541349in}}%
\pgfpathlineto{\pgfqpoint{1.962723in}{2.484229in}}%
\pgfpathlineto{\pgfqpoint{1.963588in}{2.546021in}}%
\pgfpathlineto{\pgfqpoint{1.964454in}{2.490469in}}%
\pgfpathlineto{\pgfqpoint{1.965319in}{2.564158in}}%
\pgfpathlineto{\pgfqpoint{1.967048in}{2.534768in}}%
\pgfpathlineto{\pgfqpoint{1.967912in}{2.528621in}}%
\pgfpathlineto{\pgfqpoint{1.968778in}{2.541931in}}%
\pgfpathlineto{\pgfqpoint{1.970509in}{2.516631in}}%
\pgfpathlineto{\pgfqpoint{1.971375in}{2.543807in}}%
\pgfpathlineto{\pgfqpoint{1.972240in}{2.483921in}}%
\pgfpathlineto{\pgfqpoint{1.973106in}{2.546329in}}%
\pgfpathlineto{\pgfqpoint{1.973972in}{2.426526in}}%
\pgfpathlineto{\pgfqpoint{1.974835in}{2.477527in}}%
\pgfpathlineto{\pgfqpoint{1.975702in}{2.605661in}}%
\pgfpathlineto{\pgfqpoint{1.977433in}{2.473899in}}%
\pgfpathlineto{\pgfqpoint{1.978298in}{2.499661in}}%
\pgfpathlineto{\pgfqpoint{1.979162in}{2.470087in}}%
\pgfpathlineto{\pgfqpoint{1.980028in}{2.499415in}}%
\pgfpathlineto{\pgfqpoint{1.980894in}{2.580391in}}%
\pgfpathlineto{\pgfqpoint{1.981760in}{2.505011in}}%
\pgfpathlineto{\pgfqpoint{1.982625in}{2.518907in}}%
\pgfpathlineto{\pgfqpoint{1.983491in}{2.492960in}}%
\pgfpathlineto{\pgfqpoint{1.985220in}{2.532556in}}%
\pgfpathlineto{\pgfqpoint{1.986085in}{2.506794in}}%
\pgfpathlineto{\pgfqpoint{1.986949in}{2.525793in}}%
\pgfpathlineto{\pgfqpoint{1.987814in}{2.479372in}}%
\pgfpathlineto{\pgfqpoint{1.988674in}{2.544176in}}%
\pgfpathlineto{\pgfqpoint{1.989540in}{2.514909in}}%
\pgfpathlineto{\pgfqpoint{1.990405in}{2.580299in}}%
\pgfpathlineto{\pgfqpoint{1.991267in}{2.527453in}}%
\pgfpathlineto{\pgfqpoint{1.992997in}{2.564651in}}%
\pgfpathlineto{\pgfqpoint{1.996456in}{2.479864in}}%
\pgfpathlineto{\pgfqpoint{1.998186in}{2.563175in}}%
\pgfpathlineto{\pgfqpoint{1.999051in}{2.520813in}}%
\pgfpathlineto{\pgfqpoint{1.999915in}{2.555583in}}%
\pgfpathlineto{\pgfqpoint{2.000777in}{2.501937in}}%
\pgfpathlineto{\pgfqpoint{2.001642in}{2.536369in}}%
\pgfpathlineto{\pgfqpoint{2.002508in}{2.488195in}}%
\pgfpathlineto{\pgfqpoint{2.003373in}{2.507410in}}%
\pgfpathlineto{\pgfqpoint{2.004238in}{2.491854in}}%
\pgfpathlineto{\pgfqpoint{2.005968in}{2.508947in}}%
\pgfpathlineto{\pgfqpoint{2.006833in}{2.539196in}}%
\pgfpathlineto{\pgfqpoint{2.008563in}{2.502920in}}%
\pgfpathlineto{\pgfqpoint{2.009429in}{2.576548in}}%
\pgfpathlineto{\pgfqpoint{2.010294in}{2.567111in}}%
\pgfpathlineto{\pgfqpoint{2.011158in}{2.580206in}}%
\pgfpathlineto{\pgfqpoint{2.012887in}{2.541410in}}%
\pgfpathlineto{\pgfqpoint{2.013752in}{2.549957in}}%
\pgfpathlineto{\pgfqpoint{2.015482in}{2.479926in}}%
\pgfpathlineto{\pgfqpoint{2.016347in}{2.494989in}}%
\pgfpathlineto{\pgfqpoint{2.017211in}{2.480786in}}%
\pgfpathlineto{\pgfqpoint{2.018077in}{2.505503in}}%
\pgfpathlineto{\pgfqpoint{2.018939in}{2.488103in}}%
\pgfpathlineto{\pgfqpoint{2.019804in}{2.544853in}}%
\pgfpathlineto{\pgfqpoint{2.020669in}{2.524256in}}%
\pgfpathlineto{\pgfqpoint{2.021535in}{2.540733in}}%
\pgfpathlineto{\pgfqpoint{2.022400in}{2.480663in}}%
\pgfpathlineto{\pgfqpoint{2.024995in}{2.586415in}}%
\pgfpathlineto{\pgfqpoint{2.025860in}{2.463631in}}%
\pgfpathlineto{\pgfqpoint{2.026727in}{2.570616in}}%
\pgfpathlineto{\pgfqpoint{2.029320in}{2.493268in}}%
\pgfpathlineto{\pgfqpoint{2.030186in}{2.566988in}}%
\pgfpathlineto{\pgfqpoint{2.031052in}{2.475621in}}%
\pgfpathlineto{\pgfqpoint{2.032782in}{2.544299in}}%
\pgfpathlineto{\pgfqpoint{2.033646in}{2.518599in}}%
\pgfpathlineto{\pgfqpoint{2.034510in}{2.559732in}}%
\pgfpathlineto{\pgfqpoint{2.035374in}{2.538950in}}%
\pgfpathlineto{\pgfqpoint{2.036239in}{2.614084in}}%
\pgfpathlineto{\pgfqpoint{2.037104in}{2.594040in}}%
\pgfpathlineto{\pgfqpoint{2.040565in}{2.480109in}}%
\pgfpathlineto{\pgfqpoint{2.041430in}{2.485519in}}%
\pgfpathlineto{\pgfqpoint{2.042296in}{2.549095in}}%
\pgfpathlineto{\pgfqpoint{2.043162in}{2.508085in}}%
\pgfpathlineto{\pgfqpoint{2.044026in}{2.538273in}}%
\pgfpathlineto{\pgfqpoint{2.044891in}{2.505871in}}%
\pgfpathlineto{\pgfqpoint{2.046620in}{2.622507in}}%
\pgfpathlineto{\pgfqpoint{2.047484in}{2.500860in}}%
\pgfpathlineto{\pgfqpoint{2.048349in}{2.578054in}}%
\pgfpathlineto{\pgfqpoint{2.049214in}{2.533293in}}%
\pgfpathlineto{\pgfqpoint{2.050080in}{2.597851in}}%
\pgfpathlineto{\pgfqpoint{2.050945in}{2.550324in}}%
\pgfpathlineto{\pgfqpoint{2.053538in}{2.604922in}}%
\pgfpathlineto{\pgfqpoint{2.054402in}{2.534799in}}%
\pgfpathlineto{\pgfqpoint{2.055268in}{2.577561in}}%
\pgfpathlineto{\pgfqpoint{2.056131in}{2.502735in}}%
\pgfpathlineto{\pgfqpoint{2.057861in}{2.557456in}}%
\pgfpathlineto{\pgfqpoint{2.058726in}{2.568956in}}%
\pgfpathlineto{\pgfqpoint{2.059591in}{2.494651in}}%
\pgfpathlineto{\pgfqpoint{2.060456in}{2.590228in}}%
\pgfpathlineto{\pgfqpoint{2.061322in}{2.582297in}}%
\pgfpathlineto{\pgfqpoint{2.062187in}{2.525547in}}%
\pgfpathlineto{\pgfqpoint{2.063052in}{2.529236in}}%
\pgfpathlineto{\pgfqpoint{2.064780in}{2.496341in}}%
\pgfpathlineto{\pgfqpoint{2.065646in}{2.588445in}}%
\pgfpathlineto{\pgfqpoint{2.066512in}{2.574395in}}%
\pgfpathlineto{\pgfqpoint{2.067377in}{2.604247in}}%
\pgfpathlineto{\pgfqpoint{2.068243in}{2.514786in}}%
\pgfpathlineto{\pgfqpoint{2.069109in}{2.519736in}}%
\pgfpathlineto{\pgfqpoint{2.069974in}{2.497755in}}%
\pgfpathlineto{\pgfqpoint{2.070838in}{2.513188in}}%
\pgfpathlineto{\pgfqpoint{2.072565in}{2.485643in}}%
\pgfpathlineto{\pgfqpoint{2.073429in}{2.579898in}}%
\pgfpathlineto{\pgfqpoint{2.074294in}{2.557887in}}%
\pgfpathlineto{\pgfqpoint{2.075160in}{2.546390in}}%
\pgfpathlineto{\pgfqpoint{2.076025in}{2.496495in}}%
\pgfpathlineto{\pgfqpoint{2.076890in}{2.592319in}}%
\pgfpathlineto{\pgfqpoint{2.077754in}{2.491361in}}%
\pgfpathlineto{\pgfqpoint{2.078618in}{2.520105in}}%
\pgfpathlineto{\pgfqpoint{2.079483in}{2.503228in}}%
\pgfpathlineto{\pgfqpoint{2.080348in}{2.521488in}}%
\pgfpathlineto{\pgfqpoint{2.081213in}{2.495911in}}%
\pgfpathlineto{\pgfqpoint{2.082078in}{2.556043in}}%
\pgfpathlineto{\pgfqpoint{2.082942in}{2.492313in}}%
\pgfpathlineto{\pgfqpoint{2.083807in}{2.505932in}}%
\pgfpathlineto{\pgfqpoint{2.084671in}{2.550693in}}%
\pgfpathlineto{\pgfqpoint{2.085536in}{2.436425in}}%
\pgfpathlineto{\pgfqpoint{2.086401in}{2.518907in}}%
\pgfpathlineto{\pgfqpoint{2.087265in}{2.517860in}}%
\pgfpathlineto{\pgfqpoint{2.088130in}{2.516631in}}%
\pgfpathlineto{\pgfqpoint{2.089859in}{2.547496in}}%
\pgfpathlineto{\pgfqpoint{2.090724in}{2.525424in}}%
\pgfpathlineto{\pgfqpoint{2.091589in}{2.564527in}}%
\pgfpathlineto{\pgfqpoint{2.092454in}{2.452258in}}%
\pgfpathlineto{\pgfqpoint{2.095049in}{2.583896in}}%
\pgfpathlineto{\pgfqpoint{2.095914in}{2.512636in}}%
\pgfpathlineto{\pgfqpoint{2.096780in}{2.520320in}}%
\pgfpathlineto{\pgfqpoint{2.097644in}{2.588014in}}%
\pgfpathlineto{\pgfqpoint{2.098507in}{2.578238in}}%
\pgfpathlineto{\pgfqpoint{2.099373in}{2.566618in}}%
\pgfpathlineto{\pgfqpoint{2.101101in}{2.530034in}}%
\pgfpathlineto{\pgfqpoint{2.101963in}{2.547004in}}%
\pgfpathlineto{\pgfqpoint{2.102828in}{2.533293in}}%
\pgfpathlineto{\pgfqpoint{2.103692in}{2.557456in}}%
\pgfpathlineto{\pgfqpoint{2.104558in}{2.520872in}}%
\pgfpathlineto{\pgfqpoint{2.105422in}{2.583095in}}%
\pgfpathlineto{\pgfqpoint{2.108016in}{2.477404in}}%
\pgfpathlineto{\pgfqpoint{2.109746in}{2.566003in}}%
\pgfpathlineto{\pgfqpoint{2.110611in}{2.594102in}}%
\pgfpathlineto{\pgfqpoint{2.112342in}{2.503228in}}%
\pgfpathlineto{\pgfqpoint{2.113207in}{2.499538in}}%
\pgfpathlineto{\pgfqpoint{2.114071in}{2.508146in}}%
\pgfpathlineto{\pgfqpoint{2.114938in}{2.562436in}}%
\pgfpathlineto{\pgfqpoint{2.115804in}{2.503351in}}%
\pgfpathlineto{\pgfqpoint{2.116670in}{2.512541in}}%
\pgfpathlineto{\pgfqpoint{2.117537in}{2.466890in}}%
\pgfpathlineto{\pgfqpoint{2.118403in}{2.593548in}}%
\pgfpathlineto{\pgfqpoint{2.119267in}{2.564343in}}%
\pgfpathlineto{\pgfqpoint{2.120134in}{2.567232in}}%
\pgfpathlineto{\pgfqpoint{2.120999in}{2.556964in}}%
\pgfpathlineto{\pgfqpoint{2.122731in}{2.587952in}}%
\pgfpathlineto{\pgfqpoint{2.124463in}{2.500521in}}%
\pgfpathlineto{\pgfqpoint{2.125328in}{2.582849in}}%
\pgfpathlineto{\pgfqpoint{2.126193in}{2.524069in}}%
\pgfpathlineto{\pgfqpoint{2.127057in}{2.591457in}}%
\pgfpathlineto{\pgfqpoint{2.127924in}{2.503780in}}%
\pgfpathlineto{\pgfqpoint{2.128790in}{2.525791in}}%
\pgfpathlineto{\pgfqpoint{2.129654in}{2.563912in}}%
\pgfpathlineto{\pgfqpoint{2.130520in}{2.516169in}}%
\pgfpathlineto{\pgfqpoint{2.131385in}{2.528436in}}%
\pgfpathlineto{\pgfqpoint{2.132251in}{2.492590in}}%
\pgfpathlineto{\pgfqpoint{2.133982in}{2.619002in}}%
\pgfpathlineto{\pgfqpoint{2.135713in}{2.510697in}}%
\pgfpathlineto{\pgfqpoint{2.136575in}{2.567355in}}%
\pgfpathlineto{\pgfqpoint{2.137440in}{2.538950in}}%
\pgfpathlineto{\pgfqpoint{2.138303in}{2.577746in}}%
\pgfpathlineto{\pgfqpoint{2.140031in}{2.509191in}}%
\pgfpathlineto{\pgfqpoint{2.140897in}{2.578238in}}%
\pgfpathlineto{\pgfqpoint{2.142629in}{2.520228in}}%
\pgfpathlineto{\pgfqpoint{2.143495in}{2.524870in}}%
\pgfpathlineto{\pgfqpoint{2.144360in}{2.522840in}}%
\pgfpathlineto{\pgfqpoint{2.145225in}{2.495418in}}%
\pgfpathlineto{\pgfqpoint{2.146091in}{2.539196in}}%
\pgfpathlineto{\pgfqpoint{2.147822in}{2.459819in}}%
\pgfpathlineto{\pgfqpoint{2.148687in}{2.466767in}}%
\pgfpathlineto{\pgfqpoint{2.149554in}{2.464368in}}%
\pgfpathlineto{\pgfqpoint{2.150418in}{2.473407in}}%
\pgfpathlineto{\pgfqpoint{2.151284in}{2.535507in}}%
\pgfpathlineto{\pgfqpoint{2.152149in}{2.476850in}}%
\pgfpathlineto{\pgfqpoint{2.153015in}{2.479249in}}%
\pgfpathlineto{\pgfqpoint{2.155607in}{2.547866in}}%
\pgfpathlineto{\pgfqpoint{2.156472in}{2.545652in}}%
\pgfpathlineto{\pgfqpoint{2.157337in}{2.501260in}}%
\pgfpathlineto{\pgfqpoint{2.158202in}{2.605291in}}%
\pgfpathlineto{\pgfqpoint{2.159928in}{2.486564in}}%
\pgfpathlineto{\pgfqpoint{2.160794in}{2.496955in}}%
\pgfpathlineto{\pgfqpoint{2.162520in}{2.528618in}}%
\pgfpathlineto{\pgfqpoint{2.163384in}{2.517306in}}%
\pgfpathlineto{\pgfqpoint{2.164251in}{2.454254in}}%
\pgfpathlineto{\pgfqpoint{2.165115in}{2.611562in}}%
\pgfpathlineto{\pgfqpoint{2.166844in}{2.521365in}}%
\pgfpathlineto{\pgfqpoint{2.167707in}{2.583526in}}%
\pgfpathlineto{\pgfqpoint{2.168572in}{2.516015in}}%
\pgfpathlineto{\pgfqpoint{2.169437in}{2.589797in}}%
\pgfpathlineto{\pgfqpoint{2.170301in}{2.580512in}}%
\pgfpathlineto{\pgfqpoint{2.171166in}{2.583280in}}%
\pgfpathlineto{\pgfqpoint{2.172893in}{2.520136in}}%
\pgfpathlineto{\pgfqpoint{2.173758in}{2.514448in}}%
\pgfpathlineto{\pgfqpoint{2.174620in}{2.444756in}}%
\pgfpathlineto{\pgfqpoint{2.175484in}{2.538765in}}%
\pgfpathlineto{\pgfqpoint{2.176349in}{2.510789in}}%
\pgfpathlineto{\pgfqpoint{2.177215in}{2.628839in}}%
\pgfpathlineto{\pgfqpoint{2.178945in}{2.488409in}}%
\pgfpathlineto{\pgfqpoint{2.179810in}{2.528128in}}%
\pgfpathlineto{\pgfqpoint{2.181542in}{2.491790in}}%
\pgfpathlineto{\pgfqpoint{2.182405in}{2.541962in}}%
\pgfpathlineto{\pgfqpoint{2.183270in}{2.475128in}}%
\pgfpathlineto{\pgfqpoint{2.184136in}{2.541716in}}%
\pgfpathlineto{\pgfqpoint{2.185000in}{2.531941in}}%
\pgfpathlineto{\pgfqpoint{2.185866in}{2.524500in}}%
\pgfpathlineto{\pgfqpoint{2.186731in}{2.570860in}}%
\pgfpathlineto{\pgfqpoint{2.187597in}{2.488993in}}%
\pgfpathlineto{\pgfqpoint{2.190192in}{2.630776in}}%
\pgfpathlineto{\pgfqpoint{2.191056in}{2.542024in}}%
\pgfpathlineto{\pgfqpoint{2.191922in}{2.565633in}}%
\pgfpathlineto{\pgfqpoint{2.192788in}{2.507500in}}%
\pgfpathlineto{\pgfqpoint{2.193652in}{2.550201in}}%
\pgfpathlineto{\pgfqpoint{2.194519in}{2.544420in}}%
\pgfpathlineto{\pgfqpoint{2.195384in}{2.517183in}}%
\pgfpathlineto{\pgfqpoint{2.196249in}{2.549831in}}%
\pgfpathlineto{\pgfqpoint{2.199710in}{2.483244in}}%
\pgfpathlineto{\pgfqpoint{2.202305in}{2.549831in}}%
\pgfpathlineto{\pgfqpoint{2.203171in}{2.539625in}}%
\pgfpathlineto{\pgfqpoint{2.204900in}{2.500706in}}%
\pgfpathlineto{\pgfqpoint{2.206632in}{2.562375in}}%
\pgfpathlineto{\pgfqpoint{2.207495in}{2.526222in}}%
\pgfpathlineto{\pgfqpoint{2.208361in}{2.547127in}}%
\pgfpathlineto{\pgfqpoint{2.209225in}{2.466090in}}%
\pgfpathlineto{\pgfqpoint{2.210954in}{2.504026in}}%
\pgfpathlineto{\pgfqpoint{2.211819in}{2.493789in}}%
\pgfpathlineto{\pgfqpoint{2.213551in}{2.553705in}}%
\pgfpathlineto{\pgfqpoint{2.215281in}{2.523148in}}%
\pgfpathlineto{\pgfqpoint{2.217011in}{2.547312in}}%
\pgfpathlineto{\pgfqpoint{2.217875in}{2.509683in}}%
\pgfpathlineto{\pgfqpoint{2.218741in}{2.533324in}}%
\pgfpathlineto{\pgfqpoint{2.220471in}{2.512634in}}%
\pgfpathlineto{\pgfqpoint{2.221337in}{2.537934in}}%
\pgfpathlineto{\pgfqpoint{2.222202in}{2.503718in}}%
\pgfpathlineto{\pgfqpoint{2.223065in}{2.522779in}}%
\pgfpathlineto{\pgfqpoint{2.223931in}{2.485089in}}%
\pgfpathlineto{\pgfqpoint{2.224796in}{2.549770in}}%
\pgfpathlineto{\pgfqpoint{2.225661in}{2.527851in}}%
\pgfpathlineto{\pgfqpoint{2.226526in}{2.532862in}}%
\pgfpathlineto{\pgfqpoint{2.227389in}{2.550447in}}%
\pgfpathlineto{\pgfqpoint{2.228254in}{2.518751in}}%
\pgfpathlineto{\pgfqpoint{2.229120in}{2.558809in}}%
\pgfpathlineto{\pgfqpoint{2.229986in}{2.536428in}}%
\pgfpathlineto{\pgfqpoint{2.231717in}{2.580820in}}%
\pgfpathlineto{\pgfqpoint{2.233444in}{2.519028in}}%
\pgfpathlineto{\pgfqpoint{2.234309in}{2.536367in}}%
\pgfpathlineto{\pgfqpoint{2.235174in}{2.508791in}}%
\pgfpathlineto{\pgfqpoint{2.236039in}{2.557333in}}%
\pgfpathlineto{\pgfqpoint{2.238633in}{2.474882in}}%
\pgfpathlineto{\pgfqpoint{2.239497in}{2.549587in}}%
\pgfpathlineto{\pgfqpoint{2.240361in}{2.486012in}}%
\pgfpathlineto{\pgfqpoint{2.241226in}{2.486566in}}%
\pgfpathlineto{\pgfqpoint{2.242955in}{2.599021in}}%
\pgfpathlineto{\pgfqpoint{2.243820in}{2.573258in}}%
\pgfpathlineto{\pgfqpoint{2.244685in}{2.513773in}}%
\pgfpathlineto{\pgfqpoint{2.245550in}{2.556166in}}%
\pgfpathlineto{\pgfqpoint{2.246416in}{2.553890in}}%
\pgfpathlineto{\pgfqpoint{2.247281in}{2.576886in}}%
\pgfpathlineto{\pgfqpoint{2.248147in}{2.565143in}}%
\pgfpathlineto{\pgfqpoint{2.249009in}{2.571352in}}%
\pgfpathlineto{\pgfqpoint{2.249874in}{2.516416in}}%
\pgfpathlineto{\pgfqpoint{2.250740in}{2.572951in}}%
\pgfpathlineto{\pgfqpoint{2.252471in}{2.500891in}}%
\pgfpathlineto{\pgfqpoint{2.254201in}{2.545005in}}%
\pgfpathlineto{\pgfqpoint{2.255066in}{2.503780in}}%
\pgfpathlineto{\pgfqpoint{2.255930in}{2.567232in}}%
\pgfpathlineto{\pgfqpoint{2.256794in}{2.565972in}}%
\pgfpathlineto{\pgfqpoint{2.258523in}{2.555612in}}%
\pgfpathlineto{\pgfqpoint{2.259389in}{2.602217in}}%
\pgfpathlineto{\pgfqpoint{2.260254in}{2.557580in}}%
\pgfpathlineto{\pgfqpoint{2.261118in}{2.592134in}}%
\pgfpathlineto{\pgfqpoint{2.262848in}{2.536120in}}%
\pgfpathlineto{\pgfqpoint{2.263712in}{2.553305in}}%
\pgfpathlineto{\pgfqpoint{2.264577in}{2.522286in}}%
\pgfpathlineto{\pgfqpoint{2.265441in}{2.537780in}}%
\pgfpathlineto{\pgfqpoint{2.266306in}{2.583493in}}%
\pgfpathlineto{\pgfqpoint{2.267171in}{2.510050in}}%
\pgfpathlineto{\pgfqpoint{2.268032in}{2.620906in}}%
\pgfpathlineto{\pgfqpoint{2.268898in}{2.530863in}}%
\pgfpathlineto{\pgfqpoint{2.269763in}{2.561821in}}%
\pgfpathlineto{\pgfqpoint{2.271490in}{2.505624in}}%
\pgfpathlineto{\pgfqpoint{2.273219in}{2.618879in}}%
\pgfpathlineto{\pgfqpoint{2.274084in}{2.540118in}}%
\pgfpathlineto{\pgfqpoint{2.274949in}{2.619924in}}%
\pgfpathlineto{\pgfqpoint{2.277541in}{2.553151in}}%
\pgfpathlineto{\pgfqpoint{2.278407in}{2.569752in}}%
\pgfpathlineto{\pgfqpoint{2.280137in}{2.481522in}}%
\pgfpathlineto{\pgfqpoint{2.281868in}{2.584325in}}%
\pgfpathlineto{\pgfqpoint{2.282732in}{2.480232in}}%
\pgfpathlineto{\pgfqpoint{2.283598in}{2.485150in}}%
\pgfpathlineto{\pgfqpoint{2.284462in}{2.476788in}}%
\pgfpathlineto{\pgfqpoint{2.286192in}{2.504642in}}%
\pgfpathlineto{\pgfqpoint{2.287056in}{2.497263in}}%
\pgfpathlineto{\pgfqpoint{2.288787in}{2.461356in}}%
\pgfpathlineto{\pgfqpoint{2.289652in}{2.584940in}}%
\pgfpathlineto{\pgfqpoint{2.290518in}{2.524192in}}%
\pgfpathlineto{\pgfqpoint{2.291384in}{2.565818in}}%
\pgfpathlineto{\pgfqpoint{2.293114in}{2.444756in}}%
\pgfpathlineto{\pgfqpoint{2.293978in}{2.589735in}}%
\pgfpathlineto{\pgfqpoint{2.294842in}{2.506117in}}%
\pgfpathlineto{\pgfqpoint{2.295707in}{2.520872in}}%
\pgfpathlineto{\pgfqpoint{2.297435in}{2.497324in}}%
\pgfpathlineto{\pgfqpoint{2.298302in}{2.609902in}}%
\pgfpathlineto{\pgfqpoint{2.299167in}{2.485150in}}%
\pgfpathlineto{\pgfqpoint{2.300033in}{2.608180in}}%
\pgfpathlineto{\pgfqpoint{2.300900in}{2.545036in}}%
\pgfpathlineto{\pgfqpoint{2.301766in}{2.598220in}}%
\pgfpathlineto{\pgfqpoint{2.304361in}{2.519674in}}%
\pgfpathlineto{\pgfqpoint{2.306957in}{2.547004in}}%
\pgfpathlineto{\pgfqpoint{2.307822in}{2.506486in}}%
\pgfpathlineto{\pgfqpoint{2.308688in}{2.516446in}}%
\pgfpathlineto{\pgfqpoint{2.309554in}{2.516262in}}%
\pgfpathlineto{\pgfqpoint{2.310420in}{2.518660in}}%
\pgfpathlineto{\pgfqpoint{2.311285in}{2.488993in}}%
\pgfpathlineto{\pgfqpoint{2.313017in}{2.581066in}}%
\pgfpathlineto{\pgfqpoint{2.313883in}{2.493635in}}%
\pgfpathlineto{\pgfqpoint{2.314749in}{2.515831in}}%
\pgfpathlineto{\pgfqpoint{2.315613in}{2.582972in}}%
\pgfpathlineto{\pgfqpoint{2.317342in}{2.498677in}}%
\pgfpathlineto{\pgfqpoint{2.318206in}{2.502304in}}%
\pgfpathlineto{\pgfqpoint{2.319937in}{2.566372in}}%
\pgfpathlineto{\pgfqpoint{2.320802in}{2.546144in}}%
\pgfpathlineto{\pgfqpoint{2.321668in}{2.557580in}}%
\pgfpathlineto{\pgfqpoint{2.322533in}{2.550139in}}%
\pgfpathlineto{\pgfqpoint{2.323399in}{2.572643in}}%
\pgfpathlineto{\pgfqpoint{2.325993in}{2.469717in}}%
\pgfpathlineto{\pgfqpoint{2.327723in}{2.553890in}}%
\pgfpathlineto{\pgfqpoint{2.328589in}{2.506117in}}%
\pgfpathlineto{\pgfqpoint{2.329452in}{2.625889in}}%
\pgfpathlineto{\pgfqpoint{2.331179in}{2.508146in}}%
\pgfpathlineto{\pgfqpoint{2.332043in}{2.520567in}}%
\pgfpathlineto{\pgfqpoint{2.332908in}{2.501814in}}%
\pgfpathlineto{\pgfqpoint{2.333775in}{2.519091in}}%
\pgfpathlineto{\pgfqpoint{2.334640in}{2.513188in}}%
\pgfpathlineto{\pgfqpoint{2.335506in}{2.516416in}}%
\pgfpathlineto{\pgfqpoint{2.337235in}{2.586600in}}%
\pgfpathlineto{\pgfqpoint{2.338965in}{2.486749in}}%
\pgfpathlineto{\pgfqpoint{2.339831in}{2.495295in}}%
\pgfpathlineto{\pgfqpoint{2.340692in}{2.537105in}}%
\pgfpathlineto{\pgfqpoint{2.341556in}{2.522717in}}%
\pgfpathlineto{\pgfqpoint{2.342422in}{2.454500in}}%
\pgfpathlineto{\pgfqpoint{2.343286in}{2.531879in}}%
\pgfpathlineto{\pgfqpoint{2.345881in}{2.471685in}}%
\pgfpathlineto{\pgfqpoint{2.346747in}{2.497817in}}%
\pgfpathlineto{\pgfqpoint{2.347612in}{2.494558in}}%
\pgfpathlineto{\pgfqpoint{2.348476in}{2.481645in}}%
\pgfpathlineto{\pgfqpoint{2.349341in}{2.491606in}}%
\pgfpathlineto{\pgfqpoint{2.350207in}{2.544420in}}%
\pgfpathlineto{\pgfqpoint{2.351073in}{2.519335in}}%
\pgfpathlineto{\pgfqpoint{2.351939in}{2.455668in}}%
\pgfpathlineto{\pgfqpoint{2.353668in}{2.531202in}}%
\pgfpathlineto{\pgfqpoint{2.354534in}{2.504241in}}%
\pgfpathlineto{\pgfqpoint{2.356262in}{2.552292in}}%
\pgfpathlineto{\pgfqpoint{2.357127in}{2.497386in}}%
\pgfpathlineto{\pgfqpoint{2.358859in}{2.565818in}}%
\pgfpathlineto{\pgfqpoint{2.359724in}{2.492221in}}%
\pgfpathlineto{\pgfqpoint{2.361451in}{2.571198in}}%
\pgfpathlineto{\pgfqpoint{2.362316in}{2.547127in}}%
\pgfpathlineto{\pgfqpoint{2.363181in}{2.566372in}}%
\pgfpathlineto{\pgfqpoint{2.364045in}{2.489055in}}%
\pgfpathlineto{\pgfqpoint{2.364909in}{2.576825in}}%
\pgfpathlineto{\pgfqpoint{2.365771in}{2.559424in}}%
\pgfpathlineto{\pgfqpoint{2.366638in}{2.538519in}}%
\pgfpathlineto{\pgfqpoint{2.367503in}{2.490069in}}%
\pgfpathlineto{\pgfqpoint{2.368366in}{2.495141in}}%
\pgfpathlineto{\pgfqpoint{2.370096in}{2.565880in}}%
\pgfpathlineto{\pgfqpoint{2.371826in}{2.518045in}}%
\pgfpathlineto{\pgfqpoint{2.372691in}{2.543499in}}%
\pgfpathlineto{\pgfqpoint{2.373557in}{2.492221in}}%
\pgfpathlineto{\pgfqpoint{2.375287in}{2.569077in}}%
\pgfpathlineto{\pgfqpoint{2.376151in}{2.577808in}}%
\pgfpathlineto{\pgfqpoint{2.377017in}{2.603508in}}%
\pgfpathlineto{\pgfqpoint{2.379616in}{2.537536in}}%
\pgfpathlineto{\pgfqpoint{2.380481in}{2.599421in}}%
\pgfpathlineto{\pgfqpoint{2.383076in}{2.525577in}}%
\pgfpathlineto{\pgfqpoint{2.383941in}{2.566803in}}%
\pgfpathlineto{\pgfqpoint{2.385670in}{2.453856in}}%
\pgfpathlineto{\pgfqpoint{2.386535in}{2.561515in}}%
\pgfpathlineto{\pgfqpoint{2.387399in}{2.501168in}}%
\pgfpathlineto{\pgfqpoint{2.388265in}{2.516139in}}%
\pgfpathlineto{\pgfqpoint{2.389131in}{2.567724in}}%
\pgfpathlineto{\pgfqpoint{2.389995in}{2.557857in}}%
\pgfpathlineto{\pgfqpoint{2.390860in}{2.546452in}}%
\pgfpathlineto{\pgfqpoint{2.391725in}{2.555306in}}%
\pgfpathlineto{\pgfqpoint{2.392590in}{2.544669in}}%
\pgfpathlineto{\pgfqpoint{2.393455in}{2.571229in}}%
\pgfpathlineto{\pgfqpoint{2.394320in}{2.565143in}}%
\pgfpathlineto{\pgfqpoint{2.396915in}{2.494189in}}%
\pgfpathlineto{\pgfqpoint{2.397781in}{2.548787in}}%
\pgfpathlineto{\pgfqpoint{2.398647in}{2.543807in}}%
\pgfpathlineto{\pgfqpoint{2.399511in}{2.464737in}}%
\pgfpathlineto{\pgfqpoint{2.400376in}{2.605476in}}%
\pgfpathlineto{\pgfqpoint{2.402105in}{2.529911in}}%
\pgfpathlineto{\pgfqpoint{2.403834in}{2.584940in}}%
\pgfpathlineto{\pgfqpoint{2.404699in}{2.550324in}}%
\pgfpathlineto{\pgfqpoint{2.405564in}{2.595393in}}%
\pgfpathlineto{\pgfqpoint{2.406429in}{2.484167in}}%
\pgfpathlineto{\pgfqpoint{2.408157in}{2.535199in}}%
\pgfpathlineto{\pgfqpoint{2.409022in}{2.542547in}}%
\pgfpathlineto{\pgfqpoint{2.409887in}{2.518168in}}%
\pgfpathlineto{\pgfqpoint{2.410753in}{2.545528in}}%
\pgfpathlineto{\pgfqpoint{2.411618in}{2.540025in}}%
\pgfpathlineto{\pgfqpoint{2.412483in}{2.545467in}}%
\pgfpathlineto{\pgfqpoint{2.414213in}{2.507838in}}%
\pgfpathlineto{\pgfqpoint{2.415079in}{2.557518in}}%
\pgfpathlineto{\pgfqpoint{2.415943in}{2.509868in}}%
\pgfpathlineto{\pgfqpoint{2.416808in}{2.573012in}}%
\pgfpathlineto{\pgfqpoint{2.417672in}{2.545159in}}%
\pgfpathlineto{\pgfqpoint{2.419402in}{2.596622in}}%
\pgfpathlineto{\pgfqpoint{2.421131in}{2.528097in}}%
\pgfpathlineto{\pgfqpoint{2.421995in}{2.555920in}}%
\pgfpathlineto{\pgfqpoint{2.423725in}{2.528621in}}%
\pgfpathlineto{\pgfqpoint{2.424590in}{2.575288in}}%
\pgfpathlineto{\pgfqpoint{2.425455in}{2.515463in}}%
\pgfpathlineto{\pgfqpoint{2.427184in}{2.568278in}}%
\pgfpathlineto{\pgfqpoint{2.428912in}{2.473838in}}%
\pgfpathlineto{\pgfqpoint{2.429777in}{2.533724in}}%
\pgfpathlineto{\pgfqpoint{2.430641in}{2.503105in}}%
\pgfpathlineto{\pgfqpoint{2.432370in}{2.563421in}}%
\pgfpathlineto{\pgfqpoint{2.433233in}{2.470272in}}%
\pgfpathlineto{\pgfqpoint{2.435826in}{2.575165in}}%
\pgfpathlineto{\pgfqpoint{2.436690in}{2.580637in}}%
\pgfpathlineto{\pgfqpoint{2.437555in}{2.524041in}}%
\pgfpathlineto{\pgfqpoint{2.438420in}{2.527022in}}%
\pgfpathlineto{\pgfqpoint{2.439284in}{2.535568in}}%
\pgfpathlineto{\pgfqpoint{2.440150in}{2.566341in}}%
\pgfpathlineto{\pgfqpoint{2.441015in}{2.514848in}}%
\pgfpathlineto{\pgfqpoint{2.441880in}{2.528066in}}%
\pgfpathlineto{\pgfqpoint{2.442744in}{2.481707in}}%
\pgfpathlineto{\pgfqpoint{2.443608in}{2.507962in}}%
\pgfpathlineto{\pgfqpoint{2.444474in}{2.482323in}}%
\pgfpathlineto{\pgfqpoint{2.445339in}{2.550816in}}%
\pgfpathlineto{\pgfqpoint{2.446201in}{2.545098in}}%
\pgfpathlineto{\pgfqpoint{2.447066in}{2.503595in}}%
\pgfpathlineto{\pgfqpoint{2.448795in}{2.553644in}}%
\pgfpathlineto{\pgfqpoint{2.449658in}{2.531202in}}%
\pgfpathlineto{\pgfqpoint{2.450524in}{2.577623in}}%
\pgfpathlineto{\pgfqpoint{2.452253in}{2.523210in}}%
\pgfpathlineto{\pgfqpoint{2.453117in}{2.556533in}}%
\pgfpathlineto{\pgfqpoint{2.453982in}{2.507654in}}%
\pgfpathlineto{\pgfqpoint{2.454848in}{2.562683in}}%
\pgfpathlineto{\pgfqpoint{2.455713in}{2.528066in}}%
\pgfpathlineto{\pgfqpoint{2.457441in}{2.555612in}}%
\pgfpathlineto{\pgfqpoint{2.458306in}{2.524993in}}%
\pgfpathlineto{\pgfqpoint{2.460035in}{2.576578in}}%
\pgfpathlineto{\pgfqpoint{2.461764in}{2.545652in}}%
\pgfpathlineto{\pgfqpoint{2.463494in}{2.564866in}}%
\pgfpathlineto{\pgfqpoint{2.464360in}{2.547927in}}%
\pgfpathlineto{\pgfqpoint{2.465225in}{2.568032in}}%
\pgfpathlineto{\pgfqpoint{2.466090in}{2.493850in}}%
\pgfpathlineto{\pgfqpoint{2.467821in}{2.581743in}}%
\pgfpathlineto{\pgfqpoint{2.469550in}{2.552969in}}%
\pgfpathlineto{\pgfqpoint{2.470414in}{2.611256in}}%
\pgfpathlineto{\pgfqpoint{2.473005in}{2.482507in}}%
\pgfpathlineto{\pgfqpoint{2.473871in}{2.538519in}}%
\pgfpathlineto{\pgfqpoint{2.474737in}{2.533662in}}%
\pgfpathlineto{\pgfqpoint{2.475602in}{2.531787in}}%
\pgfpathlineto{\pgfqpoint{2.476468in}{2.481707in}}%
\pgfpathlineto{\pgfqpoint{2.478200in}{2.558593in}}%
\pgfpathlineto{\pgfqpoint{2.479066in}{2.529480in}}%
\pgfpathlineto{\pgfqpoint{2.479931in}{2.575103in}}%
\pgfpathlineto{\pgfqpoint{2.480797in}{2.548910in}}%
\pgfpathlineto{\pgfqpoint{2.481662in}{2.600742in}}%
\pgfpathlineto{\pgfqpoint{2.482527in}{2.552322in}}%
\pgfpathlineto{\pgfqpoint{2.483391in}{2.593856in}}%
\pgfpathlineto{\pgfqpoint{2.485985in}{2.539196in}}%
\pgfpathlineto{\pgfqpoint{2.486849in}{2.617219in}}%
\pgfpathlineto{\pgfqpoint{2.487715in}{2.605599in}}%
\pgfpathlineto{\pgfqpoint{2.488581in}{2.581805in}}%
\pgfpathlineto{\pgfqpoint{2.489447in}{2.623184in}}%
\pgfpathlineto{\pgfqpoint{2.490310in}{2.549587in}}%
\pgfpathlineto{\pgfqpoint{2.491175in}{2.589859in}}%
\pgfpathlineto{\pgfqpoint{2.492905in}{2.512574in}}%
\pgfpathlineto{\pgfqpoint{2.493769in}{2.578669in}}%
\pgfpathlineto{\pgfqpoint{2.494635in}{2.528713in}}%
\pgfpathlineto{\pgfqpoint{2.495501in}{2.585740in}}%
\pgfpathlineto{\pgfqpoint{2.496365in}{2.576825in}}%
\pgfpathlineto{\pgfqpoint{2.497231in}{2.479064in}}%
\pgfpathlineto{\pgfqpoint{2.498962in}{2.592627in}}%
\pgfpathlineto{\pgfqpoint{2.499826in}{2.551063in}}%
\pgfpathlineto{\pgfqpoint{2.500692in}{2.565882in}}%
\pgfpathlineto{\pgfqpoint{2.501555in}{2.485612in}}%
\pgfpathlineto{\pgfqpoint{2.503285in}{2.583649in}}%
\pgfpathlineto{\pgfqpoint{2.504149in}{2.511589in}}%
\pgfpathlineto{\pgfqpoint{2.505880in}{2.553215in}}%
\pgfpathlineto{\pgfqpoint{2.507611in}{2.511651in}}%
\pgfpathlineto{\pgfqpoint{2.508476in}{2.546513in}}%
\pgfpathlineto{\pgfqpoint{2.509340in}{2.522411in}}%
\pgfpathlineto{\pgfqpoint{2.510205in}{2.584327in}}%
\pgfpathlineto{\pgfqpoint{2.511069in}{2.470949in}}%
\pgfpathlineto{\pgfqpoint{2.511934in}{2.471010in}}%
\pgfpathlineto{\pgfqpoint{2.512798in}{2.536061in}}%
\pgfpathlineto{\pgfqpoint{2.513664in}{2.464186in}}%
\pgfpathlineto{\pgfqpoint{2.514530in}{2.591398in}}%
\pgfpathlineto{\pgfqpoint{2.515396in}{2.546144in}}%
\pgfpathlineto{\pgfqpoint{2.516260in}{2.549587in}}%
\pgfpathlineto{\pgfqpoint{2.517125in}{2.527515in}}%
\pgfpathlineto{\pgfqpoint{2.517990in}{2.561330in}}%
\pgfpathlineto{\pgfqpoint{2.518855in}{2.459082in}}%
\pgfpathlineto{\pgfqpoint{2.520585in}{2.544638in}}%
\pgfpathlineto{\pgfqpoint{2.521451in}{2.482323in}}%
\pgfpathlineto{\pgfqpoint{2.522317in}{2.559917in}}%
\pgfpathlineto{\pgfqpoint{2.523180in}{2.501937in}}%
\pgfpathlineto{\pgfqpoint{2.524912in}{2.579652in}}%
\pgfpathlineto{\pgfqpoint{2.526643in}{2.510176in}}%
\pgfpathlineto{\pgfqpoint{2.527509in}{2.581989in}}%
\pgfpathlineto{\pgfqpoint{2.528374in}{2.506794in}}%
\pgfpathlineto{\pgfqpoint{2.529236in}{2.541226in}}%
\pgfpathlineto{\pgfqpoint{2.530966in}{2.484537in}}%
\pgfpathlineto{\pgfqpoint{2.531831in}{2.520505in}}%
\pgfpathlineto{\pgfqpoint{2.532696in}{2.486566in}}%
\pgfpathlineto{\pgfqpoint{2.534427in}{2.556597in}}%
\pgfpathlineto{\pgfqpoint{2.535290in}{2.495051in}}%
\pgfpathlineto{\pgfqpoint{2.536155in}{2.529421in}}%
\pgfpathlineto{\pgfqpoint{2.537021in}{2.459452in}}%
\pgfpathlineto{\pgfqpoint{2.538751in}{2.531879in}}%
\pgfpathlineto{\pgfqpoint{2.539618in}{2.507685in}}%
\pgfpathlineto{\pgfqpoint{2.540482in}{2.516754in}}%
\pgfpathlineto{\pgfqpoint{2.541346in}{2.458775in}}%
\pgfpathlineto{\pgfqpoint{2.542211in}{2.467475in}}%
\pgfpathlineto{\pgfqpoint{2.543942in}{2.544792in}}%
\pgfpathlineto{\pgfqpoint{2.544808in}{2.554783in}}%
\pgfpathlineto{\pgfqpoint{2.545672in}{2.475744in}}%
\pgfpathlineto{\pgfqpoint{2.546537in}{2.566741in}}%
\pgfpathlineto{\pgfqpoint{2.547403in}{2.501937in}}%
\pgfpathlineto{\pgfqpoint{2.548266in}{2.592442in}}%
\pgfpathlineto{\pgfqpoint{2.549131in}{2.521980in}}%
\pgfpathlineto{\pgfqpoint{2.549993in}{2.534647in}}%
\pgfpathlineto{\pgfqpoint{2.550858in}{2.541780in}}%
\pgfpathlineto{\pgfqpoint{2.551724in}{2.512359in}}%
\pgfpathlineto{\pgfqpoint{2.552590in}{2.516202in}}%
\pgfpathlineto{\pgfqpoint{2.554322in}{2.495112in}}%
\pgfpathlineto{\pgfqpoint{2.555188in}{2.501445in}}%
\pgfpathlineto{\pgfqpoint{2.556051in}{2.541287in}}%
\pgfpathlineto{\pgfqpoint{2.556916in}{2.539258in}}%
\pgfpathlineto{\pgfqpoint{2.557782in}{2.475867in}}%
\pgfpathlineto{\pgfqpoint{2.560377in}{2.577133in}}%
\pgfpathlineto{\pgfqpoint{2.562106in}{2.487980in}}%
\pgfpathlineto{\pgfqpoint{2.564700in}{2.577071in}}%
\pgfpathlineto{\pgfqpoint{2.566428in}{2.516508in}}%
\pgfpathlineto{\pgfqpoint{2.567292in}{2.612424in}}%
\pgfpathlineto{\pgfqpoint{2.568157in}{2.563021in}}%
\pgfpathlineto{\pgfqpoint{2.569022in}{2.585802in}}%
\pgfpathlineto{\pgfqpoint{2.569888in}{2.580514in}}%
\pgfpathlineto{\pgfqpoint{2.570753in}{2.568525in}}%
\pgfpathlineto{\pgfqpoint{2.571618in}{2.582359in}}%
\pgfpathlineto{\pgfqpoint{2.572482in}{2.557028in}}%
\pgfpathlineto{\pgfqpoint{2.573346in}{2.491823in}}%
\pgfpathlineto{\pgfqpoint{2.574211in}{2.592627in}}%
\pgfpathlineto{\pgfqpoint{2.575076in}{2.538950in}}%
\pgfpathlineto{\pgfqpoint{2.575939in}{2.543684in}}%
\pgfpathlineto{\pgfqpoint{2.576802in}{2.568217in}}%
\pgfpathlineto{\pgfqpoint{2.577665in}{2.555827in}}%
\pgfpathlineto{\pgfqpoint{2.578530in}{2.526960in}}%
\pgfpathlineto{\pgfqpoint{2.579394in}{2.547681in}}%
\pgfpathlineto{\pgfqpoint{2.580257in}{2.487857in}}%
\pgfpathlineto{\pgfqpoint{2.581123in}{2.552907in}}%
\pgfpathlineto{\pgfqpoint{2.582854in}{2.505011in}}%
\pgfpathlineto{\pgfqpoint{2.583719in}{2.515710in}}%
\pgfpathlineto{\pgfqpoint{2.584583in}{2.582113in}}%
\pgfpathlineto{\pgfqpoint{2.585448in}{2.473776in}}%
\pgfpathlineto{\pgfqpoint{2.586313in}{2.559486in}}%
\pgfpathlineto{\pgfqpoint{2.587177in}{2.523671in}}%
\pgfpathlineto{\pgfqpoint{2.588042in}{2.533908in}}%
\pgfpathlineto{\pgfqpoint{2.588907in}{2.533785in}}%
\pgfpathlineto{\pgfqpoint{2.590637in}{2.503043in}}%
\pgfpathlineto{\pgfqpoint{2.591500in}{2.496588in}}%
\pgfpathlineto{\pgfqpoint{2.592365in}{2.462279in}}%
\pgfpathlineto{\pgfqpoint{2.593230in}{2.537598in}}%
\pgfpathlineto{\pgfqpoint{2.594095in}{2.531941in}}%
\pgfpathlineto{\pgfqpoint{2.594961in}{2.506425in}}%
\pgfpathlineto{\pgfqpoint{2.595827in}{2.547619in}}%
\pgfpathlineto{\pgfqpoint{2.596692in}{2.528190in}}%
\pgfpathlineto{\pgfqpoint{2.598421in}{2.606584in}}%
\pgfpathlineto{\pgfqpoint{2.599286in}{2.506579in}}%
\pgfpathlineto{\pgfqpoint{2.601015in}{2.592688in}}%
\pgfpathlineto{\pgfqpoint{2.601880in}{2.512236in}}%
\pgfpathlineto{\pgfqpoint{2.602742in}{2.565205in}}%
\pgfpathlineto{\pgfqpoint{2.604469in}{2.455270in}}%
\pgfpathlineto{\pgfqpoint{2.605334in}{2.539627in}}%
\pgfpathlineto{\pgfqpoint{2.606197in}{2.474423in}}%
\pgfpathlineto{\pgfqpoint{2.607063in}{2.536615in}}%
\pgfpathlineto{\pgfqpoint{2.607928in}{2.532679in}}%
\pgfpathlineto{\pgfqpoint{2.608793in}{2.542947in}}%
\pgfpathlineto{\pgfqpoint{2.609657in}{2.541841in}}%
\pgfpathlineto{\pgfqpoint{2.610521in}{2.616852in}}%
\pgfpathlineto{\pgfqpoint{2.611387in}{2.495266in}}%
\pgfpathlineto{\pgfqpoint{2.612252in}{2.539135in}}%
\pgfpathlineto{\pgfqpoint{2.613117in}{2.470949in}}%
\pgfpathlineto{\pgfqpoint{2.613981in}{2.539135in}}%
\pgfpathlineto{\pgfqpoint{2.614846in}{2.469042in}}%
\pgfpathlineto{\pgfqpoint{2.618301in}{2.559794in}}%
\pgfpathlineto{\pgfqpoint{2.619166in}{2.492590in}}%
\pgfpathlineto{\pgfqpoint{2.620031in}{2.506609in}}%
\pgfpathlineto{\pgfqpoint{2.620894in}{2.511620in}}%
\pgfpathlineto{\pgfqpoint{2.622625in}{2.578300in}}%
\pgfpathlineto{\pgfqpoint{2.625219in}{2.455978in}}%
\pgfpathlineto{\pgfqpoint{2.626084in}{2.442113in}}%
\pgfpathlineto{\pgfqpoint{2.628682in}{2.556412in}}%
\pgfpathlineto{\pgfqpoint{2.629548in}{2.549895in}}%
\pgfpathlineto{\pgfqpoint{2.630413in}{2.499754in}}%
\pgfpathlineto{\pgfqpoint{2.631279in}{2.513927in}}%
\pgfpathlineto{\pgfqpoint{2.632145in}{2.549895in}}%
\pgfpathlineto{\pgfqpoint{2.634739in}{2.515094in}}%
\pgfpathlineto{\pgfqpoint{2.635603in}{2.561330in}}%
\pgfpathlineto{\pgfqpoint{2.636470in}{2.546575in}}%
\pgfpathlineto{\pgfqpoint{2.637336in}{2.554506in}}%
\pgfpathlineto{\pgfqpoint{2.638200in}{2.586662in}}%
\pgfpathlineto{\pgfqpoint{2.639929in}{2.501260in}}%
\pgfpathlineto{\pgfqpoint{2.641660in}{2.572889in}}%
\pgfpathlineto{\pgfqpoint{2.642524in}{2.565572in}}%
\pgfpathlineto{\pgfqpoint{2.645115in}{2.483613in}}%
\pgfpathlineto{\pgfqpoint{2.647710in}{2.574549in}}%
\pgfpathlineto{\pgfqpoint{2.648575in}{2.540795in}}%
\pgfpathlineto{\pgfqpoint{2.649441in}{2.546850in}}%
\pgfpathlineto{\pgfqpoint{2.652037in}{2.508146in}}%
\pgfpathlineto{\pgfqpoint{2.652902in}{2.538581in}}%
\pgfpathlineto{\pgfqpoint{2.653767in}{2.526714in}}%
\pgfpathlineto{\pgfqpoint{2.654633in}{2.465107in}}%
\pgfpathlineto{\pgfqpoint{2.655499in}{2.480970in}}%
\pgfpathlineto{\pgfqpoint{2.656365in}{2.552599in}}%
\pgfpathlineto{\pgfqpoint{2.657231in}{2.492590in}}%
\pgfpathlineto{\pgfqpoint{2.659827in}{2.588260in}}%
\pgfpathlineto{\pgfqpoint{2.660689in}{2.559609in}}%
\pgfpathlineto{\pgfqpoint{2.661555in}{2.497571in}}%
\pgfpathlineto{\pgfqpoint{2.662417in}{2.566126in}}%
\pgfpathlineto{\pgfqpoint{2.663283in}{2.500521in}}%
\pgfpathlineto{\pgfqpoint{2.665014in}{2.556104in}}%
\pgfpathlineto{\pgfqpoint{2.666743in}{2.458344in}}%
\pgfpathlineto{\pgfqpoint{2.667607in}{2.542085in}}%
\pgfpathlineto{\pgfqpoint{2.668473in}{2.504919in}}%
\pgfpathlineto{\pgfqpoint{2.669338in}{2.531571in}}%
\pgfpathlineto{\pgfqpoint{2.670204in}{2.487303in}}%
\pgfpathlineto{\pgfqpoint{2.671069in}{2.550816in}}%
\pgfpathlineto{\pgfqpoint{2.671934in}{2.540241in}}%
\pgfpathlineto{\pgfqpoint{2.672800in}{2.552815in}}%
\pgfpathlineto{\pgfqpoint{2.673666in}{2.515217in}}%
\pgfpathlineto{\pgfqpoint{2.674531in}{2.574734in}}%
\pgfpathlineto{\pgfqpoint{2.675397in}{2.506917in}}%
\pgfpathlineto{\pgfqpoint{2.676261in}{2.551370in}}%
\pgfpathlineto{\pgfqpoint{2.677991in}{2.535137in}}%
\pgfpathlineto{\pgfqpoint{2.678857in}{2.576948in}}%
\pgfpathlineto{\pgfqpoint{2.679722in}{2.563298in}}%
\pgfpathlineto{\pgfqpoint{2.680587in}{2.504518in}}%
\pgfpathlineto{\pgfqpoint{2.682316in}{2.591950in}}%
\pgfpathlineto{\pgfqpoint{2.683182in}{2.506486in}}%
\pgfpathlineto{\pgfqpoint{2.684047in}{2.578300in}}%
\pgfpathlineto{\pgfqpoint{2.684912in}{2.505288in}}%
\pgfpathlineto{\pgfqpoint{2.685777in}{2.522042in}}%
\pgfpathlineto{\pgfqpoint{2.686642in}{2.586969in}}%
\pgfpathlineto{\pgfqpoint{2.687507in}{2.554444in}}%
\pgfpathlineto{\pgfqpoint{2.688372in}{2.574672in}}%
\pgfpathlineto{\pgfqpoint{2.690102in}{2.513434in}}%
\pgfpathlineto{\pgfqpoint{2.690968in}{2.523456in}}%
\pgfpathlineto{\pgfqpoint{2.691834in}{2.574028in}}%
\pgfpathlineto{\pgfqpoint{2.692698in}{2.572828in}}%
\pgfpathlineto{\pgfqpoint{2.694429in}{2.584203in}}%
\pgfpathlineto{\pgfqpoint{2.697023in}{2.490900in}}%
\pgfpathlineto{\pgfqpoint{2.697889in}{2.501321in}}%
\pgfpathlineto{\pgfqpoint{2.698752in}{2.515402in}}%
\pgfpathlineto{\pgfqpoint{2.699618in}{2.497878in}}%
\pgfpathlineto{\pgfqpoint{2.700483in}{2.527391in}}%
\pgfpathlineto{\pgfqpoint{2.701346in}{2.517000in}}%
\pgfpathlineto{\pgfqpoint{2.702209in}{2.481278in}}%
\pgfpathlineto{\pgfqpoint{2.703074in}{2.487918in}}%
\pgfpathlineto{\pgfqpoint{2.703940in}{2.499908in}}%
\pgfpathlineto{\pgfqpoint{2.704806in}{2.493021in}}%
\pgfpathlineto{\pgfqpoint{2.705670in}{2.562436in}}%
\pgfpathlineto{\pgfqpoint{2.707401in}{2.526899in}}%
\pgfpathlineto{\pgfqpoint{2.708268in}{2.540672in}}%
\pgfpathlineto{\pgfqpoint{2.709134in}{2.524195in}}%
\pgfpathlineto{\pgfqpoint{2.709995in}{2.600742in}}%
\pgfpathlineto{\pgfqpoint{2.712592in}{2.506486in}}%
\pgfpathlineto{\pgfqpoint{2.713458in}{2.498032in}}%
\pgfpathlineto{\pgfqpoint{2.715189in}{2.577440in}}%
\pgfpathlineto{\pgfqpoint{2.716054in}{2.534493in}}%
\pgfpathlineto{\pgfqpoint{2.716920in}{2.581251in}}%
\pgfpathlineto{\pgfqpoint{2.718650in}{2.546144in}}%
\pgfpathlineto{\pgfqpoint{2.719515in}{2.559363in}}%
\pgfpathlineto{\pgfqpoint{2.720379in}{2.479433in}}%
\pgfpathlineto{\pgfqpoint{2.721245in}{2.563175in}}%
\pgfpathlineto{\pgfqpoint{2.722110in}{2.513249in}}%
\pgfpathlineto{\pgfqpoint{2.723839in}{2.568709in}}%
\pgfpathlineto{\pgfqpoint{2.724705in}{2.552230in}}%
\pgfpathlineto{\pgfqpoint{2.725571in}{2.488011in}}%
\pgfpathlineto{\pgfqpoint{2.726436in}{2.498371in}}%
\pgfpathlineto{\pgfqpoint{2.727301in}{2.558626in}}%
\pgfpathlineto{\pgfqpoint{2.729031in}{2.477527in}}%
\pgfpathlineto{\pgfqpoint{2.730761in}{2.530465in}}%
\pgfpathlineto{\pgfqpoint{2.731626in}{2.512880in}}%
\pgfpathlineto{\pgfqpoint{2.732492in}{2.453025in}}%
\pgfpathlineto{\pgfqpoint{2.734222in}{2.583834in}}%
\pgfpathlineto{\pgfqpoint{2.735950in}{2.530588in}}%
\pgfpathlineto{\pgfqpoint{2.736813in}{2.575411in}}%
\pgfpathlineto{\pgfqpoint{2.738539in}{2.485704in}}%
\pgfpathlineto{\pgfqpoint{2.739404in}{2.511282in}}%
\pgfpathlineto{\pgfqpoint{2.741132in}{2.598467in}}%
\pgfpathlineto{\pgfqpoint{2.741997in}{2.485458in}}%
\pgfpathlineto{\pgfqpoint{2.743725in}{2.547127in}}%
\pgfpathlineto{\pgfqpoint{2.744591in}{2.538919in}}%
\pgfpathlineto{\pgfqpoint{2.745455in}{2.519889in}}%
\pgfpathlineto{\pgfqpoint{2.746322in}{2.549218in}}%
\pgfpathlineto{\pgfqpoint{2.747187in}{2.537598in}}%
\pgfpathlineto{\pgfqpoint{2.748053in}{2.480909in}}%
\pgfpathlineto{\pgfqpoint{2.749784in}{2.559609in}}%
\pgfpathlineto{\pgfqpoint{2.750649in}{2.547127in}}%
\pgfpathlineto{\pgfqpoint{2.751515in}{2.511559in}}%
\pgfpathlineto{\pgfqpoint{2.752380in}{2.583588in}}%
\pgfpathlineto{\pgfqpoint{2.754111in}{2.494435in}}%
\pgfpathlineto{\pgfqpoint{2.754976in}{2.540672in}}%
\pgfpathlineto{\pgfqpoint{2.755840in}{2.492960in}}%
\pgfpathlineto{\pgfqpoint{2.756706in}{2.501198in}}%
\pgfpathlineto{\pgfqpoint{2.757572in}{2.503166in}}%
\pgfpathlineto{\pgfqpoint{2.758437in}{2.497755in}}%
\pgfpathlineto{\pgfqpoint{2.760166in}{2.509068in}}%
\pgfpathlineto{\pgfqpoint{2.761031in}{2.542024in}}%
\pgfpathlineto{\pgfqpoint{2.761898in}{2.472362in}}%
\pgfpathlineto{\pgfqpoint{2.762764in}{2.486012in}}%
\pgfpathlineto{\pgfqpoint{2.763629in}{2.485150in}}%
\pgfpathlineto{\pgfqpoint{2.764491in}{2.541962in}}%
\pgfpathlineto{\pgfqpoint{2.765356in}{2.479801in}}%
\pgfpathlineto{\pgfqpoint{2.766220in}{2.490192in}}%
\pgfpathlineto{\pgfqpoint{2.767951in}{2.574426in}}%
\pgfpathlineto{\pgfqpoint{2.768815in}{2.565757in}}%
\pgfpathlineto{\pgfqpoint{2.770546in}{2.473191in}}%
\pgfpathlineto{\pgfqpoint{2.772275in}{2.533047in}}%
\pgfpathlineto{\pgfqpoint{2.773139in}{2.507284in}}%
\pgfpathlineto{\pgfqpoint{2.774004in}{2.552846in}}%
\pgfpathlineto{\pgfqpoint{2.775728in}{2.449551in}}%
\pgfpathlineto{\pgfqpoint{2.776593in}{2.534155in}}%
\pgfpathlineto{\pgfqpoint{2.777458in}{2.500092in}}%
\pgfpathlineto{\pgfqpoint{2.778324in}{2.504765in}}%
\pgfpathlineto{\pgfqpoint{2.779188in}{2.552907in}}%
\pgfpathlineto{\pgfqpoint{2.780052in}{2.521980in}}%
\pgfpathlineto{\pgfqpoint{2.781780in}{2.548112in}}%
\pgfpathlineto{\pgfqpoint{2.782645in}{2.517339in}}%
\pgfpathlineto{\pgfqpoint{2.783507in}{2.558134in}}%
\pgfpathlineto{\pgfqpoint{2.785235in}{2.528928in}}%
\pgfpathlineto{\pgfqpoint{2.786965in}{2.595270in}}%
\pgfpathlineto{\pgfqpoint{2.787828in}{2.462095in}}%
\pgfpathlineto{\pgfqpoint{2.788692in}{2.587277in}}%
\pgfpathlineto{\pgfqpoint{2.789556in}{2.531081in}}%
\pgfpathlineto{\pgfqpoint{2.790421in}{2.565205in}}%
\pgfpathlineto{\pgfqpoint{2.792151in}{2.529975in}}%
\pgfpathlineto{\pgfqpoint{2.793017in}{2.505134in}}%
\pgfpathlineto{\pgfqpoint{2.793883in}{2.544238in}}%
\pgfpathlineto{\pgfqpoint{2.794748in}{2.524071in}}%
\pgfpathlineto{\pgfqpoint{2.795613in}{2.544792in}}%
\pgfpathlineto{\pgfqpoint{2.797343in}{2.498248in}}%
\pgfpathlineto{\pgfqpoint{2.798208in}{2.534401in}}%
\pgfpathlineto{\pgfqpoint{2.799072in}{2.534339in}}%
\pgfpathlineto{\pgfqpoint{2.799938in}{2.554506in}}%
\pgfpathlineto{\pgfqpoint{2.801665in}{2.504888in}}%
\pgfpathlineto{\pgfqpoint{2.802530in}{2.502368in}}%
\pgfpathlineto{\pgfqpoint{2.804259in}{2.548420in}}%
\pgfpathlineto{\pgfqpoint{2.805124in}{2.513496in}}%
\pgfpathlineto{\pgfqpoint{2.805989in}{2.583834in}}%
\pgfpathlineto{\pgfqpoint{2.806856in}{2.501445in}}%
\pgfpathlineto{\pgfqpoint{2.807721in}{2.554383in}}%
\pgfpathlineto{\pgfqpoint{2.808586in}{2.541133in}}%
\pgfpathlineto{\pgfqpoint{2.811181in}{2.499846in}}%
\pgfpathlineto{\pgfqpoint{2.812912in}{2.553584in}}%
\pgfpathlineto{\pgfqpoint{2.813775in}{2.541164in}}%
\pgfpathlineto{\pgfqpoint{2.814641in}{2.555244in}}%
\pgfpathlineto{\pgfqpoint{2.815503in}{2.465661in}}%
\pgfpathlineto{\pgfqpoint{2.816368in}{2.499231in}}%
\pgfpathlineto{\pgfqpoint{2.817232in}{2.490869in}}%
\pgfpathlineto{\pgfqpoint{2.818096in}{2.468858in}}%
\pgfpathlineto{\pgfqpoint{2.818961in}{2.540548in}}%
\pgfpathlineto{\pgfqpoint{2.819827in}{2.531633in}}%
\pgfpathlineto{\pgfqpoint{2.820692in}{2.481645in}}%
\pgfpathlineto{\pgfqpoint{2.822421in}{2.531510in}}%
\pgfpathlineto{\pgfqpoint{2.823283in}{2.528066in}}%
\pgfpathlineto{\pgfqpoint{2.824146in}{2.470764in}}%
\pgfpathlineto{\pgfqpoint{2.825873in}{2.556104in}}%
\pgfpathlineto{\pgfqpoint{2.826739in}{2.553705in}}%
\pgfpathlineto{\pgfqpoint{2.827603in}{2.494066in}}%
\pgfpathlineto{\pgfqpoint{2.828468in}{2.510974in}}%
\pgfpathlineto{\pgfqpoint{2.829334in}{2.498553in}}%
\pgfpathlineto{\pgfqpoint{2.831064in}{2.588999in}}%
\pgfpathlineto{\pgfqpoint{2.831927in}{2.531019in}}%
\pgfpathlineto{\pgfqpoint{2.832789in}{2.564835in}}%
\pgfpathlineto{\pgfqpoint{2.833654in}{2.533110in}}%
\pgfpathlineto{\pgfqpoint{2.835386in}{2.631731in}}%
\pgfpathlineto{\pgfqpoint{2.837117in}{2.525793in}}%
\pgfpathlineto{\pgfqpoint{2.837982in}{2.537967in}}%
\pgfpathlineto{\pgfqpoint{2.838847in}{2.593119in}}%
\pgfpathlineto{\pgfqpoint{2.839712in}{2.521367in}}%
\pgfpathlineto{\pgfqpoint{2.840577in}{2.576763in}}%
\pgfpathlineto{\pgfqpoint{2.841442in}{2.563483in}}%
\pgfpathlineto{\pgfqpoint{2.843172in}{2.593794in}}%
\pgfpathlineto{\pgfqpoint{2.844901in}{2.487180in}}%
\pgfpathlineto{\pgfqpoint{2.845764in}{2.548725in}}%
\pgfpathlineto{\pgfqpoint{2.846628in}{2.516385in}}%
\pgfpathlineto{\pgfqpoint{2.847492in}{2.557641in}}%
\pgfpathlineto{\pgfqpoint{2.848356in}{2.551001in}}%
\pgfpathlineto{\pgfqpoint{2.849221in}{2.528559in}}%
\pgfpathlineto{\pgfqpoint{2.850086in}{2.571722in}}%
\pgfpathlineto{\pgfqpoint{2.850950in}{2.534401in}}%
\pgfpathlineto{\pgfqpoint{2.851815in}{2.547804in}}%
\pgfpathlineto{\pgfqpoint{2.852680in}{2.599021in}}%
\pgfpathlineto{\pgfqpoint{2.854407in}{2.507962in}}%
\pgfpathlineto{\pgfqpoint{2.856137in}{2.528590in}}%
\pgfpathlineto{\pgfqpoint{2.857003in}{2.461910in}}%
\pgfpathlineto{\pgfqpoint{2.858734in}{2.535476in}}%
\pgfpathlineto{\pgfqpoint{2.859599in}{2.483000in}}%
\pgfpathlineto{\pgfqpoint{2.860465in}{2.565757in}}%
\pgfpathlineto{\pgfqpoint{2.861329in}{2.516693in}}%
\pgfpathlineto{\pgfqpoint{2.862193in}{2.612547in}}%
\pgfpathlineto{\pgfqpoint{2.863056in}{2.499785in}}%
\pgfpathlineto{\pgfqpoint{2.863921in}{2.585494in}}%
\pgfpathlineto{\pgfqpoint{2.864787in}{2.492529in}}%
\pgfpathlineto{\pgfqpoint{2.866517in}{2.573689in}}%
\pgfpathlineto{\pgfqpoint{2.869111in}{2.489270in}}%
\pgfpathlineto{\pgfqpoint{2.869976in}{2.582236in}}%
\pgfpathlineto{\pgfqpoint{2.872568in}{2.498063in}}%
\pgfpathlineto{\pgfqpoint{2.873433in}{2.549464in}}%
\pgfpathlineto{\pgfqpoint{2.874296in}{2.523579in}}%
\pgfpathlineto{\pgfqpoint{2.875161in}{2.543653in}}%
\pgfpathlineto{\pgfqpoint{2.876025in}{2.593548in}}%
\pgfpathlineto{\pgfqpoint{2.877753in}{2.487641in}}%
\pgfpathlineto{\pgfqpoint{2.878619in}{2.542455in}}%
\pgfpathlineto{\pgfqpoint{2.879484in}{2.426003in}}%
\pgfpathlineto{\pgfqpoint{2.881212in}{2.531817in}}%
\pgfpathlineto{\pgfqpoint{2.882076in}{2.567817in}}%
\pgfpathlineto{\pgfqpoint{2.883803in}{2.529175in}}%
\pgfpathlineto{\pgfqpoint{2.884668in}{2.557610in}}%
\pgfpathlineto{\pgfqpoint{2.885532in}{2.556412in}}%
\pgfpathlineto{\pgfqpoint{2.886397in}{2.483983in}}%
\pgfpathlineto{\pgfqpoint{2.887263in}{2.565603in}}%
\pgfpathlineto{\pgfqpoint{2.888127in}{2.503351in}}%
\pgfpathlineto{\pgfqpoint{2.888993in}{2.562131in}}%
\pgfpathlineto{\pgfqpoint{2.889857in}{2.507471in}}%
\pgfpathlineto{\pgfqpoint{2.890721in}{2.559673in}}%
\pgfpathlineto{\pgfqpoint{2.891587in}{2.516079in}}%
\pgfpathlineto{\pgfqpoint{2.892452in}{2.613778in}}%
\pgfpathlineto{\pgfqpoint{2.894183in}{2.510545in}}%
\pgfpathlineto{\pgfqpoint{2.895047in}{2.552907in}}%
\pgfpathlineto{\pgfqpoint{2.896779in}{2.506117in}}%
\pgfpathlineto{\pgfqpoint{2.897643in}{2.568771in}}%
\pgfpathlineto{\pgfqpoint{2.898505in}{2.470579in}}%
\pgfpathlineto{\pgfqpoint{2.899368in}{2.471993in}}%
\pgfpathlineto{\pgfqpoint{2.901099in}{2.521919in}}%
\pgfpathlineto{\pgfqpoint{2.901964in}{2.553829in}}%
\pgfpathlineto{\pgfqpoint{2.903693in}{2.490315in}}%
\pgfpathlineto{\pgfqpoint{2.904557in}{2.524500in}}%
\pgfpathlineto{\pgfqpoint{2.905420in}{2.493450in}}%
\pgfpathlineto{\pgfqpoint{2.906286in}{2.549156in}}%
\pgfpathlineto{\pgfqpoint{2.907152in}{2.501506in}}%
\pgfpathlineto{\pgfqpoint{2.908015in}{2.565633in}}%
\pgfpathlineto{\pgfqpoint{2.909746in}{2.526099in}}%
\pgfpathlineto{\pgfqpoint{2.910611in}{2.530157in}}%
\pgfpathlineto{\pgfqpoint{2.911476in}{2.540302in}}%
\pgfpathlineto{\pgfqpoint{2.912341in}{2.586969in}}%
\pgfpathlineto{\pgfqpoint{2.913205in}{2.499415in}}%
\pgfpathlineto{\pgfqpoint{2.914071in}{2.555673in}}%
\pgfpathlineto{\pgfqpoint{2.914936in}{2.535692in}}%
\pgfpathlineto{\pgfqpoint{2.915801in}{2.449151in}}%
\pgfpathlineto{\pgfqpoint{2.918398in}{2.622138in}}%
\pgfpathlineto{\pgfqpoint{2.920993in}{2.498984in}}%
\pgfpathlineto{\pgfqpoint{2.921858in}{2.516446in}}%
\pgfpathlineto{\pgfqpoint{2.922722in}{2.601971in}}%
\pgfpathlineto{\pgfqpoint{2.924453in}{2.495787in}}%
\pgfpathlineto{\pgfqpoint{2.925318in}{2.502797in}}%
\pgfpathlineto{\pgfqpoint{2.926183in}{2.553952in}}%
\pgfpathlineto{\pgfqpoint{2.927911in}{2.480293in}}%
\pgfpathlineto{\pgfqpoint{2.929642in}{2.564096in}}%
\pgfpathlineto{\pgfqpoint{2.930505in}{2.526960in}}%
\pgfpathlineto{\pgfqpoint{2.931370in}{2.585371in}}%
\pgfpathlineto{\pgfqpoint{2.932235in}{2.500183in}}%
\pgfpathlineto{\pgfqpoint{2.933966in}{2.585433in}}%
\pgfpathlineto{\pgfqpoint{2.935694in}{2.512942in}}%
\pgfpathlineto{\pgfqpoint{2.936558in}{2.511589in}}%
\pgfpathlineto{\pgfqpoint{2.938287in}{2.545467in}}%
\pgfpathlineto{\pgfqpoint{2.939153in}{2.497509in}}%
\pgfpathlineto{\pgfqpoint{2.940017in}{2.556043in}}%
\pgfpathlineto{\pgfqpoint{2.940882in}{2.518045in}}%
\pgfpathlineto{\pgfqpoint{2.941744in}{2.545652in}}%
\pgfpathlineto{\pgfqpoint{2.943475in}{2.447891in}}%
\pgfpathlineto{\pgfqpoint{2.944340in}{2.451273in}}%
\pgfpathlineto{\pgfqpoint{2.946068in}{2.566803in}}%
\pgfpathlineto{\pgfqpoint{2.946935in}{2.558811in}}%
\pgfpathlineto{\pgfqpoint{2.948660in}{2.514358in}}%
\pgfpathlineto{\pgfqpoint{2.949527in}{2.433936in}}%
\pgfpathlineto{\pgfqpoint{2.950389in}{2.572276in}}%
\pgfpathlineto{\pgfqpoint{2.951255in}{2.532987in}}%
\pgfpathlineto{\pgfqpoint{2.952121in}{2.497265in}}%
\pgfpathlineto{\pgfqpoint{2.952984in}{2.537967in}}%
\pgfpathlineto{\pgfqpoint{2.953848in}{2.475898in}}%
\pgfpathlineto{\pgfqpoint{2.955575in}{2.541410in}}%
\pgfpathlineto{\pgfqpoint{2.956440in}{2.515894in}}%
\pgfpathlineto{\pgfqpoint{2.957306in}{2.551863in}}%
\pgfpathlineto{\pgfqpoint{2.959036in}{2.507348in}}%
\pgfpathlineto{\pgfqpoint{2.961632in}{2.607138in}}%
\pgfpathlineto{\pgfqpoint{2.962497in}{2.644582in}}%
\pgfpathlineto{\pgfqpoint{2.963362in}{2.569479in}}%
\pgfpathlineto{\pgfqpoint{2.964228in}{2.596993in}}%
\pgfpathlineto{\pgfqpoint{2.965093in}{2.669605in}}%
\pgfpathlineto{\pgfqpoint{2.965957in}{2.573997in}}%
\pgfpathlineto{\pgfqpoint{2.966821in}{2.625521in}}%
\pgfpathlineto{\pgfqpoint{2.968551in}{2.572276in}}%
\pgfpathlineto{\pgfqpoint{2.970278in}{2.598253in}}%
\pgfpathlineto{\pgfqpoint{2.972009in}{2.682272in}}%
\pgfpathlineto{\pgfqpoint{2.973738in}{2.620849in}}%
\pgfpathlineto{\pgfqpoint{2.974603in}{2.660138in}}%
\pgfpathlineto{\pgfqpoint{2.975468in}{2.654911in}}%
\pgfpathlineto{\pgfqpoint{2.976333in}{2.591521in}}%
\pgfpathlineto{\pgfqpoint{2.978927in}{2.673418in}}%
\pgfpathlineto{\pgfqpoint{2.980655in}{2.613163in}}%
\pgfpathlineto{\pgfqpoint{2.983249in}{2.679198in}}%
\pgfpathlineto{\pgfqpoint{2.984114in}{2.620480in}}%
\pgfpathlineto{\pgfqpoint{2.984979in}{2.673233in}}%
\pgfpathlineto{\pgfqpoint{2.987576in}{2.559732in}}%
\pgfpathlineto{\pgfqpoint{2.990170in}{2.652513in}}%
\pgfpathlineto{\pgfqpoint{2.992763in}{2.610150in}}%
\pgfpathlineto{\pgfqpoint{2.993628in}{2.678521in}}%
\pgfpathlineto{\pgfqpoint{2.994494in}{2.626750in}}%
\pgfpathlineto{\pgfqpoint{2.995360in}{2.644641in}}%
\pgfpathlineto{\pgfqpoint{2.998820in}{2.509529in}}%
\pgfpathlineto{\pgfqpoint{2.999686in}{2.540610in}}%
\pgfpathlineto{\pgfqpoint{3.000551in}{2.535815in}}%
\pgfpathlineto{\pgfqpoint{3.001416in}{2.506363in}}%
\pgfpathlineto{\pgfqpoint{3.002282in}{2.566495in}}%
\pgfpathlineto{\pgfqpoint{3.004010in}{2.519951in}}%
\pgfpathlineto{\pgfqpoint{3.004875in}{2.507346in}}%
\pgfpathlineto{\pgfqpoint{3.005740in}{2.540087in}}%
\pgfpathlineto{\pgfqpoint{3.007470in}{2.497694in}}%
\pgfpathlineto{\pgfqpoint{3.008335in}{2.542393in}}%
\pgfpathlineto{\pgfqpoint{3.010930in}{2.479095in}}%
\pgfpathlineto{\pgfqpoint{3.012659in}{2.533908in}}%
\pgfpathlineto{\pgfqpoint{3.013525in}{2.537413in}}%
\pgfpathlineto{\pgfqpoint{3.014390in}{2.502551in}}%
\pgfpathlineto{\pgfqpoint{3.015255in}{2.542609in}}%
\pgfpathlineto{\pgfqpoint{3.017850in}{2.484383in}}%
\pgfpathlineto{\pgfqpoint{3.019578in}{2.583095in}}%
\pgfpathlineto{\pgfqpoint{3.020443in}{2.542301in}}%
\pgfpathlineto{\pgfqpoint{3.021309in}{2.558072in}}%
\pgfpathlineto{\pgfqpoint{3.023036in}{2.527145in}}%
\pgfpathlineto{\pgfqpoint{3.023900in}{2.528313in}}%
\pgfpathlineto{\pgfqpoint{3.024765in}{2.523302in}}%
\pgfpathlineto{\pgfqpoint{3.025630in}{2.476850in}}%
\pgfpathlineto{\pgfqpoint{3.027361in}{2.522625in}}%
\pgfpathlineto{\pgfqpoint{3.028227in}{2.596745in}}%
\pgfpathlineto{\pgfqpoint{3.029959in}{2.499292in}}%
\pgfpathlineto{\pgfqpoint{3.031687in}{2.573197in}}%
\pgfpathlineto{\pgfqpoint{3.032552in}{2.509622in}}%
\pgfpathlineto{\pgfqpoint{3.033415in}{2.518353in}}%
\pgfpathlineto{\pgfqpoint{3.034281in}{2.494127in}}%
\pgfpathlineto{\pgfqpoint{3.035146in}{2.603323in}}%
\pgfpathlineto{\pgfqpoint{3.036011in}{2.497078in}}%
\pgfpathlineto{\pgfqpoint{3.036876in}{2.579591in}}%
\pgfpathlineto{\pgfqpoint{3.037739in}{2.576332in}}%
\pgfpathlineto{\pgfqpoint{3.038605in}{2.569261in}}%
\pgfpathlineto{\pgfqpoint{3.039470in}{2.442726in}}%
\pgfpathlineto{\pgfqpoint{3.041199in}{2.514048in}}%
\pgfpathlineto{\pgfqpoint{3.042064in}{2.519212in}}%
\pgfpathlineto{\pgfqpoint{3.042930in}{2.500521in}}%
\pgfpathlineto{\pgfqpoint{3.043794in}{2.522779in}}%
\pgfpathlineto{\pgfqpoint{3.045523in}{2.485335in}}%
\pgfpathlineto{\pgfqpoint{3.047252in}{2.519335in}}%
\pgfpathlineto{\pgfqpoint{3.048117in}{2.516508in}}%
\pgfpathlineto{\pgfqpoint{3.048983in}{2.497417in}}%
\pgfpathlineto{\pgfqpoint{3.049848in}{2.507223in}}%
\pgfpathlineto{\pgfqpoint{3.050713in}{2.490007in}}%
\pgfpathlineto{\pgfqpoint{3.052443in}{2.551922in}}%
\pgfpathlineto{\pgfqpoint{3.053308in}{2.502243in}}%
\pgfpathlineto{\pgfqpoint{3.054173in}{2.512018in}}%
\pgfpathlineto{\pgfqpoint{3.055037in}{2.515708in}}%
\pgfpathlineto{\pgfqpoint{3.055901in}{2.502120in}}%
\pgfpathlineto{\pgfqpoint{3.057631in}{2.420959in}}%
\pgfpathlineto{\pgfqpoint{3.059359in}{2.589489in}}%
\pgfpathlineto{\pgfqpoint{3.060224in}{2.552415in}}%
\pgfpathlineto{\pgfqpoint{3.061088in}{2.559178in}}%
\pgfpathlineto{\pgfqpoint{3.061953in}{2.554937in}}%
\pgfpathlineto{\pgfqpoint{3.062818in}{2.537013in}}%
\pgfpathlineto{\pgfqpoint{3.063682in}{2.562252in}}%
\pgfpathlineto{\pgfqpoint{3.064545in}{2.542208in}}%
\pgfpathlineto{\pgfqpoint{3.065410in}{2.548849in}}%
\pgfpathlineto{\pgfqpoint{3.067139in}{2.505871in}}%
\pgfpathlineto{\pgfqpoint{3.068004in}{2.584109in}}%
\pgfpathlineto{\pgfqpoint{3.070600in}{2.488347in}}%
\pgfpathlineto{\pgfqpoint{3.073195in}{2.590166in}}%
\pgfpathlineto{\pgfqpoint{3.074926in}{2.521765in}}%
\pgfpathlineto{\pgfqpoint{3.075790in}{2.595516in}}%
\pgfpathlineto{\pgfqpoint{3.076655in}{2.523702in}}%
\pgfpathlineto{\pgfqpoint{3.077520in}{2.528867in}}%
\pgfpathlineto{\pgfqpoint{3.080116in}{2.482015in}}%
\pgfpathlineto{\pgfqpoint{3.080982in}{2.572520in}}%
\pgfpathlineto{\pgfqpoint{3.081847in}{2.514540in}}%
\pgfpathlineto{\pgfqpoint{3.082712in}{2.605476in}}%
\pgfpathlineto{\pgfqpoint{3.083578in}{2.595393in}}%
\pgfpathlineto{\pgfqpoint{3.084443in}{2.577009in}}%
\pgfpathlineto{\pgfqpoint{3.086170in}{2.511713in}}%
\pgfpathlineto{\pgfqpoint{3.087036in}{2.501321in}}%
\pgfpathlineto{\pgfqpoint{3.088767in}{2.554198in}}%
\pgfpathlineto{\pgfqpoint{3.092227in}{2.495172in}}%
\pgfpathlineto{\pgfqpoint{3.093092in}{2.560407in}}%
\pgfpathlineto{\pgfqpoint{3.093957in}{2.505409in}}%
\pgfpathlineto{\pgfqpoint{3.094822in}{2.512695in}}%
\pgfpathlineto{\pgfqpoint{3.097417in}{2.552230in}}%
\pgfpathlineto{\pgfqpoint{3.098282in}{2.523086in}}%
\pgfpathlineto{\pgfqpoint{3.099147in}{2.550078in}}%
\pgfpathlineto{\pgfqpoint{3.100012in}{2.510176in}}%
\pgfpathlineto{\pgfqpoint{3.100878in}{2.520228in}}%
\pgfpathlineto{\pgfqpoint{3.102605in}{2.490561in}}%
\pgfpathlineto{\pgfqpoint{3.103470in}{2.497170in}}%
\pgfpathlineto{\pgfqpoint{3.104334in}{2.522840in}}%
\pgfpathlineto{\pgfqpoint{3.105199in}{2.497201in}}%
\pgfpathlineto{\pgfqpoint{3.106930in}{2.563114in}}%
\pgfpathlineto{\pgfqpoint{3.107795in}{2.528867in}}%
\pgfpathlineto{\pgfqpoint{3.108661in}{2.535507in}}%
\pgfpathlineto{\pgfqpoint{3.111255in}{2.478264in}}%
\pgfpathlineto{\pgfqpoint{3.112119in}{2.530157in}}%
\pgfpathlineto{\pgfqpoint{3.113850in}{2.394399in}}%
\pgfpathlineto{\pgfqpoint{3.114714in}{2.590780in}}%
\pgfpathlineto{\pgfqpoint{3.115579in}{2.495510in}}%
\pgfpathlineto{\pgfqpoint{3.116443in}{2.585309in}}%
\pgfpathlineto{\pgfqpoint{3.117306in}{2.480109in}}%
\pgfpathlineto{\pgfqpoint{3.118172in}{2.531694in}}%
\pgfpathlineto{\pgfqpoint{3.119902in}{2.485150in}}%
\pgfpathlineto{\pgfqpoint{3.120768in}{2.543992in}}%
\pgfpathlineto{\pgfqpoint{3.121631in}{2.511589in}}%
\pgfpathlineto{\pgfqpoint{3.123362in}{2.560715in}}%
\pgfpathlineto{\pgfqpoint{3.125092in}{2.439745in}}%
\pgfpathlineto{\pgfqpoint{3.125956in}{2.455270in}}%
\pgfpathlineto{\pgfqpoint{3.127686in}{2.537598in}}%
\pgfpathlineto{\pgfqpoint{3.128550in}{2.489211in}}%
\pgfpathlineto{\pgfqpoint{3.130279in}{2.544484in}}%
\pgfpathlineto{\pgfqpoint{3.131143in}{2.416166in}}%
\pgfpathlineto{\pgfqpoint{3.133740in}{2.587216in}}%
\pgfpathlineto{\pgfqpoint{3.134605in}{2.534031in}}%
\pgfpathlineto{\pgfqpoint{3.135471in}{2.552846in}}%
\pgfpathlineto{\pgfqpoint{3.137201in}{2.494928in}}%
\pgfpathlineto{\pgfqpoint{3.138066in}{2.544607in}}%
\pgfpathlineto{\pgfqpoint{3.138930in}{2.513588in}}%
\pgfpathlineto{\pgfqpoint{3.139795in}{2.551863in}}%
\pgfpathlineto{\pgfqpoint{3.141527in}{2.489855in}}%
\pgfpathlineto{\pgfqpoint{3.142392in}{2.501814in}}%
\pgfpathlineto{\pgfqpoint{3.143259in}{2.496465in}}%
\pgfpathlineto{\pgfqpoint{3.144124in}{2.571475in}}%
\pgfpathlineto{\pgfqpoint{3.146723in}{2.463570in}}%
\pgfpathlineto{\pgfqpoint{3.147589in}{2.608673in}}%
\pgfpathlineto{\pgfqpoint{3.148455in}{2.530835in}}%
\pgfpathlineto{\pgfqpoint{3.149320in}{2.551432in}}%
\pgfpathlineto{\pgfqpoint{3.150186in}{2.554506in}}%
\pgfpathlineto{\pgfqpoint{3.151050in}{2.580237in}}%
\pgfpathlineto{\pgfqpoint{3.154510in}{2.494374in}}%
\pgfpathlineto{\pgfqpoint{3.155373in}{2.525547in}}%
\pgfpathlineto{\pgfqpoint{3.156236in}{2.492775in}}%
\pgfpathlineto{\pgfqpoint{3.157101in}{2.559609in}}%
\pgfpathlineto{\pgfqpoint{3.157967in}{2.532556in}}%
\pgfpathlineto{\pgfqpoint{3.158829in}{2.470395in}}%
\pgfpathlineto{\pgfqpoint{3.159695in}{2.510789in}}%
\pgfpathlineto{\pgfqpoint{3.161425in}{2.485766in}}%
\pgfpathlineto{\pgfqpoint{3.162292in}{2.535322in}}%
\pgfpathlineto{\pgfqpoint{3.163156in}{2.485550in}}%
\pgfpathlineto{\pgfqpoint{3.164886in}{2.631053in}}%
\pgfpathlineto{\pgfqpoint{3.165751in}{2.501260in}}%
\pgfpathlineto{\pgfqpoint{3.167480in}{2.564404in}}%
\pgfpathlineto{\pgfqpoint{3.169210in}{2.530650in}}%
\pgfpathlineto{\pgfqpoint{3.170075in}{2.526683in}}%
\pgfpathlineto{\pgfqpoint{3.170938in}{2.465291in}}%
\pgfpathlineto{\pgfqpoint{3.173530in}{2.541962in}}%
\pgfpathlineto{\pgfqpoint{3.174393in}{2.472178in}}%
\pgfpathlineto{\pgfqpoint{3.176124in}{2.519028in}}%
\pgfpathlineto{\pgfqpoint{3.176989in}{2.502150in}}%
\pgfpathlineto{\pgfqpoint{3.177853in}{2.502304in}}%
\pgfpathlineto{\pgfqpoint{3.178719in}{2.507715in}}%
\pgfpathlineto{\pgfqpoint{3.179583in}{2.523394in}}%
\pgfpathlineto{\pgfqpoint{3.180447in}{2.487241in}}%
\pgfpathlineto{\pgfqpoint{3.181311in}{2.523333in}}%
\pgfpathlineto{\pgfqpoint{3.182175in}{2.444725in}}%
\pgfpathlineto{\pgfqpoint{3.183039in}{2.562929in}}%
\pgfpathlineto{\pgfqpoint{3.183904in}{2.533724in}}%
\pgfpathlineto{\pgfqpoint{3.184769in}{2.520444in}}%
\pgfpathlineto{\pgfqpoint{3.185635in}{2.479618in}}%
\pgfpathlineto{\pgfqpoint{3.189095in}{2.562498in}}%
\pgfpathlineto{\pgfqpoint{3.190825in}{2.485273in}}%
\pgfpathlineto{\pgfqpoint{3.192553in}{2.541408in}}%
\pgfpathlineto{\pgfqpoint{3.193417in}{2.552415in}}%
\pgfpathlineto{\pgfqpoint{3.194278in}{2.499169in}}%
\pgfpathlineto{\pgfqpoint{3.195143in}{2.567786in}}%
\pgfpathlineto{\pgfqpoint{3.196006in}{2.547127in}}%
\pgfpathlineto{\pgfqpoint{3.197734in}{2.581866in}}%
\pgfpathlineto{\pgfqpoint{3.199463in}{2.542085in}}%
\pgfpathlineto{\pgfqpoint{3.200327in}{2.522042in}}%
\pgfpathlineto{\pgfqpoint{3.201193in}{2.437315in}}%
\pgfpathlineto{\pgfqpoint{3.202922in}{2.547065in}}%
\pgfpathlineto{\pgfqpoint{3.203786in}{2.495849in}}%
\pgfpathlineto{\pgfqpoint{3.204651in}{2.590043in}}%
\pgfpathlineto{\pgfqpoint{3.205515in}{2.554075in}}%
\pgfpathlineto{\pgfqpoint{3.207245in}{2.576763in}}%
\pgfpathlineto{\pgfqpoint{3.208109in}{2.535753in}}%
\pgfpathlineto{\pgfqpoint{3.208975in}{2.580022in}}%
\pgfpathlineto{\pgfqpoint{3.210707in}{2.519582in}}%
\pgfpathlineto{\pgfqpoint{3.212439in}{2.590720in}}%
\pgfpathlineto{\pgfqpoint{3.213304in}{2.612670in}}%
\pgfpathlineto{\pgfqpoint{3.215032in}{2.528097in}}%
\pgfpathlineto{\pgfqpoint{3.215896in}{2.497817in}}%
\pgfpathlineto{\pgfqpoint{3.216762in}{2.559547in}}%
\pgfpathlineto{\pgfqpoint{3.217627in}{2.556135in}}%
\pgfpathlineto{\pgfqpoint{3.218492in}{2.571044in}}%
\pgfpathlineto{\pgfqpoint{3.219357in}{2.554260in}}%
\pgfpathlineto{\pgfqpoint{3.220222in}{2.569908in}}%
\pgfpathlineto{\pgfqpoint{3.221951in}{2.530342in}}%
\pgfpathlineto{\pgfqpoint{3.222816in}{2.560284in}}%
\pgfpathlineto{\pgfqpoint{3.223681in}{2.540241in}}%
\pgfpathlineto{\pgfqpoint{3.224547in}{2.548695in}}%
\pgfpathlineto{\pgfqpoint{3.225412in}{2.577623in}}%
\pgfpathlineto{\pgfqpoint{3.226276in}{2.573505in}}%
\pgfpathlineto{\pgfqpoint{3.227141in}{2.453364in}}%
\pgfpathlineto{\pgfqpoint{3.228006in}{2.557333in}}%
\pgfpathlineto{\pgfqpoint{3.228872in}{2.540179in}}%
\pgfpathlineto{\pgfqpoint{3.229739in}{2.504488in}}%
\pgfpathlineto{\pgfqpoint{3.230605in}{2.556902in}}%
\pgfpathlineto{\pgfqpoint{3.231470in}{2.523148in}}%
\pgfpathlineto{\pgfqpoint{3.232335in}{2.590474in}}%
\pgfpathlineto{\pgfqpoint{3.233200in}{2.513619in}}%
\pgfpathlineto{\pgfqpoint{3.234064in}{2.532218in}}%
\pgfpathlineto{\pgfqpoint{3.234929in}{2.473961in}}%
\pgfpathlineto{\pgfqpoint{3.237525in}{2.566311in}}%
\pgfpathlineto{\pgfqpoint{3.240120in}{2.478387in}}%
\pgfpathlineto{\pgfqpoint{3.240984in}{2.523025in}}%
\pgfpathlineto{\pgfqpoint{3.241849in}{2.502427in}}%
\pgfpathlineto{\pgfqpoint{3.242713in}{2.527207in}}%
\pgfpathlineto{\pgfqpoint{3.243578in}{2.507746in}}%
\pgfpathlineto{\pgfqpoint{3.245308in}{2.533970in}}%
\pgfpathlineto{\pgfqpoint{3.246173in}{2.504580in}}%
\pgfpathlineto{\pgfqpoint{3.247039in}{2.543376in}}%
\pgfpathlineto{\pgfqpoint{3.247901in}{2.518414in}}%
\pgfpathlineto{\pgfqpoint{3.248767in}{2.542976in}}%
\pgfpathlineto{\pgfqpoint{3.249633in}{2.505994in}}%
\pgfpathlineto{\pgfqpoint{3.250497in}{2.569877in}}%
\pgfpathlineto{\pgfqpoint{3.251361in}{2.471562in}}%
\pgfpathlineto{\pgfqpoint{3.252226in}{2.528374in}}%
\pgfpathlineto{\pgfqpoint{3.253089in}{2.451396in}}%
\pgfpathlineto{\pgfqpoint{3.253954in}{2.548725in}}%
\pgfpathlineto{\pgfqpoint{3.254819in}{2.509006in}}%
\pgfpathlineto{\pgfqpoint{3.255685in}{2.541931in}}%
\pgfpathlineto{\pgfqpoint{3.256550in}{2.456253in}}%
\pgfpathlineto{\pgfqpoint{3.259145in}{2.531325in}}%
\pgfpathlineto{\pgfqpoint{3.260010in}{2.507223in}}%
\pgfpathlineto{\pgfqpoint{3.260876in}{2.539687in}}%
\pgfpathlineto{\pgfqpoint{3.261740in}{2.475498in}}%
\pgfpathlineto{\pgfqpoint{3.262605in}{2.541901in}}%
\pgfpathlineto{\pgfqpoint{3.263469in}{2.470025in}}%
\pgfpathlineto{\pgfqpoint{3.265199in}{2.579529in}}%
\pgfpathlineto{\pgfqpoint{3.266064in}{2.561023in}}%
\pgfpathlineto{\pgfqpoint{3.266929in}{2.545775in}}%
\pgfpathlineto{\pgfqpoint{3.267794in}{2.494466in}}%
\pgfpathlineto{\pgfqpoint{3.269522in}{2.541408in}}%
\pgfpathlineto{\pgfqpoint{3.270385in}{2.512757in}}%
\pgfpathlineto{\pgfqpoint{3.272980in}{2.562252in}}%
\pgfpathlineto{\pgfqpoint{3.273846in}{2.538211in}}%
\pgfpathlineto{\pgfqpoint{3.274709in}{2.541408in}}%
\pgfpathlineto{\pgfqpoint{3.275573in}{2.526160in}}%
\pgfpathlineto{\pgfqpoint{3.276438in}{2.475005in}}%
\pgfpathlineto{\pgfqpoint{3.278164in}{2.570983in}}%
\pgfpathlineto{\pgfqpoint{3.279893in}{2.501198in}}%
\pgfpathlineto{\pgfqpoint{3.280759in}{2.540487in}}%
\pgfpathlineto{\pgfqpoint{3.281623in}{2.493083in}}%
\pgfpathlineto{\pgfqpoint{3.282489in}{2.536736in}}%
\pgfpathlineto{\pgfqpoint{3.283353in}{2.525177in}}%
\pgfpathlineto{\pgfqpoint{3.284218in}{2.529880in}}%
\pgfpathlineto{\pgfqpoint{3.285084in}{2.500213in}}%
\pgfpathlineto{\pgfqpoint{3.285950in}{2.579652in}}%
\pgfpathlineto{\pgfqpoint{3.286815in}{2.466521in}}%
\pgfpathlineto{\pgfqpoint{3.287681in}{2.516015in}}%
\pgfpathlineto{\pgfqpoint{3.288546in}{2.500398in}}%
\pgfpathlineto{\pgfqpoint{3.289412in}{2.557025in}}%
\pgfpathlineto{\pgfqpoint{3.293738in}{2.466767in}}%
\pgfpathlineto{\pgfqpoint{3.294599in}{2.580083in}}%
\pgfpathlineto{\pgfqpoint{3.295464in}{2.571660in}}%
\pgfpathlineto{\pgfqpoint{3.296325in}{2.540518in}}%
\pgfpathlineto{\pgfqpoint{3.297192in}{2.543745in}}%
\pgfpathlineto{\pgfqpoint{3.298922in}{2.499538in}}%
\pgfpathlineto{\pgfqpoint{3.299786in}{2.526776in}}%
\pgfpathlineto{\pgfqpoint{3.300649in}{2.526160in}}%
\pgfpathlineto{\pgfqpoint{3.301516in}{2.505317in}}%
\pgfpathlineto{\pgfqpoint{3.302379in}{2.514109in}}%
\pgfpathlineto{\pgfqpoint{3.303244in}{2.551953in}}%
\pgfpathlineto{\pgfqpoint{3.304108in}{2.514602in}}%
\pgfpathlineto{\pgfqpoint{3.304975in}{2.524931in}}%
\pgfpathlineto{\pgfqpoint{3.306706in}{2.565818in}}%
\pgfpathlineto{\pgfqpoint{3.308436in}{2.489699in}}%
\pgfpathlineto{\pgfqpoint{3.310168in}{2.520595in}}%
\pgfpathlineto{\pgfqpoint{3.311034in}{2.537965in}}%
\pgfpathlineto{\pgfqpoint{3.311899in}{2.439712in}}%
\pgfpathlineto{\pgfqpoint{3.313628in}{2.585615in}}%
\pgfpathlineto{\pgfqpoint{3.314493in}{2.583586in}}%
\pgfpathlineto{\pgfqpoint{3.315359in}{2.490284in}}%
\pgfpathlineto{\pgfqpoint{3.317088in}{2.601417in}}%
\pgfpathlineto{\pgfqpoint{3.317953in}{2.515584in}}%
\pgfpathlineto{\pgfqpoint{3.318817in}{2.528374in}}%
\pgfpathlineto{\pgfqpoint{3.319683in}{2.534860in}}%
\pgfpathlineto{\pgfqpoint{3.320547in}{2.557025in}}%
\pgfpathlineto{\pgfqpoint{3.321413in}{2.513249in}}%
\pgfpathlineto{\pgfqpoint{3.323139in}{2.579529in}}%
\pgfpathlineto{\pgfqpoint{3.324004in}{2.502427in}}%
\pgfpathlineto{\pgfqpoint{3.325734in}{2.636095in}}%
\pgfpathlineto{\pgfqpoint{3.328325in}{2.502735in}}%
\pgfpathlineto{\pgfqpoint{3.329190in}{2.483983in}}%
\pgfpathlineto{\pgfqpoint{3.330054in}{2.570798in}}%
\pgfpathlineto{\pgfqpoint{3.330917in}{2.558378in}}%
\pgfpathlineto{\pgfqpoint{3.331782in}{2.548479in}}%
\pgfpathlineto{\pgfqpoint{3.334376in}{2.480601in}}%
\pgfpathlineto{\pgfqpoint{3.336106in}{2.545528in}}%
\pgfpathlineto{\pgfqpoint{3.336971in}{2.534830in}}%
\pgfpathlineto{\pgfqpoint{3.337833in}{2.515584in}}%
\pgfpathlineto{\pgfqpoint{3.338698in}{2.520872in}}%
\pgfpathlineto{\pgfqpoint{3.339562in}{2.516139in}}%
\pgfpathlineto{\pgfqpoint{3.340427in}{2.474328in}}%
\pgfpathlineto{\pgfqpoint{3.343019in}{2.573626in}}%
\pgfpathlineto{\pgfqpoint{3.343883in}{2.498399in}}%
\pgfpathlineto{\pgfqpoint{3.344746in}{2.557210in}}%
\pgfpathlineto{\pgfqpoint{3.346476in}{2.456745in}}%
\pgfpathlineto{\pgfqpoint{3.347340in}{2.581620in}}%
\pgfpathlineto{\pgfqpoint{3.349071in}{2.453056in}}%
\pgfpathlineto{\pgfqpoint{3.350801in}{2.545867in}}%
\pgfpathlineto{\pgfqpoint{3.351667in}{2.506117in}}%
\pgfpathlineto{\pgfqpoint{3.352533in}{2.537105in}}%
\pgfpathlineto{\pgfqpoint{3.353399in}{2.535168in}}%
\pgfpathlineto{\pgfqpoint{3.354265in}{2.552723in}}%
\pgfpathlineto{\pgfqpoint{3.355130in}{2.596314in}}%
\pgfpathlineto{\pgfqpoint{3.355995in}{2.491300in}}%
\pgfpathlineto{\pgfqpoint{3.356859in}{2.497694in}}%
\pgfpathlineto{\pgfqpoint{3.357724in}{2.505532in}}%
\pgfpathlineto{\pgfqpoint{3.358588in}{2.490561in}}%
\pgfpathlineto{\pgfqpoint{3.359454in}{2.518291in}}%
\pgfpathlineto{\pgfqpoint{3.360319in}{2.463570in}}%
\pgfpathlineto{\pgfqpoint{3.361184in}{2.544053in}}%
\pgfpathlineto{\pgfqpoint{3.362049in}{2.517983in}}%
\pgfpathlineto{\pgfqpoint{3.362913in}{2.521857in}}%
\pgfpathlineto{\pgfqpoint{3.364643in}{2.584817in}}%
\pgfpathlineto{\pgfqpoint{3.365508in}{2.536613in}}%
\pgfpathlineto{\pgfqpoint{3.366374in}{2.547558in}}%
\pgfpathlineto{\pgfqpoint{3.367238in}{2.564497in}}%
\pgfpathlineto{\pgfqpoint{3.368968in}{2.500213in}}%
\pgfpathlineto{\pgfqpoint{3.369832in}{2.524500in}}%
\pgfpathlineto{\pgfqpoint{3.370697in}{2.517799in}}%
\pgfpathlineto{\pgfqpoint{3.372429in}{2.488409in}}%
\pgfpathlineto{\pgfqpoint{3.373294in}{2.581374in}}%
\pgfpathlineto{\pgfqpoint{3.374159in}{2.533785in}}%
\pgfpathlineto{\pgfqpoint{3.375889in}{2.569015in}}%
\pgfpathlineto{\pgfqpoint{3.376753in}{2.480109in}}%
\pgfpathlineto{\pgfqpoint{3.377619in}{2.610456in}}%
\pgfpathlineto{\pgfqpoint{3.380214in}{2.463201in}}%
\pgfpathlineto{\pgfqpoint{3.381944in}{2.536090in}}%
\pgfpathlineto{\pgfqpoint{3.382810in}{2.553767in}}%
\pgfpathlineto{\pgfqpoint{3.383674in}{2.515584in}}%
\pgfpathlineto{\pgfqpoint{3.384537in}{2.521242in}}%
\pgfpathlineto{\pgfqpoint{3.386267in}{2.507100in}}%
\pgfpathlineto{\pgfqpoint{3.387132in}{2.510235in}}%
\pgfpathlineto{\pgfqpoint{3.388863in}{2.536982in}}%
\pgfpathlineto{\pgfqpoint{3.389728in}{2.531571in}}%
\pgfpathlineto{\pgfqpoint{3.390594in}{2.520503in}}%
\pgfpathlineto{\pgfqpoint{3.391460in}{2.491236in}}%
\pgfpathlineto{\pgfqpoint{3.392325in}{2.587891in}}%
\pgfpathlineto{\pgfqpoint{3.393191in}{2.583586in}}%
\pgfpathlineto{\pgfqpoint{3.394056in}{2.568153in}}%
\pgfpathlineto{\pgfqpoint{3.394921in}{2.518781in}}%
\pgfpathlineto{\pgfqpoint{3.395786in}{2.533200in}}%
\pgfpathlineto{\pgfqpoint{3.396649in}{2.476727in}}%
\pgfpathlineto{\pgfqpoint{3.397513in}{2.548356in}}%
\pgfpathlineto{\pgfqpoint{3.398379in}{2.513555in}}%
\pgfpathlineto{\pgfqpoint{3.400975in}{2.558347in}}%
\pgfpathlineto{\pgfqpoint{3.401841in}{2.538273in}}%
\pgfpathlineto{\pgfqpoint{3.402704in}{2.588075in}}%
\pgfpathlineto{\pgfqpoint{3.403569in}{2.537044in}}%
\pgfpathlineto{\pgfqpoint{3.404434in}{2.560592in}}%
\pgfpathlineto{\pgfqpoint{3.405299in}{2.548664in}}%
\pgfpathlineto{\pgfqpoint{3.407893in}{2.448751in}}%
\pgfpathlineto{\pgfqpoint{3.410487in}{2.493543in}}%
\pgfpathlineto{\pgfqpoint{3.411351in}{2.393722in}}%
\pgfpathlineto{\pgfqpoint{3.412214in}{2.559607in}}%
\pgfpathlineto{\pgfqpoint{3.413944in}{2.491482in}}%
\pgfpathlineto{\pgfqpoint{3.414810in}{2.492650in}}%
\pgfpathlineto{\pgfqpoint{3.415676in}{2.547864in}}%
\pgfpathlineto{\pgfqpoint{3.416540in}{2.489945in}}%
\pgfpathlineto{\pgfqpoint{3.419134in}{2.611193in}}%
\pgfpathlineto{\pgfqpoint{3.419999in}{2.501689in}}%
\pgfpathlineto{\pgfqpoint{3.420865in}{2.611254in}}%
\pgfpathlineto{\pgfqpoint{3.422597in}{2.526099in}}%
\pgfpathlineto{\pgfqpoint{3.423462in}{2.608303in}}%
\pgfpathlineto{\pgfqpoint{3.425191in}{2.518843in}}%
\pgfpathlineto{\pgfqpoint{3.426054in}{2.489638in}}%
\pgfpathlineto{\pgfqpoint{3.427782in}{2.531263in}}%
\pgfpathlineto{\pgfqpoint{3.428646in}{2.462523in}}%
\pgfpathlineto{\pgfqpoint{3.430376in}{2.595146in}}%
\pgfpathlineto{\pgfqpoint{3.431241in}{2.519397in}}%
\pgfpathlineto{\pgfqpoint{3.432107in}{2.584078in}}%
\pgfpathlineto{\pgfqpoint{3.432971in}{2.570613in}}%
\pgfpathlineto{\pgfqpoint{3.433836in}{2.603139in}}%
\pgfpathlineto{\pgfqpoint{3.438159in}{2.500090in}}%
\pgfpathlineto{\pgfqpoint{3.439025in}{2.553182in}}%
\pgfpathlineto{\pgfqpoint{3.439889in}{2.535874in}}%
\pgfpathlineto{\pgfqpoint{3.440753in}{2.537903in}}%
\pgfpathlineto{\pgfqpoint{3.441619in}{2.585800in}}%
\pgfpathlineto{\pgfqpoint{3.443347in}{2.491544in}}%
\pgfpathlineto{\pgfqpoint{3.444212in}{2.574180in}}%
\pgfpathlineto{\pgfqpoint{3.445941in}{2.506853in}}%
\pgfpathlineto{\pgfqpoint{3.448533in}{2.572273in}}%
\pgfpathlineto{\pgfqpoint{3.449396in}{2.513924in}}%
\pgfpathlineto{\pgfqpoint{3.451124in}{2.558070in}}%
\pgfpathlineto{\pgfqpoint{3.451988in}{2.494987in}}%
\pgfpathlineto{\pgfqpoint{3.452853in}{2.561421in}}%
\pgfpathlineto{\pgfqpoint{3.454582in}{2.496401in}}%
\pgfpathlineto{\pgfqpoint{3.455444in}{2.496586in}}%
\pgfpathlineto{\pgfqpoint{3.457175in}{2.523515in}}%
\pgfpathlineto{\pgfqpoint{3.458040in}{2.516660in}}%
\pgfpathlineto{\pgfqpoint{3.458905in}{2.534399in}}%
\pgfpathlineto{\pgfqpoint{3.459770in}{2.590472in}}%
\pgfpathlineto{\pgfqpoint{3.460636in}{2.516937in}}%
\pgfpathlineto{\pgfqpoint{3.461498in}{2.544605in}}%
\pgfpathlineto{\pgfqpoint{3.462364in}{2.511987in}}%
\pgfpathlineto{\pgfqpoint{3.463229in}{2.585307in}}%
\pgfpathlineto{\pgfqpoint{3.466686in}{2.457235in}}%
\pgfpathlineto{\pgfqpoint{3.467552in}{2.475003in}}%
\pgfpathlineto{\pgfqpoint{3.469282in}{2.549770in}}%
\pgfpathlineto{\pgfqpoint{3.470145in}{2.542637in}}%
\pgfpathlineto{\pgfqpoint{3.471011in}{2.522286in}}%
\pgfpathlineto{\pgfqpoint{3.471877in}{2.554904in}}%
\pgfpathlineto{\pgfqpoint{3.472743in}{2.525421in}}%
\pgfpathlineto{\pgfqpoint{3.473609in}{2.537349in}}%
\pgfpathlineto{\pgfqpoint{3.474471in}{2.496432in}}%
\pgfpathlineto{\pgfqpoint{3.475336in}{2.549524in}}%
\pgfpathlineto{\pgfqpoint{3.476201in}{2.528865in}}%
\pgfpathlineto{\pgfqpoint{3.477933in}{2.592071in}}%
\pgfpathlineto{\pgfqpoint{3.478798in}{2.528680in}}%
\pgfpathlineto{\pgfqpoint{3.479662in}{2.559545in}}%
\pgfpathlineto{\pgfqpoint{3.480525in}{2.556471in}}%
\pgfpathlineto{\pgfqpoint{3.481389in}{2.478662in}}%
\pgfpathlineto{\pgfqpoint{3.482252in}{2.588196in}}%
\pgfpathlineto{\pgfqpoint{3.483118in}{2.487362in}}%
\pgfpathlineto{\pgfqpoint{3.484847in}{2.631975in}}%
\pgfpathlineto{\pgfqpoint{3.485710in}{2.567047in}}%
\pgfpathlineto{\pgfqpoint{3.486575in}{2.567140in}}%
\pgfpathlineto{\pgfqpoint{3.487439in}{2.570860in}}%
\pgfpathlineto{\pgfqpoint{3.489170in}{2.524931in}}%
\pgfpathlineto{\pgfqpoint{3.490901in}{2.602954in}}%
\pgfpathlineto{\pgfqpoint{3.491766in}{2.591396in}}%
\pgfpathlineto{\pgfqpoint{3.496091in}{2.416349in}}%
\pgfpathlineto{\pgfqpoint{3.497819in}{2.532248in}}%
\pgfpathlineto{\pgfqpoint{3.498684in}{2.488716in}}%
\pgfpathlineto{\pgfqpoint{3.499548in}{2.621155in}}%
\pgfpathlineto{\pgfqpoint{3.502142in}{2.465045in}}%
\pgfpathlineto{\pgfqpoint{3.503870in}{2.573749in}}%
\pgfpathlineto{\pgfqpoint{3.504734in}{2.476358in}}%
\pgfpathlineto{\pgfqpoint{3.505598in}{2.549739in}}%
\pgfpathlineto{\pgfqpoint{3.506463in}{2.482199in}}%
\pgfpathlineto{\pgfqpoint{3.508192in}{2.526283in}}%
\pgfpathlineto{\pgfqpoint{3.509057in}{2.524993in}}%
\pgfpathlineto{\pgfqpoint{3.509923in}{2.514448in}}%
\pgfpathlineto{\pgfqpoint{3.510788in}{2.522840in}}%
\pgfpathlineto{\pgfqpoint{3.511653in}{2.505317in}}%
\pgfpathlineto{\pgfqpoint{3.512519in}{2.530342in}}%
\pgfpathlineto{\pgfqpoint{3.513384in}{2.530034in}}%
\pgfpathlineto{\pgfqpoint{3.514249in}{2.536859in}}%
\pgfpathlineto{\pgfqpoint{3.515113in}{2.458159in}}%
\pgfpathlineto{\pgfqpoint{3.517709in}{2.529973in}}%
\pgfpathlineto{\pgfqpoint{3.518574in}{2.504026in}}%
\pgfpathlineto{\pgfqpoint{3.519437in}{2.522071in}}%
\pgfpathlineto{\pgfqpoint{3.520301in}{2.477710in}}%
\pgfpathlineto{\pgfqpoint{3.521166in}{2.478202in}}%
\pgfpathlineto{\pgfqpoint{3.522032in}{2.479493in}}%
\pgfpathlineto{\pgfqpoint{3.522897in}{2.568399in}}%
\pgfpathlineto{\pgfqpoint{3.523762in}{2.463691in}}%
\pgfpathlineto{\pgfqpoint{3.525491in}{2.503595in}}%
\pgfpathlineto{\pgfqpoint{3.526355in}{2.515584in}}%
\pgfpathlineto{\pgfqpoint{3.527220in}{2.503287in}}%
\pgfpathlineto{\pgfqpoint{3.528086in}{2.580143in}}%
\pgfpathlineto{\pgfqpoint{3.528952in}{2.435040in}}%
\pgfpathlineto{\pgfqpoint{3.529817in}{2.505255in}}%
\pgfpathlineto{\pgfqpoint{3.530682in}{2.462831in}}%
\pgfpathlineto{\pgfqpoint{3.532412in}{2.531448in}}%
\pgfpathlineto{\pgfqpoint{3.533278in}{2.521550in}}%
\pgfpathlineto{\pgfqpoint{3.534143in}{2.545528in}}%
\pgfpathlineto{\pgfqpoint{3.535870in}{2.494127in}}%
\pgfpathlineto{\pgfqpoint{3.536735in}{2.502920in}}%
\pgfpathlineto{\pgfqpoint{3.537601in}{2.440882in}}%
\pgfpathlineto{\pgfqpoint{3.539331in}{2.535076in}}%
\pgfpathlineto{\pgfqpoint{3.541923in}{2.494805in}}%
\pgfpathlineto{\pgfqpoint{3.542787in}{2.554506in}}%
\pgfpathlineto{\pgfqpoint{3.543652in}{2.506640in}}%
\pgfpathlineto{\pgfqpoint{3.544515in}{2.527022in}}%
\pgfpathlineto{\pgfqpoint{3.545379in}{2.601663in}}%
\pgfpathlineto{\pgfqpoint{3.546245in}{2.499107in}}%
\pgfpathlineto{\pgfqpoint{3.547975in}{2.591611in}}%
\pgfpathlineto{\pgfqpoint{3.548839in}{2.587093in}}%
\pgfpathlineto{\pgfqpoint{3.551434in}{2.483798in}}%
\pgfpathlineto{\pgfqpoint{3.552299in}{2.552476in}}%
\pgfpathlineto{\pgfqpoint{3.554030in}{2.487364in}}%
\pgfpathlineto{\pgfqpoint{3.554896in}{2.571598in}}%
\pgfpathlineto{\pgfqpoint{3.555761in}{2.469227in}}%
\pgfpathlineto{\pgfqpoint{3.557492in}{2.529911in}}%
\pgfpathlineto{\pgfqpoint{3.559222in}{2.464922in}}%
\pgfpathlineto{\pgfqpoint{3.560087in}{2.506086in}}%
\pgfpathlineto{\pgfqpoint{3.560951in}{2.499169in}}%
\pgfpathlineto{\pgfqpoint{3.562682in}{2.545251in}}%
\pgfpathlineto{\pgfqpoint{3.565276in}{2.434549in}}%
\pgfpathlineto{\pgfqpoint{3.566138in}{2.435778in}}%
\pgfpathlineto{\pgfqpoint{3.567868in}{2.528559in}}%
\pgfpathlineto{\pgfqpoint{3.568733in}{2.504888in}}%
\pgfpathlineto{\pgfqpoint{3.570464in}{2.454408in}}%
\pgfpathlineto{\pgfqpoint{3.571329in}{2.516569in}}%
\pgfpathlineto{\pgfqpoint{3.572193in}{2.435748in}}%
\pgfpathlineto{\pgfqpoint{3.573923in}{2.538150in}}%
\pgfpathlineto{\pgfqpoint{3.574787in}{2.465722in}}%
\pgfpathlineto{\pgfqpoint{3.576516in}{2.542668in}}%
\pgfpathlineto{\pgfqpoint{3.577381in}{2.433749in}}%
\pgfpathlineto{\pgfqpoint{3.578247in}{2.513617in}}%
\pgfpathlineto{\pgfqpoint{3.579112in}{2.500183in}}%
\pgfpathlineto{\pgfqpoint{3.579978in}{2.531510in}}%
\pgfpathlineto{\pgfqpoint{3.580842in}{2.452625in}}%
\pgfpathlineto{\pgfqpoint{3.582573in}{2.496647in}}%
\pgfpathlineto{\pgfqpoint{3.584300in}{2.466090in}}%
\pgfpathlineto{\pgfqpoint{3.585164in}{2.513309in}}%
\pgfpathlineto{\pgfqpoint{3.586028in}{2.452961in}}%
\pgfpathlineto{\pgfqpoint{3.587756in}{2.530155in}}%
\pgfpathlineto{\pgfqpoint{3.588620in}{2.472053in}}%
\pgfpathlineto{\pgfqpoint{3.589485in}{2.546879in}}%
\pgfpathlineto{\pgfqpoint{3.590351in}{2.517981in}}%
\pgfpathlineto{\pgfqpoint{3.592078in}{2.569690in}}%
\pgfpathlineto{\pgfqpoint{3.592942in}{2.559361in}}%
\pgfpathlineto{\pgfqpoint{3.594668in}{2.502548in}}%
\pgfpathlineto{\pgfqpoint{3.595532in}{2.550599in}}%
\pgfpathlineto{\pgfqpoint{3.597262in}{2.457603in}}%
\pgfpathlineto{\pgfqpoint{3.598128in}{2.466672in}}%
\pgfpathlineto{\pgfqpoint{3.598992in}{2.426186in}}%
\pgfpathlineto{\pgfqpoint{3.599857in}{2.507959in}}%
\pgfpathlineto{\pgfqpoint{3.600722in}{2.486041in}}%
\pgfpathlineto{\pgfqpoint{3.601586in}{2.456558in}}%
\pgfpathlineto{\pgfqpoint{3.603316in}{2.559668in}}%
\pgfpathlineto{\pgfqpoint{3.604181in}{2.558501in}}%
\pgfpathlineto{\pgfqpoint{3.605045in}{2.460833in}}%
\pgfpathlineto{\pgfqpoint{3.605910in}{2.488162in}}%
\pgfpathlineto{\pgfqpoint{3.606775in}{2.490192in}}%
\pgfpathlineto{\pgfqpoint{3.607639in}{2.473438in}}%
\pgfpathlineto{\pgfqpoint{3.608504in}{2.509129in}}%
\pgfpathlineto{\pgfqpoint{3.609370in}{2.500398in}}%
\pgfpathlineto{\pgfqpoint{3.610234in}{2.503564in}}%
\pgfpathlineto{\pgfqpoint{3.611099in}{2.519151in}}%
\pgfpathlineto{\pgfqpoint{3.612827in}{2.505809in}}%
\pgfpathlineto{\pgfqpoint{3.614557in}{2.575624in}}%
\pgfpathlineto{\pgfqpoint{3.617155in}{2.486502in}}%
\pgfpathlineto{\pgfqpoint{3.618881in}{2.567478in}}%
\pgfpathlineto{\pgfqpoint{3.620609in}{2.502304in}}%
\pgfpathlineto{\pgfqpoint{3.621474in}{2.517368in}}%
\pgfpathlineto{\pgfqpoint{3.622340in}{2.482874in}}%
\pgfpathlineto{\pgfqpoint{3.623205in}{2.513678in}}%
\pgfpathlineto{\pgfqpoint{3.624932in}{2.472853in}}%
\pgfpathlineto{\pgfqpoint{3.626663in}{2.484319in}}%
\pgfpathlineto{\pgfqpoint{3.628394in}{2.513124in}}%
\pgfpathlineto{\pgfqpoint{3.630123in}{2.593854in}}%
\pgfpathlineto{\pgfqpoint{3.634446in}{2.500029in}}%
\pgfpathlineto{\pgfqpoint{3.636176in}{2.559976in}}%
\pgfpathlineto{\pgfqpoint{3.637043in}{2.546265in}}%
\pgfpathlineto{\pgfqpoint{3.637907in}{2.560161in}}%
\pgfpathlineto{\pgfqpoint{3.639640in}{2.472422in}}%
\pgfpathlineto{\pgfqpoint{3.641372in}{2.527081in}}%
\pgfpathlineto{\pgfqpoint{3.642235in}{2.475927in}}%
\pgfpathlineto{\pgfqpoint{3.645693in}{2.590657in}}%
\pgfpathlineto{\pgfqpoint{3.647418in}{2.530525in}}%
\pgfpathlineto{\pgfqpoint{3.649149in}{2.560530in}}%
\pgfpathlineto{\pgfqpoint{3.650016in}{2.552415in}}%
\pgfpathlineto{\pgfqpoint{3.650882in}{2.573995in}}%
\pgfpathlineto{\pgfqpoint{3.651747in}{2.538273in}}%
\pgfpathlineto{\pgfqpoint{3.652611in}{2.541501in}}%
\pgfpathlineto{\pgfqpoint{3.653476in}{2.570921in}}%
\pgfpathlineto{\pgfqpoint{3.654343in}{2.519459in}}%
\pgfpathlineto{\pgfqpoint{3.655208in}{2.579344in}}%
\pgfpathlineto{\pgfqpoint{3.656940in}{2.458282in}}%
\pgfpathlineto{\pgfqpoint{3.659533in}{2.585923in}}%
\pgfpathlineto{\pgfqpoint{3.660399in}{2.498615in}}%
\pgfpathlineto{\pgfqpoint{3.661264in}{2.528313in}}%
\pgfpathlineto{\pgfqpoint{3.662127in}{2.497355in}}%
\pgfpathlineto{\pgfqpoint{3.662989in}{2.569077in}}%
\pgfpathlineto{\pgfqpoint{3.663855in}{2.525300in}}%
\pgfpathlineto{\pgfqpoint{3.664721in}{2.531787in}}%
\pgfpathlineto{\pgfqpoint{3.665587in}{2.515831in}}%
\pgfpathlineto{\pgfqpoint{3.666450in}{2.534953in}}%
\pgfpathlineto{\pgfqpoint{3.667313in}{2.502551in}}%
\pgfpathlineto{\pgfqpoint{3.669910in}{2.588137in}}%
\pgfpathlineto{\pgfqpoint{3.671641in}{2.511251in}}%
\pgfpathlineto{\pgfqpoint{3.672504in}{2.519151in}}%
\pgfpathlineto{\pgfqpoint{3.673368in}{2.515708in}}%
\pgfpathlineto{\pgfqpoint{3.674235in}{2.500891in}}%
\pgfpathlineto{\pgfqpoint{3.675101in}{2.514232in}}%
\pgfpathlineto{\pgfqpoint{3.675965in}{2.485581in}}%
\pgfpathlineto{\pgfqpoint{3.677697in}{2.573441in}}%
\pgfpathlineto{\pgfqpoint{3.678563in}{2.543253in}}%
\pgfpathlineto{\pgfqpoint{3.679428in}{2.484842in}}%
\pgfpathlineto{\pgfqpoint{3.680294in}{2.560715in}}%
\pgfpathlineto{\pgfqpoint{3.681160in}{2.454747in}}%
\pgfpathlineto{\pgfqpoint{3.682026in}{2.483798in}}%
\pgfpathlineto{\pgfqpoint{3.682891in}{2.562129in}}%
\pgfpathlineto{\pgfqpoint{3.684621in}{2.471501in}}%
\pgfpathlineto{\pgfqpoint{3.685485in}{2.517799in}}%
\pgfpathlineto{\pgfqpoint{3.686350in}{2.512018in}}%
\pgfpathlineto{\pgfqpoint{3.687215in}{2.522840in}}%
\pgfpathlineto{\pgfqpoint{3.688079in}{2.507408in}}%
\pgfpathlineto{\pgfqpoint{3.688943in}{2.535936in}}%
\pgfpathlineto{\pgfqpoint{3.689808in}{2.467503in}}%
\pgfpathlineto{\pgfqpoint{3.691536in}{2.520749in}}%
\pgfpathlineto{\pgfqpoint{3.692401in}{2.477217in}}%
\pgfpathlineto{\pgfqpoint{3.693266in}{2.539779in}}%
\pgfpathlineto{\pgfqpoint{3.694131in}{2.457728in}}%
\pgfpathlineto{\pgfqpoint{3.694996in}{2.531263in}}%
\pgfpathlineto{\pgfqpoint{3.695860in}{2.524192in}}%
\pgfpathlineto{\pgfqpoint{3.696724in}{2.503533in}}%
\pgfpathlineto{\pgfqpoint{3.698452in}{2.584571in}}%
\pgfpathlineto{\pgfqpoint{3.700181in}{2.481922in}}%
\pgfpathlineto{\pgfqpoint{3.701046in}{2.502304in}}%
\pgfpathlineto{\pgfqpoint{3.701911in}{2.584263in}}%
\pgfpathlineto{\pgfqpoint{3.702776in}{2.548079in}}%
\pgfpathlineto{\pgfqpoint{3.703642in}{2.550078in}}%
\pgfpathlineto{\pgfqpoint{3.705373in}{2.524439in}}%
\pgfpathlineto{\pgfqpoint{3.706238in}{2.550139in}}%
\pgfpathlineto{\pgfqpoint{3.707103in}{2.543745in}}%
\pgfpathlineto{\pgfqpoint{3.709698in}{2.481522in}}%
\pgfpathlineto{\pgfqpoint{3.710563in}{2.479985in}}%
\pgfpathlineto{\pgfqpoint{3.711428in}{2.459142in}}%
\pgfpathlineto{\pgfqpoint{3.712293in}{2.566832in}}%
\pgfpathlineto{\pgfqpoint{3.713157in}{2.522840in}}%
\pgfpathlineto{\pgfqpoint{3.714022in}{2.539194in}}%
\pgfpathlineto{\pgfqpoint{3.715748in}{2.449182in}}%
\pgfpathlineto{\pgfqpoint{3.716611in}{2.502612in}}%
\pgfpathlineto{\pgfqpoint{3.717477in}{2.465045in}}%
\pgfpathlineto{\pgfqpoint{3.719209in}{2.556350in}}%
\pgfpathlineto{\pgfqpoint{3.720074in}{2.551186in}}%
\pgfpathlineto{\pgfqpoint{3.720938in}{2.526837in}}%
\pgfpathlineto{\pgfqpoint{3.722666in}{2.586785in}}%
\pgfpathlineto{\pgfqpoint{3.724396in}{2.529172in}}%
\pgfpathlineto{\pgfqpoint{3.726126in}{2.585433in}}%
\pgfpathlineto{\pgfqpoint{3.726990in}{2.456437in}}%
\pgfpathlineto{\pgfqpoint{3.727855in}{2.555981in}}%
\pgfpathlineto{\pgfqpoint{3.728721in}{2.555766in}}%
\pgfpathlineto{\pgfqpoint{3.729587in}{2.528374in}}%
\pgfpathlineto{\pgfqpoint{3.730452in}{2.563114in}}%
\pgfpathlineto{\pgfqpoint{3.732179in}{2.512818in}}%
\pgfpathlineto{\pgfqpoint{3.733044in}{2.522717in}}%
\pgfpathlineto{\pgfqpoint{3.733910in}{2.490592in}}%
\pgfpathlineto{\pgfqpoint{3.735640in}{2.582972in}}%
\pgfpathlineto{\pgfqpoint{3.736506in}{2.499415in}}%
\pgfpathlineto{\pgfqpoint{3.737372in}{2.516631in}}%
\pgfpathlineto{\pgfqpoint{3.738238in}{2.485366in}}%
\pgfpathlineto{\pgfqpoint{3.739103in}{2.505747in}}%
\pgfpathlineto{\pgfqpoint{3.739967in}{2.484781in}}%
\pgfpathlineto{\pgfqpoint{3.741695in}{2.527574in}}%
\pgfpathlineto{\pgfqpoint{3.742560in}{2.459265in}}%
\pgfpathlineto{\pgfqpoint{3.743425in}{2.507592in}}%
\pgfpathlineto{\pgfqpoint{3.744291in}{2.467873in}}%
\pgfpathlineto{\pgfqpoint{3.745156in}{2.556043in}}%
\pgfpathlineto{\pgfqpoint{3.746022in}{2.540856in}}%
\pgfpathlineto{\pgfqpoint{3.747753in}{2.487210in}}%
\pgfpathlineto{\pgfqpoint{3.749484in}{2.549770in}}%
\pgfpathlineto{\pgfqpoint{3.750349in}{2.526468in}}%
\pgfpathlineto{\pgfqpoint{3.751213in}{2.517245in}}%
\pgfpathlineto{\pgfqpoint{3.752078in}{2.527266in}}%
\pgfpathlineto{\pgfqpoint{3.753810in}{2.481276in}}%
\pgfpathlineto{\pgfqpoint{3.754676in}{2.563419in}}%
\pgfpathlineto{\pgfqpoint{3.755541in}{2.538027in}}%
\pgfpathlineto{\pgfqpoint{3.756406in}{2.464122in}}%
\pgfpathlineto{\pgfqpoint{3.758138in}{2.531325in}}%
\pgfpathlineto{\pgfqpoint{3.759004in}{2.531202in}}%
\pgfpathlineto{\pgfqpoint{3.759869in}{2.484935in}}%
\pgfpathlineto{\pgfqpoint{3.761597in}{2.560715in}}%
\pgfpathlineto{\pgfqpoint{3.762463in}{2.474574in}}%
\pgfpathlineto{\pgfqpoint{3.763326in}{2.478818in}}%
\pgfpathlineto{\pgfqpoint{3.765921in}{2.555119in}}%
\pgfpathlineto{\pgfqpoint{3.767650in}{2.489699in}}%
\pgfpathlineto{\pgfqpoint{3.768513in}{2.561267in}}%
\pgfpathlineto{\pgfqpoint{3.769378in}{2.482721in}}%
\pgfpathlineto{\pgfqpoint{3.771106in}{2.575039in}}%
\pgfpathlineto{\pgfqpoint{3.772835in}{2.542085in}}%
\pgfpathlineto{\pgfqpoint{3.774564in}{2.572397in}}%
\pgfpathlineto{\pgfqpoint{3.775427in}{2.521488in}}%
\pgfpathlineto{\pgfqpoint{3.776290in}{2.528128in}}%
\pgfpathlineto{\pgfqpoint{3.777155in}{2.623736in}}%
\pgfpathlineto{\pgfqpoint{3.778884in}{2.479924in}}%
\pgfpathlineto{\pgfqpoint{3.779750in}{2.562375in}}%
\pgfpathlineto{\pgfqpoint{3.780615in}{2.543253in}}%
\pgfpathlineto{\pgfqpoint{3.781479in}{2.454223in}}%
\pgfpathlineto{\pgfqpoint{3.783208in}{2.560961in}}%
\pgfpathlineto{\pgfqpoint{3.784074in}{2.488593in}}%
\pgfpathlineto{\pgfqpoint{3.784940in}{2.541223in}}%
\pgfpathlineto{\pgfqpoint{3.787530in}{2.500337in}}%
\pgfpathlineto{\pgfqpoint{3.789261in}{2.556841in}}%
\pgfpathlineto{\pgfqpoint{3.790125in}{2.567109in}}%
\pgfpathlineto{\pgfqpoint{3.791853in}{2.457420in}}%
\pgfpathlineto{\pgfqpoint{3.793584in}{2.527389in}}%
\pgfpathlineto{\pgfqpoint{3.795314in}{2.600434in}}%
\pgfpathlineto{\pgfqpoint{3.797911in}{2.458988in}}%
\pgfpathlineto{\pgfqpoint{3.798777in}{2.560530in}}%
\pgfpathlineto{\pgfqpoint{3.799640in}{2.529665in}}%
\pgfpathlineto{\pgfqpoint{3.800503in}{2.541193in}}%
\pgfpathlineto{\pgfqpoint{3.802234in}{2.507531in}}%
\pgfpathlineto{\pgfqpoint{3.803099in}{2.505809in}}%
\pgfpathlineto{\pgfqpoint{3.803964in}{2.542208in}}%
\pgfpathlineto{\pgfqpoint{3.804830in}{2.541624in}}%
\pgfpathlineto{\pgfqpoint{3.807428in}{2.482721in}}%
\pgfpathlineto{\pgfqpoint{3.809157in}{2.548479in}}%
\pgfpathlineto{\pgfqpoint{3.810020in}{2.481892in}}%
\pgfpathlineto{\pgfqpoint{3.810885in}{2.576578in}}%
\pgfpathlineto{\pgfqpoint{3.811751in}{2.495726in}}%
\pgfpathlineto{\pgfqpoint{3.812616in}{2.495849in}}%
\pgfpathlineto{\pgfqpoint{3.813481in}{2.501014in}}%
\pgfpathlineto{\pgfqpoint{3.814344in}{2.452902in}}%
\pgfpathlineto{\pgfqpoint{3.816076in}{2.521180in}}%
\pgfpathlineto{\pgfqpoint{3.817803in}{2.456684in}}%
\pgfpathlineto{\pgfqpoint{3.818667in}{2.514171in}}%
\pgfpathlineto{\pgfqpoint{3.819532in}{2.433687in}}%
\pgfpathlineto{\pgfqpoint{3.821264in}{2.605353in}}%
\pgfpathlineto{\pgfqpoint{3.822992in}{2.548171in}}%
\pgfpathlineto{\pgfqpoint{3.823858in}{2.575132in}}%
\pgfpathlineto{\pgfqpoint{3.824723in}{2.567109in}}%
\pgfpathlineto{\pgfqpoint{3.825589in}{2.580512in}}%
\pgfpathlineto{\pgfqpoint{3.826455in}{2.568615in}}%
\pgfpathlineto{\pgfqpoint{3.827320in}{2.594592in}}%
\pgfpathlineto{\pgfqpoint{3.828185in}{2.556348in}}%
\pgfpathlineto{\pgfqpoint{3.829052in}{2.567999in}}%
\pgfpathlineto{\pgfqpoint{3.830781in}{2.530771in}}%
\pgfpathlineto{\pgfqpoint{3.831644in}{2.582357in}}%
\pgfpathlineto{\pgfqpoint{3.832509in}{2.536120in}}%
\pgfpathlineto{\pgfqpoint{3.835105in}{2.650175in}}%
\pgfpathlineto{\pgfqpoint{3.835971in}{2.562190in}}%
\pgfpathlineto{\pgfqpoint{3.836836in}{2.630376in}}%
\pgfpathlineto{\pgfqpoint{3.837698in}{2.533170in}}%
\pgfpathlineto{\pgfqpoint{3.838562in}{2.567355in}}%
\pgfpathlineto{\pgfqpoint{3.841155in}{2.502120in}}%
\pgfpathlineto{\pgfqpoint{3.842018in}{2.586108in}}%
\pgfpathlineto{\pgfqpoint{3.842884in}{2.528097in}}%
\pgfpathlineto{\pgfqpoint{3.843749in}{2.548171in}}%
\pgfpathlineto{\pgfqpoint{3.844612in}{2.534583in}}%
\pgfpathlineto{\pgfqpoint{3.845478in}{2.541193in}}%
\pgfpathlineto{\pgfqpoint{3.846342in}{2.493142in}}%
\pgfpathlineto{\pgfqpoint{3.847206in}{2.560161in}}%
\pgfpathlineto{\pgfqpoint{3.848933in}{2.517491in}}%
\pgfpathlineto{\pgfqpoint{3.849796in}{2.551799in}}%
\pgfpathlineto{\pgfqpoint{3.850662in}{2.479862in}}%
\pgfpathlineto{\pgfqpoint{3.851528in}{2.584325in}}%
\pgfpathlineto{\pgfqpoint{3.852393in}{2.508729in}}%
\pgfpathlineto{\pgfqpoint{3.853258in}{2.633327in}}%
\pgfpathlineto{\pgfqpoint{3.854987in}{2.544697in}}%
\pgfpathlineto{\pgfqpoint{3.855851in}{2.525421in}}%
\pgfpathlineto{\pgfqpoint{3.856715in}{2.566801in}}%
\pgfpathlineto{\pgfqpoint{3.857580in}{2.489607in}}%
\pgfpathlineto{\pgfqpoint{3.858443in}{2.558439in}}%
\pgfpathlineto{\pgfqpoint{3.859307in}{2.485948in}}%
\pgfpathlineto{\pgfqpoint{3.860171in}{2.524069in}}%
\pgfpathlineto{\pgfqpoint{3.861036in}{2.512141in}}%
\pgfpathlineto{\pgfqpoint{3.861901in}{2.556471in}}%
\pgfpathlineto{\pgfqpoint{3.862767in}{2.548048in}}%
\pgfpathlineto{\pgfqpoint{3.863631in}{2.568215in}}%
\pgfpathlineto{\pgfqpoint{3.864495in}{2.549831in}}%
\pgfpathlineto{\pgfqpoint{3.865358in}{2.586169in}}%
\pgfpathlineto{\pgfqpoint{3.867087in}{2.483367in}}%
\pgfpathlineto{\pgfqpoint{3.867951in}{2.524192in}}%
\pgfpathlineto{\pgfqpoint{3.868815in}{2.517675in}}%
\pgfpathlineto{\pgfqpoint{3.869677in}{2.536367in}}%
\pgfpathlineto{\pgfqpoint{3.870542in}{2.506915in}}%
\pgfpathlineto{\pgfqpoint{3.871407in}{2.577377in}}%
\pgfpathlineto{\pgfqpoint{3.874002in}{2.505409in}}%
\pgfpathlineto{\pgfqpoint{3.874867in}{2.548972in}}%
\pgfpathlineto{\pgfqpoint{3.875730in}{2.506792in}}%
\pgfpathlineto{\pgfqpoint{3.879189in}{2.609779in}}%
\pgfpathlineto{\pgfqpoint{3.880052in}{2.520565in}}%
\pgfpathlineto{\pgfqpoint{3.880914in}{2.544851in}}%
\pgfpathlineto{\pgfqpoint{3.881778in}{2.500583in}}%
\pgfpathlineto{\pgfqpoint{3.882643in}{2.566924in}}%
\pgfpathlineto{\pgfqpoint{3.883507in}{2.508513in}}%
\pgfpathlineto{\pgfqpoint{3.884372in}{2.534645in}}%
\pgfpathlineto{\pgfqpoint{3.885238in}{2.481461in}}%
\pgfpathlineto{\pgfqpoint{3.886103in}{2.536459in}}%
\pgfpathlineto{\pgfqpoint{3.886970in}{2.495172in}}%
\pgfpathlineto{\pgfqpoint{3.887836in}{2.582049in}}%
\pgfpathlineto{\pgfqpoint{3.888702in}{2.479000in}}%
\pgfpathlineto{\pgfqpoint{3.889568in}{2.521917in}}%
\pgfpathlineto{\pgfqpoint{3.890434in}{2.508883in}}%
\pgfpathlineto{\pgfqpoint{3.892166in}{2.547312in}}%
\pgfpathlineto{\pgfqpoint{3.893031in}{2.531417in}}%
\pgfpathlineto{\pgfqpoint{3.893896in}{2.579344in}}%
\pgfpathlineto{\pgfqpoint{3.895627in}{2.527728in}}%
\pgfpathlineto{\pgfqpoint{3.896493in}{2.591272in}}%
\pgfpathlineto{\pgfqpoint{3.897360in}{2.536982in}}%
\pgfpathlineto{\pgfqpoint{3.898225in}{2.540118in}}%
\pgfpathlineto{\pgfqpoint{3.899091in}{2.515954in}}%
\pgfpathlineto{\pgfqpoint{3.900822in}{2.578914in}}%
\pgfpathlineto{\pgfqpoint{3.901689in}{2.484965in}}%
\pgfpathlineto{\pgfqpoint{3.903419in}{2.544913in}}%
\pgfpathlineto{\pgfqpoint{3.904283in}{2.532800in}}%
\pgfpathlineto{\pgfqpoint{3.905147in}{2.548356in}}%
\pgfpathlineto{\pgfqpoint{3.906011in}{2.478387in}}%
\pgfpathlineto{\pgfqpoint{3.906875in}{2.563727in}}%
\pgfpathlineto{\pgfqpoint{3.907739in}{2.484842in}}%
\pgfpathlineto{\pgfqpoint{3.908603in}{2.533722in}}%
\pgfpathlineto{\pgfqpoint{3.909466in}{2.506176in}}%
\pgfpathlineto{\pgfqpoint{3.910332in}{2.527020in}}%
\pgfpathlineto{\pgfqpoint{3.912926in}{2.487670in}}%
\pgfpathlineto{\pgfqpoint{3.913792in}{2.501381in}}%
\pgfpathlineto{\pgfqpoint{3.915522in}{2.566678in}}%
\pgfpathlineto{\pgfqpoint{3.916387in}{2.490990in}}%
\pgfpathlineto{\pgfqpoint{3.917253in}{2.510543in}}%
\pgfpathlineto{\pgfqpoint{3.918118in}{2.582357in}}%
\pgfpathlineto{\pgfqpoint{3.918984in}{2.575255in}}%
\pgfpathlineto{\pgfqpoint{3.919849in}{2.495233in}}%
\pgfpathlineto{\pgfqpoint{3.920712in}{2.568830in}}%
\pgfpathlineto{\pgfqpoint{3.921578in}{2.493481in}}%
\pgfpathlineto{\pgfqpoint{3.922443in}{2.530956in}}%
\pgfpathlineto{\pgfqpoint{3.923309in}{2.527820in}}%
\pgfpathlineto{\pgfqpoint{3.924173in}{2.474451in}}%
\pgfpathlineto{\pgfqpoint{3.925036in}{2.510543in}}%
\pgfpathlineto{\pgfqpoint{3.925900in}{2.482813in}}%
\pgfpathlineto{\pgfqpoint{3.927629in}{2.547494in}}%
\pgfpathlineto{\pgfqpoint{3.928493in}{2.501258in}}%
\pgfpathlineto{\pgfqpoint{3.930224in}{2.559853in}}%
\pgfpathlineto{\pgfqpoint{3.931088in}{2.517245in}}%
\pgfpathlineto{\pgfqpoint{3.931951in}{2.527512in}}%
\pgfpathlineto{\pgfqpoint{3.932817in}{2.506730in}}%
\pgfpathlineto{\pgfqpoint{3.933682in}{2.540762in}}%
\pgfpathlineto{\pgfqpoint{3.934547in}{2.520195in}}%
\pgfpathlineto{\pgfqpoint{3.937139in}{2.571596in}}%
\pgfpathlineto{\pgfqpoint{3.938004in}{2.506884in}}%
\pgfpathlineto{\pgfqpoint{3.939731in}{2.564648in}}%
\pgfpathlineto{\pgfqpoint{3.940595in}{2.511341in}}%
\pgfpathlineto{\pgfqpoint{3.941461in}{2.516814in}}%
\pgfpathlineto{\pgfqpoint{3.942327in}{2.583401in}}%
\pgfpathlineto{\pgfqpoint{3.943191in}{2.536857in}}%
\pgfpathlineto{\pgfqpoint{3.944057in}{2.543620in}}%
\pgfpathlineto{\pgfqpoint{3.945788in}{2.590164in}}%
\pgfpathlineto{\pgfqpoint{3.946653in}{2.496216in}}%
\pgfpathlineto{\pgfqpoint{3.947519in}{2.499413in}}%
\pgfpathlineto{\pgfqpoint{3.948385in}{2.460186in}}%
\pgfpathlineto{\pgfqpoint{3.949249in}{2.516260in}}%
\pgfpathlineto{\pgfqpoint{3.950114in}{2.507775in}}%
\pgfpathlineto{\pgfqpoint{3.950979in}{2.538394in}}%
\pgfpathlineto{\pgfqpoint{3.951843in}{2.530586in}}%
\pgfpathlineto{\pgfqpoint{3.952708in}{2.487362in}}%
\pgfpathlineto{\pgfqpoint{3.953573in}{2.585061in}}%
\pgfpathlineto{\pgfqpoint{3.954436in}{2.578298in}}%
\pgfpathlineto{\pgfqpoint{3.955301in}{2.615926in}}%
\pgfpathlineto{\pgfqpoint{3.957031in}{2.535197in}}%
\pgfpathlineto{\pgfqpoint{3.957894in}{2.555486in}}%
\pgfpathlineto{\pgfqpoint{3.959623in}{2.518289in}}%
\pgfpathlineto{\pgfqpoint{3.960486in}{2.542268in}}%
\pgfpathlineto{\pgfqpoint{3.961351in}{2.515030in}}%
\pgfpathlineto{\pgfqpoint{3.963081in}{2.586967in}}%
\pgfpathlineto{\pgfqpoint{3.963945in}{2.535382in}}%
\pgfpathlineto{\pgfqpoint{3.964811in}{2.588504in}}%
\pgfpathlineto{\pgfqpoint{3.965676in}{2.542514in}}%
\pgfpathlineto{\pgfqpoint{3.966542in}{2.566124in}}%
\pgfpathlineto{\pgfqpoint{3.967407in}{2.545711in}}%
\pgfpathlineto{\pgfqpoint{3.968272in}{2.552905in}}%
\pgfpathlineto{\pgfqpoint{3.969137in}{2.552351in}}%
\pgfpathlineto{\pgfqpoint{3.970002in}{2.580450in}}%
\pgfpathlineto{\pgfqpoint{3.971732in}{2.504147in}}%
\pgfpathlineto{\pgfqpoint{3.972596in}{2.507590in}}%
\pgfpathlineto{\pgfqpoint{3.973460in}{2.534520in}}%
\pgfpathlineto{\pgfqpoint{3.974325in}{2.624350in}}%
\pgfpathlineto{\pgfqpoint{3.976055in}{2.512447in}}%
\pgfpathlineto{\pgfqpoint{3.976921in}{2.574362in}}%
\pgfpathlineto{\pgfqpoint{3.978650in}{2.495631in}}%
\pgfpathlineto{\pgfqpoint{3.979516in}{2.542453in}}%
\pgfpathlineto{\pgfqpoint{3.981249in}{2.457880in}}%
\pgfpathlineto{\pgfqpoint{3.982981in}{2.577682in}}%
\pgfpathlineto{\pgfqpoint{3.983846in}{2.482505in}}%
\pgfpathlineto{\pgfqpoint{3.984712in}{2.513432in}}%
\pgfpathlineto{\pgfqpoint{3.985575in}{2.496247in}}%
\pgfpathlineto{\pgfqpoint{3.986438in}{2.441310in}}%
\pgfpathlineto{\pgfqpoint{3.987304in}{2.451825in}}%
\pgfpathlineto{\pgfqpoint{3.988169in}{2.450472in}}%
\pgfpathlineto{\pgfqpoint{3.989902in}{2.541285in}}%
\pgfpathlineto{\pgfqpoint{3.990765in}{2.445369in}}%
\pgfpathlineto{\pgfqpoint{3.991629in}{2.488039in}}%
\pgfpathlineto{\pgfqpoint{3.992494in}{2.458896in}}%
\pgfpathlineto{\pgfqpoint{3.993360in}{2.554319in}}%
\pgfpathlineto{\pgfqpoint{3.994223in}{2.530956in}}%
\pgfpathlineto{\pgfqpoint{3.995089in}{2.535566in}}%
\pgfpathlineto{\pgfqpoint{3.995954in}{2.587398in}}%
\pgfpathlineto{\pgfqpoint{3.996820in}{2.541593in}}%
\pgfpathlineto{\pgfqpoint{3.997684in}{2.549893in}}%
\pgfpathlineto{\pgfqpoint{3.998549in}{2.554811in}}%
\pgfpathlineto{\pgfqpoint{3.999414in}{2.619554in}}%
\pgfpathlineto{\pgfqpoint{4.000279in}{2.513432in}}%
\pgfpathlineto{\pgfqpoint{4.001143in}{2.524192in}}%
\pgfpathlineto{\pgfqpoint{4.002007in}{2.500337in}}%
\pgfpathlineto{\pgfqpoint{4.002874in}{2.568338in}}%
\pgfpathlineto{\pgfqpoint{4.004603in}{2.513647in}}%
\pgfpathlineto{\pgfqpoint{4.005467in}{2.522779in}}%
\pgfpathlineto{\pgfqpoint{4.006327in}{2.564343in}}%
\pgfpathlineto{\pgfqpoint{4.008057in}{2.470700in}}%
\pgfpathlineto{\pgfqpoint{4.008923in}{2.597420in}}%
\pgfpathlineto{\pgfqpoint{4.009788in}{2.578544in}}%
\pgfpathlineto{\pgfqpoint{4.012381in}{2.492160in}}%
\pgfpathlineto{\pgfqpoint{4.013246in}{2.509006in}}%
\pgfpathlineto{\pgfqpoint{4.014974in}{2.478387in}}%
\pgfpathlineto{\pgfqpoint{4.015839in}{2.570675in}}%
\pgfpathlineto{\pgfqpoint{4.016704in}{2.531140in}}%
\pgfpathlineto{\pgfqpoint{4.018434in}{2.556533in}}%
\pgfpathlineto{\pgfqpoint{4.019301in}{2.481584in}}%
\pgfpathlineto{\pgfqpoint{4.020166in}{2.485150in}}%
\pgfpathlineto{\pgfqpoint{4.021898in}{2.532800in}}%
\pgfpathlineto{\pgfqpoint{4.022764in}{2.508267in}}%
\pgfpathlineto{\pgfqpoint{4.023629in}{2.519028in}}%
\pgfpathlineto{\pgfqpoint{4.025359in}{2.558316in}}%
\pgfpathlineto{\pgfqpoint{4.026224in}{2.555058in}}%
\pgfpathlineto{\pgfqpoint{4.027087in}{2.489945in}}%
\pgfpathlineto{\pgfqpoint{4.028818in}{2.547617in}}%
\pgfpathlineto{\pgfqpoint{4.029684in}{2.484288in}}%
\pgfpathlineto{\pgfqpoint{4.030549in}{2.570365in}}%
\pgfpathlineto{\pgfqpoint{4.031412in}{2.566062in}}%
\pgfpathlineto{\pgfqpoint{4.032277in}{2.561698in}}%
\pgfpathlineto{\pgfqpoint{4.033143in}{2.522040in}}%
\pgfpathlineto{\pgfqpoint{4.034007in}{2.535197in}}%
\pgfpathlineto{\pgfqpoint{4.034872in}{2.580143in}}%
\pgfpathlineto{\pgfqpoint{4.035739in}{2.484165in}}%
\pgfpathlineto{\pgfqpoint{4.036604in}{2.546327in}}%
\pgfpathlineto{\pgfqpoint{4.037469in}{2.491298in}}%
\pgfpathlineto{\pgfqpoint{4.039199in}{2.564710in}}%
\pgfpathlineto{\pgfqpoint{4.040065in}{2.556471in}}%
\pgfpathlineto{\pgfqpoint{4.040931in}{2.503657in}}%
\pgfpathlineto{\pgfqpoint{4.041797in}{2.569998in}}%
\pgfpathlineto{\pgfqpoint{4.044392in}{2.497507in}}%
\pgfpathlineto{\pgfqpoint{4.045257in}{2.513771in}}%
\pgfpathlineto{\pgfqpoint{4.046121in}{2.563789in}}%
\pgfpathlineto{\pgfqpoint{4.046986in}{2.562190in}}%
\pgfpathlineto{\pgfqpoint{4.047850in}{2.508760in}}%
\pgfpathlineto{\pgfqpoint{4.049579in}{2.558562in}}%
\pgfpathlineto{\pgfqpoint{4.051310in}{2.501689in}}%
\pgfpathlineto{\pgfqpoint{4.053040in}{2.575778in}}%
\pgfpathlineto{\pgfqpoint{4.053906in}{2.459080in}}%
\pgfpathlineto{\pgfqpoint{4.054771in}{2.545344in}}%
\pgfpathlineto{\pgfqpoint{4.055636in}{2.535507in}}%
\pgfpathlineto{\pgfqpoint{4.056502in}{2.497140in}}%
\pgfpathlineto{\pgfqpoint{4.057366in}{2.606397in}}%
\pgfpathlineto{\pgfqpoint{4.059092in}{2.529234in}}%
\pgfpathlineto{\pgfqpoint{4.059957in}{2.535751in}}%
\pgfpathlineto{\pgfqpoint{4.060822in}{2.500583in}}%
\pgfpathlineto{\pgfqpoint{4.062552in}{2.552353in}}%
\pgfpathlineto{\pgfqpoint{4.064282in}{2.519766in}}%
\pgfpathlineto{\pgfqpoint{4.065147in}{2.533908in}}%
\pgfpathlineto{\pgfqpoint{4.066012in}{2.528128in}}%
\pgfpathlineto{\pgfqpoint{4.066875in}{2.551953in}}%
\pgfpathlineto{\pgfqpoint{4.067741in}{2.476296in}}%
\pgfpathlineto{\pgfqpoint{4.068604in}{2.512818in}}%
\pgfpathlineto{\pgfqpoint{4.069469in}{2.491975in}}%
\pgfpathlineto{\pgfqpoint{4.071198in}{2.570552in}}%
\pgfpathlineto{\pgfqpoint{4.072928in}{2.485150in}}%
\pgfpathlineto{\pgfqpoint{4.075522in}{2.542945in}}%
\pgfpathlineto{\pgfqpoint{4.076388in}{2.527574in}}%
\pgfpathlineto{\pgfqpoint{4.077253in}{2.533170in}}%
\pgfpathlineto{\pgfqpoint{4.078118in}{2.573810in}}%
\pgfpathlineto{\pgfqpoint{4.078984in}{2.566986in}}%
\pgfpathlineto{\pgfqpoint{4.079849in}{2.536243in}}%
\pgfpathlineto{\pgfqpoint{4.080713in}{2.564127in}}%
\pgfpathlineto{\pgfqpoint{4.081580in}{2.527943in}}%
\pgfpathlineto{\pgfqpoint{4.084175in}{2.581251in}}%
\pgfpathlineto{\pgfqpoint{4.085039in}{2.517429in}}%
\pgfpathlineto{\pgfqpoint{4.087632in}{2.601356in}}%
\pgfpathlineto{\pgfqpoint{4.088498in}{2.505686in}}%
\pgfpathlineto{\pgfqpoint{4.089364in}{2.523271in}}%
\pgfpathlineto{\pgfqpoint{4.090229in}{2.551645in}}%
\pgfpathlineto{\pgfqpoint{4.091095in}{2.620231in}}%
\pgfpathlineto{\pgfqpoint{4.091961in}{2.607136in}}%
\pgfpathlineto{\pgfqpoint{4.092826in}{2.559332in}}%
\pgfpathlineto{\pgfqpoint{4.093691in}{2.653126in}}%
\pgfpathlineto{\pgfqpoint{4.095420in}{2.539042in}}%
\pgfpathlineto{\pgfqpoint{4.096286in}{2.586723in}}%
\pgfpathlineto{\pgfqpoint{4.098016in}{2.545344in}}%
\pgfpathlineto{\pgfqpoint{4.098881in}{2.606890in}}%
\pgfpathlineto{\pgfqpoint{4.099746in}{2.547096in}}%
\pgfpathlineto{\pgfqpoint{4.100611in}{2.559301in}}%
\pgfpathlineto{\pgfqpoint{4.101476in}{2.548541in}}%
\pgfpathlineto{\pgfqpoint{4.102341in}{2.585954in}}%
\pgfpathlineto{\pgfqpoint{4.103206in}{2.524069in}}%
\pgfpathlineto{\pgfqpoint{4.104936in}{2.579129in}}%
\pgfpathlineto{\pgfqpoint{4.105802in}{2.563604in}}%
\pgfpathlineto{\pgfqpoint{4.107530in}{2.410814in}}%
\pgfpathlineto{\pgfqpoint{4.109259in}{2.568092in}}%
\pgfpathlineto{\pgfqpoint{4.111854in}{2.436515in}}%
\pgfpathlineto{\pgfqpoint{4.113584in}{2.551553in}}%
\pgfpathlineto{\pgfqpoint{4.115313in}{2.486318in}}%
\pgfpathlineto{\pgfqpoint{4.116177in}{2.488162in}}%
\pgfpathlineto{\pgfqpoint{4.117040in}{2.511772in}}%
\pgfpathlineto{\pgfqpoint{4.118770in}{2.444877in}}%
\pgfpathlineto{\pgfqpoint{4.120501in}{2.574978in}}%
\pgfpathlineto{\pgfqpoint{4.121367in}{2.511187in}}%
\pgfpathlineto{\pgfqpoint{4.122231in}{2.564587in}}%
\pgfpathlineto{\pgfqpoint{4.124826in}{2.494433in}}%
\pgfpathlineto{\pgfqpoint{4.125692in}{2.556164in}}%
\pgfpathlineto{\pgfqpoint{4.126557in}{2.516383in}}%
\pgfpathlineto{\pgfqpoint{4.128288in}{2.550907in}}%
\pgfpathlineto{\pgfqpoint{4.129154in}{2.530402in}}%
\pgfpathlineto{\pgfqpoint{4.130019in}{2.539440in}}%
\pgfpathlineto{\pgfqpoint{4.130884in}{2.515215in}}%
\pgfpathlineto{\pgfqpoint{4.131746in}{2.541347in}}%
\pgfpathlineto{\pgfqpoint{4.132610in}{2.455021in}}%
\pgfpathlineto{\pgfqpoint{4.134339in}{2.546573in}}%
\pgfpathlineto{\pgfqpoint{4.135203in}{2.509314in}}%
\pgfpathlineto{\pgfqpoint{4.136068in}{2.515831in}}%
\pgfpathlineto{\pgfqpoint{4.136933in}{2.591088in}}%
\pgfpathlineto{\pgfqpoint{4.137798in}{2.539071in}}%
\pgfpathlineto{\pgfqpoint{4.138664in}{2.618140in}}%
\pgfpathlineto{\pgfqpoint{4.139529in}{2.525791in}}%
\pgfpathlineto{\pgfqpoint{4.140395in}{2.562436in}}%
\pgfpathlineto{\pgfqpoint{4.141259in}{2.550385in}}%
\pgfpathlineto{\pgfqpoint{4.142124in}{2.603693in}}%
\pgfpathlineto{\pgfqpoint{4.142987in}{2.461417in}}%
\pgfpathlineto{\pgfqpoint{4.143853in}{2.552415in}}%
\pgfpathlineto{\pgfqpoint{4.144718in}{2.535137in}}%
\pgfpathlineto{\pgfqpoint{4.145583in}{2.555796in}}%
\pgfpathlineto{\pgfqpoint{4.146448in}{2.543376in}}%
\pgfpathlineto{\pgfqpoint{4.148177in}{2.593179in}}%
\pgfpathlineto{\pgfqpoint{4.149043in}{2.490438in}}%
\pgfpathlineto{\pgfqpoint{4.149908in}{2.507377in}}%
\pgfpathlineto{\pgfqpoint{4.150774in}{2.477833in}}%
\pgfpathlineto{\pgfqpoint{4.152507in}{2.532616in}}%
\pgfpathlineto{\pgfqpoint{4.153374in}{2.458649in}}%
\pgfpathlineto{\pgfqpoint{4.154234in}{2.474267in}}%
\pgfpathlineto{\pgfqpoint{4.155100in}{2.479370in}}%
\pgfpathlineto{\pgfqpoint{4.157692in}{2.563727in}}%
\pgfpathlineto{\pgfqpoint{4.158557in}{2.521303in}}%
\pgfpathlineto{\pgfqpoint{4.159420in}{2.599511in}}%
\pgfpathlineto{\pgfqpoint{4.160285in}{2.595822in}}%
\pgfpathlineto{\pgfqpoint{4.162013in}{2.539194in}}%
\pgfpathlineto{\pgfqpoint{4.162876in}{2.547679in}}%
\pgfpathlineto{\pgfqpoint{4.163741in}{2.532369in}}%
\pgfpathlineto{\pgfqpoint{4.164607in}{2.468794in}}%
\pgfpathlineto{\pgfqpoint{4.165472in}{2.562927in}}%
\pgfpathlineto{\pgfqpoint{4.166337in}{2.494772in}}%
\pgfpathlineto{\pgfqpoint{4.167203in}{2.525421in}}%
\pgfpathlineto{\pgfqpoint{4.168067in}{2.442724in}}%
\pgfpathlineto{\pgfqpoint{4.169798in}{2.495785in}}%
\pgfpathlineto{\pgfqpoint{4.170663in}{2.456251in}}%
\pgfpathlineto{\pgfqpoint{4.173258in}{2.540362in}}%
\pgfpathlineto{\pgfqpoint{4.174125in}{2.538948in}}%
\pgfpathlineto{\pgfqpoint{4.176724in}{2.462092in}}%
\pgfpathlineto{\pgfqpoint{4.177588in}{2.537534in}}%
\pgfpathlineto{\pgfqpoint{4.179317in}{2.458772in}}%
\pgfpathlineto{\pgfqpoint{4.181050in}{2.558131in}}%
\pgfpathlineto{\pgfqpoint{4.182780in}{2.477279in}}%
\pgfpathlineto{\pgfqpoint{4.183646in}{2.544482in}}%
\pgfpathlineto{\pgfqpoint{4.184511in}{2.480476in}}%
\pgfpathlineto{\pgfqpoint{4.185377in}{2.496309in}}%
\pgfpathlineto{\pgfqpoint{4.186240in}{2.508637in}}%
\pgfpathlineto{\pgfqpoint{4.187102in}{2.491606in}}%
\pgfpathlineto{\pgfqpoint{4.187968in}{2.501658in}}%
\pgfpathlineto{\pgfqpoint{4.189700in}{2.547186in}}%
\pgfpathlineto{\pgfqpoint{4.190565in}{2.479093in}}%
\pgfpathlineto{\pgfqpoint{4.191432in}{2.529234in}}%
\pgfpathlineto{\pgfqpoint{4.192297in}{2.482382in}}%
\pgfpathlineto{\pgfqpoint{4.193163in}{2.556041in}}%
\pgfpathlineto{\pgfqpoint{4.195760in}{2.489084in}}%
\pgfpathlineto{\pgfqpoint{4.196623in}{2.516321in}}%
\pgfpathlineto{\pgfqpoint{4.198352in}{2.485394in}}%
\pgfpathlineto{\pgfqpoint{4.199218in}{2.591270in}}%
\pgfpathlineto{\pgfqpoint{4.200949in}{2.526958in}}%
\pgfpathlineto{\pgfqpoint{4.201815in}{2.584876in}}%
\pgfpathlineto{\pgfqpoint{4.202680in}{2.499598in}}%
\pgfpathlineto{\pgfqpoint{4.203545in}{2.555979in}}%
\pgfpathlineto{\pgfqpoint{4.204411in}{2.502579in}}%
\pgfpathlineto{\pgfqpoint{4.206142in}{2.542760in}}%
\pgfpathlineto{\pgfqpoint{4.207008in}{2.496863in}}%
\pgfpathlineto{\pgfqpoint{4.208736in}{2.559668in}}%
\pgfpathlineto{\pgfqpoint{4.209602in}{2.549985in}}%
\pgfpathlineto{\pgfqpoint{4.210467in}{2.592871in}}%
\pgfpathlineto{\pgfqpoint{4.211332in}{2.539071in}}%
\pgfpathlineto{\pgfqpoint{4.212198in}{2.546388in}}%
\pgfpathlineto{\pgfqpoint{4.213060in}{2.515030in}}%
\pgfpathlineto{\pgfqpoint{4.213925in}{2.520411in}}%
\pgfpathlineto{\pgfqpoint{4.214789in}{2.549339in}}%
\pgfpathlineto{\pgfqpoint{4.215654in}{2.539810in}}%
\pgfpathlineto{\pgfqpoint{4.216520in}{2.512295in}}%
\pgfpathlineto{\pgfqpoint{4.218250in}{2.592625in}}%
\pgfpathlineto{\pgfqpoint{4.219112in}{2.515677in}}%
\pgfpathlineto{\pgfqpoint{4.219976in}{2.557210in}}%
\pgfpathlineto{\pgfqpoint{4.220842in}{2.548602in}}%
\pgfpathlineto{\pgfqpoint{4.221705in}{2.510420in}}%
\pgfpathlineto{\pgfqpoint{4.222571in}{2.577684in}}%
\pgfpathlineto{\pgfqpoint{4.223436in}{2.536274in}}%
\pgfpathlineto{\pgfqpoint{4.225166in}{2.585985in}}%
\pgfpathlineto{\pgfqpoint{4.226894in}{2.491298in}}%
\pgfpathlineto{\pgfqpoint{4.227760in}{2.568523in}}%
\pgfpathlineto{\pgfqpoint{4.228626in}{2.526099in}}%
\pgfpathlineto{\pgfqpoint{4.229490in}{2.532677in}}%
\pgfpathlineto{\pgfqpoint{4.230356in}{2.531325in}}%
\pgfpathlineto{\pgfqpoint{4.231219in}{2.542945in}}%
\pgfpathlineto{\pgfqpoint{4.232084in}{2.508760in}}%
\pgfpathlineto{\pgfqpoint{4.233814in}{2.539502in}}%
\pgfpathlineto{\pgfqpoint{4.234678in}{2.584509in}}%
\pgfpathlineto{\pgfqpoint{4.235543in}{2.571966in}}%
\pgfpathlineto{\pgfqpoint{4.236409in}{2.589181in}}%
\pgfpathlineto{\pgfqpoint{4.237273in}{2.520749in}}%
\pgfpathlineto{\pgfqpoint{4.238138in}{2.525514in}}%
\pgfpathlineto{\pgfqpoint{4.239867in}{2.576638in}}%
\pgfpathlineto{\pgfqpoint{4.240732in}{2.554134in}}%
\pgfpathlineto{\pgfqpoint{4.241594in}{2.606151in}}%
\pgfpathlineto{\pgfqpoint{4.242461in}{2.525360in}}%
\pgfpathlineto{\pgfqpoint{4.244190in}{2.582172in}}%
\pgfpathlineto{\pgfqpoint{4.245054in}{2.499875in}}%
\pgfpathlineto{\pgfqpoint{4.246784in}{2.591272in}}%
\pgfpathlineto{\pgfqpoint{4.247649in}{2.531017in}}%
\pgfpathlineto{\pgfqpoint{4.248514in}{2.599080in}}%
\pgfpathlineto{\pgfqpoint{4.249380in}{2.541162in}}%
\pgfpathlineto{\pgfqpoint{4.250245in}{2.559915in}}%
\pgfpathlineto{\pgfqpoint{4.251107in}{2.509191in}}%
\pgfpathlineto{\pgfqpoint{4.251972in}{2.513278in}}%
\pgfpathlineto{\pgfqpoint{4.253702in}{2.543191in}}%
\pgfpathlineto{\pgfqpoint{4.255431in}{2.494802in}}%
\pgfpathlineto{\pgfqpoint{4.256297in}{2.507592in}}%
\pgfpathlineto{\pgfqpoint{4.257163in}{2.440143in}}%
\pgfpathlineto{\pgfqpoint{4.258891in}{2.533047in}}%
\pgfpathlineto{\pgfqpoint{4.259758in}{2.458957in}}%
\pgfpathlineto{\pgfqpoint{4.260621in}{2.469841in}}%
\pgfpathlineto{\pgfqpoint{4.263215in}{2.547987in}}%
\pgfpathlineto{\pgfqpoint{4.264079in}{2.503841in}}%
\pgfpathlineto{\pgfqpoint{4.265810in}{2.579837in}}%
\pgfpathlineto{\pgfqpoint{4.266675in}{2.548602in}}%
\pgfpathlineto{\pgfqpoint{4.267541in}{2.572458in}}%
\pgfpathlineto{\pgfqpoint{4.268406in}{2.517922in}}%
\pgfpathlineto{\pgfqpoint{4.270136in}{2.551491in}}%
\pgfpathlineto{\pgfqpoint{4.271000in}{2.495141in}}%
\pgfpathlineto{\pgfqpoint{4.271865in}{2.539194in}}%
\pgfpathlineto{\pgfqpoint{4.272729in}{2.536797in}}%
\pgfpathlineto{\pgfqpoint{4.273593in}{2.509006in}}%
\pgfpathlineto{\pgfqpoint{4.274458in}{2.533416in}}%
\pgfpathlineto{\pgfqpoint{4.276188in}{2.509468in}}%
\pgfpathlineto{\pgfqpoint{4.277054in}{2.511343in}}%
\pgfpathlineto{\pgfqpoint{4.277919in}{2.507715in}}%
\pgfpathlineto{\pgfqpoint{4.279650in}{2.472301in}}%
\pgfpathlineto{\pgfqpoint{4.280515in}{2.504057in}}%
\pgfpathlineto{\pgfqpoint{4.281379in}{2.500583in}}%
\pgfpathlineto{\pgfqpoint{4.282243in}{2.461602in}}%
\pgfpathlineto{\pgfqpoint{4.283109in}{2.548695in}}%
\pgfpathlineto{\pgfqpoint{4.283974in}{2.543007in}}%
\pgfpathlineto{\pgfqpoint{4.284839in}{2.529850in}}%
\pgfpathlineto{\pgfqpoint{4.285704in}{2.491698in}}%
\pgfpathlineto{\pgfqpoint{4.286569in}{2.510912in}}%
\pgfpathlineto{\pgfqpoint{4.288300in}{2.450595in}}%
\pgfpathlineto{\pgfqpoint{4.289164in}{2.524315in}}%
\pgfpathlineto{\pgfqpoint{4.290894in}{2.462646in}}%
\pgfpathlineto{\pgfqpoint{4.292625in}{2.498399in}}%
\pgfpathlineto{\pgfqpoint{4.293491in}{2.488409in}}%
\pgfpathlineto{\pgfqpoint{4.294356in}{2.531140in}}%
\pgfpathlineto{\pgfqpoint{4.295221in}{2.460771in}}%
\pgfpathlineto{\pgfqpoint{4.296948in}{2.528005in}}%
\pgfpathlineto{\pgfqpoint{4.297812in}{2.502243in}}%
\pgfpathlineto{\pgfqpoint{4.298678in}{2.536859in}}%
\pgfpathlineto{\pgfqpoint{4.300410in}{2.463999in}}%
\pgfpathlineto{\pgfqpoint{4.302140in}{2.516783in}}%
\pgfpathlineto{\pgfqpoint{4.303005in}{2.505809in}}%
\pgfpathlineto{\pgfqpoint{4.304734in}{2.484596in}}%
\pgfpathlineto{\pgfqpoint{4.306465in}{2.575409in}}%
\pgfpathlineto{\pgfqpoint{4.307331in}{2.518104in}}%
\pgfpathlineto{\pgfqpoint{4.308196in}{2.607688in}}%
\pgfpathlineto{\pgfqpoint{4.309061in}{2.509650in}}%
\pgfpathlineto{\pgfqpoint{4.309926in}{2.533722in}}%
\pgfpathlineto{\pgfqpoint{4.311656in}{2.480047in}}%
\pgfpathlineto{\pgfqpoint{4.312522in}{2.493512in}}%
\pgfpathlineto{\pgfqpoint{4.313387in}{2.540423in}}%
\pgfpathlineto{\pgfqpoint{4.314253in}{2.533937in}}%
\pgfpathlineto{\pgfqpoint{4.315118in}{2.570059in}}%
\pgfpathlineto{\pgfqpoint{4.315983in}{2.532677in}}%
\pgfpathlineto{\pgfqpoint{4.317712in}{2.580758in}}%
\pgfpathlineto{\pgfqpoint{4.318576in}{2.495079in}}%
\pgfpathlineto{\pgfqpoint{4.320307in}{2.565326in}}%
\pgfpathlineto{\pgfqpoint{4.321170in}{2.491606in}}%
\pgfpathlineto{\pgfqpoint{4.322899in}{2.552782in}}%
\pgfpathlineto{\pgfqpoint{4.323763in}{2.556595in}}%
\pgfpathlineto{\pgfqpoint{4.324629in}{2.528865in}}%
\pgfpathlineto{\pgfqpoint{4.325492in}{2.549524in}}%
\pgfpathlineto{\pgfqpoint{4.326357in}{2.534399in}}%
\pgfpathlineto{\pgfqpoint{4.328083in}{2.558532in}}%
\pgfpathlineto{\pgfqpoint{4.329813in}{2.481584in}}%
\pgfpathlineto{\pgfqpoint{4.331544in}{2.572273in}}%
\pgfpathlineto{\pgfqpoint{4.332409in}{2.567478in}}%
\pgfpathlineto{\pgfqpoint{4.333276in}{2.580943in}}%
\pgfpathlineto{\pgfqpoint{4.334141in}{2.627364in}}%
\pgfpathlineto{\pgfqpoint{4.335868in}{2.486933in}}%
\pgfpathlineto{\pgfqpoint{4.336733in}{2.585430in}}%
\pgfpathlineto{\pgfqpoint{4.337598in}{2.474205in}}%
\pgfpathlineto{\pgfqpoint{4.338461in}{2.526037in}}%
\pgfpathlineto{\pgfqpoint{4.339326in}{2.505501in}}%
\pgfpathlineto{\pgfqpoint{4.340193in}{2.577407in}}%
\pgfpathlineto{\pgfqpoint{4.341058in}{2.537965in}}%
\pgfpathlineto{\pgfqpoint{4.341924in}{2.555796in}}%
\pgfpathlineto{\pgfqpoint{4.342786in}{2.498430in}}%
\pgfpathlineto{\pgfqpoint{4.344511in}{2.580450in}}%
\pgfpathlineto{\pgfqpoint{4.345375in}{2.520257in}}%
\pgfpathlineto{\pgfqpoint{4.346241in}{2.598467in}}%
\pgfpathlineto{\pgfqpoint{4.347106in}{2.527574in}}%
\pgfpathlineto{\pgfqpoint{4.347971in}{2.529850in}}%
\pgfpathlineto{\pgfqpoint{4.348837in}{2.516998in}}%
\pgfpathlineto{\pgfqpoint{4.349701in}{2.573349in}}%
\pgfpathlineto{\pgfqpoint{4.351432in}{2.517429in}}%
\pgfpathlineto{\pgfqpoint{4.352298in}{2.537934in}}%
\pgfpathlineto{\pgfqpoint{4.353162in}{2.613961in}}%
\pgfpathlineto{\pgfqpoint{4.354892in}{2.536120in}}%
\pgfpathlineto{\pgfqpoint{4.355757in}{2.576823in}}%
\pgfpathlineto{\pgfqpoint{4.356622in}{2.564433in}}%
\pgfpathlineto{\pgfqpoint{4.357487in}{2.522163in}}%
\pgfpathlineto{\pgfqpoint{4.358349in}{2.603875in}}%
\pgfpathlineto{\pgfqpoint{4.359212in}{2.586967in}}%
\pgfpathlineto{\pgfqpoint{4.360077in}{2.615742in}}%
\pgfpathlineto{\pgfqpoint{4.360942in}{2.614143in}}%
\pgfpathlineto{\pgfqpoint{4.362672in}{2.525483in}}%
\pgfpathlineto{\pgfqpoint{4.363537in}{2.576515in}}%
\pgfpathlineto{\pgfqpoint{4.364401in}{2.512878in}}%
\pgfpathlineto{\pgfqpoint{4.365264in}{2.518227in}}%
\pgfpathlineto{\pgfqpoint{4.366126in}{2.580204in}}%
\pgfpathlineto{\pgfqpoint{4.366989in}{2.513494in}}%
\pgfpathlineto{\pgfqpoint{4.367853in}{2.525237in}}%
\pgfpathlineto{\pgfqpoint{4.368718in}{2.497969in}}%
\pgfpathlineto{\pgfqpoint{4.369583in}{2.614697in}}%
\pgfpathlineto{\pgfqpoint{4.371313in}{2.501904in}}%
\pgfpathlineto{\pgfqpoint{4.372179in}{2.545159in}}%
\pgfpathlineto{\pgfqpoint{4.373044in}{2.537719in}}%
\pgfpathlineto{\pgfqpoint{4.373909in}{2.535813in}}%
\pgfpathlineto{\pgfqpoint{4.374774in}{2.492896in}}%
\pgfpathlineto{\pgfqpoint{4.377370in}{2.538948in}}%
\pgfpathlineto{\pgfqpoint{4.378235in}{2.506853in}}%
\pgfpathlineto{\pgfqpoint{4.379100in}{2.518658in}}%
\pgfpathlineto{\pgfqpoint{4.379961in}{2.557331in}}%
\pgfpathlineto{\pgfqpoint{4.380827in}{2.498613in}}%
\pgfpathlineto{\pgfqpoint{4.382557in}{2.585492in}}%
\pgfpathlineto{\pgfqpoint{4.383424in}{2.542391in}}%
\pgfpathlineto{\pgfqpoint{4.384290in}{2.573195in}}%
\pgfpathlineto{\pgfqpoint{4.386020in}{2.542268in}}%
\pgfpathlineto{\pgfqpoint{4.386886in}{2.564094in}}%
\pgfpathlineto{\pgfqpoint{4.387751in}{2.557701in}}%
\pgfpathlineto{\pgfqpoint{4.388616in}{2.476111in}}%
\pgfpathlineto{\pgfqpoint{4.389481in}{2.547125in}}%
\pgfpathlineto{\pgfqpoint{4.390346in}{2.489268in}}%
\pgfpathlineto{\pgfqpoint{4.391212in}{2.516444in}}%
\pgfpathlineto{\pgfqpoint{4.392077in}{2.516383in}}%
\pgfpathlineto{\pgfqpoint{4.392939in}{2.476848in}}%
\pgfpathlineto{\pgfqpoint{4.393805in}{2.498430in}}%
\pgfpathlineto{\pgfqpoint{4.394670in}{2.496401in}}%
\pgfpathlineto{\pgfqpoint{4.395533in}{2.462092in}}%
\pgfpathlineto{\pgfqpoint{4.397262in}{2.510143in}}%
\pgfpathlineto{\pgfqpoint{4.398127in}{2.509496in}}%
\pgfpathlineto{\pgfqpoint{4.398990in}{2.593115in}}%
\pgfpathlineto{\pgfqpoint{4.399855in}{2.464581in}}%
\pgfpathlineto{\pgfqpoint{4.401586in}{2.583463in}}%
\pgfpathlineto{\pgfqpoint{4.404184in}{2.477710in}}%
\pgfpathlineto{\pgfqpoint{4.405047in}{2.508206in}}%
\pgfpathlineto{\pgfqpoint{4.405913in}{2.484596in}}%
\pgfpathlineto{\pgfqpoint{4.406779in}{2.575378in}}%
\pgfpathlineto{\pgfqpoint{4.408510in}{2.484165in}}%
\pgfpathlineto{\pgfqpoint{4.410240in}{2.570244in}}%
\pgfpathlineto{\pgfqpoint{4.411104in}{2.566678in}}%
\pgfpathlineto{\pgfqpoint{4.412836in}{2.527451in}}%
\pgfpathlineto{\pgfqpoint{4.413701in}{2.482444in}}%
\pgfpathlineto{\pgfqpoint{4.414565in}{2.600925in}}%
\pgfpathlineto{\pgfqpoint{4.415430in}{2.579221in}}%
\pgfpathlineto{\pgfqpoint{4.417161in}{2.488409in}}%
\pgfpathlineto{\pgfqpoint{4.418026in}{2.557639in}}%
\pgfpathlineto{\pgfqpoint{4.418891in}{2.472053in}}%
\pgfpathlineto{\pgfqpoint{4.419756in}{2.574916in}}%
\pgfpathlineto{\pgfqpoint{4.421485in}{2.482998in}}%
\pgfpathlineto{\pgfqpoint{4.423211in}{2.532523in}}%
\pgfpathlineto{\pgfqpoint{4.424076in}{2.577315in}}%
\pgfpathlineto{\pgfqpoint{4.424939in}{2.548418in}}%
\pgfpathlineto{\pgfqpoint{4.425802in}{2.469656in}}%
\pgfpathlineto{\pgfqpoint{4.426667in}{2.496647in}}%
\pgfpathlineto{\pgfqpoint{4.427532in}{2.581987in}}%
\pgfpathlineto{\pgfqpoint{4.428398in}{2.429259in}}%
\pgfpathlineto{\pgfqpoint{4.429263in}{2.585430in}}%
\pgfpathlineto{\pgfqpoint{4.430129in}{2.557947in}}%
\pgfpathlineto{\pgfqpoint{4.430994in}{2.548110in}}%
\pgfpathlineto{\pgfqpoint{4.431859in}{2.624473in}}%
\pgfpathlineto{\pgfqpoint{4.432725in}{2.530586in}}%
\pgfpathlineto{\pgfqpoint{4.434454in}{2.624288in}}%
\pgfpathlineto{\pgfqpoint{4.435319in}{2.482967in}}%
\pgfpathlineto{\pgfqpoint{4.436184in}{2.615067in}}%
\pgfpathlineto{\pgfqpoint{4.437050in}{2.587337in}}%
\pgfpathlineto{\pgfqpoint{4.438780in}{2.554196in}}%
\pgfpathlineto{\pgfqpoint{4.439645in}{2.556471in}}%
\pgfpathlineto{\pgfqpoint{4.440508in}{2.506607in}}%
\pgfpathlineto{\pgfqpoint{4.441371in}{2.523331in}}%
\pgfpathlineto{\pgfqpoint{4.442235in}{2.490713in}}%
\pgfpathlineto{\pgfqpoint{4.443101in}{2.533475in}}%
\pgfpathlineto{\pgfqpoint{4.443964in}{2.490805in}}%
\pgfpathlineto{\pgfqpoint{4.445695in}{2.544482in}}%
\pgfpathlineto{\pgfqpoint{4.446561in}{2.501381in}}%
\pgfpathlineto{\pgfqpoint{4.448292in}{2.539687in}}%
\pgfpathlineto{\pgfqpoint{4.449156in}{2.534399in}}%
\pgfpathlineto{\pgfqpoint{4.450020in}{2.463629in}}%
\pgfpathlineto{\pgfqpoint{4.451751in}{2.532400in}}%
\pgfpathlineto{\pgfqpoint{4.452617in}{2.453669in}}%
\pgfpathlineto{\pgfqpoint{4.453484in}{2.537657in}}%
\pgfpathlineto{\pgfqpoint{4.455213in}{2.465597in}}%
\pgfpathlineto{\pgfqpoint{4.456077in}{2.547864in}}%
\pgfpathlineto{\pgfqpoint{4.456943in}{2.431196in}}%
\pgfpathlineto{\pgfqpoint{4.457807in}{2.544297in}}%
\pgfpathlineto{\pgfqpoint{4.458672in}{2.515584in}}%
\pgfpathlineto{\pgfqpoint{4.459537in}{2.486256in}}%
\pgfpathlineto{\pgfqpoint{4.461266in}{2.542914in}}%
\pgfpathlineto{\pgfqpoint{4.462131in}{2.531325in}}%
\pgfpathlineto{\pgfqpoint{4.462995in}{2.476419in}}%
\pgfpathlineto{\pgfqpoint{4.463858in}{2.542668in}}%
\pgfpathlineto{\pgfqpoint{4.464723in}{2.508452in}}%
\pgfpathlineto{\pgfqpoint{4.465588in}{2.539194in}}%
\pgfpathlineto{\pgfqpoint{4.466453in}{2.501843in}}%
\pgfpathlineto{\pgfqpoint{4.467318in}{2.594654in}}%
\pgfpathlineto{\pgfqpoint{4.468183in}{2.490007in}}%
\pgfpathlineto{\pgfqpoint{4.469047in}{2.503472in}}%
\pgfpathlineto{\pgfqpoint{4.469913in}{2.532431in}}%
\pgfpathlineto{\pgfqpoint{4.470777in}{2.497784in}}%
\pgfpathlineto{\pgfqpoint{4.471642in}{2.533722in}}%
\pgfpathlineto{\pgfqpoint{4.472507in}{2.501381in}}%
\pgfpathlineto{\pgfqpoint{4.473373in}{2.535505in}}%
\pgfpathlineto{\pgfqpoint{4.475100in}{2.510604in}}%
\pgfpathlineto{\pgfqpoint{4.476831in}{2.541223in}}%
\pgfpathlineto{\pgfqpoint{4.477695in}{2.538027in}}%
\pgfpathlineto{\pgfqpoint{4.478560in}{2.541593in}}%
\pgfpathlineto{\pgfqpoint{4.481157in}{2.490007in}}%
\pgfpathlineto{\pgfqpoint{4.482023in}{2.510297in}}%
\pgfpathlineto{\pgfqpoint{4.482888in}{2.578483in}}%
\pgfpathlineto{\pgfqpoint{4.483754in}{2.508760in}}%
\pgfpathlineto{\pgfqpoint{4.484620in}{2.560284in}}%
\pgfpathlineto{\pgfqpoint{4.485486in}{2.506884in}}%
\pgfpathlineto{\pgfqpoint{4.486350in}{2.606397in}}%
\pgfpathlineto{\pgfqpoint{4.487216in}{2.604306in}}%
\pgfpathlineto{\pgfqpoint{4.490672in}{2.490192in}}%
\pgfpathlineto{\pgfqpoint{4.491537in}{2.498307in}}%
\pgfpathlineto{\pgfqpoint{4.492402in}{2.550016in}}%
\pgfpathlineto{\pgfqpoint{4.494133in}{2.487424in}}%
\pgfpathlineto{\pgfqpoint{4.494998in}{2.552351in}}%
\pgfpathlineto{\pgfqpoint{4.495863in}{2.518289in}}%
\pgfpathlineto{\pgfqpoint{4.497591in}{2.545588in}}%
\pgfpathlineto{\pgfqpoint{4.498455in}{2.475557in}}%
\pgfpathlineto{\pgfqpoint{4.500183in}{2.540608in}}%
\pgfpathlineto{\pgfqpoint{4.501049in}{2.547494in}}%
\pgfpathlineto{\pgfqpoint{4.501914in}{2.508390in}}%
\pgfpathlineto{\pgfqpoint{4.503645in}{2.575224in}}%
\pgfpathlineto{\pgfqpoint{4.504511in}{2.526866in}}%
\pgfpathlineto{\pgfqpoint{4.505375in}{2.580143in}}%
\pgfpathlineto{\pgfqpoint{4.507970in}{2.522900in}}%
\pgfpathlineto{\pgfqpoint{4.508835in}{2.564187in}}%
\pgfpathlineto{\pgfqpoint{4.509700in}{2.522348in}}%
\pgfpathlineto{\pgfqpoint{4.510564in}{2.549277in}}%
\pgfpathlineto{\pgfqpoint{4.512294in}{2.509927in}}%
\pgfpathlineto{\pgfqpoint{4.513159in}{2.582049in}}%
\pgfpathlineto{\pgfqpoint{4.514025in}{2.496647in}}%
\pgfpathlineto{\pgfqpoint{4.514890in}{2.502058in}}%
\pgfpathlineto{\pgfqpoint{4.515753in}{2.521917in}}%
\pgfpathlineto{\pgfqpoint{4.517482in}{2.456743in}}%
\pgfpathlineto{\pgfqpoint{4.518346in}{2.509866in}}%
\pgfpathlineto{\pgfqpoint{4.519212in}{2.495172in}}%
\pgfpathlineto{\pgfqpoint{4.521806in}{2.568461in}}%
\pgfpathlineto{\pgfqpoint{4.522670in}{2.521917in}}%
\pgfpathlineto{\pgfqpoint{4.523535in}{2.532646in}}%
\pgfpathlineto{\pgfqpoint{4.524400in}{2.560284in}}%
\pgfpathlineto{\pgfqpoint{4.525263in}{2.549093in}}%
\pgfpathlineto{\pgfqpoint{4.526993in}{2.515708in}}%
\pgfpathlineto{\pgfqpoint{4.527857in}{2.546327in}}%
\pgfpathlineto{\pgfqpoint{4.528722in}{2.542576in}}%
\pgfpathlineto{\pgfqpoint{4.529586in}{2.481338in}}%
\pgfpathlineto{\pgfqpoint{4.530452in}{2.551399in}}%
\pgfpathlineto{\pgfqpoint{4.532179in}{2.480907in}}%
\pgfpathlineto{\pgfqpoint{4.533043in}{2.541991in}}%
\pgfpathlineto{\pgfqpoint{4.533909in}{2.468732in}}%
\pgfpathlineto{\pgfqpoint{4.534774in}{2.577254in}}%
\pgfpathlineto{\pgfqpoint{4.536505in}{2.502918in}}%
\pgfpathlineto{\pgfqpoint{4.537370in}{2.533845in}}%
\pgfpathlineto{\pgfqpoint{4.539100in}{2.485887in}}%
\pgfpathlineto{\pgfqpoint{4.539965in}{2.504916in}}%
\pgfpathlineto{\pgfqpoint{4.540828in}{2.586906in}}%
\pgfpathlineto{\pgfqpoint{4.541691in}{2.530586in}}%
\pgfpathlineto{\pgfqpoint{4.542555in}{2.574947in}}%
\pgfpathlineto{\pgfqpoint{4.544282in}{2.540362in}}%
\pgfpathlineto{\pgfqpoint{4.545145in}{2.499228in}}%
\pgfpathlineto{\pgfqpoint{4.546010in}{2.573993in}}%
\pgfpathlineto{\pgfqpoint{4.546874in}{2.570365in}}%
\pgfpathlineto{\pgfqpoint{4.547738in}{2.577682in}}%
\pgfpathlineto{\pgfqpoint{4.548603in}{2.564587in}}%
\pgfpathlineto{\pgfqpoint{4.549468in}{2.524991in}}%
\pgfpathlineto{\pgfqpoint{4.550331in}{2.553090in}}%
\pgfpathlineto{\pgfqpoint{4.552060in}{2.516352in}}%
\pgfpathlineto{\pgfqpoint{4.552925in}{2.589918in}}%
\pgfpathlineto{\pgfqpoint{4.553786in}{2.574485in}}%
\pgfpathlineto{\pgfqpoint{4.554650in}{2.516937in}}%
\pgfpathlineto{\pgfqpoint{4.555516in}{2.578298in}}%
\pgfpathlineto{\pgfqpoint{4.556379in}{2.576761in}}%
\pgfpathlineto{\pgfqpoint{4.557244in}{2.532308in}}%
\pgfpathlineto{\pgfqpoint{4.558109in}{2.547494in}}%
\pgfpathlineto{\pgfqpoint{4.558975in}{2.474544in}}%
\pgfpathlineto{\pgfqpoint{4.560705in}{2.547740in}}%
\pgfpathlineto{\pgfqpoint{4.562436in}{2.510174in}}%
\pgfpathlineto{\pgfqpoint{4.564162in}{2.544974in}}%
\pgfpathlineto{\pgfqpoint{4.565028in}{2.501997in}}%
\pgfpathlineto{\pgfqpoint{4.565894in}{2.508821in}}%
\pgfpathlineto{\pgfqpoint{4.566758in}{2.497140in}}%
\pgfpathlineto{\pgfqpoint{4.567622in}{2.561944in}}%
\pgfpathlineto{\pgfqpoint{4.568487in}{2.481153in}}%
\pgfpathlineto{\pgfqpoint{4.569351in}{2.575655in}}%
\pgfpathlineto{\pgfqpoint{4.570217in}{2.442665in}}%
\pgfpathlineto{\pgfqpoint{4.571948in}{2.555119in}}%
\pgfpathlineto{\pgfqpoint{4.573680in}{2.583955in}}%
\pgfpathlineto{\pgfqpoint{4.576275in}{2.467319in}}%
\pgfpathlineto{\pgfqpoint{4.578005in}{2.567907in}}%
\pgfpathlineto{\pgfqpoint{4.578869in}{2.564833in}}%
\pgfpathlineto{\pgfqpoint{4.579732in}{2.462216in}}%
\pgfpathlineto{\pgfqpoint{4.580597in}{2.611991in}}%
\pgfpathlineto{\pgfqpoint{4.582325in}{2.544913in}}%
\pgfpathlineto{\pgfqpoint{4.583189in}{2.761955in}}%
\pgfpathlineto{\pgfqpoint{4.584053in}{2.749965in}}%
\pgfpathlineto{\pgfqpoint{4.586645in}{2.866971in}}%
\pgfpathlineto{\pgfqpoint{4.587511in}{2.772346in}}%
\pgfpathlineto{\pgfqpoint{4.589242in}{2.853260in}}%
\pgfpathlineto{\pgfqpoint{4.590109in}{2.801336in}}%
\pgfpathlineto{\pgfqpoint{4.590975in}{2.809544in}}%
\pgfpathlineto{\pgfqpoint{4.592706in}{2.788823in}}%
\pgfpathlineto{\pgfqpoint{4.593571in}{2.823378in}}%
\pgfpathlineto{\pgfqpoint{4.594436in}{2.794172in}}%
\pgfpathlineto{\pgfqpoint{4.595300in}{2.816738in}}%
\pgfpathlineto{\pgfqpoint{4.597029in}{2.759802in}}%
\pgfpathlineto{\pgfqpoint{4.599625in}{2.814678in}}%
\pgfpathlineto{\pgfqpoint{4.600489in}{2.821779in}}%
\pgfpathlineto{\pgfqpoint{4.601354in}{2.771117in}}%
\pgfpathlineto{\pgfqpoint{4.602218in}{2.843115in}}%
\pgfpathlineto{\pgfqpoint{4.603082in}{2.795404in}}%
\pgfpathlineto{\pgfqpoint{4.603947in}{2.852983in}}%
\pgfpathlineto{\pgfqpoint{4.604812in}{2.751197in}}%
\pgfpathlineto{\pgfqpoint{4.605675in}{2.788394in}}%
\pgfpathlineto{\pgfqpoint{4.606541in}{2.785043in}}%
\pgfpathlineto{\pgfqpoint{4.607407in}{2.741973in}}%
\pgfpathlineto{\pgfqpoint{4.609137in}{2.801828in}}%
\pgfpathlineto{\pgfqpoint{4.610868in}{2.716088in}}%
\pgfpathlineto{\pgfqpoint{4.613459in}{2.818305in}}%
\pgfpathlineto{\pgfqpoint{4.615186in}{2.779294in}}%
\pgfpathlineto{\pgfqpoint{4.616052in}{2.796109in}}%
\pgfpathlineto{\pgfqpoint{4.616918in}{2.777326in}}%
\pgfpathlineto{\pgfqpoint{4.617784in}{2.789254in}}%
\pgfpathlineto{\pgfqpoint{4.618650in}{2.738068in}}%
\pgfpathlineto{\pgfqpoint{4.620380in}{2.774437in}}%
\pgfpathlineto{\pgfqpoint{4.621246in}{2.781323in}}%
\pgfpathlineto{\pgfqpoint{4.622110in}{2.820735in}}%
\pgfpathlineto{\pgfqpoint{4.622974in}{2.727923in}}%
\pgfpathlineto{\pgfqpoint{4.624705in}{2.782614in}}%
\pgfpathlineto{\pgfqpoint{4.626436in}{2.689833in}}%
\pgfpathlineto{\pgfqpoint{4.628165in}{2.856057in}}%
\pgfpathlineto{\pgfqpoint{4.629028in}{2.796263in}}%
\pgfpathlineto{\pgfqpoint{4.630756in}{2.815755in}}%
\pgfpathlineto{\pgfqpoint{4.631622in}{2.871151in}}%
\pgfpathlineto{\pgfqpoint{4.632489in}{2.810067in}}%
\pgfpathlineto{\pgfqpoint{4.633355in}{2.862114in}}%
\pgfpathlineto{\pgfqpoint{4.635081in}{2.778372in}}%
\pgfpathlineto{\pgfqpoint{4.636810in}{2.839734in}}%
\pgfpathlineto{\pgfqpoint{4.637675in}{2.778065in}}%
\pgfpathlineto{\pgfqpoint{4.638541in}{2.909949in}}%
\pgfpathlineto{\pgfqpoint{4.640272in}{2.786734in}}%
\pgfpathlineto{\pgfqpoint{4.641137in}{2.794911in}}%
\pgfpathlineto{\pgfqpoint{4.642003in}{2.791591in}}%
\pgfpathlineto{\pgfqpoint{4.642868in}{2.836167in}}%
\pgfpathlineto{\pgfqpoint{4.643733in}{2.825530in}}%
\pgfpathlineto{\pgfqpoint{4.644598in}{2.827252in}}%
\pgfpathlineto{\pgfqpoint{4.645460in}{2.843362in}}%
\pgfpathlineto{\pgfqpoint{4.646326in}{2.769580in}}%
\pgfpathlineto{\pgfqpoint{4.647191in}{2.801736in}}%
\pgfpathlineto{\pgfqpoint{4.648057in}{2.797495in}}%
\pgfpathlineto{\pgfqpoint{4.648923in}{2.814464in}}%
\pgfpathlineto{\pgfqpoint{4.649788in}{2.778619in}}%
\pgfpathlineto{\pgfqpoint{4.650652in}{2.813297in}}%
\pgfpathlineto{\pgfqpoint{4.652381in}{2.561084in}}%
\pgfpathlineto{\pgfqpoint{4.653247in}{2.511651in}}%
\pgfpathlineto{\pgfqpoint{4.654976in}{2.547373in}}%
\pgfpathlineto{\pgfqpoint{4.655843in}{2.539933in}}%
\pgfpathlineto{\pgfqpoint{4.656707in}{2.544544in}}%
\pgfpathlineto{\pgfqpoint{4.657572in}{2.563912in}}%
\pgfpathlineto{\pgfqpoint{4.659302in}{2.520626in}}%
\pgfpathlineto{\pgfqpoint{4.660168in}{2.622507in}}%
\pgfpathlineto{\pgfqpoint{4.661897in}{2.459080in}}%
\pgfpathlineto{\pgfqpoint{4.663625in}{2.520872in}}%
\pgfpathlineto{\pgfqpoint{4.664490in}{2.531448in}}%
\pgfpathlineto{\pgfqpoint{4.665354in}{2.530034in}}%
\pgfpathlineto{\pgfqpoint{4.666218in}{2.499015in}}%
\pgfpathlineto{\pgfqpoint{4.667082in}{2.544974in}}%
\pgfpathlineto{\pgfqpoint{4.667948in}{2.543438in}}%
\pgfpathlineto{\pgfqpoint{4.668812in}{2.521119in}}%
\pgfpathlineto{\pgfqpoint{4.669677in}{2.596129in}}%
\pgfpathlineto{\pgfqpoint{4.671407in}{2.487978in}}%
\pgfpathlineto{\pgfqpoint{4.672272in}{2.521611in}}%
\pgfpathlineto{\pgfqpoint{4.673137in}{2.473468in}}%
\pgfpathlineto{\pgfqpoint{4.674000in}{2.490315in}}%
\pgfpathlineto{\pgfqpoint{4.674866in}{2.602341in}}%
\pgfpathlineto{\pgfqpoint{4.675730in}{2.495757in}}%
\pgfpathlineto{\pgfqpoint{4.677460in}{2.579283in}}%
\pgfpathlineto{\pgfqpoint{4.678325in}{2.524131in}}%
\pgfpathlineto{\pgfqpoint{4.679190in}{2.555981in}}%
\pgfpathlineto{\pgfqpoint{4.680921in}{2.466521in}}%
\pgfpathlineto{\pgfqpoint{4.681784in}{2.513986in}}%
\pgfpathlineto{\pgfqpoint{4.682650in}{2.471439in}}%
\pgfpathlineto{\pgfqpoint{4.683516in}{2.563912in}}%
\pgfpathlineto{\pgfqpoint{4.684380in}{2.506363in}}%
\pgfpathlineto{\pgfqpoint{4.686110in}{2.555673in}}%
\pgfpathlineto{\pgfqpoint{4.687840in}{2.520503in}}%
\pgfpathlineto{\pgfqpoint{4.688705in}{2.553705in}}%
\pgfpathlineto{\pgfqpoint{4.689568in}{2.527389in}}%
\pgfpathlineto{\pgfqpoint{4.691297in}{2.627426in}}%
\pgfpathlineto{\pgfqpoint{4.693027in}{2.550139in}}%
\pgfpathlineto{\pgfqpoint{4.693889in}{2.618387in}}%
\pgfpathlineto{\pgfqpoint{4.695621in}{2.528742in}}%
\pgfpathlineto{\pgfqpoint{4.696486in}{2.549647in}}%
\pgfpathlineto{\pgfqpoint{4.697352in}{2.496524in}}%
\pgfpathlineto{\pgfqpoint{4.698218in}{2.562129in}}%
\pgfpathlineto{\pgfqpoint{4.699949in}{2.539994in}}%
\pgfpathlineto{\pgfqpoint{4.700812in}{2.542884in}}%
\pgfpathlineto{\pgfqpoint{4.701675in}{2.493481in}}%
\pgfpathlineto{\pgfqpoint{4.702540in}{2.530586in}}%
\pgfpathlineto{\pgfqpoint{4.704271in}{2.477525in}}%
\pgfpathlineto{\pgfqpoint{4.705999in}{2.528557in}}%
\pgfpathlineto{\pgfqpoint{4.706864in}{2.548846in}}%
\pgfpathlineto{\pgfqpoint{4.708594in}{2.481982in}}%
\pgfpathlineto{\pgfqpoint{4.709461in}{2.490744in}}%
\pgfpathlineto{\pgfqpoint{4.710325in}{2.555548in}}%
\pgfpathlineto{\pgfqpoint{4.711190in}{2.493941in}}%
\pgfpathlineto{\pgfqpoint{4.712056in}{2.525421in}}%
\pgfpathlineto{\pgfqpoint{4.714648in}{2.473035in}}%
\pgfpathlineto{\pgfqpoint{4.717245in}{2.545588in}}%
\pgfpathlineto{\pgfqpoint{4.718111in}{2.488961in}}%
\pgfpathlineto{\pgfqpoint{4.718976in}{2.524621in}}%
\pgfpathlineto{\pgfqpoint{4.719839in}{2.494556in}}%
\pgfpathlineto{\pgfqpoint{4.721569in}{2.539992in}}%
\pgfpathlineto{\pgfqpoint{4.722434in}{2.487485in}}%
\pgfpathlineto{\pgfqpoint{4.723300in}{2.574516in}}%
\pgfpathlineto{\pgfqpoint{4.724166in}{2.562681in}}%
\pgfpathlineto{\pgfqpoint{4.725894in}{2.515154in}}%
\pgfpathlineto{\pgfqpoint{4.726759in}{2.521240in}}%
\pgfpathlineto{\pgfqpoint{4.728487in}{2.458588in}}%
\pgfpathlineto{\pgfqpoint{4.729353in}{2.520134in}}%
\pgfpathlineto{\pgfqpoint{4.730215in}{2.486133in}}%
\pgfpathlineto{\pgfqpoint{4.731947in}{2.527574in}}%
\pgfpathlineto{\pgfqpoint{4.733678in}{2.563235in}}%
\pgfpathlineto{\pgfqpoint{4.734543in}{2.527389in}}%
\pgfpathlineto{\pgfqpoint{4.735409in}{2.572643in}}%
\pgfpathlineto{\pgfqpoint{4.737997in}{2.522594in}}%
\pgfpathlineto{\pgfqpoint{4.739729in}{2.498399in}}%
\pgfpathlineto{\pgfqpoint{4.740595in}{2.584078in}}%
\pgfpathlineto{\pgfqpoint{4.742327in}{2.517060in}}%
\pgfpathlineto{\pgfqpoint{4.743193in}{2.549154in}}%
\pgfpathlineto{\pgfqpoint{4.744060in}{2.508821in}}%
\pgfpathlineto{\pgfqpoint{4.745789in}{2.566249in}}%
\pgfpathlineto{\pgfqpoint{4.746655in}{2.451303in}}%
\pgfpathlineto{\pgfqpoint{4.748384in}{2.493327in}}%
\pgfpathlineto{\pgfqpoint{4.749246in}{2.482076in}}%
\pgfpathlineto{\pgfqpoint{4.750112in}{2.490130in}}%
\pgfpathlineto{\pgfqpoint{4.750977in}{2.561759in}}%
\pgfpathlineto{\pgfqpoint{4.751842in}{2.556010in}}%
\pgfpathlineto{\pgfqpoint{4.752708in}{2.504947in}}%
\pgfpathlineto{\pgfqpoint{4.753575in}{2.570059in}}%
\pgfpathlineto{\pgfqpoint{4.754440in}{2.558378in}}%
\pgfpathlineto{\pgfqpoint{4.756170in}{2.516814in}}%
\pgfpathlineto{\pgfqpoint{4.757901in}{2.575163in}}%
\pgfpathlineto{\pgfqpoint{4.758765in}{2.530740in}}%
\pgfpathlineto{\pgfqpoint{4.759630in}{2.567968in}}%
\pgfpathlineto{\pgfqpoint{4.760492in}{2.544051in}}%
\pgfpathlineto{\pgfqpoint{4.761358in}{2.575686in}}%
\pgfpathlineto{\pgfqpoint{4.762223in}{2.527328in}}%
\pgfpathlineto{\pgfqpoint{4.763088in}{2.556533in}}%
\pgfpathlineto{\pgfqpoint{4.763954in}{2.510420in}}%
\pgfpathlineto{\pgfqpoint{4.764818in}{2.606151in}}%
\pgfpathlineto{\pgfqpoint{4.765683in}{2.557149in}}%
\pgfpathlineto{\pgfqpoint{4.766549in}{2.560407in}}%
\pgfpathlineto{\pgfqpoint{4.767415in}{2.579283in}}%
\pgfpathlineto{\pgfqpoint{4.768280in}{2.517275in}}%
\pgfpathlineto{\pgfqpoint{4.769146in}{2.531448in}}%
\pgfpathlineto{\pgfqpoint{4.770012in}{2.517245in}}%
\pgfpathlineto{\pgfqpoint{4.770878in}{2.575532in}}%
\pgfpathlineto{\pgfqpoint{4.772610in}{2.525175in}}%
\pgfpathlineto{\pgfqpoint{4.773477in}{2.548171in}}%
\pgfpathlineto{\pgfqpoint{4.774342in}{2.491482in}}%
\pgfpathlineto{\pgfqpoint{4.775208in}{2.559361in}}%
\pgfpathlineto{\pgfqpoint{4.776073in}{2.542884in}}%
\pgfpathlineto{\pgfqpoint{4.776937in}{2.541408in}}%
\pgfpathlineto{\pgfqpoint{4.777802in}{2.525668in}}%
\pgfpathlineto{\pgfqpoint{4.778668in}{2.541162in}}%
\pgfpathlineto{\pgfqpoint{4.779534in}{2.532185in}}%
\pgfpathlineto{\pgfqpoint{4.780401in}{2.489699in}}%
\pgfpathlineto{\pgfqpoint{4.781267in}{2.535813in}}%
\pgfpathlineto{\pgfqpoint{4.782133in}{2.525976in}}%
\pgfpathlineto{\pgfqpoint{4.782998in}{2.489576in}}%
\pgfpathlineto{\pgfqpoint{4.783865in}{2.596006in}}%
\pgfpathlineto{\pgfqpoint{4.784730in}{2.484658in}}%
\pgfpathlineto{\pgfqpoint{4.786461in}{2.561821in}}%
\pgfpathlineto{\pgfqpoint{4.787327in}{2.556625in}}%
\pgfpathlineto{\pgfqpoint{4.789059in}{2.582418in}}%
\pgfpathlineto{\pgfqpoint{4.790791in}{2.508513in}}%
\pgfpathlineto{\pgfqpoint{4.791657in}{2.532677in}}%
\pgfpathlineto{\pgfqpoint{4.792520in}{2.459203in}}%
\pgfpathlineto{\pgfqpoint{4.794249in}{2.542668in}}%
\pgfpathlineto{\pgfqpoint{4.795115in}{2.454100in}}%
\pgfpathlineto{\pgfqpoint{4.796845in}{2.580789in}}%
\pgfpathlineto{\pgfqpoint{4.797709in}{2.486687in}}%
\pgfpathlineto{\pgfqpoint{4.798574in}{2.631728in}}%
\pgfpathlineto{\pgfqpoint{4.801166in}{2.531879in}}%
\pgfpathlineto{\pgfqpoint{4.802030in}{2.502181in}}%
\pgfpathlineto{\pgfqpoint{4.803757in}{2.550201in}}%
\pgfpathlineto{\pgfqpoint{4.804623in}{2.505317in}}%
\pgfpathlineto{\pgfqpoint{4.805486in}{2.517799in}}%
\pgfpathlineto{\pgfqpoint{4.806350in}{2.558963in}}%
\pgfpathlineto{\pgfqpoint{4.807214in}{2.505932in}}%
\pgfpathlineto{\pgfqpoint{4.808943in}{2.590965in}}%
\pgfpathlineto{\pgfqpoint{4.810674in}{2.506240in}}%
\pgfpathlineto{\pgfqpoint{4.813264in}{2.615436in}}%
\pgfpathlineto{\pgfqpoint{4.814993in}{2.558809in}}%
\pgfpathlineto{\pgfqpoint{4.815858in}{2.556718in}}%
\pgfpathlineto{\pgfqpoint{4.816722in}{2.491359in}}%
\pgfpathlineto{\pgfqpoint{4.817588in}{2.544482in}}%
\pgfpathlineto{\pgfqpoint{4.818454in}{2.486318in}}%
\pgfpathlineto{\pgfqpoint{4.820184in}{2.553275in}}%
\pgfpathlineto{\pgfqpoint{4.821048in}{2.519212in}}%
\pgfpathlineto{\pgfqpoint{4.821914in}{2.568830in}}%
\pgfpathlineto{\pgfqpoint{4.822781in}{2.539625in}}%
\pgfpathlineto{\pgfqpoint{4.823647in}{2.565757in}}%
\pgfpathlineto{\pgfqpoint{4.825379in}{2.466490in}}%
\pgfpathlineto{\pgfqpoint{4.826244in}{2.467934in}}%
\pgfpathlineto{\pgfqpoint{4.827110in}{2.531448in}}%
\pgfpathlineto{\pgfqpoint{4.827976in}{2.524008in}}%
\pgfpathlineto{\pgfqpoint{4.828842in}{2.502674in}}%
\pgfpathlineto{\pgfqpoint{4.829709in}{2.522163in}}%
\pgfpathlineto{\pgfqpoint{4.830573in}{2.484350in}}%
\pgfpathlineto{\pgfqpoint{4.831440in}{2.515646in}}%
\pgfpathlineto{\pgfqpoint{4.832305in}{2.510358in}}%
\pgfpathlineto{\pgfqpoint{4.833170in}{2.515769in}}%
\pgfpathlineto{\pgfqpoint{4.834897in}{2.485058in}}%
\pgfpathlineto{\pgfqpoint{4.835762in}{2.493573in}}%
\pgfpathlineto{\pgfqpoint{4.836629in}{2.486749in}}%
\pgfpathlineto{\pgfqpoint{4.838359in}{2.521057in}}%
\pgfpathlineto{\pgfqpoint{4.839223in}{2.521057in}}%
\pgfpathlineto{\pgfqpoint{4.840088in}{2.516200in}}%
\pgfpathlineto{\pgfqpoint{4.840953in}{2.545159in}}%
\pgfpathlineto{\pgfqpoint{4.841818in}{2.502089in}}%
\pgfpathlineto{\pgfqpoint{4.842682in}{2.548602in}}%
\pgfpathlineto{\pgfqpoint{4.842682in}{2.548602in}}%
\pgfusepath{stroke}%
\end{pgfscope}%
\begin{pgfscope}%
\pgfsetrectcap%
\pgfsetmiterjoin%
\pgfsetlinewidth{0.803000pt}%
\definecolor{currentstroke}{rgb}{0.000000,0.000000,0.000000}%
\pgfsetstrokecolor{currentstroke}%
\pgfsetdash{}{0pt}%
\pgfpathmoveto{\pgfqpoint{0.483776in}{2.351653in}}%
\pgfpathlineto{\pgfqpoint{0.483776in}{2.936535in}}%
\pgfusepath{stroke}%
\end{pgfscope}%
\begin{pgfscope}%
\pgfsetrectcap%
\pgfsetmiterjoin%
\pgfsetlinewidth{0.803000pt}%
\definecolor{currentstroke}{rgb}{0.000000,0.000000,0.000000}%
\pgfsetstrokecolor{currentstroke}%
\pgfsetdash{}{0pt}%
\pgfpathmoveto{\pgfqpoint{5.050249in}{2.351653in}}%
\pgfpathlineto{\pgfqpoint{5.050249in}{2.936535in}}%
\pgfusepath{stroke}%
\end{pgfscope}%
\begin{pgfscope}%
\pgfsetrectcap%
\pgfsetmiterjoin%
\pgfsetlinewidth{0.803000pt}%
\definecolor{currentstroke}{rgb}{0.000000,0.000000,0.000000}%
\pgfsetstrokecolor{currentstroke}%
\pgfsetdash{}{0pt}%
\pgfpathmoveto{\pgfqpoint{0.483776in}{2.351653in}}%
\pgfpathlineto{\pgfqpoint{5.050249in}{2.351653in}}%
\pgfusepath{stroke}%
\end{pgfscope}%
\begin{pgfscope}%
\pgfsetrectcap%
\pgfsetmiterjoin%
\pgfsetlinewidth{0.803000pt}%
\definecolor{currentstroke}{rgb}{0.000000,0.000000,0.000000}%
\pgfsetstrokecolor{currentstroke}%
\pgfsetdash{}{0pt}%
\pgfpathmoveto{\pgfqpoint{0.483776in}{2.936535in}}%
\pgfpathlineto{\pgfqpoint{5.050249in}{2.936535in}}%
\pgfusepath{stroke}%
\end{pgfscope}%
\begin{pgfscope}%
\pgfsetbuttcap%
\pgfsetmiterjoin%
\definecolor{currentfill}{rgb}{1.000000,1.000000,1.000000}%
\pgfsetfillcolor{currentfill}%
\pgfsetlinewidth{0.000000pt}%
\definecolor{currentstroke}{rgb}{0.000000,0.000000,0.000000}%
\pgfsetstrokecolor{currentstroke}%
\pgfsetstrokeopacity{0.000000}%
\pgfsetdash{}{0pt}%
\pgfpathmoveto{\pgfqpoint{0.483776in}{1.444834in}}%
\pgfpathlineto{\pgfqpoint{5.050249in}{1.444834in}}%
\pgfpathlineto{\pgfqpoint{5.050249in}{2.029715in}}%
\pgfpathlineto{\pgfqpoint{0.483776in}{2.029715in}}%
\pgfpathlineto{\pgfqpoint{0.483776in}{1.444834in}}%
\pgfpathclose%
\pgfusepath{fill}%
\end{pgfscope}%
\begin{pgfscope}%
\pgfsetbuttcap%
\pgfsetroundjoin%
\definecolor{currentfill}{rgb}{0.000000,0.000000,0.000000}%
\pgfsetfillcolor{currentfill}%
\pgfsetlinewidth{0.803000pt}%
\definecolor{currentstroke}{rgb}{0.000000,0.000000,0.000000}%
\pgfsetstrokecolor{currentstroke}%
\pgfsetdash{}{0pt}%
\pgfsys@defobject{currentmarker}{\pgfqpoint{0.000000in}{-0.048611in}}{\pgfqpoint{0.000000in}{0.000000in}}{%
\pgfpathmoveto{\pgfqpoint{0.000000in}{0.000000in}}%
\pgfpathlineto{\pgfqpoint{0.000000in}{-0.048611in}}%
\pgfusepath{stroke,fill}%
}%
\begin{pgfscope}%
\pgfsys@transformshift{0.691021in}{1.444834in}%
\pgfsys@useobject{currentmarker}{}%
\end{pgfscope}%
\end{pgfscope}%
\begin{pgfscope}%
\pgfsetbuttcap%
\pgfsetroundjoin%
\definecolor{currentfill}{rgb}{0.000000,0.000000,0.000000}%
\pgfsetfillcolor{currentfill}%
\pgfsetlinewidth{0.803000pt}%
\definecolor{currentstroke}{rgb}{0.000000,0.000000,0.000000}%
\pgfsetstrokecolor{currentstroke}%
\pgfsetdash{}{0pt}%
\pgfsys@defobject{currentmarker}{\pgfqpoint{0.000000in}{-0.048611in}}{\pgfqpoint{0.000000in}{0.000000in}}{%
\pgfpathmoveto{\pgfqpoint{0.000000in}{0.000000in}}%
\pgfpathlineto{\pgfqpoint{0.000000in}{-0.048611in}}%
\pgfusepath{stroke,fill}%
}%
\begin{pgfscope}%
\pgfsys@transformshift{1.210067in}{1.444834in}%
\pgfsys@useobject{currentmarker}{}%
\end{pgfscope}%
\end{pgfscope}%
\begin{pgfscope}%
\pgfsetbuttcap%
\pgfsetroundjoin%
\definecolor{currentfill}{rgb}{0.000000,0.000000,0.000000}%
\pgfsetfillcolor{currentfill}%
\pgfsetlinewidth{0.803000pt}%
\definecolor{currentstroke}{rgb}{0.000000,0.000000,0.000000}%
\pgfsetstrokecolor{currentstroke}%
\pgfsetdash{}{0pt}%
\pgfsys@defobject{currentmarker}{\pgfqpoint{0.000000in}{-0.048611in}}{\pgfqpoint{0.000000in}{0.000000in}}{%
\pgfpathmoveto{\pgfqpoint{0.000000in}{0.000000in}}%
\pgfpathlineto{\pgfqpoint{0.000000in}{-0.048611in}}%
\pgfusepath{stroke,fill}%
}%
\begin{pgfscope}%
\pgfsys@transformshift{1.729114in}{1.444834in}%
\pgfsys@useobject{currentmarker}{}%
\end{pgfscope}%
\end{pgfscope}%
\begin{pgfscope}%
\pgfsetbuttcap%
\pgfsetroundjoin%
\definecolor{currentfill}{rgb}{0.000000,0.000000,0.000000}%
\pgfsetfillcolor{currentfill}%
\pgfsetlinewidth{0.803000pt}%
\definecolor{currentstroke}{rgb}{0.000000,0.000000,0.000000}%
\pgfsetstrokecolor{currentstroke}%
\pgfsetdash{}{0pt}%
\pgfsys@defobject{currentmarker}{\pgfqpoint{0.000000in}{-0.048611in}}{\pgfqpoint{0.000000in}{0.000000in}}{%
\pgfpathmoveto{\pgfqpoint{0.000000in}{0.000000in}}%
\pgfpathlineto{\pgfqpoint{0.000000in}{-0.048611in}}%
\pgfusepath{stroke,fill}%
}%
\begin{pgfscope}%
\pgfsys@transformshift{2.248160in}{1.444834in}%
\pgfsys@useobject{currentmarker}{}%
\end{pgfscope}%
\end{pgfscope}%
\begin{pgfscope}%
\pgfsetbuttcap%
\pgfsetroundjoin%
\definecolor{currentfill}{rgb}{0.000000,0.000000,0.000000}%
\pgfsetfillcolor{currentfill}%
\pgfsetlinewidth{0.803000pt}%
\definecolor{currentstroke}{rgb}{0.000000,0.000000,0.000000}%
\pgfsetstrokecolor{currentstroke}%
\pgfsetdash{}{0pt}%
\pgfsys@defobject{currentmarker}{\pgfqpoint{0.000000in}{-0.048611in}}{\pgfqpoint{0.000000in}{0.000000in}}{%
\pgfpathmoveto{\pgfqpoint{0.000000in}{0.000000in}}%
\pgfpathlineto{\pgfqpoint{0.000000in}{-0.048611in}}%
\pgfusepath{stroke,fill}%
}%
\begin{pgfscope}%
\pgfsys@transformshift{2.767206in}{1.444834in}%
\pgfsys@useobject{currentmarker}{}%
\end{pgfscope}%
\end{pgfscope}%
\begin{pgfscope}%
\pgfsetbuttcap%
\pgfsetroundjoin%
\definecolor{currentfill}{rgb}{0.000000,0.000000,0.000000}%
\pgfsetfillcolor{currentfill}%
\pgfsetlinewidth{0.803000pt}%
\definecolor{currentstroke}{rgb}{0.000000,0.000000,0.000000}%
\pgfsetstrokecolor{currentstroke}%
\pgfsetdash{}{0pt}%
\pgfsys@defobject{currentmarker}{\pgfqpoint{0.000000in}{-0.048611in}}{\pgfqpoint{0.000000in}{0.000000in}}{%
\pgfpathmoveto{\pgfqpoint{0.000000in}{0.000000in}}%
\pgfpathlineto{\pgfqpoint{0.000000in}{-0.048611in}}%
\pgfusepath{stroke,fill}%
}%
\begin{pgfscope}%
\pgfsys@transformshift{3.286252in}{1.444834in}%
\pgfsys@useobject{currentmarker}{}%
\end{pgfscope}%
\end{pgfscope}%
\begin{pgfscope}%
\pgfsetbuttcap%
\pgfsetroundjoin%
\definecolor{currentfill}{rgb}{0.000000,0.000000,0.000000}%
\pgfsetfillcolor{currentfill}%
\pgfsetlinewidth{0.803000pt}%
\definecolor{currentstroke}{rgb}{0.000000,0.000000,0.000000}%
\pgfsetstrokecolor{currentstroke}%
\pgfsetdash{}{0pt}%
\pgfsys@defobject{currentmarker}{\pgfqpoint{0.000000in}{-0.048611in}}{\pgfqpoint{0.000000in}{0.000000in}}{%
\pgfpathmoveto{\pgfqpoint{0.000000in}{0.000000in}}%
\pgfpathlineto{\pgfqpoint{0.000000in}{-0.048611in}}%
\pgfusepath{stroke,fill}%
}%
\begin{pgfscope}%
\pgfsys@transformshift{3.805298in}{1.444834in}%
\pgfsys@useobject{currentmarker}{}%
\end{pgfscope}%
\end{pgfscope}%
\begin{pgfscope}%
\pgfsetbuttcap%
\pgfsetroundjoin%
\definecolor{currentfill}{rgb}{0.000000,0.000000,0.000000}%
\pgfsetfillcolor{currentfill}%
\pgfsetlinewidth{0.803000pt}%
\definecolor{currentstroke}{rgb}{0.000000,0.000000,0.000000}%
\pgfsetstrokecolor{currentstroke}%
\pgfsetdash{}{0pt}%
\pgfsys@defobject{currentmarker}{\pgfqpoint{0.000000in}{-0.048611in}}{\pgfqpoint{0.000000in}{0.000000in}}{%
\pgfpathmoveto{\pgfqpoint{0.000000in}{0.000000in}}%
\pgfpathlineto{\pgfqpoint{0.000000in}{-0.048611in}}%
\pgfusepath{stroke,fill}%
}%
\begin{pgfscope}%
\pgfsys@transformshift{4.324344in}{1.444834in}%
\pgfsys@useobject{currentmarker}{}%
\end{pgfscope}%
\end{pgfscope}%
\begin{pgfscope}%
\pgfsetbuttcap%
\pgfsetroundjoin%
\definecolor{currentfill}{rgb}{0.000000,0.000000,0.000000}%
\pgfsetfillcolor{currentfill}%
\pgfsetlinewidth{0.803000pt}%
\definecolor{currentstroke}{rgb}{0.000000,0.000000,0.000000}%
\pgfsetstrokecolor{currentstroke}%
\pgfsetdash{}{0pt}%
\pgfsys@defobject{currentmarker}{\pgfqpoint{0.000000in}{-0.048611in}}{\pgfqpoint{0.000000in}{0.000000in}}{%
\pgfpathmoveto{\pgfqpoint{0.000000in}{0.000000in}}%
\pgfpathlineto{\pgfqpoint{0.000000in}{-0.048611in}}%
\pgfusepath{stroke,fill}%
}%
\begin{pgfscope}%
\pgfsys@transformshift{4.843390in}{1.444834in}%
\pgfsys@useobject{currentmarker}{}%
\end{pgfscope}%
\end{pgfscope}%
\begin{pgfscope}%
\pgfsetbuttcap%
\pgfsetroundjoin%
\definecolor{currentfill}{rgb}{0.000000,0.000000,0.000000}%
\pgfsetfillcolor{currentfill}%
\pgfsetlinewidth{0.803000pt}%
\definecolor{currentstroke}{rgb}{0.000000,0.000000,0.000000}%
\pgfsetstrokecolor{currentstroke}%
\pgfsetdash{}{0pt}%
\pgfsys@defobject{currentmarker}{\pgfqpoint{-0.048611in}{0.000000in}}{\pgfqpoint{-0.000000in}{0.000000in}}{%
\pgfpathmoveto{\pgfqpoint{-0.000000in}{0.000000in}}%
\pgfpathlineto{\pgfqpoint{-0.048611in}{0.000000in}}%
\pgfusepath{stroke,fill}%
}%
\begin{pgfscope}%
\pgfsys@transformshift{0.483776in}{1.626011in}%
\pgfsys@useobject{currentmarker}{}%
\end{pgfscope}%
\end{pgfscope}%
\begin{pgfscope}%
\definecolor{textcolor}{rgb}{0.000000,0.000000,0.000000}%
\pgfsetstrokecolor{textcolor}%
\pgfsetfillcolor{textcolor}%
\pgftext[x=0.327525in, y=1.587455in, left, base]{\color{textcolor}\rmfamily\fontsize{8.000000}{9.600000}\selectfont \(\displaystyle {0}\)}%
\end{pgfscope}%
\begin{pgfscope}%
\pgfsetbuttcap%
\pgfsetroundjoin%
\definecolor{currentfill}{rgb}{0.000000,0.000000,0.000000}%
\pgfsetfillcolor{currentfill}%
\pgfsetlinewidth{0.803000pt}%
\definecolor{currentstroke}{rgb}{0.000000,0.000000,0.000000}%
\pgfsetstrokecolor{currentstroke}%
\pgfsetdash{}{0pt}%
\pgfsys@defobject{currentmarker}{\pgfqpoint{-0.048611in}{0.000000in}}{\pgfqpoint{-0.000000in}{0.000000in}}{%
\pgfpathmoveto{\pgfqpoint{-0.000000in}{0.000000in}}%
\pgfpathlineto{\pgfqpoint{-0.048611in}{0.000000in}}%
\pgfusepath{stroke,fill}%
}%
\begin{pgfscope}%
\pgfsys@transformshift{0.483776in}{1.833186in}%
\pgfsys@useobject{currentmarker}{}%
\end{pgfscope}%
\end{pgfscope}%
\begin{pgfscope}%
\definecolor{textcolor}{rgb}{0.000000,0.000000,0.000000}%
\pgfsetstrokecolor{textcolor}%
\pgfsetfillcolor{textcolor}%
\pgftext[x=0.327525in, y=1.794630in, left, base]{\color{textcolor}\rmfamily\fontsize{8.000000}{9.600000}\selectfont \(\displaystyle {5}\)}%
\end{pgfscope}%
\begin{pgfscope}%
\definecolor{textcolor}{rgb}{0.000000,0.000000,0.000000}%
\pgfsetstrokecolor{textcolor}%
\pgfsetfillcolor{textcolor}%
\pgftext[x=0.271969in,y=1.737274in,,bottom,rotate=90.000000]{\color{textcolor}\rmfamily\fontsize{10.000000}{12.000000}\selectfont Voltage deviation in V}%
\end{pgfscope}%
\begin{pgfscope}%
\definecolor{textcolor}{rgb}{0.000000,0.000000,0.000000}%
\pgfsetstrokecolor{textcolor}%
\pgfsetfillcolor{textcolor}%
\pgftext[x=0.483776in,y=2.071382in,left,base]{\color{textcolor}\rmfamily\fontsize{8.000000}{9.600000}\selectfont \(\displaystyle \times{10^{\ensuremath{-}6}}{}\)}%
\end{pgfscope}%
\begin{pgfscope}%
\pgfpathrectangle{\pgfqpoint{0.483776in}{1.444834in}}{\pgfqpoint{4.566474in}{0.584881in}}%
\pgfusepath{clip}%
\pgfsetrectcap%
\pgfsetroundjoin%
\pgfsetlinewidth{0.501875pt}%
\definecolor{currentstroke}{rgb}{0.000000,0.419608,0.643137}%
\pgfsetstrokecolor{currentstroke}%
\pgfsetstrokeopacity{0.700000}%
\pgfsetdash{}{0pt}%
\pgfpathmoveto{\pgfqpoint{0.691343in}{1.596619in}}%
\pgfpathlineto{\pgfqpoint{0.692205in}{1.591513in}}%
\pgfpathlineto{\pgfqpoint{0.693071in}{1.636512in}}%
\pgfpathlineto{\pgfqpoint{0.693935in}{1.590000in}}%
\pgfpathlineto{\pgfqpoint{0.694800in}{1.622502in}}%
\pgfpathlineto{\pgfqpoint{0.696532in}{1.564503in}}%
\pgfpathlineto{\pgfqpoint{0.697397in}{1.619059in}}%
\pgfpathlineto{\pgfqpoint{0.698263in}{1.610629in}}%
\pgfpathlineto{\pgfqpoint{0.699128in}{1.631764in}}%
\pgfpathlineto{\pgfqpoint{0.699993in}{1.620010in}}%
\pgfpathlineto{\pgfqpoint{0.700859in}{1.589230in}}%
\pgfpathlineto{\pgfqpoint{0.701725in}{1.635267in}}%
\pgfpathlineto{\pgfqpoint{0.702589in}{1.632299in}}%
\pgfpathlineto{\pgfqpoint{0.703453in}{1.560261in}}%
\pgfpathlineto{\pgfqpoint{0.705185in}{1.630637in}}%
\pgfpathlineto{\pgfqpoint{0.706051in}{1.606446in}}%
\pgfpathlineto{\pgfqpoint{0.706916in}{1.633962in}}%
\pgfpathlineto{\pgfqpoint{0.707780in}{1.563734in}}%
\pgfpathlineto{\pgfqpoint{0.709512in}{1.641264in}}%
\pgfpathlineto{\pgfqpoint{0.710377in}{1.594068in}}%
\pgfpathlineto{\pgfqpoint{0.711241in}{1.638055in}}%
\pgfpathlineto{\pgfqpoint{0.713837in}{1.560053in}}%
\pgfpathlineto{\pgfqpoint{0.715567in}{1.651947in}}%
\pgfpathlineto{\pgfqpoint{0.717297in}{1.619653in}}%
\pgfpathlineto{\pgfqpoint{0.718163in}{1.549188in}}%
\pgfpathlineto{\pgfqpoint{0.719030in}{1.575011in}}%
\pgfpathlineto{\pgfqpoint{0.719895in}{1.559696in}}%
\pgfpathlineto{\pgfqpoint{0.721627in}{1.606059in}}%
\pgfpathlineto{\pgfqpoint{0.722492in}{1.557498in}}%
\pgfpathlineto{\pgfqpoint{0.724219in}{1.598697in}}%
\pgfpathlineto{\pgfqpoint{0.725084in}{1.538948in}}%
\pgfpathlineto{\pgfqpoint{0.727676in}{1.610867in}}%
\pgfpathlineto{\pgfqpoint{0.728541in}{1.538208in}}%
\pgfpathlineto{\pgfqpoint{0.731135in}{1.630161in}}%
\pgfpathlineto{\pgfqpoint{0.731999in}{1.530756in}}%
\pgfpathlineto{\pgfqpoint{0.733731in}{1.628023in}}%
\pgfpathlineto{\pgfqpoint{0.734597in}{1.605970in}}%
\pgfpathlineto{\pgfqpoint{0.735460in}{1.549128in}}%
\pgfpathlineto{\pgfqpoint{0.736325in}{1.631645in}}%
\pgfpathlineto{\pgfqpoint{0.737190in}{1.582195in}}%
\pgfpathlineto{\pgfqpoint{0.738920in}{1.627845in}}%
\pgfpathlineto{\pgfqpoint{0.739784in}{1.595611in}}%
\pgfpathlineto{\pgfqpoint{0.740649in}{1.605643in}}%
\pgfpathlineto{\pgfqpoint{0.742381in}{1.518973in}}%
\pgfpathlineto{\pgfqpoint{0.744977in}{1.639186in}}%
\pgfpathlineto{\pgfqpoint{0.746705in}{1.568661in}}%
\pgfpathlineto{\pgfqpoint{0.748435in}{1.669520in}}%
\pgfpathlineto{\pgfqpoint{0.750163in}{1.617579in}}%
\pgfpathlineto{\pgfqpoint{0.751028in}{1.624822in}}%
\pgfpathlineto{\pgfqpoint{0.751891in}{1.524970in}}%
\pgfpathlineto{\pgfqpoint{0.752757in}{1.642391in}}%
\pgfpathlineto{\pgfqpoint{0.753622in}{1.534824in}}%
\pgfpathlineto{\pgfqpoint{0.754488in}{1.630578in}}%
\pgfpathlineto{\pgfqpoint{0.755354in}{1.628916in}}%
\pgfpathlineto{\pgfqpoint{0.756219in}{1.610454in}}%
\pgfpathlineto{\pgfqpoint{0.757084in}{1.687092in}}%
\pgfpathlineto{\pgfqpoint{0.759679in}{1.594722in}}%
\pgfpathlineto{\pgfqpoint{0.761407in}{1.632478in}}%
\pgfpathlineto{\pgfqpoint{0.762269in}{1.608733in}}%
\pgfpathlineto{\pgfqpoint{0.763135in}{1.634378in}}%
\pgfpathlineto{\pgfqpoint{0.764000in}{1.784806in}}%
\pgfpathlineto{\pgfqpoint{0.766596in}{1.673320in}}%
\pgfpathlineto{\pgfqpoint{0.767462in}{1.752453in}}%
\pgfpathlineto{\pgfqpoint{0.768326in}{1.742183in}}%
\pgfpathlineto{\pgfqpoint{0.769191in}{1.752869in}}%
\pgfpathlineto{\pgfqpoint{0.771788in}{1.592346in}}%
\pgfpathlineto{\pgfqpoint{0.772652in}{1.682965in}}%
\pgfpathlineto{\pgfqpoint{0.774383in}{1.598816in}}%
\pgfpathlineto{\pgfqpoint{0.775249in}{1.627607in}}%
\pgfpathlineto{\pgfqpoint{0.776979in}{1.584330in}}%
\pgfpathlineto{\pgfqpoint{0.777845in}{1.590208in}}%
\pgfpathlineto{\pgfqpoint{0.779575in}{1.572963in}}%
\pgfpathlineto{\pgfqpoint{0.780439in}{1.634553in}}%
\pgfpathlineto{\pgfqpoint{0.781305in}{1.591454in}}%
\pgfpathlineto{\pgfqpoint{0.782170in}{1.593116in}}%
\pgfpathlineto{\pgfqpoint{0.783036in}{1.580117in}}%
\pgfpathlineto{\pgfqpoint{0.783901in}{1.581600in}}%
\pgfpathlineto{\pgfqpoint{0.784766in}{1.593592in}}%
\pgfpathlineto{\pgfqpoint{0.785630in}{1.530012in}}%
\pgfpathlineto{\pgfqpoint{0.786494in}{1.608848in}}%
\pgfpathlineto{\pgfqpoint{0.787360in}{1.607245in}}%
\pgfpathlineto{\pgfqpoint{0.788225in}{1.537137in}}%
\pgfpathlineto{\pgfqpoint{0.789089in}{1.538115in}}%
\pgfpathlineto{\pgfqpoint{0.790816in}{1.602259in}}%
\pgfpathlineto{\pgfqpoint{0.791680in}{1.600210in}}%
\pgfpathlineto{\pgfqpoint{0.793412in}{1.593949in}}%
\pgfpathlineto{\pgfqpoint{0.795138in}{1.575725in}}%
\pgfpathlineto{\pgfqpoint{0.796004in}{1.655747in}}%
\pgfpathlineto{\pgfqpoint{0.798599in}{1.566879in}}%
\pgfpathlineto{\pgfqpoint{0.799464in}{1.545923in}}%
\pgfpathlineto{\pgfqpoint{0.801192in}{1.607840in}}%
\pgfpathlineto{\pgfqpoint{0.802924in}{1.552218in}}%
\pgfpathlineto{\pgfqpoint{0.803787in}{1.549902in}}%
\pgfpathlineto{\pgfqpoint{0.804652in}{1.567474in}}%
\pgfpathlineto{\pgfqpoint{0.805515in}{1.508703in}}%
\pgfpathlineto{\pgfqpoint{0.806378in}{1.522178in}}%
\pgfpathlineto{\pgfqpoint{0.807243in}{1.582017in}}%
\pgfpathlineto{\pgfqpoint{0.808107in}{1.561239in}}%
\pgfpathlineto{\pgfqpoint{0.808973in}{1.571687in}}%
\pgfpathlineto{\pgfqpoint{0.809837in}{1.500333in}}%
\pgfpathlineto{\pgfqpoint{0.811567in}{1.570025in}}%
\pgfpathlineto{\pgfqpoint{0.812431in}{1.578038in}}%
\pgfpathlineto{\pgfqpoint{0.813297in}{1.571449in}}%
\pgfpathlineto{\pgfqpoint{0.814161in}{1.632950in}}%
\pgfpathlineto{\pgfqpoint{0.815026in}{1.547496in}}%
\pgfpathlineto{\pgfqpoint{0.815891in}{1.566166in}}%
\pgfpathlineto{\pgfqpoint{0.817620in}{1.605167in}}%
\pgfpathlineto{\pgfqpoint{0.818484in}{1.600419in}}%
\pgfpathlineto{\pgfqpoint{0.819349in}{1.562306in}}%
\pgfpathlineto{\pgfqpoint{0.820213in}{1.627696in}}%
\pgfpathlineto{\pgfqpoint{0.821078in}{1.621137in}}%
\pgfpathlineto{\pgfqpoint{0.822807in}{1.664887in}}%
\pgfpathlineto{\pgfqpoint{0.823671in}{1.600538in}}%
\pgfpathlineto{\pgfqpoint{0.824536in}{1.619267in}}%
\pgfpathlineto{\pgfqpoint{0.826268in}{1.602616in}}%
\pgfpathlineto{\pgfqpoint{0.827999in}{1.618170in}}%
\pgfpathlineto{\pgfqpoint{0.828864in}{1.556729in}}%
\pgfpathlineto{\pgfqpoint{0.830596in}{1.696054in}}%
\pgfpathlineto{\pgfqpoint{0.831462in}{1.617991in}}%
\pgfpathlineto{\pgfqpoint{0.832328in}{1.622859in}}%
\pgfpathlineto{\pgfqpoint{0.833193in}{1.666196in}}%
\pgfpathlineto{\pgfqpoint{0.834058in}{1.655182in}}%
\pgfpathlineto{\pgfqpoint{0.834922in}{1.663879in}}%
\pgfpathlineto{\pgfqpoint{0.836650in}{1.590744in}}%
\pgfpathlineto{\pgfqpoint{0.837516in}{1.554352in}}%
\pgfpathlineto{\pgfqpoint{0.838382in}{1.571568in}}%
\pgfpathlineto{\pgfqpoint{0.839248in}{1.531764in}}%
\pgfpathlineto{\pgfqpoint{0.840980in}{1.572992in}}%
\pgfpathlineto{\pgfqpoint{0.842710in}{1.547228in}}%
\pgfpathlineto{\pgfqpoint{0.844440in}{1.605107in}}%
\pgfpathlineto{\pgfqpoint{0.845305in}{1.600300in}}%
\pgfpathlineto{\pgfqpoint{0.846169in}{1.545566in}}%
\pgfpathlineto{\pgfqpoint{0.847897in}{1.599292in}}%
\pgfpathlineto{\pgfqpoint{0.849627in}{1.580176in}}%
\pgfpathlineto{\pgfqpoint{0.850491in}{1.575368in}}%
\pgfpathlineto{\pgfqpoint{0.851354in}{1.557498in}}%
\pgfpathlineto{\pgfqpoint{0.853081in}{1.600359in}}%
\pgfpathlineto{\pgfqpoint{0.853945in}{1.542361in}}%
\pgfpathlineto{\pgfqpoint{0.854810in}{1.630991in}}%
\pgfpathlineto{\pgfqpoint{0.855672in}{1.627815in}}%
\pgfpathlineto{\pgfqpoint{0.856537in}{1.595254in}}%
\pgfpathlineto{\pgfqpoint{0.857402in}{1.657231in}}%
\pgfpathlineto{\pgfqpoint{0.858267in}{1.596767in}}%
\pgfpathlineto{\pgfqpoint{0.859131in}{1.636453in}}%
\pgfpathlineto{\pgfqpoint{0.860864in}{1.577741in}}%
\pgfpathlineto{\pgfqpoint{0.861730in}{1.630693in}}%
\pgfpathlineto{\pgfqpoint{0.864329in}{1.560644in}}%
\pgfpathlineto{\pgfqpoint{0.866062in}{1.576019in}}%
\pgfpathlineto{\pgfqpoint{0.866929in}{1.597983in}}%
\pgfpathlineto{\pgfqpoint{0.867794in}{1.565333in}}%
\pgfpathlineto{\pgfqpoint{0.869523in}{1.610391in}}%
\pgfpathlineto{\pgfqpoint{0.870388in}{1.588427in}}%
\pgfpathlineto{\pgfqpoint{0.871253in}{1.636393in}}%
\pgfpathlineto{\pgfqpoint{0.872119in}{1.630693in}}%
\pgfpathlineto{\pgfqpoint{0.873848in}{1.621550in}}%
\pgfpathlineto{\pgfqpoint{0.874712in}{1.570438in}}%
\pgfpathlineto{\pgfqpoint{0.876442in}{1.691008in}}%
\pgfpathlineto{\pgfqpoint{0.877309in}{1.664946in}}%
\pgfpathlineto{\pgfqpoint{0.878174in}{1.667144in}}%
\pgfpathlineto{\pgfqpoint{0.879903in}{1.651709in}}%
\pgfpathlineto{\pgfqpoint{0.881633in}{1.694749in}}%
\pgfpathlineto{\pgfqpoint{0.882498in}{1.646187in}}%
\pgfpathlineto{\pgfqpoint{0.883364in}{1.713150in}}%
\pgfpathlineto{\pgfqpoint{0.884229in}{1.703356in}}%
\pgfpathlineto{\pgfqpoint{0.885096in}{1.669282in}}%
\pgfpathlineto{\pgfqpoint{0.885960in}{1.702999in}}%
\pgfpathlineto{\pgfqpoint{0.886826in}{1.628142in}}%
\pgfpathlineto{\pgfqpoint{0.887691in}{1.653490in}}%
\pgfpathlineto{\pgfqpoint{0.888555in}{1.717129in}}%
\pgfpathlineto{\pgfqpoint{0.892011in}{1.653193in}}%
\pgfpathlineto{\pgfqpoint{0.892875in}{1.734047in}}%
\pgfpathlineto{\pgfqpoint{0.893741in}{1.654557in}}%
\pgfpathlineto{\pgfqpoint{0.894605in}{1.695102in}}%
\pgfpathlineto{\pgfqpoint{0.897202in}{1.588427in}}%
\pgfpathlineto{\pgfqpoint{0.898067in}{1.593235in}}%
\pgfpathlineto{\pgfqpoint{0.899795in}{1.651293in}}%
\pgfpathlineto{\pgfqpoint{0.900660in}{1.598310in}}%
\pgfpathlineto{\pgfqpoint{0.901525in}{1.619356in}}%
\pgfpathlineto{\pgfqpoint{0.902390in}{1.677295in}}%
\pgfpathlineto{\pgfqpoint{0.903254in}{1.612380in}}%
\pgfpathlineto{\pgfqpoint{0.904119in}{1.652244in}}%
\pgfpathlineto{\pgfqpoint{0.904984in}{1.593116in}}%
\pgfpathlineto{\pgfqpoint{0.905849in}{1.665333in}}%
\pgfpathlineto{\pgfqpoint{0.906714in}{1.656279in}}%
\pgfpathlineto{\pgfqpoint{0.907579in}{1.650106in}}%
\pgfpathlineto{\pgfqpoint{0.908445in}{1.596619in}}%
\pgfpathlineto{\pgfqpoint{0.910175in}{1.645061in}}%
\pgfpathlineto{\pgfqpoint{0.912770in}{1.569282in}}%
\pgfpathlineto{\pgfqpoint{0.914501in}{1.562127in}}%
\pgfpathlineto{\pgfqpoint{0.915365in}{1.582786in}}%
\pgfpathlineto{\pgfqpoint{0.916229in}{1.546574in}}%
\pgfpathlineto{\pgfqpoint{0.917959in}{1.613715in}}%
\pgfpathlineto{\pgfqpoint{0.918824in}{1.552690in}}%
\pgfpathlineto{\pgfqpoint{0.919687in}{1.556104in}}%
\pgfpathlineto{\pgfqpoint{0.920552in}{1.542539in}}%
\pgfpathlineto{\pgfqpoint{0.922280in}{1.625499in}}%
\pgfpathlineto{\pgfqpoint{0.923146in}{1.597094in}}%
\pgfpathlineto{\pgfqpoint{0.924010in}{1.553877in}}%
\pgfpathlineto{\pgfqpoint{0.925740in}{1.631110in}}%
\pgfpathlineto{\pgfqpoint{0.928329in}{1.557974in}}%
\pgfpathlineto{\pgfqpoint{0.930058in}{1.603802in}}%
\pgfpathlineto{\pgfqpoint{0.930923in}{1.541528in}}%
\pgfpathlineto{\pgfqpoint{0.931787in}{1.610094in}}%
\pgfpathlineto{\pgfqpoint{0.932652in}{1.555598in}}%
\pgfpathlineto{\pgfqpoint{0.933516in}{1.584092in}}%
\pgfpathlineto{\pgfqpoint{0.934381in}{1.531496in}}%
\pgfpathlineto{\pgfqpoint{0.936109in}{1.601307in}}%
\pgfpathlineto{\pgfqpoint{0.936974in}{1.599526in}}%
\pgfpathlineto{\pgfqpoint{0.937840in}{1.629269in}}%
\pgfpathlineto{\pgfqpoint{0.938706in}{1.543190in}}%
\pgfpathlineto{\pgfqpoint{0.939570in}{1.557082in}}%
\pgfpathlineto{\pgfqpoint{0.940435in}{1.640011in}}%
\pgfpathlineto{\pgfqpoint{0.943032in}{1.518553in}}%
\pgfpathlineto{\pgfqpoint{0.943897in}{1.606175in}}%
\pgfpathlineto{\pgfqpoint{0.944763in}{1.599586in}}%
\pgfpathlineto{\pgfqpoint{0.945628in}{1.627309in}}%
\pgfpathlineto{\pgfqpoint{0.946490in}{1.547228in}}%
\pgfpathlineto{\pgfqpoint{0.947355in}{1.605345in}}%
\pgfpathlineto{\pgfqpoint{0.949082in}{1.571390in}}%
\pgfpathlineto{\pgfqpoint{0.949946in}{1.566225in}}%
\pgfpathlineto{\pgfqpoint{0.950811in}{1.544703in}}%
\pgfpathlineto{\pgfqpoint{0.951674in}{1.488219in}}%
\pgfpathlineto{\pgfqpoint{0.953405in}{1.594272in}}%
\pgfpathlineto{\pgfqpoint{0.954269in}{1.610391in}}%
\pgfpathlineto{\pgfqpoint{0.955133in}{1.519564in}}%
\pgfpathlineto{\pgfqpoint{0.956863in}{1.648325in}}%
\pgfpathlineto{\pgfqpoint{0.957728in}{1.610570in}}%
\pgfpathlineto{\pgfqpoint{0.958594in}{1.615734in}}%
\pgfpathlineto{\pgfqpoint{0.961191in}{1.571033in}}%
\pgfpathlineto{\pgfqpoint{0.962055in}{1.590625in}}%
\pgfpathlineto{\pgfqpoint{0.962920in}{1.537018in}}%
\pgfpathlineto{\pgfqpoint{0.963783in}{1.619059in}}%
\pgfpathlineto{\pgfqpoint{0.964647in}{1.581481in}}%
\pgfpathlineto{\pgfqpoint{0.966375in}{1.614132in}}%
\pgfpathlineto{\pgfqpoint{0.967241in}{1.554085in}}%
\pgfpathlineto{\pgfqpoint{0.968105in}{1.590625in}}%
\pgfpathlineto{\pgfqpoint{0.968970in}{1.574654in}}%
\pgfpathlineto{\pgfqpoint{0.969836in}{1.576555in}}%
\pgfpathlineto{\pgfqpoint{0.970701in}{1.635088in}}%
\pgfpathlineto{\pgfqpoint{0.972432in}{1.491011in}}%
\pgfpathlineto{\pgfqpoint{0.975027in}{1.636215in}}%
\pgfpathlineto{\pgfqpoint{0.975889in}{1.550433in}}%
\pgfpathlineto{\pgfqpoint{0.976751in}{1.557409in}}%
\pgfpathlineto{\pgfqpoint{0.977616in}{1.638293in}}%
\pgfpathlineto{\pgfqpoint{0.978481in}{1.622918in}}%
\pgfpathlineto{\pgfqpoint{0.980211in}{1.599054in}}%
\pgfpathlineto{\pgfqpoint{0.981076in}{1.655093in}}%
\pgfpathlineto{\pgfqpoint{0.982806in}{1.611165in}}%
\pgfpathlineto{\pgfqpoint{0.983671in}{1.629745in}}%
\pgfpathlineto{\pgfqpoint{0.985400in}{1.603981in}}%
\pgfpathlineto{\pgfqpoint{0.986264in}{1.621048in}}%
\pgfpathlineto{\pgfqpoint{0.987127in}{1.532031in}}%
\pgfpathlineto{\pgfqpoint{0.988856in}{1.586051in}}%
\pgfpathlineto{\pgfqpoint{0.989722in}{1.576317in}}%
\pgfpathlineto{\pgfqpoint{0.991449in}{1.613805in}}%
\pgfpathlineto{\pgfqpoint{0.992314in}{1.593413in}}%
\pgfpathlineto{\pgfqpoint{0.993178in}{1.616151in}}%
\pgfpathlineto{\pgfqpoint{0.994043in}{1.511611in}}%
\pgfpathlineto{\pgfqpoint{0.995771in}{1.625885in}}%
\pgfpathlineto{\pgfqpoint{0.996636in}{1.581303in}}%
\pgfpathlineto{\pgfqpoint{0.997501in}{1.613953in}}%
\pgfpathlineto{\pgfqpoint{0.999232in}{1.567530in}}%
\pgfpathlineto{\pgfqpoint{1.000096in}{1.630574in}}%
\pgfpathlineto{\pgfqpoint{1.000961in}{1.555003in}}%
\pgfpathlineto{\pgfqpoint{1.001826in}{1.614370in}}%
\pgfpathlineto{\pgfqpoint{1.002691in}{1.603505in}}%
\pgfpathlineto{\pgfqpoint{1.003553in}{1.587892in}}%
\pgfpathlineto{\pgfqpoint{1.004419in}{1.598340in}}%
\pgfpathlineto{\pgfqpoint{1.005284in}{1.487181in}}%
\pgfpathlineto{\pgfqpoint{1.007014in}{1.632534in}}%
\pgfpathlineto{\pgfqpoint{1.007877in}{1.605078in}}%
\pgfpathlineto{\pgfqpoint{1.008742in}{1.613061in}}%
\pgfpathlineto{\pgfqpoint{1.009607in}{1.592759in}}%
\pgfpathlineto{\pgfqpoint{1.010472in}{1.613537in}}%
\pgfpathlineto{\pgfqpoint{1.011338in}{1.536363in}}%
\pgfpathlineto{\pgfqpoint{1.013067in}{1.643161in}}%
\pgfpathlineto{\pgfqpoint{1.013933in}{1.617753in}}%
\pgfpathlineto{\pgfqpoint{1.014798in}{1.556342in}}%
\pgfpathlineto{\pgfqpoint{1.016526in}{1.631883in}}%
\pgfpathlineto{\pgfqpoint{1.019120in}{1.587122in}}%
\pgfpathlineto{\pgfqpoint{1.019986in}{1.548980in}}%
\pgfpathlineto{\pgfqpoint{1.020851in}{1.606889in}}%
\pgfpathlineto{\pgfqpoint{1.021716in}{1.569668in}}%
\pgfpathlineto{\pgfqpoint{1.022581in}{1.569966in}}%
\pgfpathlineto{\pgfqpoint{1.023445in}{1.627666in}}%
\pgfpathlineto{\pgfqpoint{1.025173in}{1.558982in}}%
\pgfpathlineto{\pgfqpoint{1.027767in}{1.720810in}}%
\pgfpathlineto{\pgfqpoint{1.028629in}{1.710778in}}%
\pgfpathlineto{\pgfqpoint{1.029494in}{1.611343in}}%
\pgfpathlineto{\pgfqpoint{1.030357in}{1.716121in}}%
\pgfpathlineto{\pgfqpoint{1.032087in}{1.678723in}}%
\pgfpathlineto{\pgfqpoint{1.033816in}{1.636988in}}%
\pgfpathlineto{\pgfqpoint{1.034679in}{1.638888in}}%
\pgfpathlineto{\pgfqpoint{1.035545in}{1.650701in}}%
\pgfpathlineto{\pgfqpoint{1.036409in}{1.686884in}}%
\pgfpathlineto{\pgfqpoint{1.037275in}{1.645656in}}%
\pgfpathlineto{\pgfqpoint{1.038139in}{1.703535in}}%
\pgfpathlineto{\pgfqpoint{1.039005in}{1.696946in}}%
\pgfpathlineto{\pgfqpoint{1.039870in}{1.722353in}}%
\pgfpathlineto{\pgfqpoint{1.040734in}{1.708224in}}%
\pgfpathlineto{\pgfqpoint{1.041600in}{1.726153in}}%
\pgfpathlineto{\pgfqpoint{1.042466in}{1.705554in}}%
\pgfpathlineto{\pgfqpoint{1.045060in}{1.762069in}}%
\pgfpathlineto{\pgfqpoint{1.045927in}{1.694867in}}%
\pgfpathlineto{\pgfqpoint{1.046792in}{1.720989in}}%
\pgfpathlineto{\pgfqpoint{1.048523in}{1.686468in}}%
\pgfpathlineto{\pgfqpoint{1.049390in}{1.691543in}}%
\pgfpathlineto{\pgfqpoint{1.050256in}{1.751799in}}%
\pgfpathlineto{\pgfqpoint{1.051987in}{1.667739in}}%
\pgfpathlineto{\pgfqpoint{1.053719in}{1.735174in}}%
\pgfpathlineto{\pgfqpoint{1.055449in}{1.663165in}}%
\pgfpathlineto{\pgfqpoint{1.056313in}{1.710064in}}%
\pgfpathlineto{\pgfqpoint{1.058907in}{1.644823in}}%
\pgfpathlineto{\pgfqpoint{1.061503in}{1.726034in}}%
\pgfpathlineto{\pgfqpoint{1.062369in}{1.679254in}}%
\pgfpathlineto{\pgfqpoint{1.063234in}{1.552036in}}%
\pgfpathlineto{\pgfqpoint{1.064099in}{1.678302in}}%
\pgfpathlineto{\pgfqpoint{1.065827in}{1.572814in}}%
\pgfpathlineto{\pgfqpoint{1.066692in}{1.577860in}}%
\pgfpathlineto{\pgfqpoint{1.068422in}{1.552333in}}%
\pgfpathlineto{\pgfqpoint{1.069287in}{1.575722in}}%
\pgfpathlineto{\pgfqpoint{1.070151in}{1.674859in}}%
\pgfpathlineto{\pgfqpoint{1.071880in}{1.591067in}}%
\pgfpathlineto{\pgfqpoint{1.072744in}{1.634374in}}%
\pgfpathlineto{\pgfqpoint{1.074470in}{1.546455in}}%
\pgfpathlineto{\pgfqpoint{1.075335in}{1.572635in}}%
\pgfpathlineto{\pgfqpoint{1.076199in}{1.557320in}}%
\pgfpathlineto{\pgfqpoint{1.077064in}{1.613864in}}%
\pgfpathlineto{\pgfqpoint{1.077927in}{1.562425in}}%
\pgfpathlineto{\pgfqpoint{1.078792in}{1.615615in}}%
\pgfpathlineto{\pgfqpoint{1.079657in}{1.524729in}}%
\pgfpathlineto{\pgfqpoint{1.081384in}{1.599705in}}%
\pgfpathlineto{\pgfqpoint{1.082248in}{1.551322in}}%
\pgfpathlineto{\pgfqpoint{1.083114in}{1.601010in}}%
\pgfpathlineto{\pgfqpoint{1.083981in}{1.599675in}}%
\pgfpathlineto{\pgfqpoint{1.085711in}{1.559930in}}%
\pgfpathlineto{\pgfqpoint{1.086576in}{1.628763in}}%
\pgfpathlineto{\pgfqpoint{1.087440in}{1.601188in}}%
\pgfpathlineto{\pgfqpoint{1.088306in}{1.630039in}}%
\pgfpathlineto{\pgfqpoint{1.089172in}{1.609558in}}%
\pgfpathlineto{\pgfqpoint{1.090037in}{1.618226in}}%
\pgfpathlineto{\pgfqpoint{1.090903in}{1.575722in}}%
\pgfpathlineto{\pgfqpoint{1.091767in}{1.589732in}}%
\pgfpathlineto{\pgfqpoint{1.092632in}{1.551798in}}%
\pgfpathlineto{\pgfqpoint{1.093498in}{1.595845in}}%
\pgfpathlineto{\pgfqpoint{1.094362in}{1.519798in}}%
\pgfpathlineto{\pgfqpoint{1.095225in}{1.597150in}}%
\pgfpathlineto{\pgfqpoint{1.096090in}{1.559841in}}%
\pgfpathlineto{\pgfqpoint{1.097822in}{1.642506in}}%
\pgfpathlineto{\pgfqpoint{1.098687in}{1.634255in}}%
\pgfpathlineto{\pgfqpoint{1.099551in}{1.637877in}}%
\pgfpathlineto{\pgfqpoint{1.100416in}{1.591275in}}%
\pgfpathlineto{\pgfqpoint{1.101281in}{1.597388in}}%
\pgfpathlineto{\pgfqpoint{1.102145in}{1.596619in}}%
\pgfpathlineto{\pgfqpoint{1.103008in}{1.532980in}}%
\pgfpathlineto{\pgfqpoint{1.104736in}{1.582013in}}%
\pgfpathlineto{\pgfqpoint{1.105601in}{1.573584in}}%
\pgfpathlineto{\pgfqpoint{1.106466in}{1.581124in}}%
\pgfpathlineto{\pgfqpoint{1.107328in}{1.545090in}}%
\pgfpathlineto{\pgfqpoint{1.109055in}{1.576967in}}%
\pgfpathlineto{\pgfqpoint{1.109918in}{1.565333in}}%
\pgfpathlineto{\pgfqpoint{1.111647in}{1.584151in}}%
\pgfpathlineto{\pgfqpoint{1.112512in}{1.529715in}}%
\pgfpathlineto{\pgfqpoint{1.113375in}{1.581184in}}%
\pgfpathlineto{\pgfqpoint{1.114239in}{1.565690in}}%
\pgfpathlineto{\pgfqpoint{1.115102in}{1.637609in}}%
\pgfpathlineto{\pgfqpoint{1.115967in}{1.555896in}}%
\pgfpathlineto{\pgfqpoint{1.116831in}{1.653847in}}%
\pgfpathlineto{\pgfqpoint{1.117696in}{1.549307in}}%
\pgfpathlineto{\pgfqpoint{1.118561in}{1.574773in}}%
\pgfpathlineto{\pgfqpoint{1.119426in}{1.569936in}}%
\pgfpathlineto{\pgfqpoint{1.120289in}{1.573230in}}%
\pgfpathlineto{\pgfqpoint{1.122016in}{1.695254in}}%
\pgfpathlineto{\pgfqpoint{1.122881in}{1.659369in}}%
\pgfpathlineto{\pgfqpoint{1.124612in}{1.682668in}}%
\pgfpathlineto{\pgfqpoint{1.125476in}{1.655688in}}%
\pgfpathlineto{\pgfqpoint{1.126342in}{1.720751in}}%
\pgfpathlineto{\pgfqpoint{1.127208in}{1.646723in}}%
\pgfpathlineto{\pgfqpoint{1.128072in}{1.738974in}}%
\pgfpathlineto{\pgfqpoint{1.128936in}{1.629715in}}%
\pgfpathlineto{\pgfqpoint{1.129800in}{1.650106in}}%
\pgfpathlineto{\pgfqpoint{1.130665in}{1.677057in}}%
\pgfpathlineto{\pgfqpoint{1.132396in}{1.617158in}}%
\pgfpathlineto{\pgfqpoint{1.133261in}{1.670051in}}%
\pgfpathlineto{\pgfqpoint{1.134126in}{1.658566in}}%
\pgfpathlineto{\pgfqpoint{1.134991in}{1.665184in}}%
\pgfpathlineto{\pgfqpoint{1.136722in}{1.604159in}}%
\pgfpathlineto{\pgfqpoint{1.137586in}{1.624164in}}%
\pgfpathlineto{\pgfqpoint{1.138451in}{1.660525in}}%
\pgfpathlineto{\pgfqpoint{1.139316in}{1.643518in}}%
\pgfpathlineto{\pgfqpoint{1.140181in}{1.663701in}}%
\pgfpathlineto{\pgfqpoint{1.141044in}{1.645771in}}%
\pgfpathlineto{\pgfqpoint{1.141910in}{1.687505in}}%
\pgfpathlineto{\pgfqpoint{1.142776in}{1.577979in}}%
\pgfpathlineto{\pgfqpoint{1.143641in}{1.596946in}}%
\pgfpathlineto{\pgfqpoint{1.147100in}{1.737550in}}%
\pgfpathlineto{\pgfqpoint{1.148829in}{1.670825in}}%
\pgfpathlineto{\pgfqpoint{1.149692in}{1.703416in}}%
\pgfpathlineto{\pgfqpoint{1.150558in}{1.664087in}}%
\pgfpathlineto{\pgfqpoint{1.151424in}{1.682578in}}%
\pgfpathlineto{\pgfqpoint{1.152290in}{1.680738in}}%
\pgfpathlineto{\pgfqpoint{1.153155in}{1.634166in}}%
\pgfpathlineto{\pgfqpoint{1.155749in}{1.684419in}}%
\pgfpathlineto{\pgfqpoint{1.156614in}{1.665779in}}%
\pgfpathlineto{\pgfqpoint{1.157478in}{1.675960in}}%
\pgfpathlineto{\pgfqpoint{1.158343in}{1.653669in}}%
\pgfpathlineto{\pgfqpoint{1.159208in}{1.699556in}}%
\pgfpathlineto{\pgfqpoint{1.160073in}{1.631496in}}%
\pgfpathlineto{\pgfqpoint{1.160937in}{1.769665in}}%
\pgfpathlineto{\pgfqpoint{1.164394in}{1.589673in}}%
\pgfpathlineto{\pgfqpoint{1.165259in}{1.588070in}}%
\pgfpathlineto{\pgfqpoint{1.166124in}{1.592878in}}%
\pgfpathlineto{\pgfqpoint{1.166990in}{1.642774in}}%
\pgfpathlineto{\pgfqpoint{1.170449in}{1.526569in}}%
\pgfpathlineto{\pgfqpoint{1.172180in}{1.625528in}}%
\pgfpathlineto{\pgfqpoint{1.173045in}{1.610094in}}%
\pgfpathlineto{\pgfqpoint{1.173912in}{1.617277in}}%
\pgfpathlineto{\pgfqpoint{1.174778in}{1.558863in}}%
\pgfpathlineto{\pgfqpoint{1.176508in}{1.593473in}}%
\pgfpathlineto{\pgfqpoint{1.177374in}{1.607305in}}%
\pgfpathlineto{\pgfqpoint{1.178239in}{1.560882in}}%
\pgfpathlineto{\pgfqpoint{1.179969in}{1.589851in}}%
\pgfpathlineto{\pgfqpoint{1.180834in}{1.564325in}}%
\pgfpathlineto{\pgfqpoint{1.181699in}{1.618940in}}%
\pgfpathlineto{\pgfqpoint{1.182565in}{1.587122in}}%
\pgfpathlineto{\pgfqpoint{1.183430in}{1.608670in}}%
\pgfpathlineto{\pgfqpoint{1.186026in}{1.552482in}}%
\pgfpathlineto{\pgfqpoint{1.187758in}{1.605286in}}%
\pgfpathlineto{\pgfqpoint{1.189487in}{1.569192in}}%
\pgfpathlineto{\pgfqpoint{1.190352in}{1.610510in}}%
\pgfpathlineto{\pgfqpoint{1.191218in}{1.553906in}}%
\pgfpathlineto{\pgfqpoint{1.192948in}{1.677473in}}%
\pgfpathlineto{\pgfqpoint{1.193813in}{1.611046in}}%
\pgfpathlineto{\pgfqpoint{1.194678in}{1.669044in}}%
\pgfpathlineto{\pgfqpoint{1.195541in}{1.542272in}}%
\pgfpathlineto{\pgfqpoint{1.196406in}{1.629626in}}%
\pgfpathlineto{\pgfqpoint{1.198137in}{1.572933in}}%
\pgfpathlineto{\pgfqpoint{1.199001in}{1.648385in}}%
\pgfpathlineto{\pgfqpoint{1.199865in}{1.537315in}}%
\pgfpathlineto{\pgfqpoint{1.200729in}{1.593503in}}%
\pgfpathlineto{\pgfqpoint{1.201595in}{1.495819in}}%
\pgfpathlineto{\pgfqpoint{1.202459in}{1.610867in}}%
\pgfpathlineto{\pgfqpoint{1.203324in}{1.501400in}}%
\pgfpathlineto{\pgfqpoint{1.204189in}{1.568839in}}%
\pgfpathlineto{\pgfqpoint{1.205054in}{1.556996in}}%
\pgfpathlineto{\pgfqpoint{1.206785in}{1.599173in}}%
\pgfpathlineto{\pgfqpoint{1.209379in}{1.578038in}}%
\pgfpathlineto{\pgfqpoint{1.210246in}{1.499559in}}%
\pgfpathlineto{\pgfqpoint{1.211111in}{1.597094in}}%
\pgfpathlineto{\pgfqpoint{1.211977in}{1.579284in}}%
\pgfpathlineto{\pgfqpoint{1.212842in}{1.588130in}}%
\pgfpathlineto{\pgfqpoint{1.213706in}{1.620959in}}%
\pgfpathlineto{\pgfqpoint{1.214572in}{1.593205in}}%
\pgfpathlineto{\pgfqpoint{1.215438in}{1.665422in}}%
\pgfpathlineto{\pgfqpoint{1.218035in}{1.584746in}}%
\pgfpathlineto{\pgfqpoint{1.218902in}{1.684062in}}%
\pgfpathlineto{\pgfqpoint{1.219768in}{1.600181in}}%
\pgfpathlineto{\pgfqpoint{1.220634in}{1.612886in}}%
\pgfpathlineto{\pgfqpoint{1.222366in}{1.609146in}}%
\pgfpathlineto{\pgfqpoint{1.224098in}{1.615943in}}%
\pgfpathlineto{\pgfqpoint{1.224965in}{1.571271in}}%
\pgfpathlineto{\pgfqpoint{1.225831in}{1.607959in}}%
\pgfpathlineto{\pgfqpoint{1.227562in}{1.566701in}}%
\pgfpathlineto{\pgfqpoint{1.228427in}{1.595194in}}%
\pgfpathlineto{\pgfqpoint{1.229292in}{1.563288in}}%
\pgfpathlineto{\pgfqpoint{1.230156in}{1.585281in}}%
\pgfpathlineto{\pgfqpoint{1.231020in}{1.574654in}}%
\pgfpathlineto{\pgfqpoint{1.232751in}{1.587003in}}%
\pgfpathlineto{\pgfqpoint{1.233617in}{1.566552in}}%
\pgfpathlineto{\pgfqpoint{1.236211in}{1.618642in}}%
\pgfpathlineto{\pgfqpoint{1.237942in}{1.538144in}}%
\pgfpathlineto{\pgfqpoint{1.238807in}{1.618702in}}%
\pgfpathlineto{\pgfqpoint{1.239671in}{1.608432in}}%
\pgfpathlineto{\pgfqpoint{1.240537in}{1.636334in}}%
\pgfpathlineto{\pgfqpoint{1.241402in}{1.548950in}}%
\pgfpathlineto{\pgfqpoint{1.242268in}{1.557558in}}%
\pgfpathlineto{\pgfqpoint{1.243133in}{1.555539in}}%
\pgfpathlineto{\pgfqpoint{1.244864in}{1.590208in}}%
\pgfpathlineto{\pgfqpoint{1.245728in}{1.550880in}}%
\pgfpathlineto{\pgfqpoint{1.246593in}{1.602021in}}%
\pgfpathlineto{\pgfqpoint{1.247458in}{1.548771in}}%
\pgfpathlineto{\pgfqpoint{1.249187in}{1.609737in}}%
\pgfpathlineto{\pgfqpoint{1.250053in}{1.606532in}}%
\pgfpathlineto{\pgfqpoint{1.250916in}{1.563076in}}%
\pgfpathlineto{\pgfqpoint{1.251781in}{1.611815in}}%
\pgfpathlineto{\pgfqpoint{1.252647in}{1.605286in}}%
\pgfpathlineto{\pgfqpoint{1.253512in}{1.626064in}}%
\pgfpathlineto{\pgfqpoint{1.255243in}{1.576198in}}%
\pgfpathlineto{\pgfqpoint{1.256974in}{1.600002in}}%
\pgfpathlineto{\pgfqpoint{1.257840in}{1.595934in}}%
\pgfpathlineto{\pgfqpoint{1.258706in}{1.567114in}}%
\pgfpathlineto{\pgfqpoint{1.259571in}{1.619416in}}%
\pgfpathlineto{\pgfqpoint{1.261299in}{1.575841in}}%
\pgfpathlineto{\pgfqpoint{1.262165in}{1.584538in}}%
\pgfpathlineto{\pgfqpoint{1.263030in}{1.581124in}}%
\pgfpathlineto{\pgfqpoint{1.263893in}{1.550612in}}%
\pgfpathlineto{\pgfqpoint{1.264759in}{1.572784in}}%
\pgfpathlineto{\pgfqpoint{1.265625in}{1.632474in}}%
\pgfpathlineto{\pgfqpoint{1.267353in}{1.551679in}}%
\pgfpathlineto{\pgfqpoint{1.268216in}{1.559160in}}%
\pgfpathlineto{\pgfqpoint{1.269081in}{1.598102in}}%
\pgfpathlineto{\pgfqpoint{1.269946in}{1.596262in}}%
\pgfpathlineto{\pgfqpoint{1.270811in}{1.610034in}}%
\pgfpathlineto{\pgfqpoint{1.272543in}{1.531496in}}%
\pgfpathlineto{\pgfqpoint{1.273408in}{1.585162in}}%
\pgfpathlineto{\pgfqpoint{1.274272in}{1.572903in}}%
\pgfpathlineto{\pgfqpoint{1.275137in}{1.591870in}}%
\pgfpathlineto{\pgfqpoint{1.276003in}{1.576614in}}%
\pgfpathlineto{\pgfqpoint{1.276869in}{1.533247in}}%
\pgfpathlineto{\pgfqpoint{1.277734in}{1.743841in}}%
\pgfpathlineto{\pgfqpoint{1.278600in}{1.564028in}}%
\pgfpathlineto{\pgfqpoint{1.279465in}{1.601843in}}%
\pgfpathlineto{\pgfqpoint{1.280331in}{1.626123in}}%
\pgfpathlineto{\pgfqpoint{1.282062in}{1.598043in}}%
\pgfpathlineto{\pgfqpoint{1.282927in}{1.678897in}}%
\pgfpathlineto{\pgfqpoint{1.283793in}{1.586587in}}%
\pgfpathlineto{\pgfqpoint{1.284658in}{1.650757in}}%
\pgfpathlineto{\pgfqpoint{1.285521in}{1.558684in}}%
\pgfpathlineto{\pgfqpoint{1.287250in}{1.614072in}}%
\pgfpathlineto{\pgfqpoint{1.288114in}{1.622383in}}%
\pgfpathlineto{\pgfqpoint{1.290711in}{1.713686in}}%
\pgfpathlineto{\pgfqpoint{1.291576in}{1.660793in}}%
\pgfpathlineto{\pgfqpoint{1.292440in}{1.752985in}}%
\pgfpathlineto{\pgfqpoint{1.293306in}{1.742923in}}%
\pgfpathlineto{\pgfqpoint{1.294172in}{1.621316in}}%
\pgfpathlineto{\pgfqpoint{1.295904in}{1.697835in}}%
\pgfpathlineto{\pgfqpoint{1.296768in}{1.682995in}}%
\pgfpathlineto{\pgfqpoint{1.297633in}{1.572104in}}%
\pgfpathlineto{\pgfqpoint{1.298498in}{1.591573in}}%
\pgfpathlineto{\pgfqpoint{1.299363in}{1.642863in}}%
\pgfpathlineto{\pgfqpoint{1.301092in}{1.619118in}}%
\pgfpathlineto{\pgfqpoint{1.301958in}{1.625885in}}%
\pgfpathlineto{\pgfqpoint{1.305417in}{1.706472in}}%
\pgfpathlineto{\pgfqpoint{1.306282in}{1.703475in}}%
\pgfpathlineto{\pgfqpoint{1.307147in}{1.629983in}}%
\pgfpathlineto{\pgfqpoint{1.308879in}{1.746098in}}%
\pgfpathlineto{\pgfqpoint{1.309745in}{1.713597in}}%
\pgfpathlineto{\pgfqpoint{1.310610in}{1.739034in}}%
\pgfpathlineto{\pgfqpoint{1.312342in}{1.708343in}}%
\pgfpathlineto{\pgfqpoint{1.313208in}{1.716296in}}%
\pgfpathlineto{\pgfqpoint{1.314074in}{1.662039in}}%
\pgfpathlineto{\pgfqpoint{1.314939in}{1.666698in}}%
\pgfpathlineto{\pgfqpoint{1.315803in}{1.731433in}}%
\pgfpathlineto{\pgfqpoint{1.317529in}{1.640312in}}%
\pgfpathlineto{\pgfqpoint{1.318395in}{1.672368in}}%
\pgfpathlineto{\pgfqpoint{1.320124in}{1.621966in}}%
\pgfpathlineto{\pgfqpoint{1.320989in}{1.730188in}}%
\pgfpathlineto{\pgfqpoint{1.321854in}{1.709975in}}%
\pgfpathlineto{\pgfqpoint{1.322720in}{1.596678in}}%
\pgfpathlineto{\pgfqpoint{1.325313in}{1.718434in}}%
\pgfpathlineto{\pgfqpoint{1.326179in}{1.629329in}}%
\pgfpathlineto{\pgfqpoint{1.327908in}{1.685843in}}%
\pgfpathlineto{\pgfqpoint{1.328771in}{1.683556in}}%
\pgfpathlineto{\pgfqpoint{1.329636in}{1.685843in}}%
\pgfpathlineto{\pgfqpoint{1.330500in}{1.728466in}}%
\pgfpathlineto{\pgfqpoint{1.331365in}{1.699051in}}%
\pgfpathlineto{\pgfqpoint{1.332230in}{1.726507in}}%
\pgfpathlineto{\pgfqpoint{1.333954in}{1.694659in}}%
\pgfpathlineto{\pgfqpoint{1.335684in}{1.699199in}}%
\pgfpathlineto{\pgfqpoint{1.336550in}{1.646187in}}%
\pgfpathlineto{\pgfqpoint{1.337413in}{1.728347in}}%
\pgfpathlineto{\pgfqpoint{1.338278in}{1.660049in}}%
\pgfpathlineto{\pgfqpoint{1.340007in}{1.704781in}}%
\pgfpathlineto{\pgfqpoint{1.342599in}{1.546812in}}%
\pgfpathlineto{\pgfqpoint{1.343462in}{1.584538in}}%
\pgfpathlineto{\pgfqpoint{1.344329in}{1.555658in}}%
\pgfpathlineto{\pgfqpoint{1.345195in}{1.573171in}}%
\pgfpathlineto{\pgfqpoint{1.346061in}{1.543904in}}%
\pgfpathlineto{\pgfqpoint{1.346925in}{1.567828in}}%
\pgfpathlineto{\pgfqpoint{1.347786in}{1.548920in}}%
\pgfpathlineto{\pgfqpoint{1.348652in}{1.630872in}}%
\pgfpathlineto{\pgfqpoint{1.349515in}{1.562544in}}%
\pgfpathlineto{\pgfqpoint{1.351245in}{1.600121in}}%
\pgfpathlineto{\pgfqpoint{1.352107in}{1.545150in}}%
\pgfpathlineto{\pgfqpoint{1.352972in}{1.548474in}}%
\pgfpathlineto{\pgfqpoint{1.353837in}{1.554352in}}%
\pgfpathlineto{\pgfqpoint{1.354702in}{1.620483in}}%
\pgfpathlineto{\pgfqpoint{1.355566in}{1.578098in}}%
\pgfpathlineto{\pgfqpoint{1.356431in}{1.624878in}}%
\pgfpathlineto{\pgfqpoint{1.359028in}{1.559398in}}%
\pgfpathlineto{\pgfqpoint{1.359893in}{1.610778in}}%
\pgfpathlineto{\pgfqpoint{1.360758in}{1.581779in}}%
\pgfpathlineto{\pgfqpoint{1.361622in}{1.627488in}}%
\pgfpathlineto{\pgfqpoint{1.362488in}{1.572308in}}%
\pgfpathlineto{\pgfqpoint{1.365081in}{1.640904in}}%
\pgfpathlineto{\pgfqpoint{1.365947in}{1.616623in}}%
\pgfpathlineto{\pgfqpoint{1.366813in}{1.644049in}}%
\pgfpathlineto{\pgfqpoint{1.367678in}{1.567292in}}%
\pgfpathlineto{\pgfqpoint{1.368543in}{1.571211in}}%
\pgfpathlineto{\pgfqpoint{1.369408in}{1.577384in}}%
\pgfpathlineto{\pgfqpoint{1.370273in}{1.562603in}}%
\pgfpathlineto{\pgfqpoint{1.371138in}{1.595075in}}%
\pgfpathlineto{\pgfqpoint{1.372002in}{1.586914in}}%
\pgfpathlineto{\pgfqpoint{1.372866in}{1.555836in}}%
\pgfpathlineto{\pgfqpoint{1.374594in}{1.644287in}}%
\pgfpathlineto{\pgfqpoint{1.375459in}{1.558744in}}%
\pgfpathlineto{\pgfqpoint{1.377187in}{1.615794in}}%
\pgfpathlineto{\pgfqpoint{1.378052in}{1.586468in}}%
\pgfpathlineto{\pgfqpoint{1.378916in}{1.511075in}}%
\pgfpathlineto{\pgfqpoint{1.380646in}{1.578395in}}%
\pgfpathlineto{\pgfqpoint{1.381509in}{1.562782in}}%
\pgfpathlineto{\pgfqpoint{1.383238in}{1.580946in}}%
\pgfpathlineto{\pgfqpoint{1.384104in}{1.522650in}}%
\pgfpathlineto{\pgfqpoint{1.384968in}{1.536601in}}%
\pgfpathlineto{\pgfqpoint{1.387562in}{1.629031in}}%
\pgfpathlineto{\pgfqpoint{1.388427in}{1.628912in}}%
\pgfpathlineto{\pgfqpoint{1.390156in}{1.572992in}}%
\pgfpathlineto{\pgfqpoint{1.391021in}{1.565184in}}%
\pgfpathlineto{\pgfqpoint{1.391886in}{1.519564in}}%
\pgfpathlineto{\pgfqpoint{1.392749in}{1.591751in}}%
\pgfpathlineto{\pgfqpoint{1.393613in}{1.589732in}}%
\pgfpathlineto{\pgfqpoint{1.395342in}{1.589821in}}%
\pgfpathlineto{\pgfqpoint{1.396208in}{1.603743in}}%
\pgfpathlineto{\pgfqpoint{1.397073in}{1.701456in}}%
\pgfpathlineto{\pgfqpoint{1.397938in}{1.679373in}}%
\pgfpathlineto{\pgfqpoint{1.398803in}{1.709588in}}%
\pgfpathlineto{\pgfqpoint{1.400534in}{1.636036in}}%
\pgfpathlineto{\pgfqpoint{1.401400in}{1.703118in}}%
\pgfpathlineto{\pgfqpoint{1.402265in}{1.662039in}}%
\pgfpathlineto{\pgfqpoint{1.403131in}{1.725499in}}%
\pgfpathlineto{\pgfqpoint{1.403997in}{1.699021in}}%
\pgfpathlineto{\pgfqpoint{1.405729in}{1.722885in}}%
\pgfpathlineto{\pgfqpoint{1.407460in}{1.677741in}}%
\pgfpathlineto{\pgfqpoint{1.408324in}{1.732564in}}%
\pgfpathlineto{\pgfqpoint{1.410055in}{1.677295in}}%
\pgfpathlineto{\pgfqpoint{1.410921in}{1.725677in}}%
\pgfpathlineto{\pgfqpoint{1.411785in}{1.691067in}}%
\pgfpathlineto{\pgfqpoint{1.413515in}{1.736007in}}%
\pgfpathlineto{\pgfqpoint{1.414379in}{1.735144in}}%
\pgfpathlineto{\pgfqpoint{1.416108in}{1.693205in}}%
\pgfpathlineto{\pgfqpoint{1.416972in}{1.708819in}}%
\pgfpathlineto{\pgfqpoint{1.417838in}{1.653788in}}%
\pgfpathlineto{\pgfqpoint{1.418701in}{1.725856in}}%
\pgfpathlineto{\pgfqpoint{1.419566in}{1.643339in}}%
\pgfpathlineto{\pgfqpoint{1.420430in}{1.684062in}}%
\pgfpathlineto{\pgfqpoint{1.421294in}{1.639123in}}%
\pgfpathlineto{\pgfqpoint{1.422159in}{1.661384in}}%
\pgfpathlineto{\pgfqpoint{1.423024in}{1.657290in}}%
\pgfpathlineto{\pgfqpoint{1.424749in}{1.677652in}}%
\pgfpathlineto{\pgfqpoint{1.425614in}{1.627607in}}%
\pgfpathlineto{\pgfqpoint{1.427345in}{1.761295in}}%
\pgfpathlineto{\pgfqpoint{1.429937in}{1.596619in}}%
\pgfpathlineto{\pgfqpoint{1.431667in}{1.680738in}}%
\pgfpathlineto{\pgfqpoint{1.432532in}{1.670408in}}%
\pgfpathlineto{\pgfqpoint{1.433398in}{1.694659in}}%
\pgfpathlineto{\pgfqpoint{1.434265in}{1.644763in}}%
\pgfpathlineto{\pgfqpoint{1.435130in}{1.656695in}}%
\pgfpathlineto{\pgfqpoint{1.435996in}{1.673584in}}%
\pgfpathlineto{\pgfqpoint{1.436861in}{1.668746in}}%
\pgfpathlineto{\pgfqpoint{1.438592in}{1.680678in}}%
\pgfpathlineto{\pgfqpoint{1.439455in}{1.660079in}}%
\pgfpathlineto{\pgfqpoint{1.441185in}{1.691008in}}%
\pgfpathlineto{\pgfqpoint{1.442050in}{1.681035in}}%
\pgfpathlineto{\pgfqpoint{1.442915in}{1.610718in}}%
\pgfpathlineto{\pgfqpoint{1.443778in}{1.717308in}}%
\pgfpathlineto{\pgfqpoint{1.444643in}{1.674803in}}%
\pgfpathlineto{\pgfqpoint{1.445508in}{1.698578in}}%
\pgfpathlineto{\pgfqpoint{1.446371in}{1.696768in}}%
\pgfpathlineto{\pgfqpoint{1.448967in}{1.628499in}}%
\pgfpathlineto{\pgfqpoint{1.449831in}{1.705911in}}%
\pgfpathlineto{\pgfqpoint{1.450696in}{1.642629in}}%
\pgfpathlineto{\pgfqpoint{1.451559in}{1.697065in}}%
\pgfpathlineto{\pgfqpoint{1.452425in}{1.659696in}}%
\pgfpathlineto{\pgfqpoint{1.453289in}{1.662990in}}%
\pgfpathlineto{\pgfqpoint{1.455884in}{1.717724in}}%
\pgfpathlineto{\pgfqpoint{1.456749in}{1.695105in}}%
\pgfpathlineto{\pgfqpoint{1.457614in}{1.635237in}}%
\pgfpathlineto{\pgfqpoint{1.458480in}{1.651653in}}%
\pgfpathlineto{\pgfqpoint{1.459345in}{1.688933in}}%
\pgfpathlineto{\pgfqpoint{1.461941in}{1.621405in}}%
\pgfpathlineto{\pgfqpoint{1.462806in}{1.621435in}}%
\pgfpathlineto{\pgfqpoint{1.464538in}{1.688665in}}%
\pgfpathlineto{\pgfqpoint{1.467134in}{1.621583in}}%
\pgfpathlineto{\pgfqpoint{1.467999in}{1.679611in}}%
\pgfpathlineto{\pgfqpoint{1.468864in}{1.666017in}}%
\pgfpathlineto{\pgfqpoint{1.469729in}{1.655866in}}%
\pgfpathlineto{\pgfqpoint{1.470594in}{1.676942in}}%
\pgfpathlineto{\pgfqpoint{1.472326in}{1.637940in}}%
\pgfpathlineto{\pgfqpoint{1.474056in}{1.664801in}}%
\pgfpathlineto{\pgfqpoint{1.474922in}{1.633843in}}%
\pgfpathlineto{\pgfqpoint{1.475788in}{1.705851in}}%
\pgfpathlineto{\pgfqpoint{1.476654in}{1.692346in}}%
\pgfpathlineto{\pgfqpoint{1.477520in}{1.665009in}}%
\pgfpathlineto{\pgfqpoint{1.478386in}{1.711905in}}%
\pgfpathlineto{\pgfqpoint{1.479253in}{1.685073in}}%
\pgfpathlineto{\pgfqpoint{1.480117in}{1.743310in}}%
\pgfpathlineto{\pgfqpoint{1.481847in}{1.638769in}}%
\pgfpathlineto{\pgfqpoint{1.482711in}{1.696887in}}%
\pgfpathlineto{\pgfqpoint{1.483575in}{1.629983in}}%
\pgfpathlineto{\pgfqpoint{1.484441in}{1.638948in}}%
\pgfpathlineto{\pgfqpoint{1.485307in}{1.652780in}}%
\pgfpathlineto{\pgfqpoint{1.487905in}{1.591930in}}%
\pgfpathlineto{\pgfqpoint{1.488770in}{1.530131in}}%
\pgfpathlineto{\pgfqpoint{1.489635in}{1.625885in}}%
\pgfpathlineto{\pgfqpoint{1.490500in}{1.584835in}}%
\pgfpathlineto{\pgfqpoint{1.491365in}{1.585400in}}%
\pgfpathlineto{\pgfqpoint{1.492231in}{1.588725in}}%
\pgfpathlineto{\pgfqpoint{1.493097in}{1.558655in}}%
\pgfpathlineto{\pgfqpoint{1.493962in}{1.569668in}}%
\pgfpathlineto{\pgfqpoint{1.495693in}{1.537285in}}%
\pgfpathlineto{\pgfqpoint{1.497424in}{1.626064in}}%
\pgfpathlineto{\pgfqpoint{1.499153in}{1.588963in}}%
\pgfpathlineto{\pgfqpoint{1.500017in}{1.640818in}}%
\pgfpathlineto{\pgfqpoint{1.500882in}{1.518675in}}%
\pgfpathlineto{\pgfqpoint{1.502609in}{1.582995in}}%
\pgfpathlineto{\pgfqpoint{1.503473in}{1.576495in}}%
\pgfpathlineto{\pgfqpoint{1.504339in}{1.578514in}}%
\pgfpathlineto{\pgfqpoint{1.505204in}{1.566552in}}%
\pgfpathlineto{\pgfqpoint{1.506068in}{1.599292in}}%
\pgfpathlineto{\pgfqpoint{1.506933in}{1.547645in}}%
\pgfpathlineto{\pgfqpoint{1.507798in}{1.560350in}}%
\pgfpathlineto{\pgfqpoint{1.508662in}{1.660793in}}%
\pgfpathlineto{\pgfqpoint{1.509524in}{1.541026in}}%
\pgfpathlineto{\pgfqpoint{1.511254in}{1.632240in}}%
\pgfpathlineto{\pgfqpoint{1.512119in}{1.625592in}}%
\pgfpathlineto{\pgfqpoint{1.512984in}{1.610335in}}%
\pgfpathlineto{\pgfqpoint{1.513848in}{1.675279in}}%
\pgfpathlineto{\pgfqpoint{1.514713in}{1.557145in}}%
\pgfpathlineto{\pgfqpoint{1.515579in}{1.565158in}}%
\pgfpathlineto{\pgfqpoint{1.516443in}{1.587360in}}%
\pgfpathlineto{\pgfqpoint{1.519904in}{1.533753in}}%
\pgfpathlineto{\pgfqpoint{1.520767in}{1.586051in}}%
\pgfpathlineto{\pgfqpoint{1.521632in}{1.545298in}}%
\pgfpathlineto{\pgfqpoint{1.522496in}{1.549009in}}%
\pgfpathlineto{\pgfqpoint{1.523360in}{1.543190in}}%
\pgfpathlineto{\pgfqpoint{1.525956in}{1.640312in}}%
\pgfpathlineto{\pgfqpoint{1.526821in}{1.559458in}}%
\pgfpathlineto{\pgfqpoint{1.527685in}{1.579879in}}%
\pgfpathlineto{\pgfqpoint{1.528549in}{1.567828in}}%
\pgfpathlineto{\pgfqpoint{1.529411in}{1.577562in}}%
\pgfpathlineto{\pgfqpoint{1.532005in}{1.691246in}}%
\pgfpathlineto{\pgfqpoint{1.532870in}{1.615318in}}%
\pgfpathlineto{\pgfqpoint{1.533734in}{1.664798in}}%
\pgfpathlineto{\pgfqpoint{1.534599in}{1.581481in}}%
\pgfpathlineto{\pgfqpoint{1.536326in}{1.641320in}}%
\pgfpathlineto{\pgfqpoint{1.537190in}{1.610689in}}%
\pgfpathlineto{\pgfqpoint{1.538055in}{1.509294in}}%
\pgfpathlineto{\pgfqpoint{1.539786in}{1.627726in}}%
\pgfpathlineto{\pgfqpoint{1.540652in}{1.522710in}}%
\pgfpathlineto{\pgfqpoint{1.542382in}{1.608670in}}%
\pgfpathlineto{\pgfqpoint{1.543246in}{1.514429in}}%
\pgfpathlineto{\pgfqpoint{1.544111in}{1.623688in}}%
\pgfpathlineto{\pgfqpoint{1.544975in}{1.603029in}}%
\pgfpathlineto{\pgfqpoint{1.545841in}{1.584389in}}%
\pgfpathlineto{\pgfqpoint{1.546705in}{1.612053in}}%
\pgfpathlineto{\pgfqpoint{1.548435in}{1.560525in}}%
\pgfpathlineto{\pgfqpoint{1.549300in}{1.577860in}}%
\pgfpathlineto{\pgfqpoint{1.550163in}{1.543190in}}%
\pgfpathlineto{\pgfqpoint{1.551028in}{1.586527in}}%
\pgfpathlineto{\pgfqpoint{1.552758in}{1.567976in}}%
\pgfpathlineto{\pgfqpoint{1.553624in}{1.557855in}}%
\pgfpathlineto{\pgfqpoint{1.554488in}{1.583500in}}%
\pgfpathlineto{\pgfqpoint{1.555353in}{1.576019in}}%
\pgfpathlineto{\pgfqpoint{1.556218in}{1.646247in}}%
\pgfpathlineto{\pgfqpoint{1.557083in}{1.555509in}}%
\pgfpathlineto{\pgfqpoint{1.557945in}{1.591751in}}%
\pgfpathlineto{\pgfqpoint{1.559676in}{1.532890in}}%
\pgfpathlineto{\pgfqpoint{1.560542in}{1.608253in}}%
\pgfpathlineto{\pgfqpoint{1.561408in}{1.554055in}}%
\pgfpathlineto{\pgfqpoint{1.562273in}{1.575900in}}%
\pgfpathlineto{\pgfqpoint{1.563139in}{1.567292in}}%
\pgfpathlineto{\pgfqpoint{1.564005in}{1.639123in}}%
\pgfpathlineto{\pgfqpoint{1.565738in}{1.545150in}}%
\pgfpathlineto{\pgfqpoint{1.566603in}{1.562544in}}%
\pgfpathlineto{\pgfqpoint{1.567467in}{1.541707in}}%
\pgfpathlineto{\pgfqpoint{1.568332in}{1.547641in}}%
\pgfpathlineto{\pgfqpoint{1.569196in}{1.612763in}}%
\pgfpathlineto{\pgfqpoint{1.570061in}{1.571981in}}%
\pgfpathlineto{\pgfqpoint{1.572656in}{1.644525in}}%
\pgfpathlineto{\pgfqpoint{1.574383in}{1.536601in}}%
\pgfpathlineto{\pgfqpoint{1.575249in}{1.632772in}}%
\pgfpathlineto{\pgfqpoint{1.576114in}{1.569728in}}%
\pgfpathlineto{\pgfqpoint{1.576978in}{1.598459in}}%
\pgfpathlineto{\pgfqpoint{1.578706in}{1.552304in}}%
\pgfpathlineto{\pgfqpoint{1.579570in}{1.677354in}}%
\pgfpathlineto{\pgfqpoint{1.580437in}{1.575781in}}%
\pgfpathlineto{\pgfqpoint{1.581303in}{1.658060in}}%
\pgfpathlineto{\pgfqpoint{1.582170in}{1.655271in}}%
\pgfpathlineto{\pgfqpoint{1.583903in}{1.571271in}}%
\pgfpathlineto{\pgfqpoint{1.585634in}{1.555955in}}%
\pgfpathlineto{\pgfqpoint{1.586500in}{1.592227in}}%
\pgfpathlineto{\pgfqpoint{1.587366in}{1.545507in}}%
\pgfpathlineto{\pgfqpoint{1.589098in}{1.632772in}}%
\pgfpathlineto{\pgfqpoint{1.590828in}{1.576079in}}%
\pgfpathlineto{\pgfqpoint{1.591692in}{1.610927in}}%
\pgfpathlineto{\pgfqpoint{1.593423in}{1.541528in}}%
\pgfpathlineto{\pgfqpoint{1.594289in}{1.543309in}}%
\pgfpathlineto{\pgfqpoint{1.595155in}{1.572397in}}%
\pgfpathlineto{\pgfqpoint{1.596886in}{1.526867in}}%
\pgfpathlineto{\pgfqpoint{1.598615in}{1.637639in}}%
\pgfpathlineto{\pgfqpoint{1.599480in}{1.527160in}}%
\pgfpathlineto{\pgfqpoint{1.601207in}{1.591275in}}%
\pgfpathlineto{\pgfqpoint{1.602071in}{1.587892in}}%
\pgfpathlineto{\pgfqpoint{1.603803in}{1.525677in}}%
\pgfpathlineto{\pgfqpoint{1.604669in}{1.598072in}}%
\pgfpathlineto{\pgfqpoint{1.605535in}{1.575424in}}%
\pgfpathlineto{\pgfqpoint{1.608130in}{1.627369in}}%
\pgfpathlineto{\pgfqpoint{1.609860in}{1.517396in}}%
\pgfpathlineto{\pgfqpoint{1.610725in}{1.581243in}}%
\pgfpathlineto{\pgfqpoint{1.611590in}{1.564266in}}%
\pgfpathlineto{\pgfqpoint{1.614186in}{1.622944in}}%
\pgfpathlineto{\pgfqpoint{1.615051in}{1.556487in}}%
\pgfpathlineto{\pgfqpoint{1.615917in}{1.644049in}}%
\pgfpathlineto{\pgfqpoint{1.616782in}{1.597596in}}%
\pgfpathlineto{\pgfqpoint{1.617645in}{1.662987in}}%
\pgfpathlineto{\pgfqpoint{1.618510in}{1.611637in}}%
\pgfpathlineto{\pgfqpoint{1.619375in}{1.628704in}}%
\pgfpathlineto{\pgfqpoint{1.620240in}{1.622026in}}%
\pgfpathlineto{\pgfqpoint{1.621104in}{1.632415in}}%
\pgfpathlineto{\pgfqpoint{1.621969in}{1.581184in}}%
\pgfpathlineto{\pgfqpoint{1.622835in}{1.638587in}}%
\pgfpathlineto{\pgfqpoint{1.623700in}{1.544376in}}%
\pgfpathlineto{\pgfqpoint{1.625428in}{1.634196in}}%
\pgfpathlineto{\pgfqpoint{1.626293in}{1.584924in}}%
\pgfpathlineto{\pgfqpoint{1.628886in}{1.657971in}}%
\pgfpathlineto{\pgfqpoint{1.629752in}{1.593711in}}%
\pgfpathlineto{\pgfqpoint{1.630617in}{1.641320in}}%
\pgfpathlineto{\pgfqpoint{1.631481in}{1.631110in}}%
\pgfpathlineto{\pgfqpoint{1.632344in}{1.589970in}}%
\pgfpathlineto{\pgfqpoint{1.633210in}{1.608759in}}%
\pgfpathlineto{\pgfqpoint{1.634073in}{1.607126in}}%
\pgfpathlineto{\pgfqpoint{1.634936in}{1.615556in}}%
\pgfpathlineto{\pgfqpoint{1.635803in}{1.614281in}}%
\pgfpathlineto{\pgfqpoint{1.636668in}{1.590327in}}%
\pgfpathlineto{\pgfqpoint{1.639263in}{1.668627in}}%
\pgfpathlineto{\pgfqpoint{1.642725in}{1.562931in}}%
\pgfpathlineto{\pgfqpoint{1.644456in}{1.626123in}}%
\pgfpathlineto{\pgfqpoint{1.645321in}{1.586319in}}%
\pgfpathlineto{\pgfqpoint{1.647053in}{1.657822in}}%
\pgfpathlineto{\pgfqpoint{1.647918in}{1.571330in}}%
\pgfpathlineto{\pgfqpoint{1.648784in}{1.588487in}}%
\pgfpathlineto{\pgfqpoint{1.650509in}{1.609205in}}%
\pgfpathlineto{\pgfqpoint{1.651372in}{1.570973in}}%
\pgfpathlineto{\pgfqpoint{1.652236in}{1.579641in}}%
\pgfpathlineto{\pgfqpoint{1.653101in}{1.655981in}}%
\pgfpathlineto{\pgfqpoint{1.653965in}{1.608551in}}%
\pgfpathlineto{\pgfqpoint{1.654830in}{1.617456in}}%
\pgfpathlineto{\pgfqpoint{1.655694in}{1.664827in}}%
\pgfpathlineto{\pgfqpoint{1.657422in}{1.577830in}}%
\pgfpathlineto{\pgfqpoint{1.658286in}{1.591751in}}%
\pgfpathlineto{\pgfqpoint{1.659151in}{1.596975in}}%
\pgfpathlineto{\pgfqpoint{1.660879in}{1.554527in}}%
\pgfpathlineto{\pgfqpoint{1.661745in}{1.606978in}}%
\pgfpathlineto{\pgfqpoint{1.662610in}{1.560346in}}%
\pgfpathlineto{\pgfqpoint{1.663476in}{1.575781in}}%
\pgfpathlineto{\pgfqpoint{1.664342in}{1.562276in}}%
\pgfpathlineto{\pgfqpoint{1.666070in}{1.618404in}}%
\pgfpathlineto{\pgfqpoint{1.667800in}{1.530961in}}%
\pgfpathlineto{\pgfqpoint{1.668666in}{1.579224in}}%
\pgfpathlineto{\pgfqpoint{1.670393in}{1.526331in}}%
\pgfpathlineto{\pgfqpoint{1.671259in}{1.538144in}}%
\pgfpathlineto{\pgfqpoint{1.672989in}{1.620959in}}%
\pgfpathlineto{\pgfqpoint{1.674716in}{1.602200in}}%
\pgfpathlineto{\pgfqpoint{1.675581in}{1.570438in}}%
\pgfpathlineto{\pgfqpoint{1.676445in}{1.640487in}}%
\pgfpathlineto{\pgfqpoint{1.678175in}{1.584627in}}%
\pgfpathlineto{\pgfqpoint{1.679039in}{1.623866in}}%
\pgfpathlineto{\pgfqpoint{1.679905in}{1.553877in}}%
\pgfpathlineto{\pgfqpoint{1.680770in}{1.560436in}}%
\pgfpathlineto{\pgfqpoint{1.681635in}{1.651174in}}%
\pgfpathlineto{\pgfqpoint{1.683364in}{1.593562in}}%
\pgfpathlineto{\pgfqpoint{1.684229in}{1.696113in}}%
\pgfpathlineto{\pgfqpoint{1.685094in}{1.560644in}}%
\pgfpathlineto{\pgfqpoint{1.685959in}{1.649839in}}%
\pgfpathlineto{\pgfqpoint{1.686825in}{1.543666in}}%
\pgfpathlineto{\pgfqpoint{1.688554in}{1.591037in}}%
\pgfpathlineto{\pgfqpoint{1.689420in}{1.575662in}}%
\pgfpathlineto{\pgfqpoint{1.690285in}{1.516121in}}%
\pgfpathlineto{\pgfqpoint{1.692015in}{1.600300in}}%
\pgfpathlineto{\pgfqpoint{1.692880in}{1.493235in}}%
\pgfpathlineto{\pgfqpoint{1.693742in}{1.603858in}}%
\pgfpathlineto{\pgfqpoint{1.695472in}{1.550047in}}%
\pgfpathlineto{\pgfqpoint{1.696337in}{1.633660in}}%
\pgfpathlineto{\pgfqpoint{1.698064in}{1.570378in}}%
\pgfpathlineto{\pgfqpoint{1.698929in}{1.568066in}}%
\pgfpathlineto{\pgfqpoint{1.700660in}{1.622978in}}%
\pgfpathlineto{\pgfqpoint{1.702389in}{1.528380in}}%
\pgfpathlineto{\pgfqpoint{1.703252in}{1.574476in}}%
\pgfpathlineto{\pgfqpoint{1.704981in}{1.498548in}}%
\pgfpathlineto{\pgfqpoint{1.706709in}{1.615794in}}%
\pgfpathlineto{\pgfqpoint{1.707574in}{1.566995in}}%
\pgfpathlineto{\pgfqpoint{1.708439in}{1.622204in}}%
\pgfpathlineto{\pgfqpoint{1.709304in}{1.599972in}}%
\pgfpathlineto{\pgfqpoint{1.710169in}{1.613299in}}%
\pgfpathlineto{\pgfqpoint{1.711035in}{1.649452in}}%
\pgfpathlineto{\pgfqpoint{1.711900in}{1.632474in}}%
\pgfpathlineto{\pgfqpoint{1.712763in}{1.737372in}}%
\pgfpathlineto{\pgfqpoint{1.713627in}{1.640074in}}%
\pgfpathlineto{\pgfqpoint{1.714491in}{1.785933in}}%
\pgfpathlineto{\pgfqpoint{1.715355in}{1.712202in}}%
\pgfpathlineto{\pgfqpoint{1.716220in}{1.715467in}}%
\pgfpathlineto{\pgfqpoint{1.717949in}{1.652007in}}%
\pgfpathlineto{\pgfqpoint{1.718815in}{1.681154in}}%
\pgfpathlineto{\pgfqpoint{1.720545in}{1.656695in}}%
\pgfpathlineto{\pgfqpoint{1.723138in}{1.679314in}}%
\pgfpathlineto{\pgfqpoint{1.724003in}{1.672368in}}%
\pgfpathlineto{\pgfqpoint{1.725732in}{1.752568in}}%
\pgfpathlineto{\pgfqpoint{1.726596in}{1.652661in}}%
\pgfpathlineto{\pgfqpoint{1.727461in}{1.718613in}}%
\pgfpathlineto{\pgfqpoint{1.729191in}{1.622442in}}%
\pgfpathlineto{\pgfqpoint{1.730056in}{1.687327in}}%
\pgfpathlineto{\pgfqpoint{1.730922in}{1.654825in}}%
\pgfpathlineto{\pgfqpoint{1.732650in}{1.704543in}}%
\pgfpathlineto{\pgfqpoint{1.733516in}{1.641528in}}%
\pgfpathlineto{\pgfqpoint{1.735246in}{1.727875in}}%
\pgfpathlineto{\pgfqpoint{1.736112in}{1.644232in}}%
\pgfpathlineto{\pgfqpoint{1.736974in}{1.698668in}}%
\pgfpathlineto{\pgfqpoint{1.738706in}{1.611284in}}%
\pgfpathlineto{\pgfqpoint{1.739570in}{1.625116in}}%
\pgfpathlineto{\pgfqpoint{1.740436in}{1.637077in}}%
\pgfpathlineto{\pgfqpoint{1.741299in}{1.620248in}}%
\pgfpathlineto{\pgfqpoint{1.742164in}{1.641737in}}%
\pgfpathlineto{\pgfqpoint{1.743891in}{1.528410in}}%
\pgfpathlineto{\pgfqpoint{1.744756in}{1.564444in}}%
\pgfpathlineto{\pgfqpoint{1.745620in}{1.557855in}}%
\pgfpathlineto{\pgfqpoint{1.746486in}{1.574357in}}%
\pgfpathlineto{\pgfqpoint{1.747352in}{1.542212in}}%
\pgfpathlineto{\pgfqpoint{1.748217in}{1.554709in}}%
\pgfpathlineto{\pgfqpoint{1.749082in}{1.634791in}}%
\pgfpathlineto{\pgfqpoint{1.750814in}{1.572992in}}%
\pgfpathlineto{\pgfqpoint{1.751679in}{1.578217in}}%
\pgfpathlineto{\pgfqpoint{1.752545in}{1.605702in}}%
\pgfpathlineto{\pgfqpoint{1.754272in}{1.571092in}}%
\pgfpathlineto{\pgfqpoint{1.756004in}{1.635977in}}%
\pgfpathlineto{\pgfqpoint{1.756870in}{1.627220in}}%
\pgfpathlineto{\pgfqpoint{1.757735in}{1.580946in}}%
\pgfpathlineto{\pgfqpoint{1.758602in}{1.588011in}}%
\pgfpathlineto{\pgfqpoint{1.759468in}{1.598400in}}%
\pgfpathlineto{\pgfqpoint{1.760334in}{1.568954in}}%
\pgfpathlineto{\pgfqpoint{1.761200in}{1.603445in}}%
\pgfpathlineto{\pgfqpoint{1.762066in}{1.594778in}}%
\pgfpathlineto{\pgfqpoint{1.762931in}{1.574535in}}%
\pgfpathlineto{\pgfqpoint{1.763797in}{1.609146in}}%
\pgfpathlineto{\pgfqpoint{1.764663in}{1.528112in}}%
\pgfpathlineto{\pgfqpoint{1.765529in}{1.576019in}}%
\pgfpathlineto{\pgfqpoint{1.766394in}{1.553014in}}%
\pgfpathlineto{\pgfqpoint{1.767259in}{1.611815in}}%
\pgfpathlineto{\pgfqpoint{1.768988in}{1.540372in}}%
\pgfpathlineto{\pgfqpoint{1.769853in}{1.621197in}}%
\pgfpathlineto{\pgfqpoint{1.770717in}{1.571687in}}%
\pgfpathlineto{\pgfqpoint{1.772446in}{1.595313in}}%
\pgfpathlineto{\pgfqpoint{1.773309in}{1.548890in}}%
\pgfpathlineto{\pgfqpoint{1.774173in}{1.585932in}}%
\pgfpathlineto{\pgfqpoint{1.775903in}{1.560882in}}%
\pgfpathlineto{\pgfqpoint{1.776768in}{1.632772in}}%
\pgfpathlineto{\pgfqpoint{1.778496in}{1.539687in}}%
\pgfpathlineto{\pgfqpoint{1.779358in}{1.625647in}}%
\pgfpathlineto{\pgfqpoint{1.780222in}{1.554115in}}%
\pgfpathlineto{\pgfqpoint{1.782816in}{1.665125in}}%
\pgfpathlineto{\pgfqpoint{1.783681in}{1.557498in}}%
\pgfpathlineto{\pgfqpoint{1.784545in}{1.584389in}}%
\pgfpathlineto{\pgfqpoint{1.785410in}{1.542063in}}%
\pgfpathlineto{\pgfqpoint{1.786275in}{1.610510in}}%
\pgfpathlineto{\pgfqpoint{1.787138in}{1.589970in}}%
\pgfpathlineto{\pgfqpoint{1.788003in}{1.620483in}}%
\pgfpathlineto{\pgfqpoint{1.788868in}{1.504546in}}%
\pgfpathlineto{\pgfqpoint{1.789732in}{1.631764in}}%
\pgfpathlineto{\pgfqpoint{1.791463in}{1.536423in}}%
\pgfpathlineto{\pgfqpoint{1.792326in}{1.598519in}}%
\pgfpathlineto{\pgfqpoint{1.793189in}{1.526510in}}%
\pgfpathlineto{\pgfqpoint{1.794054in}{1.576852in}}%
\pgfpathlineto{\pgfqpoint{1.794918in}{1.569103in}}%
\pgfpathlineto{\pgfqpoint{1.796649in}{1.601843in}}%
\pgfpathlineto{\pgfqpoint{1.798377in}{1.549188in}}%
\pgfpathlineto{\pgfqpoint{1.799241in}{1.594778in}}%
\pgfpathlineto{\pgfqpoint{1.800107in}{1.534344in}}%
\pgfpathlineto{\pgfqpoint{1.801837in}{1.622264in}}%
\pgfpathlineto{\pgfqpoint{1.802704in}{1.582489in}}%
\pgfpathlineto{\pgfqpoint{1.803570in}{1.644406in}}%
\pgfpathlineto{\pgfqpoint{1.805299in}{1.564087in}}%
\pgfpathlineto{\pgfqpoint{1.806165in}{1.631169in}}%
\pgfpathlineto{\pgfqpoint{1.807031in}{1.606561in}}%
\pgfpathlineto{\pgfqpoint{1.807897in}{1.548474in}}%
\pgfpathlineto{\pgfqpoint{1.808763in}{1.603029in}}%
\pgfpathlineto{\pgfqpoint{1.810493in}{1.540104in}}%
\pgfpathlineto{\pgfqpoint{1.811358in}{1.552690in}}%
\pgfpathlineto{\pgfqpoint{1.812222in}{1.540580in}}%
\pgfpathlineto{\pgfqpoint{1.813089in}{1.589732in}}%
\pgfpathlineto{\pgfqpoint{1.814820in}{1.528291in}}%
\pgfpathlineto{\pgfqpoint{1.815683in}{1.566582in}}%
\pgfpathlineto{\pgfqpoint{1.816549in}{1.540967in}}%
\pgfpathlineto{\pgfqpoint{1.817413in}{1.603981in}}%
\pgfpathlineto{\pgfqpoint{1.819140in}{1.532121in}}%
\pgfpathlineto{\pgfqpoint{1.820005in}{1.625588in}}%
\pgfpathlineto{\pgfqpoint{1.821736in}{1.522948in}}%
\pgfpathlineto{\pgfqpoint{1.822603in}{1.616742in}}%
\pgfpathlineto{\pgfqpoint{1.823469in}{1.560436in}}%
\pgfpathlineto{\pgfqpoint{1.826063in}{1.659752in}}%
\pgfpathlineto{\pgfqpoint{1.826928in}{1.656695in}}%
\pgfpathlineto{\pgfqpoint{1.827794in}{1.687267in}}%
\pgfpathlineto{\pgfqpoint{1.828657in}{1.620364in}}%
\pgfpathlineto{\pgfqpoint{1.829522in}{1.620780in}}%
\pgfpathlineto{\pgfqpoint{1.830388in}{1.670289in}}%
\pgfpathlineto{\pgfqpoint{1.831253in}{1.572754in}}%
\pgfpathlineto{\pgfqpoint{1.832114in}{1.665660in}}%
\pgfpathlineto{\pgfqpoint{1.832979in}{1.639480in}}%
\pgfpathlineto{\pgfqpoint{1.833845in}{1.628202in}}%
\pgfpathlineto{\pgfqpoint{1.834712in}{1.577030in}}%
\pgfpathlineto{\pgfqpoint{1.835577in}{1.585490in}}%
\pgfpathlineto{\pgfqpoint{1.836442in}{1.698906in}}%
\pgfpathlineto{\pgfqpoint{1.838171in}{1.623662in}}%
\pgfpathlineto{\pgfqpoint{1.839899in}{1.683411in}}%
\pgfpathlineto{\pgfqpoint{1.842494in}{1.553166in}}%
\pgfpathlineto{\pgfqpoint{1.844224in}{1.654026in}}%
\pgfpathlineto{\pgfqpoint{1.845089in}{1.605435in}}%
\pgfpathlineto{\pgfqpoint{1.845955in}{1.660376in}}%
\pgfpathlineto{\pgfqpoint{1.846820in}{1.636691in}}%
\pgfpathlineto{\pgfqpoint{1.847684in}{1.568155in}}%
\pgfpathlineto{\pgfqpoint{1.848546in}{1.586825in}}%
\pgfpathlineto{\pgfqpoint{1.849412in}{1.614548in}}%
\pgfpathlineto{\pgfqpoint{1.850276in}{1.549069in}}%
\pgfpathlineto{\pgfqpoint{1.851141in}{1.651709in}}%
\pgfpathlineto{\pgfqpoint{1.852870in}{1.533218in}}%
\pgfpathlineto{\pgfqpoint{1.853735in}{1.584865in}}%
\pgfpathlineto{\pgfqpoint{1.854601in}{1.564117in}}%
\pgfpathlineto{\pgfqpoint{1.856331in}{1.622502in}}%
\pgfpathlineto{\pgfqpoint{1.857196in}{1.555479in}}%
\pgfpathlineto{\pgfqpoint{1.858925in}{1.584389in}}%
\pgfpathlineto{\pgfqpoint{1.859790in}{1.565928in}}%
\pgfpathlineto{\pgfqpoint{1.860656in}{1.588070in}}%
\pgfpathlineto{\pgfqpoint{1.862388in}{1.691365in}}%
\pgfpathlineto{\pgfqpoint{1.863254in}{1.697597in}}%
\pgfpathlineto{\pgfqpoint{1.864120in}{1.577711in}}%
\pgfpathlineto{\pgfqpoint{1.865845in}{1.625171in}}%
\pgfpathlineto{\pgfqpoint{1.867573in}{1.544674in}}%
\pgfpathlineto{\pgfqpoint{1.869304in}{1.630515in}}%
\pgfpathlineto{\pgfqpoint{1.870168in}{1.511904in}}%
\pgfpathlineto{\pgfqpoint{1.871897in}{1.582429in}}%
\pgfpathlineto{\pgfqpoint{1.872762in}{1.555241in}}%
\pgfpathlineto{\pgfqpoint{1.873626in}{1.615020in}}%
\pgfpathlineto{\pgfqpoint{1.874493in}{1.538620in}}%
\pgfpathlineto{\pgfqpoint{1.875359in}{1.553639in}}%
\pgfpathlineto{\pgfqpoint{1.877089in}{1.541647in}}%
\pgfpathlineto{\pgfqpoint{1.877953in}{1.566106in}}%
\pgfpathlineto{\pgfqpoint{1.878816in}{1.499202in}}%
\pgfpathlineto{\pgfqpoint{1.879681in}{1.584389in}}%
\pgfpathlineto{\pgfqpoint{1.880547in}{1.553490in}}%
\pgfpathlineto{\pgfqpoint{1.883143in}{1.639420in}}%
\pgfpathlineto{\pgfqpoint{1.884008in}{1.617158in}}%
\pgfpathlineto{\pgfqpoint{1.884871in}{1.640963in}}%
\pgfpathlineto{\pgfqpoint{1.885735in}{1.610183in}}%
\pgfpathlineto{\pgfqpoint{1.887463in}{1.651828in}}%
\pgfpathlineto{\pgfqpoint{1.890054in}{1.566493in}}%
\pgfpathlineto{\pgfqpoint{1.890918in}{1.605583in}}%
\pgfpathlineto{\pgfqpoint{1.891784in}{1.549128in}}%
\pgfpathlineto{\pgfqpoint{1.892650in}{1.555568in}}%
\pgfpathlineto{\pgfqpoint{1.893515in}{1.550433in}}%
\pgfpathlineto{\pgfqpoint{1.894381in}{1.604988in}}%
\pgfpathlineto{\pgfqpoint{1.895246in}{1.593146in}}%
\pgfpathlineto{\pgfqpoint{1.896112in}{1.545031in}}%
\pgfpathlineto{\pgfqpoint{1.897841in}{1.590089in}}%
\pgfpathlineto{\pgfqpoint{1.898707in}{1.553996in}}%
\pgfpathlineto{\pgfqpoint{1.900437in}{1.603743in}}%
\pgfpathlineto{\pgfqpoint{1.901303in}{1.571033in}}%
\pgfpathlineto{\pgfqpoint{1.902169in}{1.617099in}}%
\pgfpathlineto{\pgfqpoint{1.903034in}{1.605345in}}%
\pgfpathlineto{\pgfqpoint{1.904764in}{1.574298in}}%
\pgfpathlineto{\pgfqpoint{1.907360in}{1.594243in}}%
\pgfpathlineto{\pgfqpoint{1.908225in}{1.563909in}}%
\pgfpathlineto{\pgfqpoint{1.909953in}{1.634196in}}%
\pgfpathlineto{\pgfqpoint{1.910818in}{1.519266in}}%
\pgfpathlineto{\pgfqpoint{1.911684in}{1.600891in}}%
\pgfpathlineto{\pgfqpoint{1.912550in}{1.566757in}}%
\pgfpathlineto{\pgfqpoint{1.913415in}{1.606413in}}%
\pgfpathlineto{\pgfqpoint{1.915146in}{1.586349in}}%
\pgfpathlineto{\pgfqpoint{1.916877in}{1.616742in}}%
\pgfpathlineto{\pgfqpoint{1.917742in}{1.601426in}}%
\pgfpathlineto{\pgfqpoint{1.919471in}{1.551857in}}%
\pgfpathlineto{\pgfqpoint{1.921200in}{1.598221in}}%
\pgfpathlineto{\pgfqpoint{1.922931in}{1.539747in}}%
\pgfpathlineto{\pgfqpoint{1.923796in}{1.582013in}}%
\pgfpathlineto{\pgfqpoint{1.924661in}{1.549957in}}%
\pgfpathlineto{\pgfqpoint{1.925527in}{1.585575in}}%
\pgfpathlineto{\pgfqpoint{1.927259in}{1.517664in}}%
\pgfpathlineto{\pgfqpoint{1.928985in}{1.618464in}}%
\pgfpathlineto{\pgfqpoint{1.929850in}{1.610153in}}%
\pgfpathlineto{\pgfqpoint{1.930715in}{1.630366in}}%
\pgfpathlineto{\pgfqpoint{1.931580in}{1.613537in}}%
\pgfpathlineto{\pgfqpoint{1.932446in}{1.643280in}}%
\pgfpathlineto{\pgfqpoint{1.933312in}{1.537996in}}%
\pgfpathlineto{\pgfqpoint{1.935043in}{1.586230in}}%
\pgfpathlineto{\pgfqpoint{1.936775in}{1.622561in}}%
\pgfpathlineto{\pgfqpoint{1.937642in}{1.648147in}}%
\pgfpathlineto{\pgfqpoint{1.938507in}{1.611165in}}%
\pgfpathlineto{\pgfqpoint{1.939373in}{1.638174in}}%
\pgfpathlineto{\pgfqpoint{1.941967in}{1.535475in}}%
\pgfpathlineto{\pgfqpoint{1.944561in}{1.642510in}}%
\pgfpathlineto{\pgfqpoint{1.946290in}{1.580593in}}%
\pgfpathlineto{\pgfqpoint{1.947154in}{1.592465in}}%
\pgfpathlineto{\pgfqpoint{1.948018in}{1.509948in}}%
\pgfpathlineto{\pgfqpoint{1.949748in}{1.632151in}}%
\pgfpathlineto{\pgfqpoint{1.951475in}{1.623394in}}%
\pgfpathlineto{\pgfqpoint{1.952341in}{1.626748in}}%
\pgfpathlineto{\pgfqpoint{1.954072in}{1.695343in}}%
\pgfpathlineto{\pgfqpoint{1.954936in}{1.669698in}}%
\pgfpathlineto{\pgfqpoint{1.955801in}{1.729418in}}%
\pgfpathlineto{\pgfqpoint{1.956666in}{1.679909in}}%
\pgfpathlineto{\pgfqpoint{1.957531in}{1.687271in}}%
\pgfpathlineto{\pgfqpoint{1.959262in}{1.709443in}}%
\pgfpathlineto{\pgfqpoint{1.962723in}{1.617872in}}%
\pgfpathlineto{\pgfqpoint{1.963588in}{1.708402in}}%
\pgfpathlineto{\pgfqpoint{1.964454in}{1.675454in}}%
\pgfpathlineto{\pgfqpoint{1.966183in}{1.732088in}}%
\pgfpathlineto{\pgfqpoint{1.968778in}{1.655241in}}%
\pgfpathlineto{\pgfqpoint{1.969644in}{1.691900in}}%
\pgfpathlineto{\pgfqpoint{1.970509in}{1.670587in}}%
\pgfpathlineto{\pgfqpoint{1.972240in}{1.572576in}}%
\pgfpathlineto{\pgfqpoint{1.973106in}{1.555126in}}%
\pgfpathlineto{\pgfqpoint{1.974835in}{1.612291in}}%
\pgfpathlineto{\pgfqpoint{1.975702in}{1.556134in}}%
\pgfpathlineto{\pgfqpoint{1.977433in}{1.631407in}}%
\pgfpathlineto{\pgfqpoint{1.978298in}{1.564117in}}%
\pgfpathlineto{\pgfqpoint{1.980028in}{1.646544in}}%
\pgfpathlineto{\pgfqpoint{1.980894in}{1.621226in}}%
\pgfpathlineto{\pgfqpoint{1.981760in}{1.583500in}}%
\pgfpathlineto{\pgfqpoint{1.983491in}{1.675514in}}%
\pgfpathlineto{\pgfqpoint{1.986085in}{1.535594in}}%
\pgfpathlineto{\pgfqpoint{1.989540in}{1.629329in}}%
\pgfpathlineto{\pgfqpoint{1.990405in}{1.634701in}}%
\pgfpathlineto{\pgfqpoint{1.991267in}{1.618110in}}%
\pgfpathlineto{\pgfqpoint{1.992132in}{1.656755in}}%
\pgfpathlineto{\pgfqpoint{1.995590in}{1.534407in}}%
\pgfpathlineto{\pgfqpoint{1.997321in}{1.616181in}}%
\pgfpathlineto{\pgfqpoint{1.999051in}{1.575487in}}%
\pgfpathlineto{\pgfqpoint{1.999915in}{1.627135in}}%
\pgfpathlineto{\pgfqpoint{2.000777in}{1.604516in}}%
\pgfpathlineto{\pgfqpoint{2.001642in}{1.640907in}}%
\pgfpathlineto{\pgfqpoint{2.002508in}{1.563674in}}%
\pgfpathlineto{\pgfqpoint{2.003373in}{1.614965in}}%
\pgfpathlineto{\pgfqpoint{2.004238in}{1.553375in}}%
\pgfpathlineto{\pgfqpoint{2.005103in}{1.606416in}}%
\pgfpathlineto{\pgfqpoint{2.006833in}{1.553996in}}%
\pgfpathlineto{\pgfqpoint{2.007697in}{1.636810in}}%
\pgfpathlineto{\pgfqpoint{2.009429in}{1.547317in}}%
\pgfpathlineto{\pgfqpoint{2.011158in}{1.659071in}}%
\pgfpathlineto{\pgfqpoint{2.012023in}{1.604219in}}%
\pgfpathlineto{\pgfqpoint{2.012887in}{1.647496in}}%
\pgfpathlineto{\pgfqpoint{2.013752in}{1.635148in}}%
\pgfpathlineto{\pgfqpoint{2.014617in}{1.546280in}}%
\pgfpathlineto{\pgfqpoint{2.015482in}{1.565098in}}%
\pgfpathlineto{\pgfqpoint{2.016347in}{1.571539in}}%
\pgfpathlineto{\pgfqpoint{2.017211in}{1.527343in}}%
\pgfpathlineto{\pgfqpoint{2.018077in}{1.590744in}}%
\pgfpathlineto{\pgfqpoint{2.018939in}{1.532388in}}%
\pgfpathlineto{\pgfqpoint{2.020669in}{1.615913in}}%
\pgfpathlineto{\pgfqpoint{2.022400in}{1.564860in}}%
\pgfpathlineto{\pgfqpoint{2.024131in}{1.613180in}}%
\pgfpathlineto{\pgfqpoint{2.024995in}{1.612589in}}%
\pgfpathlineto{\pgfqpoint{2.025860in}{1.606000in}}%
\pgfpathlineto{\pgfqpoint{2.026727in}{1.576852in}}%
\pgfpathlineto{\pgfqpoint{2.027592in}{1.589201in}}%
\pgfpathlineto{\pgfqpoint{2.028455in}{1.633515in}}%
\pgfpathlineto{\pgfqpoint{2.029320in}{1.559815in}}%
\pgfpathlineto{\pgfqpoint{2.031052in}{1.662336in}}%
\pgfpathlineto{\pgfqpoint{2.032782in}{1.576614in}}%
\pgfpathlineto{\pgfqpoint{2.033646in}{1.581838in}}%
\pgfpathlineto{\pgfqpoint{2.035374in}{1.642566in}}%
\pgfpathlineto{\pgfqpoint{2.036239in}{1.641677in}}%
\pgfpathlineto{\pgfqpoint{2.037104in}{1.593532in}}%
\pgfpathlineto{\pgfqpoint{2.037969in}{1.662068in}}%
\pgfpathlineto{\pgfqpoint{2.039700in}{1.617396in}}%
\pgfpathlineto{\pgfqpoint{2.040565in}{1.627428in}}%
\pgfpathlineto{\pgfqpoint{2.041430in}{1.605464in}}%
\pgfpathlineto{\pgfqpoint{2.042296in}{1.665601in}}%
\pgfpathlineto{\pgfqpoint{2.043162in}{1.614013in}}%
\pgfpathlineto{\pgfqpoint{2.044026in}{1.683884in}}%
\pgfpathlineto{\pgfqpoint{2.044891in}{1.652836in}}%
\pgfpathlineto{\pgfqpoint{2.045756in}{1.676224in}}%
\pgfpathlineto{\pgfqpoint{2.047484in}{1.553252in}}%
\pgfpathlineto{\pgfqpoint{2.049214in}{1.666132in}}%
\pgfpathlineto{\pgfqpoint{2.050080in}{1.635263in}}%
\pgfpathlineto{\pgfqpoint{2.050945in}{1.585754in}}%
\pgfpathlineto{\pgfqpoint{2.051810in}{1.666549in}}%
\pgfpathlineto{\pgfqpoint{2.052674in}{1.613358in}}%
\pgfpathlineto{\pgfqpoint{2.053538in}{1.621907in}}%
\pgfpathlineto{\pgfqpoint{2.054402in}{1.640844in}}%
\pgfpathlineto{\pgfqpoint{2.056131in}{1.565511in}}%
\pgfpathlineto{\pgfqpoint{2.056996in}{1.606710in}}%
\pgfpathlineto{\pgfqpoint{2.057861in}{1.591989in}}%
\pgfpathlineto{\pgfqpoint{2.058726in}{1.600776in}}%
\pgfpathlineto{\pgfqpoint{2.060456in}{1.647020in}}%
\pgfpathlineto{\pgfqpoint{2.061322in}{1.612618in}}%
\pgfpathlineto{\pgfqpoint{2.062187in}{1.525264in}}%
\pgfpathlineto{\pgfqpoint{2.063052in}{1.558034in}}%
\pgfpathlineto{\pgfqpoint{2.063916in}{1.540550in}}%
\pgfpathlineto{\pgfqpoint{2.065646in}{1.622085in}}%
\pgfpathlineto{\pgfqpoint{2.068243in}{1.570676in}}%
\pgfpathlineto{\pgfqpoint{2.069109in}{1.557379in}}%
\pgfpathlineto{\pgfqpoint{2.069974in}{1.572576in}}%
\pgfpathlineto{\pgfqpoint{2.070838in}{1.535058in}}%
\pgfpathlineto{\pgfqpoint{2.071701in}{1.645061in}}%
\pgfpathlineto{\pgfqpoint{2.072565in}{1.546752in}}%
\pgfpathlineto{\pgfqpoint{2.074294in}{1.607662in}}%
\pgfpathlineto{\pgfqpoint{2.075160in}{1.575190in}}%
\pgfpathlineto{\pgfqpoint{2.076025in}{1.638085in}}%
\pgfpathlineto{\pgfqpoint{2.076890in}{1.516537in}}%
\pgfpathlineto{\pgfqpoint{2.077754in}{1.545090in}}%
\pgfpathlineto{\pgfqpoint{2.079483in}{1.643399in}}%
\pgfpathlineto{\pgfqpoint{2.080348in}{1.600300in}}%
\pgfpathlineto{\pgfqpoint{2.081213in}{1.610629in}}%
\pgfpathlineto{\pgfqpoint{2.082078in}{1.632950in}}%
\pgfpathlineto{\pgfqpoint{2.084671in}{1.588844in}}%
\pgfpathlineto{\pgfqpoint{2.086401in}{1.645001in}}%
\pgfpathlineto{\pgfqpoint{2.088130in}{1.598935in}}%
\pgfpathlineto{\pgfqpoint{2.088995in}{1.643339in}}%
\pgfpathlineto{\pgfqpoint{2.089859in}{1.585073in}}%
\pgfpathlineto{\pgfqpoint{2.090724in}{1.639599in}}%
\pgfpathlineto{\pgfqpoint{2.091589in}{1.620308in}}%
\pgfpathlineto{\pgfqpoint{2.092454in}{1.519032in}}%
\pgfpathlineto{\pgfqpoint{2.094184in}{1.676109in}}%
\pgfpathlineto{\pgfqpoint{2.095914in}{1.573766in}}%
\pgfpathlineto{\pgfqpoint{2.096780in}{1.661622in}}%
\pgfpathlineto{\pgfqpoint{2.098507in}{1.588427in}}%
\pgfpathlineto{\pgfqpoint{2.099373in}{1.620483in}}%
\pgfpathlineto{\pgfqpoint{2.100239in}{1.600240in}}%
\pgfpathlineto{\pgfqpoint{2.101101in}{1.659900in}}%
\pgfpathlineto{\pgfqpoint{2.102828in}{1.592997in}}%
\pgfpathlineto{\pgfqpoint{2.103692in}{1.597745in}}%
\pgfpathlineto{\pgfqpoint{2.106286in}{1.577800in}}%
\pgfpathlineto{\pgfqpoint{2.107151in}{1.533396in}}%
\pgfpathlineto{\pgfqpoint{2.109746in}{1.621018in}}%
\pgfpathlineto{\pgfqpoint{2.110611in}{1.609384in}}%
\pgfpathlineto{\pgfqpoint{2.112342in}{1.552452in}}%
\pgfpathlineto{\pgfqpoint{2.113207in}{1.598638in}}%
\pgfpathlineto{\pgfqpoint{2.114938in}{1.528410in}}%
\pgfpathlineto{\pgfqpoint{2.115804in}{1.606591in}}%
\pgfpathlineto{\pgfqpoint{2.118403in}{1.541439in}}%
\pgfpathlineto{\pgfqpoint{2.120134in}{1.635560in}}%
\pgfpathlineto{\pgfqpoint{2.120999in}{1.611250in}}%
\pgfpathlineto{\pgfqpoint{2.122731in}{1.634196in}}%
\pgfpathlineto{\pgfqpoint{2.123596in}{1.551798in}}%
\pgfpathlineto{\pgfqpoint{2.125328in}{1.587178in}}%
\pgfpathlineto{\pgfqpoint{2.126193in}{1.558208in}}%
\pgfpathlineto{\pgfqpoint{2.128790in}{1.592699in}}%
\pgfpathlineto{\pgfqpoint{2.131385in}{1.521405in}}%
\pgfpathlineto{\pgfqpoint{2.133116in}{1.635382in}}%
\pgfpathlineto{\pgfqpoint{2.133982in}{1.639182in}}%
\pgfpathlineto{\pgfqpoint{2.135713in}{1.606413in}}%
\pgfpathlineto{\pgfqpoint{2.136575in}{1.624164in}}%
\pgfpathlineto{\pgfqpoint{2.137440in}{1.573911in}}%
\pgfpathlineto{\pgfqpoint{2.138303in}{1.673852in}}%
\pgfpathlineto{\pgfqpoint{2.139169in}{1.661444in}}%
\pgfpathlineto{\pgfqpoint{2.140031in}{1.675811in}}%
\pgfpathlineto{\pgfqpoint{2.141763in}{1.598757in}}%
\pgfpathlineto{\pgfqpoint{2.142629in}{1.601902in}}%
\pgfpathlineto{\pgfqpoint{2.143495in}{1.625647in}}%
\pgfpathlineto{\pgfqpoint{2.145225in}{1.598578in}}%
\pgfpathlineto{\pgfqpoint{2.146091in}{1.610748in}}%
\pgfpathlineto{\pgfqpoint{2.147822in}{1.567709in}}%
\pgfpathlineto{\pgfqpoint{2.149554in}{1.639985in}}%
\pgfpathlineto{\pgfqpoint{2.150418in}{1.659960in}}%
\pgfpathlineto{\pgfqpoint{2.152149in}{1.572933in}}%
\pgfpathlineto{\pgfqpoint{2.153015in}{1.587479in}}%
\pgfpathlineto{\pgfqpoint{2.153879in}{1.650225in}}%
\pgfpathlineto{\pgfqpoint{2.154743in}{1.648682in}}%
\pgfpathlineto{\pgfqpoint{2.156472in}{1.524937in}}%
\pgfpathlineto{\pgfqpoint{2.158202in}{1.609737in}}%
\pgfpathlineto{\pgfqpoint{2.159066in}{1.578838in}}%
\pgfpathlineto{\pgfqpoint{2.159928in}{1.615080in}}%
\pgfpathlineto{\pgfqpoint{2.160794in}{1.543012in}}%
\pgfpathlineto{\pgfqpoint{2.162520in}{1.600712in}}%
\pgfpathlineto{\pgfqpoint{2.163384in}{1.562127in}}%
\pgfpathlineto{\pgfqpoint{2.165115in}{1.621907in}}%
\pgfpathlineto{\pgfqpoint{2.165980in}{1.622383in}}%
\pgfpathlineto{\pgfqpoint{2.166844in}{1.684062in}}%
\pgfpathlineto{\pgfqpoint{2.167707in}{1.675990in}}%
\pgfpathlineto{\pgfqpoint{2.168572in}{1.635650in}}%
\pgfpathlineto{\pgfqpoint{2.169437in}{1.776075in}}%
\pgfpathlineto{\pgfqpoint{2.171166in}{1.645150in}}%
\pgfpathlineto{\pgfqpoint{2.172029in}{1.701754in}}%
\pgfpathlineto{\pgfqpoint{2.172893in}{1.655093in}}%
\pgfpathlineto{\pgfqpoint{2.173758in}{1.659250in}}%
\pgfpathlineto{\pgfqpoint{2.174620in}{1.675752in}}%
\pgfpathlineto{\pgfqpoint{2.175484in}{1.639480in}}%
\pgfpathlineto{\pgfqpoint{2.176349in}{1.738677in}}%
\pgfpathlineto{\pgfqpoint{2.177215in}{1.738558in}}%
\pgfpathlineto{\pgfqpoint{2.178080in}{1.610064in}}%
\pgfpathlineto{\pgfqpoint{2.179810in}{1.673673in}}%
\pgfpathlineto{\pgfqpoint{2.180676in}{1.665035in}}%
\pgfpathlineto{\pgfqpoint{2.181542in}{1.682400in}}%
\pgfpathlineto{\pgfqpoint{2.182405in}{1.653550in}}%
\pgfpathlineto{\pgfqpoint{2.183270in}{1.703178in}}%
\pgfpathlineto{\pgfqpoint{2.184136in}{1.698013in}}%
\pgfpathlineto{\pgfqpoint{2.185000in}{1.679046in}}%
\pgfpathlineto{\pgfqpoint{2.185866in}{1.689286in}}%
\pgfpathlineto{\pgfqpoint{2.186731in}{1.718851in}}%
\pgfpathlineto{\pgfqpoint{2.188460in}{1.644406in}}%
\pgfpathlineto{\pgfqpoint{2.189326in}{1.767825in}}%
\pgfpathlineto{\pgfqpoint{2.190192in}{1.745771in}}%
\pgfpathlineto{\pgfqpoint{2.191056in}{1.766757in}}%
\pgfpathlineto{\pgfqpoint{2.192788in}{1.635977in}}%
\pgfpathlineto{\pgfqpoint{2.193652in}{1.664530in}}%
\pgfpathlineto{\pgfqpoint{2.194519in}{1.571624in}}%
\pgfpathlineto{\pgfqpoint{2.195384in}{1.702821in}}%
\pgfpathlineto{\pgfqpoint{2.196249in}{1.688334in}}%
\pgfpathlineto{\pgfqpoint{2.197115in}{1.633690in}}%
\pgfpathlineto{\pgfqpoint{2.197982in}{1.664470in}}%
\pgfpathlineto{\pgfqpoint{2.198846in}{1.625469in}}%
\pgfpathlineto{\pgfqpoint{2.199710in}{1.695756in}}%
\pgfpathlineto{\pgfqpoint{2.200575in}{1.678659in}}%
\pgfpathlineto{\pgfqpoint{2.201440in}{1.712853in}}%
\pgfpathlineto{\pgfqpoint{2.202305in}{1.628853in}}%
\pgfpathlineto{\pgfqpoint{2.203171in}{1.666430in}}%
\pgfpathlineto{\pgfqpoint{2.204035in}{1.661711in}}%
\pgfpathlineto{\pgfqpoint{2.204900in}{1.651888in}}%
\pgfpathlineto{\pgfqpoint{2.207495in}{1.703178in}}%
\pgfpathlineto{\pgfqpoint{2.208361in}{1.647849in}}%
\pgfpathlineto{\pgfqpoint{2.209225in}{1.655003in}}%
\pgfpathlineto{\pgfqpoint{2.210090in}{1.677354in}}%
\pgfpathlineto{\pgfqpoint{2.210954in}{1.675633in}}%
\pgfpathlineto{\pgfqpoint{2.211819in}{1.646723in}}%
\pgfpathlineto{\pgfqpoint{2.212685in}{1.712675in}}%
\pgfpathlineto{\pgfqpoint{2.214415in}{1.655152in}}%
\pgfpathlineto{\pgfqpoint{2.215281in}{1.697894in}}%
\pgfpathlineto{\pgfqpoint{2.216146in}{1.694749in}}%
\pgfpathlineto{\pgfqpoint{2.218741in}{1.639152in}}%
\pgfpathlineto{\pgfqpoint{2.220471in}{1.682043in}}%
\pgfpathlineto{\pgfqpoint{2.221337in}{1.592937in}}%
\pgfpathlineto{\pgfqpoint{2.222202in}{1.687148in}}%
\pgfpathlineto{\pgfqpoint{2.223065in}{1.615407in}}%
\pgfpathlineto{\pgfqpoint{2.223931in}{1.693856in}}%
\pgfpathlineto{\pgfqpoint{2.226526in}{1.634136in}}%
\pgfpathlineto{\pgfqpoint{2.227389in}{1.703948in}}%
\pgfpathlineto{\pgfqpoint{2.228254in}{1.659841in}}%
\pgfpathlineto{\pgfqpoint{2.229986in}{1.711429in}}%
\pgfpathlineto{\pgfqpoint{2.230853in}{1.709648in}}%
\pgfpathlineto{\pgfqpoint{2.231717in}{1.702583in}}%
\pgfpathlineto{\pgfqpoint{2.232580in}{1.704513in}}%
\pgfpathlineto{\pgfqpoint{2.234309in}{1.592878in}}%
\pgfpathlineto{\pgfqpoint{2.235174in}{1.671981in}}%
\pgfpathlineto{\pgfqpoint{2.236039in}{1.641915in}}%
\pgfpathlineto{\pgfqpoint{2.236904in}{1.713567in}}%
\pgfpathlineto{\pgfqpoint{2.239497in}{1.648920in}}%
\pgfpathlineto{\pgfqpoint{2.240361in}{1.638829in}}%
\pgfpathlineto{\pgfqpoint{2.241226in}{1.609681in}}%
\pgfpathlineto{\pgfqpoint{2.242955in}{1.710362in}}%
\pgfpathlineto{\pgfqpoint{2.243820in}{1.633367in}}%
\pgfpathlineto{\pgfqpoint{2.245550in}{1.721818in}}%
\pgfpathlineto{\pgfqpoint{2.246416in}{1.648563in}}%
\pgfpathlineto{\pgfqpoint{2.247281in}{1.655539in}}%
\pgfpathlineto{\pgfqpoint{2.248147in}{1.700151in}}%
\pgfpathlineto{\pgfqpoint{2.250740in}{1.647080in}}%
\pgfpathlineto{\pgfqpoint{2.253336in}{1.713329in}}%
\pgfpathlineto{\pgfqpoint{2.254201in}{1.628763in}}%
\pgfpathlineto{\pgfqpoint{2.255930in}{1.682281in}}%
\pgfpathlineto{\pgfqpoint{2.256794in}{1.631199in}}%
\pgfpathlineto{\pgfqpoint{2.258523in}{1.720513in}}%
\pgfpathlineto{\pgfqpoint{2.259389in}{1.710064in}}%
\pgfpathlineto{\pgfqpoint{2.260254in}{1.629329in}}%
\pgfpathlineto{\pgfqpoint{2.261984in}{1.718018in}}%
\pgfpathlineto{\pgfqpoint{2.263712in}{1.692194in}}%
\pgfpathlineto{\pgfqpoint{2.265441in}{1.736301in}}%
\pgfpathlineto{\pgfqpoint{2.266306in}{1.686583in}}%
\pgfpathlineto{\pgfqpoint{2.267171in}{1.753041in}}%
\pgfpathlineto{\pgfqpoint{2.268898in}{1.661146in}}%
\pgfpathlineto{\pgfqpoint{2.269763in}{1.709469in}}%
\pgfpathlineto{\pgfqpoint{2.270626in}{1.675514in}}%
\pgfpathlineto{\pgfqpoint{2.271490in}{1.682459in}}%
\pgfpathlineto{\pgfqpoint{2.273219in}{1.728942in}}%
\pgfpathlineto{\pgfqpoint{2.274949in}{1.758979in}}%
\pgfpathlineto{\pgfqpoint{2.275812in}{1.690502in}}%
\pgfpathlineto{\pgfqpoint{2.276676in}{1.705967in}}%
\pgfpathlineto{\pgfqpoint{2.277541in}{1.762005in}}%
\pgfpathlineto{\pgfqpoint{2.280137in}{1.646158in}}%
\pgfpathlineto{\pgfqpoint{2.281003in}{1.669754in}}%
\pgfpathlineto{\pgfqpoint{2.281868in}{1.631050in}}%
\pgfpathlineto{\pgfqpoint{2.283598in}{1.680619in}}%
\pgfpathlineto{\pgfqpoint{2.285328in}{1.639807in}}%
\pgfpathlineto{\pgfqpoint{2.286192in}{1.646425in}}%
\pgfpathlineto{\pgfqpoint{2.287056in}{1.641439in}}%
\pgfpathlineto{\pgfqpoint{2.287921in}{1.519683in}}%
\pgfpathlineto{\pgfqpoint{2.289652in}{1.626982in}}%
\pgfpathlineto{\pgfqpoint{2.292249in}{1.540282in}}%
\pgfpathlineto{\pgfqpoint{2.293114in}{1.568954in}}%
\pgfpathlineto{\pgfqpoint{2.293978in}{1.555360in}}%
\pgfpathlineto{\pgfqpoint{2.295707in}{1.577324in}}%
\pgfpathlineto{\pgfqpoint{2.296571in}{1.563909in}}%
\pgfpathlineto{\pgfqpoint{2.297435in}{1.624636in}}%
\pgfpathlineto{\pgfqpoint{2.298302in}{1.577800in}}%
\pgfpathlineto{\pgfqpoint{2.300033in}{1.635620in}}%
\pgfpathlineto{\pgfqpoint{2.300900in}{1.645830in}}%
\pgfpathlineto{\pgfqpoint{2.301766in}{1.549957in}}%
\pgfpathlineto{\pgfqpoint{2.304361in}{1.617158in}}%
\pgfpathlineto{\pgfqpoint{2.305227in}{1.598578in}}%
\pgfpathlineto{\pgfqpoint{2.306092in}{1.611756in}}%
\pgfpathlineto{\pgfqpoint{2.306957in}{1.540223in}}%
\pgfpathlineto{\pgfqpoint{2.307822in}{1.559160in}}%
\pgfpathlineto{\pgfqpoint{2.308688in}{1.610242in}}%
\pgfpathlineto{\pgfqpoint{2.309554in}{1.543785in}}%
\pgfpathlineto{\pgfqpoint{2.310420in}{1.559279in}}%
\pgfpathlineto{\pgfqpoint{2.311285in}{1.601010in}}%
\pgfpathlineto{\pgfqpoint{2.312152in}{1.585694in}}%
\pgfpathlineto{\pgfqpoint{2.313883in}{1.634166in}}%
\pgfpathlineto{\pgfqpoint{2.314749in}{1.650936in}}%
\pgfpathlineto{\pgfqpoint{2.316478in}{1.568776in}}%
\pgfpathlineto{\pgfqpoint{2.317342in}{1.639063in}}%
\pgfpathlineto{\pgfqpoint{2.318206in}{1.563819in}}%
\pgfpathlineto{\pgfqpoint{2.319072in}{1.569490in}}%
\pgfpathlineto{\pgfqpoint{2.319937in}{1.569430in}}%
\pgfpathlineto{\pgfqpoint{2.321668in}{1.649809in}}%
\pgfpathlineto{\pgfqpoint{2.323399in}{1.577651in}}%
\pgfpathlineto{\pgfqpoint{2.324264in}{1.632950in}}%
\pgfpathlineto{\pgfqpoint{2.325127in}{1.617040in}}%
\pgfpathlineto{\pgfqpoint{2.325993in}{1.637639in}}%
\pgfpathlineto{\pgfqpoint{2.326858in}{1.570676in}}%
\pgfpathlineto{\pgfqpoint{2.327723in}{1.596797in}}%
\pgfpathlineto{\pgfqpoint{2.328589in}{1.670349in}}%
\pgfpathlineto{\pgfqpoint{2.330317in}{1.605554in}}%
\pgfpathlineto{\pgfqpoint{2.332043in}{1.597570in}}%
\pgfpathlineto{\pgfqpoint{2.332908in}{1.609562in}}%
\pgfpathlineto{\pgfqpoint{2.333775in}{1.581838in}}%
\pgfpathlineto{\pgfqpoint{2.334640in}{1.607008in}}%
\pgfpathlineto{\pgfqpoint{2.335506in}{1.509175in}}%
\pgfpathlineto{\pgfqpoint{2.336370in}{1.599288in}}%
\pgfpathlineto{\pgfqpoint{2.337235in}{1.562693in}}%
\pgfpathlineto{\pgfqpoint{2.338100in}{1.639182in}}%
\pgfpathlineto{\pgfqpoint{2.341556in}{1.502170in}}%
\pgfpathlineto{\pgfqpoint{2.342422in}{1.507126in}}%
\pgfpathlineto{\pgfqpoint{2.343286in}{1.533158in}}%
\pgfpathlineto{\pgfqpoint{2.344150in}{1.529774in}}%
\pgfpathlineto{\pgfqpoint{2.345881in}{1.588427in}}%
\pgfpathlineto{\pgfqpoint{2.346747in}{1.533575in}}%
\pgfpathlineto{\pgfqpoint{2.350207in}{1.597567in}}%
\pgfpathlineto{\pgfqpoint{2.353668in}{1.526034in}}%
\pgfpathlineto{\pgfqpoint{2.356262in}{1.624253in}}%
\pgfpathlineto{\pgfqpoint{2.357127in}{1.584746in}}%
\pgfpathlineto{\pgfqpoint{2.358859in}{1.608521in}}%
\pgfpathlineto{\pgfqpoint{2.359724in}{1.511373in}}%
\pgfpathlineto{\pgfqpoint{2.361451in}{1.633928in}}%
\pgfpathlineto{\pgfqpoint{2.362316in}{1.638293in}}%
\pgfpathlineto{\pgfqpoint{2.363181in}{1.613953in}}%
\pgfpathlineto{\pgfqpoint{2.364045in}{1.553936in}}%
\pgfpathlineto{\pgfqpoint{2.364909in}{1.655152in}}%
\pgfpathlineto{\pgfqpoint{2.365771in}{1.649720in}}%
\pgfpathlineto{\pgfqpoint{2.368366in}{1.568359in}}%
\pgfpathlineto{\pgfqpoint{2.369231in}{1.587713in}}%
\pgfpathlineto{\pgfqpoint{2.370096in}{1.650698in}}%
\pgfpathlineto{\pgfqpoint{2.370961in}{1.596410in}}%
\pgfpathlineto{\pgfqpoint{2.371826in}{1.603386in}}%
\pgfpathlineto{\pgfqpoint{2.372691in}{1.634553in}}%
\pgfpathlineto{\pgfqpoint{2.373557in}{1.555182in}}%
\pgfpathlineto{\pgfqpoint{2.374422in}{1.618404in}}%
\pgfpathlineto{\pgfqpoint{2.375287in}{1.613329in}}%
\pgfpathlineto{\pgfqpoint{2.376151in}{1.599943in}}%
\pgfpathlineto{\pgfqpoint{2.377017in}{1.616266in}}%
\pgfpathlineto{\pgfqpoint{2.377884in}{1.607008in}}%
\pgfpathlineto{\pgfqpoint{2.378749in}{1.567649in}}%
\pgfpathlineto{\pgfqpoint{2.380481in}{1.639926in}}%
\pgfpathlineto{\pgfqpoint{2.381346in}{1.568006in}}%
\pgfpathlineto{\pgfqpoint{2.383076in}{1.620721in}}%
\pgfpathlineto{\pgfqpoint{2.383941in}{1.551266in}}%
\pgfpathlineto{\pgfqpoint{2.384807in}{1.656372in}}%
\pgfpathlineto{\pgfqpoint{2.385670in}{1.638710in}}%
\pgfpathlineto{\pgfqpoint{2.389131in}{1.694808in}}%
\pgfpathlineto{\pgfqpoint{2.389995in}{1.636929in}}%
\pgfpathlineto{\pgfqpoint{2.390860in}{1.736840in}}%
\pgfpathlineto{\pgfqpoint{2.391725in}{1.659547in}}%
\pgfpathlineto{\pgfqpoint{2.392590in}{1.660614in}}%
\pgfpathlineto{\pgfqpoint{2.393455in}{1.670884in}}%
\pgfpathlineto{\pgfqpoint{2.394320in}{1.605970in}}%
\pgfpathlineto{\pgfqpoint{2.395187in}{1.676406in}}%
\pgfpathlineto{\pgfqpoint{2.396051in}{1.675990in}}%
\pgfpathlineto{\pgfqpoint{2.397781in}{1.649809in}}%
\pgfpathlineto{\pgfqpoint{2.398647in}{1.698311in}}%
\pgfpathlineto{\pgfqpoint{2.399511in}{1.665452in}}%
\pgfpathlineto{\pgfqpoint{2.400376in}{1.685129in}}%
\pgfpathlineto{\pgfqpoint{2.401240in}{1.675514in}}%
\pgfpathlineto{\pgfqpoint{2.402105in}{1.652895in}}%
\pgfpathlineto{\pgfqpoint{2.403834in}{1.692521in}}%
\pgfpathlineto{\pgfqpoint{2.404699in}{1.658655in}}%
\pgfpathlineto{\pgfqpoint{2.405564in}{1.684835in}}%
\pgfpathlineto{\pgfqpoint{2.407293in}{1.632831in}}%
\pgfpathlineto{\pgfqpoint{2.408157in}{1.667501in}}%
\pgfpathlineto{\pgfqpoint{2.409022in}{1.636245in}}%
\pgfpathlineto{\pgfqpoint{2.409887in}{1.662217in}}%
\pgfpathlineto{\pgfqpoint{2.410753in}{1.658238in}}%
\pgfpathlineto{\pgfqpoint{2.411618in}{1.620780in}}%
\pgfpathlineto{\pgfqpoint{2.412483in}{1.648623in}}%
\pgfpathlineto{\pgfqpoint{2.413348in}{1.644971in}}%
\pgfpathlineto{\pgfqpoint{2.415079in}{1.579760in}}%
\pgfpathlineto{\pgfqpoint{2.416808in}{1.522353in}}%
\pgfpathlineto{\pgfqpoint{2.417672in}{1.645711in}}%
\pgfpathlineto{\pgfqpoint{2.419402in}{1.555003in}}%
\pgfpathlineto{\pgfqpoint{2.421131in}{1.637579in}}%
\pgfpathlineto{\pgfqpoint{2.421995in}{1.605940in}}%
\pgfpathlineto{\pgfqpoint{2.422861in}{1.658774in}}%
\pgfpathlineto{\pgfqpoint{2.424590in}{1.574833in}}%
\pgfpathlineto{\pgfqpoint{2.425455in}{1.586735in}}%
\pgfpathlineto{\pgfqpoint{2.426320in}{1.567590in}}%
\pgfpathlineto{\pgfqpoint{2.427184in}{1.595551in}}%
\pgfpathlineto{\pgfqpoint{2.428048in}{1.543874in}}%
\pgfpathlineto{\pgfqpoint{2.428912in}{1.624521in}}%
\pgfpathlineto{\pgfqpoint{2.429777in}{1.559160in}}%
\pgfpathlineto{\pgfqpoint{2.430641in}{1.564087in}}%
\pgfpathlineto{\pgfqpoint{2.432370in}{1.531585in}}%
\pgfpathlineto{\pgfqpoint{2.433233in}{1.564384in}}%
\pgfpathlineto{\pgfqpoint{2.434097in}{1.538918in}}%
\pgfpathlineto{\pgfqpoint{2.434962in}{1.552304in}}%
\pgfpathlineto{\pgfqpoint{2.435826in}{1.589970in}}%
\pgfpathlineto{\pgfqpoint{2.436690in}{1.559755in}}%
\pgfpathlineto{\pgfqpoint{2.437555in}{1.590119in}}%
\pgfpathlineto{\pgfqpoint{2.438420in}{1.517902in}}%
\pgfpathlineto{\pgfqpoint{2.440150in}{1.622323in}}%
\pgfpathlineto{\pgfqpoint{2.441015in}{1.546812in}}%
\pgfpathlineto{\pgfqpoint{2.441880in}{1.589524in}}%
\pgfpathlineto{\pgfqpoint{2.442744in}{1.586527in}}%
\pgfpathlineto{\pgfqpoint{2.444474in}{1.534761in}}%
\pgfpathlineto{\pgfqpoint{2.445339in}{1.653133in}}%
\pgfpathlineto{\pgfqpoint{2.446201in}{1.580232in}}%
\pgfpathlineto{\pgfqpoint{2.447066in}{1.594570in}}%
\pgfpathlineto{\pgfqpoint{2.448795in}{1.611577in}}%
\pgfpathlineto{\pgfqpoint{2.449658in}{1.575305in}}%
\pgfpathlineto{\pgfqpoint{2.450524in}{1.607008in}}%
\pgfpathlineto{\pgfqpoint{2.451388in}{1.575751in}}%
\pgfpathlineto{\pgfqpoint{2.452253in}{1.638885in}}%
\pgfpathlineto{\pgfqpoint{2.453982in}{1.592937in}}%
\pgfpathlineto{\pgfqpoint{2.454848in}{1.617040in}}%
\pgfpathlineto{\pgfqpoint{2.455713in}{1.606651in}}%
\pgfpathlineto{\pgfqpoint{2.457441in}{1.655212in}}%
\pgfpathlineto{\pgfqpoint{2.458306in}{1.601545in}}%
\pgfpathlineto{\pgfqpoint{2.460035in}{1.631288in}}%
\pgfpathlineto{\pgfqpoint{2.460899in}{1.626004in}}%
\pgfpathlineto{\pgfqpoint{2.462629in}{1.696708in}}%
\pgfpathlineto{\pgfqpoint{2.464360in}{1.559160in}}%
\pgfpathlineto{\pgfqpoint{2.465225in}{1.629150in}}%
\pgfpathlineto{\pgfqpoint{2.466090in}{1.564295in}}%
\pgfpathlineto{\pgfqpoint{2.467821in}{1.608908in}}%
\pgfpathlineto{\pgfqpoint{2.468685in}{1.579938in}}%
\pgfpathlineto{\pgfqpoint{2.469550in}{1.615794in}}%
\pgfpathlineto{\pgfqpoint{2.470414in}{1.552780in}}%
\pgfpathlineto{\pgfqpoint{2.471277in}{1.565098in}}%
\pgfpathlineto{\pgfqpoint{2.472140in}{1.555777in}}%
\pgfpathlineto{\pgfqpoint{2.473871in}{1.629150in}}%
\pgfpathlineto{\pgfqpoint{2.474737in}{1.576019in}}%
\pgfpathlineto{\pgfqpoint{2.475602in}{1.617129in}}%
\pgfpathlineto{\pgfqpoint{2.476468in}{1.565630in}}%
\pgfpathlineto{\pgfqpoint{2.477334in}{1.663106in}}%
\pgfpathlineto{\pgfqpoint{2.478200in}{1.628912in}}%
\pgfpathlineto{\pgfqpoint{2.479066in}{1.656636in}}%
\pgfpathlineto{\pgfqpoint{2.479931in}{1.627488in}}%
\pgfpathlineto{\pgfqpoint{2.480797in}{1.632891in}}%
\pgfpathlineto{\pgfqpoint{2.481662in}{1.615139in}}%
\pgfpathlineto{\pgfqpoint{2.483391in}{1.644287in}}%
\pgfpathlineto{\pgfqpoint{2.485119in}{1.610867in}}%
\pgfpathlineto{\pgfqpoint{2.486849in}{1.680619in}}%
\pgfpathlineto{\pgfqpoint{2.487715in}{1.666787in}}%
\pgfpathlineto{\pgfqpoint{2.489447in}{1.545566in}}%
\pgfpathlineto{\pgfqpoint{2.490310in}{1.633898in}}%
\pgfpathlineto{\pgfqpoint{2.491175in}{1.633069in}}%
\pgfpathlineto{\pgfqpoint{2.492039in}{1.570293in}}%
\pgfpathlineto{\pgfqpoint{2.493769in}{1.641677in}}%
\pgfpathlineto{\pgfqpoint{2.494635in}{1.580176in}}%
\pgfpathlineto{\pgfqpoint{2.495501in}{1.655212in}}%
\pgfpathlineto{\pgfqpoint{2.496365in}{1.564325in}}%
\pgfpathlineto{\pgfqpoint{2.497231in}{1.625707in}}%
\pgfpathlineto{\pgfqpoint{2.498096in}{1.562960in}}%
\pgfpathlineto{\pgfqpoint{2.498962in}{1.629418in}}%
\pgfpathlineto{\pgfqpoint{2.500692in}{1.578157in}}%
\pgfpathlineto{\pgfqpoint{2.501555in}{1.604189in}}%
\pgfpathlineto{\pgfqpoint{2.502421in}{1.576198in}}%
\pgfpathlineto{\pgfqpoint{2.503285in}{1.600478in}}%
\pgfpathlineto{\pgfqpoint{2.504149in}{1.600121in}}%
\pgfpathlineto{\pgfqpoint{2.505014in}{1.609027in}}%
\pgfpathlineto{\pgfqpoint{2.505880in}{1.569192in}}%
\pgfpathlineto{\pgfqpoint{2.507611in}{1.609324in}}%
\pgfpathlineto{\pgfqpoint{2.508476in}{1.571628in}}%
\pgfpathlineto{\pgfqpoint{2.509340in}{1.606000in}}%
\pgfpathlineto{\pgfqpoint{2.510205in}{1.594127in}}%
\pgfpathlineto{\pgfqpoint{2.511069in}{1.516805in}}%
\pgfpathlineto{\pgfqpoint{2.511934in}{1.587181in}}%
\pgfpathlineto{\pgfqpoint{2.512798in}{1.532329in}}%
\pgfpathlineto{\pgfqpoint{2.514530in}{1.607662in}}%
\pgfpathlineto{\pgfqpoint{2.515396in}{1.597213in}}%
\pgfpathlineto{\pgfqpoint{2.517125in}{1.646901in}}%
\pgfpathlineto{\pgfqpoint{2.517990in}{1.513540in}}%
\pgfpathlineto{\pgfqpoint{2.519720in}{1.617694in}}%
\pgfpathlineto{\pgfqpoint{2.522317in}{1.580355in}}%
\pgfpathlineto{\pgfqpoint{2.524046in}{1.665244in}}%
\pgfpathlineto{\pgfqpoint{2.524912in}{1.651947in}}%
\pgfpathlineto{\pgfqpoint{2.525777in}{1.621316in}}%
\pgfpathlineto{\pgfqpoint{2.526643in}{1.643280in}}%
\pgfpathlineto{\pgfqpoint{2.527509in}{1.597303in}}%
\pgfpathlineto{\pgfqpoint{2.528374in}{1.659547in}}%
\pgfpathlineto{\pgfqpoint{2.530102in}{1.556253in}}%
\pgfpathlineto{\pgfqpoint{2.531831in}{1.621970in}}%
\pgfpathlineto{\pgfqpoint{2.533562in}{1.543845in}}%
\pgfpathlineto{\pgfqpoint{2.534427in}{1.613065in}}%
\pgfpathlineto{\pgfqpoint{2.537021in}{1.540729in}}%
\pgfpathlineto{\pgfqpoint{2.537886in}{1.634910in}}%
\pgfpathlineto{\pgfqpoint{2.541346in}{1.540996in}}%
\pgfpathlineto{\pgfqpoint{2.542211in}{1.587241in}}%
\pgfpathlineto{\pgfqpoint{2.543077in}{1.549485in}}%
\pgfpathlineto{\pgfqpoint{2.543942in}{1.614429in}}%
\pgfpathlineto{\pgfqpoint{2.544808in}{1.569906in}}%
\pgfpathlineto{\pgfqpoint{2.545672in}{1.583322in}}%
\pgfpathlineto{\pgfqpoint{2.547403in}{1.556134in}}%
\pgfpathlineto{\pgfqpoint{2.548266in}{1.600597in}}%
\pgfpathlineto{\pgfqpoint{2.549131in}{1.588992in}}%
\pgfpathlineto{\pgfqpoint{2.549993in}{1.523781in}}%
\pgfpathlineto{\pgfqpoint{2.551724in}{1.580860in}}%
\pgfpathlineto{\pgfqpoint{2.552590in}{1.594960in}}%
\pgfpathlineto{\pgfqpoint{2.553455in}{1.594306in}}%
\pgfpathlineto{\pgfqpoint{2.554322in}{1.585400in}}%
\pgfpathlineto{\pgfqpoint{2.555188in}{1.625056in}}%
\pgfpathlineto{\pgfqpoint{2.556916in}{1.595551in}}%
\pgfpathlineto{\pgfqpoint{2.557782in}{1.650225in}}%
\pgfpathlineto{\pgfqpoint{2.558645in}{1.594276in}}%
\pgfpathlineto{\pgfqpoint{2.559511in}{1.650106in}}%
\pgfpathlineto{\pgfqpoint{2.560377in}{1.541234in}}%
\pgfpathlineto{\pgfqpoint{2.561241in}{1.568333in}}%
\pgfpathlineto{\pgfqpoint{2.562106in}{1.592465in}}%
\pgfpathlineto{\pgfqpoint{2.562970in}{1.670944in}}%
\pgfpathlineto{\pgfqpoint{2.564700in}{1.623335in}}%
\pgfpathlineto{\pgfqpoint{2.565564in}{1.665482in}}%
\pgfpathlineto{\pgfqpoint{2.566428in}{1.629567in}}%
\pgfpathlineto{\pgfqpoint{2.567292in}{1.637877in}}%
\pgfpathlineto{\pgfqpoint{2.568157in}{1.609205in}}%
\pgfpathlineto{\pgfqpoint{2.569022in}{1.627904in}}%
\pgfpathlineto{\pgfqpoint{2.569888in}{1.569728in}}%
\pgfpathlineto{\pgfqpoint{2.570753in}{1.598786in}}%
\pgfpathlineto{\pgfqpoint{2.571618in}{1.535415in}}%
\pgfpathlineto{\pgfqpoint{2.572482in}{1.611522in}}%
\pgfpathlineto{\pgfqpoint{2.573346in}{1.556134in}}%
\pgfpathlineto{\pgfqpoint{2.575939in}{1.639242in}}%
\pgfpathlineto{\pgfqpoint{2.576802in}{1.560763in}}%
\pgfpathlineto{\pgfqpoint{2.577665in}{1.598459in}}%
\pgfpathlineto{\pgfqpoint{2.578530in}{1.557498in}}%
\pgfpathlineto{\pgfqpoint{2.579394in}{1.561179in}}%
\pgfpathlineto{\pgfqpoint{2.581123in}{1.638472in}}%
\pgfpathlineto{\pgfqpoint{2.582854in}{1.558093in}}%
\pgfpathlineto{\pgfqpoint{2.585448in}{1.637996in}}%
\pgfpathlineto{\pgfqpoint{2.588042in}{1.526272in}}%
\pgfpathlineto{\pgfqpoint{2.588907in}{1.622323in}}%
\pgfpathlineto{\pgfqpoint{2.589773in}{1.558893in}}%
\pgfpathlineto{\pgfqpoint{2.590637in}{1.588844in}}%
\pgfpathlineto{\pgfqpoint{2.591500in}{1.568304in}}%
\pgfpathlineto{\pgfqpoint{2.593230in}{1.696708in}}%
\pgfpathlineto{\pgfqpoint{2.594961in}{1.626599in}}%
\pgfpathlineto{\pgfqpoint{2.597557in}{1.724610in}}%
\pgfpathlineto{\pgfqpoint{2.598421in}{1.643934in}}%
\pgfpathlineto{\pgfqpoint{2.600150in}{1.705613in}}%
\pgfpathlineto{\pgfqpoint{2.601015in}{1.706149in}}%
\pgfpathlineto{\pgfqpoint{2.601880in}{1.716240in}}%
\pgfpathlineto{\pgfqpoint{2.602742in}{1.652720in}}%
\pgfpathlineto{\pgfqpoint{2.603605in}{1.678306in}}%
\pgfpathlineto{\pgfqpoint{2.604469in}{1.618408in}}%
\pgfpathlineto{\pgfqpoint{2.605334in}{1.666077in}}%
\pgfpathlineto{\pgfqpoint{2.606197in}{1.659696in}}%
\pgfpathlineto{\pgfqpoint{2.607063in}{1.623811in}}%
\pgfpathlineto{\pgfqpoint{2.607928in}{1.679909in}}%
\pgfpathlineto{\pgfqpoint{2.608793in}{1.656431in}}%
\pgfpathlineto{\pgfqpoint{2.609657in}{1.700449in}}%
\pgfpathlineto{\pgfqpoint{2.611387in}{1.669579in}}%
\pgfpathlineto{\pgfqpoint{2.613117in}{1.712202in}}%
\pgfpathlineto{\pgfqpoint{2.613981in}{1.659547in}}%
\pgfpathlineto{\pgfqpoint{2.614846in}{1.685133in}}%
\pgfpathlineto{\pgfqpoint{2.615710in}{1.650315in}}%
\pgfpathlineto{\pgfqpoint{2.617439in}{1.707216in}}%
\pgfpathlineto{\pgfqpoint{2.618301in}{1.715645in}}%
\pgfpathlineto{\pgfqpoint{2.620031in}{1.657350in}}%
\pgfpathlineto{\pgfqpoint{2.621760in}{1.674149in}}%
\pgfpathlineto{\pgfqpoint{2.622625in}{1.658952in}}%
\pgfpathlineto{\pgfqpoint{2.623489in}{1.681154in}}%
\pgfpathlineto{\pgfqpoint{2.625219in}{1.591190in}}%
\pgfpathlineto{\pgfqpoint{2.626084in}{1.600835in}}%
\pgfpathlineto{\pgfqpoint{2.626950in}{1.686974in}}%
\pgfpathlineto{\pgfqpoint{2.627816in}{1.647318in}}%
\pgfpathlineto{\pgfqpoint{2.628682in}{1.652423in}}%
\pgfpathlineto{\pgfqpoint{2.630413in}{1.635564in}}%
\pgfpathlineto{\pgfqpoint{2.631279in}{1.681571in}}%
\pgfpathlineto{\pgfqpoint{2.632145in}{1.628767in}}%
\pgfpathlineto{\pgfqpoint{2.633874in}{1.669877in}}%
\pgfpathlineto{\pgfqpoint{2.634739in}{1.691573in}}%
\pgfpathlineto{\pgfqpoint{2.636470in}{1.646961in}}%
\pgfpathlineto{\pgfqpoint{2.637336in}{1.723658in}}%
\pgfpathlineto{\pgfqpoint{2.638200in}{1.681214in}}%
\pgfpathlineto{\pgfqpoint{2.639064in}{1.733099in}}%
\pgfpathlineto{\pgfqpoint{2.639929in}{1.633724in}}%
\pgfpathlineto{\pgfqpoint{2.642524in}{1.725677in}}%
\pgfpathlineto{\pgfqpoint{2.643388in}{1.694689in}}%
\pgfpathlineto{\pgfqpoint{2.644252in}{1.706740in}}%
\pgfpathlineto{\pgfqpoint{2.645115in}{1.640015in}}%
\pgfpathlineto{\pgfqpoint{2.645978in}{1.734047in}}%
\pgfpathlineto{\pgfqpoint{2.647710in}{1.694749in}}%
\pgfpathlineto{\pgfqpoint{2.648575in}{1.718434in}}%
\pgfpathlineto{\pgfqpoint{2.649441in}{1.693265in}}%
\pgfpathlineto{\pgfqpoint{2.650307in}{1.715051in}}%
\pgfpathlineto{\pgfqpoint{2.651172in}{1.669222in}}%
\pgfpathlineto{\pgfqpoint{2.652037in}{1.766936in}}%
\pgfpathlineto{\pgfqpoint{2.652902in}{1.682697in}}%
\pgfpathlineto{\pgfqpoint{2.653767in}{1.682935in}}%
\pgfpathlineto{\pgfqpoint{2.654633in}{1.686974in}}%
\pgfpathlineto{\pgfqpoint{2.655499in}{1.679314in}}%
\pgfpathlineto{\pgfqpoint{2.657231in}{1.768836in}}%
\pgfpathlineto{\pgfqpoint{2.658097in}{1.681452in}}%
\pgfpathlineto{\pgfqpoint{2.658961in}{1.689108in}}%
\pgfpathlineto{\pgfqpoint{2.659827in}{1.682638in}}%
\pgfpathlineto{\pgfqpoint{2.660689in}{1.633129in}}%
\pgfpathlineto{\pgfqpoint{2.662417in}{1.706324in}}%
\pgfpathlineto{\pgfqpoint{2.663283in}{1.672636in}}%
\pgfpathlineto{\pgfqpoint{2.664149in}{1.682876in}}%
\pgfpathlineto{\pgfqpoint{2.665014in}{1.629329in}}%
\pgfpathlineto{\pgfqpoint{2.665878in}{1.651977in}}%
\pgfpathlineto{\pgfqpoint{2.666743in}{1.648087in}}%
\pgfpathlineto{\pgfqpoint{2.669338in}{1.553758in}}%
\pgfpathlineto{\pgfqpoint{2.670204in}{1.637788in}}%
\pgfpathlineto{\pgfqpoint{2.672800in}{1.542182in}}%
\pgfpathlineto{\pgfqpoint{2.673666in}{1.592937in}}%
\pgfpathlineto{\pgfqpoint{2.674531in}{1.562960in}}%
\pgfpathlineto{\pgfqpoint{2.675397in}{1.640312in}}%
\pgfpathlineto{\pgfqpoint{2.676261in}{1.611224in}}%
\pgfpathlineto{\pgfqpoint{2.677125in}{1.670587in}}%
\pgfpathlineto{\pgfqpoint{2.677991in}{1.614191in}}%
\pgfpathlineto{\pgfqpoint{2.678857in}{1.619594in}}%
\pgfpathlineto{\pgfqpoint{2.679722in}{1.634850in}}%
\pgfpathlineto{\pgfqpoint{2.681452in}{1.721937in}}%
\pgfpathlineto{\pgfqpoint{2.683182in}{1.595075in}}%
\pgfpathlineto{\pgfqpoint{2.684047in}{1.592406in}}%
\pgfpathlineto{\pgfqpoint{2.684912in}{1.621107in}}%
\pgfpathlineto{\pgfqpoint{2.685777in}{1.614013in}}%
\pgfpathlineto{\pgfqpoint{2.686642in}{1.605524in}}%
\pgfpathlineto{\pgfqpoint{2.687507in}{1.620959in}}%
\pgfpathlineto{\pgfqpoint{2.688372in}{1.662276in}}%
\pgfpathlineto{\pgfqpoint{2.690102in}{1.595194in}}%
\pgfpathlineto{\pgfqpoint{2.690968in}{1.588725in}}%
\pgfpathlineto{\pgfqpoint{2.692698in}{1.633307in}}%
\pgfpathlineto{\pgfqpoint{2.693563in}{1.573111in}}%
\pgfpathlineto{\pgfqpoint{2.694429in}{1.623245in}}%
\pgfpathlineto{\pgfqpoint{2.695292in}{1.597451in}}%
\pgfpathlineto{\pgfqpoint{2.696157in}{1.643399in}}%
\pgfpathlineto{\pgfqpoint{2.697023in}{1.570914in}}%
\pgfpathlineto{\pgfqpoint{2.697889in}{1.578633in}}%
\pgfpathlineto{\pgfqpoint{2.698752in}{1.561001in}}%
\pgfpathlineto{\pgfqpoint{2.699618in}{1.573052in}}%
\pgfpathlineto{\pgfqpoint{2.701346in}{1.532388in}}%
\pgfpathlineto{\pgfqpoint{2.702209in}{1.587181in}}%
\pgfpathlineto{\pgfqpoint{2.703074in}{1.566404in}}%
\pgfpathlineto{\pgfqpoint{2.705670in}{1.648801in}}%
\pgfpathlineto{\pgfqpoint{2.707401in}{1.629567in}}%
\pgfpathlineto{\pgfqpoint{2.708268in}{1.537404in}}%
\pgfpathlineto{\pgfqpoint{2.709995in}{1.621018in}}%
\pgfpathlineto{\pgfqpoint{2.711726in}{1.597213in}}%
\pgfpathlineto{\pgfqpoint{2.713458in}{1.656904in}}%
\pgfpathlineto{\pgfqpoint{2.714324in}{1.656104in}}%
\pgfpathlineto{\pgfqpoint{2.715189in}{1.623275in}}%
\pgfpathlineto{\pgfqpoint{2.716054in}{1.637817in}}%
\pgfpathlineto{\pgfqpoint{2.716920in}{1.727102in}}%
\pgfpathlineto{\pgfqpoint{2.718650in}{1.574416in}}%
\pgfpathlineto{\pgfqpoint{2.719515in}{1.592465in}}%
\pgfpathlineto{\pgfqpoint{2.720379in}{1.571449in}}%
\pgfpathlineto{\pgfqpoint{2.721245in}{1.648444in}}%
\pgfpathlineto{\pgfqpoint{2.722110in}{1.627904in}}%
\pgfpathlineto{\pgfqpoint{2.722975in}{1.600805in}}%
\pgfpathlineto{\pgfqpoint{2.723839in}{1.680857in}}%
\pgfpathlineto{\pgfqpoint{2.725571in}{1.587241in}}%
\pgfpathlineto{\pgfqpoint{2.726436in}{1.611343in}}%
\pgfpathlineto{\pgfqpoint{2.727301in}{1.530816in}}%
\pgfpathlineto{\pgfqpoint{2.728166in}{1.533932in}}%
\pgfpathlineto{\pgfqpoint{2.729031in}{1.543845in}}%
\pgfpathlineto{\pgfqpoint{2.729896in}{1.588368in}}%
\pgfpathlineto{\pgfqpoint{2.730761in}{1.561417in}}%
\pgfpathlineto{\pgfqpoint{2.731626in}{1.610391in}}%
\pgfpathlineto{\pgfqpoint{2.732492in}{1.540401in}}%
\pgfpathlineto{\pgfqpoint{2.733357in}{1.632831in}}%
\pgfpathlineto{\pgfqpoint{2.734222in}{1.621554in}}%
\pgfpathlineto{\pgfqpoint{2.735086in}{1.636691in}}%
\pgfpathlineto{\pgfqpoint{2.735950in}{1.555896in}}%
\pgfpathlineto{\pgfqpoint{2.736813in}{1.636126in}}%
\pgfpathlineto{\pgfqpoint{2.737677in}{1.632891in}}%
\pgfpathlineto{\pgfqpoint{2.738539in}{1.545388in}}%
\pgfpathlineto{\pgfqpoint{2.739404in}{1.594391in}}%
\pgfpathlineto{\pgfqpoint{2.740269in}{1.584032in}}%
\pgfpathlineto{\pgfqpoint{2.741132in}{1.581303in}}%
\pgfpathlineto{\pgfqpoint{2.741997in}{1.601635in}}%
\pgfpathlineto{\pgfqpoint{2.742860in}{1.593175in}}%
\pgfpathlineto{\pgfqpoint{2.744591in}{1.651888in}}%
\pgfpathlineto{\pgfqpoint{2.745455in}{1.665422in}}%
\pgfpathlineto{\pgfqpoint{2.746322in}{1.583381in}}%
\pgfpathlineto{\pgfqpoint{2.747187in}{1.597094in}}%
\pgfpathlineto{\pgfqpoint{2.748053in}{1.559458in}}%
\pgfpathlineto{\pgfqpoint{2.750649in}{1.628023in}}%
\pgfpathlineto{\pgfqpoint{2.751515in}{1.545269in}}%
\pgfpathlineto{\pgfqpoint{2.753245in}{1.638353in}}%
\pgfpathlineto{\pgfqpoint{2.754111in}{1.564028in}}%
\pgfpathlineto{\pgfqpoint{2.755840in}{1.633277in}}%
\pgfpathlineto{\pgfqpoint{2.757572in}{1.581422in}}%
\pgfpathlineto{\pgfqpoint{2.758437in}{1.572814in}}%
\pgfpathlineto{\pgfqpoint{2.759302in}{1.651055in}}%
\pgfpathlineto{\pgfqpoint{2.760166in}{1.646306in}}%
\pgfpathlineto{\pgfqpoint{2.761898in}{1.504784in}}%
\pgfpathlineto{\pgfqpoint{2.764491in}{1.601307in}}%
\pgfpathlineto{\pgfqpoint{2.765356in}{1.560227in}}%
\pgfpathlineto{\pgfqpoint{2.766220in}{1.568657in}}%
\pgfpathlineto{\pgfqpoint{2.767085in}{1.562365in}}%
\pgfpathlineto{\pgfqpoint{2.767951in}{1.640636in}}%
\pgfpathlineto{\pgfqpoint{2.769681in}{1.534404in}}%
\pgfpathlineto{\pgfqpoint{2.770546in}{1.609707in}}%
\pgfpathlineto{\pgfqpoint{2.771411in}{1.546633in}}%
\pgfpathlineto{\pgfqpoint{2.772275in}{1.628615in}}%
\pgfpathlineto{\pgfqpoint{2.773139in}{1.548890in}}%
\pgfpathlineto{\pgfqpoint{2.774004in}{1.602438in}}%
\pgfpathlineto{\pgfqpoint{2.774866in}{1.536809in}}%
\pgfpathlineto{\pgfqpoint{2.775728in}{1.607008in}}%
\pgfpathlineto{\pgfqpoint{2.776593in}{1.577030in}}%
\pgfpathlineto{\pgfqpoint{2.779188in}{1.648682in}}%
\pgfpathlineto{\pgfqpoint{2.780052in}{1.557528in}}%
\pgfpathlineto{\pgfqpoint{2.780917in}{1.626837in}}%
\pgfpathlineto{\pgfqpoint{2.781780in}{1.595849in}}%
\pgfpathlineto{\pgfqpoint{2.782645in}{1.635029in}}%
\pgfpathlineto{\pgfqpoint{2.784371in}{1.566463in}}%
\pgfpathlineto{\pgfqpoint{2.786101in}{1.666965in}}%
\pgfpathlineto{\pgfqpoint{2.786965in}{1.542272in}}%
\pgfpathlineto{\pgfqpoint{2.787828in}{1.643339in}}%
\pgfpathlineto{\pgfqpoint{2.788692in}{1.628737in}}%
\pgfpathlineto{\pgfqpoint{2.789556in}{1.621375in}}%
\pgfpathlineto{\pgfqpoint{2.791286in}{1.555185in}}%
\pgfpathlineto{\pgfqpoint{2.793017in}{1.617396in}}%
\pgfpathlineto{\pgfqpoint{2.793883in}{1.576911in}}%
\pgfpathlineto{\pgfqpoint{2.794748in}{1.612767in}}%
\pgfpathlineto{\pgfqpoint{2.795613in}{1.532448in}}%
\pgfpathlineto{\pgfqpoint{2.796478in}{1.542420in}}%
\pgfpathlineto{\pgfqpoint{2.797343in}{1.623097in}}%
\pgfpathlineto{\pgfqpoint{2.799072in}{1.541974in}}%
\pgfpathlineto{\pgfqpoint{2.800803in}{1.623037in}}%
\pgfpathlineto{\pgfqpoint{2.801665in}{1.549723in}}%
\pgfpathlineto{\pgfqpoint{2.803395in}{1.601162in}}%
\pgfpathlineto{\pgfqpoint{2.804259in}{1.582433in}}%
\pgfpathlineto{\pgfqpoint{2.805989in}{1.631496in}}%
\pgfpathlineto{\pgfqpoint{2.806856in}{1.626242in}}%
\pgfpathlineto{\pgfqpoint{2.807721in}{1.641975in}}%
\pgfpathlineto{\pgfqpoint{2.809451in}{1.607959in}}%
\pgfpathlineto{\pgfqpoint{2.810316in}{1.642034in}}%
\pgfpathlineto{\pgfqpoint{2.811181in}{1.609681in}}%
\pgfpathlineto{\pgfqpoint{2.812047in}{1.647556in}}%
\pgfpathlineto{\pgfqpoint{2.814641in}{1.568125in}}%
\pgfpathlineto{\pgfqpoint{2.816368in}{1.610748in}}%
\pgfpathlineto{\pgfqpoint{2.818096in}{1.571449in}}%
\pgfpathlineto{\pgfqpoint{2.818961in}{1.589673in}}%
\pgfpathlineto{\pgfqpoint{2.819827in}{1.546276in}}%
\pgfpathlineto{\pgfqpoint{2.821555in}{1.630812in}}%
\pgfpathlineto{\pgfqpoint{2.822421in}{1.618672in}}%
\pgfpathlineto{\pgfqpoint{2.823283in}{1.622978in}}%
\pgfpathlineto{\pgfqpoint{2.824146in}{1.646425in}}%
\pgfpathlineto{\pgfqpoint{2.825008in}{1.576108in}}%
\pgfpathlineto{\pgfqpoint{2.825873in}{1.610272in}}%
\pgfpathlineto{\pgfqpoint{2.826739in}{1.605345in}}%
\pgfpathlineto{\pgfqpoint{2.827603in}{1.598191in}}%
\pgfpathlineto{\pgfqpoint{2.828468in}{1.574179in}}%
\pgfpathlineto{\pgfqpoint{2.829334in}{1.625469in}}%
\pgfpathlineto{\pgfqpoint{2.830199in}{1.600419in}}%
\pgfpathlineto{\pgfqpoint{2.831927in}{1.648563in}}%
\pgfpathlineto{\pgfqpoint{2.833654in}{1.580176in}}%
\pgfpathlineto{\pgfqpoint{2.834521in}{1.626064in}}%
\pgfpathlineto{\pgfqpoint{2.837117in}{1.564890in}}%
\pgfpathlineto{\pgfqpoint{2.837982in}{1.681809in}}%
\pgfpathlineto{\pgfqpoint{2.839712in}{1.585043in}}%
\pgfpathlineto{\pgfqpoint{2.841442in}{1.614132in}}%
\pgfpathlineto{\pgfqpoint{2.842307in}{1.517485in}}%
\pgfpathlineto{\pgfqpoint{2.843172in}{1.586170in}}%
\pgfpathlineto{\pgfqpoint{2.844037in}{1.562336in}}%
\pgfpathlineto{\pgfqpoint{2.845764in}{1.636096in}}%
\pgfpathlineto{\pgfqpoint{2.846628in}{1.606799in}}%
\pgfpathlineto{\pgfqpoint{2.847492in}{1.632891in}}%
\pgfpathlineto{\pgfqpoint{2.848356in}{1.626183in}}%
\pgfpathlineto{\pgfqpoint{2.849221in}{1.610927in}}%
\pgfpathlineto{\pgfqpoint{2.851815in}{1.696351in}}%
\pgfpathlineto{\pgfqpoint{2.852680in}{1.621375in}}%
\pgfpathlineto{\pgfqpoint{2.853545in}{1.631377in}}%
\pgfpathlineto{\pgfqpoint{2.855272in}{1.698430in}}%
\pgfpathlineto{\pgfqpoint{2.856137in}{1.659101in}}%
\pgfpathlineto{\pgfqpoint{2.857003in}{1.662871in}}%
\pgfpathlineto{\pgfqpoint{2.857868in}{1.700032in}}%
\pgfpathlineto{\pgfqpoint{2.858734in}{1.628559in}}%
\pgfpathlineto{\pgfqpoint{2.859599in}{1.702940in}}%
\pgfpathlineto{\pgfqpoint{2.860465in}{1.633218in}}%
\pgfpathlineto{\pgfqpoint{2.862193in}{1.710897in}}%
\pgfpathlineto{\pgfqpoint{2.863921in}{1.657350in}}%
\pgfpathlineto{\pgfqpoint{2.864787in}{1.722770in}}%
\pgfpathlineto{\pgfqpoint{2.865652in}{1.693622in}}%
\pgfpathlineto{\pgfqpoint{2.866517in}{1.726153in}}%
\pgfpathlineto{\pgfqpoint{2.868246in}{1.693503in}}%
\pgfpathlineto{\pgfqpoint{2.869111in}{1.732921in}}%
\pgfpathlineto{\pgfqpoint{2.869976in}{1.659458in}}%
\pgfpathlineto{\pgfqpoint{2.870840in}{1.669936in}}%
\pgfpathlineto{\pgfqpoint{2.871704in}{1.664652in}}%
\pgfpathlineto{\pgfqpoint{2.873433in}{1.698608in}}%
\pgfpathlineto{\pgfqpoint{2.874296in}{1.677533in}}%
\pgfpathlineto{\pgfqpoint{2.875161in}{1.720275in}}%
\pgfpathlineto{\pgfqpoint{2.876025in}{1.545864in}}%
\pgfpathlineto{\pgfqpoint{2.876888in}{1.578276in}}%
\pgfpathlineto{\pgfqpoint{2.877753in}{1.571152in}}%
\pgfpathlineto{\pgfqpoint{2.878619in}{1.590327in}}%
\pgfpathlineto{\pgfqpoint{2.879484in}{1.547079in}}%
\pgfpathlineto{\pgfqpoint{2.880348in}{1.557022in}}%
\pgfpathlineto{\pgfqpoint{2.881212in}{1.648444in}}%
\pgfpathlineto{\pgfqpoint{2.882076in}{1.637639in}}%
\pgfpathlineto{\pgfqpoint{2.882941in}{1.609146in}}%
\pgfpathlineto{\pgfqpoint{2.883803in}{1.648325in}}%
\pgfpathlineto{\pgfqpoint{2.885532in}{1.471419in}}%
\pgfpathlineto{\pgfqpoint{2.887263in}{1.631050in}}%
\pgfpathlineto{\pgfqpoint{2.888127in}{1.630340in}}%
\pgfpathlineto{\pgfqpoint{2.889857in}{1.587836in}}%
\pgfpathlineto{\pgfqpoint{2.890721in}{1.603271in}}%
\pgfpathlineto{\pgfqpoint{2.891587in}{1.652810in}}%
\pgfpathlineto{\pgfqpoint{2.892452in}{1.650761in}}%
\pgfpathlineto{\pgfqpoint{2.893318in}{1.530667in}}%
\pgfpathlineto{\pgfqpoint{2.895047in}{1.597094in}}%
\pgfpathlineto{\pgfqpoint{2.895913in}{1.586289in}}%
\pgfpathlineto{\pgfqpoint{2.896779in}{1.600776in}}%
\pgfpathlineto{\pgfqpoint{2.898505in}{1.578217in}}%
\pgfpathlineto{\pgfqpoint{2.899368in}{1.584805in}}%
\pgfpathlineto{\pgfqpoint{2.900234in}{1.519564in}}%
\pgfpathlineto{\pgfqpoint{2.901964in}{1.700980in}}%
\pgfpathlineto{\pgfqpoint{2.902827in}{1.577562in}}%
\pgfpathlineto{\pgfqpoint{2.903693in}{1.603118in}}%
\pgfpathlineto{\pgfqpoint{2.904557in}{1.632058in}}%
\pgfpathlineto{\pgfqpoint{2.906286in}{1.540044in}}%
\pgfpathlineto{\pgfqpoint{2.909746in}{1.660376in}}%
\pgfpathlineto{\pgfqpoint{2.910611in}{1.618940in}}%
\pgfpathlineto{\pgfqpoint{2.911476in}{1.684776in}}%
\pgfpathlineto{\pgfqpoint{2.912341in}{1.610570in}}%
\pgfpathlineto{\pgfqpoint{2.913205in}{1.634493in}}%
\pgfpathlineto{\pgfqpoint{2.914071in}{1.579522in}}%
\pgfpathlineto{\pgfqpoint{2.914936in}{1.615377in}}%
\pgfpathlineto{\pgfqpoint{2.915801in}{1.539747in}}%
\pgfpathlineto{\pgfqpoint{2.916667in}{1.584330in}}%
\pgfpathlineto{\pgfqpoint{2.917532in}{1.688602in}}%
\pgfpathlineto{\pgfqpoint{2.920128in}{1.576049in}}%
\pgfpathlineto{\pgfqpoint{2.921858in}{1.635798in}}%
\pgfpathlineto{\pgfqpoint{2.922722in}{1.567471in}}%
\pgfpathlineto{\pgfqpoint{2.923588in}{1.648325in}}%
\pgfpathlineto{\pgfqpoint{2.926183in}{1.574892in}}%
\pgfpathlineto{\pgfqpoint{2.927911in}{1.622502in}}%
\pgfpathlineto{\pgfqpoint{2.928776in}{1.632355in}}%
\pgfpathlineto{\pgfqpoint{2.931370in}{1.737015in}}%
\pgfpathlineto{\pgfqpoint{2.932235in}{1.743901in}}%
\pgfpathlineto{\pgfqpoint{2.933100in}{1.724015in}}%
\pgfpathlineto{\pgfqpoint{2.933966in}{1.677473in}}%
\pgfpathlineto{\pgfqpoint{2.934831in}{1.708640in}}%
\pgfpathlineto{\pgfqpoint{2.936558in}{1.674179in}}%
\pgfpathlineto{\pgfqpoint{2.937423in}{1.727161in}}%
\pgfpathlineto{\pgfqpoint{2.938287in}{1.718970in}}%
\pgfpathlineto{\pgfqpoint{2.941744in}{1.682935in}}%
\pgfpathlineto{\pgfqpoint{2.942610in}{1.700389in}}%
\pgfpathlineto{\pgfqpoint{2.944340in}{1.648325in}}%
\pgfpathlineto{\pgfqpoint{2.945203in}{1.710778in}}%
\pgfpathlineto{\pgfqpoint{2.946068in}{1.680236in}}%
\pgfpathlineto{\pgfqpoint{2.947799in}{1.702765in}}%
\pgfpathlineto{\pgfqpoint{2.948660in}{1.646697in}}%
\pgfpathlineto{\pgfqpoint{2.949527in}{1.711314in}}%
\pgfpathlineto{\pgfqpoint{2.951255in}{1.546221in}}%
\pgfpathlineto{\pgfqpoint{2.952121in}{1.618467in}}%
\pgfpathlineto{\pgfqpoint{2.953848in}{1.528826in}}%
\pgfpathlineto{\pgfqpoint{2.956440in}{1.656283in}}%
\pgfpathlineto{\pgfqpoint{2.957306in}{1.539513in}}%
\pgfpathlineto{\pgfqpoint{2.958170in}{1.570442in}}%
\pgfpathlineto{\pgfqpoint{2.959036in}{1.537732in}}%
\pgfpathlineto{\pgfqpoint{2.959901in}{1.575130in}}%
\pgfpathlineto{\pgfqpoint{2.960766in}{1.532478in}}%
\pgfpathlineto{\pgfqpoint{2.962497in}{1.632240in}}%
\pgfpathlineto{\pgfqpoint{2.964228in}{1.541766in}}%
\pgfpathlineto{\pgfqpoint{2.965957in}{1.593770in}}%
\pgfpathlineto{\pgfqpoint{2.966821in}{1.613303in}}%
\pgfpathlineto{\pgfqpoint{2.967687in}{1.544112in}}%
\pgfpathlineto{\pgfqpoint{2.968551in}{1.601014in}}%
\pgfpathlineto{\pgfqpoint{2.970278in}{1.549783in}}%
\pgfpathlineto{\pgfqpoint{2.971143in}{1.642212in}}%
\pgfpathlineto{\pgfqpoint{2.972009in}{1.635267in}}%
\pgfpathlineto{\pgfqpoint{2.973738in}{1.590803in}}%
\pgfpathlineto{\pgfqpoint{2.974603in}{1.607900in}}%
\pgfpathlineto{\pgfqpoint{2.975468in}{1.573349in}}%
\pgfpathlineto{\pgfqpoint{2.976333in}{1.619832in}}%
\pgfpathlineto{\pgfqpoint{2.977198in}{1.580444in}}%
\pgfpathlineto{\pgfqpoint{2.978062in}{1.583857in}}%
\pgfpathlineto{\pgfqpoint{2.978927in}{1.607483in}}%
\pgfpathlineto{\pgfqpoint{2.979791in}{1.535356in}}%
\pgfpathlineto{\pgfqpoint{2.980655in}{1.614429in}}%
\pgfpathlineto{\pgfqpoint{2.981520in}{1.535534in}}%
\pgfpathlineto{\pgfqpoint{2.982384in}{1.582195in}}%
\pgfpathlineto{\pgfqpoint{2.983249in}{1.568185in}}%
\pgfpathlineto{\pgfqpoint{2.984114in}{1.579730in}}%
\pgfpathlineto{\pgfqpoint{2.984979in}{1.561239in}}%
\pgfpathlineto{\pgfqpoint{2.985844in}{1.586943in}}%
\pgfpathlineto{\pgfqpoint{2.986710in}{1.560703in}}%
\pgfpathlineto{\pgfqpoint{2.988440in}{1.602319in}}%
\pgfpathlineto{\pgfqpoint{2.989305in}{1.560882in}}%
\pgfpathlineto{\pgfqpoint{2.990170in}{1.679314in}}%
\pgfpathlineto{\pgfqpoint{2.991898in}{1.545626in}}%
\pgfpathlineto{\pgfqpoint{2.993628in}{1.629656in}}%
\pgfpathlineto{\pgfqpoint{2.994494in}{1.629150in}}%
\pgfpathlineto{\pgfqpoint{2.995360in}{1.637579in}}%
\pgfpathlineto{\pgfqpoint{2.996226in}{1.611756in}}%
\pgfpathlineto{\pgfqpoint{2.997090in}{1.627726in}}%
\pgfpathlineto{\pgfqpoint{2.997954in}{1.617396in}}%
\pgfpathlineto{\pgfqpoint{2.998820in}{1.624134in}}%
\pgfpathlineto{\pgfqpoint{2.999686in}{1.595432in}}%
\pgfpathlineto{\pgfqpoint{3.000551in}{1.619475in}}%
\pgfpathlineto{\pgfqpoint{3.002282in}{1.589851in}}%
\pgfpathlineto{\pgfqpoint{3.004010in}{1.620721in}}%
\pgfpathlineto{\pgfqpoint{3.004875in}{1.649512in}}%
\pgfpathlineto{\pgfqpoint{3.006606in}{1.583084in}}%
\pgfpathlineto{\pgfqpoint{3.007470in}{1.676109in}}%
\pgfpathlineto{\pgfqpoint{3.008335in}{1.643547in}}%
\pgfpathlineto{\pgfqpoint{3.009201in}{1.699556in}}%
\pgfpathlineto{\pgfqpoint{3.010065in}{1.650761in}}%
\pgfpathlineto{\pgfqpoint{3.010930in}{1.693503in}}%
\pgfpathlineto{\pgfqpoint{3.011793in}{1.669579in}}%
\pgfpathlineto{\pgfqpoint{3.014390in}{1.712381in}}%
\pgfpathlineto{\pgfqpoint{3.015255in}{1.715467in}}%
\pgfpathlineto{\pgfqpoint{3.016120in}{1.647318in}}%
\pgfpathlineto{\pgfqpoint{3.016985in}{1.657588in}}%
\pgfpathlineto{\pgfqpoint{3.018714in}{1.706502in}}%
\pgfpathlineto{\pgfqpoint{3.019578in}{1.708700in}}%
\pgfpathlineto{\pgfqpoint{3.021309in}{1.660495in}}%
\pgfpathlineto{\pgfqpoint{3.022174in}{1.719386in}}%
\pgfpathlineto{\pgfqpoint{3.024765in}{1.650136in}}%
\pgfpathlineto{\pgfqpoint{3.026495in}{1.731969in}}%
\pgfpathlineto{\pgfqpoint{3.027361in}{1.728436in}}%
\pgfpathlineto{\pgfqpoint{3.029093in}{1.675990in}}%
\pgfpathlineto{\pgfqpoint{3.029959in}{1.725499in}}%
\pgfpathlineto{\pgfqpoint{3.031687in}{1.615913in}}%
\pgfpathlineto{\pgfqpoint{3.032552in}{1.694749in}}%
\pgfpathlineto{\pgfqpoint{3.033415in}{1.655688in}}%
\pgfpathlineto{\pgfqpoint{3.034281in}{1.731880in}}%
\pgfpathlineto{\pgfqpoint{3.035146in}{1.684181in}}%
\pgfpathlineto{\pgfqpoint{3.036011in}{1.773465in}}%
\pgfpathlineto{\pgfqpoint{3.038605in}{1.540401in}}%
\pgfpathlineto{\pgfqpoint{3.040334in}{1.616802in}}%
\pgfpathlineto{\pgfqpoint{3.041199in}{1.709826in}}%
\pgfpathlineto{\pgfqpoint{3.042064in}{1.678778in}}%
\pgfpathlineto{\pgfqpoint{3.042930in}{1.709469in}}%
\pgfpathlineto{\pgfqpoint{3.043794in}{1.654264in}}%
\pgfpathlineto{\pgfqpoint{3.045523in}{1.715407in}}%
\pgfpathlineto{\pgfqpoint{3.046388in}{1.682697in}}%
\pgfpathlineto{\pgfqpoint{3.047252in}{1.713269in}}%
\pgfpathlineto{\pgfqpoint{3.048117in}{1.620661in}}%
\pgfpathlineto{\pgfqpoint{3.048983in}{1.708164in}}%
\pgfpathlineto{\pgfqpoint{3.049848in}{1.687327in}}%
\pgfpathlineto{\pgfqpoint{3.050713in}{1.707629in}}%
\pgfpathlineto{\pgfqpoint{3.051578in}{1.644466in}}%
\pgfpathlineto{\pgfqpoint{3.052443in}{1.702821in}}%
\pgfpathlineto{\pgfqpoint{3.053308in}{1.699140in}}%
\pgfpathlineto{\pgfqpoint{3.054173in}{1.726685in}}%
\pgfpathlineto{\pgfqpoint{3.055037in}{1.642744in}}%
\pgfpathlineto{\pgfqpoint{3.056766in}{1.695697in}}%
\pgfpathlineto{\pgfqpoint{3.057631in}{1.695756in}}%
\pgfpathlineto{\pgfqpoint{3.058495in}{1.767289in}}%
\pgfpathlineto{\pgfqpoint{3.061953in}{1.646663in}}%
\pgfpathlineto{\pgfqpoint{3.062818in}{1.721699in}}%
\pgfpathlineto{\pgfqpoint{3.063682in}{1.682757in}}%
\pgfpathlineto{\pgfqpoint{3.064545in}{1.689346in}}%
\pgfpathlineto{\pgfqpoint{3.065410in}{1.663939in}}%
\pgfpathlineto{\pgfqpoint{3.066275in}{1.670527in}}%
\pgfpathlineto{\pgfqpoint{3.067139in}{1.719680in}}%
\pgfpathlineto{\pgfqpoint{3.068004in}{1.701337in}}%
\pgfpathlineto{\pgfqpoint{3.068870in}{1.632950in}}%
\pgfpathlineto{\pgfqpoint{3.071465in}{1.714872in}}%
\pgfpathlineto{\pgfqpoint{3.072330in}{1.704900in}}%
\pgfpathlineto{\pgfqpoint{3.074059in}{1.765571in}}%
\pgfpathlineto{\pgfqpoint{3.074926in}{1.748028in}}%
\pgfpathlineto{\pgfqpoint{3.075790in}{1.677771in}}%
\pgfpathlineto{\pgfqpoint{3.076655in}{1.679730in}}%
\pgfpathlineto{\pgfqpoint{3.078385in}{1.694927in}}%
\pgfpathlineto{\pgfqpoint{3.080116in}{1.666846in}}%
\pgfpathlineto{\pgfqpoint{3.081847in}{1.728942in}}%
\pgfpathlineto{\pgfqpoint{3.082712in}{1.725261in}}%
\pgfpathlineto{\pgfqpoint{3.087036in}{1.586408in}}%
\pgfpathlineto{\pgfqpoint{3.088767in}{1.666936in}}%
\pgfpathlineto{\pgfqpoint{3.090498in}{1.548295in}}%
\pgfpathlineto{\pgfqpoint{3.092227in}{1.673376in}}%
\pgfpathlineto{\pgfqpoint{3.093092in}{1.656279in}}%
\pgfpathlineto{\pgfqpoint{3.093957in}{1.580827in}}%
\pgfpathlineto{\pgfqpoint{3.094822in}{1.613537in}}%
\pgfpathlineto{\pgfqpoint{3.095687in}{1.599824in}}%
\pgfpathlineto{\pgfqpoint{3.096552in}{1.618464in}}%
\pgfpathlineto{\pgfqpoint{3.097417in}{1.579581in}}%
\pgfpathlineto{\pgfqpoint{3.098282in}{1.585545in}}%
\pgfpathlineto{\pgfqpoint{3.099147in}{1.581243in}}%
\pgfpathlineto{\pgfqpoint{3.100878in}{1.612351in}}%
\pgfpathlineto{\pgfqpoint{3.101742in}{1.608729in}}%
\pgfpathlineto{\pgfqpoint{3.102605in}{1.613596in}}%
\pgfpathlineto{\pgfqpoint{3.104334in}{1.675514in}}%
\pgfpathlineto{\pgfqpoint{3.106065in}{1.583203in}}%
\pgfpathlineto{\pgfqpoint{3.106930in}{1.616032in}}%
\pgfpathlineto{\pgfqpoint{3.107795in}{1.576108in}}%
\pgfpathlineto{\pgfqpoint{3.108661in}{1.637758in}}%
\pgfpathlineto{\pgfqpoint{3.110391in}{1.586051in}}%
\pgfpathlineto{\pgfqpoint{3.111255in}{1.598400in}}%
\pgfpathlineto{\pgfqpoint{3.113850in}{1.547641in}}%
\pgfpathlineto{\pgfqpoint{3.114714in}{1.563790in}}%
\pgfpathlineto{\pgfqpoint{3.115579in}{1.610391in}}%
\pgfpathlineto{\pgfqpoint{3.117306in}{1.519058in}}%
\pgfpathlineto{\pgfqpoint{3.118172in}{1.519207in}}%
\pgfpathlineto{\pgfqpoint{3.121631in}{1.631407in}}%
\pgfpathlineto{\pgfqpoint{3.124228in}{1.554412in}}%
\pgfpathlineto{\pgfqpoint{3.125092in}{1.564801in}}%
\pgfpathlineto{\pgfqpoint{3.125956in}{1.628321in}}%
\pgfpathlineto{\pgfqpoint{3.128550in}{1.524435in}}%
\pgfpathlineto{\pgfqpoint{3.129414in}{1.619743in}}%
\pgfpathlineto{\pgfqpoint{3.132007in}{1.530697in}}%
\pgfpathlineto{\pgfqpoint{3.132874in}{1.622918in}}%
\pgfpathlineto{\pgfqpoint{3.133740in}{1.614310in}}%
\pgfpathlineto{\pgfqpoint{3.134605in}{1.557677in}}%
\pgfpathlineto{\pgfqpoint{3.135471in}{1.585519in}}%
\pgfpathlineto{\pgfqpoint{3.136337in}{1.549366in}}%
\pgfpathlineto{\pgfqpoint{3.138066in}{1.646961in}}%
\pgfpathlineto{\pgfqpoint{3.139795in}{1.587538in}}%
\pgfpathlineto{\pgfqpoint{3.141527in}{1.551118in}}%
\pgfpathlineto{\pgfqpoint{3.142392in}{1.604397in}}%
\pgfpathlineto{\pgfqpoint{3.143259in}{1.591632in}}%
\pgfpathlineto{\pgfqpoint{3.144124in}{1.583143in}}%
\pgfpathlineto{\pgfqpoint{3.144990in}{1.593651in}}%
\pgfpathlineto{\pgfqpoint{3.145857in}{1.569757in}}%
\pgfpathlineto{\pgfqpoint{3.146723in}{1.768241in}}%
\pgfpathlineto{\pgfqpoint{3.148455in}{1.674714in}}%
\pgfpathlineto{\pgfqpoint{3.149320in}{1.662395in}}%
\pgfpathlineto{\pgfqpoint{3.150186in}{1.708878in}}%
\pgfpathlineto{\pgfqpoint{3.151050in}{1.636542in}}%
\pgfpathlineto{\pgfqpoint{3.151916in}{1.731378in}}%
\pgfpathlineto{\pgfqpoint{3.152782in}{1.710600in}}%
\pgfpathlineto{\pgfqpoint{3.154510in}{1.685133in}}%
\pgfpathlineto{\pgfqpoint{3.155373in}{1.695403in}}%
\pgfpathlineto{\pgfqpoint{3.156236in}{1.731259in}}%
\pgfpathlineto{\pgfqpoint{3.157967in}{1.662544in}}%
\pgfpathlineto{\pgfqpoint{3.158829in}{1.730664in}}%
\pgfpathlineto{\pgfqpoint{3.160561in}{1.696768in}}%
\pgfpathlineto{\pgfqpoint{3.161425in}{1.698370in}}%
\pgfpathlineto{\pgfqpoint{3.162292in}{1.682995in}}%
\pgfpathlineto{\pgfqpoint{3.163156in}{1.689941in}}%
\pgfpathlineto{\pgfqpoint{3.164021in}{1.805699in}}%
\pgfpathlineto{\pgfqpoint{3.164886in}{1.689881in}}%
\pgfpathlineto{\pgfqpoint{3.166615in}{1.797686in}}%
\pgfpathlineto{\pgfqpoint{3.167480in}{1.676376in}}%
\pgfpathlineto{\pgfqpoint{3.168345in}{1.727637in}}%
\pgfpathlineto{\pgfqpoint{3.169210in}{1.715645in}}%
\pgfpathlineto{\pgfqpoint{3.170075in}{1.681482in}}%
\pgfpathlineto{\pgfqpoint{3.172667in}{1.721937in}}%
\pgfpathlineto{\pgfqpoint{3.173530in}{1.569787in}}%
\pgfpathlineto{\pgfqpoint{3.174393in}{1.598162in}}%
\pgfpathlineto{\pgfqpoint{3.175258in}{1.628853in}}%
\pgfpathlineto{\pgfqpoint{3.176124in}{1.620066in}}%
\pgfpathlineto{\pgfqpoint{3.176989in}{1.551709in}}%
\pgfpathlineto{\pgfqpoint{3.177853in}{1.635858in}}%
\pgfpathlineto{\pgfqpoint{3.178719in}{1.576138in}}%
\pgfpathlineto{\pgfqpoint{3.179583in}{1.583114in}}%
\pgfpathlineto{\pgfqpoint{3.180447in}{1.584865in}}%
\pgfpathlineto{\pgfqpoint{3.181311in}{1.581779in}}%
\pgfpathlineto{\pgfqpoint{3.182175in}{1.583084in}}%
\pgfpathlineto{\pgfqpoint{3.183039in}{1.607364in}}%
\pgfpathlineto{\pgfqpoint{3.184769in}{1.551207in}}%
\pgfpathlineto{\pgfqpoint{3.185635in}{1.564266in}}%
\pgfpathlineto{\pgfqpoint{3.186500in}{1.600210in}}%
\pgfpathlineto{\pgfqpoint{3.187365in}{1.583203in}}%
\pgfpathlineto{\pgfqpoint{3.188230in}{1.658179in}}%
\pgfpathlineto{\pgfqpoint{3.189960in}{1.554055in}}%
\pgfpathlineto{\pgfqpoint{3.190825in}{1.638349in}}%
\pgfpathlineto{\pgfqpoint{3.191688in}{1.623331in}}%
\pgfpathlineto{\pgfqpoint{3.192553in}{1.622145in}}%
\pgfpathlineto{\pgfqpoint{3.193417in}{1.543428in}}%
\pgfpathlineto{\pgfqpoint{3.196006in}{1.620661in}}%
\pgfpathlineto{\pgfqpoint{3.197734in}{1.655331in}}%
\pgfpathlineto{\pgfqpoint{3.200327in}{1.574416in}}%
\pgfpathlineto{\pgfqpoint{3.201193in}{1.575662in}}%
\pgfpathlineto{\pgfqpoint{3.202058in}{1.563730in}}%
\pgfpathlineto{\pgfqpoint{3.204651in}{1.627488in}}%
\pgfpathlineto{\pgfqpoint{3.205515in}{1.625380in}}%
\pgfpathlineto{\pgfqpoint{3.206381in}{1.622264in}}%
\pgfpathlineto{\pgfqpoint{3.207245in}{1.585222in}}%
\pgfpathlineto{\pgfqpoint{3.208109in}{1.687892in}}%
\pgfpathlineto{\pgfqpoint{3.208975in}{1.560941in}}%
\pgfpathlineto{\pgfqpoint{3.210707in}{1.614072in}}%
\pgfpathlineto{\pgfqpoint{3.211573in}{1.596857in}}%
\pgfpathlineto{\pgfqpoint{3.212439in}{1.677533in}}%
\pgfpathlineto{\pgfqpoint{3.213304in}{1.569490in}}%
\pgfpathlineto{\pgfqpoint{3.214169in}{1.627131in}}%
\pgfpathlineto{\pgfqpoint{3.215032in}{1.603743in}}%
\pgfpathlineto{\pgfqpoint{3.215896in}{1.541707in}}%
\pgfpathlineto{\pgfqpoint{3.216762in}{1.613953in}}%
\pgfpathlineto{\pgfqpoint{3.217627in}{1.596529in}}%
\pgfpathlineto{\pgfqpoint{3.218492in}{1.630872in}}%
\pgfpathlineto{\pgfqpoint{3.219357in}{1.617515in}}%
\pgfpathlineto{\pgfqpoint{3.220222in}{1.569192in}}%
\pgfpathlineto{\pgfqpoint{3.221087in}{1.610510in}}%
\pgfpathlineto{\pgfqpoint{3.221951in}{1.606383in}}%
\pgfpathlineto{\pgfqpoint{3.222816in}{1.558327in}}%
\pgfpathlineto{\pgfqpoint{3.224547in}{1.640844in}}%
\pgfpathlineto{\pgfqpoint{3.225412in}{1.508699in}}%
\pgfpathlineto{\pgfqpoint{3.226276in}{1.523305in}}%
\pgfpathlineto{\pgfqpoint{3.227141in}{1.574119in}}%
\pgfpathlineto{\pgfqpoint{3.228006in}{1.572516in}}%
\pgfpathlineto{\pgfqpoint{3.228872in}{1.560584in}}%
\pgfpathlineto{\pgfqpoint{3.230605in}{1.679373in}}%
\pgfpathlineto{\pgfqpoint{3.234064in}{1.528321in}}%
\pgfpathlineto{\pgfqpoint{3.235794in}{1.666549in}}%
\pgfpathlineto{\pgfqpoint{3.236660in}{1.635947in}}%
\pgfpathlineto{\pgfqpoint{3.237525in}{1.565809in}}%
\pgfpathlineto{\pgfqpoint{3.238390in}{1.589970in}}%
\pgfpathlineto{\pgfqpoint{3.239254in}{1.531258in}}%
\pgfpathlineto{\pgfqpoint{3.240120in}{1.613418in}}%
\pgfpathlineto{\pgfqpoint{3.240984in}{1.601545in}}%
\pgfpathlineto{\pgfqpoint{3.241849in}{1.586587in}}%
\pgfpathlineto{\pgfqpoint{3.242713in}{1.524550in}}%
\pgfpathlineto{\pgfqpoint{3.244443in}{1.652007in}}%
\pgfpathlineto{\pgfqpoint{3.245308in}{1.586825in}}%
\pgfpathlineto{\pgfqpoint{3.246173in}{1.644317in}}%
\pgfpathlineto{\pgfqpoint{3.247039in}{1.602140in}}%
\pgfpathlineto{\pgfqpoint{3.247901in}{1.664649in}}%
\pgfpathlineto{\pgfqpoint{3.248767in}{1.659900in}}%
\pgfpathlineto{\pgfqpoint{3.249633in}{1.653431in}}%
\pgfpathlineto{\pgfqpoint{3.250497in}{1.620840in}}%
\pgfpathlineto{\pgfqpoint{3.251361in}{1.644644in}}%
\pgfpathlineto{\pgfqpoint{3.252226in}{1.612529in}}%
\pgfpathlineto{\pgfqpoint{3.253089in}{1.648117in}}%
\pgfpathlineto{\pgfqpoint{3.253954in}{1.639420in}}%
\pgfpathlineto{\pgfqpoint{3.254819in}{1.656338in}}%
\pgfpathlineto{\pgfqpoint{3.255685in}{1.541736in}}%
\pgfpathlineto{\pgfqpoint{3.256550in}{1.628972in}}%
\pgfpathlineto{\pgfqpoint{3.257415in}{1.628377in}}%
\pgfpathlineto{\pgfqpoint{3.258280in}{1.633422in}}%
\pgfpathlineto{\pgfqpoint{3.259145in}{1.603326in}}%
\pgfpathlineto{\pgfqpoint{3.260010in}{1.702583in}}%
\pgfpathlineto{\pgfqpoint{3.260876in}{1.617515in}}%
\pgfpathlineto{\pgfqpoint{3.261740in}{1.711726in}}%
\pgfpathlineto{\pgfqpoint{3.262605in}{1.594986in}}%
\pgfpathlineto{\pgfqpoint{3.264334in}{1.739034in}}%
\pgfpathlineto{\pgfqpoint{3.266929in}{1.672903in}}%
\pgfpathlineto{\pgfqpoint{3.267794in}{1.714694in}}%
\pgfpathlineto{\pgfqpoint{3.268657in}{1.714634in}}%
\pgfpathlineto{\pgfqpoint{3.269522in}{1.666995in}}%
\pgfpathlineto{\pgfqpoint{3.272114in}{1.721788in}}%
\pgfpathlineto{\pgfqpoint{3.273846in}{1.680202in}}%
\pgfpathlineto{\pgfqpoint{3.274709in}{1.704305in}}%
\pgfpathlineto{\pgfqpoint{3.275573in}{1.674030in}}%
\pgfpathlineto{\pgfqpoint{3.277300in}{1.732088in}}%
\pgfpathlineto{\pgfqpoint{3.279893in}{1.700032in}}%
\pgfpathlineto{\pgfqpoint{3.280759in}{1.705732in}}%
\pgfpathlineto{\pgfqpoint{3.282489in}{1.652007in}}%
\pgfpathlineto{\pgfqpoint{3.283353in}{1.655033in}}%
\pgfpathlineto{\pgfqpoint{3.285084in}{1.704245in}}%
\pgfpathlineto{\pgfqpoint{3.285950in}{1.686260in}}%
\pgfpathlineto{\pgfqpoint{3.287681in}{1.704781in}}%
\pgfpathlineto{\pgfqpoint{3.289412in}{1.714456in}}%
\pgfpathlineto{\pgfqpoint{3.290277in}{1.708402in}}%
\pgfpathlineto{\pgfqpoint{3.291141in}{1.693116in}}%
\pgfpathlineto{\pgfqpoint{3.292006in}{1.645120in}}%
\pgfpathlineto{\pgfqpoint{3.292872in}{1.708997in}}%
\pgfpathlineto{\pgfqpoint{3.293738in}{1.658893in}}%
\pgfpathlineto{\pgfqpoint{3.294599in}{1.748355in}}%
\pgfpathlineto{\pgfqpoint{3.295464in}{1.686260in}}%
\pgfpathlineto{\pgfqpoint{3.296325in}{1.761771in}}%
\pgfpathlineto{\pgfqpoint{3.297192in}{1.670944in}}%
\pgfpathlineto{\pgfqpoint{3.298058in}{1.707246in}}%
\pgfpathlineto{\pgfqpoint{3.298922in}{1.552036in}}%
\pgfpathlineto{\pgfqpoint{3.299786in}{1.637460in}}%
\pgfpathlineto{\pgfqpoint{3.300649in}{1.632266in}}%
\pgfpathlineto{\pgfqpoint{3.301516in}{1.640785in}}%
\pgfpathlineto{\pgfqpoint{3.302379in}{1.585932in}}%
\pgfpathlineto{\pgfqpoint{3.303244in}{1.620155in}}%
\pgfpathlineto{\pgfqpoint{3.304108in}{1.559398in}}%
\pgfpathlineto{\pgfqpoint{3.304975in}{1.672071in}}%
\pgfpathlineto{\pgfqpoint{3.305841in}{1.653133in}}%
\pgfpathlineto{\pgfqpoint{3.307572in}{1.596053in}}%
\pgfpathlineto{\pgfqpoint{3.308436in}{1.611518in}}%
\pgfpathlineto{\pgfqpoint{3.310168in}{1.580440in}}%
\pgfpathlineto{\pgfqpoint{3.311899in}{1.655684in}}%
\pgfpathlineto{\pgfqpoint{3.312762in}{1.607450in}}%
\pgfpathlineto{\pgfqpoint{3.313628in}{1.656930in}}%
\pgfpathlineto{\pgfqpoint{3.315359in}{1.587594in}}%
\pgfpathlineto{\pgfqpoint{3.317088in}{1.686970in}}%
\pgfpathlineto{\pgfqpoint{3.318817in}{1.627309in}}%
\pgfpathlineto{\pgfqpoint{3.319683in}{1.647611in}}%
\pgfpathlineto{\pgfqpoint{3.322275in}{1.549779in}}%
\pgfpathlineto{\pgfqpoint{3.324004in}{1.632058in}}%
\pgfpathlineto{\pgfqpoint{3.324870in}{1.643871in}}%
\pgfpathlineto{\pgfqpoint{3.325734in}{1.622026in}}%
\pgfpathlineto{\pgfqpoint{3.326597in}{1.625796in}}%
\pgfpathlineto{\pgfqpoint{3.327461in}{1.644763in}}%
\pgfpathlineto{\pgfqpoint{3.328325in}{1.560941in}}%
\pgfpathlineto{\pgfqpoint{3.329190in}{1.568627in}}%
\pgfpathlineto{\pgfqpoint{3.330917in}{1.608015in}}%
\pgfpathlineto{\pgfqpoint{3.331782in}{1.587773in}}%
\pgfpathlineto{\pgfqpoint{3.332647in}{1.618761in}}%
\pgfpathlineto{\pgfqpoint{3.335241in}{1.569073in}}%
\pgfpathlineto{\pgfqpoint{3.336971in}{1.644168in}}%
\pgfpathlineto{\pgfqpoint{3.337833in}{1.599407in}}%
\pgfpathlineto{\pgfqpoint{3.338698in}{1.608283in}}%
\pgfpathlineto{\pgfqpoint{3.340427in}{1.531139in}}%
\pgfpathlineto{\pgfqpoint{3.343019in}{1.652598in}}%
\pgfpathlineto{\pgfqpoint{3.343883in}{1.582132in}}%
\pgfpathlineto{\pgfqpoint{3.344746in}{1.648266in}}%
\pgfpathlineto{\pgfqpoint{3.345611in}{1.549128in}}%
\pgfpathlineto{\pgfqpoint{3.346476in}{1.659782in}}%
\pgfpathlineto{\pgfqpoint{3.347340in}{1.632653in}}%
\pgfpathlineto{\pgfqpoint{3.349071in}{1.615496in}}%
\pgfpathlineto{\pgfqpoint{3.351667in}{1.565987in}}%
\pgfpathlineto{\pgfqpoint{3.353399in}{1.660555in}}%
\pgfpathlineto{\pgfqpoint{3.354265in}{1.700623in}}%
\pgfpathlineto{\pgfqpoint{3.355130in}{1.542034in}}%
\pgfpathlineto{\pgfqpoint{3.356859in}{1.639063in}}%
\pgfpathlineto{\pgfqpoint{3.358588in}{1.576436in}}%
\pgfpathlineto{\pgfqpoint{3.359454in}{1.583441in}}%
\pgfpathlineto{\pgfqpoint{3.360319in}{1.552274in}}%
\pgfpathlineto{\pgfqpoint{3.361184in}{1.661860in}}%
\pgfpathlineto{\pgfqpoint{3.363780in}{1.570676in}}%
\pgfpathlineto{\pgfqpoint{3.364643in}{1.627191in}}%
\pgfpathlineto{\pgfqpoint{3.365508in}{1.620721in}}%
\pgfpathlineto{\pgfqpoint{3.366374in}{1.595016in}}%
\pgfpathlineto{\pgfqpoint{3.367238in}{1.608045in}}%
\pgfpathlineto{\pgfqpoint{3.368103in}{1.656279in}}%
\pgfpathlineto{\pgfqpoint{3.369832in}{1.626923in}}%
\pgfpathlineto{\pgfqpoint{3.370697in}{1.524193in}}%
\pgfpathlineto{\pgfqpoint{3.373294in}{1.619356in}}%
\pgfpathlineto{\pgfqpoint{3.374159in}{1.600329in}}%
\pgfpathlineto{\pgfqpoint{3.375889in}{1.636155in}}%
\pgfpathlineto{\pgfqpoint{3.376753in}{1.633928in}}%
\pgfpathlineto{\pgfqpoint{3.377619in}{1.628023in}}%
\pgfpathlineto{\pgfqpoint{3.378486in}{1.608670in}}%
\pgfpathlineto{\pgfqpoint{3.379350in}{1.548831in}}%
\pgfpathlineto{\pgfqpoint{3.380214in}{1.567649in}}%
\pgfpathlineto{\pgfqpoint{3.381080in}{1.631760in}}%
\pgfpathlineto{\pgfqpoint{3.381944in}{1.602731in}}%
\pgfpathlineto{\pgfqpoint{3.382810in}{1.626893in}}%
\pgfpathlineto{\pgfqpoint{3.385402in}{1.592461in}}%
\pgfpathlineto{\pgfqpoint{3.387132in}{1.692075in}}%
\pgfpathlineto{\pgfqpoint{3.388863in}{1.557201in}}%
\pgfpathlineto{\pgfqpoint{3.390594in}{1.654141in}}%
\pgfpathlineto{\pgfqpoint{3.391460in}{1.637223in}}%
\pgfpathlineto{\pgfqpoint{3.392325in}{1.640368in}}%
\pgfpathlineto{\pgfqpoint{3.393191in}{1.657108in}}%
\pgfpathlineto{\pgfqpoint{3.394056in}{1.605758in}}%
\pgfpathlineto{\pgfqpoint{3.394921in}{1.633601in}}%
\pgfpathlineto{\pgfqpoint{3.395786in}{1.586259in}}%
\pgfpathlineto{\pgfqpoint{3.400110in}{1.681746in}}%
\pgfpathlineto{\pgfqpoint{3.401841in}{1.565154in}}%
\pgfpathlineto{\pgfqpoint{3.402704in}{1.538055in}}%
\pgfpathlineto{\pgfqpoint{3.404434in}{1.630277in}}%
\pgfpathlineto{\pgfqpoint{3.406164in}{1.587654in}}%
\pgfpathlineto{\pgfqpoint{3.407029in}{1.606472in}}%
\pgfpathlineto{\pgfqpoint{3.407893in}{1.589405in}}%
\pgfpathlineto{\pgfqpoint{3.408757in}{1.548355in}}%
\pgfpathlineto{\pgfqpoint{3.409621in}{1.596024in}}%
\pgfpathlineto{\pgfqpoint{3.410487in}{1.519088in}}%
\pgfpathlineto{\pgfqpoint{3.412214in}{1.628377in}}%
\pgfpathlineto{\pgfqpoint{3.413080in}{1.580767in}}%
\pgfpathlineto{\pgfqpoint{3.413944in}{1.588364in}}%
\pgfpathlineto{\pgfqpoint{3.414810in}{1.593767in}}%
\pgfpathlineto{\pgfqpoint{3.415676in}{1.583794in}}%
\pgfpathlineto{\pgfqpoint{3.416540in}{1.624933in}}%
\pgfpathlineto{\pgfqpoint{3.417405in}{1.581626in}}%
\pgfpathlineto{\pgfqpoint{3.418269in}{1.653784in}}%
\pgfpathlineto{\pgfqpoint{3.419134in}{1.648976in}}%
\pgfpathlineto{\pgfqpoint{3.419999in}{1.626358in}}%
\pgfpathlineto{\pgfqpoint{3.420865in}{1.652598in}}%
\pgfpathlineto{\pgfqpoint{3.421731in}{1.611518in}}%
\pgfpathlineto{\pgfqpoint{3.422597in}{1.666311in}}%
\pgfpathlineto{\pgfqpoint{3.425191in}{1.599348in}}%
\pgfpathlineto{\pgfqpoint{3.426054in}{1.664411in}}%
\pgfpathlineto{\pgfqpoint{3.427782in}{1.566935in}}%
\pgfpathlineto{\pgfqpoint{3.428646in}{1.636568in}}%
\pgfpathlineto{\pgfqpoint{3.429510in}{1.624815in}}%
\pgfpathlineto{\pgfqpoint{3.430376in}{1.601724in}}%
\pgfpathlineto{\pgfqpoint{3.432107in}{1.649155in}}%
\pgfpathlineto{\pgfqpoint{3.433836in}{1.626417in}}%
\pgfpathlineto{\pgfqpoint{3.434701in}{1.645652in}}%
\pgfpathlineto{\pgfqpoint{3.435566in}{1.639714in}}%
\pgfpathlineto{\pgfqpoint{3.436430in}{1.645325in}}%
\pgfpathlineto{\pgfqpoint{3.437295in}{1.602791in}}%
\pgfpathlineto{\pgfqpoint{3.438159in}{1.655862in}}%
\pgfpathlineto{\pgfqpoint{3.439025in}{1.590323in}}%
\pgfpathlineto{\pgfqpoint{3.439889in}{1.630396in}}%
\pgfpathlineto{\pgfqpoint{3.440753in}{1.626536in}}%
\pgfpathlineto{\pgfqpoint{3.441619in}{1.625290in}}%
\pgfpathlineto{\pgfqpoint{3.442483in}{1.632058in}}%
\pgfpathlineto{\pgfqpoint{3.444212in}{1.596024in}}%
\pgfpathlineto{\pgfqpoint{3.445076in}{1.625052in}}%
\pgfpathlineto{\pgfqpoint{3.445941in}{1.572397in}}%
\pgfpathlineto{\pgfqpoint{3.446805in}{1.627369in}}%
\pgfpathlineto{\pgfqpoint{3.447669in}{1.626655in}}%
\pgfpathlineto{\pgfqpoint{3.448533in}{1.601069in}}%
\pgfpathlineto{\pgfqpoint{3.450259in}{1.667586in}}%
\pgfpathlineto{\pgfqpoint{3.451124in}{1.605996in}}%
\pgfpathlineto{\pgfqpoint{3.451988in}{1.635084in}}%
\pgfpathlineto{\pgfqpoint{3.452853in}{1.592372in}}%
\pgfpathlineto{\pgfqpoint{3.454582in}{1.631998in}}%
\pgfpathlineto{\pgfqpoint{3.456310in}{1.581597in}}%
\pgfpathlineto{\pgfqpoint{3.457175in}{1.674383in}}%
\pgfpathlineto{\pgfqpoint{3.458040in}{1.619769in}}%
\pgfpathlineto{\pgfqpoint{3.458905in}{1.645116in}}%
\pgfpathlineto{\pgfqpoint{3.459770in}{1.578451in}}%
\pgfpathlineto{\pgfqpoint{3.460636in}{1.650460in}}%
\pgfpathlineto{\pgfqpoint{3.462364in}{1.627811in}}%
\pgfpathlineto{\pgfqpoint{3.463229in}{1.650222in}}%
\pgfpathlineto{\pgfqpoint{3.464094in}{1.647017in}}%
\pgfpathlineto{\pgfqpoint{3.464958in}{1.513031in}}%
\pgfpathlineto{\pgfqpoint{3.465821in}{1.551441in}}%
\pgfpathlineto{\pgfqpoint{3.467552in}{1.615671in}}%
\pgfpathlineto{\pgfqpoint{3.468416in}{1.605996in}}%
\pgfpathlineto{\pgfqpoint{3.469282in}{1.648768in}}%
\pgfpathlineto{\pgfqpoint{3.470145in}{1.611161in}}%
\pgfpathlineto{\pgfqpoint{3.471011in}{1.624815in}}%
\pgfpathlineto{\pgfqpoint{3.472743in}{1.605996in}}%
\pgfpathlineto{\pgfqpoint{3.473609in}{1.616207in}}%
\pgfpathlineto{\pgfqpoint{3.474471in}{1.561295in}}%
\pgfpathlineto{\pgfqpoint{3.475336in}{1.602434in}}%
\pgfpathlineto{\pgfqpoint{3.476201in}{1.598039in}}%
\pgfpathlineto{\pgfqpoint{3.477067in}{1.603561in}}%
\pgfpathlineto{\pgfqpoint{3.477933in}{1.624339in}}%
\pgfpathlineto{\pgfqpoint{3.478798in}{1.620925in}}%
\pgfpathlineto{\pgfqpoint{3.480525in}{1.574770in}}%
\pgfpathlineto{\pgfqpoint{3.481389in}{1.580793in}}%
\pgfpathlineto{\pgfqpoint{3.482252in}{1.605520in}}%
\pgfpathlineto{\pgfqpoint{3.483118in}{1.696645in}}%
\pgfpathlineto{\pgfqpoint{3.484847in}{1.621609in}}%
\pgfpathlineto{\pgfqpoint{3.485710in}{1.647433in}}%
\pgfpathlineto{\pgfqpoint{3.486575in}{1.594481in}}%
\pgfpathlineto{\pgfqpoint{3.487439in}{1.623271in}}%
\pgfpathlineto{\pgfqpoint{3.488304in}{1.569192in}}%
\pgfpathlineto{\pgfqpoint{3.489170in}{1.572397in}}%
\pgfpathlineto{\pgfqpoint{3.490901in}{1.650698in}}%
\pgfpathlineto{\pgfqpoint{3.491766in}{1.611756in}}%
\pgfpathlineto{\pgfqpoint{3.492631in}{1.629150in}}%
\pgfpathlineto{\pgfqpoint{3.494360in}{1.542301in}}%
\pgfpathlineto{\pgfqpoint{3.495226in}{1.569073in}}%
\pgfpathlineto{\pgfqpoint{3.496091in}{1.531258in}}%
\pgfpathlineto{\pgfqpoint{3.498684in}{1.660138in}}%
\pgfpathlineto{\pgfqpoint{3.499548in}{1.655212in}}%
\pgfpathlineto{\pgfqpoint{3.502142in}{1.593056in}}%
\pgfpathlineto{\pgfqpoint{3.503006in}{1.613091in}}%
\pgfpathlineto{\pgfqpoint{3.504734in}{1.674800in}}%
\pgfpathlineto{\pgfqpoint{3.506463in}{1.622204in}}%
\pgfpathlineto{\pgfqpoint{3.507329in}{1.646158in}}%
\pgfpathlineto{\pgfqpoint{3.508192in}{1.570854in}}%
\pgfpathlineto{\pgfqpoint{3.509923in}{1.639301in}}%
\pgfpathlineto{\pgfqpoint{3.510788in}{1.631998in}}%
\pgfpathlineto{\pgfqpoint{3.512519in}{1.550314in}}%
\pgfpathlineto{\pgfqpoint{3.513384in}{1.610034in}}%
\pgfpathlineto{\pgfqpoint{3.515113in}{1.565511in}}%
\pgfpathlineto{\pgfqpoint{3.515979in}{1.568954in}}%
\pgfpathlineto{\pgfqpoint{3.516844in}{1.620572in}}%
\pgfpathlineto{\pgfqpoint{3.517709in}{1.575484in}}%
\pgfpathlineto{\pgfqpoint{3.518574in}{1.614723in}}%
\pgfpathlineto{\pgfqpoint{3.519437in}{1.599259in}}%
\pgfpathlineto{\pgfqpoint{3.520301in}{1.553163in}}%
\pgfpathlineto{\pgfqpoint{3.523762in}{1.688691in}}%
\pgfpathlineto{\pgfqpoint{3.525491in}{1.580470in}}%
\pgfpathlineto{\pgfqpoint{3.526355in}{1.590472in}}%
\pgfpathlineto{\pgfqpoint{3.527220in}{1.629146in}}%
\pgfpathlineto{\pgfqpoint{3.529817in}{1.544852in}}%
\pgfpathlineto{\pgfqpoint{3.530682in}{1.618523in}}%
\pgfpathlineto{\pgfqpoint{3.531546in}{1.579314in}}%
\pgfpathlineto{\pgfqpoint{3.533278in}{1.636453in}}%
\pgfpathlineto{\pgfqpoint{3.534143in}{1.578038in}}%
\pgfpathlineto{\pgfqpoint{3.535006in}{1.596916in}}%
\pgfpathlineto{\pgfqpoint{3.536735in}{1.558803in}}%
\pgfpathlineto{\pgfqpoint{3.537601in}{1.584984in}}%
\pgfpathlineto{\pgfqpoint{3.538466in}{1.584508in}}%
\pgfpathlineto{\pgfqpoint{3.539331in}{1.584805in}}%
\pgfpathlineto{\pgfqpoint{3.540196in}{1.617396in}}%
\pgfpathlineto{\pgfqpoint{3.541059in}{1.578455in}}%
\pgfpathlineto{\pgfqpoint{3.541923in}{1.588427in}}%
\pgfpathlineto{\pgfqpoint{3.543652in}{1.562603in}}%
\pgfpathlineto{\pgfqpoint{3.544515in}{1.621847in}}%
\pgfpathlineto{\pgfqpoint{3.545379in}{1.555271in}}%
\pgfpathlineto{\pgfqpoint{3.547110in}{1.654855in}}%
\pgfpathlineto{\pgfqpoint{3.547975in}{1.619297in}}%
\pgfpathlineto{\pgfqpoint{3.548839in}{1.727221in}}%
\pgfpathlineto{\pgfqpoint{3.550568in}{1.595105in}}%
\pgfpathlineto{\pgfqpoint{3.551434in}{1.662039in}}%
\pgfpathlineto{\pgfqpoint{3.552299in}{1.614310in}}%
\pgfpathlineto{\pgfqpoint{3.553165in}{1.624402in}}%
\pgfpathlineto{\pgfqpoint{3.554896in}{1.585728in}}%
\pgfpathlineto{\pgfqpoint{3.557492in}{1.686141in}}%
\pgfpathlineto{\pgfqpoint{3.558357in}{1.639896in}}%
\pgfpathlineto{\pgfqpoint{3.559222in}{1.663879in}}%
\pgfpathlineto{\pgfqpoint{3.560087in}{1.533158in}}%
\pgfpathlineto{\pgfqpoint{3.561816in}{1.641201in}}%
\pgfpathlineto{\pgfqpoint{3.562682in}{1.627488in}}%
\pgfpathlineto{\pgfqpoint{3.563548in}{1.579938in}}%
\pgfpathlineto{\pgfqpoint{3.565276in}{1.663109in}}%
\pgfpathlineto{\pgfqpoint{3.566138in}{1.712083in}}%
\pgfpathlineto{\pgfqpoint{3.567868in}{1.648801in}}%
\pgfpathlineto{\pgfqpoint{3.569599in}{1.703475in}}%
\pgfpathlineto{\pgfqpoint{3.570464in}{1.708878in}}%
\pgfpathlineto{\pgfqpoint{3.571329in}{1.677711in}}%
\pgfpathlineto{\pgfqpoint{3.572193in}{1.686200in}}%
\pgfpathlineto{\pgfqpoint{3.573058in}{1.773168in}}%
\pgfpathlineto{\pgfqpoint{3.573923in}{1.633724in}}%
\pgfpathlineto{\pgfqpoint{3.575651in}{1.738260in}}%
\pgfpathlineto{\pgfqpoint{3.576516in}{1.670587in}}%
\pgfpathlineto{\pgfqpoint{3.577381in}{1.704483in}}%
\pgfpathlineto{\pgfqpoint{3.579112in}{1.666757in}}%
\pgfpathlineto{\pgfqpoint{3.579978in}{1.689524in}}%
\pgfpathlineto{\pgfqpoint{3.580842in}{1.742001in}}%
\pgfpathlineto{\pgfqpoint{3.582573in}{1.686022in}}%
\pgfpathlineto{\pgfqpoint{3.583437in}{1.661830in}}%
\pgfpathlineto{\pgfqpoint{3.585164in}{1.683170in}}%
\pgfpathlineto{\pgfqpoint{3.586028in}{1.675246in}}%
\pgfpathlineto{\pgfqpoint{3.586892in}{1.734460in}}%
\pgfpathlineto{\pgfqpoint{3.587756in}{1.622974in}}%
\pgfpathlineto{\pgfqpoint{3.590351in}{1.807893in}}%
\pgfpathlineto{\pgfqpoint{3.591214in}{1.710596in}}%
\pgfpathlineto{\pgfqpoint{3.592078in}{1.746035in}}%
\pgfpathlineto{\pgfqpoint{3.592942in}{1.667170in}}%
\pgfpathlineto{\pgfqpoint{3.593804in}{1.689461in}}%
\pgfpathlineto{\pgfqpoint{3.594668in}{1.758384in}}%
\pgfpathlineto{\pgfqpoint{3.595532in}{1.703680in}}%
\pgfpathlineto{\pgfqpoint{3.596397in}{1.734044in}}%
\pgfpathlineto{\pgfqpoint{3.598128in}{1.669873in}}%
\pgfpathlineto{\pgfqpoint{3.598992in}{1.740160in}}%
\pgfpathlineto{\pgfqpoint{3.599857in}{1.707510in}}%
\pgfpathlineto{\pgfqpoint{3.600722in}{1.712734in}}%
\pgfpathlineto{\pgfqpoint{3.601586in}{1.690413in}}%
\pgfpathlineto{\pgfqpoint{3.603316in}{1.759157in}}%
\pgfpathlineto{\pgfqpoint{3.604181in}{1.749363in}}%
\pgfpathlineto{\pgfqpoint{3.605045in}{1.675157in}}%
\pgfpathlineto{\pgfqpoint{3.605910in}{1.713626in}}%
\pgfpathlineto{\pgfqpoint{3.606775in}{1.700092in}}%
\pgfpathlineto{\pgfqpoint{3.607639in}{1.650017in}}%
\pgfpathlineto{\pgfqpoint{3.609370in}{1.735471in}}%
\pgfpathlineto{\pgfqpoint{3.610234in}{1.721818in}}%
\pgfpathlineto{\pgfqpoint{3.613692in}{1.745801in}}%
\pgfpathlineto{\pgfqpoint{3.614557in}{1.747582in}}%
\pgfpathlineto{\pgfqpoint{3.615422in}{1.701575in}}%
\pgfpathlineto{\pgfqpoint{3.616289in}{1.754528in}}%
\pgfpathlineto{\pgfqpoint{3.617155in}{1.690800in}}%
\pgfpathlineto{\pgfqpoint{3.618017in}{1.690948in}}%
\pgfpathlineto{\pgfqpoint{3.618881in}{1.732266in}}%
\pgfpathlineto{\pgfqpoint{3.621474in}{1.670081in}}%
\pgfpathlineto{\pgfqpoint{3.623205in}{1.759098in}}%
\pgfpathlineto{\pgfqpoint{3.624067in}{1.694273in}}%
\pgfpathlineto{\pgfqpoint{3.624932in}{1.731909in}}%
\pgfpathlineto{\pgfqpoint{3.625798in}{1.697240in}}%
\pgfpathlineto{\pgfqpoint{3.626663in}{1.699051in}}%
\pgfpathlineto{\pgfqpoint{3.628394in}{1.772216in}}%
\pgfpathlineto{\pgfqpoint{3.629258in}{1.707331in}}%
\pgfpathlineto{\pgfqpoint{3.630123in}{1.801066in}}%
\pgfpathlineto{\pgfqpoint{3.631852in}{1.723361in}}%
\pgfpathlineto{\pgfqpoint{3.632716in}{1.728823in}}%
\pgfpathlineto{\pgfqpoint{3.633580in}{1.626834in}}%
\pgfpathlineto{\pgfqpoint{3.635312in}{1.803799in}}%
\pgfpathlineto{\pgfqpoint{3.636176in}{1.698991in}}%
\pgfpathlineto{\pgfqpoint{3.637043in}{1.717661in}}%
\pgfpathlineto{\pgfqpoint{3.638773in}{1.752330in}}%
\pgfpathlineto{\pgfqpoint{3.640506in}{1.683170in}}%
\pgfpathlineto{\pgfqpoint{3.641372in}{1.743603in}}%
\pgfpathlineto{\pgfqpoint{3.643098in}{1.719888in}}%
\pgfpathlineto{\pgfqpoint{3.643962in}{1.776075in}}%
\pgfpathlineto{\pgfqpoint{3.645693in}{1.681954in}}%
\pgfpathlineto{\pgfqpoint{3.647418in}{1.784743in}}%
\pgfpathlineto{\pgfqpoint{3.649149in}{1.667263in}}%
\pgfpathlineto{\pgfqpoint{3.650016in}{1.743931in}}%
\pgfpathlineto{\pgfqpoint{3.650882in}{1.728288in}}%
\pgfpathlineto{\pgfqpoint{3.652611in}{1.668360in}}%
\pgfpathlineto{\pgfqpoint{3.654343in}{1.780233in}}%
\pgfpathlineto{\pgfqpoint{3.656940in}{1.689286in}}%
\pgfpathlineto{\pgfqpoint{3.657803in}{1.785159in}}%
\pgfpathlineto{\pgfqpoint{3.658669in}{1.668389in}}%
\pgfpathlineto{\pgfqpoint{3.659533in}{1.697567in}}%
\pgfpathlineto{\pgfqpoint{3.660399in}{1.722175in}}%
\pgfpathlineto{\pgfqpoint{3.661264in}{1.667263in}}%
\pgfpathlineto{\pgfqpoint{3.662127in}{1.752628in}}%
\pgfpathlineto{\pgfqpoint{3.664721in}{1.643071in}}%
\pgfpathlineto{\pgfqpoint{3.665587in}{1.757911in}}%
\pgfpathlineto{\pgfqpoint{3.667313in}{1.675811in}}%
\pgfpathlineto{\pgfqpoint{3.668179in}{1.687803in}}%
\pgfpathlineto{\pgfqpoint{3.669045in}{1.728674in}}%
\pgfpathlineto{\pgfqpoint{3.669910in}{1.672606in}}%
\pgfpathlineto{\pgfqpoint{3.670775in}{1.682816in}}%
\pgfpathlineto{\pgfqpoint{3.672504in}{1.721758in}}%
\pgfpathlineto{\pgfqpoint{3.673368in}{1.648385in}}%
\pgfpathlineto{\pgfqpoint{3.674235in}{1.663463in}}%
\pgfpathlineto{\pgfqpoint{3.675965in}{1.711131in}}%
\pgfpathlineto{\pgfqpoint{3.676831in}{1.714277in}}%
\pgfpathlineto{\pgfqpoint{3.677697in}{1.728347in}}%
\pgfpathlineto{\pgfqpoint{3.678563in}{1.683943in}}%
\pgfpathlineto{\pgfqpoint{3.679428in}{1.744079in}}%
\pgfpathlineto{\pgfqpoint{3.680294in}{1.677711in}}%
\pgfpathlineto{\pgfqpoint{3.681160in}{1.691335in}}%
\pgfpathlineto{\pgfqpoint{3.682026in}{1.649750in}}%
\pgfpathlineto{\pgfqpoint{3.682891in}{1.692729in}}%
\pgfpathlineto{\pgfqpoint{3.683756in}{1.679225in}}%
\pgfpathlineto{\pgfqpoint{3.684621in}{1.711667in}}%
\pgfpathlineto{\pgfqpoint{3.685485in}{1.656041in}}%
\pgfpathlineto{\pgfqpoint{3.686350in}{1.728585in}}%
\pgfpathlineto{\pgfqpoint{3.688079in}{1.654438in}}%
\pgfpathlineto{\pgfqpoint{3.688943in}{1.735234in}}%
\pgfpathlineto{\pgfqpoint{3.689808in}{1.668508in}}%
\pgfpathlineto{\pgfqpoint{3.690672in}{1.669843in}}%
\pgfpathlineto{\pgfqpoint{3.691536in}{1.719501in}}%
\pgfpathlineto{\pgfqpoint{3.692401in}{1.709529in}}%
\pgfpathlineto{\pgfqpoint{3.693266in}{1.620572in}}%
\pgfpathlineto{\pgfqpoint{3.694996in}{1.709469in}}%
\pgfpathlineto{\pgfqpoint{3.695860in}{1.680202in}}%
\pgfpathlineto{\pgfqpoint{3.696724in}{1.803680in}}%
\pgfpathlineto{\pgfqpoint{3.698452in}{1.707157in}}%
\pgfpathlineto{\pgfqpoint{3.699316in}{1.732683in}}%
\pgfpathlineto{\pgfqpoint{3.700181in}{1.727221in}}%
\pgfpathlineto{\pgfqpoint{3.701046in}{1.666727in}}%
\pgfpathlineto{\pgfqpoint{3.701911in}{1.743544in}}%
\pgfpathlineto{\pgfqpoint{3.704508in}{1.682281in}}%
\pgfpathlineto{\pgfqpoint{3.705373in}{1.675514in}}%
\pgfpathlineto{\pgfqpoint{3.706238in}{1.721818in}}%
\pgfpathlineto{\pgfqpoint{3.707103in}{1.670646in}}%
\pgfpathlineto{\pgfqpoint{3.707967in}{1.675930in}}%
\pgfpathlineto{\pgfqpoint{3.708833in}{1.689524in}}%
\pgfpathlineto{\pgfqpoint{3.709698in}{1.644704in}}%
\pgfpathlineto{\pgfqpoint{3.711428in}{1.764143in}}%
\pgfpathlineto{\pgfqpoint{3.712293in}{1.544822in}}%
\pgfpathlineto{\pgfqpoint{3.713157in}{1.676343in}}%
\pgfpathlineto{\pgfqpoint{3.714022in}{1.652955in}}%
\pgfpathlineto{\pgfqpoint{3.714886in}{1.569073in}}%
\pgfpathlineto{\pgfqpoint{3.715748in}{1.632831in}}%
\pgfpathlineto{\pgfqpoint{3.716611in}{1.596827in}}%
\pgfpathlineto{\pgfqpoint{3.718343in}{1.684657in}}%
\pgfpathlineto{\pgfqpoint{3.719209in}{1.671331in}}%
\pgfpathlineto{\pgfqpoint{3.720074in}{1.713567in}}%
\pgfpathlineto{\pgfqpoint{3.722666in}{1.634493in}}%
\pgfpathlineto{\pgfqpoint{3.723531in}{1.692194in}}%
\pgfpathlineto{\pgfqpoint{3.725261in}{1.642923in}}%
\pgfpathlineto{\pgfqpoint{3.726126in}{1.636631in}}%
\pgfpathlineto{\pgfqpoint{3.726990in}{1.675692in}}%
\pgfpathlineto{\pgfqpoint{3.727855in}{1.644823in}}%
\pgfpathlineto{\pgfqpoint{3.729587in}{1.686260in}}%
\pgfpathlineto{\pgfqpoint{3.730452in}{1.619475in}}%
\pgfpathlineto{\pgfqpoint{3.731315in}{1.667025in}}%
\pgfpathlineto{\pgfqpoint{3.732179in}{1.656338in}}%
\pgfpathlineto{\pgfqpoint{3.733910in}{1.598281in}}%
\pgfpathlineto{\pgfqpoint{3.734774in}{1.687505in}}%
\pgfpathlineto{\pgfqpoint{3.736506in}{1.562901in}}%
\pgfpathlineto{\pgfqpoint{3.737372in}{1.653431in}}%
\pgfpathlineto{\pgfqpoint{3.738238in}{1.592997in}}%
\pgfpathlineto{\pgfqpoint{3.739103in}{1.629031in}}%
\pgfpathlineto{\pgfqpoint{3.739967in}{1.592818in}}%
\pgfpathlineto{\pgfqpoint{3.740831in}{1.642120in}}%
\pgfpathlineto{\pgfqpoint{3.741695in}{1.596500in}}%
\pgfpathlineto{\pgfqpoint{3.743425in}{1.663225in}}%
\pgfpathlineto{\pgfqpoint{3.745156in}{1.560793in}}%
\pgfpathlineto{\pgfqpoint{3.746887in}{1.628972in}}%
\pgfpathlineto{\pgfqpoint{3.747753in}{1.626064in}}%
\pgfpathlineto{\pgfqpoint{3.748618in}{1.688572in}}%
\pgfpathlineto{\pgfqpoint{3.750349in}{1.602880in}}%
\pgfpathlineto{\pgfqpoint{3.752078in}{1.638706in}}%
\pgfpathlineto{\pgfqpoint{3.752944in}{1.585575in}}%
\pgfpathlineto{\pgfqpoint{3.753810in}{1.628317in}}%
\pgfpathlineto{\pgfqpoint{3.754676in}{1.613180in}}%
\pgfpathlineto{\pgfqpoint{3.756406in}{1.679727in}}%
\pgfpathlineto{\pgfqpoint{3.757272in}{1.620155in}}%
\pgfpathlineto{\pgfqpoint{3.758138in}{1.666311in}}%
\pgfpathlineto{\pgfqpoint{3.759004in}{1.561949in}}%
\pgfpathlineto{\pgfqpoint{3.759869in}{1.652627in}}%
\pgfpathlineto{\pgfqpoint{3.760734in}{1.642625in}}%
\pgfpathlineto{\pgfqpoint{3.761597in}{1.573822in}}%
\pgfpathlineto{\pgfqpoint{3.762463in}{1.581303in}}%
\pgfpathlineto{\pgfqpoint{3.763326in}{1.634136in}}%
\pgfpathlineto{\pgfqpoint{3.764191in}{1.625320in}}%
\pgfpathlineto{\pgfqpoint{3.765921in}{1.648798in}}%
\pgfpathlineto{\pgfqpoint{3.766786in}{1.575097in}}%
\pgfpathlineto{\pgfqpoint{3.768513in}{1.666311in}}%
\pgfpathlineto{\pgfqpoint{3.769378in}{1.593116in}}%
\pgfpathlineto{\pgfqpoint{3.770241in}{1.678362in}}%
\pgfpathlineto{\pgfqpoint{3.771106in}{1.670940in}}%
\pgfpathlineto{\pgfqpoint{3.772835in}{1.563611in}}%
\pgfpathlineto{\pgfqpoint{3.773700in}{1.652717in}}%
\pgfpathlineto{\pgfqpoint{3.774564in}{1.620423in}}%
\pgfpathlineto{\pgfqpoint{3.776290in}{1.697270in}}%
\pgfpathlineto{\pgfqpoint{3.778019in}{1.584508in}}%
\pgfpathlineto{\pgfqpoint{3.778884in}{1.626953in}}%
\pgfpathlineto{\pgfqpoint{3.779750in}{1.625588in}}%
\pgfpathlineto{\pgfqpoint{3.780615in}{1.637817in}}%
\pgfpathlineto{\pgfqpoint{3.781479in}{1.632177in}}%
\pgfpathlineto{\pgfqpoint{3.782344in}{1.587773in}}%
\pgfpathlineto{\pgfqpoint{3.783208in}{1.599943in}}%
\pgfpathlineto{\pgfqpoint{3.784074in}{1.618107in}}%
\pgfpathlineto{\pgfqpoint{3.784940in}{1.589018in}}%
\pgfpathlineto{\pgfqpoint{3.785802in}{1.590561in}}%
\pgfpathlineto{\pgfqpoint{3.786665in}{1.606234in}}%
\pgfpathlineto{\pgfqpoint{3.787530in}{1.689818in}}%
\pgfpathlineto{\pgfqpoint{3.788396in}{1.602880in}}%
\pgfpathlineto{\pgfqpoint{3.789261in}{1.611101in}}%
\pgfpathlineto{\pgfqpoint{3.790125in}{1.592461in}}%
\pgfpathlineto{\pgfqpoint{3.790989in}{1.597805in}}%
\pgfpathlineto{\pgfqpoint{3.792718in}{1.561295in}}%
\pgfpathlineto{\pgfqpoint{3.793584in}{1.520334in}}%
\pgfpathlineto{\pgfqpoint{3.794449in}{1.629091in}}%
\pgfpathlineto{\pgfqpoint{3.795314in}{1.584449in}}%
\pgfpathlineto{\pgfqpoint{3.796179in}{1.609796in}}%
\pgfpathlineto{\pgfqpoint{3.797911in}{1.585724in}}%
\pgfpathlineto{\pgfqpoint{3.798777in}{1.601486in}}%
\pgfpathlineto{\pgfqpoint{3.800503in}{1.576908in}}%
\pgfpathlineto{\pgfqpoint{3.802234in}{1.697835in}}%
\pgfpathlineto{\pgfqpoint{3.805695in}{1.636334in}}%
\pgfpathlineto{\pgfqpoint{3.806562in}{1.724904in}}%
\pgfpathlineto{\pgfqpoint{3.807428in}{1.714307in}}%
\pgfpathlineto{\pgfqpoint{3.808293in}{1.705848in}}%
\pgfpathlineto{\pgfqpoint{3.809157in}{1.717070in}}%
\pgfpathlineto{\pgfqpoint{3.810020in}{1.674328in}}%
\pgfpathlineto{\pgfqpoint{3.810885in}{1.746217in}}%
\pgfpathlineto{\pgfqpoint{3.812616in}{1.636334in}}%
\pgfpathlineto{\pgfqpoint{3.813481in}{1.705316in}}%
\pgfpathlineto{\pgfqpoint{3.814344in}{1.639866in}}%
\pgfpathlineto{\pgfqpoint{3.815209in}{1.682459in}}%
\pgfpathlineto{\pgfqpoint{3.816076in}{1.670706in}}%
\pgfpathlineto{\pgfqpoint{3.816939in}{1.636780in}}%
\pgfpathlineto{\pgfqpoint{3.818667in}{1.697954in}}%
\pgfpathlineto{\pgfqpoint{3.819532in}{1.685903in}}%
\pgfpathlineto{\pgfqpoint{3.820398in}{1.689286in}}%
\pgfpathlineto{\pgfqpoint{3.821264in}{1.702970in}}%
\pgfpathlineto{\pgfqpoint{3.822129in}{1.695994in}}%
\pgfpathlineto{\pgfqpoint{3.822992in}{1.753695in}}%
\pgfpathlineto{\pgfqpoint{3.823858in}{1.723182in}}%
\pgfpathlineto{\pgfqpoint{3.824723in}{1.726507in}}%
\pgfpathlineto{\pgfqpoint{3.825589in}{1.705729in}}%
\pgfpathlineto{\pgfqpoint{3.826455in}{1.740220in}}%
\pgfpathlineto{\pgfqpoint{3.827320in}{1.614426in}}%
\pgfpathlineto{\pgfqpoint{3.828185in}{1.762184in}}%
\pgfpathlineto{\pgfqpoint{3.829052in}{1.693916in}}%
\pgfpathlineto{\pgfqpoint{3.829917in}{1.702464in}}%
\pgfpathlineto{\pgfqpoint{3.830781in}{1.693767in}}%
\pgfpathlineto{\pgfqpoint{3.831644in}{1.693916in}}%
\pgfpathlineto{\pgfqpoint{3.832509in}{1.733452in}}%
\pgfpathlineto{\pgfqpoint{3.833375in}{1.729980in}}%
\pgfpathlineto{\pgfqpoint{3.834240in}{1.723718in}}%
\pgfpathlineto{\pgfqpoint{3.835105in}{1.754409in}}%
\pgfpathlineto{\pgfqpoint{3.835971in}{1.709202in}}%
\pgfpathlineto{\pgfqpoint{3.836836in}{1.712972in}}%
\pgfpathlineto{\pgfqpoint{3.837698in}{1.720096in}}%
\pgfpathlineto{\pgfqpoint{3.838562in}{1.651531in}}%
\pgfpathlineto{\pgfqpoint{3.840290in}{1.701367in}}%
\pgfpathlineto{\pgfqpoint{3.841155in}{1.710953in}}%
\pgfpathlineto{\pgfqpoint{3.842018in}{1.668508in}}%
\pgfpathlineto{\pgfqpoint{3.843749in}{1.728347in}}%
\pgfpathlineto{\pgfqpoint{3.844612in}{1.716237in}}%
\pgfpathlineto{\pgfqpoint{3.845478in}{1.661354in}}%
\pgfpathlineto{\pgfqpoint{3.847206in}{1.701099in}}%
\pgfpathlineto{\pgfqpoint{3.849796in}{1.629061in}}%
\pgfpathlineto{\pgfqpoint{3.850662in}{1.697478in}}%
\pgfpathlineto{\pgfqpoint{3.851528in}{1.688929in}}%
\pgfpathlineto{\pgfqpoint{3.853258in}{1.727990in}}%
\pgfpathlineto{\pgfqpoint{3.856715in}{1.568121in}}%
\pgfpathlineto{\pgfqpoint{3.857580in}{1.585813in}}%
\pgfpathlineto{\pgfqpoint{3.858443in}{1.580172in}}%
\pgfpathlineto{\pgfqpoint{3.859307in}{1.614634in}}%
\pgfpathlineto{\pgfqpoint{3.860171in}{1.568657in}}%
\pgfpathlineto{\pgfqpoint{3.861901in}{1.628525in}}%
\pgfpathlineto{\pgfqpoint{3.862767in}{1.576432in}}%
\pgfpathlineto{\pgfqpoint{3.863631in}{1.654022in}}%
\pgfpathlineto{\pgfqpoint{3.864495in}{1.577503in}}%
\pgfpathlineto{\pgfqpoint{3.865358in}{1.702524in}}%
\pgfpathlineto{\pgfqpoint{3.866223in}{1.667025in}}%
\pgfpathlineto{\pgfqpoint{3.867087in}{1.663463in}}%
\pgfpathlineto{\pgfqpoint{3.867951in}{1.666965in}}%
\pgfpathlineto{\pgfqpoint{3.868815in}{1.748917in}}%
\pgfpathlineto{\pgfqpoint{3.869677in}{1.677295in}}%
\pgfpathlineto{\pgfqpoint{3.870542in}{1.731612in}}%
\pgfpathlineto{\pgfqpoint{3.872272in}{1.676878in}}%
\pgfpathlineto{\pgfqpoint{3.873137in}{1.706205in}}%
\pgfpathlineto{\pgfqpoint{3.874002in}{1.623896in}}%
\pgfpathlineto{\pgfqpoint{3.874867in}{1.697775in}}%
\pgfpathlineto{\pgfqpoint{3.875730in}{1.686970in}}%
\pgfpathlineto{\pgfqpoint{3.878324in}{1.638736in}}%
\pgfpathlineto{\pgfqpoint{3.879189in}{1.673138in}}%
\pgfpathlineto{\pgfqpoint{3.880052in}{1.664351in}}%
\pgfpathlineto{\pgfqpoint{3.880914in}{1.568865in}}%
\pgfpathlineto{\pgfqpoint{3.881778in}{1.630039in}}%
\pgfpathlineto{\pgfqpoint{3.882643in}{1.618166in}}%
\pgfpathlineto{\pgfqpoint{3.883507in}{1.618166in}}%
\pgfpathlineto{\pgfqpoint{3.884372in}{1.630753in}}%
\pgfpathlineto{\pgfqpoint{3.886103in}{1.570259in}}%
\pgfpathlineto{\pgfqpoint{3.887836in}{1.664143in}}%
\pgfpathlineto{\pgfqpoint{3.890434in}{1.572308in}}%
\pgfpathlineto{\pgfqpoint{3.891300in}{1.610332in}}%
\pgfpathlineto{\pgfqpoint{3.892166in}{1.584211in}}%
\pgfpathlineto{\pgfqpoint{3.893896in}{1.652479in}}%
\pgfpathlineto{\pgfqpoint{3.895627in}{1.598043in}}%
\pgfpathlineto{\pgfqpoint{3.896493in}{1.632653in}}%
\pgfpathlineto{\pgfqpoint{3.897360in}{1.584151in}}%
\pgfpathlineto{\pgfqpoint{3.898225in}{1.637520in}}%
\pgfpathlineto{\pgfqpoint{3.899958in}{1.600802in}}%
\pgfpathlineto{\pgfqpoint{3.900822in}{1.644109in}}%
\pgfpathlineto{\pgfqpoint{3.901689in}{1.599586in}}%
\pgfpathlineto{\pgfqpoint{3.904283in}{1.673316in}}%
\pgfpathlineto{\pgfqpoint{3.905147in}{1.590030in}}%
\pgfpathlineto{\pgfqpoint{3.906011in}{1.661146in}}%
\pgfpathlineto{\pgfqpoint{3.906875in}{1.562008in}}%
\pgfpathlineto{\pgfqpoint{3.907739in}{1.652241in}}%
\pgfpathlineto{\pgfqpoint{3.908603in}{1.632827in}}%
\pgfpathlineto{\pgfqpoint{3.912926in}{1.583556in}}%
\pgfpathlineto{\pgfqpoint{3.913792in}{1.588542in}}%
\pgfpathlineto{\pgfqpoint{3.914657in}{1.674562in}}%
\pgfpathlineto{\pgfqpoint{3.917253in}{1.597329in}}%
\pgfpathlineto{\pgfqpoint{3.918118in}{1.648976in}}%
\pgfpathlineto{\pgfqpoint{3.918984in}{1.580708in}}%
\pgfpathlineto{\pgfqpoint{3.919849in}{1.666549in}}%
\pgfpathlineto{\pgfqpoint{3.920712in}{1.621907in}}%
\pgfpathlineto{\pgfqpoint{3.921578in}{1.638944in}}%
\pgfpathlineto{\pgfqpoint{3.924173in}{1.583854in}}%
\pgfpathlineto{\pgfqpoint{3.925036in}{1.568597in}}%
\pgfpathlineto{\pgfqpoint{3.928493in}{1.631552in}}%
\pgfpathlineto{\pgfqpoint{3.929358in}{1.643514in}}%
\pgfpathlineto{\pgfqpoint{3.930224in}{1.585397in}}%
\pgfpathlineto{\pgfqpoint{3.932817in}{1.652122in}}%
\pgfpathlineto{\pgfqpoint{3.933682in}{1.595191in}}%
\pgfpathlineto{\pgfqpoint{3.934547in}{1.662273in}}%
\pgfpathlineto{\pgfqpoint{3.935411in}{1.655773in}}%
\pgfpathlineto{\pgfqpoint{3.936274in}{1.691540in}}%
\pgfpathlineto{\pgfqpoint{3.937139in}{1.688334in}}%
\pgfpathlineto{\pgfqpoint{3.939731in}{1.593410in}}%
\pgfpathlineto{\pgfqpoint{3.943191in}{1.668624in}}%
\pgfpathlineto{\pgfqpoint{3.944922in}{1.639476in}}%
\pgfpathlineto{\pgfqpoint{3.945788in}{1.578094in}}%
\pgfpathlineto{\pgfqpoint{3.946653in}{1.631403in}}%
\pgfpathlineto{\pgfqpoint{3.947519in}{1.526714in}}%
\pgfpathlineto{\pgfqpoint{3.949249in}{1.626120in}}%
\pgfpathlineto{\pgfqpoint{3.950114in}{1.595131in}}%
\pgfpathlineto{\pgfqpoint{3.950979in}{1.627901in}}%
\pgfpathlineto{\pgfqpoint{3.951843in}{1.591688in}}%
\pgfpathlineto{\pgfqpoint{3.953573in}{1.663935in}}%
\pgfpathlineto{\pgfqpoint{3.954436in}{1.636062in}}%
\pgfpathlineto{\pgfqpoint{3.957031in}{1.702907in}}%
\pgfpathlineto{\pgfqpoint{3.959623in}{1.639982in}}%
\pgfpathlineto{\pgfqpoint{3.960486in}{1.647136in}}%
\pgfpathlineto{\pgfqpoint{3.961351in}{1.680794in}}%
\pgfpathlineto{\pgfqpoint{3.962216in}{1.670702in}}%
\pgfpathlineto{\pgfqpoint{3.963081in}{1.698426in}}%
\pgfpathlineto{\pgfqpoint{3.963945in}{1.655297in}}%
\pgfpathlineto{\pgfqpoint{3.964811in}{1.701691in}}%
\pgfpathlineto{\pgfqpoint{3.965676in}{1.672721in}}%
\pgfpathlineto{\pgfqpoint{3.966542in}{1.719144in}}%
\pgfpathlineto{\pgfqpoint{3.967407in}{1.653784in}}%
\pgfpathlineto{\pgfqpoint{3.968272in}{1.730779in}}%
\pgfpathlineto{\pgfqpoint{3.969137in}{1.634757in}}%
\pgfpathlineto{\pgfqpoint{3.970867in}{1.708636in}}%
\pgfpathlineto{\pgfqpoint{3.971732in}{1.615314in}}%
\pgfpathlineto{\pgfqpoint{3.973460in}{1.716411in}}%
\pgfpathlineto{\pgfqpoint{3.975191in}{1.653427in}}%
\pgfpathlineto{\pgfqpoint{3.976055in}{1.680407in}}%
\pgfpathlineto{\pgfqpoint{3.978650in}{1.581478in}}%
\pgfpathlineto{\pgfqpoint{3.979516in}{1.591748in}}%
\pgfpathlineto{\pgfqpoint{3.980382in}{1.624339in}}%
\pgfpathlineto{\pgfqpoint{3.981249in}{1.586226in}}%
\pgfpathlineto{\pgfqpoint{3.982115in}{1.628730in}}%
\pgfpathlineto{\pgfqpoint{3.982981in}{1.557792in}}%
\pgfpathlineto{\pgfqpoint{3.983846in}{1.558030in}}%
\pgfpathlineto{\pgfqpoint{3.984712in}{1.585218in}}%
\pgfpathlineto{\pgfqpoint{3.985575in}{1.550906in}}%
\pgfpathlineto{\pgfqpoint{3.986438in}{1.552330in}}%
\pgfpathlineto{\pgfqpoint{3.987304in}{1.539568in}}%
\pgfpathlineto{\pgfqpoint{3.988169in}{1.578302in}}%
\pgfpathlineto{\pgfqpoint{3.989902in}{1.565273in}}%
\pgfpathlineto{\pgfqpoint{3.990765in}{1.625112in}}%
\pgfpathlineto{\pgfqpoint{3.991629in}{1.596321in}}%
\pgfpathlineto{\pgfqpoint{3.992494in}{1.614158in}}%
\pgfpathlineto{\pgfqpoint{3.993360in}{1.528466in}}%
\pgfpathlineto{\pgfqpoint{3.994223in}{1.631522in}}%
\pgfpathlineto{\pgfqpoint{3.995954in}{1.575424in}}%
\pgfpathlineto{\pgfqpoint{3.996820in}{1.676700in}}%
\pgfpathlineto{\pgfqpoint{3.997684in}{1.590621in}}%
\pgfpathlineto{\pgfqpoint{3.998549in}{1.643395in}}%
\pgfpathlineto{\pgfqpoint{4.000279in}{1.584032in}}%
\pgfpathlineto{\pgfqpoint{4.001143in}{1.611280in}}%
\pgfpathlineto{\pgfqpoint{4.002007in}{1.593707in}}%
\pgfpathlineto{\pgfqpoint{4.002874in}{1.614307in}}%
\pgfpathlineto{\pgfqpoint{4.003737in}{1.607896in}}%
\pgfpathlineto{\pgfqpoint{4.004603in}{1.565303in}}%
\pgfpathlineto{\pgfqpoint{4.006327in}{1.606651in}}%
\pgfpathlineto{\pgfqpoint{4.008057in}{1.564500in}}%
\pgfpathlineto{\pgfqpoint{4.008923in}{1.628079in}}%
\pgfpathlineto{\pgfqpoint{4.009788in}{1.527220in}}%
\pgfpathlineto{\pgfqpoint{4.010652in}{1.626536in}}%
\pgfpathlineto{\pgfqpoint{4.011517in}{1.572487in}}%
\pgfpathlineto{\pgfqpoint{4.013246in}{1.615675in}}%
\pgfpathlineto{\pgfqpoint{4.014108in}{1.552066in}}%
\pgfpathlineto{\pgfqpoint{4.014974in}{1.660908in}}%
\pgfpathlineto{\pgfqpoint{4.016704in}{1.578421in}}%
\pgfpathlineto{\pgfqpoint{4.017569in}{1.670940in}}%
\pgfpathlineto{\pgfqpoint{4.018434in}{1.661444in}}%
\pgfpathlineto{\pgfqpoint{4.019301in}{1.558803in}}%
\pgfpathlineto{\pgfqpoint{4.020166in}{1.628615in}}%
\pgfpathlineto{\pgfqpoint{4.021032in}{1.575365in}}%
\pgfpathlineto{\pgfqpoint{4.022764in}{1.614664in}}%
\pgfpathlineto{\pgfqpoint{4.023629in}{1.597180in}}%
\pgfpathlineto{\pgfqpoint{4.024494in}{1.672364in}}%
\pgfpathlineto{\pgfqpoint{4.025359in}{1.588899in}}%
\pgfpathlineto{\pgfqpoint{4.026224in}{1.592670in}}%
\pgfpathlineto{\pgfqpoint{4.027087in}{1.579934in}}%
\pgfpathlineto{\pgfqpoint{4.027953in}{1.660492in}}%
\pgfpathlineto{\pgfqpoint{4.028818in}{1.578332in}}%
\pgfpathlineto{\pgfqpoint{4.030549in}{1.632500in}}%
\pgfpathlineto{\pgfqpoint{4.032277in}{1.595369in}}%
\pgfpathlineto{\pgfqpoint{4.033143in}{1.613741in}}%
\pgfpathlineto{\pgfqpoint{4.034872in}{1.584802in}}%
\pgfpathlineto{\pgfqpoint{4.035739in}{1.557465in}}%
\pgfpathlineto{\pgfqpoint{4.037469in}{1.631582in}}%
\pgfpathlineto{\pgfqpoint{4.038334in}{1.615909in}}%
\pgfpathlineto{\pgfqpoint{4.039199in}{1.679667in}}%
\pgfpathlineto{\pgfqpoint{4.040065in}{1.620185in}}%
\pgfpathlineto{\pgfqpoint{4.040931in}{1.624933in}}%
\pgfpathlineto{\pgfqpoint{4.041797in}{1.674324in}}%
\pgfpathlineto{\pgfqpoint{4.043526in}{1.603680in}}%
\pgfpathlineto{\pgfqpoint{4.044392in}{1.611280in}}%
\pgfpathlineto{\pgfqpoint{4.045257in}{1.635114in}}%
\pgfpathlineto{\pgfqpoint{4.046121in}{1.627250in}}%
\pgfpathlineto{\pgfqpoint{4.046986in}{1.661384in}}%
\pgfpathlineto{\pgfqpoint{4.048715in}{1.632177in}}%
\pgfpathlineto{\pgfqpoint{4.049579in}{1.652271in}}%
\pgfpathlineto{\pgfqpoint{4.050444in}{1.596024in}}%
\pgfpathlineto{\pgfqpoint{4.051310in}{1.647195in}}%
\pgfpathlineto{\pgfqpoint{4.052176in}{1.633660in}}%
\pgfpathlineto{\pgfqpoint{4.053906in}{1.577384in}}%
\pgfpathlineto{\pgfqpoint{4.054771in}{1.613715in}}%
\pgfpathlineto{\pgfqpoint{4.055636in}{1.569192in}}%
\pgfpathlineto{\pgfqpoint{4.057366in}{1.702880in}}%
\pgfpathlineto{\pgfqpoint{4.058228in}{1.656695in}}%
\pgfpathlineto{\pgfqpoint{4.059092in}{1.726269in}}%
\pgfpathlineto{\pgfqpoint{4.060822in}{1.601664in}}%
\pgfpathlineto{\pgfqpoint{4.062552in}{1.581541in}}%
\pgfpathlineto{\pgfqpoint{4.063417in}{1.632712in}}%
\pgfpathlineto{\pgfqpoint{4.064282in}{1.573022in}}%
\pgfpathlineto{\pgfqpoint{4.066012in}{1.665363in}}%
\pgfpathlineto{\pgfqpoint{4.066875in}{1.599824in}}%
\pgfpathlineto{\pgfqpoint{4.067741in}{1.611994in}}%
\pgfpathlineto{\pgfqpoint{4.068604in}{1.620334in}}%
\pgfpathlineto{\pgfqpoint{4.069469in}{1.606829in}}%
\pgfpathlineto{\pgfqpoint{4.070334in}{1.607896in}}%
\pgfpathlineto{\pgfqpoint{4.071198in}{1.639242in}}%
\pgfpathlineto{\pgfqpoint{4.072063in}{1.597983in}}%
\pgfpathlineto{\pgfqpoint{4.072928in}{1.606591in}}%
\pgfpathlineto{\pgfqpoint{4.073792in}{1.572219in}}%
\pgfpathlineto{\pgfqpoint{4.074657in}{1.664589in}}%
\pgfpathlineto{\pgfqpoint{4.075522in}{1.587297in}}%
\pgfpathlineto{\pgfqpoint{4.076388in}{1.604869in}}%
\pgfpathlineto{\pgfqpoint{4.078118in}{1.664500in}}%
\pgfpathlineto{\pgfqpoint{4.078984in}{1.600831in}}%
\pgfpathlineto{\pgfqpoint{4.079849in}{1.670289in}}%
\pgfpathlineto{\pgfqpoint{4.080713in}{1.623807in}}%
\pgfpathlineto{\pgfqpoint{4.082444in}{1.759336in}}%
\pgfpathlineto{\pgfqpoint{4.083309in}{1.695994in}}%
\pgfpathlineto{\pgfqpoint{4.084175in}{1.709707in}}%
\pgfpathlineto{\pgfqpoint{4.085039in}{1.681984in}}%
\pgfpathlineto{\pgfqpoint{4.085903in}{1.696113in}}%
\pgfpathlineto{\pgfqpoint{4.086768in}{1.759038in}}%
\pgfpathlineto{\pgfqpoint{4.088498in}{1.660495in}}%
\pgfpathlineto{\pgfqpoint{4.090229in}{1.741733in}}%
\pgfpathlineto{\pgfqpoint{4.091095in}{1.737312in}}%
\pgfpathlineto{\pgfqpoint{4.091961in}{1.720632in}}%
\pgfpathlineto{\pgfqpoint{4.092826in}{1.757465in}}%
\pgfpathlineto{\pgfqpoint{4.094556in}{1.713686in}}%
\pgfpathlineto{\pgfqpoint{4.095420in}{1.734285in}}%
\pgfpathlineto{\pgfqpoint{4.096286in}{1.705256in}}%
\pgfpathlineto{\pgfqpoint{4.097151in}{1.613656in}}%
\pgfpathlineto{\pgfqpoint{4.098881in}{1.656398in}}%
\pgfpathlineto{\pgfqpoint{4.099746in}{1.620661in}}%
\pgfpathlineto{\pgfqpoint{4.101476in}{1.683408in}}%
\pgfpathlineto{\pgfqpoint{4.103206in}{1.607361in}}%
\pgfpathlineto{\pgfqpoint{4.104071in}{1.666668in}}%
\pgfpathlineto{\pgfqpoint{4.104936in}{1.666311in}}%
\pgfpathlineto{\pgfqpoint{4.106665in}{1.535947in}}%
\pgfpathlineto{\pgfqpoint{4.107530in}{1.642090in}}%
\pgfpathlineto{\pgfqpoint{4.109259in}{1.595875in}}%
\pgfpathlineto{\pgfqpoint{4.110125in}{1.585278in}}%
\pgfpathlineto{\pgfqpoint{4.110990in}{1.611458in}}%
\pgfpathlineto{\pgfqpoint{4.112719in}{1.588840in}}%
\pgfpathlineto{\pgfqpoint{4.113584in}{1.619709in}}%
\pgfpathlineto{\pgfqpoint{4.114449in}{1.605937in}}%
\pgfpathlineto{\pgfqpoint{4.115313in}{1.544614in}}%
\pgfpathlineto{\pgfqpoint{4.116177in}{1.621193in}}%
\pgfpathlineto{\pgfqpoint{4.117040in}{1.586999in}}%
\pgfpathlineto{\pgfqpoint{4.117905in}{1.599348in}}%
\pgfpathlineto{\pgfqpoint{4.118770in}{1.579905in}}%
\pgfpathlineto{\pgfqpoint{4.119636in}{1.628020in}}%
\pgfpathlineto{\pgfqpoint{4.120501in}{1.620657in}}%
\pgfpathlineto{\pgfqpoint{4.121367in}{1.608815in}}%
\pgfpathlineto{\pgfqpoint{4.122231in}{1.610982in}}%
\pgfpathlineto{\pgfqpoint{4.123096in}{1.643692in}}%
\pgfpathlineto{\pgfqpoint{4.124826in}{1.562600in}}%
\pgfpathlineto{\pgfqpoint{4.125692in}{1.591450in}}%
\pgfpathlineto{\pgfqpoint{4.126557in}{1.741287in}}%
\pgfpathlineto{\pgfqpoint{4.128288in}{1.664054in}}%
\pgfpathlineto{\pgfqpoint{4.129154in}{1.671119in}}%
\pgfpathlineto{\pgfqpoint{4.130019in}{1.719501in}}%
\pgfpathlineto{\pgfqpoint{4.130884in}{1.657911in}}%
\pgfpathlineto{\pgfqpoint{4.131746in}{1.665954in}}%
\pgfpathlineto{\pgfqpoint{4.132610in}{1.692135in}}%
\pgfpathlineto{\pgfqpoint{4.133474in}{1.659603in}}%
\pgfpathlineto{\pgfqpoint{4.134339in}{1.692789in}}%
\pgfpathlineto{\pgfqpoint{4.135203in}{1.673614in}}%
\pgfpathlineto{\pgfqpoint{4.136933in}{1.790086in}}%
\pgfpathlineto{\pgfqpoint{4.138664in}{1.686970in}}%
\pgfpathlineto{\pgfqpoint{4.139529in}{1.744258in}}%
\pgfpathlineto{\pgfqpoint{4.140395in}{1.696143in}}%
\pgfpathlineto{\pgfqpoint{4.142124in}{1.727577in}}%
\pgfpathlineto{\pgfqpoint{4.142987in}{1.679135in}}%
\pgfpathlineto{\pgfqpoint{4.143853in}{1.731021in}}%
\pgfpathlineto{\pgfqpoint{4.145583in}{1.695935in}}%
\pgfpathlineto{\pgfqpoint{4.147312in}{1.721550in}}%
\pgfpathlineto{\pgfqpoint{4.149043in}{1.658298in}}%
\pgfpathlineto{\pgfqpoint{4.149908in}{1.651293in}}%
\pgfpathlineto{\pgfqpoint{4.152507in}{1.693916in}}%
\pgfpathlineto{\pgfqpoint{4.153374in}{1.652895in}}%
\pgfpathlineto{\pgfqpoint{4.154234in}{1.709678in}}%
\pgfpathlineto{\pgfqpoint{4.155100in}{1.653074in}}%
\pgfpathlineto{\pgfqpoint{4.157692in}{1.704721in}}%
\pgfpathlineto{\pgfqpoint{4.159420in}{1.669427in}}%
\pgfpathlineto{\pgfqpoint{4.160285in}{1.743128in}}%
\pgfpathlineto{\pgfqpoint{4.161148in}{1.693737in}}%
\pgfpathlineto{\pgfqpoint{4.162013in}{1.726090in}}%
\pgfpathlineto{\pgfqpoint{4.163741in}{1.650727in}}%
\pgfpathlineto{\pgfqpoint{4.164607in}{1.696704in}}%
\pgfpathlineto{\pgfqpoint{4.167203in}{1.652419in}}%
\pgfpathlineto{\pgfqpoint{4.168932in}{1.684534in}}%
\pgfpathlineto{\pgfqpoint{4.169798in}{1.670881in}}%
\pgfpathlineto{\pgfqpoint{4.170663in}{1.636330in}}%
\pgfpathlineto{\pgfqpoint{4.171528in}{1.650400in}}%
\pgfpathlineto{\pgfqpoint{4.172393in}{1.612406in}}%
\pgfpathlineto{\pgfqpoint{4.173258in}{1.631165in}}%
\pgfpathlineto{\pgfqpoint{4.174125in}{1.618464in}}%
\pgfpathlineto{\pgfqpoint{4.174991in}{1.695280in}}%
\pgfpathlineto{\pgfqpoint{4.175858in}{1.616474in}}%
\pgfpathlineto{\pgfqpoint{4.176724in}{1.688275in}}%
\pgfpathlineto{\pgfqpoint{4.178452in}{1.621758in}}%
\pgfpathlineto{\pgfqpoint{4.180184in}{1.681508in}}%
\pgfpathlineto{\pgfqpoint{4.181050in}{1.630217in}}%
\pgfpathlineto{\pgfqpoint{4.181915in}{1.702286in}}%
\pgfpathlineto{\pgfqpoint{4.182780in}{1.664292in}}%
\pgfpathlineto{\pgfqpoint{4.183646in}{1.714039in}}%
\pgfpathlineto{\pgfqpoint{4.184511in}{1.700326in}}%
\pgfpathlineto{\pgfqpoint{4.186240in}{1.688989in}}%
\pgfpathlineto{\pgfqpoint{4.187102in}{1.654319in}}%
\pgfpathlineto{\pgfqpoint{4.187968in}{1.662957in}}%
\pgfpathlineto{\pgfqpoint{4.189700in}{1.704126in}}%
\pgfpathlineto{\pgfqpoint{4.190565in}{1.652598in}}%
\pgfpathlineto{\pgfqpoint{4.191432in}{1.693856in}}%
\pgfpathlineto{\pgfqpoint{4.193163in}{1.643454in}}%
\pgfpathlineto{\pgfqpoint{4.194030in}{1.659187in}}%
\pgfpathlineto{\pgfqpoint{4.194895in}{1.645087in}}%
\pgfpathlineto{\pgfqpoint{4.196623in}{1.662511in}}%
\pgfpathlineto{\pgfqpoint{4.197486in}{1.650460in}}%
\pgfpathlineto{\pgfqpoint{4.199218in}{1.742235in}}%
\pgfpathlineto{\pgfqpoint{4.200083in}{1.643395in}}%
\pgfpathlineto{\pgfqpoint{4.201815in}{1.717601in}}%
\pgfpathlineto{\pgfqpoint{4.202680in}{1.724607in}}%
\pgfpathlineto{\pgfqpoint{4.203545in}{1.705015in}}%
\pgfpathlineto{\pgfqpoint{4.204411in}{1.719055in}}%
\pgfpathlineto{\pgfqpoint{4.205277in}{1.751795in}}%
\pgfpathlineto{\pgfqpoint{4.207008in}{1.655624in}}%
\pgfpathlineto{\pgfqpoint{4.207872in}{1.702048in}}%
\pgfpathlineto{\pgfqpoint{4.208736in}{1.693916in}}%
\pgfpathlineto{\pgfqpoint{4.209602in}{1.657941in}}%
\pgfpathlineto{\pgfqpoint{4.210467in}{1.738022in}}%
\pgfpathlineto{\pgfqpoint{4.212198in}{1.631582in}}%
\pgfpathlineto{\pgfqpoint{4.213060in}{1.697002in}}%
\pgfpathlineto{\pgfqpoint{4.213925in}{1.696913in}}%
\pgfpathlineto{\pgfqpoint{4.214789in}{1.697359in}}%
\pgfpathlineto{\pgfqpoint{4.216520in}{1.714396in}}%
\pgfpathlineto{\pgfqpoint{4.218250in}{1.668151in}}%
\pgfpathlineto{\pgfqpoint{4.219112in}{1.690770in}}%
\pgfpathlineto{\pgfqpoint{4.219976in}{1.675335in}}%
\pgfpathlineto{\pgfqpoint{4.220842in}{1.759157in}}%
\pgfpathlineto{\pgfqpoint{4.221705in}{1.711429in}}%
\pgfpathlineto{\pgfqpoint{4.222571in}{1.740339in}}%
\pgfpathlineto{\pgfqpoint{4.223436in}{1.702702in}}%
\pgfpathlineto{\pgfqpoint{4.224302in}{1.735293in}}%
\pgfpathlineto{\pgfqpoint{4.225166in}{1.662987in}}%
\pgfpathlineto{\pgfqpoint{4.226894in}{1.701869in}}%
\pgfpathlineto{\pgfqpoint{4.227760in}{1.684003in}}%
\pgfpathlineto{\pgfqpoint{4.228626in}{1.704543in}}%
\pgfpathlineto{\pgfqpoint{4.229490in}{1.683705in}}%
\pgfpathlineto{\pgfqpoint{4.230356in}{1.702315in}}%
\pgfpathlineto{\pgfqpoint{4.231219in}{1.644287in}}%
\pgfpathlineto{\pgfqpoint{4.232949in}{1.718137in}}%
\pgfpathlineto{\pgfqpoint{4.233814in}{1.701159in}}%
\pgfpathlineto{\pgfqpoint{4.234678in}{1.734996in}}%
\pgfpathlineto{\pgfqpoint{4.235543in}{1.681865in}}%
\pgfpathlineto{\pgfqpoint{4.236409in}{1.709053in}}%
\pgfpathlineto{\pgfqpoint{4.237273in}{1.702345in}}%
\pgfpathlineto{\pgfqpoint{4.238138in}{1.690056in}}%
\pgfpathlineto{\pgfqpoint{4.239867in}{1.729117in}}%
\pgfpathlineto{\pgfqpoint{4.240732in}{1.700088in}}%
\pgfpathlineto{\pgfqpoint{4.241594in}{1.714396in}}%
\pgfpathlineto{\pgfqpoint{4.242461in}{1.702167in}}%
\pgfpathlineto{\pgfqpoint{4.243325in}{1.717601in}}%
\pgfpathlineto{\pgfqpoint{4.244190in}{1.714515in}}%
\pgfpathlineto{\pgfqpoint{4.245054in}{1.715909in}}%
\pgfpathlineto{\pgfqpoint{4.245919in}{1.752509in}}%
\pgfpathlineto{\pgfqpoint{4.246784in}{1.707748in}}%
\pgfpathlineto{\pgfqpoint{4.248514in}{1.738439in}}%
\pgfpathlineto{\pgfqpoint{4.249380in}{1.750698in}}%
\pgfpathlineto{\pgfqpoint{4.250245in}{1.713210in}}%
\pgfpathlineto{\pgfqpoint{4.251107in}{1.741882in}}%
\pgfpathlineto{\pgfqpoint{4.252837in}{1.699199in}}%
\pgfpathlineto{\pgfqpoint{4.253702in}{1.633422in}}%
\pgfpathlineto{\pgfqpoint{4.254566in}{1.640279in}}%
\pgfpathlineto{\pgfqpoint{4.255431in}{1.651174in}}%
\pgfpathlineto{\pgfqpoint{4.256297in}{1.639242in}}%
\pgfpathlineto{\pgfqpoint{4.257163in}{1.642090in}}%
\pgfpathlineto{\pgfqpoint{4.258027in}{1.704662in}}%
\pgfpathlineto{\pgfqpoint{4.258891in}{1.625528in}}%
\pgfpathlineto{\pgfqpoint{4.259758in}{1.636274in}}%
\pgfpathlineto{\pgfqpoint{4.260621in}{1.672071in}}%
\pgfpathlineto{\pgfqpoint{4.261484in}{1.654587in}}%
\pgfpathlineto{\pgfqpoint{4.263215in}{1.733809in}}%
\pgfpathlineto{\pgfqpoint{4.264079in}{1.666965in}}%
\pgfpathlineto{\pgfqpoint{4.264945in}{1.756547in}}%
\pgfpathlineto{\pgfqpoint{4.265810in}{1.737253in}}%
\pgfpathlineto{\pgfqpoint{4.266675in}{1.667973in}}%
\pgfpathlineto{\pgfqpoint{4.268406in}{1.723301in}}%
\pgfpathlineto{\pgfqpoint{4.270136in}{1.676997in}}%
\pgfpathlineto{\pgfqpoint{4.271865in}{1.714872in}}%
\pgfpathlineto{\pgfqpoint{4.273593in}{1.646277in}}%
\pgfpathlineto{\pgfqpoint{4.275324in}{1.746336in}}%
\pgfpathlineto{\pgfqpoint{4.276188in}{1.727518in}}%
\pgfpathlineto{\pgfqpoint{4.277054in}{1.667382in}}%
\pgfpathlineto{\pgfqpoint{4.277919in}{1.737520in}}%
\pgfpathlineto{\pgfqpoint{4.278785in}{1.654383in}}%
\pgfpathlineto{\pgfqpoint{4.279650in}{1.744674in}}%
\pgfpathlineto{\pgfqpoint{4.281379in}{1.639955in}}%
\pgfpathlineto{\pgfqpoint{4.282243in}{1.641082in}}%
\pgfpathlineto{\pgfqpoint{4.285704in}{1.527339in}}%
\pgfpathlineto{\pgfqpoint{4.289164in}{1.687386in}}%
\pgfpathlineto{\pgfqpoint{4.290894in}{1.535709in}}%
\pgfpathlineto{\pgfqpoint{4.291759in}{1.626953in}}%
\pgfpathlineto{\pgfqpoint{4.292625in}{1.623569in}}%
\pgfpathlineto{\pgfqpoint{4.293491in}{1.587535in}}%
\pgfpathlineto{\pgfqpoint{4.295221in}{1.611399in}}%
\pgfpathlineto{\pgfqpoint{4.296085in}{1.688810in}}%
\pgfpathlineto{\pgfqpoint{4.297812in}{1.619118in}}%
\pgfpathlineto{\pgfqpoint{4.298678in}{1.632593in}}%
\pgfpathlineto{\pgfqpoint{4.299544in}{1.615020in}}%
\pgfpathlineto{\pgfqpoint{4.301276in}{1.684713in}}%
\pgfpathlineto{\pgfqpoint{4.302140in}{1.599824in}}%
\pgfpathlineto{\pgfqpoint{4.303005in}{1.654319in}}%
\pgfpathlineto{\pgfqpoint{4.303869in}{1.611518in}}%
\pgfpathlineto{\pgfqpoint{4.304734in}{1.657762in}}%
\pgfpathlineto{\pgfqpoint{4.306465in}{1.614307in}}%
\pgfpathlineto{\pgfqpoint{4.307331in}{1.661797in}}%
\pgfpathlineto{\pgfqpoint{4.308196in}{1.655565in}}%
\pgfpathlineto{\pgfqpoint{4.311656in}{1.603326in}}%
\pgfpathlineto{\pgfqpoint{4.313387in}{1.646898in}}%
\pgfpathlineto{\pgfqpoint{4.314253in}{1.600177in}}%
\pgfpathlineto{\pgfqpoint{4.316847in}{1.676581in}}%
\pgfpathlineto{\pgfqpoint{4.317712in}{1.664887in}}%
\pgfpathlineto{\pgfqpoint{4.318576in}{1.607866in}}%
\pgfpathlineto{\pgfqpoint{4.319442in}{1.631701in}}%
\pgfpathlineto{\pgfqpoint{4.320307in}{1.580470in}}%
\pgfpathlineto{\pgfqpoint{4.321170in}{1.589346in}}%
\pgfpathlineto{\pgfqpoint{4.323763in}{1.683289in}}%
\pgfpathlineto{\pgfqpoint{4.324629in}{1.679667in}}%
\pgfpathlineto{\pgfqpoint{4.328083in}{1.604988in}}%
\pgfpathlineto{\pgfqpoint{4.329813in}{1.619947in}}%
\pgfpathlineto{\pgfqpoint{4.330678in}{1.602702in}}%
\pgfpathlineto{\pgfqpoint{4.331544in}{1.611458in}}%
\pgfpathlineto{\pgfqpoint{4.332409in}{1.654557in}}%
\pgfpathlineto{\pgfqpoint{4.333276in}{1.583735in}}%
\pgfpathlineto{\pgfqpoint{4.334141in}{1.618285in}}%
\pgfpathlineto{\pgfqpoint{4.335005in}{1.574119in}}%
\pgfpathlineto{\pgfqpoint{4.336733in}{1.645533in}}%
\pgfpathlineto{\pgfqpoint{4.338461in}{1.594540in}}%
\pgfpathlineto{\pgfqpoint{4.339326in}{1.645652in}}%
\pgfpathlineto{\pgfqpoint{4.340193in}{1.591037in}}%
\pgfpathlineto{\pgfqpoint{4.341058in}{1.646128in}}%
\pgfpathlineto{\pgfqpoint{4.341924in}{1.582132in}}%
\pgfpathlineto{\pgfqpoint{4.342786in}{1.616385in}}%
\pgfpathlineto{\pgfqpoint{4.343648in}{1.575008in}}%
\pgfpathlineto{\pgfqpoint{4.345375in}{1.652717in}}%
\pgfpathlineto{\pgfqpoint{4.347106in}{1.587713in}}%
\pgfpathlineto{\pgfqpoint{4.347971in}{1.623450in}}%
\pgfpathlineto{\pgfqpoint{4.348837in}{1.739625in}}%
\pgfpathlineto{\pgfqpoint{4.349701in}{1.621907in}}%
\pgfpathlineto{\pgfqpoint{4.350566in}{1.623628in}}%
\pgfpathlineto{\pgfqpoint{4.351432in}{1.614901in}}%
\pgfpathlineto{\pgfqpoint{4.352298in}{1.621609in}}%
\pgfpathlineto{\pgfqpoint{4.353162in}{1.585635in}}%
\pgfpathlineto{\pgfqpoint{4.354027in}{1.623539in}}%
\pgfpathlineto{\pgfqpoint{4.354892in}{1.597507in}}%
\pgfpathlineto{\pgfqpoint{4.355757in}{1.684118in}}%
\pgfpathlineto{\pgfqpoint{4.356622in}{1.605788in}}%
\pgfpathlineto{\pgfqpoint{4.357487in}{1.613890in}}%
\pgfpathlineto{\pgfqpoint{4.359212in}{1.686137in}}%
\pgfpathlineto{\pgfqpoint{4.360077in}{1.668267in}}%
\pgfpathlineto{\pgfqpoint{4.360942in}{1.686375in}}%
\pgfpathlineto{\pgfqpoint{4.361807in}{1.604096in}}%
\pgfpathlineto{\pgfqpoint{4.362672in}{1.653665in}}%
\pgfpathlineto{\pgfqpoint{4.363537in}{1.593410in}}%
\pgfpathlineto{\pgfqpoint{4.365264in}{1.638290in}}%
\pgfpathlineto{\pgfqpoint{4.366126in}{1.585159in}}%
\pgfpathlineto{\pgfqpoint{4.366989in}{1.593529in}}%
\pgfpathlineto{\pgfqpoint{4.367853in}{1.581775in}}%
\pgfpathlineto{\pgfqpoint{4.368718in}{1.632917in}}%
\pgfpathlineto{\pgfqpoint{4.369583in}{1.582548in}}%
\pgfpathlineto{\pgfqpoint{4.371313in}{1.632058in}}%
\pgfpathlineto{\pgfqpoint{4.372179in}{1.587237in}}%
\pgfpathlineto{\pgfqpoint{4.373044in}{1.617720in}}%
\pgfpathlineto{\pgfqpoint{4.373909in}{1.540576in}}%
\pgfpathlineto{\pgfqpoint{4.374774in}{1.663935in}}%
\pgfpathlineto{\pgfqpoint{4.375640in}{1.614723in}}%
\pgfpathlineto{\pgfqpoint{4.376505in}{1.648381in}}%
\pgfpathlineto{\pgfqpoint{4.377370in}{1.630217in}}%
\pgfpathlineto{\pgfqpoint{4.378235in}{1.533987in}}%
\pgfpathlineto{\pgfqpoint{4.379100in}{1.655446in}}%
\pgfpathlineto{\pgfqpoint{4.379961in}{1.654554in}}%
\pgfpathlineto{\pgfqpoint{4.380827in}{1.636092in}}%
\pgfpathlineto{\pgfqpoint{4.381692in}{1.655386in}}%
\pgfpathlineto{\pgfqpoint{4.382557in}{1.586880in}}%
\pgfpathlineto{\pgfqpoint{4.384290in}{1.657168in}}%
\pgfpathlineto{\pgfqpoint{4.386020in}{1.634252in}}%
\pgfpathlineto{\pgfqpoint{4.386886in}{1.653486in}}%
\pgfpathlineto{\pgfqpoint{4.388616in}{1.572870in}}%
\pgfpathlineto{\pgfqpoint{4.389481in}{1.621371in}}%
\pgfpathlineto{\pgfqpoint{4.390346in}{1.600474in}}%
\pgfpathlineto{\pgfqpoint{4.391212in}{1.667200in}}%
\pgfpathlineto{\pgfqpoint{4.392939in}{1.592342in}}%
\pgfpathlineto{\pgfqpoint{4.394670in}{1.609588in}}%
\pgfpathlineto{\pgfqpoint{4.395533in}{1.672781in}}%
\pgfpathlineto{\pgfqpoint{4.396398in}{1.557554in}}%
\pgfpathlineto{\pgfqpoint{4.397262in}{1.576402in}}%
\pgfpathlineto{\pgfqpoint{4.398127in}{1.609971in}}%
\pgfpathlineto{\pgfqpoint{4.399855in}{1.567526in}}%
\pgfpathlineto{\pgfqpoint{4.400721in}{1.625584in}}%
\pgfpathlineto{\pgfqpoint{4.401586in}{1.619412in}}%
\pgfpathlineto{\pgfqpoint{4.402453in}{1.578391in}}%
\pgfpathlineto{\pgfqpoint{4.403318in}{1.619828in}}%
\pgfpathlineto{\pgfqpoint{4.405913in}{1.588721in}}%
\pgfpathlineto{\pgfqpoint{4.406779in}{1.586999in}}%
\pgfpathlineto{\pgfqpoint{4.407644in}{1.606056in}}%
\pgfpathlineto{\pgfqpoint{4.408510in}{1.679370in}}%
\pgfpathlineto{\pgfqpoint{4.409376in}{1.593588in}}%
\pgfpathlineto{\pgfqpoint{4.410240in}{1.623093in}}%
\pgfpathlineto{\pgfqpoint{4.412836in}{1.587178in}}%
\pgfpathlineto{\pgfqpoint{4.413701in}{1.681299in}}%
\pgfpathlineto{\pgfqpoint{4.414565in}{1.680143in}}%
\pgfpathlineto{\pgfqpoint{4.416295in}{1.590889in}}%
\pgfpathlineto{\pgfqpoint{4.418026in}{1.640249in}}%
\pgfpathlineto{\pgfqpoint{4.420621in}{1.561473in}}%
\pgfpathlineto{\pgfqpoint{4.422347in}{1.634965in}}%
\pgfpathlineto{\pgfqpoint{4.423211in}{1.612942in}}%
\pgfpathlineto{\pgfqpoint{4.424076in}{1.631760in}}%
\pgfpathlineto{\pgfqpoint{4.425802in}{1.513656in}}%
\pgfpathlineto{\pgfqpoint{4.427532in}{1.626417in}}%
\pgfpathlineto{\pgfqpoint{4.428398in}{1.605639in}}%
\pgfpathlineto{\pgfqpoint{4.429263in}{1.667557in}}%
\pgfpathlineto{\pgfqpoint{4.430129in}{1.631731in}}%
\pgfpathlineto{\pgfqpoint{4.430994in}{1.670583in}}%
\pgfpathlineto{\pgfqpoint{4.431859in}{1.648679in}}%
\pgfpathlineto{\pgfqpoint{4.432725in}{1.650043in}}%
\pgfpathlineto{\pgfqpoint{4.433588in}{1.676934in}}%
\pgfpathlineto{\pgfqpoint{4.435319in}{1.561384in}}%
\pgfpathlineto{\pgfqpoint{4.437050in}{1.617690in}}%
\pgfpathlineto{\pgfqpoint{4.437915in}{1.643038in}}%
\pgfpathlineto{\pgfqpoint{4.438780in}{1.626358in}}%
\pgfpathlineto{\pgfqpoint{4.439645in}{1.533690in}}%
\pgfpathlineto{\pgfqpoint{4.441371in}{1.603918in}}%
\pgfpathlineto{\pgfqpoint{4.442235in}{1.586345in}}%
\pgfpathlineto{\pgfqpoint{4.443101in}{1.597091in}}%
\pgfpathlineto{\pgfqpoint{4.443964in}{1.635322in}}%
\pgfpathlineto{\pgfqpoint{4.445695in}{1.585278in}}%
\pgfpathlineto{\pgfqpoint{4.446561in}{1.590026in}}%
\pgfpathlineto{\pgfqpoint{4.447427in}{1.619590in}}%
\pgfpathlineto{\pgfqpoint{4.448292in}{1.616564in}}%
\pgfpathlineto{\pgfqpoint{4.449156in}{1.561473in}}%
\pgfpathlineto{\pgfqpoint{4.450020in}{1.600355in}}%
\pgfpathlineto{\pgfqpoint{4.450885in}{1.538498in}}%
\pgfpathlineto{\pgfqpoint{4.452617in}{1.619531in}}%
\pgfpathlineto{\pgfqpoint{4.454349in}{1.530455in}}%
\pgfpathlineto{\pgfqpoint{4.455213in}{1.619471in}}%
\pgfpathlineto{\pgfqpoint{4.456077in}{1.587237in}}%
\pgfpathlineto{\pgfqpoint{4.456943in}{1.612525in}}%
\pgfpathlineto{\pgfqpoint{4.459537in}{1.556249in}}%
\pgfpathlineto{\pgfqpoint{4.460402in}{1.619828in}}%
\pgfpathlineto{\pgfqpoint{4.461266in}{1.599615in}}%
\pgfpathlineto{\pgfqpoint{4.462131in}{1.616861in}}%
\pgfpathlineto{\pgfqpoint{4.462995in}{1.554170in}}%
\pgfpathlineto{\pgfqpoint{4.464723in}{1.600415in}}%
\pgfpathlineto{\pgfqpoint{4.465588in}{1.599348in}}%
\pgfpathlineto{\pgfqpoint{4.466453in}{1.616207in}}%
\pgfpathlineto{\pgfqpoint{4.468183in}{1.573584in}}%
\pgfpathlineto{\pgfqpoint{4.469913in}{1.617690in}}%
\pgfpathlineto{\pgfqpoint{4.471642in}{1.555713in}}%
\pgfpathlineto{\pgfqpoint{4.472507in}{1.641673in}}%
\pgfpathlineto{\pgfqpoint{4.473373in}{1.611220in}}%
\pgfpathlineto{\pgfqpoint{4.474235in}{1.656160in}}%
\pgfpathlineto{\pgfqpoint{4.475100in}{1.636096in}}%
\pgfpathlineto{\pgfqpoint{4.475966in}{1.690651in}}%
\pgfpathlineto{\pgfqpoint{4.479425in}{1.563730in}}%
\pgfpathlineto{\pgfqpoint{4.480290in}{1.563671in}}%
\pgfpathlineto{\pgfqpoint{4.482023in}{1.662332in}}%
\pgfpathlineto{\pgfqpoint{4.482888in}{1.576967in}}%
\pgfpathlineto{\pgfqpoint{4.483754in}{1.645771in}}%
\pgfpathlineto{\pgfqpoint{4.484620in}{1.631998in}}%
\pgfpathlineto{\pgfqpoint{4.485486in}{1.640190in}}%
\pgfpathlineto{\pgfqpoint{4.487216in}{1.691480in}}%
\pgfpathlineto{\pgfqpoint{4.488081in}{1.582608in}}%
\pgfpathlineto{\pgfqpoint{4.488945in}{1.613299in}}%
\pgfpathlineto{\pgfqpoint{4.489808in}{1.613477in}}%
\pgfpathlineto{\pgfqpoint{4.491537in}{1.702464in}}%
\pgfpathlineto{\pgfqpoint{4.492402in}{1.632088in}}%
\pgfpathlineto{\pgfqpoint{4.493266in}{1.636925in}}%
\pgfpathlineto{\pgfqpoint{4.494133in}{1.725555in}}%
\pgfpathlineto{\pgfqpoint{4.494998in}{1.601958in}}%
\pgfpathlineto{\pgfqpoint{4.495863in}{1.705193in}}%
\pgfpathlineto{\pgfqpoint{4.498455in}{1.627841in}}%
\pgfpathlineto{\pgfqpoint{4.500183in}{1.672662in}}%
\pgfpathlineto{\pgfqpoint{4.501914in}{1.591896in}}%
\pgfpathlineto{\pgfqpoint{4.504511in}{1.646660in}}%
\pgfpathlineto{\pgfqpoint{4.506240in}{1.596972in}}%
\pgfpathlineto{\pgfqpoint{4.507970in}{1.679013in}}%
\pgfpathlineto{\pgfqpoint{4.508835in}{1.586851in}}%
\pgfpathlineto{\pgfqpoint{4.509700in}{1.623152in}}%
\pgfpathlineto{\pgfqpoint{4.510564in}{1.587118in}}%
\pgfpathlineto{\pgfqpoint{4.511429in}{1.604126in}}%
\pgfpathlineto{\pgfqpoint{4.512294in}{1.651824in}}%
\pgfpathlineto{\pgfqpoint{4.514025in}{1.583913in}}%
\pgfpathlineto{\pgfqpoint{4.514890in}{1.578570in}}%
\pgfpathlineto{\pgfqpoint{4.516619in}{1.494153in}}%
\pgfpathlineto{\pgfqpoint{4.517482in}{1.600474in}}%
\pgfpathlineto{\pgfqpoint{4.518346in}{1.595964in}}%
\pgfpathlineto{\pgfqpoint{4.519212in}{1.544317in}}%
\pgfpathlineto{\pgfqpoint{4.520942in}{1.602196in}}%
\pgfpathlineto{\pgfqpoint{4.522670in}{1.640309in}}%
\pgfpathlineto{\pgfqpoint{4.523535in}{1.516712in}}%
\pgfpathlineto{\pgfqpoint{4.524400in}{1.658889in}}%
\pgfpathlineto{\pgfqpoint{4.525263in}{1.569545in}}%
\pgfpathlineto{\pgfqpoint{4.526128in}{1.576848in}}%
\pgfpathlineto{\pgfqpoint{4.526993in}{1.576908in}}%
\pgfpathlineto{\pgfqpoint{4.530452in}{1.612972in}}%
\pgfpathlineto{\pgfqpoint{4.531316in}{1.615909in}}%
\pgfpathlineto{\pgfqpoint{4.532179in}{1.653486in}}%
\pgfpathlineto{\pgfqpoint{4.533043in}{1.538855in}}%
\pgfpathlineto{\pgfqpoint{4.533909in}{1.671059in}}%
\pgfpathlineto{\pgfqpoint{4.534774in}{1.578867in}}%
\pgfpathlineto{\pgfqpoint{4.535638in}{1.600534in}}%
\pgfpathlineto{\pgfqpoint{4.536505in}{1.631284in}}%
\pgfpathlineto{\pgfqpoint{4.538236in}{1.588364in}}%
\pgfpathlineto{\pgfqpoint{4.539965in}{1.620836in}}%
\pgfpathlineto{\pgfqpoint{4.540828in}{1.603858in}}%
\pgfpathlineto{\pgfqpoint{4.541691in}{1.605580in}}%
\pgfpathlineto{\pgfqpoint{4.542555in}{1.668386in}}%
\pgfpathlineto{\pgfqpoint{4.543419in}{1.656989in}}%
\pgfpathlineto{\pgfqpoint{4.544282in}{1.603590in}}%
\pgfpathlineto{\pgfqpoint{4.545145in}{1.673134in}}%
\pgfpathlineto{\pgfqpoint{4.546010in}{1.623446in}}%
\pgfpathlineto{\pgfqpoint{4.546874in}{1.689015in}}%
\pgfpathlineto{\pgfqpoint{4.548603in}{1.624279in}}%
\pgfpathlineto{\pgfqpoint{4.549468in}{1.653962in}}%
\pgfpathlineto{\pgfqpoint{4.550331in}{1.650400in}}%
\pgfpathlineto{\pgfqpoint{4.552060in}{1.598277in}}%
\pgfpathlineto{\pgfqpoint{4.552925in}{1.677113in}}%
\pgfpathlineto{\pgfqpoint{4.554650in}{1.615909in}}%
\pgfpathlineto{\pgfqpoint{4.555516in}{1.608963in}}%
\pgfpathlineto{\pgfqpoint{4.556379in}{1.629741in}}%
\pgfpathlineto{\pgfqpoint{4.558975in}{1.542268in}}%
\pgfpathlineto{\pgfqpoint{4.559839in}{1.631820in}}%
\pgfpathlineto{\pgfqpoint{4.560705in}{1.553873in}}%
\pgfpathlineto{\pgfqpoint{4.561571in}{1.574889in}}%
\pgfpathlineto{\pgfqpoint{4.562436in}{1.586940in}}%
\pgfpathlineto{\pgfqpoint{4.563300in}{1.585605in}}%
\pgfpathlineto{\pgfqpoint{4.565028in}{1.599288in}}%
\pgfpathlineto{\pgfqpoint{4.565894in}{1.579046in}}%
\pgfpathlineto{\pgfqpoint{4.568487in}{1.701929in}}%
\pgfpathlineto{\pgfqpoint{4.569351in}{1.615734in}}%
\pgfpathlineto{\pgfqpoint{4.570217in}{1.620304in}}%
\pgfpathlineto{\pgfqpoint{4.571084in}{1.668211in}}%
\pgfpathlineto{\pgfqpoint{4.572814in}{1.612793in}}%
\pgfpathlineto{\pgfqpoint{4.574545in}{1.659365in}}%
\pgfpathlineto{\pgfqpoint{4.575410in}{1.571713in}}%
\pgfpathlineto{\pgfqpoint{4.576275in}{1.668386in}}%
\pgfpathlineto{\pgfqpoint{4.578869in}{1.566697in}}%
\pgfpathlineto{\pgfqpoint{4.579732in}{1.740752in}}%
\pgfpathlineto{\pgfqpoint{4.580597in}{1.610566in}}%
\pgfpathlineto{\pgfqpoint{4.581462in}{1.640368in}}%
\pgfpathlineto{\pgfqpoint{4.583189in}{1.565154in}}%
\pgfpathlineto{\pgfqpoint{4.584053in}{1.624993in}}%
\pgfpathlineto{\pgfqpoint{4.584916in}{1.540993in}}%
\pgfpathlineto{\pgfqpoint{4.586645in}{1.604691in}}%
\pgfpathlineto{\pgfqpoint{4.587511in}{1.604750in}}%
\pgfpathlineto{\pgfqpoint{4.588377in}{1.568478in}}%
\pgfpathlineto{\pgfqpoint{4.589242in}{1.643930in}}%
\pgfpathlineto{\pgfqpoint{4.590109in}{1.624815in}}%
\pgfpathlineto{\pgfqpoint{4.590975in}{1.634965in}}%
\pgfpathlineto{\pgfqpoint{4.592706in}{1.583675in}}%
\pgfpathlineto{\pgfqpoint{4.593571in}{1.643633in}}%
\pgfpathlineto{\pgfqpoint{4.597029in}{1.521579in}}%
\pgfpathlineto{\pgfqpoint{4.597894in}{1.623863in}}%
\pgfpathlineto{\pgfqpoint{4.598760in}{1.595667in}}%
\pgfpathlineto{\pgfqpoint{4.599625in}{1.525141in}}%
\pgfpathlineto{\pgfqpoint{4.601354in}{1.578927in}}%
\pgfpathlineto{\pgfqpoint{4.602218in}{1.581422in}}%
\pgfpathlineto{\pgfqpoint{4.603082in}{1.654438in}}%
\pgfpathlineto{\pgfqpoint{4.603947in}{1.555955in}}%
\pgfpathlineto{\pgfqpoint{4.604812in}{1.566106in}}%
\pgfpathlineto{\pgfqpoint{4.605675in}{1.615675in}}%
\pgfpathlineto{\pgfqpoint{4.606541in}{1.543398in}}%
\pgfpathlineto{\pgfqpoint{4.607407in}{1.545566in}}%
\pgfpathlineto{\pgfqpoint{4.608272in}{1.525855in}}%
\pgfpathlineto{\pgfqpoint{4.609137in}{1.532920in}}%
\pgfpathlineto{\pgfqpoint{4.610002in}{1.610213in}}%
\pgfpathlineto{\pgfqpoint{4.610868in}{1.520482in}}%
\pgfpathlineto{\pgfqpoint{4.612594in}{1.649036in}}%
\pgfpathlineto{\pgfqpoint{4.614321in}{1.542833in}}%
\pgfpathlineto{\pgfqpoint{4.616052in}{1.607837in}}%
\pgfpathlineto{\pgfqpoint{4.616918in}{1.558268in}}%
\pgfpathlineto{\pgfqpoint{4.617784in}{1.558446in}}%
\pgfpathlineto{\pgfqpoint{4.618650in}{1.532682in}}%
\pgfpathlineto{\pgfqpoint{4.619515in}{1.610213in}}%
\pgfpathlineto{\pgfqpoint{4.620380in}{1.605226in}}%
\pgfpathlineto{\pgfqpoint{4.621246in}{1.644406in}}%
\pgfpathlineto{\pgfqpoint{4.622110in}{1.629269in}}%
\pgfpathlineto{\pgfqpoint{4.622974in}{1.587088in}}%
\pgfpathlineto{\pgfqpoint{4.624705in}{1.604334in}}%
\pgfpathlineto{\pgfqpoint{4.625570in}{1.596053in}}%
\pgfpathlineto{\pgfqpoint{4.626436in}{1.676997in}}%
\pgfpathlineto{\pgfqpoint{4.627301in}{1.630098in}}%
\pgfpathlineto{\pgfqpoint{4.628165in}{1.649512in}}%
\pgfpathlineto{\pgfqpoint{4.629028in}{1.638766in}}%
\pgfpathlineto{\pgfqpoint{4.629894in}{1.594778in}}%
\pgfpathlineto{\pgfqpoint{4.630756in}{1.650519in}}%
\pgfpathlineto{\pgfqpoint{4.631622in}{1.645890in}}%
\pgfpathlineto{\pgfqpoint{4.632489in}{1.659930in}}%
\pgfpathlineto{\pgfqpoint{4.633355in}{1.641082in}}%
\pgfpathlineto{\pgfqpoint{4.634220in}{1.574357in}}%
\pgfpathlineto{\pgfqpoint{4.635081in}{1.574387in}}%
\pgfpathlineto{\pgfqpoint{4.635945in}{1.651293in}}%
\pgfpathlineto{\pgfqpoint{4.637675in}{1.621847in}}%
\pgfpathlineto{\pgfqpoint{4.638541in}{1.663225in}}%
\pgfpathlineto{\pgfqpoint{4.640272in}{1.518969in}}%
\pgfpathlineto{\pgfqpoint{4.641137in}{1.622442in}}%
\pgfpathlineto{\pgfqpoint{4.642003in}{1.608313in}}%
\pgfpathlineto{\pgfqpoint{4.642868in}{1.618464in}}%
\pgfpathlineto{\pgfqpoint{4.643733in}{1.669992in}}%
\pgfpathlineto{\pgfqpoint{4.647191in}{1.526331in}}%
\pgfpathlineto{\pgfqpoint{4.648057in}{1.622799in}}%
\pgfpathlineto{\pgfqpoint{4.648923in}{1.557587in}}%
\pgfpathlineto{\pgfqpoint{4.649788in}{1.576852in}}%
\pgfpathlineto{\pgfqpoint{4.650652in}{1.569430in}}%
\pgfpathlineto{\pgfqpoint{4.652381in}{1.617337in}}%
\pgfpathlineto{\pgfqpoint{4.654112in}{1.569044in}}%
\pgfpathlineto{\pgfqpoint{4.654976in}{1.563909in}}%
\pgfpathlineto{\pgfqpoint{4.655843in}{1.602493in}}%
\pgfpathlineto{\pgfqpoint{4.656707in}{1.542536in}}%
\pgfpathlineto{\pgfqpoint{4.658437in}{1.609618in}}%
\pgfpathlineto{\pgfqpoint{4.660168in}{1.562008in}}%
\pgfpathlineto{\pgfqpoint{4.662761in}{1.647552in}}%
\pgfpathlineto{\pgfqpoint{4.663625in}{1.573970in}}%
\pgfpathlineto{\pgfqpoint{4.664490in}{1.575841in}}%
\pgfpathlineto{\pgfqpoint{4.665354in}{1.560525in}}%
\pgfpathlineto{\pgfqpoint{4.666218in}{1.589316in}}%
\pgfpathlineto{\pgfqpoint{4.667082in}{1.580589in}}%
\pgfpathlineto{\pgfqpoint{4.667948in}{1.532682in}}%
\pgfpathlineto{\pgfqpoint{4.668812in}{1.640190in}}%
\pgfpathlineto{\pgfqpoint{4.669677in}{1.561592in}}%
\pgfpathlineto{\pgfqpoint{4.670542in}{1.599318in}}%
\pgfpathlineto{\pgfqpoint{4.672272in}{1.567590in}}%
\pgfpathlineto{\pgfqpoint{4.674000in}{1.589613in}}%
\pgfpathlineto{\pgfqpoint{4.674866in}{1.629269in}}%
\pgfpathlineto{\pgfqpoint{4.675730in}{1.566816in}}%
\pgfpathlineto{\pgfqpoint{4.677460in}{1.611369in}}%
\pgfpathlineto{\pgfqpoint{4.678325in}{1.575722in}}%
\pgfpathlineto{\pgfqpoint{4.679190in}{1.618583in}}%
\pgfpathlineto{\pgfqpoint{4.680055in}{1.569222in}}%
\pgfpathlineto{\pgfqpoint{4.682650in}{1.632474in}}%
\pgfpathlineto{\pgfqpoint{4.683516in}{1.626123in}}%
\pgfpathlineto{\pgfqpoint{4.684380in}{1.570616in}}%
\pgfpathlineto{\pgfqpoint{4.685244in}{1.606532in}}%
\pgfpathlineto{\pgfqpoint{4.686110in}{1.592878in}}%
\pgfpathlineto{\pgfqpoint{4.687840in}{1.641023in}}%
\pgfpathlineto{\pgfqpoint{4.688705in}{1.609439in}}%
\pgfpathlineto{\pgfqpoint{4.690432in}{1.680084in}}%
\pgfpathlineto{\pgfqpoint{4.693027in}{1.540398in}}%
\pgfpathlineto{\pgfqpoint{4.693889in}{1.631582in}}%
\pgfpathlineto{\pgfqpoint{4.694755in}{1.576729in}}%
\pgfpathlineto{\pgfqpoint{4.695621in}{1.627901in}}%
\pgfpathlineto{\pgfqpoint{4.696486in}{1.590651in}}%
\pgfpathlineto{\pgfqpoint{4.697352in}{1.612823in}}%
\pgfpathlineto{\pgfqpoint{4.698218in}{1.591275in}}%
\pgfpathlineto{\pgfqpoint{4.699949in}{1.619888in}}%
\pgfpathlineto{\pgfqpoint{4.700812in}{1.620245in}}%
\pgfpathlineto{\pgfqpoint{4.701675in}{1.545919in}}%
\pgfpathlineto{\pgfqpoint{4.702540in}{1.595905in}}%
\pgfpathlineto{\pgfqpoint{4.704271in}{1.538736in}}%
\pgfpathlineto{\pgfqpoint{4.705999in}{1.698128in}}%
\pgfpathlineto{\pgfqpoint{4.708594in}{1.665835in}}%
\pgfpathlineto{\pgfqpoint{4.709461in}{1.728700in}}%
\pgfpathlineto{\pgfqpoint{4.710325in}{1.727395in}}%
\pgfpathlineto{\pgfqpoint{4.711190in}{1.626179in}}%
\pgfpathlineto{\pgfqpoint{4.712921in}{1.698485in}}%
\pgfpathlineto{\pgfqpoint{4.713785in}{1.645533in}}%
\pgfpathlineto{\pgfqpoint{4.714648in}{1.686137in}}%
\pgfpathlineto{\pgfqpoint{4.715513in}{1.637371in}}%
\pgfpathlineto{\pgfqpoint{4.718111in}{1.704926in}}%
\pgfpathlineto{\pgfqpoint{4.718976in}{1.633303in}}%
\pgfpathlineto{\pgfqpoint{4.719839in}{1.690472in}}%
\pgfpathlineto{\pgfqpoint{4.720705in}{1.653516in}}%
\pgfpathlineto{\pgfqpoint{4.721569in}{1.667557in}}%
\pgfpathlineto{\pgfqpoint{4.723300in}{1.744373in}}%
\pgfpathlineto{\pgfqpoint{4.724166in}{1.724666in}}%
\pgfpathlineto{\pgfqpoint{4.725030in}{1.648887in}}%
\pgfpathlineto{\pgfqpoint{4.725894in}{1.706260in}}%
\pgfpathlineto{\pgfqpoint{4.726759in}{1.659008in}}%
\pgfpathlineto{\pgfqpoint{4.728487in}{1.700326in}}%
\pgfpathlineto{\pgfqpoint{4.729353in}{1.613001in}}%
\pgfpathlineto{\pgfqpoint{4.731080in}{1.673078in}}%
\pgfpathlineto{\pgfqpoint{4.731947in}{1.624993in}}%
\pgfpathlineto{\pgfqpoint{4.732812in}{1.688453in}}%
\pgfpathlineto{\pgfqpoint{4.733678in}{1.683289in}}%
\pgfpathlineto{\pgfqpoint{4.734543in}{1.639093in}}%
\pgfpathlineto{\pgfqpoint{4.736271in}{1.720453in}}%
\pgfpathlineto{\pgfqpoint{4.737133in}{1.678868in}}%
\pgfpathlineto{\pgfqpoint{4.737997in}{1.724904in}}%
\pgfpathlineto{\pgfqpoint{4.739729in}{1.650846in}}%
\pgfpathlineto{\pgfqpoint{4.740595in}{1.762719in}}%
\pgfpathlineto{\pgfqpoint{4.741460in}{1.633660in}}%
\pgfpathlineto{\pgfqpoint{4.742327in}{1.707153in}}%
\pgfpathlineto{\pgfqpoint{4.743193in}{1.637460in}}%
\pgfpathlineto{\pgfqpoint{4.744060in}{1.705967in}}%
\pgfpathlineto{\pgfqpoint{4.744925in}{1.665363in}}%
\pgfpathlineto{\pgfqpoint{4.745789in}{1.686200in}}%
\pgfpathlineto{\pgfqpoint{4.746655in}{1.600567in}}%
\pgfpathlineto{\pgfqpoint{4.747518in}{1.692492in}}%
\pgfpathlineto{\pgfqpoint{4.748384in}{1.674268in}}%
\pgfpathlineto{\pgfqpoint{4.749246in}{1.644644in}}%
\pgfpathlineto{\pgfqpoint{4.750977in}{1.746630in}}%
\pgfpathlineto{\pgfqpoint{4.751842in}{1.683527in}}%
\pgfpathlineto{\pgfqpoint{4.752708in}{1.762184in}}%
\pgfpathlineto{\pgfqpoint{4.754440in}{1.688751in}}%
\pgfpathlineto{\pgfqpoint{4.755305in}{1.705550in}}%
\pgfpathlineto{\pgfqpoint{4.756170in}{1.664173in}}%
\pgfpathlineto{\pgfqpoint{4.757035in}{1.667438in}}%
\pgfpathlineto{\pgfqpoint{4.757901in}{1.693797in}}%
\pgfpathlineto{\pgfqpoint{4.758765in}{1.688453in}}%
\pgfpathlineto{\pgfqpoint{4.759630in}{1.818401in}}%
\pgfpathlineto{\pgfqpoint{4.761358in}{1.609558in}}%
\pgfpathlineto{\pgfqpoint{4.762223in}{1.631582in}}%
\pgfpathlineto{\pgfqpoint{4.763088in}{1.602702in}}%
\pgfpathlineto{\pgfqpoint{4.763954in}{1.705907in}}%
\pgfpathlineto{\pgfqpoint{4.766549in}{1.592640in}}%
\pgfpathlineto{\pgfqpoint{4.767415in}{1.659484in}}%
\pgfpathlineto{\pgfqpoint{4.770012in}{1.557911in}}%
\pgfpathlineto{\pgfqpoint{4.771744in}{1.640309in}}%
\pgfpathlineto{\pgfqpoint{4.773477in}{1.607896in}}%
\pgfpathlineto{\pgfqpoint{4.774342in}{1.566043in}}%
\pgfpathlineto{\pgfqpoint{4.775208in}{1.612853in}}%
\pgfpathlineto{\pgfqpoint{4.776073in}{1.585694in}}%
\pgfpathlineto{\pgfqpoint{4.776937in}{1.643811in}}%
\pgfpathlineto{\pgfqpoint{4.777802in}{1.559603in}}%
\pgfpathlineto{\pgfqpoint{4.778668in}{1.640249in}}%
\pgfpathlineto{\pgfqpoint{4.781267in}{1.552806in}}%
\pgfpathlineto{\pgfqpoint{4.782133in}{1.636895in}}%
\pgfpathlineto{\pgfqpoint{4.783865in}{1.561176in}}%
\pgfpathlineto{\pgfqpoint{4.784730in}{1.649036in}}%
\pgfpathlineto{\pgfqpoint{4.785596in}{1.624874in}}%
\pgfpathlineto{\pgfqpoint{4.786461in}{1.628198in}}%
\pgfpathlineto{\pgfqpoint{4.787327in}{1.585813in}}%
\pgfpathlineto{\pgfqpoint{4.788193in}{1.670940in}}%
\pgfpathlineto{\pgfqpoint{4.789925in}{1.574948in}}%
\pgfpathlineto{\pgfqpoint{4.790791in}{1.606115in}}%
\pgfpathlineto{\pgfqpoint{4.792520in}{1.527815in}}%
\pgfpathlineto{\pgfqpoint{4.793384in}{1.622855in}}%
\pgfpathlineto{\pgfqpoint{4.794249in}{1.544703in}}%
\pgfpathlineto{\pgfqpoint{4.795115in}{1.583794in}}%
\pgfpathlineto{\pgfqpoint{4.795980in}{1.704126in}}%
\pgfpathlineto{\pgfqpoint{4.796845in}{1.663225in}}%
\pgfpathlineto{\pgfqpoint{4.798574in}{1.786464in}}%
\pgfpathlineto{\pgfqpoint{4.801166in}{1.644585in}}%
\pgfpathlineto{\pgfqpoint{4.802893in}{1.697894in}}%
\pgfpathlineto{\pgfqpoint{4.803757in}{1.675365in}}%
\pgfpathlineto{\pgfqpoint{4.804623in}{1.705967in}}%
\pgfpathlineto{\pgfqpoint{4.806350in}{1.532742in}}%
\pgfpathlineto{\pgfqpoint{4.808078in}{1.650698in}}%
\pgfpathlineto{\pgfqpoint{4.808943in}{1.592997in}}%
\pgfpathlineto{\pgfqpoint{4.809809in}{1.595075in}}%
\pgfpathlineto{\pgfqpoint{4.812399in}{1.733452in}}%
\pgfpathlineto{\pgfqpoint{4.813264in}{1.684032in}}%
\pgfpathlineto{\pgfqpoint{4.814129in}{1.690651in}}%
\pgfpathlineto{\pgfqpoint{4.814993in}{1.743841in}}%
\pgfpathlineto{\pgfqpoint{4.815858in}{1.669814in}}%
\pgfpathlineto{\pgfqpoint{4.816722in}{1.745147in}}%
\pgfpathlineto{\pgfqpoint{4.818454in}{1.616207in}}%
\pgfpathlineto{\pgfqpoint{4.820184in}{1.693945in}}%
\pgfpathlineto{\pgfqpoint{4.821048in}{1.661146in}}%
\pgfpathlineto{\pgfqpoint{4.821914in}{1.690710in}}%
\pgfpathlineto{\pgfqpoint{4.822781in}{1.690562in}}%
\pgfpathlineto{\pgfqpoint{4.824513in}{1.620840in}}%
\pgfpathlineto{\pgfqpoint{4.825379in}{1.630782in}}%
\pgfpathlineto{\pgfqpoint{4.826244in}{1.629269in}}%
\pgfpathlineto{\pgfqpoint{4.827110in}{1.714575in}}%
\pgfpathlineto{\pgfqpoint{4.827976in}{1.582846in}}%
\pgfpathlineto{\pgfqpoint{4.830573in}{1.694154in}}%
\pgfpathlineto{\pgfqpoint{4.831440in}{1.735115in}}%
\pgfpathlineto{\pgfqpoint{4.832305in}{1.532385in}}%
\pgfpathlineto{\pgfqpoint{4.833170in}{1.534523in}}%
\pgfpathlineto{\pgfqpoint{4.834033in}{1.589375in}}%
\pgfpathlineto{\pgfqpoint{4.834897in}{1.574000in}}%
\pgfpathlineto{\pgfqpoint{4.835762in}{1.569014in}}%
\pgfpathlineto{\pgfqpoint{4.836629in}{1.612882in}}%
\pgfpathlineto{\pgfqpoint{4.838359in}{1.565273in}}%
\pgfpathlineto{\pgfqpoint{4.839223in}{1.508937in}}%
\pgfpathlineto{\pgfqpoint{4.840953in}{1.639004in}}%
\pgfpathlineto{\pgfqpoint{4.842682in}{1.551917in}}%
\pgfpathlineto{\pgfqpoint{4.842682in}{1.551917in}}%
\pgfusepath{stroke}%
\end{pgfscope}%
\begin{pgfscope}%
\pgfsetrectcap%
\pgfsetmiterjoin%
\pgfsetlinewidth{0.803000pt}%
\definecolor{currentstroke}{rgb}{0.000000,0.000000,0.000000}%
\pgfsetstrokecolor{currentstroke}%
\pgfsetdash{}{0pt}%
\pgfpathmoveto{\pgfqpoint{0.483776in}{1.444834in}}%
\pgfpathlineto{\pgfqpoint{0.483776in}{2.029715in}}%
\pgfusepath{stroke}%
\end{pgfscope}%
\begin{pgfscope}%
\pgfsetrectcap%
\pgfsetmiterjoin%
\pgfsetlinewidth{0.803000pt}%
\definecolor{currentstroke}{rgb}{0.000000,0.000000,0.000000}%
\pgfsetstrokecolor{currentstroke}%
\pgfsetdash{}{0pt}%
\pgfpathmoveto{\pgfqpoint{5.050249in}{1.444834in}}%
\pgfpathlineto{\pgfqpoint{5.050249in}{2.029715in}}%
\pgfusepath{stroke}%
\end{pgfscope}%
\begin{pgfscope}%
\pgfsetrectcap%
\pgfsetmiterjoin%
\pgfsetlinewidth{0.803000pt}%
\definecolor{currentstroke}{rgb}{0.000000,0.000000,0.000000}%
\pgfsetstrokecolor{currentstroke}%
\pgfsetdash{}{0pt}%
\pgfpathmoveto{\pgfqpoint{0.483776in}{1.444834in}}%
\pgfpathlineto{\pgfqpoint{5.050249in}{1.444834in}}%
\pgfusepath{stroke}%
\end{pgfscope}%
\begin{pgfscope}%
\pgfsetrectcap%
\pgfsetmiterjoin%
\pgfsetlinewidth{0.803000pt}%
\definecolor{currentstroke}{rgb}{0.000000,0.000000,0.000000}%
\pgfsetstrokecolor{currentstroke}%
\pgfsetdash{}{0pt}%
\pgfpathmoveto{\pgfqpoint{0.483776in}{2.029715in}}%
\pgfpathlineto{\pgfqpoint{5.050249in}{2.029715in}}%
\pgfusepath{stroke}%
\end{pgfscope}%
\begin{pgfscope}%
\pgfsetbuttcap%
\pgfsetmiterjoin%
\definecolor{currentfill}{rgb}{1.000000,1.000000,1.000000}%
\pgfsetfillcolor{currentfill}%
\pgfsetlinewidth{0.000000pt}%
\definecolor{currentstroke}{rgb}{0.000000,0.000000,0.000000}%
\pgfsetstrokecolor{currentstroke}%
\pgfsetstrokeopacity{0.000000}%
\pgfsetdash{}{0pt}%
\pgfpathmoveto{\pgfqpoint{0.483776in}{0.538014in}}%
\pgfpathlineto{\pgfqpoint{5.050249in}{0.538014in}}%
\pgfpathlineto{\pgfqpoint{5.050249in}{1.122895in}}%
\pgfpathlineto{\pgfqpoint{0.483776in}{1.122895in}}%
\pgfpathlineto{\pgfqpoint{0.483776in}{0.538014in}}%
\pgfpathclose%
\pgfusepath{fill}%
\end{pgfscope}%
\begin{pgfscope}%
\pgfsetbuttcap%
\pgfsetroundjoin%
\definecolor{currentfill}{rgb}{0.000000,0.000000,0.000000}%
\pgfsetfillcolor{currentfill}%
\pgfsetlinewidth{0.803000pt}%
\definecolor{currentstroke}{rgb}{0.000000,0.000000,0.000000}%
\pgfsetstrokecolor{currentstroke}%
\pgfsetdash{}{0pt}%
\pgfsys@defobject{currentmarker}{\pgfqpoint{0.000000in}{-0.048611in}}{\pgfqpoint{0.000000in}{0.000000in}}{%
\pgfpathmoveto{\pgfqpoint{0.000000in}{0.000000in}}%
\pgfpathlineto{\pgfqpoint{0.000000in}{-0.048611in}}%
\pgfusepath{stroke,fill}%
}%
\begin{pgfscope}%
\pgfsys@transformshift{0.691021in}{0.538014in}%
\pgfsys@useobject{currentmarker}{}%
\end{pgfscope}%
\end{pgfscope}%
\begin{pgfscope}%
\definecolor{textcolor}{rgb}{0.000000,0.000000,0.000000}%
\pgfsetstrokecolor{textcolor}%
\pgfsetfillcolor{textcolor}%
\pgftext[x=0.691021in,y=0.440792in,,top]{\color{textcolor}\rmfamily\fontsize{8.000000}{9.600000}\selectfont \(\displaystyle {06{:}00}\)}%
\end{pgfscope}%
\begin{pgfscope}%
\pgfsetbuttcap%
\pgfsetroundjoin%
\definecolor{currentfill}{rgb}{0.000000,0.000000,0.000000}%
\pgfsetfillcolor{currentfill}%
\pgfsetlinewidth{0.803000pt}%
\definecolor{currentstroke}{rgb}{0.000000,0.000000,0.000000}%
\pgfsetstrokecolor{currentstroke}%
\pgfsetdash{}{0pt}%
\pgfsys@defobject{currentmarker}{\pgfqpoint{0.000000in}{-0.048611in}}{\pgfqpoint{0.000000in}{0.000000in}}{%
\pgfpathmoveto{\pgfqpoint{0.000000in}{0.000000in}}%
\pgfpathlineto{\pgfqpoint{0.000000in}{-0.048611in}}%
\pgfusepath{stroke,fill}%
}%
\begin{pgfscope}%
\pgfsys@transformshift{1.210067in}{0.538014in}%
\pgfsys@useobject{currentmarker}{}%
\end{pgfscope}%
\end{pgfscope}%
\begin{pgfscope}%
\definecolor{textcolor}{rgb}{0.000000,0.000000,0.000000}%
\pgfsetstrokecolor{textcolor}%
\pgfsetfillcolor{textcolor}%
\pgftext[x=1.210067in,y=0.440792in,,top]{\color{textcolor}\rmfamily\fontsize{8.000000}{9.600000}\selectfont \(\displaystyle {09{:}00}\)}%
\end{pgfscope}%
\begin{pgfscope}%
\pgfsetbuttcap%
\pgfsetroundjoin%
\definecolor{currentfill}{rgb}{0.000000,0.000000,0.000000}%
\pgfsetfillcolor{currentfill}%
\pgfsetlinewidth{0.803000pt}%
\definecolor{currentstroke}{rgb}{0.000000,0.000000,0.000000}%
\pgfsetstrokecolor{currentstroke}%
\pgfsetdash{}{0pt}%
\pgfsys@defobject{currentmarker}{\pgfqpoint{0.000000in}{-0.048611in}}{\pgfqpoint{0.000000in}{0.000000in}}{%
\pgfpathmoveto{\pgfqpoint{0.000000in}{0.000000in}}%
\pgfpathlineto{\pgfqpoint{0.000000in}{-0.048611in}}%
\pgfusepath{stroke,fill}%
}%
\begin{pgfscope}%
\pgfsys@transformshift{1.729114in}{0.538014in}%
\pgfsys@useobject{currentmarker}{}%
\end{pgfscope}%
\end{pgfscope}%
\begin{pgfscope}%
\definecolor{textcolor}{rgb}{0.000000,0.000000,0.000000}%
\pgfsetstrokecolor{textcolor}%
\pgfsetfillcolor{textcolor}%
\pgftext[x=1.729114in,y=0.440792in,,top]{\color{textcolor}\rmfamily\fontsize{8.000000}{9.600000}\selectfont \(\displaystyle {12{:}00}\)}%
\end{pgfscope}%
\begin{pgfscope}%
\pgfsetbuttcap%
\pgfsetroundjoin%
\definecolor{currentfill}{rgb}{0.000000,0.000000,0.000000}%
\pgfsetfillcolor{currentfill}%
\pgfsetlinewidth{0.803000pt}%
\definecolor{currentstroke}{rgb}{0.000000,0.000000,0.000000}%
\pgfsetstrokecolor{currentstroke}%
\pgfsetdash{}{0pt}%
\pgfsys@defobject{currentmarker}{\pgfqpoint{0.000000in}{-0.048611in}}{\pgfqpoint{0.000000in}{0.000000in}}{%
\pgfpathmoveto{\pgfqpoint{0.000000in}{0.000000in}}%
\pgfpathlineto{\pgfqpoint{0.000000in}{-0.048611in}}%
\pgfusepath{stroke,fill}%
}%
\begin{pgfscope}%
\pgfsys@transformshift{2.248160in}{0.538014in}%
\pgfsys@useobject{currentmarker}{}%
\end{pgfscope}%
\end{pgfscope}%
\begin{pgfscope}%
\definecolor{textcolor}{rgb}{0.000000,0.000000,0.000000}%
\pgfsetstrokecolor{textcolor}%
\pgfsetfillcolor{textcolor}%
\pgftext[x=2.248160in,y=0.440792in,,top]{\color{textcolor}\rmfamily\fontsize{8.000000}{9.600000}\selectfont \(\displaystyle {15{:}00}\)}%
\end{pgfscope}%
\begin{pgfscope}%
\pgfsetbuttcap%
\pgfsetroundjoin%
\definecolor{currentfill}{rgb}{0.000000,0.000000,0.000000}%
\pgfsetfillcolor{currentfill}%
\pgfsetlinewidth{0.803000pt}%
\definecolor{currentstroke}{rgb}{0.000000,0.000000,0.000000}%
\pgfsetstrokecolor{currentstroke}%
\pgfsetdash{}{0pt}%
\pgfsys@defobject{currentmarker}{\pgfqpoint{0.000000in}{-0.048611in}}{\pgfqpoint{0.000000in}{0.000000in}}{%
\pgfpathmoveto{\pgfqpoint{0.000000in}{0.000000in}}%
\pgfpathlineto{\pgfqpoint{0.000000in}{-0.048611in}}%
\pgfusepath{stroke,fill}%
}%
\begin{pgfscope}%
\pgfsys@transformshift{2.767206in}{0.538014in}%
\pgfsys@useobject{currentmarker}{}%
\end{pgfscope}%
\end{pgfscope}%
\begin{pgfscope}%
\definecolor{textcolor}{rgb}{0.000000,0.000000,0.000000}%
\pgfsetstrokecolor{textcolor}%
\pgfsetfillcolor{textcolor}%
\pgftext[x=2.767206in,y=0.440792in,,top]{\color{textcolor}\rmfamily\fontsize{8.000000}{9.600000}\selectfont \(\displaystyle {18{:}00}\)}%
\end{pgfscope}%
\begin{pgfscope}%
\pgfsetbuttcap%
\pgfsetroundjoin%
\definecolor{currentfill}{rgb}{0.000000,0.000000,0.000000}%
\pgfsetfillcolor{currentfill}%
\pgfsetlinewidth{0.803000pt}%
\definecolor{currentstroke}{rgb}{0.000000,0.000000,0.000000}%
\pgfsetstrokecolor{currentstroke}%
\pgfsetdash{}{0pt}%
\pgfsys@defobject{currentmarker}{\pgfqpoint{0.000000in}{-0.048611in}}{\pgfqpoint{0.000000in}{0.000000in}}{%
\pgfpathmoveto{\pgfqpoint{0.000000in}{0.000000in}}%
\pgfpathlineto{\pgfqpoint{0.000000in}{-0.048611in}}%
\pgfusepath{stroke,fill}%
}%
\begin{pgfscope}%
\pgfsys@transformshift{3.286252in}{0.538014in}%
\pgfsys@useobject{currentmarker}{}%
\end{pgfscope}%
\end{pgfscope}%
\begin{pgfscope}%
\definecolor{textcolor}{rgb}{0.000000,0.000000,0.000000}%
\pgfsetstrokecolor{textcolor}%
\pgfsetfillcolor{textcolor}%
\pgftext[x=3.286252in,y=0.440792in,,top]{\color{textcolor}\rmfamily\fontsize{8.000000}{9.600000}\selectfont \(\displaystyle {21{:}00}\)}%
\end{pgfscope}%
\begin{pgfscope}%
\pgfsetbuttcap%
\pgfsetroundjoin%
\definecolor{currentfill}{rgb}{0.000000,0.000000,0.000000}%
\pgfsetfillcolor{currentfill}%
\pgfsetlinewidth{0.803000pt}%
\definecolor{currentstroke}{rgb}{0.000000,0.000000,0.000000}%
\pgfsetstrokecolor{currentstroke}%
\pgfsetdash{}{0pt}%
\pgfsys@defobject{currentmarker}{\pgfqpoint{0.000000in}{-0.048611in}}{\pgfqpoint{0.000000in}{0.000000in}}{%
\pgfpathmoveto{\pgfqpoint{0.000000in}{0.000000in}}%
\pgfpathlineto{\pgfqpoint{0.000000in}{-0.048611in}}%
\pgfusepath{stroke,fill}%
}%
\begin{pgfscope}%
\pgfsys@transformshift{3.805298in}{0.538014in}%
\pgfsys@useobject{currentmarker}{}%
\end{pgfscope}%
\end{pgfscope}%
\begin{pgfscope}%
\definecolor{textcolor}{rgb}{0.000000,0.000000,0.000000}%
\pgfsetstrokecolor{textcolor}%
\pgfsetfillcolor{textcolor}%
\pgftext[x=3.805298in,y=0.440792in,,top]{\color{textcolor}\rmfamily\fontsize{8.000000}{9.600000}\selectfont \(\displaystyle {00{:}00}\)}%
\end{pgfscope}%
\begin{pgfscope}%
\pgfsetbuttcap%
\pgfsetroundjoin%
\definecolor{currentfill}{rgb}{0.000000,0.000000,0.000000}%
\pgfsetfillcolor{currentfill}%
\pgfsetlinewidth{0.803000pt}%
\definecolor{currentstroke}{rgb}{0.000000,0.000000,0.000000}%
\pgfsetstrokecolor{currentstroke}%
\pgfsetdash{}{0pt}%
\pgfsys@defobject{currentmarker}{\pgfqpoint{0.000000in}{-0.048611in}}{\pgfqpoint{0.000000in}{0.000000in}}{%
\pgfpathmoveto{\pgfqpoint{0.000000in}{0.000000in}}%
\pgfpathlineto{\pgfqpoint{0.000000in}{-0.048611in}}%
\pgfusepath{stroke,fill}%
}%
\begin{pgfscope}%
\pgfsys@transformshift{4.324344in}{0.538014in}%
\pgfsys@useobject{currentmarker}{}%
\end{pgfscope}%
\end{pgfscope}%
\begin{pgfscope}%
\definecolor{textcolor}{rgb}{0.000000,0.000000,0.000000}%
\pgfsetstrokecolor{textcolor}%
\pgfsetfillcolor{textcolor}%
\pgftext[x=4.324344in,y=0.440792in,,top]{\color{textcolor}\rmfamily\fontsize{8.000000}{9.600000}\selectfont \(\displaystyle {03{:}00}\)}%
\end{pgfscope}%
\begin{pgfscope}%
\pgfsetbuttcap%
\pgfsetroundjoin%
\definecolor{currentfill}{rgb}{0.000000,0.000000,0.000000}%
\pgfsetfillcolor{currentfill}%
\pgfsetlinewidth{0.803000pt}%
\definecolor{currentstroke}{rgb}{0.000000,0.000000,0.000000}%
\pgfsetstrokecolor{currentstroke}%
\pgfsetdash{}{0pt}%
\pgfsys@defobject{currentmarker}{\pgfqpoint{0.000000in}{-0.048611in}}{\pgfqpoint{0.000000in}{0.000000in}}{%
\pgfpathmoveto{\pgfqpoint{0.000000in}{0.000000in}}%
\pgfpathlineto{\pgfqpoint{0.000000in}{-0.048611in}}%
\pgfusepath{stroke,fill}%
}%
\begin{pgfscope}%
\pgfsys@transformshift{4.843390in}{0.538014in}%
\pgfsys@useobject{currentmarker}{}%
\end{pgfscope}%
\end{pgfscope}%
\begin{pgfscope}%
\definecolor{textcolor}{rgb}{0.000000,0.000000,0.000000}%
\pgfsetstrokecolor{textcolor}%
\pgfsetfillcolor{textcolor}%
\pgftext[x=4.843390in,y=0.440792in,,top]{\color{textcolor}\rmfamily\fontsize{8.000000}{9.600000}\selectfont \(\displaystyle {06{:}00}\)}%
\end{pgfscope}%
\begin{pgfscope}%
\definecolor{textcolor}{rgb}{0.000000,0.000000,0.000000}%
\pgfsetstrokecolor{textcolor}%
\pgfsetfillcolor{textcolor}%
\pgftext[x=2.767012in,y=0.286570in,,top]{\color{textcolor}\rmfamily\fontsize{10.000000}{12.000000}\selectfont Time (UTC)}%
\end{pgfscope}%
\begin{pgfscope}%
\pgfsetbuttcap%
\pgfsetroundjoin%
\definecolor{currentfill}{rgb}{0.000000,0.000000,0.000000}%
\pgfsetfillcolor{currentfill}%
\pgfsetlinewidth{0.803000pt}%
\definecolor{currentstroke}{rgb}{0.000000,0.000000,0.000000}%
\pgfsetstrokecolor{currentstroke}%
\pgfsetdash{}{0pt}%
\pgfsys@defobject{currentmarker}{\pgfqpoint{-0.048611in}{0.000000in}}{\pgfqpoint{-0.000000in}{0.000000in}}{%
\pgfpathmoveto{\pgfqpoint{-0.000000in}{0.000000in}}%
\pgfpathlineto{\pgfqpoint{-0.048611in}{0.000000in}}%
\pgfusepath{stroke,fill}%
}%
\begin{pgfscope}%
\pgfsys@transformshift{0.483776in}{0.719191in}%
\pgfsys@useobject{currentmarker}{}%
\end{pgfscope}%
\end{pgfscope}%
\begin{pgfscope}%
\definecolor{textcolor}{rgb}{0.000000,0.000000,0.000000}%
\pgfsetstrokecolor{textcolor}%
\pgfsetfillcolor{textcolor}%
\pgftext[x=0.327525in, y=0.680636in, left, base]{\color{textcolor}\rmfamily\fontsize{8.000000}{9.600000}\selectfont \(\displaystyle {0}\)}%
\end{pgfscope}%
\begin{pgfscope}%
\pgfsetbuttcap%
\pgfsetroundjoin%
\definecolor{currentfill}{rgb}{0.000000,0.000000,0.000000}%
\pgfsetfillcolor{currentfill}%
\pgfsetlinewidth{0.803000pt}%
\definecolor{currentstroke}{rgb}{0.000000,0.000000,0.000000}%
\pgfsetstrokecolor{currentstroke}%
\pgfsetdash{}{0pt}%
\pgfsys@defobject{currentmarker}{\pgfqpoint{-0.048611in}{0.000000in}}{\pgfqpoint{-0.000000in}{0.000000in}}{%
\pgfpathmoveto{\pgfqpoint{-0.000000in}{0.000000in}}%
\pgfpathlineto{\pgfqpoint{-0.048611in}{0.000000in}}%
\pgfusepath{stroke,fill}%
}%
\begin{pgfscope}%
\pgfsys@transformshift{0.483776in}{0.926366in}%
\pgfsys@useobject{currentmarker}{}%
\end{pgfscope}%
\end{pgfscope}%
\begin{pgfscope}%
\definecolor{textcolor}{rgb}{0.000000,0.000000,0.000000}%
\pgfsetstrokecolor{textcolor}%
\pgfsetfillcolor{textcolor}%
\pgftext[x=0.327525in, y=0.887811in, left, base]{\color{textcolor}\rmfamily\fontsize{8.000000}{9.600000}\selectfont \(\displaystyle {5}\)}%
\end{pgfscope}%
\begin{pgfscope}%
\definecolor{textcolor}{rgb}{0.000000,0.000000,0.000000}%
\pgfsetstrokecolor{textcolor}%
\pgfsetfillcolor{textcolor}%
\pgftext[x=0.483776in,y=1.164562in,left,base]{\color{textcolor}\rmfamily\fontsize{8.000000}{9.600000}\selectfont \(\displaystyle \times{10^{\ensuremath{-}6}}{}\)}%
\end{pgfscope}%
\begin{pgfscope}%
\pgfpathrectangle{\pgfqpoint{0.483776in}{0.538014in}}{\pgfqpoint{4.566474in}{0.584881in}}%
\pgfusepath{clip}%
\pgfsetrectcap%
\pgfsetroundjoin%
\pgfsetlinewidth{0.501875pt}%
\definecolor{currentstroke}{rgb}{0.000000,0.419608,0.643137}%
\pgfsetstrokecolor{currentstroke}%
\pgfsetstrokeopacity{0.700000}%
\pgfsetdash{}{0pt}%
\pgfpathmoveto{\pgfqpoint{0.691343in}{0.723309in}}%
\pgfpathlineto{\pgfqpoint{0.692205in}{0.730232in}}%
\pgfpathlineto{\pgfqpoint{0.693935in}{0.701294in}}%
\pgfpathlineto{\pgfqpoint{0.694800in}{0.716533in}}%
\pgfpathlineto{\pgfqpoint{0.695666in}{0.686239in}}%
\pgfpathlineto{\pgfqpoint{0.696532in}{0.687741in}}%
\pgfpathlineto{\pgfqpoint{0.698263in}{0.732136in}}%
\pgfpathlineto{\pgfqpoint{0.699128in}{0.727776in}}%
\pgfpathlineto{\pgfqpoint{0.699993in}{0.774404in}}%
\pgfpathlineto{\pgfqpoint{0.701725in}{0.645250in}}%
\pgfpathlineto{\pgfqpoint{0.703453in}{0.750814in}}%
\pgfpathlineto{\pgfqpoint{0.704319in}{0.749056in}}%
\pgfpathlineto{\pgfqpoint{0.705185in}{0.769605in}}%
\pgfpathlineto{\pgfqpoint{0.709512in}{0.701108in}}%
\pgfpathlineto{\pgfqpoint{0.711241in}{0.778473in}}%
\pgfpathlineto{\pgfqpoint{0.712105in}{0.705834in}}%
\pgfpathlineto{\pgfqpoint{0.712971in}{0.751585in}}%
\pgfpathlineto{\pgfqpoint{0.713837in}{0.688691in}}%
\pgfpathlineto{\pgfqpoint{0.714702in}{0.689609in}}%
\pgfpathlineto{\pgfqpoint{0.715567in}{0.766787in}}%
\pgfpathlineto{\pgfqpoint{0.717297in}{0.689609in}}%
\pgfpathlineto{\pgfqpoint{0.718163in}{0.747850in}}%
\pgfpathlineto{\pgfqpoint{0.719030in}{0.739681in}}%
\pgfpathlineto{\pgfqpoint{0.719895in}{0.652319in}}%
\pgfpathlineto{\pgfqpoint{0.720762in}{0.737703in}}%
\pgfpathlineto{\pgfqpoint{0.721627in}{0.672612in}}%
\pgfpathlineto{\pgfqpoint{0.723356in}{0.735287in}}%
\pgfpathlineto{\pgfqpoint{0.725084in}{0.736384in}}%
\pgfpathlineto{\pgfqpoint{0.725947in}{0.694920in}}%
\pgfpathlineto{\pgfqpoint{0.727676in}{0.751951in}}%
\pgfpathlineto{\pgfqpoint{0.728541in}{0.646309in}}%
\pgfpathlineto{\pgfqpoint{0.729407in}{0.755760in}}%
\pgfpathlineto{\pgfqpoint{0.730271in}{0.732867in}}%
\pgfpathlineto{\pgfqpoint{0.731135in}{0.762134in}}%
\pgfpathlineto{\pgfqpoint{0.732865in}{0.715030in}}%
\pgfpathlineto{\pgfqpoint{0.733731in}{0.722026in}}%
\pgfpathlineto{\pgfqpoint{0.735460in}{0.682540in}}%
\pgfpathlineto{\pgfqpoint{0.736325in}{0.782501in}}%
\pgfpathlineto{\pgfqpoint{0.737190in}{0.721036in}}%
\pgfpathlineto{\pgfqpoint{0.738920in}{0.778546in}}%
\pgfpathlineto{\pgfqpoint{0.740649in}{0.685873in}}%
\pgfpathlineto{\pgfqpoint{0.741516in}{0.685906in}}%
\pgfpathlineto{\pgfqpoint{0.742381in}{0.661365in}}%
\pgfpathlineto{\pgfqpoint{0.744112in}{0.728434in}}%
\pgfpathlineto{\pgfqpoint{0.744977in}{0.676640in}}%
\pgfpathlineto{\pgfqpoint{0.745840in}{0.733452in}}%
\pgfpathlineto{\pgfqpoint{0.746705in}{0.708985in}}%
\pgfpathlineto{\pgfqpoint{0.748435in}{0.780925in}}%
\pgfpathlineto{\pgfqpoint{0.750163in}{0.711949in}}%
\pgfpathlineto{\pgfqpoint{0.751028in}{0.710739in}}%
\pgfpathlineto{\pgfqpoint{0.751891in}{0.701656in}}%
\pgfpathlineto{\pgfqpoint{0.752757in}{0.767518in}}%
\pgfpathlineto{\pgfqpoint{0.753622in}{0.661804in}}%
\pgfpathlineto{\pgfqpoint{0.754488in}{0.702391in}}%
\pgfpathlineto{\pgfqpoint{0.755354in}{0.695577in}}%
\pgfpathlineto{\pgfqpoint{0.756219in}{0.680376in}}%
\pgfpathlineto{\pgfqpoint{0.757950in}{0.793269in}}%
\pgfpathlineto{\pgfqpoint{0.758816in}{0.693965in}}%
\pgfpathlineto{\pgfqpoint{0.759679in}{0.695870in}}%
\pgfpathlineto{\pgfqpoint{0.760542in}{0.741402in}}%
\pgfpathlineto{\pgfqpoint{0.761407in}{0.661584in}}%
\pgfpathlineto{\pgfqpoint{0.763135in}{0.730374in}}%
\pgfpathlineto{\pgfqpoint{0.764000in}{0.670744in}}%
\pgfpathlineto{\pgfqpoint{0.764865in}{0.676494in}}%
\pgfpathlineto{\pgfqpoint{0.765730in}{0.703454in}}%
\pgfpathlineto{\pgfqpoint{0.766596in}{0.664150in}}%
\pgfpathlineto{\pgfqpoint{0.768326in}{0.763892in}}%
\pgfpathlineto{\pgfqpoint{0.770057in}{0.681366in}}%
\pgfpathlineto{\pgfqpoint{0.770922in}{0.749900in}}%
\pgfpathlineto{\pgfqpoint{0.771788in}{0.703674in}}%
\pgfpathlineto{\pgfqpoint{0.772652in}{0.790597in}}%
\pgfpathlineto{\pgfqpoint{0.774383in}{0.694554in}}%
\pgfpathlineto{\pgfqpoint{0.776115in}{0.752576in}}%
\pgfpathlineto{\pgfqpoint{0.776979in}{0.711185in}}%
\pgfpathlineto{\pgfqpoint{0.777845in}{0.739279in}}%
\pgfpathlineto{\pgfqpoint{0.778710in}{0.680708in}}%
\pgfpathlineto{\pgfqpoint{0.779575in}{0.777190in}}%
\pgfpathlineto{\pgfqpoint{0.780439in}{0.774884in}}%
\pgfpathlineto{\pgfqpoint{0.781305in}{0.700856in}}%
\pgfpathlineto{\pgfqpoint{0.782170in}{0.760892in}}%
\pgfpathlineto{\pgfqpoint{0.783036in}{0.713601in}}%
\pgfpathlineto{\pgfqpoint{0.783901in}{0.722099in}}%
\pgfpathlineto{\pgfqpoint{0.785630in}{0.682576in}}%
\pgfpathlineto{\pgfqpoint{0.786494in}{0.747484in}}%
\pgfpathlineto{\pgfqpoint{0.787360in}{0.605032in}}%
\pgfpathlineto{\pgfqpoint{0.788225in}{0.615727in}}%
\pgfpathlineto{\pgfqpoint{0.789089in}{0.747338in}}%
\pgfpathlineto{\pgfqpoint{0.789954in}{0.655104in}}%
\pgfpathlineto{\pgfqpoint{0.791680in}{0.787081in}}%
\pgfpathlineto{\pgfqpoint{0.793412in}{0.719022in}}%
\pgfpathlineto{\pgfqpoint{0.794274in}{0.721255in}}%
\pgfpathlineto{\pgfqpoint{0.795138in}{0.700961in}}%
\pgfpathlineto{\pgfqpoint{0.796004in}{0.820449in}}%
\pgfpathlineto{\pgfqpoint{0.796868in}{0.703637in}}%
\pgfpathlineto{\pgfqpoint{0.797734in}{0.729497in}}%
\pgfpathlineto{\pgfqpoint{0.798599in}{0.747152in}}%
\pgfpathlineto{\pgfqpoint{0.799464in}{0.743343in}}%
\pgfpathlineto{\pgfqpoint{0.800327in}{0.693856in}}%
\pgfpathlineto{\pgfqpoint{0.801192in}{0.755321in}}%
\pgfpathlineto{\pgfqpoint{0.802924in}{0.712022in}}%
\pgfpathlineto{\pgfqpoint{0.803787in}{0.714880in}}%
\pgfpathlineto{\pgfqpoint{0.804652in}{0.653196in}}%
\pgfpathlineto{\pgfqpoint{0.805515in}{0.665028in}}%
\pgfpathlineto{\pgfqpoint{0.808107in}{0.727483in}}%
\pgfpathlineto{\pgfqpoint{0.809837in}{0.608435in}}%
\pgfpathlineto{\pgfqpoint{0.810702in}{0.681439in}}%
\pgfpathlineto{\pgfqpoint{0.811567in}{0.658803in}}%
\pgfpathlineto{\pgfqpoint{0.812431in}{0.718510in}}%
\pgfpathlineto{\pgfqpoint{0.813297in}{0.682576in}}%
\pgfpathlineto{\pgfqpoint{0.814161in}{0.740671in}}%
\pgfpathlineto{\pgfqpoint{0.815026in}{0.689682in}}%
\pgfpathlineto{\pgfqpoint{0.817620in}{0.783272in}}%
\pgfpathlineto{\pgfqpoint{0.819349in}{0.683786in}}%
\pgfpathlineto{\pgfqpoint{0.820213in}{0.773381in}}%
\pgfpathlineto{\pgfqpoint{0.821078in}{0.738913in}}%
\pgfpathlineto{\pgfqpoint{0.821942in}{0.767559in}}%
\pgfpathlineto{\pgfqpoint{0.822807in}{0.760599in}}%
\pgfpathlineto{\pgfqpoint{0.823671in}{0.723821in}}%
\pgfpathlineto{\pgfqpoint{0.824536in}{0.777263in}}%
\pgfpathlineto{\pgfqpoint{0.826268in}{0.719095in}}%
\pgfpathlineto{\pgfqpoint{0.827133in}{0.746859in}}%
\pgfpathlineto{\pgfqpoint{0.828864in}{0.724990in}}%
\pgfpathlineto{\pgfqpoint{0.830596in}{0.855140in}}%
\pgfpathlineto{\pgfqpoint{0.831462in}{0.760266in}}%
\pgfpathlineto{\pgfqpoint{0.832328in}{0.803014in}}%
\pgfpathlineto{\pgfqpoint{0.833193in}{0.746494in}}%
\pgfpathlineto{\pgfqpoint{0.834058in}{0.759316in}}%
\pgfpathlineto{\pgfqpoint{0.834922in}{0.777336in}}%
\pgfpathlineto{\pgfqpoint{0.836650in}{0.714076in}}%
\pgfpathlineto{\pgfqpoint{0.838382in}{0.684590in}}%
\pgfpathlineto{\pgfqpoint{0.840114in}{0.723529in}}%
\pgfpathlineto{\pgfqpoint{0.840980in}{0.720378in}}%
\pgfpathlineto{\pgfqpoint{0.841845in}{0.730927in}}%
\pgfpathlineto{\pgfqpoint{0.843575in}{0.677265in}}%
\pgfpathlineto{\pgfqpoint{0.845305in}{0.697705in}}%
\pgfpathlineto{\pgfqpoint{0.846169in}{0.675836in}}%
\pgfpathlineto{\pgfqpoint{0.847897in}{0.724406in}}%
\pgfpathlineto{\pgfqpoint{0.848762in}{0.724406in}}%
\pgfpathlineto{\pgfqpoint{0.849627in}{0.695139in}}%
\pgfpathlineto{\pgfqpoint{0.851354in}{0.746421in}}%
\pgfpathlineto{\pgfqpoint{0.852218in}{0.710194in}}%
\pgfpathlineto{\pgfqpoint{0.853081in}{0.734077in}}%
\pgfpathlineto{\pgfqpoint{0.853945in}{0.705359in}}%
\pgfpathlineto{\pgfqpoint{0.854810in}{0.759316in}}%
\pgfpathlineto{\pgfqpoint{0.855672in}{0.754736in}}%
\pgfpathlineto{\pgfqpoint{0.856537in}{0.740854in}}%
\pgfpathlineto{\pgfqpoint{0.857402in}{0.761549in}}%
\pgfpathlineto{\pgfqpoint{0.858267in}{0.749279in}}%
\pgfpathlineto{\pgfqpoint{0.859131in}{0.670379in}}%
\pgfpathlineto{\pgfqpoint{0.861730in}{0.738292in}}%
\pgfpathlineto{\pgfqpoint{0.862597in}{0.751147in}}%
\pgfpathlineto{\pgfqpoint{0.864329in}{0.720012in}}%
\pgfpathlineto{\pgfqpoint{0.865195in}{0.678694in}}%
\pgfpathlineto{\pgfqpoint{0.866062in}{0.750745in}}%
\pgfpathlineto{\pgfqpoint{0.866929in}{0.734849in}}%
\pgfpathlineto{\pgfqpoint{0.867794in}{0.774737in}}%
\pgfpathlineto{\pgfqpoint{0.868659in}{0.702175in}}%
\pgfpathlineto{\pgfqpoint{0.870388in}{0.747850in}}%
\pgfpathlineto{\pgfqpoint{0.871253in}{0.663200in}}%
\pgfpathlineto{\pgfqpoint{0.872119in}{0.745178in}}%
\pgfpathlineto{\pgfqpoint{0.872983in}{0.674740in}}%
\pgfpathlineto{\pgfqpoint{0.873848in}{0.766901in}}%
\pgfpathlineto{\pgfqpoint{0.874712in}{0.721368in}}%
\pgfpathlineto{\pgfqpoint{0.875577in}{0.750270in}}%
\pgfpathlineto{\pgfqpoint{0.877309in}{0.707084in}}%
\pgfpathlineto{\pgfqpoint{0.878174in}{0.699024in}}%
\pgfpathlineto{\pgfqpoint{0.879035in}{0.658255in}}%
\pgfpathlineto{\pgfqpoint{0.879903in}{0.667853in}}%
\pgfpathlineto{\pgfqpoint{0.881633in}{0.731990in}}%
\pgfpathlineto{\pgfqpoint{0.882498in}{0.705911in}}%
\pgfpathlineto{\pgfqpoint{0.883364in}{0.746680in}}%
\pgfpathlineto{\pgfqpoint{0.884229in}{0.721661in}}%
\pgfpathlineto{\pgfqpoint{0.885096in}{0.727154in}}%
\pgfpathlineto{\pgfqpoint{0.885960in}{0.755508in}}%
\pgfpathlineto{\pgfqpoint{0.886826in}{0.653675in}}%
\pgfpathlineto{\pgfqpoint{0.888555in}{0.733566in}}%
\pgfpathlineto{\pgfqpoint{0.891148in}{0.786350in}}%
\pgfpathlineto{\pgfqpoint{0.892875in}{0.743530in}}%
\pgfpathlineto{\pgfqpoint{0.893741in}{0.754850in}}%
\pgfpathlineto{\pgfqpoint{0.894605in}{0.728588in}}%
\pgfpathlineto{\pgfqpoint{0.896337in}{0.756019in}}%
\pgfpathlineto{\pgfqpoint{0.898067in}{0.704372in}}%
\pgfpathlineto{\pgfqpoint{0.901525in}{0.744041in}}%
\pgfpathlineto{\pgfqpoint{0.903254in}{0.711697in}}%
\pgfpathlineto{\pgfqpoint{0.904119in}{0.713528in}}%
\pgfpathlineto{\pgfqpoint{0.904984in}{0.687010in}}%
\pgfpathlineto{\pgfqpoint{0.906714in}{0.753238in}}%
\pgfpathlineto{\pgfqpoint{0.907579in}{0.773308in}}%
\pgfpathlineto{\pgfqpoint{0.909309in}{0.708107in}}%
\pgfpathlineto{\pgfqpoint{0.910175in}{0.764700in}}%
\pgfpathlineto{\pgfqpoint{0.911039in}{0.744114in}}%
\pgfpathlineto{\pgfqpoint{0.911905in}{0.653382in}}%
\pgfpathlineto{\pgfqpoint{0.912770in}{0.718364in}}%
\pgfpathlineto{\pgfqpoint{0.913635in}{0.653127in}}%
\pgfpathlineto{\pgfqpoint{0.914501in}{0.730049in}}%
\pgfpathlineto{\pgfqpoint{0.915365in}{0.654446in}}%
\pgfpathlineto{\pgfqpoint{0.917094in}{0.728620in}}%
\pgfpathlineto{\pgfqpoint{0.917959in}{0.706386in}}%
\pgfpathlineto{\pgfqpoint{0.918824in}{0.650232in}}%
\pgfpathlineto{\pgfqpoint{0.919687in}{0.701806in}}%
\pgfpathlineto{\pgfqpoint{0.920552in}{0.664406in}}%
\pgfpathlineto{\pgfqpoint{0.922280in}{0.757229in}}%
\pgfpathlineto{\pgfqpoint{0.924010in}{0.651295in}}%
\pgfpathlineto{\pgfqpoint{0.924875in}{0.746201in}}%
\pgfpathlineto{\pgfqpoint{0.925740in}{0.730415in}}%
\pgfpathlineto{\pgfqpoint{0.926603in}{0.781916in}}%
\pgfpathlineto{\pgfqpoint{0.927464in}{0.683161in}}%
\pgfpathlineto{\pgfqpoint{0.928329in}{0.696860in}}%
\pgfpathlineto{\pgfqpoint{0.929193in}{0.704591in}}%
\pgfpathlineto{\pgfqpoint{0.930058in}{0.660086in}}%
\pgfpathlineto{\pgfqpoint{0.931787in}{0.724665in}}%
\pgfpathlineto{\pgfqpoint{0.933516in}{0.679612in}}%
\pgfpathlineto{\pgfqpoint{0.934381in}{0.716423in}}%
\pgfpathlineto{\pgfqpoint{0.935246in}{0.702066in}}%
\pgfpathlineto{\pgfqpoint{0.936974in}{0.782907in}}%
\pgfpathlineto{\pgfqpoint{0.939570in}{0.666607in}}%
\pgfpathlineto{\pgfqpoint{0.940435in}{0.745657in}}%
\pgfpathlineto{\pgfqpoint{0.942166in}{0.707194in}}%
\pgfpathlineto{\pgfqpoint{0.943032in}{0.768988in}}%
\pgfpathlineto{\pgfqpoint{0.943897in}{0.750416in}}%
\pgfpathlineto{\pgfqpoint{0.944763in}{0.775911in}}%
\pgfpathlineto{\pgfqpoint{0.945628in}{0.719428in}}%
\pgfpathlineto{\pgfqpoint{0.946490in}{0.730488in}}%
\pgfpathlineto{\pgfqpoint{0.947355in}{0.733712in}}%
\pgfpathlineto{\pgfqpoint{0.948218in}{0.769865in}}%
\pgfpathlineto{\pgfqpoint{0.949946in}{0.658547in}}%
\pgfpathlineto{\pgfqpoint{0.950811in}{0.714117in}}%
\pgfpathlineto{\pgfqpoint{0.951674in}{0.637194in}}%
\pgfpathlineto{\pgfqpoint{0.952539in}{0.719135in}}%
\pgfpathlineto{\pgfqpoint{0.953405in}{0.686425in}}%
\pgfpathlineto{\pgfqpoint{0.954269in}{0.714738in}}%
\pgfpathlineto{\pgfqpoint{0.955133in}{0.672100in}}%
\pgfpathlineto{\pgfqpoint{0.955998in}{0.761403in}}%
\pgfpathlineto{\pgfqpoint{0.956863in}{0.704372in}}%
\pgfpathlineto{\pgfqpoint{0.957728in}{0.761111in}}%
\pgfpathlineto{\pgfqpoint{0.958594in}{0.675507in}}%
\pgfpathlineto{\pgfqpoint{0.959459in}{0.765212in}}%
\pgfpathlineto{\pgfqpoint{0.960325in}{0.701587in}}%
\pgfpathlineto{\pgfqpoint{0.961191in}{0.744187in}}%
\pgfpathlineto{\pgfqpoint{0.962055in}{0.679572in}}%
\pgfpathlineto{\pgfqpoint{0.962920in}{0.699938in}}%
\pgfpathlineto{\pgfqpoint{0.963783in}{0.766568in}}%
\pgfpathlineto{\pgfqpoint{0.964647in}{0.713345in}}%
\pgfpathlineto{\pgfqpoint{0.965512in}{0.720012in}}%
\pgfpathlineto{\pgfqpoint{0.968105in}{0.654186in}}%
\pgfpathlineto{\pgfqpoint{0.970701in}{0.725798in}}%
\pgfpathlineto{\pgfqpoint{0.972432in}{0.677558in}}%
\pgfpathlineto{\pgfqpoint{0.973297in}{0.720268in}}%
\pgfpathlineto{\pgfqpoint{0.974162in}{0.684517in}}%
\pgfpathlineto{\pgfqpoint{0.975027in}{0.759316in}}%
\pgfpathlineto{\pgfqpoint{0.975889in}{0.702943in}}%
\pgfpathlineto{\pgfqpoint{0.976751in}{0.739388in}}%
\pgfpathlineto{\pgfqpoint{0.978481in}{0.704884in}}%
\pgfpathlineto{\pgfqpoint{0.979346in}{0.715067in}}%
\pgfpathlineto{\pgfqpoint{0.981076in}{0.809242in}}%
\pgfpathlineto{\pgfqpoint{0.981941in}{0.768143in}}%
\pgfpathlineto{\pgfqpoint{0.984535in}{0.715213in}}%
\pgfpathlineto{\pgfqpoint{0.985400in}{0.741037in}}%
\pgfpathlineto{\pgfqpoint{0.986264in}{0.740891in}}%
\pgfpathlineto{\pgfqpoint{0.987991in}{0.701367in}}%
\pgfpathlineto{\pgfqpoint{0.988856in}{0.761111in}}%
\pgfpathlineto{\pgfqpoint{0.989722in}{0.733821in}}%
\pgfpathlineto{\pgfqpoint{0.990585in}{0.663127in}}%
\pgfpathlineto{\pgfqpoint{0.991449in}{0.756019in}}%
\pgfpathlineto{\pgfqpoint{0.992314in}{0.699280in}}%
\pgfpathlineto{\pgfqpoint{0.993178in}{0.737740in}}%
\pgfpathlineto{\pgfqpoint{0.994043in}{0.669023in}}%
\pgfpathlineto{\pgfqpoint{0.996636in}{0.735251in}}%
\pgfpathlineto{\pgfqpoint{0.997501in}{0.713053in}}%
\pgfpathlineto{\pgfqpoint{1.000961in}{0.750562in}}%
\pgfpathlineto{\pgfqpoint{1.001826in}{0.732210in}}%
\pgfpathlineto{\pgfqpoint{1.002691in}{0.752503in}}%
\pgfpathlineto{\pgfqpoint{1.003553in}{0.714628in}}%
\pgfpathlineto{\pgfqpoint{1.004419in}{0.783491in}}%
\pgfpathlineto{\pgfqpoint{1.005284in}{0.710121in}}%
\pgfpathlineto{\pgfqpoint{1.006149in}{0.733529in}}%
\pgfpathlineto{\pgfqpoint{1.007014in}{0.695399in}}%
\pgfpathlineto{\pgfqpoint{1.007877in}{0.762686in}}%
\pgfpathlineto{\pgfqpoint{1.008742in}{0.689576in}}%
\pgfpathlineto{\pgfqpoint{1.009607in}{0.748548in}}%
\pgfpathlineto{\pgfqpoint{1.011338in}{0.687083in}}%
\pgfpathlineto{\pgfqpoint{1.012203in}{0.694042in}}%
\pgfpathlineto{\pgfqpoint{1.013067in}{0.735580in}}%
\pgfpathlineto{\pgfqpoint{1.013933in}{0.722099in}}%
\pgfpathlineto{\pgfqpoint{1.014798in}{0.685029in}}%
\pgfpathlineto{\pgfqpoint{1.015662in}{0.746421in}}%
\pgfpathlineto{\pgfqpoint{1.016526in}{0.731146in}}%
\pgfpathlineto{\pgfqpoint{1.017391in}{0.740891in}}%
\pgfpathlineto{\pgfqpoint{1.020851in}{0.668182in}}%
\pgfpathlineto{\pgfqpoint{1.022581in}{0.687595in}}%
\pgfpathlineto{\pgfqpoint{1.023445in}{0.749133in}}%
\pgfpathlineto{\pgfqpoint{1.026037in}{0.674407in}}%
\pgfpathlineto{\pgfqpoint{1.027767in}{0.768070in}}%
\pgfpathlineto{\pgfqpoint{1.029494in}{0.734516in}}%
\pgfpathlineto{\pgfqpoint{1.030357in}{0.725798in}}%
\pgfpathlineto{\pgfqpoint{1.031221in}{0.743928in}}%
\pgfpathlineto{\pgfqpoint{1.034679in}{0.669059in}}%
\pgfpathlineto{\pgfqpoint{1.035545in}{0.706897in}}%
\pgfpathlineto{\pgfqpoint{1.036409in}{0.705395in}}%
\pgfpathlineto{\pgfqpoint{1.037275in}{0.706934in}}%
\pgfpathlineto{\pgfqpoint{1.038139in}{0.687302in}}%
\pgfpathlineto{\pgfqpoint{1.040734in}{0.746973in}}%
\pgfpathlineto{\pgfqpoint{1.041600in}{0.711916in}}%
\pgfpathlineto{\pgfqpoint{1.042466in}{0.772025in}}%
\pgfpathlineto{\pgfqpoint{1.043331in}{0.735507in}}%
\pgfpathlineto{\pgfqpoint{1.044195in}{0.803200in}}%
\pgfpathlineto{\pgfqpoint{1.045927in}{0.680270in}}%
\pgfpathlineto{\pgfqpoint{1.046792in}{0.684188in}}%
\pgfpathlineto{\pgfqpoint{1.047657in}{0.731771in}}%
\pgfpathlineto{\pgfqpoint{1.049390in}{0.692759in}}%
\pgfpathlineto{\pgfqpoint{1.050256in}{0.744187in}}%
\pgfpathlineto{\pgfqpoint{1.051987in}{0.642980in}}%
\pgfpathlineto{\pgfqpoint{1.052852in}{0.681260in}}%
\pgfpathlineto{\pgfqpoint{1.053719in}{0.660565in}}%
\pgfpathlineto{\pgfqpoint{1.056313in}{0.771294in}}%
\pgfpathlineto{\pgfqpoint{1.057179in}{0.766422in}}%
\pgfpathlineto{\pgfqpoint{1.058042in}{0.690161in}}%
\pgfpathlineto{\pgfqpoint{1.058907in}{0.750562in}}%
\pgfpathlineto{\pgfqpoint{1.059773in}{0.680781in}}%
\pgfpathlineto{\pgfqpoint{1.060637in}{0.712614in}}%
\pgfpathlineto{\pgfqpoint{1.062369in}{0.669721in}}%
\pgfpathlineto{\pgfqpoint{1.063234in}{0.676936in}}%
\pgfpathlineto{\pgfqpoint{1.064962in}{0.771587in}}%
\pgfpathlineto{\pgfqpoint{1.065827in}{0.659830in}}%
\pgfpathlineto{\pgfqpoint{1.066692in}{0.682138in}}%
\pgfpathlineto{\pgfqpoint{1.067557in}{0.745105in}}%
\pgfpathlineto{\pgfqpoint{1.068422in}{0.636240in}}%
\pgfpathlineto{\pgfqpoint{1.070151in}{0.752763in}}%
\pgfpathlineto{\pgfqpoint{1.073608in}{0.646975in}}%
\pgfpathlineto{\pgfqpoint{1.074470in}{0.784994in}}%
\pgfpathlineto{\pgfqpoint{1.075335in}{0.659867in}}%
\pgfpathlineto{\pgfqpoint{1.076199in}{0.704664in}}%
\pgfpathlineto{\pgfqpoint{1.077064in}{0.680197in}}%
\pgfpathlineto{\pgfqpoint{1.079657in}{0.732136in}}%
\pgfpathlineto{\pgfqpoint{1.081384in}{0.673018in}}%
\pgfpathlineto{\pgfqpoint{1.083981in}{0.753859in}}%
\pgfpathlineto{\pgfqpoint{1.085711in}{0.664045in}}%
\pgfpathlineto{\pgfqpoint{1.087440in}{0.773970in}}%
\pgfpathlineto{\pgfqpoint{1.088306in}{0.790857in}}%
\pgfpathlineto{\pgfqpoint{1.089172in}{0.718770in}}%
\pgfpathlineto{\pgfqpoint{1.090037in}{0.751553in}}%
\pgfpathlineto{\pgfqpoint{1.091767in}{0.683457in}}%
\pgfpathlineto{\pgfqpoint{1.095225in}{0.738990in}}%
\pgfpathlineto{\pgfqpoint{1.096956in}{0.696426in}}%
\pgfpathlineto{\pgfqpoint{1.097822in}{0.734556in}}%
\pgfpathlineto{\pgfqpoint{1.098687in}{0.723163in}}%
\pgfpathlineto{\pgfqpoint{1.100416in}{0.743384in}}%
\pgfpathlineto{\pgfqpoint{1.101281in}{0.768659in}}%
\pgfpathlineto{\pgfqpoint{1.102145in}{0.675032in}}%
\pgfpathlineto{\pgfqpoint{1.103008in}{0.687156in}}%
\pgfpathlineto{\pgfqpoint{1.104736in}{0.725656in}}%
\pgfpathlineto{\pgfqpoint{1.105601in}{0.702066in}}%
\pgfpathlineto{\pgfqpoint{1.106466in}{0.712834in}}%
\pgfpathlineto{\pgfqpoint{1.108192in}{0.692394in}}%
\pgfpathlineto{\pgfqpoint{1.109055in}{0.796387in}}%
\pgfpathlineto{\pgfqpoint{1.110782in}{0.715363in}}%
\pgfpathlineto{\pgfqpoint{1.112512in}{0.694116in}}%
\pgfpathlineto{\pgfqpoint{1.114239in}{0.690599in}}%
\pgfpathlineto{\pgfqpoint{1.115102in}{0.721368in}}%
\pgfpathlineto{\pgfqpoint{1.115967in}{0.717008in}}%
\pgfpathlineto{\pgfqpoint{1.116831in}{0.740964in}}%
\pgfpathlineto{\pgfqpoint{1.117696in}{0.729497in}}%
\pgfpathlineto{\pgfqpoint{1.118561in}{0.642980in}}%
\pgfpathlineto{\pgfqpoint{1.119426in}{0.672685in}}%
\pgfpathlineto{\pgfqpoint{1.120289in}{0.645067in}}%
\pgfpathlineto{\pgfqpoint{1.122016in}{0.727337in}}%
\pgfpathlineto{\pgfqpoint{1.122881in}{0.703454in}}%
\pgfpathlineto{\pgfqpoint{1.124612in}{0.771806in}}%
\pgfpathlineto{\pgfqpoint{1.125476in}{0.692979in}}%
\pgfpathlineto{\pgfqpoint{1.126342in}{0.761330in}}%
\pgfpathlineto{\pgfqpoint{1.128072in}{0.700198in}}%
\pgfpathlineto{\pgfqpoint{1.128936in}{0.725396in}}%
\pgfpathlineto{\pgfqpoint{1.130665in}{0.657231in}}%
\pgfpathlineto{\pgfqpoint{1.132396in}{0.717816in}}%
\pgfpathlineto{\pgfqpoint{1.133261in}{0.681187in}}%
\pgfpathlineto{\pgfqpoint{1.134991in}{0.716058in}}%
\pgfpathlineto{\pgfqpoint{1.135856in}{0.729976in}}%
\pgfpathlineto{\pgfqpoint{1.136722in}{0.672466in}}%
\pgfpathlineto{\pgfqpoint{1.137586in}{0.683640in}}%
\pgfpathlineto{\pgfqpoint{1.140181in}{0.745544in}}%
\pgfpathlineto{\pgfqpoint{1.141044in}{0.741296in}}%
\pgfpathlineto{\pgfqpoint{1.141910in}{0.729794in}}%
\pgfpathlineto{\pgfqpoint{1.143641in}{0.656606in}}%
\pgfpathlineto{\pgfqpoint{1.144506in}{0.711112in}}%
\pgfpathlineto{\pgfqpoint{1.145371in}{0.676607in}}%
\pgfpathlineto{\pgfqpoint{1.146235in}{0.767120in}}%
\pgfpathlineto{\pgfqpoint{1.147100in}{0.765983in}}%
\pgfpathlineto{\pgfqpoint{1.148829in}{0.697778in}}%
\pgfpathlineto{\pgfqpoint{1.149692in}{0.749133in}}%
\pgfpathlineto{\pgfqpoint{1.151424in}{0.709135in}}%
\pgfpathlineto{\pgfqpoint{1.153155in}{0.748694in}}%
\pgfpathlineto{\pgfqpoint{1.154020in}{0.690892in}}%
\pgfpathlineto{\pgfqpoint{1.154885in}{0.743895in}}%
\pgfpathlineto{\pgfqpoint{1.155749in}{0.740817in}}%
\pgfpathlineto{\pgfqpoint{1.156614in}{0.771806in}}%
\pgfpathlineto{\pgfqpoint{1.159208in}{0.574742in}}%
\pgfpathlineto{\pgfqpoint{1.160937in}{0.763567in}}%
\pgfpathlineto{\pgfqpoint{1.162667in}{0.719135in}}%
\pgfpathlineto{\pgfqpoint{1.163531in}{0.779098in}}%
\pgfpathlineto{\pgfqpoint{1.165259in}{0.705801in}}%
\pgfpathlineto{\pgfqpoint{1.166990in}{0.769426in}}%
\pgfpathlineto{\pgfqpoint{1.167854in}{0.744849in}}%
\pgfpathlineto{\pgfqpoint{1.168720in}{0.741808in}}%
\pgfpathlineto{\pgfqpoint{1.169584in}{0.682430in}}%
\pgfpathlineto{\pgfqpoint{1.171315in}{0.720784in}}%
\pgfpathlineto{\pgfqpoint{1.172180in}{0.700417in}}%
\pgfpathlineto{\pgfqpoint{1.173912in}{0.728401in}}%
\pgfpathlineto{\pgfqpoint{1.174778in}{0.726972in}}%
\pgfpathlineto{\pgfqpoint{1.176508in}{0.668657in}}%
\pgfpathlineto{\pgfqpoint{1.178239in}{0.722026in}}%
\pgfpathlineto{\pgfqpoint{1.179103in}{0.707742in}}%
\pgfpathlineto{\pgfqpoint{1.181699in}{0.761330in}}%
\pgfpathlineto{\pgfqpoint{1.182565in}{0.723894in}}%
\pgfpathlineto{\pgfqpoint{1.183430in}{0.790378in}}%
\pgfpathlineto{\pgfqpoint{1.185160in}{0.677704in}}%
\pgfpathlineto{\pgfqpoint{1.186891in}{0.733968in}}%
\pgfpathlineto{\pgfqpoint{1.188623in}{0.779058in}}%
\pgfpathlineto{\pgfqpoint{1.189487in}{0.690197in}}%
\pgfpathlineto{\pgfqpoint{1.190352in}{0.744041in}}%
\pgfpathlineto{\pgfqpoint{1.191218in}{0.659465in}}%
\pgfpathlineto{\pgfqpoint{1.192083in}{0.837668in}}%
\pgfpathlineto{\pgfqpoint{1.192948in}{0.716277in}}%
\pgfpathlineto{\pgfqpoint{1.193813in}{0.736164in}}%
\pgfpathlineto{\pgfqpoint{1.194678in}{0.747484in}}%
\pgfpathlineto{\pgfqpoint{1.195541in}{0.708692in}}%
\pgfpathlineto{\pgfqpoint{1.196406in}{0.730378in}}%
\pgfpathlineto{\pgfqpoint{1.198137in}{0.703089in}}%
\pgfpathlineto{\pgfqpoint{1.199001in}{0.763088in}}%
\pgfpathlineto{\pgfqpoint{1.199865in}{0.700304in}}%
\pgfpathlineto{\pgfqpoint{1.200729in}{0.746348in}}%
\pgfpathlineto{\pgfqpoint{1.201595in}{0.691586in}}%
\pgfpathlineto{\pgfqpoint{1.202459in}{0.747923in}}%
\pgfpathlineto{\pgfqpoint{1.203324in}{0.673489in}}%
\pgfpathlineto{\pgfqpoint{1.204189in}{0.702683in}}%
\pgfpathlineto{\pgfqpoint{1.205054in}{0.696089in}}%
\pgfpathlineto{\pgfqpoint{1.205919in}{0.693710in}}%
\pgfpathlineto{\pgfqpoint{1.207651in}{0.749718in}}%
\pgfpathlineto{\pgfqpoint{1.208516in}{0.715140in}}%
\pgfpathlineto{\pgfqpoint{1.209379in}{0.720232in}}%
\pgfpathlineto{\pgfqpoint{1.211111in}{0.680891in}}%
\pgfpathlineto{\pgfqpoint{1.211977in}{0.735141in}}%
\pgfpathlineto{\pgfqpoint{1.212842in}{0.730049in}}%
\pgfpathlineto{\pgfqpoint{1.213706in}{0.732136in}}%
\pgfpathlineto{\pgfqpoint{1.214572in}{0.690599in}}%
\pgfpathlineto{\pgfqpoint{1.217169in}{0.817558in}}%
\pgfpathlineto{\pgfqpoint{1.218035in}{0.682174in}}%
\pgfpathlineto{\pgfqpoint{1.218902in}{0.796826in}}%
\pgfpathlineto{\pgfqpoint{1.220634in}{0.682942in}}%
\pgfpathlineto{\pgfqpoint{1.222366in}{0.742247in}}%
\pgfpathlineto{\pgfqpoint{1.223232in}{0.735507in}}%
\pgfpathlineto{\pgfqpoint{1.224098in}{0.737520in}}%
\pgfpathlineto{\pgfqpoint{1.224965in}{0.768399in}}%
\pgfpathlineto{\pgfqpoint{1.225831in}{0.674553in}}%
\pgfpathlineto{\pgfqpoint{1.228427in}{0.727045in}}%
\pgfpathlineto{\pgfqpoint{1.230156in}{0.686165in}}%
\pgfpathlineto{\pgfqpoint{1.231020in}{0.728913in}}%
\pgfpathlineto{\pgfqpoint{1.232751in}{0.696568in}}%
\pgfpathlineto{\pgfqpoint{1.233617in}{0.701952in}}%
\pgfpathlineto{\pgfqpoint{1.234482in}{0.668913in}}%
\pgfpathlineto{\pgfqpoint{1.236211in}{0.728182in}}%
\pgfpathlineto{\pgfqpoint{1.237076in}{0.737817in}}%
\pgfpathlineto{\pgfqpoint{1.237942in}{0.678548in}}%
\pgfpathlineto{\pgfqpoint{1.238807in}{0.765910in}}%
\pgfpathlineto{\pgfqpoint{1.239671in}{0.757339in}}%
\pgfpathlineto{\pgfqpoint{1.240537in}{0.761257in}}%
\pgfpathlineto{\pgfqpoint{1.241402in}{0.677558in}}%
\pgfpathlineto{\pgfqpoint{1.242268in}{0.686019in}}%
\pgfpathlineto{\pgfqpoint{1.243999in}{0.733895in}}%
\pgfpathlineto{\pgfqpoint{1.244864in}{0.687668in}}%
\pgfpathlineto{\pgfqpoint{1.245728in}{0.748069in}}%
\pgfpathlineto{\pgfqpoint{1.246593in}{0.706678in}}%
\pgfpathlineto{\pgfqpoint{1.247458in}{0.716788in}}%
\pgfpathlineto{\pgfqpoint{1.249187in}{0.708660in}}%
\pgfpathlineto{\pgfqpoint{1.250053in}{0.695179in}}%
\pgfpathlineto{\pgfqpoint{1.250916in}{0.756864in}}%
\pgfpathlineto{\pgfqpoint{1.251781in}{0.736643in}}%
\pgfpathlineto{\pgfqpoint{1.252647in}{0.747484in}}%
\pgfpathlineto{\pgfqpoint{1.253512in}{0.704445in}}%
\pgfpathlineto{\pgfqpoint{1.254377in}{0.776971in}}%
\pgfpathlineto{\pgfqpoint{1.255243in}{0.721807in}}%
\pgfpathlineto{\pgfqpoint{1.256109in}{0.762906in}}%
\pgfpathlineto{\pgfqpoint{1.256974in}{0.742100in}}%
\pgfpathlineto{\pgfqpoint{1.257840in}{0.781112in}}%
\pgfpathlineto{\pgfqpoint{1.258706in}{0.671370in}}%
\pgfpathlineto{\pgfqpoint{1.259571in}{0.697924in}}%
\pgfpathlineto{\pgfqpoint{1.260436in}{0.721734in}}%
\pgfpathlineto{\pgfqpoint{1.261299in}{0.698988in}}%
\pgfpathlineto{\pgfqpoint{1.262165in}{0.755946in}}%
\pgfpathlineto{\pgfqpoint{1.263030in}{0.675032in}}%
\pgfpathlineto{\pgfqpoint{1.264759in}{0.711404in}}%
\pgfpathlineto{\pgfqpoint{1.265625in}{0.719464in}}%
\pgfpathlineto{\pgfqpoint{1.266490in}{0.798328in}}%
\pgfpathlineto{\pgfqpoint{1.269081in}{0.714336in}}%
\pgfpathlineto{\pgfqpoint{1.269946in}{0.713970in}}%
\pgfpathlineto{\pgfqpoint{1.270811in}{0.757229in}}%
\pgfpathlineto{\pgfqpoint{1.271677in}{0.747558in}}%
\pgfpathlineto{\pgfqpoint{1.272543in}{0.735507in}}%
\pgfpathlineto{\pgfqpoint{1.274272in}{0.647852in}}%
\pgfpathlineto{\pgfqpoint{1.276003in}{0.772244in}}%
\pgfpathlineto{\pgfqpoint{1.276869in}{0.718510in}}%
\pgfpathlineto{\pgfqpoint{1.277734in}{0.799757in}}%
\pgfpathlineto{\pgfqpoint{1.279465in}{0.745471in}}%
\pgfpathlineto{\pgfqpoint{1.281196in}{0.705874in}}%
\pgfpathlineto{\pgfqpoint{1.282062in}{0.709244in}}%
\pgfpathlineto{\pgfqpoint{1.282927in}{0.805433in}}%
\pgfpathlineto{\pgfqpoint{1.283793in}{0.702797in}}%
\pgfpathlineto{\pgfqpoint{1.284658in}{0.785802in}}%
\pgfpathlineto{\pgfqpoint{1.285521in}{0.688732in}}%
\pgfpathlineto{\pgfqpoint{1.286386in}{0.704810in}}%
\pgfpathlineto{\pgfqpoint{1.287250in}{0.708254in}}%
\pgfpathlineto{\pgfqpoint{1.288114in}{0.726606in}}%
\pgfpathlineto{\pgfqpoint{1.288980in}{0.771002in}}%
\pgfpathlineto{\pgfqpoint{1.289846in}{0.705874in}}%
\pgfpathlineto{\pgfqpoint{1.290711in}{0.804223in}}%
\pgfpathlineto{\pgfqpoint{1.291576in}{0.756202in}}%
\pgfpathlineto{\pgfqpoint{1.292440in}{0.800342in}}%
\pgfpathlineto{\pgfqpoint{1.293306in}{0.799059in}}%
\pgfpathlineto{\pgfqpoint{1.295037in}{0.714628in}}%
\pgfpathlineto{\pgfqpoint{1.295904in}{0.815836in}}%
\pgfpathlineto{\pgfqpoint{1.297633in}{0.645652in}}%
\pgfpathlineto{\pgfqpoint{1.298498in}{0.659611in}}%
\pgfpathlineto{\pgfqpoint{1.299363in}{0.623092in}}%
\pgfpathlineto{\pgfqpoint{1.300229in}{0.707815in}}%
\pgfpathlineto{\pgfqpoint{1.301092in}{0.675324in}}%
\pgfpathlineto{\pgfqpoint{1.301958in}{0.697413in}}%
\pgfpathlineto{\pgfqpoint{1.302823in}{0.697120in}}%
\pgfpathlineto{\pgfqpoint{1.304552in}{0.714263in}}%
\pgfpathlineto{\pgfqpoint{1.305417in}{0.762979in}}%
\pgfpathlineto{\pgfqpoint{1.307147in}{0.655177in}}%
\pgfpathlineto{\pgfqpoint{1.308013in}{0.760892in}}%
\pgfpathlineto{\pgfqpoint{1.308879in}{0.757302in}}%
\pgfpathlineto{\pgfqpoint{1.309745in}{0.756019in}}%
\pgfpathlineto{\pgfqpoint{1.310610in}{0.707121in}}%
\pgfpathlineto{\pgfqpoint{1.311476in}{0.765764in}}%
\pgfpathlineto{\pgfqpoint{1.312342in}{0.714409in}}%
\pgfpathlineto{\pgfqpoint{1.313208in}{0.784153in}}%
\pgfpathlineto{\pgfqpoint{1.314074in}{0.733566in}}%
\pgfpathlineto{\pgfqpoint{1.315803in}{0.792834in}}%
\pgfpathlineto{\pgfqpoint{1.318395in}{0.699134in}}%
\pgfpathlineto{\pgfqpoint{1.320124in}{0.739429in}}%
\pgfpathlineto{\pgfqpoint{1.321854in}{0.700636in}}%
\pgfpathlineto{\pgfqpoint{1.322720in}{0.723163in}}%
\pgfpathlineto{\pgfqpoint{1.323583in}{0.709025in}}%
\pgfpathlineto{\pgfqpoint{1.324448in}{0.724592in}}%
\pgfpathlineto{\pgfqpoint{1.326179in}{0.681918in}}%
\pgfpathlineto{\pgfqpoint{1.329636in}{0.757704in}}%
\pgfpathlineto{\pgfqpoint{1.330500in}{0.768842in}}%
\pgfpathlineto{\pgfqpoint{1.332230in}{0.725583in}}%
\pgfpathlineto{\pgfqpoint{1.333092in}{0.738146in}}%
\pgfpathlineto{\pgfqpoint{1.333954in}{0.735214in}}%
\pgfpathlineto{\pgfqpoint{1.334819in}{0.674995in}}%
\pgfpathlineto{\pgfqpoint{1.336550in}{0.743457in}}%
\pgfpathlineto{\pgfqpoint{1.337413in}{0.754923in}}%
\pgfpathlineto{\pgfqpoint{1.338278in}{0.709317in}}%
\pgfpathlineto{\pgfqpoint{1.340007in}{0.745251in}}%
\pgfpathlineto{\pgfqpoint{1.340869in}{0.709390in}}%
\pgfpathlineto{\pgfqpoint{1.341734in}{0.729684in}}%
\pgfpathlineto{\pgfqpoint{1.342599in}{0.692394in}}%
\pgfpathlineto{\pgfqpoint{1.343462in}{0.752868in}}%
\pgfpathlineto{\pgfqpoint{1.344329in}{0.670415in}}%
\pgfpathlineto{\pgfqpoint{1.346061in}{0.761549in}}%
\pgfpathlineto{\pgfqpoint{1.347786in}{0.699792in}}%
\pgfpathlineto{\pgfqpoint{1.348652in}{0.755654in}}%
\pgfpathlineto{\pgfqpoint{1.349515in}{0.709902in}}%
\pgfpathlineto{\pgfqpoint{1.350380in}{0.747484in}}%
\pgfpathlineto{\pgfqpoint{1.352107in}{0.668365in}}%
\pgfpathlineto{\pgfqpoint{1.352972in}{0.706240in}}%
\pgfpathlineto{\pgfqpoint{1.353837in}{0.646752in}}%
\pgfpathlineto{\pgfqpoint{1.354702in}{0.781478in}}%
\pgfpathlineto{\pgfqpoint{1.356431in}{0.717227in}}%
\pgfpathlineto{\pgfqpoint{1.357296in}{0.790524in}}%
\pgfpathlineto{\pgfqpoint{1.359028in}{0.696495in}}%
\pgfpathlineto{\pgfqpoint{1.359893in}{0.784003in}}%
\pgfpathlineto{\pgfqpoint{1.360758in}{0.655141in}}%
\pgfpathlineto{\pgfqpoint{1.362488in}{0.752211in}}%
\pgfpathlineto{\pgfqpoint{1.363353in}{0.750672in}}%
\pgfpathlineto{\pgfqpoint{1.364217in}{0.719354in}}%
\pgfpathlineto{\pgfqpoint{1.365081in}{0.794885in}}%
\pgfpathlineto{\pgfqpoint{1.365947in}{0.732762in}}%
\pgfpathlineto{\pgfqpoint{1.366813in}{0.761955in}}%
\pgfpathlineto{\pgfqpoint{1.368543in}{0.666936in}}%
\pgfpathlineto{\pgfqpoint{1.369408in}{0.677964in}}%
\pgfpathlineto{\pgfqpoint{1.370273in}{0.704810in}}%
\pgfpathlineto{\pgfqpoint{1.371138in}{0.683859in}}%
\pgfpathlineto{\pgfqpoint{1.372002in}{0.747996in}}%
\pgfpathlineto{\pgfqpoint{1.372866in}{0.696458in}}%
\pgfpathlineto{\pgfqpoint{1.373729in}{0.736424in}}%
\pgfpathlineto{\pgfqpoint{1.374594in}{0.731698in}}%
\pgfpathlineto{\pgfqpoint{1.375459in}{0.692248in}}%
\pgfpathlineto{\pgfqpoint{1.376322in}{0.697924in}}%
\pgfpathlineto{\pgfqpoint{1.377187in}{0.726570in}}%
\pgfpathlineto{\pgfqpoint{1.378916in}{0.625837in}}%
\pgfpathlineto{\pgfqpoint{1.379781in}{0.713089in}}%
\pgfpathlineto{\pgfqpoint{1.380646in}{0.681074in}}%
\pgfpathlineto{\pgfqpoint{1.381509in}{0.740817in}}%
\pgfpathlineto{\pgfqpoint{1.382373in}{0.684152in}}%
\pgfpathlineto{\pgfqpoint{1.383238in}{0.710966in}}%
\pgfpathlineto{\pgfqpoint{1.384968in}{0.682211in}}%
\pgfpathlineto{\pgfqpoint{1.385831in}{0.717008in}}%
\pgfpathlineto{\pgfqpoint{1.386696in}{0.699244in}}%
\pgfpathlineto{\pgfqpoint{1.387562in}{0.786569in}}%
\pgfpathlineto{\pgfqpoint{1.389291in}{0.703089in}}%
\pgfpathlineto{\pgfqpoint{1.390156in}{0.704884in}}%
\pgfpathlineto{\pgfqpoint{1.391021in}{0.743237in}}%
\pgfpathlineto{\pgfqpoint{1.391886in}{0.715213in}}%
\pgfpathlineto{\pgfqpoint{1.393613in}{0.731990in}}%
\pgfpathlineto{\pgfqpoint{1.395342in}{0.699426in}}%
\pgfpathlineto{\pgfqpoint{1.396208in}{0.669794in}}%
\pgfpathlineto{\pgfqpoint{1.397073in}{0.716131in}}%
\pgfpathlineto{\pgfqpoint{1.397938in}{0.664922in}}%
\pgfpathlineto{\pgfqpoint{1.399668in}{0.738438in}}%
\pgfpathlineto{\pgfqpoint{1.400534in}{0.716058in}}%
\pgfpathlineto{\pgfqpoint{1.401400in}{0.721478in}}%
\pgfpathlineto{\pgfqpoint{1.402265in}{0.776605in}}%
\pgfpathlineto{\pgfqpoint{1.403131in}{0.747119in}}%
\pgfpathlineto{\pgfqpoint{1.403997in}{0.760015in}}%
\pgfpathlineto{\pgfqpoint{1.404863in}{0.751845in}}%
\pgfpathlineto{\pgfqpoint{1.406594in}{0.716788in}}%
\pgfpathlineto{\pgfqpoint{1.408324in}{0.750635in}}%
\pgfpathlineto{\pgfqpoint{1.409190in}{0.735580in}}%
\pgfpathlineto{\pgfqpoint{1.410055in}{0.800196in}}%
\pgfpathlineto{\pgfqpoint{1.411785in}{0.757375in}}%
\pgfpathlineto{\pgfqpoint{1.412651in}{0.760965in}}%
\pgfpathlineto{\pgfqpoint{1.413515in}{0.709390in}}%
\pgfpathlineto{\pgfqpoint{1.414379in}{0.782322in}}%
\pgfpathlineto{\pgfqpoint{1.415244in}{0.766312in}}%
\pgfpathlineto{\pgfqpoint{1.416108in}{0.701075in}}%
\pgfpathlineto{\pgfqpoint{1.416972in}{0.785798in}}%
\pgfpathlineto{\pgfqpoint{1.418701in}{0.702723in}}%
\pgfpathlineto{\pgfqpoint{1.419566in}{0.747558in}}%
\pgfpathlineto{\pgfqpoint{1.420430in}{0.717596in}}%
\pgfpathlineto{\pgfqpoint{1.421294in}{0.727743in}}%
\pgfpathlineto{\pgfqpoint{1.422159in}{0.709025in}}%
\pgfpathlineto{\pgfqpoint{1.423024in}{0.618732in}}%
\pgfpathlineto{\pgfqpoint{1.423884in}{0.720638in}}%
\pgfpathlineto{\pgfqpoint{1.424749in}{0.649610in}}%
\pgfpathlineto{\pgfqpoint{1.426480in}{0.770417in}}%
\pgfpathlineto{\pgfqpoint{1.427345in}{0.746790in}}%
\pgfpathlineto{\pgfqpoint{1.428206in}{0.736936in}}%
\pgfpathlineto{\pgfqpoint{1.429937in}{0.632102in}}%
\pgfpathlineto{\pgfqpoint{1.431667in}{0.753932in}}%
\pgfpathlineto{\pgfqpoint{1.434265in}{0.685288in}}%
\pgfpathlineto{\pgfqpoint{1.435996in}{0.743164in}}%
\pgfpathlineto{\pgfqpoint{1.436861in}{0.699317in}}%
\pgfpathlineto{\pgfqpoint{1.437726in}{0.711039in}}%
\pgfpathlineto{\pgfqpoint{1.438592in}{0.734703in}}%
\pgfpathlineto{\pgfqpoint{1.439455in}{0.680124in}}%
\pgfpathlineto{\pgfqpoint{1.440320in}{0.717706in}}%
\pgfpathlineto{\pgfqpoint{1.441185in}{0.698988in}}%
\pgfpathlineto{\pgfqpoint{1.442050in}{0.707815in}}%
\pgfpathlineto{\pgfqpoint{1.443778in}{0.689353in}}%
\pgfpathlineto{\pgfqpoint{1.447237in}{0.758472in}}%
\pgfpathlineto{\pgfqpoint{1.448103in}{0.729351in}}%
\pgfpathlineto{\pgfqpoint{1.448967in}{0.658949in}}%
\pgfpathlineto{\pgfqpoint{1.449831in}{0.719095in}}%
\pgfpathlineto{\pgfqpoint{1.451559in}{0.667155in}}%
\pgfpathlineto{\pgfqpoint{1.452425in}{0.738398in}}%
\pgfpathlineto{\pgfqpoint{1.453289in}{0.696129in}}%
\pgfpathlineto{\pgfqpoint{1.454154in}{0.802356in}}%
\pgfpathlineto{\pgfqpoint{1.455019in}{0.748142in}}%
\pgfpathlineto{\pgfqpoint{1.455884in}{0.775797in}}%
\pgfpathlineto{\pgfqpoint{1.457614in}{0.724698in}}%
\pgfpathlineto{\pgfqpoint{1.458480in}{0.639021in}}%
\pgfpathlineto{\pgfqpoint{1.459345in}{0.733160in}}%
\pgfpathlineto{\pgfqpoint{1.460209in}{0.719095in}}%
\pgfpathlineto{\pgfqpoint{1.461941in}{0.665433in}}%
\pgfpathlineto{\pgfqpoint{1.462806in}{0.686092in}}%
\pgfpathlineto{\pgfqpoint{1.463672in}{0.749279in}}%
\pgfpathlineto{\pgfqpoint{1.464538in}{0.745138in}}%
\pgfpathlineto{\pgfqpoint{1.466268in}{0.663639in}}%
\pgfpathlineto{\pgfqpoint{1.467999in}{0.720049in}}%
\pgfpathlineto{\pgfqpoint{1.468864in}{0.744553in}}%
\pgfpathlineto{\pgfqpoint{1.470594in}{0.690193in}}%
\pgfpathlineto{\pgfqpoint{1.471459in}{0.759349in}}%
\pgfpathlineto{\pgfqpoint{1.472326in}{0.608544in}}%
\pgfpathlineto{\pgfqpoint{1.473192in}{0.706971in}}%
\pgfpathlineto{\pgfqpoint{1.474056in}{0.696860in}}%
\pgfpathlineto{\pgfqpoint{1.474922in}{0.674699in}}%
\pgfpathlineto{\pgfqpoint{1.475788in}{0.686312in}}%
\pgfpathlineto{\pgfqpoint{1.478386in}{0.755873in}}%
\pgfpathlineto{\pgfqpoint{1.480117in}{0.723675in}}%
\pgfpathlineto{\pgfqpoint{1.481847in}{0.744955in}}%
\pgfpathlineto{\pgfqpoint{1.483575in}{0.674261in}}%
\pgfpathlineto{\pgfqpoint{1.484441in}{0.652172in}}%
\pgfpathlineto{\pgfqpoint{1.485307in}{0.731438in}}%
\pgfpathlineto{\pgfqpoint{1.486174in}{0.719387in}}%
\pgfpathlineto{\pgfqpoint{1.487040in}{0.634847in}}%
\pgfpathlineto{\pgfqpoint{1.489635in}{0.749060in}}%
\pgfpathlineto{\pgfqpoint{1.492231in}{0.682795in}}%
\pgfpathlineto{\pgfqpoint{1.493097in}{0.779902in}}%
\pgfpathlineto{\pgfqpoint{1.494827in}{0.691769in}}%
\pgfpathlineto{\pgfqpoint{1.497424in}{0.778546in}}%
\pgfpathlineto{\pgfqpoint{1.499153in}{0.691696in}}%
\pgfpathlineto{\pgfqpoint{1.500017in}{0.726493in}}%
\pgfpathlineto{\pgfqpoint{1.500882in}{0.724808in}}%
\pgfpathlineto{\pgfqpoint{1.501745in}{0.636678in}}%
\pgfpathlineto{\pgfqpoint{1.504339in}{0.758691in}}%
\pgfpathlineto{\pgfqpoint{1.506068in}{0.727958in}}%
\pgfpathlineto{\pgfqpoint{1.506933in}{0.764002in}}%
\pgfpathlineto{\pgfqpoint{1.507798in}{0.699240in}}%
\pgfpathlineto{\pgfqpoint{1.508662in}{0.782355in}}%
\pgfpathlineto{\pgfqpoint{1.509524in}{0.764440in}}%
\pgfpathlineto{\pgfqpoint{1.510389in}{0.697482in}}%
\pgfpathlineto{\pgfqpoint{1.511254in}{0.786675in}}%
\pgfpathlineto{\pgfqpoint{1.512984in}{0.732096in}}%
\pgfpathlineto{\pgfqpoint{1.513848in}{0.779789in}}%
\pgfpathlineto{\pgfqpoint{1.514713in}{0.653342in}}%
\pgfpathlineto{\pgfqpoint{1.515579in}{0.688508in}}%
\pgfpathlineto{\pgfqpoint{1.516443in}{0.733160in}}%
\pgfpathlineto{\pgfqpoint{1.517308in}{0.701513in}}%
\pgfpathlineto{\pgfqpoint{1.518172in}{0.706532in}}%
\pgfpathlineto{\pgfqpoint{1.519904in}{0.720853in}}%
\pgfpathlineto{\pgfqpoint{1.520767in}{0.726387in}}%
\pgfpathlineto{\pgfqpoint{1.522496in}{0.689609in}}%
\pgfpathlineto{\pgfqpoint{1.524226in}{0.764554in}}%
\pgfpathlineto{\pgfqpoint{1.525091in}{0.718839in}}%
\pgfpathlineto{\pgfqpoint{1.525956in}{0.758472in}}%
\pgfpathlineto{\pgfqpoint{1.527685in}{0.715213in}}%
\pgfpathlineto{\pgfqpoint{1.529411in}{0.771038in}}%
\pgfpathlineto{\pgfqpoint{1.531141in}{0.738803in}}%
\pgfpathlineto{\pgfqpoint{1.532005in}{0.744005in}}%
\pgfpathlineto{\pgfqpoint{1.532870in}{0.778034in}}%
\pgfpathlineto{\pgfqpoint{1.533734in}{0.770783in}}%
\pgfpathlineto{\pgfqpoint{1.534599in}{0.715176in}}%
\pgfpathlineto{\pgfqpoint{1.535463in}{0.724592in}}%
\pgfpathlineto{\pgfqpoint{1.537190in}{0.696056in}}%
\pgfpathlineto{\pgfqpoint{1.538055in}{0.649866in}}%
\pgfpathlineto{\pgfqpoint{1.539786in}{0.731990in}}%
\pgfpathlineto{\pgfqpoint{1.540652in}{0.695691in}}%
\pgfpathlineto{\pgfqpoint{1.541516in}{0.748621in}}%
\pgfpathlineto{\pgfqpoint{1.543246in}{0.706313in}}%
\pgfpathlineto{\pgfqpoint{1.544975in}{0.729538in}}%
\pgfpathlineto{\pgfqpoint{1.545841in}{0.719866in}}%
\pgfpathlineto{\pgfqpoint{1.546705in}{0.685288in}}%
\pgfpathlineto{\pgfqpoint{1.547570in}{0.710893in}}%
\pgfpathlineto{\pgfqpoint{1.548435in}{0.671735in}}%
\pgfpathlineto{\pgfqpoint{1.549300in}{0.751001in}}%
\pgfpathlineto{\pgfqpoint{1.550163in}{0.738803in}}%
\pgfpathlineto{\pgfqpoint{1.551028in}{0.725104in}}%
\pgfpathlineto{\pgfqpoint{1.551893in}{0.729205in}}%
\pgfpathlineto{\pgfqpoint{1.552758in}{0.682576in}}%
\pgfpathlineto{\pgfqpoint{1.553624in}{0.693088in}}%
\pgfpathlineto{\pgfqpoint{1.554488in}{0.768801in}}%
\pgfpathlineto{\pgfqpoint{1.555353in}{0.730269in}}%
\pgfpathlineto{\pgfqpoint{1.556218in}{0.745251in}}%
\pgfpathlineto{\pgfqpoint{1.557083in}{0.677119in}}%
\pgfpathlineto{\pgfqpoint{1.557945in}{0.706313in}}%
\pgfpathlineto{\pgfqpoint{1.558810in}{0.692394in}}%
\pgfpathlineto{\pgfqpoint{1.559676in}{0.745324in}}%
\pgfpathlineto{\pgfqpoint{1.561408in}{0.660123in}}%
\pgfpathlineto{\pgfqpoint{1.562273in}{0.776459in}}%
\pgfpathlineto{\pgfqpoint{1.563139in}{0.704262in}}%
\pgfpathlineto{\pgfqpoint{1.564005in}{0.744959in}}%
\pgfpathlineto{\pgfqpoint{1.564872in}{0.654739in}}%
\pgfpathlineto{\pgfqpoint{1.565738in}{0.687302in}}%
\pgfpathlineto{\pgfqpoint{1.567467in}{0.605069in}}%
\pgfpathlineto{\pgfqpoint{1.569196in}{0.751260in}}%
\pgfpathlineto{\pgfqpoint{1.570061in}{0.715582in}}%
\pgfpathlineto{\pgfqpoint{1.570924in}{0.717121in}}%
\pgfpathlineto{\pgfqpoint{1.571789in}{0.723163in}}%
\pgfpathlineto{\pgfqpoint{1.573519in}{0.692175in}}%
\pgfpathlineto{\pgfqpoint{1.574383in}{0.751001in}}%
\pgfpathlineto{\pgfqpoint{1.576114in}{0.698363in}}%
\pgfpathlineto{\pgfqpoint{1.576978in}{0.755873in}}%
\pgfpathlineto{\pgfqpoint{1.577842in}{0.649720in}}%
\pgfpathlineto{\pgfqpoint{1.578706in}{0.665214in}}%
\pgfpathlineto{\pgfqpoint{1.580437in}{0.688585in}}%
\pgfpathlineto{\pgfqpoint{1.581303in}{0.742100in}}%
\pgfpathlineto{\pgfqpoint{1.583036in}{0.642980in}}%
\pgfpathlineto{\pgfqpoint{1.583903in}{0.704591in}}%
\pgfpathlineto{\pgfqpoint{1.584769in}{0.624042in}}%
\pgfpathlineto{\pgfqpoint{1.585634in}{0.732648in}}%
\pgfpathlineto{\pgfqpoint{1.587366in}{0.672247in}}%
\pgfpathlineto{\pgfqpoint{1.588232in}{0.772943in}}%
\pgfpathlineto{\pgfqpoint{1.589962in}{0.721222in}}%
\pgfpathlineto{\pgfqpoint{1.590828in}{0.732356in}}%
\pgfpathlineto{\pgfqpoint{1.591692in}{0.720378in}}%
\pgfpathlineto{\pgfqpoint{1.592558in}{0.663346in}}%
\pgfpathlineto{\pgfqpoint{1.595155in}{0.723163in}}%
\pgfpathlineto{\pgfqpoint{1.596021in}{0.694335in}}%
\pgfpathlineto{\pgfqpoint{1.596886in}{0.612357in}}%
\pgfpathlineto{\pgfqpoint{1.597751in}{0.751845in}}%
\pgfpathlineto{\pgfqpoint{1.598615in}{0.740233in}}%
\pgfpathlineto{\pgfqpoint{1.601207in}{0.700417in}}%
\pgfpathlineto{\pgfqpoint{1.602071in}{0.719647in}}%
\pgfpathlineto{\pgfqpoint{1.602937in}{0.703714in}}%
\pgfpathlineto{\pgfqpoint{1.603803in}{0.710747in}}%
\pgfpathlineto{\pgfqpoint{1.604669in}{0.728035in}}%
\pgfpathlineto{\pgfqpoint{1.606401in}{0.687010in}}%
\pgfpathlineto{\pgfqpoint{1.608130in}{0.716167in}}%
\pgfpathlineto{\pgfqpoint{1.608995in}{0.690745in}}%
\pgfpathlineto{\pgfqpoint{1.609860in}{0.740598in}}%
\pgfpathlineto{\pgfqpoint{1.610725in}{0.651332in}}%
\pgfpathlineto{\pgfqpoint{1.614186in}{0.758001in}}%
\pgfpathlineto{\pgfqpoint{1.615051in}{0.702175in}}%
\pgfpathlineto{\pgfqpoint{1.615917in}{0.756718in}}%
\pgfpathlineto{\pgfqpoint{1.616782in}{0.744886in}}%
\pgfpathlineto{\pgfqpoint{1.617645in}{0.751626in}}%
\pgfpathlineto{\pgfqpoint{1.619375in}{0.653935in}}%
\pgfpathlineto{\pgfqpoint{1.621104in}{0.709025in}}%
\pgfpathlineto{\pgfqpoint{1.621969in}{0.681334in}}%
\pgfpathlineto{\pgfqpoint{1.622835in}{0.792505in}}%
\pgfpathlineto{\pgfqpoint{1.624564in}{0.702431in}}%
\pgfpathlineto{\pgfqpoint{1.625428in}{0.734922in}}%
\pgfpathlineto{\pgfqpoint{1.627159in}{0.688951in}}%
\pgfpathlineto{\pgfqpoint{1.628022in}{0.718218in}}%
\pgfpathlineto{\pgfqpoint{1.628886in}{0.810598in}}%
\pgfpathlineto{\pgfqpoint{1.629752in}{0.703199in}}%
\pgfpathlineto{\pgfqpoint{1.630617in}{0.730049in}}%
\pgfpathlineto{\pgfqpoint{1.631481in}{0.778327in}}%
\pgfpathlineto{\pgfqpoint{1.632344in}{0.706313in}}%
\pgfpathlineto{\pgfqpoint{1.633210in}{0.769865in}}%
\pgfpathlineto{\pgfqpoint{1.634073in}{0.699573in}}%
\pgfpathlineto{\pgfqpoint{1.634936in}{0.700856in}}%
\pgfpathlineto{\pgfqpoint{1.636668in}{0.751476in}}%
\pgfpathlineto{\pgfqpoint{1.637533in}{0.732429in}}%
\pgfpathlineto{\pgfqpoint{1.638397in}{0.779975in}}%
\pgfpathlineto{\pgfqpoint{1.639263in}{0.726350in}}%
\pgfpathlineto{\pgfqpoint{1.640129in}{0.786277in}}%
\pgfpathlineto{\pgfqpoint{1.641860in}{0.644409in}}%
\pgfpathlineto{\pgfqpoint{1.642725in}{0.689682in}}%
\pgfpathlineto{\pgfqpoint{1.643590in}{0.751037in}}%
\pgfpathlineto{\pgfqpoint{1.644456in}{0.699426in}}%
\pgfpathlineto{\pgfqpoint{1.645321in}{0.719793in}}%
\pgfpathlineto{\pgfqpoint{1.646187in}{0.779683in}}%
\pgfpathlineto{\pgfqpoint{1.647053in}{0.772212in}}%
\pgfpathlineto{\pgfqpoint{1.647918in}{0.705468in}}%
\pgfpathlineto{\pgfqpoint{1.648784in}{0.710012in}}%
\pgfpathlineto{\pgfqpoint{1.649645in}{0.681220in}}%
\pgfpathlineto{\pgfqpoint{1.650509in}{0.730488in}}%
\pgfpathlineto{\pgfqpoint{1.651372in}{0.706240in}}%
\pgfpathlineto{\pgfqpoint{1.654830in}{0.790378in}}%
\pgfpathlineto{\pgfqpoint{1.655694in}{0.722578in}}%
\pgfpathlineto{\pgfqpoint{1.657422in}{0.783784in}}%
\pgfpathlineto{\pgfqpoint{1.660016in}{0.683713in}}%
\pgfpathlineto{\pgfqpoint{1.660879in}{0.763604in}}%
\pgfpathlineto{\pgfqpoint{1.662610in}{0.699280in}}%
\pgfpathlineto{\pgfqpoint{1.663476in}{0.724738in}}%
\pgfpathlineto{\pgfqpoint{1.664342in}{0.654665in}}%
\pgfpathlineto{\pgfqpoint{1.666070in}{0.722505in}}%
\pgfpathlineto{\pgfqpoint{1.667800in}{0.731442in}}%
\pgfpathlineto{\pgfqpoint{1.668666in}{0.685142in}}%
\pgfpathlineto{\pgfqpoint{1.669531in}{0.704372in}}%
\pgfpathlineto{\pgfqpoint{1.670393in}{0.692613in}}%
\pgfpathlineto{\pgfqpoint{1.671259in}{0.707669in}}%
\pgfpathlineto{\pgfqpoint{1.672123in}{0.695764in}}%
\pgfpathlineto{\pgfqpoint{1.672989in}{0.732136in}}%
\pgfpathlineto{\pgfqpoint{1.673852in}{0.661698in}}%
\pgfpathlineto{\pgfqpoint{1.675581in}{0.724519in}}%
\pgfpathlineto{\pgfqpoint{1.676445in}{0.739502in}}%
\pgfpathlineto{\pgfqpoint{1.677310in}{0.725985in}}%
\pgfpathlineto{\pgfqpoint{1.678175in}{0.694042in}}%
\pgfpathlineto{\pgfqpoint{1.679039in}{0.708034in}}%
\pgfpathlineto{\pgfqpoint{1.679905in}{0.689170in}}%
\pgfpathlineto{\pgfqpoint{1.681635in}{0.700490in}}%
\pgfpathlineto{\pgfqpoint{1.682499in}{0.749352in}}%
\pgfpathlineto{\pgfqpoint{1.683364in}{0.702943in}}%
\pgfpathlineto{\pgfqpoint{1.684229in}{0.812429in}}%
\pgfpathlineto{\pgfqpoint{1.685959in}{0.707742in}}%
\pgfpathlineto{\pgfqpoint{1.686825in}{0.678987in}}%
\pgfpathlineto{\pgfqpoint{1.687689in}{0.681626in}}%
\pgfpathlineto{\pgfqpoint{1.688554in}{0.678037in}}%
\pgfpathlineto{\pgfqpoint{1.689420in}{0.659172in}}%
\pgfpathlineto{\pgfqpoint{1.690285in}{0.696349in}}%
\pgfpathlineto{\pgfqpoint{1.691149in}{0.682174in}}%
\pgfpathlineto{\pgfqpoint{1.692015in}{0.637450in}}%
\pgfpathlineto{\pgfqpoint{1.693742in}{0.756316in}}%
\pgfpathlineto{\pgfqpoint{1.695472in}{0.683348in}}%
\pgfpathlineto{\pgfqpoint{1.696337in}{0.754740in}}%
\pgfpathlineto{\pgfqpoint{1.697201in}{0.730123in}}%
\pgfpathlineto{\pgfqpoint{1.698064in}{0.756864in}}%
\pgfpathlineto{\pgfqpoint{1.698929in}{0.662136in}}%
\pgfpathlineto{\pgfqpoint{1.699794in}{0.726972in}}%
\pgfpathlineto{\pgfqpoint{1.700660in}{0.653455in}}%
\pgfpathlineto{\pgfqpoint{1.701524in}{0.715286in}}%
\pgfpathlineto{\pgfqpoint{1.702389in}{0.686165in}}%
\pgfpathlineto{\pgfqpoint{1.703252in}{0.697595in}}%
\pgfpathlineto{\pgfqpoint{1.704117in}{0.630563in}}%
\pgfpathlineto{\pgfqpoint{1.705846in}{0.700563in}}%
\pgfpathlineto{\pgfqpoint{1.706709in}{0.651222in}}%
\pgfpathlineto{\pgfqpoint{1.707574in}{0.750124in}}%
\pgfpathlineto{\pgfqpoint{1.709304in}{0.664629in}}%
\pgfpathlineto{\pgfqpoint{1.712763in}{0.783089in}}%
\pgfpathlineto{\pgfqpoint{1.713627in}{0.703089in}}%
\pgfpathlineto{\pgfqpoint{1.714491in}{0.709464in}}%
\pgfpathlineto{\pgfqpoint{1.715355in}{0.695837in}}%
\pgfpathlineto{\pgfqpoint{1.716220in}{0.733639in}}%
\pgfpathlineto{\pgfqpoint{1.717949in}{0.639244in}}%
\pgfpathlineto{\pgfqpoint{1.719680in}{0.706020in}}%
\pgfpathlineto{\pgfqpoint{1.720545in}{0.714409in}}%
\pgfpathlineto{\pgfqpoint{1.721409in}{0.706971in}}%
\pgfpathlineto{\pgfqpoint{1.723138in}{0.630344in}}%
\pgfpathlineto{\pgfqpoint{1.724003in}{0.754005in}}%
\pgfpathlineto{\pgfqpoint{1.724868in}{0.683676in}}%
\pgfpathlineto{\pgfqpoint{1.725732in}{0.787779in}}%
\pgfpathlineto{\pgfqpoint{1.726596in}{0.667594in}}%
\pgfpathlineto{\pgfqpoint{1.728326in}{0.740452in}}%
\pgfpathlineto{\pgfqpoint{1.729191in}{0.678329in}}%
\pgfpathlineto{\pgfqpoint{1.730922in}{0.769353in}}%
\pgfpathlineto{\pgfqpoint{1.731786in}{0.722834in}}%
\pgfpathlineto{\pgfqpoint{1.732650in}{0.794885in}}%
\pgfpathlineto{\pgfqpoint{1.733516in}{0.735653in}}%
\pgfpathlineto{\pgfqpoint{1.734382in}{0.737520in}}%
\pgfpathlineto{\pgfqpoint{1.735246in}{0.771952in}}%
\pgfpathlineto{\pgfqpoint{1.736974in}{0.735872in}}%
\pgfpathlineto{\pgfqpoint{1.737840in}{0.764002in}}%
\pgfpathlineto{\pgfqpoint{1.739570in}{0.683088in}}%
\pgfpathlineto{\pgfqpoint{1.741299in}{0.716127in}}%
\pgfpathlineto{\pgfqpoint{1.743026in}{0.685800in}}%
\pgfpathlineto{\pgfqpoint{1.743891in}{0.725835in}}%
\pgfpathlineto{\pgfqpoint{1.744756in}{0.686019in}}%
\pgfpathlineto{\pgfqpoint{1.745620in}{0.722611in}}%
\pgfpathlineto{\pgfqpoint{1.746486in}{0.683786in}}%
\pgfpathlineto{\pgfqpoint{1.747352in}{0.722465in}}%
\pgfpathlineto{\pgfqpoint{1.748217in}{0.717519in}}%
\pgfpathlineto{\pgfqpoint{1.749082in}{0.732210in}}%
\pgfpathlineto{\pgfqpoint{1.749948in}{0.779829in}}%
\pgfpathlineto{\pgfqpoint{1.750814in}{0.704884in}}%
\pgfpathlineto{\pgfqpoint{1.751679in}{0.711843in}}%
\pgfpathlineto{\pgfqpoint{1.752545in}{0.727045in}}%
\pgfpathlineto{\pgfqpoint{1.753409in}{0.702577in}}%
\pgfpathlineto{\pgfqpoint{1.754272in}{0.737374in}}%
\pgfpathlineto{\pgfqpoint{1.756004in}{0.693677in}}%
\pgfpathlineto{\pgfqpoint{1.756870in}{0.774299in}}%
\pgfpathlineto{\pgfqpoint{1.758602in}{0.690526in}}%
\pgfpathlineto{\pgfqpoint{1.759468in}{0.692394in}}%
\pgfpathlineto{\pgfqpoint{1.761200in}{0.736936in}}%
\pgfpathlineto{\pgfqpoint{1.762066in}{0.726826in}}%
\pgfpathlineto{\pgfqpoint{1.762931in}{0.730305in}}%
\pgfpathlineto{\pgfqpoint{1.763797in}{0.713491in}}%
\pgfpathlineto{\pgfqpoint{1.764663in}{0.716350in}}%
\pgfpathlineto{\pgfqpoint{1.765529in}{0.709537in}}%
\pgfpathlineto{\pgfqpoint{1.766394in}{0.666059in}}%
\pgfpathlineto{\pgfqpoint{1.768988in}{0.766933in}}%
\pgfpathlineto{\pgfqpoint{1.770717in}{0.695066in}}%
\pgfpathlineto{\pgfqpoint{1.771582in}{0.734516in}}%
\pgfpathlineto{\pgfqpoint{1.773309in}{0.658401in}}%
\pgfpathlineto{\pgfqpoint{1.774173in}{0.747338in}}%
\pgfpathlineto{\pgfqpoint{1.775039in}{0.719939in}}%
\pgfpathlineto{\pgfqpoint{1.775903in}{0.728839in}}%
\pgfpathlineto{\pgfqpoint{1.776768in}{0.723017in}}%
\pgfpathlineto{\pgfqpoint{1.777633in}{0.707888in}}%
\pgfpathlineto{\pgfqpoint{1.778496in}{0.726460in}}%
\pgfpathlineto{\pgfqpoint{1.779358in}{0.709317in}}%
\pgfpathlineto{\pgfqpoint{1.780222in}{0.660196in}}%
\pgfpathlineto{\pgfqpoint{1.781951in}{0.721405in}}%
\pgfpathlineto{\pgfqpoint{1.782816in}{0.733127in}}%
\pgfpathlineto{\pgfqpoint{1.783681in}{0.694554in}}%
\pgfpathlineto{\pgfqpoint{1.784545in}{0.707523in}}%
\pgfpathlineto{\pgfqpoint{1.785410in}{0.660342in}}%
\pgfpathlineto{\pgfqpoint{1.788003in}{0.764262in}}%
\pgfpathlineto{\pgfqpoint{1.788868in}{0.626312in}}%
\pgfpathlineto{\pgfqpoint{1.789732in}{0.705176in}}%
\pgfpathlineto{\pgfqpoint{1.790596in}{0.703527in}}%
\pgfpathlineto{\pgfqpoint{1.792326in}{0.780268in}}%
\pgfpathlineto{\pgfqpoint{1.794054in}{0.641697in}}%
\pgfpathlineto{\pgfqpoint{1.794918in}{0.714701in}}%
\pgfpathlineto{\pgfqpoint{1.795782in}{0.685471in}}%
\pgfpathlineto{\pgfqpoint{1.796649in}{0.737082in}}%
\pgfpathlineto{\pgfqpoint{1.797512in}{0.726826in}}%
\pgfpathlineto{\pgfqpoint{1.798377in}{0.672356in}}%
\pgfpathlineto{\pgfqpoint{1.800107in}{0.738219in}}%
\pgfpathlineto{\pgfqpoint{1.800973in}{0.700746in}}%
\pgfpathlineto{\pgfqpoint{1.801837in}{0.724738in}}%
\pgfpathlineto{\pgfqpoint{1.802704in}{0.713970in}}%
\pgfpathlineto{\pgfqpoint{1.803570in}{0.679685in}}%
\pgfpathlineto{\pgfqpoint{1.806165in}{0.762979in}}%
\pgfpathlineto{\pgfqpoint{1.807031in}{0.784848in}}%
\pgfpathlineto{\pgfqpoint{1.807897in}{0.683567in}}%
\pgfpathlineto{\pgfqpoint{1.808763in}{0.776020in}}%
\pgfpathlineto{\pgfqpoint{1.810493in}{0.695874in}}%
\pgfpathlineto{\pgfqpoint{1.811358in}{0.700157in}}%
\pgfpathlineto{\pgfqpoint{1.813954in}{0.746055in}}%
\pgfpathlineto{\pgfqpoint{1.815683in}{0.700304in}}%
\pgfpathlineto{\pgfqpoint{1.816549in}{0.700523in}}%
\pgfpathlineto{\pgfqpoint{1.817413in}{0.749791in}}%
\pgfpathlineto{\pgfqpoint{1.819140in}{0.687814in}}%
\pgfpathlineto{\pgfqpoint{1.820870in}{0.744187in}}%
\pgfpathlineto{\pgfqpoint{1.821736in}{0.680562in}}%
\pgfpathlineto{\pgfqpoint{1.822603in}{0.817265in}}%
\pgfpathlineto{\pgfqpoint{1.823469in}{0.697413in}}%
\pgfpathlineto{\pgfqpoint{1.825198in}{0.781258in}}%
\pgfpathlineto{\pgfqpoint{1.826928in}{0.759170in}}%
\pgfpathlineto{\pgfqpoint{1.827794in}{0.763823in}}%
\pgfpathlineto{\pgfqpoint{1.828657in}{0.697120in}}%
\pgfpathlineto{\pgfqpoint{1.830388in}{0.765837in}}%
\pgfpathlineto{\pgfqpoint{1.831253in}{0.703089in}}%
\pgfpathlineto{\pgfqpoint{1.832979in}{0.790524in}}%
\pgfpathlineto{\pgfqpoint{1.834712in}{0.686677in}}%
\pgfpathlineto{\pgfqpoint{1.835577in}{0.741695in}}%
\pgfpathlineto{\pgfqpoint{1.837307in}{0.676494in}}%
\pgfpathlineto{\pgfqpoint{1.838171in}{0.736603in}}%
\pgfpathlineto{\pgfqpoint{1.839035in}{0.731182in}}%
\pgfpathlineto{\pgfqpoint{1.839899in}{0.755102in}}%
\pgfpathlineto{\pgfqpoint{1.840763in}{0.749937in}}%
\pgfpathlineto{\pgfqpoint{1.842494in}{0.693637in}}%
\pgfpathlineto{\pgfqpoint{1.843359in}{0.756275in}}%
\pgfpathlineto{\pgfqpoint{1.844224in}{0.753234in}}%
\pgfpathlineto{\pgfqpoint{1.845089in}{0.787008in}}%
\pgfpathlineto{\pgfqpoint{1.845955in}{0.751037in}}%
\pgfpathlineto{\pgfqpoint{1.846820in}{0.763417in}}%
\pgfpathlineto{\pgfqpoint{1.849412in}{0.708619in}}%
\pgfpathlineto{\pgfqpoint{1.850276in}{0.715286in}}%
\pgfpathlineto{\pgfqpoint{1.851141in}{0.676754in}}%
\pgfpathlineto{\pgfqpoint{1.852007in}{0.684371in}}%
\pgfpathlineto{\pgfqpoint{1.852870in}{0.725652in}}%
\pgfpathlineto{\pgfqpoint{1.853735in}{0.689170in}}%
\pgfpathlineto{\pgfqpoint{1.854601in}{0.770011in}}%
\pgfpathlineto{\pgfqpoint{1.855465in}{0.744736in}}%
\pgfpathlineto{\pgfqpoint{1.856331in}{0.780121in}}%
\pgfpathlineto{\pgfqpoint{1.857196in}{0.642687in}}%
\pgfpathlineto{\pgfqpoint{1.858062in}{0.670342in}}%
\pgfpathlineto{\pgfqpoint{1.858925in}{0.719574in}}%
\pgfpathlineto{\pgfqpoint{1.859790in}{0.691403in}}%
\pgfpathlineto{\pgfqpoint{1.861521in}{0.761184in}}%
\pgfpathlineto{\pgfqpoint{1.862388in}{0.707779in}}%
\pgfpathlineto{\pgfqpoint{1.864982in}{0.755544in}}%
\pgfpathlineto{\pgfqpoint{1.865845in}{0.765399in}}%
\pgfpathlineto{\pgfqpoint{1.867573in}{0.688549in}}%
\pgfpathlineto{\pgfqpoint{1.868440in}{0.757010in}}%
\pgfpathlineto{\pgfqpoint{1.871032in}{0.642468in}}%
\pgfpathlineto{\pgfqpoint{1.871897in}{0.742908in}}%
\pgfpathlineto{\pgfqpoint{1.872762in}{0.681845in}}%
\pgfpathlineto{\pgfqpoint{1.873626in}{0.710600in}}%
\pgfpathlineto{\pgfqpoint{1.874493in}{0.632431in}}%
\pgfpathlineto{\pgfqpoint{1.876224in}{0.680781in}}%
\pgfpathlineto{\pgfqpoint{1.877089in}{0.678512in}}%
\pgfpathlineto{\pgfqpoint{1.877953in}{0.658401in}}%
\pgfpathlineto{\pgfqpoint{1.878816in}{0.703016in}}%
\pgfpathlineto{\pgfqpoint{1.879681in}{0.676534in}}%
\pgfpathlineto{\pgfqpoint{1.882278in}{0.776459in}}%
\pgfpathlineto{\pgfqpoint{1.884008in}{0.741954in}}%
\pgfpathlineto{\pgfqpoint{1.884871in}{0.777815in}}%
\pgfpathlineto{\pgfqpoint{1.885735in}{0.774664in}}%
\pgfpathlineto{\pgfqpoint{1.886600in}{0.774518in}}%
\pgfpathlineto{\pgfqpoint{1.887463in}{0.838728in}}%
\pgfpathlineto{\pgfqpoint{1.888327in}{0.769020in}}%
\pgfpathlineto{\pgfqpoint{1.889190in}{0.777336in}}%
\pgfpathlineto{\pgfqpoint{1.892650in}{0.694335in}}%
\pgfpathlineto{\pgfqpoint{1.893515in}{0.682138in}}%
\pgfpathlineto{\pgfqpoint{1.894381in}{0.762540in}}%
\pgfpathlineto{\pgfqpoint{1.895246in}{0.659392in}}%
\pgfpathlineto{\pgfqpoint{1.896975in}{0.721368in}}%
\pgfpathlineto{\pgfqpoint{1.897841in}{0.720451in}}%
\pgfpathlineto{\pgfqpoint{1.898707in}{0.711697in}}%
\pgfpathlineto{\pgfqpoint{1.899572in}{0.790524in}}%
\pgfpathlineto{\pgfqpoint{1.901303in}{0.668438in}}%
\pgfpathlineto{\pgfqpoint{1.902169in}{0.747265in}}%
\pgfpathlineto{\pgfqpoint{1.903034in}{0.740452in}}%
\pgfpathlineto{\pgfqpoint{1.903899in}{0.632285in}}%
\pgfpathlineto{\pgfqpoint{1.905629in}{0.783199in}}%
\pgfpathlineto{\pgfqpoint{1.906494in}{0.716642in}}%
\pgfpathlineto{\pgfqpoint{1.907360in}{0.791003in}}%
\pgfpathlineto{\pgfqpoint{1.908225in}{0.690526in}}%
\pgfpathlineto{\pgfqpoint{1.909089in}{0.739721in}}%
\pgfpathlineto{\pgfqpoint{1.909953in}{0.715948in}}%
\pgfpathlineto{\pgfqpoint{1.910818in}{0.649354in}}%
\pgfpathlineto{\pgfqpoint{1.911684in}{0.783824in}}%
\pgfpathlineto{\pgfqpoint{1.912550in}{0.700344in}}%
\pgfpathlineto{\pgfqpoint{1.913415in}{0.774226in}}%
\pgfpathlineto{\pgfqpoint{1.916011in}{0.673384in}}%
\pgfpathlineto{\pgfqpoint{1.916877in}{0.737561in}}%
\pgfpathlineto{\pgfqpoint{1.917742in}{0.733127in}}%
\pgfpathlineto{\pgfqpoint{1.918607in}{0.726095in}}%
\pgfpathlineto{\pgfqpoint{1.919471in}{0.729867in}}%
\pgfpathlineto{\pgfqpoint{1.920335in}{0.676534in}}%
\pgfpathlineto{\pgfqpoint{1.921200in}{0.719574in}}%
\pgfpathlineto{\pgfqpoint{1.922066in}{0.699573in}}%
\pgfpathlineto{\pgfqpoint{1.922931in}{0.728328in}}%
\pgfpathlineto{\pgfqpoint{1.923796in}{0.701554in}}%
\pgfpathlineto{\pgfqpoint{1.924661in}{0.710783in}}%
\pgfpathlineto{\pgfqpoint{1.926393in}{0.662981in}}%
\pgfpathlineto{\pgfqpoint{1.927259in}{0.671881in}}%
\pgfpathlineto{\pgfqpoint{1.928122in}{0.732908in}}%
\pgfpathlineto{\pgfqpoint{1.928985in}{0.684594in}}%
\pgfpathlineto{\pgfqpoint{1.929850in}{0.704006in}}%
\pgfpathlineto{\pgfqpoint{1.930715in}{0.774006in}}%
\pgfpathlineto{\pgfqpoint{1.932446in}{0.701075in}}%
\pgfpathlineto{\pgfqpoint{1.933312in}{0.733931in}}%
\pgfpathlineto{\pgfqpoint{1.934179in}{0.697778in}}%
\pgfpathlineto{\pgfqpoint{1.935043in}{0.733493in}}%
\pgfpathlineto{\pgfqpoint{1.936775in}{0.697559in}}%
\pgfpathlineto{\pgfqpoint{1.938507in}{0.754992in}}%
\pgfpathlineto{\pgfqpoint{1.939373in}{0.754517in}}%
\pgfpathlineto{\pgfqpoint{1.941103in}{0.712318in}}%
\pgfpathlineto{\pgfqpoint{1.941967in}{0.727703in}}%
\pgfpathlineto{\pgfqpoint{1.943696in}{0.712428in}}%
\pgfpathlineto{\pgfqpoint{1.945425in}{0.745357in}}%
\pgfpathlineto{\pgfqpoint{1.946290in}{0.697153in}}%
\pgfpathlineto{\pgfqpoint{1.947154in}{0.734808in}}%
\pgfpathlineto{\pgfqpoint{1.948018in}{0.685211in}}%
\pgfpathlineto{\pgfqpoint{1.948883in}{0.698363in}}%
\pgfpathlineto{\pgfqpoint{1.949748in}{0.736603in}}%
\pgfpathlineto{\pgfqpoint{1.950610in}{0.728068in}}%
\pgfpathlineto{\pgfqpoint{1.951475in}{0.722684in}}%
\pgfpathlineto{\pgfqpoint{1.952341in}{0.707190in}}%
\pgfpathlineto{\pgfqpoint{1.953207in}{0.658620in}}%
\pgfpathlineto{\pgfqpoint{1.954936in}{0.720816in}}%
\pgfpathlineto{\pgfqpoint{1.955801in}{0.723529in}}%
\pgfpathlineto{\pgfqpoint{1.956666in}{0.755175in}}%
\pgfpathlineto{\pgfqpoint{1.957531in}{0.707555in}}%
\pgfpathlineto{\pgfqpoint{1.958396in}{0.784807in}}%
\pgfpathlineto{\pgfqpoint{1.961857in}{0.699865in}}%
\pgfpathlineto{\pgfqpoint{1.963588in}{0.742027in}}%
\pgfpathlineto{\pgfqpoint{1.964454in}{0.805433in}}%
\pgfpathlineto{\pgfqpoint{1.965319in}{0.735141in}}%
\pgfpathlineto{\pgfqpoint{1.966183in}{0.738146in}}%
\pgfpathlineto{\pgfqpoint{1.967048in}{0.697705in}}%
\pgfpathlineto{\pgfqpoint{1.967912in}{0.748694in}}%
\pgfpathlineto{\pgfqpoint{1.968778in}{0.694335in}}%
\pgfpathlineto{\pgfqpoint{1.969644in}{0.744443in}}%
\pgfpathlineto{\pgfqpoint{1.970509in}{0.738657in}}%
\pgfpathlineto{\pgfqpoint{1.971375in}{0.725323in}}%
\pgfpathlineto{\pgfqpoint{1.972240in}{0.766531in}}%
\pgfpathlineto{\pgfqpoint{1.973972in}{0.689755in}}%
\pgfpathlineto{\pgfqpoint{1.975702in}{0.737886in}}%
\pgfpathlineto{\pgfqpoint{1.976567in}{0.727081in}}%
\pgfpathlineto{\pgfqpoint{1.978298in}{0.699426in}}%
\pgfpathlineto{\pgfqpoint{1.980028in}{0.783491in}}%
\pgfpathlineto{\pgfqpoint{1.980894in}{0.677631in}}%
\pgfpathlineto{\pgfqpoint{1.981760in}{0.720597in}}%
\pgfpathlineto{\pgfqpoint{1.982625in}{0.706751in}}%
\pgfpathlineto{\pgfqpoint{1.983491in}{0.765066in}}%
\pgfpathlineto{\pgfqpoint{1.986949in}{0.629500in}}%
\pgfpathlineto{\pgfqpoint{1.987814in}{0.730488in}}%
\pgfpathlineto{\pgfqpoint{1.988674in}{0.716898in}}%
\pgfpathlineto{\pgfqpoint{1.989540in}{0.793821in}}%
\pgfpathlineto{\pgfqpoint{1.990405in}{0.717300in}}%
\pgfpathlineto{\pgfqpoint{1.992132in}{0.785067in}}%
\pgfpathlineto{\pgfqpoint{1.992997in}{0.756385in}}%
\pgfpathlineto{\pgfqpoint{1.993859in}{0.760193in}}%
\pgfpathlineto{\pgfqpoint{1.996456in}{0.693271in}}%
\pgfpathlineto{\pgfqpoint{1.997321in}{0.727118in}}%
\pgfpathlineto{\pgfqpoint{1.998186in}{0.686239in}}%
\pgfpathlineto{\pgfqpoint{1.999051in}{0.700669in}}%
\pgfpathlineto{\pgfqpoint{1.999915in}{0.760705in}}%
\pgfpathlineto{\pgfqpoint{2.001642in}{0.717885in}}%
\pgfpathlineto{\pgfqpoint{2.002508in}{0.766235in}}%
\pgfpathlineto{\pgfqpoint{2.003373in}{0.714295in}}%
\pgfpathlineto{\pgfqpoint{2.004238in}{0.730155in}}%
\pgfpathlineto{\pgfqpoint{2.005103in}{0.678325in}}%
\pgfpathlineto{\pgfqpoint{2.006833in}{0.728986in}}%
\pgfpathlineto{\pgfqpoint{2.007697in}{0.699719in}}%
\pgfpathlineto{\pgfqpoint{2.008563in}{0.757668in}}%
\pgfpathlineto{\pgfqpoint{2.009429in}{0.738032in}}%
\pgfpathlineto{\pgfqpoint{2.011158in}{0.758252in}}%
\pgfpathlineto{\pgfqpoint{2.012023in}{0.720962in}}%
\pgfpathlineto{\pgfqpoint{2.013752in}{0.785944in}}%
\pgfpathlineto{\pgfqpoint{2.014617in}{0.675138in}}%
\pgfpathlineto{\pgfqpoint{2.015482in}{0.704039in}}%
\pgfpathlineto{\pgfqpoint{2.016347in}{0.737740in}}%
\pgfpathlineto{\pgfqpoint{2.017211in}{0.700669in}}%
\pgfpathlineto{\pgfqpoint{2.018077in}{0.722757in}}%
\pgfpathlineto{\pgfqpoint{2.019804in}{0.633235in}}%
\pgfpathlineto{\pgfqpoint{2.020669in}{0.785652in}}%
\pgfpathlineto{\pgfqpoint{2.022400in}{0.676201in}}%
\pgfpathlineto{\pgfqpoint{2.024131in}{0.789574in}}%
\pgfpathlineto{\pgfqpoint{2.024995in}{0.756531in}}%
\pgfpathlineto{\pgfqpoint{2.026727in}{0.700267in}}%
\pgfpathlineto{\pgfqpoint{2.028455in}{0.734808in}}%
\pgfpathlineto{\pgfqpoint{2.029320in}{0.695248in}}%
\pgfpathlineto{\pgfqpoint{2.031052in}{0.766568in}}%
\pgfpathlineto{\pgfqpoint{2.031917in}{0.696203in}}%
\pgfpathlineto{\pgfqpoint{2.032782in}{0.711843in}}%
\pgfpathlineto{\pgfqpoint{2.033646in}{0.715067in}}%
\pgfpathlineto{\pgfqpoint{2.034510in}{0.730415in}}%
\pgfpathlineto{\pgfqpoint{2.035374in}{0.832979in}}%
\pgfpathlineto{\pgfqpoint{2.037104in}{0.754663in}}%
\pgfpathlineto{\pgfqpoint{2.037969in}{0.749279in}}%
\pgfpathlineto{\pgfqpoint{2.038834in}{0.658657in}}%
\pgfpathlineto{\pgfqpoint{2.039700in}{0.775030in}}%
\pgfpathlineto{\pgfqpoint{2.040565in}{0.694189in}}%
\pgfpathlineto{\pgfqpoint{2.041430in}{0.702870in}}%
\pgfpathlineto{\pgfqpoint{2.042296in}{0.704737in}}%
\pgfpathlineto{\pgfqpoint{2.043162in}{0.712468in}}%
\pgfpathlineto{\pgfqpoint{2.044026in}{0.787998in}}%
\pgfpathlineto{\pgfqpoint{2.044891in}{0.787487in}}%
\pgfpathlineto{\pgfqpoint{2.045756in}{0.806461in}}%
\pgfpathlineto{\pgfqpoint{2.048349in}{0.703933in}}%
\pgfpathlineto{\pgfqpoint{2.049214in}{0.818329in}}%
\pgfpathlineto{\pgfqpoint{2.050080in}{0.762321in}}%
\pgfpathlineto{\pgfqpoint{2.051810in}{0.803273in}}%
\pgfpathlineto{\pgfqpoint{2.052674in}{0.736022in}}%
\pgfpathlineto{\pgfqpoint{2.054402in}{0.801552in}}%
\pgfpathlineto{\pgfqpoint{2.055268in}{0.716460in}}%
\pgfpathlineto{\pgfqpoint{2.056131in}{0.729976in}}%
\pgfpathlineto{\pgfqpoint{2.056996in}{0.800488in}}%
\pgfpathlineto{\pgfqpoint{2.057861in}{0.770669in}}%
\pgfpathlineto{\pgfqpoint{2.058726in}{0.686239in}}%
\pgfpathlineto{\pgfqpoint{2.059591in}{0.760965in}}%
\pgfpathlineto{\pgfqpoint{2.060456in}{0.744480in}}%
\pgfpathlineto{\pgfqpoint{2.061322in}{0.739461in}}%
\pgfpathlineto{\pgfqpoint{2.063052in}{0.667667in}}%
\pgfpathlineto{\pgfqpoint{2.063916in}{0.684444in}}%
\pgfpathlineto{\pgfqpoint{2.064780in}{0.767595in}}%
\pgfpathlineto{\pgfqpoint{2.065646in}{0.755508in}}%
\pgfpathlineto{\pgfqpoint{2.067377in}{0.711441in}}%
\pgfpathlineto{\pgfqpoint{2.068243in}{0.762686in}}%
\pgfpathlineto{\pgfqpoint{2.070838in}{0.644482in}}%
\pgfpathlineto{\pgfqpoint{2.073429in}{0.735580in}}%
\pgfpathlineto{\pgfqpoint{2.074294in}{0.744809in}}%
\pgfpathlineto{\pgfqpoint{2.075160in}{0.724406in}}%
\pgfpathlineto{\pgfqpoint{2.076025in}{0.739023in}}%
\pgfpathlineto{\pgfqpoint{2.076890in}{0.681183in}}%
\pgfpathlineto{\pgfqpoint{2.080348in}{0.739023in}}%
\pgfpathlineto{\pgfqpoint{2.081213in}{0.754298in}}%
\pgfpathlineto{\pgfqpoint{2.082078in}{0.690745in}}%
\pgfpathlineto{\pgfqpoint{2.083807in}{0.747229in}}%
\pgfpathlineto{\pgfqpoint{2.085536in}{0.691549in}}%
\pgfpathlineto{\pgfqpoint{2.086401in}{0.752686in}}%
\pgfpathlineto{\pgfqpoint{2.087265in}{0.721880in}}%
\pgfpathlineto{\pgfqpoint{2.089859in}{0.768582in}}%
\pgfpathlineto{\pgfqpoint{2.090724in}{0.757485in}}%
\pgfpathlineto{\pgfqpoint{2.091589in}{0.697299in}}%
\pgfpathlineto{\pgfqpoint{2.094184in}{0.754590in}}%
\pgfpathlineto{\pgfqpoint{2.095914in}{0.704185in}}%
\pgfpathlineto{\pgfqpoint{2.096780in}{0.803639in}}%
\pgfpathlineto{\pgfqpoint{2.099373in}{0.724519in}}%
\pgfpathlineto{\pgfqpoint{2.100239in}{0.764335in}}%
\pgfpathlineto{\pgfqpoint{2.101101in}{0.762321in}}%
\pgfpathlineto{\pgfqpoint{2.101963in}{0.760819in}}%
\pgfpathlineto{\pgfqpoint{2.102828in}{0.716131in}}%
\pgfpathlineto{\pgfqpoint{2.103692in}{0.782761in}}%
\pgfpathlineto{\pgfqpoint{2.104558in}{0.765837in}}%
\pgfpathlineto{\pgfqpoint{2.107151in}{0.699938in}}%
\pgfpathlineto{\pgfqpoint{2.108016in}{0.718510in}}%
\pgfpathlineto{\pgfqpoint{2.108880in}{0.795250in}}%
\pgfpathlineto{\pgfqpoint{2.109746in}{0.790670in}}%
\pgfpathlineto{\pgfqpoint{2.110611in}{0.729278in}}%
\pgfpathlineto{\pgfqpoint{2.111477in}{0.749864in}}%
\pgfpathlineto{\pgfqpoint{2.112342in}{0.689426in}}%
\pgfpathlineto{\pgfqpoint{2.114071in}{0.753786in}}%
\pgfpathlineto{\pgfqpoint{2.114938in}{0.677521in}}%
\pgfpathlineto{\pgfqpoint{2.115804in}{0.755508in}}%
\pgfpathlineto{\pgfqpoint{2.116670in}{0.691111in}}%
\pgfpathlineto{\pgfqpoint{2.117537in}{0.702577in}}%
\pgfpathlineto{\pgfqpoint{2.120134in}{0.820562in}}%
\pgfpathlineto{\pgfqpoint{2.121865in}{0.775947in}}%
\pgfpathlineto{\pgfqpoint{2.122731in}{0.764189in}}%
\pgfpathlineto{\pgfqpoint{2.124463in}{0.681260in}}%
\pgfpathlineto{\pgfqpoint{2.125328in}{0.710162in}}%
\pgfpathlineto{\pgfqpoint{2.126193in}{0.743164in}}%
\pgfpathlineto{\pgfqpoint{2.127924in}{0.686060in}}%
\pgfpathlineto{\pgfqpoint{2.128790in}{0.744740in}}%
\pgfpathlineto{\pgfqpoint{2.129654in}{0.710820in}}%
\pgfpathlineto{\pgfqpoint{2.130520in}{0.721588in}}%
\pgfpathlineto{\pgfqpoint{2.131385in}{0.663529in}}%
\pgfpathlineto{\pgfqpoint{2.132251in}{0.820197in}}%
\pgfpathlineto{\pgfqpoint{2.133116in}{0.812027in}}%
\pgfpathlineto{\pgfqpoint{2.134847in}{0.769061in}}%
\pgfpathlineto{\pgfqpoint{2.135713in}{0.691517in}}%
\pgfpathlineto{\pgfqpoint{2.137440in}{0.770783in}}%
\pgfpathlineto{\pgfqpoint{2.138303in}{0.731442in}}%
\pgfpathlineto{\pgfqpoint{2.139169in}{0.798255in}}%
\pgfpathlineto{\pgfqpoint{2.140031in}{0.724885in}}%
\pgfpathlineto{\pgfqpoint{2.140897in}{0.749096in}}%
\pgfpathlineto{\pgfqpoint{2.141763in}{0.694481in}}%
\pgfpathlineto{\pgfqpoint{2.142629in}{0.742100in}}%
\pgfpathlineto{\pgfqpoint{2.143495in}{0.711002in}}%
\pgfpathlineto{\pgfqpoint{2.144360in}{0.733346in}}%
\pgfpathlineto{\pgfqpoint{2.145225in}{0.711770in}}%
\pgfpathlineto{\pgfqpoint{2.146091in}{0.742612in}}%
\pgfpathlineto{\pgfqpoint{2.146956in}{0.709829in}}%
\pgfpathlineto{\pgfqpoint{2.147822in}{0.723748in}}%
\pgfpathlineto{\pgfqpoint{2.148687in}{0.696974in}}%
\pgfpathlineto{\pgfqpoint{2.149554in}{0.747484in}}%
\pgfpathlineto{\pgfqpoint{2.150418in}{0.743822in}}%
\pgfpathlineto{\pgfqpoint{2.151284in}{0.766495in}}%
\pgfpathlineto{\pgfqpoint{2.152149in}{0.736237in}}%
\pgfpathlineto{\pgfqpoint{2.153015in}{0.650414in}}%
\pgfpathlineto{\pgfqpoint{2.153879in}{0.776897in}}%
\pgfpathlineto{\pgfqpoint{2.154743in}{0.746275in}}%
\pgfpathlineto{\pgfqpoint{2.155607in}{0.699426in}}%
\pgfpathlineto{\pgfqpoint{2.156472in}{0.709975in}}%
\pgfpathlineto{\pgfqpoint{2.157337in}{0.840965in}}%
\pgfpathlineto{\pgfqpoint{2.159066in}{0.705436in}}%
\pgfpathlineto{\pgfqpoint{2.159928in}{0.724410in}}%
\pgfpathlineto{\pgfqpoint{2.160794in}{0.698842in}}%
\pgfpathlineto{\pgfqpoint{2.161658in}{0.745584in}}%
\pgfpathlineto{\pgfqpoint{2.162520in}{0.742580in}}%
\pgfpathlineto{\pgfqpoint{2.163384in}{0.673237in}}%
\pgfpathlineto{\pgfqpoint{2.164251in}{0.830307in}}%
\pgfpathlineto{\pgfqpoint{2.165115in}{0.738073in}}%
\pgfpathlineto{\pgfqpoint{2.165980in}{0.754078in}}%
\pgfpathlineto{\pgfqpoint{2.166844in}{0.735360in}}%
\pgfpathlineto{\pgfqpoint{2.167707in}{0.673530in}}%
\pgfpathlineto{\pgfqpoint{2.169437in}{0.806059in}}%
\pgfpathlineto{\pgfqpoint{2.171166in}{0.651076in}}%
\pgfpathlineto{\pgfqpoint{2.172029in}{0.709866in}}%
\pgfpathlineto{\pgfqpoint{2.172893in}{0.731771in}}%
\pgfpathlineto{\pgfqpoint{2.174620in}{0.668073in}}%
\pgfpathlineto{\pgfqpoint{2.175484in}{0.683201in}}%
\pgfpathlineto{\pgfqpoint{2.176349in}{0.739356in}}%
\pgfpathlineto{\pgfqpoint{2.178080in}{0.691809in}}%
\pgfpathlineto{\pgfqpoint{2.178945in}{0.712614in}}%
\pgfpathlineto{\pgfqpoint{2.179810in}{0.633202in}}%
\pgfpathlineto{\pgfqpoint{2.180676in}{0.652140in}}%
\pgfpathlineto{\pgfqpoint{2.181542in}{0.731588in}}%
\pgfpathlineto{\pgfqpoint{2.182405in}{0.721661in}}%
\pgfpathlineto{\pgfqpoint{2.183270in}{0.739940in}}%
\pgfpathlineto{\pgfqpoint{2.184136in}{0.709244in}}%
\pgfpathlineto{\pgfqpoint{2.185000in}{0.739940in}}%
\pgfpathlineto{\pgfqpoint{2.185866in}{0.737963in}}%
\pgfpathlineto{\pgfqpoint{2.186731in}{0.752503in}}%
\pgfpathlineto{\pgfqpoint{2.187597in}{0.668073in}}%
\pgfpathlineto{\pgfqpoint{2.189326in}{0.859793in}}%
\pgfpathlineto{\pgfqpoint{2.191056in}{0.733237in}}%
\pgfpathlineto{\pgfqpoint{2.192788in}{0.767339in}}%
\pgfpathlineto{\pgfqpoint{2.195384in}{0.717048in}}%
\pgfpathlineto{\pgfqpoint{2.196249in}{0.744082in}}%
\pgfpathlineto{\pgfqpoint{2.197115in}{0.712103in}}%
\pgfpathlineto{\pgfqpoint{2.197982in}{0.752909in}}%
\pgfpathlineto{\pgfqpoint{2.198846in}{0.686571in}}%
\pgfpathlineto{\pgfqpoint{2.199710in}{0.755069in}}%
\pgfpathlineto{\pgfqpoint{2.200575in}{0.665876in}}%
\pgfpathlineto{\pgfqpoint{2.202305in}{0.760161in}}%
\pgfpathlineto{\pgfqpoint{2.204035in}{0.674886in}}%
\pgfpathlineto{\pgfqpoint{2.204900in}{0.740598in}}%
\pgfpathlineto{\pgfqpoint{2.205766in}{0.727743in}}%
\pgfpathlineto{\pgfqpoint{2.206632in}{0.767486in}}%
\pgfpathlineto{\pgfqpoint{2.207495in}{0.764335in}}%
\pgfpathlineto{\pgfqpoint{2.208361in}{0.753713in}}%
\pgfpathlineto{\pgfqpoint{2.209225in}{0.690599in}}%
\pgfpathlineto{\pgfqpoint{2.210090in}{0.693750in}}%
\pgfpathlineto{\pgfqpoint{2.210954in}{0.788437in}}%
\pgfpathlineto{\pgfqpoint{2.211819in}{0.748329in}}%
\pgfpathlineto{\pgfqpoint{2.212685in}{0.771919in}}%
\pgfpathlineto{\pgfqpoint{2.213551in}{0.763823in}}%
\pgfpathlineto{\pgfqpoint{2.215281in}{0.733200in}}%
\pgfpathlineto{\pgfqpoint{2.216146in}{0.743895in}}%
\pgfpathlineto{\pgfqpoint{2.217875in}{0.688732in}}%
\pgfpathlineto{\pgfqpoint{2.218741in}{0.729319in}}%
\pgfpathlineto{\pgfqpoint{2.219607in}{0.681882in}}%
\pgfpathlineto{\pgfqpoint{2.220471in}{0.755142in}}%
\pgfpathlineto{\pgfqpoint{2.221337in}{0.696755in}}%
\pgfpathlineto{\pgfqpoint{2.222202in}{0.734410in}}%
\pgfpathlineto{\pgfqpoint{2.223931in}{0.679100in}}%
\pgfpathlineto{\pgfqpoint{2.224796in}{0.703714in}}%
\pgfpathlineto{\pgfqpoint{2.225661in}{0.763750in}}%
\pgfpathlineto{\pgfqpoint{2.226526in}{0.712505in}}%
\pgfpathlineto{\pgfqpoint{2.227389in}{0.764887in}}%
\pgfpathlineto{\pgfqpoint{2.228254in}{0.731186in}}%
\pgfpathlineto{\pgfqpoint{2.229120in}{0.732835in}}%
\pgfpathlineto{\pgfqpoint{2.229986in}{0.789720in}}%
\pgfpathlineto{\pgfqpoint{2.232580in}{0.727524in}}%
\pgfpathlineto{\pgfqpoint{2.234309in}{0.774518in}}%
\pgfpathlineto{\pgfqpoint{2.235174in}{0.783126in}}%
\pgfpathlineto{\pgfqpoint{2.236039in}{0.672137in}}%
\pgfpathlineto{\pgfqpoint{2.236904in}{0.728255in}}%
\pgfpathlineto{\pgfqpoint{2.237768in}{0.717925in}}%
\pgfpathlineto{\pgfqpoint{2.238633in}{0.670379in}}%
\pgfpathlineto{\pgfqpoint{2.239497in}{0.691330in}}%
\pgfpathlineto{\pgfqpoint{2.240361in}{0.667886in}}%
\pgfpathlineto{\pgfqpoint{2.242090in}{0.752576in}}%
\pgfpathlineto{\pgfqpoint{2.242955in}{0.747192in}}%
\pgfpathlineto{\pgfqpoint{2.244685in}{0.714921in}}%
\pgfpathlineto{\pgfqpoint{2.245550in}{0.757193in}}%
\pgfpathlineto{\pgfqpoint{2.246416in}{0.718437in}}%
\pgfpathlineto{\pgfqpoint{2.247281in}{0.734297in}}%
\pgfpathlineto{\pgfqpoint{2.248147in}{0.658072in}}%
\pgfpathlineto{\pgfqpoint{2.249874in}{0.769500in}}%
\pgfpathlineto{\pgfqpoint{2.251606in}{0.657889in}}%
\pgfpathlineto{\pgfqpoint{2.253336in}{0.727962in}}%
\pgfpathlineto{\pgfqpoint{2.254201in}{0.710600in}}%
\pgfpathlineto{\pgfqpoint{2.255930in}{0.733054in}}%
\pgfpathlineto{\pgfqpoint{2.257658in}{0.667963in}}%
\pgfpathlineto{\pgfqpoint{2.258523in}{0.766203in}}%
\pgfpathlineto{\pgfqpoint{2.259389in}{0.683567in}}%
\pgfpathlineto{\pgfqpoint{2.260254in}{0.693969in}}%
\pgfpathlineto{\pgfqpoint{2.261118in}{0.803493in}}%
\pgfpathlineto{\pgfqpoint{2.262848in}{0.730821in}}%
\pgfpathlineto{\pgfqpoint{2.263712in}{0.773787in}}%
\pgfpathlineto{\pgfqpoint{2.264577in}{0.730711in}}%
\pgfpathlineto{\pgfqpoint{2.265441in}{0.750676in}}%
\pgfpathlineto{\pgfqpoint{2.266306in}{0.711591in}}%
\pgfpathlineto{\pgfqpoint{2.267171in}{0.712399in}}%
\pgfpathlineto{\pgfqpoint{2.268898in}{0.720345in}}%
\pgfpathlineto{\pgfqpoint{2.269763in}{0.677525in}}%
\pgfpathlineto{\pgfqpoint{2.271490in}{0.769792in}}%
\pgfpathlineto{\pgfqpoint{2.272353in}{0.762321in}}%
\pgfpathlineto{\pgfqpoint{2.273219in}{0.779683in}}%
\pgfpathlineto{\pgfqpoint{2.274084in}{0.770015in}}%
\pgfpathlineto{\pgfqpoint{2.274949in}{0.802689in}}%
\pgfpathlineto{\pgfqpoint{2.275812in}{0.715619in}}%
\pgfpathlineto{\pgfqpoint{2.276676in}{0.735949in}}%
\pgfpathlineto{\pgfqpoint{2.277541in}{0.743643in}}%
\pgfpathlineto{\pgfqpoint{2.280137in}{0.664191in}}%
\pgfpathlineto{\pgfqpoint{2.281003in}{0.773568in}}%
\pgfpathlineto{\pgfqpoint{2.281868in}{0.652651in}}%
\pgfpathlineto{\pgfqpoint{2.283598in}{0.714446in}}%
\pgfpathlineto{\pgfqpoint{2.284462in}{0.717998in}}%
\pgfpathlineto{\pgfqpoint{2.286192in}{0.664629in}}%
\pgfpathlineto{\pgfqpoint{2.287056in}{0.671662in}}%
\pgfpathlineto{\pgfqpoint{2.287921in}{0.670013in}}%
\pgfpathlineto{\pgfqpoint{2.290518in}{0.750233in}}%
\pgfpathlineto{\pgfqpoint{2.292249in}{0.625983in}}%
\pgfpathlineto{\pgfqpoint{2.293114in}{0.734849in}}%
\pgfpathlineto{\pgfqpoint{2.293978in}{0.702650in}}%
\pgfpathlineto{\pgfqpoint{2.294842in}{0.746607in}}%
\pgfpathlineto{\pgfqpoint{2.296571in}{0.656095in}}%
\pgfpathlineto{\pgfqpoint{2.297435in}{0.777011in}}%
\pgfpathlineto{\pgfqpoint{2.298302in}{0.676900in}}%
\pgfpathlineto{\pgfqpoint{2.299167in}{0.794227in}}%
\pgfpathlineto{\pgfqpoint{2.300033in}{0.722542in}}%
\pgfpathlineto{\pgfqpoint{2.300900in}{0.779171in}}%
\pgfpathlineto{\pgfqpoint{2.301766in}{0.766129in}}%
\pgfpathlineto{\pgfqpoint{2.302632in}{0.712834in}}%
\pgfpathlineto{\pgfqpoint{2.303497in}{0.716277in}}%
\pgfpathlineto{\pgfqpoint{2.304361in}{0.735580in}}%
\pgfpathlineto{\pgfqpoint{2.305227in}{0.697522in}}%
\pgfpathlineto{\pgfqpoint{2.306092in}{0.729684in}}%
\pgfpathlineto{\pgfqpoint{2.306957in}{0.673895in}}%
\pgfpathlineto{\pgfqpoint{2.308688in}{0.733493in}}%
\pgfpathlineto{\pgfqpoint{2.310420in}{0.632650in}}%
\pgfpathlineto{\pgfqpoint{2.311285in}{0.741370in}}%
\pgfpathlineto{\pgfqpoint{2.312152in}{0.739685in}}%
\pgfpathlineto{\pgfqpoint{2.313017in}{0.690088in}}%
\pgfpathlineto{\pgfqpoint{2.313883in}{0.798401in}}%
\pgfpathlineto{\pgfqpoint{2.315613in}{0.681553in}}%
\pgfpathlineto{\pgfqpoint{2.316478in}{0.707230in}}%
\pgfpathlineto{\pgfqpoint{2.317342in}{0.736278in}}%
\pgfpathlineto{\pgfqpoint{2.318206in}{0.707669in}}%
\pgfpathlineto{\pgfqpoint{2.319072in}{0.714738in}}%
\pgfpathlineto{\pgfqpoint{2.319937in}{0.696129in}}%
\pgfpathlineto{\pgfqpoint{2.321668in}{0.776130in}}%
\pgfpathlineto{\pgfqpoint{2.323399in}{0.712907in}}%
\pgfpathlineto{\pgfqpoint{2.324264in}{0.706276in}}%
\pgfpathlineto{\pgfqpoint{2.325993in}{0.728328in}}%
\pgfpathlineto{\pgfqpoint{2.326858in}{0.719647in}}%
\pgfpathlineto{\pgfqpoint{2.327723in}{0.697559in}}%
\pgfpathlineto{\pgfqpoint{2.328589in}{0.844482in}}%
\pgfpathlineto{\pgfqpoint{2.329452in}{0.694481in}}%
\pgfpathlineto{\pgfqpoint{2.330317in}{0.737520in}}%
\pgfpathlineto{\pgfqpoint{2.332043in}{0.662429in}}%
\pgfpathlineto{\pgfqpoint{2.332908in}{0.681439in}}%
\pgfpathlineto{\pgfqpoint{2.333775in}{0.656826in}}%
\pgfpathlineto{\pgfqpoint{2.334640in}{0.790451in}}%
\pgfpathlineto{\pgfqpoint{2.335506in}{0.719720in}}%
\pgfpathlineto{\pgfqpoint{2.336370in}{0.768769in}}%
\pgfpathlineto{\pgfqpoint{2.337235in}{0.629280in}}%
\pgfpathlineto{\pgfqpoint{2.338965in}{0.738146in}}%
\pgfpathlineto{\pgfqpoint{2.339831in}{0.675178in}}%
\pgfpathlineto{\pgfqpoint{2.340692in}{0.712176in}}%
\pgfpathlineto{\pgfqpoint{2.341556in}{0.651222in}}%
\pgfpathlineto{\pgfqpoint{2.342422in}{0.697486in}}%
\pgfpathlineto{\pgfqpoint{2.343286in}{0.632285in}}%
\pgfpathlineto{\pgfqpoint{2.345015in}{0.692394in}}%
\pgfpathlineto{\pgfqpoint{2.345881in}{0.654885in}}%
\pgfpathlineto{\pgfqpoint{2.346747in}{0.718802in}}%
\pgfpathlineto{\pgfqpoint{2.347612in}{0.666241in}}%
\pgfpathlineto{\pgfqpoint{2.348476in}{0.691225in}}%
\pgfpathlineto{\pgfqpoint{2.349341in}{0.756279in}}%
\pgfpathlineto{\pgfqpoint{2.350207in}{0.651921in}}%
\pgfpathlineto{\pgfqpoint{2.351073in}{0.746827in}}%
\pgfpathlineto{\pgfqpoint{2.353668in}{0.631993in}}%
\pgfpathlineto{\pgfqpoint{2.355397in}{0.712834in}}%
\pgfpathlineto{\pgfqpoint{2.356262in}{0.712907in}}%
\pgfpathlineto{\pgfqpoint{2.357127in}{0.649720in}}%
\pgfpathlineto{\pgfqpoint{2.357993in}{0.742100in}}%
\pgfpathlineto{\pgfqpoint{2.358859in}{0.723236in}}%
\pgfpathlineto{\pgfqpoint{2.359724in}{0.686275in}}%
\pgfpathlineto{\pgfqpoint{2.361451in}{0.805653in}}%
\pgfpathlineto{\pgfqpoint{2.362316in}{0.692723in}}%
\pgfpathlineto{\pgfqpoint{2.363181in}{0.694116in}}%
\pgfpathlineto{\pgfqpoint{2.364045in}{0.783784in}}%
\pgfpathlineto{\pgfqpoint{2.364909in}{0.697413in}}%
\pgfpathlineto{\pgfqpoint{2.365771in}{0.700636in}}%
\pgfpathlineto{\pgfqpoint{2.366638in}{0.652323in}}%
\pgfpathlineto{\pgfqpoint{2.368366in}{0.777523in}}%
\pgfpathlineto{\pgfqpoint{2.369231in}{0.691846in}}%
\pgfpathlineto{\pgfqpoint{2.370096in}{0.734045in}}%
\pgfpathlineto{\pgfqpoint{2.370961in}{0.692248in}}%
\pgfpathlineto{\pgfqpoint{2.371826in}{0.761074in}}%
\pgfpathlineto{\pgfqpoint{2.373557in}{0.726533in}}%
\pgfpathlineto{\pgfqpoint{2.375287in}{0.746754in}}%
\pgfpathlineto{\pgfqpoint{2.376151in}{0.726022in}}%
\pgfpathlineto{\pgfqpoint{2.377017in}{0.751260in}}%
\pgfpathlineto{\pgfqpoint{2.377884in}{0.688951in}}%
\pgfpathlineto{\pgfqpoint{2.378749in}{0.711697in}}%
\pgfpathlineto{\pgfqpoint{2.379616in}{0.772025in}}%
\pgfpathlineto{\pgfqpoint{2.380481in}{0.728913in}}%
\pgfpathlineto{\pgfqpoint{2.381346in}{0.747960in}}%
\pgfpathlineto{\pgfqpoint{2.382212in}{0.734297in}}%
\pgfpathlineto{\pgfqpoint{2.383076in}{0.765212in}}%
\pgfpathlineto{\pgfqpoint{2.384807in}{0.669608in}}%
\pgfpathlineto{\pgfqpoint{2.385670in}{0.773783in}}%
\pgfpathlineto{\pgfqpoint{2.386535in}{0.749425in}}%
\pgfpathlineto{\pgfqpoint{2.388265in}{0.809059in}}%
\pgfpathlineto{\pgfqpoint{2.389995in}{0.712208in}}%
\pgfpathlineto{\pgfqpoint{2.390860in}{0.724369in}}%
\pgfpathlineto{\pgfqpoint{2.391725in}{0.714076in}}%
\pgfpathlineto{\pgfqpoint{2.392590in}{0.790158in}}%
\pgfpathlineto{\pgfqpoint{2.393455in}{0.706313in}}%
\pgfpathlineto{\pgfqpoint{2.394320in}{0.725762in}}%
\pgfpathlineto{\pgfqpoint{2.395187in}{0.689389in}}%
\pgfpathlineto{\pgfqpoint{2.396051in}{0.692467in}}%
\pgfpathlineto{\pgfqpoint{2.396915in}{0.772870in}}%
\pgfpathlineto{\pgfqpoint{2.397781in}{0.709317in}}%
\pgfpathlineto{\pgfqpoint{2.398647in}{0.716496in}}%
\pgfpathlineto{\pgfqpoint{2.399511in}{0.707303in}}%
\pgfpathlineto{\pgfqpoint{2.401240in}{0.748694in}}%
\pgfpathlineto{\pgfqpoint{2.402968in}{0.735766in}}%
\pgfpathlineto{\pgfqpoint{2.403834in}{0.712395in}}%
\pgfpathlineto{\pgfqpoint{2.404699in}{0.733456in}}%
\pgfpathlineto{\pgfqpoint{2.405564in}{0.651442in}}%
\pgfpathlineto{\pgfqpoint{2.406429in}{0.752357in}}%
\pgfpathlineto{\pgfqpoint{2.408157in}{0.658182in}}%
\pgfpathlineto{\pgfqpoint{2.411618in}{0.774737in}}%
\pgfpathlineto{\pgfqpoint{2.413348in}{0.695837in}}%
\pgfpathlineto{\pgfqpoint{2.414213in}{0.719501in}}%
\pgfpathlineto{\pgfqpoint{2.415079in}{0.668584in}}%
\pgfpathlineto{\pgfqpoint{2.415943in}{0.709317in}}%
\pgfpathlineto{\pgfqpoint{2.416808in}{0.669465in}}%
\pgfpathlineto{\pgfqpoint{2.417672in}{0.749758in}}%
\pgfpathlineto{\pgfqpoint{2.418538in}{0.732835in}}%
\pgfpathlineto{\pgfqpoint{2.419402in}{0.702321in}}%
\pgfpathlineto{\pgfqpoint{2.421131in}{0.754225in}}%
\pgfpathlineto{\pgfqpoint{2.421995in}{0.701733in}}%
\pgfpathlineto{\pgfqpoint{2.422861in}{0.723529in}}%
\pgfpathlineto{\pgfqpoint{2.423725in}{0.691111in}}%
\pgfpathlineto{\pgfqpoint{2.424590in}{0.692979in}}%
\pgfpathlineto{\pgfqpoint{2.425455in}{0.737301in}}%
\pgfpathlineto{\pgfqpoint{2.426320in}{0.671918in}}%
\pgfpathlineto{\pgfqpoint{2.427184in}{0.672393in}}%
\pgfpathlineto{\pgfqpoint{2.428912in}{0.754480in}}%
\pgfpathlineto{\pgfqpoint{2.430641in}{0.692394in}}%
\pgfpathlineto{\pgfqpoint{2.431505in}{0.680562in}}%
\pgfpathlineto{\pgfqpoint{2.432370in}{0.699646in}}%
\pgfpathlineto{\pgfqpoint{2.434097in}{0.681220in}}%
\pgfpathlineto{\pgfqpoint{2.435826in}{0.777117in}}%
\pgfpathlineto{\pgfqpoint{2.436690in}{0.648729in}}%
\pgfpathlineto{\pgfqpoint{2.437555in}{0.739169in}}%
\pgfpathlineto{\pgfqpoint{2.438420in}{0.685946in}}%
\pgfpathlineto{\pgfqpoint{2.439284in}{0.731844in}}%
\pgfpathlineto{\pgfqpoint{2.441015in}{0.690672in}}%
\pgfpathlineto{\pgfqpoint{2.442744in}{0.747704in}}%
\pgfpathlineto{\pgfqpoint{2.443608in}{0.671881in}}%
\pgfpathlineto{\pgfqpoint{2.446201in}{0.787560in}}%
\pgfpathlineto{\pgfqpoint{2.447066in}{0.666059in}}%
\pgfpathlineto{\pgfqpoint{2.448795in}{0.732908in}}%
\pgfpathlineto{\pgfqpoint{2.449658in}{0.737926in}}%
\pgfpathlineto{\pgfqpoint{2.450524in}{0.618293in}}%
\pgfpathlineto{\pgfqpoint{2.451388in}{0.749060in}}%
\pgfpathlineto{\pgfqpoint{2.452253in}{0.682397in}}%
\pgfpathlineto{\pgfqpoint{2.453117in}{0.733419in}}%
\pgfpathlineto{\pgfqpoint{2.453982in}{0.724812in}}%
\pgfpathlineto{\pgfqpoint{2.454848in}{0.734958in}}%
\pgfpathlineto{\pgfqpoint{2.455713in}{0.726314in}}%
\pgfpathlineto{\pgfqpoint{2.456577in}{0.793455in}}%
\pgfpathlineto{\pgfqpoint{2.457441in}{0.724994in}}%
\pgfpathlineto{\pgfqpoint{2.458306in}{0.733273in}}%
\pgfpathlineto{\pgfqpoint{2.459170in}{0.752503in}}%
\pgfpathlineto{\pgfqpoint{2.460035in}{0.716642in}}%
\pgfpathlineto{\pgfqpoint{2.462629in}{0.783784in}}%
\pgfpathlineto{\pgfqpoint{2.463494in}{0.785944in}}%
\pgfpathlineto{\pgfqpoint{2.464360in}{0.713199in}}%
\pgfpathlineto{\pgfqpoint{2.465225in}{0.733346in}}%
\pgfpathlineto{\pgfqpoint{2.466090in}{0.737594in}}%
\pgfpathlineto{\pgfqpoint{2.466956in}{0.689097in}}%
\pgfpathlineto{\pgfqpoint{2.467821in}{0.735580in}}%
\pgfpathlineto{\pgfqpoint{2.468685in}{0.728986in}}%
\pgfpathlineto{\pgfqpoint{2.469550in}{0.747631in}}%
\pgfpathlineto{\pgfqpoint{2.471277in}{0.729278in}}%
\pgfpathlineto{\pgfqpoint{2.472140in}{0.642907in}}%
\pgfpathlineto{\pgfqpoint{2.473871in}{0.723200in}}%
\pgfpathlineto{\pgfqpoint{2.474737in}{0.695764in}}%
\pgfpathlineto{\pgfqpoint{2.475602in}{0.705801in}}%
\pgfpathlineto{\pgfqpoint{2.476468in}{0.656350in}}%
\pgfpathlineto{\pgfqpoint{2.478200in}{0.775874in}}%
\pgfpathlineto{\pgfqpoint{2.479066in}{0.734995in}}%
\pgfpathlineto{\pgfqpoint{2.479931in}{0.760965in}}%
\pgfpathlineto{\pgfqpoint{2.480797in}{0.731990in}}%
\pgfpathlineto{\pgfqpoint{2.481662in}{0.746827in}}%
\pgfpathlineto{\pgfqpoint{2.482527in}{0.707815in}}%
\pgfpathlineto{\pgfqpoint{2.483391in}{0.750306in}}%
\pgfpathlineto{\pgfqpoint{2.485119in}{0.703162in}}%
\pgfpathlineto{\pgfqpoint{2.486849in}{0.783053in}}%
\pgfpathlineto{\pgfqpoint{2.487715in}{0.774957in}}%
\pgfpathlineto{\pgfqpoint{2.489447in}{0.708400in}}%
\pgfpathlineto{\pgfqpoint{2.490310in}{0.743310in}}%
\pgfpathlineto{\pgfqpoint{2.492039in}{0.702537in}}%
\pgfpathlineto{\pgfqpoint{2.492905in}{0.727374in}}%
\pgfpathlineto{\pgfqpoint{2.493769in}{0.677119in}}%
\pgfpathlineto{\pgfqpoint{2.495501in}{0.749096in}}%
\pgfpathlineto{\pgfqpoint{2.496365in}{0.682430in}}%
\pgfpathlineto{\pgfqpoint{2.497231in}{0.769719in}}%
\pgfpathlineto{\pgfqpoint{2.498096in}{0.708692in}}%
\pgfpathlineto{\pgfqpoint{2.498962in}{0.752868in}}%
\pgfpathlineto{\pgfqpoint{2.499826in}{0.671512in}}%
\pgfpathlineto{\pgfqpoint{2.500692in}{0.715213in}}%
\pgfpathlineto{\pgfqpoint{2.501555in}{0.705176in}}%
\pgfpathlineto{\pgfqpoint{2.502421in}{0.707998in}}%
\pgfpathlineto{\pgfqpoint{2.503285in}{0.639390in}}%
\pgfpathlineto{\pgfqpoint{2.504149in}{0.747704in}}%
\pgfpathlineto{\pgfqpoint{2.505014in}{0.716825in}}%
\pgfpathlineto{\pgfqpoint{2.505880in}{0.707888in}}%
\pgfpathlineto{\pgfqpoint{2.506746in}{0.684225in}}%
\pgfpathlineto{\pgfqpoint{2.508476in}{0.751878in}}%
\pgfpathlineto{\pgfqpoint{2.509340in}{0.737886in}}%
\pgfpathlineto{\pgfqpoint{2.510205in}{0.741621in}}%
\pgfpathlineto{\pgfqpoint{2.511069in}{0.635103in}}%
\pgfpathlineto{\pgfqpoint{2.511934in}{0.710487in}}%
\pgfpathlineto{\pgfqpoint{2.512798in}{0.695358in}}%
\pgfpathlineto{\pgfqpoint{2.513664in}{0.745138in}}%
\pgfpathlineto{\pgfqpoint{2.514530in}{0.725652in}}%
\pgfpathlineto{\pgfqpoint{2.515396in}{0.746348in}}%
\pgfpathlineto{\pgfqpoint{2.516260in}{0.725981in}}%
\pgfpathlineto{\pgfqpoint{2.517125in}{0.756385in}}%
\pgfpathlineto{\pgfqpoint{2.518855in}{0.660927in}}%
\pgfpathlineto{\pgfqpoint{2.519720in}{0.737520in}}%
\pgfpathlineto{\pgfqpoint{2.520585in}{0.715140in}}%
\pgfpathlineto{\pgfqpoint{2.521451in}{0.722246in}}%
\pgfpathlineto{\pgfqpoint{2.522317in}{0.714076in}}%
\pgfpathlineto{\pgfqpoint{2.524046in}{0.762540in}}%
\pgfpathlineto{\pgfqpoint{2.524912in}{0.761403in}}%
\pgfpathlineto{\pgfqpoint{2.525777in}{0.691403in}}%
\pgfpathlineto{\pgfqpoint{2.527509in}{0.753453in}}%
\pgfpathlineto{\pgfqpoint{2.528374in}{0.696605in}}%
\pgfpathlineto{\pgfqpoint{2.529236in}{0.712208in}}%
\pgfpathlineto{\pgfqpoint{2.530102in}{0.696787in}}%
\pgfpathlineto{\pgfqpoint{2.530966in}{0.734333in}}%
\pgfpathlineto{\pgfqpoint{2.531831in}{0.680010in}}%
\pgfpathlineto{\pgfqpoint{2.533562in}{0.804772in}}%
\pgfpathlineto{\pgfqpoint{2.535290in}{0.705541in}}%
\pgfpathlineto{\pgfqpoint{2.536155in}{0.667959in}}%
\pgfpathlineto{\pgfqpoint{2.537886in}{0.759426in}}%
\pgfpathlineto{\pgfqpoint{2.540482in}{0.650378in}}%
\pgfpathlineto{\pgfqpoint{2.542211in}{0.750156in}}%
\pgfpathlineto{\pgfqpoint{2.543077in}{0.679572in}}%
\pgfpathlineto{\pgfqpoint{2.543942in}{0.773893in}}%
\pgfpathlineto{\pgfqpoint{2.544808in}{0.736164in}}%
\pgfpathlineto{\pgfqpoint{2.545672in}{0.759974in}}%
\pgfpathlineto{\pgfqpoint{2.546537in}{0.684956in}}%
\pgfpathlineto{\pgfqpoint{2.547403in}{0.725579in}}%
\pgfpathlineto{\pgfqpoint{2.548266in}{0.713565in}}%
\pgfpathlineto{\pgfqpoint{2.549131in}{0.731438in}}%
\pgfpathlineto{\pgfqpoint{2.549993in}{0.666237in}}%
\pgfpathlineto{\pgfqpoint{2.551724in}{0.733744in}}%
\pgfpathlineto{\pgfqpoint{2.552590in}{0.675576in}}%
\pgfpathlineto{\pgfqpoint{2.554322in}{0.749791in}}%
\pgfpathlineto{\pgfqpoint{2.555188in}{0.710560in}}%
\pgfpathlineto{\pgfqpoint{2.556051in}{0.749718in}}%
\pgfpathlineto{\pgfqpoint{2.556916in}{0.698217in}}%
\pgfpathlineto{\pgfqpoint{2.557782in}{0.726972in}}%
\pgfpathlineto{\pgfqpoint{2.558645in}{0.688326in}}%
\pgfpathlineto{\pgfqpoint{2.559511in}{0.742576in}}%
\pgfpathlineto{\pgfqpoint{2.560377in}{0.676786in}}%
\pgfpathlineto{\pgfqpoint{2.561241in}{0.677850in}}%
\pgfpathlineto{\pgfqpoint{2.562106in}{0.698728in}}%
\pgfpathlineto{\pgfqpoint{2.562970in}{0.764920in}}%
\pgfpathlineto{\pgfqpoint{2.563836in}{0.763417in}}%
\pgfpathlineto{\pgfqpoint{2.564700in}{0.655469in}}%
\pgfpathlineto{\pgfqpoint{2.566428in}{0.736972in}}%
\pgfpathlineto{\pgfqpoint{2.568157in}{0.782501in}}%
\pgfpathlineto{\pgfqpoint{2.569022in}{0.768472in}}%
\pgfpathlineto{\pgfqpoint{2.569888in}{0.701660in}}%
\pgfpathlineto{\pgfqpoint{2.570753in}{0.761842in}}%
\pgfpathlineto{\pgfqpoint{2.571618in}{0.701733in}}%
\pgfpathlineto{\pgfqpoint{2.573346in}{0.730780in}}%
\pgfpathlineto{\pgfqpoint{2.574211in}{0.685727in}}%
\pgfpathlineto{\pgfqpoint{2.575076in}{0.748841in}}%
\pgfpathlineto{\pgfqpoint{2.575939in}{0.721734in}}%
\pgfpathlineto{\pgfqpoint{2.578530in}{0.771952in}}%
\pgfpathlineto{\pgfqpoint{2.580257in}{0.715140in}}%
\pgfpathlineto{\pgfqpoint{2.581123in}{0.763527in}}%
\pgfpathlineto{\pgfqpoint{2.582854in}{0.641916in}}%
\pgfpathlineto{\pgfqpoint{2.583719in}{0.758033in}}%
\pgfpathlineto{\pgfqpoint{2.586313in}{0.684152in}}%
\pgfpathlineto{\pgfqpoint{2.587177in}{0.670306in}}%
\pgfpathlineto{\pgfqpoint{2.589773in}{0.703162in}}%
\pgfpathlineto{\pgfqpoint{2.590637in}{0.698399in}}%
\pgfpathlineto{\pgfqpoint{2.591500in}{0.769167in}}%
\pgfpathlineto{\pgfqpoint{2.592365in}{0.748069in}}%
\pgfpathlineto{\pgfqpoint{2.593230in}{0.673603in}}%
\pgfpathlineto{\pgfqpoint{2.594095in}{0.706532in}}%
\pgfpathlineto{\pgfqpoint{2.594961in}{0.696129in}}%
\pgfpathlineto{\pgfqpoint{2.595827in}{0.711039in}}%
\pgfpathlineto{\pgfqpoint{2.596692in}{0.747923in}}%
\pgfpathlineto{\pgfqpoint{2.598421in}{0.715652in}}%
\pgfpathlineto{\pgfqpoint{2.599286in}{0.771660in}}%
\pgfpathlineto{\pgfqpoint{2.600150in}{0.758435in}}%
\pgfpathlineto{\pgfqpoint{2.601015in}{0.755540in}}%
\pgfpathlineto{\pgfqpoint{2.601880in}{0.742978in}}%
\pgfpathlineto{\pgfqpoint{2.602742in}{0.698655in}}%
\pgfpathlineto{\pgfqpoint{2.604469in}{0.733050in}}%
\pgfpathlineto{\pgfqpoint{2.605334in}{0.713638in}}%
\pgfpathlineto{\pgfqpoint{2.606197in}{0.726566in}}%
\pgfpathlineto{\pgfqpoint{2.607063in}{0.664333in}}%
\pgfpathlineto{\pgfqpoint{2.607928in}{0.767080in}}%
\pgfpathlineto{\pgfqpoint{2.608793in}{0.745576in}}%
\pgfpathlineto{\pgfqpoint{2.609657in}{0.672502in}}%
\pgfpathlineto{\pgfqpoint{2.611387in}{0.771952in}}%
\pgfpathlineto{\pgfqpoint{2.614846in}{0.717227in}}%
\pgfpathlineto{\pgfqpoint{2.615710in}{0.714994in}}%
\pgfpathlineto{\pgfqpoint{2.616574in}{0.680306in}}%
\pgfpathlineto{\pgfqpoint{2.617439in}{0.754736in}}%
\pgfpathlineto{\pgfqpoint{2.618301in}{0.737447in}}%
\pgfpathlineto{\pgfqpoint{2.619166in}{0.680855in}}%
\pgfpathlineto{\pgfqpoint{2.620894in}{0.754663in}}%
\pgfpathlineto{\pgfqpoint{2.621760in}{0.756019in}}%
\pgfpathlineto{\pgfqpoint{2.624354in}{0.678873in}}%
\pgfpathlineto{\pgfqpoint{2.625219in}{0.688910in}}%
\pgfpathlineto{\pgfqpoint{2.626950in}{0.725981in}}%
\pgfpathlineto{\pgfqpoint{2.627816in}{0.753088in}}%
\pgfpathlineto{\pgfqpoint{2.628682in}{0.720816in}}%
\pgfpathlineto{\pgfqpoint{2.629548in}{0.734370in}}%
\pgfpathlineto{\pgfqpoint{2.630413in}{0.729643in}}%
\pgfpathlineto{\pgfqpoint{2.632145in}{0.653967in}}%
\pgfpathlineto{\pgfqpoint{2.634739in}{0.764846in}}%
\pgfpathlineto{\pgfqpoint{2.635603in}{0.695139in}}%
\pgfpathlineto{\pgfqpoint{2.636470in}{0.709975in}}%
\pgfpathlineto{\pgfqpoint{2.637336in}{0.827960in}}%
\pgfpathlineto{\pgfqpoint{2.639064in}{0.727264in}}%
\pgfpathlineto{\pgfqpoint{2.639929in}{0.722319in}}%
\pgfpathlineto{\pgfqpoint{2.641660in}{0.789574in}}%
\pgfpathlineto{\pgfqpoint{2.642524in}{0.721551in}}%
\pgfpathlineto{\pgfqpoint{2.643388in}{0.757595in}}%
\pgfpathlineto{\pgfqpoint{2.644252in}{0.673310in}}%
\pgfpathlineto{\pgfqpoint{2.645115in}{0.675397in}}%
\pgfpathlineto{\pgfqpoint{2.646844in}{0.731479in}}%
\pgfpathlineto{\pgfqpoint{2.647710in}{0.699244in}}%
\pgfpathlineto{\pgfqpoint{2.648575in}{0.748548in}}%
\pgfpathlineto{\pgfqpoint{2.649441in}{0.701294in}}%
\pgfpathlineto{\pgfqpoint{2.650307in}{0.772723in}}%
\pgfpathlineto{\pgfqpoint{2.651172in}{0.682430in}}%
\pgfpathlineto{\pgfqpoint{2.652902in}{0.749060in}}%
\pgfpathlineto{\pgfqpoint{2.653767in}{0.723748in}}%
\pgfpathlineto{\pgfqpoint{2.654633in}{0.663968in}}%
\pgfpathlineto{\pgfqpoint{2.655499in}{0.762906in}}%
\pgfpathlineto{\pgfqpoint{2.656365in}{0.732283in}}%
\pgfpathlineto{\pgfqpoint{2.657231in}{0.780085in}}%
\pgfpathlineto{\pgfqpoint{2.658097in}{0.710048in}}%
\pgfpathlineto{\pgfqpoint{2.658961in}{0.824371in}}%
\pgfpathlineto{\pgfqpoint{2.660689in}{0.687302in}}%
\pgfpathlineto{\pgfqpoint{2.661555in}{0.704116in}}%
\pgfpathlineto{\pgfqpoint{2.662417in}{0.765691in}}%
\pgfpathlineto{\pgfqpoint{2.664149in}{0.706240in}}%
\pgfpathlineto{\pgfqpoint{2.665878in}{0.746275in}}%
\pgfpathlineto{\pgfqpoint{2.667607in}{0.699938in}}%
\pgfpathlineto{\pgfqpoint{2.668473in}{0.736716in}}%
\pgfpathlineto{\pgfqpoint{2.669338in}{0.670087in}}%
\pgfpathlineto{\pgfqpoint{2.670204in}{0.762906in}}%
\pgfpathlineto{\pgfqpoint{2.671069in}{0.685398in}}%
\pgfpathlineto{\pgfqpoint{2.671934in}{0.763271in}}%
\pgfpathlineto{\pgfqpoint{2.672800in}{0.713638in}}%
\pgfpathlineto{\pgfqpoint{2.674531in}{0.775834in}}%
\pgfpathlineto{\pgfqpoint{2.676261in}{0.712537in}}%
\pgfpathlineto{\pgfqpoint{2.677125in}{0.812685in}}%
\pgfpathlineto{\pgfqpoint{2.678857in}{0.706751in}}%
\pgfpathlineto{\pgfqpoint{2.681452in}{0.808877in}}%
\pgfpathlineto{\pgfqpoint{2.682316in}{0.720378in}}%
\pgfpathlineto{\pgfqpoint{2.683182in}{0.731954in}}%
\pgfpathlineto{\pgfqpoint{2.684047in}{0.753672in}}%
\pgfpathlineto{\pgfqpoint{2.684912in}{0.732502in}}%
\pgfpathlineto{\pgfqpoint{2.685777in}{0.739132in}}%
\pgfpathlineto{\pgfqpoint{2.686642in}{0.711551in}}%
\pgfpathlineto{\pgfqpoint{2.687507in}{0.779829in}}%
\pgfpathlineto{\pgfqpoint{2.688372in}{0.721807in}}%
\pgfpathlineto{\pgfqpoint{2.689237in}{0.730780in}}%
\pgfpathlineto{\pgfqpoint{2.690102in}{0.735616in}}%
\pgfpathlineto{\pgfqpoint{2.690968in}{0.717227in}}%
\pgfpathlineto{\pgfqpoint{2.691834in}{0.810671in}}%
\pgfpathlineto{\pgfqpoint{2.694429in}{0.727483in}}%
\pgfpathlineto{\pgfqpoint{2.695292in}{0.684773in}}%
\pgfpathlineto{\pgfqpoint{2.696157in}{0.690892in}}%
\pgfpathlineto{\pgfqpoint{2.697023in}{0.739315in}}%
\pgfpathlineto{\pgfqpoint{2.697889in}{0.671037in}}%
\pgfpathlineto{\pgfqpoint{2.698752in}{0.735872in}}%
\pgfpathlineto{\pgfqpoint{2.699618in}{0.718766in}}%
\pgfpathlineto{\pgfqpoint{2.700483in}{0.686385in}}%
\pgfpathlineto{\pgfqpoint{2.701346in}{0.691842in}}%
\pgfpathlineto{\pgfqpoint{2.702209in}{0.705359in}}%
\pgfpathlineto{\pgfqpoint{2.703074in}{0.689755in}}%
\pgfpathlineto{\pgfqpoint{2.703940in}{0.739681in}}%
\pgfpathlineto{\pgfqpoint{2.704806in}{0.697413in}}%
\pgfpathlineto{\pgfqpoint{2.705670in}{0.768655in}}%
\pgfpathlineto{\pgfqpoint{2.706536in}{0.754517in}}%
\pgfpathlineto{\pgfqpoint{2.707401in}{0.726972in}}%
\pgfpathlineto{\pgfqpoint{2.709134in}{0.764518in}}%
\pgfpathlineto{\pgfqpoint{2.709995in}{0.737374in}}%
\pgfpathlineto{\pgfqpoint{2.710861in}{0.746128in}}%
\pgfpathlineto{\pgfqpoint{2.712592in}{0.687814in}}%
\pgfpathlineto{\pgfqpoint{2.713458in}{0.782355in}}%
\pgfpathlineto{\pgfqpoint{2.714324in}{0.713857in}}%
\pgfpathlineto{\pgfqpoint{2.715189in}{0.756969in}}%
\pgfpathlineto{\pgfqpoint{2.716054in}{0.738146in}}%
\pgfpathlineto{\pgfqpoint{2.716920in}{0.796826in}}%
\pgfpathlineto{\pgfqpoint{2.717785in}{0.729059in}}%
\pgfpathlineto{\pgfqpoint{2.718650in}{0.752357in}}%
\pgfpathlineto{\pgfqpoint{2.719515in}{0.662502in}}%
\pgfpathlineto{\pgfqpoint{2.722110in}{0.768363in}}%
\pgfpathlineto{\pgfqpoint{2.722975in}{0.736164in}}%
\pgfpathlineto{\pgfqpoint{2.723839in}{0.753896in}}%
\pgfpathlineto{\pgfqpoint{2.724705in}{0.713857in}}%
\pgfpathlineto{\pgfqpoint{2.726436in}{0.753672in}}%
\pgfpathlineto{\pgfqpoint{2.728166in}{0.700121in}}%
\pgfpathlineto{\pgfqpoint{2.729031in}{0.701587in}}%
\pgfpathlineto{\pgfqpoint{2.729896in}{0.778765in}}%
\pgfpathlineto{\pgfqpoint{2.730761in}{0.679060in}}%
\pgfpathlineto{\pgfqpoint{2.732492in}{0.727191in}}%
\pgfpathlineto{\pgfqpoint{2.733357in}{0.733858in}}%
\pgfpathlineto{\pgfqpoint{2.735086in}{0.794406in}}%
\pgfpathlineto{\pgfqpoint{2.735950in}{0.715359in}}%
\pgfpathlineto{\pgfqpoint{2.736813in}{0.748768in}}%
\pgfpathlineto{\pgfqpoint{2.737677in}{0.691038in}}%
\pgfpathlineto{\pgfqpoint{2.739404in}{0.782687in}}%
\pgfpathlineto{\pgfqpoint{2.740269in}{0.688183in}}%
\pgfpathlineto{\pgfqpoint{2.741132in}{0.697632in}}%
\pgfpathlineto{\pgfqpoint{2.741997in}{0.692248in}}%
\pgfpathlineto{\pgfqpoint{2.742860in}{0.707669in}}%
\pgfpathlineto{\pgfqpoint{2.743725in}{0.818215in}}%
\pgfpathlineto{\pgfqpoint{2.745455in}{0.693896in}}%
\pgfpathlineto{\pgfqpoint{2.746322in}{0.723967in}}%
\pgfpathlineto{\pgfqpoint{2.747187in}{0.717154in}}%
\pgfpathlineto{\pgfqpoint{2.748918in}{0.766568in}}%
\pgfpathlineto{\pgfqpoint{2.749784in}{0.746936in}}%
\pgfpathlineto{\pgfqpoint{2.750649in}{0.734077in}}%
\pgfpathlineto{\pgfqpoint{2.752380in}{0.766495in}}%
\pgfpathlineto{\pgfqpoint{2.753245in}{0.758179in}}%
\pgfpathlineto{\pgfqpoint{2.754976in}{0.682576in}}%
\pgfpathlineto{\pgfqpoint{2.755840in}{0.732283in}}%
\pgfpathlineto{\pgfqpoint{2.756706in}{0.696422in}}%
\pgfpathlineto{\pgfqpoint{2.758437in}{0.800853in}}%
\pgfpathlineto{\pgfqpoint{2.761898in}{0.648729in}}%
\pgfpathlineto{\pgfqpoint{2.763629in}{0.736790in}}%
\pgfpathlineto{\pgfqpoint{2.764491in}{0.693385in}}%
\pgfpathlineto{\pgfqpoint{2.765356in}{0.731479in}}%
\pgfpathlineto{\pgfqpoint{2.766220in}{0.680051in}}%
\pgfpathlineto{\pgfqpoint{2.767951in}{0.769134in}}%
\pgfpathlineto{\pgfqpoint{2.768815in}{0.769426in}}%
\pgfpathlineto{\pgfqpoint{2.769681in}{0.728109in}}%
\pgfpathlineto{\pgfqpoint{2.770546in}{0.774006in}}%
\pgfpathlineto{\pgfqpoint{2.772275in}{0.705143in}}%
\pgfpathlineto{\pgfqpoint{2.773139in}{0.684078in}}%
\pgfpathlineto{\pgfqpoint{2.774004in}{0.708546in}}%
\pgfpathlineto{\pgfqpoint{2.774866in}{0.678914in}}%
\pgfpathlineto{\pgfqpoint{2.775728in}{0.709866in}}%
\pgfpathlineto{\pgfqpoint{2.776593in}{0.700450in}}%
\pgfpathlineto{\pgfqpoint{2.777458in}{0.749645in}}%
\pgfpathlineto{\pgfqpoint{2.778324in}{0.727191in}}%
\pgfpathlineto{\pgfqpoint{2.779188in}{0.763344in}}%
\pgfpathlineto{\pgfqpoint{2.781780in}{0.717519in}}%
\pgfpathlineto{\pgfqpoint{2.782645in}{0.790962in}}%
\pgfpathlineto{\pgfqpoint{2.783507in}{0.720597in}}%
\pgfpathlineto{\pgfqpoint{2.784371in}{0.735945in}}%
\pgfpathlineto{\pgfqpoint{2.785235in}{0.701111in}}%
\pgfpathlineto{\pgfqpoint{2.786101in}{0.753932in}}%
\pgfpathlineto{\pgfqpoint{2.786965in}{0.732867in}}%
\pgfpathlineto{\pgfqpoint{2.787828in}{0.756312in}}%
\pgfpathlineto{\pgfqpoint{2.788692in}{0.741841in}}%
\pgfpathlineto{\pgfqpoint{2.789556in}{0.786017in}}%
\pgfpathlineto{\pgfqpoint{2.790421in}{0.648949in}}%
\pgfpathlineto{\pgfqpoint{2.793017in}{0.756092in}}%
\pgfpathlineto{\pgfqpoint{2.793883in}{0.701879in}}%
\pgfpathlineto{\pgfqpoint{2.794748in}{0.736311in}}%
\pgfpathlineto{\pgfqpoint{2.795613in}{0.702391in}}%
\pgfpathlineto{\pgfqpoint{2.796478in}{0.710487in}}%
\pgfpathlineto{\pgfqpoint{2.797343in}{0.779240in}}%
\pgfpathlineto{\pgfqpoint{2.798208in}{0.726200in}}%
\pgfpathlineto{\pgfqpoint{2.799072in}{0.728766in}}%
\pgfpathlineto{\pgfqpoint{2.799938in}{0.705286in}}%
\pgfpathlineto{\pgfqpoint{2.800803in}{0.718802in}}%
\pgfpathlineto{\pgfqpoint{2.802530in}{0.681585in}}%
\pgfpathlineto{\pgfqpoint{2.803395in}{0.789972in}}%
\pgfpathlineto{\pgfqpoint{2.804259in}{0.694920in}}%
\pgfpathlineto{\pgfqpoint{2.805124in}{0.743895in}}%
\pgfpathlineto{\pgfqpoint{2.805989in}{0.737740in}}%
\pgfpathlineto{\pgfqpoint{2.806856in}{0.701185in}}%
\pgfpathlineto{\pgfqpoint{2.808586in}{0.747079in}}%
\pgfpathlineto{\pgfqpoint{2.809451in}{0.683344in}}%
\pgfpathlineto{\pgfqpoint{2.810316in}{0.734370in}}%
\pgfpathlineto{\pgfqpoint{2.811181in}{0.731877in}}%
\pgfpathlineto{\pgfqpoint{2.812047in}{0.708985in}}%
\pgfpathlineto{\pgfqpoint{2.812912in}{0.710341in}}%
\pgfpathlineto{\pgfqpoint{2.813775in}{0.738142in}}%
\pgfpathlineto{\pgfqpoint{2.814641in}{0.624335in}}%
\pgfpathlineto{\pgfqpoint{2.815503in}{0.754371in}}%
\pgfpathlineto{\pgfqpoint{2.816368in}{0.751220in}}%
\pgfpathlineto{\pgfqpoint{2.818096in}{0.706167in}}%
\pgfpathlineto{\pgfqpoint{2.818961in}{0.714336in}}%
\pgfpathlineto{\pgfqpoint{2.819827in}{0.703787in}}%
\pgfpathlineto{\pgfqpoint{2.822421in}{0.765910in}}%
\pgfpathlineto{\pgfqpoint{2.823283in}{0.712208in}}%
\pgfpathlineto{\pgfqpoint{2.824146in}{0.798839in}}%
\pgfpathlineto{\pgfqpoint{2.825873in}{0.702212in}}%
\pgfpathlineto{\pgfqpoint{2.827603in}{0.724738in}}%
\pgfpathlineto{\pgfqpoint{2.828468in}{0.691773in}}%
\pgfpathlineto{\pgfqpoint{2.829334in}{0.739940in}}%
\pgfpathlineto{\pgfqpoint{2.830199in}{0.739169in}}%
\pgfpathlineto{\pgfqpoint{2.831064in}{0.672539in}}%
\pgfpathlineto{\pgfqpoint{2.831927in}{0.783857in}}%
\pgfpathlineto{\pgfqpoint{2.833654in}{0.657045in}}%
\pgfpathlineto{\pgfqpoint{2.834521in}{0.778400in}}%
\pgfpathlineto{\pgfqpoint{2.836252in}{0.708692in}}%
\pgfpathlineto{\pgfqpoint{2.837117in}{0.764002in}}%
\pgfpathlineto{\pgfqpoint{2.838847in}{0.692207in}}%
\pgfpathlineto{\pgfqpoint{2.839712in}{0.747558in}}%
\pgfpathlineto{\pgfqpoint{2.841442in}{0.702577in}}%
\pgfpathlineto{\pgfqpoint{2.842307in}{0.675215in}}%
\pgfpathlineto{\pgfqpoint{2.843172in}{0.697266in}}%
\pgfpathlineto{\pgfqpoint{2.844037in}{0.663127in}}%
\pgfpathlineto{\pgfqpoint{2.845764in}{0.738584in}}%
\pgfpathlineto{\pgfqpoint{2.847492in}{0.682247in}}%
\pgfpathlineto{\pgfqpoint{2.848356in}{0.735653in}}%
\pgfpathlineto{\pgfqpoint{2.849221in}{0.733931in}}%
\pgfpathlineto{\pgfqpoint{2.850086in}{0.717154in}}%
\pgfpathlineto{\pgfqpoint{2.850950in}{0.665287in}}%
\pgfpathlineto{\pgfqpoint{2.851815in}{0.688841in}}%
\pgfpathlineto{\pgfqpoint{2.852680in}{0.653455in}}%
\pgfpathlineto{\pgfqpoint{2.855272in}{0.716642in}}%
\pgfpathlineto{\pgfqpoint{2.856137in}{0.698217in}}%
\pgfpathlineto{\pgfqpoint{2.857003in}{0.655506in}}%
\pgfpathlineto{\pgfqpoint{2.857868in}{0.747484in}}%
\pgfpathlineto{\pgfqpoint{2.858734in}{0.683234in}}%
\pgfpathlineto{\pgfqpoint{2.859599in}{0.757156in}}%
\pgfpathlineto{\pgfqpoint{2.860465in}{0.711624in}}%
\pgfpathlineto{\pgfqpoint{2.861329in}{0.765764in}}%
\pgfpathlineto{\pgfqpoint{2.862193in}{0.705103in}}%
\pgfpathlineto{\pgfqpoint{2.863056in}{0.730488in}}%
\pgfpathlineto{\pgfqpoint{2.863921in}{0.680672in}}%
\pgfpathlineto{\pgfqpoint{2.864787in}{0.763417in}}%
\pgfpathlineto{\pgfqpoint{2.865652in}{0.744553in}}%
\pgfpathlineto{\pgfqpoint{2.866517in}{0.744261in}}%
\pgfpathlineto{\pgfqpoint{2.867382in}{0.706605in}}%
\pgfpathlineto{\pgfqpoint{2.868246in}{0.739315in}}%
\pgfpathlineto{\pgfqpoint{2.869111in}{0.732210in}}%
\pgfpathlineto{\pgfqpoint{2.870840in}{0.679827in}}%
\pgfpathlineto{\pgfqpoint{2.872568in}{0.753015in}}%
\pgfpathlineto{\pgfqpoint{2.873433in}{0.691001in}}%
\pgfpathlineto{\pgfqpoint{2.875161in}{0.760819in}}%
\pgfpathlineto{\pgfqpoint{2.876025in}{0.683051in}}%
\pgfpathlineto{\pgfqpoint{2.876888in}{0.706751in}}%
\pgfpathlineto{\pgfqpoint{2.877753in}{0.763198in}}%
\pgfpathlineto{\pgfqpoint{2.879484in}{0.687375in}}%
\pgfpathlineto{\pgfqpoint{2.881212in}{0.775103in}}%
\pgfpathlineto{\pgfqpoint{2.882941in}{0.711953in}}%
\pgfpathlineto{\pgfqpoint{2.883803in}{0.740525in}}%
\pgfpathlineto{\pgfqpoint{2.884668in}{0.735872in}}%
\pgfpathlineto{\pgfqpoint{2.885532in}{0.649903in}}%
\pgfpathlineto{\pgfqpoint{2.886397in}{0.786277in}}%
\pgfpathlineto{\pgfqpoint{2.888127in}{0.703820in}}%
\pgfpathlineto{\pgfqpoint{2.888993in}{0.687116in}}%
\pgfpathlineto{\pgfqpoint{2.889857in}{0.699605in}}%
\pgfpathlineto{\pgfqpoint{2.891587in}{0.779569in}}%
\pgfpathlineto{\pgfqpoint{2.893318in}{0.731584in}}%
\pgfpathlineto{\pgfqpoint{2.894183in}{0.735799in}}%
\pgfpathlineto{\pgfqpoint{2.895047in}{0.732356in}}%
\pgfpathlineto{\pgfqpoint{2.895913in}{0.740452in}}%
\pgfpathlineto{\pgfqpoint{2.898505in}{0.658328in}}%
\pgfpathlineto{\pgfqpoint{2.899368in}{0.756019in}}%
\pgfpathlineto{\pgfqpoint{2.900234in}{0.630783in}}%
\pgfpathlineto{\pgfqpoint{2.901099in}{0.740013in}}%
\pgfpathlineto{\pgfqpoint{2.901964in}{0.727780in}}%
\pgfpathlineto{\pgfqpoint{2.902827in}{0.685215in}}%
\pgfpathlineto{\pgfqpoint{2.905420in}{0.759828in}}%
\pgfpathlineto{\pgfqpoint{2.906286in}{0.675836in}}%
\pgfpathlineto{\pgfqpoint{2.907152in}{0.694042in}}%
\pgfpathlineto{\pgfqpoint{2.908881in}{0.756791in}}%
\pgfpathlineto{\pgfqpoint{2.909746in}{0.702650in}}%
\pgfpathlineto{\pgfqpoint{2.911476in}{0.776861in}}%
\pgfpathlineto{\pgfqpoint{2.914071in}{0.695874in}}%
\pgfpathlineto{\pgfqpoint{2.914936in}{0.712687in}}%
\pgfpathlineto{\pgfqpoint{2.915801in}{0.741881in}}%
\pgfpathlineto{\pgfqpoint{2.916667in}{0.714336in}}%
\pgfpathlineto{\pgfqpoint{2.917532in}{0.791003in}}%
\pgfpathlineto{\pgfqpoint{2.918398in}{0.698951in}}%
\pgfpathlineto{\pgfqpoint{2.919262in}{0.726022in}}%
\pgfpathlineto{\pgfqpoint{2.920993in}{0.709281in}}%
\pgfpathlineto{\pgfqpoint{2.921858in}{0.743384in}}%
\pgfpathlineto{\pgfqpoint{2.922722in}{0.696129in}}%
\pgfpathlineto{\pgfqpoint{2.923588in}{0.798766in}}%
\pgfpathlineto{\pgfqpoint{2.924453in}{0.775322in}}%
\pgfpathlineto{\pgfqpoint{2.926183in}{0.675836in}}%
\pgfpathlineto{\pgfqpoint{2.927046in}{0.792246in}}%
\pgfpathlineto{\pgfqpoint{2.927911in}{0.750562in}}%
\pgfpathlineto{\pgfqpoint{2.928776in}{0.769207in}}%
\pgfpathlineto{\pgfqpoint{2.929642in}{0.721880in}}%
\pgfpathlineto{\pgfqpoint{2.931370in}{0.798035in}}%
\pgfpathlineto{\pgfqpoint{2.932235in}{0.770709in}}%
\pgfpathlineto{\pgfqpoint{2.933966in}{0.707669in}}%
\pgfpathlineto{\pgfqpoint{2.935694in}{0.745836in}}%
\pgfpathlineto{\pgfqpoint{2.936558in}{0.754225in}}%
\pgfpathlineto{\pgfqpoint{2.937423in}{0.733383in}}%
\pgfpathlineto{\pgfqpoint{2.939153in}{0.754371in}}%
\pgfpathlineto{\pgfqpoint{2.940017in}{0.721697in}}%
\pgfpathlineto{\pgfqpoint{2.941744in}{0.775322in}}%
\pgfpathlineto{\pgfqpoint{2.942610in}{0.718766in}}%
\pgfpathlineto{\pgfqpoint{2.943475in}{0.728839in}}%
\pgfpathlineto{\pgfqpoint{2.944340in}{0.707596in}}%
\pgfpathlineto{\pgfqpoint{2.945203in}{0.778985in}}%
\pgfpathlineto{\pgfqpoint{2.946068in}{0.774551in}}%
\pgfpathlineto{\pgfqpoint{2.946935in}{0.742868in}}%
\pgfpathlineto{\pgfqpoint{2.947799in}{0.774551in}}%
\pgfpathlineto{\pgfqpoint{2.948660in}{0.692865in}}%
\pgfpathlineto{\pgfqpoint{2.950389in}{0.740192in}}%
\pgfpathlineto{\pgfqpoint{2.951255in}{0.702756in}}%
\pgfpathlineto{\pgfqpoint{2.952121in}{0.717921in}}%
\pgfpathlineto{\pgfqpoint{2.952984in}{0.708838in}}%
\pgfpathlineto{\pgfqpoint{2.953848in}{0.665945in}}%
\pgfpathlineto{\pgfqpoint{2.955575in}{0.740265in}}%
\pgfpathlineto{\pgfqpoint{2.956440in}{0.738178in}}%
\pgfpathlineto{\pgfqpoint{2.957306in}{0.687116in}}%
\pgfpathlineto{\pgfqpoint{2.958170in}{0.762719in}}%
\pgfpathlineto{\pgfqpoint{2.959036in}{0.659534in}}%
\pgfpathlineto{\pgfqpoint{2.961632in}{0.744809in}}%
\pgfpathlineto{\pgfqpoint{2.963362in}{0.695724in}}%
\pgfpathlineto{\pgfqpoint{2.965093in}{0.763856in}}%
\pgfpathlineto{\pgfqpoint{2.966821in}{0.705761in}}%
\pgfpathlineto{\pgfqpoint{2.967687in}{0.725981in}}%
\pgfpathlineto{\pgfqpoint{2.968551in}{0.719351in}}%
\pgfpathlineto{\pgfqpoint{2.969413in}{0.686604in}}%
\pgfpathlineto{\pgfqpoint{2.970278in}{0.714588in}}%
\pgfpathlineto{\pgfqpoint{2.971143in}{0.781108in}}%
\pgfpathlineto{\pgfqpoint{2.972009in}{0.760340in}}%
\pgfpathlineto{\pgfqpoint{2.972874in}{0.760851in}}%
\pgfpathlineto{\pgfqpoint{2.973738in}{0.789972in}}%
\pgfpathlineto{\pgfqpoint{2.975468in}{0.698765in}}%
\pgfpathlineto{\pgfqpoint{2.976333in}{0.751805in}}%
\pgfpathlineto{\pgfqpoint{2.977198in}{0.731730in}}%
\pgfpathlineto{\pgfqpoint{2.978062in}{0.761988in}}%
\pgfpathlineto{\pgfqpoint{2.979791in}{0.692759in}}%
\pgfpathlineto{\pgfqpoint{2.980655in}{0.696495in}}%
\pgfpathlineto{\pgfqpoint{2.982384in}{0.762353in}}%
\pgfpathlineto{\pgfqpoint{2.983249in}{0.658839in}}%
\pgfpathlineto{\pgfqpoint{2.984114in}{0.780341in}}%
\pgfpathlineto{\pgfqpoint{2.984979in}{0.689938in}}%
\pgfpathlineto{\pgfqpoint{2.985844in}{0.742978in}}%
\pgfpathlineto{\pgfqpoint{2.986710in}{0.606607in}}%
\pgfpathlineto{\pgfqpoint{2.989305in}{0.744114in}}%
\pgfpathlineto{\pgfqpoint{2.990170in}{0.744224in}}%
\pgfpathlineto{\pgfqpoint{2.991034in}{0.794479in}}%
\pgfpathlineto{\pgfqpoint{2.991898in}{0.698436in}}%
\pgfpathlineto{\pgfqpoint{2.992763in}{0.753234in}}%
\pgfpathlineto{\pgfqpoint{2.993628in}{0.747631in}}%
\pgfpathlineto{\pgfqpoint{2.994494in}{0.741296in}}%
\pgfpathlineto{\pgfqpoint{2.995360in}{0.846751in}}%
\pgfpathlineto{\pgfqpoint{2.997090in}{0.709646in}}%
\pgfpathlineto{\pgfqpoint{2.997954in}{0.775030in}}%
\pgfpathlineto{\pgfqpoint{2.998820in}{0.763052in}}%
\pgfpathlineto{\pgfqpoint{2.999686in}{0.708436in}}%
\pgfpathlineto{\pgfqpoint{3.000551in}{0.734150in}}%
\pgfpathlineto{\pgfqpoint{3.001416in}{0.716642in}}%
\pgfpathlineto{\pgfqpoint{3.002282in}{0.719939in}}%
\pgfpathlineto{\pgfqpoint{3.003146in}{0.726899in}}%
\pgfpathlineto{\pgfqpoint{3.004010in}{0.702431in}}%
\pgfpathlineto{\pgfqpoint{3.005740in}{0.766568in}}%
\pgfpathlineto{\pgfqpoint{3.008335in}{0.709975in}}%
\pgfpathlineto{\pgfqpoint{3.009201in}{0.738475in}}%
\pgfpathlineto{\pgfqpoint{3.010065in}{0.698363in}}%
\pgfpathlineto{\pgfqpoint{3.010930in}{0.732429in}}%
\pgfpathlineto{\pgfqpoint{3.011793in}{0.691330in}}%
\pgfpathlineto{\pgfqpoint{3.013525in}{0.718181in}}%
\pgfpathlineto{\pgfqpoint{3.014390in}{0.708254in}}%
\pgfpathlineto{\pgfqpoint{3.015255in}{0.752868in}}%
\pgfpathlineto{\pgfqpoint{3.016120in}{0.699938in}}%
\pgfpathlineto{\pgfqpoint{3.016985in}{0.705176in}}%
\pgfpathlineto{\pgfqpoint{3.019578in}{0.795762in}}%
\pgfpathlineto{\pgfqpoint{3.021309in}{0.701075in}}%
\pgfpathlineto{\pgfqpoint{3.023036in}{0.764664in}}%
\pgfpathlineto{\pgfqpoint{3.023900in}{0.721515in}}%
\pgfpathlineto{\pgfqpoint{3.024765in}{0.726826in}}%
\pgfpathlineto{\pgfqpoint{3.025630in}{0.729574in}}%
\pgfpathlineto{\pgfqpoint{3.026495in}{0.787852in}}%
\pgfpathlineto{\pgfqpoint{3.027361in}{0.786350in}}%
\pgfpathlineto{\pgfqpoint{3.028227in}{0.753859in}}%
\pgfpathlineto{\pgfqpoint{3.029093in}{0.776824in}}%
\pgfpathlineto{\pgfqpoint{3.029959in}{0.702504in}}%
\pgfpathlineto{\pgfqpoint{3.030824in}{0.740525in}}%
\pgfpathlineto{\pgfqpoint{3.031687in}{0.733785in}}%
\pgfpathlineto{\pgfqpoint{3.032552in}{0.726899in}}%
\pgfpathlineto{\pgfqpoint{3.033415in}{0.696349in}}%
\pgfpathlineto{\pgfqpoint{3.034281in}{0.818694in}}%
\pgfpathlineto{\pgfqpoint{3.035146in}{0.714299in}}%
\pgfpathlineto{\pgfqpoint{3.036011in}{0.853491in}}%
\pgfpathlineto{\pgfqpoint{3.038605in}{0.687814in}}%
\pgfpathlineto{\pgfqpoint{3.039470in}{0.753640in}}%
\pgfpathlineto{\pgfqpoint{3.040334in}{0.700271in}}%
\pgfpathlineto{\pgfqpoint{3.041199in}{0.777669in}}%
\pgfpathlineto{\pgfqpoint{3.042064in}{0.729538in}}%
\pgfpathlineto{\pgfqpoint{3.042930in}{0.782541in}}%
\pgfpathlineto{\pgfqpoint{3.044659in}{0.740415in}}%
\pgfpathlineto{\pgfqpoint{3.046388in}{0.727889in}}%
\pgfpathlineto{\pgfqpoint{3.047252in}{0.795214in}}%
\pgfpathlineto{\pgfqpoint{3.048983in}{0.719208in}}%
\pgfpathlineto{\pgfqpoint{3.049848in}{0.713970in}}%
\pgfpathlineto{\pgfqpoint{3.050713in}{0.796168in}}%
\pgfpathlineto{\pgfqpoint{3.051578in}{0.781295in}}%
\pgfpathlineto{\pgfqpoint{3.052443in}{0.726022in}}%
\pgfpathlineto{\pgfqpoint{3.053308in}{0.777303in}}%
\pgfpathlineto{\pgfqpoint{3.055037in}{0.686644in}}%
\pgfpathlineto{\pgfqpoint{3.055901in}{0.746827in}}%
\pgfpathlineto{\pgfqpoint{3.056766in}{0.612470in}}%
\pgfpathlineto{\pgfqpoint{3.058495in}{0.827229in}}%
\pgfpathlineto{\pgfqpoint{3.059359in}{0.681845in}}%
\pgfpathlineto{\pgfqpoint{3.060224in}{0.810817in}}%
\pgfpathlineto{\pgfqpoint{3.061088in}{0.732502in}}%
\pgfpathlineto{\pgfqpoint{3.061953in}{0.744041in}}%
\pgfpathlineto{\pgfqpoint{3.062818in}{0.760453in}}%
\pgfpathlineto{\pgfqpoint{3.063682in}{0.740525in}}%
\pgfpathlineto{\pgfqpoint{3.064545in}{0.743603in}}%
\pgfpathlineto{\pgfqpoint{3.065410in}{0.728255in}}%
\pgfpathlineto{\pgfqpoint{3.066275in}{0.678366in}}%
\pgfpathlineto{\pgfqpoint{3.067139in}{0.809940in}}%
\pgfpathlineto{\pgfqpoint{3.068870in}{0.702797in}}%
\pgfpathlineto{\pgfqpoint{3.069735in}{0.733054in}}%
\pgfpathlineto{\pgfqpoint{3.070600in}{0.717816in}}%
\pgfpathlineto{\pgfqpoint{3.072330in}{0.745397in}}%
\pgfpathlineto{\pgfqpoint{3.073195in}{0.712103in}}%
\pgfpathlineto{\pgfqpoint{3.074926in}{0.733566in}}%
\pgfpathlineto{\pgfqpoint{3.075790in}{0.709866in}}%
\pgfpathlineto{\pgfqpoint{3.076655in}{0.723821in}}%
\pgfpathlineto{\pgfqpoint{3.077520in}{0.708473in}}%
\pgfpathlineto{\pgfqpoint{3.078385in}{0.663200in}}%
\pgfpathlineto{\pgfqpoint{3.079251in}{0.664849in}}%
\pgfpathlineto{\pgfqpoint{3.080982in}{0.740452in}}%
\pgfpathlineto{\pgfqpoint{3.081847in}{0.815763in}}%
\pgfpathlineto{\pgfqpoint{3.082712in}{0.750416in}}%
\pgfpathlineto{\pgfqpoint{3.083578in}{0.754590in}}%
\pgfpathlineto{\pgfqpoint{3.084443in}{0.780487in}}%
\pgfpathlineto{\pgfqpoint{3.086170in}{0.735872in}}%
\pgfpathlineto{\pgfqpoint{3.087036in}{0.812174in}}%
\pgfpathlineto{\pgfqpoint{3.090498in}{0.680051in}}%
\pgfpathlineto{\pgfqpoint{3.093092in}{0.750051in}}%
\pgfpathlineto{\pgfqpoint{3.094822in}{0.685142in}}%
\pgfpathlineto{\pgfqpoint{3.095687in}{0.693677in}}%
\pgfpathlineto{\pgfqpoint{3.096552in}{0.734995in}}%
\pgfpathlineto{\pgfqpoint{3.097417in}{0.725177in}}%
\pgfpathlineto{\pgfqpoint{3.099147in}{0.756641in}}%
\pgfpathlineto{\pgfqpoint{3.100012in}{0.682722in}}%
\pgfpathlineto{\pgfqpoint{3.100878in}{0.725177in}}%
\pgfpathlineto{\pgfqpoint{3.101742in}{0.685435in}}%
\pgfpathlineto{\pgfqpoint{3.103470in}{0.738292in}}%
\pgfpathlineto{\pgfqpoint{3.104334in}{0.731588in}}%
\pgfpathlineto{\pgfqpoint{3.105199in}{0.695764in}}%
\pgfpathlineto{\pgfqpoint{3.106065in}{0.713272in}}%
\pgfpathlineto{\pgfqpoint{3.106930in}{0.694554in}}%
\pgfpathlineto{\pgfqpoint{3.107795in}{0.759682in}}%
\pgfpathlineto{\pgfqpoint{3.108661in}{0.742100in}}%
\pgfpathlineto{\pgfqpoint{3.109526in}{0.693823in}}%
\pgfpathlineto{\pgfqpoint{3.110391in}{0.705801in}}%
\pgfpathlineto{\pgfqpoint{3.111255in}{0.688878in}}%
\pgfpathlineto{\pgfqpoint{3.112119in}{0.751512in}}%
\pgfpathlineto{\pgfqpoint{3.112985in}{0.665068in}}%
\pgfpathlineto{\pgfqpoint{3.113850in}{0.777121in}}%
\pgfpathlineto{\pgfqpoint{3.114714in}{0.748914in}}%
\pgfpathlineto{\pgfqpoint{3.115579in}{0.749206in}}%
\pgfpathlineto{\pgfqpoint{3.117306in}{0.717925in}}%
\pgfpathlineto{\pgfqpoint{3.118172in}{0.723711in}}%
\pgfpathlineto{\pgfqpoint{3.119038in}{0.707961in}}%
\pgfpathlineto{\pgfqpoint{3.119902in}{0.749206in}}%
\pgfpathlineto{\pgfqpoint{3.120768in}{0.720122in}}%
\pgfpathlineto{\pgfqpoint{3.122496in}{0.767559in}}%
\pgfpathlineto{\pgfqpoint{3.123362in}{0.732173in}}%
\pgfpathlineto{\pgfqpoint{3.124228in}{0.744918in}}%
\pgfpathlineto{\pgfqpoint{3.125956in}{0.698290in}}%
\pgfpathlineto{\pgfqpoint{3.126821in}{0.730853in}}%
\pgfpathlineto{\pgfqpoint{3.127686in}{0.658507in}}%
\pgfpathlineto{\pgfqpoint{3.128550in}{0.684184in}}%
\pgfpathlineto{\pgfqpoint{3.129414in}{0.751512in}}%
\pgfpathlineto{\pgfqpoint{3.130279in}{0.678179in}}%
\pgfpathlineto{\pgfqpoint{3.132007in}{0.750229in}}%
\pgfpathlineto{\pgfqpoint{3.133740in}{0.713491in}}%
\pgfpathlineto{\pgfqpoint{3.134605in}{0.763271in}}%
\pgfpathlineto{\pgfqpoint{3.135471in}{0.698582in}}%
\pgfpathlineto{\pgfqpoint{3.136337in}{0.731292in}}%
\pgfpathlineto{\pgfqpoint{3.137201in}{0.717081in}}%
\pgfpathlineto{\pgfqpoint{3.138930in}{0.758983in}}%
\pgfpathlineto{\pgfqpoint{3.140660in}{0.678288in}}%
\pgfpathlineto{\pgfqpoint{3.141527in}{0.714076in}}%
\pgfpathlineto{\pgfqpoint{3.142392in}{0.676348in}}%
\pgfpathlineto{\pgfqpoint{3.143259in}{0.751001in}}%
\pgfpathlineto{\pgfqpoint{3.144124in}{0.746348in}}%
\pgfpathlineto{\pgfqpoint{3.144990in}{0.696276in}}%
\pgfpathlineto{\pgfqpoint{3.145857in}{0.770450in}}%
\pgfpathlineto{\pgfqpoint{3.147589in}{0.718949in}}%
\pgfpathlineto{\pgfqpoint{3.148455in}{0.724040in}}%
\pgfpathlineto{\pgfqpoint{3.149320in}{0.720195in}}%
\pgfpathlineto{\pgfqpoint{3.151050in}{0.758033in}}%
\pgfpathlineto{\pgfqpoint{3.151916in}{0.716935in}}%
\pgfpathlineto{\pgfqpoint{3.152782in}{0.761549in}}%
\pgfpathlineto{\pgfqpoint{3.155373in}{0.632650in}}%
\pgfpathlineto{\pgfqpoint{3.156236in}{0.735507in}}%
\pgfpathlineto{\pgfqpoint{3.157101in}{0.723821in}}%
\pgfpathlineto{\pgfqpoint{3.158829in}{0.714263in}}%
\pgfpathlineto{\pgfqpoint{3.159695in}{0.737155in}}%
\pgfpathlineto{\pgfqpoint{3.160561in}{0.685873in}}%
\pgfpathlineto{\pgfqpoint{3.161425in}{0.692540in}}%
\pgfpathlineto{\pgfqpoint{3.162292in}{0.696056in}}%
\pgfpathlineto{\pgfqpoint{3.163156in}{0.712761in}}%
\pgfpathlineto{\pgfqpoint{3.164021in}{0.808146in}}%
\pgfpathlineto{\pgfqpoint{3.165751in}{0.718437in}}%
\pgfpathlineto{\pgfqpoint{3.166615in}{0.767559in}}%
\pgfpathlineto{\pgfqpoint{3.167480in}{0.672320in}}%
\pgfpathlineto{\pgfqpoint{3.168345in}{0.734004in}}%
\pgfpathlineto{\pgfqpoint{3.169210in}{0.732136in}}%
\pgfpathlineto{\pgfqpoint{3.170075in}{0.722246in}}%
\pgfpathlineto{\pgfqpoint{3.170938in}{0.686202in}}%
\pgfpathlineto{\pgfqpoint{3.171802in}{0.743968in}}%
\pgfpathlineto{\pgfqpoint{3.173530in}{0.650085in}}%
\pgfpathlineto{\pgfqpoint{3.174393in}{0.762467in}}%
\pgfpathlineto{\pgfqpoint{3.175258in}{0.686352in}}%
\pgfpathlineto{\pgfqpoint{3.176124in}{0.712103in}}%
\pgfpathlineto{\pgfqpoint{3.176989in}{0.695399in}}%
\pgfpathlineto{\pgfqpoint{3.177853in}{0.707523in}}%
\pgfpathlineto{\pgfqpoint{3.178719in}{0.664922in}}%
\pgfpathlineto{\pgfqpoint{3.180447in}{0.750891in}}%
\pgfpathlineto{\pgfqpoint{3.181311in}{0.711989in}}%
\pgfpathlineto{\pgfqpoint{3.183039in}{0.768509in}}%
\pgfpathlineto{\pgfqpoint{3.183904in}{0.646569in}}%
\pgfpathlineto{\pgfqpoint{3.184769in}{0.655835in}}%
\pgfpathlineto{\pgfqpoint{3.185635in}{0.720232in}}%
\pgfpathlineto{\pgfqpoint{3.186500in}{0.711331in}}%
\pgfpathlineto{\pgfqpoint{3.188230in}{0.715838in}}%
\pgfpathlineto{\pgfqpoint{3.189095in}{0.682138in}}%
\pgfpathlineto{\pgfqpoint{3.189960in}{0.697815in}}%
\pgfpathlineto{\pgfqpoint{3.190825in}{0.769832in}}%
\pgfpathlineto{\pgfqpoint{3.192553in}{0.722871in}}%
\pgfpathlineto{\pgfqpoint{3.193417in}{0.743968in}}%
\pgfpathlineto{\pgfqpoint{3.194278in}{0.743603in}}%
\pgfpathlineto{\pgfqpoint{3.195143in}{0.733712in}}%
\pgfpathlineto{\pgfqpoint{3.196006in}{0.778254in}}%
\pgfpathlineto{\pgfqpoint{3.196869in}{0.728511in}}%
\pgfpathlineto{\pgfqpoint{3.197734in}{0.735507in}}%
\pgfpathlineto{\pgfqpoint{3.198598in}{0.755873in}}%
\pgfpathlineto{\pgfqpoint{3.199463in}{0.644701in}}%
\pgfpathlineto{\pgfqpoint{3.200327in}{0.658035in}}%
\pgfpathlineto{\pgfqpoint{3.201193in}{0.714628in}}%
\pgfpathlineto{\pgfqpoint{3.202922in}{0.685581in}}%
\pgfpathlineto{\pgfqpoint{3.205515in}{0.784555in}}%
\pgfpathlineto{\pgfqpoint{3.206381in}{0.685361in}}%
\pgfpathlineto{\pgfqpoint{3.207245in}{0.758472in}}%
\pgfpathlineto{\pgfqpoint{3.208975in}{0.718802in}}%
\pgfpathlineto{\pgfqpoint{3.209840in}{0.762906in}}%
\pgfpathlineto{\pgfqpoint{3.210707in}{0.706093in}}%
\pgfpathlineto{\pgfqpoint{3.211573in}{0.757960in}}%
\pgfpathlineto{\pgfqpoint{3.212439in}{0.757595in}}%
\pgfpathlineto{\pgfqpoint{3.213304in}{0.758106in}}%
\pgfpathlineto{\pgfqpoint{3.214169in}{0.788071in}}%
\pgfpathlineto{\pgfqpoint{3.215896in}{0.721734in}}%
\pgfpathlineto{\pgfqpoint{3.216762in}{0.742320in}}%
\pgfpathlineto{\pgfqpoint{3.217627in}{0.734849in}}%
\pgfpathlineto{\pgfqpoint{3.218492in}{0.694079in}}%
\pgfpathlineto{\pgfqpoint{3.220222in}{0.742174in}}%
\pgfpathlineto{\pgfqpoint{3.221087in}{0.750928in}}%
\pgfpathlineto{\pgfqpoint{3.222816in}{0.716131in}}%
\pgfpathlineto{\pgfqpoint{3.224547in}{0.774080in}}%
\pgfpathlineto{\pgfqpoint{3.225412in}{0.685508in}}%
\pgfpathlineto{\pgfqpoint{3.226276in}{0.686239in}}%
\pgfpathlineto{\pgfqpoint{3.227141in}{0.760745in}}%
\pgfpathlineto{\pgfqpoint{3.228006in}{0.745361in}}%
\pgfpathlineto{\pgfqpoint{3.228872in}{0.702723in}}%
\pgfpathlineto{\pgfqpoint{3.229739in}{0.794665in}}%
\pgfpathlineto{\pgfqpoint{3.230605in}{0.742320in}}%
\pgfpathlineto{\pgfqpoint{3.231470in}{0.787081in}}%
\pgfpathlineto{\pgfqpoint{3.232335in}{0.677594in}}%
\pgfpathlineto{\pgfqpoint{3.233200in}{0.690892in}}%
\pgfpathlineto{\pgfqpoint{3.234064in}{0.687887in}}%
\pgfpathlineto{\pgfqpoint{3.235794in}{0.796460in}}%
\pgfpathlineto{\pgfqpoint{3.237525in}{0.702833in}}%
\pgfpathlineto{\pgfqpoint{3.239254in}{0.740379in}}%
\pgfpathlineto{\pgfqpoint{3.242713in}{0.594410in}}%
\pgfpathlineto{\pgfqpoint{3.243578in}{0.725396in}}%
\pgfpathlineto{\pgfqpoint{3.244443in}{0.713345in}}%
\pgfpathlineto{\pgfqpoint{3.245308in}{0.691184in}}%
\pgfpathlineto{\pgfqpoint{3.246173in}{0.750197in}}%
\pgfpathlineto{\pgfqpoint{3.247039in}{0.653200in}}%
\pgfpathlineto{\pgfqpoint{3.247901in}{0.758366in}}%
\pgfpathlineto{\pgfqpoint{3.248767in}{0.711185in}}%
\pgfpathlineto{\pgfqpoint{3.250497in}{0.743895in}}%
\pgfpathlineto{\pgfqpoint{3.251361in}{0.714628in}}%
\pgfpathlineto{\pgfqpoint{3.252226in}{0.646277in}}%
\pgfpathlineto{\pgfqpoint{3.253089in}{0.803785in}}%
\pgfpathlineto{\pgfqpoint{3.253954in}{0.686681in}}%
\pgfpathlineto{\pgfqpoint{3.254819in}{0.703641in}}%
\pgfpathlineto{\pgfqpoint{3.255685in}{0.635655in}}%
\pgfpathlineto{\pgfqpoint{3.256550in}{0.726277in}}%
\pgfpathlineto{\pgfqpoint{3.257415in}{0.691078in}}%
\pgfpathlineto{\pgfqpoint{3.258280in}{0.732981in}}%
\pgfpathlineto{\pgfqpoint{3.259145in}{0.717852in}}%
\pgfpathlineto{\pgfqpoint{3.260010in}{0.746461in}}%
\pgfpathlineto{\pgfqpoint{3.260876in}{0.689207in}}%
\pgfpathlineto{\pgfqpoint{3.262605in}{0.721734in}}%
\pgfpathlineto{\pgfqpoint{3.263469in}{0.692613in}}%
\pgfpathlineto{\pgfqpoint{3.264334in}{0.772577in}}%
\pgfpathlineto{\pgfqpoint{3.266929in}{0.679133in}}%
\pgfpathlineto{\pgfqpoint{3.267794in}{0.784263in}}%
\pgfpathlineto{\pgfqpoint{3.268657in}{0.713678in}}%
\pgfpathlineto{\pgfqpoint{3.270385in}{0.745361in}}%
\pgfpathlineto{\pgfqpoint{3.272980in}{0.695070in}}%
\pgfpathlineto{\pgfqpoint{3.273846in}{0.794592in}}%
\pgfpathlineto{\pgfqpoint{3.275573in}{0.709317in}}%
\pgfpathlineto{\pgfqpoint{3.276438in}{0.769280in}}%
\pgfpathlineto{\pgfqpoint{3.277300in}{0.764042in}}%
\pgfpathlineto{\pgfqpoint{3.278164in}{0.707815in}}%
\pgfpathlineto{\pgfqpoint{3.279029in}{0.778765in}}%
\pgfpathlineto{\pgfqpoint{3.280759in}{0.668950in}}%
\pgfpathlineto{\pgfqpoint{3.281623in}{0.760819in}}%
\pgfpathlineto{\pgfqpoint{3.282489in}{0.637486in}}%
\pgfpathlineto{\pgfqpoint{3.284218in}{0.734556in}}%
\pgfpathlineto{\pgfqpoint{3.285084in}{0.722249in}}%
\pgfpathlineto{\pgfqpoint{3.286815in}{0.765910in}}%
\pgfpathlineto{\pgfqpoint{3.287681in}{0.700490in}}%
\pgfpathlineto{\pgfqpoint{3.288546in}{0.768988in}}%
\pgfpathlineto{\pgfqpoint{3.289412in}{0.754777in}}%
\pgfpathlineto{\pgfqpoint{3.290277in}{0.711039in}}%
\pgfpathlineto{\pgfqpoint{3.291141in}{0.750855in}}%
\pgfpathlineto{\pgfqpoint{3.292872in}{0.701440in}}%
\pgfpathlineto{\pgfqpoint{3.293738in}{0.676827in}}%
\pgfpathlineto{\pgfqpoint{3.294599in}{0.730013in}}%
\pgfpathlineto{\pgfqpoint{3.295464in}{0.692394in}}%
\pgfpathlineto{\pgfqpoint{3.296325in}{0.819498in}}%
\pgfpathlineto{\pgfqpoint{3.298058in}{0.687595in}}%
\pgfpathlineto{\pgfqpoint{3.298922in}{0.666643in}}%
\pgfpathlineto{\pgfqpoint{3.300649in}{0.746534in}}%
\pgfpathlineto{\pgfqpoint{3.302379in}{0.691809in}}%
\pgfpathlineto{\pgfqpoint{3.303244in}{0.711185in}}%
\pgfpathlineto{\pgfqpoint{3.304108in}{0.665287in}}%
\pgfpathlineto{\pgfqpoint{3.304975in}{0.786569in}}%
\pgfpathlineto{\pgfqpoint{3.305841in}{0.779537in}}%
\pgfpathlineto{\pgfqpoint{3.306706in}{0.656241in}}%
\pgfpathlineto{\pgfqpoint{3.307572in}{0.701627in}}%
\pgfpathlineto{\pgfqpoint{3.308436in}{0.676096in}}%
\pgfpathlineto{\pgfqpoint{3.309302in}{0.717267in}}%
\pgfpathlineto{\pgfqpoint{3.310168in}{0.672433in}}%
\pgfpathlineto{\pgfqpoint{3.313628in}{0.786354in}}%
\pgfpathlineto{\pgfqpoint{3.314493in}{0.671150in}}%
\pgfpathlineto{\pgfqpoint{3.317088in}{0.832540in}}%
\pgfpathlineto{\pgfqpoint{3.317953in}{0.673128in}}%
\pgfpathlineto{\pgfqpoint{3.319683in}{0.762102in}}%
\pgfpathlineto{\pgfqpoint{3.320547in}{0.730598in}}%
\pgfpathlineto{\pgfqpoint{3.321413in}{0.743237in}}%
\pgfpathlineto{\pgfqpoint{3.322275in}{0.786350in}}%
\pgfpathlineto{\pgfqpoint{3.323139in}{0.657926in}}%
\pgfpathlineto{\pgfqpoint{3.324870in}{0.774006in}}%
\pgfpathlineto{\pgfqpoint{3.325734in}{0.750270in}}%
\pgfpathlineto{\pgfqpoint{3.326597in}{0.768070in}}%
\pgfpathlineto{\pgfqpoint{3.327461in}{0.671772in}}%
\pgfpathlineto{\pgfqpoint{3.328325in}{0.675617in}}%
\pgfpathlineto{\pgfqpoint{3.329190in}{0.688074in}}%
\pgfpathlineto{\pgfqpoint{3.330054in}{0.675548in}}%
\pgfpathlineto{\pgfqpoint{3.330917in}{0.686279in}}%
\pgfpathlineto{\pgfqpoint{3.331782in}{0.717779in}}%
\pgfpathlineto{\pgfqpoint{3.332647in}{0.676534in}}%
\pgfpathlineto{\pgfqpoint{3.333513in}{0.706532in}}%
\pgfpathlineto{\pgfqpoint{3.334376in}{0.682284in}}%
\pgfpathlineto{\pgfqpoint{3.336971in}{0.763714in}}%
\pgfpathlineto{\pgfqpoint{3.341293in}{0.672653in}}%
\pgfpathlineto{\pgfqpoint{3.342154in}{0.785254in}}%
\pgfpathlineto{\pgfqpoint{3.343883in}{0.636646in}}%
\pgfpathlineto{\pgfqpoint{3.344746in}{0.704299in}}%
\pgfpathlineto{\pgfqpoint{3.345611in}{0.648291in}}%
\pgfpathlineto{\pgfqpoint{3.346476in}{0.765691in}}%
\pgfpathlineto{\pgfqpoint{3.347340in}{0.738584in}}%
\pgfpathlineto{\pgfqpoint{3.348206in}{0.727191in}}%
\pgfpathlineto{\pgfqpoint{3.349071in}{0.728364in}}%
\pgfpathlineto{\pgfqpoint{3.349936in}{0.685581in}}%
\pgfpathlineto{\pgfqpoint{3.350801in}{0.721661in}}%
\pgfpathlineto{\pgfqpoint{3.351667in}{0.666716in}}%
\pgfpathlineto{\pgfqpoint{3.352533in}{0.743676in}}%
\pgfpathlineto{\pgfqpoint{3.353399in}{0.714921in}}%
\pgfpathlineto{\pgfqpoint{3.354265in}{0.789135in}}%
\pgfpathlineto{\pgfqpoint{3.355130in}{0.662210in}}%
\pgfpathlineto{\pgfqpoint{3.355995in}{0.687522in}}%
\pgfpathlineto{\pgfqpoint{3.356859in}{0.684631in}}%
\pgfpathlineto{\pgfqpoint{3.357724in}{0.675105in}}%
\pgfpathlineto{\pgfqpoint{3.359454in}{0.629280in}}%
\pgfpathlineto{\pgfqpoint{3.360319in}{0.711404in}}%
\pgfpathlineto{\pgfqpoint{3.361184in}{0.678402in}}%
\pgfpathlineto{\pgfqpoint{3.362913in}{0.773235in}}%
\pgfpathlineto{\pgfqpoint{3.365508in}{0.709646in}}%
\pgfpathlineto{\pgfqpoint{3.366374in}{0.702723in}}%
\pgfpathlineto{\pgfqpoint{3.367238in}{0.705216in}}%
\pgfpathlineto{\pgfqpoint{3.368103in}{0.713897in}}%
\pgfpathlineto{\pgfqpoint{3.368968in}{0.698696in}}%
\pgfpathlineto{\pgfqpoint{3.369832in}{0.790085in}}%
\pgfpathlineto{\pgfqpoint{3.370697in}{0.635290in}}%
\pgfpathlineto{\pgfqpoint{3.372429in}{0.748621in}}%
\pgfpathlineto{\pgfqpoint{3.373294in}{0.658255in}}%
\pgfpathlineto{\pgfqpoint{3.375025in}{0.741589in}}%
\pgfpathlineto{\pgfqpoint{3.375889in}{0.705947in}}%
\pgfpathlineto{\pgfqpoint{3.376753in}{0.775947in}}%
\pgfpathlineto{\pgfqpoint{3.377619in}{0.696203in}}%
\pgfpathlineto{\pgfqpoint{3.378486in}{0.697413in}}%
\pgfpathlineto{\pgfqpoint{3.379350in}{0.690819in}}%
\pgfpathlineto{\pgfqpoint{3.380214in}{0.666680in}}%
\pgfpathlineto{\pgfqpoint{3.381080in}{0.732323in}}%
\pgfpathlineto{\pgfqpoint{3.382810in}{0.656939in}}%
\pgfpathlineto{\pgfqpoint{3.383674in}{0.704006in}}%
\pgfpathlineto{\pgfqpoint{3.384537in}{0.680124in}}%
\pgfpathlineto{\pgfqpoint{3.387132in}{0.784044in}}%
\pgfpathlineto{\pgfqpoint{3.388863in}{0.724665in}}%
\pgfpathlineto{\pgfqpoint{3.389728in}{0.756498in}}%
\pgfpathlineto{\pgfqpoint{3.390594in}{0.691005in}}%
\pgfpathlineto{\pgfqpoint{3.391460in}{0.718916in}}%
\pgfpathlineto{\pgfqpoint{3.392325in}{0.713386in}}%
\pgfpathlineto{\pgfqpoint{3.393191in}{0.728295in}}%
\pgfpathlineto{\pgfqpoint{3.394056in}{0.682397in}}%
\pgfpathlineto{\pgfqpoint{3.394921in}{0.737634in}}%
\pgfpathlineto{\pgfqpoint{3.395786in}{0.706020in}}%
\pgfpathlineto{\pgfqpoint{3.396649in}{0.717304in}}%
\pgfpathlineto{\pgfqpoint{3.397513in}{0.751114in}}%
\pgfpathlineto{\pgfqpoint{3.398379in}{0.684484in}}%
\pgfpathlineto{\pgfqpoint{3.399244in}{0.821334in}}%
\pgfpathlineto{\pgfqpoint{3.402704in}{0.681991in}}%
\pgfpathlineto{\pgfqpoint{3.404434in}{0.715400in}}%
\pgfpathlineto{\pgfqpoint{3.406164in}{0.666022in}}%
\pgfpathlineto{\pgfqpoint{3.407893in}{0.698696in}}%
\pgfpathlineto{\pgfqpoint{3.409621in}{0.727670in}}%
\pgfpathlineto{\pgfqpoint{3.410487in}{0.631554in}}%
\pgfpathlineto{\pgfqpoint{3.411351in}{0.747306in}}%
\pgfpathlineto{\pgfqpoint{3.412214in}{0.729172in}}%
\pgfpathlineto{\pgfqpoint{3.413080in}{0.665730in}}%
\pgfpathlineto{\pgfqpoint{3.413944in}{0.752544in}}%
\pgfpathlineto{\pgfqpoint{3.415676in}{0.686279in}}%
\pgfpathlineto{\pgfqpoint{3.418269in}{0.778075in}}%
\pgfpathlineto{\pgfqpoint{3.419134in}{0.752982in}}%
\pgfpathlineto{\pgfqpoint{3.419999in}{0.778294in}}%
\pgfpathlineto{\pgfqpoint{3.420865in}{0.658515in}}%
\pgfpathlineto{\pgfqpoint{3.421731in}{0.664045in}}%
\pgfpathlineto{\pgfqpoint{3.422597in}{0.763421in}}%
\pgfpathlineto{\pgfqpoint{3.423462in}{0.671296in}}%
\pgfpathlineto{\pgfqpoint{3.424327in}{0.692873in}}%
\pgfpathlineto{\pgfqpoint{3.425191in}{0.667195in}}%
\pgfpathlineto{\pgfqpoint{3.426054in}{0.728807in}}%
\pgfpathlineto{\pgfqpoint{3.427782in}{0.671004in}}%
\pgfpathlineto{\pgfqpoint{3.428646in}{0.804191in}}%
\pgfpathlineto{\pgfqpoint{3.430376in}{0.671516in}}%
\pgfpathlineto{\pgfqpoint{3.432107in}{0.768915in}}%
\pgfpathlineto{\pgfqpoint{3.433836in}{0.742360in}}%
\pgfpathlineto{\pgfqpoint{3.434701in}{0.673972in}}%
\pgfpathlineto{\pgfqpoint{3.435566in}{0.822763in}}%
\pgfpathlineto{\pgfqpoint{3.437295in}{0.672653in}}%
\pgfpathlineto{\pgfqpoint{3.438159in}{0.756425in}}%
\pgfpathlineto{\pgfqpoint{3.439025in}{0.677160in}}%
\pgfpathlineto{\pgfqpoint{3.439889in}{0.691444in}}%
\pgfpathlineto{\pgfqpoint{3.440753in}{0.739136in}}%
\pgfpathlineto{\pgfqpoint{3.441619in}{0.718952in}}%
\pgfpathlineto{\pgfqpoint{3.442483in}{0.744740in}}%
\pgfpathlineto{\pgfqpoint{3.443347in}{0.720711in}}%
\pgfpathlineto{\pgfqpoint{3.444212in}{0.736351in}}%
\pgfpathlineto{\pgfqpoint{3.445076in}{0.706719in}}%
\pgfpathlineto{\pgfqpoint{3.445941in}{0.724958in}}%
\pgfpathlineto{\pgfqpoint{3.446805in}{0.655766in}}%
\pgfpathlineto{\pgfqpoint{3.447669in}{0.742726in}}%
\pgfpathlineto{\pgfqpoint{3.448533in}{0.669063in}}%
\pgfpathlineto{\pgfqpoint{3.450259in}{0.761298in}}%
\pgfpathlineto{\pgfqpoint{3.451988in}{0.702910in}}%
\pgfpathlineto{\pgfqpoint{3.452853in}{0.699979in}}%
\pgfpathlineto{\pgfqpoint{3.453718in}{0.682580in}}%
\pgfpathlineto{\pgfqpoint{3.454582in}{0.696462in}}%
\pgfpathlineto{\pgfqpoint{3.455444in}{0.691078in}}%
\pgfpathlineto{\pgfqpoint{3.456310in}{0.718331in}}%
\pgfpathlineto{\pgfqpoint{3.457175in}{0.718039in}}%
\pgfpathlineto{\pgfqpoint{3.458905in}{0.745584in}}%
\pgfpathlineto{\pgfqpoint{3.460636in}{0.675292in}}%
\pgfpathlineto{\pgfqpoint{3.462364in}{0.800528in}}%
\pgfpathlineto{\pgfqpoint{3.463229in}{0.731629in}}%
\pgfpathlineto{\pgfqpoint{3.464094in}{0.740639in}}%
\pgfpathlineto{\pgfqpoint{3.464958in}{0.737634in}}%
\pgfpathlineto{\pgfqpoint{3.465821in}{0.583679in}}%
\pgfpathlineto{\pgfqpoint{3.467552in}{0.726574in}}%
\pgfpathlineto{\pgfqpoint{3.468416in}{0.718624in}}%
\pgfpathlineto{\pgfqpoint{3.469282in}{0.756206in}}%
\pgfpathlineto{\pgfqpoint{3.471877in}{0.711445in}}%
\pgfpathlineto{\pgfqpoint{3.472743in}{0.718404in}}%
\pgfpathlineto{\pgfqpoint{3.473609in}{0.696097in}}%
\pgfpathlineto{\pgfqpoint{3.474471in}{0.711445in}}%
\pgfpathlineto{\pgfqpoint{3.475336in}{0.685804in}}%
\pgfpathlineto{\pgfqpoint{3.476201in}{0.758626in}}%
\pgfpathlineto{\pgfqpoint{3.477067in}{0.743570in}}%
\pgfpathlineto{\pgfqpoint{3.477933in}{0.741995in}}%
\pgfpathlineto{\pgfqpoint{3.478798in}{0.750676in}}%
\pgfpathlineto{\pgfqpoint{3.480525in}{0.711445in}}%
\pgfpathlineto{\pgfqpoint{3.481389in}{0.767161in}}%
\pgfpathlineto{\pgfqpoint{3.482252in}{0.697416in}}%
\pgfpathlineto{\pgfqpoint{3.483118in}{0.809136in}}%
\pgfpathlineto{\pgfqpoint{3.484847in}{0.722323in}}%
\pgfpathlineto{\pgfqpoint{3.486575in}{0.760453in}}%
\pgfpathlineto{\pgfqpoint{3.488304in}{0.711185in}}%
\pgfpathlineto{\pgfqpoint{3.489170in}{0.717706in}}%
\pgfpathlineto{\pgfqpoint{3.490036in}{0.782176in}}%
\pgfpathlineto{\pgfqpoint{3.490901in}{0.762029in}}%
\pgfpathlineto{\pgfqpoint{3.491766in}{0.707340in}}%
\pgfpathlineto{\pgfqpoint{3.492631in}{0.712687in}}%
\pgfpathlineto{\pgfqpoint{3.493495in}{0.747411in}}%
\pgfpathlineto{\pgfqpoint{3.494360in}{0.621001in}}%
\pgfpathlineto{\pgfqpoint{3.496091in}{0.683348in}}%
\pgfpathlineto{\pgfqpoint{3.497819in}{0.676680in}}%
\pgfpathlineto{\pgfqpoint{3.498684in}{0.750599in}}%
\pgfpathlineto{\pgfqpoint{3.499548in}{0.718145in}}%
\pgfpathlineto{\pgfqpoint{3.500413in}{0.742174in}}%
\pgfpathlineto{\pgfqpoint{3.502142in}{0.680051in}}%
\pgfpathlineto{\pgfqpoint{3.503006in}{0.777011in}}%
\pgfpathlineto{\pgfqpoint{3.503870in}{0.715692in}}%
\pgfpathlineto{\pgfqpoint{3.504734in}{0.765033in}}%
\pgfpathlineto{\pgfqpoint{3.505598in}{0.711551in}}%
\pgfpathlineto{\pgfqpoint{3.507329in}{0.743457in}}%
\pgfpathlineto{\pgfqpoint{3.508192in}{0.682320in}}%
\pgfpathlineto{\pgfqpoint{3.509057in}{0.774737in}}%
\pgfpathlineto{\pgfqpoint{3.509923in}{0.769280in}}%
\pgfpathlineto{\pgfqpoint{3.511653in}{0.706240in}}%
\pgfpathlineto{\pgfqpoint{3.513384in}{0.741589in}}%
\pgfpathlineto{\pgfqpoint{3.515979in}{0.659392in}}%
\pgfpathlineto{\pgfqpoint{3.516844in}{0.662835in}}%
\pgfpathlineto{\pgfqpoint{3.517709in}{0.657633in}}%
\pgfpathlineto{\pgfqpoint{3.519437in}{0.724081in}}%
\pgfpathlineto{\pgfqpoint{3.521166in}{0.678037in}}%
\pgfpathlineto{\pgfqpoint{3.522897in}{0.765216in}}%
\pgfpathlineto{\pgfqpoint{3.523762in}{0.713532in}}%
\pgfpathlineto{\pgfqpoint{3.524627in}{0.743091in}}%
\pgfpathlineto{\pgfqpoint{3.526355in}{0.672141in}}%
\pgfpathlineto{\pgfqpoint{3.527220in}{0.743607in}}%
\pgfpathlineto{\pgfqpoint{3.528086in}{0.684411in}}%
\pgfpathlineto{\pgfqpoint{3.528952in}{0.706865in}}%
\pgfpathlineto{\pgfqpoint{3.529817in}{0.702431in}}%
\pgfpathlineto{\pgfqpoint{3.530682in}{0.688220in}}%
\pgfpathlineto{\pgfqpoint{3.531546in}{0.740013in}}%
\pgfpathlineto{\pgfqpoint{3.532412in}{0.679722in}}%
\pgfpathlineto{\pgfqpoint{3.534143in}{0.740598in}}%
\pgfpathlineto{\pgfqpoint{3.535870in}{0.696129in}}%
\pgfpathlineto{\pgfqpoint{3.536735in}{0.596278in}}%
\pgfpathlineto{\pgfqpoint{3.540196in}{0.778473in}}%
\pgfpathlineto{\pgfqpoint{3.541923in}{0.655141in}}%
\pgfpathlineto{\pgfqpoint{3.542787in}{0.754956in}}%
\pgfpathlineto{\pgfqpoint{3.543652in}{0.694408in}}%
\pgfpathlineto{\pgfqpoint{3.545379in}{0.762832in}}%
\pgfpathlineto{\pgfqpoint{3.546245in}{0.714336in}}%
\pgfpathlineto{\pgfqpoint{3.547110in}{0.796899in}}%
\pgfpathlineto{\pgfqpoint{3.547975in}{0.792319in}}%
\pgfpathlineto{\pgfqpoint{3.550568in}{0.670087in}}%
\pgfpathlineto{\pgfqpoint{3.552299in}{0.739315in}}%
\pgfpathlineto{\pgfqpoint{3.553165in}{0.692613in}}%
\pgfpathlineto{\pgfqpoint{3.554030in}{0.737447in}}%
\pgfpathlineto{\pgfqpoint{3.554896in}{0.691842in}}%
\pgfpathlineto{\pgfqpoint{3.555761in}{0.738621in}}%
\pgfpathlineto{\pgfqpoint{3.556626in}{0.702212in}}%
\pgfpathlineto{\pgfqpoint{3.557492in}{0.718364in}}%
\pgfpathlineto{\pgfqpoint{3.558357in}{0.714921in}}%
\pgfpathlineto{\pgfqpoint{3.559222in}{0.714774in}}%
\pgfpathlineto{\pgfqpoint{3.560951in}{0.657597in}}%
\pgfpathlineto{\pgfqpoint{3.561816in}{0.724738in}}%
\pgfpathlineto{\pgfqpoint{3.562682in}{0.707669in}}%
\pgfpathlineto{\pgfqpoint{3.563548in}{0.630929in}}%
\pgfpathlineto{\pgfqpoint{3.564412in}{0.682942in}}%
\pgfpathlineto{\pgfqpoint{3.565276in}{0.665178in}}%
\pgfpathlineto{\pgfqpoint{3.567003in}{0.745836in}}%
\pgfpathlineto{\pgfqpoint{3.567868in}{0.694664in}}%
\pgfpathlineto{\pgfqpoint{3.568733in}{0.715067in}}%
\pgfpathlineto{\pgfqpoint{3.569599in}{0.671662in}}%
\pgfpathlineto{\pgfqpoint{3.571329in}{0.721734in}}%
\pgfpathlineto{\pgfqpoint{3.572193in}{0.633860in}}%
\pgfpathlineto{\pgfqpoint{3.573058in}{0.780820in}}%
\pgfpathlineto{\pgfqpoint{3.573923in}{0.636678in}}%
\pgfpathlineto{\pgfqpoint{3.575651in}{0.744082in}}%
\pgfpathlineto{\pgfqpoint{3.576516in}{0.640674in}}%
\pgfpathlineto{\pgfqpoint{3.577381in}{0.732615in}}%
\pgfpathlineto{\pgfqpoint{3.579112in}{0.691882in}}%
\pgfpathlineto{\pgfqpoint{3.579978in}{0.683567in}}%
\pgfpathlineto{\pgfqpoint{3.580842in}{0.687635in}}%
\pgfpathlineto{\pgfqpoint{3.581708in}{0.746461in}}%
\pgfpathlineto{\pgfqpoint{3.582573in}{0.659392in}}%
\pgfpathlineto{\pgfqpoint{3.583437in}{0.740087in}}%
\pgfpathlineto{\pgfqpoint{3.584300in}{0.665145in}}%
\pgfpathlineto{\pgfqpoint{3.585164in}{0.781957in}}%
\pgfpathlineto{\pgfqpoint{3.586028in}{0.679393in}}%
\pgfpathlineto{\pgfqpoint{3.586892in}{0.753973in}}%
\pgfpathlineto{\pgfqpoint{3.587756in}{0.687343in}}%
\pgfpathlineto{\pgfqpoint{3.588620in}{0.774047in}}%
\pgfpathlineto{\pgfqpoint{3.589485in}{0.762252in}}%
\pgfpathlineto{\pgfqpoint{3.591214in}{0.760786in}}%
\pgfpathlineto{\pgfqpoint{3.592942in}{0.710349in}}%
\pgfpathlineto{\pgfqpoint{3.593804in}{0.750237in}}%
\pgfpathlineto{\pgfqpoint{3.594668in}{0.703828in}}%
\pgfpathlineto{\pgfqpoint{3.595532in}{0.750822in}}%
\pgfpathlineto{\pgfqpoint{3.596397in}{0.736355in}}%
\pgfpathlineto{\pgfqpoint{3.597262in}{0.693864in}}%
\pgfpathlineto{\pgfqpoint{3.598992in}{0.714852in}}%
\pgfpathlineto{\pgfqpoint{3.599857in}{0.711664in}}%
\pgfpathlineto{\pgfqpoint{3.600722in}{0.683900in}}%
\pgfpathlineto{\pgfqpoint{3.603316in}{0.811187in}}%
\pgfpathlineto{\pgfqpoint{3.605045in}{0.732835in}}%
\pgfpathlineto{\pgfqpoint{3.607639in}{0.697047in}}%
\pgfpathlineto{\pgfqpoint{3.610234in}{0.760380in}}%
\pgfpathlineto{\pgfqpoint{3.611099in}{0.741808in}}%
\pgfpathlineto{\pgfqpoint{3.612827in}{0.707779in}}%
\pgfpathlineto{\pgfqpoint{3.613692in}{0.808584in}}%
\pgfpathlineto{\pgfqpoint{3.614557in}{0.801698in}}%
\pgfpathlineto{\pgfqpoint{3.615422in}{0.682028in}}%
\pgfpathlineto{\pgfqpoint{3.616289in}{0.743310in}}%
\pgfpathlineto{\pgfqpoint{3.618017in}{0.677890in}}%
\pgfpathlineto{\pgfqpoint{3.618881in}{0.786496in}}%
\pgfpathlineto{\pgfqpoint{3.619745in}{0.755654in}}%
\pgfpathlineto{\pgfqpoint{3.621474in}{0.686206in}}%
\pgfpathlineto{\pgfqpoint{3.622340in}{0.692617in}}%
\pgfpathlineto{\pgfqpoint{3.623205in}{0.754631in}}%
\pgfpathlineto{\pgfqpoint{3.624067in}{0.702066in}}%
\pgfpathlineto{\pgfqpoint{3.624932in}{0.713751in}}%
\pgfpathlineto{\pgfqpoint{3.625798in}{0.673863in}}%
\pgfpathlineto{\pgfqpoint{3.626663in}{0.680603in}}%
\pgfpathlineto{\pgfqpoint{3.628394in}{0.729724in}}%
\pgfpathlineto{\pgfqpoint{3.629258in}{0.814228in}}%
\pgfpathlineto{\pgfqpoint{3.630123in}{0.763644in}}%
\pgfpathlineto{\pgfqpoint{3.630987in}{0.792432in}}%
\pgfpathlineto{\pgfqpoint{3.632716in}{0.698696in}}%
\pgfpathlineto{\pgfqpoint{3.633580in}{0.701627in}}%
\pgfpathlineto{\pgfqpoint{3.634446in}{0.722359in}}%
\pgfpathlineto{\pgfqpoint{3.635312in}{0.714961in}}%
\pgfpathlineto{\pgfqpoint{3.637043in}{0.756243in}}%
\pgfpathlineto{\pgfqpoint{3.638773in}{0.704997in}}%
\pgfpathlineto{\pgfqpoint{3.639640in}{0.610091in}}%
\pgfpathlineto{\pgfqpoint{3.641372in}{0.751334in}}%
\pgfpathlineto{\pgfqpoint{3.642235in}{0.720345in}}%
\pgfpathlineto{\pgfqpoint{3.643962in}{0.768001in}}%
\pgfpathlineto{\pgfqpoint{3.644829in}{0.767892in}}%
\pgfpathlineto{\pgfqpoint{3.645693in}{0.691078in}}%
\pgfpathlineto{\pgfqpoint{3.647418in}{0.752690in}}%
\pgfpathlineto{\pgfqpoint{3.648283in}{0.739794in}}%
\pgfpathlineto{\pgfqpoint{3.649149in}{0.706093in}}%
\pgfpathlineto{\pgfqpoint{3.650016in}{0.735986in}}%
\pgfpathlineto{\pgfqpoint{3.651747in}{0.719354in}}%
\pgfpathlineto{\pgfqpoint{3.652611in}{0.739867in}}%
\pgfpathlineto{\pgfqpoint{3.653476in}{0.726204in}}%
\pgfpathlineto{\pgfqpoint{3.654343in}{0.755215in}}%
\pgfpathlineto{\pgfqpoint{3.655208in}{0.742100in}}%
\pgfpathlineto{\pgfqpoint{3.656074in}{0.680599in}}%
\pgfpathlineto{\pgfqpoint{3.656940in}{0.681553in}}%
\pgfpathlineto{\pgfqpoint{3.657803in}{0.746754in}}%
\pgfpathlineto{\pgfqpoint{3.658669in}{0.741516in}}%
\pgfpathlineto{\pgfqpoint{3.659533in}{0.738146in}}%
\pgfpathlineto{\pgfqpoint{3.661264in}{0.702650in}}%
\pgfpathlineto{\pgfqpoint{3.662127in}{0.777596in}}%
\pgfpathlineto{\pgfqpoint{3.662989in}{0.696020in}}%
\pgfpathlineto{\pgfqpoint{3.663855in}{0.716496in}}%
\pgfpathlineto{\pgfqpoint{3.664721in}{0.722651in}}%
\pgfpathlineto{\pgfqpoint{3.665587in}{0.745032in}}%
\pgfpathlineto{\pgfqpoint{3.666450in}{0.682284in}}%
\pgfpathlineto{\pgfqpoint{3.667313in}{0.714555in}}%
\pgfpathlineto{\pgfqpoint{3.668179in}{0.649135in}}%
\pgfpathlineto{\pgfqpoint{3.669045in}{0.734703in}}%
\pgfpathlineto{\pgfqpoint{3.670775in}{0.637303in}}%
\pgfpathlineto{\pgfqpoint{3.671641in}{0.681187in}}%
\pgfpathlineto{\pgfqpoint{3.672504in}{0.665730in}}%
\pgfpathlineto{\pgfqpoint{3.674235in}{0.727962in}}%
\pgfpathlineto{\pgfqpoint{3.675101in}{0.611845in}}%
\pgfpathlineto{\pgfqpoint{3.676831in}{0.750822in}}%
\pgfpathlineto{\pgfqpoint{3.678563in}{0.745032in}}%
\pgfpathlineto{\pgfqpoint{3.679428in}{0.729538in}}%
\pgfpathlineto{\pgfqpoint{3.680294in}{0.732063in}}%
\pgfpathlineto{\pgfqpoint{3.681160in}{0.661479in}}%
\pgfpathlineto{\pgfqpoint{3.682891in}{0.746973in}}%
\pgfpathlineto{\pgfqpoint{3.683756in}{0.688732in}}%
\pgfpathlineto{\pgfqpoint{3.685485in}{0.742872in}}%
\pgfpathlineto{\pgfqpoint{3.687215in}{0.628622in}}%
\pgfpathlineto{\pgfqpoint{3.688079in}{0.735766in}}%
\pgfpathlineto{\pgfqpoint{3.688943in}{0.689649in}}%
\pgfpathlineto{\pgfqpoint{3.689808in}{0.695179in}}%
\pgfpathlineto{\pgfqpoint{3.692401in}{0.763677in}}%
\pgfpathlineto{\pgfqpoint{3.693266in}{0.671516in}}%
\pgfpathlineto{\pgfqpoint{3.694131in}{0.710710in}}%
\pgfpathlineto{\pgfqpoint{3.694996in}{0.703349in}}%
\pgfpathlineto{\pgfqpoint{3.695860in}{0.710308in}}%
\pgfpathlineto{\pgfqpoint{3.697589in}{0.782761in}}%
\pgfpathlineto{\pgfqpoint{3.698452in}{0.692650in}}%
\pgfpathlineto{\pgfqpoint{3.700181in}{0.762540in}}%
\pgfpathlineto{\pgfqpoint{3.701046in}{0.725839in}}%
\pgfpathlineto{\pgfqpoint{3.701911in}{0.785546in}}%
\pgfpathlineto{\pgfqpoint{3.704508in}{0.676242in}}%
\pgfpathlineto{\pgfqpoint{3.705373in}{0.763823in}}%
\pgfpathlineto{\pgfqpoint{3.706238in}{0.757083in}}%
\pgfpathlineto{\pgfqpoint{3.707967in}{0.666790in}}%
\pgfpathlineto{\pgfqpoint{3.708833in}{0.669794in}}%
\pgfpathlineto{\pgfqpoint{3.709698in}{0.704080in}}%
\pgfpathlineto{\pgfqpoint{3.710563in}{0.693677in}}%
\pgfpathlineto{\pgfqpoint{3.711428in}{0.737269in}}%
\pgfpathlineto{\pgfqpoint{3.712293in}{0.715911in}}%
\pgfpathlineto{\pgfqpoint{3.713157in}{0.788733in}}%
\pgfpathlineto{\pgfqpoint{3.714886in}{0.671589in}}%
\pgfpathlineto{\pgfqpoint{3.716611in}{0.706459in}}%
\pgfpathlineto{\pgfqpoint{3.717477in}{0.684261in}}%
\pgfpathlineto{\pgfqpoint{3.718343in}{0.749352in}}%
\pgfpathlineto{\pgfqpoint{3.719209in}{0.732356in}}%
\pgfpathlineto{\pgfqpoint{3.720938in}{0.760672in}}%
\pgfpathlineto{\pgfqpoint{3.721802in}{0.690380in}}%
\pgfpathlineto{\pgfqpoint{3.724396in}{0.774591in}}%
\pgfpathlineto{\pgfqpoint{3.726126in}{0.730561in}}%
\pgfpathlineto{\pgfqpoint{3.727855in}{0.733566in}}%
\pgfpathlineto{\pgfqpoint{3.728721in}{0.716350in}}%
\pgfpathlineto{\pgfqpoint{3.729587in}{0.766531in}}%
\pgfpathlineto{\pgfqpoint{3.730452in}{0.703162in}}%
\pgfpathlineto{\pgfqpoint{3.731315in}{0.721442in}}%
\pgfpathlineto{\pgfqpoint{3.732179in}{0.719391in}}%
\pgfpathlineto{\pgfqpoint{3.733044in}{0.678841in}}%
\pgfpathlineto{\pgfqpoint{3.734774in}{0.741735in}}%
\pgfpathlineto{\pgfqpoint{3.735640in}{0.721880in}}%
\pgfpathlineto{\pgfqpoint{3.736506in}{0.646167in}}%
\pgfpathlineto{\pgfqpoint{3.738238in}{0.721222in}}%
\pgfpathlineto{\pgfqpoint{3.739103in}{0.657305in}}%
\pgfpathlineto{\pgfqpoint{3.739967in}{0.762394in}}%
\pgfpathlineto{\pgfqpoint{3.742560in}{0.678402in}}%
\pgfpathlineto{\pgfqpoint{3.744291in}{0.723675in}}%
\pgfpathlineto{\pgfqpoint{3.745156in}{0.654885in}}%
\pgfpathlineto{\pgfqpoint{3.746022in}{0.757266in}}%
\pgfpathlineto{\pgfqpoint{3.746887in}{0.731406in}}%
\pgfpathlineto{\pgfqpoint{3.747753in}{0.687229in}}%
\pgfpathlineto{\pgfqpoint{3.748618in}{0.785729in}}%
\pgfpathlineto{\pgfqpoint{3.750349in}{0.722286in}}%
\pgfpathlineto{\pgfqpoint{3.751213in}{0.730455in}}%
\pgfpathlineto{\pgfqpoint{3.752078in}{0.723788in}}%
\pgfpathlineto{\pgfqpoint{3.752944in}{0.682909in}}%
\pgfpathlineto{\pgfqpoint{3.754676in}{0.739648in}}%
\pgfpathlineto{\pgfqpoint{3.755541in}{0.674411in}}%
\pgfpathlineto{\pgfqpoint{3.756406in}{0.759576in}}%
\pgfpathlineto{\pgfqpoint{3.757272in}{0.695106in}}%
\pgfpathlineto{\pgfqpoint{3.759004in}{0.728109in}}%
\pgfpathlineto{\pgfqpoint{3.760734in}{0.722542in}}%
\pgfpathlineto{\pgfqpoint{3.762463in}{0.678914in}}%
\pgfpathlineto{\pgfqpoint{3.764191in}{0.779317in}}%
\pgfpathlineto{\pgfqpoint{3.765055in}{0.682836in}}%
\pgfpathlineto{\pgfqpoint{3.765921in}{0.795802in}}%
\pgfpathlineto{\pgfqpoint{3.766786in}{0.677525in}}%
\pgfpathlineto{\pgfqpoint{3.768513in}{0.749685in}}%
\pgfpathlineto{\pgfqpoint{3.769378in}{0.710674in}}%
\pgfpathlineto{\pgfqpoint{3.771970in}{0.766203in}}%
\pgfpathlineto{\pgfqpoint{3.772835in}{0.712541in}}%
\pgfpathlineto{\pgfqpoint{3.773700in}{0.799830in}}%
\pgfpathlineto{\pgfqpoint{3.775427in}{0.719866in}}%
\pgfpathlineto{\pgfqpoint{3.776290in}{0.778806in}}%
\pgfpathlineto{\pgfqpoint{3.778019in}{0.728328in}}%
\pgfpathlineto{\pgfqpoint{3.778884in}{0.757156in}}%
\pgfpathlineto{\pgfqpoint{3.780615in}{0.685215in}}%
\pgfpathlineto{\pgfqpoint{3.781479in}{0.736132in}}%
\pgfpathlineto{\pgfqpoint{3.782344in}{0.695399in}}%
\pgfpathlineto{\pgfqpoint{3.783208in}{0.743676in}}%
\pgfpathlineto{\pgfqpoint{3.784074in}{0.682690in}}%
\pgfpathlineto{\pgfqpoint{3.785802in}{0.768549in}}%
\pgfpathlineto{\pgfqpoint{3.786665in}{0.686352in}}%
\pgfpathlineto{\pgfqpoint{3.787530in}{0.783386in}}%
\pgfpathlineto{\pgfqpoint{3.788396in}{0.711957in}}%
\pgfpathlineto{\pgfqpoint{3.789261in}{0.727012in}}%
\pgfpathlineto{\pgfqpoint{3.791853in}{0.658588in}}%
\pgfpathlineto{\pgfqpoint{3.792718in}{0.758220in}}%
\pgfpathlineto{\pgfqpoint{3.793584in}{0.694668in}}%
\pgfpathlineto{\pgfqpoint{3.794449in}{0.740160in}}%
\pgfpathlineto{\pgfqpoint{3.795314in}{0.654154in}}%
\pgfpathlineto{\pgfqpoint{3.797911in}{0.751845in}}%
\pgfpathlineto{\pgfqpoint{3.798777in}{0.648952in}}%
\pgfpathlineto{\pgfqpoint{3.799640in}{0.698330in}}%
\pgfpathlineto{\pgfqpoint{3.800503in}{0.668844in}}%
\pgfpathlineto{\pgfqpoint{3.801369in}{0.674740in}}%
\pgfpathlineto{\pgfqpoint{3.802234in}{0.695472in}}%
\pgfpathlineto{\pgfqpoint{3.803099in}{0.751037in}}%
\pgfpathlineto{\pgfqpoint{3.803964in}{0.696755in}}%
\pgfpathlineto{\pgfqpoint{3.804830in}{0.760234in}}%
\pgfpathlineto{\pgfqpoint{3.806562in}{0.672287in}}%
\pgfpathlineto{\pgfqpoint{3.808293in}{0.744411in}}%
\pgfpathlineto{\pgfqpoint{3.809157in}{0.695618in}}%
\pgfpathlineto{\pgfqpoint{3.810020in}{0.737009in}}%
\pgfpathlineto{\pgfqpoint{3.810885in}{0.642103in}}%
\pgfpathlineto{\pgfqpoint{3.811751in}{0.706240in}}%
\pgfpathlineto{\pgfqpoint{3.812616in}{0.694444in}}%
\pgfpathlineto{\pgfqpoint{3.813481in}{0.611626in}}%
\pgfpathlineto{\pgfqpoint{3.815209in}{0.726350in}}%
\pgfpathlineto{\pgfqpoint{3.816076in}{0.716423in}}%
\pgfpathlineto{\pgfqpoint{3.816939in}{0.634445in}}%
\pgfpathlineto{\pgfqpoint{3.817803in}{0.722725in}}%
\pgfpathlineto{\pgfqpoint{3.818667in}{0.705728in}}%
\pgfpathlineto{\pgfqpoint{3.819532in}{0.726314in}}%
\pgfpathlineto{\pgfqpoint{3.820398in}{0.705655in}}%
\pgfpathlineto{\pgfqpoint{3.821264in}{0.757229in}}%
\pgfpathlineto{\pgfqpoint{3.822129in}{0.708660in}}%
\pgfpathlineto{\pgfqpoint{3.822992in}{0.719907in}}%
\pgfpathlineto{\pgfqpoint{3.823858in}{0.717048in}}%
\pgfpathlineto{\pgfqpoint{3.824723in}{0.751187in}}%
\pgfpathlineto{\pgfqpoint{3.825589in}{0.740858in}}%
\pgfpathlineto{\pgfqpoint{3.826455in}{0.692873in}}%
\pgfpathlineto{\pgfqpoint{3.828185in}{0.771408in}}%
\pgfpathlineto{\pgfqpoint{3.829917in}{0.672214in}}%
\pgfpathlineto{\pgfqpoint{3.830781in}{0.742580in}}%
\pgfpathlineto{\pgfqpoint{3.831644in}{0.698001in}}%
\pgfpathlineto{\pgfqpoint{3.832509in}{0.723861in}}%
\pgfpathlineto{\pgfqpoint{3.833375in}{0.790930in}}%
\pgfpathlineto{\pgfqpoint{3.834240in}{0.740013in}}%
\pgfpathlineto{\pgfqpoint{3.835971in}{0.769905in}}%
\pgfpathlineto{\pgfqpoint{3.836836in}{0.739575in}}%
\pgfpathlineto{\pgfqpoint{3.837698in}{0.789793in}}%
\pgfpathlineto{\pgfqpoint{3.838562in}{0.652286in}}%
\pgfpathlineto{\pgfqpoint{3.839427in}{0.740379in}}%
\pgfpathlineto{\pgfqpoint{3.840290in}{0.715838in}}%
\pgfpathlineto{\pgfqpoint{3.841155in}{0.744520in}}%
\pgfpathlineto{\pgfqpoint{3.843749in}{0.710272in}}%
\pgfpathlineto{\pgfqpoint{3.844612in}{0.730748in}}%
\pgfpathlineto{\pgfqpoint{3.845478in}{0.672360in}}%
\pgfpathlineto{\pgfqpoint{3.846342in}{0.726976in}}%
\pgfpathlineto{\pgfqpoint{3.848071in}{0.701919in}}%
\pgfpathlineto{\pgfqpoint{3.848933in}{0.746607in}}%
\pgfpathlineto{\pgfqpoint{3.849796in}{0.714263in}}%
\pgfpathlineto{\pgfqpoint{3.850662in}{0.715327in}}%
\pgfpathlineto{\pgfqpoint{3.851528in}{0.691444in}}%
\pgfpathlineto{\pgfqpoint{3.852393in}{0.767599in}}%
\pgfpathlineto{\pgfqpoint{3.853258in}{0.725218in}}%
\pgfpathlineto{\pgfqpoint{3.854122in}{0.758147in}}%
\pgfpathlineto{\pgfqpoint{3.855851in}{0.661519in}}%
\pgfpathlineto{\pgfqpoint{3.856715in}{0.666976in}}%
\pgfpathlineto{\pgfqpoint{3.857580in}{0.747964in}}%
\pgfpathlineto{\pgfqpoint{3.858443in}{0.660090in}}%
\pgfpathlineto{\pgfqpoint{3.859307in}{0.677452in}}%
\pgfpathlineto{\pgfqpoint{3.860171in}{0.734410in}}%
\pgfpathlineto{\pgfqpoint{3.861901in}{0.673863in}}%
\pgfpathlineto{\pgfqpoint{3.862767in}{0.690603in}}%
\pgfpathlineto{\pgfqpoint{3.863631in}{0.756279in}}%
\pgfpathlineto{\pgfqpoint{3.866223in}{0.659319in}}%
\pgfpathlineto{\pgfqpoint{3.867087in}{0.693092in}}%
\pgfpathlineto{\pgfqpoint{3.867951in}{0.676461in}}%
\pgfpathlineto{\pgfqpoint{3.870542in}{0.756425in}}%
\pgfpathlineto{\pgfqpoint{3.872272in}{0.675876in}}%
\pgfpathlineto{\pgfqpoint{3.873137in}{0.732835in}}%
\pgfpathlineto{\pgfqpoint{3.874002in}{0.654300in}}%
\pgfpathlineto{\pgfqpoint{3.874867in}{0.719172in}}%
\pgfpathlineto{\pgfqpoint{3.875730in}{0.694595in}}%
\pgfpathlineto{\pgfqpoint{3.877459in}{0.715400in}}%
\pgfpathlineto{\pgfqpoint{3.878324in}{0.783751in}}%
\pgfpathlineto{\pgfqpoint{3.879189in}{0.689174in}}%
\pgfpathlineto{\pgfqpoint{3.880052in}{0.764887in}}%
\pgfpathlineto{\pgfqpoint{3.880914in}{0.716058in}}%
\pgfpathlineto{\pgfqpoint{3.881778in}{0.720272in}}%
\pgfpathlineto{\pgfqpoint{3.882643in}{0.720491in}}%
\pgfpathlineto{\pgfqpoint{3.883507in}{0.748329in}}%
\pgfpathlineto{\pgfqpoint{3.884372in}{0.647158in}}%
\pgfpathlineto{\pgfqpoint{3.885238in}{0.715619in}}%
\pgfpathlineto{\pgfqpoint{3.886103in}{0.696828in}}%
\pgfpathlineto{\pgfqpoint{3.886970in}{0.709870in}}%
\pgfpathlineto{\pgfqpoint{3.887836in}{0.751114in}}%
\pgfpathlineto{\pgfqpoint{3.888702in}{0.701042in}}%
\pgfpathlineto{\pgfqpoint{3.890434in}{0.728109in}}%
\pgfpathlineto{\pgfqpoint{3.891300in}{0.685325in}}%
\pgfpathlineto{\pgfqpoint{3.892166in}{0.690088in}}%
\pgfpathlineto{\pgfqpoint{3.893031in}{0.736205in}}%
\pgfpathlineto{\pgfqpoint{3.893896in}{0.653057in}}%
\pgfpathlineto{\pgfqpoint{3.895627in}{0.760161in}}%
\pgfpathlineto{\pgfqpoint{3.896493in}{0.736863in}}%
\pgfpathlineto{\pgfqpoint{3.897360in}{0.619576in}}%
\pgfpathlineto{\pgfqpoint{3.899091in}{0.711737in}}%
\pgfpathlineto{\pgfqpoint{3.899958in}{0.711591in}}%
\pgfpathlineto{\pgfqpoint{3.900822in}{0.707084in}}%
\pgfpathlineto{\pgfqpoint{3.901689in}{0.684777in}}%
\pgfpathlineto{\pgfqpoint{3.902555in}{0.739648in}}%
\pgfpathlineto{\pgfqpoint{3.903419in}{0.676315in}}%
\pgfpathlineto{\pgfqpoint{3.904283in}{0.739502in}}%
\pgfpathlineto{\pgfqpoint{3.905147in}{0.688732in}}%
\pgfpathlineto{\pgfqpoint{3.906011in}{0.692873in}}%
\pgfpathlineto{\pgfqpoint{3.906875in}{0.684777in}}%
\pgfpathlineto{\pgfqpoint{3.907739in}{0.750456in}}%
\pgfpathlineto{\pgfqpoint{3.910332in}{0.654925in}}%
\pgfpathlineto{\pgfqpoint{3.911197in}{0.658515in}}%
\pgfpathlineto{\pgfqpoint{3.912061in}{0.709870in}}%
\pgfpathlineto{\pgfqpoint{3.912926in}{0.682324in}}%
\pgfpathlineto{\pgfqpoint{3.913792in}{0.701115in}}%
\pgfpathlineto{\pgfqpoint{3.914657in}{0.697599in}}%
\pgfpathlineto{\pgfqpoint{3.916387in}{0.701846in}}%
\pgfpathlineto{\pgfqpoint{3.917253in}{0.667232in}}%
\pgfpathlineto{\pgfqpoint{3.918984in}{0.733054in}}%
\pgfpathlineto{\pgfqpoint{3.919849in}{0.727487in}}%
\pgfpathlineto{\pgfqpoint{3.920712in}{0.705363in}}%
\pgfpathlineto{\pgfqpoint{3.921578in}{0.753201in}}%
\pgfpathlineto{\pgfqpoint{3.922443in}{0.659172in}}%
\pgfpathlineto{\pgfqpoint{3.923309in}{0.706865in}}%
\pgfpathlineto{\pgfqpoint{3.925036in}{0.660675in}}%
\pgfpathlineto{\pgfqpoint{3.926765in}{0.747086in}}%
\pgfpathlineto{\pgfqpoint{3.927629in}{0.702983in}}%
\pgfpathlineto{\pgfqpoint{3.928493in}{0.766389in}}%
\pgfpathlineto{\pgfqpoint{3.929358in}{0.647673in}}%
\pgfpathlineto{\pgfqpoint{3.930224in}{0.720418in}}%
\pgfpathlineto{\pgfqpoint{3.931951in}{0.663716in}}%
\pgfpathlineto{\pgfqpoint{3.932817in}{0.759503in}}%
\pgfpathlineto{\pgfqpoint{3.933682in}{0.737195in}}%
\pgfpathlineto{\pgfqpoint{3.934547in}{0.738625in}}%
\pgfpathlineto{\pgfqpoint{3.935411in}{0.737342in}}%
\pgfpathlineto{\pgfqpoint{3.936274in}{0.752434in}}%
\pgfpathlineto{\pgfqpoint{3.937139in}{0.744082in}}%
\pgfpathlineto{\pgfqpoint{3.938004in}{0.699613in}}%
\pgfpathlineto{\pgfqpoint{3.938870in}{0.740237in}}%
\pgfpathlineto{\pgfqpoint{3.939731in}{0.705915in}}%
\pgfpathlineto{\pgfqpoint{3.940595in}{0.738990in}}%
\pgfpathlineto{\pgfqpoint{3.941461in}{0.717820in}}%
\pgfpathlineto{\pgfqpoint{3.943191in}{0.788039in}}%
\pgfpathlineto{\pgfqpoint{3.944057in}{0.787747in}}%
\pgfpathlineto{\pgfqpoint{3.944922in}{0.764010in}}%
\pgfpathlineto{\pgfqpoint{3.945788in}{0.659944in}}%
\pgfpathlineto{\pgfqpoint{3.946653in}{0.673351in}}%
\pgfpathlineto{\pgfqpoint{3.947519in}{0.701335in}}%
\pgfpathlineto{\pgfqpoint{3.948385in}{0.700421in}}%
\pgfpathlineto{\pgfqpoint{3.949249in}{0.711079in}}%
\pgfpathlineto{\pgfqpoint{3.950114in}{0.744926in}}%
\pgfpathlineto{\pgfqpoint{3.950979in}{0.709102in}}%
\pgfpathlineto{\pgfqpoint{3.951843in}{0.714523in}}%
\pgfpathlineto{\pgfqpoint{3.952708in}{0.760640in}}%
\pgfpathlineto{\pgfqpoint{3.953573in}{0.731665in}}%
\pgfpathlineto{\pgfqpoint{3.954436in}{0.777709in}}%
\pgfpathlineto{\pgfqpoint{3.955301in}{0.736538in}}%
\pgfpathlineto{\pgfqpoint{3.956166in}{0.764156in}}%
\pgfpathlineto{\pgfqpoint{3.957031in}{0.751666in}}%
\pgfpathlineto{\pgfqpoint{3.957894in}{0.719614in}}%
\pgfpathlineto{\pgfqpoint{3.958759in}{0.723131in}}%
\pgfpathlineto{\pgfqpoint{3.959623in}{0.741995in}}%
\pgfpathlineto{\pgfqpoint{3.960486in}{0.694119in}}%
\pgfpathlineto{\pgfqpoint{3.962216in}{0.772910in}}%
\pgfpathlineto{\pgfqpoint{3.963081in}{0.748954in}}%
\pgfpathlineto{\pgfqpoint{3.963945in}{0.764083in}}%
\pgfpathlineto{\pgfqpoint{3.964811in}{0.726464in}}%
\pgfpathlineto{\pgfqpoint{3.965676in}{0.790418in}}%
\pgfpathlineto{\pgfqpoint{3.966542in}{0.776426in}}%
\pgfpathlineto{\pgfqpoint{3.967407in}{0.649322in}}%
\pgfpathlineto{\pgfqpoint{3.968272in}{0.698590in}}%
\pgfpathlineto{\pgfqpoint{3.969137in}{0.697599in}}%
\pgfpathlineto{\pgfqpoint{3.970867in}{0.745365in}}%
\pgfpathlineto{\pgfqpoint{3.971732in}{0.770969in}}%
\pgfpathlineto{\pgfqpoint{3.972596in}{0.739615in}}%
\pgfpathlineto{\pgfqpoint{3.973460in}{0.799871in}}%
\pgfpathlineto{\pgfqpoint{3.974325in}{0.745036in}}%
\pgfpathlineto{\pgfqpoint{3.975191in}{0.762434in}}%
\pgfpathlineto{\pgfqpoint{3.976055in}{0.755037in}}%
\pgfpathlineto{\pgfqpoint{3.976921in}{0.709468in}}%
\pgfpathlineto{\pgfqpoint{3.977787in}{0.735035in}}%
\pgfpathlineto{\pgfqpoint{3.978650in}{0.733606in}}%
\pgfpathlineto{\pgfqpoint{3.979516in}{0.649212in}}%
\pgfpathlineto{\pgfqpoint{3.981249in}{0.748589in}}%
\pgfpathlineto{\pgfqpoint{3.982981in}{0.690932in}}%
\pgfpathlineto{\pgfqpoint{3.983846in}{0.699394in}}%
\pgfpathlineto{\pgfqpoint{3.984712in}{0.709870in}}%
\pgfpathlineto{\pgfqpoint{3.985575in}{0.630823in}}%
\pgfpathlineto{\pgfqpoint{3.986438in}{0.682397in}}%
\pgfpathlineto{\pgfqpoint{3.987304in}{0.640308in}}%
\pgfpathlineto{\pgfqpoint{3.988169in}{0.702837in}}%
\pgfpathlineto{\pgfqpoint{3.989902in}{0.672872in}}%
\pgfpathlineto{\pgfqpoint{3.990765in}{0.729245in}}%
\pgfpathlineto{\pgfqpoint{3.991629in}{0.640235in}}%
\pgfpathlineto{\pgfqpoint{3.992494in}{0.726501in}}%
\pgfpathlineto{\pgfqpoint{3.994223in}{0.664889in}}%
\pgfpathlineto{\pgfqpoint{3.995089in}{0.769979in}}%
\pgfpathlineto{\pgfqpoint{3.995954in}{0.684704in}}%
\pgfpathlineto{\pgfqpoint{3.996820in}{0.773860in}}%
\pgfpathlineto{\pgfqpoint{3.997684in}{0.730528in}}%
\pgfpathlineto{\pgfqpoint{3.998549in}{0.755694in}}%
\pgfpathlineto{\pgfqpoint{3.999414in}{0.750822in}}%
\pgfpathlineto{\pgfqpoint{4.000279in}{0.665547in}}%
\pgfpathlineto{\pgfqpoint{4.002007in}{0.725071in}}%
\pgfpathlineto{\pgfqpoint{4.002874in}{0.698294in}}%
\pgfpathlineto{\pgfqpoint{4.004603in}{0.719281in}}%
\pgfpathlineto{\pgfqpoint{4.005467in}{0.714153in}}%
\pgfpathlineto{\pgfqpoint{4.007191in}{0.667122in}}%
\pgfpathlineto{\pgfqpoint{4.008923in}{0.738917in}}%
\pgfpathlineto{\pgfqpoint{4.009788in}{0.703495in}}%
\pgfpathlineto{\pgfqpoint{4.010652in}{0.706426in}}%
\pgfpathlineto{\pgfqpoint{4.011517in}{0.742100in}}%
\pgfpathlineto{\pgfqpoint{4.013246in}{0.697559in}}%
\pgfpathlineto{\pgfqpoint{4.014108in}{0.692175in}}%
\pgfpathlineto{\pgfqpoint{4.014974in}{0.734154in}}%
\pgfpathlineto{\pgfqpoint{4.015839in}{0.708879in}}%
\pgfpathlineto{\pgfqpoint{4.016704in}{0.773276in}}%
\pgfpathlineto{\pgfqpoint{4.017569in}{0.725181in}}%
\pgfpathlineto{\pgfqpoint{4.018434in}{0.751772in}}%
\pgfpathlineto{\pgfqpoint{4.019301in}{0.700636in}}%
\pgfpathlineto{\pgfqpoint{4.020166in}{0.736132in}}%
\pgfpathlineto{\pgfqpoint{4.021032in}{0.732689in}}%
\pgfpathlineto{\pgfqpoint{4.021898in}{0.737122in}}%
\pgfpathlineto{\pgfqpoint{4.023629in}{0.705915in}}%
\pgfpathlineto{\pgfqpoint{4.024494in}{0.740492in}}%
\pgfpathlineto{\pgfqpoint{4.025359in}{0.730821in}}%
\pgfpathlineto{\pgfqpoint{4.026224in}{0.710527in}}%
\pgfpathlineto{\pgfqpoint{4.027953in}{0.750603in}}%
\pgfpathlineto{\pgfqpoint{4.028818in}{0.692727in}}%
\pgfpathlineto{\pgfqpoint{4.029684in}{0.777271in}}%
\pgfpathlineto{\pgfqpoint{4.031412in}{0.680566in}}%
\pgfpathlineto{\pgfqpoint{4.032277in}{0.682105in}}%
\pgfpathlineto{\pgfqpoint{4.034007in}{0.771517in}}%
\pgfpathlineto{\pgfqpoint{4.036604in}{0.708477in}}%
\pgfpathlineto{\pgfqpoint{4.037469in}{0.749320in}}%
\pgfpathlineto{\pgfqpoint{4.038334in}{0.705915in}}%
\pgfpathlineto{\pgfqpoint{4.039199in}{0.709577in}}%
\pgfpathlineto{\pgfqpoint{4.040065in}{0.696682in}}%
\pgfpathlineto{\pgfqpoint{4.040931in}{0.738734in}}%
\pgfpathlineto{\pgfqpoint{4.042662in}{0.685110in}}%
\pgfpathlineto{\pgfqpoint{4.045257in}{0.753348in}}%
\pgfpathlineto{\pgfqpoint{4.046121in}{0.736497in}}%
\pgfpathlineto{\pgfqpoint{4.046986in}{0.737926in}}%
\pgfpathlineto{\pgfqpoint{4.047850in}{0.756718in}}%
\pgfpathlineto{\pgfqpoint{4.048715in}{0.691517in}}%
\pgfpathlineto{\pgfqpoint{4.049579in}{0.785546in}}%
\pgfpathlineto{\pgfqpoint{4.050444in}{0.706792in}}%
\pgfpathlineto{\pgfqpoint{4.052176in}{0.787706in}}%
\pgfpathlineto{\pgfqpoint{4.053040in}{0.681443in}}%
\pgfpathlineto{\pgfqpoint{4.053906in}{0.740087in}}%
\pgfpathlineto{\pgfqpoint{4.054771in}{0.711478in}}%
\pgfpathlineto{\pgfqpoint{4.055636in}{0.645948in}}%
\pgfpathlineto{\pgfqpoint{4.056502in}{0.775363in}}%
\pgfpathlineto{\pgfqpoint{4.058228in}{0.714409in}}%
\pgfpathlineto{\pgfqpoint{4.059092in}{0.730382in}}%
\pgfpathlineto{\pgfqpoint{4.060822in}{0.690380in}}%
\pgfpathlineto{\pgfqpoint{4.061687in}{0.740306in}}%
\pgfpathlineto{\pgfqpoint{4.062552in}{0.685910in}}%
\pgfpathlineto{\pgfqpoint{4.063417in}{0.733493in}}%
\pgfpathlineto{\pgfqpoint{4.064282in}{0.721953in}}%
\pgfpathlineto{\pgfqpoint{4.065147in}{0.712432in}}%
\pgfpathlineto{\pgfqpoint{4.066012in}{0.764408in}}%
\pgfpathlineto{\pgfqpoint{4.067741in}{0.705874in}}%
\pgfpathlineto{\pgfqpoint{4.068604in}{0.738438in}}%
\pgfpathlineto{\pgfqpoint{4.069469in}{0.672762in}}%
\pgfpathlineto{\pgfqpoint{4.071198in}{0.700344in}}%
\pgfpathlineto{\pgfqpoint{4.072928in}{0.676315in}}%
\pgfpathlineto{\pgfqpoint{4.073792in}{0.688585in}}%
\pgfpathlineto{\pgfqpoint{4.074657in}{0.729136in}}%
\pgfpathlineto{\pgfqpoint{4.075522in}{0.720638in}}%
\pgfpathlineto{\pgfqpoint{4.076388in}{0.705582in}}%
\pgfpathlineto{\pgfqpoint{4.078118in}{0.761663in}}%
\pgfpathlineto{\pgfqpoint{4.079849in}{0.695252in}}%
\pgfpathlineto{\pgfqpoint{4.080713in}{0.698622in}}%
\pgfpathlineto{\pgfqpoint{4.081580in}{0.700381in}}%
\pgfpathlineto{\pgfqpoint{4.082444in}{0.782468in}}%
\pgfpathlineto{\pgfqpoint{4.084175in}{0.686608in}}%
\pgfpathlineto{\pgfqpoint{4.085039in}{0.725802in}}%
\pgfpathlineto{\pgfqpoint{4.085903in}{0.695106in}}%
\pgfpathlineto{\pgfqpoint{4.086768in}{0.739063in}}%
\pgfpathlineto{\pgfqpoint{4.087632in}{0.728035in}}%
\pgfpathlineto{\pgfqpoint{4.088498in}{0.652798in}}%
\pgfpathlineto{\pgfqpoint{4.090229in}{0.740858in}}%
\pgfpathlineto{\pgfqpoint{4.091095in}{0.708915in}}%
\pgfpathlineto{\pgfqpoint{4.093691in}{0.825029in}}%
\pgfpathlineto{\pgfqpoint{4.094556in}{0.743895in}}%
\pgfpathlineto{\pgfqpoint{4.095420in}{0.805653in}}%
\pgfpathlineto{\pgfqpoint{4.097151in}{0.707450in}}%
\pgfpathlineto{\pgfqpoint{4.098016in}{0.758439in}}%
\pgfpathlineto{\pgfqpoint{4.098881in}{0.741589in}}%
\pgfpathlineto{\pgfqpoint{4.099746in}{0.688732in}}%
\pgfpathlineto{\pgfqpoint{4.100611in}{0.695216in}}%
\pgfpathlineto{\pgfqpoint{4.101476in}{0.725437in}}%
\pgfpathlineto{\pgfqpoint{4.102341in}{0.687781in}}%
\pgfpathlineto{\pgfqpoint{4.103206in}{0.764923in}}%
\pgfpathlineto{\pgfqpoint{4.104071in}{0.738073in}}%
\pgfpathlineto{\pgfqpoint{4.104936in}{0.753128in}}%
\pgfpathlineto{\pgfqpoint{4.107530in}{0.608771in}}%
\pgfpathlineto{\pgfqpoint{4.108395in}{0.726939in}}%
\pgfpathlineto{\pgfqpoint{4.110125in}{0.670310in}}%
\pgfpathlineto{\pgfqpoint{4.110990in}{0.689722in}}%
\pgfpathlineto{\pgfqpoint{4.111854in}{0.645473in}}%
\pgfpathlineto{\pgfqpoint{4.113584in}{0.713386in}}%
\pgfpathlineto{\pgfqpoint{4.114449in}{0.703422in}}%
\pgfpathlineto{\pgfqpoint{4.115313in}{0.678037in}}%
\pgfpathlineto{\pgfqpoint{4.116177in}{0.733972in}}%
\pgfpathlineto{\pgfqpoint{4.117040in}{0.677562in}}%
\pgfpathlineto{\pgfqpoint{4.117905in}{0.691371in}}%
\pgfpathlineto{\pgfqpoint{4.119636in}{0.766243in}}%
\pgfpathlineto{\pgfqpoint{4.120501in}{0.755475in}}%
\pgfpathlineto{\pgfqpoint{4.122231in}{0.690859in}}%
\pgfpathlineto{\pgfqpoint{4.123096in}{0.727012in}}%
\pgfpathlineto{\pgfqpoint{4.123961in}{0.683900in}}%
\pgfpathlineto{\pgfqpoint{4.125692in}{0.767745in}}%
\pgfpathlineto{\pgfqpoint{4.126557in}{0.697563in}}%
\pgfpathlineto{\pgfqpoint{4.127422in}{0.716244in}}%
\pgfpathlineto{\pgfqpoint{4.129154in}{0.692617in}}%
\pgfpathlineto{\pgfqpoint{4.130019in}{0.702983in}}%
\pgfpathlineto{\pgfqpoint{4.130884in}{0.667195in}}%
\pgfpathlineto{\pgfqpoint{4.132610in}{0.711810in}}%
\pgfpathlineto{\pgfqpoint{4.133474in}{0.753348in}}%
\pgfpathlineto{\pgfqpoint{4.134339in}{0.635143in}}%
\pgfpathlineto{\pgfqpoint{4.135203in}{0.724373in}}%
\pgfpathlineto{\pgfqpoint{4.136068in}{0.722067in}}%
\pgfpathlineto{\pgfqpoint{4.137798in}{0.785180in}}%
\pgfpathlineto{\pgfqpoint{4.139529in}{0.708952in}}%
\pgfpathlineto{\pgfqpoint{4.141259in}{0.722761in}}%
\pgfpathlineto{\pgfqpoint{4.142124in}{0.721661in}}%
\pgfpathlineto{\pgfqpoint{4.143853in}{0.747484in}}%
\pgfpathlineto{\pgfqpoint{4.144718in}{0.714482in}}%
\pgfpathlineto{\pgfqpoint{4.145583in}{0.772577in}}%
\pgfpathlineto{\pgfqpoint{4.148177in}{0.673310in}}%
\pgfpathlineto{\pgfqpoint{4.149043in}{0.710820in}}%
\pgfpathlineto{\pgfqpoint{4.150774in}{0.667195in}}%
\pgfpathlineto{\pgfqpoint{4.151640in}{0.725656in}}%
\pgfpathlineto{\pgfqpoint{4.152507in}{0.711883in}}%
\pgfpathlineto{\pgfqpoint{4.153374in}{0.688147in}}%
\pgfpathlineto{\pgfqpoint{4.154234in}{0.744667in}}%
\pgfpathlineto{\pgfqpoint{4.155100in}{0.669100in}}%
\pgfpathlineto{\pgfqpoint{4.156830in}{0.737780in}}%
\pgfpathlineto{\pgfqpoint{4.157692in}{0.674703in}}%
\pgfpathlineto{\pgfqpoint{4.158557in}{0.725583in}}%
\pgfpathlineto{\pgfqpoint{4.159420in}{0.702618in}}%
\pgfpathlineto{\pgfqpoint{4.160285in}{0.716354in}}%
\pgfpathlineto{\pgfqpoint{4.161148in}{0.709943in}}%
\pgfpathlineto{\pgfqpoint{4.162013in}{0.726793in}}%
\pgfpathlineto{\pgfqpoint{4.162876in}{0.658149in}}%
\pgfpathlineto{\pgfqpoint{4.164607in}{0.693937in}}%
\pgfpathlineto{\pgfqpoint{4.165472in}{0.658441in}}%
\pgfpathlineto{\pgfqpoint{4.166337in}{0.766535in}}%
\pgfpathlineto{\pgfqpoint{4.167203in}{0.637490in}}%
\pgfpathlineto{\pgfqpoint{4.168067in}{0.728515in}}%
\pgfpathlineto{\pgfqpoint{4.169798in}{0.666318in}}%
\pgfpathlineto{\pgfqpoint{4.170663in}{0.693717in}}%
\pgfpathlineto{\pgfqpoint{4.171528in}{0.661811in}}%
\pgfpathlineto{\pgfqpoint{4.173258in}{0.693937in}}%
\pgfpathlineto{\pgfqpoint{4.174125in}{0.646719in}}%
\pgfpathlineto{\pgfqpoint{4.174991in}{0.695951in}}%
\pgfpathlineto{\pgfqpoint{4.175858in}{0.625805in}}%
\pgfpathlineto{\pgfqpoint{4.177588in}{0.662396in}}%
\pgfpathlineto{\pgfqpoint{4.178452in}{0.646610in}}%
\pgfpathlineto{\pgfqpoint{4.180184in}{0.713166in}}%
\pgfpathlineto{\pgfqpoint{4.181050in}{0.677086in}}%
\pgfpathlineto{\pgfqpoint{4.181915in}{0.735693in}}%
\pgfpathlineto{\pgfqpoint{4.183646in}{0.668077in}}%
\pgfpathlineto{\pgfqpoint{4.185377in}{0.736059in}}%
\pgfpathlineto{\pgfqpoint{4.186240in}{0.682909in}}%
\pgfpathlineto{\pgfqpoint{4.187102in}{0.729099in}}%
\pgfpathlineto{\pgfqpoint{4.187968in}{0.684265in}}%
\pgfpathlineto{\pgfqpoint{4.188833in}{0.707271in}}%
\pgfpathlineto{\pgfqpoint{4.189700in}{0.697965in}}%
\pgfpathlineto{\pgfqpoint{4.190565in}{0.717048in}}%
\pgfpathlineto{\pgfqpoint{4.191432in}{0.680822in}}%
\pgfpathlineto{\pgfqpoint{4.192297in}{0.724779in}}%
\pgfpathlineto{\pgfqpoint{4.193163in}{0.678589in}}%
\pgfpathlineto{\pgfqpoint{4.194030in}{0.692581in}}%
\pgfpathlineto{\pgfqpoint{4.194895in}{0.680237in}}%
\pgfpathlineto{\pgfqpoint{4.195760in}{0.702289in}}%
\pgfpathlineto{\pgfqpoint{4.196623in}{0.637344in}}%
\pgfpathlineto{\pgfqpoint{4.197486in}{0.645513in}}%
\pgfpathlineto{\pgfqpoint{4.199218in}{0.775330in}}%
\pgfpathlineto{\pgfqpoint{4.200083in}{0.675072in}}%
\pgfpathlineto{\pgfqpoint{4.201815in}{0.744155in}}%
\pgfpathlineto{\pgfqpoint{4.202680in}{0.661081in}}%
\pgfpathlineto{\pgfqpoint{4.204411in}{0.779870in}}%
\pgfpathlineto{\pgfqpoint{4.206142in}{0.638367in}}%
\pgfpathlineto{\pgfqpoint{4.208736in}{0.766974in}}%
\pgfpathlineto{\pgfqpoint{4.210467in}{0.695585in}}%
\pgfpathlineto{\pgfqpoint{4.212198in}{0.659871in}}%
\pgfpathlineto{\pgfqpoint{4.213925in}{0.786975in}}%
\pgfpathlineto{\pgfqpoint{4.215654in}{0.664191in}}%
\pgfpathlineto{\pgfqpoint{4.217385in}{0.721774in}}%
\pgfpathlineto{\pgfqpoint{4.218250in}{0.698184in}}%
\pgfpathlineto{\pgfqpoint{4.219976in}{0.707157in}}%
\pgfpathlineto{\pgfqpoint{4.220842in}{0.701700in}}%
\pgfpathlineto{\pgfqpoint{4.221705in}{0.686681in}}%
\pgfpathlineto{\pgfqpoint{4.223436in}{0.756937in}}%
\pgfpathlineto{\pgfqpoint{4.224302in}{0.775582in}}%
\pgfpathlineto{\pgfqpoint{4.226031in}{0.682544in}}%
\pgfpathlineto{\pgfqpoint{4.226894in}{0.705951in}}%
\pgfpathlineto{\pgfqpoint{4.227760in}{0.634924in}}%
\pgfpathlineto{\pgfqpoint{4.229490in}{0.710308in}}%
\pgfpathlineto{\pgfqpoint{4.230356in}{0.718697in}}%
\pgfpathlineto{\pgfqpoint{4.232084in}{0.643313in}}%
\pgfpathlineto{\pgfqpoint{4.232949in}{0.732469in}}%
\pgfpathlineto{\pgfqpoint{4.233814in}{0.688110in}}%
\pgfpathlineto{\pgfqpoint{4.234678in}{0.758074in}}%
\pgfpathlineto{\pgfqpoint{4.236409in}{0.658186in}}%
\pgfpathlineto{\pgfqpoint{4.238138in}{0.724998in}}%
\pgfpathlineto{\pgfqpoint{4.239003in}{0.711006in}}%
\pgfpathlineto{\pgfqpoint{4.240732in}{0.740419in}}%
\pgfpathlineto{\pgfqpoint{4.243325in}{0.711372in}}%
\pgfpathlineto{\pgfqpoint{4.244190in}{0.734191in}}%
\pgfpathlineto{\pgfqpoint{4.245054in}{0.733825in}}%
\pgfpathlineto{\pgfqpoint{4.245919in}{0.739502in}}%
\pgfpathlineto{\pgfqpoint{4.246784in}{0.710235in}}%
\pgfpathlineto{\pgfqpoint{4.248514in}{0.754631in}}%
\pgfpathlineto{\pgfqpoint{4.249380in}{0.720126in}}%
\pgfpathlineto{\pgfqpoint{4.250245in}{0.638331in}}%
\pgfpathlineto{\pgfqpoint{4.251107in}{0.766682in}}%
\pgfpathlineto{\pgfqpoint{4.251972in}{0.749320in}}%
\pgfpathlineto{\pgfqpoint{4.252837in}{0.679356in}}%
\pgfpathlineto{\pgfqpoint{4.253702in}{0.727085in}}%
\pgfpathlineto{\pgfqpoint{4.255431in}{0.650418in}}%
\pgfpathlineto{\pgfqpoint{4.257163in}{0.720126in}}%
\pgfpathlineto{\pgfqpoint{4.258027in}{0.719281in}}%
\pgfpathlineto{\pgfqpoint{4.258891in}{0.724300in}}%
\pgfpathlineto{\pgfqpoint{4.259758in}{0.652578in}}%
\pgfpathlineto{\pgfqpoint{4.262349in}{0.770015in}}%
\pgfpathlineto{\pgfqpoint{4.263215in}{0.700198in}}%
\pgfpathlineto{\pgfqpoint{4.265810in}{0.789720in}}%
\pgfpathlineto{\pgfqpoint{4.267541in}{0.704080in}}%
\pgfpathlineto{\pgfqpoint{4.268406in}{0.742872in}}%
\pgfpathlineto{\pgfqpoint{4.269270in}{0.675511in}}%
\pgfpathlineto{\pgfqpoint{4.271000in}{0.745876in}}%
\pgfpathlineto{\pgfqpoint{4.272729in}{0.645327in}}%
\pgfpathlineto{\pgfqpoint{4.273593in}{0.719647in}}%
\pgfpathlineto{\pgfqpoint{4.274458in}{0.694335in}}%
\pgfpathlineto{\pgfqpoint{4.276188in}{0.757887in}}%
\pgfpathlineto{\pgfqpoint{4.277054in}{0.625325in}}%
\pgfpathlineto{\pgfqpoint{4.277919in}{0.687010in}}%
\pgfpathlineto{\pgfqpoint{4.278785in}{0.624335in}}%
\pgfpathlineto{\pgfqpoint{4.280515in}{0.760819in}}%
\pgfpathlineto{\pgfqpoint{4.281379in}{0.653053in}}%
\pgfpathlineto{\pgfqpoint{4.282243in}{0.750416in}}%
\pgfpathlineto{\pgfqpoint{4.284839in}{0.667561in}}%
\pgfpathlineto{\pgfqpoint{4.285704in}{0.681626in}}%
\pgfpathlineto{\pgfqpoint{4.286569in}{0.667707in}}%
\pgfpathlineto{\pgfqpoint{4.287434in}{0.614411in}}%
\pgfpathlineto{\pgfqpoint{4.289164in}{0.701993in}}%
\pgfpathlineto{\pgfqpoint{4.290029in}{0.691882in}}%
\pgfpathlineto{\pgfqpoint{4.290894in}{0.705436in}}%
\pgfpathlineto{\pgfqpoint{4.291759in}{0.663899in}}%
\pgfpathlineto{\pgfqpoint{4.292625in}{0.695252in}}%
\pgfpathlineto{\pgfqpoint{4.293491in}{0.658953in}}%
\pgfpathlineto{\pgfqpoint{4.296948in}{0.707377in}}%
\pgfpathlineto{\pgfqpoint{4.298678in}{0.697193in}}%
\pgfpathlineto{\pgfqpoint{4.300410in}{0.599871in}}%
\pgfpathlineto{\pgfqpoint{4.301276in}{0.681886in}}%
\pgfpathlineto{\pgfqpoint{4.302140in}{0.681699in}}%
\pgfpathlineto{\pgfqpoint{4.303005in}{0.682653in}}%
\pgfpathlineto{\pgfqpoint{4.303869in}{0.638148in}}%
\pgfpathlineto{\pgfqpoint{4.304734in}{0.725802in}}%
\pgfpathlineto{\pgfqpoint{4.305599in}{0.684119in}}%
\pgfpathlineto{\pgfqpoint{4.306465in}{0.762581in}}%
\pgfpathlineto{\pgfqpoint{4.307331in}{0.709139in}}%
\pgfpathlineto{\pgfqpoint{4.308196in}{0.714742in}}%
\pgfpathlineto{\pgfqpoint{4.309061in}{0.709358in}}%
\pgfpathlineto{\pgfqpoint{4.309926in}{0.658039in}}%
\pgfpathlineto{\pgfqpoint{4.311656in}{0.724446in}}%
\pgfpathlineto{\pgfqpoint{4.312522in}{0.695658in}}%
\pgfpathlineto{\pgfqpoint{4.313387in}{0.731885in}}%
\pgfpathlineto{\pgfqpoint{4.314253in}{0.728515in}}%
\pgfpathlineto{\pgfqpoint{4.315118in}{0.677598in}}%
\pgfpathlineto{\pgfqpoint{4.316847in}{0.735730in}}%
\pgfpathlineto{\pgfqpoint{4.318576in}{0.713605in}}%
\pgfpathlineto{\pgfqpoint{4.319442in}{0.723752in}}%
\pgfpathlineto{\pgfqpoint{4.320307in}{0.720711in}}%
\pgfpathlineto{\pgfqpoint{4.321170in}{0.696901in}}%
\pgfpathlineto{\pgfqpoint{4.322035in}{0.744045in}}%
\pgfpathlineto{\pgfqpoint{4.323763in}{0.708002in}}%
\pgfpathlineto{\pgfqpoint{4.324629in}{0.750895in}}%
\pgfpathlineto{\pgfqpoint{4.326357in}{0.678991in}}%
\pgfpathlineto{\pgfqpoint{4.328083in}{0.741954in}}%
\pgfpathlineto{\pgfqpoint{4.328948in}{0.672982in}}%
\pgfpathlineto{\pgfqpoint{4.329813in}{0.756718in}}%
\pgfpathlineto{\pgfqpoint{4.330678in}{0.735986in}}%
\pgfpathlineto{\pgfqpoint{4.331544in}{0.705326in}}%
\pgfpathlineto{\pgfqpoint{4.332409in}{0.762102in}}%
\pgfpathlineto{\pgfqpoint{4.333276in}{0.701554in}}%
\pgfpathlineto{\pgfqpoint{4.334141in}{0.717194in}}%
\pgfpathlineto{\pgfqpoint{4.335005in}{0.662835in}}%
\pgfpathlineto{\pgfqpoint{4.335868in}{0.766353in}}%
\pgfpathlineto{\pgfqpoint{4.336733in}{0.654852in}}%
\pgfpathlineto{\pgfqpoint{4.337598in}{0.677744in}}%
\pgfpathlineto{\pgfqpoint{4.338461in}{0.707230in}}%
\pgfpathlineto{\pgfqpoint{4.339326in}{0.787560in}}%
\pgfpathlineto{\pgfqpoint{4.341058in}{0.672835in}}%
\pgfpathlineto{\pgfqpoint{4.341924in}{0.684484in}}%
\pgfpathlineto{\pgfqpoint{4.342786in}{0.693165in}}%
\pgfpathlineto{\pgfqpoint{4.344511in}{0.732104in}}%
\pgfpathlineto{\pgfqpoint{4.345375in}{0.739794in}}%
\pgfpathlineto{\pgfqpoint{4.347106in}{0.702285in}}%
\pgfpathlineto{\pgfqpoint{4.347971in}{0.723898in}}%
\pgfpathlineto{\pgfqpoint{4.348837in}{0.809063in}}%
\pgfpathlineto{\pgfqpoint{4.350566in}{0.684594in}}%
\pgfpathlineto{\pgfqpoint{4.352298in}{0.736424in}}%
\pgfpathlineto{\pgfqpoint{4.353162in}{0.686279in}}%
\pgfpathlineto{\pgfqpoint{4.355757in}{0.770677in}}%
\pgfpathlineto{\pgfqpoint{4.356622in}{0.718550in}}%
\pgfpathlineto{\pgfqpoint{4.358349in}{0.793642in}}%
\pgfpathlineto{\pgfqpoint{4.359212in}{0.772325in}}%
\pgfpathlineto{\pgfqpoint{4.360077in}{0.803314in}}%
\pgfpathlineto{\pgfqpoint{4.363537in}{0.683315in}}%
\pgfpathlineto{\pgfqpoint{4.365264in}{0.719834in}}%
\pgfpathlineto{\pgfqpoint{4.366126in}{0.656501in}}%
\pgfpathlineto{\pgfqpoint{4.366989in}{0.728441in}}%
\pgfpathlineto{\pgfqpoint{4.367853in}{0.694375in}}%
\pgfpathlineto{\pgfqpoint{4.368718in}{0.755914in}}%
\pgfpathlineto{\pgfqpoint{4.370449in}{0.698330in}}%
\pgfpathlineto{\pgfqpoint{4.371313in}{0.758878in}}%
\pgfpathlineto{\pgfqpoint{4.373044in}{0.663314in}}%
\pgfpathlineto{\pgfqpoint{4.373909in}{0.665510in}}%
\pgfpathlineto{\pgfqpoint{4.374774in}{0.745438in}}%
\pgfpathlineto{\pgfqpoint{4.375640in}{0.682617in}}%
\pgfpathlineto{\pgfqpoint{4.376505in}{0.738588in}}%
\pgfpathlineto{\pgfqpoint{4.378235in}{0.665108in}}%
\pgfpathlineto{\pgfqpoint{4.380827in}{0.736392in}}%
\pgfpathlineto{\pgfqpoint{4.383424in}{0.708846in}}%
\pgfpathlineto{\pgfqpoint{4.384290in}{0.673278in}}%
\pgfpathlineto{\pgfqpoint{4.386020in}{0.743132in}}%
\pgfpathlineto{\pgfqpoint{4.387751in}{0.693238in}}%
\pgfpathlineto{\pgfqpoint{4.388616in}{0.692325in}}%
\pgfpathlineto{\pgfqpoint{4.389481in}{0.706426in}}%
\pgfpathlineto{\pgfqpoint{4.390346in}{0.695512in}}%
\pgfpathlineto{\pgfqpoint{4.391212in}{0.719395in}}%
\pgfpathlineto{\pgfqpoint{4.392077in}{0.709285in}}%
\pgfpathlineto{\pgfqpoint{4.392939in}{0.660346in}}%
\pgfpathlineto{\pgfqpoint{4.394670in}{0.722213in}}%
\pgfpathlineto{\pgfqpoint{4.395533in}{0.719834in}}%
\pgfpathlineto{\pgfqpoint{4.397262in}{0.662729in}}%
\pgfpathlineto{\pgfqpoint{4.398127in}{0.756575in}}%
\pgfpathlineto{\pgfqpoint{4.398990in}{0.703681in}}%
\pgfpathlineto{\pgfqpoint{4.401586in}{0.745730in}}%
\pgfpathlineto{\pgfqpoint{4.403318in}{0.685987in}}%
\pgfpathlineto{\pgfqpoint{4.405047in}{0.725693in}}%
\pgfpathlineto{\pgfqpoint{4.405913in}{0.703056in}}%
\pgfpathlineto{\pgfqpoint{4.406779in}{0.716829in}}%
\pgfpathlineto{\pgfqpoint{4.407644in}{0.632581in}}%
\pgfpathlineto{\pgfqpoint{4.408510in}{0.805620in}}%
\pgfpathlineto{\pgfqpoint{4.409376in}{0.718404in}}%
\pgfpathlineto{\pgfqpoint{4.410240in}{0.723642in}}%
\pgfpathlineto{\pgfqpoint{4.411970in}{0.701152in}}%
\pgfpathlineto{\pgfqpoint{4.412836in}{0.691225in}}%
\pgfpathlineto{\pgfqpoint{4.413701in}{0.771408in}}%
\pgfpathlineto{\pgfqpoint{4.415430in}{0.682836in}}%
\pgfpathlineto{\pgfqpoint{4.418891in}{0.762581in}}%
\pgfpathlineto{\pgfqpoint{4.419756in}{0.660382in}}%
\pgfpathlineto{\pgfqpoint{4.420621in}{0.677744in}}%
\pgfpathlineto{\pgfqpoint{4.422347in}{0.735474in}}%
\pgfpathlineto{\pgfqpoint{4.423211in}{0.718624in}}%
\pgfpathlineto{\pgfqpoint{4.424076in}{0.748073in}}%
\pgfpathlineto{\pgfqpoint{4.424939in}{0.680051in}}%
\pgfpathlineto{\pgfqpoint{4.425802in}{0.693092in}}%
\pgfpathlineto{\pgfqpoint{4.426667in}{0.704339in}}%
\pgfpathlineto{\pgfqpoint{4.428398in}{0.771408in}}%
\pgfpathlineto{\pgfqpoint{4.430129in}{0.725583in}}%
\pgfpathlineto{\pgfqpoint{4.430994in}{0.772764in}}%
\pgfpathlineto{\pgfqpoint{4.431859in}{0.711299in}}%
\pgfpathlineto{\pgfqpoint{4.432725in}{0.781445in}}%
\pgfpathlineto{\pgfqpoint{4.433588in}{0.777307in}}%
\pgfpathlineto{\pgfqpoint{4.436184in}{0.663095in}}%
\pgfpathlineto{\pgfqpoint{4.437050in}{0.757781in}}%
\pgfpathlineto{\pgfqpoint{4.437915in}{0.750456in}}%
\pgfpathlineto{\pgfqpoint{4.438780in}{0.726866in}}%
\pgfpathlineto{\pgfqpoint{4.439645in}{0.665108in}}%
\pgfpathlineto{\pgfqpoint{4.442235in}{0.746940in}}%
\pgfpathlineto{\pgfqpoint{4.443964in}{0.699759in}}%
\pgfpathlineto{\pgfqpoint{4.444829in}{0.723642in}}%
\pgfpathlineto{\pgfqpoint{4.445695in}{0.680676in}}%
\pgfpathlineto{\pgfqpoint{4.446561in}{0.699101in}}%
\pgfpathlineto{\pgfqpoint{4.447427in}{0.698257in}}%
\pgfpathlineto{\pgfqpoint{4.449156in}{0.629028in}}%
\pgfpathlineto{\pgfqpoint{4.450020in}{0.698846in}}%
\pgfpathlineto{\pgfqpoint{4.450885in}{0.683753in}}%
\pgfpathlineto{\pgfqpoint{4.451751in}{0.715546in}}%
\pgfpathlineto{\pgfqpoint{4.452617in}{0.702947in}}%
\pgfpathlineto{\pgfqpoint{4.453484in}{0.717706in}}%
\pgfpathlineto{\pgfqpoint{4.454349in}{0.684411in}}%
\pgfpathlineto{\pgfqpoint{4.455213in}{0.740675in}}%
\pgfpathlineto{\pgfqpoint{4.456077in}{0.674082in}}%
\pgfpathlineto{\pgfqpoint{4.456943in}{0.756133in}}%
\pgfpathlineto{\pgfqpoint{4.457807in}{0.742872in}}%
\pgfpathlineto{\pgfqpoint{4.458672in}{0.730894in}}%
\pgfpathlineto{\pgfqpoint{4.459537in}{0.672762in}}%
\pgfpathlineto{\pgfqpoint{4.461266in}{0.715765in}}%
\pgfpathlineto{\pgfqpoint{4.462131in}{0.700417in}}%
\pgfpathlineto{\pgfqpoint{4.462995in}{0.737342in}}%
\pgfpathlineto{\pgfqpoint{4.464723in}{0.702581in}}%
\pgfpathlineto{\pgfqpoint{4.465588in}{0.691371in}}%
\pgfpathlineto{\pgfqpoint{4.466453in}{0.749758in}}%
\pgfpathlineto{\pgfqpoint{4.467318in}{0.693604in}}%
\pgfpathlineto{\pgfqpoint{4.468183in}{0.706792in}}%
\pgfpathlineto{\pgfqpoint{4.469047in}{0.671995in}}%
\pgfpathlineto{\pgfqpoint{4.470777in}{0.742068in}}%
\pgfpathlineto{\pgfqpoint{4.471642in}{0.702800in}}%
\pgfpathlineto{\pgfqpoint{4.472507in}{0.785180in}}%
\pgfpathlineto{\pgfqpoint{4.473373in}{0.651701in}}%
\pgfpathlineto{\pgfqpoint{4.474235in}{0.675840in}}%
\pgfpathlineto{\pgfqpoint{4.476831in}{0.794665in}}%
\pgfpathlineto{\pgfqpoint{4.478560in}{0.700783in}}%
\pgfpathlineto{\pgfqpoint{4.479425in}{0.709244in}}%
\pgfpathlineto{\pgfqpoint{4.480290in}{0.653788in}}%
\pgfpathlineto{\pgfqpoint{4.482023in}{0.761078in}}%
\pgfpathlineto{\pgfqpoint{4.482888in}{0.653569in}}%
\pgfpathlineto{\pgfqpoint{4.483754in}{0.749685in}}%
\pgfpathlineto{\pgfqpoint{4.484620in}{0.703349in}}%
\pgfpathlineto{\pgfqpoint{4.485486in}{0.770052in}}%
\pgfpathlineto{\pgfqpoint{4.486350in}{0.730382in}}%
\pgfpathlineto{\pgfqpoint{4.487216in}{0.800675in}}%
\pgfpathlineto{\pgfqpoint{4.488081in}{0.680124in}}%
\pgfpathlineto{\pgfqpoint{4.488945in}{0.695033in}}%
\pgfpathlineto{\pgfqpoint{4.489808in}{0.657451in}}%
\pgfpathlineto{\pgfqpoint{4.492402in}{0.737853in}}%
\pgfpathlineto{\pgfqpoint{4.493266in}{0.654889in}}%
\pgfpathlineto{\pgfqpoint{4.496727in}{0.757343in}}%
\pgfpathlineto{\pgfqpoint{4.497591in}{0.682178in}}%
\pgfpathlineto{\pgfqpoint{4.499321in}{0.776499in}}%
\pgfpathlineto{\pgfqpoint{4.501049in}{0.699686in}}%
\pgfpathlineto{\pgfqpoint{4.501914in}{0.670200in}}%
\pgfpathlineto{\pgfqpoint{4.504511in}{0.759211in}}%
\pgfpathlineto{\pgfqpoint{4.505375in}{0.690421in}}%
\pgfpathlineto{\pgfqpoint{4.506240in}{0.716171in}}%
\pgfpathlineto{\pgfqpoint{4.507105in}{0.689799in}}%
\pgfpathlineto{\pgfqpoint{4.507970in}{0.727711in}}%
\pgfpathlineto{\pgfqpoint{4.508835in}{0.682617in}}%
\pgfpathlineto{\pgfqpoint{4.509700in}{0.716573in}}%
\pgfpathlineto{\pgfqpoint{4.510564in}{0.654998in}}%
\pgfpathlineto{\pgfqpoint{4.512294in}{0.728186in}}%
\pgfpathlineto{\pgfqpoint{4.513159in}{0.657085in}}%
\pgfpathlineto{\pgfqpoint{4.514025in}{0.677890in}}%
\pgfpathlineto{\pgfqpoint{4.514890in}{0.639577in}}%
\pgfpathlineto{\pgfqpoint{4.515753in}{0.670346in}}%
\pgfpathlineto{\pgfqpoint{4.516619in}{0.612032in}}%
\pgfpathlineto{\pgfqpoint{4.518346in}{0.712322in}}%
\pgfpathlineto{\pgfqpoint{4.519212in}{0.686060in}}%
\pgfpathlineto{\pgfqpoint{4.520076in}{0.750822in}}%
\pgfpathlineto{\pgfqpoint{4.521806in}{0.682434in}}%
\pgfpathlineto{\pgfqpoint{4.522670in}{0.685767in}}%
\pgfpathlineto{\pgfqpoint{4.523535in}{0.694741in}}%
\pgfpathlineto{\pgfqpoint{4.524400in}{0.728368in}}%
\pgfpathlineto{\pgfqpoint{4.525263in}{0.716463in}}%
\pgfpathlineto{\pgfqpoint{4.526128in}{0.669100in}}%
\pgfpathlineto{\pgfqpoint{4.527857in}{0.701042in}}%
\pgfpathlineto{\pgfqpoint{4.528722in}{0.646719in}}%
\pgfpathlineto{\pgfqpoint{4.529586in}{0.654852in}}%
\pgfpathlineto{\pgfqpoint{4.530452in}{0.663387in}}%
\pgfpathlineto{\pgfqpoint{4.532179in}{0.753900in}}%
\pgfpathlineto{\pgfqpoint{4.533043in}{0.689211in}}%
\pgfpathlineto{\pgfqpoint{4.533909in}{0.730236in}}%
\pgfpathlineto{\pgfqpoint{4.535638in}{0.664962in}}%
\pgfpathlineto{\pgfqpoint{4.537370in}{0.696170in}}%
\pgfpathlineto{\pgfqpoint{4.538236in}{0.695878in}}%
\pgfpathlineto{\pgfqpoint{4.539100in}{0.697892in}}%
\pgfpathlineto{\pgfqpoint{4.539965in}{0.734962in}}%
\pgfpathlineto{\pgfqpoint{4.541691in}{0.688845in}}%
\pgfpathlineto{\pgfqpoint{4.542555in}{0.743205in}}%
\pgfpathlineto{\pgfqpoint{4.544282in}{0.709285in}}%
\pgfpathlineto{\pgfqpoint{4.545145in}{0.723683in}}%
\pgfpathlineto{\pgfqpoint{4.546010in}{0.706467in}}%
\pgfpathlineto{\pgfqpoint{4.546874in}{0.768663in}}%
\pgfpathlineto{\pgfqpoint{4.548603in}{0.714376in}}%
\pgfpathlineto{\pgfqpoint{4.549468in}{0.682324in}}%
\pgfpathlineto{\pgfqpoint{4.551196in}{0.726354in}}%
\pgfpathlineto{\pgfqpoint{4.552060in}{0.722400in}}%
\pgfpathlineto{\pgfqpoint{4.552925in}{0.725770in}}%
\pgfpathlineto{\pgfqpoint{4.553786in}{0.744301in}}%
\pgfpathlineto{\pgfqpoint{4.554650in}{0.721299in}}%
\pgfpathlineto{\pgfqpoint{4.555516in}{0.728368in}}%
\pgfpathlineto{\pgfqpoint{4.556379in}{0.747598in}}%
\pgfpathlineto{\pgfqpoint{4.558975in}{0.665182in}}%
\pgfpathlineto{\pgfqpoint{4.559839in}{0.678881in}}%
\pgfpathlineto{\pgfqpoint{4.560705in}{0.674999in}}%
\pgfpathlineto{\pgfqpoint{4.561571in}{0.682470in}}%
\pgfpathlineto{\pgfqpoint{4.562436in}{0.721921in}}%
\pgfpathlineto{\pgfqpoint{4.563300in}{0.705289in}}%
\pgfpathlineto{\pgfqpoint{4.564162in}{0.657195in}}%
\pgfpathlineto{\pgfqpoint{4.565028in}{0.667195in}}%
\pgfpathlineto{\pgfqpoint{4.566758in}{0.696426in}}%
\pgfpathlineto{\pgfqpoint{4.567622in}{0.674301in}}%
\pgfpathlineto{\pgfqpoint{4.568487in}{0.741077in}}%
\pgfpathlineto{\pgfqpoint{4.569351in}{0.617010in}}%
\pgfpathlineto{\pgfqpoint{4.571948in}{0.751480in}}%
\pgfpathlineto{\pgfqpoint{4.573680in}{0.686023in}}%
\pgfpathlineto{\pgfqpoint{4.574545in}{0.723788in}}%
\pgfpathlineto{\pgfqpoint{4.575410in}{0.644815in}}%
\pgfpathlineto{\pgfqpoint{4.577141in}{0.746867in}}%
\pgfpathlineto{\pgfqpoint{4.578005in}{0.711226in}}%
\pgfpathlineto{\pgfqpoint{4.578869in}{0.583057in}}%
\pgfpathlineto{\pgfqpoint{4.579732in}{0.765293in}}%
\pgfpathlineto{\pgfqpoint{4.583189in}{0.584852in}}%
\pgfpathlineto{\pgfqpoint{4.584916in}{0.663825in}}%
\pgfpathlineto{\pgfqpoint{4.586645in}{0.674374in}}%
\pgfpathlineto{\pgfqpoint{4.587511in}{0.767266in}}%
\pgfpathlineto{\pgfqpoint{4.588377in}{0.662031in}}%
\pgfpathlineto{\pgfqpoint{4.590109in}{0.758001in}}%
\pgfpathlineto{\pgfqpoint{4.591841in}{0.690786in}}%
\pgfpathlineto{\pgfqpoint{4.592706in}{0.751041in}}%
\pgfpathlineto{\pgfqpoint{4.593571in}{0.670127in}}%
\pgfpathlineto{\pgfqpoint{4.594436in}{0.730821in}}%
\pgfpathlineto{\pgfqpoint{4.595300in}{0.667195in}}%
\pgfpathlineto{\pgfqpoint{4.596165in}{0.673716in}}%
\pgfpathlineto{\pgfqpoint{4.597894in}{0.684927in}}%
\pgfpathlineto{\pgfqpoint{4.598760in}{0.634266in}}%
\pgfpathlineto{\pgfqpoint{4.599625in}{0.689357in}}%
\pgfpathlineto{\pgfqpoint{4.600489in}{0.647194in}}%
\pgfpathlineto{\pgfqpoint{4.603082in}{0.732689in}}%
\pgfpathlineto{\pgfqpoint{4.603947in}{0.667301in}}%
\pgfpathlineto{\pgfqpoint{4.604812in}{0.711514in}}%
\pgfpathlineto{\pgfqpoint{4.607407in}{0.643089in}}%
\pgfpathlineto{\pgfqpoint{4.609137in}{0.657816in}}%
\pgfpathlineto{\pgfqpoint{4.610002in}{0.688585in}}%
\pgfpathlineto{\pgfqpoint{4.610868in}{0.675803in}}%
\pgfpathlineto{\pgfqpoint{4.613459in}{0.739429in}}%
\pgfpathlineto{\pgfqpoint{4.615186in}{0.612251in}}%
\pgfpathlineto{\pgfqpoint{4.616918in}{0.717231in}}%
\pgfpathlineto{\pgfqpoint{4.617784in}{0.652286in}}%
\pgfpathlineto{\pgfqpoint{4.619515in}{0.716131in}}%
\pgfpathlineto{\pgfqpoint{4.620380in}{0.676607in}}%
\pgfpathlineto{\pgfqpoint{4.621246in}{0.683238in}}%
\pgfpathlineto{\pgfqpoint{4.622110in}{0.681845in}}%
\pgfpathlineto{\pgfqpoint{4.622974in}{0.716756in}}%
\pgfpathlineto{\pgfqpoint{4.623839in}{0.596684in}}%
\pgfpathlineto{\pgfqpoint{4.626436in}{0.691627in}}%
\pgfpathlineto{\pgfqpoint{4.627301in}{0.696243in}}%
\pgfpathlineto{\pgfqpoint{4.629028in}{0.756571in}}%
\pgfpathlineto{\pgfqpoint{4.629894in}{0.735141in}}%
\pgfpathlineto{\pgfqpoint{4.630756in}{0.748991in}}%
\pgfpathlineto{\pgfqpoint{4.631622in}{0.712103in}}%
\pgfpathlineto{\pgfqpoint{4.632489in}{0.776313in}}%
\pgfpathlineto{\pgfqpoint{4.633355in}{0.633312in}}%
\pgfpathlineto{\pgfqpoint{4.634220in}{0.728547in}}%
\pgfpathlineto{\pgfqpoint{4.635081in}{0.724812in}}%
\pgfpathlineto{\pgfqpoint{4.635945in}{0.712505in}}%
\pgfpathlineto{\pgfqpoint{4.637675in}{0.805141in}}%
\pgfpathlineto{\pgfqpoint{4.640272in}{0.662798in}}%
\pgfpathlineto{\pgfqpoint{4.642003in}{0.771002in}}%
\pgfpathlineto{\pgfqpoint{4.642868in}{0.724227in}}%
\pgfpathlineto{\pgfqpoint{4.643733in}{0.748256in}}%
\pgfpathlineto{\pgfqpoint{4.645460in}{0.692357in}}%
\pgfpathlineto{\pgfqpoint{4.646326in}{0.722651in}}%
\pgfpathlineto{\pgfqpoint{4.648057in}{0.678914in}}%
\pgfpathlineto{\pgfqpoint{4.649788in}{0.705322in}}%
\pgfpathlineto{\pgfqpoint{4.650652in}{0.654812in}}%
\pgfpathlineto{\pgfqpoint{4.651516in}{0.706020in}}%
\pgfpathlineto{\pgfqpoint{4.652381in}{0.671918in}}%
\pgfpathlineto{\pgfqpoint{4.653247in}{0.699207in}}%
\pgfpathlineto{\pgfqpoint{4.654112in}{0.685435in}}%
\pgfpathlineto{\pgfqpoint{4.654976in}{0.698915in}}%
\pgfpathlineto{\pgfqpoint{4.655843in}{0.794300in}}%
\pgfpathlineto{\pgfqpoint{4.657572in}{0.703787in}}%
\pgfpathlineto{\pgfqpoint{4.658437in}{0.734264in}}%
\pgfpathlineto{\pgfqpoint{4.659302in}{0.727707in}}%
\pgfpathlineto{\pgfqpoint{4.660168in}{0.661040in}}%
\pgfpathlineto{\pgfqpoint{4.662761in}{0.792944in}}%
\pgfpathlineto{\pgfqpoint{4.664490in}{0.709281in}}%
\pgfpathlineto{\pgfqpoint{4.665354in}{0.700929in}}%
\pgfpathlineto{\pgfqpoint{4.666218in}{0.665693in}}%
\pgfpathlineto{\pgfqpoint{4.667082in}{0.722578in}}%
\pgfpathlineto{\pgfqpoint{4.667948in}{0.684923in}}%
\pgfpathlineto{\pgfqpoint{4.668812in}{0.759686in}}%
\pgfpathlineto{\pgfqpoint{4.669677in}{0.605438in}}%
\pgfpathlineto{\pgfqpoint{4.670542in}{0.711957in}}%
\pgfpathlineto{\pgfqpoint{4.672272in}{0.620234in}}%
\pgfpathlineto{\pgfqpoint{4.673137in}{0.728182in}}%
\pgfpathlineto{\pgfqpoint{4.674000in}{0.705801in}}%
\pgfpathlineto{\pgfqpoint{4.674866in}{0.702504in}}%
\pgfpathlineto{\pgfqpoint{4.675730in}{0.720930in}}%
\pgfpathlineto{\pgfqpoint{4.676595in}{0.765764in}}%
\pgfpathlineto{\pgfqpoint{4.677460in}{0.705070in}}%
\pgfpathlineto{\pgfqpoint{4.678325in}{0.723017in}}%
\pgfpathlineto{\pgfqpoint{4.680055in}{0.690672in}}%
\pgfpathlineto{\pgfqpoint{4.681784in}{0.726533in}}%
\pgfpathlineto{\pgfqpoint{4.682650in}{0.791368in}}%
\pgfpathlineto{\pgfqpoint{4.683516in}{0.663310in}}%
\pgfpathlineto{\pgfqpoint{4.685244in}{0.745324in}}%
\pgfpathlineto{\pgfqpoint{4.686110in}{0.715984in}}%
\pgfpathlineto{\pgfqpoint{4.686974in}{0.747890in}}%
\pgfpathlineto{\pgfqpoint{4.688705in}{0.684192in}}%
\pgfpathlineto{\pgfqpoint{4.689568in}{0.704153in}}%
\pgfpathlineto{\pgfqpoint{4.690432in}{0.772139in}}%
\pgfpathlineto{\pgfqpoint{4.691297in}{0.755361in}}%
\pgfpathlineto{\pgfqpoint{4.692162in}{0.719062in}}%
\pgfpathlineto{\pgfqpoint{4.693027in}{0.756425in}}%
\pgfpathlineto{\pgfqpoint{4.693889in}{0.706280in}}%
\pgfpathlineto{\pgfqpoint{4.695621in}{0.745584in}}%
\pgfpathlineto{\pgfqpoint{4.697352in}{0.727597in}}%
\pgfpathlineto{\pgfqpoint{4.699083in}{0.671077in}}%
\pgfpathlineto{\pgfqpoint{4.699949in}{0.739356in}}%
\pgfpathlineto{\pgfqpoint{4.701675in}{0.685621in}}%
\pgfpathlineto{\pgfqpoint{4.702540in}{0.710235in}}%
\pgfpathlineto{\pgfqpoint{4.703406in}{0.618845in}}%
\pgfpathlineto{\pgfqpoint{4.704271in}{0.634632in}}%
\pgfpathlineto{\pgfqpoint{4.705135in}{0.649687in}}%
\pgfpathlineto{\pgfqpoint{4.706864in}{0.724633in}}%
\pgfpathlineto{\pgfqpoint{4.707730in}{0.707490in}}%
\pgfpathlineto{\pgfqpoint{4.708594in}{0.714596in}}%
\pgfpathlineto{\pgfqpoint{4.709461in}{0.694818in}}%
\pgfpathlineto{\pgfqpoint{4.710325in}{0.733095in}}%
\pgfpathlineto{\pgfqpoint{4.712056in}{0.652838in}}%
\pgfpathlineto{\pgfqpoint{4.713785in}{0.750164in}}%
\pgfpathlineto{\pgfqpoint{4.715513in}{0.657012in}}%
\pgfpathlineto{\pgfqpoint{4.716380in}{0.707234in}}%
\pgfpathlineto{\pgfqpoint{4.717245in}{0.613096in}}%
\pgfpathlineto{\pgfqpoint{4.718111in}{0.691557in}}%
\pgfpathlineto{\pgfqpoint{4.719839in}{0.655071in}}%
\pgfpathlineto{\pgfqpoint{4.720705in}{0.722400in}}%
\pgfpathlineto{\pgfqpoint{4.721569in}{0.652655in}}%
\pgfpathlineto{\pgfqpoint{4.723300in}{0.752032in}}%
\pgfpathlineto{\pgfqpoint{4.725030in}{0.713240in}}%
\pgfpathlineto{\pgfqpoint{4.725894in}{0.719030in}}%
\pgfpathlineto{\pgfqpoint{4.726759in}{0.718185in}}%
\pgfpathlineto{\pgfqpoint{4.729353in}{0.606867in}}%
\pgfpathlineto{\pgfqpoint{4.730215in}{0.724706in}}%
\pgfpathlineto{\pgfqpoint{4.731080in}{0.682434in}}%
\pgfpathlineto{\pgfqpoint{4.731947in}{0.692873in}}%
\pgfpathlineto{\pgfqpoint{4.732812in}{0.756133in}}%
\pgfpathlineto{\pgfqpoint{4.733678in}{0.739063in}}%
\pgfpathlineto{\pgfqpoint{4.734543in}{0.691663in}}%
\pgfpathlineto{\pgfqpoint{4.735409in}{0.703641in}}%
\pgfpathlineto{\pgfqpoint{4.736271in}{0.729976in}}%
\pgfpathlineto{\pgfqpoint{4.737133in}{0.720784in}}%
\pgfpathlineto{\pgfqpoint{4.737997in}{0.696462in}}%
\pgfpathlineto{\pgfqpoint{4.739729in}{0.740858in}}%
\pgfpathlineto{\pgfqpoint{4.741460in}{0.698111in}}%
\pgfpathlineto{\pgfqpoint{4.742327in}{0.779870in}}%
\pgfpathlineto{\pgfqpoint{4.743193in}{0.719208in}}%
\pgfpathlineto{\pgfqpoint{4.744060in}{0.754923in}}%
\pgfpathlineto{\pgfqpoint{4.746655in}{0.660049in}}%
\pgfpathlineto{\pgfqpoint{4.747518in}{0.667451in}}%
\pgfpathlineto{\pgfqpoint{4.748384in}{0.718291in}}%
\pgfpathlineto{\pgfqpoint{4.749246in}{0.715911in}}%
\pgfpathlineto{\pgfqpoint{4.750112in}{0.716902in}}%
\pgfpathlineto{\pgfqpoint{4.752708in}{0.762800in}}%
\pgfpathlineto{\pgfqpoint{4.754440in}{0.694814in}}%
\pgfpathlineto{\pgfqpoint{4.755305in}{0.745803in}}%
\pgfpathlineto{\pgfqpoint{4.757035in}{0.659505in}}%
\pgfpathlineto{\pgfqpoint{4.758765in}{0.757343in}}%
\pgfpathlineto{\pgfqpoint{4.760492in}{0.780747in}}%
\pgfpathlineto{\pgfqpoint{4.761358in}{0.704778in}}%
\pgfpathlineto{\pgfqpoint{4.763954in}{0.829978in}}%
\pgfpathlineto{\pgfqpoint{4.766549in}{0.688037in}}%
\pgfpathlineto{\pgfqpoint{4.767415in}{0.744593in}}%
\pgfpathlineto{\pgfqpoint{4.770012in}{0.682617in}}%
\pgfpathlineto{\pgfqpoint{4.770878in}{0.769134in}}%
\pgfpathlineto{\pgfqpoint{4.771744in}{0.715034in}}%
\pgfpathlineto{\pgfqpoint{4.772610in}{0.717048in}}%
\pgfpathlineto{\pgfqpoint{4.773477in}{0.729063in}}%
\pgfpathlineto{\pgfqpoint{4.774342in}{0.690640in}}%
\pgfpathlineto{\pgfqpoint{4.775208in}{0.744447in}}%
\pgfpathlineto{\pgfqpoint{4.776073in}{0.691444in}}%
\pgfpathlineto{\pgfqpoint{4.777802in}{0.765106in}}%
\pgfpathlineto{\pgfqpoint{4.779534in}{0.682690in}}%
\pgfpathlineto{\pgfqpoint{4.780401in}{0.721847in}}%
\pgfpathlineto{\pgfqpoint{4.781267in}{0.637929in}}%
\pgfpathlineto{\pgfqpoint{4.782998in}{0.800967in}}%
\pgfpathlineto{\pgfqpoint{4.783865in}{0.735766in}}%
\pgfpathlineto{\pgfqpoint{4.784730in}{0.735986in}}%
\pgfpathlineto{\pgfqpoint{4.785596in}{0.732360in}}%
\pgfpathlineto{\pgfqpoint{4.786461in}{0.795875in}}%
\pgfpathlineto{\pgfqpoint{4.787327in}{0.751187in}}%
\pgfpathlineto{\pgfqpoint{4.788193in}{0.795583in}}%
\pgfpathlineto{\pgfqpoint{4.789059in}{0.704778in}}%
\pgfpathlineto{\pgfqpoint{4.789925in}{0.742799in}}%
\pgfpathlineto{\pgfqpoint{4.792520in}{0.696974in}}%
\pgfpathlineto{\pgfqpoint{4.793384in}{0.776719in}}%
\pgfpathlineto{\pgfqpoint{4.795115in}{0.689503in}}%
\pgfpathlineto{\pgfqpoint{4.795980in}{0.753055in}}%
\pgfpathlineto{\pgfqpoint{4.796845in}{0.721076in}}%
\pgfpathlineto{\pgfqpoint{4.798574in}{0.788843in}}%
\pgfpathlineto{\pgfqpoint{4.801166in}{0.717633in}}%
\pgfpathlineto{\pgfqpoint{4.802893in}{0.681553in}}%
\pgfpathlineto{\pgfqpoint{4.804623in}{0.715692in}}%
\pgfpathlineto{\pgfqpoint{4.805486in}{0.705801in}}%
\pgfpathlineto{\pgfqpoint{4.806350in}{0.681845in}}%
\pgfpathlineto{\pgfqpoint{4.808078in}{0.821845in}}%
\pgfpathlineto{\pgfqpoint{4.809809in}{0.668365in}}%
\pgfpathlineto{\pgfqpoint{4.810674in}{0.799392in}}%
\pgfpathlineto{\pgfqpoint{4.811537in}{0.732944in}}%
\pgfpathlineto{\pgfqpoint{4.812399in}{0.772066in}}%
\pgfpathlineto{\pgfqpoint{4.814129in}{0.665693in}}%
\pgfpathlineto{\pgfqpoint{4.815858in}{0.760088in}}%
\pgfpathlineto{\pgfqpoint{4.816722in}{0.747562in}}%
\pgfpathlineto{\pgfqpoint{4.817588in}{0.734264in}}%
\pgfpathlineto{\pgfqpoint{4.818454in}{0.675219in}}%
\pgfpathlineto{\pgfqpoint{4.821048in}{0.732506in}}%
\pgfpathlineto{\pgfqpoint{4.821914in}{0.672580in}}%
\pgfpathlineto{\pgfqpoint{4.823647in}{0.715984in}}%
\pgfpathlineto{\pgfqpoint{4.824513in}{0.682064in}}%
\pgfpathlineto{\pgfqpoint{4.825379in}{0.698915in}}%
\pgfpathlineto{\pgfqpoint{4.826244in}{0.750672in}}%
\pgfpathlineto{\pgfqpoint{4.827110in}{0.723935in}}%
\pgfpathlineto{\pgfqpoint{4.827976in}{0.640747in}}%
\pgfpathlineto{\pgfqpoint{4.828842in}{0.669063in}}%
\pgfpathlineto{\pgfqpoint{4.829709in}{0.649760in}}%
\pgfpathlineto{\pgfqpoint{4.831440in}{0.691078in}}%
\pgfpathlineto{\pgfqpoint{4.832305in}{0.693312in}}%
\pgfpathlineto{\pgfqpoint{4.833170in}{0.643897in}}%
\pgfpathlineto{\pgfqpoint{4.834033in}{0.695252in}}%
\pgfpathlineto{\pgfqpoint{4.834897in}{0.650491in}}%
\pgfpathlineto{\pgfqpoint{4.835762in}{0.656022in}}%
\pgfpathlineto{\pgfqpoint{4.837494in}{0.745105in}}%
\pgfpathlineto{\pgfqpoint{4.839223in}{0.638733in}}%
\pgfpathlineto{\pgfqpoint{4.840088in}{0.720930in}}%
\pgfpathlineto{\pgfqpoint{4.840953in}{0.710381in}}%
\pgfpathlineto{\pgfqpoint{4.841818in}{0.762321in}}%
\pgfpathlineto{\pgfqpoint{4.842682in}{0.699098in}}%
\pgfpathlineto{\pgfqpoint{4.842682in}{0.699098in}}%
\pgfusepath{stroke}%
\end{pgfscope}%
\begin{pgfscope}%
\pgfsetrectcap%
\pgfsetmiterjoin%
\pgfsetlinewidth{0.803000pt}%
\definecolor{currentstroke}{rgb}{0.000000,0.000000,0.000000}%
\pgfsetstrokecolor{currentstroke}%
\pgfsetdash{}{0pt}%
\pgfpathmoveto{\pgfqpoint{0.483776in}{0.538014in}}%
\pgfpathlineto{\pgfqpoint{0.483776in}{1.122895in}}%
\pgfusepath{stroke}%
\end{pgfscope}%
\begin{pgfscope}%
\pgfsetrectcap%
\pgfsetmiterjoin%
\pgfsetlinewidth{0.803000pt}%
\definecolor{currentstroke}{rgb}{0.000000,0.000000,0.000000}%
\pgfsetstrokecolor{currentstroke}%
\pgfsetdash{}{0pt}%
\pgfpathmoveto{\pgfqpoint{5.050249in}{0.538014in}}%
\pgfpathlineto{\pgfqpoint{5.050249in}{1.122895in}}%
\pgfusepath{stroke}%
\end{pgfscope}%
\begin{pgfscope}%
\pgfsetrectcap%
\pgfsetmiterjoin%
\pgfsetlinewidth{0.803000pt}%
\definecolor{currentstroke}{rgb}{0.000000,0.000000,0.000000}%
\pgfsetstrokecolor{currentstroke}%
\pgfsetdash{}{0pt}%
\pgfpathmoveto{\pgfqpoint{0.483776in}{0.538014in}}%
\pgfpathlineto{\pgfqpoint{5.050249in}{0.538014in}}%
\pgfusepath{stroke}%
\end{pgfscope}%
\begin{pgfscope}%
\pgfsetrectcap%
\pgfsetmiterjoin%
\pgfsetlinewidth{0.803000pt}%
\definecolor{currentstroke}{rgb}{0.000000,0.000000,0.000000}%
\pgfsetstrokecolor{currentstroke}%
\pgfsetdash{}{0pt}%
\pgfpathmoveto{\pgfqpoint{0.483776in}{1.122895in}}%
\pgfpathlineto{\pgfqpoint{5.050249in}{1.122895in}}%
\pgfusepath{stroke}%
\end{pgfscope}%
\end{pgfpicture}%
\makeatother%
\endgroup%

    \caption{Popcorn noise in different samples of the LM399 over a \qty{24}{\hour} period.}
    \label{fig:popcorn_noise_lm399}
\end{figure}

Figure \ref{fig:popcorn_noise_lm399} shows two samples of the LM399, that exhibit popcorn noise, while the last one does not.

The sources of popcorn noise in semiconductor devices are not yet fully understood, but some sources have been identified. Defects in the semiconductor crystal lattice and contamination of the semiconductor material have been linked to popcorn noise \cite{technote_ti_popcorn_noise}. This problem has improved over the years as manufacturing processes and wafer quality has evolved. Unfortunately the LM399 is built around a process from 1991, as can be seen etched into the die \cite{lm399_richi}.

The popcorn noise caused by defects and contamination can be reduced by lowering the strain on the lattice and removing surface contaminants on the die. This can be can be achieved using a high-temperature burn-in process. Manufacturers like Fluke and Keysight use similar techniques in their products. Fluke, for example, uses a period of \qty{60}{\day} burn-in for their references \cite{zener_popcorn_noise}.

Fortunately, the LM399 is a heated reference, which regulates its die to \qty{90}{\celsius} when turned on, so it is only required put the diodes in a simple test circuit and wait. The use of a separate test setup instead of the final circuit has both advantages and disadvantages. The disadvantage is, that the Zener diode will subjected to mechanical stress when soldered, this stress will not be removed by the burn-in process as it happens after the testing, when diode is soldered into the final circuit, but this mainly affects the voltage drift properties of the Zener diode and not the popcorn noise. The drift of the diode is also only of secondary concern in our setup, as the drift is mainly caused by the reference resistors used and are typically at least an order of magnitude worse than the the drift of the diode judging by the data sheet \cite{datasheet_LM399,datasheet_VPR}.

The advantages of testing the Zener diodes separately, on the other hand, are, that more diodes can be tested at the same time, as a special compact test fixture can be used. It is also simpler to remove the diodes from the test fixture, because they be socketed. Therefore for our application a separate test board was used. Building this test setup is detailed in the next sections.

\subsection{Building a Test Setup for Zener Diodes}
There are several ways to measure the popcorn noise of semiconductor devices. The most trivial one is to directly monitor the device in the time-domain. In this case, the Zener voltage can be monitored with a long-scale multimeter. It requires a low noise DMM, that can reliably distinguish between both voltage levels, which are about \qty{4}{\micro \volt} apart.
A related option is to use a second reference, whose voltage is similar to the device under test (DUT). Measuring the the voltage difference between the two references, less resolution is required. Directly comparing the difference of two references using a millivolt meter is commonly done when intercomparing primary voltage references. This method, however, increases the measurement noise by a factor of $\sqrt{2}$, if both references produce the same level of uncorrelated noise. The noise of the LM399 with a \qty{100}{\plc} integration time (\qty{2}{\second}) is about \qty{1.5}{\micro \volt_{pp}} as can be determined from the data in figure \ref{fig:popcorn_noise_lm399}.

\begin{figure}[ht]
    \centering
    %%% Creator: Matplotlib, PGF backend
%%
%% To include the figure in your LaTeX document, write
%%   \input{<filename>.pgf}
%%
%% Make sure the required packages are loaded in your preamble
%%   \usepackage{pgf}
%%
%% Also ensure that all the required font packages are loaded; for instance,
%% the lmodern package is sometimes necessary when using math font.
%%   \usepackage{lmodern}
%%
%% Figures using additional raster images can only be included by \input if
%% they are in the same directory as the main LaTeX file. For loading figures
%% from other directories you can use the `import` package
%%   \usepackage{import}
%%
%% and then include the figures with
%%   \import{<path to file>}{<filename>.pgf}
%%
%% Matplotlib used the following preamble
%%   \usepackage{fontspec}
%%
\begingroup%
\makeatletter%
\begin{pgfpicture}%
\pgfpathrectangle{\pgfpointorigin}{\pgfqpoint{5.200000in}{3.210000in}}%
\pgfusepath{use as bounding box, clip}%
\begin{pgfscope}%
\pgfsetbuttcap%
\pgfsetmiterjoin%
\definecolor{currentfill}{rgb}{1.000000,1.000000,1.000000}%
\pgfsetfillcolor{currentfill}%
\pgfsetlinewidth{0.000000pt}%
\definecolor{currentstroke}{rgb}{1.000000,1.000000,1.000000}%
\pgfsetstrokecolor{currentstroke}%
\pgfsetdash{}{0pt}%
\pgfpathmoveto{\pgfqpoint{0.000000in}{0.000000in}}%
\pgfpathlineto{\pgfqpoint{5.200000in}{0.000000in}}%
\pgfpathlineto{\pgfqpoint{5.200000in}{3.210000in}}%
\pgfpathlineto{\pgfqpoint{0.000000in}{3.210000in}}%
\pgfpathlineto{\pgfqpoint{0.000000in}{0.000000in}}%
\pgfpathclose%
\pgfusepath{fill}%
\end{pgfscope}%
\begin{pgfscope}%
\pgfsetbuttcap%
\pgfsetmiterjoin%
\definecolor{currentfill}{rgb}{1.000000,1.000000,1.000000}%
\pgfsetfillcolor{currentfill}%
\pgfsetlinewidth{0.000000pt}%
\definecolor{currentstroke}{rgb}{0.000000,0.000000,0.000000}%
\pgfsetstrokecolor{currentstroke}%
\pgfsetstrokeopacity{0.000000}%
\pgfsetdash{}{0pt}%
\pgfpathmoveto{\pgfqpoint{0.483776in}{2.351653in}}%
\pgfpathlineto{\pgfqpoint{5.050249in}{2.351653in}}%
\pgfpathlineto{\pgfqpoint{5.050249in}{2.936535in}}%
\pgfpathlineto{\pgfqpoint{0.483776in}{2.936535in}}%
\pgfpathlineto{\pgfqpoint{0.483776in}{2.351653in}}%
\pgfpathclose%
\pgfusepath{fill}%
\end{pgfscope}%
\begin{pgfscope}%
\pgfsetbuttcap%
\pgfsetroundjoin%
\definecolor{currentfill}{rgb}{0.000000,0.000000,0.000000}%
\pgfsetfillcolor{currentfill}%
\pgfsetlinewidth{0.803000pt}%
\definecolor{currentstroke}{rgb}{0.000000,0.000000,0.000000}%
\pgfsetstrokecolor{currentstroke}%
\pgfsetdash{}{0pt}%
\pgfsys@defobject{currentmarker}{\pgfqpoint{0.000000in}{-0.048611in}}{\pgfqpoint{0.000000in}{0.000000in}}{%
\pgfpathmoveto{\pgfqpoint{0.000000in}{0.000000in}}%
\pgfpathlineto{\pgfqpoint{0.000000in}{-0.048611in}}%
\pgfusepath{stroke,fill}%
}%
\begin{pgfscope}%
\pgfsys@transformshift{0.691021in}{2.351653in}%
\pgfsys@useobject{currentmarker}{}%
\end{pgfscope}%
\end{pgfscope}%
\begin{pgfscope}%
\pgfsetbuttcap%
\pgfsetroundjoin%
\definecolor{currentfill}{rgb}{0.000000,0.000000,0.000000}%
\pgfsetfillcolor{currentfill}%
\pgfsetlinewidth{0.803000pt}%
\definecolor{currentstroke}{rgb}{0.000000,0.000000,0.000000}%
\pgfsetstrokecolor{currentstroke}%
\pgfsetdash{}{0pt}%
\pgfsys@defobject{currentmarker}{\pgfqpoint{0.000000in}{-0.048611in}}{\pgfqpoint{0.000000in}{0.000000in}}{%
\pgfpathmoveto{\pgfqpoint{0.000000in}{0.000000in}}%
\pgfpathlineto{\pgfqpoint{0.000000in}{-0.048611in}}%
\pgfusepath{stroke,fill}%
}%
\begin{pgfscope}%
\pgfsys@transformshift{1.210067in}{2.351653in}%
\pgfsys@useobject{currentmarker}{}%
\end{pgfscope}%
\end{pgfscope}%
\begin{pgfscope}%
\pgfsetbuttcap%
\pgfsetroundjoin%
\definecolor{currentfill}{rgb}{0.000000,0.000000,0.000000}%
\pgfsetfillcolor{currentfill}%
\pgfsetlinewidth{0.803000pt}%
\definecolor{currentstroke}{rgb}{0.000000,0.000000,0.000000}%
\pgfsetstrokecolor{currentstroke}%
\pgfsetdash{}{0pt}%
\pgfsys@defobject{currentmarker}{\pgfqpoint{0.000000in}{-0.048611in}}{\pgfqpoint{0.000000in}{0.000000in}}{%
\pgfpathmoveto{\pgfqpoint{0.000000in}{0.000000in}}%
\pgfpathlineto{\pgfqpoint{0.000000in}{-0.048611in}}%
\pgfusepath{stroke,fill}%
}%
\begin{pgfscope}%
\pgfsys@transformshift{1.729114in}{2.351653in}%
\pgfsys@useobject{currentmarker}{}%
\end{pgfscope}%
\end{pgfscope}%
\begin{pgfscope}%
\pgfsetbuttcap%
\pgfsetroundjoin%
\definecolor{currentfill}{rgb}{0.000000,0.000000,0.000000}%
\pgfsetfillcolor{currentfill}%
\pgfsetlinewidth{0.803000pt}%
\definecolor{currentstroke}{rgb}{0.000000,0.000000,0.000000}%
\pgfsetstrokecolor{currentstroke}%
\pgfsetdash{}{0pt}%
\pgfsys@defobject{currentmarker}{\pgfqpoint{0.000000in}{-0.048611in}}{\pgfqpoint{0.000000in}{0.000000in}}{%
\pgfpathmoveto{\pgfqpoint{0.000000in}{0.000000in}}%
\pgfpathlineto{\pgfqpoint{0.000000in}{-0.048611in}}%
\pgfusepath{stroke,fill}%
}%
\begin{pgfscope}%
\pgfsys@transformshift{2.248160in}{2.351653in}%
\pgfsys@useobject{currentmarker}{}%
\end{pgfscope}%
\end{pgfscope}%
\begin{pgfscope}%
\pgfsetbuttcap%
\pgfsetroundjoin%
\definecolor{currentfill}{rgb}{0.000000,0.000000,0.000000}%
\pgfsetfillcolor{currentfill}%
\pgfsetlinewidth{0.803000pt}%
\definecolor{currentstroke}{rgb}{0.000000,0.000000,0.000000}%
\pgfsetstrokecolor{currentstroke}%
\pgfsetdash{}{0pt}%
\pgfsys@defobject{currentmarker}{\pgfqpoint{0.000000in}{-0.048611in}}{\pgfqpoint{0.000000in}{0.000000in}}{%
\pgfpathmoveto{\pgfqpoint{0.000000in}{0.000000in}}%
\pgfpathlineto{\pgfqpoint{0.000000in}{-0.048611in}}%
\pgfusepath{stroke,fill}%
}%
\begin{pgfscope}%
\pgfsys@transformshift{2.767206in}{2.351653in}%
\pgfsys@useobject{currentmarker}{}%
\end{pgfscope}%
\end{pgfscope}%
\begin{pgfscope}%
\pgfsetbuttcap%
\pgfsetroundjoin%
\definecolor{currentfill}{rgb}{0.000000,0.000000,0.000000}%
\pgfsetfillcolor{currentfill}%
\pgfsetlinewidth{0.803000pt}%
\definecolor{currentstroke}{rgb}{0.000000,0.000000,0.000000}%
\pgfsetstrokecolor{currentstroke}%
\pgfsetdash{}{0pt}%
\pgfsys@defobject{currentmarker}{\pgfqpoint{0.000000in}{-0.048611in}}{\pgfqpoint{0.000000in}{0.000000in}}{%
\pgfpathmoveto{\pgfqpoint{0.000000in}{0.000000in}}%
\pgfpathlineto{\pgfqpoint{0.000000in}{-0.048611in}}%
\pgfusepath{stroke,fill}%
}%
\begin{pgfscope}%
\pgfsys@transformshift{3.286252in}{2.351653in}%
\pgfsys@useobject{currentmarker}{}%
\end{pgfscope}%
\end{pgfscope}%
\begin{pgfscope}%
\pgfsetbuttcap%
\pgfsetroundjoin%
\definecolor{currentfill}{rgb}{0.000000,0.000000,0.000000}%
\pgfsetfillcolor{currentfill}%
\pgfsetlinewidth{0.803000pt}%
\definecolor{currentstroke}{rgb}{0.000000,0.000000,0.000000}%
\pgfsetstrokecolor{currentstroke}%
\pgfsetdash{}{0pt}%
\pgfsys@defobject{currentmarker}{\pgfqpoint{0.000000in}{-0.048611in}}{\pgfqpoint{0.000000in}{0.000000in}}{%
\pgfpathmoveto{\pgfqpoint{0.000000in}{0.000000in}}%
\pgfpathlineto{\pgfqpoint{0.000000in}{-0.048611in}}%
\pgfusepath{stroke,fill}%
}%
\begin{pgfscope}%
\pgfsys@transformshift{3.805298in}{2.351653in}%
\pgfsys@useobject{currentmarker}{}%
\end{pgfscope}%
\end{pgfscope}%
\begin{pgfscope}%
\pgfsetbuttcap%
\pgfsetroundjoin%
\definecolor{currentfill}{rgb}{0.000000,0.000000,0.000000}%
\pgfsetfillcolor{currentfill}%
\pgfsetlinewidth{0.803000pt}%
\definecolor{currentstroke}{rgb}{0.000000,0.000000,0.000000}%
\pgfsetstrokecolor{currentstroke}%
\pgfsetdash{}{0pt}%
\pgfsys@defobject{currentmarker}{\pgfqpoint{0.000000in}{-0.048611in}}{\pgfqpoint{0.000000in}{0.000000in}}{%
\pgfpathmoveto{\pgfqpoint{0.000000in}{0.000000in}}%
\pgfpathlineto{\pgfqpoint{0.000000in}{-0.048611in}}%
\pgfusepath{stroke,fill}%
}%
\begin{pgfscope}%
\pgfsys@transformshift{4.324344in}{2.351653in}%
\pgfsys@useobject{currentmarker}{}%
\end{pgfscope}%
\end{pgfscope}%
\begin{pgfscope}%
\pgfsetbuttcap%
\pgfsetroundjoin%
\definecolor{currentfill}{rgb}{0.000000,0.000000,0.000000}%
\pgfsetfillcolor{currentfill}%
\pgfsetlinewidth{0.803000pt}%
\definecolor{currentstroke}{rgb}{0.000000,0.000000,0.000000}%
\pgfsetstrokecolor{currentstroke}%
\pgfsetdash{}{0pt}%
\pgfsys@defobject{currentmarker}{\pgfqpoint{0.000000in}{-0.048611in}}{\pgfqpoint{0.000000in}{0.000000in}}{%
\pgfpathmoveto{\pgfqpoint{0.000000in}{0.000000in}}%
\pgfpathlineto{\pgfqpoint{0.000000in}{-0.048611in}}%
\pgfusepath{stroke,fill}%
}%
\begin{pgfscope}%
\pgfsys@transformshift{4.843390in}{2.351653in}%
\pgfsys@useobject{currentmarker}{}%
\end{pgfscope}%
\end{pgfscope}%
\begin{pgfscope}%
\pgfsetbuttcap%
\pgfsetroundjoin%
\definecolor{currentfill}{rgb}{0.000000,0.000000,0.000000}%
\pgfsetfillcolor{currentfill}%
\pgfsetlinewidth{0.803000pt}%
\definecolor{currentstroke}{rgb}{0.000000,0.000000,0.000000}%
\pgfsetstrokecolor{currentstroke}%
\pgfsetdash{}{0pt}%
\pgfsys@defobject{currentmarker}{\pgfqpoint{-0.048611in}{0.000000in}}{\pgfqpoint{-0.000000in}{0.000000in}}{%
\pgfpathmoveto{\pgfqpoint{-0.000000in}{0.000000in}}%
\pgfpathlineto{\pgfqpoint{-0.048611in}{0.000000in}}%
\pgfusepath{stroke,fill}%
}%
\begin{pgfscope}%
\pgfsys@transformshift{0.483776in}{2.532831in}%
\pgfsys@useobject{currentmarker}{}%
\end{pgfscope}%
\end{pgfscope}%
\begin{pgfscope}%
\definecolor{textcolor}{rgb}{0.000000,0.000000,0.000000}%
\pgfsetstrokecolor{textcolor}%
\pgfsetfillcolor{textcolor}%
\pgftext[x=0.327525in, y=2.494275in, left, base]{\color{textcolor}\rmfamily\fontsize{8.000000}{9.600000}\selectfont \(\displaystyle {0}\)}%
\end{pgfscope}%
\begin{pgfscope}%
\pgfsetbuttcap%
\pgfsetroundjoin%
\definecolor{currentfill}{rgb}{0.000000,0.000000,0.000000}%
\pgfsetfillcolor{currentfill}%
\pgfsetlinewidth{0.803000pt}%
\definecolor{currentstroke}{rgb}{0.000000,0.000000,0.000000}%
\pgfsetstrokecolor{currentstroke}%
\pgfsetdash{}{0pt}%
\pgfsys@defobject{currentmarker}{\pgfqpoint{-0.048611in}{0.000000in}}{\pgfqpoint{-0.000000in}{0.000000in}}{%
\pgfpathmoveto{\pgfqpoint{-0.000000in}{0.000000in}}%
\pgfpathlineto{\pgfqpoint{-0.048611in}{0.000000in}}%
\pgfusepath{stroke,fill}%
}%
\begin{pgfscope}%
\pgfsys@transformshift{0.483776in}{2.740006in}%
\pgfsys@useobject{currentmarker}{}%
\end{pgfscope}%
\end{pgfscope}%
\begin{pgfscope}%
\definecolor{textcolor}{rgb}{0.000000,0.000000,0.000000}%
\pgfsetstrokecolor{textcolor}%
\pgfsetfillcolor{textcolor}%
\pgftext[x=0.327525in, y=2.701450in, left, base]{\color{textcolor}\rmfamily\fontsize{8.000000}{9.600000}\selectfont \(\displaystyle {5}\)}%
\end{pgfscope}%
\begin{pgfscope}%
\definecolor{textcolor}{rgb}{0.000000,0.000000,0.000000}%
\pgfsetstrokecolor{textcolor}%
\pgfsetfillcolor{textcolor}%
\pgftext[x=0.483776in,y=2.978201in,left,base]{\color{textcolor}\rmfamily\fontsize{8.000000}{9.600000}\selectfont \(\displaystyle \times{10^{\ensuremath{-}6}}{}\)}%
\end{pgfscope}%
\begin{pgfscope}%
\pgfpathrectangle{\pgfqpoint{0.483776in}{2.351653in}}{\pgfqpoint{4.566474in}{0.584881in}}%
\pgfusepath{clip}%
\pgfsetrectcap%
\pgfsetroundjoin%
\pgfsetlinewidth{0.501875pt}%
\definecolor{currentstroke}{rgb}{0.121569,0.466667,0.705882}%
\pgfsetstrokecolor{currentstroke}%
\pgfsetstrokeopacity{0.700000}%
\pgfsetdash{}{0pt}%
\pgfpathmoveto{\pgfqpoint{0.691343in}{2.530219in}}%
\pgfpathlineto{\pgfqpoint{0.692205in}{2.507777in}}%
\pgfpathlineto{\pgfqpoint{0.693071in}{2.590474in}}%
\pgfpathlineto{\pgfqpoint{0.694800in}{2.506363in}}%
\pgfpathlineto{\pgfqpoint{0.695666in}{2.545836in}}%
\pgfpathlineto{\pgfqpoint{0.696532in}{2.498186in}}%
\pgfpathlineto{\pgfqpoint{0.697397in}{2.501383in}}%
\pgfpathlineto{\pgfqpoint{0.699128in}{2.538888in}}%
\pgfpathlineto{\pgfqpoint{0.699993in}{2.536184in}}%
\pgfpathlineto{\pgfqpoint{0.700859in}{2.601850in}}%
\pgfpathlineto{\pgfqpoint{0.701725in}{2.538336in}}%
\pgfpathlineto{\pgfqpoint{0.702589in}{2.553523in}}%
\pgfpathlineto{\pgfqpoint{0.703453in}{2.510576in}}%
\pgfpathlineto{\pgfqpoint{0.706051in}{2.564068in}}%
\pgfpathlineto{\pgfqpoint{0.707780in}{2.536184in}}%
\pgfpathlineto{\pgfqpoint{0.708646in}{2.529913in}}%
\pgfpathlineto{\pgfqpoint{0.709512in}{2.574367in}}%
\pgfpathlineto{\pgfqpoint{0.710377in}{2.521305in}}%
\pgfpathlineto{\pgfqpoint{0.712105in}{2.585309in}}%
\pgfpathlineto{\pgfqpoint{0.712971in}{2.517924in}}%
\pgfpathlineto{\pgfqpoint{0.713837in}{2.551432in}}%
\pgfpathlineto{\pgfqpoint{0.714702in}{2.479803in}}%
\pgfpathlineto{\pgfqpoint{0.716430in}{2.549464in}}%
\pgfpathlineto{\pgfqpoint{0.719030in}{2.475436in}}%
\pgfpathlineto{\pgfqpoint{0.719895in}{2.469535in}}%
\pgfpathlineto{\pgfqpoint{0.720762in}{2.400426in}}%
\pgfpathlineto{\pgfqpoint{0.721627in}{2.506486in}}%
\pgfpathlineto{\pgfqpoint{0.722492in}{2.468950in}}%
\pgfpathlineto{\pgfqpoint{0.723356in}{2.490807in}}%
\pgfpathlineto{\pgfqpoint{0.724219in}{2.547435in}}%
\pgfpathlineto{\pgfqpoint{0.725084in}{2.460004in}}%
\pgfpathlineto{\pgfqpoint{0.725947in}{2.531019in}}%
\pgfpathlineto{\pgfqpoint{0.726812in}{2.496280in}}%
\pgfpathlineto{\pgfqpoint{0.728541in}{2.519030in}}%
\pgfpathlineto{\pgfqpoint{0.729407in}{2.497203in}}%
\pgfpathlineto{\pgfqpoint{0.730271in}{2.584758in}}%
\pgfpathlineto{\pgfqpoint{0.732865in}{2.466092in}}%
\pgfpathlineto{\pgfqpoint{0.733731in}{2.535384in}}%
\pgfpathlineto{\pgfqpoint{0.734597in}{2.484044in}}%
\pgfpathlineto{\pgfqpoint{0.735460in}{2.484475in}}%
\pgfpathlineto{\pgfqpoint{0.736325in}{2.528682in}}%
\pgfpathlineto{\pgfqpoint{0.737190in}{2.491331in}}%
\pgfpathlineto{\pgfqpoint{0.738056in}{2.494989in}}%
\pgfpathlineto{\pgfqpoint{0.738920in}{2.504888in}}%
\pgfpathlineto{\pgfqpoint{0.740649in}{2.546144in}}%
\pgfpathlineto{\pgfqpoint{0.742381in}{2.482078in}}%
\pgfpathlineto{\pgfqpoint{0.743246in}{2.512636in}}%
\pgfpathlineto{\pgfqpoint{0.744112in}{2.503659in}}%
\pgfpathlineto{\pgfqpoint{0.744977in}{2.529852in}}%
\pgfpathlineto{\pgfqpoint{0.745840in}{2.522658in}}%
\pgfpathlineto{\pgfqpoint{0.746705in}{2.526285in}}%
\pgfpathlineto{\pgfqpoint{0.747569in}{2.494989in}}%
\pgfpathlineto{\pgfqpoint{0.748435in}{2.558934in}}%
\pgfpathlineto{\pgfqpoint{0.749299in}{2.542886in}}%
\pgfpathlineto{\pgfqpoint{0.750163in}{2.552294in}}%
\pgfpathlineto{\pgfqpoint{0.751028in}{2.540704in}}%
\pgfpathlineto{\pgfqpoint{0.752757in}{2.494006in}}%
\pgfpathlineto{\pgfqpoint{0.753622in}{2.550357in}}%
\pgfpathlineto{\pgfqpoint{0.754488in}{2.440761in}}%
\pgfpathlineto{\pgfqpoint{0.755354in}{2.493268in}}%
\pgfpathlineto{\pgfqpoint{0.756219in}{2.471287in}}%
\pgfpathlineto{\pgfqpoint{0.757950in}{2.535078in}}%
\pgfpathlineto{\pgfqpoint{0.758816in}{2.534524in}}%
\pgfpathlineto{\pgfqpoint{0.759679in}{2.531450in}}%
\pgfpathlineto{\pgfqpoint{0.760542in}{2.495420in}}%
\pgfpathlineto{\pgfqpoint{0.761407in}{2.534339in}}%
\pgfpathlineto{\pgfqpoint{0.762269in}{2.509685in}}%
\pgfpathlineto{\pgfqpoint{0.763135in}{2.514727in}}%
\pgfpathlineto{\pgfqpoint{0.764000in}{2.521429in}}%
\pgfpathlineto{\pgfqpoint{0.764865in}{2.551617in}}%
\pgfpathlineto{\pgfqpoint{0.765730in}{2.507256in}}%
\pgfpathlineto{\pgfqpoint{0.766596in}{2.566988in}}%
\pgfpathlineto{\pgfqpoint{0.767462in}{2.500277in}}%
\pgfpathlineto{\pgfqpoint{0.769191in}{2.567172in}}%
\pgfpathlineto{\pgfqpoint{0.770922in}{2.489763in}}%
\pgfpathlineto{\pgfqpoint{0.771788in}{2.603141in}}%
\pgfpathlineto{\pgfqpoint{0.772652in}{2.517216in}}%
\pgfpathlineto{\pgfqpoint{0.773518in}{2.550939in}}%
\pgfpathlineto{\pgfqpoint{0.774383in}{2.502920in}}%
\pgfpathlineto{\pgfqpoint{0.775249in}{2.504857in}}%
\pgfpathlineto{\pgfqpoint{0.776979in}{2.541778in}}%
\pgfpathlineto{\pgfqpoint{0.779575in}{2.501568in}}%
\pgfpathlineto{\pgfqpoint{0.780439in}{2.532310in}}%
\pgfpathlineto{\pgfqpoint{0.781305in}{2.524870in}}%
\pgfpathlineto{\pgfqpoint{0.782170in}{2.511128in}}%
\pgfpathlineto{\pgfqpoint{0.783036in}{2.593610in}}%
\pgfpathlineto{\pgfqpoint{0.784766in}{2.501321in}}%
\pgfpathlineto{\pgfqpoint{0.785630in}{2.528990in}}%
\pgfpathlineto{\pgfqpoint{0.787360in}{2.487180in}}%
\pgfpathlineto{\pgfqpoint{0.788225in}{2.466951in}}%
\pgfpathlineto{\pgfqpoint{0.789089in}{2.410386in}}%
\pgfpathlineto{\pgfqpoint{0.789954in}{2.538704in}}%
\pgfpathlineto{\pgfqpoint{0.790816in}{2.520444in}}%
\pgfpathlineto{\pgfqpoint{0.791680in}{2.480601in}}%
\pgfpathlineto{\pgfqpoint{0.792546in}{2.576702in}}%
\pgfpathlineto{\pgfqpoint{0.793412in}{2.571906in}}%
\pgfpathlineto{\pgfqpoint{0.794274in}{2.527576in}}%
\pgfpathlineto{\pgfqpoint{0.795138in}{2.567480in}}%
\pgfpathlineto{\pgfqpoint{0.796004in}{2.509254in}}%
\pgfpathlineto{\pgfqpoint{0.796868in}{2.534216in}}%
\pgfpathlineto{\pgfqpoint{0.798599in}{2.477897in}}%
\pgfpathlineto{\pgfqpoint{0.799464in}{2.543440in}}%
\pgfpathlineto{\pgfqpoint{0.800327in}{2.537967in}}%
\pgfpathlineto{\pgfqpoint{0.801192in}{2.472855in}}%
\pgfpathlineto{\pgfqpoint{0.802058in}{2.568894in}}%
\pgfpathlineto{\pgfqpoint{0.803787in}{2.451429in}}%
\pgfpathlineto{\pgfqpoint{0.804652in}{2.464616in}}%
\pgfpathlineto{\pgfqpoint{0.805515in}{2.514727in}}%
\pgfpathlineto{\pgfqpoint{0.806378in}{2.440576in}}%
\pgfpathlineto{\pgfqpoint{0.808107in}{2.534401in}}%
\pgfpathlineto{\pgfqpoint{0.808973in}{2.535938in}}%
\pgfpathlineto{\pgfqpoint{0.809837in}{2.465722in}}%
\pgfpathlineto{\pgfqpoint{0.812431in}{2.522350in}}%
\pgfpathlineto{\pgfqpoint{0.813297in}{2.496218in}}%
\pgfpathlineto{\pgfqpoint{0.814161in}{2.511713in}}%
\pgfpathlineto{\pgfqpoint{0.815026in}{2.557949in}}%
\pgfpathlineto{\pgfqpoint{0.817620in}{2.473284in}}%
\pgfpathlineto{\pgfqpoint{0.818484in}{2.579837in}}%
\pgfpathlineto{\pgfqpoint{0.819349in}{2.551309in}}%
\pgfpathlineto{\pgfqpoint{0.820213in}{2.504272in}}%
\pgfpathlineto{\pgfqpoint{0.821078in}{2.576763in}}%
\pgfpathlineto{\pgfqpoint{0.821942in}{2.542455in}}%
\pgfpathlineto{\pgfqpoint{0.822807in}{2.555642in}}%
\pgfpathlineto{\pgfqpoint{0.823671in}{2.672556in}}%
\pgfpathlineto{\pgfqpoint{0.824536in}{2.516693in}}%
\pgfpathlineto{\pgfqpoint{0.825401in}{2.582297in}}%
\pgfpathlineto{\pgfqpoint{0.826268in}{2.535630in}}%
\pgfpathlineto{\pgfqpoint{0.827133in}{2.539596in}}%
\pgfpathlineto{\pgfqpoint{0.827999in}{2.555244in}}%
\pgfpathlineto{\pgfqpoint{0.828864in}{2.551924in}}%
\pgfpathlineto{\pgfqpoint{0.829730in}{2.505565in}}%
\pgfpathlineto{\pgfqpoint{0.831462in}{2.639784in}}%
\pgfpathlineto{\pgfqpoint{0.832328in}{2.563391in}}%
\pgfpathlineto{\pgfqpoint{0.833193in}{2.597730in}}%
\pgfpathlineto{\pgfqpoint{0.835786in}{2.510053in}}%
\pgfpathlineto{\pgfqpoint{0.836650in}{2.530650in}}%
\pgfpathlineto{\pgfqpoint{0.837516in}{2.501814in}}%
\pgfpathlineto{\pgfqpoint{0.838382in}{2.518968in}}%
\pgfpathlineto{\pgfqpoint{0.840114in}{2.487303in}}%
\pgfpathlineto{\pgfqpoint{0.840980in}{2.522042in}}%
\pgfpathlineto{\pgfqpoint{0.841845in}{2.510145in}}%
\pgfpathlineto{\pgfqpoint{0.842710in}{2.472239in}}%
\pgfpathlineto{\pgfqpoint{0.843575in}{2.501014in}}%
\pgfpathlineto{\pgfqpoint{0.844440in}{2.458220in}}%
\pgfpathlineto{\pgfqpoint{0.845305in}{2.550509in}}%
\pgfpathlineto{\pgfqpoint{0.847033in}{2.473961in}}%
\pgfpathlineto{\pgfqpoint{0.847897in}{2.532125in}}%
\pgfpathlineto{\pgfqpoint{0.849627in}{2.491485in}}%
\pgfpathlineto{\pgfqpoint{0.850491in}{2.489763in}}%
\pgfpathlineto{\pgfqpoint{0.851354in}{2.469073in}}%
\pgfpathlineto{\pgfqpoint{0.853081in}{2.535568in}}%
\pgfpathlineto{\pgfqpoint{0.853945in}{2.515033in}}%
\pgfpathlineto{\pgfqpoint{0.854810in}{2.520690in}}%
\pgfpathlineto{\pgfqpoint{0.855672in}{2.572858in}}%
\pgfpathlineto{\pgfqpoint{0.856537in}{2.558072in}}%
\pgfpathlineto{\pgfqpoint{0.857402in}{2.494250in}}%
\pgfpathlineto{\pgfqpoint{0.858267in}{2.561330in}}%
\pgfpathlineto{\pgfqpoint{0.859131in}{2.549218in}}%
\pgfpathlineto{\pgfqpoint{0.859997in}{2.479187in}}%
\pgfpathlineto{\pgfqpoint{0.860864in}{2.482630in}}%
\pgfpathlineto{\pgfqpoint{0.862597in}{2.580022in}}%
\pgfpathlineto{\pgfqpoint{0.863463in}{2.521796in}}%
\pgfpathlineto{\pgfqpoint{0.864329in}{2.544176in}}%
\pgfpathlineto{\pgfqpoint{0.866062in}{2.483798in}}%
\pgfpathlineto{\pgfqpoint{0.866929in}{2.519643in}}%
\pgfpathlineto{\pgfqpoint{0.867794in}{2.512511in}}%
\pgfpathlineto{\pgfqpoint{0.868659in}{2.511035in}}%
\pgfpathlineto{\pgfqpoint{0.869523in}{2.503841in}}%
\pgfpathlineto{\pgfqpoint{0.870388in}{2.487210in}}%
\pgfpathlineto{\pgfqpoint{0.872119in}{2.555920in}}%
\pgfpathlineto{\pgfqpoint{0.873848in}{2.522902in}}%
\pgfpathlineto{\pgfqpoint{0.874712in}{2.512141in}}%
\pgfpathlineto{\pgfqpoint{0.875577in}{2.538581in}}%
\pgfpathlineto{\pgfqpoint{0.876442in}{2.496955in}}%
\pgfpathlineto{\pgfqpoint{0.877309in}{2.512388in}}%
\pgfpathlineto{\pgfqpoint{0.878174in}{2.500275in}}%
\pgfpathlineto{\pgfqpoint{0.879035in}{2.516323in}}%
\pgfpathlineto{\pgfqpoint{0.879903in}{2.487333in}}%
\pgfpathlineto{\pgfqpoint{0.880769in}{2.493696in}}%
\pgfpathlineto{\pgfqpoint{0.881633in}{2.531694in}}%
\pgfpathlineto{\pgfqpoint{0.882498in}{2.495787in}}%
\pgfpathlineto{\pgfqpoint{0.884229in}{2.560499in}}%
\pgfpathlineto{\pgfqpoint{0.885096in}{2.504395in}}%
\pgfpathlineto{\pgfqpoint{0.886826in}{2.583126in}}%
\pgfpathlineto{\pgfqpoint{0.887691in}{2.422498in}}%
\pgfpathlineto{\pgfqpoint{0.889418in}{2.543961in}}%
\pgfpathlineto{\pgfqpoint{0.890283in}{2.565880in}}%
\pgfpathlineto{\pgfqpoint{0.891148in}{2.557395in}}%
\pgfpathlineto{\pgfqpoint{0.892011in}{2.564035in}}%
\pgfpathlineto{\pgfqpoint{0.893741in}{2.529726in}}%
\pgfpathlineto{\pgfqpoint{0.894605in}{2.551553in}}%
\pgfpathlineto{\pgfqpoint{0.896337in}{2.531725in}}%
\pgfpathlineto{\pgfqpoint{0.897202in}{2.561146in}}%
\pgfpathlineto{\pgfqpoint{0.898931in}{2.473561in}}%
\pgfpathlineto{\pgfqpoint{0.899795in}{2.509252in}}%
\pgfpathlineto{\pgfqpoint{0.900660in}{2.595085in}}%
\pgfpathlineto{\pgfqpoint{0.901525in}{2.502735in}}%
\pgfpathlineto{\pgfqpoint{0.902390in}{2.559547in}}%
\pgfpathlineto{\pgfqpoint{0.903254in}{2.503872in}}%
\pgfpathlineto{\pgfqpoint{0.904119in}{2.575288in}}%
\pgfpathlineto{\pgfqpoint{0.905849in}{2.545405in}}%
\pgfpathlineto{\pgfqpoint{0.907579in}{2.600250in}}%
\pgfpathlineto{\pgfqpoint{0.908445in}{2.501075in}}%
\pgfpathlineto{\pgfqpoint{0.910175in}{2.834938in}}%
\pgfpathlineto{\pgfqpoint{0.911039in}{2.755746in}}%
\pgfpathlineto{\pgfqpoint{0.911905in}{2.798847in}}%
\pgfpathlineto{\pgfqpoint{0.912770in}{2.696475in}}%
\pgfpathlineto{\pgfqpoint{0.913635in}{2.736931in}}%
\pgfpathlineto{\pgfqpoint{0.914501in}{2.692170in}}%
\pgfpathlineto{\pgfqpoint{0.916229in}{2.784951in}}%
\pgfpathlineto{\pgfqpoint{0.917094in}{2.781508in}}%
\pgfpathlineto{\pgfqpoint{0.917959in}{2.753809in}}%
\pgfpathlineto{\pgfqpoint{0.918824in}{2.549341in}}%
\pgfpathlineto{\pgfqpoint{0.919687in}{2.820183in}}%
\pgfpathlineto{\pgfqpoint{0.920552in}{2.780525in}}%
\pgfpathlineto{\pgfqpoint{0.921417in}{2.732813in}}%
\pgfpathlineto{\pgfqpoint{0.922280in}{2.833032in}}%
\pgfpathlineto{\pgfqpoint{0.924010in}{2.523517in}}%
\pgfpathlineto{\pgfqpoint{0.925740in}{2.485827in}}%
\pgfpathlineto{\pgfqpoint{0.927464in}{2.528467in}}%
\pgfpathlineto{\pgfqpoint{0.929193in}{2.484229in}}%
\pgfpathlineto{\pgfqpoint{0.930058in}{2.556166in}}%
\pgfpathlineto{\pgfqpoint{0.931787in}{2.469717in}}%
\pgfpathlineto{\pgfqpoint{0.932652in}{2.481645in}}%
\pgfpathlineto{\pgfqpoint{0.934381in}{2.541685in}}%
\pgfpathlineto{\pgfqpoint{0.935246in}{2.506055in}}%
\pgfpathlineto{\pgfqpoint{0.936109in}{2.512634in}}%
\pgfpathlineto{\pgfqpoint{0.936974in}{2.520934in}}%
\pgfpathlineto{\pgfqpoint{0.937840in}{2.507223in}}%
\pgfpathlineto{\pgfqpoint{0.938706in}{2.528251in}}%
\pgfpathlineto{\pgfqpoint{0.939570in}{2.454285in}}%
\pgfpathlineto{\pgfqpoint{0.941301in}{2.497938in}}%
\pgfpathlineto{\pgfqpoint{0.943032in}{2.479862in}}%
\pgfpathlineto{\pgfqpoint{0.943897in}{2.565664in}}%
\pgfpathlineto{\pgfqpoint{0.944763in}{2.555612in}}%
\pgfpathlineto{\pgfqpoint{0.945628in}{2.568953in}}%
\pgfpathlineto{\pgfqpoint{0.946490in}{2.480109in}}%
\pgfpathlineto{\pgfqpoint{0.947355in}{2.483921in}}%
\pgfpathlineto{\pgfqpoint{0.948218in}{2.525177in}}%
\pgfpathlineto{\pgfqpoint{0.951674in}{2.469410in}}%
\pgfpathlineto{\pgfqpoint{0.954269in}{2.527820in}}%
\pgfpathlineto{\pgfqpoint{0.955133in}{2.489270in}}%
\pgfpathlineto{\pgfqpoint{0.958594in}{2.554198in}}%
\pgfpathlineto{\pgfqpoint{0.959459in}{2.458344in}}%
\pgfpathlineto{\pgfqpoint{0.960325in}{2.558103in}}%
\pgfpathlineto{\pgfqpoint{0.961191in}{2.541041in}}%
\pgfpathlineto{\pgfqpoint{0.962055in}{2.594040in}}%
\pgfpathlineto{\pgfqpoint{0.963783in}{2.480847in}}%
\pgfpathlineto{\pgfqpoint{0.964647in}{2.514786in}}%
\pgfpathlineto{\pgfqpoint{0.965512in}{2.468581in}}%
\pgfpathlineto{\pgfqpoint{0.966375in}{2.564712in}}%
\pgfpathlineto{\pgfqpoint{0.967241in}{2.549279in}}%
\pgfpathlineto{\pgfqpoint{0.968105in}{2.561577in}}%
\pgfpathlineto{\pgfqpoint{0.968970in}{2.459267in}}%
\pgfpathlineto{\pgfqpoint{0.969836in}{2.547373in}}%
\pgfpathlineto{\pgfqpoint{0.970701in}{2.468858in}}%
\pgfpathlineto{\pgfqpoint{0.971567in}{2.494743in}}%
\pgfpathlineto{\pgfqpoint{0.972432in}{2.494681in}}%
\pgfpathlineto{\pgfqpoint{0.973297in}{2.508392in}}%
\pgfpathlineto{\pgfqpoint{0.974162in}{2.473899in}}%
\pgfpathlineto{\pgfqpoint{0.975027in}{2.506517in}}%
\pgfpathlineto{\pgfqpoint{0.975889in}{2.595454in}}%
\pgfpathlineto{\pgfqpoint{0.977616in}{2.480724in}}%
\pgfpathlineto{\pgfqpoint{0.978481in}{2.523517in}}%
\pgfpathlineto{\pgfqpoint{0.979346in}{2.518476in}}%
\pgfpathlineto{\pgfqpoint{0.980211in}{2.480539in}}%
\pgfpathlineto{\pgfqpoint{0.981076in}{2.562929in}}%
\pgfpathlineto{\pgfqpoint{0.981941in}{2.547404in}}%
\pgfpathlineto{\pgfqpoint{0.982806in}{2.574980in}}%
\pgfpathlineto{\pgfqpoint{0.984535in}{2.477404in}}%
\pgfpathlineto{\pgfqpoint{0.985400in}{2.521980in}}%
\pgfpathlineto{\pgfqpoint{0.986264in}{2.462156in}}%
\pgfpathlineto{\pgfqpoint{0.987127in}{2.527022in}}%
\pgfpathlineto{\pgfqpoint{0.987991in}{2.429569in}}%
\pgfpathlineto{\pgfqpoint{0.988856in}{2.514879in}}%
\pgfpathlineto{\pgfqpoint{0.989722in}{2.511158in}}%
\pgfpathlineto{\pgfqpoint{0.990585in}{2.512942in}}%
\pgfpathlineto{\pgfqpoint{0.991449in}{2.500860in}}%
\pgfpathlineto{\pgfqpoint{0.992314in}{2.526714in}}%
\pgfpathlineto{\pgfqpoint{0.993178in}{2.514602in}}%
\pgfpathlineto{\pgfqpoint{0.994043in}{2.534309in}}%
\pgfpathlineto{\pgfqpoint{0.994907in}{2.517062in}}%
\pgfpathlineto{\pgfqpoint{0.995771in}{2.468919in}}%
\pgfpathlineto{\pgfqpoint{0.997501in}{2.512880in}}%
\pgfpathlineto{\pgfqpoint{0.998367in}{2.528467in}}%
\pgfpathlineto{\pgfqpoint{1.000096in}{2.502920in}}%
\pgfpathlineto{\pgfqpoint{1.000961in}{2.560530in}}%
\pgfpathlineto{\pgfqpoint{1.001826in}{2.543684in}}%
\pgfpathlineto{\pgfqpoint{1.002691in}{2.549464in}}%
\pgfpathlineto{\pgfqpoint{1.004419in}{2.516323in}}%
\pgfpathlineto{\pgfqpoint{1.005284in}{2.516508in}}%
\pgfpathlineto{\pgfqpoint{1.006149in}{2.506548in}}%
\pgfpathlineto{\pgfqpoint{1.007014in}{2.552538in}}%
\pgfpathlineto{\pgfqpoint{1.007877in}{2.514355in}}%
\pgfpathlineto{\pgfqpoint{1.009607in}{2.556841in}}%
\pgfpathlineto{\pgfqpoint{1.012203in}{2.436700in}}%
\pgfpathlineto{\pgfqpoint{1.013933in}{2.590597in}}%
\pgfpathlineto{\pgfqpoint{1.014798in}{2.534278in}}%
\pgfpathlineto{\pgfqpoint{1.015662in}{2.474453in}}%
\pgfpathlineto{\pgfqpoint{1.017391in}{2.553277in}}%
\pgfpathlineto{\pgfqpoint{1.018255in}{2.585556in}}%
\pgfpathlineto{\pgfqpoint{1.019120in}{2.565328in}}%
\pgfpathlineto{\pgfqpoint{1.019986in}{2.489671in}}%
\pgfpathlineto{\pgfqpoint{1.020851in}{2.530711in}}%
\pgfpathlineto{\pgfqpoint{1.021716in}{2.492344in}}%
\pgfpathlineto{\pgfqpoint{1.022581in}{2.496311in}}%
\pgfpathlineto{\pgfqpoint{1.023445in}{2.487487in}}%
\pgfpathlineto{\pgfqpoint{1.025173in}{2.552353in}}%
\pgfpathlineto{\pgfqpoint{1.026903in}{2.506917in}}%
\pgfpathlineto{\pgfqpoint{1.027767in}{2.494620in}}%
\pgfpathlineto{\pgfqpoint{1.028629in}{2.617773in}}%
\pgfpathlineto{\pgfqpoint{1.030357in}{2.505072in}}%
\pgfpathlineto{\pgfqpoint{1.031221in}{2.541410in}}%
\pgfpathlineto{\pgfqpoint{1.032087in}{2.529636in}}%
\pgfpathlineto{\pgfqpoint{1.033816in}{2.555060in}}%
\pgfpathlineto{\pgfqpoint{1.035545in}{2.463631in}}%
\pgfpathlineto{\pgfqpoint{1.037275in}{2.479372in}}%
\pgfpathlineto{\pgfqpoint{1.038139in}{2.472239in}}%
\pgfpathlineto{\pgfqpoint{1.040734in}{2.553767in}}%
\pgfpathlineto{\pgfqpoint{1.041600in}{2.555858in}}%
\pgfpathlineto{\pgfqpoint{1.042466in}{2.565143in}}%
\pgfpathlineto{\pgfqpoint{1.043331in}{2.563360in}}%
\pgfpathlineto{\pgfqpoint{1.044195in}{2.552661in}}%
\pgfpathlineto{\pgfqpoint{1.045060in}{2.594777in}}%
\pgfpathlineto{\pgfqpoint{1.045927in}{2.473561in}}%
\pgfpathlineto{\pgfqpoint{1.046792in}{2.524993in}}%
\pgfpathlineto{\pgfqpoint{1.047657in}{2.520259in}}%
\pgfpathlineto{\pgfqpoint{1.048523in}{2.540395in}}%
\pgfpathlineto{\pgfqpoint{1.050256in}{2.485827in}}%
\pgfpathlineto{\pgfqpoint{1.051122in}{2.507777in}}%
\pgfpathlineto{\pgfqpoint{1.051987in}{2.445187in}}%
\pgfpathlineto{\pgfqpoint{1.054584in}{2.557025in}}%
\pgfpathlineto{\pgfqpoint{1.055449in}{2.498707in}}%
\pgfpathlineto{\pgfqpoint{1.057179in}{2.536921in}}%
\pgfpathlineto{\pgfqpoint{1.059773in}{2.497386in}}%
\pgfpathlineto{\pgfqpoint{1.060637in}{2.446323in}}%
\pgfpathlineto{\pgfqpoint{1.061503in}{2.553952in}}%
\pgfpathlineto{\pgfqpoint{1.062369in}{2.518414in}}%
\pgfpathlineto{\pgfqpoint{1.063234in}{2.519643in}}%
\pgfpathlineto{\pgfqpoint{1.064099in}{2.507900in}}%
\pgfpathlineto{\pgfqpoint{1.064962in}{2.556595in}}%
\pgfpathlineto{\pgfqpoint{1.068422in}{2.469410in}}%
\pgfpathlineto{\pgfqpoint{1.069287in}{2.471808in}}%
\pgfpathlineto{\pgfqpoint{1.071017in}{2.570183in}}%
\pgfpathlineto{\pgfqpoint{1.073608in}{2.452563in}}%
\pgfpathlineto{\pgfqpoint{1.074470in}{2.531263in}}%
\pgfpathlineto{\pgfqpoint{1.075335in}{2.509252in}}%
\pgfpathlineto{\pgfqpoint{1.076199in}{2.456499in}}%
\pgfpathlineto{\pgfqpoint{1.077064in}{2.521734in}}%
\pgfpathlineto{\pgfqpoint{1.077927in}{2.519766in}}%
\pgfpathlineto{\pgfqpoint{1.078792in}{2.559301in}}%
\pgfpathlineto{\pgfqpoint{1.080521in}{2.451149in}}%
\pgfpathlineto{\pgfqpoint{1.081384in}{2.500706in}}%
\pgfpathlineto{\pgfqpoint{1.082248in}{2.474759in}}%
\pgfpathlineto{\pgfqpoint{1.083981in}{2.563789in}}%
\pgfpathlineto{\pgfqpoint{1.084846in}{2.555612in}}%
\pgfpathlineto{\pgfqpoint{1.085711in}{2.565880in}}%
\pgfpathlineto{\pgfqpoint{1.086576in}{2.458957in}}%
\pgfpathlineto{\pgfqpoint{1.088306in}{2.596252in}}%
\pgfpathlineto{\pgfqpoint{1.090037in}{2.520626in}}%
\pgfpathlineto{\pgfqpoint{1.090903in}{2.523885in}}%
\pgfpathlineto{\pgfqpoint{1.092632in}{2.492160in}}%
\pgfpathlineto{\pgfqpoint{1.093498in}{2.452625in}}%
\pgfpathlineto{\pgfqpoint{1.094362in}{2.517121in}}%
\pgfpathlineto{\pgfqpoint{1.095225in}{2.508760in}}%
\pgfpathlineto{\pgfqpoint{1.096090in}{2.539133in}}%
\pgfpathlineto{\pgfqpoint{1.097822in}{2.469040in}}%
\pgfpathlineto{\pgfqpoint{1.099551in}{2.585125in}}%
\pgfpathlineto{\pgfqpoint{1.101281in}{2.538273in}}%
\pgfpathlineto{\pgfqpoint{1.102145in}{2.544728in}}%
\pgfpathlineto{\pgfqpoint{1.103871in}{2.487118in}}%
\pgfpathlineto{\pgfqpoint{1.104736in}{2.521119in}}%
\pgfpathlineto{\pgfqpoint{1.106466in}{2.483182in}}%
\pgfpathlineto{\pgfqpoint{1.107328in}{2.523271in}}%
\pgfpathlineto{\pgfqpoint{1.108192in}{2.516446in}}%
\pgfpathlineto{\pgfqpoint{1.109055in}{2.521611in}}%
\pgfpathlineto{\pgfqpoint{1.109918in}{2.573933in}}%
\pgfpathlineto{\pgfqpoint{1.110782in}{2.491913in}}%
\pgfpathlineto{\pgfqpoint{1.111647in}{2.517306in}}%
\pgfpathlineto{\pgfqpoint{1.113375in}{2.447583in}}%
\pgfpathlineto{\pgfqpoint{1.115967in}{2.557518in}}%
\pgfpathlineto{\pgfqpoint{1.117696in}{2.504734in}}%
\pgfpathlineto{\pgfqpoint{1.118561in}{2.575780in}}%
\pgfpathlineto{\pgfqpoint{1.120289in}{2.467321in}}%
\pgfpathlineto{\pgfqpoint{1.122016in}{2.537629in}}%
\pgfpathlineto{\pgfqpoint{1.122881in}{2.487610in}}%
\pgfpathlineto{\pgfqpoint{1.123746in}{2.492898in}}%
\pgfpathlineto{\pgfqpoint{1.124612in}{2.501229in}}%
\pgfpathlineto{\pgfqpoint{1.126342in}{2.547681in}}%
\pgfpathlineto{\pgfqpoint{1.127208in}{2.494066in}}%
\pgfpathlineto{\pgfqpoint{1.128072in}{2.512757in}}%
\pgfpathlineto{\pgfqpoint{1.128936in}{2.510912in}}%
\pgfpathlineto{\pgfqpoint{1.129800in}{2.459880in}}%
\pgfpathlineto{\pgfqpoint{1.130665in}{2.506055in}}%
\pgfpathlineto{\pgfqpoint{1.131530in}{2.488501in}}%
\pgfpathlineto{\pgfqpoint{1.133261in}{2.538396in}}%
\pgfpathlineto{\pgfqpoint{1.134126in}{2.495264in}}%
\pgfpathlineto{\pgfqpoint{1.134991in}{2.518045in}}%
\pgfpathlineto{\pgfqpoint{1.135856in}{2.459388in}}%
\pgfpathlineto{\pgfqpoint{1.136722in}{2.518014in}}%
\pgfpathlineto{\pgfqpoint{1.137586in}{2.437869in}}%
\pgfpathlineto{\pgfqpoint{1.138451in}{2.535137in}}%
\pgfpathlineto{\pgfqpoint{1.139316in}{2.480478in}}%
\pgfpathlineto{\pgfqpoint{1.140181in}{2.499415in}}%
\pgfpathlineto{\pgfqpoint{1.141044in}{2.486841in}}%
\pgfpathlineto{\pgfqpoint{1.142776in}{2.569507in}}%
\pgfpathlineto{\pgfqpoint{1.144506in}{2.463385in}}%
\pgfpathlineto{\pgfqpoint{1.145371in}{2.495295in}}%
\pgfpathlineto{\pgfqpoint{1.146235in}{2.469625in}}%
\pgfpathlineto{\pgfqpoint{1.147966in}{2.566434in}}%
\pgfpathlineto{\pgfqpoint{1.148829in}{2.487364in}}%
\pgfpathlineto{\pgfqpoint{1.149692in}{2.508085in}}%
\pgfpathlineto{\pgfqpoint{1.150558in}{2.502489in}}%
\pgfpathlineto{\pgfqpoint{1.152290in}{2.558932in}}%
\pgfpathlineto{\pgfqpoint{1.154885in}{2.468427in}}%
\pgfpathlineto{\pgfqpoint{1.155749in}{2.533508in}}%
\pgfpathlineto{\pgfqpoint{1.156614in}{2.523517in}}%
\pgfpathlineto{\pgfqpoint{1.158343in}{2.492590in}}%
\pgfpathlineto{\pgfqpoint{1.159208in}{2.547312in}}%
\pgfpathlineto{\pgfqpoint{1.160073in}{2.530065in}}%
\pgfpathlineto{\pgfqpoint{1.160937in}{2.482384in}}%
\pgfpathlineto{\pgfqpoint{1.161802in}{2.611931in}}%
\pgfpathlineto{\pgfqpoint{1.162667in}{2.610702in}}%
\pgfpathlineto{\pgfqpoint{1.164394in}{2.600065in}}%
\pgfpathlineto{\pgfqpoint{1.165259in}{2.497694in}}%
\pgfpathlineto{\pgfqpoint{1.166124in}{2.528190in}}%
\pgfpathlineto{\pgfqpoint{1.166990in}{2.603877in}}%
\pgfpathlineto{\pgfqpoint{1.167854in}{2.583219in}}%
\pgfpathlineto{\pgfqpoint{1.171315in}{2.496095in}}%
\pgfpathlineto{\pgfqpoint{1.172180in}{2.515431in}}%
\pgfpathlineto{\pgfqpoint{1.173045in}{2.574488in}}%
\pgfpathlineto{\pgfqpoint{1.173912in}{2.559178in}}%
\pgfpathlineto{\pgfqpoint{1.174778in}{2.559178in}}%
\pgfpathlineto{\pgfqpoint{1.177374in}{2.461664in}}%
\pgfpathlineto{\pgfqpoint{1.178239in}{2.509806in}}%
\pgfpathlineto{\pgfqpoint{1.179103in}{2.441097in}}%
\pgfpathlineto{\pgfqpoint{1.180834in}{2.542886in}}%
\pgfpathlineto{\pgfqpoint{1.181699in}{2.588876in}}%
\pgfpathlineto{\pgfqpoint{1.183430in}{2.523271in}}%
\pgfpathlineto{\pgfqpoint{1.184295in}{2.576702in}}%
\pgfpathlineto{\pgfqpoint{1.186026in}{2.486135in}}%
\pgfpathlineto{\pgfqpoint{1.186891in}{2.590782in}}%
\pgfpathlineto{\pgfqpoint{1.187758in}{2.588937in}}%
\pgfpathlineto{\pgfqpoint{1.189487in}{2.513496in}}%
\pgfpathlineto{\pgfqpoint{1.190352in}{2.593117in}}%
\pgfpathlineto{\pgfqpoint{1.191218in}{2.492744in}}%
\pgfpathlineto{\pgfqpoint{1.193813in}{2.581743in}}%
\pgfpathlineto{\pgfqpoint{1.194678in}{2.539442in}}%
\pgfpathlineto{\pgfqpoint{1.195541in}{2.600681in}}%
\pgfpathlineto{\pgfqpoint{1.196406in}{2.472362in}}%
\pgfpathlineto{\pgfqpoint{1.197272in}{2.600927in}}%
\pgfpathlineto{\pgfqpoint{1.199001in}{2.544299in}}%
\pgfpathlineto{\pgfqpoint{1.199865in}{2.545898in}}%
\pgfpathlineto{\pgfqpoint{1.200729in}{2.532772in}}%
\pgfpathlineto{\pgfqpoint{1.201595in}{2.496280in}}%
\pgfpathlineto{\pgfqpoint{1.203324in}{2.535661in}}%
\pgfpathlineto{\pgfqpoint{1.204189in}{2.471441in}}%
\pgfpathlineto{\pgfqpoint{1.205054in}{2.528069in}}%
\pgfpathlineto{\pgfqpoint{1.205919in}{2.463018in}}%
\pgfpathlineto{\pgfqpoint{1.206785in}{2.517370in}}%
\pgfpathlineto{\pgfqpoint{1.207651in}{2.503505in}}%
\pgfpathlineto{\pgfqpoint{1.208516in}{2.554752in}}%
\pgfpathlineto{\pgfqpoint{1.210246in}{2.513219in}}%
\pgfpathlineto{\pgfqpoint{1.211111in}{2.493206in}}%
\pgfpathlineto{\pgfqpoint{1.212842in}{2.554044in}}%
\pgfpathlineto{\pgfqpoint{1.214572in}{2.510114in}}%
\pgfpathlineto{\pgfqpoint{1.216303in}{2.592011in}}%
\pgfpathlineto{\pgfqpoint{1.217169in}{2.535445in}}%
\pgfpathlineto{\pgfqpoint{1.218035in}{2.588322in}}%
\pgfpathlineto{\pgfqpoint{1.218902in}{2.547558in}}%
\pgfpathlineto{\pgfqpoint{1.219768in}{2.608888in}}%
\pgfpathlineto{\pgfqpoint{1.220634in}{2.502674in}}%
\pgfpathlineto{\pgfqpoint{1.221500in}{2.524010in}}%
\pgfpathlineto{\pgfqpoint{1.222366in}{2.494281in}}%
\pgfpathlineto{\pgfqpoint{1.224098in}{2.561330in}}%
\pgfpathlineto{\pgfqpoint{1.224965in}{2.537413in}}%
\pgfpathlineto{\pgfqpoint{1.225831in}{2.549464in}}%
\pgfpathlineto{\pgfqpoint{1.226697in}{2.474515in}}%
\pgfpathlineto{\pgfqpoint{1.227562in}{2.481401in}}%
\pgfpathlineto{\pgfqpoint{1.228427in}{2.565205in}}%
\pgfpathlineto{\pgfqpoint{1.229292in}{2.541102in}}%
\pgfpathlineto{\pgfqpoint{1.230156in}{2.479864in}}%
\pgfpathlineto{\pgfqpoint{1.231020in}{2.523825in}}%
\pgfpathlineto{\pgfqpoint{1.231885in}{2.497817in}}%
\pgfpathlineto{\pgfqpoint{1.233617in}{2.549772in}}%
\pgfpathlineto{\pgfqpoint{1.235347in}{2.460004in}}%
\pgfpathlineto{\pgfqpoint{1.237076in}{2.572951in}}%
\pgfpathlineto{\pgfqpoint{1.237942in}{2.579098in}}%
\pgfpathlineto{\pgfqpoint{1.238807in}{2.480355in}}%
\pgfpathlineto{\pgfqpoint{1.240537in}{2.557580in}}%
\pgfpathlineto{\pgfqpoint{1.241402in}{2.549064in}}%
\pgfpathlineto{\pgfqpoint{1.243133in}{2.489086in}}%
\pgfpathlineto{\pgfqpoint{1.243999in}{2.544115in}}%
\pgfpathlineto{\pgfqpoint{1.244864in}{2.501137in}}%
\pgfpathlineto{\pgfqpoint{1.245728in}{2.508239in}}%
\pgfpathlineto{\pgfqpoint{1.246593in}{2.553092in}}%
\pgfpathlineto{\pgfqpoint{1.247458in}{2.468981in}}%
\pgfpathlineto{\pgfqpoint{1.248323in}{2.473376in}}%
\pgfpathlineto{\pgfqpoint{1.249187in}{2.476358in}}%
\pgfpathlineto{\pgfqpoint{1.250053in}{2.558686in}}%
\pgfpathlineto{\pgfqpoint{1.250916in}{2.495326in}}%
\pgfpathlineto{\pgfqpoint{1.251781in}{2.508760in}}%
\pgfpathlineto{\pgfqpoint{1.252647in}{2.540179in}}%
\pgfpathlineto{\pgfqpoint{1.253512in}{2.526776in}}%
\pgfpathlineto{\pgfqpoint{1.254377in}{2.550878in}}%
\pgfpathlineto{\pgfqpoint{1.255243in}{2.489517in}}%
\pgfpathlineto{\pgfqpoint{1.256109in}{2.575103in}}%
\pgfpathlineto{\pgfqpoint{1.257840in}{2.519797in}}%
\pgfpathlineto{\pgfqpoint{1.258706in}{2.577746in}}%
\pgfpathlineto{\pgfqpoint{1.259571in}{2.474944in}}%
\pgfpathlineto{\pgfqpoint{1.260436in}{2.496095in}}%
\pgfpathlineto{\pgfqpoint{1.262165in}{2.559178in}}%
\pgfpathlineto{\pgfqpoint{1.263030in}{2.549033in}}%
\pgfpathlineto{\pgfqpoint{1.265625in}{2.428707in}}%
\pgfpathlineto{\pgfqpoint{1.267353in}{2.603416in}}%
\pgfpathlineto{\pgfqpoint{1.268216in}{2.590351in}}%
\pgfpathlineto{\pgfqpoint{1.269081in}{2.517983in}}%
\pgfpathlineto{\pgfqpoint{1.269946in}{2.536767in}}%
\pgfpathlineto{\pgfqpoint{1.270811in}{2.558809in}}%
\pgfpathlineto{\pgfqpoint{1.274272in}{2.496280in}}%
\pgfpathlineto{\pgfqpoint{1.275137in}{2.497694in}}%
\pgfpathlineto{\pgfqpoint{1.276003in}{2.529113in}}%
\pgfpathlineto{\pgfqpoint{1.276869in}{2.496341in}}%
\pgfpathlineto{\pgfqpoint{1.278600in}{2.620108in}}%
\pgfpathlineto{\pgfqpoint{1.280331in}{2.511035in}}%
\pgfpathlineto{\pgfqpoint{1.281196in}{2.514786in}}%
\pgfpathlineto{\pgfqpoint{1.282062in}{2.492098in}}%
\pgfpathlineto{\pgfqpoint{1.283793in}{2.574734in}}%
\pgfpathlineto{\pgfqpoint{1.284658in}{2.573320in}}%
\pgfpathlineto{\pgfqpoint{1.285521in}{2.592871in}}%
\pgfpathlineto{\pgfqpoint{1.287250in}{2.515033in}}%
\pgfpathlineto{\pgfqpoint{1.288980in}{2.538611in}}%
\pgfpathlineto{\pgfqpoint{1.290711in}{2.564404in}}%
\pgfpathlineto{\pgfqpoint{1.292440in}{2.523333in}}%
\pgfpathlineto{\pgfqpoint{1.293306in}{2.579775in}}%
\pgfpathlineto{\pgfqpoint{1.295037in}{2.525362in}}%
\pgfpathlineto{\pgfqpoint{1.295904in}{2.533354in}}%
\pgfpathlineto{\pgfqpoint{1.296768in}{2.555612in}}%
\pgfpathlineto{\pgfqpoint{1.297633in}{2.509868in}}%
\pgfpathlineto{\pgfqpoint{1.298498in}{2.390250in}}%
\pgfpathlineto{\pgfqpoint{1.300229in}{2.534830in}}%
\pgfpathlineto{\pgfqpoint{1.301092in}{2.457851in}}%
\pgfpathlineto{\pgfqpoint{1.301958in}{2.461356in}}%
\pgfpathlineto{\pgfqpoint{1.302823in}{2.454562in}}%
\pgfpathlineto{\pgfqpoint{1.303687in}{2.514355in}}%
\pgfpathlineto{\pgfqpoint{1.304552in}{2.512880in}}%
\pgfpathlineto{\pgfqpoint{1.305417in}{2.529788in}}%
\pgfpathlineto{\pgfqpoint{1.306282in}{2.518168in}}%
\pgfpathlineto{\pgfqpoint{1.307147in}{2.530096in}}%
\pgfpathlineto{\pgfqpoint{1.308013in}{2.465261in}}%
\pgfpathlineto{\pgfqpoint{1.308879in}{2.566064in}}%
\pgfpathlineto{\pgfqpoint{1.309745in}{2.500583in}}%
\pgfpathlineto{\pgfqpoint{1.310610in}{2.543930in}}%
\pgfpathlineto{\pgfqpoint{1.311476in}{2.524192in}}%
\pgfpathlineto{\pgfqpoint{1.312342in}{2.533016in}}%
\pgfpathlineto{\pgfqpoint{1.313208in}{2.486872in}}%
\pgfpathlineto{\pgfqpoint{1.314074in}{2.542945in}}%
\pgfpathlineto{\pgfqpoint{1.314939in}{2.530188in}}%
\pgfpathlineto{\pgfqpoint{1.315803in}{2.527266in}}%
\pgfpathlineto{\pgfqpoint{1.316665in}{2.527943in}}%
\pgfpathlineto{\pgfqpoint{1.317529in}{2.505655in}}%
\pgfpathlineto{\pgfqpoint{1.318395in}{2.424220in}}%
\pgfpathlineto{\pgfqpoint{1.319260in}{2.523086in}}%
\pgfpathlineto{\pgfqpoint{1.320124in}{2.516015in}}%
\pgfpathlineto{\pgfqpoint{1.320989in}{2.513863in}}%
\pgfpathlineto{\pgfqpoint{1.321854in}{2.544851in}}%
\pgfpathlineto{\pgfqpoint{1.322720in}{2.466213in}}%
\pgfpathlineto{\pgfqpoint{1.323583in}{2.516139in}}%
\pgfpathlineto{\pgfqpoint{1.324448in}{2.476450in}}%
\pgfpathlineto{\pgfqpoint{1.325313in}{2.547004in}}%
\pgfpathlineto{\pgfqpoint{1.327043in}{2.459234in}}%
\pgfpathlineto{\pgfqpoint{1.329636in}{2.539563in}}%
\pgfpathlineto{\pgfqpoint{1.330500in}{2.495541in}}%
\pgfpathlineto{\pgfqpoint{1.331365in}{2.500213in}}%
\pgfpathlineto{\pgfqpoint{1.332230in}{2.566986in}}%
\pgfpathlineto{\pgfqpoint{1.333954in}{2.479862in}}%
\pgfpathlineto{\pgfqpoint{1.336550in}{2.549924in}}%
\pgfpathlineto{\pgfqpoint{1.337413in}{2.541778in}}%
\pgfpathlineto{\pgfqpoint{1.338278in}{2.584448in}}%
\pgfpathlineto{\pgfqpoint{1.339144in}{2.495541in}}%
\pgfpathlineto{\pgfqpoint{1.340007in}{2.551984in}}%
\pgfpathlineto{\pgfqpoint{1.341734in}{2.497324in}}%
\pgfpathlineto{\pgfqpoint{1.342599in}{2.526283in}}%
\pgfpathlineto{\pgfqpoint{1.343462in}{2.516385in}}%
\pgfpathlineto{\pgfqpoint{1.344329in}{2.526960in}}%
\pgfpathlineto{\pgfqpoint{1.345195in}{2.458097in}}%
\pgfpathlineto{\pgfqpoint{1.346925in}{2.544238in}}%
\pgfpathlineto{\pgfqpoint{1.347786in}{2.547927in}}%
\pgfpathlineto{\pgfqpoint{1.348652in}{2.473838in}}%
\pgfpathlineto{\pgfqpoint{1.349515in}{2.560900in}}%
\pgfpathlineto{\pgfqpoint{1.350380in}{2.489024in}}%
\pgfpathlineto{\pgfqpoint{1.351245in}{2.504395in}}%
\pgfpathlineto{\pgfqpoint{1.352107in}{2.505994in}}%
\pgfpathlineto{\pgfqpoint{1.353837in}{2.451088in}}%
\pgfpathlineto{\pgfqpoint{1.354702in}{2.470518in}}%
\pgfpathlineto{\pgfqpoint{1.355566in}{2.519643in}}%
\pgfpathlineto{\pgfqpoint{1.356431in}{2.515279in}}%
\pgfpathlineto{\pgfqpoint{1.357296in}{2.496218in}}%
\pgfpathlineto{\pgfqpoint{1.358162in}{2.511282in}}%
\pgfpathlineto{\pgfqpoint{1.359028in}{2.508331in}}%
\pgfpathlineto{\pgfqpoint{1.359893in}{2.448322in}}%
\pgfpathlineto{\pgfqpoint{1.360758in}{2.508146in}}%
\pgfpathlineto{\pgfqpoint{1.361622in}{2.501445in}}%
\pgfpathlineto{\pgfqpoint{1.362488in}{2.516169in}}%
\pgfpathlineto{\pgfqpoint{1.363353in}{2.571167in}}%
\pgfpathlineto{\pgfqpoint{1.364217in}{2.531817in}}%
\pgfpathlineto{\pgfqpoint{1.365947in}{2.565695in}}%
\pgfpathlineto{\pgfqpoint{1.367678in}{2.532800in}}%
\pgfpathlineto{\pgfqpoint{1.368543in}{2.521796in}}%
\pgfpathlineto{\pgfqpoint{1.369408in}{2.476912in}}%
\pgfpathlineto{\pgfqpoint{1.370273in}{2.503841in}}%
\pgfpathlineto{\pgfqpoint{1.371138in}{2.437562in}}%
\pgfpathlineto{\pgfqpoint{1.372866in}{2.504826in}}%
\pgfpathlineto{\pgfqpoint{1.373729in}{2.458282in}}%
\pgfpathlineto{\pgfqpoint{1.374594in}{2.502181in}}%
\pgfpathlineto{\pgfqpoint{1.375459in}{2.433564in}}%
\pgfpathlineto{\pgfqpoint{1.376322in}{2.510974in}}%
\pgfpathlineto{\pgfqpoint{1.377187in}{2.485766in}}%
\pgfpathlineto{\pgfqpoint{1.378052in}{2.514355in}}%
\pgfpathlineto{\pgfqpoint{1.379781in}{2.456684in}}%
\pgfpathlineto{\pgfqpoint{1.380646in}{2.523456in}}%
\pgfpathlineto{\pgfqpoint{1.381509in}{2.473838in}}%
\pgfpathlineto{\pgfqpoint{1.382373in}{2.523517in}}%
\pgfpathlineto{\pgfqpoint{1.383238in}{2.497263in}}%
\pgfpathlineto{\pgfqpoint{1.384104in}{2.500644in}}%
\pgfpathlineto{\pgfqpoint{1.384968in}{2.542270in}}%
\pgfpathlineto{\pgfqpoint{1.386696in}{2.466951in}}%
\pgfpathlineto{\pgfqpoint{1.387562in}{2.531941in}}%
\pgfpathlineto{\pgfqpoint{1.388427in}{2.443986in}}%
\pgfpathlineto{\pgfqpoint{1.389291in}{2.557395in}}%
\pgfpathlineto{\pgfqpoint{1.391021in}{2.413398in}}%
\pgfpathlineto{\pgfqpoint{1.391886in}{2.555365in}}%
\pgfpathlineto{\pgfqpoint{1.393613in}{2.480447in}}%
\pgfpathlineto{\pgfqpoint{1.394476in}{2.490623in}}%
\pgfpathlineto{\pgfqpoint{1.395342in}{2.549095in}}%
\pgfpathlineto{\pgfqpoint{1.396208in}{2.440820in}}%
\pgfpathlineto{\pgfqpoint{1.397938in}{2.504549in}}%
\pgfpathlineto{\pgfqpoint{1.398803in}{2.506178in}}%
\pgfpathlineto{\pgfqpoint{1.399668in}{2.575840in}}%
\pgfpathlineto{\pgfqpoint{1.400534in}{2.484996in}}%
\pgfpathlineto{\pgfqpoint{1.401400in}{2.519520in}}%
\pgfpathlineto{\pgfqpoint{1.402265in}{2.492529in}}%
\pgfpathlineto{\pgfqpoint{1.403997in}{2.552292in}}%
\pgfpathlineto{\pgfqpoint{1.404863in}{2.549524in}}%
\pgfpathlineto{\pgfqpoint{1.405729in}{2.520934in}}%
\pgfpathlineto{\pgfqpoint{1.406594in}{2.584386in}}%
\pgfpathlineto{\pgfqpoint{1.408324in}{2.495911in}}%
\pgfpathlineto{\pgfqpoint{1.409190in}{2.571229in}}%
\pgfpathlineto{\pgfqpoint{1.410055in}{2.483798in}}%
\pgfpathlineto{\pgfqpoint{1.410921in}{2.584755in}}%
\pgfpathlineto{\pgfqpoint{1.411785in}{2.509437in}}%
\pgfpathlineto{\pgfqpoint{1.412651in}{2.555673in}}%
\pgfpathlineto{\pgfqpoint{1.413515in}{2.524808in}}%
\pgfpathlineto{\pgfqpoint{1.414379in}{2.569569in}}%
\pgfpathlineto{\pgfqpoint{1.416108in}{2.494127in}}%
\pgfpathlineto{\pgfqpoint{1.417838in}{2.592565in}}%
\pgfpathlineto{\pgfqpoint{1.420430in}{2.485519in}}%
\pgfpathlineto{\pgfqpoint{1.422159in}{2.513894in}}%
\pgfpathlineto{\pgfqpoint{1.423024in}{2.522840in}}%
\pgfpathlineto{\pgfqpoint{1.423884in}{2.509622in}}%
\pgfpathlineto{\pgfqpoint{1.424749in}{2.541408in}}%
\pgfpathlineto{\pgfqpoint{1.425614in}{2.494250in}}%
\pgfpathlineto{\pgfqpoint{1.428206in}{2.572828in}}%
\pgfpathlineto{\pgfqpoint{1.429937in}{2.484290in}}%
\pgfpathlineto{\pgfqpoint{1.430801in}{2.498984in}}%
\pgfpathlineto{\pgfqpoint{1.431667in}{2.473068in}}%
\pgfpathlineto{\pgfqpoint{1.433398in}{2.559363in}}%
\pgfpathlineto{\pgfqpoint{1.435130in}{2.445738in}}%
\pgfpathlineto{\pgfqpoint{1.436861in}{2.532923in}}%
\pgfpathlineto{\pgfqpoint{1.437726in}{2.485827in}}%
\pgfpathlineto{\pgfqpoint{1.438592in}{2.524531in}}%
\pgfpathlineto{\pgfqpoint{1.439455in}{2.508944in}}%
\pgfpathlineto{\pgfqpoint{1.441185in}{2.562190in}}%
\pgfpathlineto{\pgfqpoint{1.442915in}{2.521919in}}%
\pgfpathlineto{\pgfqpoint{1.443778in}{2.534031in}}%
\pgfpathlineto{\pgfqpoint{1.444643in}{2.471993in}}%
\pgfpathlineto{\pgfqpoint{1.445508in}{2.539258in}}%
\pgfpathlineto{\pgfqpoint{1.446371in}{2.518291in}}%
\pgfpathlineto{\pgfqpoint{1.447237in}{2.535815in}}%
\pgfpathlineto{\pgfqpoint{1.448967in}{2.472362in}}%
\pgfpathlineto{\pgfqpoint{1.449831in}{2.456991in}}%
\pgfpathlineto{\pgfqpoint{1.452425in}{2.482754in}}%
\pgfpathlineto{\pgfqpoint{1.454154in}{2.525054in}}%
\pgfpathlineto{\pgfqpoint{1.455884in}{2.479126in}}%
\pgfpathlineto{\pgfqpoint{1.457614in}{2.545898in}}%
\pgfpathlineto{\pgfqpoint{1.460209in}{2.482938in}}%
\pgfpathlineto{\pgfqpoint{1.461074in}{2.455208in}}%
\pgfpathlineto{\pgfqpoint{1.461941in}{2.465599in}}%
\pgfpathlineto{\pgfqpoint{1.462806in}{2.448076in}}%
\pgfpathlineto{\pgfqpoint{1.463672in}{2.556658in}}%
\pgfpathlineto{\pgfqpoint{1.465402in}{2.512205in}}%
\pgfpathlineto{\pgfqpoint{1.466268in}{2.491300in}}%
\pgfpathlineto{\pgfqpoint{1.467134in}{2.506517in}}%
\pgfpathlineto{\pgfqpoint{1.467999in}{2.468796in}}%
\pgfpathlineto{\pgfqpoint{1.468864in}{2.540548in}}%
\pgfpathlineto{\pgfqpoint{1.469729in}{2.523671in}}%
\pgfpathlineto{\pgfqpoint{1.471459in}{2.480049in}}%
\pgfpathlineto{\pgfqpoint{1.472326in}{2.510853in}}%
\pgfpathlineto{\pgfqpoint{1.473192in}{2.456193in}}%
\pgfpathlineto{\pgfqpoint{1.474056in}{2.485889in}}%
\pgfpathlineto{\pgfqpoint{1.474922in}{2.474269in}}%
\pgfpathlineto{\pgfqpoint{1.475788in}{2.514111in}}%
\pgfpathlineto{\pgfqpoint{1.476654in}{2.474084in}}%
\pgfpathlineto{\pgfqpoint{1.479253in}{2.593948in}}%
\pgfpathlineto{\pgfqpoint{1.480117in}{2.539442in}}%
\pgfpathlineto{\pgfqpoint{1.480982in}{2.580453in}}%
\pgfpathlineto{\pgfqpoint{1.482711in}{2.496465in}}%
\pgfpathlineto{\pgfqpoint{1.483575in}{2.508177in}}%
\pgfpathlineto{\pgfqpoint{1.484441in}{2.472362in}}%
\pgfpathlineto{\pgfqpoint{1.485307in}{2.481217in}}%
\pgfpathlineto{\pgfqpoint{1.486174in}{2.477835in}}%
\pgfpathlineto{\pgfqpoint{1.487040in}{2.458898in}}%
\pgfpathlineto{\pgfqpoint{1.487905in}{2.484414in}}%
\pgfpathlineto{\pgfqpoint{1.488770in}{2.463539in}}%
\pgfpathlineto{\pgfqpoint{1.490500in}{2.534953in}}%
\pgfpathlineto{\pgfqpoint{1.493097in}{2.472239in}}%
\pgfpathlineto{\pgfqpoint{1.494827in}{2.556474in}}%
\pgfpathlineto{\pgfqpoint{1.495693in}{2.460650in}}%
\pgfpathlineto{\pgfqpoint{1.496558in}{2.578916in}}%
\pgfpathlineto{\pgfqpoint{1.497424in}{2.561392in}}%
\pgfpathlineto{\pgfqpoint{1.498288in}{2.559486in}}%
\pgfpathlineto{\pgfqpoint{1.499153in}{2.591213in}}%
\pgfpathlineto{\pgfqpoint{1.500017in}{2.470764in}}%
\pgfpathlineto{\pgfqpoint{1.500882in}{2.548604in}}%
\pgfpathlineto{\pgfqpoint{1.501745in}{2.520751in}}%
\pgfpathlineto{\pgfqpoint{1.502609in}{2.524410in}}%
\pgfpathlineto{\pgfqpoint{1.503473in}{2.562991in}}%
\pgfpathlineto{\pgfqpoint{1.504339in}{2.553092in}}%
\pgfpathlineto{\pgfqpoint{1.505204in}{2.498309in}}%
\pgfpathlineto{\pgfqpoint{1.506068in}{2.556843in}}%
\pgfpathlineto{\pgfqpoint{1.506933in}{2.487980in}}%
\pgfpathlineto{\pgfqpoint{1.507798in}{2.543070in}}%
\pgfpathlineto{\pgfqpoint{1.508662in}{2.502368in}}%
\pgfpathlineto{\pgfqpoint{1.509524in}{2.596316in}}%
\pgfpathlineto{\pgfqpoint{1.510389in}{2.557274in}}%
\pgfpathlineto{\pgfqpoint{1.511254in}{2.595149in}}%
\pgfpathlineto{\pgfqpoint{1.512984in}{2.526470in}}%
\pgfpathlineto{\pgfqpoint{1.513848in}{2.562685in}}%
\pgfpathlineto{\pgfqpoint{1.515579in}{2.478758in}}%
\pgfpathlineto{\pgfqpoint{1.516443in}{2.510730in}}%
\pgfpathlineto{\pgfqpoint{1.517308in}{2.489978in}}%
\pgfpathlineto{\pgfqpoint{1.519038in}{2.544484in}}%
\pgfpathlineto{\pgfqpoint{1.520767in}{2.504888in}}%
\pgfpathlineto{\pgfqpoint{1.521632in}{2.560684in}}%
\pgfpathlineto{\pgfqpoint{1.523360in}{2.465415in}}%
\pgfpathlineto{\pgfqpoint{1.525091in}{2.574857in}}%
\pgfpathlineto{\pgfqpoint{1.525956in}{2.500277in}}%
\pgfpathlineto{\pgfqpoint{1.526821in}{2.501506in}}%
\pgfpathlineto{\pgfqpoint{1.527685in}{2.506794in}}%
\pgfpathlineto{\pgfqpoint{1.528549in}{2.544546in}}%
\pgfpathlineto{\pgfqpoint{1.529411in}{2.449613in}}%
\pgfpathlineto{\pgfqpoint{1.530275in}{2.581435in}}%
\pgfpathlineto{\pgfqpoint{1.531141in}{2.550909in}}%
\pgfpathlineto{\pgfqpoint{1.532005in}{2.530157in}}%
\pgfpathlineto{\pgfqpoint{1.533734in}{2.571044in}}%
\pgfpathlineto{\pgfqpoint{1.535463in}{2.535630in}}%
\pgfpathlineto{\pgfqpoint{1.536326in}{2.576148in}}%
\pgfpathlineto{\pgfqpoint{1.538921in}{2.482446in}}%
\pgfpathlineto{\pgfqpoint{1.539786in}{2.525547in}}%
\pgfpathlineto{\pgfqpoint{1.541516in}{2.492221in}}%
\pgfpathlineto{\pgfqpoint{1.542382in}{2.558993in}}%
\pgfpathlineto{\pgfqpoint{1.544111in}{2.482138in}}%
\pgfpathlineto{\pgfqpoint{1.544975in}{2.546634in}}%
\pgfpathlineto{\pgfqpoint{1.545841in}{2.526376in}}%
\pgfpathlineto{\pgfqpoint{1.546705in}{2.464737in}}%
\pgfpathlineto{\pgfqpoint{1.548435in}{2.522963in}}%
\pgfpathlineto{\pgfqpoint{1.549300in}{2.480293in}}%
\pgfpathlineto{\pgfqpoint{1.550163in}{2.524993in}}%
\pgfpathlineto{\pgfqpoint{1.551028in}{2.502612in}}%
\pgfpathlineto{\pgfqpoint{1.551893in}{2.580022in}}%
\pgfpathlineto{\pgfqpoint{1.552758in}{2.480940in}}%
\pgfpathlineto{\pgfqpoint{1.553624in}{2.537536in}}%
\pgfpathlineto{\pgfqpoint{1.554488in}{2.523948in}}%
\pgfpathlineto{\pgfqpoint{1.555353in}{2.496311in}}%
\pgfpathlineto{\pgfqpoint{1.556218in}{2.564651in}}%
\pgfpathlineto{\pgfqpoint{1.557945in}{2.447768in}}%
\pgfpathlineto{\pgfqpoint{1.558810in}{2.471808in}}%
\pgfpathlineto{\pgfqpoint{1.559676in}{2.526899in}}%
\pgfpathlineto{\pgfqpoint{1.560542in}{2.510912in}}%
\pgfpathlineto{\pgfqpoint{1.562273in}{2.451488in}}%
\pgfpathlineto{\pgfqpoint{1.564872in}{2.531325in}}%
\pgfpathlineto{\pgfqpoint{1.565738in}{2.475313in}}%
\pgfpathlineto{\pgfqpoint{1.566603in}{2.499169in}}%
\pgfpathlineto{\pgfqpoint{1.567467in}{2.453056in}}%
\pgfpathlineto{\pgfqpoint{1.568332in}{2.461541in}}%
\pgfpathlineto{\pgfqpoint{1.570061in}{2.556595in}}%
\pgfpathlineto{\pgfqpoint{1.572656in}{2.504211in}}%
\pgfpathlineto{\pgfqpoint{1.574383in}{2.516139in}}%
\pgfpathlineto{\pgfqpoint{1.575249in}{2.557826in}}%
\pgfpathlineto{\pgfqpoint{1.576978in}{2.518845in}}%
\pgfpathlineto{\pgfqpoint{1.577842in}{2.571291in}}%
\pgfpathlineto{\pgfqpoint{1.578706in}{2.508023in}}%
\pgfpathlineto{\pgfqpoint{1.579570in}{2.521919in}}%
\pgfpathlineto{\pgfqpoint{1.580437in}{2.571167in}}%
\pgfpathlineto{\pgfqpoint{1.581303in}{2.518014in}}%
\pgfpathlineto{\pgfqpoint{1.583036in}{2.545713in}}%
\pgfpathlineto{\pgfqpoint{1.583903in}{2.472116in}}%
\pgfpathlineto{\pgfqpoint{1.585634in}{2.504703in}}%
\pgfpathlineto{\pgfqpoint{1.586500in}{2.456745in}}%
\pgfpathlineto{\pgfqpoint{1.589098in}{2.578115in}}%
\pgfpathlineto{\pgfqpoint{1.592558in}{2.513373in}}%
\pgfpathlineto{\pgfqpoint{1.593423in}{2.437746in}}%
\pgfpathlineto{\pgfqpoint{1.595155in}{2.552476in}}%
\pgfpathlineto{\pgfqpoint{1.596021in}{2.514540in}}%
\pgfpathlineto{\pgfqpoint{1.596886in}{2.545959in}}%
\pgfpathlineto{\pgfqpoint{1.597751in}{2.462033in}}%
\pgfpathlineto{\pgfqpoint{1.598615in}{2.550201in}}%
\pgfpathlineto{\pgfqpoint{1.599480in}{2.533477in}}%
\pgfpathlineto{\pgfqpoint{1.600345in}{2.569015in}}%
\pgfpathlineto{\pgfqpoint{1.601207in}{2.548849in}}%
\pgfpathlineto{\pgfqpoint{1.602937in}{2.490407in}}%
\pgfpathlineto{\pgfqpoint{1.603803in}{2.548541in}}%
\pgfpathlineto{\pgfqpoint{1.604669in}{2.500275in}}%
\pgfpathlineto{\pgfqpoint{1.606401in}{2.543684in}}%
\pgfpathlineto{\pgfqpoint{1.607265in}{2.521242in}}%
\pgfpathlineto{\pgfqpoint{1.608995in}{2.567909in}}%
\pgfpathlineto{\pgfqpoint{1.610725in}{2.447583in}}%
\pgfpathlineto{\pgfqpoint{1.611590in}{2.499107in}}%
\pgfpathlineto{\pgfqpoint{1.612456in}{2.452779in}}%
\pgfpathlineto{\pgfqpoint{1.615051in}{2.592625in}}%
\pgfpathlineto{\pgfqpoint{1.615917in}{2.516015in}}%
\pgfpathlineto{\pgfqpoint{1.618510in}{2.560407in}}%
\pgfpathlineto{\pgfqpoint{1.619375in}{2.481368in}}%
\pgfpathlineto{\pgfqpoint{1.621104in}{2.557210in}}%
\pgfpathlineto{\pgfqpoint{1.621969in}{2.490161in}}%
\pgfpathlineto{\pgfqpoint{1.622835in}{2.536490in}}%
\pgfpathlineto{\pgfqpoint{1.623700in}{2.536305in}}%
\pgfpathlineto{\pgfqpoint{1.624564in}{2.527759in}}%
\pgfpathlineto{\pgfqpoint{1.625428in}{2.554136in}}%
\pgfpathlineto{\pgfqpoint{1.626293in}{2.545375in}}%
\pgfpathlineto{\pgfqpoint{1.627159in}{2.509868in}}%
\pgfpathlineto{\pgfqpoint{1.628022in}{2.532002in}}%
\pgfpathlineto{\pgfqpoint{1.628886in}{2.490376in}}%
\pgfpathlineto{\pgfqpoint{1.630617in}{2.552661in}}%
\pgfpathlineto{\pgfqpoint{1.631481in}{2.547342in}}%
\pgfpathlineto{\pgfqpoint{1.632344in}{2.523210in}}%
\pgfpathlineto{\pgfqpoint{1.633210in}{2.562991in}}%
\pgfpathlineto{\pgfqpoint{1.634073in}{2.518168in}}%
\pgfpathlineto{\pgfqpoint{1.634936in}{2.527022in}}%
\pgfpathlineto{\pgfqpoint{1.635803in}{2.512880in}}%
\pgfpathlineto{\pgfqpoint{1.637533in}{2.601235in}}%
\pgfpathlineto{\pgfqpoint{1.638397in}{2.523733in}}%
\pgfpathlineto{\pgfqpoint{1.639263in}{2.545528in}}%
\pgfpathlineto{\pgfqpoint{1.640129in}{2.581620in}}%
\pgfpathlineto{\pgfqpoint{1.643590in}{2.477466in}}%
\pgfpathlineto{\pgfqpoint{1.644456in}{2.554567in}}%
\pgfpathlineto{\pgfqpoint{1.645321in}{2.502612in}}%
\pgfpathlineto{\pgfqpoint{1.646187in}{2.505994in}}%
\pgfpathlineto{\pgfqpoint{1.647053in}{2.579591in}}%
\pgfpathlineto{\pgfqpoint{1.647918in}{2.570860in}}%
\pgfpathlineto{\pgfqpoint{1.648784in}{2.467382in}}%
\pgfpathlineto{\pgfqpoint{1.649645in}{2.526345in}}%
\pgfpathlineto{\pgfqpoint{1.650509in}{2.522411in}}%
\pgfpathlineto{\pgfqpoint{1.651372in}{2.532125in}}%
\pgfpathlineto{\pgfqpoint{1.652236in}{2.555920in}}%
\pgfpathlineto{\pgfqpoint{1.653101in}{2.516508in}}%
\pgfpathlineto{\pgfqpoint{1.653965in}{2.589612in}}%
\pgfpathlineto{\pgfqpoint{1.654830in}{2.512757in}}%
\pgfpathlineto{\pgfqpoint{1.655694in}{2.524931in}}%
\pgfpathlineto{\pgfqpoint{1.656558in}{2.545159in}}%
\pgfpathlineto{\pgfqpoint{1.657422in}{2.538858in}}%
\pgfpathlineto{\pgfqpoint{1.658286in}{2.523456in}}%
\pgfpathlineto{\pgfqpoint{1.659151in}{2.566372in}}%
\pgfpathlineto{\pgfqpoint{1.660879in}{2.469902in}}%
\pgfpathlineto{\pgfqpoint{1.661745in}{2.543253in}}%
\pgfpathlineto{\pgfqpoint{1.662610in}{2.497571in}}%
\pgfpathlineto{\pgfqpoint{1.664342in}{2.546881in}}%
\pgfpathlineto{\pgfqpoint{1.665207in}{2.473530in}}%
\pgfpathlineto{\pgfqpoint{1.666935in}{2.508791in}}%
\pgfpathlineto{\pgfqpoint{1.667800in}{2.519397in}}%
\pgfpathlineto{\pgfqpoint{1.668666in}{2.543745in}}%
\pgfpathlineto{\pgfqpoint{1.669531in}{2.467873in}}%
\pgfpathlineto{\pgfqpoint{1.670393in}{2.500891in}}%
\pgfpathlineto{\pgfqpoint{1.671259in}{2.443157in}}%
\pgfpathlineto{\pgfqpoint{1.673852in}{2.542270in}}%
\pgfpathlineto{\pgfqpoint{1.674716in}{2.449551in}}%
\pgfpathlineto{\pgfqpoint{1.676445in}{2.591703in}}%
\pgfpathlineto{\pgfqpoint{1.678175in}{2.498677in}}%
\pgfpathlineto{\pgfqpoint{1.679039in}{2.475067in}}%
\pgfpathlineto{\pgfqpoint{1.679905in}{2.483859in}}%
\pgfpathlineto{\pgfqpoint{1.680770in}{2.524931in}}%
\pgfpathlineto{\pgfqpoint{1.681635in}{2.524685in}}%
\pgfpathlineto{\pgfqpoint{1.683364in}{2.482046in}}%
\pgfpathlineto{\pgfqpoint{1.684229in}{2.514909in}}%
\pgfpathlineto{\pgfqpoint{1.685094in}{2.607382in}}%
\pgfpathlineto{\pgfqpoint{1.686825in}{2.507715in}}%
\pgfpathlineto{\pgfqpoint{1.687689in}{2.520013in}}%
\pgfpathlineto{\pgfqpoint{1.688554in}{2.416441in}}%
\pgfpathlineto{\pgfqpoint{1.689420in}{2.540056in}}%
\pgfpathlineto{\pgfqpoint{1.690285in}{2.524562in}}%
\pgfpathlineto{\pgfqpoint{1.692015in}{2.456376in}}%
\pgfpathlineto{\pgfqpoint{1.692880in}{2.473653in}}%
\pgfpathlineto{\pgfqpoint{1.694607in}{2.549893in}}%
\pgfpathlineto{\pgfqpoint{1.695472in}{2.466490in}}%
\pgfpathlineto{\pgfqpoint{1.697201in}{2.527820in}}%
\pgfpathlineto{\pgfqpoint{1.698064in}{2.509406in}}%
\pgfpathlineto{\pgfqpoint{1.698929in}{2.563850in}}%
\pgfpathlineto{\pgfqpoint{1.700660in}{2.499354in}}%
\pgfpathlineto{\pgfqpoint{1.702389in}{2.446139in}}%
\pgfpathlineto{\pgfqpoint{1.703252in}{2.463631in}}%
\pgfpathlineto{\pgfqpoint{1.704117in}{2.459573in}}%
\pgfpathlineto{\pgfqpoint{1.704981in}{2.473530in}}%
\pgfpathlineto{\pgfqpoint{1.705846in}{2.449182in}}%
\pgfpathlineto{\pgfqpoint{1.708439in}{2.545036in}}%
\pgfpathlineto{\pgfqpoint{1.710169in}{2.497324in}}%
\pgfpathlineto{\pgfqpoint{1.711035in}{2.526283in}}%
\pgfpathlineto{\pgfqpoint{1.711900in}{2.520965in}}%
\pgfpathlineto{\pgfqpoint{1.712763in}{2.464553in}}%
\pgfpathlineto{\pgfqpoint{1.713627in}{2.533293in}}%
\pgfpathlineto{\pgfqpoint{1.714491in}{2.483983in}}%
\pgfpathlineto{\pgfqpoint{1.715355in}{2.519766in}}%
\pgfpathlineto{\pgfqpoint{1.716220in}{2.509622in}}%
\pgfpathlineto{\pgfqpoint{1.717084in}{2.447614in}}%
\pgfpathlineto{\pgfqpoint{1.717949in}{2.518106in}}%
\pgfpathlineto{\pgfqpoint{1.718815in}{2.506363in}}%
\pgfpathlineto{\pgfqpoint{1.721409in}{2.477158in}}%
\pgfpathlineto{\pgfqpoint{1.722274in}{2.529175in}}%
\pgfpathlineto{\pgfqpoint{1.723138in}{2.509806in}}%
\pgfpathlineto{\pgfqpoint{1.725732in}{2.559178in}}%
\pgfpathlineto{\pgfqpoint{1.727461in}{2.496649in}}%
\pgfpathlineto{\pgfqpoint{1.728326in}{2.579775in}}%
\pgfpathlineto{\pgfqpoint{1.729191in}{2.561084in}}%
\pgfpathlineto{\pgfqpoint{1.730056in}{2.524685in}}%
\pgfpathlineto{\pgfqpoint{1.730922in}{2.548572in}}%
\pgfpathlineto{\pgfqpoint{1.731786in}{2.529788in}}%
\pgfpathlineto{\pgfqpoint{1.732650in}{2.572704in}}%
\pgfpathlineto{\pgfqpoint{1.733516in}{2.560653in}}%
\pgfpathlineto{\pgfqpoint{1.736112in}{2.616390in}}%
\pgfpathlineto{\pgfqpoint{1.738706in}{2.513865in}}%
\pgfpathlineto{\pgfqpoint{1.739570in}{2.553154in}}%
\pgfpathlineto{\pgfqpoint{1.742164in}{2.498617in}}%
\pgfpathlineto{\pgfqpoint{1.743026in}{2.518229in}}%
\pgfpathlineto{\pgfqpoint{1.743891in}{2.454716in}}%
\pgfpathlineto{\pgfqpoint{1.744756in}{2.525793in}}%
\pgfpathlineto{\pgfqpoint{1.745620in}{2.463385in}}%
\pgfpathlineto{\pgfqpoint{1.746486in}{2.465599in}}%
\pgfpathlineto{\pgfqpoint{1.747352in}{2.541901in}}%
\pgfpathlineto{\pgfqpoint{1.748217in}{2.493021in}}%
\pgfpathlineto{\pgfqpoint{1.750814in}{2.554013in}}%
\pgfpathlineto{\pgfqpoint{1.751679in}{2.497878in}}%
\pgfpathlineto{\pgfqpoint{1.752545in}{2.511682in}}%
\pgfpathlineto{\pgfqpoint{1.753409in}{2.501137in}}%
\pgfpathlineto{\pgfqpoint{1.755138in}{2.568248in}}%
\pgfpathlineto{\pgfqpoint{1.756004in}{2.524870in}}%
\pgfpathlineto{\pgfqpoint{1.756870in}{2.595331in}}%
\pgfpathlineto{\pgfqpoint{1.757735in}{2.543376in}}%
\pgfpathlineto{\pgfqpoint{1.758602in}{2.561146in}}%
\pgfpathlineto{\pgfqpoint{1.759468in}{2.461263in}}%
\pgfpathlineto{\pgfqpoint{1.760334in}{2.542270in}}%
\pgfpathlineto{\pgfqpoint{1.761200in}{2.535014in}}%
\pgfpathlineto{\pgfqpoint{1.762931in}{2.571537in}}%
\pgfpathlineto{\pgfqpoint{1.763797in}{2.567971in}}%
\pgfpathlineto{\pgfqpoint{1.764663in}{2.472209in}}%
\pgfpathlineto{\pgfqpoint{1.765529in}{2.490500in}}%
\pgfpathlineto{\pgfqpoint{1.766394in}{2.449305in}}%
\pgfpathlineto{\pgfqpoint{1.768123in}{2.529665in}}%
\pgfpathlineto{\pgfqpoint{1.768988in}{2.514817in}}%
\pgfpathlineto{\pgfqpoint{1.771582in}{2.563483in}}%
\pgfpathlineto{\pgfqpoint{1.772446in}{2.537782in}}%
\pgfpathlineto{\pgfqpoint{1.773309in}{2.573751in}}%
\pgfpathlineto{\pgfqpoint{1.775039in}{2.521488in}}%
\pgfpathlineto{\pgfqpoint{1.775903in}{2.547127in}}%
\pgfpathlineto{\pgfqpoint{1.776768in}{2.535630in}}%
\pgfpathlineto{\pgfqpoint{1.777633in}{2.552292in}}%
\pgfpathlineto{\pgfqpoint{1.778496in}{2.549187in}}%
\pgfpathlineto{\pgfqpoint{1.781086in}{2.499631in}}%
\pgfpathlineto{\pgfqpoint{1.781951in}{2.544053in}}%
\pgfpathlineto{\pgfqpoint{1.782816in}{2.533231in}}%
\pgfpathlineto{\pgfqpoint{1.783681in}{2.476450in}}%
\pgfpathlineto{\pgfqpoint{1.784545in}{2.485827in}}%
\pgfpathlineto{\pgfqpoint{1.785410in}{2.498984in}}%
\pgfpathlineto{\pgfqpoint{1.786275in}{2.448137in}}%
\pgfpathlineto{\pgfqpoint{1.787138in}{2.509437in}}%
\pgfpathlineto{\pgfqpoint{1.788003in}{2.475313in}}%
\pgfpathlineto{\pgfqpoint{1.788868in}{2.550509in}}%
\pgfpathlineto{\pgfqpoint{1.789732in}{2.486443in}}%
\pgfpathlineto{\pgfqpoint{1.792326in}{2.530281in}}%
\pgfpathlineto{\pgfqpoint{1.793189in}{2.525054in}}%
\pgfpathlineto{\pgfqpoint{1.794054in}{2.558010in}}%
\pgfpathlineto{\pgfqpoint{1.794918in}{2.438116in}}%
\pgfpathlineto{\pgfqpoint{1.798377in}{2.578423in}}%
\pgfpathlineto{\pgfqpoint{1.800107in}{2.517922in}}%
\pgfpathlineto{\pgfqpoint{1.800973in}{2.558932in}}%
\pgfpathlineto{\pgfqpoint{1.802704in}{2.529726in}}%
\pgfpathlineto{\pgfqpoint{1.803570in}{2.550201in}}%
\pgfpathlineto{\pgfqpoint{1.804436in}{2.509314in}}%
\pgfpathlineto{\pgfqpoint{1.805299in}{2.516446in}}%
\pgfpathlineto{\pgfqpoint{1.807031in}{2.538765in}}%
\pgfpathlineto{\pgfqpoint{1.807897in}{2.593486in}}%
\pgfpathlineto{\pgfqpoint{1.809628in}{2.546881in}}%
\pgfpathlineto{\pgfqpoint{1.810493in}{2.561269in}}%
\pgfpathlineto{\pgfqpoint{1.811358in}{2.497447in}}%
\pgfpathlineto{\pgfqpoint{1.813089in}{2.555981in}}%
\pgfpathlineto{\pgfqpoint{1.813954in}{2.530465in}}%
\pgfpathlineto{\pgfqpoint{1.814820in}{2.539073in}}%
\pgfpathlineto{\pgfqpoint{1.815683in}{2.481093in}}%
\pgfpathlineto{\pgfqpoint{1.816549in}{2.505196in}}%
\pgfpathlineto{\pgfqpoint{1.817413in}{2.448630in}}%
\pgfpathlineto{\pgfqpoint{1.818276in}{2.530835in}}%
\pgfpathlineto{\pgfqpoint{1.819140in}{2.515433in}}%
\pgfpathlineto{\pgfqpoint{1.820870in}{2.492221in}}%
\pgfpathlineto{\pgfqpoint{1.821736in}{2.506209in}}%
\pgfpathlineto{\pgfqpoint{1.822603in}{2.570000in}}%
\pgfpathlineto{\pgfqpoint{1.824334in}{2.476727in}}%
\pgfpathlineto{\pgfqpoint{1.825198in}{2.502735in}}%
\pgfpathlineto{\pgfqpoint{1.827794in}{2.567786in}}%
\pgfpathlineto{\pgfqpoint{1.829522in}{2.543622in}}%
\pgfpathlineto{\pgfqpoint{1.830388in}{2.547250in}}%
\pgfpathlineto{\pgfqpoint{1.831253in}{2.598774in}}%
\pgfpathlineto{\pgfqpoint{1.832979in}{2.550632in}}%
\pgfpathlineto{\pgfqpoint{1.833845in}{2.550878in}}%
\pgfpathlineto{\pgfqpoint{1.835577in}{2.484721in}}%
\pgfpathlineto{\pgfqpoint{1.836442in}{2.543193in}}%
\pgfpathlineto{\pgfqpoint{1.837307in}{2.486012in}}%
\pgfpathlineto{\pgfqpoint{1.838171in}{2.510514in}}%
\pgfpathlineto{\pgfqpoint{1.839035in}{2.570800in}}%
\pgfpathlineto{\pgfqpoint{1.840763in}{2.530681in}}%
\pgfpathlineto{\pgfqpoint{1.841628in}{2.560101in}}%
\pgfpathlineto{\pgfqpoint{1.843359in}{2.469289in}}%
\pgfpathlineto{\pgfqpoint{1.844224in}{2.589060in}}%
\pgfpathlineto{\pgfqpoint{1.845955in}{2.507777in}}%
\pgfpathlineto{\pgfqpoint{1.847684in}{2.547866in}}%
\pgfpathlineto{\pgfqpoint{1.848546in}{2.508454in}}%
\pgfpathlineto{\pgfqpoint{1.849412in}{2.554814in}}%
\pgfpathlineto{\pgfqpoint{1.851141in}{2.485889in}}%
\pgfpathlineto{\pgfqpoint{1.852007in}{2.499354in}}%
\pgfpathlineto{\pgfqpoint{1.852870in}{2.477343in}}%
\pgfpathlineto{\pgfqpoint{1.853735in}{2.533785in}}%
\pgfpathlineto{\pgfqpoint{1.854601in}{2.483675in}}%
\pgfpathlineto{\pgfqpoint{1.856331in}{2.568709in}}%
\pgfpathlineto{\pgfqpoint{1.858062in}{2.471870in}}%
\pgfpathlineto{\pgfqpoint{1.858925in}{2.485581in}}%
\pgfpathlineto{\pgfqpoint{1.859790in}{2.547250in}}%
\pgfpathlineto{\pgfqpoint{1.861521in}{2.488901in}}%
\pgfpathlineto{\pgfqpoint{1.864120in}{2.573074in}}%
\pgfpathlineto{\pgfqpoint{1.865845in}{2.553952in}}%
\pgfpathlineto{\pgfqpoint{1.866708in}{2.563789in}}%
\pgfpathlineto{\pgfqpoint{1.867573in}{2.524746in}}%
\pgfpathlineto{\pgfqpoint{1.868440in}{2.526222in}}%
\pgfpathlineto{\pgfqpoint{1.869304in}{2.567201in}}%
\pgfpathlineto{\pgfqpoint{1.870168in}{2.533108in}}%
\pgfpathlineto{\pgfqpoint{1.871032in}{2.546942in}}%
\pgfpathlineto{\pgfqpoint{1.871897in}{2.469964in}}%
\pgfpathlineto{\pgfqpoint{1.872762in}{2.577500in}}%
\pgfpathlineto{\pgfqpoint{1.873626in}{2.457851in}}%
\pgfpathlineto{\pgfqpoint{1.874493in}{2.517552in}}%
\pgfpathlineto{\pgfqpoint{1.876224in}{2.411522in}}%
\pgfpathlineto{\pgfqpoint{1.877953in}{2.467873in}}%
\pgfpathlineto{\pgfqpoint{1.878816in}{2.454839in}}%
\pgfpathlineto{\pgfqpoint{1.879681in}{2.437500in}}%
\pgfpathlineto{\pgfqpoint{1.880547in}{2.555612in}}%
\pgfpathlineto{\pgfqpoint{1.881412in}{2.481892in}}%
\pgfpathlineto{\pgfqpoint{1.882278in}{2.492283in}}%
\pgfpathlineto{\pgfqpoint{1.883143in}{2.490469in}}%
\pgfpathlineto{\pgfqpoint{1.884008in}{2.479985in}}%
\pgfpathlineto{\pgfqpoint{1.886600in}{2.607382in}}%
\pgfpathlineto{\pgfqpoint{1.888327in}{2.556073in}}%
\pgfpathlineto{\pgfqpoint{1.889190in}{2.573997in}}%
\pgfpathlineto{\pgfqpoint{1.890054in}{2.630686in}}%
\pgfpathlineto{\pgfqpoint{1.890918in}{2.542886in}}%
\pgfpathlineto{\pgfqpoint{1.891784in}{2.558380in}}%
\pgfpathlineto{\pgfqpoint{1.892650in}{2.552846in}}%
\pgfpathlineto{\pgfqpoint{1.893515in}{2.509745in}}%
\pgfpathlineto{\pgfqpoint{1.894381in}{2.537044in}}%
\pgfpathlineto{\pgfqpoint{1.895246in}{2.471316in}}%
\pgfpathlineto{\pgfqpoint{1.896112in}{2.488655in}}%
\pgfpathlineto{\pgfqpoint{1.896975in}{2.454593in}}%
\pgfpathlineto{\pgfqpoint{1.897841in}{2.558932in}}%
\pgfpathlineto{\pgfqpoint{1.898707in}{2.497940in}}%
\pgfpathlineto{\pgfqpoint{1.899572in}{2.535507in}}%
\pgfpathlineto{\pgfqpoint{1.900437in}{2.507838in}}%
\pgfpathlineto{\pgfqpoint{1.901303in}{2.569815in}}%
\pgfpathlineto{\pgfqpoint{1.903034in}{2.469779in}}%
\pgfpathlineto{\pgfqpoint{1.903899in}{2.507592in}}%
\pgfpathlineto{\pgfqpoint{1.904764in}{2.496034in}}%
\pgfpathlineto{\pgfqpoint{1.906494in}{2.579652in}}%
\pgfpathlineto{\pgfqpoint{1.907360in}{2.453179in}}%
\pgfpathlineto{\pgfqpoint{1.908225in}{2.587706in}}%
\pgfpathlineto{\pgfqpoint{1.909953in}{2.515708in}}%
\pgfpathlineto{\pgfqpoint{1.910818in}{2.562498in}}%
\pgfpathlineto{\pgfqpoint{1.911684in}{2.495541in}}%
\pgfpathlineto{\pgfqpoint{1.912550in}{2.571966in}}%
\pgfpathlineto{\pgfqpoint{1.915146in}{2.508944in}}%
\pgfpathlineto{\pgfqpoint{1.916011in}{2.524993in}}%
\pgfpathlineto{\pgfqpoint{1.916877in}{2.478541in}}%
\pgfpathlineto{\pgfqpoint{1.918607in}{2.558993in}}%
\pgfpathlineto{\pgfqpoint{1.920335in}{2.533354in}}%
\pgfpathlineto{\pgfqpoint{1.921200in}{2.478695in}}%
\pgfpathlineto{\pgfqpoint{1.922066in}{2.543684in}}%
\pgfpathlineto{\pgfqpoint{1.923796in}{2.479924in}}%
\pgfpathlineto{\pgfqpoint{1.924661in}{2.536120in}}%
\pgfpathlineto{\pgfqpoint{1.925527in}{2.483367in}}%
\pgfpathlineto{\pgfqpoint{1.927259in}{2.521488in}}%
\pgfpathlineto{\pgfqpoint{1.928122in}{2.519705in}}%
\pgfpathlineto{\pgfqpoint{1.928985in}{2.510604in}}%
\pgfpathlineto{\pgfqpoint{1.930715in}{2.524469in}}%
\pgfpathlineto{\pgfqpoint{1.931580in}{2.596499in}}%
\pgfpathlineto{\pgfqpoint{1.933312in}{2.499107in}}%
\pgfpathlineto{\pgfqpoint{1.934179in}{2.502858in}}%
\pgfpathlineto{\pgfqpoint{1.936775in}{2.465107in}}%
\pgfpathlineto{\pgfqpoint{1.937642in}{2.576148in}}%
\pgfpathlineto{\pgfqpoint{1.938507in}{2.495849in}}%
\pgfpathlineto{\pgfqpoint{1.939373in}{2.539073in}}%
\pgfpathlineto{\pgfqpoint{1.941967in}{2.491546in}}%
\pgfpathlineto{\pgfqpoint{1.942833in}{2.475990in}}%
\pgfpathlineto{\pgfqpoint{1.943696in}{2.561946in}}%
\pgfpathlineto{\pgfqpoint{1.944561in}{2.502060in}}%
\pgfpathlineto{\pgfqpoint{1.945425in}{2.508823in}}%
\pgfpathlineto{\pgfqpoint{1.946290in}{2.498617in}}%
\pgfpathlineto{\pgfqpoint{1.947154in}{2.509070in}}%
\pgfpathlineto{\pgfqpoint{1.948018in}{2.489824in}}%
\pgfpathlineto{\pgfqpoint{1.948883in}{2.515833in}}%
\pgfpathlineto{\pgfqpoint{1.949748in}{2.515310in}}%
\pgfpathlineto{\pgfqpoint{1.953207in}{2.469104in}}%
\pgfpathlineto{\pgfqpoint{1.954072in}{2.548112in}}%
\pgfpathlineto{\pgfqpoint{1.954936in}{2.475713in}}%
\pgfpathlineto{\pgfqpoint{1.956666in}{2.599267in}}%
\pgfpathlineto{\pgfqpoint{1.957531in}{2.548112in}}%
\pgfpathlineto{\pgfqpoint{1.958396in}{2.548420in}}%
\pgfpathlineto{\pgfqpoint{1.959262in}{2.527668in}}%
\pgfpathlineto{\pgfqpoint{1.960126in}{2.541349in}}%
\pgfpathlineto{\pgfqpoint{1.962723in}{2.484229in}}%
\pgfpathlineto{\pgfqpoint{1.963588in}{2.546021in}}%
\pgfpathlineto{\pgfqpoint{1.964454in}{2.490469in}}%
\pgfpathlineto{\pgfqpoint{1.965319in}{2.564158in}}%
\pgfpathlineto{\pgfqpoint{1.967048in}{2.534768in}}%
\pgfpathlineto{\pgfqpoint{1.967912in}{2.528621in}}%
\pgfpathlineto{\pgfqpoint{1.968778in}{2.541931in}}%
\pgfpathlineto{\pgfqpoint{1.970509in}{2.516631in}}%
\pgfpathlineto{\pgfqpoint{1.971375in}{2.543807in}}%
\pgfpathlineto{\pgfqpoint{1.972240in}{2.483921in}}%
\pgfpathlineto{\pgfqpoint{1.973106in}{2.546329in}}%
\pgfpathlineto{\pgfqpoint{1.973972in}{2.426526in}}%
\pgfpathlineto{\pgfqpoint{1.974835in}{2.477527in}}%
\pgfpathlineto{\pgfqpoint{1.975702in}{2.605661in}}%
\pgfpathlineto{\pgfqpoint{1.977433in}{2.473899in}}%
\pgfpathlineto{\pgfqpoint{1.978298in}{2.499661in}}%
\pgfpathlineto{\pgfqpoint{1.979162in}{2.470087in}}%
\pgfpathlineto{\pgfqpoint{1.980028in}{2.499415in}}%
\pgfpathlineto{\pgfqpoint{1.980894in}{2.580391in}}%
\pgfpathlineto{\pgfqpoint{1.981760in}{2.505011in}}%
\pgfpathlineto{\pgfqpoint{1.982625in}{2.518907in}}%
\pgfpathlineto{\pgfqpoint{1.983491in}{2.492960in}}%
\pgfpathlineto{\pgfqpoint{1.985220in}{2.532556in}}%
\pgfpathlineto{\pgfqpoint{1.986085in}{2.506794in}}%
\pgfpathlineto{\pgfqpoint{1.986949in}{2.525793in}}%
\pgfpathlineto{\pgfqpoint{1.987814in}{2.479372in}}%
\pgfpathlineto{\pgfqpoint{1.988674in}{2.544176in}}%
\pgfpathlineto{\pgfqpoint{1.989540in}{2.514909in}}%
\pgfpathlineto{\pgfqpoint{1.990405in}{2.580299in}}%
\pgfpathlineto{\pgfqpoint{1.991267in}{2.527453in}}%
\pgfpathlineto{\pgfqpoint{1.992997in}{2.564651in}}%
\pgfpathlineto{\pgfqpoint{1.996456in}{2.479864in}}%
\pgfpathlineto{\pgfqpoint{1.998186in}{2.563175in}}%
\pgfpathlineto{\pgfqpoint{1.999051in}{2.520813in}}%
\pgfpathlineto{\pgfqpoint{1.999915in}{2.555583in}}%
\pgfpathlineto{\pgfqpoint{2.000777in}{2.501937in}}%
\pgfpathlineto{\pgfqpoint{2.001642in}{2.536369in}}%
\pgfpathlineto{\pgfqpoint{2.002508in}{2.488195in}}%
\pgfpathlineto{\pgfqpoint{2.003373in}{2.507410in}}%
\pgfpathlineto{\pgfqpoint{2.004238in}{2.491854in}}%
\pgfpathlineto{\pgfqpoint{2.005968in}{2.508947in}}%
\pgfpathlineto{\pgfqpoint{2.006833in}{2.539196in}}%
\pgfpathlineto{\pgfqpoint{2.008563in}{2.502920in}}%
\pgfpathlineto{\pgfqpoint{2.009429in}{2.576548in}}%
\pgfpathlineto{\pgfqpoint{2.010294in}{2.567111in}}%
\pgfpathlineto{\pgfqpoint{2.011158in}{2.580206in}}%
\pgfpathlineto{\pgfqpoint{2.012887in}{2.541410in}}%
\pgfpathlineto{\pgfqpoint{2.013752in}{2.549957in}}%
\pgfpathlineto{\pgfqpoint{2.015482in}{2.479926in}}%
\pgfpathlineto{\pgfqpoint{2.016347in}{2.494989in}}%
\pgfpathlineto{\pgfqpoint{2.017211in}{2.480786in}}%
\pgfpathlineto{\pgfqpoint{2.018077in}{2.505503in}}%
\pgfpathlineto{\pgfqpoint{2.018939in}{2.488103in}}%
\pgfpathlineto{\pgfqpoint{2.019804in}{2.544853in}}%
\pgfpathlineto{\pgfqpoint{2.020669in}{2.524256in}}%
\pgfpathlineto{\pgfqpoint{2.021535in}{2.540733in}}%
\pgfpathlineto{\pgfqpoint{2.022400in}{2.480663in}}%
\pgfpathlineto{\pgfqpoint{2.024995in}{2.586415in}}%
\pgfpathlineto{\pgfqpoint{2.025860in}{2.463631in}}%
\pgfpathlineto{\pgfqpoint{2.026727in}{2.570616in}}%
\pgfpathlineto{\pgfqpoint{2.029320in}{2.493268in}}%
\pgfpathlineto{\pgfqpoint{2.030186in}{2.566988in}}%
\pgfpathlineto{\pgfqpoint{2.031052in}{2.475621in}}%
\pgfpathlineto{\pgfqpoint{2.032782in}{2.544299in}}%
\pgfpathlineto{\pgfqpoint{2.033646in}{2.518599in}}%
\pgfpathlineto{\pgfqpoint{2.034510in}{2.559732in}}%
\pgfpathlineto{\pgfqpoint{2.035374in}{2.538950in}}%
\pgfpathlineto{\pgfqpoint{2.036239in}{2.614084in}}%
\pgfpathlineto{\pgfqpoint{2.037104in}{2.594040in}}%
\pgfpathlineto{\pgfqpoint{2.040565in}{2.480109in}}%
\pgfpathlineto{\pgfqpoint{2.041430in}{2.485519in}}%
\pgfpathlineto{\pgfqpoint{2.042296in}{2.549095in}}%
\pgfpathlineto{\pgfqpoint{2.043162in}{2.508085in}}%
\pgfpathlineto{\pgfqpoint{2.044026in}{2.538273in}}%
\pgfpathlineto{\pgfqpoint{2.044891in}{2.505871in}}%
\pgfpathlineto{\pgfqpoint{2.046620in}{2.622507in}}%
\pgfpathlineto{\pgfqpoint{2.047484in}{2.500860in}}%
\pgfpathlineto{\pgfqpoint{2.048349in}{2.578054in}}%
\pgfpathlineto{\pgfqpoint{2.049214in}{2.533293in}}%
\pgfpathlineto{\pgfqpoint{2.050080in}{2.597851in}}%
\pgfpathlineto{\pgfqpoint{2.050945in}{2.550324in}}%
\pgfpathlineto{\pgfqpoint{2.053538in}{2.604922in}}%
\pgfpathlineto{\pgfqpoint{2.054402in}{2.534799in}}%
\pgfpathlineto{\pgfqpoint{2.055268in}{2.577561in}}%
\pgfpathlineto{\pgfqpoint{2.056131in}{2.502735in}}%
\pgfpathlineto{\pgfqpoint{2.057861in}{2.557456in}}%
\pgfpathlineto{\pgfqpoint{2.058726in}{2.568956in}}%
\pgfpathlineto{\pgfqpoint{2.059591in}{2.494651in}}%
\pgfpathlineto{\pgfqpoint{2.060456in}{2.590228in}}%
\pgfpathlineto{\pgfqpoint{2.061322in}{2.582297in}}%
\pgfpathlineto{\pgfqpoint{2.062187in}{2.525547in}}%
\pgfpathlineto{\pgfqpoint{2.063052in}{2.529236in}}%
\pgfpathlineto{\pgfqpoint{2.064780in}{2.496341in}}%
\pgfpathlineto{\pgfqpoint{2.065646in}{2.588445in}}%
\pgfpathlineto{\pgfqpoint{2.066512in}{2.574395in}}%
\pgfpathlineto{\pgfqpoint{2.067377in}{2.604247in}}%
\pgfpathlineto{\pgfqpoint{2.068243in}{2.514786in}}%
\pgfpathlineto{\pgfqpoint{2.069109in}{2.519736in}}%
\pgfpathlineto{\pgfqpoint{2.069974in}{2.497755in}}%
\pgfpathlineto{\pgfqpoint{2.070838in}{2.513188in}}%
\pgfpathlineto{\pgfqpoint{2.072565in}{2.485643in}}%
\pgfpathlineto{\pgfqpoint{2.073429in}{2.579898in}}%
\pgfpathlineto{\pgfqpoint{2.074294in}{2.557887in}}%
\pgfpathlineto{\pgfqpoint{2.075160in}{2.546390in}}%
\pgfpathlineto{\pgfqpoint{2.076025in}{2.496495in}}%
\pgfpathlineto{\pgfqpoint{2.076890in}{2.592319in}}%
\pgfpathlineto{\pgfqpoint{2.077754in}{2.491361in}}%
\pgfpathlineto{\pgfqpoint{2.078618in}{2.520105in}}%
\pgfpathlineto{\pgfqpoint{2.079483in}{2.503228in}}%
\pgfpathlineto{\pgfqpoint{2.080348in}{2.521488in}}%
\pgfpathlineto{\pgfqpoint{2.081213in}{2.495911in}}%
\pgfpathlineto{\pgfqpoint{2.082078in}{2.556043in}}%
\pgfpathlineto{\pgfqpoint{2.082942in}{2.492313in}}%
\pgfpathlineto{\pgfqpoint{2.083807in}{2.505932in}}%
\pgfpathlineto{\pgfqpoint{2.084671in}{2.550693in}}%
\pgfpathlineto{\pgfqpoint{2.085536in}{2.436425in}}%
\pgfpathlineto{\pgfqpoint{2.086401in}{2.518907in}}%
\pgfpathlineto{\pgfqpoint{2.087265in}{2.517860in}}%
\pgfpathlineto{\pgfqpoint{2.088130in}{2.516631in}}%
\pgfpathlineto{\pgfqpoint{2.089859in}{2.547496in}}%
\pgfpathlineto{\pgfqpoint{2.090724in}{2.525424in}}%
\pgfpathlineto{\pgfqpoint{2.091589in}{2.564527in}}%
\pgfpathlineto{\pgfqpoint{2.092454in}{2.452258in}}%
\pgfpathlineto{\pgfqpoint{2.095049in}{2.583896in}}%
\pgfpathlineto{\pgfqpoint{2.095914in}{2.512636in}}%
\pgfpathlineto{\pgfqpoint{2.096780in}{2.520320in}}%
\pgfpathlineto{\pgfqpoint{2.097644in}{2.588014in}}%
\pgfpathlineto{\pgfqpoint{2.098507in}{2.578238in}}%
\pgfpathlineto{\pgfqpoint{2.099373in}{2.566618in}}%
\pgfpathlineto{\pgfqpoint{2.101101in}{2.530034in}}%
\pgfpathlineto{\pgfqpoint{2.101963in}{2.547004in}}%
\pgfpathlineto{\pgfqpoint{2.102828in}{2.533293in}}%
\pgfpathlineto{\pgfqpoint{2.103692in}{2.557456in}}%
\pgfpathlineto{\pgfqpoint{2.104558in}{2.520872in}}%
\pgfpathlineto{\pgfqpoint{2.105422in}{2.583095in}}%
\pgfpathlineto{\pgfqpoint{2.108016in}{2.477404in}}%
\pgfpathlineto{\pgfqpoint{2.109746in}{2.566003in}}%
\pgfpathlineto{\pgfqpoint{2.110611in}{2.594102in}}%
\pgfpathlineto{\pgfqpoint{2.112342in}{2.503228in}}%
\pgfpathlineto{\pgfqpoint{2.113207in}{2.499538in}}%
\pgfpathlineto{\pgfqpoint{2.114071in}{2.508146in}}%
\pgfpathlineto{\pgfqpoint{2.114938in}{2.562436in}}%
\pgfpathlineto{\pgfqpoint{2.115804in}{2.503351in}}%
\pgfpathlineto{\pgfqpoint{2.116670in}{2.512541in}}%
\pgfpathlineto{\pgfqpoint{2.117537in}{2.466890in}}%
\pgfpathlineto{\pgfqpoint{2.118403in}{2.593548in}}%
\pgfpathlineto{\pgfqpoint{2.119267in}{2.564343in}}%
\pgfpathlineto{\pgfqpoint{2.120134in}{2.567232in}}%
\pgfpathlineto{\pgfqpoint{2.120999in}{2.556964in}}%
\pgfpathlineto{\pgfqpoint{2.122731in}{2.587952in}}%
\pgfpathlineto{\pgfqpoint{2.124463in}{2.500521in}}%
\pgfpathlineto{\pgfqpoint{2.125328in}{2.582849in}}%
\pgfpathlineto{\pgfqpoint{2.126193in}{2.524069in}}%
\pgfpathlineto{\pgfqpoint{2.127057in}{2.591457in}}%
\pgfpathlineto{\pgfqpoint{2.127924in}{2.503780in}}%
\pgfpathlineto{\pgfqpoint{2.128790in}{2.525791in}}%
\pgfpathlineto{\pgfqpoint{2.129654in}{2.563912in}}%
\pgfpathlineto{\pgfqpoint{2.130520in}{2.516169in}}%
\pgfpathlineto{\pgfqpoint{2.131385in}{2.528436in}}%
\pgfpathlineto{\pgfqpoint{2.132251in}{2.492590in}}%
\pgfpathlineto{\pgfqpoint{2.133982in}{2.619002in}}%
\pgfpathlineto{\pgfqpoint{2.135713in}{2.510697in}}%
\pgfpathlineto{\pgfqpoint{2.136575in}{2.567355in}}%
\pgfpathlineto{\pgfqpoint{2.137440in}{2.538950in}}%
\pgfpathlineto{\pgfqpoint{2.138303in}{2.577746in}}%
\pgfpathlineto{\pgfqpoint{2.140031in}{2.509191in}}%
\pgfpathlineto{\pgfqpoint{2.140897in}{2.578238in}}%
\pgfpathlineto{\pgfqpoint{2.142629in}{2.520228in}}%
\pgfpathlineto{\pgfqpoint{2.143495in}{2.524870in}}%
\pgfpathlineto{\pgfqpoint{2.144360in}{2.522840in}}%
\pgfpathlineto{\pgfqpoint{2.145225in}{2.495418in}}%
\pgfpathlineto{\pgfqpoint{2.146091in}{2.539196in}}%
\pgfpathlineto{\pgfqpoint{2.147822in}{2.459819in}}%
\pgfpathlineto{\pgfqpoint{2.148687in}{2.466767in}}%
\pgfpathlineto{\pgfqpoint{2.149554in}{2.464368in}}%
\pgfpathlineto{\pgfqpoint{2.150418in}{2.473407in}}%
\pgfpathlineto{\pgfqpoint{2.151284in}{2.535507in}}%
\pgfpathlineto{\pgfqpoint{2.152149in}{2.476850in}}%
\pgfpathlineto{\pgfqpoint{2.153015in}{2.479249in}}%
\pgfpathlineto{\pgfqpoint{2.155607in}{2.547866in}}%
\pgfpathlineto{\pgfqpoint{2.156472in}{2.545652in}}%
\pgfpathlineto{\pgfqpoint{2.157337in}{2.501260in}}%
\pgfpathlineto{\pgfqpoint{2.158202in}{2.605291in}}%
\pgfpathlineto{\pgfqpoint{2.159928in}{2.486564in}}%
\pgfpathlineto{\pgfqpoint{2.160794in}{2.496955in}}%
\pgfpathlineto{\pgfqpoint{2.162520in}{2.528618in}}%
\pgfpathlineto{\pgfqpoint{2.163384in}{2.517306in}}%
\pgfpathlineto{\pgfqpoint{2.164251in}{2.454254in}}%
\pgfpathlineto{\pgfqpoint{2.165115in}{2.611562in}}%
\pgfpathlineto{\pgfqpoint{2.166844in}{2.521365in}}%
\pgfpathlineto{\pgfqpoint{2.167707in}{2.583526in}}%
\pgfpathlineto{\pgfqpoint{2.168572in}{2.516015in}}%
\pgfpathlineto{\pgfqpoint{2.169437in}{2.589797in}}%
\pgfpathlineto{\pgfqpoint{2.170301in}{2.580512in}}%
\pgfpathlineto{\pgfqpoint{2.171166in}{2.583280in}}%
\pgfpathlineto{\pgfqpoint{2.172893in}{2.520136in}}%
\pgfpathlineto{\pgfqpoint{2.173758in}{2.514448in}}%
\pgfpathlineto{\pgfqpoint{2.174620in}{2.444756in}}%
\pgfpathlineto{\pgfqpoint{2.175484in}{2.538765in}}%
\pgfpathlineto{\pgfqpoint{2.176349in}{2.510789in}}%
\pgfpathlineto{\pgfqpoint{2.177215in}{2.628839in}}%
\pgfpathlineto{\pgfqpoint{2.178945in}{2.488409in}}%
\pgfpathlineto{\pgfqpoint{2.179810in}{2.528128in}}%
\pgfpathlineto{\pgfqpoint{2.181542in}{2.491790in}}%
\pgfpathlineto{\pgfqpoint{2.182405in}{2.541962in}}%
\pgfpathlineto{\pgfqpoint{2.183270in}{2.475128in}}%
\pgfpathlineto{\pgfqpoint{2.184136in}{2.541716in}}%
\pgfpathlineto{\pgfqpoint{2.185000in}{2.531941in}}%
\pgfpathlineto{\pgfqpoint{2.185866in}{2.524500in}}%
\pgfpathlineto{\pgfqpoint{2.186731in}{2.570860in}}%
\pgfpathlineto{\pgfqpoint{2.187597in}{2.488993in}}%
\pgfpathlineto{\pgfqpoint{2.190192in}{2.630776in}}%
\pgfpathlineto{\pgfqpoint{2.191056in}{2.542024in}}%
\pgfpathlineto{\pgfqpoint{2.191922in}{2.565633in}}%
\pgfpathlineto{\pgfqpoint{2.192788in}{2.507500in}}%
\pgfpathlineto{\pgfqpoint{2.193652in}{2.550201in}}%
\pgfpathlineto{\pgfqpoint{2.194519in}{2.544420in}}%
\pgfpathlineto{\pgfqpoint{2.195384in}{2.517183in}}%
\pgfpathlineto{\pgfqpoint{2.196249in}{2.549831in}}%
\pgfpathlineto{\pgfqpoint{2.199710in}{2.483244in}}%
\pgfpathlineto{\pgfqpoint{2.202305in}{2.549831in}}%
\pgfpathlineto{\pgfqpoint{2.203171in}{2.539625in}}%
\pgfpathlineto{\pgfqpoint{2.204900in}{2.500706in}}%
\pgfpathlineto{\pgfqpoint{2.206632in}{2.562375in}}%
\pgfpathlineto{\pgfqpoint{2.207495in}{2.526222in}}%
\pgfpathlineto{\pgfqpoint{2.208361in}{2.547127in}}%
\pgfpathlineto{\pgfqpoint{2.209225in}{2.466090in}}%
\pgfpathlineto{\pgfqpoint{2.210954in}{2.504026in}}%
\pgfpathlineto{\pgfqpoint{2.211819in}{2.493789in}}%
\pgfpathlineto{\pgfqpoint{2.213551in}{2.553705in}}%
\pgfpathlineto{\pgfqpoint{2.215281in}{2.523148in}}%
\pgfpathlineto{\pgfqpoint{2.217011in}{2.547312in}}%
\pgfpathlineto{\pgfqpoint{2.217875in}{2.509683in}}%
\pgfpathlineto{\pgfqpoint{2.218741in}{2.533324in}}%
\pgfpathlineto{\pgfqpoint{2.220471in}{2.512634in}}%
\pgfpathlineto{\pgfqpoint{2.221337in}{2.537934in}}%
\pgfpathlineto{\pgfqpoint{2.222202in}{2.503718in}}%
\pgfpathlineto{\pgfqpoint{2.223065in}{2.522779in}}%
\pgfpathlineto{\pgfqpoint{2.223931in}{2.485089in}}%
\pgfpathlineto{\pgfqpoint{2.224796in}{2.549770in}}%
\pgfpathlineto{\pgfqpoint{2.225661in}{2.527851in}}%
\pgfpathlineto{\pgfqpoint{2.226526in}{2.532862in}}%
\pgfpathlineto{\pgfqpoint{2.227389in}{2.550447in}}%
\pgfpathlineto{\pgfqpoint{2.228254in}{2.518751in}}%
\pgfpathlineto{\pgfqpoint{2.229120in}{2.558809in}}%
\pgfpathlineto{\pgfqpoint{2.229986in}{2.536428in}}%
\pgfpathlineto{\pgfqpoint{2.231717in}{2.580820in}}%
\pgfpathlineto{\pgfqpoint{2.233444in}{2.519028in}}%
\pgfpathlineto{\pgfqpoint{2.234309in}{2.536367in}}%
\pgfpathlineto{\pgfqpoint{2.235174in}{2.508791in}}%
\pgfpathlineto{\pgfqpoint{2.236039in}{2.557333in}}%
\pgfpathlineto{\pgfqpoint{2.238633in}{2.474882in}}%
\pgfpathlineto{\pgfqpoint{2.239497in}{2.549587in}}%
\pgfpathlineto{\pgfqpoint{2.240361in}{2.486012in}}%
\pgfpathlineto{\pgfqpoint{2.241226in}{2.486566in}}%
\pgfpathlineto{\pgfqpoint{2.242955in}{2.599021in}}%
\pgfpathlineto{\pgfqpoint{2.243820in}{2.573258in}}%
\pgfpathlineto{\pgfqpoint{2.244685in}{2.513773in}}%
\pgfpathlineto{\pgfqpoint{2.245550in}{2.556166in}}%
\pgfpathlineto{\pgfqpoint{2.246416in}{2.553890in}}%
\pgfpathlineto{\pgfqpoint{2.247281in}{2.576886in}}%
\pgfpathlineto{\pgfqpoint{2.248147in}{2.565143in}}%
\pgfpathlineto{\pgfqpoint{2.249009in}{2.571352in}}%
\pgfpathlineto{\pgfqpoint{2.249874in}{2.516416in}}%
\pgfpathlineto{\pgfqpoint{2.250740in}{2.572951in}}%
\pgfpathlineto{\pgfqpoint{2.252471in}{2.500891in}}%
\pgfpathlineto{\pgfqpoint{2.254201in}{2.545005in}}%
\pgfpathlineto{\pgfqpoint{2.255066in}{2.503780in}}%
\pgfpathlineto{\pgfqpoint{2.255930in}{2.567232in}}%
\pgfpathlineto{\pgfqpoint{2.256794in}{2.565972in}}%
\pgfpathlineto{\pgfqpoint{2.258523in}{2.555612in}}%
\pgfpathlineto{\pgfqpoint{2.259389in}{2.602217in}}%
\pgfpathlineto{\pgfqpoint{2.260254in}{2.557580in}}%
\pgfpathlineto{\pgfqpoint{2.261118in}{2.592134in}}%
\pgfpathlineto{\pgfqpoint{2.262848in}{2.536120in}}%
\pgfpathlineto{\pgfqpoint{2.263712in}{2.553305in}}%
\pgfpathlineto{\pgfqpoint{2.264577in}{2.522286in}}%
\pgfpathlineto{\pgfqpoint{2.265441in}{2.537780in}}%
\pgfpathlineto{\pgfqpoint{2.266306in}{2.583493in}}%
\pgfpathlineto{\pgfqpoint{2.267171in}{2.510050in}}%
\pgfpathlineto{\pgfqpoint{2.268032in}{2.620906in}}%
\pgfpathlineto{\pgfqpoint{2.268898in}{2.530863in}}%
\pgfpathlineto{\pgfqpoint{2.269763in}{2.561821in}}%
\pgfpathlineto{\pgfqpoint{2.271490in}{2.505624in}}%
\pgfpathlineto{\pgfqpoint{2.273219in}{2.618879in}}%
\pgfpathlineto{\pgfqpoint{2.274084in}{2.540118in}}%
\pgfpathlineto{\pgfqpoint{2.274949in}{2.619924in}}%
\pgfpathlineto{\pgfqpoint{2.277541in}{2.553151in}}%
\pgfpathlineto{\pgfqpoint{2.278407in}{2.569752in}}%
\pgfpathlineto{\pgfqpoint{2.280137in}{2.481522in}}%
\pgfpathlineto{\pgfqpoint{2.281868in}{2.584325in}}%
\pgfpathlineto{\pgfqpoint{2.282732in}{2.480232in}}%
\pgfpathlineto{\pgfqpoint{2.283598in}{2.485150in}}%
\pgfpathlineto{\pgfqpoint{2.284462in}{2.476788in}}%
\pgfpathlineto{\pgfqpoint{2.286192in}{2.504642in}}%
\pgfpathlineto{\pgfqpoint{2.287056in}{2.497263in}}%
\pgfpathlineto{\pgfqpoint{2.288787in}{2.461356in}}%
\pgfpathlineto{\pgfqpoint{2.289652in}{2.584940in}}%
\pgfpathlineto{\pgfqpoint{2.290518in}{2.524192in}}%
\pgfpathlineto{\pgfqpoint{2.291384in}{2.565818in}}%
\pgfpathlineto{\pgfqpoint{2.293114in}{2.444756in}}%
\pgfpathlineto{\pgfqpoint{2.293978in}{2.589735in}}%
\pgfpathlineto{\pgfqpoint{2.294842in}{2.506117in}}%
\pgfpathlineto{\pgfqpoint{2.295707in}{2.520872in}}%
\pgfpathlineto{\pgfqpoint{2.297435in}{2.497324in}}%
\pgfpathlineto{\pgfqpoint{2.298302in}{2.609902in}}%
\pgfpathlineto{\pgfqpoint{2.299167in}{2.485150in}}%
\pgfpathlineto{\pgfqpoint{2.300033in}{2.608180in}}%
\pgfpathlineto{\pgfqpoint{2.300900in}{2.545036in}}%
\pgfpathlineto{\pgfqpoint{2.301766in}{2.598220in}}%
\pgfpathlineto{\pgfqpoint{2.304361in}{2.519674in}}%
\pgfpathlineto{\pgfqpoint{2.306957in}{2.547004in}}%
\pgfpathlineto{\pgfqpoint{2.307822in}{2.506486in}}%
\pgfpathlineto{\pgfqpoint{2.308688in}{2.516446in}}%
\pgfpathlineto{\pgfqpoint{2.309554in}{2.516262in}}%
\pgfpathlineto{\pgfqpoint{2.310420in}{2.518660in}}%
\pgfpathlineto{\pgfqpoint{2.311285in}{2.488993in}}%
\pgfpathlineto{\pgfqpoint{2.313017in}{2.581066in}}%
\pgfpathlineto{\pgfqpoint{2.313883in}{2.493635in}}%
\pgfpathlineto{\pgfqpoint{2.314749in}{2.515831in}}%
\pgfpathlineto{\pgfqpoint{2.315613in}{2.582972in}}%
\pgfpathlineto{\pgfqpoint{2.317342in}{2.498677in}}%
\pgfpathlineto{\pgfqpoint{2.318206in}{2.502304in}}%
\pgfpathlineto{\pgfqpoint{2.319937in}{2.566372in}}%
\pgfpathlineto{\pgfqpoint{2.320802in}{2.546144in}}%
\pgfpathlineto{\pgfqpoint{2.321668in}{2.557580in}}%
\pgfpathlineto{\pgfqpoint{2.322533in}{2.550139in}}%
\pgfpathlineto{\pgfqpoint{2.323399in}{2.572643in}}%
\pgfpathlineto{\pgfqpoint{2.325993in}{2.469717in}}%
\pgfpathlineto{\pgfqpoint{2.327723in}{2.553890in}}%
\pgfpathlineto{\pgfqpoint{2.328589in}{2.506117in}}%
\pgfpathlineto{\pgfqpoint{2.329452in}{2.625889in}}%
\pgfpathlineto{\pgfqpoint{2.331179in}{2.508146in}}%
\pgfpathlineto{\pgfqpoint{2.332043in}{2.520567in}}%
\pgfpathlineto{\pgfqpoint{2.332908in}{2.501814in}}%
\pgfpathlineto{\pgfqpoint{2.333775in}{2.519091in}}%
\pgfpathlineto{\pgfqpoint{2.334640in}{2.513188in}}%
\pgfpathlineto{\pgfqpoint{2.335506in}{2.516416in}}%
\pgfpathlineto{\pgfqpoint{2.337235in}{2.586600in}}%
\pgfpathlineto{\pgfqpoint{2.338965in}{2.486749in}}%
\pgfpathlineto{\pgfqpoint{2.339831in}{2.495295in}}%
\pgfpathlineto{\pgfqpoint{2.340692in}{2.537105in}}%
\pgfpathlineto{\pgfqpoint{2.341556in}{2.522717in}}%
\pgfpathlineto{\pgfqpoint{2.342422in}{2.454500in}}%
\pgfpathlineto{\pgfqpoint{2.343286in}{2.531879in}}%
\pgfpathlineto{\pgfqpoint{2.345881in}{2.471685in}}%
\pgfpathlineto{\pgfqpoint{2.346747in}{2.497817in}}%
\pgfpathlineto{\pgfqpoint{2.347612in}{2.494558in}}%
\pgfpathlineto{\pgfqpoint{2.348476in}{2.481645in}}%
\pgfpathlineto{\pgfqpoint{2.349341in}{2.491606in}}%
\pgfpathlineto{\pgfqpoint{2.350207in}{2.544420in}}%
\pgfpathlineto{\pgfqpoint{2.351073in}{2.519335in}}%
\pgfpathlineto{\pgfqpoint{2.351939in}{2.455668in}}%
\pgfpathlineto{\pgfqpoint{2.353668in}{2.531202in}}%
\pgfpathlineto{\pgfqpoint{2.354534in}{2.504241in}}%
\pgfpathlineto{\pgfqpoint{2.356262in}{2.552292in}}%
\pgfpathlineto{\pgfqpoint{2.357127in}{2.497386in}}%
\pgfpathlineto{\pgfqpoint{2.358859in}{2.565818in}}%
\pgfpathlineto{\pgfqpoint{2.359724in}{2.492221in}}%
\pgfpathlineto{\pgfqpoint{2.361451in}{2.571198in}}%
\pgfpathlineto{\pgfqpoint{2.362316in}{2.547127in}}%
\pgfpathlineto{\pgfqpoint{2.363181in}{2.566372in}}%
\pgfpathlineto{\pgfqpoint{2.364045in}{2.489055in}}%
\pgfpathlineto{\pgfqpoint{2.364909in}{2.576825in}}%
\pgfpathlineto{\pgfqpoint{2.365771in}{2.559424in}}%
\pgfpathlineto{\pgfqpoint{2.366638in}{2.538519in}}%
\pgfpathlineto{\pgfqpoint{2.367503in}{2.490069in}}%
\pgfpathlineto{\pgfqpoint{2.368366in}{2.495141in}}%
\pgfpathlineto{\pgfqpoint{2.370096in}{2.565880in}}%
\pgfpathlineto{\pgfqpoint{2.371826in}{2.518045in}}%
\pgfpathlineto{\pgfqpoint{2.372691in}{2.543499in}}%
\pgfpathlineto{\pgfqpoint{2.373557in}{2.492221in}}%
\pgfpathlineto{\pgfqpoint{2.375287in}{2.569077in}}%
\pgfpathlineto{\pgfqpoint{2.376151in}{2.577808in}}%
\pgfpathlineto{\pgfqpoint{2.377017in}{2.603508in}}%
\pgfpathlineto{\pgfqpoint{2.379616in}{2.537536in}}%
\pgfpathlineto{\pgfqpoint{2.380481in}{2.599421in}}%
\pgfpathlineto{\pgfqpoint{2.383076in}{2.525577in}}%
\pgfpathlineto{\pgfqpoint{2.383941in}{2.566803in}}%
\pgfpathlineto{\pgfqpoint{2.385670in}{2.453856in}}%
\pgfpathlineto{\pgfqpoint{2.386535in}{2.561515in}}%
\pgfpathlineto{\pgfqpoint{2.387399in}{2.501168in}}%
\pgfpathlineto{\pgfqpoint{2.388265in}{2.516139in}}%
\pgfpathlineto{\pgfqpoint{2.389131in}{2.567724in}}%
\pgfpathlineto{\pgfqpoint{2.389995in}{2.557857in}}%
\pgfpathlineto{\pgfqpoint{2.390860in}{2.546452in}}%
\pgfpathlineto{\pgfqpoint{2.391725in}{2.555306in}}%
\pgfpathlineto{\pgfqpoint{2.392590in}{2.544669in}}%
\pgfpathlineto{\pgfqpoint{2.393455in}{2.571229in}}%
\pgfpathlineto{\pgfqpoint{2.394320in}{2.565143in}}%
\pgfpathlineto{\pgfqpoint{2.396915in}{2.494189in}}%
\pgfpathlineto{\pgfqpoint{2.397781in}{2.548787in}}%
\pgfpathlineto{\pgfqpoint{2.398647in}{2.543807in}}%
\pgfpathlineto{\pgfqpoint{2.399511in}{2.464737in}}%
\pgfpathlineto{\pgfqpoint{2.400376in}{2.605476in}}%
\pgfpathlineto{\pgfqpoint{2.402105in}{2.529911in}}%
\pgfpathlineto{\pgfqpoint{2.403834in}{2.584940in}}%
\pgfpathlineto{\pgfqpoint{2.404699in}{2.550324in}}%
\pgfpathlineto{\pgfqpoint{2.405564in}{2.595393in}}%
\pgfpathlineto{\pgfqpoint{2.406429in}{2.484167in}}%
\pgfpathlineto{\pgfqpoint{2.408157in}{2.535199in}}%
\pgfpathlineto{\pgfqpoint{2.409022in}{2.542547in}}%
\pgfpathlineto{\pgfqpoint{2.409887in}{2.518168in}}%
\pgfpathlineto{\pgfqpoint{2.410753in}{2.545528in}}%
\pgfpathlineto{\pgfqpoint{2.411618in}{2.540025in}}%
\pgfpathlineto{\pgfqpoint{2.412483in}{2.545467in}}%
\pgfpathlineto{\pgfqpoint{2.414213in}{2.507838in}}%
\pgfpathlineto{\pgfqpoint{2.415079in}{2.557518in}}%
\pgfpathlineto{\pgfqpoint{2.415943in}{2.509868in}}%
\pgfpathlineto{\pgfqpoint{2.416808in}{2.573012in}}%
\pgfpathlineto{\pgfqpoint{2.417672in}{2.545159in}}%
\pgfpathlineto{\pgfqpoint{2.419402in}{2.596622in}}%
\pgfpathlineto{\pgfqpoint{2.421131in}{2.528097in}}%
\pgfpathlineto{\pgfqpoint{2.421995in}{2.555920in}}%
\pgfpathlineto{\pgfqpoint{2.423725in}{2.528621in}}%
\pgfpathlineto{\pgfqpoint{2.424590in}{2.575288in}}%
\pgfpathlineto{\pgfqpoint{2.425455in}{2.515463in}}%
\pgfpathlineto{\pgfqpoint{2.427184in}{2.568278in}}%
\pgfpathlineto{\pgfqpoint{2.428912in}{2.473838in}}%
\pgfpathlineto{\pgfqpoint{2.429777in}{2.533724in}}%
\pgfpathlineto{\pgfqpoint{2.430641in}{2.503105in}}%
\pgfpathlineto{\pgfqpoint{2.432370in}{2.563421in}}%
\pgfpathlineto{\pgfqpoint{2.433233in}{2.470272in}}%
\pgfpathlineto{\pgfqpoint{2.435826in}{2.575165in}}%
\pgfpathlineto{\pgfqpoint{2.436690in}{2.580637in}}%
\pgfpathlineto{\pgfqpoint{2.437555in}{2.524041in}}%
\pgfpathlineto{\pgfqpoint{2.438420in}{2.527022in}}%
\pgfpathlineto{\pgfqpoint{2.439284in}{2.535568in}}%
\pgfpathlineto{\pgfqpoint{2.440150in}{2.566341in}}%
\pgfpathlineto{\pgfqpoint{2.441015in}{2.514848in}}%
\pgfpathlineto{\pgfqpoint{2.441880in}{2.528066in}}%
\pgfpathlineto{\pgfqpoint{2.442744in}{2.481707in}}%
\pgfpathlineto{\pgfqpoint{2.443608in}{2.507962in}}%
\pgfpathlineto{\pgfqpoint{2.444474in}{2.482323in}}%
\pgfpathlineto{\pgfqpoint{2.445339in}{2.550816in}}%
\pgfpathlineto{\pgfqpoint{2.446201in}{2.545098in}}%
\pgfpathlineto{\pgfqpoint{2.447066in}{2.503595in}}%
\pgfpathlineto{\pgfqpoint{2.448795in}{2.553644in}}%
\pgfpathlineto{\pgfqpoint{2.449658in}{2.531202in}}%
\pgfpathlineto{\pgfqpoint{2.450524in}{2.577623in}}%
\pgfpathlineto{\pgfqpoint{2.452253in}{2.523210in}}%
\pgfpathlineto{\pgfqpoint{2.453117in}{2.556533in}}%
\pgfpathlineto{\pgfqpoint{2.453982in}{2.507654in}}%
\pgfpathlineto{\pgfqpoint{2.454848in}{2.562683in}}%
\pgfpathlineto{\pgfqpoint{2.455713in}{2.528066in}}%
\pgfpathlineto{\pgfqpoint{2.457441in}{2.555612in}}%
\pgfpathlineto{\pgfqpoint{2.458306in}{2.524993in}}%
\pgfpathlineto{\pgfqpoint{2.460035in}{2.576578in}}%
\pgfpathlineto{\pgfqpoint{2.461764in}{2.545652in}}%
\pgfpathlineto{\pgfqpoint{2.463494in}{2.564866in}}%
\pgfpathlineto{\pgfqpoint{2.464360in}{2.547927in}}%
\pgfpathlineto{\pgfqpoint{2.465225in}{2.568032in}}%
\pgfpathlineto{\pgfqpoint{2.466090in}{2.493850in}}%
\pgfpathlineto{\pgfqpoint{2.467821in}{2.581743in}}%
\pgfpathlineto{\pgfqpoint{2.469550in}{2.552969in}}%
\pgfpathlineto{\pgfqpoint{2.470414in}{2.611256in}}%
\pgfpathlineto{\pgfqpoint{2.473005in}{2.482507in}}%
\pgfpathlineto{\pgfqpoint{2.473871in}{2.538519in}}%
\pgfpathlineto{\pgfqpoint{2.474737in}{2.533662in}}%
\pgfpathlineto{\pgfqpoint{2.475602in}{2.531787in}}%
\pgfpathlineto{\pgfqpoint{2.476468in}{2.481707in}}%
\pgfpathlineto{\pgfqpoint{2.478200in}{2.558593in}}%
\pgfpathlineto{\pgfqpoint{2.479066in}{2.529480in}}%
\pgfpathlineto{\pgfqpoint{2.479931in}{2.575103in}}%
\pgfpathlineto{\pgfqpoint{2.480797in}{2.548910in}}%
\pgfpathlineto{\pgfqpoint{2.481662in}{2.600742in}}%
\pgfpathlineto{\pgfqpoint{2.482527in}{2.552322in}}%
\pgfpathlineto{\pgfqpoint{2.483391in}{2.593856in}}%
\pgfpathlineto{\pgfqpoint{2.485985in}{2.539196in}}%
\pgfpathlineto{\pgfqpoint{2.486849in}{2.617219in}}%
\pgfpathlineto{\pgfqpoint{2.487715in}{2.605599in}}%
\pgfpathlineto{\pgfqpoint{2.488581in}{2.581805in}}%
\pgfpathlineto{\pgfqpoint{2.489447in}{2.623184in}}%
\pgfpathlineto{\pgfqpoint{2.490310in}{2.549587in}}%
\pgfpathlineto{\pgfqpoint{2.491175in}{2.589859in}}%
\pgfpathlineto{\pgfqpoint{2.492905in}{2.512574in}}%
\pgfpathlineto{\pgfqpoint{2.493769in}{2.578669in}}%
\pgfpathlineto{\pgfqpoint{2.494635in}{2.528713in}}%
\pgfpathlineto{\pgfqpoint{2.495501in}{2.585740in}}%
\pgfpathlineto{\pgfqpoint{2.496365in}{2.576825in}}%
\pgfpathlineto{\pgfqpoint{2.497231in}{2.479064in}}%
\pgfpathlineto{\pgfqpoint{2.498962in}{2.592627in}}%
\pgfpathlineto{\pgfqpoint{2.499826in}{2.551063in}}%
\pgfpathlineto{\pgfqpoint{2.500692in}{2.565882in}}%
\pgfpathlineto{\pgfqpoint{2.501555in}{2.485612in}}%
\pgfpathlineto{\pgfqpoint{2.503285in}{2.583649in}}%
\pgfpathlineto{\pgfqpoint{2.504149in}{2.511589in}}%
\pgfpathlineto{\pgfqpoint{2.505880in}{2.553215in}}%
\pgfpathlineto{\pgfqpoint{2.507611in}{2.511651in}}%
\pgfpathlineto{\pgfqpoint{2.508476in}{2.546513in}}%
\pgfpathlineto{\pgfqpoint{2.509340in}{2.522411in}}%
\pgfpathlineto{\pgfqpoint{2.510205in}{2.584327in}}%
\pgfpathlineto{\pgfqpoint{2.511069in}{2.470949in}}%
\pgfpathlineto{\pgfqpoint{2.511934in}{2.471010in}}%
\pgfpathlineto{\pgfqpoint{2.512798in}{2.536061in}}%
\pgfpathlineto{\pgfqpoint{2.513664in}{2.464186in}}%
\pgfpathlineto{\pgfqpoint{2.514530in}{2.591398in}}%
\pgfpathlineto{\pgfqpoint{2.515396in}{2.546144in}}%
\pgfpathlineto{\pgfqpoint{2.516260in}{2.549587in}}%
\pgfpathlineto{\pgfqpoint{2.517125in}{2.527515in}}%
\pgfpathlineto{\pgfqpoint{2.517990in}{2.561330in}}%
\pgfpathlineto{\pgfqpoint{2.518855in}{2.459082in}}%
\pgfpathlineto{\pgfqpoint{2.520585in}{2.544638in}}%
\pgfpathlineto{\pgfqpoint{2.521451in}{2.482323in}}%
\pgfpathlineto{\pgfqpoint{2.522317in}{2.559917in}}%
\pgfpathlineto{\pgfqpoint{2.523180in}{2.501937in}}%
\pgfpathlineto{\pgfqpoint{2.524912in}{2.579652in}}%
\pgfpathlineto{\pgfqpoint{2.526643in}{2.510176in}}%
\pgfpathlineto{\pgfqpoint{2.527509in}{2.581989in}}%
\pgfpathlineto{\pgfqpoint{2.528374in}{2.506794in}}%
\pgfpathlineto{\pgfqpoint{2.529236in}{2.541226in}}%
\pgfpathlineto{\pgfqpoint{2.530966in}{2.484537in}}%
\pgfpathlineto{\pgfqpoint{2.531831in}{2.520505in}}%
\pgfpathlineto{\pgfqpoint{2.532696in}{2.486566in}}%
\pgfpathlineto{\pgfqpoint{2.534427in}{2.556597in}}%
\pgfpathlineto{\pgfqpoint{2.535290in}{2.495051in}}%
\pgfpathlineto{\pgfqpoint{2.536155in}{2.529421in}}%
\pgfpathlineto{\pgfqpoint{2.537021in}{2.459452in}}%
\pgfpathlineto{\pgfqpoint{2.538751in}{2.531879in}}%
\pgfpathlineto{\pgfqpoint{2.539618in}{2.507685in}}%
\pgfpathlineto{\pgfqpoint{2.540482in}{2.516754in}}%
\pgfpathlineto{\pgfqpoint{2.541346in}{2.458775in}}%
\pgfpathlineto{\pgfqpoint{2.542211in}{2.467475in}}%
\pgfpathlineto{\pgfqpoint{2.543942in}{2.544792in}}%
\pgfpathlineto{\pgfqpoint{2.544808in}{2.554783in}}%
\pgfpathlineto{\pgfqpoint{2.545672in}{2.475744in}}%
\pgfpathlineto{\pgfqpoint{2.546537in}{2.566741in}}%
\pgfpathlineto{\pgfqpoint{2.547403in}{2.501937in}}%
\pgfpathlineto{\pgfqpoint{2.548266in}{2.592442in}}%
\pgfpathlineto{\pgfqpoint{2.549131in}{2.521980in}}%
\pgfpathlineto{\pgfqpoint{2.549993in}{2.534647in}}%
\pgfpathlineto{\pgfqpoint{2.550858in}{2.541780in}}%
\pgfpathlineto{\pgfqpoint{2.551724in}{2.512359in}}%
\pgfpathlineto{\pgfqpoint{2.552590in}{2.516202in}}%
\pgfpathlineto{\pgfqpoint{2.554322in}{2.495112in}}%
\pgfpathlineto{\pgfqpoint{2.555188in}{2.501445in}}%
\pgfpathlineto{\pgfqpoint{2.556051in}{2.541287in}}%
\pgfpathlineto{\pgfqpoint{2.556916in}{2.539258in}}%
\pgfpathlineto{\pgfqpoint{2.557782in}{2.475867in}}%
\pgfpathlineto{\pgfqpoint{2.560377in}{2.577133in}}%
\pgfpathlineto{\pgfqpoint{2.562106in}{2.487980in}}%
\pgfpathlineto{\pgfqpoint{2.564700in}{2.577071in}}%
\pgfpathlineto{\pgfqpoint{2.566428in}{2.516508in}}%
\pgfpathlineto{\pgfqpoint{2.567292in}{2.612424in}}%
\pgfpathlineto{\pgfqpoint{2.568157in}{2.563021in}}%
\pgfpathlineto{\pgfqpoint{2.569022in}{2.585802in}}%
\pgfpathlineto{\pgfqpoint{2.569888in}{2.580514in}}%
\pgfpathlineto{\pgfqpoint{2.570753in}{2.568525in}}%
\pgfpathlineto{\pgfqpoint{2.571618in}{2.582359in}}%
\pgfpathlineto{\pgfqpoint{2.572482in}{2.557028in}}%
\pgfpathlineto{\pgfqpoint{2.573346in}{2.491823in}}%
\pgfpathlineto{\pgfqpoint{2.574211in}{2.592627in}}%
\pgfpathlineto{\pgfqpoint{2.575076in}{2.538950in}}%
\pgfpathlineto{\pgfqpoint{2.575939in}{2.543684in}}%
\pgfpathlineto{\pgfqpoint{2.576802in}{2.568217in}}%
\pgfpathlineto{\pgfqpoint{2.577665in}{2.555827in}}%
\pgfpathlineto{\pgfqpoint{2.578530in}{2.526960in}}%
\pgfpathlineto{\pgfqpoint{2.579394in}{2.547681in}}%
\pgfpathlineto{\pgfqpoint{2.580257in}{2.487857in}}%
\pgfpathlineto{\pgfqpoint{2.581123in}{2.552907in}}%
\pgfpathlineto{\pgfqpoint{2.582854in}{2.505011in}}%
\pgfpathlineto{\pgfqpoint{2.583719in}{2.515710in}}%
\pgfpathlineto{\pgfqpoint{2.584583in}{2.582113in}}%
\pgfpathlineto{\pgfqpoint{2.585448in}{2.473776in}}%
\pgfpathlineto{\pgfqpoint{2.586313in}{2.559486in}}%
\pgfpathlineto{\pgfqpoint{2.587177in}{2.523671in}}%
\pgfpathlineto{\pgfqpoint{2.588042in}{2.533908in}}%
\pgfpathlineto{\pgfqpoint{2.588907in}{2.533785in}}%
\pgfpathlineto{\pgfqpoint{2.590637in}{2.503043in}}%
\pgfpathlineto{\pgfqpoint{2.591500in}{2.496588in}}%
\pgfpathlineto{\pgfqpoint{2.592365in}{2.462279in}}%
\pgfpathlineto{\pgfqpoint{2.593230in}{2.537598in}}%
\pgfpathlineto{\pgfqpoint{2.594095in}{2.531941in}}%
\pgfpathlineto{\pgfqpoint{2.594961in}{2.506425in}}%
\pgfpathlineto{\pgfqpoint{2.595827in}{2.547619in}}%
\pgfpathlineto{\pgfqpoint{2.596692in}{2.528190in}}%
\pgfpathlineto{\pgfqpoint{2.598421in}{2.606584in}}%
\pgfpathlineto{\pgfqpoint{2.599286in}{2.506579in}}%
\pgfpathlineto{\pgfqpoint{2.601015in}{2.592688in}}%
\pgfpathlineto{\pgfqpoint{2.601880in}{2.512236in}}%
\pgfpathlineto{\pgfqpoint{2.602742in}{2.565205in}}%
\pgfpathlineto{\pgfqpoint{2.604469in}{2.455270in}}%
\pgfpathlineto{\pgfqpoint{2.605334in}{2.539627in}}%
\pgfpathlineto{\pgfqpoint{2.606197in}{2.474423in}}%
\pgfpathlineto{\pgfqpoint{2.607063in}{2.536615in}}%
\pgfpathlineto{\pgfqpoint{2.607928in}{2.532679in}}%
\pgfpathlineto{\pgfqpoint{2.608793in}{2.542947in}}%
\pgfpathlineto{\pgfqpoint{2.609657in}{2.541841in}}%
\pgfpathlineto{\pgfqpoint{2.610521in}{2.616852in}}%
\pgfpathlineto{\pgfqpoint{2.611387in}{2.495266in}}%
\pgfpathlineto{\pgfqpoint{2.612252in}{2.539135in}}%
\pgfpathlineto{\pgfqpoint{2.613117in}{2.470949in}}%
\pgfpathlineto{\pgfqpoint{2.613981in}{2.539135in}}%
\pgfpathlineto{\pgfqpoint{2.614846in}{2.469042in}}%
\pgfpathlineto{\pgfqpoint{2.618301in}{2.559794in}}%
\pgfpathlineto{\pgfqpoint{2.619166in}{2.492590in}}%
\pgfpathlineto{\pgfqpoint{2.620031in}{2.506609in}}%
\pgfpathlineto{\pgfqpoint{2.620894in}{2.511620in}}%
\pgfpathlineto{\pgfqpoint{2.622625in}{2.578300in}}%
\pgfpathlineto{\pgfqpoint{2.625219in}{2.455978in}}%
\pgfpathlineto{\pgfqpoint{2.626084in}{2.442113in}}%
\pgfpathlineto{\pgfqpoint{2.628682in}{2.556412in}}%
\pgfpathlineto{\pgfqpoint{2.629548in}{2.549895in}}%
\pgfpathlineto{\pgfqpoint{2.630413in}{2.499754in}}%
\pgfpathlineto{\pgfqpoint{2.631279in}{2.513927in}}%
\pgfpathlineto{\pgfqpoint{2.632145in}{2.549895in}}%
\pgfpathlineto{\pgfqpoint{2.634739in}{2.515094in}}%
\pgfpathlineto{\pgfqpoint{2.635603in}{2.561330in}}%
\pgfpathlineto{\pgfqpoint{2.636470in}{2.546575in}}%
\pgfpathlineto{\pgfqpoint{2.637336in}{2.554506in}}%
\pgfpathlineto{\pgfqpoint{2.638200in}{2.586662in}}%
\pgfpathlineto{\pgfqpoint{2.639929in}{2.501260in}}%
\pgfpathlineto{\pgfqpoint{2.641660in}{2.572889in}}%
\pgfpathlineto{\pgfqpoint{2.642524in}{2.565572in}}%
\pgfpathlineto{\pgfqpoint{2.645115in}{2.483613in}}%
\pgfpathlineto{\pgfqpoint{2.647710in}{2.574549in}}%
\pgfpathlineto{\pgfqpoint{2.648575in}{2.540795in}}%
\pgfpathlineto{\pgfqpoint{2.649441in}{2.546850in}}%
\pgfpathlineto{\pgfqpoint{2.652037in}{2.508146in}}%
\pgfpathlineto{\pgfqpoint{2.652902in}{2.538581in}}%
\pgfpathlineto{\pgfqpoint{2.653767in}{2.526714in}}%
\pgfpathlineto{\pgfqpoint{2.654633in}{2.465107in}}%
\pgfpathlineto{\pgfqpoint{2.655499in}{2.480970in}}%
\pgfpathlineto{\pgfqpoint{2.656365in}{2.552599in}}%
\pgfpathlineto{\pgfqpoint{2.657231in}{2.492590in}}%
\pgfpathlineto{\pgfqpoint{2.659827in}{2.588260in}}%
\pgfpathlineto{\pgfqpoint{2.660689in}{2.559609in}}%
\pgfpathlineto{\pgfqpoint{2.661555in}{2.497571in}}%
\pgfpathlineto{\pgfqpoint{2.662417in}{2.566126in}}%
\pgfpathlineto{\pgfqpoint{2.663283in}{2.500521in}}%
\pgfpathlineto{\pgfqpoint{2.665014in}{2.556104in}}%
\pgfpathlineto{\pgfqpoint{2.666743in}{2.458344in}}%
\pgfpathlineto{\pgfqpoint{2.667607in}{2.542085in}}%
\pgfpathlineto{\pgfqpoint{2.668473in}{2.504919in}}%
\pgfpathlineto{\pgfqpoint{2.669338in}{2.531571in}}%
\pgfpathlineto{\pgfqpoint{2.670204in}{2.487303in}}%
\pgfpathlineto{\pgfqpoint{2.671069in}{2.550816in}}%
\pgfpathlineto{\pgfqpoint{2.671934in}{2.540241in}}%
\pgfpathlineto{\pgfqpoint{2.672800in}{2.552815in}}%
\pgfpathlineto{\pgfqpoint{2.673666in}{2.515217in}}%
\pgfpathlineto{\pgfqpoint{2.674531in}{2.574734in}}%
\pgfpathlineto{\pgfqpoint{2.675397in}{2.506917in}}%
\pgfpathlineto{\pgfqpoint{2.676261in}{2.551370in}}%
\pgfpathlineto{\pgfqpoint{2.677991in}{2.535137in}}%
\pgfpathlineto{\pgfqpoint{2.678857in}{2.576948in}}%
\pgfpathlineto{\pgfqpoint{2.679722in}{2.563298in}}%
\pgfpathlineto{\pgfqpoint{2.680587in}{2.504518in}}%
\pgfpathlineto{\pgfqpoint{2.682316in}{2.591950in}}%
\pgfpathlineto{\pgfqpoint{2.683182in}{2.506486in}}%
\pgfpathlineto{\pgfqpoint{2.684047in}{2.578300in}}%
\pgfpathlineto{\pgfqpoint{2.684912in}{2.505288in}}%
\pgfpathlineto{\pgfqpoint{2.685777in}{2.522042in}}%
\pgfpathlineto{\pgfqpoint{2.686642in}{2.586969in}}%
\pgfpathlineto{\pgfqpoint{2.687507in}{2.554444in}}%
\pgfpathlineto{\pgfqpoint{2.688372in}{2.574672in}}%
\pgfpathlineto{\pgfqpoint{2.690102in}{2.513434in}}%
\pgfpathlineto{\pgfqpoint{2.690968in}{2.523456in}}%
\pgfpathlineto{\pgfqpoint{2.691834in}{2.574028in}}%
\pgfpathlineto{\pgfqpoint{2.692698in}{2.572828in}}%
\pgfpathlineto{\pgfqpoint{2.694429in}{2.584203in}}%
\pgfpathlineto{\pgfqpoint{2.697023in}{2.490900in}}%
\pgfpathlineto{\pgfqpoint{2.697889in}{2.501321in}}%
\pgfpathlineto{\pgfqpoint{2.698752in}{2.515402in}}%
\pgfpathlineto{\pgfqpoint{2.699618in}{2.497878in}}%
\pgfpathlineto{\pgfqpoint{2.700483in}{2.527391in}}%
\pgfpathlineto{\pgfqpoint{2.701346in}{2.517000in}}%
\pgfpathlineto{\pgfqpoint{2.702209in}{2.481278in}}%
\pgfpathlineto{\pgfqpoint{2.703074in}{2.487918in}}%
\pgfpathlineto{\pgfqpoint{2.703940in}{2.499908in}}%
\pgfpathlineto{\pgfqpoint{2.704806in}{2.493021in}}%
\pgfpathlineto{\pgfqpoint{2.705670in}{2.562436in}}%
\pgfpathlineto{\pgfqpoint{2.707401in}{2.526899in}}%
\pgfpathlineto{\pgfqpoint{2.708268in}{2.540672in}}%
\pgfpathlineto{\pgfqpoint{2.709134in}{2.524195in}}%
\pgfpathlineto{\pgfqpoint{2.709995in}{2.600742in}}%
\pgfpathlineto{\pgfqpoint{2.712592in}{2.506486in}}%
\pgfpathlineto{\pgfqpoint{2.713458in}{2.498032in}}%
\pgfpathlineto{\pgfqpoint{2.715189in}{2.577440in}}%
\pgfpathlineto{\pgfqpoint{2.716054in}{2.534493in}}%
\pgfpathlineto{\pgfqpoint{2.716920in}{2.581251in}}%
\pgfpathlineto{\pgfqpoint{2.718650in}{2.546144in}}%
\pgfpathlineto{\pgfqpoint{2.719515in}{2.559363in}}%
\pgfpathlineto{\pgfqpoint{2.720379in}{2.479433in}}%
\pgfpathlineto{\pgfqpoint{2.721245in}{2.563175in}}%
\pgfpathlineto{\pgfqpoint{2.722110in}{2.513249in}}%
\pgfpathlineto{\pgfqpoint{2.723839in}{2.568709in}}%
\pgfpathlineto{\pgfqpoint{2.724705in}{2.552230in}}%
\pgfpathlineto{\pgfqpoint{2.725571in}{2.488011in}}%
\pgfpathlineto{\pgfqpoint{2.726436in}{2.498371in}}%
\pgfpathlineto{\pgfqpoint{2.727301in}{2.558626in}}%
\pgfpathlineto{\pgfqpoint{2.729031in}{2.477527in}}%
\pgfpathlineto{\pgfqpoint{2.730761in}{2.530465in}}%
\pgfpathlineto{\pgfqpoint{2.731626in}{2.512880in}}%
\pgfpathlineto{\pgfqpoint{2.732492in}{2.453025in}}%
\pgfpathlineto{\pgfqpoint{2.734222in}{2.583834in}}%
\pgfpathlineto{\pgfqpoint{2.735950in}{2.530588in}}%
\pgfpathlineto{\pgfqpoint{2.736813in}{2.575411in}}%
\pgfpathlineto{\pgfqpoint{2.738539in}{2.485704in}}%
\pgfpathlineto{\pgfqpoint{2.739404in}{2.511282in}}%
\pgfpathlineto{\pgfqpoint{2.741132in}{2.598467in}}%
\pgfpathlineto{\pgfqpoint{2.741997in}{2.485458in}}%
\pgfpathlineto{\pgfqpoint{2.743725in}{2.547127in}}%
\pgfpathlineto{\pgfqpoint{2.744591in}{2.538919in}}%
\pgfpathlineto{\pgfqpoint{2.745455in}{2.519889in}}%
\pgfpathlineto{\pgfqpoint{2.746322in}{2.549218in}}%
\pgfpathlineto{\pgfqpoint{2.747187in}{2.537598in}}%
\pgfpathlineto{\pgfqpoint{2.748053in}{2.480909in}}%
\pgfpathlineto{\pgfqpoint{2.749784in}{2.559609in}}%
\pgfpathlineto{\pgfqpoint{2.750649in}{2.547127in}}%
\pgfpathlineto{\pgfqpoint{2.751515in}{2.511559in}}%
\pgfpathlineto{\pgfqpoint{2.752380in}{2.583588in}}%
\pgfpathlineto{\pgfqpoint{2.754111in}{2.494435in}}%
\pgfpathlineto{\pgfqpoint{2.754976in}{2.540672in}}%
\pgfpathlineto{\pgfqpoint{2.755840in}{2.492960in}}%
\pgfpathlineto{\pgfqpoint{2.756706in}{2.501198in}}%
\pgfpathlineto{\pgfqpoint{2.757572in}{2.503166in}}%
\pgfpathlineto{\pgfqpoint{2.758437in}{2.497755in}}%
\pgfpathlineto{\pgfqpoint{2.760166in}{2.509068in}}%
\pgfpathlineto{\pgfqpoint{2.761031in}{2.542024in}}%
\pgfpathlineto{\pgfqpoint{2.761898in}{2.472362in}}%
\pgfpathlineto{\pgfqpoint{2.762764in}{2.486012in}}%
\pgfpathlineto{\pgfqpoint{2.763629in}{2.485150in}}%
\pgfpathlineto{\pgfqpoint{2.764491in}{2.541962in}}%
\pgfpathlineto{\pgfqpoint{2.765356in}{2.479801in}}%
\pgfpathlineto{\pgfqpoint{2.766220in}{2.490192in}}%
\pgfpathlineto{\pgfqpoint{2.767951in}{2.574426in}}%
\pgfpathlineto{\pgfqpoint{2.768815in}{2.565757in}}%
\pgfpathlineto{\pgfqpoint{2.770546in}{2.473191in}}%
\pgfpathlineto{\pgfqpoint{2.772275in}{2.533047in}}%
\pgfpathlineto{\pgfqpoint{2.773139in}{2.507284in}}%
\pgfpathlineto{\pgfqpoint{2.774004in}{2.552846in}}%
\pgfpathlineto{\pgfqpoint{2.775728in}{2.449551in}}%
\pgfpathlineto{\pgfqpoint{2.776593in}{2.534155in}}%
\pgfpathlineto{\pgfqpoint{2.777458in}{2.500092in}}%
\pgfpathlineto{\pgfqpoint{2.778324in}{2.504765in}}%
\pgfpathlineto{\pgfqpoint{2.779188in}{2.552907in}}%
\pgfpathlineto{\pgfqpoint{2.780052in}{2.521980in}}%
\pgfpathlineto{\pgfqpoint{2.781780in}{2.548112in}}%
\pgfpathlineto{\pgfqpoint{2.782645in}{2.517339in}}%
\pgfpathlineto{\pgfqpoint{2.783507in}{2.558134in}}%
\pgfpathlineto{\pgfqpoint{2.785235in}{2.528928in}}%
\pgfpathlineto{\pgfqpoint{2.786965in}{2.595270in}}%
\pgfpathlineto{\pgfqpoint{2.787828in}{2.462095in}}%
\pgfpathlineto{\pgfqpoint{2.788692in}{2.587277in}}%
\pgfpathlineto{\pgfqpoint{2.789556in}{2.531081in}}%
\pgfpathlineto{\pgfqpoint{2.790421in}{2.565205in}}%
\pgfpathlineto{\pgfqpoint{2.792151in}{2.529975in}}%
\pgfpathlineto{\pgfqpoint{2.793017in}{2.505134in}}%
\pgfpathlineto{\pgfqpoint{2.793883in}{2.544238in}}%
\pgfpathlineto{\pgfqpoint{2.794748in}{2.524071in}}%
\pgfpathlineto{\pgfqpoint{2.795613in}{2.544792in}}%
\pgfpathlineto{\pgfqpoint{2.797343in}{2.498248in}}%
\pgfpathlineto{\pgfqpoint{2.798208in}{2.534401in}}%
\pgfpathlineto{\pgfqpoint{2.799072in}{2.534339in}}%
\pgfpathlineto{\pgfqpoint{2.799938in}{2.554506in}}%
\pgfpathlineto{\pgfqpoint{2.801665in}{2.504888in}}%
\pgfpathlineto{\pgfqpoint{2.802530in}{2.502368in}}%
\pgfpathlineto{\pgfqpoint{2.804259in}{2.548420in}}%
\pgfpathlineto{\pgfqpoint{2.805124in}{2.513496in}}%
\pgfpathlineto{\pgfqpoint{2.805989in}{2.583834in}}%
\pgfpathlineto{\pgfqpoint{2.806856in}{2.501445in}}%
\pgfpathlineto{\pgfqpoint{2.807721in}{2.554383in}}%
\pgfpathlineto{\pgfqpoint{2.808586in}{2.541133in}}%
\pgfpathlineto{\pgfqpoint{2.811181in}{2.499846in}}%
\pgfpathlineto{\pgfqpoint{2.812912in}{2.553584in}}%
\pgfpathlineto{\pgfqpoint{2.813775in}{2.541164in}}%
\pgfpathlineto{\pgfqpoint{2.814641in}{2.555244in}}%
\pgfpathlineto{\pgfqpoint{2.815503in}{2.465661in}}%
\pgfpathlineto{\pgfqpoint{2.816368in}{2.499231in}}%
\pgfpathlineto{\pgfqpoint{2.817232in}{2.490869in}}%
\pgfpathlineto{\pgfqpoint{2.818096in}{2.468858in}}%
\pgfpathlineto{\pgfqpoint{2.818961in}{2.540548in}}%
\pgfpathlineto{\pgfqpoint{2.819827in}{2.531633in}}%
\pgfpathlineto{\pgfqpoint{2.820692in}{2.481645in}}%
\pgfpathlineto{\pgfqpoint{2.822421in}{2.531510in}}%
\pgfpathlineto{\pgfqpoint{2.823283in}{2.528066in}}%
\pgfpathlineto{\pgfqpoint{2.824146in}{2.470764in}}%
\pgfpathlineto{\pgfqpoint{2.825873in}{2.556104in}}%
\pgfpathlineto{\pgfqpoint{2.826739in}{2.553705in}}%
\pgfpathlineto{\pgfqpoint{2.827603in}{2.494066in}}%
\pgfpathlineto{\pgfqpoint{2.828468in}{2.510974in}}%
\pgfpathlineto{\pgfqpoint{2.829334in}{2.498553in}}%
\pgfpathlineto{\pgfqpoint{2.831064in}{2.588999in}}%
\pgfpathlineto{\pgfqpoint{2.831927in}{2.531019in}}%
\pgfpathlineto{\pgfqpoint{2.832789in}{2.564835in}}%
\pgfpathlineto{\pgfqpoint{2.833654in}{2.533110in}}%
\pgfpathlineto{\pgfqpoint{2.835386in}{2.631731in}}%
\pgfpathlineto{\pgfqpoint{2.837117in}{2.525793in}}%
\pgfpathlineto{\pgfqpoint{2.837982in}{2.537967in}}%
\pgfpathlineto{\pgfqpoint{2.838847in}{2.593119in}}%
\pgfpathlineto{\pgfqpoint{2.839712in}{2.521367in}}%
\pgfpathlineto{\pgfqpoint{2.840577in}{2.576763in}}%
\pgfpathlineto{\pgfqpoint{2.841442in}{2.563483in}}%
\pgfpathlineto{\pgfqpoint{2.843172in}{2.593794in}}%
\pgfpathlineto{\pgfqpoint{2.844901in}{2.487180in}}%
\pgfpathlineto{\pgfqpoint{2.845764in}{2.548725in}}%
\pgfpathlineto{\pgfqpoint{2.846628in}{2.516385in}}%
\pgfpathlineto{\pgfqpoint{2.847492in}{2.557641in}}%
\pgfpathlineto{\pgfqpoint{2.848356in}{2.551001in}}%
\pgfpathlineto{\pgfqpoint{2.849221in}{2.528559in}}%
\pgfpathlineto{\pgfqpoint{2.850086in}{2.571722in}}%
\pgfpathlineto{\pgfqpoint{2.850950in}{2.534401in}}%
\pgfpathlineto{\pgfqpoint{2.851815in}{2.547804in}}%
\pgfpathlineto{\pgfqpoint{2.852680in}{2.599021in}}%
\pgfpathlineto{\pgfqpoint{2.854407in}{2.507962in}}%
\pgfpathlineto{\pgfqpoint{2.856137in}{2.528590in}}%
\pgfpathlineto{\pgfqpoint{2.857003in}{2.461910in}}%
\pgfpathlineto{\pgfqpoint{2.858734in}{2.535476in}}%
\pgfpathlineto{\pgfqpoint{2.859599in}{2.483000in}}%
\pgfpathlineto{\pgfqpoint{2.860465in}{2.565757in}}%
\pgfpathlineto{\pgfqpoint{2.861329in}{2.516693in}}%
\pgfpathlineto{\pgfqpoint{2.862193in}{2.612547in}}%
\pgfpathlineto{\pgfqpoint{2.863056in}{2.499785in}}%
\pgfpathlineto{\pgfqpoint{2.863921in}{2.585494in}}%
\pgfpathlineto{\pgfqpoint{2.864787in}{2.492529in}}%
\pgfpathlineto{\pgfqpoint{2.866517in}{2.573689in}}%
\pgfpathlineto{\pgfqpoint{2.869111in}{2.489270in}}%
\pgfpathlineto{\pgfqpoint{2.869976in}{2.582236in}}%
\pgfpathlineto{\pgfqpoint{2.872568in}{2.498063in}}%
\pgfpathlineto{\pgfqpoint{2.873433in}{2.549464in}}%
\pgfpathlineto{\pgfqpoint{2.874296in}{2.523579in}}%
\pgfpathlineto{\pgfqpoint{2.875161in}{2.543653in}}%
\pgfpathlineto{\pgfqpoint{2.876025in}{2.593548in}}%
\pgfpathlineto{\pgfqpoint{2.877753in}{2.487641in}}%
\pgfpathlineto{\pgfqpoint{2.878619in}{2.542455in}}%
\pgfpathlineto{\pgfqpoint{2.879484in}{2.426003in}}%
\pgfpathlineto{\pgfqpoint{2.881212in}{2.531817in}}%
\pgfpathlineto{\pgfqpoint{2.882076in}{2.567817in}}%
\pgfpathlineto{\pgfqpoint{2.883803in}{2.529175in}}%
\pgfpathlineto{\pgfqpoint{2.884668in}{2.557610in}}%
\pgfpathlineto{\pgfqpoint{2.885532in}{2.556412in}}%
\pgfpathlineto{\pgfqpoint{2.886397in}{2.483983in}}%
\pgfpathlineto{\pgfqpoint{2.887263in}{2.565603in}}%
\pgfpathlineto{\pgfqpoint{2.888127in}{2.503351in}}%
\pgfpathlineto{\pgfqpoint{2.888993in}{2.562131in}}%
\pgfpathlineto{\pgfqpoint{2.889857in}{2.507471in}}%
\pgfpathlineto{\pgfqpoint{2.890721in}{2.559673in}}%
\pgfpathlineto{\pgfqpoint{2.891587in}{2.516079in}}%
\pgfpathlineto{\pgfqpoint{2.892452in}{2.613778in}}%
\pgfpathlineto{\pgfqpoint{2.894183in}{2.510545in}}%
\pgfpathlineto{\pgfqpoint{2.895047in}{2.552907in}}%
\pgfpathlineto{\pgfqpoint{2.896779in}{2.506117in}}%
\pgfpathlineto{\pgfqpoint{2.897643in}{2.568771in}}%
\pgfpathlineto{\pgfqpoint{2.898505in}{2.470579in}}%
\pgfpathlineto{\pgfqpoint{2.899368in}{2.471993in}}%
\pgfpathlineto{\pgfqpoint{2.901099in}{2.521919in}}%
\pgfpathlineto{\pgfqpoint{2.901964in}{2.553829in}}%
\pgfpathlineto{\pgfqpoint{2.903693in}{2.490315in}}%
\pgfpathlineto{\pgfqpoint{2.904557in}{2.524500in}}%
\pgfpathlineto{\pgfqpoint{2.905420in}{2.493450in}}%
\pgfpathlineto{\pgfqpoint{2.906286in}{2.549156in}}%
\pgfpathlineto{\pgfqpoint{2.907152in}{2.501506in}}%
\pgfpathlineto{\pgfqpoint{2.908015in}{2.565633in}}%
\pgfpathlineto{\pgfqpoint{2.909746in}{2.526099in}}%
\pgfpathlineto{\pgfqpoint{2.910611in}{2.530157in}}%
\pgfpathlineto{\pgfqpoint{2.911476in}{2.540302in}}%
\pgfpathlineto{\pgfqpoint{2.912341in}{2.586969in}}%
\pgfpathlineto{\pgfqpoint{2.913205in}{2.499415in}}%
\pgfpathlineto{\pgfqpoint{2.914071in}{2.555673in}}%
\pgfpathlineto{\pgfqpoint{2.914936in}{2.535692in}}%
\pgfpathlineto{\pgfqpoint{2.915801in}{2.449151in}}%
\pgfpathlineto{\pgfqpoint{2.918398in}{2.622138in}}%
\pgfpathlineto{\pgfqpoint{2.920993in}{2.498984in}}%
\pgfpathlineto{\pgfqpoint{2.921858in}{2.516446in}}%
\pgfpathlineto{\pgfqpoint{2.922722in}{2.601971in}}%
\pgfpathlineto{\pgfqpoint{2.924453in}{2.495787in}}%
\pgfpathlineto{\pgfqpoint{2.925318in}{2.502797in}}%
\pgfpathlineto{\pgfqpoint{2.926183in}{2.553952in}}%
\pgfpathlineto{\pgfqpoint{2.927911in}{2.480293in}}%
\pgfpathlineto{\pgfqpoint{2.929642in}{2.564096in}}%
\pgfpathlineto{\pgfqpoint{2.930505in}{2.526960in}}%
\pgfpathlineto{\pgfqpoint{2.931370in}{2.585371in}}%
\pgfpathlineto{\pgfqpoint{2.932235in}{2.500183in}}%
\pgfpathlineto{\pgfqpoint{2.933966in}{2.585433in}}%
\pgfpathlineto{\pgfqpoint{2.935694in}{2.512942in}}%
\pgfpathlineto{\pgfqpoint{2.936558in}{2.511589in}}%
\pgfpathlineto{\pgfqpoint{2.938287in}{2.545467in}}%
\pgfpathlineto{\pgfqpoint{2.939153in}{2.497509in}}%
\pgfpathlineto{\pgfqpoint{2.940017in}{2.556043in}}%
\pgfpathlineto{\pgfqpoint{2.940882in}{2.518045in}}%
\pgfpathlineto{\pgfqpoint{2.941744in}{2.545652in}}%
\pgfpathlineto{\pgfqpoint{2.943475in}{2.447891in}}%
\pgfpathlineto{\pgfqpoint{2.944340in}{2.451273in}}%
\pgfpathlineto{\pgfqpoint{2.946068in}{2.566803in}}%
\pgfpathlineto{\pgfqpoint{2.946935in}{2.558811in}}%
\pgfpathlineto{\pgfqpoint{2.948660in}{2.514358in}}%
\pgfpathlineto{\pgfqpoint{2.949527in}{2.433936in}}%
\pgfpathlineto{\pgfqpoint{2.950389in}{2.572276in}}%
\pgfpathlineto{\pgfqpoint{2.951255in}{2.532987in}}%
\pgfpathlineto{\pgfqpoint{2.952121in}{2.497265in}}%
\pgfpathlineto{\pgfqpoint{2.952984in}{2.537967in}}%
\pgfpathlineto{\pgfqpoint{2.953848in}{2.475898in}}%
\pgfpathlineto{\pgfqpoint{2.955575in}{2.541410in}}%
\pgfpathlineto{\pgfqpoint{2.956440in}{2.515894in}}%
\pgfpathlineto{\pgfqpoint{2.957306in}{2.551863in}}%
\pgfpathlineto{\pgfqpoint{2.959036in}{2.507348in}}%
\pgfpathlineto{\pgfqpoint{2.961632in}{2.607138in}}%
\pgfpathlineto{\pgfqpoint{2.962497in}{2.644582in}}%
\pgfpathlineto{\pgfqpoint{2.963362in}{2.569479in}}%
\pgfpathlineto{\pgfqpoint{2.964228in}{2.596993in}}%
\pgfpathlineto{\pgfqpoint{2.965093in}{2.669605in}}%
\pgfpathlineto{\pgfqpoint{2.965957in}{2.573997in}}%
\pgfpathlineto{\pgfqpoint{2.966821in}{2.625521in}}%
\pgfpathlineto{\pgfqpoint{2.968551in}{2.572276in}}%
\pgfpathlineto{\pgfqpoint{2.970278in}{2.598253in}}%
\pgfpathlineto{\pgfqpoint{2.972009in}{2.682272in}}%
\pgfpathlineto{\pgfqpoint{2.973738in}{2.620849in}}%
\pgfpathlineto{\pgfqpoint{2.974603in}{2.660138in}}%
\pgfpathlineto{\pgfqpoint{2.975468in}{2.654911in}}%
\pgfpathlineto{\pgfqpoint{2.976333in}{2.591521in}}%
\pgfpathlineto{\pgfqpoint{2.978927in}{2.673418in}}%
\pgfpathlineto{\pgfqpoint{2.980655in}{2.613163in}}%
\pgfpathlineto{\pgfqpoint{2.983249in}{2.679198in}}%
\pgfpathlineto{\pgfqpoint{2.984114in}{2.620480in}}%
\pgfpathlineto{\pgfqpoint{2.984979in}{2.673233in}}%
\pgfpathlineto{\pgfqpoint{2.987576in}{2.559732in}}%
\pgfpathlineto{\pgfqpoint{2.990170in}{2.652513in}}%
\pgfpathlineto{\pgfqpoint{2.992763in}{2.610150in}}%
\pgfpathlineto{\pgfqpoint{2.993628in}{2.678521in}}%
\pgfpathlineto{\pgfqpoint{2.994494in}{2.626750in}}%
\pgfpathlineto{\pgfqpoint{2.995360in}{2.644641in}}%
\pgfpathlineto{\pgfqpoint{2.998820in}{2.509529in}}%
\pgfpathlineto{\pgfqpoint{2.999686in}{2.540610in}}%
\pgfpathlineto{\pgfqpoint{3.000551in}{2.535815in}}%
\pgfpathlineto{\pgfqpoint{3.001416in}{2.506363in}}%
\pgfpathlineto{\pgfqpoint{3.002282in}{2.566495in}}%
\pgfpathlineto{\pgfqpoint{3.004010in}{2.519951in}}%
\pgfpathlineto{\pgfqpoint{3.004875in}{2.507346in}}%
\pgfpathlineto{\pgfqpoint{3.005740in}{2.540087in}}%
\pgfpathlineto{\pgfqpoint{3.007470in}{2.497694in}}%
\pgfpathlineto{\pgfqpoint{3.008335in}{2.542393in}}%
\pgfpathlineto{\pgfqpoint{3.010930in}{2.479095in}}%
\pgfpathlineto{\pgfqpoint{3.012659in}{2.533908in}}%
\pgfpathlineto{\pgfqpoint{3.013525in}{2.537413in}}%
\pgfpathlineto{\pgfqpoint{3.014390in}{2.502551in}}%
\pgfpathlineto{\pgfqpoint{3.015255in}{2.542609in}}%
\pgfpathlineto{\pgfqpoint{3.017850in}{2.484383in}}%
\pgfpathlineto{\pgfqpoint{3.019578in}{2.583095in}}%
\pgfpathlineto{\pgfqpoint{3.020443in}{2.542301in}}%
\pgfpathlineto{\pgfqpoint{3.021309in}{2.558072in}}%
\pgfpathlineto{\pgfqpoint{3.023036in}{2.527145in}}%
\pgfpathlineto{\pgfqpoint{3.023900in}{2.528313in}}%
\pgfpathlineto{\pgfqpoint{3.024765in}{2.523302in}}%
\pgfpathlineto{\pgfqpoint{3.025630in}{2.476850in}}%
\pgfpathlineto{\pgfqpoint{3.027361in}{2.522625in}}%
\pgfpathlineto{\pgfqpoint{3.028227in}{2.596745in}}%
\pgfpathlineto{\pgfqpoint{3.029959in}{2.499292in}}%
\pgfpathlineto{\pgfqpoint{3.031687in}{2.573197in}}%
\pgfpathlineto{\pgfqpoint{3.032552in}{2.509622in}}%
\pgfpathlineto{\pgfqpoint{3.033415in}{2.518353in}}%
\pgfpathlineto{\pgfqpoint{3.034281in}{2.494127in}}%
\pgfpathlineto{\pgfqpoint{3.035146in}{2.603323in}}%
\pgfpathlineto{\pgfqpoint{3.036011in}{2.497078in}}%
\pgfpathlineto{\pgfqpoint{3.036876in}{2.579591in}}%
\pgfpathlineto{\pgfqpoint{3.037739in}{2.576332in}}%
\pgfpathlineto{\pgfqpoint{3.038605in}{2.569261in}}%
\pgfpathlineto{\pgfqpoint{3.039470in}{2.442726in}}%
\pgfpathlineto{\pgfqpoint{3.041199in}{2.514048in}}%
\pgfpathlineto{\pgfqpoint{3.042064in}{2.519212in}}%
\pgfpathlineto{\pgfqpoint{3.042930in}{2.500521in}}%
\pgfpathlineto{\pgfqpoint{3.043794in}{2.522779in}}%
\pgfpathlineto{\pgfqpoint{3.045523in}{2.485335in}}%
\pgfpathlineto{\pgfqpoint{3.047252in}{2.519335in}}%
\pgfpathlineto{\pgfqpoint{3.048117in}{2.516508in}}%
\pgfpathlineto{\pgfqpoint{3.048983in}{2.497417in}}%
\pgfpathlineto{\pgfqpoint{3.049848in}{2.507223in}}%
\pgfpathlineto{\pgfqpoint{3.050713in}{2.490007in}}%
\pgfpathlineto{\pgfqpoint{3.052443in}{2.551922in}}%
\pgfpathlineto{\pgfqpoint{3.053308in}{2.502243in}}%
\pgfpathlineto{\pgfqpoint{3.054173in}{2.512018in}}%
\pgfpathlineto{\pgfqpoint{3.055037in}{2.515708in}}%
\pgfpathlineto{\pgfqpoint{3.055901in}{2.502120in}}%
\pgfpathlineto{\pgfqpoint{3.057631in}{2.420959in}}%
\pgfpathlineto{\pgfqpoint{3.059359in}{2.589489in}}%
\pgfpathlineto{\pgfqpoint{3.060224in}{2.552415in}}%
\pgfpathlineto{\pgfqpoint{3.061088in}{2.559178in}}%
\pgfpathlineto{\pgfqpoint{3.061953in}{2.554937in}}%
\pgfpathlineto{\pgfqpoint{3.062818in}{2.537013in}}%
\pgfpathlineto{\pgfqpoint{3.063682in}{2.562252in}}%
\pgfpathlineto{\pgfqpoint{3.064545in}{2.542208in}}%
\pgfpathlineto{\pgfqpoint{3.065410in}{2.548849in}}%
\pgfpathlineto{\pgfqpoint{3.067139in}{2.505871in}}%
\pgfpathlineto{\pgfqpoint{3.068004in}{2.584109in}}%
\pgfpathlineto{\pgfqpoint{3.070600in}{2.488347in}}%
\pgfpathlineto{\pgfqpoint{3.073195in}{2.590166in}}%
\pgfpathlineto{\pgfqpoint{3.074926in}{2.521765in}}%
\pgfpathlineto{\pgfqpoint{3.075790in}{2.595516in}}%
\pgfpathlineto{\pgfqpoint{3.076655in}{2.523702in}}%
\pgfpathlineto{\pgfqpoint{3.077520in}{2.528867in}}%
\pgfpathlineto{\pgfqpoint{3.080116in}{2.482015in}}%
\pgfpathlineto{\pgfqpoint{3.080982in}{2.572520in}}%
\pgfpathlineto{\pgfqpoint{3.081847in}{2.514540in}}%
\pgfpathlineto{\pgfqpoint{3.082712in}{2.605476in}}%
\pgfpathlineto{\pgfqpoint{3.083578in}{2.595393in}}%
\pgfpathlineto{\pgfqpoint{3.084443in}{2.577009in}}%
\pgfpathlineto{\pgfqpoint{3.086170in}{2.511713in}}%
\pgfpathlineto{\pgfqpoint{3.087036in}{2.501321in}}%
\pgfpathlineto{\pgfqpoint{3.088767in}{2.554198in}}%
\pgfpathlineto{\pgfqpoint{3.092227in}{2.495172in}}%
\pgfpathlineto{\pgfqpoint{3.093092in}{2.560407in}}%
\pgfpathlineto{\pgfqpoint{3.093957in}{2.505409in}}%
\pgfpathlineto{\pgfqpoint{3.094822in}{2.512695in}}%
\pgfpathlineto{\pgfqpoint{3.097417in}{2.552230in}}%
\pgfpathlineto{\pgfqpoint{3.098282in}{2.523086in}}%
\pgfpathlineto{\pgfqpoint{3.099147in}{2.550078in}}%
\pgfpathlineto{\pgfqpoint{3.100012in}{2.510176in}}%
\pgfpathlineto{\pgfqpoint{3.100878in}{2.520228in}}%
\pgfpathlineto{\pgfqpoint{3.102605in}{2.490561in}}%
\pgfpathlineto{\pgfqpoint{3.103470in}{2.497170in}}%
\pgfpathlineto{\pgfqpoint{3.104334in}{2.522840in}}%
\pgfpathlineto{\pgfqpoint{3.105199in}{2.497201in}}%
\pgfpathlineto{\pgfqpoint{3.106930in}{2.563114in}}%
\pgfpathlineto{\pgfqpoint{3.107795in}{2.528867in}}%
\pgfpathlineto{\pgfqpoint{3.108661in}{2.535507in}}%
\pgfpathlineto{\pgfqpoint{3.111255in}{2.478264in}}%
\pgfpathlineto{\pgfqpoint{3.112119in}{2.530157in}}%
\pgfpathlineto{\pgfqpoint{3.113850in}{2.394399in}}%
\pgfpathlineto{\pgfqpoint{3.114714in}{2.590780in}}%
\pgfpathlineto{\pgfqpoint{3.115579in}{2.495510in}}%
\pgfpathlineto{\pgfqpoint{3.116443in}{2.585309in}}%
\pgfpathlineto{\pgfqpoint{3.117306in}{2.480109in}}%
\pgfpathlineto{\pgfqpoint{3.118172in}{2.531694in}}%
\pgfpathlineto{\pgfqpoint{3.119902in}{2.485150in}}%
\pgfpathlineto{\pgfqpoint{3.120768in}{2.543992in}}%
\pgfpathlineto{\pgfqpoint{3.121631in}{2.511589in}}%
\pgfpathlineto{\pgfqpoint{3.123362in}{2.560715in}}%
\pgfpathlineto{\pgfqpoint{3.125092in}{2.439745in}}%
\pgfpathlineto{\pgfqpoint{3.125956in}{2.455270in}}%
\pgfpathlineto{\pgfqpoint{3.127686in}{2.537598in}}%
\pgfpathlineto{\pgfqpoint{3.128550in}{2.489211in}}%
\pgfpathlineto{\pgfqpoint{3.130279in}{2.544484in}}%
\pgfpathlineto{\pgfqpoint{3.131143in}{2.416166in}}%
\pgfpathlineto{\pgfqpoint{3.133740in}{2.587216in}}%
\pgfpathlineto{\pgfqpoint{3.134605in}{2.534031in}}%
\pgfpathlineto{\pgfqpoint{3.135471in}{2.552846in}}%
\pgfpathlineto{\pgfqpoint{3.137201in}{2.494928in}}%
\pgfpathlineto{\pgfqpoint{3.138066in}{2.544607in}}%
\pgfpathlineto{\pgfqpoint{3.138930in}{2.513588in}}%
\pgfpathlineto{\pgfqpoint{3.139795in}{2.551863in}}%
\pgfpathlineto{\pgfqpoint{3.141527in}{2.489855in}}%
\pgfpathlineto{\pgfqpoint{3.142392in}{2.501814in}}%
\pgfpathlineto{\pgfqpoint{3.143259in}{2.496465in}}%
\pgfpathlineto{\pgfqpoint{3.144124in}{2.571475in}}%
\pgfpathlineto{\pgfqpoint{3.146723in}{2.463570in}}%
\pgfpathlineto{\pgfqpoint{3.147589in}{2.608673in}}%
\pgfpathlineto{\pgfqpoint{3.148455in}{2.530835in}}%
\pgfpathlineto{\pgfqpoint{3.149320in}{2.551432in}}%
\pgfpathlineto{\pgfqpoint{3.150186in}{2.554506in}}%
\pgfpathlineto{\pgfqpoint{3.151050in}{2.580237in}}%
\pgfpathlineto{\pgfqpoint{3.154510in}{2.494374in}}%
\pgfpathlineto{\pgfqpoint{3.155373in}{2.525547in}}%
\pgfpathlineto{\pgfqpoint{3.156236in}{2.492775in}}%
\pgfpathlineto{\pgfqpoint{3.157101in}{2.559609in}}%
\pgfpathlineto{\pgfqpoint{3.157967in}{2.532556in}}%
\pgfpathlineto{\pgfqpoint{3.158829in}{2.470395in}}%
\pgfpathlineto{\pgfqpoint{3.159695in}{2.510789in}}%
\pgfpathlineto{\pgfqpoint{3.161425in}{2.485766in}}%
\pgfpathlineto{\pgfqpoint{3.162292in}{2.535322in}}%
\pgfpathlineto{\pgfqpoint{3.163156in}{2.485550in}}%
\pgfpathlineto{\pgfqpoint{3.164886in}{2.631053in}}%
\pgfpathlineto{\pgfqpoint{3.165751in}{2.501260in}}%
\pgfpathlineto{\pgfqpoint{3.167480in}{2.564404in}}%
\pgfpathlineto{\pgfqpoint{3.169210in}{2.530650in}}%
\pgfpathlineto{\pgfqpoint{3.170075in}{2.526683in}}%
\pgfpathlineto{\pgfqpoint{3.170938in}{2.465291in}}%
\pgfpathlineto{\pgfqpoint{3.173530in}{2.541962in}}%
\pgfpathlineto{\pgfqpoint{3.174393in}{2.472178in}}%
\pgfpathlineto{\pgfqpoint{3.176124in}{2.519028in}}%
\pgfpathlineto{\pgfqpoint{3.176989in}{2.502150in}}%
\pgfpathlineto{\pgfqpoint{3.177853in}{2.502304in}}%
\pgfpathlineto{\pgfqpoint{3.178719in}{2.507715in}}%
\pgfpathlineto{\pgfqpoint{3.179583in}{2.523394in}}%
\pgfpathlineto{\pgfqpoint{3.180447in}{2.487241in}}%
\pgfpathlineto{\pgfqpoint{3.181311in}{2.523333in}}%
\pgfpathlineto{\pgfqpoint{3.182175in}{2.444725in}}%
\pgfpathlineto{\pgfqpoint{3.183039in}{2.562929in}}%
\pgfpathlineto{\pgfqpoint{3.183904in}{2.533724in}}%
\pgfpathlineto{\pgfqpoint{3.184769in}{2.520444in}}%
\pgfpathlineto{\pgfqpoint{3.185635in}{2.479618in}}%
\pgfpathlineto{\pgfqpoint{3.189095in}{2.562498in}}%
\pgfpathlineto{\pgfqpoint{3.190825in}{2.485273in}}%
\pgfpathlineto{\pgfqpoint{3.192553in}{2.541408in}}%
\pgfpathlineto{\pgfqpoint{3.193417in}{2.552415in}}%
\pgfpathlineto{\pgfqpoint{3.194278in}{2.499169in}}%
\pgfpathlineto{\pgfqpoint{3.195143in}{2.567786in}}%
\pgfpathlineto{\pgfqpoint{3.196006in}{2.547127in}}%
\pgfpathlineto{\pgfqpoint{3.197734in}{2.581866in}}%
\pgfpathlineto{\pgfqpoint{3.199463in}{2.542085in}}%
\pgfpathlineto{\pgfqpoint{3.200327in}{2.522042in}}%
\pgfpathlineto{\pgfqpoint{3.201193in}{2.437315in}}%
\pgfpathlineto{\pgfqpoint{3.202922in}{2.547065in}}%
\pgfpathlineto{\pgfqpoint{3.203786in}{2.495849in}}%
\pgfpathlineto{\pgfqpoint{3.204651in}{2.590043in}}%
\pgfpathlineto{\pgfqpoint{3.205515in}{2.554075in}}%
\pgfpathlineto{\pgfqpoint{3.207245in}{2.576763in}}%
\pgfpathlineto{\pgfqpoint{3.208109in}{2.535753in}}%
\pgfpathlineto{\pgfqpoint{3.208975in}{2.580022in}}%
\pgfpathlineto{\pgfqpoint{3.210707in}{2.519582in}}%
\pgfpathlineto{\pgfqpoint{3.212439in}{2.590720in}}%
\pgfpathlineto{\pgfqpoint{3.213304in}{2.612670in}}%
\pgfpathlineto{\pgfqpoint{3.215032in}{2.528097in}}%
\pgfpathlineto{\pgfqpoint{3.215896in}{2.497817in}}%
\pgfpathlineto{\pgfqpoint{3.216762in}{2.559547in}}%
\pgfpathlineto{\pgfqpoint{3.217627in}{2.556135in}}%
\pgfpathlineto{\pgfqpoint{3.218492in}{2.571044in}}%
\pgfpathlineto{\pgfqpoint{3.219357in}{2.554260in}}%
\pgfpathlineto{\pgfqpoint{3.220222in}{2.569908in}}%
\pgfpathlineto{\pgfqpoint{3.221951in}{2.530342in}}%
\pgfpathlineto{\pgfqpoint{3.222816in}{2.560284in}}%
\pgfpathlineto{\pgfqpoint{3.223681in}{2.540241in}}%
\pgfpathlineto{\pgfqpoint{3.224547in}{2.548695in}}%
\pgfpathlineto{\pgfqpoint{3.225412in}{2.577623in}}%
\pgfpathlineto{\pgfqpoint{3.226276in}{2.573505in}}%
\pgfpathlineto{\pgfqpoint{3.227141in}{2.453364in}}%
\pgfpathlineto{\pgfqpoint{3.228006in}{2.557333in}}%
\pgfpathlineto{\pgfqpoint{3.228872in}{2.540179in}}%
\pgfpathlineto{\pgfqpoint{3.229739in}{2.504488in}}%
\pgfpathlineto{\pgfqpoint{3.230605in}{2.556902in}}%
\pgfpathlineto{\pgfqpoint{3.231470in}{2.523148in}}%
\pgfpathlineto{\pgfqpoint{3.232335in}{2.590474in}}%
\pgfpathlineto{\pgfqpoint{3.233200in}{2.513619in}}%
\pgfpathlineto{\pgfqpoint{3.234064in}{2.532218in}}%
\pgfpathlineto{\pgfqpoint{3.234929in}{2.473961in}}%
\pgfpathlineto{\pgfqpoint{3.237525in}{2.566311in}}%
\pgfpathlineto{\pgfqpoint{3.240120in}{2.478387in}}%
\pgfpathlineto{\pgfqpoint{3.240984in}{2.523025in}}%
\pgfpathlineto{\pgfqpoint{3.241849in}{2.502427in}}%
\pgfpathlineto{\pgfqpoint{3.242713in}{2.527207in}}%
\pgfpathlineto{\pgfqpoint{3.243578in}{2.507746in}}%
\pgfpathlineto{\pgfqpoint{3.245308in}{2.533970in}}%
\pgfpathlineto{\pgfqpoint{3.246173in}{2.504580in}}%
\pgfpathlineto{\pgfqpoint{3.247039in}{2.543376in}}%
\pgfpathlineto{\pgfqpoint{3.247901in}{2.518414in}}%
\pgfpathlineto{\pgfqpoint{3.248767in}{2.542976in}}%
\pgfpathlineto{\pgfqpoint{3.249633in}{2.505994in}}%
\pgfpathlineto{\pgfqpoint{3.250497in}{2.569877in}}%
\pgfpathlineto{\pgfqpoint{3.251361in}{2.471562in}}%
\pgfpathlineto{\pgfqpoint{3.252226in}{2.528374in}}%
\pgfpathlineto{\pgfqpoint{3.253089in}{2.451396in}}%
\pgfpathlineto{\pgfqpoint{3.253954in}{2.548725in}}%
\pgfpathlineto{\pgfqpoint{3.254819in}{2.509006in}}%
\pgfpathlineto{\pgfqpoint{3.255685in}{2.541931in}}%
\pgfpathlineto{\pgfqpoint{3.256550in}{2.456253in}}%
\pgfpathlineto{\pgfqpoint{3.259145in}{2.531325in}}%
\pgfpathlineto{\pgfqpoint{3.260010in}{2.507223in}}%
\pgfpathlineto{\pgfqpoint{3.260876in}{2.539687in}}%
\pgfpathlineto{\pgfqpoint{3.261740in}{2.475498in}}%
\pgfpathlineto{\pgfqpoint{3.262605in}{2.541901in}}%
\pgfpathlineto{\pgfqpoint{3.263469in}{2.470025in}}%
\pgfpathlineto{\pgfqpoint{3.265199in}{2.579529in}}%
\pgfpathlineto{\pgfqpoint{3.266064in}{2.561023in}}%
\pgfpathlineto{\pgfqpoint{3.266929in}{2.545775in}}%
\pgfpathlineto{\pgfqpoint{3.267794in}{2.494466in}}%
\pgfpathlineto{\pgfqpoint{3.269522in}{2.541408in}}%
\pgfpathlineto{\pgfqpoint{3.270385in}{2.512757in}}%
\pgfpathlineto{\pgfqpoint{3.272980in}{2.562252in}}%
\pgfpathlineto{\pgfqpoint{3.273846in}{2.538211in}}%
\pgfpathlineto{\pgfqpoint{3.274709in}{2.541408in}}%
\pgfpathlineto{\pgfqpoint{3.275573in}{2.526160in}}%
\pgfpathlineto{\pgfqpoint{3.276438in}{2.475005in}}%
\pgfpathlineto{\pgfqpoint{3.278164in}{2.570983in}}%
\pgfpathlineto{\pgfqpoint{3.279893in}{2.501198in}}%
\pgfpathlineto{\pgfqpoint{3.280759in}{2.540487in}}%
\pgfpathlineto{\pgfqpoint{3.281623in}{2.493083in}}%
\pgfpathlineto{\pgfqpoint{3.282489in}{2.536736in}}%
\pgfpathlineto{\pgfqpoint{3.283353in}{2.525177in}}%
\pgfpathlineto{\pgfqpoint{3.284218in}{2.529880in}}%
\pgfpathlineto{\pgfqpoint{3.285084in}{2.500213in}}%
\pgfpathlineto{\pgfqpoint{3.285950in}{2.579652in}}%
\pgfpathlineto{\pgfqpoint{3.286815in}{2.466521in}}%
\pgfpathlineto{\pgfqpoint{3.287681in}{2.516015in}}%
\pgfpathlineto{\pgfqpoint{3.288546in}{2.500398in}}%
\pgfpathlineto{\pgfqpoint{3.289412in}{2.557025in}}%
\pgfpathlineto{\pgfqpoint{3.293738in}{2.466767in}}%
\pgfpathlineto{\pgfqpoint{3.294599in}{2.580083in}}%
\pgfpathlineto{\pgfqpoint{3.295464in}{2.571660in}}%
\pgfpathlineto{\pgfqpoint{3.296325in}{2.540518in}}%
\pgfpathlineto{\pgfqpoint{3.297192in}{2.543745in}}%
\pgfpathlineto{\pgfqpoint{3.298922in}{2.499538in}}%
\pgfpathlineto{\pgfqpoint{3.299786in}{2.526776in}}%
\pgfpathlineto{\pgfqpoint{3.300649in}{2.526160in}}%
\pgfpathlineto{\pgfqpoint{3.301516in}{2.505317in}}%
\pgfpathlineto{\pgfqpoint{3.302379in}{2.514109in}}%
\pgfpathlineto{\pgfqpoint{3.303244in}{2.551953in}}%
\pgfpathlineto{\pgfqpoint{3.304108in}{2.514602in}}%
\pgfpathlineto{\pgfqpoint{3.304975in}{2.524931in}}%
\pgfpathlineto{\pgfqpoint{3.306706in}{2.565818in}}%
\pgfpathlineto{\pgfqpoint{3.308436in}{2.489699in}}%
\pgfpathlineto{\pgfqpoint{3.310168in}{2.520595in}}%
\pgfpathlineto{\pgfqpoint{3.311034in}{2.537965in}}%
\pgfpathlineto{\pgfqpoint{3.311899in}{2.439712in}}%
\pgfpathlineto{\pgfqpoint{3.313628in}{2.585615in}}%
\pgfpathlineto{\pgfqpoint{3.314493in}{2.583586in}}%
\pgfpathlineto{\pgfqpoint{3.315359in}{2.490284in}}%
\pgfpathlineto{\pgfqpoint{3.317088in}{2.601417in}}%
\pgfpathlineto{\pgfqpoint{3.317953in}{2.515584in}}%
\pgfpathlineto{\pgfqpoint{3.318817in}{2.528374in}}%
\pgfpathlineto{\pgfqpoint{3.319683in}{2.534860in}}%
\pgfpathlineto{\pgfqpoint{3.320547in}{2.557025in}}%
\pgfpathlineto{\pgfqpoint{3.321413in}{2.513249in}}%
\pgfpathlineto{\pgfqpoint{3.323139in}{2.579529in}}%
\pgfpathlineto{\pgfqpoint{3.324004in}{2.502427in}}%
\pgfpathlineto{\pgfqpoint{3.325734in}{2.636095in}}%
\pgfpathlineto{\pgfqpoint{3.328325in}{2.502735in}}%
\pgfpathlineto{\pgfqpoint{3.329190in}{2.483983in}}%
\pgfpathlineto{\pgfqpoint{3.330054in}{2.570798in}}%
\pgfpathlineto{\pgfqpoint{3.330917in}{2.558378in}}%
\pgfpathlineto{\pgfqpoint{3.331782in}{2.548479in}}%
\pgfpathlineto{\pgfqpoint{3.334376in}{2.480601in}}%
\pgfpathlineto{\pgfqpoint{3.336106in}{2.545528in}}%
\pgfpathlineto{\pgfqpoint{3.336971in}{2.534830in}}%
\pgfpathlineto{\pgfqpoint{3.337833in}{2.515584in}}%
\pgfpathlineto{\pgfqpoint{3.338698in}{2.520872in}}%
\pgfpathlineto{\pgfqpoint{3.339562in}{2.516139in}}%
\pgfpathlineto{\pgfqpoint{3.340427in}{2.474328in}}%
\pgfpathlineto{\pgfqpoint{3.343019in}{2.573626in}}%
\pgfpathlineto{\pgfqpoint{3.343883in}{2.498399in}}%
\pgfpathlineto{\pgfqpoint{3.344746in}{2.557210in}}%
\pgfpathlineto{\pgfqpoint{3.346476in}{2.456745in}}%
\pgfpathlineto{\pgfqpoint{3.347340in}{2.581620in}}%
\pgfpathlineto{\pgfqpoint{3.349071in}{2.453056in}}%
\pgfpathlineto{\pgfqpoint{3.350801in}{2.545867in}}%
\pgfpathlineto{\pgfqpoint{3.351667in}{2.506117in}}%
\pgfpathlineto{\pgfqpoint{3.352533in}{2.537105in}}%
\pgfpathlineto{\pgfqpoint{3.353399in}{2.535168in}}%
\pgfpathlineto{\pgfqpoint{3.354265in}{2.552723in}}%
\pgfpathlineto{\pgfqpoint{3.355130in}{2.596314in}}%
\pgfpathlineto{\pgfqpoint{3.355995in}{2.491300in}}%
\pgfpathlineto{\pgfqpoint{3.356859in}{2.497694in}}%
\pgfpathlineto{\pgfqpoint{3.357724in}{2.505532in}}%
\pgfpathlineto{\pgfqpoint{3.358588in}{2.490561in}}%
\pgfpathlineto{\pgfqpoint{3.359454in}{2.518291in}}%
\pgfpathlineto{\pgfqpoint{3.360319in}{2.463570in}}%
\pgfpathlineto{\pgfqpoint{3.361184in}{2.544053in}}%
\pgfpathlineto{\pgfqpoint{3.362049in}{2.517983in}}%
\pgfpathlineto{\pgfqpoint{3.362913in}{2.521857in}}%
\pgfpathlineto{\pgfqpoint{3.364643in}{2.584817in}}%
\pgfpathlineto{\pgfqpoint{3.365508in}{2.536613in}}%
\pgfpathlineto{\pgfqpoint{3.366374in}{2.547558in}}%
\pgfpathlineto{\pgfqpoint{3.367238in}{2.564497in}}%
\pgfpathlineto{\pgfqpoint{3.368968in}{2.500213in}}%
\pgfpathlineto{\pgfqpoint{3.369832in}{2.524500in}}%
\pgfpathlineto{\pgfqpoint{3.370697in}{2.517799in}}%
\pgfpathlineto{\pgfqpoint{3.372429in}{2.488409in}}%
\pgfpathlineto{\pgfqpoint{3.373294in}{2.581374in}}%
\pgfpathlineto{\pgfqpoint{3.374159in}{2.533785in}}%
\pgfpathlineto{\pgfqpoint{3.375889in}{2.569015in}}%
\pgfpathlineto{\pgfqpoint{3.376753in}{2.480109in}}%
\pgfpathlineto{\pgfqpoint{3.377619in}{2.610456in}}%
\pgfpathlineto{\pgfqpoint{3.380214in}{2.463201in}}%
\pgfpathlineto{\pgfqpoint{3.381944in}{2.536090in}}%
\pgfpathlineto{\pgfqpoint{3.382810in}{2.553767in}}%
\pgfpathlineto{\pgfqpoint{3.383674in}{2.515584in}}%
\pgfpathlineto{\pgfqpoint{3.384537in}{2.521242in}}%
\pgfpathlineto{\pgfqpoint{3.386267in}{2.507100in}}%
\pgfpathlineto{\pgfqpoint{3.387132in}{2.510235in}}%
\pgfpathlineto{\pgfqpoint{3.388863in}{2.536982in}}%
\pgfpathlineto{\pgfqpoint{3.389728in}{2.531571in}}%
\pgfpathlineto{\pgfqpoint{3.390594in}{2.520503in}}%
\pgfpathlineto{\pgfqpoint{3.391460in}{2.491236in}}%
\pgfpathlineto{\pgfqpoint{3.392325in}{2.587891in}}%
\pgfpathlineto{\pgfqpoint{3.393191in}{2.583586in}}%
\pgfpathlineto{\pgfqpoint{3.394056in}{2.568153in}}%
\pgfpathlineto{\pgfqpoint{3.394921in}{2.518781in}}%
\pgfpathlineto{\pgfqpoint{3.395786in}{2.533200in}}%
\pgfpathlineto{\pgfqpoint{3.396649in}{2.476727in}}%
\pgfpathlineto{\pgfqpoint{3.397513in}{2.548356in}}%
\pgfpathlineto{\pgfqpoint{3.398379in}{2.513555in}}%
\pgfpathlineto{\pgfqpoint{3.400975in}{2.558347in}}%
\pgfpathlineto{\pgfqpoint{3.401841in}{2.538273in}}%
\pgfpathlineto{\pgfqpoint{3.402704in}{2.588075in}}%
\pgfpathlineto{\pgfqpoint{3.403569in}{2.537044in}}%
\pgfpathlineto{\pgfqpoint{3.404434in}{2.560592in}}%
\pgfpathlineto{\pgfqpoint{3.405299in}{2.548664in}}%
\pgfpathlineto{\pgfqpoint{3.407893in}{2.448751in}}%
\pgfpathlineto{\pgfqpoint{3.410487in}{2.493543in}}%
\pgfpathlineto{\pgfqpoint{3.411351in}{2.393722in}}%
\pgfpathlineto{\pgfqpoint{3.412214in}{2.559607in}}%
\pgfpathlineto{\pgfqpoint{3.413944in}{2.491482in}}%
\pgfpathlineto{\pgfqpoint{3.414810in}{2.492650in}}%
\pgfpathlineto{\pgfqpoint{3.415676in}{2.547864in}}%
\pgfpathlineto{\pgfqpoint{3.416540in}{2.489945in}}%
\pgfpathlineto{\pgfqpoint{3.419134in}{2.611193in}}%
\pgfpathlineto{\pgfqpoint{3.419999in}{2.501689in}}%
\pgfpathlineto{\pgfqpoint{3.420865in}{2.611254in}}%
\pgfpathlineto{\pgfqpoint{3.422597in}{2.526099in}}%
\pgfpathlineto{\pgfqpoint{3.423462in}{2.608303in}}%
\pgfpathlineto{\pgfqpoint{3.425191in}{2.518843in}}%
\pgfpathlineto{\pgfqpoint{3.426054in}{2.489638in}}%
\pgfpathlineto{\pgfqpoint{3.427782in}{2.531263in}}%
\pgfpathlineto{\pgfqpoint{3.428646in}{2.462523in}}%
\pgfpathlineto{\pgfqpoint{3.430376in}{2.595146in}}%
\pgfpathlineto{\pgfqpoint{3.431241in}{2.519397in}}%
\pgfpathlineto{\pgfqpoint{3.432107in}{2.584078in}}%
\pgfpathlineto{\pgfqpoint{3.432971in}{2.570613in}}%
\pgfpathlineto{\pgfqpoint{3.433836in}{2.603139in}}%
\pgfpathlineto{\pgfqpoint{3.438159in}{2.500090in}}%
\pgfpathlineto{\pgfqpoint{3.439025in}{2.553182in}}%
\pgfpathlineto{\pgfqpoint{3.439889in}{2.535874in}}%
\pgfpathlineto{\pgfqpoint{3.440753in}{2.537903in}}%
\pgfpathlineto{\pgfqpoint{3.441619in}{2.585800in}}%
\pgfpathlineto{\pgfqpoint{3.443347in}{2.491544in}}%
\pgfpathlineto{\pgfqpoint{3.444212in}{2.574180in}}%
\pgfpathlineto{\pgfqpoint{3.445941in}{2.506853in}}%
\pgfpathlineto{\pgfqpoint{3.448533in}{2.572273in}}%
\pgfpathlineto{\pgfqpoint{3.449396in}{2.513924in}}%
\pgfpathlineto{\pgfqpoint{3.451124in}{2.558070in}}%
\pgfpathlineto{\pgfqpoint{3.451988in}{2.494987in}}%
\pgfpathlineto{\pgfqpoint{3.452853in}{2.561421in}}%
\pgfpathlineto{\pgfqpoint{3.454582in}{2.496401in}}%
\pgfpathlineto{\pgfqpoint{3.455444in}{2.496586in}}%
\pgfpathlineto{\pgfqpoint{3.457175in}{2.523515in}}%
\pgfpathlineto{\pgfqpoint{3.458040in}{2.516660in}}%
\pgfpathlineto{\pgfqpoint{3.458905in}{2.534399in}}%
\pgfpathlineto{\pgfqpoint{3.459770in}{2.590472in}}%
\pgfpathlineto{\pgfqpoint{3.460636in}{2.516937in}}%
\pgfpathlineto{\pgfqpoint{3.461498in}{2.544605in}}%
\pgfpathlineto{\pgfqpoint{3.462364in}{2.511987in}}%
\pgfpathlineto{\pgfqpoint{3.463229in}{2.585307in}}%
\pgfpathlineto{\pgfqpoint{3.466686in}{2.457235in}}%
\pgfpathlineto{\pgfqpoint{3.467552in}{2.475003in}}%
\pgfpathlineto{\pgfqpoint{3.469282in}{2.549770in}}%
\pgfpathlineto{\pgfqpoint{3.470145in}{2.542637in}}%
\pgfpathlineto{\pgfqpoint{3.471011in}{2.522286in}}%
\pgfpathlineto{\pgfqpoint{3.471877in}{2.554904in}}%
\pgfpathlineto{\pgfqpoint{3.472743in}{2.525421in}}%
\pgfpathlineto{\pgfqpoint{3.473609in}{2.537349in}}%
\pgfpathlineto{\pgfqpoint{3.474471in}{2.496432in}}%
\pgfpathlineto{\pgfqpoint{3.475336in}{2.549524in}}%
\pgfpathlineto{\pgfqpoint{3.476201in}{2.528865in}}%
\pgfpathlineto{\pgfqpoint{3.477933in}{2.592071in}}%
\pgfpathlineto{\pgfqpoint{3.478798in}{2.528680in}}%
\pgfpathlineto{\pgfqpoint{3.479662in}{2.559545in}}%
\pgfpathlineto{\pgfqpoint{3.480525in}{2.556471in}}%
\pgfpathlineto{\pgfqpoint{3.481389in}{2.478662in}}%
\pgfpathlineto{\pgfqpoint{3.482252in}{2.588196in}}%
\pgfpathlineto{\pgfqpoint{3.483118in}{2.487362in}}%
\pgfpathlineto{\pgfqpoint{3.484847in}{2.631975in}}%
\pgfpathlineto{\pgfqpoint{3.485710in}{2.567047in}}%
\pgfpathlineto{\pgfqpoint{3.486575in}{2.567140in}}%
\pgfpathlineto{\pgfqpoint{3.487439in}{2.570860in}}%
\pgfpathlineto{\pgfqpoint{3.489170in}{2.524931in}}%
\pgfpathlineto{\pgfqpoint{3.490901in}{2.602954in}}%
\pgfpathlineto{\pgfqpoint{3.491766in}{2.591396in}}%
\pgfpathlineto{\pgfqpoint{3.496091in}{2.416349in}}%
\pgfpathlineto{\pgfqpoint{3.497819in}{2.532248in}}%
\pgfpathlineto{\pgfqpoint{3.498684in}{2.488716in}}%
\pgfpathlineto{\pgfqpoint{3.499548in}{2.621155in}}%
\pgfpathlineto{\pgfqpoint{3.502142in}{2.465045in}}%
\pgfpathlineto{\pgfqpoint{3.503870in}{2.573749in}}%
\pgfpathlineto{\pgfqpoint{3.504734in}{2.476358in}}%
\pgfpathlineto{\pgfqpoint{3.505598in}{2.549739in}}%
\pgfpathlineto{\pgfqpoint{3.506463in}{2.482199in}}%
\pgfpathlineto{\pgfqpoint{3.508192in}{2.526283in}}%
\pgfpathlineto{\pgfqpoint{3.509057in}{2.524993in}}%
\pgfpathlineto{\pgfqpoint{3.509923in}{2.514448in}}%
\pgfpathlineto{\pgfqpoint{3.510788in}{2.522840in}}%
\pgfpathlineto{\pgfqpoint{3.511653in}{2.505317in}}%
\pgfpathlineto{\pgfqpoint{3.512519in}{2.530342in}}%
\pgfpathlineto{\pgfqpoint{3.513384in}{2.530034in}}%
\pgfpathlineto{\pgfqpoint{3.514249in}{2.536859in}}%
\pgfpathlineto{\pgfqpoint{3.515113in}{2.458159in}}%
\pgfpathlineto{\pgfqpoint{3.517709in}{2.529973in}}%
\pgfpathlineto{\pgfqpoint{3.518574in}{2.504026in}}%
\pgfpathlineto{\pgfqpoint{3.519437in}{2.522071in}}%
\pgfpathlineto{\pgfqpoint{3.520301in}{2.477710in}}%
\pgfpathlineto{\pgfqpoint{3.521166in}{2.478202in}}%
\pgfpathlineto{\pgfqpoint{3.522032in}{2.479493in}}%
\pgfpathlineto{\pgfqpoint{3.522897in}{2.568399in}}%
\pgfpathlineto{\pgfqpoint{3.523762in}{2.463691in}}%
\pgfpathlineto{\pgfqpoint{3.525491in}{2.503595in}}%
\pgfpathlineto{\pgfqpoint{3.526355in}{2.515584in}}%
\pgfpathlineto{\pgfqpoint{3.527220in}{2.503287in}}%
\pgfpathlineto{\pgfqpoint{3.528086in}{2.580143in}}%
\pgfpathlineto{\pgfqpoint{3.528952in}{2.435040in}}%
\pgfpathlineto{\pgfqpoint{3.529817in}{2.505255in}}%
\pgfpathlineto{\pgfqpoint{3.530682in}{2.462831in}}%
\pgfpathlineto{\pgfqpoint{3.532412in}{2.531448in}}%
\pgfpathlineto{\pgfqpoint{3.533278in}{2.521550in}}%
\pgfpathlineto{\pgfqpoint{3.534143in}{2.545528in}}%
\pgfpathlineto{\pgfqpoint{3.535870in}{2.494127in}}%
\pgfpathlineto{\pgfqpoint{3.536735in}{2.502920in}}%
\pgfpathlineto{\pgfqpoint{3.537601in}{2.440882in}}%
\pgfpathlineto{\pgfqpoint{3.539331in}{2.535076in}}%
\pgfpathlineto{\pgfqpoint{3.541923in}{2.494805in}}%
\pgfpathlineto{\pgfqpoint{3.542787in}{2.554506in}}%
\pgfpathlineto{\pgfqpoint{3.543652in}{2.506640in}}%
\pgfpathlineto{\pgfqpoint{3.544515in}{2.527022in}}%
\pgfpathlineto{\pgfqpoint{3.545379in}{2.601663in}}%
\pgfpathlineto{\pgfqpoint{3.546245in}{2.499107in}}%
\pgfpathlineto{\pgfqpoint{3.547975in}{2.591611in}}%
\pgfpathlineto{\pgfqpoint{3.548839in}{2.587093in}}%
\pgfpathlineto{\pgfqpoint{3.551434in}{2.483798in}}%
\pgfpathlineto{\pgfqpoint{3.552299in}{2.552476in}}%
\pgfpathlineto{\pgfqpoint{3.554030in}{2.487364in}}%
\pgfpathlineto{\pgfqpoint{3.554896in}{2.571598in}}%
\pgfpathlineto{\pgfqpoint{3.555761in}{2.469227in}}%
\pgfpathlineto{\pgfqpoint{3.557492in}{2.529911in}}%
\pgfpathlineto{\pgfqpoint{3.559222in}{2.464922in}}%
\pgfpathlineto{\pgfqpoint{3.560087in}{2.506086in}}%
\pgfpathlineto{\pgfqpoint{3.560951in}{2.499169in}}%
\pgfpathlineto{\pgfqpoint{3.562682in}{2.545251in}}%
\pgfpathlineto{\pgfqpoint{3.565276in}{2.434549in}}%
\pgfpathlineto{\pgfqpoint{3.566138in}{2.435778in}}%
\pgfpathlineto{\pgfqpoint{3.567868in}{2.528559in}}%
\pgfpathlineto{\pgfqpoint{3.568733in}{2.504888in}}%
\pgfpathlineto{\pgfqpoint{3.570464in}{2.454408in}}%
\pgfpathlineto{\pgfqpoint{3.571329in}{2.516569in}}%
\pgfpathlineto{\pgfqpoint{3.572193in}{2.435748in}}%
\pgfpathlineto{\pgfqpoint{3.573923in}{2.538150in}}%
\pgfpathlineto{\pgfqpoint{3.574787in}{2.465722in}}%
\pgfpathlineto{\pgfqpoint{3.576516in}{2.542668in}}%
\pgfpathlineto{\pgfqpoint{3.577381in}{2.433749in}}%
\pgfpathlineto{\pgfqpoint{3.578247in}{2.513617in}}%
\pgfpathlineto{\pgfqpoint{3.579112in}{2.500183in}}%
\pgfpathlineto{\pgfqpoint{3.579978in}{2.531510in}}%
\pgfpathlineto{\pgfqpoint{3.580842in}{2.452625in}}%
\pgfpathlineto{\pgfqpoint{3.582573in}{2.496647in}}%
\pgfpathlineto{\pgfqpoint{3.584300in}{2.466090in}}%
\pgfpathlineto{\pgfqpoint{3.585164in}{2.513309in}}%
\pgfpathlineto{\pgfqpoint{3.586028in}{2.452961in}}%
\pgfpathlineto{\pgfqpoint{3.587756in}{2.530155in}}%
\pgfpathlineto{\pgfqpoint{3.588620in}{2.472053in}}%
\pgfpathlineto{\pgfqpoint{3.589485in}{2.546879in}}%
\pgfpathlineto{\pgfqpoint{3.590351in}{2.517981in}}%
\pgfpathlineto{\pgfqpoint{3.592078in}{2.569690in}}%
\pgfpathlineto{\pgfqpoint{3.592942in}{2.559361in}}%
\pgfpathlineto{\pgfqpoint{3.594668in}{2.502548in}}%
\pgfpathlineto{\pgfqpoint{3.595532in}{2.550599in}}%
\pgfpathlineto{\pgfqpoint{3.597262in}{2.457603in}}%
\pgfpathlineto{\pgfqpoint{3.598128in}{2.466672in}}%
\pgfpathlineto{\pgfqpoint{3.598992in}{2.426186in}}%
\pgfpathlineto{\pgfqpoint{3.599857in}{2.507959in}}%
\pgfpathlineto{\pgfqpoint{3.600722in}{2.486041in}}%
\pgfpathlineto{\pgfqpoint{3.601586in}{2.456558in}}%
\pgfpathlineto{\pgfqpoint{3.603316in}{2.559668in}}%
\pgfpathlineto{\pgfqpoint{3.604181in}{2.558501in}}%
\pgfpathlineto{\pgfqpoint{3.605045in}{2.460833in}}%
\pgfpathlineto{\pgfqpoint{3.605910in}{2.488162in}}%
\pgfpathlineto{\pgfqpoint{3.606775in}{2.490192in}}%
\pgfpathlineto{\pgfqpoint{3.607639in}{2.473438in}}%
\pgfpathlineto{\pgfqpoint{3.608504in}{2.509129in}}%
\pgfpathlineto{\pgfqpoint{3.609370in}{2.500398in}}%
\pgfpathlineto{\pgfqpoint{3.610234in}{2.503564in}}%
\pgfpathlineto{\pgfqpoint{3.611099in}{2.519151in}}%
\pgfpathlineto{\pgfqpoint{3.612827in}{2.505809in}}%
\pgfpathlineto{\pgfqpoint{3.614557in}{2.575624in}}%
\pgfpathlineto{\pgfqpoint{3.617155in}{2.486502in}}%
\pgfpathlineto{\pgfqpoint{3.618881in}{2.567478in}}%
\pgfpathlineto{\pgfqpoint{3.620609in}{2.502304in}}%
\pgfpathlineto{\pgfqpoint{3.621474in}{2.517368in}}%
\pgfpathlineto{\pgfqpoint{3.622340in}{2.482874in}}%
\pgfpathlineto{\pgfqpoint{3.623205in}{2.513678in}}%
\pgfpathlineto{\pgfqpoint{3.624932in}{2.472853in}}%
\pgfpathlineto{\pgfqpoint{3.626663in}{2.484319in}}%
\pgfpathlineto{\pgfqpoint{3.628394in}{2.513124in}}%
\pgfpathlineto{\pgfqpoint{3.630123in}{2.593854in}}%
\pgfpathlineto{\pgfqpoint{3.634446in}{2.500029in}}%
\pgfpathlineto{\pgfqpoint{3.636176in}{2.559976in}}%
\pgfpathlineto{\pgfqpoint{3.637043in}{2.546265in}}%
\pgfpathlineto{\pgfqpoint{3.637907in}{2.560161in}}%
\pgfpathlineto{\pgfqpoint{3.639640in}{2.472422in}}%
\pgfpathlineto{\pgfqpoint{3.641372in}{2.527081in}}%
\pgfpathlineto{\pgfqpoint{3.642235in}{2.475927in}}%
\pgfpathlineto{\pgfqpoint{3.645693in}{2.590657in}}%
\pgfpathlineto{\pgfqpoint{3.647418in}{2.530525in}}%
\pgfpathlineto{\pgfqpoint{3.649149in}{2.560530in}}%
\pgfpathlineto{\pgfqpoint{3.650016in}{2.552415in}}%
\pgfpathlineto{\pgfqpoint{3.650882in}{2.573995in}}%
\pgfpathlineto{\pgfqpoint{3.651747in}{2.538273in}}%
\pgfpathlineto{\pgfqpoint{3.652611in}{2.541501in}}%
\pgfpathlineto{\pgfqpoint{3.653476in}{2.570921in}}%
\pgfpathlineto{\pgfqpoint{3.654343in}{2.519459in}}%
\pgfpathlineto{\pgfqpoint{3.655208in}{2.579344in}}%
\pgfpathlineto{\pgfqpoint{3.656940in}{2.458282in}}%
\pgfpathlineto{\pgfqpoint{3.659533in}{2.585923in}}%
\pgfpathlineto{\pgfqpoint{3.660399in}{2.498615in}}%
\pgfpathlineto{\pgfqpoint{3.661264in}{2.528313in}}%
\pgfpathlineto{\pgfqpoint{3.662127in}{2.497355in}}%
\pgfpathlineto{\pgfqpoint{3.662989in}{2.569077in}}%
\pgfpathlineto{\pgfqpoint{3.663855in}{2.525300in}}%
\pgfpathlineto{\pgfqpoint{3.664721in}{2.531787in}}%
\pgfpathlineto{\pgfqpoint{3.665587in}{2.515831in}}%
\pgfpathlineto{\pgfqpoint{3.666450in}{2.534953in}}%
\pgfpathlineto{\pgfqpoint{3.667313in}{2.502551in}}%
\pgfpathlineto{\pgfqpoint{3.669910in}{2.588137in}}%
\pgfpathlineto{\pgfqpoint{3.671641in}{2.511251in}}%
\pgfpathlineto{\pgfqpoint{3.672504in}{2.519151in}}%
\pgfpathlineto{\pgfqpoint{3.673368in}{2.515708in}}%
\pgfpathlineto{\pgfqpoint{3.674235in}{2.500891in}}%
\pgfpathlineto{\pgfqpoint{3.675101in}{2.514232in}}%
\pgfpathlineto{\pgfqpoint{3.675965in}{2.485581in}}%
\pgfpathlineto{\pgfqpoint{3.677697in}{2.573441in}}%
\pgfpathlineto{\pgfqpoint{3.678563in}{2.543253in}}%
\pgfpathlineto{\pgfqpoint{3.679428in}{2.484842in}}%
\pgfpathlineto{\pgfqpoint{3.680294in}{2.560715in}}%
\pgfpathlineto{\pgfqpoint{3.681160in}{2.454747in}}%
\pgfpathlineto{\pgfqpoint{3.682026in}{2.483798in}}%
\pgfpathlineto{\pgfqpoint{3.682891in}{2.562129in}}%
\pgfpathlineto{\pgfqpoint{3.684621in}{2.471501in}}%
\pgfpathlineto{\pgfqpoint{3.685485in}{2.517799in}}%
\pgfpathlineto{\pgfqpoint{3.686350in}{2.512018in}}%
\pgfpathlineto{\pgfqpoint{3.687215in}{2.522840in}}%
\pgfpathlineto{\pgfqpoint{3.688079in}{2.507408in}}%
\pgfpathlineto{\pgfqpoint{3.688943in}{2.535936in}}%
\pgfpathlineto{\pgfqpoint{3.689808in}{2.467503in}}%
\pgfpathlineto{\pgfqpoint{3.691536in}{2.520749in}}%
\pgfpathlineto{\pgfqpoint{3.692401in}{2.477217in}}%
\pgfpathlineto{\pgfqpoint{3.693266in}{2.539779in}}%
\pgfpathlineto{\pgfqpoint{3.694131in}{2.457728in}}%
\pgfpathlineto{\pgfqpoint{3.694996in}{2.531263in}}%
\pgfpathlineto{\pgfqpoint{3.695860in}{2.524192in}}%
\pgfpathlineto{\pgfqpoint{3.696724in}{2.503533in}}%
\pgfpathlineto{\pgfqpoint{3.698452in}{2.584571in}}%
\pgfpathlineto{\pgfqpoint{3.700181in}{2.481922in}}%
\pgfpathlineto{\pgfqpoint{3.701046in}{2.502304in}}%
\pgfpathlineto{\pgfqpoint{3.701911in}{2.584263in}}%
\pgfpathlineto{\pgfqpoint{3.702776in}{2.548079in}}%
\pgfpathlineto{\pgfqpoint{3.703642in}{2.550078in}}%
\pgfpathlineto{\pgfqpoint{3.705373in}{2.524439in}}%
\pgfpathlineto{\pgfqpoint{3.706238in}{2.550139in}}%
\pgfpathlineto{\pgfqpoint{3.707103in}{2.543745in}}%
\pgfpathlineto{\pgfqpoint{3.709698in}{2.481522in}}%
\pgfpathlineto{\pgfqpoint{3.710563in}{2.479985in}}%
\pgfpathlineto{\pgfqpoint{3.711428in}{2.459142in}}%
\pgfpathlineto{\pgfqpoint{3.712293in}{2.566832in}}%
\pgfpathlineto{\pgfqpoint{3.713157in}{2.522840in}}%
\pgfpathlineto{\pgfqpoint{3.714022in}{2.539194in}}%
\pgfpathlineto{\pgfqpoint{3.715748in}{2.449182in}}%
\pgfpathlineto{\pgfqpoint{3.716611in}{2.502612in}}%
\pgfpathlineto{\pgfqpoint{3.717477in}{2.465045in}}%
\pgfpathlineto{\pgfqpoint{3.719209in}{2.556350in}}%
\pgfpathlineto{\pgfqpoint{3.720074in}{2.551186in}}%
\pgfpathlineto{\pgfqpoint{3.720938in}{2.526837in}}%
\pgfpathlineto{\pgfqpoint{3.722666in}{2.586785in}}%
\pgfpathlineto{\pgfqpoint{3.724396in}{2.529172in}}%
\pgfpathlineto{\pgfqpoint{3.726126in}{2.585433in}}%
\pgfpathlineto{\pgfqpoint{3.726990in}{2.456437in}}%
\pgfpathlineto{\pgfqpoint{3.727855in}{2.555981in}}%
\pgfpathlineto{\pgfqpoint{3.728721in}{2.555766in}}%
\pgfpathlineto{\pgfqpoint{3.729587in}{2.528374in}}%
\pgfpathlineto{\pgfqpoint{3.730452in}{2.563114in}}%
\pgfpathlineto{\pgfqpoint{3.732179in}{2.512818in}}%
\pgfpathlineto{\pgfqpoint{3.733044in}{2.522717in}}%
\pgfpathlineto{\pgfqpoint{3.733910in}{2.490592in}}%
\pgfpathlineto{\pgfqpoint{3.735640in}{2.582972in}}%
\pgfpathlineto{\pgfqpoint{3.736506in}{2.499415in}}%
\pgfpathlineto{\pgfqpoint{3.737372in}{2.516631in}}%
\pgfpathlineto{\pgfqpoint{3.738238in}{2.485366in}}%
\pgfpathlineto{\pgfqpoint{3.739103in}{2.505747in}}%
\pgfpathlineto{\pgfqpoint{3.739967in}{2.484781in}}%
\pgfpathlineto{\pgfqpoint{3.741695in}{2.527574in}}%
\pgfpathlineto{\pgfqpoint{3.742560in}{2.459265in}}%
\pgfpathlineto{\pgfqpoint{3.743425in}{2.507592in}}%
\pgfpathlineto{\pgfqpoint{3.744291in}{2.467873in}}%
\pgfpathlineto{\pgfqpoint{3.745156in}{2.556043in}}%
\pgfpathlineto{\pgfqpoint{3.746022in}{2.540856in}}%
\pgfpathlineto{\pgfqpoint{3.747753in}{2.487210in}}%
\pgfpathlineto{\pgfqpoint{3.749484in}{2.549770in}}%
\pgfpathlineto{\pgfqpoint{3.750349in}{2.526468in}}%
\pgfpathlineto{\pgfqpoint{3.751213in}{2.517245in}}%
\pgfpathlineto{\pgfqpoint{3.752078in}{2.527266in}}%
\pgfpathlineto{\pgfqpoint{3.753810in}{2.481276in}}%
\pgfpathlineto{\pgfqpoint{3.754676in}{2.563419in}}%
\pgfpathlineto{\pgfqpoint{3.755541in}{2.538027in}}%
\pgfpathlineto{\pgfqpoint{3.756406in}{2.464122in}}%
\pgfpathlineto{\pgfqpoint{3.758138in}{2.531325in}}%
\pgfpathlineto{\pgfqpoint{3.759004in}{2.531202in}}%
\pgfpathlineto{\pgfqpoint{3.759869in}{2.484935in}}%
\pgfpathlineto{\pgfqpoint{3.761597in}{2.560715in}}%
\pgfpathlineto{\pgfqpoint{3.762463in}{2.474574in}}%
\pgfpathlineto{\pgfqpoint{3.763326in}{2.478818in}}%
\pgfpathlineto{\pgfqpoint{3.765921in}{2.555119in}}%
\pgfpathlineto{\pgfqpoint{3.767650in}{2.489699in}}%
\pgfpathlineto{\pgfqpoint{3.768513in}{2.561267in}}%
\pgfpathlineto{\pgfqpoint{3.769378in}{2.482721in}}%
\pgfpathlineto{\pgfqpoint{3.771106in}{2.575039in}}%
\pgfpathlineto{\pgfqpoint{3.772835in}{2.542085in}}%
\pgfpathlineto{\pgfqpoint{3.774564in}{2.572397in}}%
\pgfpathlineto{\pgfqpoint{3.775427in}{2.521488in}}%
\pgfpathlineto{\pgfqpoint{3.776290in}{2.528128in}}%
\pgfpathlineto{\pgfqpoint{3.777155in}{2.623736in}}%
\pgfpathlineto{\pgfqpoint{3.778884in}{2.479924in}}%
\pgfpathlineto{\pgfqpoint{3.779750in}{2.562375in}}%
\pgfpathlineto{\pgfqpoint{3.780615in}{2.543253in}}%
\pgfpathlineto{\pgfqpoint{3.781479in}{2.454223in}}%
\pgfpathlineto{\pgfqpoint{3.783208in}{2.560961in}}%
\pgfpathlineto{\pgfqpoint{3.784074in}{2.488593in}}%
\pgfpathlineto{\pgfqpoint{3.784940in}{2.541223in}}%
\pgfpathlineto{\pgfqpoint{3.787530in}{2.500337in}}%
\pgfpathlineto{\pgfqpoint{3.789261in}{2.556841in}}%
\pgfpathlineto{\pgfqpoint{3.790125in}{2.567109in}}%
\pgfpathlineto{\pgfqpoint{3.791853in}{2.457420in}}%
\pgfpathlineto{\pgfqpoint{3.793584in}{2.527389in}}%
\pgfpathlineto{\pgfqpoint{3.795314in}{2.600434in}}%
\pgfpathlineto{\pgfqpoint{3.797911in}{2.458988in}}%
\pgfpathlineto{\pgfqpoint{3.798777in}{2.560530in}}%
\pgfpathlineto{\pgfqpoint{3.799640in}{2.529665in}}%
\pgfpathlineto{\pgfqpoint{3.800503in}{2.541193in}}%
\pgfpathlineto{\pgfqpoint{3.802234in}{2.507531in}}%
\pgfpathlineto{\pgfqpoint{3.803099in}{2.505809in}}%
\pgfpathlineto{\pgfqpoint{3.803964in}{2.542208in}}%
\pgfpathlineto{\pgfqpoint{3.804830in}{2.541624in}}%
\pgfpathlineto{\pgfqpoint{3.807428in}{2.482721in}}%
\pgfpathlineto{\pgfqpoint{3.809157in}{2.548479in}}%
\pgfpathlineto{\pgfqpoint{3.810020in}{2.481892in}}%
\pgfpathlineto{\pgfqpoint{3.810885in}{2.576578in}}%
\pgfpathlineto{\pgfqpoint{3.811751in}{2.495726in}}%
\pgfpathlineto{\pgfqpoint{3.812616in}{2.495849in}}%
\pgfpathlineto{\pgfqpoint{3.813481in}{2.501014in}}%
\pgfpathlineto{\pgfqpoint{3.814344in}{2.452902in}}%
\pgfpathlineto{\pgfqpoint{3.816076in}{2.521180in}}%
\pgfpathlineto{\pgfqpoint{3.817803in}{2.456684in}}%
\pgfpathlineto{\pgfqpoint{3.818667in}{2.514171in}}%
\pgfpathlineto{\pgfqpoint{3.819532in}{2.433687in}}%
\pgfpathlineto{\pgfqpoint{3.821264in}{2.605353in}}%
\pgfpathlineto{\pgfqpoint{3.822992in}{2.548171in}}%
\pgfpathlineto{\pgfqpoint{3.823858in}{2.575132in}}%
\pgfpathlineto{\pgfqpoint{3.824723in}{2.567109in}}%
\pgfpathlineto{\pgfqpoint{3.825589in}{2.580512in}}%
\pgfpathlineto{\pgfqpoint{3.826455in}{2.568615in}}%
\pgfpathlineto{\pgfqpoint{3.827320in}{2.594592in}}%
\pgfpathlineto{\pgfqpoint{3.828185in}{2.556348in}}%
\pgfpathlineto{\pgfqpoint{3.829052in}{2.567999in}}%
\pgfpathlineto{\pgfqpoint{3.830781in}{2.530771in}}%
\pgfpathlineto{\pgfqpoint{3.831644in}{2.582357in}}%
\pgfpathlineto{\pgfqpoint{3.832509in}{2.536120in}}%
\pgfpathlineto{\pgfqpoint{3.835105in}{2.650175in}}%
\pgfpathlineto{\pgfqpoint{3.835971in}{2.562190in}}%
\pgfpathlineto{\pgfqpoint{3.836836in}{2.630376in}}%
\pgfpathlineto{\pgfqpoint{3.837698in}{2.533170in}}%
\pgfpathlineto{\pgfqpoint{3.838562in}{2.567355in}}%
\pgfpathlineto{\pgfqpoint{3.841155in}{2.502120in}}%
\pgfpathlineto{\pgfqpoint{3.842018in}{2.586108in}}%
\pgfpathlineto{\pgfqpoint{3.842884in}{2.528097in}}%
\pgfpathlineto{\pgfqpoint{3.843749in}{2.548171in}}%
\pgfpathlineto{\pgfqpoint{3.844612in}{2.534583in}}%
\pgfpathlineto{\pgfqpoint{3.845478in}{2.541193in}}%
\pgfpathlineto{\pgfqpoint{3.846342in}{2.493142in}}%
\pgfpathlineto{\pgfqpoint{3.847206in}{2.560161in}}%
\pgfpathlineto{\pgfqpoint{3.848933in}{2.517491in}}%
\pgfpathlineto{\pgfqpoint{3.849796in}{2.551799in}}%
\pgfpathlineto{\pgfqpoint{3.850662in}{2.479862in}}%
\pgfpathlineto{\pgfqpoint{3.851528in}{2.584325in}}%
\pgfpathlineto{\pgfqpoint{3.852393in}{2.508729in}}%
\pgfpathlineto{\pgfqpoint{3.853258in}{2.633327in}}%
\pgfpathlineto{\pgfqpoint{3.854987in}{2.544697in}}%
\pgfpathlineto{\pgfqpoint{3.855851in}{2.525421in}}%
\pgfpathlineto{\pgfqpoint{3.856715in}{2.566801in}}%
\pgfpathlineto{\pgfqpoint{3.857580in}{2.489607in}}%
\pgfpathlineto{\pgfqpoint{3.858443in}{2.558439in}}%
\pgfpathlineto{\pgfqpoint{3.859307in}{2.485948in}}%
\pgfpathlineto{\pgfqpoint{3.860171in}{2.524069in}}%
\pgfpathlineto{\pgfqpoint{3.861036in}{2.512141in}}%
\pgfpathlineto{\pgfqpoint{3.861901in}{2.556471in}}%
\pgfpathlineto{\pgfqpoint{3.862767in}{2.548048in}}%
\pgfpathlineto{\pgfqpoint{3.863631in}{2.568215in}}%
\pgfpathlineto{\pgfqpoint{3.864495in}{2.549831in}}%
\pgfpathlineto{\pgfqpoint{3.865358in}{2.586169in}}%
\pgfpathlineto{\pgfqpoint{3.867087in}{2.483367in}}%
\pgfpathlineto{\pgfqpoint{3.867951in}{2.524192in}}%
\pgfpathlineto{\pgfqpoint{3.868815in}{2.517675in}}%
\pgfpathlineto{\pgfqpoint{3.869677in}{2.536367in}}%
\pgfpathlineto{\pgfqpoint{3.870542in}{2.506915in}}%
\pgfpathlineto{\pgfqpoint{3.871407in}{2.577377in}}%
\pgfpathlineto{\pgfqpoint{3.874002in}{2.505409in}}%
\pgfpathlineto{\pgfqpoint{3.874867in}{2.548972in}}%
\pgfpathlineto{\pgfqpoint{3.875730in}{2.506792in}}%
\pgfpathlineto{\pgfqpoint{3.879189in}{2.609779in}}%
\pgfpathlineto{\pgfqpoint{3.880052in}{2.520565in}}%
\pgfpathlineto{\pgfqpoint{3.880914in}{2.544851in}}%
\pgfpathlineto{\pgfqpoint{3.881778in}{2.500583in}}%
\pgfpathlineto{\pgfqpoint{3.882643in}{2.566924in}}%
\pgfpathlineto{\pgfqpoint{3.883507in}{2.508513in}}%
\pgfpathlineto{\pgfqpoint{3.884372in}{2.534645in}}%
\pgfpathlineto{\pgfqpoint{3.885238in}{2.481461in}}%
\pgfpathlineto{\pgfqpoint{3.886103in}{2.536459in}}%
\pgfpathlineto{\pgfqpoint{3.886970in}{2.495172in}}%
\pgfpathlineto{\pgfqpoint{3.887836in}{2.582049in}}%
\pgfpathlineto{\pgfqpoint{3.888702in}{2.479000in}}%
\pgfpathlineto{\pgfqpoint{3.889568in}{2.521917in}}%
\pgfpathlineto{\pgfqpoint{3.890434in}{2.508883in}}%
\pgfpathlineto{\pgfqpoint{3.892166in}{2.547312in}}%
\pgfpathlineto{\pgfqpoint{3.893031in}{2.531417in}}%
\pgfpathlineto{\pgfqpoint{3.893896in}{2.579344in}}%
\pgfpathlineto{\pgfqpoint{3.895627in}{2.527728in}}%
\pgfpathlineto{\pgfqpoint{3.896493in}{2.591272in}}%
\pgfpathlineto{\pgfqpoint{3.897360in}{2.536982in}}%
\pgfpathlineto{\pgfqpoint{3.898225in}{2.540118in}}%
\pgfpathlineto{\pgfqpoint{3.899091in}{2.515954in}}%
\pgfpathlineto{\pgfqpoint{3.900822in}{2.578914in}}%
\pgfpathlineto{\pgfqpoint{3.901689in}{2.484965in}}%
\pgfpathlineto{\pgfqpoint{3.903419in}{2.544913in}}%
\pgfpathlineto{\pgfqpoint{3.904283in}{2.532800in}}%
\pgfpathlineto{\pgfqpoint{3.905147in}{2.548356in}}%
\pgfpathlineto{\pgfqpoint{3.906011in}{2.478387in}}%
\pgfpathlineto{\pgfqpoint{3.906875in}{2.563727in}}%
\pgfpathlineto{\pgfqpoint{3.907739in}{2.484842in}}%
\pgfpathlineto{\pgfqpoint{3.908603in}{2.533722in}}%
\pgfpathlineto{\pgfqpoint{3.909466in}{2.506176in}}%
\pgfpathlineto{\pgfqpoint{3.910332in}{2.527020in}}%
\pgfpathlineto{\pgfqpoint{3.912926in}{2.487670in}}%
\pgfpathlineto{\pgfqpoint{3.913792in}{2.501381in}}%
\pgfpathlineto{\pgfqpoint{3.915522in}{2.566678in}}%
\pgfpathlineto{\pgfqpoint{3.916387in}{2.490990in}}%
\pgfpathlineto{\pgfqpoint{3.917253in}{2.510543in}}%
\pgfpathlineto{\pgfqpoint{3.918118in}{2.582357in}}%
\pgfpathlineto{\pgfqpoint{3.918984in}{2.575255in}}%
\pgfpathlineto{\pgfqpoint{3.919849in}{2.495233in}}%
\pgfpathlineto{\pgfqpoint{3.920712in}{2.568830in}}%
\pgfpathlineto{\pgfqpoint{3.921578in}{2.493481in}}%
\pgfpathlineto{\pgfqpoint{3.922443in}{2.530956in}}%
\pgfpathlineto{\pgfqpoint{3.923309in}{2.527820in}}%
\pgfpathlineto{\pgfqpoint{3.924173in}{2.474451in}}%
\pgfpathlineto{\pgfqpoint{3.925036in}{2.510543in}}%
\pgfpathlineto{\pgfqpoint{3.925900in}{2.482813in}}%
\pgfpathlineto{\pgfqpoint{3.927629in}{2.547494in}}%
\pgfpathlineto{\pgfqpoint{3.928493in}{2.501258in}}%
\pgfpathlineto{\pgfqpoint{3.930224in}{2.559853in}}%
\pgfpathlineto{\pgfqpoint{3.931088in}{2.517245in}}%
\pgfpathlineto{\pgfqpoint{3.931951in}{2.527512in}}%
\pgfpathlineto{\pgfqpoint{3.932817in}{2.506730in}}%
\pgfpathlineto{\pgfqpoint{3.933682in}{2.540762in}}%
\pgfpathlineto{\pgfqpoint{3.934547in}{2.520195in}}%
\pgfpathlineto{\pgfqpoint{3.937139in}{2.571596in}}%
\pgfpathlineto{\pgfqpoint{3.938004in}{2.506884in}}%
\pgfpathlineto{\pgfqpoint{3.939731in}{2.564648in}}%
\pgfpathlineto{\pgfqpoint{3.940595in}{2.511341in}}%
\pgfpathlineto{\pgfqpoint{3.941461in}{2.516814in}}%
\pgfpathlineto{\pgfqpoint{3.942327in}{2.583401in}}%
\pgfpathlineto{\pgfqpoint{3.943191in}{2.536857in}}%
\pgfpathlineto{\pgfqpoint{3.944057in}{2.543620in}}%
\pgfpathlineto{\pgfqpoint{3.945788in}{2.590164in}}%
\pgfpathlineto{\pgfqpoint{3.946653in}{2.496216in}}%
\pgfpathlineto{\pgfqpoint{3.947519in}{2.499413in}}%
\pgfpathlineto{\pgfqpoint{3.948385in}{2.460186in}}%
\pgfpathlineto{\pgfqpoint{3.949249in}{2.516260in}}%
\pgfpathlineto{\pgfqpoint{3.950114in}{2.507775in}}%
\pgfpathlineto{\pgfqpoint{3.950979in}{2.538394in}}%
\pgfpathlineto{\pgfqpoint{3.951843in}{2.530586in}}%
\pgfpathlineto{\pgfqpoint{3.952708in}{2.487362in}}%
\pgfpathlineto{\pgfqpoint{3.953573in}{2.585061in}}%
\pgfpathlineto{\pgfqpoint{3.954436in}{2.578298in}}%
\pgfpathlineto{\pgfqpoint{3.955301in}{2.615926in}}%
\pgfpathlineto{\pgfqpoint{3.957031in}{2.535197in}}%
\pgfpathlineto{\pgfqpoint{3.957894in}{2.555486in}}%
\pgfpathlineto{\pgfqpoint{3.959623in}{2.518289in}}%
\pgfpathlineto{\pgfqpoint{3.960486in}{2.542268in}}%
\pgfpathlineto{\pgfqpoint{3.961351in}{2.515030in}}%
\pgfpathlineto{\pgfqpoint{3.963081in}{2.586967in}}%
\pgfpathlineto{\pgfqpoint{3.963945in}{2.535382in}}%
\pgfpathlineto{\pgfqpoint{3.964811in}{2.588504in}}%
\pgfpathlineto{\pgfqpoint{3.965676in}{2.542514in}}%
\pgfpathlineto{\pgfqpoint{3.966542in}{2.566124in}}%
\pgfpathlineto{\pgfqpoint{3.967407in}{2.545711in}}%
\pgfpathlineto{\pgfqpoint{3.968272in}{2.552905in}}%
\pgfpathlineto{\pgfqpoint{3.969137in}{2.552351in}}%
\pgfpathlineto{\pgfqpoint{3.970002in}{2.580450in}}%
\pgfpathlineto{\pgfqpoint{3.971732in}{2.504147in}}%
\pgfpathlineto{\pgfqpoint{3.972596in}{2.507590in}}%
\pgfpathlineto{\pgfqpoint{3.973460in}{2.534520in}}%
\pgfpathlineto{\pgfqpoint{3.974325in}{2.624350in}}%
\pgfpathlineto{\pgfqpoint{3.976055in}{2.512447in}}%
\pgfpathlineto{\pgfqpoint{3.976921in}{2.574362in}}%
\pgfpathlineto{\pgfqpoint{3.978650in}{2.495631in}}%
\pgfpathlineto{\pgfqpoint{3.979516in}{2.542453in}}%
\pgfpathlineto{\pgfqpoint{3.981249in}{2.457880in}}%
\pgfpathlineto{\pgfqpoint{3.982981in}{2.577682in}}%
\pgfpathlineto{\pgfqpoint{3.983846in}{2.482505in}}%
\pgfpathlineto{\pgfqpoint{3.984712in}{2.513432in}}%
\pgfpathlineto{\pgfqpoint{3.985575in}{2.496247in}}%
\pgfpathlineto{\pgfqpoint{3.986438in}{2.441310in}}%
\pgfpathlineto{\pgfqpoint{3.987304in}{2.451825in}}%
\pgfpathlineto{\pgfqpoint{3.988169in}{2.450472in}}%
\pgfpathlineto{\pgfqpoint{3.989902in}{2.541285in}}%
\pgfpathlineto{\pgfqpoint{3.990765in}{2.445369in}}%
\pgfpathlineto{\pgfqpoint{3.991629in}{2.488039in}}%
\pgfpathlineto{\pgfqpoint{3.992494in}{2.458896in}}%
\pgfpathlineto{\pgfqpoint{3.993360in}{2.554319in}}%
\pgfpathlineto{\pgfqpoint{3.994223in}{2.530956in}}%
\pgfpathlineto{\pgfqpoint{3.995089in}{2.535566in}}%
\pgfpathlineto{\pgfqpoint{3.995954in}{2.587398in}}%
\pgfpathlineto{\pgfqpoint{3.996820in}{2.541593in}}%
\pgfpathlineto{\pgfqpoint{3.997684in}{2.549893in}}%
\pgfpathlineto{\pgfqpoint{3.998549in}{2.554811in}}%
\pgfpathlineto{\pgfqpoint{3.999414in}{2.619554in}}%
\pgfpathlineto{\pgfqpoint{4.000279in}{2.513432in}}%
\pgfpathlineto{\pgfqpoint{4.001143in}{2.524192in}}%
\pgfpathlineto{\pgfqpoint{4.002007in}{2.500337in}}%
\pgfpathlineto{\pgfqpoint{4.002874in}{2.568338in}}%
\pgfpathlineto{\pgfqpoint{4.004603in}{2.513647in}}%
\pgfpathlineto{\pgfqpoint{4.005467in}{2.522779in}}%
\pgfpathlineto{\pgfqpoint{4.006327in}{2.564343in}}%
\pgfpathlineto{\pgfqpoint{4.008057in}{2.470700in}}%
\pgfpathlineto{\pgfqpoint{4.008923in}{2.597420in}}%
\pgfpathlineto{\pgfqpoint{4.009788in}{2.578544in}}%
\pgfpathlineto{\pgfqpoint{4.012381in}{2.492160in}}%
\pgfpathlineto{\pgfqpoint{4.013246in}{2.509006in}}%
\pgfpathlineto{\pgfqpoint{4.014974in}{2.478387in}}%
\pgfpathlineto{\pgfqpoint{4.015839in}{2.570675in}}%
\pgfpathlineto{\pgfqpoint{4.016704in}{2.531140in}}%
\pgfpathlineto{\pgfqpoint{4.018434in}{2.556533in}}%
\pgfpathlineto{\pgfqpoint{4.019301in}{2.481584in}}%
\pgfpathlineto{\pgfqpoint{4.020166in}{2.485150in}}%
\pgfpathlineto{\pgfqpoint{4.021898in}{2.532800in}}%
\pgfpathlineto{\pgfqpoint{4.022764in}{2.508267in}}%
\pgfpathlineto{\pgfqpoint{4.023629in}{2.519028in}}%
\pgfpathlineto{\pgfqpoint{4.025359in}{2.558316in}}%
\pgfpathlineto{\pgfqpoint{4.026224in}{2.555058in}}%
\pgfpathlineto{\pgfqpoint{4.027087in}{2.489945in}}%
\pgfpathlineto{\pgfqpoint{4.028818in}{2.547617in}}%
\pgfpathlineto{\pgfqpoint{4.029684in}{2.484288in}}%
\pgfpathlineto{\pgfqpoint{4.030549in}{2.570365in}}%
\pgfpathlineto{\pgfqpoint{4.031412in}{2.566062in}}%
\pgfpathlineto{\pgfqpoint{4.032277in}{2.561698in}}%
\pgfpathlineto{\pgfqpoint{4.033143in}{2.522040in}}%
\pgfpathlineto{\pgfqpoint{4.034007in}{2.535197in}}%
\pgfpathlineto{\pgfqpoint{4.034872in}{2.580143in}}%
\pgfpathlineto{\pgfqpoint{4.035739in}{2.484165in}}%
\pgfpathlineto{\pgfqpoint{4.036604in}{2.546327in}}%
\pgfpathlineto{\pgfqpoint{4.037469in}{2.491298in}}%
\pgfpathlineto{\pgfqpoint{4.039199in}{2.564710in}}%
\pgfpathlineto{\pgfqpoint{4.040065in}{2.556471in}}%
\pgfpathlineto{\pgfqpoint{4.040931in}{2.503657in}}%
\pgfpathlineto{\pgfqpoint{4.041797in}{2.569998in}}%
\pgfpathlineto{\pgfqpoint{4.044392in}{2.497507in}}%
\pgfpathlineto{\pgfqpoint{4.045257in}{2.513771in}}%
\pgfpathlineto{\pgfqpoint{4.046121in}{2.563789in}}%
\pgfpathlineto{\pgfqpoint{4.046986in}{2.562190in}}%
\pgfpathlineto{\pgfqpoint{4.047850in}{2.508760in}}%
\pgfpathlineto{\pgfqpoint{4.049579in}{2.558562in}}%
\pgfpathlineto{\pgfqpoint{4.051310in}{2.501689in}}%
\pgfpathlineto{\pgfqpoint{4.053040in}{2.575778in}}%
\pgfpathlineto{\pgfqpoint{4.053906in}{2.459080in}}%
\pgfpathlineto{\pgfqpoint{4.054771in}{2.545344in}}%
\pgfpathlineto{\pgfqpoint{4.055636in}{2.535507in}}%
\pgfpathlineto{\pgfqpoint{4.056502in}{2.497140in}}%
\pgfpathlineto{\pgfqpoint{4.057366in}{2.606397in}}%
\pgfpathlineto{\pgfqpoint{4.059092in}{2.529234in}}%
\pgfpathlineto{\pgfqpoint{4.059957in}{2.535751in}}%
\pgfpathlineto{\pgfqpoint{4.060822in}{2.500583in}}%
\pgfpathlineto{\pgfqpoint{4.062552in}{2.552353in}}%
\pgfpathlineto{\pgfqpoint{4.064282in}{2.519766in}}%
\pgfpathlineto{\pgfqpoint{4.065147in}{2.533908in}}%
\pgfpathlineto{\pgfqpoint{4.066012in}{2.528128in}}%
\pgfpathlineto{\pgfqpoint{4.066875in}{2.551953in}}%
\pgfpathlineto{\pgfqpoint{4.067741in}{2.476296in}}%
\pgfpathlineto{\pgfqpoint{4.068604in}{2.512818in}}%
\pgfpathlineto{\pgfqpoint{4.069469in}{2.491975in}}%
\pgfpathlineto{\pgfqpoint{4.071198in}{2.570552in}}%
\pgfpathlineto{\pgfqpoint{4.072928in}{2.485150in}}%
\pgfpathlineto{\pgfqpoint{4.075522in}{2.542945in}}%
\pgfpathlineto{\pgfqpoint{4.076388in}{2.527574in}}%
\pgfpathlineto{\pgfqpoint{4.077253in}{2.533170in}}%
\pgfpathlineto{\pgfqpoint{4.078118in}{2.573810in}}%
\pgfpathlineto{\pgfqpoint{4.078984in}{2.566986in}}%
\pgfpathlineto{\pgfqpoint{4.079849in}{2.536243in}}%
\pgfpathlineto{\pgfqpoint{4.080713in}{2.564127in}}%
\pgfpathlineto{\pgfqpoint{4.081580in}{2.527943in}}%
\pgfpathlineto{\pgfqpoint{4.084175in}{2.581251in}}%
\pgfpathlineto{\pgfqpoint{4.085039in}{2.517429in}}%
\pgfpathlineto{\pgfqpoint{4.087632in}{2.601356in}}%
\pgfpathlineto{\pgfqpoint{4.088498in}{2.505686in}}%
\pgfpathlineto{\pgfqpoint{4.089364in}{2.523271in}}%
\pgfpathlineto{\pgfqpoint{4.090229in}{2.551645in}}%
\pgfpathlineto{\pgfqpoint{4.091095in}{2.620231in}}%
\pgfpathlineto{\pgfqpoint{4.091961in}{2.607136in}}%
\pgfpathlineto{\pgfqpoint{4.092826in}{2.559332in}}%
\pgfpathlineto{\pgfqpoint{4.093691in}{2.653126in}}%
\pgfpathlineto{\pgfqpoint{4.095420in}{2.539042in}}%
\pgfpathlineto{\pgfqpoint{4.096286in}{2.586723in}}%
\pgfpathlineto{\pgfqpoint{4.098016in}{2.545344in}}%
\pgfpathlineto{\pgfqpoint{4.098881in}{2.606890in}}%
\pgfpathlineto{\pgfqpoint{4.099746in}{2.547096in}}%
\pgfpathlineto{\pgfqpoint{4.100611in}{2.559301in}}%
\pgfpathlineto{\pgfqpoint{4.101476in}{2.548541in}}%
\pgfpathlineto{\pgfqpoint{4.102341in}{2.585954in}}%
\pgfpathlineto{\pgfqpoint{4.103206in}{2.524069in}}%
\pgfpathlineto{\pgfqpoint{4.104936in}{2.579129in}}%
\pgfpathlineto{\pgfqpoint{4.105802in}{2.563604in}}%
\pgfpathlineto{\pgfqpoint{4.107530in}{2.410814in}}%
\pgfpathlineto{\pgfqpoint{4.109259in}{2.568092in}}%
\pgfpathlineto{\pgfqpoint{4.111854in}{2.436515in}}%
\pgfpathlineto{\pgfqpoint{4.113584in}{2.551553in}}%
\pgfpathlineto{\pgfqpoint{4.115313in}{2.486318in}}%
\pgfpathlineto{\pgfqpoint{4.116177in}{2.488162in}}%
\pgfpathlineto{\pgfqpoint{4.117040in}{2.511772in}}%
\pgfpathlineto{\pgfqpoint{4.118770in}{2.444877in}}%
\pgfpathlineto{\pgfqpoint{4.120501in}{2.574978in}}%
\pgfpathlineto{\pgfqpoint{4.121367in}{2.511187in}}%
\pgfpathlineto{\pgfqpoint{4.122231in}{2.564587in}}%
\pgfpathlineto{\pgfqpoint{4.124826in}{2.494433in}}%
\pgfpathlineto{\pgfqpoint{4.125692in}{2.556164in}}%
\pgfpathlineto{\pgfqpoint{4.126557in}{2.516383in}}%
\pgfpathlineto{\pgfqpoint{4.128288in}{2.550907in}}%
\pgfpathlineto{\pgfqpoint{4.129154in}{2.530402in}}%
\pgfpathlineto{\pgfqpoint{4.130019in}{2.539440in}}%
\pgfpathlineto{\pgfqpoint{4.130884in}{2.515215in}}%
\pgfpathlineto{\pgfqpoint{4.131746in}{2.541347in}}%
\pgfpathlineto{\pgfqpoint{4.132610in}{2.455021in}}%
\pgfpathlineto{\pgfqpoint{4.134339in}{2.546573in}}%
\pgfpathlineto{\pgfqpoint{4.135203in}{2.509314in}}%
\pgfpathlineto{\pgfqpoint{4.136068in}{2.515831in}}%
\pgfpathlineto{\pgfqpoint{4.136933in}{2.591088in}}%
\pgfpathlineto{\pgfqpoint{4.137798in}{2.539071in}}%
\pgfpathlineto{\pgfqpoint{4.138664in}{2.618140in}}%
\pgfpathlineto{\pgfqpoint{4.139529in}{2.525791in}}%
\pgfpathlineto{\pgfqpoint{4.140395in}{2.562436in}}%
\pgfpathlineto{\pgfqpoint{4.141259in}{2.550385in}}%
\pgfpathlineto{\pgfqpoint{4.142124in}{2.603693in}}%
\pgfpathlineto{\pgfqpoint{4.142987in}{2.461417in}}%
\pgfpathlineto{\pgfqpoint{4.143853in}{2.552415in}}%
\pgfpathlineto{\pgfqpoint{4.144718in}{2.535137in}}%
\pgfpathlineto{\pgfqpoint{4.145583in}{2.555796in}}%
\pgfpathlineto{\pgfqpoint{4.146448in}{2.543376in}}%
\pgfpathlineto{\pgfqpoint{4.148177in}{2.593179in}}%
\pgfpathlineto{\pgfqpoint{4.149043in}{2.490438in}}%
\pgfpathlineto{\pgfqpoint{4.149908in}{2.507377in}}%
\pgfpathlineto{\pgfqpoint{4.150774in}{2.477833in}}%
\pgfpathlineto{\pgfqpoint{4.152507in}{2.532616in}}%
\pgfpathlineto{\pgfqpoint{4.153374in}{2.458649in}}%
\pgfpathlineto{\pgfqpoint{4.154234in}{2.474267in}}%
\pgfpathlineto{\pgfqpoint{4.155100in}{2.479370in}}%
\pgfpathlineto{\pgfqpoint{4.157692in}{2.563727in}}%
\pgfpathlineto{\pgfqpoint{4.158557in}{2.521303in}}%
\pgfpathlineto{\pgfqpoint{4.159420in}{2.599511in}}%
\pgfpathlineto{\pgfqpoint{4.160285in}{2.595822in}}%
\pgfpathlineto{\pgfqpoint{4.162013in}{2.539194in}}%
\pgfpathlineto{\pgfqpoint{4.162876in}{2.547679in}}%
\pgfpathlineto{\pgfqpoint{4.163741in}{2.532369in}}%
\pgfpathlineto{\pgfqpoint{4.164607in}{2.468794in}}%
\pgfpathlineto{\pgfqpoint{4.165472in}{2.562927in}}%
\pgfpathlineto{\pgfqpoint{4.166337in}{2.494772in}}%
\pgfpathlineto{\pgfqpoint{4.167203in}{2.525421in}}%
\pgfpathlineto{\pgfqpoint{4.168067in}{2.442724in}}%
\pgfpathlineto{\pgfqpoint{4.169798in}{2.495785in}}%
\pgfpathlineto{\pgfqpoint{4.170663in}{2.456251in}}%
\pgfpathlineto{\pgfqpoint{4.173258in}{2.540362in}}%
\pgfpathlineto{\pgfqpoint{4.174125in}{2.538948in}}%
\pgfpathlineto{\pgfqpoint{4.176724in}{2.462092in}}%
\pgfpathlineto{\pgfqpoint{4.177588in}{2.537534in}}%
\pgfpathlineto{\pgfqpoint{4.179317in}{2.458772in}}%
\pgfpathlineto{\pgfqpoint{4.181050in}{2.558131in}}%
\pgfpathlineto{\pgfqpoint{4.182780in}{2.477279in}}%
\pgfpathlineto{\pgfqpoint{4.183646in}{2.544482in}}%
\pgfpathlineto{\pgfqpoint{4.184511in}{2.480476in}}%
\pgfpathlineto{\pgfqpoint{4.185377in}{2.496309in}}%
\pgfpathlineto{\pgfqpoint{4.186240in}{2.508637in}}%
\pgfpathlineto{\pgfqpoint{4.187102in}{2.491606in}}%
\pgfpathlineto{\pgfqpoint{4.187968in}{2.501658in}}%
\pgfpathlineto{\pgfqpoint{4.189700in}{2.547186in}}%
\pgfpathlineto{\pgfqpoint{4.190565in}{2.479093in}}%
\pgfpathlineto{\pgfqpoint{4.191432in}{2.529234in}}%
\pgfpathlineto{\pgfqpoint{4.192297in}{2.482382in}}%
\pgfpathlineto{\pgfqpoint{4.193163in}{2.556041in}}%
\pgfpathlineto{\pgfqpoint{4.195760in}{2.489084in}}%
\pgfpathlineto{\pgfqpoint{4.196623in}{2.516321in}}%
\pgfpathlineto{\pgfqpoint{4.198352in}{2.485394in}}%
\pgfpathlineto{\pgfqpoint{4.199218in}{2.591270in}}%
\pgfpathlineto{\pgfqpoint{4.200949in}{2.526958in}}%
\pgfpathlineto{\pgfqpoint{4.201815in}{2.584876in}}%
\pgfpathlineto{\pgfqpoint{4.202680in}{2.499598in}}%
\pgfpathlineto{\pgfqpoint{4.203545in}{2.555979in}}%
\pgfpathlineto{\pgfqpoint{4.204411in}{2.502579in}}%
\pgfpathlineto{\pgfqpoint{4.206142in}{2.542760in}}%
\pgfpathlineto{\pgfqpoint{4.207008in}{2.496863in}}%
\pgfpathlineto{\pgfqpoint{4.208736in}{2.559668in}}%
\pgfpathlineto{\pgfqpoint{4.209602in}{2.549985in}}%
\pgfpathlineto{\pgfqpoint{4.210467in}{2.592871in}}%
\pgfpathlineto{\pgfqpoint{4.211332in}{2.539071in}}%
\pgfpathlineto{\pgfqpoint{4.212198in}{2.546388in}}%
\pgfpathlineto{\pgfqpoint{4.213060in}{2.515030in}}%
\pgfpathlineto{\pgfqpoint{4.213925in}{2.520411in}}%
\pgfpathlineto{\pgfqpoint{4.214789in}{2.549339in}}%
\pgfpathlineto{\pgfqpoint{4.215654in}{2.539810in}}%
\pgfpathlineto{\pgfqpoint{4.216520in}{2.512295in}}%
\pgfpathlineto{\pgfqpoint{4.218250in}{2.592625in}}%
\pgfpathlineto{\pgfqpoint{4.219112in}{2.515677in}}%
\pgfpathlineto{\pgfqpoint{4.219976in}{2.557210in}}%
\pgfpathlineto{\pgfqpoint{4.220842in}{2.548602in}}%
\pgfpathlineto{\pgfqpoint{4.221705in}{2.510420in}}%
\pgfpathlineto{\pgfqpoint{4.222571in}{2.577684in}}%
\pgfpathlineto{\pgfqpoint{4.223436in}{2.536274in}}%
\pgfpathlineto{\pgfqpoint{4.225166in}{2.585985in}}%
\pgfpathlineto{\pgfqpoint{4.226894in}{2.491298in}}%
\pgfpathlineto{\pgfqpoint{4.227760in}{2.568523in}}%
\pgfpathlineto{\pgfqpoint{4.228626in}{2.526099in}}%
\pgfpathlineto{\pgfqpoint{4.229490in}{2.532677in}}%
\pgfpathlineto{\pgfqpoint{4.230356in}{2.531325in}}%
\pgfpathlineto{\pgfqpoint{4.231219in}{2.542945in}}%
\pgfpathlineto{\pgfqpoint{4.232084in}{2.508760in}}%
\pgfpathlineto{\pgfqpoint{4.233814in}{2.539502in}}%
\pgfpathlineto{\pgfqpoint{4.234678in}{2.584509in}}%
\pgfpathlineto{\pgfqpoint{4.235543in}{2.571966in}}%
\pgfpathlineto{\pgfqpoint{4.236409in}{2.589181in}}%
\pgfpathlineto{\pgfqpoint{4.237273in}{2.520749in}}%
\pgfpathlineto{\pgfqpoint{4.238138in}{2.525514in}}%
\pgfpathlineto{\pgfqpoint{4.239867in}{2.576638in}}%
\pgfpathlineto{\pgfqpoint{4.240732in}{2.554134in}}%
\pgfpathlineto{\pgfqpoint{4.241594in}{2.606151in}}%
\pgfpathlineto{\pgfqpoint{4.242461in}{2.525360in}}%
\pgfpathlineto{\pgfqpoint{4.244190in}{2.582172in}}%
\pgfpathlineto{\pgfqpoint{4.245054in}{2.499875in}}%
\pgfpathlineto{\pgfqpoint{4.246784in}{2.591272in}}%
\pgfpathlineto{\pgfqpoint{4.247649in}{2.531017in}}%
\pgfpathlineto{\pgfqpoint{4.248514in}{2.599080in}}%
\pgfpathlineto{\pgfqpoint{4.249380in}{2.541162in}}%
\pgfpathlineto{\pgfqpoint{4.250245in}{2.559915in}}%
\pgfpathlineto{\pgfqpoint{4.251107in}{2.509191in}}%
\pgfpathlineto{\pgfqpoint{4.251972in}{2.513278in}}%
\pgfpathlineto{\pgfqpoint{4.253702in}{2.543191in}}%
\pgfpathlineto{\pgfqpoint{4.255431in}{2.494802in}}%
\pgfpathlineto{\pgfqpoint{4.256297in}{2.507592in}}%
\pgfpathlineto{\pgfqpoint{4.257163in}{2.440143in}}%
\pgfpathlineto{\pgfqpoint{4.258891in}{2.533047in}}%
\pgfpathlineto{\pgfqpoint{4.259758in}{2.458957in}}%
\pgfpathlineto{\pgfqpoint{4.260621in}{2.469841in}}%
\pgfpathlineto{\pgfqpoint{4.263215in}{2.547987in}}%
\pgfpathlineto{\pgfqpoint{4.264079in}{2.503841in}}%
\pgfpathlineto{\pgfqpoint{4.265810in}{2.579837in}}%
\pgfpathlineto{\pgfqpoint{4.266675in}{2.548602in}}%
\pgfpathlineto{\pgfqpoint{4.267541in}{2.572458in}}%
\pgfpathlineto{\pgfqpoint{4.268406in}{2.517922in}}%
\pgfpathlineto{\pgfqpoint{4.270136in}{2.551491in}}%
\pgfpathlineto{\pgfqpoint{4.271000in}{2.495141in}}%
\pgfpathlineto{\pgfqpoint{4.271865in}{2.539194in}}%
\pgfpathlineto{\pgfqpoint{4.272729in}{2.536797in}}%
\pgfpathlineto{\pgfqpoint{4.273593in}{2.509006in}}%
\pgfpathlineto{\pgfqpoint{4.274458in}{2.533416in}}%
\pgfpathlineto{\pgfqpoint{4.276188in}{2.509468in}}%
\pgfpathlineto{\pgfqpoint{4.277054in}{2.511343in}}%
\pgfpathlineto{\pgfqpoint{4.277919in}{2.507715in}}%
\pgfpathlineto{\pgfqpoint{4.279650in}{2.472301in}}%
\pgfpathlineto{\pgfqpoint{4.280515in}{2.504057in}}%
\pgfpathlineto{\pgfqpoint{4.281379in}{2.500583in}}%
\pgfpathlineto{\pgfqpoint{4.282243in}{2.461602in}}%
\pgfpathlineto{\pgfqpoint{4.283109in}{2.548695in}}%
\pgfpathlineto{\pgfqpoint{4.283974in}{2.543007in}}%
\pgfpathlineto{\pgfqpoint{4.284839in}{2.529850in}}%
\pgfpathlineto{\pgfqpoint{4.285704in}{2.491698in}}%
\pgfpathlineto{\pgfqpoint{4.286569in}{2.510912in}}%
\pgfpathlineto{\pgfqpoint{4.288300in}{2.450595in}}%
\pgfpathlineto{\pgfqpoint{4.289164in}{2.524315in}}%
\pgfpathlineto{\pgfqpoint{4.290894in}{2.462646in}}%
\pgfpathlineto{\pgfqpoint{4.292625in}{2.498399in}}%
\pgfpathlineto{\pgfqpoint{4.293491in}{2.488409in}}%
\pgfpathlineto{\pgfqpoint{4.294356in}{2.531140in}}%
\pgfpathlineto{\pgfqpoint{4.295221in}{2.460771in}}%
\pgfpathlineto{\pgfqpoint{4.296948in}{2.528005in}}%
\pgfpathlineto{\pgfqpoint{4.297812in}{2.502243in}}%
\pgfpathlineto{\pgfqpoint{4.298678in}{2.536859in}}%
\pgfpathlineto{\pgfqpoint{4.300410in}{2.463999in}}%
\pgfpathlineto{\pgfqpoint{4.302140in}{2.516783in}}%
\pgfpathlineto{\pgfqpoint{4.303005in}{2.505809in}}%
\pgfpathlineto{\pgfqpoint{4.304734in}{2.484596in}}%
\pgfpathlineto{\pgfqpoint{4.306465in}{2.575409in}}%
\pgfpathlineto{\pgfqpoint{4.307331in}{2.518104in}}%
\pgfpathlineto{\pgfqpoint{4.308196in}{2.607688in}}%
\pgfpathlineto{\pgfqpoint{4.309061in}{2.509650in}}%
\pgfpathlineto{\pgfqpoint{4.309926in}{2.533722in}}%
\pgfpathlineto{\pgfqpoint{4.311656in}{2.480047in}}%
\pgfpathlineto{\pgfqpoint{4.312522in}{2.493512in}}%
\pgfpathlineto{\pgfqpoint{4.313387in}{2.540423in}}%
\pgfpathlineto{\pgfqpoint{4.314253in}{2.533937in}}%
\pgfpathlineto{\pgfqpoint{4.315118in}{2.570059in}}%
\pgfpathlineto{\pgfqpoint{4.315983in}{2.532677in}}%
\pgfpathlineto{\pgfqpoint{4.317712in}{2.580758in}}%
\pgfpathlineto{\pgfqpoint{4.318576in}{2.495079in}}%
\pgfpathlineto{\pgfqpoint{4.320307in}{2.565326in}}%
\pgfpathlineto{\pgfqpoint{4.321170in}{2.491606in}}%
\pgfpathlineto{\pgfqpoint{4.322899in}{2.552782in}}%
\pgfpathlineto{\pgfqpoint{4.323763in}{2.556595in}}%
\pgfpathlineto{\pgfqpoint{4.324629in}{2.528865in}}%
\pgfpathlineto{\pgfqpoint{4.325492in}{2.549524in}}%
\pgfpathlineto{\pgfqpoint{4.326357in}{2.534399in}}%
\pgfpathlineto{\pgfqpoint{4.328083in}{2.558532in}}%
\pgfpathlineto{\pgfqpoint{4.329813in}{2.481584in}}%
\pgfpathlineto{\pgfqpoint{4.331544in}{2.572273in}}%
\pgfpathlineto{\pgfqpoint{4.332409in}{2.567478in}}%
\pgfpathlineto{\pgfqpoint{4.333276in}{2.580943in}}%
\pgfpathlineto{\pgfqpoint{4.334141in}{2.627364in}}%
\pgfpathlineto{\pgfqpoint{4.335868in}{2.486933in}}%
\pgfpathlineto{\pgfqpoint{4.336733in}{2.585430in}}%
\pgfpathlineto{\pgfqpoint{4.337598in}{2.474205in}}%
\pgfpathlineto{\pgfqpoint{4.338461in}{2.526037in}}%
\pgfpathlineto{\pgfqpoint{4.339326in}{2.505501in}}%
\pgfpathlineto{\pgfqpoint{4.340193in}{2.577407in}}%
\pgfpathlineto{\pgfqpoint{4.341058in}{2.537965in}}%
\pgfpathlineto{\pgfqpoint{4.341924in}{2.555796in}}%
\pgfpathlineto{\pgfqpoint{4.342786in}{2.498430in}}%
\pgfpathlineto{\pgfqpoint{4.344511in}{2.580450in}}%
\pgfpathlineto{\pgfqpoint{4.345375in}{2.520257in}}%
\pgfpathlineto{\pgfqpoint{4.346241in}{2.598467in}}%
\pgfpathlineto{\pgfqpoint{4.347106in}{2.527574in}}%
\pgfpathlineto{\pgfqpoint{4.347971in}{2.529850in}}%
\pgfpathlineto{\pgfqpoint{4.348837in}{2.516998in}}%
\pgfpathlineto{\pgfqpoint{4.349701in}{2.573349in}}%
\pgfpathlineto{\pgfqpoint{4.351432in}{2.517429in}}%
\pgfpathlineto{\pgfqpoint{4.352298in}{2.537934in}}%
\pgfpathlineto{\pgfqpoint{4.353162in}{2.613961in}}%
\pgfpathlineto{\pgfqpoint{4.354892in}{2.536120in}}%
\pgfpathlineto{\pgfqpoint{4.355757in}{2.576823in}}%
\pgfpathlineto{\pgfqpoint{4.356622in}{2.564433in}}%
\pgfpathlineto{\pgfqpoint{4.357487in}{2.522163in}}%
\pgfpathlineto{\pgfqpoint{4.358349in}{2.603875in}}%
\pgfpathlineto{\pgfqpoint{4.359212in}{2.586967in}}%
\pgfpathlineto{\pgfqpoint{4.360077in}{2.615742in}}%
\pgfpathlineto{\pgfqpoint{4.360942in}{2.614143in}}%
\pgfpathlineto{\pgfqpoint{4.362672in}{2.525483in}}%
\pgfpathlineto{\pgfqpoint{4.363537in}{2.576515in}}%
\pgfpathlineto{\pgfqpoint{4.364401in}{2.512878in}}%
\pgfpathlineto{\pgfqpoint{4.365264in}{2.518227in}}%
\pgfpathlineto{\pgfqpoint{4.366126in}{2.580204in}}%
\pgfpathlineto{\pgfqpoint{4.366989in}{2.513494in}}%
\pgfpathlineto{\pgfqpoint{4.367853in}{2.525237in}}%
\pgfpathlineto{\pgfqpoint{4.368718in}{2.497969in}}%
\pgfpathlineto{\pgfqpoint{4.369583in}{2.614697in}}%
\pgfpathlineto{\pgfqpoint{4.371313in}{2.501904in}}%
\pgfpathlineto{\pgfqpoint{4.372179in}{2.545159in}}%
\pgfpathlineto{\pgfqpoint{4.373044in}{2.537719in}}%
\pgfpathlineto{\pgfqpoint{4.373909in}{2.535813in}}%
\pgfpathlineto{\pgfqpoint{4.374774in}{2.492896in}}%
\pgfpathlineto{\pgfqpoint{4.377370in}{2.538948in}}%
\pgfpathlineto{\pgfqpoint{4.378235in}{2.506853in}}%
\pgfpathlineto{\pgfqpoint{4.379100in}{2.518658in}}%
\pgfpathlineto{\pgfqpoint{4.379961in}{2.557331in}}%
\pgfpathlineto{\pgfqpoint{4.380827in}{2.498613in}}%
\pgfpathlineto{\pgfqpoint{4.382557in}{2.585492in}}%
\pgfpathlineto{\pgfqpoint{4.383424in}{2.542391in}}%
\pgfpathlineto{\pgfqpoint{4.384290in}{2.573195in}}%
\pgfpathlineto{\pgfqpoint{4.386020in}{2.542268in}}%
\pgfpathlineto{\pgfqpoint{4.386886in}{2.564094in}}%
\pgfpathlineto{\pgfqpoint{4.387751in}{2.557701in}}%
\pgfpathlineto{\pgfqpoint{4.388616in}{2.476111in}}%
\pgfpathlineto{\pgfqpoint{4.389481in}{2.547125in}}%
\pgfpathlineto{\pgfqpoint{4.390346in}{2.489268in}}%
\pgfpathlineto{\pgfqpoint{4.391212in}{2.516444in}}%
\pgfpathlineto{\pgfqpoint{4.392077in}{2.516383in}}%
\pgfpathlineto{\pgfqpoint{4.392939in}{2.476848in}}%
\pgfpathlineto{\pgfqpoint{4.393805in}{2.498430in}}%
\pgfpathlineto{\pgfqpoint{4.394670in}{2.496401in}}%
\pgfpathlineto{\pgfqpoint{4.395533in}{2.462092in}}%
\pgfpathlineto{\pgfqpoint{4.397262in}{2.510143in}}%
\pgfpathlineto{\pgfqpoint{4.398127in}{2.509496in}}%
\pgfpathlineto{\pgfqpoint{4.398990in}{2.593115in}}%
\pgfpathlineto{\pgfqpoint{4.399855in}{2.464581in}}%
\pgfpathlineto{\pgfqpoint{4.401586in}{2.583463in}}%
\pgfpathlineto{\pgfqpoint{4.404184in}{2.477710in}}%
\pgfpathlineto{\pgfqpoint{4.405047in}{2.508206in}}%
\pgfpathlineto{\pgfqpoint{4.405913in}{2.484596in}}%
\pgfpathlineto{\pgfqpoint{4.406779in}{2.575378in}}%
\pgfpathlineto{\pgfqpoint{4.408510in}{2.484165in}}%
\pgfpathlineto{\pgfqpoint{4.410240in}{2.570244in}}%
\pgfpathlineto{\pgfqpoint{4.411104in}{2.566678in}}%
\pgfpathlineto{\pgfqpoint{4.412836in}{2.527451in}}%
\pgfpathlineto{\pgfqpoint{4.413701in}{2.482444in}}%
\pgfpathlineto{\pgfqpoint{4.414565in}{2.600925in}}%
\pgfpathlineto{\pgfqpoint{4.415430in}{2.579221in}}%
\pgfpathlineto{\pgfqpoint{4.417161in}{2.488409in}}%
\pgfpathlineto{\pgfqpoint{4.418026in}{2.557639in}}%
\pgfpathlineto{\pgfqpoint{4.418891in}{2.472053in}}%
\pgfpathlineto{\pgfqpoint{4.419756in}{2.574916in}}%
\pgfpathlineto{\pgfqpoint{4.421485in}{2.482998in}}%
\pgfpathlineto{\pgfqpoint{4.423211in}{2.532523in}}%
\pgfpathlineto{\pgfqpoint{4.424076in}{2.577315in}}%
\pgfpathlineto{\pgfqpoint{4.424939in}{2.548418in}}%
\pgfpathlineto{\pgfqpoint{4.425802in}{2.469656in}}%
\pgfpathlineto{\pgfqpoint{4.426667in}{2.496647in}}%
\pgfpathlineto{\pgfqpoint{4.427532in}{2.581987in}}%
\pgfpathlineto{\pgfqpoint{4.428398in}{2.429259in}}%
\pgfpathlineto{\pgfqpoint{4.429263in}{2.585430in}}%
\pgfpathlineto{\pgfqpoint{4.430129in}{2.557947in}}%
\pgfpathlineto{\pgfqpoint{4.430994in}{2.548110in}}%
\pgfpathlineto{\pgfqpoint{4.431859in}{2.624473in}}%
\pgfpathlineto{\pgfqpoint{4.432725in}{2.530586in}}%
\pgfpathlineto{\pgfqpoint{4.434454in}{2.624288in}}%
\pgfpathlineto{\pgfqpoint{4.435319in}{2.482967in}}%
\pgfpathlineto{\pgfqpoint{4.436184in}{2.615067in}}%
\pgfpathlineto{\pgfqpoint{4.437050in}{2.587337in}}%
\pgfpathlineto{\pgfqpoint{4.438780in}{2.554196in}}%
\pgfpathlineto{\pgfqpoint{4.439645in}{2.556471in}}%
\pgfpathlineto{\pgfqpoint{4.440508in}{2.506607in}}%
\pgfpathlineto{\pgfqpoint{4.441371in}{2.523331in}}%
\pgfpathlineto{\pgfqpoint{4.442235in}{2.490713in}}%
\pgfpathlineto{\pgfqpoint{4.443101in}{2.533475in}}%
\pgfpathlineto{\pgfqpoint{4.443964in}{2.490805in}}%
\pgfpathlineto{\pgfqpoint{4.445695in}{2.544482in}}%
\pgfpathlineto{\pgfqpoint{4.446561in}{2.501381in}}%
\pgfpathlineto{\pgfqpoint{4.448292in}{2.539687in}}%
\pgfpathlineto{\pgfqpoint{4.449156in}{2.534399in}}%
\pgfpathlineto{\pgfqpoint{4.450020in}{2.463629in}}%
\pgfpathlineto{\pgfqpoint{4.451751in}{2.532400in}}%
\pgfpathlineto{\pgfqpoint{4.452617in}{2.453669in}}%
\pgfpathlineto{\pgfqpoint{4.453484in}{2.537657in}}%
\pgfpathlineto{\pgfqpoint{4.455213in}{2.465597in}}%
\pgfpathlineto{\pgfqpoint{4.456077in}{2.547864in}}%
\pgfpathlineto{\pgfqpoint{4.456943in}{2.431196in}}%
\pgfpathlineto{\pgfqpoint{4.457807in}{2.544297in}}%
\pgfpathlineto{\pgfqpoint{4.458672in}{2.515584in}}%
\pgfpathlineto{\pgfqpoint{4.459537in}{2.486256in}}%
\pgfpathlineto{\pgfqpoint{4.461266in}{2.542914in}}%
\pgfpathlineto{\pgfqpoint{4.462131in}{2.531325in}}%
\pgfpathlineto{\pgfqpoint{4.462995in}{2.476419in}}%
\pgfpathlineto{\pgfqpoint{4.463858in}{2.542668in}}%
\pgfpathlineto{\pgfqpoint{4.464723in}{2.508452in}}%
\pgfpathlineto{\pgfqpoint{4.465588in}{2.539194in}}%
\pgfpathlineto{\pgfqpoint{4.466453in}{2.501843in}}%
\pgfpathlineto{\pgfqpoint{4.467318in}{2.594654in}}%
\pgfpathlineto{\pgfqpoint{4.468183in}{2.490007in}}%
\pgfpathlineto{\pgfqpoint{4.469047in}{2.503472in}}%
\pgfpathlineto{\pgfqpoint{4.469913in}{2.532431in}}%
\pgfpathlineto{\pgfqpoint{4.470777in}{2.497784in}}%
\pgfpathlineto{\pgfqpoint{4.471642in}{2.533722in}}%
\pgfpathlineto{\pgfqpoint{4.472507in}{2.501381in}}%
\pgfpathlineto{\pgfqpoint{4.473373in}{2.535505in}}%
\pgfpathlineto{\pgfqpoint{4.475100in}{2.510604in}}%
\pgfpathlineto{\pgfqpoint{4.476831in}{2.541223in}}%
\pgfpathlineto{\pgfqpoint{4.477695in}{2.538027in}}%
\pgfpathlineto{\pgfqpoint{4.478560in}{2.541593in}}%
\pgfpathlineto{\pgfqpoint{4.481157in}{2.490007in}}%
\pgfpathlineto{\pgfqpoint{4.482023in}{2.510297in}}%
\pgfpathlineto{\pgfqpoint{4.482888in}{2.578483in}}%
\pgfpathlineto{\pgfqpoint{4.483754in}{2.508760in}}%
\pgfpathlineto{\pgfqpoint{4.484620in}{2.560284in}}%
\pgfpathlineto{\pgfqpoint{4.485486in}{2.506884in}}%
\pgfpathlineto{\pgfqpoint{4.486350in}{2.606397in}}%
\pgfpathlineto{\pgfqpoint{4.487216in}{2.604306in}}%
\pgfpathlineto{\pgfqpoint{4.490672in}{2.490192in}}%
\pgfpathlineto{\pgfqpoint{4.491537in}{2.498307in}}%
\pgfpathlineto{\pgfqpoint{4.492402in}{2.550016in}}%
\pgfpathlineto{\pgfqpoint{4.494133in}{2.487424in}}%
\pgfpathlineto{\pgfqpoint{4.494998in}{2.552351in}}%
\pgfpathlineto{\pgfqpoint{4.495863in}{2.518289in}}%
\pgfpathlineto{\pgfqpoint{4.497591in}{2.545588in}}%
\pgfpathlineto{\pgfqpoint{4.498455in}{2.475557in}}%
\pgfpathlineto{\pgfqpoint{4.500183in}{2.540608in}}%
\pgfpathlineto{\pgfqpoint{4.501049in}{2.547494in}}%
\pgfpathlineto{\pgfqpoint{4.501914in}{2.508390in}}%
\pgfpathlineto{\pgfqpoint{4.503645in}{2.575224in}}%
\pgfpathlineto{\pgfqpoint{4.504511in}{2.526866in}}%
\pgfpathlineto{\pgfqpoint{4.505375in}{2.580143in}}%
\pgfpathlineto{\pgfqpoint{4.507970in}{2.522900in}}%
\pgfpathlineto{\pgfqpoint{4.508835in}{2.564187in}}%
\pgfpathlineto{\pgfqpoint{4.509700in}{2.522348in}}%
\pgfpathlineto{\pgfqpoint{4.510564in}{2.549277in}}%
\pgfpathlineto{\pgfqpoint{4.512294in}{2.509927in}}%
\pgfpathlineto{\pgfqpoint{4.513159in}{2.582049in}}%
\pgfpathlineto{\pgfqpoint{4.514025in}{2.496647in}}%
\pgfpathlineto{\pgfqpoint{4.514890in}{2.502058in}}%
\pgfpathlineto{\pgfqpoint{4.515753in}{2.521917in}}%
\pgfpathlineto{\pgfqpoint{4.517482in}{2.456743in}}%
\pgfpathlineto{\pgfqpoint{4.518346in}{2.509866in}}%
\pgfpathlineto{\pgfqpoint{4.519212in}{2.495172in}}%
\pgfpathlineto{\pgfqpoint{4.521806in}{2.568461in}}%
\pgfpathlineto{\pgfqpoint{4.522670in}{2.521917in}}%
\pgfpathlineto{\pgfqpoint{4.523535in}{2.532646in}}%
\pgfpathlineto{\pgfqpoint{4.524400in}{2.560284in}}%
\pgfpathlineto{\pgfqpoint{4.525263in}{2.549093in}}%
\pgfpathlineto{\pgfqpoint{4.526993in}{2.515708in}}%
\pgfpathlineto{\pgfqpoint{4.527857in}{2.546327in}}%
\pgfpathlineto{\pgfqpoint{4.528722in}{2.542576in}}%
\pgfpathlineto{\pgfqpoint{4.529586in}{2.481338in}}%
\pgfpathlineto{\pgfqpoint{4.530452in}{2.551399in}}%
\pgfpathlineto{\pgfqpoint{4.532179in}{2.480907in}}%
\pgfpathlineto{\pgfqpoint{4.533043in}{2.541991in}}%
\pgfpathlineto{\pgfqpoint{4.533909in}{2.468732in}}%
\pgfpathlineto{\pgfqpoint{4.534774in}{2.577254in}}%
\pgfpathlineto{\pgfqpoint{4.536505in}{2.502918in}}%
\pgfpathlineto{\pgfqpoint{4.537370in}{2.533845in}}%
\pgfpathlineto{\pgfqpoint{4.539100in}{2.485887in}}%
\pgfpathlineto{\pgfqpoint{4.539965in}{2.504916in}}%
\pgfpathlineto{\pgfqpoint{4.540828in}{2.586906in}}%
\pgfpathlineto{\pgfqpoint{4.541691in}{2.530586in}}%
\pgfpathlineto{\pgfqpoint{4.542555in}{2.574947in}}%
\pgfpathlineto{\pgfqpoint{4.544282in}{2.540362in}}%
\pgfpathlineto{\pgfqpoint{4.545145in}{2.499228in}}%
\pgfpathlineto{\pgfqpoint{4.546010in}{2.573993in}}%
\pgfpathlineto{\pgfqpoint{4.546874in}{2.570365in}}%
\pgfpathlineto{\pgfqpoint{4.547738in}{2.577682in}}%
\pgfpathlineto{\pgfqpoint{4.548603in}{2.564587in}}%
\pgfpathlineto{\pgfqpoint{4.549468in}{2.524991in}}%
\pgfpathlineto{\pgfqpoint{4.550331in}{2.553090in}}%
\pgfpathlineto{\pgfqpoint{4.552060in}{2.516352in}}%
\pgfpathlineto{\pgfqpoint{4.552925in}{2.589918in}}%
\pgfpathlineto{\pgfqpoint{4.553786in}{2.574485in}}%
\pgfpathlineto{\pgfqpoint{4.554650in}{2.516937in}}%
\pgfpathlineto{\pgfqpoint{4.555516in}{2.578298in}}%
\pgfpathlineto{\pgfqpoint{4.556379in}{2.576761in}}%
\pgfpathlineto{\pgfqpoint{4.557244in}{2.532308in}}%
\pgfpathlineto{\pgfqpoint{4.558109in}{2.547494in}}%
\pgfpathlineto{\pgfqpoint{4.558975in}{2.474544in}}%
\pgfpathlineto{\pgfqpoint{4.560705in}{2.547740in}}%
\pgfpathlineto{\pgfqpoint{4.562436in}{2.510174in}}%
\pgfpathlineto{\pgfqpoint{4.564162in}{2.544974in}}%
\pgfpathlineto{\pgfqpoint{4.565028in}{2.501997in}}%
\pgfpathlineto{\pgfqpoint{4.565894in}{2.508821in}}%
\pgfpathlineto{\pgfqpoint{4.566758in}{2.497140in}}%
\pgfpathlineto{\pgfqpoint{4.567622in}{2.561944in}}%
\pgfpathlineto{\pgfqpoint{4.568487in}{2.481153in}}%
\pgfpathlineto{\pgfqpoint{4.569351in}{2.575655in}}%
\pgfpathlineto{\pgfqpoint{4.570217in}{2.442665in}}%
\pgfpathlineto{\pgfqpoint{4.571948in}{2.555119in}}%
\pgfpathlineto{\pgfqpoint{4.573680in}{2.583955in}}%
\pgfpathlineto{\pgfqpoint{4.576275in}{2.467319in}}%
\pgfpathlineto{\pgfqpoint{4.578005in}{2.567907in}}%
\pgfpathlineto{\pgfqpoint{4.578869in}{2.564833in}}%
\pgfpathlineto{\pgfqpoint{4.579732in}{2.462216in}}%
\pgfpathlineto{\pgfqpoint{4.580597in}{2.611991in}}%
\pgfpathlineto{\pgfqpoint{4.582325in}{2.544913in}}%
\pgfpathlineto{\pgfqpoint{4.583189in}{2.761955in}}%
\pgfpathlineto{\pgfqpoint{4.584053in}{2.749965in}}%
\pgfpathlineto{\pgfqpoint{4.586645in}{2.866971in}}%
\pgfpathlineto{\pgfqpoint{4.587511in}{2.772346in}}%
\pgfpathlineto{\pgfqpoint{4.589242in}{2.853260in}}%
\pgfpathlineto{\pgfqpoint{4.590109in}{2.801336in}}%
\pgfpathlineto{\pgfqpoint{4.590975in}{2.809544in}}%
\pgfpathlineto{\pgfqpoint{4.592706in}{2.788823in}}%
\pgfpathlineto{\pgfqpoint{4.593571in}{2.823378in}}%
\pgfpathlineto{\pgfqpoint{4.594436in}{2.794172in}}%
\pgfpathlineto{\pgfqpoint{4.595300in}{2.816738in}}%
\pgfpathlineto{\pgfqpoint{4.597029in}{2.759802in}}%
\pgfpathlineto{\pgfqpoint{4.599625in}{2.814678in}}%
\pgfpathlineto{\pgfqpoint{4.600489in}{2.821779in}}%
\pgfpathlineto{\pgfqpoint{4.601354in}{2.771117in}}%
\pgfpathlineto{\pgfqpoint{4.602218in}{2.843115in}}%
\pgfpathlineto{\pgfqpoint{4.603082in}{2.795404in}}%
\pgfpathlineto{\pgfqpoint{4.603947in}{2.852983in}}%
\pgfpathlineto{\pgfqpoint{4.604812in}{2.751197in}}%
\pgfpathlineto{\pgfqpoint{4.605675in}{2.788394in}}%
\pgfpathlineto{\pgfqpoint{4.606541in}{2.785043in}}%
\pgfpathlineto{\pgfqpoint{4.607407in}{2.741973in}}%
\pgfpathlineto{\pgfqpoint{4.609137in}{2.801828in}}%
\pgfpathlineto{\pgfqpoint{4.610868in}{2.716088in}}%
\pgfpathlineto{\pgfqpoint{4.613459in}{2.818305in}}%
\pgfpathlineto{\pgfqpoint{4.615186in}{2.779294in}}%
\pgfpathlineto{\pgfqpoint{4.616052in}{2.796109in}}%
\pgfpathlineto{\pgfqpoint{4.616918in}{2.777326in}}%
\pgfpathlineto{\pgfqpoint{4.617784in}{2.789254in}}%
\pgfpathlineto{\pgfqpoint{4.618650in}{2.738068in}}%
\pgfpathlineto{\pgfqpoint{4.620380in}{2.774437in}}%
\pgfpathlineto{\pgfqpoint{4.621246in}{2.781323in}}%
\pgfpathlineto{\pgfqpoint{4.622110in}{2.820735in}}%
\pgfpathlineto{\pgfqpoint{4.622974in}{2.727923in}}%
\pgfpathlineto{\pgfqpoint{4.624705in}{2.782614in}}%
\pgfpathlineto{\pgfqpoint{4.626436in}{2.689833in}}%
\pgfpathlineto{\pgfqpoint{4.628165in}{2.856057in}}%
\pgfpathlineto{\pgfqpoint{4.629028in}{2.796263in}}%
\pgfpathlineto{\pgfqpoint{4.630756in}{2.815755in}}%
\pgfpathlineto{\pgfqpoint{4.631622in}{2.871151in}}%
\pgfpathlineto{\pgfqpoint{4.632489in}{2.810067in}}%
\pgfpathlineto{\pgfqpoint{4.633355in}{2.862114in}}%
\pgfpathlineto{\pgfqpoint{4.635081in}{2.778372in}}%
\pgfpathlineto{\pgfqpoint{4.636810in}{2.839734in}}%
\pgfpathlineto{\pgfqpoint{4.637675in}{2.778065in}}%
\pgfpathlineto{\pgfqpoint{4.638541in}{2.909949in}}%
\pgfpathlineto{\pgfqpoint{4.640272in}{2.786734in}}%
\pgfpathlineto{\pgfqpoint{4.641137in}{2.794911in}}%
\pgfpathlineto{\pgfqpoint{4.642003in}{2.791591in}}%
\pgfpathlineto{\pgfqpoint{4.642868in}{2.836167in}}%
\pgfpathlineto{\pgfqpoint{4.643733in}{2.825530in}}%
\pgfpathlineto{\pgfqpoint{4.644598in}{2.827252in}}%
\pgfpathlineto{\pgfqpoint{4.645460in}{2.843362in}}%
\pgfpathlineto{\pgfqpoint{4.646326in}{2.769580in}}%
\pgfpathlineto{\pgfqpoint{4.647191in}{2.801736in}}%
\pgfpathlineto{\pgfqpoint{4.648057in}{2.797495in}}%
\pgfpathlineto{\pgfqpoint{4.648923in}{2.814464in}}%
\pgfpathlineto{\pgfqpoint{4.649788in}{2.778619in}}%
\pgfpathlineto{\pgfqpoint{4.650652in}{2.813297in}}%
\pgfpathlineto{\pgfqpoint{4.652381in}{2.561084in}}%
\pgfpathlineto{\pgfqpoint{4.653247in}{2.511651in}}%
\pgfpathlineto{\pgfqpoint{4.654976in}{2.547373in}}%
\pgfpathlineto{\pgfqpoint{4.655843in}{2.539933in}}%
\pgfpathlineto{\pgfqpoint{4.656707in}{2.544544in}}%
\pgfpathlineto{\pgfqpoint{4.657572in}{2.563912in}}%
\pgfpathlineto{\pgfqpoint{4.659302in}{2.520626in}}%
\pgfpathlineto{\pgfqpoint{4.660168in}{2.622507in}}%
\pgfpathlineto{\pgfqpoint{4.661897in}{2.459080in}}%
\pgfpathlineto{\pgfqpoint{4.663625in}{2.520872in}}%
\pgfpathlineto{\pgfqpoint{4.664490in}{2.531448in}}%
\pgfpathlineto{\pgfqpoint{4.665354in}{2.530034in}}%
\pgfpathlineto{\pgfqpoint{4.666218in}{2.499015in}}%
\pgfpathlineto{\pgfqpoint{4.667082in}{2.544974in}}%
\pgfpathlineto{\pgfqpoint{4.667948in}{2.543438in}}%
\pgfpathlineto{\pgfqpoint{4.668812in}{2.521119in}}%
\pgfpathlineto{\pgfqpoint{4.669677in}{2.596129in}}%
\pgfpathlineto{\pgfqpoint{4.671407in}{2.487978in}}%
\pgfpathlineto{\pgfqpoint{4.672272in}{2.521611in}}%
\pgfpathlineto{\pgfqpoint{4.673137in}{2.473468in}}%
\pgfpathlineto{\pgfqpoint{4.674000in}{2.490315in}}%
\pgfpathlineto{\pgfqpoint{4.674866in}{2.602341in}}%
\pgfpathlineto{\pgfqpoint{4.675730in}{2.495757in}}%
\pgfpathlineto{\pgfqpoint{4.677460in}{2.579283in}}%
\pgfpathlineto{\pgfqpoint{4.678325in}{2.524131in}}%
\pgfpathlineto{\pgfqpoint{4.679190in}{2.555981in}}%
\pgfpathlineto{\pgfqpoint{4.680921in}{2.466521in}}%
\pgfpathlineto{\pgfqpoint{4.681784in}{2.513986in}}%
\pgfpathlineto{\pgfqpoint{4.682650in}{2.471439in}}%
\pgfpathlineto{\pgfqpoint{4.683516in}{2.563912in}}%
\pgfpathlineto{\pgfqpoint{4.684380in}{2.506363in}}%
\pgfpathlineto{\pgfqpoint{4.686110in}{2.555673in}}%
\pgfpathlineto{\pgfqpoint{4.687840in}{2.520503in}}%
\pgfpathlineto{\pgfqpoint{4.688705in}{2.553705in}}%
\pgfpathlineto{\pgfqpoint{4.689568in}{2.527389in}}%
\pgfpathlineto{\pgfqpoint{4.691297in}{2.627426in}}%
\pgfpathlineto{\pgfqpoint{4.693027in}{2.550139in}}%
\pgfpathlineto{\pgfqpoint{4.693889in}{2.618387in}}%
\pgfpathlineto{\pgfqpoint{4.695621in}{2.528742in}}%
\pgfpathlineto{\pgfqpoint{4.696486in}{2.549647in}}%
\pgfpathlineto{\pgfqpoint{4.697352in}{2.496524in}}%
\pgfpathlineto{\pgfqpoint{4.698218in}{2.562129in}}%
\pgfpathlineto{\pgfqpoint{4.699949in}{2.539994in}}%
\pgfpathlineto{\pgfqpoint{4.700812in}{2.542884in}}%
\pgfpathlineto{\pgfqpoint{4.701675in}{2.493481in}}%
\pgfpathlineto{\pgfqpoint{4.702540in}{2.530586in}}%
\pgfpathlineto{\pgfqpoint{4.704271in}{2.477525in}}%
\pgfpathlineto{\pgfqpoint{4.705999in}{2.528557in}}%
\pgfpathlineto{\pgfqpoint{4.706864in}{2.548846in}}%
\pgfpathlineto{\pgfqpoint{4.708594in}{2.481982in}}%
\pgfpathlineto{\pgfqpoint{4.709461in}{2.490744in}}%
\pgfpathlineto{\pgfqpoint{4.710325in}{2.555548in}}%
\pgfpathlineto{\pgfqpoint{4.711190in}{2.493941in}}%
\pgfpathlineto{\pgfqpoint{4.712056in}{2.525421in}}%
\pgfpathlineto{\pgfqpoint{4.714648in}{2.473035in}}%
\pgfpathlineto{\pgfqpoint{4.717245in}{2.545588in}}%
\pgfpathlineto{\pgfqpoint{4.718111in}{2.488961in}}%
\pgfpathlineto{\pgfqpoint{4.718976in}{2.524621in}}%
\pgfpathlineto{\pgfqpoint{4.719839in}{2.494556in}}%
\pgfpathlineto{\pgfqpoint{4.721569in}{2.539992in}}%
\pgfpathlineto{\pgfqpoint{4.722434in}{2.487485in}}%
\pgfpathlineto{\pgfqpoint{4.723300in}{2.574516in}}%
\pgfpathlineto{\pgfqpoint{4.724166in}{2.562681in}}%
\pgfpathlineto{\pgfqpoint{4.725894in}{2.515154in}}%
\pgfpathlineto{\pgfqpoint{4.726759in}{2.521240in}}%
\pgfpathlineto{\pgfqpoint{4.728487in}{2.458588in}}%
\pgfpathlineto{\pgfqpoint{4.729353in}{2.520134in}}%
\pgfpathlineto{\pgfqpoint{4.730215in}{2.486133in}}%
\pgfpathlineto{\pgfqpoint{4.731947in}{2.527574in}}%
\pgfpathlineto{\pgfqpoint{4.733678in}{2.563235in}}%
\pgfpathlineto{\pgfqpoint{4.734543in}{2.527389in}}%
\pgfpathlineto{\pgfqpoint{4.735409in}{2.572643in}}%
\pgfpathlineto{\pgfqpoint{4.737997in}{2.522594in}}%
\pgfpathlineto{\pgfqpoint{4.739729in}{2.498399in}}%
\pgfpathlineto{\pgfqpoint{4.740595in}{2.584078in}}%
\pgfpathlineto{\pgfqpoint{4.742327in}{2.517060in}}%
\pgfpathlineto{\pgfqpoint{4.743193in}{2.549154in}}%
\pgfpathlineto{\pgfqpoint{4.744060in}{2.508821in}}%
\pgfpathlineto{\pgfqpoint{4.745789in}{2.566249in}}%
\pgfpathlineto{\pgfqpoint{4.746655in}{2.451303in}}%
\pgfpathlineto{\pgfqpoint{4.748384in}{2.493327in}}%
\pgfpathlineto{\pgfqpoint{4.749246in}{2.482076in}}%
\pgfpathlineto{\pgfqpoint{4.750112in}{2.490130in}}%
\pgfpathlineto{\pgfqpoint{4.750977in}{2.561759in}}%
\pgfpathlineto{\pgfqpoint{4.751842in}{2.556010in}}%
\pgfpathlineto{\pgfqpoint{4.752708in}{2.504947in}}%
\pgfpathlineto{\pgfqpoint{4.753575in}{2.570059in}}%
\pgfpathlineto{\pgfqpoint{4.754440in}{2.558378in}}%
\pgfpathlineto{\pgfqpoint{4.756170in}{2.516814in}}%
\pgfpathlineto{\pgfqpoint{4.757901in}{2.575163in}}%
\pgfpathlineto{\pgfqpoint{4.758765in}{2.530740in}}%
\pgfpathlineto{\pgfqpoint{4.759630in}{2.567968in}}%
\pgfpathlineto{\pgfqpoint{4.760492in}{2.544051in}}%
\pgfpathlineto{\pgfqpoint{4.761358in}{2.575686in}}%
\pgfpathlineto{\pgfqpoint{4.762223in}{2.527328in}}%
\pgfpathlineto{\pgfqpoint{4.763088in}{2.556533in}}%
\pgfpathlineto{\pgfqpoint{4.763954in}{2.510420in}}%
\pgfpathlineto{\pgfqpoint{4.764818in}{2.606151in}}%
\pgfpathlineto{\pgfqpoint{4.765683in}{2.557149in}}%
\pgfpathlineto{\pgfqpoint{4.766549in}{2.560407in}}%
\pgfpathlineto{\pgfqpoint{4.767415in}{2.579283in}}%
\pgfpathlineto{\pgfqpoint{4.768280in}{2.517275in}}%
\pgfpathlineto{\pgfqpoint{4.769146in}{2.531448in}}%
\pgfpathlineto{\pgfqpoint{4.770012in}{2.517245in}}%
\pgfpathlineto{\pgfqpoint{4.770878in}{2.575532in}}%
\pgfpathlineto{\pgfqpoint{4.772610in}{2.525175in}}%
\pgfpathlineto{\pgfqpoint{4.773477in}{2.548171in}}%
\pgfpathlineto{\pgfqpoint{4.774342in}{2.491482in}}%
\pgfpathlineto{\pgfqpoint{4.775208in}{2.559361in}}%
\pgfpathlineto{\pgfqpoint{4.776073in}{2.542884in}}%
\pgfpathlineto{\pgfqpoint{4.776937in}{2.541408in}}%
\pgfpathlineto{\pgfqpoint{4.777802in}{2.525668in}}%
\pgfpathlineto{\pgfqpoint{4.778668in}{2.541162in}}%
\pgfpathlineto{\pgfqpoint{4.779534in}{2.532185in}}%
\pgfpathlineto{\pgfqpoint{4.780401in}{2.489699in}}%
\pgfpathlineto{\pgfqpoint{4.781267in}{2.535813in}}%
\pgfpathlineto{\pgfqpoint{4.782133in}{2.525976in}}%
\pgfpathlineto{\pgfqpoint{4.782998in}{2.489576in}}%
\pgfpathlineto{\pgfqpoint{4.783865in}{2.596006in}}%
\pgfpathlineto{\pgfqpoint{4.784730in}{2.484658in}}%
\pgfpathlineto{\pgfqpoint{4.786461in}{2.561821in}}%
\pgfpathlineto{\pgfqpoint{4.787327in}{2.556625in}}%
\pgfpathlineto{\pgfqpoint{4.789059in}{2.582418in}}%
\pgfpathlineto{\pgfqpoint{4.790791in}{2.508513in}}%
\pgfpathlineto{\pgfqpoint{4.791657in}{2.532677in}}%
\pgfpathlineto{\pgfqpoint{4.792520in}{2.459203in}}%
\pgfpathlineto{\pgfqpoint{4.794249in}{2.542668in}}%
\pgfpathlineto{\pgfqpoint{4.795115in}{2.454100in}}%
\pgfpathlineto{\pgfqpoint{4.796845in}{2.580789in}}%
\pgfpathlineto{\pgfqpoint{4.797709in}{2.486687in}}%
\pgfpathlineto{\pgfqpoint{4.798574in}{2.631728in}}%
\pgfpathlineto{\pgfqpoint{4.801166in}{2.531879in}}%
\pgfpathlineto{\pgfqpoint{4.802030in}{2.502181in}}%
\pgfpathlineto{\pgfqpoint{4.803757in}{2.550201in}}%
\pgfpathlineto{\pgfqpoint{4.804623in}{2.505317in}}%
\pgfpathlineto{\pgfqpoint{4.805486in}{2.517799in}}%
\pgfpathlineto{\pgfqpoint{4.806350in}{2.558963in}}%
\pgfpathlineto{\pgfqpoint{4.807214in}{2.505932in}}%
\pgfpathlineto{\pgfqpoint{4.808943in}{2.590965in}}%
\pgfpathlineto{\pgfqpoint{4.810674in}{2.506240in}}%
\pgfpathlineto{\pgfqpoint{4.813264in}{2.615436in}}%
\pgfpathlineto{\pgfqpoint{4.814993in}{2.558809in}}%
\pgfpathlineto{\pgfqpoint{4.815858in}{2.556718in}}%
\pgfpathlineto{\pgfqpoint{4.816722in}{2.491359in}}%
\pgfpathlineto{\pgfqpoint{4.817588in}{2.544482in}}%
\pgfpathlineto{\pgfqpoint{4.818454in}{2.486318in}}%
\pgfpathlineto{\pgfqpoint{4.820184in}{2.553275in}}%
\pgfpathlineto{\pgfqpoint{4.821048in}{2.519212in}}%
\pgfpathlineto{\pgfqpoint{4.821914in}{2.568830in}}%
\pgfpathlineto{\pgfqpoint{4.822781in}{2.539625in}}%
\pgfpathlineto{\pgfqpoint{4.823647in}{2.565757in}}%
\pgfpathlineto{\pgfqpoint{4.825379in}{2.466490in}}%
\pgfpathlineto{\pgfqpoint{4.826244in}{2.467934in}}%
\pgfpathlineto{\pgfqpoint{4.827110in}{2.531448in}}%
\pgfpathlineto{\pgfqpoint{4.827976in}{2.524008in}}%
\pgfpathlineto{\pgfqpoint{4.828842in}{2.502674in}}%
\pgfpathlineto{\pgfqpoint{4.829709in}{2.522163in}}%
\pgfpathlineto{\pgfqpoint{4.830573in}{2.484350in}}%
\pgfpathlineto{\pgfqpoint{4.831440in}{2.515646in}}%
\pgfpathlineto{\pgfqpoint{4.832305in}{2.510358in}}%
\pgfpathlineto{\pgfqpoint{4.833170in}{2.515769in}}%
\pgfpathlineto{\pgfqpoint{4.834897in}{2.485058in}}%
\pgfpathlineto{\pgfqpoint{4.835762in}{2.493573in}}%
\pgfpathlineto{\pgfqpoint{4.836629in}{2.486749in}}%
\pgfpathlineto{\pgfqpoint{4.838359in}{2.521057in}}%
\pgfpathlineto{\pgfqpoint{4.839223in}{2.521057in}}%
\pgfpathlineto{\pgfqpoint{4.840088in}{2.516200in}}%
\pgfpathlineto{\pgfqpoint{4.840953in}{2.545159in}}%
\pgfpathlineto{\pgfqpoint{4.841818in}{2.502089in}}%
\pgfpathlineto{\pgfqpoint{4.842682in}{2.548602in}}%
\pgfpathlineto{\pgfqpoint{4.842682in}{2.548602in}}%
\pgfusepath{stroke}%
\end{pgfscope}%
\begin{pgfscope}%
\pgfsetrectcap%
\pgfsetmiterjoin%
\pgfsetlinewidth{0.803000pt}%
\definecolor{currentstroke}{rgb}{0.000000,0.000000,0.000000}%
\pgfsetstrokecolor{currentstroke}%
\pgfsetdash{}{0pt}%
\pgfpathmoveto{\pgfqpoint{0.483776in}{2.351653in}}%
\pgfpathlineto{\pgfqpoint{0.483776in}{2.936535in}}%
\pgfusepath{stroke}%
\end{pgfscope}%
\begin{pgfscope}%
\pgfsetrectcap%
\pgfsetmiterjoin%
\pgfsetlinewidth{0.803000pt}%
\definecolor{currentstroke}{rgb}{0.000000,0.000000,0.000000}%
\pgfsetstrokecolor{currentstroke}%
\pgfsetdash{}{0pt}%
\pgfpathmoveto{\pgfqpoint{5.050249in}{2.351653in}}%
\pgfpathlineto{\pgfqpoint{5.050249in}{2.936535in}}%
\pgfusepath{stroke}%
\end{pgfscope}%
\begin{pgfscope}%
\pgfsetrectcap%
\pgfsetmiterjoin%
\pgfsetlinewidth{0.803000pt}%
\definecolor{currentstroke}{rgb}{0.000000,0.000000,0.000000}%
\pgfsetstrokecolor{currentstroke}%
\pgfsetdash{}{0pt}%
\pgfpathmoveto{\pgfqpoint{0.483776in}{2.351653in}}%
\pgfpathlineto{\pgfqpoint{5.050249in}{2.351653in}}%
\pgfusepath{stroke}%
\end{pgfscope}%
\begin{pgfscope}%
\pgfsetrectcap%
\pgfsetmiterjoin%
\pgfsetlinewidth{0.803000pt}%
\definecolor{currentstroke}{rgb}{0.000000,0.000000,0.000000}%
\pgfsetstrokecolor{currentstroke}%
\pgfsetdash{}{0pt}%
\pgfpathmoveto{\pgfqpoint{0.483776in}{2.936535in}}%
\pgfpathlineto{\pgfqpoint{5.050249in}{2.936535in}}%
\pgfusepath{stroke}%
\end{pgfscope}%
\begin{pgfscope}%
\pgfsetbuttcap%
\pgfsetmiterjoin%
\definecolor{currentfill}{rgb}{1.000000,1.000000,1.000000}%
\pgfsetfillcolor{currentfill}%
\pgfsetlinewidth{0.000000pt}%
\definecolor{currentstroke}{rgb}{0.000000,0.000000,0.000000}%
\pgfsetstrokecolor{currentstroke}%
\pgfsetstrokeopacity{0.000000}%
\pgfsetdash{}{0pt}%
\pgfpathmoveto{\pgfqpoint{0.483776in}{1.444834in}}%
\pgfpathlineto{\pgfqpoint{5.050249in}{1.444834in}}%
\pgfpathlineto{\pgfqpoint{5.050249in}{2.029715in}}%
\pgfpathlineto{\pgfqpoint{0.483776in}{2.029715in}}%
\pgfpathlineto{\pgfqpoint{0.483776in}{1.444834in}}%
\pgfpathclose%
\pgfusepath{fill}%
\end{pgfscope}%
\begin{pgfscope}%
\pgfsetbuttcap%
\pgfsetroundjoin%
\definecolor{currentfill}{rgb}{0.000000,0.000000,0.000000}%
\pgfsetfillcolor{currentfill}%
\pgfsetlinewidth{0.803000pt}%
\definecolor{currentstroke}{rgb}{0.000000,0.000000,0.000000}%
\pgfsetstrokecolor{currentstroke}%
\pgfsetdash{}{0pt}%
\pgfsys@defobject{currentmarker}{\pgfqpoint{0.000000in}{-0.048611in}}{\pgfqpoint{0.000000in}{0.000000in}}{%
\pgfpathmoveto{\pgfqpoint{0.000000in}{0.000000in}}%
\pgfpathlineto{\pgfqpoint{0.000000in}{-0.048611in}}%
\pgfusepath{stroke,fill}%
}%
\begin{pgfscope}%
\pgfsys@transformshift{0.691021in}{1.444834in}%
\pgfsys@useobject{currentmarker}{}%
\end{pgfscope}%
\end{pgfscope}%
\begin{pgfscope}%
\pgfsetbuttcap%
\pgfsetroundjoin%
\definecolor{currentfill}{rgb}{0.000000,0.000000,0.000000}%
\pgfsetfillcolor{currentfill}%
\pgfsetlinewidth{0.803000pt}%
\definecolor{currentstroke}{rgb}{0.000000,0.000000,0.000000}%
\pgfsetstrokecolor{currentstroke}%
\pgfsetdash{}{0pt}%
\pgfsys@defobject{currentmarker}{\pgfqpoint{0.000000in}{-0.048611in}}{\pgfqpoint{0.000000in}{0.000000in}}{%
\pgfpathmoveto{\pgfqpoint{0.000000in}{0.000000in}}%
\pgfpathlineto{\pgfqpoint{0.000000in}{-0.048611in}}%
\pgfusepath{stroke,fill}%
}%
\begin{pgfscope}%
\pgfsys@transformshift{1.210067in}{1.444834in}%
\pgfsys@useobject{currentmarker}{}%
\end{pgfscope}%
\end{pgfscope}%
\begin{pgfscope}%
\pgfsetbuttcap%
\pgfsetroundjoin%
\definecolor{currentfill}{rgb}{0.000000,0.000000,0.000000}%
\pgfsetfillcolor{currentfill}%
\pgfsetlinewidth{0.803000pt}%
\definecolor{currentstroke}{rgb}{0.000000,0.000000,0.000000}%
\pgfsetstrokecolor{currentstroke}%
\pgfsetdash{}{0pt}%
\pgfsys@defobject{currentmarker}{\pgfqpoint{0.000000in}{-0.048611in}}{\pgfqpoint{0.000000in}{0.000000in}}{%
\pgfpathmoveto{\pgfqpoint{0.000000in}{0.000000in}}%
\pgfpathlineto{\pgfqpoint{0.000000in}{-0.048611in}}%
\pgfusepath{stroke,fill}%
}%
\begin{pgfscope}%
\pgfsys@transformshift{1.729114in}{1.444834in}%
\pgfsys@useobject{currentmarker}{}%
\end{pgfscope}%
\end{pgfscope}%
\begin{pgfscope}%
\pgfsetbuttcap%
\pgfsetroundjoin%
\definecolor{currentfill}{rgb}{0.000000,0.000000,0.000000}%
\pgfsetfillcolor{currentfill}%
\pgfsetlinewidth{0.803000pt}%
\definecolor{currentstroke}{rgb}{0.000000,0.000000,0.000000}%
\pgfsetstrokecolor{currentstroke}%
\pgfsetdash{}{0pt}%
\pgfsys@defobject{currentmarker}{\pgfqpoint{0.000000in}{-0.048611in}}{\pgfqpoint{0.000000in}{0.000000in}}{%
\pgfpathmoveto{\pgfqpoint{0.000000in}{0.000000in}}%
\pgfpathlineto{\pgfqpoint{0.000000in}{-0.048611in}}%
\pgfusepath{stroke,fill}%
}%
\begin{pgfscope}%
\pgfsys@transformshift{2.248160in}{1.444834in}%
\pgfsys@useobject{currentmarker}{}%
\end{pgfscope}%
\end{pgfscope}%
\begin{pgfscope}%
\pgfsetbuttcap%
\pgfsetroundjoin%
\definecolor{currentfill}{rgb}{0.000000,0.000000,0.000000}%
\pgfsetfillcolor{currentfill}%
\pgfsetlinewidth{0.803000pt}%
\definecolor{currentstroke}{rgb}{0.000000,0.000000,0.000000}%
\pgfsetstrokecolor{currentstroke}%
\pgfsetdash{}{0pt}%
\pgfsys@defobject{currentmarker}{\pgfqpoint{0.000000in}{-0.048611in}}{\pgfqpoint{0.000000in}{0.000000in}}{%
\pgfpathmoveto{\pgfqpoint{0.000000in}{0.000000in}}%
\pgfpathlineto{\pgfqpoint{0.000000in}{-0.048611in}}%
\pgfusepath{stroke,fill}%
}%
\begin{pgfscope}%
\pgfsys@transformshift{2.767206in}{1.444834in}%
\pgfsys@useobject{currentmarker}{}%
\end{pgfscope}%
\end{pgfscope}%
\begin{pgfscope}%
\pgfsetbuttcap%
\pgfsetroundjoin%
\definecolor{currentfill}{rgb}{0.000000,0.000000,0.000000}%
\pgfsetfillcolor{currentfill}%
\pgfsetlinewidth{0.803000pt}%
\definecolor{currentstroke}{rgb}{0.000000,0.000000,0.000000}%
\pgfsetstrokecolor{currentstroke}%
\pgfsetdash{}{0pt}%
\pgfsys@defobject{currentmarker}{\pgfqpoint{0.000000in}{-0.048611in}}{\pgfqpoint{0.000000in}{0.000000in}}{%
\pgfpathmoveto{\pgfqpoint{0.000000in}{0.000000in}}%
\pgfpathlineto{\pgfqpoint{0.000000in}{-0.048611in}}%
\pgfusepath{stroke,fill}%
}%
\begin{pgfscope}%
\pgfsys@transformshift{3.286252in}{1.444834in}%
\pgfsys@useobject{currentmarker}{}%
\end{pgfscope}%
\end{pgfscope}%
\begin{pgfscope}%
\pgfsetbuttcap%
\pgfsetroundjoin%
\definecolor{currentfill}{rgb}{0.000000,0.000000,0.000000}%
\pgfsetfillcolor{currentfill}%
\pgfsetlinewidth{0.803000pt}%
\definecolor{currentstroke}{rgb}{0.000000,0.000000,0.000000}%
\pgfsetstrokecolor{currentstroke}%
\pgfsetdash{}{0pt}%
\pgfsys@defobject{currentmarker}{\pgfqpoint{0.000000in}{-0.048611in}}{\pgfqpoint{0.000000in}{0.000000in}}{%
\pgfpathmoveto{\pgfqpoint{0.000000in}{0.000000in}}%
\pgfpathlineto{\pgfqpoint{0.000000in}{-0.048611in}}%
\pgfusepath{stroke,fill}%
}%
\begin{pgfscope}%
\pgfsys@transformshift{3.805298in}{1.444834in}%
\pgfsys@useobject{currentmarker}{}%
\end{pgfscope}%
\end{pgfscope}%
\begin{pgfscope}%
\pgfsetbuttcap%
\pgfsetroundjoin%
\definecolor{currentfill}{rgb}{0.000000,0.000000,0.000000}%
\pgfsetfillcolor{currentfill}%
\pgfsetlinewidth{0.803000pt}%
\definecolor{currentstroke}{rgb}{0.000000,0.000000,0.000000}%
\pgfsetstrokecolor{currentstroke}%
\pgfsetdash{}{0pt}%
\pgfsys@defobject{currentmarker}{\pgfqpoint{0.000000in}{-0.048611in}}{\pgfqpoint{0.000000in}{0.000000in}}{%
\pgfpathmoveto{\pgfqpoint{0.000000in}{0.000000in}}%
\pgfpathlineto{\pgfqpoint{0.000000in}{-0.048611in}}%
\pgfusepath{stroke,fill}%
}%
\begin{pgfscope}%
\pgfsys@transformshift{4.324344in}{1.444834in}%
\pgfsys@useobject{currentmarker}{}%
\end{pgfscope}%
\end{pgfscope}%
\begin{pgfscope}%
\pgfsetbuttcap%
\pgfsetroundjoin%
\definecolor{currentfill}{rgb}{0.000000,0.000000,0.000000}%
\pgfsetfillcolor{currentfill}%
\pgfsetlinewidth{0.803000pt}%
\definecolor{currentstroke}{rgb}{0.000000,0.000000,0.000000}%
\pgfsetstrokecolor{currentstroke}%
\pgfsetdash{}{0pt}%
\pgfsys@defobject{currentmarker}{\pgfqpoint{0.000000in}{-0.048611in}}{\pgfqpoint{0.000000in}{0.000000in}}{%
\pgfpathmoveto{\pgfqpoint{0.000000in}{0.000000in}}%
\pgfpathlineto{\pgfqpoint{0.000000in}{-0.048611in}}%
\pgfusepath{stroke,fill}%
}%
\begin{pgfscope}%
\pgfsys@transformshift{4.843390in}{1.444834in}%
\pgfsys@useobject{currentmarker}{}%
\end{pgfscope}%
\end{pgfscope}%
\begin{pgfscope}%
\pgfsetbuttcap%
\pgfsetroundjoin%
\definecolor{currentfill}{rgb}{0.000000,0.000000,0.000000}%
\pgfsetfillcolor{currentfill}%
\pgfsetlinewidth{0.803000pt}%
\definecolor{currentstroke}{rgb}{0.000000,0.000000,0.000000}%
\pgfsetstrokecolor{currentstroke}%
\pgfsetdash{}{0pt}%
\pgfsys@defobject{currentmarker}{\pgfqpoint{-0.048611in}{0.000000in}}{\pgfqpoint{-0.000000in}{0.000000in}}{%
\pgfpathmoveto{\pgfqpoint{-0.000000in}{0.000000in}}%
\pgfpathlineto{\pgfqpoint{-0.048611in}{0.000000in}}%
\pgfusepath{stroke,fill}%
}%
\begin{pgfscope}%
\pgfsys@transformshift{0.483776in}{1.626011in}%
\pgfsys@useobject{currentmarker}{}%
\end{pgfscope}%
\end{pgfscope}%
\begin{pgfscope}%
\definecolor{textcolor}{rgb}{0.000000,0.000000,0.000000}%
\pgfsetstrokecolor{textcolor}%
\pgfsetfillcolor{textcolor}%
\pgftext[x=0.327525in, y=1.587455in, left, base]{\color{textcolor}\rmfamily\fontsize{8.000000}{9.600000}\selectfont \(\displaystyle {0}\)}%
\end{pgfscope}%
\begin{pgfscope}%
\pgfsetbuttcap%
\pgfsetroundjoin%
\definecolor{currentfill}{rgb}{0.000000,0.000000,0.000000}%
\pgfsetfillcolor{currentfill}%
\pgfsetlinewidth{0.803000pt}%
\definecolor{currentstroke}{rgb}{0.000000,0.000000,0.000000}%
\pgfsetstrokecolor{currentstroke}%
\pgfsetdash{}{0pt}%
\pgfsys@defobject{currentmarker}{\pgfqpoint{-0.048611in}{0.000000in}}{\pgfqpoint{-0.000000in}{0.000000in}}{%
\pgfpathmoveto{\pgfqpoint{-0.000000in}{0.000000in}}%
\pgfpathlineto{\pgfqpoint{-0.048611in}{0.000000in}}%
\pgfusepath{stroke,fill}%
}%
\begin{pgfscope}%
\pgfsys@transformshift{0.483776in}{1.833186in}%
\pgfsys@useobject{currentmarker}{}%
\end{pgfscope}%
\end{pgfscope}%
\begin{pgfscope}%
\definecolor{textcolor}{rgb}{0.000000,0.000000,0.000000}%
\pgfsetstrokecolor{textcolor}%
\pgfsetfillcolor{textcolor}%
\pgftext[x=0.327525in, y=1.794630in, left, base]{\color{textcolor}\rmfamily\fontsize{8.000000}{9.600000}\selectfont \(\displaystyle {5}\)}%
\end{pgfscope}%
\begin{pgfscope}%
\definecolor{textcolor}{rgb}{0.000000,0.000000,0.000000}%
\pgfsetstrokecolor{textcolor}%
\pgfsetfillcolor{textcolor}%
\pgftext[x=0.271969in,y=1.737274in,,bottom,rotate=90.000000]{\color{textcolor}\rmfamily\fontsize{10.000000}{12.000000}\selectfont Voltage deviation in V}%
\end{pgfscope}%
\begin{pgfscope}%
\definecolor{textcolor}{rgb}{0.000000,0.000000,0.000000}%
\pgfsetstrokecolor{textcolor}%
\pgfsetfillcolor{textcolor}%
\pgftext[x=0.483776in,y=2.071382in,left,base]{\color{textcolor}\rmfamily\fontsize{8.000000}{9.600000}\selectfont \(\displaystyle \times{10^{\ensuremath{-}6}}{}\)}%
\end{pgfscope}%
\begin{pgfscope}%
\pgfpathrectangle{\pgfqpoint{0.483776in}{1.444834in}}{\pgfqpoint{4.566474in}{0.584881in}}%
\pgfusepath{clip}%
\pgfsetrectcap%
\pgfsetroundjoin%
\pgfsetlinewidth{0.501875pt}%
\definecolor{currentstroke}{rgb}{0.000000,0.419608,0.643137}%
\pgfsetstrokecolor{currentstroke}%
\pgfsetstrokeopacity{0.700000}%
\pgfsetdash{}{0pt}%
\pgfpathmoveto{\pgfqpoint{0.691343in}{1.596619in}}%
\pgfpathlineto{\pgfqpoint{0.692205in}{1.591513in}}%
\pgfpathlineto{\pgfqpoint{0.693071in}{1.636512in}}%
\pgfpathlineto{\pgfqpoint{0.693935in}{1.590000in}}%
\pgfpathlineto{\pgfqpoint{0.694800in}{1.622502in}}%
\pgfpathlineto{\pgfqpoint{0.696532in}{1.564503in}}%
\pgfpathlineto{\pgfqpoint{0.697397in}{1.619059in}}%
\pgfpathlineto{\pgfqpoint{0.698263in}{1.610629in}}%
\pgfpathlineto{\pgfqpoint{0.699128in}{1.631764in}}%
\pgfpathlineto{\pgfqpoint{0.699993in}{1.620010in}}%
\pgfpathlineto{\pgfqpoint{0.700859in}{1.589230in}}%
\pgfpathlineto{\pgfqpoint{0.701725in}{1.635267in}}%
\pgfpathlineto{\pgfqpoint{0.702589in}{1.632299in}}%
\pgfpathlineto{\pgfqpoint{0.703453in}{1.560261in}}%
\pgfpathlineto{\pgfqpoint{0.705185in}{1.630637in}}%
\pgfpathlineto{\pgfqpoint{0.706051in}{1.606446in}}%
\pgfpathlineto{\pgfqpoint{0.706916in}{1.633962in}}%
\pgfpathlineto{\pgfqpoint{0.707780in}{1.563734in}}%
\pgfpathlineto{\pgfqpoint{0.709512in}{1.641264in}}%
\pgfpathlineto{\pgfqpoint{0.710377in}{1.594068in}}%
\pgfpathlineto{\pgfqpoint{0.711241in}{1.638055in}}%
\pgfpathlineto{\pgfqpoint{0.713837in}{1.560053in}}%
\pgfpathlineto{\pgfqpoint{0.715567in}{1.651947in}}%
\pgfpathlineto{\pgfqpoint{0.717297in}{1.619653in}}%
\pgfpathlineto{\pgfqpoint{0.718163in}{1.549188in}}%
\pgfpathlineto{\pgfqpoint{0.719030in}{1.575011in}}%
\pgfpathlineto{\pgfqpoint{0.719895in}{1.559696in}}%
\pgfpathlineto{\pgfqpoint{0.721627in}{1.606059in}}%
\pgfpathlineto{\pgfqpoint{0.722492in}{1.557498in}}%
\pgfpathlineto{\pgfqpoint{0.724219in}{1.598697in}}%
\pgfpathlineto{\pgfqpoint{0.725084in}{1.538948in}}%
\pgfpathlineto{\pgfqpoint{0.727676in}{1.610867in}}%
\pgfpathlineto{\pgfqpoint{0.728541in}{1.538208in}}%
\pgfpathlineto{\pgfqpoint{0.731135in}{1.630161in}}%
\pgfpathlineto{\pgfqpoint{0.731999in}{1.530756in}}%
\pgfpathlineto{\pgfqpoint{0.733731in}{1.628023in}}%
\pgfpathlineto{\pgfqpoint{0.734597in}{1.605970in}}%
\pgfpathlineto{\pgfqpoint{0.735460in}{1.549128in}}%
\pgfpathlineto{\pgfqpoint{0.736325in}{1.631645in}}%
\pgfpathlineto{\pgfqpoint{0.737190in}{1.582195in}}%
\pgfpathlineto{\pgfqpoint{0.738920in}{1.627845in}}%
\pgfpathlineto{\pgfqpoint{0.739784in}{1.595611in}}%
\pgfpathlineto{\pgfqpoint{0.740649in}{1.605643in}}%
\pgfpathlineto{\pgfqpoint{0.742381in}{1.518973in}}%
\pgfpathlineto{\pgfqpoint{0.744977in}{1.639186in}}%
\pgfpathlineto{\pgfqpoint{0.746705in}{1.568661in}}%
\pgfpathlineto{\pgfqpoint{0.748435in}{1.669520in}}%
\pgfpathlineto{\pgfqpoint{0.750163in}{1.617579in}}%
\pgfpathlineto{\pgfqpoint{0.751028in}{1.624822in}}%
\pgfpathlineto{\pgfqpoint{0.751891in}{1.524970in}}%
\pgfpathlineto{\pgfqpoint{0.752757in}{1.642391in}}%
\pgfpathlineto{\pgfqpoint{0.753622in}{1.534824in}}%
\pgfpathlineto{\pgfqpoint{0.754488in}{1.630578in}}%
\pgfpathlineto{\pgfqpoint{0.755354in}{1.628916in}}%
\pgfpathlineto{\pgfqpoint{0.756219in}{1.610454in}}%
\pgfpathlineto{\pgfqpoint{0.757084in}{1.687092in}}%
\pgfpathlineto{\pgfqpoint{0.759679in}{1.594722in}}%
\pgfpathlineto{\pgfqpoint{0.761407in}{1.632478in}}%
\pgfpathlineto{\pgfqpoint{0.762269in}{1.608733in}}%
\pgfpathlineto{\pgfqpoint{0.763135in}{1.634378in}}%
\pgfpathlineto{\pgfqpoint{0.764000in}{1.784806in}}%
\pgfpathlineto{\pgfqpoint{0.766596in}{1.673320in}}%
\pgfpathlineto{\pgfqpoint{0.767462in}{1.752453in}}%
\pgfpathlineto{\pgfqpoint{0.768326in}{1.742183in}}%
\pgfpathlineto{\pgfqpoint{0.769191in}{1.752869in}}%
\pgfpathlineto{\pgfqpoint{0.771788in}{1.592346in}}%
\pgfpathlineto{\pgfqpoint{0.772652in}{1.682965in}}%
\pgfpathlineto{\pgfqpoint{0.774383in}{1.598816in}}%
\pgfpathlineto{\pgfqpoint{0.775249in}{1.627607in}}%
\pgfpathlineto{\pgfqpoint{0.776979in}{1.584330in}}%
\pgfpathlineto{\pgfqpoint{0.777845in}{1.590208in}}%
\pgfpathlineto{\pgfqpoint{0.779575in}{1.572963in}}%
\pgfpathlineto{\pgfqpoint{0.780439in}{1.634553in}}%
\pgfpathlineto{\pgfqpoint{0.781305in}{1.591454in}}%
\pgfpathlineto{\pgfqpoint{0.782170in}{1.593116in}}%
\pgfpathlineto{\pgfqpoint{0.783036in}{1.580117in}}%
\pgfpathlineto{\pgfqpoint{0.783901in}{1.581600in}}%
\pgfpathlineto{\pgfqpoint{0.784766in}{1.593592in}}%
\pgfpathlineto{\pgfqpoint{0.785630in}{1.530012in}}%
\pgfpathlineto{\pgfqpoint{0.786494in}{1.608848in}}%
\pgfpathlineto{\pgfqpoint{0.787360in}{1.607245in}}%
\pgfpathlineto{\pgfqpoint{0.788225in}{1.537137in}}%
\pgfpathlineto{\pgfqpoint{0.789089in}{1.538115in}}%
\pgfpathlineto{\pgfqpoint{0.790816in}{1.602259in}}%
\pgfpathlineto{\pgfqpoint{0.791680in}{1.600210in}}%
\pgfpathlineto{\pgfqpoint{0.793412in}{1.593949in}}%
\pgfpathlineto{\pgfqpoint{0.795138in}{1.575725in}}%
\pgfpathlineto{\pgfqpoint{0.796004in}{1.655747in}}%
\pgfpathlineto{\pgfqpoint{0.798599in}{1.566879in}}%
\pgfpathlineto{\pgfqpoint{0.799464in}{1.545923in}}%
\pgfpathlineto{\pgfqpoint{0.801192in}{1.607840in}}%
\pgfpathlineto{\pgfqpoint{0.802924in}{1.552218in}}%
\pgfpathlineto{\pgfqpoint{0.803787in}{1.549902in}}%
\pgfpathlineto{\pgfqpoint{0.804652in}{1.567474in}}%
\pgfpathlineto{\pgfqpoint{0.805515in}{1.508703in}}%
\pgfpathlineto{\pgfqpoint{0.806378in}{1.522178in}}%
\pgfpathlineto{\pgfqpoint{0.807243in}{1.582017in}}%
\pgfpathlineto{\pgfqpoint{0.808107in}{1.561239in}}%
\pgfpathlineto{\pgfqpoint{0.808973in}{1.571687in}}%
\pgfpathlineto{\pgfqpoint{0.809837in}{1.500333in}}%
\pgfpathlineto{\pgfqpoint{0.811567in}{1.570025in}}%
\pgfpathlineto{\pgfqpoint{0.812431in}{1.578038in}}%
\pgfpathlineto{\pgfqpoint{0.813297in}{1.571449in}}%
\pgfpathlineto{\pgfqpoint{0.814161in}{1.632950in}}%
\pgfpathlineto{\pgfqpoint{0.815026in}{1.547496in}}%
\pgfpathlineto{\pgfqpoint{0.815891in}{1.566166in}}%
\pgfpathlineto{\pgfqpoint{0.817620in}{1.605167in}}%
\pgfpathlineto{\pgfqpoint{0.818484in}{1.600419in}}%
\pgfpathlineto{\pgfqpoint{0.819349in}{1.562306in}}%
\pgfpathlineto{\pgfqpoint{0.820213in}{1.627696in}}%
\pgfpathlineto{\pgfqpoint{0.821078in}{1.621137in}}%
\pgfpathlineto{\pgfqpoint{0.822807in}{1.664887in}}%
\pgfpathlineto{\pgfqpoint{0.823671in}{1.600538in}}%
\pgfpathlineto{\pgfqpoint{0.824536in}{1.619267in}}%
\pgfpathlineto{\pgfqpoint{0.826268in}{1.602616in}}%
\pgfpathlineto{\pgfqpoint{0.827999in}{1.618170in}}%
\pgfpathlineto{\pgfqpoint{0.828864in}{1.556729in}}%
\pgfpathlineto{\pgfqpoint{0.830596in}{1.696054in}}%
\pgfpathlineto{\pgfqpoint{0.831462in}{1.617991in}}%
\pgfpathlineto{\pgfqpoint{0.832328in}{1.622859in}}%
\pgfpathlineto{\pgfqpoint{0.833193in}{1.666196in}}%
\pgfpathlineto{\pgfqpoint{0.834058in}{1.655182in}}%
\pgfpathlineto{\pgfqpoint{0.834922in}{1.663879in}}%
\pgfpathlineto{\pgfqpoint{0.836650in}{1.590744in}}%
\pgfpathlineto{\pgfqpoint{0.837516in}{1.554352in}}%
\pgfpathlineto{\pgfqpoint{0.838382in}{1.571568in}}%
\pgfpathlineto{\pgfqpoint{0.839248in}{1.531764in}}%
\pgfpathlineto{\pgfqpoint{0.840980in}{1.572992in}}%
\pgfpathlineto{\pgfqpoint{0.842710in}{1.547228in}}%
\pgfpathlineto{\pgfqpoint{0.844440in}{1.605107in}}%
\pgfpathlineto{\pgfqpoint{0.845305in}{1.600300in}}%
\pgfpathlineto{\pgfqpoint{0.846169in}{1.545566in}}%
\pgfpathlineto{\pgfqpoint{0.847897in}{1.599292in}}%
\pgfpathlineto{\pgfqpoint{0.849627in}{1.580176in}}%
\pgfpathlineto{\pgfqpoint{0.850491in}{1.575368in}}%
\pgfpathlineto{\pgfqpoint{0.851354in}{1.557498in}}%
\pgfpathlineto{\pgfqpoint{0.853081in}{1.600359in}}%
\pgfpathlineto{\pgfqpoint{0.853945in}{1.542361in}}%
\pgfpathlineto{\pgfqpoint{0.854810in}{1.630991in}}%
\pgfpathlineto{\pgfqpoint{0.855672in}{1.627815in}}%
\pgfpathlineto{\pgfqpoint{0.856537in}{1.595254in}}%
\pgfpathlineto{\pgfqpoint{0.857402in}{1.657231in}}%
\pgfpathlineto{\pgfqpoint{0.858267in}{1.596767in}}%
\pgfpathlineto{\pgfqpoint{0.859131in}{1.636453in}}%
\pgfpathlineto{\pgfqpoint{0.860864in}{1.577741in}}%
\pgfpathlineto{\pgfqpoint{0.861730in}{1.630693in}}%
\pgfpathlineto{\pgfqpoint{0.864329in}{1.560644in}}%
\pgfpathlineto{\pgfqpoint{0.866062in}{1.576019in}}%
\pgfpathlineto{\pgfqpoint{0.866929in}{1.597983in}}%
\pgfpathlineto{\pgfqpoint{0.867794in}{1.565333in}}%
\pgfpathlineto{\pgfqpoint{0.869523in}{1.610391in}}%
\pgfpathlineto{\pgfqpoint{0.870388in}{1.588427in}}%
\pgfpathlineto{\pgfqpoint{0.871253in}{1.636393in}}%
\pgfpathlineto{\pgfqpoint{0.872119in}{1.630693in}}%
\pgfpathlineto{\pgfqpoint{0.873848in}{1.621550in}}%
\pgfpathlineto{\pgfqpoint{0.874712in}{1.570438in}}%
\pgfpathlineto{\pgfqpoint{0.876442in}{1.691008in}}%
\pgfpathlineto{\pgfqpoint{0.877309in}{1.664946in}}%
\pgfpathlineto{\pgfqpoint{0.878174in}{1.667144in}}%
\pgfpathlineto{\pgfqpoint{0.879903in}{1.651709in}}%
\pgfpathlineto{\pgfqpoint{0.881633in}{1.694749in}}%
\pgfpathlineto{\pgfqpoint{0.882498in}{1.646187in}}%
\pgfpathlineto{\pgfqpoint{0.883364in}{1.713150in}}%
\pgfpathlineto{\pgfqpoint{0.884229in}{1.703356in}}%
\pgfpathlineto{\pgfqpoint{0.885096in}{1.669282in}}%
\pgfpathlineto{\pgfqpoint{0.885960in}{1.702999in}}%
\pgfpathlineto{\pgfqpoint{0.886826in}{1.628142in}}%
\pgfpathlineto{\pgfqpoint{0.887691in}{1.653490in}}%
\pgfpathlineto{\pgfqpoint{0.888555in}{1.717129in}}%
\pgfpathlineto{\pgfqpoint{0.892011in}{1.653193in}}%
\pgfpathlineto{\pgfqpoint{0.892875in}{1.734047in}}%
\pgfpathlineto{\pgfqpoint{0.893741in}{1.654557in}}%
\pgfpathlineto{\pgfqpoint{0.894605in}{1.695102in}}%
\pgfpathlineto{\pgfqpoint{0.897202in}{1.588427in}}%
\pgfpathlineto{\pgfqpoint{0.898067in}{1.593235in}}%
\pgfpathlineto{\pgfqpoint{0.899795in}{1.651293in}}%
\pgfpathlineto{\pgfqpoint{0.900660in}{1.598310in}}%
\pgfpathlineto{\pgfqpoint{0.901525in}{1.619356in}}%
\pgfpathlineto{\pgfqpoint{0.902390in}{1.677295in}}%
\pgfpathlineto{\pgfqpoint{0.903254in}{1.612380in}}%
\pgfpathlineto{\pgfqpoint{0.904119in}{1.652244in}}%
\pgfpathlineto{\pgfqpoint{0.904984in}{1.593116in}}%
\pgfpathlineto{\pgfqpoint{0.905849in}{1.665333in}}%
\pgfpathlineto{\pgfqpoint{0.906714in}{1.656279in}}%
\pgfpathlineto{\pgfqpoint{0.907579in}{1.650106in}}%
\pgfpathlineto{\pgfqpoint{0.908445in}{1.596619in}}%
\pgfpathlineto{\pgfqpoint{0.910175in}{1.645061in}}%
\pgfpathlineto{\pgfqpoint{0.912770in}{1.569282in}}%
\pgfpathlineto{\pgfqpoint{0.914501in}{1.562127in}}%
\pgfpathlineto{\pgfqpoint{0.915365in}{1.582786in}}%
\pgfpathlineto{\pgfqpoint{0.916229in}{1.546574in}}%
\pgfpathlineto{\pgfqpoint{0.917959in}{1.613715in}}%
\pgfpathlineto{\pgfqpoint{0.918824in}{1.552690in}}%
\pgfpathlineto{\pgfqpoint{0.919687in}{1.556104in}}%
\pgfpathlineto{\pgfqpoint{0.920552in}{1.542539in}}%
\pgfpathlineto{\pgfqpoint{0.922280in}{1.625499in}}%
\pgfpathlineto{\pgfqpoint{0.923146in}{1.597094in}}%
\pgfpathlineto{\pgfqpoint{0.924010in}{1.553877in}}%
\pgfpathlineto{\pgfqpoint{0.925740in}{1.631110in}}%
\pgfpathlineto{\pgfqpoint{0.928329in}{1.557974in}}%
\pgfpathlineto{\pgfqpoint{0.930058in}{1.603802in}}%
\pgfpathlineto{\pgfqpoint{0.930923in}{1.541528in}}%
\pgfpathlineto{\pgfqpoint{0.931787in}{1.610094in}}%
\pgfpathlineto{\pgfqpoint{0.932652in}{1.555598in}}%
\pgfpathlineto{\pgfqpoint{0.933516in}{1.584092in}}%
\pgfpathlineto{\pgfqpoint{0.934381in}{1.531496in}}%
\pgfpathlineto{\pgfqpoint{0.936109in}{1.601307in}}%
\pgfpathlineto{\pgfqpoint{0.936974in}{1.599526in}}%
\pgfpathlineto{\pgfqpoint{0.937840in}{1.629269in}}%
\pgfpathlineto{\pgfqpoint{0.938706in}{1.543190in}}%
\pgfpathlineto{\pgfqpoint{0.939570in}{1.557082in}}%
\pgfpathlineto{\pgfqpoint{0.940435in}{1.640011in}}%
\pgfpathlineto{\pgfqpoint{0.943032in}{1.518553in}}%
\pgfpathlineto{\pgfqpoint{0.943897in}{1.606175in}}%
\pgfpathlineto{\pgfqpoint{0.944763in}{1.599586in}}%
\pgfpathlineto{\pgfqpoint{0.945628in}{1.627309in}}%
\pgfpathlineto{\pgfqpoint{0.946490in}{1.547228in}}%
\pgfpathlineto{\pgfqpoint{0.947355in}{1.605345in}}%
\pgfpathlineto{\pgfqpoint{0.949082in}{1.571390in}}%
\pgfpathlineto{\pgfqpoint{0.949946in}{1.566225in}}%
\pgfpathlineto{\pgfqpoint{0.950811in}{1.544703in}}%
\pgfpathlineto{\pgfqpoint{0.951674in}{1.488219in}}%
\pgfpathlineto{\pgfqpoint{0.953405in}{1.594272in}}%
\pgfpathlineto{\pgfqpoint{0.954269in}{1.610391in}}%
\pgfpathlineto{\pgfqpoint{0.955133in}{1.519564in}}%
\pgfpathlineto{\pgfqpoint{0.956863in}{1.648325in}}%
\pgfpathlineto{\pgfqpoint{0.957728in}{1.610570in}}%
\pgfpathlineto{\pgfqpoint{0.958594in}{1.615734in}}%
\pgfpathlineto{\pgfqpoint{0.961191in}{1.571033in}}%
\pgfpathlineto{\pgfqpoint{0.962055in}{1.590625in}}%
\pgfpathlineto{\pgfqpoint{0.962920in}{1.537018in}}%
\pgfpathlineto{\pgfqpoint{0.963783in}{1.619059in}}%
\pgfpathlineto{\pgfqpoint{0.964647in}{1.581481in}}%
\pgfpathlineto{\pgfqpoint{0.966375in}{1.614132in}}%
\pgfpathlineto{\pgfqpoint{0.967241in}{1.554085in}}%
\pgfpathlineto{\pgfqpoint{0.968105in}{1.590625in}}%
\pgfpathlineto{\pgfqpoint{0.968970in}{1.574654in}}%
\pgfpathlineto{\pgfqpoint{0.969836in}{1.576555in}}%
\pgfpathlineto{\pgfqpoint{0.970701in}{1.635088in}}%
\pgfpathlineto{\pgfqpoint{0.972432in}{1.491011in}}%
\pgfpathlineto{\pgfqpoint{0.975027in}{1.636215in}}%
\pgfpathlineto{\pgfqpoint{0.975889in}{1.550433in}}%
\pgfpathlineto{\pgfqpoint{0.976751in}{1.557409in}}%
\pgfpathlineto{\pgfqpoint{0.977616in}{1.638293in}}%
\pgfpathlineto{\pgfqpoint{0.978481in}{1.622918in}}%
\pgfpathlineto{\pgfqpoint{0.980211in}{1.599054in}}%
\pgfpathlineto{\pgfqpoint{0.981076in}{1.655093in}}%
\pgfpathlineto{\pgfqpoint{0.982806in}{1.611165in}}%
\pgfpathlineto{\pgfqpoint{0.983671in}{1.629745in}}%
\pgfpathlineto{\pgfqpoint{0.985400in}{1.603981in}}%
\pgfpathlineto{\pgfqpoint{0.986264in}{1.621048in}}%
\pgfpathlineto{\pgfqpoint{0.987127in}{1.532031in}}%
\pgfpathlineto{\pgfqpoint{0.988856in}{1.586051in}}%
\pgfpathlineto{\pgfqpoint{0.989722in}{1.576317in}}%
\pgfpathlineto{\pgfqpoint{0.991449in}{1.613805in}}%
\pgfpathlineto{\pgfqpoint{0.992314in}{1.593413in}}%
\pgfpathlineto{\pgfqpoint{0.993178in}{1.616151in}}%
\pgfpathlineto{\pgfqpoint{0.994043in}{1.511611in}}%
\pgfpathlineto{\pgfqpoint{0.995771in}{1.625885in}}%
\pgfpathlineto{\pgfqpoint{0.996636in}{1.581303in}}%
\pgfpathlineto{\pgfqpoint{0.997501in}{1.613953in}}%
\pgfpathlineto{\pgfqpoint{0.999232in}{1.567530in}}%
\pgfpathlineto{\pgfqpoint{1.000096in}{1.630574in}}%
\pgfpathlineto{\pgfqpoint{1.000961in}{1.555003in}}%
\pgfpathlineto{\pgfqpoint{1.001826in}{1.614370in}}%
\pgfpathlineto{\pgfqpoint{1.002691in}{1.603505in}}%
\pgfpathlineto{\pgfqpoint{1.003553in}{1.587892in}}%
\pgfpathlineto{\pgfqpoint{1.004419in}{1.598340in}}%
\pgfpathlineto{\pgfqpoint{1.005284in}{1.487181in}}%
\pgfpathlineto{\pgfqpoint{1.007014in}{1.632534in}}%
\pgfpathlineto{\pgfqpoint{1.007877in}{1.605078in}}%
\pgfpathlineto{\pgfqpoint{1.008742in}{1.613061in}}%
\pgfpathlineto{\pgfqpoint{1.009607in}{1.592759in}}%
\pgfpathlineto{\pgfqpoint{1.010472in}{1.613537in}}%
\pgfpathlineto{\pgfqpoint{1.011338in}{1.536363in}}%
\pgfpathlineto{\pgfqpoint{1.013067in}{1.643161in}}%
\pgfpathlineto{\pgfqpoint{1.013933in}{1.617753in}}%
\pgfpathlineto{\pgfqpoint{1.014798in}{1.556342in}}%
\pgfpathlineto{\pgfqpoint{1.016526in}{1.631883in}}%
\pgfpathlineto{\pgfqpoint{1.019120in}{1.587122in}}%
\pgfpathlineto{\pgfqpoint{1.019986in}{1.548980in}}%
\pgfpathlineto{\pgfqpoint{1.020851in}{1.606889in}}%
\pgfpathlineto{\pgfqpoint{1.021716in}{1.569668in}}%
\pgfpathlineto{\pgfqpoint{1.022581in}{1.569966in}}%
\pgfpathlineto{\pgfqpoint{1.023445in}{1.627666in}}%
\pgfpathlineto{\pgfqpoint{1.025173in}{1.558982in}}%
\pgfpathlineto{\pgfqpoint{1.027767in}{1.720810in}}%
\pgfpathlineto{\pgfqpoint{1.028629in}{1.710778in}}%
\pgfpathlineto{\pgfqpoint{1.029494in}{1.611343in}}%
\pgfpathlineto{\pgfqpoint{1.030357in}{1.716121in}}%
\pgfpathlineto{\pgfqpoint{1.032087in}{1.678723in}}%
\pgfpathlineto{\pgfqpoint{1.033816in}{1.636988in}}%
\pgfpathlineto{\pgfqpoint{1.034679in}{1.638888in}}%
\pgfpathlineto{\pgfqpoint{1.035545in}{1.650701in}}%
\pgfpathlineto{\pgfqpoint{1.036409in}{1.686884in}}%
\pgfpathlineto{\pgfqpoint{1.037275in}{1.645656in}}%
\pgfpathlineto{\pgfqpoint{1.038139in}{1.703535in}}%
\pgfpathlineto{\pgfqpoint{1.039005in}{1.696946in}}%
\pgfpathlineto{\pgfqpoint{1.039870in}{1.722353in}}%
\pgfpathlineto{\pgfqpoint{1.040734in}{1.708224in}}%
\pgfpathlineto{\pgfqpoint{1.041600in}{1.726153in}}%
\pgfpathlineto{\pgfqpoint{1.042466in}{1.705554in}}%
\pgfpathlineto{\pgfqpoint{1.045060in}{1.762069in}}%
\pgfpathlineto{\pgfqpoint{1.045927in}{1.694867in}}%
\pgfpathlineto{\pgfqpoint{1.046792in}{1.720989in}}%
\pgfpathlineto{\pgfqpoint{1.048523in}{1.686468in}}%
\pgfpathlineto{\pgfqpoint{1.049390in}{1.691543in}}%
\pgfpathlineto{\pgfqpoint{1.050256in}{1.751799in}}%
\pgfpathlineto{\pgfqpoint{1.051987in}{1.667739in}}%
\pgfpathlineto{\pgfqpoint{1.053719in}{1.735174in}}%
\pgfpathlineto{\pgfqpoint{1.055449in}{1.663165in}}%
\pgfpathlineto{\pgfqpoint{1.056313in}{1.710064in}}%
\pgfpathlineto{\pgfqpoint{1.058907in}{1.644823in}}%
\pgfpathlineto{\pgfqpoint{1.061503in}{1.726034in}}%
\pgfpathlineto{\pgfqpoint{1.062369in}{1.679254in}}%
\pgfpathlineto{\pgfqpoint{1.063234in}{1.552036in}}%
\pgfpathlineto{\pgfqpoint{1.064099in}{1.678302in}}%
\pgfpathlineto{\pgfqpoint{1.065827in}{1.572814in}}%
\pgfpathlineto{\pgfqpoint{1.066692in}{1.577860in}}%
\pgfpathlineto{\pgfqpoint{1.068422in}{1.552333in}}%
\pgfpathlineto{\pgfqpoint{1.069287in}{1.575722in}}%
\pgfpathlineto{\pgfqpoint{1.070151in}{1.674859in}}%
\pgfpathlineto{\pgfqpoint{1.071880in}{1.591067in}}%
\pgfpathlineto{\pgfqpoint{1.072744in}{1.634374in}}%
\pgfpathlineto{\pgfqpoint{1.074470in}{1.546455in}}%
\pgfpathlineto{\pgfqpoint{1.075335in}{1.572635in}}%
\pgfpathlineto{\pgfqpoint{1.076199in}{1.557320in}}%
\pgfpathlineto{\pgfqpoint{1.077064in}{1.613864in}}%
\pgfpathlineto{\pgfqpoint{1.077927in}{1.562425in}}%
\pgfpathlineto{\pgfqpoint{1.078792in}{1.615615in}}%
\pgfpathlineto{\pgfqpoint{1.079657in}{1.524729in}}%
\pgfpathlineto{\pgfqpoint{1.081384in}{1.599705in}}%
\pgfpathlineto{\pgfqpoint{1.082248in}{1.551322in}}%
\pgfpathlineto{\pgfqpoint{1.083114in}{1.601010in}}%
\pgfpathlineto{\pgfqpoint{1.083981in}{1.599675in}}%
\pgfpathlineto{\pgfqpoint{1.085711in}{1.559930in}}%
\pgfpathlineto{\pgfqpoint{1.086576in}{1.628763in}}%
\pgfpathlineto{\pgfqpoint{1.087440in}{1.601188in}}%
\pgfpathlineto{\pgfqpoint{1.088306in}{1.630039in}}%
\pgfpathlineto{\pgfqpoint{1.089172in}{1.609558in}}%
\pgfpathlineto{\pgfqpoint{1.090037in}{1.618226in}}%
\pgfpathlineto{\pgfqpoint{1.090903in}{1.575722in}}%
\pgfpathlineto{\pgfqpoint{1.091767in}{1.589732in}}%
\pgfpathlineto{\pgfqpoint{1.092632in}{1.551798in}}%
\pgfpathlineto{\pgfqpoint{1.093498in}{1.595845in}}%
\pgfpathlineto{\pgfqpoint{1.094362in}{1.519798in}}%
\pgfpathlineto{\pgfqpoint{1.095225in}{1.597150in}}%
\pgfpathlineto{\pgfqpoint{1.096090in}{1.559841in}}%
\pgfpathlineto{\pgfqpoint{1.097822in}{1.642506in}}%
\pgfpathlineto{\pgfqpoint{1.098687in}{1.634255in}}%
\pgfpathlineto{\pgfqpoint{1.099551in}{1.637877in}}%
\pgfpathlineto{\pgfqpoint{1.100416in}{1.591275in}}%
\pgfpathlineto{\pgfqpoint{1.101281in}{1.597388in}}%
\pgfpathlineto{\pgfqpoint{1.102145in}{1.596619in}}%
\pgfpathlineto{\pgfqpoint{1.103008in}{1.532980in}}%
\pgfpathlineto{\pgfqpoint{1.104736in}{1.582013in}}%
\pgfpathlineto{\pgfqpoint{1.105601in}{1.573584in}}%
\pgfpathlineto{\pgfqpoint{1.106466in}{1.581124in}}%
\pgfpathlineto{\pgfqpoint{1.107328in}{1.545090in}}%
\pgfpathlineto{\pgfqpoint{1.109055in}{1.576967in}}%
\pgfpathlineto{\pgfqpoint{1.109918in}{1.565333in}}%
\pgfpathlineto{\pgfqpoint{1.111647in}{1.584151in}}%
\pgfpathlineto{\pgfqpoint{1.112512in}{1.529715in}}%
\pgfpathlineto{\pgfqpoint{1.113375in}{1.581184in}}%
\pgfpathlineto{\pgfqpoint{1.114239in}{1.565690in}}%
\pgfpathlineto{\pgfqpoint{1.115102in}{1.637609in}}%
\pgfpathlineto{\pgfqpoint{1.115967in}{1.555896in}}%
\pgfpathlineto{\pgfqpoint{1.116831in}{1.653847in}}%
\pgfpathlineto{\pgfqpoint{1.117696in}{1.549307in}}%
\pgfpathlineto{\pgfqpoint{1.118561in}{1.574773in}}%
\pgfpathlineto{\pgfqpoint{1.119426in}{1.569936in}}%
\pgfpathlineto{\pgfqpoint{1.120289in}{1.573230in}}%
\pgfpathlineto{\pgfqpoint{1.122016in}{1.695254in}}%
\pgfpathlineto{\pgfqpoint{1.122881in}{1.659369in}}%
\pgfpathlineto{\pgfqpoint{1.124612in}{1.682668in}}%
\pgfpathlineto{\pgfqpoint{1.125476in}{1.655688in}}%
\pgfpathlineto{\pgfqpoint{1.126342in}{1.720751in}}%
\pgfpathlineto{\pgfqpoint{1.127208in}{1.646723in}}%
\pgfpathlineto{\pgfqpoint{1.128072in}{1.738974in}}%
\pgfpathlineto{\pgfqpoint{1.128936in}{1.629715in}}%
\pgfpathlineto{\pgfqpoint{1.129800in}{1.650106in}}%
\pgfpathlineto{\pgfqpoint{1.130665in}{1.677057in}}%
\pgfpathlineto{\pgfqpoint{1.132396in}{1.617158in}}%
\pgfpathlineto{\pgfqpoint{1.133261in}{1.670051in}}%
\pgfpathlineto{\pgfqpoint{1.134126in}{1.658566in}}%
\pgfpathlineto{\pgfqpoint{1.134991in}{1.665184in}}%
\pgfpathlineto{\pgfqpoint{1.136722in}{1.604159in}}%
\pgfpathlineto{\pgfqpoint{1.137586in}{1.624164in}}%
\pgfpathlineto{\pgfqpoint{1.138451in}{1.660525in}}%
\pgfpathlineto{\pgfqpoint{1.139316in}{1.643518in}}%
\pgfpathlineto{\pgfqpoint{1.140181in}{1.663701in}}%
\pgfpathlineto{\pgfqpoint{1.141044in}{1.645771in}}%
\pgfpathlineto{\pgfqpoint{1.141910in}{1.687505in}}%
\pgfpathlineto{\pgfqpoint{1.142776in}{1.577979in}}%
\pgfpathlineto{\pgfqpoint{1.143641in}{1.596946in}}%
\pgfpathlineto{\pgfqpoint{1.147100in}{1.737550in}}%
\pgfpathlineto{\pgfqpoint{1.148829in}{1.670825in}}%
\pgfpathlineto{\pgfqpoint{1.149692in}{1.703416in}}%
\pgfpathlineto{\pgfqpoint{1.150558in}{1.664087in}}%
\pgfpathlineto{\pgfqpoint{1.151424in}{1.682578in}}%
\pgfpathlineto{\pgfqpoint{1.152290in}{1.680738in}}%
\pgfpathlineto{\pgfqpoint{1.153155in}{1.634166in}}%
\pgfpathlineto{\pgfqpoint{1.155749in}{1.684419in}}%
\pgfpathlineto{\pgfqpoint{1.156614in}{1.665779in}}%
\pgfpathlineto{\pgfqpoint{1.157478in}{1.675960in}}%
\pgfpathlineto{\pgfqpoint{1.158343in}{1.653669in}}%
\pgfpathlineto{\pgfqpoint{1.159208in}{1.699556in}}%
\pgfpathlineto{\pgfqpoint{1.160073in}{1.631496in}}%
\pgfpathlineto{\pgfqpoint{1.160937in}{1.769665in}}%
\pgfpathlineto{\pgfqpoint{1.164394in}{1.589673in}}%
\pgfpathlineto{\pgfqpoint{1.165259in}{1.588070in}}%
\pgfpathlineto{\pgfqpoint{1.166124in}{1.592878in}}%
\pgfpathlineto{\pgfqpoint{1.166990in}{1.642774in}}%
\pgfpathlineto{\pgfqpoint{1.170449in}{1.526569in}}%
\pgfpathlineto{\pgfqpoint{1.172180in}{1.625528in}}%
\pgfpathlineto{\pgfqpoint{1.173045in}{1.610094in}}%
\pgfpathlineto{\pgfqpoint{1.173912in}{1.617277in}}%
\pgfpathlineto{\pgfqpoint{1.174778in}{1.558863in}}%
\pgfpathlineto{\pgfqpoint{1.176508in}{1.593473in}}%
\pgfpathlineto{\pgfqpoint{1.177374in}{1.607305in}}%
\pgfpathlineto{\pgfqpoint{1.178239in}{1.560882in}}%
\pgfpathlineto{\pgfqpoint{1.179969in}{1.589851in}}%
\pgfpathlineto{\pgfqpoint{1.180834in}{1.564325in}}%
\pgfpathlineto{\pgfqpoint{1.181699in}{1.618940in}}%
\pgfpathlineto{\pgfqpoint{1.182565in}{1.587122in}}%
\pgfpathlineto{\pgfqpoint{1.183430in}{1.608670in}}%
\pgfpathlineto{\pgfqpoint{1.186026in}{1.552482in}}%
\pgfpathlineto{\pgfqpoint{1.187758in}{1.605286in}}%
\pgfpathlineto{\pgfqpoint{1.189487in}{1.569192in}}%
\pgfpathlineto{\pgfqpoint{1.190352in}{1.610510in}}%
\pgfpathlineto{\pgfqpoint{1.191218in}{1.553906in}}%
\pgfpathlineto{\pgfqpoint{1.192948in}{1.677473in}}%
\pgfpathlineto{\pgfqpoint{1.193813in}{1.611046in}}%
\pgfpathlineto{\pgfqpoint{1.194678in}{1.669044in}}%
\pgfpathlineto{\pgfqpoint{1.195541in}{1.542272in}}%
\pgfpathlineto{\pgfqpoint{1.196406in}{1.629626in}}%
\pgfpathlineto{\pgfqpoint{1.198137in}{1.572933in}}%
\pgfpathlineto{\pgfqpoint{1.199001in}{1.648385in}}%
\pgfpathlineto{\pgfqpoint{1.199865in}{1.537315in}}%
\pgfpathlineto{\pgfqpoint{1.200729in}{1.593503in}}%
\pgfpathlineto{\pgfqpoint{1.201595in}{1.495819in}}%
\pgfpathlineto{\pgfqpoint{1.202459in}{1.610867in}}%
\pgfpathlineto{\pgfqpoint{1.203324in}{1.501400in}}%
\pgfpathlineto{\pgfqpoint{1.204189in}{1.568839in}}%
\pgfpathlineto{\pgfqpoint{1.205054in}{1.556996in}}%
\pgfpathlineto{\pgfqpoint{1.206785in}{1.599173in}}%
\pgfpathlineto{\pgfqpoint{1.209379in}{1.578038in}}%
\pgfpathlineto{\pgfqpoint{1.210246in}{1.499559in}}%
\pgfpathlineto{\pgfqpoint{1.211111in}{1.597094in}}%
\pgfpathlineto{\pgfqpoint{1.211977in}{1.579284in}}%
\pgfpathlineto{\pgfqpoint{1.212842in}{1.588130in}}%
\pgfpathlineto{\pgfqpoint{1.213706in}{1.620959in}}%
\pgfpathlineto{\pgfqpoint{1.214572in}{1.593205in}}%
\pgfpathlineto{\pgfqpoint{1.215438in}{1.665422in}}%
\pgfpathlineto{\pgfqpoint{1.218035in}{1.584746in}}%
\pgfpathlineto{\pgfqpoint{1.218902in}{1.684062in}}%
\pgfpathlineto{\pgfqpoint{1.219768in}{1.600181in}}%
\pgfpathlineto{\pgfqpoint{1.220634in}{1.612886in}}%
\pgfpathlineto{\pgfqpoint{1.222366in}{1.609146in}}%
\pgfpathlineto{\pgfqpoint{1.224098in}{1.615943in}}%
\pgfpathlineto{\pgfqpoint{1.224965in}{1.571271in}}%
\pgfpathlineto{\pgfqpoint{1.225831in}{1.607959in}}%
\pgfpathlineto{\pgfqpoint{1.227562in}{1.566701in}}%
\pgfpathlineto{\pgfqpoint{1.228427in}{1.595194in}}%
\pgfpathlineto{\pgfqpoint{1.229292in}{1.563288in}}%
\pgfpathlineto{\pgfqpoint{1.230156in}{1.585281in}}%
\pgfpathlineto{\pgfqpoint{1.231020in}{1.574654in}}%
\pgfpathlineto{\pgfqpoint{1.232751in}{1.587003in}}%
\pgfpathlineto{\pgfqpoint{1.233617in}{1.566552in}}%
\pgfpathlineto{\pgfqpoint{1.236211in}{1.618642in}}%
\pgfpathlineto{\pgfqpoint{1.237942in}{1.538144in}}%
\pgfpathlineto{\pgfqpoint{1.238807in}{1.618702in}}%
\pgfpathlineto{\pgfqpoint{1.239671in}{1.608432in}}%
\pgfpathlineto{\pgfqpoint{1.240537in}{1.636334in}}%
\pgfpathlineto{\pgfqpoint{1.241402in}{1.548950in}}%
\pgfpathlineto{\pgfqpoint{1.242268in}{1.557558in}}%
\pgfpathlineto{\pgfqpoint{1.243133in}{1.555539in}}%
\pgfpathlineto{\pgfqpoint{1.244864in}{1.590208in}}%
\pgfpathlineto{\pgfqpoint{1.245728in}{1.550880in}}%
\pgfpathlineto{\pgfqpoint{1.246593in}{1.602021in}}%
\pgfpathlineto{\pgfqpoint{1.247458in}{1.548771in}}%
\pgfpathlineto{\pgfqpoint{1.249187in}{1.609737in}}%
\pgfpathlineto{\pgfqpoint{1.250053in}{1.606532in}}%
\pgfpathlineto{\pgfqpoint{1.250916in}{1.563076in}}%
\pgfpathlineto{\pgfqpoint{1.251781in}{1.611815in}}%
\pgfpathlineto{\pgfqpoint{1.252647in}{1.605286in}}%
\pgfpathlineto{\pgfqpoint{1.253512in}{1.626064in}}%
\pgfpathlineto{\pgfqpoint{1.255243in}{1.576198in}}%
\pgfpathlineto{\pgfqpoint{1.256974in}{1.600002in}}%
\pgfpathlineto{\pgfqpoint{1.257840in}{1.595934in}}%
\pgfpathlineto{\pgfqpoint{1.258706in}{1.567114in}}%
\pgfpathlineto{\pgfqpoint{1.259571in}{1.619416in}}%
\pgfpathlineto{\pgfqpoint{1.261299in}{1.575841in}}%
\pgfpathlineto{\pgfqpoint{1.262165in}{1.584538in}}%
\pgfpathlineto{\pgfqpoint{1.263030in}{1.581124in}}%
\pgfpathlineto{\pgfqpoint{1.263893in}{1.550612in}}%
\pgfpathlineto{\pgfqpoint{1.264759in}{1.572784in}}%
\pgfpathlineto{\pgfqpoint{1.265625in}{1.632474in}}%
\pgfpathlineto{\pgfqpoint{1.267353in}{1.551679in}}%
\pgfpathlineto{\pgfqpoint{1.268216in}{1.559160in}}%
\pgfpathlineto{\pgfqpoint{1.269081in}{1.598102in}}%
\pgfpathlineto{\pgfqpoint{1.269946in}{1.596262in}}%
\pgfpathlineto{\pgfqpoint{1.270811in}{1.610034in}}%
\pgfpathlineto{\pgfqpoint{1.272543in}{1.531496in}}%
\pgfpathlineto{\pgfqpoint{1.273408in}{1.585162in}}%
\pgfpathlineto{\pgfqpoint{1.274272in}{1.572903in}}%
\pgfpathlineto{\pgfqpoint{1.275137in}{1.591870in}}%
\pgfpathlineto{\pgfqpoint{1.276003in}{1.576614in}}%
\pgfpathlineto{\pgfqpoint{1.276869in}{1.533247in}}%
\pgfpathlineto{\pgfqpoint{1.277734in}{1.743841in}}%
\pgfpathlineto{\pgfqpoint{1.278600in}{1.564028in}}%
\pgfpathlineto{\pgfqpoint{1.279465in}{1.601843in}}%
\pgfpathlineto{\pgfqpoint{1.280331in}{1.626123in}}%
\pgfpathlineto{\pgfqpoint{1.282062in}{1.598043in}}%
\pgfpathlineto{\pgfqpoint{1.282927in}{1.678897in}}%
\pgfpathlineto{\pgfqpoint{1.283793in}{1.586587in}}%
\pgfpathlineto{\pgfqpoint{1.284658in}{1.650757in}}%
\pgfpathlineto{\pgfqpoint{1.285521in}{1.558684in}}%
\pgfpathlineto{\pgfqpoint{1.287250in}{1.614072in}}%
\pgfpathlineto{\pgfqpoint{1.288114in}{1.622383in}}%
\pgfpathlineto{\pgfqpoint{1.290711in}{1.713686in}}%
\pgfpathlineto{\pgfqpoint{1.291576in}{1.660793in}}%
\pgfpathlineto{\pgfqpoint{1.292440in}{1.752985in}}%
\pgfpathlineto{\pgfqpoint{1.293306in}{1.742923in}}%
\pgfpathlineto{\pgfqpoint{1.294172in}{1.621316in}}%
\pgfpathlineto{\pgfqpoint{1.295904in}{1.697835in}}%
\pgfpathlineto{\pgfqpoint{1.296768in}{1.682995in}}%
\pgfpathlineto{\pgfqpoint{1.297633in}{1.572104in}}%
\pgfpathlineto{\pgfqpoint{1.298498in}{1.591573in}}%
\pgfpathlineto{\pgfqpoint{1.299363in}{1.642863in}}%
\pgfpathlineto{\pgfqpoint{1.301092in}{1.619118in}}%
\pgfpathlineto{\pgfqpoint{1.301958in}{1.625885in}}%
\pgfpathlineto{\pgfqpoint{1.305417in}{1.706472in}}%
\pgfpathlineto{\pgfqpoint{1.306282in}{1.703475in}}%
\pgfpathlineto{\pgfqpoint{1.307147in}{1.629983in}}%
\pgfpathlineto{\pgfqpoint{1.308879in}{1.746098in}}%
\pgfpathlineto{\pgfqpoint{1.309745in}{1.713597in}}%
\pgfpathlineto{\pgfqpoint{1.310610in}{1.739034in}}%
\pgfpathlineto{\pgfqpoint{1.312342in}{1.708343in}}%
\pgfpathlineto{\pgfqpoint{1.313208in}{1.716296in}}%
\pgfpathlineto{\pgfqpoint{1.314074in}{1.662039in}}%
\pgfpathlineto{\pgfqpoint{1.314939in}{1.666698in}}%
\pgfpathlineto{\pgfqpoint{1.315803in}{1.731433in}}%
\pgfpathlineto{\pgfqpoint{1.317529in}{1.640312in}}%
\pgfpathlineto{\pgfqpoint{1.318395in}{1.672368in}}%
\pgfpathlineto{\pgfqpoint{1.320124in}{1.621966in}}%
\pgfpathlineto{\pgfqpoint{1.320989in}{1.730188in}}%
\pgfpathlineto{\pgfqpoint{1.321854in}{1.709975in}}%
\pgfpathlineto{\pgfqpoint{1.322720in}{1.596678in}}%
\pgfpathlineto{\pgfqpoint{1.325313in}{1.718434in}}%
\pgfpathlineto{\pgfqpoint{1.326179in}{1.629329in}}%
\pgfpathlineto{\pgfqpoint{1.327908in}{1.685843in}}%
\pgfpathlineto{\pgfqpoint{1.328771in}{1.683556in}}%
\pgfpathlineto{\pgfqpoint{1.329636in}{1.685843in}}%
\pgfpathlineto{\pgfqpoint{1.330500in}{1.728466in}}%
\pgfpathlineto{\pgfqpoint{1.331365in}{1.699051in}}%
\pgfpathlineto{\pgfqpoint{1.332230in}{1.726507in}}%
\pgfpathlineto{\pgfqpoint{1.333954in}{1.694659in}}%
\pgfpathlineto{\pgfqpoint{1.335684in}{1.699199in}}%
\pgfpathlineto{\pgfqpoint{1.336550in}{1.646187in}}%
\pgfpathlineto{\pgfqpoint{1.337413in}{1.728347in}}%
\pgfpathlineto{\pgfqpoint{1.338278in}{1.660049in}}%
\pgfpathlineto{\pgfqpoint{1.340007in}{1.704781in}}%
\pgfpathlineto{\pgfqpoint{1.342599in}{1.546812in}}%
\pgfpathlineto{\pgfqpoint{1.343462in}{1.584538in}}%
\pgfpathlineto{\pgfqpoint{1.344329in}{1.555658in}}%
\pgfpathlineto{\pgfqpoint{1.345195in}{1.573171in}}%
\pgfpathlineto{\pgfqpoint{1.346061in}{1.543904in}}%
\pgfpathlineto{\pgfqpoint{1.346925in}{1.567828in}}%
\pgfpathlineto{\pgfqpoint{1.347786in}{1.548920in}}%
\pgfpathlineto{\pgfqpoint{1.348652in}{1.630872in}}%
\pgfpathlineto{\pgfqpoint{1.349515in}{1.562544in}}%
\pgfpathlineto{\pgfqpoint{1.351245in}{1.600121in}}%
\pgfpathlineto{\pgfqpoint{1.352107in}{1.545150in}}%
\pgfpathlineto{\pgfqpoint{1.352972in}{1.548474in}}%
\pgfpathlineto{\pgfqpoint{1.353837in}{1.554352in}}%
\pgfpathlineto{\pgfqpoint{1.354702in}{1.620483in}}%
\pgfpathlineto{\pgfqpoint{1.355566in}{1.578098in}}%
\pgfpathlineto{\pgfqpoint{1.356431in}{1.624878in}}%
\pgfpathlineto{\pgfqpoint{1.359028in}{1.559398in}}%
\pgfpathlineto{\pgfqpoint{1.359893in}{1.610778in}}%
\pgfpathlineto{\pgfqpoint{1.360758in}{1.581779in}}%
\pgfpathlineto{\pgfqpoint{1.361622in}{1.627488in}}%
\pgfpathlineto{\pgfqpoint{1.362488in}{1.572308in}}%
\pgfpathlineto{\pgfqpoint{1.365081in}{1.640904in}}%
\pgfpathlineto{\pgfqpoint{1.365947in}{1.616623in}}%
\pgfpathlineto{\pgfqpoint{1.366813in}{1.644049in}}%
\pgfpathlineto{\pgfqpoint{1.367678in}{1.567292in}}%
\pgfpathlineto{\pgfqpoint{1.368543in}{1.571211in}}%
\pgfpathlineto{\pgfqpoint{1.369408in}{1.577384in}}%
\pgfpathlineto{\pgfqpoint{1.370273in}{1.562603in}}%
\pgfpathlineto{\pgfqpoint{1.371138in}{1.595075in}}%
\pgfpathlineto{\pgfqpoint{1.372002in}{1.586914in}}%
\pgfpathlineto{\pgfqpoint{1.372866in}{1.555836in}}%
\pgfpathlineto{\pgfqpoint{1.374594in}{1.644287in}}%
\pgfpathlineto{\pgfqpoint{1.375459in}{1.558744in}}%
\pgfpathlineto{\pgfqpoint{1.377187in}{1.615794in}}%
\pgfpathlineto{\pgfqpoint{1.378052in}{1.586468in}}%
\pgfpathlineto{\pgfqpoint{1.378916in}{1.511075in}}%
\pgfpathlineto{\pgfqpoint{1.380646in}{1.578395in}}%
\pgfpathlineto{\pgfqpoint{1.381509in}{1.562782in}}%
\pgfpathlineto{\pgfqpoint{1.383238in}{1.580946in}}%
\pgfpathlineto{\pgfqpoint{1.384104in}{1.522650in}}%
\pgfpathlineto{\pgfqpoint{1.384968in}{1.536601in}}%
\pgfpathlineto{\pgfqpoint{1.387562in}{1.629031in}}%
\pgfpathlineto{\pgfqpoint{1.388427in}{1.628912in}}%
\pgfpathlineto{\pgfqpoint{1.390156in}{1.572992in}}%
\pgfpathlineto{\pgfqpoint{1.391021in}{1.565184in}}%
\pgfpathlineto{\pgfqpoint{1.391886in}{1.519564in}}%
\pgfpathlineto{\pgfqpoint{1.392749in}{1.591751in}}%
\pgfpathlineto{\pgfqpoint{1.393613in}{1.589732in}}%
\pgfpathlineto{\pgfqpoint{1.395342in}{1.589821in}}%
\pgfpathlineto{\pgfqpoint{1.396208in}{1.603743in}}%
\pgfpathlineto{\pgfqpoint{1.397073in}{1.701456in}}%
\pgfpathlineto{\pgfqpoint{1.397938in}{1.679373in}}%
\pgfpathlineto{\pgfqpoint{1.398803in}{1.709588in}}%
\pgfpathlineto{\pgfqpoint{1.400534in}{1.636036in}}%
\pgfpathlineto{\pgfqpoint{1.401400in}{1.703118in}}%
\pgfpathlineto{\pgfqpoint{1.402265in}{1.662039in}}%
\pgfpathlineto{\pgfqpoint{1.403131in}{1.725499in}}%
\pgfpathlineto{\pgfqpoint{1.403997in}{1.699021in}}%
\pgfpathlineto{\pgfqpoint{1.405729in}{1.722885in}}%
\pgfpathlineto{\pgfqpoint{1.407460in}{1.677741in}}%
\pgfpathlineto{\pgfqpoint{1.408324in}{1.732564in}}%
\pgfpathlineto{\pgfqpoint{1.410055in}{1.677295in}}%
\pgfpathlineto{\pgfqpoint{1.410921in}{1.725677in}}%
\pgfpathlineto{\pgfqpoint{1.411785in}{1.691067in}}%
\pgfpathlineto{\pgfqpoint{1.413515in}{1.736007in}}%
\pgfpathlineto{\pgfqpoint{1.414379in}{1.735144in}}%
\pgfpathlineto{\pgfqpoint{1.416108in}{1.693205in}}%
\pgfpathlineto{\pgfqpoint{1.416972in}{1.708819in}}%
\pgfpathlineto{\pgfqpoint{1.417838in}{1.653788in}}%
\pgfpathlineto{\pgfqpoint{1.418701in}{1.725856in}}%
\pgfpathlineto{\pgfqpoint{1.419566in}{1.643339in}}%
\pgfpathlineto{\pgfqpoint{1.420430in}{1.684062in}}%
\pgfpathlineto{\pgfqpoint{1.421294in}{1.639123in}}%
\pgfpathlineto{\pgfqpoint{1.422159in}{1.661384in}}%
\pgfpathlineto{\pgfqpoint{1.423024in}{1.657290in}}%
\pgfpathlineto{\pgfqpoint{1.424749in}{1.677652in}}%
\pgfpathlineto{\pgfqpoint{1.425614in}{1.627607in}}%
\pgfpathlineto{\pgfqpoint{1.427345in}{1.761295in}}%
\pgfpathlineto{\pgfqpoint{1.429937in}{1.596619in}}%
\pgfpathlineto{\pgfqpoint{1.431667in}{1.680738in}}%
\pgfpathlineto{\pgfqpoint{1.432532in}{1.670408in}}%
\pgfpathlineto{\pgfqpoint{1.433398in}{1.694659in}}%
\pgfpathlineto{\pgfqpoint{1.434265in}{1.644763in}}%
\pgfpathlineto{\pgfqpoint{1.435130in}{1.656695in}}%
\pgfpathlineto{\pgfqpoint{1.435996in}{1.673584in}}%
\pgfpathlineto{\pgfqpoint{1.436861in}{1.668746in}}%
\pgfpathlineto{\pgfqpoint{1.438592in}{1.680678in}}%
\pgfpathlineto{\pgfqpoint{1.439455in}{1.660079in}}%
\pgfpathlineto{\pgfqpoint{1.441185in}{1.691008in}}%
\pgfpathlineto{\pgfqpoint{1.442050in}{1.681035in}}%
\pgfpathlineto{\pgfqpoint{1.442915in}{1.610718in}}%
\pgfpathlineto{\pgfqpoint{1.443778in}{1.717308in}}%
\pgfpathlineto{\pgfqpoint{1.444643in}{1.674803in}}%
\pgfpathlineto{\pgfqpoint{1.445508in}{1.698578in}}%
\pgfpathlineto{\pgfqpoint{1.446371in}{1.696768in}}%
\pgfpathlineto{\pgfqpoint{1.448967in}{1.628499in}}%
\pgfpathlineto{\pgfqpoint{1.449831in}{1.705911in}}%
\pgfpathlineto{\pgfqpoint{1.450696in}{1.642629in}}%
\pgfpathlineto{\pgfqpoint{1.451559in}{1.697065in}}%
\pgfpathlineto{\pgfqpoint{1.452425in}{1.659696in}}%
\pgfpathlineto{\pgfqpoint{1.453289in}{1.662990in}}%
\pgfpathlineto{\pgfqpoint{1.455884in}{1.717724in}}%
\pgfpathlineto{\pgfqpoint{1.456749in}{1.695105in}}%
\pgfpathlineto{\pgfqpoint{1.457614in}{1.635237in}}%
\pgfpathlineto{\pgfqpoint{1.458480in}{1.651653in}}%
\pgfpathlineto{\pgfqpoint{1.459345in}{1.688933in}}%
\pgfpathlineto{\pgfqpoint{1.461941in}{1.621405in}}%
\pgfpathlineto{\pgfqpoint{1.462806in}{1.621435in}}%
\pgfpathlineto{\pgfqpoint{1.464538in}{1.688665in}}%
\pgfpathlineto{\pgfqpoint{1.467134in}{1.621583in}}%
\pgfpathlineto{\pgfqpoint{1.467999in}{1.679611in}}%
\pgfpathlineto{\pgfqpoint{1.468864in}{1.666017in}}%
\pgfpathlineto{\pgfqpoint{1.469729in}{1.655866in}}%
\pgfpathlineto{\pgfqpoint{1.470594in}{1.676942in}}%
\pgfpathlineto{\pgfqpoint{1.472326in}{1.637940in}}%
\pgfpathlineto{\pgfqpoint{1.474056in}{1.664801in}}%
\pgfpathlineto{\pgfqpoint{1.474922in}{1.633843in}}%
\pgfpathlineto{\pgfqpoint{1.475788in}{1.705851in}}%
\pgfpathlineto{\pgfqpoint{1.476654in}{1.692346in}}%
\pgfpathlineto{\pgfqpoint{1.477520in}{1.665009in}}%
\pgfpathlineto{\pgfqpoint{1.478386in}{1.711905in}}%
\pgfpathlineto{\pgfqpoint{1.479253in}{1.685073in}}%
\pgfpathlineto{\pgfqpoint{1.480117in}{1.743310in}}%
\pgfpathlineto{\pgfqpoint{1.481847in}{1.638769in}}%
\pgfpathlineto{\pgfqpoint{1.482711in}{1.696887in}}%
\pgfpathlineto{\pgfqpoint{1.483575in}{1.629983in}}%
\pgfpathlineto{\pgfqpoint{1.484441in}{1.638948in}}%
\pgfpathlineto{\pgfqpoint{1.485307in}{1.652780in}}%
\pgfpathlineto{\pgfqpoint{1.487905in}{1.591930in}}%
\pgfpathlineto{\pgfqpoint{1.488770in}{1.530131in}}%
\pgfpathlineto{\pgfqpoint{1.489635in}{1.625885in}}%
\pgfpathlineto{\pgfqpoint{1.490500in}{1.584835in}}%
\pgfpathlineto{\pgfqpoint{1.491365in}{1.585400in}}%
\pgfpathlineto{\pgfqpoint{1.492231in}{1.588725in}}%
\pgfpathlineto{\pgfqpoint{1.493097in}{1.558655in}}%
\pgfpathlineto{\pgfqpoint{1.493962in}{1.569668in}}%
\pgfpathlineto{\pgfqpoint{1.495693in}{1.537285in}}%
\pgfpathlineto{\pgfqpoint{1.497424in}{1.626064in}}%
\pgfpathlineto{\pgfqpoint{1.499153in}{1.588963in}}%
\pgfpathlineto{\pgfqpoint{1.500017in}{1.640818in}}%
\pgfpathlineto{\pgfqpoint{1.500882in}{1.518675in}}%
\pgfpathlineto{\pgfqpoint{1.502609in}{1.582995in}}%
\pgfpathlineto{\pgfqpoint{1.503473in}{1.576495in}}%
\pgfpathlineto{\pgfqpoint{1.504339in}{1.578514in}}%
\pgfpathlineto{\pgfqpoint{1.505204in}{1.566552in}}%
\pgfpathlineto{\pgfqpoint{1.506068in}{1.599292in}}%
\pgfpathlineto{\pgfqpoint{1.506933in}{1.547645in}}%
\pgfpathlineto{\pgfqpoint{1.507798in}{1.560350in}}%
\pgfpathlineto{\pgfqpoint{1.508662in}{1.660793in}}%
\pgfpathlineto{\pgfqpoint{1.509524in}{1.541026in}}%
\pgfpathlineto{\pgfqpoint{1.511254in}{1.632240in}}%
\pgfpathlineto{\pgfqpoint{1.512119in}{1.625592in}}%
\pgfpathlineto{\pgfqpoint{1.512984in}{1.610335in}}%
\pgfpathlineto{\pgfqpoint{1.513848in}{1.675279in}}%
\pgfpathlineto{\pgfqpoint{1.514713in}{1.557145in}}%
\pgfpathlineto{\pgfqpoint{1.515579in}{1.565158in}}%
\pgfpathlineto{\pgfqpoint{1.516443in}{1.587360in}}%
\pgfpathlineto{\pgfqpoint{1.519904in}{1.533753in}}%
\pgfpathlineto{\pgfqpoint{1.520767in}{1.586051in}}%
\pgfpathlineto{\pgfqpoint{1.521632in}{1.545298in}}%
\pgfpathlineto{\pgfqpoint{1.522496in}{1.549009in}}%
\pgfpathlineto{\pgfqpoint{1.523360in}{1.543190in}}%
\pgfpathlineto{\pgfqpoint{1.525956in}{1.640312in}}%
\pgfpathlineto{\pgfqpoint{1.526821in}{1.559458in}}%
\pgfpathlineto{\pgfqpoint{1.527685in}{1.579879in}}%
\pgfpathlineto{\pgfqpoint{1.528549in}{1.567828in}}%
\pgfpathlineto{\pgfqpoint{1.529411in}{1.577562in}}%
\pgfpathlineto{\pgfqpoint{1.532005in}{1.691246in}}%
\pgfpathlineto{\pgfqpoint{1.532870in}{1.615318in}}%
\pgfpathlineto{\pgfqpoint{1.533734in}{1.664798in}}%
\pgfpathlineto{\pgfqpoint{1.534599in}{1.581481in}}%
\pgfpathlineto{\pgfqpoint{1.536326in}{1.641320in}}%
\pgfpathlineto{\pgfqpoint{1.537190in}{1.610689in}}%
\pgfpathlineto{\pgfqpoint{1.538055in}{1.509294in}}%
\pgfpathlineto{\pgfqpoint{1.539786in}{1.627726in}}%
\pgfpathlineto{\pgfqpoint{1.540652in}{1.522710in}}%
\pgfpathlineto{\pgfqpoint{1.542382in}{1.608670in}}%
\pgfpathlineto{\pgfqpoint{1.543246in}{1.514429in}}%
\pgfpathlineto{\pgfqpoint{1.544111in}{1.623688in}}%
\pgfpathlineto{\pgfqpoint{1.544975in}{1.603029in}}%
\pgfpathlineto{\pgfqpoint{1.545841in}{1.584389in}}%
\pgfpathlineto{\pgfqpoint{1.546705in}{1.612053in}}%
\pgfpathlineto{\pgfqpoint{1.548435in}{1.560525in}}%
\pgfpathlineto{\pgfqpoint{1.549300in}{1.577860in}}%
\pgfpathlineto{\pgfqpoint{1.550163in}{1.543190in}}%
\pgfpathlineto{\pgfqpoint{1.551028in}{1.586527in}}%
\pgfpathlineto{\pgfqpoint{1.552758in}{1.567976in}}%
\pgfpathlineto{\pgfqpoint{1.553624in}{1.557855in}}%
\pgfpathlineto{\pgfqpoint{1.554488in}{1.583500in}}%
\pgfpathlineto{\pgfqpoint{1.555353in}{1.576019in}}%
\pgfpathlineto{\pgfqpoint{1.556218in}{1.646247in}}%
\pgfpathlineto{\pgfqpoint{1.557083in}{1.555509in}}%
\pgfpathlineto{\pgfqpoint{1.557945in}{1.591751in}}%
\pgfpathlineto{\pgfqpoint{1.559676in}{1.532890in}}%
\pgfpathlineto{\pgfqpoint{1.560542in}{1.608253in}}%
\pgfpathlineto{\pgfqpoint{1.561408in}{1.554055in}}%
\pgfpathlineto{\pgfqpoint{1.562273in}{1.575900in}}%
\pgfpathlineto{\pgfqpoint{1.563139in}{1.567292in}}%
\pgfpathlineto{\pgfqpoint{1.564005in}{1.639123in}}%
\pgfpathlineto{\pgfqpoint{1.565738in}{1.545150in}}%
\pgfpathlineto{\pgfqpoint{1.566603in}{1.562544in}}%
\pgfpathlineto{\pgfqpoint{1.567467in}{1.541707in}}%
\pgfpathlineto{\pgfqpoint{1.568332in}{1.547641in}}%
\pgfpathlineto{\pgfqpoint{1.569196in}{1.612763in}}%
\pgfpathlineto{\pgfqpoint{1.570061in}{1.571981in}}%
\pgfpathlineto{\pgfqpoint{1.572656in}{1.644525in}}%
\pgfpathlineto{\pgfqpoint{1.574383in}{1.536601in}}%
\pgfpathlineto{\pgfqpoint{1.575249in}{1.632772in}}%
\pgfpathlineto{\pgfqpoint{1.576114in}{1.569728in}}%
\pgfpathlineto{\pgfqpoint{1.576978in}{1.598459in}}%
\pgfpathlineto{\pgfqpoint{1.578706in}{1.552304in}}%
\pgfpathlineto{\pgfqpoint{1.579570in}{1.677354in}}%
\pgfpathlineto{\pgfqpoint{1.580437in}{1.575781in}}%
\pgfpathlineto{\pgfqpoint{1.581303in}{1.658060in}}%
\pgfpathlineto{\pgfqpoint{1.582170in}{1.655271in}}%
\pgfpathlineto{\pgfqpoint{1.583903in}{1.571271in}}%
\pgfpathlineto{\pgfqpoint{1.585634in}{1.555955in}}%
\pgfpathlineto{\pgfqpoint{1.586500in}{1.592227in}}%
\pgfpathlineto{\pgfqpoint{1.587366in}{1.545507in}}%
\pgfpathlineto{\pgfqpoint{1.589098in}{1.632772in}}%
\pgfpathlineto{\pgfqpoint{1.590828in}{1.576079in}}%
\pgfpathlineto{\pgfqpoint{1.591692in}{1.610927in}}%
\pgfpathlineto{\pgfqpoint{1.593423in}{1.541528in}}%
\pgfpathlineto{\pgfqpoint{1.594289in}{1.543309in}}%
\pgfpathlineto{\pgfqpoint{1.595155in}{1.572397in}}%
\pgfpathlineto{\pgfqpoint{1.596886in}{1.526867in}}%
\pgfpathlineto{\pgfqpoint{1.598615in}{1.637639in}}%
\pgfpathlineto{\pgfqpoint{1.599480in}{1.527160in}}%
\pgfpathlineto{\pgfqpoint{1.601207in}{1.591275in}}%
\pgfpathlineto{\pgfqpoint{1.602071in}{1.587892in}}%
\pgfpathlineto{\pgfqpoint{1.603803in}{1.525677in}}%
\pgfpathlineto{\pgfqpoint{1.604669in}{1.598072in}}%
\pgfpathlineto{\pgfqpoint{1.605535in}{1.575424in}}%
\pgfpathlineto{\pgfqpoint{1.608130in}{1.627369in}}%
\pgfpathlineto{\pgfqpoint{1.609860in}{1.517396in}}%
\pgfpathlineto{\pgfqpoint{1.610725in}{1.581243in}}%
\pgfpathlineto{\pgfqpoint{1.611590in}{1.564266in}}%
\pgfpathlineto{\pgfqpoint{1.614186in}{1.622944in}}%
\pgfpathlineto{\pgfqpoint{1.615051in}{1.556487in}}%
\pgfpathlineto{\pgfqpoint{1.615917in}{1.644049in}}%
\pgfpathlineto{\pgfqpoint{1.616782in}{1.597596in}}%
\pgfpathlineto{\pgfqpoint{1.617645in}{1.662987in}}%
\pgfpathlineto{\pgfqpoint{1.618510in}{1.611637in}}%
\pgfpathlineto{\pgfqpoint{1.619375in}{1.628704in}}%
\pgfpathlineto{\pgfqpoint{1.620240in}{1.622026in}}%
\pgfpathlineto{\pgfqpoint{1.621104in}{1.632415in}}%
\pgfpathlineto{\pgfqpoint{1.621969in}{1.581184in}}%
\pgfpathlineto{\pgfqpoint{1.622835in}{1.638587in}}%
\pgfpathlineto{\pgfqpoint{1.623700in}{1.544376in}}%
\pgfpathlineto{\pgfqpoint{1.625428in}{1.634196in}}%
\pgfpathlineto{\pgfqpoint{1.626293in}{1.584924in}}%
\pgfpathlineto{\pgfqpoint{1.628886in}{1.657971in}}%
\pgfpathlineto{\pgfqpoint{1.629752in}{1.593711in}}%
\pgfpathlineto{\pgfqpoint{1.630617in}{1.641320in}}%
\pgfpathlineto{\pgfqpoint{1.631481in}{1.631110in}}%
\pgfpathlineto{\pgfqpoint{1.632344in}{1.589970in}}%
\pgfpathlineto{\pgfqpoint{1.633210in}{1.608759in}}%
\pgfpathlineto{\pgfqpoint{1.634073in}{1.607126in}}%
\pgfpathlineto{\pgfqpoint{1.634936in}{1.615556in}}%
\pgfpathlineto{\pgfqpoint{1.635803in}{1.614281in}}%
\pgfpathlineto{\pgfqpoint{1.636668in}{1.590327in}}%
\pgfpathlineto{\pgfqpoint{1.639263in}{1.668627in}}%
\pgfpathlineto{\pgfqpoint{1.642725in}{1.562931in}}%
\pgfpathlineto{\pgfqpoint{1.644456in}{1.626123in}}%
\pgfpathlineto{\pgfqpoint{1.645321in}{1.586319in}}%
\pgfpathlineto{\pgfqpoint{1.647053in}{1.657822in}}%
\pgfpathlineto{\pgfqpoint{1.647918in}{1.571330in}}%
\pgfpathlineto{\pgfqpoint{1.648784in}{1.588487in}}%
\pgfpathlineto{\pgfqpoint{1.650509in}{1.609205in}}%
\pgfpathlineto{\pgfqpoint{1.651372in}{1.570973in}}%
\pgfpathlineto{\pgfqpoint{1.652236in}{1.579641in}}%
\pgfpathlineto{\pgfqpoint{1.653101in}{1.655981in}}%
\pgfpathlineto{\pgfqpoint{1.653965in}{1.608551in}}%
\pgfpathlineto{\pgfqpoint{1.654830in}{1.617456in}}%
\pgfpathlineto{\pgfqpoint{1.655694in}{1.664827in}}%
\pgfpathlineto{\pgfqpoint{1.657422in}{1.577830in}}%
\pgfpathlineto{\pgfqpoint{1.658286in}{1.591751in}}%
\pgfpathlineto{\pgfqpoint{1.659151in}{1.596975in}}%
\pgfpathlineto{\pgfqpoint{1.660879in}{1.554527in}}%
\pgfpathlineto{\pgfqpoint{1.661745in}{1.606978in}}%
\pgfpathlineto{\pgfqpoint{1.662610in}{1.560346in}}%
\pgfpathlineto{\pgfqpoint{1.663476in}{1.575781in}}%
\pgfpathlineto{\pgfqpoint{1.664342in}{1.562276in}}%
\pgfpathlineto{\pgfqpoint{1.666070in}{1.618404in}}%
\pgfpathlineto{\pgfqpoint{1.667800in}{1.530961in}}%
\pgfpathlineto{\pgfqpoint{1.668666in}{1.579224in}}%
\pgfpathlineto{\pgfqpoint{1.670393in}{1.526331in}}%
\pgfpathlineto{\pgfqpoint{1.671259in}{1.538144in}}%
\pgfpathlineto{\pgfqpoint{1.672989in}{1.620959in}}%
\pgfpathlineto{\pgfqpoint{1.674716in}{1.602200in}}%
\pgfpathlineto{\pgfqpoint{1.675581in}{1.570438in}}%
\pgfpathlineto{\pgfqpoint{1.676445in}{1.640487in}}%
\pgfpathlineto{\pgfqpoint{1.678175in}{1.584627in}}%
\pgfpathlineto{\pgfqpoint{1.679039in}{1.623866in}}%
\pgfpathlineto{\pgfqpoint{1.679905in}{1.553877in}}%
\pgfpathlineto{\pgfqpoint{1.680770in}{1.560436in}}%
\pgfpathlineto{\pgfqpoint{1.681635in}{1.651174in}}%
\pgfpathlineto{\pgfqpoint{1.683364in}{1.593562in}}%
\pgfpathlineto{\pgfqpoint{1.684229in}{1.696113in}}%
\pgfpathlineto{\pgfqpoint{1.685094in}{1.560644in}}%
\pgfpathlineto{\pgfqpoint{1.685959in}{1.649839in}}%
\pgfpathlineto{\pgfqpoint{1.686825in}{1.543666in}}%
\pgfpathlineto{\pgfqpoint{1.688554in}{1.591037in}}%
\pgfpathlineto{\pgfqpoint{1.689420in}{1.575662in}}%
\pgfpathlineto{\pgfqpoint{1.690285in}{1.516121in}}%
\pgfpathlineto{\pgfqpoint{1.692015in}{1.600300in}}%
\pgfpathlineto{\pgfqpoint{1.692880in}{1.493235in}}%
\pgfpathlineto{\pgfqpoint{1.693742in}{1.603858in}}%
\pgfpathlineto{\pgfqpoint{1.695472in}{1.550047in}}%
\pgfpathlineto{\pgfqpoint{1.696337in}{1.633660in}}%
\pgfpathlineto{\pgfqpoint{1.698064in}{1.570378in}}%
\pgfpathlineto{\pgfqpoint{1.698929in}{1.568066in}}%
\pgfpathlineto{\pgfqpoint{1.700660in}{1.622978in}}%
\pgfpathlineto{\pgfqpoint{1.702389in}{1.528380in}}%
\pgfpathlineto{\pgfqpoint{1.703252in}{1.574476in}}%
\pgfpathlineto{\pgfqpoint{1.704981in}{1.498548in}}%
\pgfpathlineto{\pgfqpoint{1.706709in}{1.615794in}}%
\pgfpathlineto{\pgfqpoint{1.707574in}{1.566995in}}%
\pgfpathlineto{\pgfqpoint{1.708439in}{1.622204in}}%
\pgfpathlineto{\pgfqpoint{1.709304in}{1.599972in}}%
\pgfpathlineto{\pgfqpoint{1.710169in}{1.613299in}}%
\pgfpathlineto{\pgfqpoint{1.711035in}{1.649452in}}%
\pgfpathlineto{\pgfqpoint{1.711900in}{1.632474in}}%
\pgfpathlineto{\pgfqpoint{1.712763in}{1.737372in}}%
\pgfpathlineto{\pgfqpoint{1.713627in}{1.640074in}}%
\pgfpathlineto{\pgfqpoint{1.714491in}{1.785933in}}%
\pgfpathlineto{\pgfqpoint{1.715355in}{1.712202in}}%
\pgfpathlineto{\pgfqpoint{1.716220in}{1.715467in}}%
\pgfpathlineto{\pgfqpoint{1.717949in}{1.652007in}}%
\pgfpathlineto{\pgfqpoint{1.718815in}{1.681154in}}%
\pgfpathlineto{\pgfqpoint{1.720545in}{1.656695in}}%
\pgfpathlineto{\pgfqpoint{1.723138in}{1.679314in}}%
\pgfpathlineto{\pgfqpoint{1.724003in}{1.672368in}}%
\pgfpathlineto{\pgfqpoint{1.725732in}{1.752568in}}%
\pgfpathlineto{\pgfqpoint{1.726596in}{1.652661in}}%
\pgfpathlineto{\pgfqpoint{1.727461in}{1.718613in}}%
\pgfpathlineto{\pgfqpoint{1.729191in}{1.622442in}}%
\pgfpathlineto{\pgfqpoint{1.730056in}{1.687327in}}%
\pgfpathlineto{\pgfqpoint{1.730922in}{1.654825in}}%
\pgfpathlineto{\pgfqpoint{1.732650in}{1.704543in}}%
\pgfpathlineto{\pgfqpoint{1.733516in}{1.641528in}}%
\pgfpathlineto{\pgfqpoint{1.735246in}{1.727875in}}%
\pgfpathlineto{\pgfqpoint{1.736112in}{1.644232in}}%
\pgfpathlineto{\pgfqpoint{1.736974in}{1.698668in}}%
\pgfpathlineto{\pgfqpoint{1.738706in}{1.611284in}}%
\pgfpathlineto{\pgfqpoint{1.739570in}{1.625116in}}%
\pgfpathlineto{\pgfqpoint{1.740436in}{1.637077in}}%
\pgfpathlineto{\pgfqpoint{1.741299in}{1.620248in}}%
\pgfpathlineto{\pgfqpoint{1.742164in}{1.641737in}}%
\pgfpathlineto{\pgfqpoint{1.743891in}{1.528410in}}%
\pgfpathlineto{\pgfqpoint{1.744756in}{1.564444in}}%
\pgfpathlineto{\pgfqpoint{1.745620in}{1.557855in}}%
\pgfpathlineto{\pgfqpoint{1.746486in}{1.574357in}}%
\pgfpathlineto{\pgfqpoint{1.747352in}{1.542212in}}%
\pgfpathlineto{\pgfqpoint{1.748217in}{1.554709in}}%
\pgfpathlineto{\pgfqpoint{1.749082in}{1.634791in}}%
\pgfpathlineto{\pgfqpoint{1.750814in}{1.572992in}}%
\pgfpathlineto{\pgfqpoint{1.751679in}{1.578217in}}%
\pgfpathlineto{\pgfqpoint{1.752545in}{1.605702in}}%
\pgfpathlineto{\pgfqpoint{1.754272in}{1.571092in}}%
\pgfpathlineto{\pgfqpoint{1.756004in}{1.635977in}}%
\pgfpathlineto{\pgfqpoint{1.756870in}{1.627220in}}%
\pgfpathlineto{\pgfqpoint{1.757735in}{1.580946in}}%
\pgfpathlineto{\pgfqpoint{1.758602in}{1.588011in}}%
\pgfpathlineto{\pgfqpoint{1.759468in}{1.598400in}}%
\pgfpathlineto{\pgfqpoint{1.760334in}{1.568954in}}%
\pgfpathlineto{\pgfqpoint{1.761200in}{1.603445in}}%
\pgfpathlineto{\pgfqpoint{1.762066in}{1.594778in}}%
\pgfpathlineto{\pgfqpoint{1.762931in}{1.574535in}}%
\pgfpathlineto{\pgfqpoint{1.763797in}{1.609146in}}%
\pgfpathlineto{\pgfqpoint{1.764663in}{1.528112in}}%
\pgfpathlineto{\pgfqpoint{1.765529in}{1.576019in}}%
\pgfpathlineto{\pgfqpoint{1.766394in}{1.553014in}}%
\pgfpathlineto{\pgfqpoint{1.767259in}{1.611815in}}%
\pgfpathlineto{\pgfqpoint{1.768988in}{1.540372in}}%
\pgfpathlineto{\pgfqpoint{1.769853in}{1.621197in}}%
\pgfpathlineto{\pgfqpoint{1.770717in}{1.571687in}}%
\pgfpathlineto{\pgfqpoint{1.772446in}{1.595313in}}%
\pgfpathlineto{\pgfqpoint{1.773309in}{1.548890in}}%
\pgfpathlineto{\pgfqpoint{1.774173in}{1.585932in}}%
\pgfpathlineto{\pgfqpoint{1.775903in}{1.560882in}}%
\pgfpathlineto{\pgfqpoint{1.776768in}{1.632772in}}%
\pgfpathlineto{\pgfqpoint{1.778496in}{1.539687in}}%
\pgfpathlineto{\pgfqpoint{1.779358in}{1.625647in}}%
\pgfpathlineto{\pgfqpoint{1.780222in}{1.554115in}}%
\pgfpathlineto{\pgfqpoint{1.782816in}{1.665125in}}%
\pgfpathlineto{\pgfqpoint{1.783681in}{1.557498in}}%
\pgfpathlineto{\pgfqpoint{1.784545in}{1.584389in}}%
\pgfpathlineto{\pgfqpoint{1.785410in}{1.542063in}}%
\pgfpathlineto{\pgfqpoint{1.786275in}{1.610510in}}%
\pgfpathlineto{\pgfqpoint{1.787138in}{1.589970in}}%
\pgfpathlineto{\pgfqpoint{1.788003in}{1.620483in}}%
\pgfpathlineto{\pgfqpoint{1.788868in}{1.504546in}}%
\pgfpathlineto{\pgfqpoint{1.789732in}{1.631764in}}%
\pgfpathlineto{\pgfqpoint{1.791463in}{1.536423in}}%
\pgfpathlineto{\pgfqpoint{1.792326in}{1.598519in}}%
\pgfpathlineto{\pgfqpoint{1.793189in}{1.526510in}}%
\pgfpathlineto{\pgfqpoint{1.794054in}{1.576852in}}%
\pgfpathlineto{\pgfqpoint{1.794918in}{1.569103in}}%
\pgfpathlineto{\pgfqpoint{1.796649in}{1.601843in}}%
\pgfpathlineto{\pgfqpoint{1.798377in}{1.549188in}}%
\pgfpathlineto{\pgfqpoint{1.799241in}{1.594778in}}%
\pgfpathlineto{\pgfqpoint{1.800107in}{1.534344in}}%
\pgfpathlineto{\pgfqpoint{1.801837in}{1.622264in}}%
\pgfpathlineto{\pgfqpoint{1.802704in}{1.582489in}}%
\pgfpathlineto{\pgfqpoint{1.803570in}{1.644406in}}%
\pgfpathlineto{\pgfqpoint{1.805299in}{1.564087in}}%
\pgfpathlineto{\pgfqpoint{1.806165in}{1.631169in}}%
\pgfpathlineto{\pgfqpoint{1.807031in}{1.606561in}}%
\pgfpathlineto{\pgfqpoint{1.807897in}{1.548474in}}%
\pgfpathlineto{\pgfqpoint{1.808763in}{1.603029in}}%
\pgfpathlineto{\pgfqpoint{1.810493in}{1.540104in}}%
\pgfpathlineto{\pgfqpoint{1.811358in}{1.552690in}}%
\pgfpathlineto{\pgfqpoint{1.812222in}{1.540580in}}%
\pgfpathlineto{\pgfqpoint{1.813089in}{1.589732in}}%
\pgfpathlineto{\pgfqpoint{1.814820in}{1.528291in}}%
\pgfpathlineto{\pgfqpoint{1.815683in}{1.566582in}}%
\pgfpathlineto{\pgfqpoint{1.816549in}{1.540967in}}%
\pgfpathlineto{\pgfqpoint{1.817413in}{1.603981in}}%
\pgfpathlineto{\pgfqpoint{1.819140in}{1.532121in}}%
\pgfpathlineto{\pgfqpoint{1.820005in}{1.625588in}}%
\pgfpathlineto{\pgfqpoint{1.821736in}{1.522948in}}%
\pgfpathlineto{\pgfqpoint{1.822603in}{1.616742in}}%
\pgfpathlineto{\pgfqpoint{1.823469in}{1.560436in}}%
\pgfpathlineto{\pgfqpoint{1.826063in}{1.659752in}}%
\pgfpathlineto{\pgfqpoint{1.826928in}{1.656695in}}%
\pgfpathlineto{\pgfqpoint{1.827794in}{1.687267in}}%
\pgfpathlineto{\pgfqpoint{1.828657in}{1.620364in}}%
\pgfpathlineto{\pgfqpoint{1.829522in}{1.620780in}}%
\pgfpathlineto{\pgfqpoint{1.830388in}{1.670289in}}%
\pgfpathlineto{\pgfqpoint{1.831253in}{1.572754in}}%
\pgfpathlineto{\pgfqpoint{1.832114in}{1.665660in}}%
\pgfpathlineto{\pgfqpoint{1.832979in}{1.639480in}}%
\pgfpathlineto{\pgfqpoint{1.833845in}{1.628202in}}%
\pgfpathlineto{\pgfqpoint{1.834712in}{1.577030in}}%
\pgfpathlineto{\pgfqpoint{1.835577in}{1.585490in}}%
\pgfpathlineto{\pgfqpoint{1.836442in}{1.698906in}}%
\pgfpathlineto{\pgfqpoint{1.838171in}{1.623662in}}%
\pgfpathlineto{\pgfqpoint{1.839899in}{1.683411in}}%
\pgfpathlineto{\pgfqpoint{1.842494in}{1.553166in}}%
\pgfpathlineto{\pgfqpoint{1.844224in}{1.654026in}}%
\pgfpathlineto{\pgfqpoint{1.845089in}{1.605435in}}%
\pgfpathlineto{\pgfqpoint{1.845955in}{1.660376in}}%
\pgfpathlineto{\pgfqpoint{1.846820in}{1.636691in}}%
\pgfpathlineto{\pgfqpoint{1.847684in}{1.568155in}}%
\pgfpathlineto{\pgfqpoint{1.848546in}{1.586825in}}%
\pgfpathlineto{\pgfqpoint{1.849412in}{1.614548in}}%
\pgfpathlineto{\pgfqpoint{1.850276in}{1.549069in}}%
\pgfpathlineto{\pgfqpoint{1.851141in}{1.651709in}}%
\pgfpathlineto{\pgfqpoint{1.852870in}{1.533218in}}%
\pgfpathlineto{\pgfqpoint{1.853735in}{1.584865in}}%
\pgfpathlineto{\pgfqpoint{1.854601in}{1.564117in}}%
\pgfpathlineto{\pgfqpoint{1.856331in}{1.622502in}}%
\pgfpathlineto{\pgfqpoint{1.857196in}{1.555479in}}%
\pgfpathlineto{\pgfqpoint{1.858925in}{1.584389in}}%
\pgfpathlineto{\pgfqpoint{1.859790in}{1.565928in}}%
\pgfpathlineto{\pgfqpoint{1.860656in}{1.588070in}}%
\pgfpathlineto{\pgfqpoint{1.862388in}{1.691365in}}%
\pgfpathlineto{\pgfqpoint{1.863254in}{1.697597in}}%
\pgfpathlineto{\pgfqpoint{1.864120in}{1.577711in}}%
\pgfpathlineto{\pgfqpoint{1.865845in}{1.625171in}}%
\pgfpathlineto{\pgfqpoint{1.867573in}{1.544674in}}%
\pgfpathlineto{\pgfqpoint{1.869304in}{1.630515in}}%
\pgfpathlineto{\pgfqpoint{1.870168in}{1.511904in}}%
\pgfpathlineto{\pgfqpoint{1.871897in}{1.582429in}}%
\pgfpathlineto{\pgfqpoint{1.872762in}{1.555241in}}%
\pgfpathlineto{\pgfqpoint{1.873626in}{1.615020in}}%
\pgfpathlineto{\pgfqpoint{1.874493in}{1.538620in}}%
\pgfpathlineto{\pgfqpoint{1.875359in}{1.553639in}}%
\pgfpathlineto{\pgfqpoint{1.877089in}{1.541647in}}%
\pgfpathlineto{\pgfqpoint{1.877953in}{1.566106in}}%
\pgfpathlineto{\pgfqpoint{1.878816in}{1.499202in}}%
\pgfpathlineto{\pgfqpoint{1.879681in}{1.584389in}}%
\pgfpathlineto{\pgfqpoint{1.880547in}{1.553490in}}%
\pgfpathlineto{\pgfqpoint{1.883143in}{1.639420in}}%
\pgfpathlineto{\pgfqpoint{1.884008in}{1.617158in}}%
\pgfpathlineto{\pgfqpoint{1.884871in}{1.640963in}}%
\pgfpathlineto{\pgfqpoint{1.885735in}{1.610183in}}%
\pgfpathlineto{\pgfqpoint{1.887463in}{1.651828in}}%
\pgfpathlineto{\pgfqpoint{1.890054in}{1.566493in}}%
\pgfpathlineto{\pgfqpoint{1.890918in}{1.605583in}}%
\pgfpathlineto{\pgfqpoint{1.891784in}{1.549128in}}%
\pgfpathlineto{\pgfqpoint{1.892650in}{1.555568in}}%
\pgfpathlineto{\pgfqpoint{1.893515in}{1.550433in}}%
\pgfpathlineto{\pgfqpoint{1.894381in}{1.604988in}}%
\pgfpathlineto{\pgfqpoint{1.895246in}{1.593146in}}%
\pgfpathlineto{\pgfqpoint{1.896112in}{1.545031in}}%
\pgfpathlineto{\pgfqpoint{1.897841in}{1.590089in}}%
\pgfpathlineto{\pgfqpoint{1.898707in}{1.553996in}}%
\pgfpathlineto{\pgfqpoint{1.900437in}{1.603743in}}%
\pgfpathlineto{\pgfqpoint{1.901303in}{1.571033in}}%
\pgfpathlineto{\pgfqpoint{1.902169in}{1.617099in}}%
\pgfpathlineto{\pgfqpoint{1.903034in}{1.605345in}}%
\pgfpathlineto{\pgfqpoint{1.904764in}{1.574298in}}%
\pgfpathlineto{\pgfqpoint{1.907360in}{1.594243in}}%
\pgfpathlineto{\pgfqpoint{1.908225in}{1.563909in}}%
\pgfpathlineto{\pgfqpoint{1.909953in}{1.634196in}}%
\pgfpathlineto{\pgfqpoint{1.910818in}{1.519266in}}%
\pgfpathlineto{\pgfqpoint{1.911684in}{1.600891in}}%
\pgfpathlineto{\pgfqpoint{1.912550in}{1.566757in}}%
\pgfpathlineto{\pgfqpoint{1.913415in}{1.606413in}}%
\pgfpathlineto{\pgfqpoint{1.915146in}{1.586349in}}%
\pgfpathlineto{\pgfqpoint{1.916877in}{1.616742in}}%
\pgfpathlineto{\pgfqpoint{1.917742in}{1.601426in}}%
\pgfpathlineto{\pgfqpoint{1.919471in}{1.551857in}}%
\pgfpathlineto{\pgfqpoint{1.921200in}{1.598221in}}%
\pgfpathlineto{\pgfqpoint{1.922931in}{1.539747in}}%
\pgfpathlineto{\pgfqpoint{1.923796in}{1.582013in}}%
\pgfpathlineto{\pgfqpoint{1.924661in}{1.549957in}}%
\pgfpathlineto{\pgfqpoint{1.925527in}{1.585575in}}%
\pgfpathlineto{\pgfqpoint{1.927259in}{1.517664in}}%
\pgfpathlineto{\pgfqpoint{1.928985in}{1.618464in}}%
\pgfpathlineto{\pgfqpoint{1.929850in}{1.610153in}}%
\pgfpathlineto{\pgfqpoint{1.930715in}{1.630366in}}%
\pgfpathlineto{\pgfqpoint{1.931580in}{1.613537in}}%
\pgfpathlineto{\pgfqpoint{1.932446in}{1.643280in}}%
\pgfpathlineto{\pgfqpoint{1.933312in}{1.537996in}}%
\pgfpathlineto{\pgfqpoint{1.935043in}{1.586230in}}%
\pgfpathlineto{\pgfqpoint{1.936775in}{1.622561in}}%
\pgfpathlineto{\pgfqpoint{1.937642in}{1.648147in}}%
\pgfpathlineto{\pgfqpoint{1.938507in}{1.611165in}}%
\pgfpathlineto{\pgfqpoint{1.939373in}{1.638174in}}%
\pgfpathlineto{\pgfqpoint{1.941967in}{1.535475in}}%
\pgfpathlineto{\pgfqpoint{1.944561in}{1.642510in}}%
\pgfpathlineto{\pgfqpoint{1.946290in}{1.580593in}}%
\pgfpathlineto{\pgfqpoint{1.947154in}{1.592465in}}%
\pgfpathlineto{\pgfqpoint{1.948018in}{1.509948in}}%
\pgfpathlineto{\pgfqpoint{1.949748in}{1.632151in}}%
\pgfpathlineto{\pgfqpoint{1.951475in}{1.623394in}}%
\pgfpathlineto{\pgfqpoint{1.952341in}{1.626748in}}%
\pgfpathlineto{\pgfqpoint{1.954072in}{1.695343in}}%
\pgfpathlineto{\pgfqpoint{1.954936in}{1.669698in}}%
\pgfpathlineto{\pgfqpoint{1.955801in}{1.729418in}}%
\pgfpathlineto{\pgfqpoint{1.956666in}{1.679909in}}%
\pgfpathlineto{\pgfqpoint{1.957531in}{1.687271in}}%
\pgfpathlineto{\pgfqpoint{1.959262in}{1.709443in}}%
\pgfpathlineto{\pgfqpoint{1.962723in}{1.617872in}}%
\pgfpathlineto{\pgfqpoint{1.963588in}{1.708402in}}%
\pgfpathlineto{\pgfqpoint{1.964454in}{1.675454in}}%
\pgfpathlineto{\pgfqpoint{1.966183in}{1.732088in}}%
\pgfpathlineto{\pgfqpoint{1.968778in}{1.655241in}}%
\pgfpathlineto{\pgfqpoint{1.969644in}{1.691900in}}%
\pgfpathlineto{\pgfqpoint{1.970509in}{1.670587in}}%
\pgfpathlineto{\pgfqpoint{1.972240in}{1.572576in}}%
\pgfpathlineto{\pgfqpoint{1.973106in}{1.555126in}}%
\pgfpathlineto{\pgfqpoint{1.974835in}{1.612291in}}%
\pgfpathlineto{\pgfqpoint{1.975702in}{1.556134in}}%
\pgfpathlineto{\pgfqpoint{1.977433in}{1.631407in}}%
\pgfpathlineto{\pgfqpoint{1.978298in}{1.564117in}}%
\pgfpathlineto{\pgfqpoint{1.980028in}{1.646544in}}%
\pgfpathlineto{\pgfqpoint{1.980894in}{1.621226in}}%
\pgfpathlineto{\pgfqpoint{1.981760in}{1.583500in}}%
\pgfpathlineto{\pgfqpoint{1.983491in}{1.675514in}}%
\pgfpathlineto{\pgfqpoint{1.986085in}{1.535594in}}%
\pgfpathlineto{\pgfqpoint{1.989540in}{1.629329in}}%
\pgfpathlineto{\pgfqpoint{1.990405in}{1.634701in}}%
\pgfpathlineto{\pgfqpoint{1.991267in}{1.618110in}}%
\pgfpathlineto{\pgfqpoint{1.992132in}{1.656755in}}%
\pgfpathlineto{\pgfqpoint{1.995590in}{1.534407in}}%
\pgfpathlineto{\pgfqpoint{1.997321in}{1.616181in}}%
\pgfpathlineto{\pgfqpoint{1.999051in}{1.575487in}}%
\pgfpathlineto{\pgfqpoint{1.999915in}{1.627135in}}%
\pgfpathlineto{\pgfqpoint{2.000777in}{1.604516in}}%
\pgfpathlineto{\pgfqpoint{2.001642in}{1.640907in}}%
\pgfpathlineto{\pgfqpoint{2.002508in}{1.563674in}}%
\pgfpathlineto{\pgfqpoint{2.003373in}{1.614965in}}%
\pgfpathlineto{\pgfqpoint{2.004238in}{1.553375in}}%
\pgfpathlineto{\pgfqpoint{2.005103in}{1.606416in}}%
\pgfpathlineto{\pgfqpoint{2.006833in}{1.553996in}}%
\pgfpathlineto{\pgfqpoint{2.007697in}{1.636810in}}%
\pgfpathlineto{\pgfqpoint{2.009429in}{1.547317in}}%
\pgfpathlineto{\pgfqpoint{2.011158in}{1.659071in}}%
\pgfpathlineto{\pgfqpoint{2.012023in}{1.604219in}}%
\pgfpathlineto{\pgfqpoint{2.012887in}{1.647496in}}%
\pgfpathlineto{\pgfqpoint{2.013752in}{1.635148in}}%
\pgfpathlineto{\pgfqpoint{2.014617in}{1.546280in}}%
\pgfpathlineto{\pgfqpoint{2.015482in}{1.565098in}}%
\pgfpathlineto{\pgfqpoint{2.016347in}{1.571539in}}%
\pgfpathlineto{\pgfqpoint{2.017211in}{1.527343in}}%
\pgfpathlineto{\pgfqpoint{2.018077in}{1.590744in}}%
\pgfpathlineto{\pgfqpoint{2.018939in}{1.532388in}}%
\pgfpathlineto{\pgfqpoint{2.020669in}{1.615913in}}%
\pgfpathlineto{\pgfqpoint{2.022400in}{1.564860in}}%
\pgfpathlineto{\pgfqpoint{2.024131in}{1.613180in}}%
\pgfpathlineto{\pgfqpoint{2.024995in}{1.612589in}}%
\pgfpathlineto{\pgfqpoint{2.025860in}{1.606000in}}%
\pgfpathlineto{\pgfqpoint{2.026727in}{1.576852in}}%
\pgfpathlineto{\pgfqpoint{2.027592in}{1.589201in}}%
\pgfpathlineto{\pgfqpoint{2.028455in}{1.633515in}}%
\pgfpathlineto{\pgfqpoint{2.029320in}{1.559815in}}%
\pgfpathlineto{\pgfqpoint{2.031052in}{1.662336in}}%
\pgfpathlineto{\pgfqpoint{2.032782in}{1.576614in}}%
\pgfpathlineto{\pgfqpoint{2.033646in}{1.581838in}}%
\pgfpathlineto{\pgfqpoint{2.035374in}{1.642566in}}%
\pgfpathlineto{\pgfqpoint{2.036239in}{1.641677in}}%
\pgfpathlineto{\pgfqpoint{2.037104in}{1.593532in}}%
\pgfpathlineto{\pgfqpoint{2.037969in}{1.662068in}}%
\pgfpathlineto{\pgfqpoint{2.039700in}{1.617396in}}%
\pgfpathlineto{\pgfqpoint{2.040565in}{1.627428in}}%
\pgfpathlineto{\pgfqpoint{2.041430in}{1.605464in}}%
\pgfpathlineto{\pgfqpoint{2.042296in}{1.665601in}}%
\pgfpathlineto{\pgfqpoint{2.043162in}{1.614013in}}%
\pgfpathlineto{\pgfqpoint{2.044026in}{1.683884in}}%
\pgfpathlineto{\pgfqpoint{2.044891in}{1.652836in}}%
\pgfpathlineto{\pgfqpoint{2.045756in}{1.676224in}}%
\pgfpathlineto{\pgfqpoint{2.047484in}{1.553252in}}%
\pgfpathlineto{\pgfqpoint{2.049214in}{1.666132in}}%
\pgfpathlineto{\pgfqpoint{2.050080in}{1.635263in}}%
\pgfpathlineto{\pgfqpoint{2.050945in}{1.585754in}}%
\pgfpathlineto{\pgfqpoint{2.051810in}{1.666549in}}%
\pgfpathlineto{\pgfqpoint{2.052674in}{1.613358in}}%
\pgfpathlineto{\pgfqpoint{2.053538in}{1.621907in}}%
\pgfpathlineto{\pgfqpoint{2.054402in}{1.640844in}}%
\pgfpathlineto{\pgfqpoint{2.056131in}{1.565511in}}%
\pgfpathlineto{\pgfqpoint{2.056996in}{1.606710in}}%
\pgfpathlineto{\pgfqpoint{2.057861in}{1.591989in}}%
\pgfpathlineto{\pgfqpoint{2.058726in}{1.600776in}}%
\pgfpathlineto{\pgfqpoint{2.060456in}{1.647020in}}%
\pgfpathlineto{\pgfqpoint{2.061322in}{1.612618in}}%
\pgfpathlineto{\pgfqpoint{2.062187in}{1.525264in}}%
\pgfpathlineto{\pgfqpoint{2.063052in}{1.558034in}}%
\pgfpathlineto{\pgfqpoint{2.063916in}{1.540550in}}%
\pgfpathlineto{\pgfqpoint{2.065646in}{1.622085in}}%
\pgfpathlineto{\pgfqpoint{2.068243in}{1.570676in}}%
\pgfpathlineto{\pgfqpoint{2.069109in}{1.557379in}}%
\pgfpathlineto{\pgfqpoint{2.069974in}{1.572576in}}%
\pgfpathlineto{\pgfqpoint{2.070838in}{1.535058in}}%
\pgfpathlineto{\pgfqpoint{2.071701in}{1.645061in}}%
\pgfpathlineto{\pgfqpoint{2.072565in}{1.546752in}}%
\pgfpathlineto{\pgfqpoint{2.074294in}{1.607662in}}%
\pgfpathlineto{\pgfqpoint{2.075160in}{1.575190in}}%
\pgfpathlineto{\pgfqpoint{2.076025in}{1.638085in}}%
\pgfpathlineto{\pgfqpoint{2.076890in}{1.516537in}}%
\pgfpathlineto{\pgfqpoint{2.077754in}{1.545090in}}%
\pgfpathlineto{\pgfqpoint{2.079483in}{1.643399in}}%
\pgfpathlineto{\pgfqpoint{2.080348in}{1.600300in}}%
\pgfpathlineto{\pgfqpoint{2.081213in}{1.610629in}}%
\pgfpathlineto{\pgfqpoint{2.082078in}{1.632950in}}%
\pgfpathlineto{\pgfqpoint{2.084671in}{1.588844in}}%
\pgfpathlineto{\pgfqpoint{2.086401in}{1.645001in}}%
\pgfpathlineto{\pgfqpoint{2.088130in}{1.598935in}}%
\pgfpathlineto{\pgfqpoint{2.088995in}{1.643339in}}%
\pgfpathlineto{\pgfqpoint{2.089859in}{1.585073in}}%
\pgfpathlineto{\pgfqpoint{2.090724in}{1.639599in}}%
\pgfpathlineto{\pgfqpoint{2.091589in}{1.620308in}}%
\pgfpathlineto{\pgfqpoint{2.092454in}{1.519032in}}%
\pgfpathlineto{\pgfqpoint{2.094184in}{1.676109in}}%
\pgfpathlineto{\pgfqpoint{2.095914in}{1.573766in}}%
\pgfpathlineto{\pgfqpoint{2.096780in}{1.661622in}}%
\pgfpathlineto{\pgfqpoint{2.098507in}{1.588427in}}%
\pgfpathlineto{\pgfqpoint{2.099373in}{1.620483in}}%
\pgfpathlineto{\pgfqpoint{2.100239in}{1.600240in}}%
\pgfpathlineto{\pgfqpoint{2.101101in}{1.659900in}}%
\pgfpathlineto{\pgfqpoint{2.102828in}{1.592997in}}%
\pgfpathlineto{\pgfqpoint{2.103692in}{1.597745in}}%
\pgfpathlineto{\pgfqpoint{2.106286in}{1.577800in}}%
\pgfpathlineto{\pgfqpoint{2.107151in}{1.533396in}}%
\pgfpathlineto{\pgfqpoint{2.109746in}{1.621018in}}%
\pgfpathlineto{\pgfqpoint{2.110611in}{1.609384in}}%
\pgfpathlineto{\pgfqpoint{2.112342in}{1.552452in}}%
\pgfpathlineto{\pgfqpoint{2.113207in}{1.598638in}}%
\pgfpathlineto{\pgfqpoint{2.114938in}{1.528410in}}%
\pgfpathlineto{\pgfqpoint{2.115804in}{1.606591in}}%
\pgfpathlineto{\pgfqpoint{2.118403in}{1.541439in}}%
\pgfpathlineto{\pgfqpoint{2.120134in}{1.635560in}}%
\pgfpathlineto{\pgfqpoint{2.120999in}{1.611250in}}%
\pgfpathlineto{\pgfqpoint{2.122731in}{1.634196in}}%
\pgfpathlineto{\pgfqpoint{2.123596in}{1.551798in}}%
\pgfpathlineto{\pgfqpoint{2.125328in}{1.587178in}}%
\pgfpathlineto{\pgfqpoint{2.126193in}{1.558208in}}%
\pgfpathlineto{\pgfqpoint{2.128790in}{1.592699in}}%
\pgfpathlineto{\pgfqpoint{2.131385in}{1.521405in}}%
\pgfpathlineto{\pgfqpoint{2.133116in}{1.635382in}}%
\pgfpathlineto{\pgfqpoint{2.133982in}{1.639182in}}%
\pgfpathlineto{\pgfqpoint{2.135713in}{1.606413in}}%
\pgfpathlineto{\pgfqpoint{2.136575in}{1.624164in}}%
\pgfpathlineto{\pgfqpoint{2.137440in}{1.573911in}}%
\pgfpathlineto{\pgfqpoint{2.138303in}{1.673852in}}%
\pgfpathlineto{\pgfqpoint{2.139169in}{1.661444in}}%
\pgfpathlineto{\pgfqpoint{2.140031in}{1.675811in}}%
\pgfpathlineto{\pgfqpoint{2.141763in}{1.598757in}}%
\pgfpathlineto{\pgfqpoint{2.142629in}{1.601902in}}%
\pgfpathlineto{\pgfqpoint{2.143495in}{1.625647in}}%
\pgfpathlineto{\pgfqpoint{2.145225in}{1.598578in}}%
\pgfpathlineto{\pgfqpoint{2.146091in}{1.610748in}}%
\pgfpathlineto{\pgfqpoint{2.147822in}{1.567709in}}%
\pgfpathlineto{\pgfqpoint{2.149554in}{1.639985in}}%
\pgfpathlineto{\pgfqpoint{2.150418in}{1.659960in}}%
\pgfpathlineto{\pgfqpoint{2.152149in}{1.572933in}}%
\pgfpathlineto{\pgfqpoint{2.153015in}{1.587479in}}%
\pgfpathlineto{\pgfqpoint{2.153879in}{1.650225in}}%
\pgfpathlineto{\pgfqpoint{2.154743in}{1.648682in}}%
\pgfpathlineto{\pgfqpoint{2.156472in}{1.524937in}}%
\pgfpathlineto{\pgfqpoint{2.158202in}{1.609737in}}%
\pgfpathlineto{\pgfqpoint{2.159066in}{1.578838in}}%
\pgfpathlineto{\pgfqpoint{2.159928in}{1.615080in}}%
\pgfpathlineto{\pgfqpoint{2.160794in}{1.543012in}}%
\pgfpathlineto{\pgfqpoint{2.162520in}{1.600712in}}%
\pgfpathlineto{\pgfqpoint{2.163384in}{1.562127in}}%
\pgfpathlineto{\pgfqpoint{2.165115in}{1.621907in}}%
\pgfpathlineto{\pgfqpoint{2.165980in}{1.622383in}}%
\pgfpathlineto{\pgfqpoint{2.166844in}{1.684062in}}%
\pgfpathlineto{\pgfqpoint{2.167707in}{1.675990in}}%
\pgfpathlineto{\pgfqpoint{2.168572in}{1.635650in}}%
\pgfpathlineto{\pgfqpoint{2.169437in}{1.776075in}}%
\pgfpathlineto{\pgfqpoint{2.171166in}{1.645150in}}%
\pgfpathlineto{\pgfqpoint{2.172029in}{1.701754in}}%
\pgfpathlineto{\pgfqpoint{2.172893in}{1.655093in}}%
\pgfpathlineto{\pgfqpoint{2.173758in}{1.659250in}}%
\pgfpathlineto{\pgfqpoint{2.174620in}{1.675752in}}%
\pgfpathlineto{\pgfqpoint{2.175484in}{1.639480in}}%
\pgfpathlineto{\pgfqpoint{2.176349in}{1.738677in}}%
\pgfpathlineto{\pgfqpoint{2.177215in}{1.738558in}}%
\pgfpathlineto{\pgfqpoint{2.178080in}{1.610064in}}%
\pgfpathlineto{\pgfqpoint{2.179810in}{1.673673in}}%
\pgfpathlineto{\pgfqpoint{2.180676in}{1.665035in}}%
\pgfpathlineto{\pgfqpoint{2.181542in}{1.682400in}}%
\pgfpathlineto{\pgfqpoint{2.182405in}{1.653550in}}%
\pgfpathlineto{\pgfqpoint{2.183270in}{1.703178in}}%
\pgfpathlineto{\pgfqpoint{2.184136in}{1.698013in}}%
\pgfpathlineto{\pgfqpoint{2.185000in}{1.679046in}}%
\pgfpathlineto{\pgfqpoint{2.185866in}{1.689286in}}%
\pgfpathlineto{\pgfqpoint{2.186731in}{1.718851in}}%
\pgfpathlineto{\pgfqpoint{2.188460in}{1.644406in}}%
\pgfpathlineto{\pgfqpoint{2.189326in}{1.767825in}}%
\pgfpathlineto{\pgfqpoint{2.190192in}{1.745771in}}%
\pgfpathlineto{\pgfqpoint{2.191056in}{1.766757in}}%
\pgfpathlineto{\pgfqpoint{2.192788in}{1.635977in}}%
\pgfpathlineto{\pgfqpoint{2.193652in}{1.664530in}}%
\pgfpathlineto{\pgfqpoint{2.194519in}{1.571624in}}%
\pgfpathlineto{\pgfqpoint{2.195384in}{1.702821in}}%
\pgfpathlineto{\pgfqpoint{2.196249in}{1.688334in}}%
\pgfpathlineto{\pgfqpoint{2.197115in}{1.633690in}}%
\pgfpathlineto{\pgfqpoint{2.197982in}{1.664470in}}%
\pgfpathlineto{\pgfqpoint{2.198846in}{1.625469in}}%
\pgfpathlineto{\pgfqpoint{2.199710in}{1.695756in}}%
\pgfpathlineto{\pgfqpoint{2.200575in}{1.678659in}}%
\pgfpathlineto{\pgfqpoint{2.201440in}{1.712853in}}%
\pgfpathlineto{\pgfqpoint{2.202305in}{1.628853in}}%
\pgfpathlineto{\pgfqpoint{2.203171in}{1.666430in}}%
\pgfpathlineto{\pgfqpoint{2.204035in}{1.661711in}}%
\pgfpathlineto{\pgfqpoint{2.204900in}{1.651888in}}%
\pgfpathlineto{\pgfqpoint{2.207495in}{1.703178in}}%
\pgfpathlineto{\pgfqpoint{2.208361in}{1.647849in}}%
\pgfpathlineto{\pgfqpoint{2.209225in}{1.655003in}}%
\pgfpathlineto{\pgfqpoint{2.210090in}{1.677354in}}%
\pgfpathlineto{\pgfqpoint{2.210954in}{1.675633in}}%
\pgfpathlineto{\pgfqpoint{2.211819in}{1.646723in}}%
\pgfpathlineto{\pgfqpoint{2.212685in}{1.712675in}}%
\pgfpathlineto{\pgfqpoint{2.214415in}{1.655152in}}%
\pgfpathlineto{\pgfqpoint{2.215281in}{1.697894in}}%
\pgfpathlineto{\pgfqpoint{2.216146in}{1.694749in}}%
\pgfpathlineto{\pgfqpoint{2.218741in}{1.639152in}}%
\pgfpathlineto{\pgfqpoint{2.220471in}{1.682043in}}%
\pgfpathlineto{\pgfqpoint{2.221337in}{1.592937in}}%
\pgfpathlineto{\pgfqpoint{2.222202in}{1.687148in}}%
\pgfpathlineto{\pgfqpoint{2.223065in}{1.615407in}}%
\pgfpathlineto{\pgfqpoint{2.223931in}{1.693856in}}%
\pgfpathlineto{\pgfqpoint{2.226526in}{1.634136in}}%
\pgfpathlineto{\pgfqpoint{2.227389in}{1.703948in}}%
\pgfpathlineto{\pgfqpoint{2.228254in}{1.659841in}}%
\pgfpathlineto{\pgfqpoint{2.229986in}{1.711429in}}%
\pgfpathlineto{\pgfqpoint{2.230853in}{1.709648in}}%
\pgfpathlineto{\pgfqpoint{2.231717in}{1.702583in}}%
\pgfpathlineto{\pgfqpoint{2.232580in}{1.704513in}}%
\pgfpathlineto{\pgfqpoint{2.234309in}{1.592878in}}%
\pgfpathlineto{\pgfqpoint{2.235174in}{1.671981in}}%
\pgfpathlineto{\pgfqpoint{2.236039in}{1.641915in}}%
\pgfpathlineto{\pgfqpoint{2.236904in}{1.713567in}}%
\pgfpathlineto{\pgfqpoint{2.239497in}{1.648920in}}%
\pgfpathlineto{\pgfqpoint{2.240361in}{1.638829in}}%
\pgfpathlineto{\pgfqpoint{2.241226in}{1.609681in}}%
\pgfpathlineto{\pgfqpoint{2.242955in}{1.710362in}}%
\pgfpathlineto{\pgfqpoint{2.243820in}{1.633367in}}%
\pgfpathlineto{\pgfqpoint{2.245550in}{1.721818in}}%
\pgfpathlineto{\pgfqpoint{2.246416in}{1.648563in}}%
\pgfpathlineto{\pgfqpoint{2.247281in}{1.655539in}}%
\pgfpathlineto{\pgfqpoint{2.248147in}{1.700151in}}%
\pgfpathlineto{\pgfqpoint{2.250740in}{1.647080in}}%
\pgfpathlineto{\pgfqpoint{2.253336in}{1.713329in}}%
\pgfpathlineto{\pgfqpoint{2.254201in}{1.628763in}}%
\pgfpathlineto{\pgfqpoint{2.255930in}{1.682281in}}%
\pgfpathlineto{\pgfqpoint{2.256794in}{1.631199in}}%
\pgfpathlineto{\pgfqpoint{2.258523in}{1.720513in}}%
\pgfpathlineto{\pgfqpoint{2.259389in}{1.710064in}}%
\pgfpathlineto{\pgfqpoint{2.260254in}{1.629329in}}%
\pgfpathlineto{\pgfqpoint{2.261984in}{1.718018in}}%
\pgfpathlineto{\pgfqpoint{2.263712in}{1.692194in}}%
\pgfpathlineto{\pgfqpoint{2.265441in}{1.736301in}}%
\pgfpathlineto{\pgfqpoint{2.266306in}{1.686583in}}%
\pgfpathlineto{\pgfqpoint{2.267171in}{1.753041in}}%
\pgfpathlineto{\pgfqpoint{2.268898in}{1.661146in}}%
\pgfpathlineto{\pgfqpoint{2.269763in}{1.709469in}}%
\pgfpathlineto{\pgfqpoint{2.270626in}{1.675514in}}%
\pgfpathlineto{\pgfqpoint{2.271490in}{1.682459in}}%
\pgfpathlineto{\pgfqpoint{2.273219in}{1.728942in}}%
\pgfpathlineto{\pgfqpoint{2.274949in}{1.758979in}}%
\pgfpathlineto{\pgfqpoint{2.275812in}{1.690502in}}%
\pgfpathlineto{\pgfqpoint{2.276676in}{1.705967in}}%
\pgfpathlineto{\pgfqpoint{2.277541in}{1.762005in}}%
\pgfpathlineto{\pgfqpoint{2.280137in}{1.646158in}}%
\pgfpathlineto{\pgfqpoint{2.281003in}{1.669754in}}%
\pgfpathlineto{\pgfqpoint{2.281868in}{1.631050in}}%
\pgfpathlineto{\pgfqpoint{2.283598in}{1.680619in}}%
\pgfpathlineto{\pgfqpoint{2.285328in}{1.639807in}}%
\pgfpathlineto{\pgfqpoint{2.286192in}{1.646425in}}%
\pgfpathlineto{\pgfqpoint{2.287056in}{1.641439in}}%
\pgfpathlineto{\pgfqpoint{2.287921in}{1.519683in}}%
\pgfpathlineto{\pgfqpoint{2.289652in}{1.626982in}}%
\pgfpathlineto{\pgfqpoint{2.292249in}{1.540282in}}%
\pgfpathlineto{\pgfqpoint{2.293114in}{1.568954in}}%
\pgfpathlineto{\pgfqpoint{2.293978in}{1.555360in}}%
\pgfpathlineto{\pgfqpoint{2.295707in}{1.577324in}}%
\pgfpathlineto{\pgfqpoint{2.296571in}{1.563909in}}%
\pgfpathlineto{\pgfqpoint{2.297435in}{1.624636in}}%
\pgfpathlineto{\pgfqpoint{2.298302in}{1.577800in}}%
\pgfpathlineto{\pgfqpoint{2.300033in}{1.635620in}}%
\pgfpathlineto{\pgfqpoint{2.300900in}{1.645830in}}%
\pgfpathlineto{\pgfqpoint{2.301766in}{1.549957in}}%
\pgfpathlineto{\pgfqpoint{2.304361in}{1.617158in}}%
\pgfpathlineto{\pgfqpoint{2.305227in}{1.598578in}}%
\pgfpathlineto{\pgfqpoint{2.306092in}{1.611756in}}%
\pgfpathlineto{\pgfqpoint{2.306957in}{1.540223in}}%
\pgfpathlineto{\pgfqpoint{2.307822in}{1.559160in}}%
\pgfpathlineto{\pgfqpoint{2.308688in}{1.610242in}}%
\pgfpathlineto{\pgfqpoint{2.309554in}{1.543785in}}%
\pgfpathlineto{\pgfqpoint{2.310420in}{1.559279in}}%
\pgfpathlineto{\pgfqpoint{2.311285in}{1.601010in}}%
\pgfpathlineto{\pgfqpoint{2.312152in}{1.585694in}}%
\pgfpathlineto{\pgfqpoint{2.313883in}{1.634166in}}%
\pgfpathlineto{\pgfqpoint{2.314749in}{1.650936in}}%
\pgfpathlineto{\pgfqpoint{2.316478in}{1.568776in}}%
\pgfpathlineto{\pgfqpoint{2.317342in}{1.639063in}}%
\pgfpathlineto{\pgfqpoint{2.318206in}{1.563819in}}%
\pgfpathlineto{\pgfqpoint{2.319072in}{1.569490in}}%
\pgfpathlineto{\pgfqpoint{2.319937in}{1.569430in}}%
\pgfpathlineto{\pgfqpoint{2.321668in}{1.649809in}}%
\pgfpathlineto{\pgfqpoint{2.323399in}{1.577651in}}%
\pgfpathlineto{\pgfqpoint{2.324264in}{1.632950in}}%
\pgfpathlineto{\pgfqpoint{2.325127in}{1.617040in}}%
\pgfpathlineto{\pgfqpoint{2.325993in}{1.637639in}}%
\pgfpathlineto{\pgfqpoint{2.326858in}{1.570676in}}%
\pgfpathlineto{\pgfqpoint{2.327723in}{1.596797in}}%
\pgfpathlineto{\pgfqpoint{2.328589in}{1.670349in}}%
\pgfpathlineto{\pgfqpoint{2.330317in}{1.605554in}}%
\pgfpathlineto{\pgfqpoint{2.332043in}{1.597570in}}%
\pgfpathlineto{\pgfqpoint{2.332908in}{1.609562in}}%
\pgfpathlineto{\pgfqpoint{2.333775in}{1.581838in}}%
\pgfpathlineto{\pgfqpoint{2.334640in}{1.607008in}}%
\pgfpathlineto{\pgfqpoint{2.335506in}{1.509175in}}%
\pgfpathlineto{\pgfqpoint{2.336370in}{1.599288in}}%
\pgfpathlineto{\pgfqpoint{2.337235in}{1.562693in}}%
\pgfpathlineto{\pgfqpoint{2.338100in}{1.639182in}}%
\pgfpathlineto{\pgfqpoint{2.341556in}{1.502170in}}%
\pgfpathlineto{\pgfqpoint{2.342422in}{1.507126in}}%
\pgfpathlineto{\pgfqpoint{2.343286in}{1.533158in}}%
\pgfpathlineto{\pgfqpoint{2.344150in}{1.529774in}}%
\pgfpathlineto{\pgfqpoint{2.345881in}{1.588427in}}%
\pgfpathlineto{\pgfqpoint{2.346747in}{1.533575in}}%
\pgfpathlineto{\pgfqpoint{2.350207in}{1.597567in}}%
\pgfpathlineto{\pgfqpoint{2.353668in}{1.526034in}}%
\pgfpathlineto{\pgfqpoint{2.356262in}{1.624253in}}%
\pgfpathlineto{\pgfqpoint{2.357127in}{1.584746in}}%
\pgfpathlineto{\pgfqpoint{2.358859in}{1.608521in}}%
\pgfpathlineto{\pgfqpoint{2.359724in}{1.511373in}}%
\pgfpathlineto{\pgfqpoint{2.361451in}{1.633928in}}%
\pgfpathlineto{\pgfqpoint{2.362316in}{1.638293in}}%
\pgfpathlineto{\pgfqpoint{2.363181in}{1.613953in}}%
\pgfpathlineto{\pgfqpoint{2.364045in}{1.553936in}}%
\pgfpathlineto{\pgfqpoint{2.364909in}{1.655152in}}%
\pgfpathlineto{\pgfqpoint{2.365771in}{1.649720in}}%
\pgfpathlineto{\pgfqpoint{2.368366in}{1.568359in}}%
\pgfpathlineto{\pgfqpoint{2.369231in}{1.587713in}}%
\pgfpathlineto{\pgfqpoint{2.370096in}{1.650698in}}%
\pgfpathlineto{\pgfqpoint{2.370961in}{1.596410in}}%
\pgfpathlineto{\pgfqpoint{2.371826in}{1.603386in}}%
\pgfpathlineto{\pgfqpoint{2.372691in}{1.634553in}}%
\pgfpathlineto{\pgfqpoint{2.373557in}{1.555182in}}%
\pgfpathlineto{\pgfqpoint{2.374422in}{1.618404in}}%
\pgfpathlineto{\pgfqpoint{2.375287in}{1.613329in}}%
\pgfpathlineto{\pgfqpoint{2.376151in}{1.599943in}}%
\pgfpathlineto{\pgfqpoint{2.377017in}{1.616266in}}%
\pgfpathlineto{\pgfqpoint{2.377884in}{1.607008in}}%
\pgfpathlineto{\pgfqpoint{2.378749in}{1.567649in}}%
\pgfpathlineto{\pgfqpoint{2.380481in}{1.639926in}}%
\pgfpathlineto{\pgfqpoint{2.381346in}{1.568006in}}%
\pgfpathlineto{\pgfqpoint{2.383076in}{1.620721in}}%
\pgfpathlineto{\pgfqpoint{2.383941in}{1.551266in}}%
\pgfpathlineto{\pgfqpoint{2.384807in}{1.656372in}}%
\pgfpathlineto{\pgfqpoint{2.385670in}{1.638710in}}%
\pgfpathlineto{\pgfqpoint{2.389131in}{1.694808in}}%
\pgfpathlineto{\pgfqpoint{2.389995in}{1.636929in}}%
\pgfpathlineto{\pgfqpoint{2.390860in}{1.736840in}}%
\pgfpathlineto{\pgfqpoint{2.391725in}{1.659547in}}%
\pgfpathlineto{\pgfqpoint{2.392590in}{1.660614in}}%
\pgfpathlineto{\pgfqpoint{2.393455in}{1.670884in}}%
\pgfpathlineto{\pgfqpoint{2.394320in}{1.605970in}}%
\pgfpathlineto{\pgfqpoint{2.395187in}{1.676406in}}%
\pgfpathlineto{\pgfqpoint{2.396051in}{1.675990in}}%
\pgfpathlineto{\pgfqpoint{2.397781in}{1.649809in}}%
\pgfpathlineto{\pgfqpoint{2.398647in}{1.698311in}}%
\pgfpathlineto{\pgfqpoint{2.399511in}{1.665452in}}%
\pgfpathlineto{\pgfqpoint{2.400376in}{1.685129in}}%
\pgfpathlineto{\pgfqpoint{2.401240in}{1.675514in}}%
\pgfpathlineto{\pgfqpoint{2.402105in}{1.652895in}}%
\pgfpathlineto{\pgfqpoint{2.403834in}{1.692521in}}%
\pgfpathlineto{\pgfqpoint{2.404699in}{1.658655in}}%
\pgfpathlineto{\pgfqpoint{2.405564in}{1.684835in}}%
\pgfpathlineto{\pgfqpoint{2.407293in}{1.632831in}}%
\pgfpathlineto{\pgfqpoint{2.408157in}{1.667501in}}%
\pgfpathlineto{\pgfqpoint{2.409022in}{1.636245in}}%
\pgfpathlineto{\pgfqpoint{2.409887in}{1.662217in}}%
\pgfpathlineto{\pgfqpoint{2.410753in}{1.658238in}}%
\pgfpathlineto{\pgfqpoint{2.411618in}{1.620780in}}%
\pgfpathlineto{\pgfqpoint{2.412483in}{1.648623in}}%
\pgfpathlineto{\pgfqpoint{2.413348in}{1.644971in}}%
\pgfpathlineto{\pgfqpoint{2.415079in}{1.579760in}}%
\pgfpathlineto{\pgfqpoint{2.416808in}{1.522353in}}%
\pgfpathlineto{\pgfqpoint{2.417672in}{1.645711in}}%
\pgfpathlineto{\pgfqpoint{2.419402in}{1.555003in}}%
\pgfpathlineto{\pgfqpoint{2.421131in}{1.637579in}}%
\pgfpathlineto{\pgfqpoint{2.421995in}{1.605940in}}%
\pgfpathlineto{\pgfqpoint{2.422861in}{1.658774in}}%
\pgfpathlineto{\pgfqpoint{2.424590in}{1.574833in}}%
\pgfpathlineto{\pgfqpoint{2.425455in}{1.586735in}}%
\pgfpathlineto{\pgfqpoint{2.426320in}{1.567590in}}%
\pgfpathlineto{\pgfqpoint{2.427184in}{1.595551in}}%
\pgfpathlineto{\pgfqpoint{2.428048in}{1.543874in}}%
\pgfpathlineto{\pgfqpoint{2.428912in}{1.624521in}}%
\pgfpathlineto{\pgfqpoint{2.429777in}{1.559160in}}%
\pgfpathlineto{\pgfqpoint{2.430641in}{1.564087in}}%
\pgfpathlineto{\pgfqpoint{2.432370in}{1.531585in}}%
\pgfpathlineto{\pgfqpoint{2.433233in}{1.564384in}}%
\pgfpathlineto{\pgfqpoint{2.434097in}{1.538918in}}%
\pgfpathlineto{\pgfqpoint{2.434962in}{1.552304in}}%
\pgfpathlineto{\pgfqpoint{2.435826in}{1.589970in}}%
\pgfpathlineto{\pgfqpoint{2.436690in}{1.559755in}}%
\pgfpathlineto{\pgfqpoint{2.437555in}{1.590119in}}%
\pgfpathlineto{\pgfqpoint{2.438420in}{1.517902in}}%
\pgfpathlineto{\pgfqpoint{2.440150in}{1.622323in}}%
\pgfpathlineto{\pgfqpoint{2.441015in}{1.546812in}}%
\pgfpathlineto{\pgfqpoint{2.441880in}{1.589524in}}%
\pgfpathlineto{\pgfqpoint{2.442744in}{1.586527in}}%
\pgfpathlineto{\pgfqpoint{2.444474in}{1.534761in}}%
\pgfpathlineto{\pgfqpoint{2.445339in}{1.653133in}}%
\pgfpathlineto{\pgfqpoint{2.446201in}{1.580232in}}%
\pgfpathlineto{\pgfqpoint{2.447066in}{1.594570in}}%
\pgfpathlineto{\pgfqpoint{2.448795in}{1.611577in}}%
\pgfpathlineto{\pgfqpoint{2.449658in}{1.575305in}}%
\pgfpathlineto{\pgfqpoint{2.450524in}{1.607008in}}%
\pgfpathlineto{\pgfqpoint{2.451388in}{1.575751in}}%
\pgfpathlineto{\pgfqpoint{2.452253in}{1.638885in}}%
\pgfpathlineto{\pgfqpoint{2.453982in}{1.592937in}}%
\pgfpathlineto{\pgfqpoint{2.454848in}{1.617040in}}%
\pgfpathlineto{\pgfqpoint{2.455713in}{1.606651in}}%
\pgfpathlineto{\pgfqpoint{2.457441in}{1.655212in}}%
\pgfpathlineto{\pgfqpoint{2.458306in}{1.601545in}}%
\pgfpathlineto{\pgfqpoint{2.460035in}{1.631288in}}%
\pgfpathlineto{\pgfqpoint{2.460899in}{1.626004in}}%
\pgfpathlineto{\pgfqpoint{2.462629in}{1.696708in}}%
\pgfpathlineto{\pgfqpoint{2.464360in}{1.559160in}}%
\pgfpathlineto{\pgfqpoint{2.465225in}{1.629150in}}%
\pgfpathlineto{\pgfqpoint{2.466090in}{1.564295in}}%
\pgfpathlineto{\pgfqpoint{2.467821in}{1.608908in}}%
\pgfpathlineto{\pgfqpoint{2.468685in}{1.579938in}}%
\pgfpathlineto{\pgfqpoint{2.469550in}{1.615794in}}%
\pgfpathlineto{\pgfqpoint{2.470414in}{1.552780in}}%
\pgfpathlineto{\pgfqpoint{2.471277in}{1.565098in}}%
\pgfpathlineto{\pgfqpoint{2.472140in}{1.555777in}}%
\pgfpathlineto{\pgfqpoint{2.473871in}{1.629150in}}%
\pgfpathlineto{\pgfqpoint{2.474737in}{1.576019in}}%
\pgfpathlineto{\pgfqpoint{2.475602in}{1.617129in}}%
\pgfpathlineto{\pgfqpoint{2.476468in}{1.565630in}}%
\pgfpathlineto{\pgfqpoint{2.477334in}{1.663106in}}%
\pgfpathlineto{\pgfqpoint{2.478200in}{1.628912in}}%
\pgfpathlineto{\pgfqpoint{2.479066in}{1.656636in}}%
\pgfpathlineto{\pgfqpoint{2.479931in}{1.627488in}}%
\pgfpathlineto{\pgfqpoint{2.480797in}{1.632891in}}%
\pgfpathlineto{\pgfqpoint{2.481662in}{1.615139in}}%
\pgfpathlineto{\pgfqpoint{2.483391in}{1.644287in}}%
\pgfpathlineto{\pgfqpoint{2.485119in}{1.610867in}}%
\pgfpathlineto{\pgfqpoint{2.486849in}{1.680619in}}%
\pgfpathlineto{\pgfqpoint{2.487715in}{1.666787in}}%
\pgfpathlineto{\pgfqpoint{2.489447in}{1.545566in}}%
\pgfpathlineto{\pgfqpoint{2.490310in}{1.633898in}}%
\pgfpathlineto{\pgfqpoint{2.491175in}{1.633069in}}%
\pgfpathlineto{\pgfqpoint{2.492039in}{1.570293in}}%
\pgfpathlineto{\pgfqpoint{2.493769in}{1.641677in}}%
\pgfpathlineto{\pgfqpoint{2.494635in}{1.580176in}}%
\pgfpathlineto{\pgfqpoint{2.495501in}{1.655212in}}%
\pgfpathlineto{\pgfqpoint{2.496365in}{1.564325in}}%
\pgfpathlineto{\pgfqpoint{2.497231in}{1.625707in}}%
\pgfpathlineto{\pgfqpoint{2.498096in}{1.562960in}}%
\pgfpathlineto{\pgfqpoint{2.498962in}{1.629418in}}%
\pgfpathlineto{\pgfqpoint{2.500692in}{1.578157in}}%
\pgfpathlineto{\pgfqpoint{2.501555in}{1.604189in}}%
\pgfpathlineto{\pgfqpoint{2.502421in}{1.576198in}}%
\pgfpathlineto{\pgfqpoint{2.503285in}{1.600478in}}%
\pgfpathlineto{\pgfqpoint{2.504149in}{1.600121in}}%
\pgfpathlineto{\pgfqpoint{2.505014in}{1.609027in}}%
\pgfpathlineto{\pgfqpoint{2.505880in}{1.569192in}}%
\pgfpathlineto{\pgfqpoint{2.507611in}{1.609324in}}%
\pgfpathlineto{\pgfqpoint{2.508476in}{1.571628in}}%
\pgfpathlineto{\pgfqpoint{2.509340in}{1.606000in}}%
\pgfpathlineto{\pgfqpoint{2.510205in}{1.594127in}}%
\pgfpathlineto{\pgfqpoint{2.511069in}{1.516805in}}%
\pgfpathlineto{\pgfqpoint{2.511934in}{1.587181in}}%
\pgfpathlineto{\pgfqpoint{2.512798in}{1.532329in}}%
\pgfpathlineto{\pgfqpoint{2.514530in}{1.607662in}}%
\pgfpathlineto{\pgfqpoint{2.515396in}{1.597213in}}%
\pgfpathlineto{\pgfqpoint{2.517125in}{1.646901in}}%
\pgfpathlineto{\pgfqpoint{2.517990in}{1.513540in}}%
\pgfpathlineto{\pgfqpoint{2.519720in}{1.617694in}}%
\pgfpathlineto{\pgfqpoint{2.522317in}{1.580355in}}%
\pgfpathlineto{\pgfqpoint{2.524046in}{1.665244in}}%
\pgfpathlineto{\pgfqpoint{2.524912in}{1.651947in}}%
\pgfpathlineto{\pgfqpoint{2.525777in}{1.621316in}}%
\pgfpathlineto{\pgfqpoint{2.526643in}{1.643280in}}%
\pgfpathlineto{\pgfqpoint{2.527509in}{1.597303in}}%
\pgfpathlineto{\pgfqpoint{2.528374in}{1.659547in}}%
\pgfpathlineto{\pgfqpoint{2.530102in}{1.556253in}}%
\pgfpathlineto{\pgfqpoint{2.531831in}{1.621970in}}%
\pgfpathlineto{\pgfqpoint{2.533562in}{1.543845in}}%
\pgfpathlineto{\pgfqpoint{2.534427in}{1.613065in}}%
\pgfpathlineto{\pgfqpoint{2.537021in}{1.540729in}}%
\pgfpathlineto{\pgfqpoint{2.537886in}{1.634910in}}%
\pgfpathlineto{\pgfqpoint{2.541346in}{1.540996in}}%
\pgfpathlineto{\pgfqpoint{2.542211in}{1.587241in}}%
\pgfpathlineto{\pgfqpoint{2.543077in}{1.549485in}}%
\pgfpathlineto{\pgfqpoint{2.543942in}{1.614429in}}%
\pgfpathlineto{\pgfqpoint{2.544808in}{1.569906in}}%
\pgfpathlineto{\pgfqpoint{2.545672in}{1.583322in}}%
\pgfpathlineto{\pgfqpoint{2.547403in}{1.556134in}}%
\pgfpathlineto{\pgfqpoint{2.548266in}{1.600597in}}%
\pgfpathlineto{\pgfqpoint{2.549131in}{1.588992in}}%
\pgfpathlineto{\pgfqpoint{2.549993in}{1.523781in}}%
\pgfpathlineto{\pgfqpoint{2.551724in}{1.580860in}}%
\pgfpathlineto{\pgfqpoint{2.552590in}{1.594960in}}%
\pgfpathlineto{\pgfqpoint{2.553455in}{1.594306in}}%
\pgfpathlineto{\pgfqpoint{2.554322in}{1.585400in}}%
\pgfpathlineto{\pgfqpoint{2.555188in}{1.625056in}}%
\pgfpathlineto{\pgfqpoint{2.556916in}{1.595551in}}%
\pgfpathlineto{\pgfqpoint{2.557782in}{1.650225in}}%
\pgfpathlineto{\pgfqpoint{2.558645in}{1.594276in}}%
\pgfpathlineto{\pgfqpoint{2.559511in}{1.650106in}}%
\pgfpathlineto{\pgfqpoint{2.560377in}{1.541234in}}%
\pgfpathlineto{\pgfqpoint{2.561241in}{1.568333in}}%
\pgfpathlineto{\pgfqpoint{2.562106in}{1.592465in}}%
\pgfpathlineto{\pgfqpoint{2.562970in}{1.670944in}}%
\pgfpathlineto{\pgfqpoint{2.564700in}{1.623335in}}%
\pgfpathlineto{\pgfqpoint{2.565564in}{1.665482in}}%
\pgfpathlineto{\pgfqpoint{2.566428in}{1.629567in}}%
\pgfpathlineto{\pgfqpoint{2.567292in}{1.637877in}}%
\pgfpathlineto{\pgfqpoint{2.568157in}{1.609205in}}%
\pgfpathlineto{\pgfqpoint{2.569022in}{1.627904in}}%
\pgfpathlineto{\pgfqpoint{2.569888in}{1.569728in}}%
\pgfpathlineto{\pgfqpoint{2.570753in}{1.598786in}}%
\pgfpathlineto{\pgfqpoint{2.571618in}{1.535415in}}%
\pgfpathlineto{\pgfqpoint{2.572482in}{1.611522in}}%
\pgfpathlineto{\pgfqpoint{2.573346in}{1.556134in}}%
\pgfpathlineto{\pgfqpoint{2.575939in}{1.639242in}}%
\pgfpathlineto{\pgfqpoint{2.576802in}{1.560763in}}%
\pgfpathlineto{\pgfqpoint{2.577665in}{1.598459in}}%
\pgfpathlineto{\pgfqpoint{2.578530in}{1.557498in}}%
\pgfpathlineto{\pgfqpoint{2.579394in}{1.561179in}}%
\pgfpathlineto{\pgfqpoint{2.581123in}{1.638472in}}%
\pgfpathlineto{\pgfqpoint{2.582854in}{1.558093in}}%
\pgfpathlineto{\pgfqpoint{2.585448in}{1.637996in}}%
\pgfpathlineto{\pgfqpoint{2.588042in}{1.526272in}}%
\pgfpathlineto{\pgfqpoint{2.588907in}{1.622323in}}%
\pgfpathlineto{\pgfqpoint{2.589773in}{1.558893in}}%
\pgfpathlineto{\pgfqpoint{2.590637in}{1.588844in}}%
\pgfpathlineto{\pgfqpoint{2.591500in}{1.568304in}}%
\pgfpathlineto{\pgfqpoint{2.593230in}{1.696708in}}%
\pgfpathlineto{\pgfqpoint{2.594961in}{1.626599in}}%
\pgfpathlineto{\pgfqpoint{2.597557in}{1.724610in}}%
\pgfpathlineto{\pgfqpoint{2.598421in}{1.643934in}}%
\pgfpathlineto{\pgfqpoint{2.600150in}{1.705613in}}%
\pgfpathlineto{\pgfqpoint{2.601015in}{1.706149in}}%
\pgfpathlineto{\pgfqpoint{2.601880in}{1.716240in}}%
\pgfpathlineto{\pgfqpoint{2.602742in}{1.652720in}}%
\pgfpathlineto{\pgfqpoint{2.603605in}{1.678306in}}%
\pgfpathlineto{\pgfqpoint{2.604469in}{1.618408in}}%
\pgfpathlineto{\pgfqpoint{2.605334in}{1.666077in}}%
\pgfpathlineto{\pgfqpoint{2.606197in}{1.659696in}}%
\pgfpathlineto{\pgfqpoint{2.607063in}{1.623811in}}%
\pgfpathlineto{\pgfqpoint{2.607928in}{1.679909in}}%
\pgfpathlineto{\pgfqpoint{2.608793in}{1.656431in}}%
\pgfpathlineto{\pgfqpoint{2.609657in}{1.700449in}}%
\pgfpathlineto{\pgfqpoint{2.611387in}{1.669579in}}%
\pgfpathlineto{\pgfqpoint{2.613117in}{1.712202in}}%
\pgfpathlineto{\pgfqpoint{2.613981in}{1.659547in}}%
\pgfpathlineto{\pgfqpoint{2.614846in}{1.685133in}}%
\pgfpathlineto{\pgfqpoint{2.615710in}{1.650315in}}%
\pgfpathlineto{\pgfqpoint{2.617439in}{1.707216in}}%
\pgfpathlineto{\pgfqpoint{2.618301in}{1.715645in}}%
\pgfpathlineto{\pgfqpoint{2.620031in}{1.657350in}}%
\pgfpathlineto{\pgfqpoint{2.621760in}{1.674149in}}%
\pgfpathlineto{\pgfqpoint{2.622625in}{1.658952in}}%
\pgfpathlineto{\pgfqpoint{2.623489in}{1.681154in}}%
\pgfpathlineto{\pgfqpoint{2.625219in}{1.591190in}}%
\pgfpathlineto{\pgfqpoint{2.626084in}{1.600835in}}%
\pgfpathlineto{\pgfqpoint{2.626950in}{1.686974in}}%
\pgfpathlineto{\pgfqpoint{2.627816in}{1.647318in}}%
\pgfpathlineto{\pgfqpoint{2.628682in}{1.652423in}}%
\pgfpathlineto{\pgfqpoint{2.630413in}{1.635564in}}%
\pgfpathlineto{\pgfqpoint{2.631279in}{1.681571in}}%
\pgfpathlineto{\pgfqpoint{2.632145in}{1.628767in}}%
\pgfpathlineto{\pgfqpoint{2.633874in}{1.669877in}}%
\pgfpathlineto{\pgfqpoint{2.634739in}{1.691573in}}%
\pgfpathlineto{\pgfqpoint{2.636470in}{1.646961in}}%
\pgfpathlineto{\pgfqpoint{2.637336in}{1.723658in}}%
\pgfpathlineto{\pgfqpoint{2.638200in}{1.681214in}}%
\pgfpathlineto{\pgfqpoint{2.639064in}{1.733099in}}%
\pgfpathlineto{\pgfqpoint{2.639929in}{1.633724in}}%
\pgfpathlineto{\pgfqpoint{2.642524in}{1.725677in}}%
\pgfpathlineto{\pgfqpoint{2.643388in}{1.694689in}}%
\pgfpathlineto{\pgfqpoint{2.644252in}{1.706740in}}%
\pgfpathlineto{\pgfqpoint{2.645115in}{1.640015in}}%
\pgfpathlineto{\pgfqpoint{2.645978in}{1.734047in}}%
\pgfpathlineto{\pgfqpoint{2.647710in}{1.694749in}}%
\pgfpathlineto{\pgfqpoint{2.648575in}{1.718434in}}%
\pgfpathlineto{\pgfqpoint{2.649441in}{1.693265in}}%
\pgfpathlineto{\pgfqpoint{2.650307in}{1.715051in}}%
\pgfpathlineto{\pgfqpoint{2.651172in}{1.669222in}}%
\pgfpathlineto{\pgfqpoint{2.652037in}{1.766936in}}%
\pgfpathlineto{\pgfqpoint{2.652902in}{1.682697in}}%
\pgfpathlineto{\pgfqpoint{2.653767in}{1.682935in}}%
\pgfpathlineto{\pgfqpoint{2.654633in}{1.686974in}}%
\pgfpathlineto{\pgfqpoint{2.655499in}{1.679314in}}%
\pgfpathlineto{\pgfqpoint{2.657231in}{1.768836in}}%
\pgfpathlineto{\pgfqpoint{2.658097in}{1.681452in}}%
\pgfpathlineto{\pgfqpoint{2.658961in}{1.689108in}}%
\pgfpathlineto{\pgfqpoint{2.659827in}{1.682638in}}%
\pgfpathlineto{\pgfqpoint{2.660689in}{1.633129in}}%
\pgfpathlineto{\pgfqpoint{2.662417in}{1.706324in}}%
\pgfpathlineto{\pgfqpoint{2.663283in}{1.672636in}}%
\pgfpathlineto{\pgfqpoint{2.664149in}{1.682876in}}%
\pgfpathlineto{\pgfqpoint{2.665014in}{1.629329in}}%
\pgfpathlineto{\pgfqpoint{2.665878in}{1.651977in}}%
\pgfpathlineto{\pgfqpoint{2.666743in}{1.648087in}}%
\pgfpathlineto{\pgfqpoint{2.669338in}{1.553758in}}%
\pgfpathlineto{\pgfqpoint{2.670204in}{1.637788in}}%
\pgfpathlineto{\pgfqpoint{2.672800in}{1.542182in}}%
\pgfpathlineto{\pgfqpoint{2.673666in}{1.592937in}}%
\pgfpathlineto{\pgfqpoint{2.674531in}{1.562960in}}%
\pgfpathlineto{\pgfqpoint{2.675397in}{1.640312in}}%
\pgfpathlineto{\pgfqpoint{2.676261in}{1.611224in}}%
\pgfpathlineto{\pgfqpoint{2.677125in}{1.670587in}}%
\pgfpathlineto{\pgfqpoint{2.677991in}{1.614191in}}%
\pgfpathlineto{\pgfqpoint{2.678857in}{1.619594in}}%
\pgfpathlineto{\pgfqpoint{2.679722in}{1.634850in}}%
\pgfpathlineto{\pgfqpoint{2.681452in}{1.721937in}}%
\pgfpathlineto{\pgfqpoint{2.683182in}{1.595075in}}%
\pgfpathlineto{\pgfqpoint{2.684047in}{1.592406in}}%
\pgfpathlineto{\pgfqpoint{2.684912in}{1.621107in}}%
\pgfpathlineto{\pgfqpoint{2.685777in}{1.614013in}}%
\pgfpathlineto{\pgfqpoint{2.686642in}{1.605524in}}%
\pgfpathlineto{\pgfqpoint{2.687507in}{1.620959in}}%
\pgfpathlineto{\pgfqpoint{2.688372in}{1.662276in}}%
\pgfpathlineto{\pgfqpoint{2.690102in}{1.595194in}}%
\pgfpathlineto{\pgfqpoint{2.690968in}{1.588725in}}%
\pgfpathlineto{\pgfqpoint{2.692698in}{1.633307in}}%
\pgfpathlineto{\pgfqpoint{2.693563in}{1.573111in}}%
\pgfpathlineto{\pgfqpoint{2.694429in}{1.623245in}}%
\pgfpathlineto{\pgfqpoint{2.695292in}{1.597451in}}%
\pgfpathlineto{\pgfqpoint{2.696157in}{1.643399in}}%
\pgfpathlineto{\pgfqpoint{2.697023in}{1.570914in}}%
\pgfpathlineto{\pgfqpoint{2.697889in}{1.578633in}}%
\pgfpathlineto{\pgfqpoint{2.698752in}{1.561001in}}%
\pgfpathlineto{\pgfqpoint{2.699618in}{1.573052in}}%
\pgfpathlineto{\pgfqpoint{2.701346in}{1.532388in}}%
\pgfpathlineto{\pgfqpoint{2.702209in}{1.587181in}}%
\pgfpathlineto{\pgfqpoint{2.703074in}{1.566404in}}%
\pgfpathlineto{\pgfqpoint{2.705670in}{1.648801in}}%
\pgfpathlineto{\pgfqpoint{2.707401in}{1.629567in}}%
\pgfpathlineto{\pgfqpoint{2.708268in}{1.537404in}}%
\pgfpathlineto{\pgfqpoint{2.709995in}{1.621018in}}%
\pgfpathlineto{\pgfqpoint{2.711726in}{1.597213in}}%
\pgfpathlineto{\pgfqpoint{2.713458in}{1.656904in}}%
\pgfpathlineto{\pgfqpoint{2.714324in}{1.656104in}}%
\pgfpathlineto{\pgfqpoint{2.715189in}{1.623275in}}%
\pgfpathlineto{\pgfqpoint{2.716054in}{1.637817in}}%
\pgfpathlineto{\pgfqpoint{2.716920in}{1.727102in}}%
\pgfpathlineto{\pgfqpoint{2.718650in}{1.574416in}}%
\pgfpathlineto{\pgfqpoint{2.719515in}{1.592465in}}%
\pgfpathlineto{\pgfqpoint{2.720379in}{1.571449in}}%
\pgfpathlineto{\pgfqpoint{2.721245in}{1.648444in}}%
\pgfpathlineto{\pgfqpoint{2.722110in}{1.627904in}}%
\pgfpathlineto{\pgfqpoint{2.722975in}{1.600805in}}%
\pgfpathlineto{\pgfqpoint{2.723839in}{1.680857in}}%
\pgfpathlineto{\pgfqpoint{2.725571in}{1.587241in}}%
\pgfpathlineto{\pgfqpoint{2.726436in}{1.611343in}}%
\pgfpathlineto{\pgfqpoint{2.727301in}{1.530816in}}%
\pgfpathlineto{\pgfqpoint{2.728166in}{1.533932in}}%
\pgfpathlineto{\pgfqpoint{2.729031in}{1.543845in}}%
\pgfpathlineto{\pgfqpoint{2.729896in}{1.588368in}}%
\pgfpathlineto{\pgfqpoint{2.730761in}{1.561417in}}%
\pgfpathlineto{\pgfqpoint{2.731626in}{1.610391in}}%
\pgfpathlineto{\pgfqpoint{2.732492in}{1.540401in}}%
\pgfpathlineto{\pgfqpoint{2.733357in}{1.632831in}}%
\pgfpathlineto{\pgfqpoint{2.734222in}{1.621554in}}%
\pgfpathlineto{\pgfqpoint{2.735086in}{1.636691in}}%
\pgfpathlineto{\pgfqpoint{2.735950in}{1.555896in}}%
\pgfpathlineto{\pgfqpoint{2.736813in}{1.636126in}}%
\pgfpathlineto{\pgfqpoint{2.737677in}{1.632891in}}%
\pgfpathlineto{\pgfqpoint{2.738539in}{1.545388in}}%
\pgfpathlineto{\pgfqpoint{2.739404in}{1.594391in}}%
\pgfpathlineto{\pgfqpoint{2.740269in}{1.584032in}}%
\pgfpathlineto{\pgfqpoint{2.741132in}{1.581303in}}%
\pgfpathlineto{\pgfqpoint{2.741997in}{1.601635in}}%
\pgfpathlineto{\pgfqpoint{2.742860in}{1.593175in}}%
\pgfpathlineto{\pgfqpoint{2.744591in}{1.651888in}}%
\pgfpathlineto{\pgfqpoint{2.745455in}{1.665422in}}%
\pgfpathlineto{\pgfqpoint{2.746322in}{1.583381in}}%
\pgfpathlineto{\pgfqpoint{2.747187in}{1.597094in}}%
\pgfpathlineto{\pgfqpoint{2.748053in}{1.559458in}}%
\pgfpathlineto{\pgfqpoint{2.750649in}{1.628023in}}%
\pgfpathlineto{\pgfqpoint{2.751515in}{1.545269in}}%
\pgfpathlineto{\pgfqpoint{2.753245in}{1.638353in}}%
\pgfpathlineto{\pgfqpoint{2.754111in}{1.564028in}}%
\pgfpathlineto{\pgfqpoint{2.755840in}{1.633277in}}%
\pgfpathlineto{\pgfqpoint{2.757572in}{1.581422in}}%
\pgfpathlineto{\pgfqpoint{2.758437in}{1.572814in}}%
\pgfpathlineto{\pgfqpoint{2.759302in}{1.651055in}}%
\pgfpathlineto{\pgfqpoint{2.760166in}{1.646306in}}%
\pgfpathlineto{\pgfqpoint{2.761898in}{1.504784in}}%
\pgfpathlineto{\pgfqpoint{2.764491in}{1.601307in}}%
\pgfpathlineto{\pgfqpoint{2.765356in}{1.560227in}}%
\pgfpathlineto{\pgfqpoint{2.766220in}{1.568657in}}%
\pgfpathlineto{\pgfqpoint{2.767085in}{1.562365in}}%
\pgfpathlineto{\pgfqpoint{2.767951in}{1.640636in}}%
\pgfpathlineto{\pgfqpoint{2.769681in}{1.534404in}}%
\pgfpathlineto{\pgfqpoint{2.770546in}{1.609707in}}%
\pgfpathlineto{\pgfqpoint{2.771411in}{1.546633in}}%
\pgfpathlineto{\pgfqpoint{2.772275in}{1.628615in}}%
\pgfpathlineto{\pgfqpoint{2.773139in}{1.548890in}}%
\pgfpathlineto{\pgfqpoint{2.774004in}{1.602438in}}%
\pgfpathlineto{\pgfqpoint{2.774866in}{1.536809in}}%
\pgfpathlineto{\pgfqpoint{2.775728in}{1.607008in}}%
\pgfpathlineto{\pgfqpoint{2.776593in}{1.577030in}}%
\pgfpathlineto{\pgfqpoint{2.779188in}{1.648682in}}%
\pgfpathlineto{\pgfqpoint{2.780052in}{1.557528in}}%
\pgfpathlineto{\pgfqpoint{2.780917in}{1.626837in}}%
\pgfpathlineto{\pgfqpoint{2.781780in}{1.595849in}}%
\pgfpathlineto{\pgfqpoint{2.782645in}{1.635029in}}%
\pgfpathlineto{\pgfqpoint{2.784371in}{1.566463in}}%
\pgfpathlineto{\pgfqpoint{2.786101in}{1.666965in}}%
\pgfpathlineto{\pgfqpoint{2.786965in}{1.542272in}}%
\pgfpathlineto{\pgfqpoint{2.787828in}{1.643339in}}%
\pgfpathlineto{\pgfqpoint{2.788692in}{1.628737in}}%
\pgfpathlineto{\pgfqpoint{2.789556in}{1.621375in}}%
\pgfpathlineto{\pgfqpoint{2.791286in}{1.555185in}}%
\pgfpathlineto{\pgfqpoint{2.793017in}{1.617396in}}%
\pgfpathlineto{\pgfqpoint{2.793883in}{1.576911in}}%
\pgfpathlineto{\pgfqpoint{2.794748in}{1.612767in}}%
\pgfpathlineto{\pgfqpoint{2.795613in}{1.532448in}}%
\pgfpathlineto{\pgfqpoint{2.796478in}{1.542420in}}%
\pgfpathlineto{\pgfqpoint{2.797343in}{1.623097in}}%
\pgfpathlineto{\pgfqpoint{2.799072in}{1.541974in}}%
\pgfpathlineto{\pgfqpoint{2.800803in}{1.623037in}}%
\pgfpathlineto{\pgfqpoint{2.801665in}{1.549723in}}%
\pgfpathlineto{\pgfqpoint{2.803395in}{1.601162in}}%
\pgfpathlineto{\pgfqpoint{2.804259in}{1.582433in}}%
\pgfpathlineto{\pgfqpoint{2.805989in}{1.631496in}}%
\pgfpathlineto{\pgfqpoint{2.806856in}{1.626242in}}%
\pgfpathlineto{\pgfqpoint{2.807721in}{1.641975in}}%
\pgfpathlineto{\pgfqpoint{2.809451in}{1.607959in}}%
\pgfpathlineto{\pgfqpoint{2.810316in}{1.642034in}}%
\pgfpathlineto{\pgfqpoint{2.811181in}{1.609681in}}%
\pgfpathlineto{\pgfqpoint{2.812047in}{1.647556in}}%
\pgfpathlineto{\pgfqpoint{2.814641in}{1.568125in}}%
\pgfpathlineto{\pgfqpoint{2.816368in}{1.610748in}}%
\pgfpathlineto{\pgfqpoint{2.818096in}{1.571449in}}%
\pgfpathlineto{\pgfqpoint{2.818961in}{1.589673in}}%
\pgfpathlineto{\pgfqpoint{2.819827in}{1.546276in}}%
\pgfpathlineto{\pgfqpoint{2.821555in}{1.630812in}}%
\pgfpathlineto{\pgfqpoint{2.822421in}{1.618672in}}%
\pgfpathlineto{\pgfqpoint{2.823283in}{1.622978in}}%
\pgfpathlineto{\pgfqpoint{2.824146in}{1.646425in}}%
\pgfpathlineto{\pgfqpoint{2.825008in}{1.576108in}}%
\pgfpathlineto{\pgfqpoint{2.825873in}{1.610272in}}%
\pgfpathlineto{\pgfqpoint{2.826739in}{1.605345in}}%
\pgfpathlineto{\pgfqpoint{2.827603in}{1.598191in}}%
\pgfpathlineto{\pgfqpoint{2.828468in}{1.574179in}}%
\pgfpathlineto{\pgfqpoint{2.829334in}{1.625469in}}%
\pgfpathlineto{\pgfqpoint{2.830199in}{1.600419in}}%
\pgfpathlineto{\pgfqpoint{2.831927in}{1.648563in}}%
\pgfpathlineto{\pgfqpoint{2.833654in}{1.580176in}}%
\pgfpathlineto{\pgfqpoint{2.834521in}{1.626064in}}%
\pgfpathlineto{\pgfqpoint{2.837117in}{1.564890in}}%
\pgfpathlineto{\pgfqpoint{2.837982in}{1.681809in}}%
\pgfpathlineto{\pgfqpoint{2.839712in}{1.585043in}}%
\pgfpathlineto{\pgfqpoint{2.841442in}{1.614132in}}%
\pgfpathlineto{\pgfqpoint{2.842307in}{1.517485in}}%
\pgfpathlineto{\pgfqpoint{2.843172in}{1.586170in}}%
\pgfpathlineto{\pgfqpoint{2.844037in}{1.562336in}}%
\pgfpathlineto{\pgfqpoint{2.845764in}{1.636096in}}%
\pgfpathlineto{\pgfqpoint{2.846628in}{1.606799in}}%
\pgfpathlineto{\pgfqpoint{2.847492in}{1.632891in}}%
\pgfpathlineto{\pgfqpoint{2.848356in}{1.626183in}}%
\pgfpathlineto{\pgfqpoint{2.849221in}{1.610927in}}%
\pgfpathlineto{\pgfqpoint{2.851815in}{1.696351in}}%
\pgfpathlineto{\pgfqpoint{2.852680in}{1.621375in}}%
\pgfpathlineto{\pgfqpoint{2.853545in}{1.631377in}}%
\pgfpathlineto{\pgfqpoint{2.855272in}{1.698430in}}%
\pgfpathlineto{\pgfqpoint{2.856137in}{1.659101in}}%
\pgfpathlineto{\pgfqpoint{2.857003in}{1.662871in}}%
\pgfpathlineto{\pgfqpoint{2.857868in}{1.700032in}}%
\pgfpathlineto{\pgfqpoint{2.858734in}{1.628559in}}%
\pgfpathlineto{\pgfqpoint{2.859599in}{1.702940in}}%
\pgfpathlineto{\pgfqpoint{2.860465in}{1.633218in}}%
\pgfpathlineto{\pgfqpoint{2.862193in}{1.710897in}}%
\pgfpathlineto{\pgfqpoint{2.863921in}{1.657350in}}%
\pgfpathlineto{\pgfqpoint{2.864787in}{1.722770in}}%
\pgfpathlineto{\pgfqpoint{2.865652in}{1.693622in}}%
\pgfpathlineto{\pgfqpoint{2.866517in}{1.726153in}}%
\pgfpathlineto{\pgfqpoint{2.868246in}{1.693503in}}%
\pgfpathlineto{\pgfqpoint{2.869111in}{1.732921in}}%
\pgfpathlineto{\pgfqpoint{2.869976in}{1.659458in}}%
\pgfpathlineto{\pgfqpoint{2.870840in}{1.669936in}}%
\pgfpathlineto{\pgfqpoint{2.871704in}{1.664652in}}%
\pgfpathlineto{\pgfqpoint{2.873433in}{1.698608in}}%
\pgfpathlineto{\pgfqpoint{2.874296in}{1.677533in}}%
\pgfpathlineto{\pgfqpoint{2.875161in}{1.720275in}}%
\pgfpathlineto{\pgfqpoint{2.876025in}{1.545864in}}%
\pgfpathlineto{\pgfqpoint{2.876888in}{1.578276in}}%
\pgfpathlineto{\pgfqpoint{2.877753in}{1.571152in}}%
\pgfpathlineto{\pgfqpoint{2.878619in}{1.590327in}}%
\pgfpathlineto{\pgfqpoint{2.879484in}{1.547079in}}%
\pgfpathlineto{\pgfqpoint{2.880348in}{1.557022in}}%
\pgfpathlineto{\pgfqpoint{2.881212in}{1.648444in}}%
\pgfpathlineto{\pgfqpoint{2.882076in}{1.637639in}}%
\pgfpathlineto{\pgfqpoint{2.882941in}{1.609146in}}%
\pgfpathlineto{\pgfqpoint{2.883803in}{1.648325in}}%
\pgfpathlineto{\pgfqpoint{2.885532in}{1.471419in}}%
\pgfpathlineto{\pgfqpoint{2.887263in}{1.631050in}}%
\pgfpathlineto{\pgfqpoint{2.888127in}{1.630340in}}%
\pgfpathlineto{\pgfqpoint{2.889857in}{1.587836in}}%
\pgfpathlineto{\pgfqpoint{2.890721in}{1.603271in}}%
\pgfpathlineto{\pgfqpoint{2.891587in}{1.652810in}}%
\pgfpathlineto{\pgfqpoint{2.892452in}{1.650761in}}%
\pgfpathlineto{\pgfqpoint{2.893318in}{1.530667in}}%
\pgfpathlineto{\pgfqpoint{2.895047in}{1.597094in}}%
\pgfpathlineto{\pgfqpoint{2.895913in}{1.586289in}}%
\pgfpathlineto{\pgfqpoint{2.896779in}{1.600776in}}%
\pgfpathlineto{\pgfqpoint{2.898505in}{1.578217in}}%
\pgfpathlineto{\pgfqpoint{2.899368in}{1.584805in}}%
\pgfpathlineto{\pgfqpoint{2.900234in}{1.519564in}}%
\pgfpathlineto{\pgfqpoint{2.901964in}{1.700980in}}%
\pgfpathlineto{\pgfqpoint{2.902827in}{1.577562in}}%
\pgfpathlineto{\pgfqpoint{2.903693in}{1.603118in}}%
\pgfpathlineto{\pgfqpoint{2.904557in}{1.632058in}}%
\pgfpathlineto{\pgfqpoint{2.906286in}{1.540044in}}%
\pgfpathlineto{\pgfqpoint{2.909746in}{1.660376in}}%
\pgfpathlineto{\pgfqpoint{2.910611in}{1.618940in}}%
\pgfpathlineto{\pgfqpoint{2.911476in}{1.684776in}}%
\pgfpathlineto{\pgfqpoint{2.912341in}{1.610570in}}%
\pgfpathlineto{\pgfqpoint{2.913205in}{1.634493in}}%
\pgfpathlineto{\pgfqpoint{2.914071in}{1.579522in}}%
\pgfpathlineto{\pgfqpoint{2.914936in}{1.615377in}}%
\pgfpathlineto{\pgfqpoint{2.915801in}{1.539747in}}%
\pgfpathlineto{\pgfqpoint{2.916667in}{1.584330in}}%
\pgfpathlineto{\pgfqpoint{2.917532in}{1.688602in}}%
\pgfpathlineto{\pgfqpoint{2.920128in}{1.576049in}}%
\pgfpathlineto{\pgfqpoint{2.921858in}{1.635798in}}%
\pgfpathlineto{\pgfqpoint{2.922722in}{1.567471in}}%
\pgfpathlineto{\pgfqpoint{2.923588in}{1.648325in}}%
\pgfpathlineto{\pgfqpoint{2.926183in}{1.574892in}}%
\pgfpathlineto{\pgfqpoint{2.927911in}{1.622502in}}%
\pgfpathlineto{\pgfqpoint{2.928776in}{1.632355in}}%
\pgfpathlineto{\pgfqpoint{2.931370in}{1.737015in}}%
\pgfpathlineto{\pgfqpoint{2.932235in}{1.743901in}}%
\pgfpathlineto{\pgfqpoint{2.933100in}{1.724015in}}%
\pgfpathlineto{\pgfqpoint{2.933966in}{1.677473in}}%
\pgfpathlineto{\pgfqpoint{2.934831in}{1.708640in}}%
\pgfpathlineto{\pgfqpoint{2.936558in}{1.674179in}}%
\pgfpathlineto{\pgfqpoint{2.937423in}{1.727161in}}%
\pgfpathlineto{\pgfqpoint{2.938287in}{1.718970in}}%
\pgfpathlineto{\pgfqpoint{2.941744in}{1.682935in}}%
\pgfpathlineto{\pgfqpoint{2.942610in}{1.700389in}}%
\pgfpathlineto{\pgfqpoint{2.944340in}{1.648325in}}%
\pgfpathlineto{\pgfqpoint{2.945203in}{1.710778in}}%
\pgfpathlineto{\pgfqpoint{2.946068in}{1.680236in}}%
\pgfpathlineto{\pgfqpoint{2.947799in}{1.702765in}}%
\pgfpathlineto{\pgfqpoint{2.948660in}{1.646697in}}%
\pgfpathlineto{\pgfqpoint{2.949527in}{1.711314in}}%
\pgfpathlineto{\pgfqpoint{2.951255in}{1.546221in}}%
\pgfpathlineto{\pgfqpoint{2.952121in}{1.618467in}}%
\pgfpathlineto{\pgfqpoint{2.953848in}{1.528826in}}%
\pgfpathlineto{\pgfqpoint{2.956440in}{1.656283in}}%
\pgfpathlineto{\pgfqpoint{2.957306in}{1.539513in}}%
\pgfpathlineto{\pgfqpoint{2.958170in}{1.570442in}}%
\pgfpathlineto{\pgfqpoint{2.959036in}{1.537732in}}%
\pgfpathlineto{\pgfqpoint{2.959901in}{1.575130in}}%
\pgfpathlineto{\pgfqpoint{2.960766in}{1.532478in}}%
\pgfpathlineto{\pgfqpoint{2.962497in}{1.632240in}}%
\pgfpathlineto{\pgfqpoint{2.964228in}{1.541766in}}%
\pgfpathlineto{\pgfqpoint{2.965957in}{1.593770in}}%
\pgfpathlineto{\pgfqpoint{2.966821in}{1.613303in}}%
\pgfpathlineto{\pgfqpoint{2.967687in}{1.544112in}}%
\pgfpathlineto{\pgfqpoint{2.968551in}{1.601014in}}%
\pgfpathlineto{\pgfqpoint{2.970278in}{1.549783in}}%
\pgfpathlineto{\pgfqpoint{2.971143in}{1.642212in}}%
\pgfpathlineto{\pgfqpoint{2.972009in}{1.635267in}}%
\pgfpathlineto{\pgfqpoint{2.973738in}{1.590803in}}%
\pgfpathlineto{\pgfqpoint{2.974603in}{1.607900in}}%
\pgfpathlineto{\pgfqpoint{2.975468in}{1.573349in}}%
\pgfpathlineto{\pgfqpoint{2.976333in}{1.619832in}}%
\pgfpathlineto{\pgfqpoint{2.977198in}{1.580444in}}%
\pgfpathlineto{\pgfqpoint{2.978062in}{1.583857in}}%
\pgfpathlineto{\pgfqpoint{2.978927in}{1.607483in}}%
\pgfpathlineto{\pgfqpoint{2.979791in}{1.535356in}}%
\pgfpathlineto{\pgfqpoint{2.980655in}{1.614429in}}%
\pgfpathlineto{\pgfqpoint{2.981520in}{1.535534in}}%
\pgfpathlineto{\pgfqpoint{2.982384in}{1.582195in}}%
\pgfpathlineto{\pgfqpoint{2.983249in}{1.568185in}}%
\pgfpathlineto{\pgfqpoint{2.984114in}{1.579730in}}%
\pgfpathlineto{\pgfqpoint{2.984979in}{1.561239in}}%
\pgfpathlineto{\pgfqpoint{2.985844in}{1.586943in}}%
\pgfpathlineto{\pgfqpoint{2.986710in}{1.560703in}}%
\pgfpathlineto{\pgfqpoint{2.988440in}{1.602319in}}%
\pgfpathlineto{\pgfqpoint{2.989305in}{1.560882in}}%
\pgfpathlineto{\pgfqpoint{2.990170in}{1.679314in}}%
\pgfpathlineto{\pgfqpoint{2.991898in}{1.545626in}}%
\pgfpathlineto{\pgfqpoint{2.993628in}{1.629656in}}%
\pgfpathlineto{\pgfqpoint{2.994494in}{1.629150in}}%
\pgfpathlineto{\pgfqpoint{2.995360in}{1.637579in}}%
\pgfpathlineto{\pgfqpoint{2.996226in}{1.611756in}}%
\pgfpathlineto{\pgfqpoint{2.997090in}{1.627726in}}%
\pgfpathlineto{\pgfqpoint{2.997954in}{1.617396in}}%
\pgfpathlineto{\pgfqpoint{2.998820in}{1.624134in}}%
\pgfpathlineto{\pgfqpoint{2.999686in}{1.595432in}}%
\pgfpathlineto{\pgfqpoint{3.000551in}{1.619475in}}%
\pgfpathlineto{\pgfqpoint{3.002282in}{1.589851in}}%
\pgfpathlineto{\pgfqpoint{3.004010in}{1.620721in}}%
\pgfpathlineto{\pgfqpoint{3.004875in}{1.649512in}}%
\pgfpathlineto{\pgfqpoint{3.006606in}{1.583084in}}%
\pgfpathlineto{\pgfqpoint{3.007470in}{1.676109in}}%
\pgfpathlineto{\pgfqpoint{3.008335in}{1.643547in}}%
\pgfpathlineto{\pgfqpoint{3.009201in}{1.699556in}}%
\pgfpathlineto{\pgfqpoint{3.010065in}{1.650761in}}%
\pgfpathlineto{\pgfqpoint{3.010930in}{1.693503in}}%
\pgfpathlineto{\pgfqpoint{3.011793in}{1.669579in}}%
\pgfpathlineto{\pgfqpoint{3.014390in}{1.712381in}}%
\pgfpathlineto{\pgfqpoint{3.015255in}{1.715467in}}%
\pgfpathlineto{\pgfqpoint{3.016120in}{1.647318in}}%
\pgfpathlineto{\pgfqpoint{3.016985in}{1.657588in}}%
\pgfpathlineto{\pgfqpoint{3.018714in}{1.706502in}}%
\pgfpathlineto{\pgfqpoint{3.019578in}{1.708700in}}%
\pgfpathlineto{\pgfqpoint{3.021309in}{1.660495in}}%
\pgfpathlineto{\pgfqpoint{3.022174in}{1.719386in}}%
\pgfpathlineto{\pgfqpoint{3.024765in}{1.650136in}}%
\pgfpathlineto{\pgfqpoint{3.026495in}{1.731969in}}%
\pgfpathlineto{\pgfqpoint{3.027361in}{1.728436in}}%
\pgfpathlineto{\pgfqpoint{3.029093in}{1.675990in}}%
\pgfpathlineto{\pgfqpoint{3.029959in}{1.725499in}}%
\pgfpathlineto{\pgfqpoint{3.031687in}{1.615913in}}%
\pgfpathlineto{\pgfqpoint{3.032552in}{1.694749in}}%
\pgfpathlineto{\pgfqpoint{3.033415in}{1.655688in}}%
\pgfpathlineto{\pgfqpoint{3.034281in}{1.731880in}}%
\pgfpathlineto{\pgfqpoint{3.035146in}{1.684181in}}%
\pgfpathlineto{\pgfqpoint{3.036011in}{1.773465in}}%
\pgfpathlineto{\pgfqpoint{3.038605in}{1.540401in}}%
\pgfpathlineto{\pgfqpoint{3.040334in}{1.616802in}}%
\pgfpathlineto{\pgfqpoint{3.041199in}{1.709826in}}%
\pgfpathlineto{\pgfqpoint{3.042064in}{1.678778in}}%
\pgfpathlineto{\pgfqpoint{3.042930in}{1.709469in}}%
\pgfpathlineto{\pgfqpoint{3.043794in}{1.654264in}}%
\pgfpathlineto{\pgfqpoint{3.045523in}{1.715407in}}%
\pgfpathlineto{\pgfqpoint{3.046388in}{1.682697in}}%
\pgfpathlineto{\pgfqpoint{3.047252in}{1.713269in}}%
\pgfpathlineto{\pgfqpoint{3.048117in}{1.620661in}}%
\pgfpathlineto{\pgfqpoint{3.048983in}{1.708164in}}%
\pgfpathlineto{\pgfqpoint{3.049848in}{1.687327in}}%
\pgfpathlineto{\pgfqpoint{3.050713in}{1.707629in}}%
\pgfpathlineto{\pgfqpoint{3.051578in}{1.644466in}}%
\pgfpathlineto{\pgfqpoint{3.052443in}{1.702821in}}%
\pgfpathlineto{\pgfqpoint{3.053308in}{1.699140in}}%
\pgfpathlineto{\pgfqpoint{3.054173in}{1.726685in}}%
\pgfpathlineto{\pgfqpoint{3.055037in}{1.642744in}}%
\pgfpathlineto{\pgfqpoint{3.056766in}{1.695697in}}%
\pgfpathlineto{\pgfqpoint{3.057631in}{1.695756in}}%
\pgfpathlineto{\pgfqpoint{3.058495in}{1.767289in}}%
\pgfpathlineto{\pgfqpoint{3.061953in}{1.646663in}}%
\pgfpathlineto{\pgfqpoint{3.062818in}{1.721699in}}%
\pgfpathlineto{\pgfqpoint{3.063682in}{1.682757in}}%
\pgfpathlineto{\pgfqpoint{3.064545in}{1.689346in}}%
\pgfpathlineto{\pgfqpoint{3.065410in}{1.663939in}}%
\pgfpathlineto{\pgfqpoint{3.066275in}{1.670527in}}%
\pgfpathlineto{\pgfqpoint{3.067139in}{1.719680in}}%
\pgfpathlineto{\pgfqpoint{3.068004in}{1.701337in}}%
\pgfpathlineto{\pgfqpoint{3.068870in}{1.632950in}}%
\pgfpathlineto{\pgfqpoint{3.071465in}{1.714872in}}%
\pgfpathlineto{\pgfqpoint{3.072330in}{1.704900in}}%
\pgfpathlineto{\pgfqpoint{3.074059in}{1.765571in}}%
\pgfpathlineto{\pgfqpoint{3.074926in}{1.748028in}}%
\pgfpathlineto{\pgfqpoint{3.075790in}{1.677771in}}%
\pgfpathlineto{\pgfqpoint{3.076655in}{1.679730in}}%
\pgfpathlineto{\pgfqpoint{3.078385in}{1.694927in}}%
\pgfpathlineto{\pgfqpoint{3.080116in}{1.666846in}}%
\pgfpathlineto{\pgfqpoint{3.081847in}{1.728942in}}%
\pgfpathlineto{\pgfqpoint{3.082712in}{1.725261in}}%
\pgfpathlineto{\pgfqpoint{3.087036in}{1.586408in}}%
\pgfpathlineto{\pgfqpoint{3.088767in}{1.666936in}}%
\pgfpathlineto{\pgfqpoint{3.090498in}{1.548295in}}%
\pgfpathlineto{\pgfqpoint{3.092227in}{1.673376in}}%
\pgfpathlineto{\pgfqpoint{3.093092in}{1.656279in}}%
\pgfpathlineto{\pgfqpoint{3.093957in}{1.580827in}}%
\pgfpathlineto{\pgfqpoint{3.094822in}{1.613537in}}%
\pgfpathlineto{\pgfqpoint{3.095687in}{1.599824in}}%
\pgfpathlineto{\pgfqpoint{3.096552in}{1.618464in}}%
\pgfpathlineto{\pgfqpoint{3.097417in}{1.579581in}}%
\pgfpathlineto{\pgfqpoint{3.098282in}{1.585545in}}%
\pgfpathlineto{\pgfqpoint{3.099147in}{1.581243in}}%
\pgfpathlineto{\pgfqpoint{3.100878in}{1.612351in}}%
\pgfpathlineto{\pgfqpoint{3.101742in}{1.608729in}}%
\pgfpathlineto{\pgfqpoint{3.102605in}{1.613596in}}%
\pgfpathlineto{\pgfqpoint{3.104334in}{1.675514in}}%
\pgfpathlineto{\pgfqpoint{3.106065in}{1.583203in}}%
\pgfpathlineto{\pgfqpoint{3.106930in}{1.616032in}}%
\pgfpathlineto{\pgfqpoint{3.107795in}{1.576108in}}%
\pgfpathlineto{\pgfqpoint{3.108661in}{1.637758in}}%
\pgfpathlineto{\pgfqpoint{3.110391in}{1.586051in}}%
\pgfpathlineto{\pgfqpoint{3.111255in}{1.598400in}}%
\pgfpathlineto{\pgfqpoint{3.113850in}{1.547641in}}%
\pgfpathlineto{\pgfqpoint{3.114714in}{1.563790in}}%
\pgfpathlineto{\pgfqpoint{3.115579in}{1.610391in}}%
\pgfpathlineto{\pgfqpoint{3.117306in}{1.519058in}}%
\pgfpathlineto{\pgfqpoint{3.118172in}{1.519207in}}%
\pgfpathlineto{\pgfqpoint{3.121631in}{1.631407in}}%
\pgfpathlineto{\pgfqpoint{3.124228in}{1.554412in}}%
\pgfpathlineto{\pgfqpoint{3.125092in}{1.564801in}}%
\pgfpathlineto{\pgfqpoint{3.125956in}{1.628321in}}%
\pgfpathlineto{\pgfqpoint{3.128550in}{1.524435in}}%
\pgfpathlineto{\pgfqpoint{3.129414in}{1.619743in}}%
\pgfpathlineto{\pgfqpoint{3.132007in}{1.530697in}}%
\pgfpathlineto{\pgfqpoint{3.132874in}{1.622918in}}%
\pgfpathlineto{\pgfqpoint{3.133740in}{1.614310in}}%
\pgfpathlineto{\pgfqpoint{3.134605in}{1.557677in}}%
\pgfpathlineto{\pgfqpoint{3.135471in}{1.585519in}}%
\pgfpathlineto{\pgfqpoint{3.136337in}{1.549366in}}%
\pgfpathlineto{\pgfqpoint{3.138066in}{1.646961in}}%
\pgfpathlineto{\pgfqpoint{3.139795in}{1.587538in}}%
\pgfpathlineto{\pgfqpoint{3.141527in}{1.551118in}}%
\pgfpathlineto{\pgfqpoint{3.142392in}{1.604397in}}%
\pgfpathlineto{\pgfqpoint{3.143259in}{1.591632in}}%
\pgfpathlineto{\pgfqpoint{3.144124in}{1.583143in}}%
\pgfpathlineto{\pgfqpoint{3.144990in}{1.593651in}}%
\pgfpathlineto{\pgfqpoint{3.145857in}{1.569757in}}%
\pgfpathlineto{\pgfqpoint{3.146723in}{1.768241in}}%
\pgfpathlineto{\pgfqpoint{3.148455in}{1.674714in}}%
\pgfpathlineto{\pgfqpoint{3.149320in}{1.662395in}}%
\pgfpathlineto{\pgfqpoint{3.150186in}{1.708878in}}%
\pgfpathlineto{\pgfqpoint{3.151050in}{1.636542in}}%
\pgfpathlineto{\pgfqpoint{3.151916in}{1.731378in}}%
\pgfpathlineto{\pgfqpoint{3.152782in}{1.710600in}}%
\pgfpathlineto{\pgfqpoint{3.154510in}{1.685133in}}%
\pgfpathlineto{\pgfqpoint{3.155373in}{1.695403in}}%
\pgfpathlineto{\pgfqpoint{3.156236in}{1.731259in}}%
\pgfpathlineto{\pgfqpoint{3.157967in}{1.662544in}}%
\pgfpathlineto{\pgfqpoint{3.158829in}{1.730664in}}%
\pgfpathlineto{\pgfqpoint{3.160561in}{1.696768in}}%
\pgfpathlineto{\pgfqpoint{3.161425in}{1.698370in}}%
\pgfpathlineto{\pgfqpoint{3.162292in}{1.682995in}}%
\pgfpathlineto{\pgfqpoint{3.163156in}{1.689941in}}%
\pgfpathlineto{\pgfqpoint{3.164021in}{1.805699in}}%
\pgfpathlineto{\pgfqpoint{3.164886in}{1.689881in}}%
\pgfpathlineto{\pgfqpoint{3.166615in}{1.797686in}}%
\pgfpathlineto{\pgfqpoint{3.167480in}{1.676376in}}%
\pgfpathlineto{\pgfqpoint{3.168345in}{1.727637in}}%
\pgfpathlineto{\pgfqpoint{3.169210in}{1.715645in}}%
\pgfpathlineto{\pgfqpoint{3.170075in}{1.681482in}}%
\pgfpathlineto{\pgfqpoint{3.172667in}{1.721937in}}%
\pgfpathlineto{\pgfqpoint{3.173530in}{1.569787in}}%
\pgfpathlineto{\pgfqpoint{3.174393in}{1.598162in}}%
\pgfpathlineto{\pgfqpoint{3.175258in}{1.628853in}}%
\pgfpathlineto{\pgfqpoint{3.176124in}{1.620066in}}%
\pgfpathlineto{\pgfqpoint{3.176989in}{1.551709in}}%
\pgfpathlineto{\pgfqpoint{3.177853in}{1.635858in}}%
\pgfpathlineto{\pgfqpoint{3.178719in}{1.576138in}}%
\pgfpathlineto{\pgfqpoint{3.179583in}{1.583114in}}%
\pgfpathlineto{\pgfqpoint{3.180447in}{1.584865in}}%
\pgfpathlineto{\pgfqpoint{3.181311in}{1.581779in}}%
\pgfpathlineto{\pgfqpoint{3.182175in}{1.583084in}}%
\pgfpathlineto{\pgfqpoint{3.183039in}{1.607364in}}%
\pgfpathlineto{\pgfqpoint{3.184769in}{1.551207in}}%
\pgfpathlineto{\pgfqpoint{3.185635in}{1.564266in}}%
\pgfpathlineto{\pgfqpoint{3.186500in}{1.600210in}}%
\pgfpathlineto{\pgfqpoint{3.187365in}{1.583203in}}%
\pgfpathlineto{\pgfqpoint{3.188230in}{1.658179in}}%
\pgfpathlineto{\pgfqpoint{3.189960in}{1.554055in}}%
\pgfpathlineto{\pgfqpoint{3.190825in}{1.638349in}}%
\pgfpathlineto{\pgfqpoint{3.191688in}{1.623331in}}%
\pgfpathlineto{\pgfqpoint{3.192553in}{1.622145in}}%
\pgfpathlineto{\pgfqpoint{3.193417in}{1.543428in}}%
\pgfpathlineto{\pgfqpoint{3.196006in}{1.620661in}}%
\pgfpathlineto{\pgfqpoint{3.197734in}{1.655331in}}%
\pgfpathlineto{\pgfqpoint{3.200327in}{1.574416in}}%
\pgfpathlineto{\pgfqpoint{3.201193in}{1.575662in}}%
\pgfpathlineto{\pgfqpoint{3.202058in}{1.563730in}}%
\pgfpathlineto{\pgfqpoint{3.204651in}{1.627488in}}%
\pgfpathlineto{\pgfqpoint{3.205515in}{1.625380in}}%
\pgfpathlineto{\pgfqpoint{3.206381in}{1.622264in}}%
\pgfpathlineto{\pgfqpoint{3.207245in}{1.585222in}}%
\pgfpathlineto{\pgfqpoint{3.208109in}{1.687892in}}%
\pgfpathlineto{\pgfqpoint{3.208975in}{1.560941in}}%
\pgfpathlineto{\pgfqpoint{3.210707in}{1.614072in}}%
\pgfpathlineto{\pgfqpoint{3.211573in}{1.596857in}}%
\pgfpathlineto{\pgfqpoint{3.212439in}{1.677533in}}%
\pgfpathlineto{\pgfqpoint{3.213304in}{1.569490in}}%
\pgfpathlineto{\pgfqpoint{3.214169in}{1.627131in}}%
\pgfpathlineto{\pgfqpoint{3.215032in}{1.603743in}}%
\pgfpathlineto{\pgfqpoint{3.215896in}{1.541707in}}%
\pgfpathlineto{\pgfqpoint{3.216762in}{1.613953in}}%
\pgfpathlineto{\pgfqpoint{3.217627in}{1.596529in}}%
\pgfpathlineto{\pgfqpoint{3.218492in}{1.630872in}}%
\pgfpathlineto{\pgfqpoint{3.219357in}{1.617515in}}%
\pgfpathlineto{\pgfqpoint{3.220222in}{1.569192in}}%
\pgfpathlineto{\pgfqpoint{3.221087in}{1.610510in}}%
\pgfpathlineto{\pgfqpoint{3.221951in}{1.606383in}}%
\pgfpathlineto{\pgfqpoint{3.222816in}{1.558327in}}%
\pgfpathlineto{\pgfqpoint{3.224547in}{1.640844in}}%
\pgfpathlineto{\pgfqpoint{3.225412in}{1.508699in}}%
\pgfpathlineto{\pgfqpoint{3.226276in}{1.523305in}}%
\pgfpathlineto{\pgfqpoint{3.227141in}{1.574119in}}%
\pgfpathlineto{\pgfqpoint{3.228006in}{1.572516in}}%
\pgfpathlineto{\pgfqpoint{3.228872in}{1.560584in}}%
\pgfpathlineto{\pgfqpoint{3.230605in}{1.679373in}}%
\pgfpathlineto{\pgfqpoint{3.234064in}{1.528321in}}%
\pgfpathlineto{\pgfqpoint{3.235794in}{1.666549in}}%
\pgfpathlineto{\pgfqpoint{3.236660in}{1.635947in}}%
\pgfpathlineto{\pgfqpoint{3.237525in}{1.565809in}}%
\pgfpathlineto{\pgfqpoint{3.238390in}{1.589970in}}%
\pgfpathlineto{\pgfqpoint{3.239254in}{1.531258in}}%
\pgfpathlineto{\pgfqpoint{3.240120in}{1.613418in}}%
\pgfpathlineto{\pgfqpoint{3.240984in}{1.601545in}}%
\pgfpathlineto{\pgfqpoint{3.241849in}{1.586587in}}%
\pgfpathlineto{\pgfqpoint{3.242713in}{1.524550in}}%
\pgfpathlineto{\pgfqpoint{3.244443in}{1.652007in}}%
\pgfpathlineto{\pgfqpoint{3.245308in}{1.586825in}}%
\pgfpathlineto{\pgfqpoint{3.246173in}{1.644317in}}%
\pgfpathlineto{\pgfqpoint{3.247039in}{1.602140in}}%
\pgfpathlineto{\pgfqpoint{3.247901in}{1.664649in}}%
\pgfpathlineto{\pgfqpoint{3.248767in}{1.659900in}}%
\pgfpathlineto{\pgfqpoint{3.249633in}{1.653431in}}%
\pgfpathlineto{\pgfqpoint{3.250497in}{1.620840in}}%
\pgfpathlineto{\pgfqpoint{3.251361in}{1.644644in}}%
\pgfpathlineto{\pgfqpoint{3.252226in}{1.612529in}}%
\pgfpathlineto{\pgfqpoint{3.253089in}{1.648117in}}%
\pgfpathlineto{\pgfqpoint{3.253954in}{1.639420in}}%
\pgfpathlineto{\pgfqpoint{3.254819in}{1.656338in}}%
\pgfpathlineto{\pgfqpoint{3.255685in}{1.541736in}}%
\pgfpathlineto{\pgfqpoint{3.256550in}{1.628972in}}%
\pgfpathlineto{\pgfqpoint{3.257415in}{1.628377in}}%
\pgfpathlineto{\pgfqpoint{3.258280in}{1.633422in}}%
\pgfpathlineto{\pgfqpoint{3.259145in}{1.603326in}}%
\pgfpathlineto{\pgfqpoint{3.260010in}{1.702583in}}%
\pgfpathlineto{\pgfqpoint{3.260876in}{1.617515in}}%
\pgfpathlineto{\pgfqpoint{3.261740in}{1.711726in}}%
\pgfpathlineto{\pgfqpoint{3.262605in}{1.594986in}}%
\pgfpathlineto{\pgfqpoint{3.264334in}{1.739034in}}%
\pgfpathlineto{\pgfqpoint{3.266929in}{1.672903in}}%
\pgfpathlineto{\pgfqpoint{3.267794in}{1.714694in}}%
\pgfpathlineto{\pgfqpoint{3.268657in}{1.714634in}}%
\pgfpathlineto{\pgfqpoint{3.269522in}{1.666995in}}%
\pgfpathlineto{\pgfqpoint{3.272114in}{1.721788in}}%
\pgfpathlineto{\pgfqpoint{3.273846in}{1.680202in}}%
\pgfpathlineto{\pgfqpoint{3.274709in}{1.704305in}}%
\pgfpathlineto{\pgfqpoint{3.275573in}{1.674030in}}%
\pgfpathlineto{\pgfqpoint{3.277300in}{1.732088in}}%
\pgfpathlineto{\pgfqpoint{3.279893in}{1.700032in}}%
\pgfpathlineto{\pgfqpoint{3.280759in}{1.705732in}}%
\pgfpathlineto{\pgfqpoint{3.282489in}{1.652007in}}%
\pgfpathlineto{\pgfqpoint{3.283353in}{1.655033in}}%
\pgfpathlineto{\pgfqpoint{3.285084in}{1.704245in}}%
\pgfpathlineto{\pgfqpoint{3.285950in}{1.686260in}}%
\pgfpathlineto{\pgfqpoint{3.287681in}{1.704781in}}%
\pgfpathlineto{\pgfqpoint{3.289412in}{1.714456in}}%
\pgfpathlineto{\pgfqpoint{3.290277in}{1.708402in}}%
\pgfpathlineto{\pgfqpoint{3.291141in}{1.693116in}}%
\pgfpathlineto{\pgfqpoint{3.292006in}{1.645120in}}%
\pgfpathlineto{\pgfqpoint{3.292872in}{1.708997in}}%
\pgfpathlineto{\pgfqpoint{3.293738in}{1.658893in}}%
\pgfpathlineto{\pgfqpoint{3.294599in}{1.748355in}}%
\pgfpathlineto{\pgfqpoint{3.295464in}{1.686260in}}%
\pgfpathlineto{\pgfqpoint{3.296325in}{1.761771in}}%
\pgfpathlineto{\pgfqpoint{3.297192in}{1.670944in}}%
\pgfpathlineto{\pgfqpoint{3.298058in}{1.707246in}}%
\pgfpathlineto{\pgfqpoint{3.298922in}{1.552036in}}%
\pgfpathlineto{\pgfqpoint{3.299786in}{1.637460in}}%
\pgfpathlineto{\pgfqpoint{3.300649in}{1.632266in}}%
\pgfpathlineto{\pgfqpoint{3.301516in}{1.640785in}}%
\pgfpathlineto{\pgfqpoint{3.302379in}{1.585932in}}%
\pgfpathlineto{\pgfqpoint{3.303244in}{1.620155in}}%
\pgfpathlineto{\pgfqpoint{3.304108in}{1.559398in}}%
\pgfpathlineto{\pgfqpoint{3.304975in}{1.672071in}}%
\pgfpathlineto{\pgfqpoint{3.305841in}{1.653133in}}%
\pgfpathlineto{\pgfqpoint{3.307572in}{1.596053in}}%
\pgfpathlineto{\pgfqpoint{3.308436in}{1.611518in}}%
\pgfpathlineto{\pgfqpoint{3.310168in}{1.580440in}}%
\pgfpathlineto{\pgfqpoint{3.311899in}{1.655684in}}%
\pgfpathlineto{\pgfqpoint{3.312762in}{1.607450in}}%
\pgfpathlineto{\pgfqpoint{3.313628in}{1.656930in}}%
\pgfpathlineto{\pgfqpoint{3.315359in}{1.587594in}}%
\pgfpathlineto{\pgfqpoint{3.317088in}{1.686970in}}%
\pgfpathlineto{\pgfqpoint{3.318817in}{1.627309in}}%
\pgfpathlineto{\pgfqpoint{3.319683in}{1.647611in}}%
\pgfpathlineto{\pgfqpoint{3.322275in}{1.549779in}}%
\pgfpathlineto{\pgfqpoint{3.324004in}{1.632058in}}%
\pgfpathlineto{\pgfqpoint{3.324870in}{1.643871in}}%
\pgfpathlineto{\pgfqpoint{3.325734in}{1.622026in}}%
\pgfpathlineto{\pgfqpoint{3.326597in}{1.625796in}}%
\pgfpathlineto{\pgfqpoint{3.327461in}{1.644763in}}%
\pgfpathlineto{\pgfqpoint{3.328325in}{1.560941in}}%
\pgfpathlineto{\pgfqpoint{3.329190in}{1.568627in}}%
\pgfpathlineto{\pgfqpoint{3.330917in}{1.608015in}}%
\pgfpathlineto{\pgfqpoint{3.331782in}{1.587773in}}%
\pgfpathlineto{\pgfqpoint{3.332647in}{1.618761in}}%
\pgfpathlineto{\pgfqpoint{3.335241in}{1.569073in}}%
\pgfpathlineto{\pgfqpoint{3.336971in}{1.644168in}}%
\pgfpathlineto{\pgfqpoint{3.337833in}{1.599407in}}%
\pgfpathlineto{\pgfqpoint{3.338698in}{1.608283in}}%
\pgfpathlineto{\pgfqpoint{3.340427in}{1.531139in}}%
\pgfpathlineto{\pgfqpoint{3.343019in}{1.652598in}}%
\pgfpathlineto{\pgfqpoint{3.343883in}{1.582132in}}%
\pgfpathlineto{\pgfqpoint{3.344746in}{1.648266in}}%
\pgfpathlineto{\pgfqpoint{3.345611in}{1.549128in}}%
\pgfpathlineto{\pgfqpoint{3.346476in}{1.659782in}}%
\pgfpathlineto{\pgfqpoint{3.347340in}{1.632653in}}%
\pgfpathlineto{\pgfqpoint{3.349071in}{1.615496in}}%
\pgfpathlineto{\pgfqpoint{3.351667in}{1.565987in}}%
\pgfpathlineto{\pgfqpoint{3.353399in}{1.660555in}}%
\pgfpathlineto{\pgfqpoint{3.354265in}{1.700623in}}%
\pgfpathlineto{\pgfqpoint{3.355130in}{1.542034in}}%
\pgfpathlineto{\pgfqpoint{3.356859in}{1.639063in}}%
\pgfpathlineto{\pgfqpoint{3.358588in}{1.576436in}}%
\pgfpathlineto{\pgfqpoint{3.359454in}{1.583441in}}%
\pgfpathlineto{\pgfqpoint{3.360319in}{1.552274in}}%
\pgfpathlineto{\pgfqpoint{3.361184in}{1.661860in}}%
\pgfpathlineto{\pgfqpoint{3.363780in}{1.570676in}}%
\pgfpathlineto{\pgfqpoint{3.364643in}{1.627191in}}%
\pgfpathlineto{\pgfqpoint{3.365508in}{1.620721in}}%
\pgfpathlineto{\pgfqpoint{3.366374in}{1.595016in}}%
\pgfpathlineto{\pgfqpoint{3.367238in}{1.608045in}}%
\pgfpathlineto{\pgfqpoint{3.368103in}{1.656279in}}%
\pgfpathlineto{\pgfqpoint{3.369832in}{1.626923in}}%
\pgfpathlineto{\pgfqpoint{3.370697in}{1.524193in}}%
\pgfpathlineto{\pgfqpoint{3.373294in}{1.619356in}}%
\pgfpathlineto{\pgfqpoint{3.374159in}{1.600329in}}%
\pgfpathlineto{\pgfqpoint{3.375889in}{1.636155in}}%
\pgfpathlineto{\pgfqpoint{3.376753in}{1.633928in}}%
\pgfpathlineto{\pgfqpoint{3.377619in}{1.628023in}}%
\pgfpathlineto{\pgfqpoint{3.378486in}{1.608670in}}%
\pgfpathlineto{\pgfqpoint{3.379350in}{1.548831in}}%
\pgfpathlineto{\pgfqpoint{3.380214in}{1.567649in}}%
\pgfpathlineto{\pgfqpoint{3.381080in}{1.631760in}}%
\pgfpathlineto{\pgfqpoint{3.381944in}{1.602731in}}%
\pgfpathlineto{\pgfqpoint{3.382810in}{1.626893in}}%
\pgfpathlineto{\pgfqpoint{3.385402in}{1.592461in}}%
\pgfpathlineto{\pgfqpoint{3.387132in}{1.692075in}}%
\pgfpathlineto{\pgfqpoint{3.388863in}{1.557201in}}%
\pgfpathlineto{\pgfqpoint{3.390594in}{1.654141in}}%
\pgfpathlineto{\pgfqpoint{3.391460in}{1.637223in}}%
\pgfpathlineto{\pgfqpoint{3.392325in}{1.640368in}}%
\pgfpathlineto{\pgfqpoint{3.393191in}{1.657108in}}%
\pgfpathlineto{\pgfqpoint{3.394056in}{1.605758in}}%
\pgfpathlineto{\pgfqpoint{3.394921in}{1.633601in}}%
\pgfpathlineto{\pgfqpoint{3.395786in}{1.586259in}}%
\pgfpathlineto{\pgfqpoint{3.400110in}{1.681746in}}%
\pgfpathlineto{\pgfqpoint{3.401841in}{1.565154in}}%
\pgfpathlineto{\pgfqpoint{3.402704in}{1.538055in}}%
\pgfpathlineto{\pgfqpoint{3.404434in}{1.630277in}}%
\pgfpathlineto{\pgfqpoint{3.406164in}{1.587654in}}%
\pgfpathlineto{\pgfqpoint{3.407029in}{1.606472in}}%
\pgfpathlineto{\pgfqpoint{3.407893in}{1.589405in}}%
\pgfpathlineto{\pgfqpoint{3.408757in}{1.548355in}}%
\pgfpathlineto{\pgfqpoint{3.409621in}{1.596024in}}%
\pgfpathlineto{\pgfqpoint{3.410487in}{1.519088in}}%
\pgfpathlineto{\pgfqpoint{3.412214in}{1.628377in}}%
\pgfpathlineto{\pgfqpoint{3.413080in}{1.580767in}}%
\pgfpathlineto{\pgfqpoint{3.413944in}{1.588364in}}%
\pgfpathlineto{\pgfqpoint{3.414810in}{1.593767in}}%
\pgfpathlineto{\pgfqpoint{3.415676in}{1.583794in}}%
\pgfpathlineto{\pgfqpoint{3.416540in}{1.624933in}}%
\pgfpathlineto{\pgfqpoint{3.417405in}{1.581626in}}%
\pgfpathlineto{\pgfqpoint{3.418269in}{1.653784in}}%
\pgfpathlineto{\pgfqpoint{3.419134in}{1.648976in}}%
\pgfpathlineto{\pgfqpoint{3.419999in}{1.626358in}}%
\pgfpathlineto{\pgfqpoint{3.420865in}{1.652598in}}%
\pgfpathlineto{\pgfqpoint{3.421731in}{1.611518in}}%
\pgfpathlineto{\pgfqpoint{3.422597in}{1.666311in}}%
\pgfpathlineto{\pgfqpoint{3.425191in}{1.599348in}}%
\pgfpathlineto{\pgfqpoint{3.426054in}{1.664411in}}%
\pgfpathlineto{\pgfqpoint{3.427782in}{1.566935in}}%
\pgfpathlineto{\pgfqpoint{3.428646in}{1.636568in}}%
\pgfpathlineto{\pgfqpoint{3.429510in}{1.624815in}}%
\pgfpathlineto{\pgfqpoint{3.430376in}{1.601724in}}%
\pgfpathlineto{\pgfqpoint{3.432107in}{1.649155in}}%
\pgfpathlineto{\pgfqpoint{3.433836in}{1.626417in}}%
\pgfpathlineto{\pgfqpoint{3.434701in}{1.645652in}}%
\pgfpathlineto{\pgfqpoint{3.435566in}{1.639714in}}%
\pgfpathlineto{\pgfqpoint{3.436430in}{1.645325in}}%
\pgfpathlineto{\pgfqpoint{3.437295in}{1.602791in}}%
\pgfpathlineto{\pgfqpoint{3.438159in}{1.655862in}}%
\pgfpathlineto{\pgfqpoint{3.439025in}{1.590323in}}%
\pgfpathlineto{\pgfqpoint{3.439889in}{1.630396in}}%
\pgfpathlineto{\pgfqpoint{3.440753in}{1.626536in}}%
\pgfpathlineto{\pgfqpoint{3.441619in}{1.625290in}}%
\pgfpathlineto{\pgfqpoint{3.442483in}{1.632058in}}%
\pgfpathlineto{\pgfqpoint{3.444212in}{1.596024in}}%
\pgfpathlineto{\pgfqpoint{3.445076in}{1.625052in}}%
\pgfpathlineto{\pgfqpoint{3.445941in}{1.572397in}}%
\pgfpathlineto{\pgfqpoint{3.446805in}{1.627369in}}%
\pgfpathlineto{\pgfqpoint{3.447669in}{1.626655in}}%
\pgfpathlineto{\pgfqpoint{3.448533in}{1.601069in}}%
\pgfpathlineto{\pgfqpoint{3.450259in}{1.667586in}}%
\pgfpathlineto{\pgfqpoint{3.451124in}{1.605996in}}%
\pgfpathlineto{\pgfqpoint{3.451988in}{1.635084in}}%
\pgfpathlineto{\pgfqpoint{3.452853in}{1.592372in}}%
\pgfpathlineto{\pgfqpoint{3.454582in}{1.631998in}}%
\pgfpathlineto{\pgfqpoint{3.456310in}{1.581597in}}%
\pgfpathlineto{\pgfqpoint{3.457175in}{1.674383in}}%
\pgfpathlineto{\pgfqpoint{3.458040in}{1.619769in}}%
\pgfpathlineto{\pgfqpoint{3.458905in}{1.645116in}}%
\pgfpathlineto{\pgfqpoint{3.459770in}{1.578451in}}%
\pgfpathlineto{\pgfqpoint{3.460636in}{1.650460in}}%
\pgfpathlineto{\pgfqpoint{3.462364in}{1.627811in}}%
\pgfpathlineto{\pgfqpoint{3.463229in}{1.650222in}}%
\pgfpathlineto{\pgfqpoint{3.464094in}{1.647017in}}%
\pgfpathlineto{\pgfqpoint{3.464958in}{1.513031in}}%
\pgfpathlineto{\pgfqpoint{3.465821in}{1.551441in}}%
\pgfpathlineto{\pgfqpoint{3.467552in}{1.615671in}}%
\pgfpathlineto{\pgfqpoint{3.468416in}{1.605996in}}%
\pgfpathlineto{\pgfqpoint{3.469282in}{1.648768in}}%
\pgfpathlineto{\pgfqpoint{3.470145in}{1.611161in}}%
\pgfpathlineto{\pgfqpoint{3.471011in}{1.624815in}}%
\pgfpathlineto{\pgfqpoint{3.472743in}{1.605996in}}%
\pgfpathlineto{\pgfqpoint{3.473609in}{1.616207in}}%
\pgfpathlineto{\pgfqpoint{3.474471in}{1.561295in}}%
\pgfpathlineto{\pgfqpoint{3.475336in}{1.602434in}}%
\pgfpathlineto{\pgfqpoint{3.476201in}{1.598039in}}%
\pgfpathlineto{\pgfqpoint{3.477067in}{1.603561in}}%
\pgfpathlineto{\pgfqpoint{3.477933in}{1.624339in}}%
\pgfpathlineto{\pgfqpoint{3.478798in}{1.620925in}}%
\pgfpathlineto{\pgfqpoint{3.480525in}{1.574770in}}%
\pgfpathlineto{\pgfqpoint{3.481389in}{1.580793in}}%
\pgfpathlineto{\pgfqpoint{3.482252in}{1.605520in}}%
\pgfpathlineto{\pgfqpoint{3.483118in}{1.696645in}}%
\pgfpathlineto{\pgfqpoint{3.484847in}{1.621609in}}%
\pgfpathlineto{\pgfqpoint{3.485710in}{1.647433in}}%
\pgfpathlineto{\pgfqpoint{3.486575in}{1.594481in}}%
\pgfpathlineto{\pgfqpoint{3.487439in}{1.623271in}}%
\pgfpathlineto{\pgfqpoint{3.488304in}{1.569192in}}%
\pgfpathlineto{\pgfqpoint{3.489170in}{1.572397in}}%
\pgfpathlineto{\pgfqpoint{3.490901in}{1.650698in}}%
\pgfpathlineto{\pgfqpoint{3.491766in}{1.611756in}}%
\pgfpathlineto{\pgfqpoint{3.492631in}{1.629150in}}%
\pgfpathlineto{\pgfqpoint{3.494360in}{1.542301in}}%
\pgfpathlineto{\pgfqpoint{3.495226in}{1.569073in}}%
\pgfpathlineto{\pgfqpoint{3.496091in}{1.531258in}}%
\pgfpathlineto{\pgfqpoint{3.498684in}{1.660138in}}%
\pgfpathlineto{\pgfqpoint{3.499548in}{1.655212in}}%
\pgfpathlineto{\pgfqpoint{3.502142in}{1.593056in}}%
\pgfpathlineto{\pgfqpoint{3.503006in}{1.613091in}}%
\pgfpathlineto{\pgfqpoint{3.504734in}{1.674800in}}%
\pgfpathlineto{\pgfqpoint{3.506463in}{1.622204in}}%
\pgfpathlineto{\pgfqpoint{3.507329in}{1.646158in}}%
\pgfpathlineto{\pgfqpoint{3.508192in}{1.570854in}}%
\pgfpathlineto{\pgfqpoint{3.509923in}{1.639301in}}%
\pgfpathlineto{\pgfqpoint{3.510788in}{1.631998in}}%
\pgfpathlineto{\pgfqpoint{3.512519in}{1.550314in}}%
\pgfpathlineto{\pgfqpoint{3.513384in}{1.610034in}}%
\pgfpathlineto{\pgfqpoint{3.515113in}{1.565511in}}%
\pgfpathlineto{\pgfqpoint{3.515979in}{1.568954in}}%
\pgfpathlineto{\pgfqpoint{3.516844in}{1.620572in}}%
\pgfpathlineto{\pgfqpoint{3.517709in}{1.575484in}}%
\pgfpathlineto{\pgfqpoint{3.518574in}{1.614723in}}%
\pgfpathlineto{\pgfqpoint{3.519437in}{1.599259in}}%
\pgfpathlineto{\pgfqpoint{3.520301in}{1.553163in}}%
\pgfpathlineto{\pgfqpoint{3.523762in}{1.688691in}}%
\pgfpathlineto{\pgfqpoint{3.525491in}{1.580470in}}%
\pgfpathlineto{\pgfqpoint{3.526355in}{1.590472in}}%
\pgfpathlineto{\pgfqpoint{3.527220in}{1.629146in}}%
\pgfpathlineto{\pgfqpoint{3.529817in}{1.544852in}}%
\pgfpathlineto{\pgfqpoint{3.530682in}{1.618523in}}%
\pgfpathlineto{\pgfqpoint{3.531546in}{1.579314in}}%
\pgfpathlineto{\pgfqpoint{3.533278in}{1.636453in}}%
\pgfpathlineto{\pgfqpoint{3.534143in}{1.578038in}}%
\pgfpathlineto{\pgfqpoint{3.535006in}{1.596916in}}%
\pgfpathlineto{\pgfqpoint{3.536735in}{1.558803in}}%
\pgfpathlineto{\pgfqpoint{3.537601in}{1.584984in}}%
\pgfpathlineto{\pgfqpoint{3.538466in}{1.584508in}}%
\pgfpathlineto{\pgfqpoint{3.539331in}{1.584805in}}%
\pgfpathlineto{\pgfqpoint{3.540196in}{1.617396in}}%
\pgfpathlineto{\pgfqpoint{3.541059in}{1.578455in}}%
\pgfpathlineto{\pgfqpoint{3.541923in}{1.588427in}}%
\pgfpathlineto{\pgfqpoint{3.543652in}{1.562603in}}%
\pgfpathlineto{\pgfqpoint{3.544515in}{1.621847in}}%
\pgfpathlineto{\pgfqpoint{3.545379in}{1.555271in}}%
\pgfpathlineto{\pgfqpoint{3.547110in}{1.654855in}}%
\pgfpathlineto{\pgfqpoint{3.547975in}{1.619297in}}%
\pgfpathlineto{\pgfqpoint{3.548839in}{1.727221in}}%
\pgfpathlineto{\pgfqpoint{3.550568in}{1.595105in}}%
\pgfpathlineto{\pgfqpoint{3.551434in}{1.662039in}}%
\pgfpathlineto{\pgfqpoint{3.552299in}{1.614310in}}%
\pgfpathlineto{\pgfqpoint{3.553165in}{1.624402in}}%
\pgfpathlineto{\pgfqpoint{3.554896in}{1.585728in}}%
\pgfpathlineto{\pgfqpoint{3.557492in}{1.686141in}}%
\pgfpathlineto{\pgfqpoint{3.558357in}{1.639896in}}%
\pgfpathlineto{\pgfqpoint{3.559222in}{1.663879in}}%
\pgfpathlineto{\pgfqpoint{3.560087in}{1.533158in}}%
\pgfpathlineto{\pgfqpoint{3.561816in}{1.641201in}}%
\pgfpathlineto{\pgfqpoint{3.562682in}{1.627488in}}%
\pgfpathlineto{\pgfqpoint{3.563548in}{1.579938in}}%
\pgfpathlineto{\pgfqpoint{3.565276in}{1.663109in}}%
\pgfpathlineto{\pgfqpoint{3.566138in}{1.712083in}}%
\pgfpathlineto{\pgfqpoint{3.567868in}{1.648801in}}%
\pgfpathlineto{\pgfqpoint{3.569599in}{1.703475in}}%
\pgfpathlineto{\pgfqpoint{3.570464in}{1.708878in}}%
\pgfpathlineto{\pgfqpoint{3.571329in}{1.677711in}}%
\pgfpathlineto{\pgfqpoint{3.572193in}{1.686200in}}%
\pgfpathlineto{\pgfqpoint{3.573058in}{1.773168in}}%
\pgfpathlineto{\pgfqpoint{3.573923in}{1.633724in}}%
\pgfpathlineto{\pgfqpoint{3.575651in}{1.738260in}}%
\pgfpathlineto{\pgfqpoint{3.576516in}{1.670587in}}%
\pgfpathlineto{\pgfqpoint{3.577381in}{1.704483in}}%
\pgfpathlineto{\pgfqpoint{3.579112in}{1.666757in}}%
\pgfpathlineto{\pgfqpoint{3.579978in}{1.689524in}}%
\pgfpathlineto{\pgfqpoint{3.580842in}{1.742001in}}%
\pgfpathlineto{\pgfqpoint{3.582573in}{1.686022in}}%
\pgfpathlineto{\pgfqpoint{3.583437in}{1.661830in}}%
\pgfpathlineto{\pgfqpoint{3.585164in}{1.683170in}}%
\pgfpathlineto{\pgfqpoint{3.586028in}{1.675246in}}%
\pgfpathlineto{\pgfqpoint{3.586892in}{1.734460in}}%
\pgfpathlineto{\pgfqpoint{3.587756in}{1.622974in}}%
\pgfpathlineto{\pgfqpoint{3.590351in}{1.807893in}}%
\pgfpathlineto{\pgfqpoint{3.591214in}{1.710596in}}%
\pgfpathlineto{\pgfqpoint{3.592078in}{1.746035in}}%
\pgfpathlineto{\pgfqpoint{3.592942in}{1.667170in}}%
\pgfpathlineto{\pgfqpoint{3.593804in}{1.689461in}}%
\pgfpathlineto{\pgfqpoint{3.594668in}{1.758384in}}%
\pgfpathlineto{\pgfqpoint{3.595532in}{1.703680in}}%
\pgfpathlineto{\pgfqpoint{3.596397in}{1.734044in}}%
\pgfpathlineto{\pgfqpoint{3.598128in}{1.669873in}}%
\pgfpathlineto{\pgfqpoint{3.598992in}{1.740160in}}%
\pgfpathlineto{\pgfqpoint{3.599857in}{1.707510in}}%
\pgfpathlineto{\pgfqpoint{3.600722in}{1.712734in}}%
\pgfpathlineto{\pgfqpoint{3.601586in}{1.690413in}}%
\pgfpathlineto{\pgfqpoint{3.603316in}{1.759157in}}%
\pgfpathlineto{\pgfqpoint{3.604181in}{1.749363in}}%
\pgfpathlineto{\pgfqpoint{3.605045in}{1.675157in}}%
\pgfpathlineto{\pgfqpoint{3.605910in}{1.713626in}}%
\pgfpathlineto{\pgfqpoint{3.606775in}{1.700092in}}%
\pgfpathlineto{\pgfqpoint{3.607639in}{1.650017in}}%
\pgfpathlineto{\pgfqpoint{3.609370in}{1.735471in}}%
\pgfpathlineto{\pgfqpoint{3.610234in}{1.721818in}}%
\pgfpathlineto{\pgfqpoint{3.613692in}{1.745801in}}%
\pgfpathlineto{\pgfqpoint{3.614557in}{1.747582in}}%
\pgfpathlineto{\pgfqpoint{3.615422in}{1.701575in}}%
\pgfpathlineto{\pgfqpoint{3.616289in}{1.754528in}}%
\pgfpathlineto{\pgfqpoint{3.617155in}{1.690800in}}%
\pgfpathlineto{\pgfqpoint{3.618017in}{1.690948in}}%
\pgfpathlineto{\pgfqpoint{3.618881in}{1.732266in}}%
\pgfpathlineto{\pgfqpoint{3.621474in}{1.670081in}}%
\pgfpathlineto{\pgfqpoint{3.623205in}{1.759098in}}%
\pgfpathlineto{\pgfqpoint{3.624067in}{1.694273in}}%
\pgfpathlineto{\pgfqpoint{3.624932in}{1.731909in}}%
\pgfpathlineto{\pgfqpoint{3.625798in}{1.697240in}}%
\pgfpathlineto{\pgfqpoint{3.626663in}{1.699051in}}%
\pgfpathlineto{\pgfqpoint{3.628394in}{1.772216in}}%
\pgfpathlineto{\pgfqpoint{3.629258in}{1.707331in}}%
\pgfpathlineto{\pgfqpoint{3.630123in}{1.801066in}}%
\pgfpathlineto{\pgfqpoint{3.631852in}{1.723361in}}%
\pgfpathlineto{\pgfqpoint{3.632716in}{1.728823in}}%
\pgfpathlineto{\pgfqpoint{3.633580in}{1.626834in}}%
\pgfpathlineto{\pgfqpoint{3.635312in}{1.803799in}}%
\pgfpathlineto{\pgfqpoint{3.636176in}{1.698991in}}%
\pgfpathlineto{\pgfqpoint{3.637043in}{1.717661in}}%
\pgfpathlineto{\pgfqpoint{3.638773in}{1.752330in}}%
\pgfpathlineto{\pgfqpoint{3.640506in}{1.683170in}}%
\pgfpathlineto{\pgfqpoint{3.641372in}{1.743603in}}%
\pgfpathlineto{\pgfqpoint{3.643098in}{1.719888in}}%
\pgfpathlineto{\pgfqpoint{3.643962in}{1.776075in}}%
\pgfpathlineto{\pgfqpoint{3.645693in}{1.681954in}}%
\pgfpathlineto{\pgfqpoint{3.647418in}{1.784743in}}%
\pgfpathlineto{\pgfqpoint{3.649149in}{1.667263in}}%
\pgfpathlineto{\pgfqpoint{3.650016in}{1.743931in}}%
\pgfpathlineto{\pgfqpoint{3.650882in}{1.728288in}}%
\pgfpathlineto{\pgfqpoint{3.652611in}{1.668360in}}%
\pgfpathlineto{\pgfqpoint{3.654343in}{1.780233in}}%
\pgfpathlineto{\pgfqpoint{3.656940in}{1.689286in}}%
\pgfpathlineto{\pgfqpoint{3.657803in}{1.785159in}}%
\pgfpathlineto{\pgfqpoint{3.658669in}{1.668389in}}%
\pgfpathlineto{\pgfqpoint{3.659533in}{1.697567in}}%
\pgfpathlineto{\pgfqpoint{3.660399in}{1.722175in}}%
\pgfpathlineto{\pgfqpoint{3.661264in}{1.667263in}}%
\pgfpathlineto{\pgfqpoint{3.662127in}{1.752628in}}%
\pgfpathlineto{\pgfqpoint{3.664721in}{1.643071in}}%
\pgfpathlineto{\pgfqpoint{3.665587in}{1.757911in}}%
\pgfpathlineto{\pgfqpoint{3.667313in}{1.675811in}}%
\pgfpathlineto{\pgfqpoint{3.668179in}{1.687803in}}%
\pgfpathlineto{\pgfqpoint{3.669045in}{1.728674in}}%
\pgfpathlineto{\pgfqpoint{3.669910in}{1.672606in}}%
\pgfpathlineto{\pgfqpoint{3.670775in}{1.682816in}}%
\pgfpathlineto{\pgfqpoint{3.672504in}{1.721758in}}%
\pgfpathlineto{\pgfqpoint{3.673368in}{1.648385in}}%
\pgfpathlineto{\pgfqpoint{3.674235in}{1.663463in}}%
\pgfpathlineto{\pgfqpoint{3.675965in}{1.711131in}}%
\pgfpathlineto{\pgfqpoint{3.676831in}{1.714277in}}%
\pgfpathlineto{\pgfqpoint{3.677697in}{1.728347in}}%
\pgfpathlineto{\pgfqpoint{3.678563in}{1.683943in}}%
\pgfpathlineto{\pgfqpoint{3.679428in}{1.744079in}}%
\pgfpathlineto{\pgfqpoint{3.680294in}{1.677711in}}%
\pgfpathlineto{\pgfqpoint{3.681160in}{1.691335in}}%
\pgfpathlineto{\pgfqpoint{3.682026in}{1.649750in}}%
\pgfpathlineto{\pgfqpoint{3.682891in}{1.692729in}}%
\pgfpathlineto{\pgfqpoint{3.683756in}{1.679225in}}%
\pgfpathlineto{\pgfqpoint{3.684621in}{1.711667in}}%
\pgfpathlineto{\pgfqpoint{3.685485in}{1.656041in}}%
\pgfpathlineto{\pgfqpoint{3.686350in}{1.728585in}}%
\pgfpathlineto{\pgfqpoint{3.688079in}{1.654438in}}%
\pgfpathlineto{\pgfqpoint{3.688943in}{1.735234in}}%
\pgfpathlineto{\pgfqpoint{3.689808in}{1.668508in}}%
\pgfpathlineto{\pgfqpoint{3.690672in}{1.669843in}}%
\pgfpathlineto{\pgfqpoint{3.691536in}{1.719501in}}%
\pgfpathlineto{\pgfqpoint{3.692401in}{1.709529in}}%
\pgfpathlineto{\pgfqpoint{3.693266in}{1.620572in}}%
\pgfpathlineto{\pgfqpoint{3.694996in}{1.709469in}}%
\pgfpathlineto{\pgfqpoint{3.695860in}{1.680202in}}%
\pgfpathlineto{\pgfqpoint{3.696724in}{1.803680in}}%
\pgfpathlineto{\pgfqpoint{3.698452in}{1.707157in}}%
\pgfpathlineto{\pgfqpoint{3.699316in}{1.732683in}}%
\pgfpathlineto{\pgfqpoint{3.700181in}{1.727221in}}%
\pgfpathlineto{\pgfqpoint{3.701046in}{1.666727in}}%
\pgfpathlineto{\pgfqpoint{3.701911in}{1.743544in}}%
\pgfpathlineto{\pgfqpoint{3.704508in}{1.682281in}}%
\pgfpathlineto{\pgfqpoint{3.705373in}{1.675514in}}%
\pgfpathlineto{\pgfqpoint{3.706238in}{1.721818in}}%
\pgfpathlineto{\pgfqpoint{3.707103in}{1.670646in}}%
\pgfpathlineto{\pgfqpoint{3.707967in}{1.675930in}}%
\pgfpathlineto{\pgfqpoint{3.708833in}{1.689524in}}%
\pgfpathlineto{\pgfqpoint{3.709698in}{1.644704in}}%
\pgfpathlineto{\pgfqpoint{3.711428in}{1.764143in}}%
\pgfpathlineto{\pgfqpoint{3.712293in}{1.544822in}}%
\pgfpathlineto{\pgfqpoint{3.713157in}{1.676343in}}%
\pgfpathlineto{\pgfqpoint{3.714022in}{1.652955in}}%
\pgfpathlineto{\pgfqpoint{3.714886in}{1.569073in}}%
\pgfpathlineto{\pgfqpoint{3.715748in}{1.632831in}}%
\pgfpathlineto{\pgfqpoint{3.716611in}{1.596827in}}%
\pgfpathlineto{\pgfqpoint{3.718343in}{1.684657in}}%
\pgfpathlineto{\pgfqpoint{3.719209in}{1.671331in}}%
\pgfpathlineto{\pgfqpoint{3.720074in}{1.713567in}}%
\pgfpathlineto{\pgfqpoint{3.722666in}{1.634493in}}%
\pgfpathlineto{\pgfqpoint{3.723531in}{1.692194in}}%
\pgfpathlineto{\pgfqpoint{3.725261in}{1.642923in}}%
\pgfpathlineto{\pgfqpoint{3.726126in}{1.636631in}}%
\pgfpathlineto{\pgfqpoint{3.726990in}{1.675692in}}%
\pgfpathlineto{\pgfqpoint{3.727855in}{1.644823in}}%
\pgfpathlineto{\pgfqpoint{3.729587in}{1.686260in}}%
\pgfpathlineto{\pgfqpoint{3.730452in}{1.619475in}}%
\pgfpathlineto{\pgfqpoint{3.731315in}{1.667025in}}%
\pgfpathlineto{\pgfqpoint{3.732179in}{1.656338in}}%
\pgfpathlineto{\pgfqpoint{3.733910in}{1.598281in}}%
\pgfpathlineto{\pgfqpoint{3.734774in}{1.687505in}}%
\pgfpathlineto{\pgfqpoint{3.736506in}{1.562901in}}%
\pgfpathlineto{\pgfqpoint{3.737372in}{1.653431in}}%
\pgfpathlineto{\pgfqpoint{3.738238in}{1.592997in}}%
\pgfpathlineto{\pgfqpoint{3.739103in}{1.629031in}}%
\pgfpathlineto{\pgfqpoint{3.739967in}{1.592818in}}%
\pgfpathlineto{\pgfqpoint{3.740831in}{1.642120in}}%
\pgfpathlineto{\pgfqpoint{3.741695in}{1.596500in}}%
\pgfpathlineto{\pgfqpoint{3.743425in}{1.663225in}}%
\pgfpathlineto{\pgfqpoint{3.745156in}{1.560793in}}%
\pgfpathlineto{\pgfqpoint{3.746887in}{1.628972in}}%
\pgfpathlineto{\pgfqpoint{3.747753in}{1.626064in}}%
\pgfpathlineto{\pgfqpoint{3.748618in}{1.688572in}}%
\pgfpathlineto{\pgfqpoint{3.750349in}{1.602880in}}%
\pgfpathlineto{\pgfqpoint{3.752078in}{1.638706in}}%
\pgfpathlineto{\pgfqpoint{3.752944in}{1.585575in}}%
\pgfpathlineto{\pgfqpoint{3.753810in}{1.628317in}}%
\pgfpathlineto{\pgfqpoint{3.754676in}{1.613180in}}%
\pgfpathlineto{\pgfqpoint{3.756406in}{1.679727in}}%
\pgfpathlineto{\pgfqpoint{3.757272in}{1.620155in}}%
\pgfpathlineto{\pgfqpoint{3.758138in}{1.666311in}}%
\pgfpathlineto{\pgfqpoint{3.759004in}{1.561949in}}%
\pgfpathlineto{\pgfqpoint{3.759869in}{1.652627in}}%
\pgfpathlineto{\pgfqpoint{3.760734in}{1.642625in}}%
\pgfpathlineto{\pgfqpoint{3.761597in}{1.573822in}}%
\pgfpathlineto{\pgfqpoint{3.762463in}{1.581303in}}%
\pgfpathlineto{\pgfqpoint{3.763326in}{1.634136in}}%
\pgfpathlineto{\pgfqpoint{3.764191in}{1.625320in}}%
\pgfpathlineto{\pgfqpoint{3.765921in}{1.648798in}}%
\pgfpathlineto{\pgfqpoint{3.766786in}{1.575097in}}%
\pgfpathlineto{\pgfqpoint{3.768513in}{1.666311in}}%
\pgfpathlineto{\pgfqpoint{3.769378in}{1.593116in}}%
\pgfpathlineto{\pgfqpoint{3.770241in}{1.678362in}}%
\pgfpathlineto{\pgfqpoint{3.771106in}{1.670940in}}%
\pgfpathlineto{\pgfqpoint{3.772835in}{1.563611in}}%
\pgfpathlineto{\pgfqpoint{3.773700in}{1.652717in}}%
\pgfpathlineto{\pgfqpoint{3.774564in}{1.620423in}}%
\pgfpathlineto{\pgfqpoint{3.776290in}{1.697270in}}%
\pgfpathlineto{\pgfqpoint{3.778019in}{1.584508in}}%
\pgfpathlineto{\pgfqpoint{3.778884in}{1.626953in}}%
\pgfpathlineto{\pgfqpoint{3.779750in}{1.625588in}}%
\pgfpathlineto{\pgfqpoint{3.780615in}{1.637817in}}%
\pgfpathlineto{\pgfqpoint{3.781479in}{1.632177in}}%
\pgfpathlineto{\pgfqpoint{3.782344in}{1.587773in}}%
\pgfpathlineto{\pgfqpoint{3.783208in}{1.599943in}}%
\pgfpathlineto{\pgfqpoint{3.784074in}{1.618107in}}%
\pgfpathlineto{\pgfqpoint{3.784940in}{1.589018in}}%
\pgfpathlineto{\pgfqpoint{3.785802in}{1.590561in}}%
\pgfpathlineto{\pgfqpoint{3.786665in}{1.606234in}}%
\pgfpathlineto{\pgfqpoint{3.787530in}{1.689818in}}%
\pgfpathlineto{\pgfqpoint{3.788396in}{1.602880in}}%
\pgfpathlineto{\pgfqpoint{3.789261in}{1.611101in}}%
\pgfpathlineto{\pgfqpoint{3.790125in}{1.592461in}}%
\pgfpathlineto{\pgfqpoint{3.790989in}{1.597805in}}%
\pgfpathlineto{\pgfqpoint{3.792718in}{1.561295in}}%
\pgfpathlineto{\pgfqpoint{3.793584in}{1.520334in}}%
\pgfpathlineto{\pgfqpoint{3.794449in}{1.629091in}}%
\pgfpathlineto{\pgfqpoint{3.795314in}{1.584449in}}%
\pgfpathlineto{\pgfqpoint{3.796179in}{1.609796in}}%
\pgfpathlineto{\pgfqpoint{3.797911in}{1.585724in}}%
\pgfpathlineto{\pgfqpoint{3.798777in}{1.601486in}}%
\pgfpathlineto{\pgfqpoint{3.800503in}{1.576908in}}%
\pgfpathlineto{\pgfqpoint{3.802234in}{1.697835in}}%
\pgfpathlineto{\pgfqpoint{3.805695in}{1.636334in}}%
\pgfpathlineto{\pgfqpoint{3.806562in}{1.724904in}}%
\pgfpathlineto{\pgfqpoint{3.807428in}{1.714307in}}%
\pgfpathlineto{\pgfqpoint{3.808293in}{1.705848in}}%
\pgfpathlineto{\pgfqpoint{3.809157in}{1.717070in}}%
\pgfpathlineto{\pgfqpoint{3.810020in}{1.674328in}}%
\pgfpathlineto{\pgfqpoint{3.810885in}{1.746217in}}%
\pgfpathlineto{\pgfqpoint{3.812616in}{1.636334in}}%
\pgfpathlineto{\pgfqpoint{3.813481in}{1.705316in}}%
\pgfpathlineto{\pgfqpoint{3.814344in}{1.639866in}}%
\pgfpathlineto{\pgfqpoint{3.815209in}{1.682459in}}%
\pgfpathlineto{\pgfqpoint{3.816076in}{1.670706in}}%
\pgfpathlineto{\pgfqpoint{3.816939in}{1.636780in}}%
\pgfpathlineto{\pgfqpoint{3.818667in}{1.697954in}}%
\pgfpathlineto{\pgfqpoint{3.819532in}{1.685903in}}%
\pgfpathlineto{\pgfqpoint{3.820398in}{1.689286in}}%
\pgfpathlineto{\pgfqpoint{3.821264in}{1.702970in}}%
\pgfpathlineto{\pgfqpoint{3.822129in}{1.695994in}}%
\pgfpathlineto{\pgfqpoint{3.822992in}{1.753695in}}%
\pgfpathlineto{\pgfqpoint{3.823858in}{1.723182in}}%
\pgfpathlineto{\pgfqpoint{3.824723in}{1.726507in}}%
\pgfpathlineto{\pgfqpoint{3.825589in}{1.705729in}}%
\pgfpathlineto{\pgfqpoint{3.826455in}{1.740220in}}%
\pgfpathlineto{\pgfqpoint{3.827320in}{1.614426in}}%
\pgfpathlineto{\pgfqpoint{3.828185in}{1.762184in}}%
\pgfpathlineto{\pgfqpoint{3.829052in}{1.693916in}}%
\pgfpathlineto{\pgfqpoint{3.829917in}{1.702464in}}%
\pgfpathlineto{\pgfqpoint{3.830781in}{1.693767in}}%
\pgfpathlineto{\pgfqpoint{3.831644in}{1.693916in}}%
\pgfpathlineto{\pgfqpoint{3.832509in}{1.733452in}}%
\pgfpathlineto{\pgfqpoint{3.833375in}{1.729980in}}%
\pgfpathlineto{\pgfqpoint{3.834240in}{1.723718in}}%
\pgfpathlineto{\pgfqpoint{3.835105in}{1.754409in}}%
\pgfpathlineto{\pgfqpoint{3.835971in}{1.709202in}}%
\pgfpathlineto{\pgfqpoint{3.836836in}{1.712972in}}%
\pgfpathlineto{\pgfqpoint{3.837698in}{1.720096in}}%
\pgfpathlineto{\pgfqpoint{3.838562in}{1.651531in}}%
\pgfpathlineto{\pgfqpoint{3.840290in}{1.701367in}}%
\pgfpathlineto{\pgfqpoint{3.841155in}{1.710953in}}%
\pgfpathlineto{\pgfqpoint{3.842018in}{1.668508in}}%
\pgfpathlineto{\pgfqpoint{3.843749in}{1.728347in}}%
\pgfpathlineto{\pgfqpoint{3.844612in}{1.716237in}}%
\pgfpathlineto{\pgfqpoint{3.845478in}{1.661354in}}%
\pgfpathlineto{\pgfqpoint{3.847206in}{1.701099in}}%
\pgfpathlineto{\pgfqpoint{3.849796in}{1.629061in}}%
\pgfpathlineto{\pgfqpoint{3.850662in}{1.697478in}}%
\pgfpathlineto{\pgfqpoint{3.851528in}{1.688929in}}%
\pgfpathlineto{\pgfqpoint{3.853258in}{1.727990in}}%
\pgfpathlineto{\pgfqpoint{3.856715in}{1.568121in}}%
\pgfpathlineto{\pgfqpoint{3.857580in}{1.585813in}}%
\pgfpathlineto{\pgfqpoint{3.858443in}{1.580172in}}%
\pgfpathlineto{\pgfqpoint{3.859307in}{1.614634in}}%
\pgfpathlineto{\pgfqpoint{3.860171in}{1.568657in}}%
\pgfpathlineto{\pgfqpoint{3.861901in}{1.628525in}}%
\pgfpathlineto{\pgfqpoint{3.862767in}{1.576432in}}%
\pgfpathlineto{\pgfqpoint{3.863631in}{1.654022in}}%
\pgfpathlineto{\pgfqpoint{3.864495in}{1.577503in}}%
\pgfpathlineto{\pgfqpoint{3.865358in}{1.702524in}}%
\pgfpathlineto{\pgfqpoint{3.866223in}{1.667025in}}%
\pgfpathlineto{\pgfqpoint{3.867087in}{1.663463in}}%
\pgfpathlineto{\pgfqpoint{3.867951in}{1.666965in}}%
\pgfpathlineto{\pgfqpoint{3.868815in}{1.748917in}}%
\pgfpathlineto{\pgfqpoint{3.869677in}{1.677295in}}%
\pgfpathlineto{\pgfqpoint{3.870542in}{1.731612in}}%
\pgfpathlineto{\pgfqpoint{3.872272in}{1.676878in}}%
\pgfpathlineto{\pgfqpoint{3.873137in}{1.706205in}}%
\pgfpathlineto{\pgfqpoint{3.874002in}{1.623896in}}%
\pgfpathlineto{\pgfqpoint{3.874867in}{1.697775in}}%
\pgfpathlineto{\pgfqpoint{3.875730in}{1.686970in}}%
\pgfpathlineto{\pgfqpoint{3.878324in}{1.638736in}}%
\pgfpathlineto{\pgfqpoint{3.879189in}{1.673138in}}%
\pgfpathlineto{\pgfqpoint{3.880052in}{1.664351in}}%
\pgfpathlineto{\pgfqpoint{3.880914in}{1.568865in}}%
\pgfpathlineto{\pgfqpoint{3.881778in}{1.630039in}}%
\pgfpathlineto{\pgfqpoint{3.882643in}{1.618166in}}%
\pgfpathlineto{\pgfqpoint{3.883507in}{1.618166in}}%
\pgfpathlineto{\pgfqpoint{3.884372in}{1.630753in}}%
\pgfpathlineto{\pgfqpoint{3.886103in}{1.570259in}}%
\pgfpathlineto{\pgfqpoint{3.887836in}{1.664143in}}%
\pgfpathlineto{\pgfqpoint{3.890434in}{1.572308in}}%
\pgfpathlineto{\pgfqpoint{3.891300in}{1.610332in}}%
\pgfpathlineto{\pgfqpoint{3.892166in}{1.584211in}}%
\pgfpathlineto{\pgfqpoint{3.893896in}{1.652479in}}%
\pgfpathlineto{\pgfqpoint{3.895627in}{1.598043in}}%
\pgfpathlineto{\pgfqpoint{3.896493in}{1.632653in}}%
\pgfpathlineto{\pgfqpoint{3.897360in}{1.584151in}}%
\pgfpathlineto{\pgfqpoint{3.898225in}{1.637520in}}%
\pgfpathlineto{\pgfqpoint{3.899958in}{1.600802in}}%
\pgfpathlineto{\pgfqpoint{3.900822in}{1.644109in}}%
\pgfpathlineto{\pgfqpoint{3.901689in}{1.599586in}}%
\pgfpathlineto{\pgfqpoint{3.904283in}{1.673316in}}%
\pgfpathlineto{\pgfqpoint{3.905147in}{1.590030in}}%
\pgfpathlineto{\pgfqpoint{3.906011in}{1.661146in}}%
\pgfpathlineto{\pgfqpoint{3.906875in}{1.562008in}}%
\pgfpathlineto{\pgfqpoint{3.907739in}{1.652241in}}%
\pgfpathlineto{\pgfqpoint{3.908603in}{1.632827in}}%
\pgfpathlineto{\pgfqpoint{3.912926in}{1.583556in}}%
\pgfpathlineto{\pgfqpoint{3.913792in}{1.588542in}}%
\pgfpathlineto{\pgfqpoint{3.914657in}{1.674562in}}%
\pgfpathlineto{\pgfqpoint{3.917253in}{1.597329in}}%
\pgfpathlineto{\pgfqpoint{3.918118in}{1.648976in}}%
\pgfpathlineto{\pgfqpoint{3.918984in}{1.580708in}}%
\pgfpathlineto{\pgfqpoint{3.919849in}{1.666549in}}%
\pgfpathlineto{\pgfqpoint{3.920712in}{1.621907in}}%
\pgfpathlineto{\pgfqpoint{3.921578in}{1.638944in}}%
\pgfpathlineto{\pgfqpoint{3.924173in}{1.583854in}}%
\pgfpathlineto{\pgfqpoint{3.925036in}{1.568597in}}%
\pgfpathlineto{\pgfqpoint{3.928493in}{1.631552in}}%
\pgfpathlineto{\pgfqpoint{3.929358in}{1.643514in}}%
\pgfpathlineto{\pgfqpoint{3.930224in}{1.585397in}}%
\pgfpathlineto{\pgfqpoint{3.932817in}{1.652122in}}%
\pgfpathlineto{\pgfqpoint{3.933682in}{1.595191in}}%
\pgfpathlineto{\pgfqpoint{3.934547in}{1.662273in}}%
\pgfpathlineto{\pgfqpoint{3.935411in}{1.655773in}}%
\pgfpathlineto{\pgfqpoint{3.936274in}{1.691540in}}%
\pgfpathlineto{\pgfqpoint{3.937139in}{1.688334in}}%
\pgfpathlineto{\pgfqpoint{3.939731in}{1.593410in}}%
\pgfpathlineto{\pgfqpoint{3.943191in}{1.668624in}}%
\pgfpathlineto{\pgfqpoint{3.944922in}{1.639476in}}%
\pgfpathlineto{\pgfqpoint{3.945788in}{1.578094in}}%
\pgfpathlineto{\pgfqpoint{3.946653in}{1.631403in}}%
\pgfpathlineto{\pgfqpoint{3.947519in}{1.526714in}}%
\pgfpathlineto{\pgfqpoint{3.949249in}{1.626120in}}%
\pgfpathlineto{\pgfqpoint{3.950114in}{1.595131in}}%
\pgfpathlineto{\pgfqpoint{3.950979in}{1.627901in}}%
\pgfpathlineto{\pgfqpoint{3.951843in}{1.591688in}}%
\pgfpathlineto{\pgfqpoint{3.953573in}{1.663935in}}%
\pgfpathlineto{\pgfqpoint{3.954436in}{1.636062in}}%
\pgfpathlineto{\pgfqpoint{3.957031in}{1.702907in}}%
\pgfpathlineto{\pgfqpoint{3.959623in}{1.639982in}}%
\pgfpathlineto{\pgfqpoint{3.960486in}{1.647136in}}%
\pgfpathlineto{\pgfqpoint{3.961351in}{1.680794in}}%
\pgfpathlineto{\pgfqpoint{3.962216in}{1.670702in}}%
\pgfpathlineto{\pgfqpoint{3.963081in}{1.698426in}}%
\pgfpathlineto{\pgfqpoint{3.963945in}{1.655297in}}%
\pgfpathlineto{\pgfqpoint{3.964811in}{1.701691in}}%
\pgfpathlineto{\pgfqpoint{3.965676in}{1.672721in}}%
\pgfpathlineto{\pgfqpoint{3.966542in}{1.719144in}}%
\pgfpathlineto{\pgfqpoint{3.967407in}{1.653784in}}%
\pgfpathlineto{\pgfqpoint{3.968272in}{1.730779in}}%
\pgfpathlineto{\pgfqpoint{3.969137in}{1.634757in}}%
\pgfpathlineto{\pgfqpoint{3.970867in}{1.708636in}}%
\pgfpathlineto{\pgfqpoint{3.971732in}{1.615314in}}%
\pgfpathlineto{\pgfqpoint{3.973460in}{1.716411in}}%
\pgfpathlineto{\pgfqpoint{3.975191in}{1.653427in}}%
\pgfpathlineto{\pgfqpoint{3.976055in}{1.680407in}}%
\pgfpathlineto{\pgfqpoint{3.978650in}{1.581478in}}%
\pgfpathlineto{\pgfqpoint{3.979516in}{1.591748in}}%
\pgfpathlineto{\pgfqpoint{3.980382in}{1.624339in}}%
\pgfpathlineto{\pgfqpoint{3.981249in}{1.586226in}}%
\pgfpathlineto{\pgfqpoint{3.982115in}{1.628730in}}%
\pgfpathlineto{\pgfqpoint{3.982981in}{1.557792in}}%
\pgfpathlineto{\pgfqpoint{3.983846in}{1.558030in}}%
\pgfpathlineto{\pgfqpoint{3.984712in}{1.585218in}}%
\pgfpathlineto{\pgfqpoint{3.985575in}{1.550906in}}%
\pgfpathlineto{\pgfqpoint{3.986438in}{1.552330in}}%
\pgfpathlineto{\pgfqpoint{3.987304in}{1.539568in}}%
\pgfpathlineto{\pgfqpoint{3.988169in}{1.578302in}}%
\pgfpathlineto{\pgfqpoint{3.989902in}{1.565273in}}%
\pgfpathlineto{\pgfqpoint{3.990765in}{1.625112in}}%
\pgfpathlineto{\pgfqpoint{3.991629in}{1.596321in}}%
\pgfpathlineto{\pgfqpoint{3.992494in}{1.614158in}}%
\pgfpathlineto{\pgfqpoint{3.993360in}{1.528466in}}%
\pgfpathlineto{\pgfqpoint{3.994223in}{1.631522in}}%
\pgfpathlineto{\pgfqpoint{3.995954in}{1.575424in}}%
\pgfpathlineto{\pgfqpoint{3.996820in}{1.676700in}}%
\pgfpathlineto{\pgfqpoint{3.997684in}{1.590621in}}%
\pgfpathlineto{\pgfqpoint{3.998549in}{1.643395in}}%
\pgfpathlineto{\pgfqpoint{4.000279in}{1.584032in}}%
\pgfpathlineto{\pgfqpoint{4.001143in}{1.611280in}}%
\pgfpathlineto{\pgfqpoint{4.002007in}{1.593707in}}%
\pgfpathlineto{\pgfqpoint{4.002874in}{1.614307in}}%
\pgfpathlineto{\pgfqpoint{4.003737in}{1.607896in}}%
\pgfpathlineto{\pgfqpoint{4.004603in}{1.565303in}}%
\pgfpathlineto{\pgfqpoint{4.006327in}{1.606651in}}%
\pgfpathlineto{\pgfqpoint{4.008057in}{1.564500in}}%
\pgfpathlineto{\pgfqpoint{4.008923in}{1.628079in}}%
\pgfpathlineto{\pgfqpoint{4.009788in}{1.527220in}}%
\pgfpathlineto{\pgfqpoint{4.010652in}{1.626536in}}%
\pgfpathlineto{\pgfqpoint{4.011517in}{1.572487in}}%
\pgfpathlineto{\pgfqpoint{4.013246in}{1.615675in}}%
\pgfpathlineto{\pgfqpoint{4.014108in}{1.552066in}}%
\pgfpathlineto{\pgfqpoint{4.014974in}{1.660908in}}%
\pgfpathlineto{\pgfqpoint{4.016704in}{1.578421in}}%
\pgfpathlineto{\pgfqpoint{4.017569in}{1.670940in}}%
\pgfpathlineto{\pgfqpoint{4.018434in}{1.661444in}}%
\pgfpathlineto{\pgfqpoint{4.019301in}{1.558803in}}%
\pgfpathlineto{\pgfqpoint{4.020166in}{1.628615in}}%
\pgfpathlineto{\pgfqpoint{4.021032in}{1.575365in}}%
\pgfpathlineto{\pgfqpoint{4.022764in}{1.614664in}}%
\pgfpathlineto{\pgfqpoint{4.023629in}{1.597180in}}%
\pgfpathlineto{\pgfqpoint{4.024494in}{1.672364in}}%
\pgfpathlineto{\pgfqpoint{4.025359in}{1.588899in}}%
\pgfpathlineto{\pgfqpoint{4.026224in}{1.592670in}}%
\pgfpathlineto{\pgfqpoint{4.027087in}{1.579934in}}%
\pgfpathlineto{\pgfqpoint{4.027953in}{1.660492in}}%
\pgfpathlineto{\pgfqpoint{4.028818in}{1.578332in}}%
\pgfpathlineto{\pgfqpoint{4.030549in}{1.632500in}}%
\pgfpathlineto{\pgfqpoint{4.032277in}{1.595369in}}%
\pgfpathlineto{\pgfqpoint{4.033143in}{1.613741in}}%
\pgfpathlineto{\pgfqpoint{4.034872in}{1.584802in}}%
\pgfpathlineto{\pgfqpoint{4.035739in}{1.557465in}}%
\pgfpathlineto{\pgfqpoint{4.037469in}{1.631582in}}%
\pgfpathlineto{\pgfqpoint{4.038334in}{1.615909in}}%
\pgfpathlineto{\pgfqpoint{4.039199in}{1.679667in}}%
\pgfpathlineto{\pgfqpoint{4.040065in}{1.620185in}}%
\pgfpathlineto{\pgfqpoint{4.040931in}{1.624933in}}%
\pgfpathlineto{\pgfqpoint{4.041797in}{1.674324in}}%
\pgfpathlineto{\pgfqpoint{4.043526in}{1.603680in}}%
\pgfpathlineto{\pgfqpoint{4.044392in}{1.611280in}}%
\pgfpathlineto{\pgfqpoint{4.045257in}{1.635114in}}%
\pgfpathlineto{\pgfqpoint{4.046121in}{1.627250in}}%
\pgfpathlineto{\pgfqpoint{4.046986in}{1.661384in}}%
\pgfpathlineto{\pgfqpoint{4.048715in}{1.632177in}}%
\pgfpathlineto{\pgfqpoint{4.049579in}{1.652271in}}%
\pgfpathlineto{\pgfqpoint{4.050444in}{1.596024in}}%
\pgfpathlineto{\pgfqpoint{4.051310in}{1.647195in}}%
\pgfpathlineto{\pgfqpoint{4.052176in}{1.633660in}}%
\pgfpathlineto{\pgfqpoint{4.053906in}{1.577384in}}%
\pgfpathlineto{\pgfqpoint{4.054771in}{1.613715in}}%
\pgfpathlineto{\pgfqpoint{4.055636in}{1.569192in}}%
\pgfpathlineto{\pgfqpoint{4.057366in}{1.702880in}}%
\pgfpathlineto{\pgfqpoint{4.058228in}{1.656695in}}%
\pgfpathlineto{\pgfqpoint{4.059092in}{1.726269in}}%
\pgfpathlineto{\pgfqpoint{4.060822in}{1.601664in}}%
\pgfpathlineto{\pgfqpoint{4.062552in}{1.581541in}}%
\pgfpathlineto{\pgfqpoint{4.063417in}{1.632712in}}%
\pgfpathlineto{\pgfqpoint{4.064282in}{1.573022in}}%
\pgfpathlineto{\pgfqpoint{4.066012in}{1.665363in}}%
\pgfpathlineto{\pgfqpoint{4.066875in}{1.599824in}}%
\pgfpathlineto{\pgfqpoint{4.067741in}{1.611994in}}%
\pgfpathlineto{\pgfqpoint{4.068604in}{1.620334in}}%
\pgfpathlineto{\pgfqpoint{4.069469in}{1.606829in}}%
\pgfpathlineto{\pgfqpoint{4.070334in}{1.607896in}}%
\pgfpathlineto{\pgfqpoint{4.071198in}{1.639242in}}%
\pgfpathlineto{\pgfqpoint{4.072063in}{1.597983in}}%
\pgfpathlineto{\pgfqpoint{4.072928in}{1.606591in}}%
\pgfpathlineto{\pgfqpoint{4.073792in}{1.572219in}}%
\pgfpathlineto{\pgfqpoint{4.074657in}{1.664589in}}%
\pgfpathlineto{\pgfqpoint{4.075522in}{1.587297in}}%
\pgfpathlineto{\pgfqpoint{4.076388in}{1.604869in}}%
\pgfpathlineto{\pgfqpoint{4.078118in}{1.664500in}}%
\pgfpathlineto{\pgfqpoint{4.078984in}{1.600831in}}%
\pgfpathlineto{\pgfqpoint{4.079849in}{1.670289in}}%
\pgfpathlineto{\pgfqpoint{4.080713in}{1.623807in}}%
\pgfpathlineto{\pgfqpoint{4.082444in}{1.759336in}}%
\pgfpathlineto{\pgfqpoint{4.083309in}{1.695994in}}%
\pgfpathlineto{\pgfqpoint{4.084175in}{1.709707in}}%
\pgfpathlineto{\pgfqpoint{4.085039in}{1.681984in}}%
\pgfpathlineto{\pgfqpoint{4.085903in}{1.696113in}}%
\pgfpathlineto{\pgfqpoint{4.086768in}{1.759038in}}%
\pgfpathlineto{\pgfqpoint{4.088498in}{1.660495in}}%
\pgfpathlineto{\pgfqpoint{4.090229in}{1.741733in}}%
\pgfpathlineto{\pgfqpoint{4.091095in}{1.737312in}}%
\pgfpathlineto{\pgfqpoint{4.091961in}{1.720632in}}%
\pgfpathlineto{\pgfqpoint{4.092826in}{1.757465in}}%
\pgfpathlineto{\pgfqpoint{4.094556in}{1.713686in}}%
\pgfpathlineto{\pgfqpoint{4.095420in}{1.734285in}}%
\pgfpathlineto{\pgfqpoint{4.096286in}{1.705256in}}%
\pgfpathlineto{\pgfqpoint{4.097151in}{1.613656in}}%
\pgfpathlineto{\pgfqpoint{4.098881in}{1.656398in}}%
\pgfpathlineto{\pgfqpoint{4.099746in}{1.620661in}}%
\pgfpathlineto{\pgfqpoint{4.101476in}{1.683408in}}%
\pgfpathlineto{\pgfqpoint{4.103206in}{1.607361in}}%
\pgfpathlineto{\pgfqpoint{4.104071in}{1.666668in}}%
\pgfpathlineto{\pgfqpoint{4.104936in}{1.666311in}}%
\pgfpathlineto{\pgfqpoint{4.106665in}{1.535947in}}%
\pgfpathlineto{\pgfqpoint{4.107530in}{1.642090in}}%
\pgfpathlineto{\pgfqpoint{4.109259in}{1.595875in}}%
\pgfpathlineto{\pgfqpoint{4.110125in}{1.585278in}}%
\pgfpathlineto{\pgfqpoint{4.110990in}{1.611458in}}%
\pgfpathlineto{\pgfqpoint{4.112719in}{1.588840in}}%
\pgfpathlineto{\pgfqpoint{4.113584in}{1.619709in}}%
\pgfpathlineto{\pgfqpoint{4.114449in}{1.605937in}}%
\pgfpathlineto{\pgfqpoint{4.115313in}{1.544614in}}%
\pgfpathlineto{\pgfqpoint{4.116177in}{1.621193in}}%
\pgfpathlineto{\pgfqpoint{4.117040in}{1.586999in}}%
\pgfpathlineto{\pgfqpoint{4.117905in}{1.599348in}}%
\pgfpathlineto{\pgfqpoint{4.118770in}{1.579905in}}%
\pgfpathlineto{\pgfqpoint{4.119636in}{1.628020in}}%
\pgfpathlineto{\pgfqpoint{4.120501in}{1.620657in}}%
\pgfpathlineto{\pgfqpoint{4.121367in}{1.608815in}}%
\pgfpathlineto{\pgfqpoint{4.122231in}{1.610982in}}%
\pgfpathlineto{\pgfqpoint{4.123096in}{1.643692in}}%
\pgfpathlineto{\pgfqpoint{4.124826in}{1.562600in}}%
\pgfpathlineto{\pgfqpoint{4.125692in}{1.591450in}}%
\pgfpathlineto{\pgfqpoint{4.126557in}{1.741287in}}%
\pgfpathlineto{\pgfqpoint{4.128288in}{1.664054in}}%
\pgfpathlineto{\pgfqpoint{4.129154in}{1.671119in}}%
\pgfpathlineto{\pgfqpoint{4.130019in}{1.719501in}}%
\pgfpathlineto{\pgfqpoint{4.130884in}{1.657911in}}%
\pgfpathlineto{\pgfqpoint{4.131746in}{1.665954in}}%
\pgfpathlineto{\pgfqpoint{4.132610in}{1.692135in}}%
\pgfpathlineto{\pgfqpoint{4.133474in}{1.659603in}}%
\pgfpathlineto{\pgfqpoint{4.134339in}{1.692789in}}%
\pgfpathlineto{\pgfqpoint{4.135203in}{1.673614in}}%
\pgfpathlineto{\pgfqpoint{4.136933in}{1.790086in}}%
\pgfpathlineto{\pgfqpoint{4.138664in}{1.686970in}}%
\pgfpathlineto{\pgfqpoint{4.139529in}{1.744258in}}%
\pgfpathlineto{\pgfqpoint{4.140395in}{1.696143in}}%
\pgfpathlineto{\pgfqpoint{4.142124in}{1.727577in}}%
\pgfpathlineto{\pgfqpoint{4.142987in}{1.679135in}}%
\pgfpathlineto{\pgfqpoint{4.143853in}{1.731021in}}%
\pgfpathlineto{\pgfqpoint{4.145583in}{1.695935in}}%
\pgfpathlineto{\pgfqpoint{4.147312in}{1.721550in}}%
\pgfpathlineto{\pgfqpoint{4.149043in}{1.658298in}}%
\pgfpathlineto{\pgfqpoint{4.149908in}{1.651293in}}%
\pgfpathlineto{\pgfqpoint{4.152507in}{1.693916in}}%
\pgfpathlineto{\pgfqpoint{4.153374in}{1.652895in}}%
\pgfpathlineto{\pgfqpoint{4.154234in}{1.709678in}}%
\pgfpathlineto{\pgfqpoint{4.155100in}{1.653074in}}%
\pgfpathlineto{\pgfqpoint{4.157692in}{1.704721in}}%
\pgfpathlineto{\pgfqpoint{4.159420in}{1.669427in}}%
\pgfpathlineto{\pgfqpoint{4.160285in}{1.743128in}}%
\pgfpathlineto{\pgfqpoint{4.161148in}{1.693737in}}%
\pgfpathlineto{\pgfqpoint{4.162013in}{1.726090in}}%
\pgfpathlineto{\pgfqpoint{4.163741in}{1.650727in}}%
\pgfpathlineto{\pgfqpoint{4.164607in}{1.696704in}}%
\pgfpathlineto{\pgfqpoint{4.167203in}{1.652419in}}%
\pgfpathlineto{\pgfqpoint{4.168932in}{1.684534in}}%
\pgfpathlineto{\pgfqpoint{4.169798in}{1.670881in}}%
\pgfpathlineto{\pgfqpoint{4.170663in}{1.636330in}}%
\pgfpathlineto{\pgfqpoint{4.171528in}{1.650400in}}%
\pgfpathlineto{\pgfqpoint{4.172393in}{1.612406in}}%
\pgfpathlineto{\pgfqpoint{4.173258in}{1.631165in}}%
\pgfpathlineto{\pgfqpoint{4.174125in}{1.618464in}}%
\pgfpathlineto{\pgfqpoint{4.174991in}{1.695280in}}%
\pgfpathlineto{\pgfqpoint{4.175858in}{1.616474in}}%
\pgfpathlineto{\pgfqpoint{4.176724in}{1.688275in}}%
\pgfpathlineto{\pgfqpoint{4.178452in}{1.621758in}}%
\pgfpathlineto{\pgfqpoint{4.180184in}{1.681508in}}%
\pgfpathlineto{\pgfqpoint{4.181050in}{1.630217in}}%
\pgfpathlineto{\pgfqpoint{4.181915in}{1.702286in}}%
\pgfpathlineto{\pgfqpoint{4.182780in}{1.664292in}}%
\pgfpathlineto{\pgfqpoint{4.183646in}{1.714039in}}%
\pgfpathlineto{\pgfqpoint{4.184511in}{1.700326in}}%
\pgfpathlineto{\pgfqpoint{4.186240in}{1.688989in}}%
\pgfpathlineto{\pgfqpoint{4.187102in}{1.654319in}}%
\pgfpathlineto{\pgfqpoint{4.187968in}{1.662957in}}%
\pgfpathlineto{\pgfqpoint{4.189700in}{1.704126in}}%
\pgfpathlineto{\pgfqpoint{4.190565in}{1.652598in}}%
\pgfpathlineto{\pgfqpoint{4.191432in}{1.693856in}}%
\pgfpathlineto{\pgfqpoint{4.193163in}{1.643454in}}%
\pgfpathlineto{\pgfqpoint{4.194030in}{1.659187in}}%
\pgfpathlineto{\pgfqpoint{4.194895in}{1.645087in}}%
\pgfpathlineto{\pgfqpoint{4.196623in}{1.662511in}}%
\pgfpathlineto{\pgfqpoint{4.197486in}{1.650460in}}%
\pgfpathlineto{\pgfqpoint{4.199218in}{1.742235in}}%
\pgfpathlineto{\pgfqpoint{4.200083in}{1.643395in}}%
\pgfpathlineto{\pgfqpoint{4.201815in}{1.717601in}}%
\pgfpathlineto{\pgfqpoint{4.202680in}{1.724607in}}%
\pgfpathlineto{\pgfqpoint{4.203545in}{1.705015in}}%
\pgfpathlineto{\pgfqpoint{4.204411in}{1.719055in}}%
\pgfpathlineto{\pgfqpoint{4.205277in}{1.751795in}}%
\pgfpathlineto{\pgfqpoint{4.207008in}{1.655624in}}%
\pgfpathlineto{\pgfqpoint{4.207872in}{1.702048in}}%
\pgfpathlineto{\pgfqpoint{4.208736in}{1.693916in}}%
\pgfpathlineto{\pgfqpoint{4.209602in}{1.657941in}}%
\pgfpathlineto{\pgfqpoint{4.210467in}{1.738022in}}%
\pgfpathlineto{\pgfqpoint{4.212198in}{1.631582in}}%
\pgfpathlineto{\pgfqpoint{4.213060in}{1.697002in}}%
\pgfpathlineto{\pgfqpoint{4.213925in}{1.696913in}}%
\pgfpathlineto{\pgfqpoint{4.214789in}{1.697359in}}%
\pgfpathlineto{\pgfqpoint{4.216520in}{1.714396in}}%
\pgfpathlineto{\pgfqpoint{4.218250in}{1.668151in}}%
\pgfpathlineto{\pgfqpoint{4.219112in}{1.690770in}}%
\pgfpathlineto{\pgfqpoint{4.219976in}{1.675335in}}%
\pgfpathlineto{\pgfqpoint{4.220842in}{1.759157in}}%
\pgfpathlineto{\pgfqpoint{4.221705in}{1.711429in}}%
\pgfpathlineto{\pgfqpoint{4.222571in}{1.740339in}}%
\pgfpathlineto{\pgfqpoint{4.223436in}{1.702702in}}%
\pgfpathlineto{\pgfqpoint{4.224302in}{1.735293in}}%
\pgfpathlineto{\pgfqpoint{4.225166in}{1.662987in}}%
\pgfpathlineto{\pgfqpoint{4.226894in}{1.701869in}}%
\pgfpathlineto{\pgfqpoint{4.227760in}{1.684003in}}%
\pgfpathlineto{\pgfqpoint{4.228626in}{1.704543in}}%
\pgfpathlineto{\pgfqpoint{4.229490in}{1.683705in}}%
\pgfpathlineto{\pgfqpoint{4.230356in}{1.702315in}}%
\pgfpathlineto{\pgfqpoint{4.231219in}{1.644287in}}%
\pgfpathlineto{\pgfqpoint{4.232949in}{1.718137in}}%
\pgfpathlineto{\pgfqpoint{4.233814in}{1.701159in}}%
\pgfpathlineto{\pgfqpoint{4.234678in}{1.734996in}}%
\pgfpathlineto{\pgfqpoint{4.235543in}{1.681865in}}%
\pgfpathlineto{\pgfqpoint{4.236409in}{1.709053in}}%
\pgfpathlineto{\pgfqpoint{4.237273in}{1.702345in}}%
\pgfpathlineto{\pgfqpoint{4.238138in}{1.690056in}}%
\pgfpathlineto{\pgfqpoint{4.239867in}{1.729117in}}%
\pgfpathlineto{\pgfqpoint{4.240732in}{1.700088in}}%
\pgfpathlineto{\pgfqpoint{4.241594in}{1.714396in}}%
\pgfpathlineto{\pgfqpoint{4.242461in}{1.702167in}}%
\pgfpathlineto{\pgfqpoint{4.243325in}{1.717601in}}%
\pgfpathlineto{\pgfqpoint{4.244190in}{1.714515in}}%
\pgfpathlineto{\pgfqpoint{4.245054in}{1.715909in}}%
\pgfpathlineto{\pgfqpoint{4.245919in}{1.752509in}}%
\pgfpathlineto{\pgfqpoint{4.246784in}{1.707748in}}%
\pgfpathlineto{\pgfqpoint{4.248514in}{1.738439in}}%
\pgfpathlineto{\pgfqpoint{4.249380in}{1.750698in}}%
\pgfpathlineto{\pgfqpoint{4.250245in}{1.713210in}}%
\pgfpathlineto{\pgfqpoint{4.251107in}{1.741882in}}%
\pgfpathlineto{\pgfqpoint{4.252837in}{1.699199in}}%
\pgfpathlineto{\pgfqpoint{4.253702in}{1.633422in}}%
\pgfpathlineto{\pgfqpoint{4.254566in}{1.640279in}}%
\pgfpathlineto{\pgfqpoint{4.255431in}{1.651174in}}%
\pgfpathlineto{\pgfqpoint{4.256297in}{1.639242in}}%
\pgfpathlineto{\pgfqpoint{4.257163in}{1.642090in}}%
\pgfpathlineto{\pgfqpoint{4.258027in}{1.704662in}}%
\pgfpathlineto{\pgfqpoint{4.258891in}{1.625528in}}%
\pgfpathlineto{\pgfqpoint{4.259758in}{1.636274in}}%
\pgfpathlineto{\pgfqpoint{4.260621in}{1.672071in}}%
\pgfpathlineto{\pgfqpoint{4.261484in}{1.654587in}}%
\pgfpathlineto{\pgfqpoint{4.263215in}{1.733809in}}%
\pgfpathlineto{\pgfqpoint{4.264079in}{1.666965in}}%
\pgfpathlineto{\pgfqpoint{4.264945in}{1.756547in}}%
\pgfpathlineto{\pgfqpoint{4.265810in}{1.737253in}}%
\pgfpathlineto{\pgfqpoint{4.266675in}{1.667973in}}%
\pgfpathlineto{\pgfqpoint{4.268406in}{1.723301in}}%
\pgfpathlineto{\pgfqpoint{4.270136in}{1.676997in}}%
\pgfpathlineto{\pgfqpoint{4.271865in}{1.714872in}}%
\pgfpathlineto{\pgfqpoint{4.273593in}{1.646277in}}%
\pgfpathlineto{\pgfqpoint{4.275324in}{1.746336in}}%
\pgfpathlineto{\pgfqpoint{4.276188in}{1.727518in}}%
\pgfpathlineto{\pgfqpoint{4.277054in}{1.667382in}}%
\pgfpathlineto{\pgfqpoint{4.277919in}{1.737520in}}%
\pgfpathlineto{\pgfqpoint{4.278785in}{1.654383in}}%
\pgfpathlineto{\pgfqpoint{4.279650in}{1.744674in}}%
\pgfpathlineto{\pgfqpoint{4.281379in}{1.639955in}}%
\pgfpathlineto{\pgfqpoint{4.282243in}{1.641082in}}%
\pgfpathlineto{\pgfqpoint{4.285704in}{1.527339in}}%
\pgfpathlineto{\pgfqpoint{4.289164in}{1.687386in}}%
\pgfpathlineto{\pgfqpoint{4.290894in}{1.535709in}}%
\pgfpathlineto{\pgfqpoint{4.291759in}{1.626953in}}%
\pgfpathlineto{\pgfqpoint{4.292625in}{1.623569in}}%
\pgfpathlineto{\pgfqpoint{4.293491in}{1.587535in}}%
\pgfpathlineto{\pgfqpoint{4.295221in}{1.611399in}}%
\pgfpathlineto{\pgfqpoint{4.296085in}{1.688810in}}%
\pgfpathlineto{\pgfqpoint{4.297812in}{1.619118in}}%
\pgfpathlineto{\pgfqpoint{4.298678in}{1.632593in}}%
\pgfpathlineto{\pgfqpoint{4.299544in}{1.615020in}}%
\pgfpathlineto{\pgfqpoint{4.301276in}{1.684713in}}%
\pgfpathlineto{\pgfqpoint{4.302140in}{1.599824in}}%
\pgfpathlineto{\pgfqpoint{4.303005in}{1.654319in}}%
\pgfpathlineto{\pgfqpoint{4.303869in}{1.611518in}}%
\pgfpathlineto{\pgfqpoint{4.304734in}{1.657762in}}%
\pgfpathlineto{\pgfqpoint{4.306465in}{1.614307in}}%
\pgfpathlineto{\pgfqpoint{4.307331in}{1.661797in}}%
\pgfpathlineto{\pgfqpoint{4.308196in}{1.655565in}}%
\pgfpathlineto{\pgfqpoint{4.311656in}{1.603326in}}%
\pgfpathlineto{\pgfqpoint{4.313387in}{1.646898in}}%
\pgfpathlineto{\pgfqpoint{4.314253in}{1.600177in}}%
\pgfpathlineto{\pgfqpoint{4.316847in}{1.676581in}}%
\pgfpathlineto{\pgfqpoint{4.317712in}{1.664887in}}%
\pgfpathlineto{\pgfqpoint{4.318576in}{1.607866in}}%
\pgfpathlineto{\pgfqpoint{4.319442in}{1.631701in}}%
\pgfpathlineto{\pgfqpoint{4.320307in}{1.580470in}}%
\pgfpathlineto{\pgfqpoint{4.321170in}{1.589346in}}%
\pgfpathlineto{\pgfqpoint{4.323763in}{1.683289in}}%
\pgfpathlineto{\pgfqpoint{4.324629in}{1.679667in}}%
\pgfpathlineto{\pgfqpoint{4.328083in}{1.604988in}}%
\pgfpathlineto{\pgfqpoint{4.329813in}{1.619947in}}%
\pgfpathlineto{\pgfqpoint{4.330678in}{1.602702in}}%
\pgfpathlineto{\pgfqpoint{4.331544in}{1.611458in}}%
\pgfpathlineto{\pgfqpoint{4.332409in}{1.654557in}}%
\pgfpathlineto{\pgfqpoint{4.333276in}{1.583735in}}%
\pgfpathlineto{\pgfqpoint{4.334141in}{1.618285in}}%
\pgfpathlineto{\pgfqpoint{4.335005in}{1.574119in}}%
\pgfpathlineto{\pgfqpoint{4.336733in}{1.645533in}}%
\pgfpathlineto{\pgfqpoint{4.338461in}{1.594540in}}%
\pgfpathlineto{\pgfqpoint{4.339326in}{1.645652in}}%
\pgfpathlineto{\pgfqpoint{4.340193in}{1.591037in}}%
\pgfpathlineto{\pgfqpoint{4.341058in}{1.646128in}}%
\pgfpathlineto{\pgfqpoint{4.341924in}{1.582132in}}%
\pgfpathlineto{\pgfqpoint{4.342786in}{1.616385in}}%
\pgfpathlineto{\pgfqpoint{4.343648in}{1.575008in}}%
\pgfpathlineto{\pgfqpoint{4.345375in}{1.652717in}}%
\pgfpathlineto{\pgfqpoint{4.347106in}{1.587713in}}%
\pgfpathlineto{\pgfqpoint{4.347971in}{1.623450in}}%
\pgfpathlineto{\pgfqpoint{4.348837in}{1.739625in}}%
\pgfpathlineto{\pgfqpoint{4.349701in}{1.621907in}}%
\pgfpathlineto{\pgfqpoint{4.350566in}{1.623628in}}%
\pgfpathlineto{\pgfqpoint{4.351432in}{1.614901in}}%
\pgfpathlineto{\pgfqpoint{4.352298in}{1.621609in}}%
\pgfpathlineto{\pgfqpoint{4.353162in}{1.585635in}}%
\pgfpathlineto{\pgfqpoint{4.354027in}{1.623539in}}%
\pgfpathlineto{\pgfqpoint{4.354892in}{1.597507in}}%
\pgfpathlineto{\pgfqpoint{4.355757in}{1.684118in}}%
\pgfpathlineto{\pgfqpoint{4.356622in}{1.605788in}}%
\pgfpathlineto{\pgfqpoint{4.357487in}{1.613890in}}%
\pgfpathlineto{\pgfqpoint{4.359212in}{1.686137in}}%
\pgfpathlineto{\pgfqpoint{4.360077in}{1.668267in}}%
\pgfpathlineto{\pgfqpoint{4.360942in}{1.686375in}}%
\pgfpathlineto{\pgfqpoint{4.361807in}{1.604096in}}%
\pgfpathlineto{\pgfqpoint{4.362672in}{1.653665in}}%
\pgfpathlineto{\pgfqpoint{4.363537in}{1.593410in}}%
\pgfpathlineto{\pgfqpoint{4.365264in}{1.638290in}}%
\pgfpathlineto{\pgfqpoint{4.366126in}{1.585159in}}%
\pgfpathlineto{\pgfqpoint{4.366989in}{1.593529in}}%
\pgfpathlineto{\pgfqpoint{4.367853in}{1.581775in}}%
\pgfpathlineto{\pgfqpoint{4.368718in}{1.632917in}}%
\pgfpathlineto{\pgfqpoint{4.369583in}{1.582548in}}%
\pgfpathlineto{\pgfqpoint{4.371313in}{1.632058in}}%
\pgfpathlineto{\pgfqpoint{4.372179in}{1.587237in}}%
\pgfpathlineto{\pgfqpoint{4.373044in}{1.617720in}}%
\pgfpathlineto{\pgfqpoint{4.373909in}{1.540576in}}%
\pgfpathlineto{\pgfqpoint{4.374774in}{1.663935in}}%
\pgfpathlineto{\pgfqpoint{4.375640in}{1.614723in}}%
\pgfpathlineto{\pgfqpoint{4.376505in}{1.648381in}}%
\pgfpathlineto{\pgfqpoint{4.377370in}{1.630217in}}%
\pgfpathlineto{\pgfqpoint{4.378235in}{1.533987in}}%
\pgfpathlineto{\pgfqpoint{4.379100in}{1.655446in}}%
\pgfpathlineto{\pgfqpoint{4.379961in}{1.654554in}}%
\pgfpathlineto{\pgfqpoint{4.380827in}{1.636092in}}%
\pgfpathlineto{\pgfqpoint{4.381692in}{1.655386in}}%
\pgfpathlineto{\pgfqpoint{4.382557in}{1.586880in}}%
\pgfpathlineto{\pgfqpoint{4.384290in}{1.657168in}}%
\pgfpathlineto{\pgfqpoint{4.386020in}{1.634252in}}%
\pgfpathlineto{\pgfqpoint{4.386886in}{1.653486in}}%
\pgfpathlineto{\pgfqpoint{4.388616in}{1.572870in}}%
\pgfpathlineto{\pgfqpoint{4.389481in}{1.621371in}}%
\pgfpathlineto{\pgfqpoint{4.390346in}{1.600474in}}%
\pgfpathlineto{\pgfqpoint{4.391212in}{1.667200in}}%
\pgfpathlineto{\pgfqpoint{4.392939in}{1.592342in}}%
\pgfpathlineto{\pgfqpoint{4.394670in}{1.609588in}}%
\pgfpathlineto{\pgfqpoint{4.395533in}{1.672781in}}%
\pgfpathlineto{\pgfqpoint{4.396398in}{1.557554in}}%
\pgfpathlineto{\pgfqpoint{4.397262in}{1.576402in}}%
\pgfpathlineto{\pgfqpoint{4.398127in}{1.609971in}}%
\pgfpathlineto{\pgfqpoint{4.399855in}{1.567526in}}%
\pgfpathlineto{\pgfqpoint{4.400721in}{1.625584in}}%
\pgfpathlineto{\pgfqpoint{4.401586in}{1.619412in}}%
\pgfpathlineto{\pgfqpoint{4.402453in}{1.578391in}}%
\pgfpathlineto{\pgfqpoint{4.403318in}{1.619828in}}%
\pgfpathlineto{\pgfqpoint{4.405913in}{1.588721in}}%
\pgfpathlineto{\pgfqpoint{4.406779in}{1.586999in}}%
\pgfpathlineto{\pgfqpoint{4.407644in}{1.606056in}}%
\pgfpathlineto{\pgfqpoint{4.408510in}{1.679370in}}%
\pgfpathlineto{\pgfqpoint{4.409376in}{1.593588in}}%
\pgfpathlineto{\pgfqpoint{4.410240in}{1.623093in}}%
\pgfpathlineto{\pgfqpoint{4.412836in}{1.587178in}}%
\pgfpathlineto{\pgfqpoint{4.413701in}{1.681299in}}%
\pgfpathlineto{\pgfqpoint{4.414565in}{1.680143in}}%
\pgfpathlineto{\pgfqpoint{4.416295in}{1.590889in}}%
\pgfpathlineto{\pgfqpoint{4.418026in}{1.640249in}}%
\pgfpathlineto{\pgfqpoint{4.420621in}{1.561473in}}%
\pgfpathlineto{\pgfqpoint{4.422347in}{1.634965in}}%
\pgfpathlineto{\pgfqpoint{4.423211in}{1.612942in}}%
\pgfpathlineto{\pgfqpoint{4.424076in}{1.631760in}}%
\pgfpathlineto{\pgfqpoint{4.425802in}{1.513656in}}%
\pgfpathlineto{\pgfqpoint{4.427532in}{1.626417in}}%
\pgfpathlineto{\pgfqpoint{4.428398in}{1.605639in}}%
\pgfpathlineto{\pgfqpoint{4.429263in}{1.667557in}}%
\pgfpathlineto{\pgfqpoint{4.430129in}{1.631731in}}%
\pgfpathlineto{\pgfqpoint{4.430994in}{1.670583in}}%
\pgfpathlineto{\pgfqpoint{4.431859in}{1.648679in}}%
\pgfpathlineto{\pgfqpoint{4.432725in}{1.650043in}}%
\pgfpathlineto{\pgfqpoint{4.433588in}{1.676934in}}%
\pgfpathlineto{\pgfqpoint{4.435319in}{1.561384in}}%
\pgfpathlineto{\pgfqpoint{4.437050in}{1.617690in}}%
\pgfpathlineto{\pgfqpoint{4.437915in}{1.643038in}}%
\pgfpathlineto{\pgfqpoint{4.438780in}{1.626358in}}%
\pgfpathlineto{\pgfqpoint{4.439645in}{1.533690in}}%
\pgfpathlineto{\pgfqpoint{4.441371in}{1.603918in}}%
\pgfpathlineto{\pgfqpoint{4.442235in}{1.586345in}}%
\pgfpathlineto{\pgfqpoint{4.443101in}{1.597091in}}%
\pgfpathlineto{\pgfqpoint{4.443964in}{1.635322in}}%
\pgfpathlineto{\pgfqpoint{4.445695in}{1.585278in}}%
\pgfpathlineto{\pgfqpoint{4.446561in}{1.590026in}}%
\pgfpathlineto{\pgfqpoint{4.447427in}{1.619590in}}%
\pgfpathlineto{\pgfqpoint{4.448292in}{1.616564in}}%
\pgfpathlineto{\pgfqpoint{4.449156in}{1.561473in}}%
\pgfpathlineto{\pgfqpoint{4.450020in}{1.600355in}}%
\pgfpathlineto{\pgfqpoint{4.450885in}{1.538498in}}%
\pgfpathlineto{\pgfqpoint{4.452617in}{1.619531in}}%
\pgfpathlineto{\pgfqpoint{4.454349in}{1.530455in}}%
\pgfpathlineto{\pgfqpoint{4.455213in}{1.619471in}}%
\pgfpathlineto{\pgfqpoint{4.456077in}{1.587237in}}%
\pgfpathlineto{\pgfqpoint{4.456943in}{1.612525in}}%
\pgfpathlineto{\pgfqpoint{4.459537in}{1.556249in}}%
\pgfpathlineto{\pgfqpoint{4.460402in}{1.619828in}}%
\pgfpathlineto{\pgfqpoint{4.461266in}{1.599615in}}%
\pgfpathlineto{\pgfqpoint{4.462131in}{1.616861in}}%
\pgfpathlineto{\pgfqpoint{4.462995in}{1.554170in}}%
\pgfpathlineto{\pgfqpoint{4.464723in}{1.600415in}}%
\pgfpathlineto{\pgfqpoint{4.465588in}{1.599348in}}%
\pgfpathlineto{\pgfqpoint{4.466453in}{1.616207in}}%
\pgfpathlineto{\pgfqpoint{4.468183in}{1.573584in}}%
\pgfpathlineto{\pgfqpoint{4.469913in}{1.617690in}}%
\pgfpathlineto{\pgfqpoint{4.471642in}{1.555713in}}%
\pgfpathlineto{\pgfqpoint{4.472507in}{1.641673in}}%
\pgfpathlineto{\pgfqpoint{4.473373in}{1.611220in}}%
\pgfpathlineto{\pgfqpoint{4.474235in}{1.656160in}}%
\pgfpathlineto{\pgfqpoint{4.475100in}{1.636096in}}%
\pgfpathlineto{\pgfqpoint{4.475966in}{1.690651in}}%
\pgfpathlineto{\pgfqpoint{4.479425in}{1.563730in}}%
\pgfpathlineto{\pgfqpoint{4.480290in}{1.563671in}}%
\pgfpathlineto{\pgfqpoint{4.482023in}{1.662332in}}%
\pgfpathlineto{\pgfqpoint{4.482888in}{1.576967in}}%
\pgfpathlineto{\pgfqpoint{4.483754in}{1.645771in}}%
\pgfpathlineto{\pgfqpoint{4.484620in}{1.631998in}}%
\pgfpathlineto{\pgfqpoint{4.485486in}{1.640190in}}%
\pgfpathlineto{\pgfqpoint{4.487216in}{1.691480in}}%
\pgfpathlineto{\pgfqpoint{4.488081in}{1.582608in}}%
\pgfpathlineto{\pgfqpoint{4.488945in}{1.613299in}}%
\pgfpathlineto{\pgfqpoint{4.489808in}{1.613477in}}%
\pgfpathlineto{\pgfqpoint{4.491537in}{1.702464in}}%
\pgfpathlineto{\pgfqpoint{4.492402in}{1.632088in}}%
\pgfpathlineto{\pgfqpoint{4.493266in}{1.636925in}}%
\pgfpathlineto{\pgfqpoint{4.494133in}{1.725555in}}%
\pgfpathlineto{\pgfqpoint{4.494998in}{1.601958in}}%
\pgfpathlineto{\pgfqpoint{4.495863in}{1.705193in}}%
\pgfpathlineto{\pgfqpoint{4.498455in}{1.627841in}}%
\pgfpathlineto{\pgfqpoint{4.500183in}{1.672662in}}%
\pgfpathlineto{\pgfqpoint{4.501914in}{1.591896in}}%
\pgfpathlineto{\pgfqpoint{4.504511in}{1.646660in}}%
\pgfpathlineto{\pgfqpoint{4.506240in}{1.596972in}}%
\pgfpathlineto{\pgfqpoint{4.507970in}{1.679013in}}%
\pgfpathlineto{\pgfqpoint{4.508835in}{1.586851in}}%
\pgfpathlineto{\pgfqpoint{4.509700in}{1.623152in}}%
\pgfpathlineto{\pgfqpoint{4.510564in}{1.587118in}}%
\pgfpathlineto{\pgfqpoint{4.511429in}{1.604126in}}%
\pgfpathlineto{\pgfqpoint{4.512294in}{1.651824in}}%
\pgfpathlineto{\pgfqpoint{4.514025in}{1.583913in}}%
\pgfpathlineto{\pgfqpoint{4.514890in}{1.578570in}}%
\pgfpathlineto{\pgfqpoint{4.516619in}{1.494153in}}%
\pgfpathlineto{\pgfqpoint{4.517482in}{1.600474in}}%
\pgfpathlineto{\pgfqpoint{4.518346in}{1.595964in}}%
\pgfpathlineto{\pgfqpoint{4.519212in}{1.544317in}}%
\pgfpathlineto{\pgfqpoint{4.520942in}{1.602196in}}%
\pgfpathlineto{\pgfqpoint{4.522670in}{1.640309in}}%
\pgfpathlineto{\pgfqpoint{4.523535in}{1.516712in}}%
\pgfpathlineto{\pgfqpoint{4.524400in}{1.658889in}}%
\pgfpathlineto{\pgfqpoint{4.525263in}{1.569545in}}%
\pgfpathlineto{\pgfqpoint{4.526128in}{1.576848in}}%
\pgfpathlineto{\pgfqpoint{4.526993in}{1.576908in}}%
\pgfpathlineto{\pgfqpoint{4.530452in}{1.612972in}}%
\pgfpathlineto{\pgfqpoint{4.531316in}{1.615909in}}%
\pgfpathlineto{\pgfqpoint{4.532179in}{1.653486in}}%
\pgfpathlineto{\pgfqpoint{4.533043in}{1.538855in}}%
\pgfpathlineto{\pgfqpoint{4.533909in}{1.671059in}}%
\pgfpathlineto{\pgfqpoint{4.534774in}{1.578867in}}%
\pgfpathlineto{\pgfqpoint{4.535638in}{1.600534in}}%
\pgfpathlineto{\pgfqpoint{4.536505in}{1.631284in}}%
\pgfpathlineto{\pgfqpoint{4.538236in}{1.588364in}}%
\pgfpathlineto{\pgfqpoint{4.539965in}{1.620836in}}%
\pgfpathlineto{\pgfqpoint{4.540828in}{1.603858in}}%
\pgfpathlineto{\pgfqpoint{4.541691in}{1.605580in}}%
\pgfpathlineto{\pgfqpoint{4.542555in}{1.668386in}}%
\pgfpathlineto{\pgfqpoint{4.543419in}{1.656989in}}%
\pgfpathlineto{\pgfqpoint{4.544282in}{1.603590in}}%
\pgfpathlineto{\pgfqpoint{4.545145in}{1.673134in}}%
\pgfpathlineto{\pgfqpoint{4.546010in}{1.623446in}}%
\pgfpathlineto{\pgfqpoint{4.546874in}{1.689015in}}%
\pgfpathlineto{\pgfqpoint{4.548603in}{1.624279in}}%
\pgfpathlineto{\pgfqpoint{4.549468in}{1.653962in}}%
\pgfpathlineto{\pgfqpoint{4.550331in}{1.650400in}}%
\pgfpathlineto{\pgfqpoint{4.552060in}{1.598277in}}%
\pgfpathlineto{\pgfqpoint{4.552925in}{1.677113in}}%
\pgfpathlineto{\pgfqpoint{4.554650in}{1.615909in}}%
\pgfpathlineto{\pgfqpoint{4.555516in}{1.608963in}}%
\pgfpathlineto{\pgfqpoint{4.556379in}{1.629741in}}%
\pgfpathlineto{\pgfqpoint{4.558975in}{1.542268in}}%
\pgfpathlineto{\pgfqpoint{4.559839in}{1.631820in}}%
\pgfpathlineto{\pgfqpoint{4.560705in}{1.553873in}}%
\pgfpathlineto{\pgfqpoint{4.561571in}{1.574889in}}%
\pgfpathlineto{\pgfqpoint{4.562436in}{1.586940in}}%
\pgfpathlineto{\pgfqpoint{4.563300in}{1.585605in}}%
\pgfpathlineto{\pgfqpoint{4.565028in}{1.599288in}}%
\pgfpathlineto{\pgfqpoint{4.565894in}{1.579046in}}%
\pgfpathlineto{\pgfqpoint{4.568487in}{1.701929in}}%
\pgfpathlineto{\pgfqpoint{4.569351in}{1.615734in}}%
\pgfpathlineto{\pgfqpoint{4.570217in}{1.620304in}}%
\pgfpathlineto{\pgfqpoint{4.571084in}{1.668211in}}%
\pgfpathlineto{\pgfqpoint{4.572814in}{1.612793in}}%
\pgfpathlineto{\pgfqpoint{4.574545in}{1.659365in}}%
\pgfpathlineto{\pgfqpoint{4.575410in}{1.571713in}}%
\pgfpathlineto{\pgfqpoint{4.576275in}{1.668386in}}%
\pgfpathlineto{\pgfqpoint{4.578869in}{1.566697in}}%
\pgfpathlineto{\pgfqpoint{4.579732in}{1.740752in}}%
\pgfpathlineto{\pgfqpoint{4.580597in}{1.610566in}}%
\pgfpathlineto{\pgfqpoint{4.581462in}{1.640368in}}%
\pgfpathlineto{\pgfqpoint{4.583189in}{1.565154in}}%
\pgfpathlineto{\pgfqpoint{4.584053in}{1.624993in}}%
\pgfpathlineto{\pgfqpoint{4.584916in}{1.540993in}}%
\pgfpathlineto{\pgfqpoint{4.586645in}{1.604691in}}%
\pgfpathlineto{\pgfqpoint{4.587511in}{1.604750in}}%
\pgfpathlineto{\pgfqpoint{4.588377in}{1.568478in}}%
\pgfpathlineto{\pgfqpoint{4.589242in}{1.643930in}}%
\pgfpathlineto{\pgfqpoint{4.590109in}{1.624815in}}%
\pgfpathlineto{\pgfqpoint{4.590975in}{1.634965in}}%
\pgfpathlineto{\pgfqpoint{4.592706in}{1.583675in}}%
\pgfpathlineto{\pgfqpoint{4.593571in}{1.643633in}}%
\pgfpathlineto{\pgfqpoint{4.597029in}{1.521579in}}%
\pgfpathlineto{\pgfqpoint{4.597894in}{1.623863in}}%
\pgfpathlineto{\pgfqpoint{4.598760in}{1.595667in}}%
\pgfpathlineto{\pgfqpoint{4.599625in}{1.525141in}}%
\pgfpathlineto{\pgfqpoint{4.601354in}{1.578927in}}%
\pgfpathlineto{\pgfqpoint{4.602218in}{1.581422in}}%
\pgfpathlineto{\pgfqpoint{4.603082in}{1.654438in}}%
\pgfpathlineto{\pgfqpoint{4.603947in}{1.555955in}}%
\pgfpathlineto{\pgfqpoint{4.604812in}{1.566106in}}%
\pgfpathlineto{\pgfqpoint{4.605675in}{1.615675in}}%
\pgfpathlineto{\pgfqpoint{4.606541in}{1.543398in}}%
\pgfpathlineto{\pgfqpoint{4.607407in}{1.545566in}}%
\pgfpathlineto{\pgfqpoint{4.608272in}{1.525855in}}%
\pgfpathlineto{\pgfqpoint{4.609137in}{1.532920in}}%
\pgfpathlineto{\pgfqpoint{4.610002in}{1.610213in}}%
\pgfpathlineto{\pgfqpoint{4.610868in}{1.520482in}}%
\pgfpathlineto{\pgfqpoint{4.612594in}{1.649036in}}%
\pgfpathlineto{\pgfqpoint{4.614321in}{1.542833in}}%
\pgfpathlineto{\pgfqpoint{4.616052in}{1.607837in}}%
\pgfpathlineto{\pgfqpoint{4.616918in}{1.558268in}}%
\pgfpathlineto{\pgfqpoint{4.617784in}{1.558446in}}%
\pgfpathlineto{\pgfqpoint{4.618650in}{1.532682in}}%
\pgfpathlineto{\pgfqpoint{4.619515in}{1.610213in}}%
\pgfpathlineto{\pgfqpoint{4.620380in}{1.605226in}}%
\pgfpathlineto{\pgfqpoint{4.621246in}{1.644406in}}%
\pgfpathlineto{\pgfqpoint{4.622110in}{1.629269in}}%
\pgfpathlineto{\pgfqpoint{4.622974in}{1.587088in}}%
\pgfpathlineto{\pgfqpoint{4.624705in}{1.604334in}}%
\pgfpathlineto{\pgfqpoint{4.625570in}{1.596053in}}%
\pgfpathlineto{\pgfqpoint{4.626436in}{1.676997in}}%
\pgfpathlineto{\pgfqpoint{4.627301in}{1.630098in}}%
\pgfpathlineto{\pgfqpoint{4.628165in}{1.649512in}}%
\pgfpathlineto{\pgfqpoint{4.629028in}{1.638766in}}%
\pgfpathlineto{\pgfqpoint{4.629894in}{1.594778in}}%
\pgfpathlineto{\pgfqpoint{4.630756in}{1.650519in}}%
\pgfpathlineto{\pgfqpoint{4.631622in}{1.645890in}}%
\pgfpathlineto{\pgfqpoint{4.632489in}{1.659930in}}%
\pgfpathlineto{\pgfqpoint{4.633355in}{1.641082in}}%
\pgfpathlineto{\pgfqpoint{4.634220in}{1.574357in}}%
\pgfpathlineto{\pgfqpoint{4.635081in}{1.574387in}}%
\pgfpathlineto{\pgfqpoint{4.635945in}{1.651293in}}%
\pgfpathlineto{\pgfqpoint{4.637675in}{1.621847in}}%
\pgfpathlineto{\pgfqpoint{4.638541in}{1.663225in}}%
\pgfpathlineto{\pgfqpoint{4.640272in}{1.518969in}}%
\pgfpathlineto{\pgfqpoint{4.641137in}{1.622442in}}%
\pgfpathlineto{\pgfqpoint{4.642003in}{1.608313in}}%
\pgfpathlineto{\pgfqpoint{4.642868in}{1.618464in}}%
\pgfpathlineto{\pgfqpoint{4.643733in}{1.669992in}}%
\pgfpathlineto{\pgfqpoint{4.647191in}{1.526331in}}%
\pgfpathlineto{\pgfqpoint{4.648057in}{1.622799in}}%
\pgfpathlineto{\pgfqpoint{4.648923in}{1.557587in}}%
\pgfpathlineto{\pgfqpoint{4.649788in}{1.576852in}}%
\pgfpathlineto{\pgfqpoint{4.650652in}{1.569430in}}%
\pgfpathlineto{\pgfqpoint{4.652381in}{1.617337in}}%
\pgfpathlineto{\pgfqpoint{4.654112in}{1.569044in}}%
\pgfpathlineto{\pgfqpoint{4.654976in}{1.563909in}}%
\pgfpathlineto{\pgfqpoint{4.655843in}{1.602493in}}%
\pgfpathlineto{\pgfqpoint{4.656707in}{1.542536in}}%
\pgfpathlineto{\pgfqpoint{4.658437in}{1.609618in}}%
\pgfpathlineto{\pgfqpoint{4.660168in}{1.562008in}}%
\pgfpathlineto{\pgfqpoint{4.662761in}{1.647552in}}%
\pgfpathlineto{\pgfqpoint{4.663625in}{1.573970in}}%
\pgfpathlineto{\pgfqpoint{4.664490in}{1.575841in}}%
\pgfpathlineto{\pgfqpoint{4.665354in}{1.560525in}}%
\pgfpathlineto{\pgfqpoint{4.666218in}{1.589316in}}%
\pgfpathlineto{\pgfqpoint{4.667082in}{1.580589in}}%
\pgfpathlineto{\pgfqpoint{4.667948in}{1.532682in}}%
\pgfpathlineto{\pgfqpoint{4.668812in}{1.640190in}}%
\pgfpathlineto{\pgfqpoint{4.669677in}{1.561592in}}%
\pgfpathlineto{\pgfqpoint{4.670542in}{1.599318in}}%
\pgfpathlineto{\pgfqpoint{4.672272in}{1.567590in}}%
\pgfpathlineto{\pgfqpoint{4.674000in}{1.589613in}}%
\pgfpathlineto{\pgfqpoint{4.674866in}{1.629269in}}%
\pgfpathlineto{\pgfqpoint{4.675730in}{1.566816in}}%
\pgfpathlineto{\pgfqpoint{4.677460in}{1.611369in}}%
\pgfpathlineto{\pgfqpoint{4.678325in}{1.575722in}}%
\pgfpathlineto{\pgfqpoint{4.679190in}{1.618583in}}%
\pgfpathlineto{\pgfqpoint{4.680055in}{1.569222in}}%
\pgfpathlineto{\pgfqpoint{4.682650in}{1.632474in}}%
\pgfpathlineto{\pgfqpoint{4.683516in}{1.626123in}}%
\pgfpathlineto{\pgfqpoint{4.684380in}{1.570616in}}%
\pgfpathlineto{\pgfqpoint{4.685244in}{1.606532in}}%
\pgfpathlineto{\pgfqpoint{4.686110in}{1.592878in}}%
\pgfpathlineto{\pgfqpoint{4.687840in}{1.641023in}}%
\pgfpathlineto{\pgfqpoint{4.688705in}{1.609439in}}%
\pgfpathlineto{\pgfqpoint{4.690432in}{1.680084in}}%
\pgfpathlineto{\pgfqpoint{4.693027in}{1.540398in}}%
\pgfpathlineto{\pgfqpoint{4.693889in}{1.631582in}}%
\pgfpathlineto{\pgfqpoint{4.694755in}{1.576729in}}%
\pgfpathlineto{\pgfqpoint{4.695621in}{1.627901in}}%
\pgfpathlineto{\pgfqpoint{4.696486in}{1.590651in}}%
\pgfpathlineto{\pgfqpoint{4.697352in}{1.612823in}}%
\pgfpathlineto{\pgfqpoint{4.698218in}{1.591275in}}%
\pgfpathlineto{\pgfqpoint{4.699949in}{1.619888in}}%
\pgfpathlineto{\pgfqpoint{4.700812in}{1.620245in}}%
\pgfpathlineto{\pgfqpoint{4.701675in}{1.545919in}}%
\pgfpathlineto{\pgfqpoint{4.702540in}{1.595905in}}%
\pgfpathlineto{\pgfqpoint{4.704271in}{1.538736in}}%
\pgfpathlineto{\pgfqpoint{4.705999in}{1.698128in}}%
\pgfpathlineto{\pgfqpoint{4.708594in}{1.665835in}}%
\pgfpathlineto{\pgfqpoint{4.709461in}{1.728700in}}%
\pgfpathlineto{\pgfqpoint{4.710325in}{1.727395in}}%
\pgfpathlineto{\pgfqpoint{4.711190in}{1.626179in}}%
\pgfpathlineto{\pgfqpoint{4.712921in}{1.698485in}}%
\pgfpathlineto{\pgfqpoint{4.713785in}{1.645533in}}%
\pgfpathlineto{\pgfqpoint{4.714648in}{1.686137in}}%
\pgfpathlineto{\pgfqpoint{4.715513in}{1.637371in}}%
\pgfpathlineto{\pgfqpoint{4.718111in}{1.704926in}}%
\pgfpathlineto{\pgfqpoint{4.718976in}{1.633303in}}%
\pgfpathlineto{\pgfqpoint{4.719839in}{1.690472in}}%
\pgfpathlineto{\pgfqpoint{4.720705in}{1.653516in}}%
\pgfpathlineto{\pgfqpoint{4.721569in}{1.667557in}}%
\pgfpathlineto{\pgfqpoint{4.723300in}{1.744373in}}%
\pgfpathlineto{\pgfqpoint{4.724166in}{1.724666in}}%
\pgfpathlineto{\pgfqpoint{4.725030in}{1.648887in}}%
\pgfpathlineto{\pgfqpoint{4.725894in}{1.706260in}}%
\pgfpathlineto{\pgfqpoint{4.726759in}{1.659008in}}%
\pgfpathlineto{\pgfqpoint{4.728487in}{1.700326in}}%
\pgfpathlineto{\pgfqpoint{4.729353in}{1.613001in}}%
\pgfpathlineto{\pgfqpoint{4.731080in}{1.673078in}}%
\pgfpathlineto{\pgfqpoint{4.731947in}{1.624993in}}%
\pgfpathlineto{\pgfqpoint{4.732812in}{1.688453in}}%
\pgfpathlineto{\pgfqpoint{4.733678in}{1.683289in}}%
\pgfpathlineto{\pgfqpoint{4.734543in}{1.639093in}}%
\pgfpathlineto{\pgfqpoint{4.736271in}{1.720453in}}%
\pgfpathlineto{\pgfqpoint{4.737133in}{1.678868in}}%
\pgfpathlineto{\pgfqpoint{4.737997in}{1.724904in}}%
\pgfpathlineto{\pgfqpoint{4.739729in}{1.650846in}}%
\pgfpathlineto{\pgfqpoint{4.740595in}{1.762719in}}%
\pgfpathlineto{\pgfqpoint{4.741460in}{1.633660in}}%
\pgfpathlineto{\pgfqpoint{4.742327in}{1.707153in}}%
\pgfpathlineto{\pgfqpoint{4.743193in}{1.637460in}}%
\pgfpathlineto{\pgfqpoint{4.744060in}{1.705967in}}%
\pgfpathlineto{\pgfqpoint{4.744925in}{1.665363in}}%
\pgfpathlineto{\pgfqpoint{4.745789in}{1.686200in}}%
\pgfpathlineto{\pgfqpoint{4.746655in}{1.600567in}}%
\pgfpathlineto{\pgfqpoint{4.747518in}{1.692492in}}%
\pgfpathlineto{\pgfqpoint{4.748384in}{1.674268in}}%
\pgfpathlineto{\pgfqpoint{4.749246in}{1.644644in}}%
\pgfpathlineto{\pgfqpoint{4.750977in}{1.746630in}}%
\pgfpathlineto{\pgfqpoint{4.751842in}{1.683527in}}%
\pgfpathlineto{\pgfqpoint{4.752708in}{1.762184in}}%
\pgfpathlineto{\pgfqpoint{4.754440in}{1.688751in}}%
\pgfpathlineto{\pgfqpoint{4.755305in}{1.705550in}}%
\pgfpathlineto{\pgfqpoint{4.756170in}{1.664173in}}%
\pgfpathlineto{\pgfqpoint{4.757035in}{1.667438in}}%
\pgfpathlineto{\pgfqpoint{4.757901in}{1.693797in}}%
\pgfpathlineto{\pgfqpoint{4.758765in}{1.688453in}}%
\pgfpathlineto{\pgfqpoint{4.759630in}{1.818401in}}%
\pgfpathlineto{\pgfqpoint{4.761358in}{1.609558in}}%
\pgfpathlineto{\pgfqpoint{4.762223in}{1.631582in}}%
\pgfpathlineto{\pgfqpoint{4.763088in}{1.602702in}}%
\pgfpathlineto{\pgfqpoint{4.763954in}{1.705907in}}%
\pgfpathlineto{\pgfqpoint{4.766549in}{1.592640in}}%
\pgfpathlineto{\pgfqpoint{4.767415in}{1.659484in}}%
\pgfpathlineto{\pgfqpoint{4.770012in}{1.557911in}}%
\pgfpathlineto{\pgfqpoint{4.771744in}{1.640309in}}%
\pgfpathlineto{\pgfqpoint{4.773477in}{1.607896in}}%
\pgfpathlineto{\pgfqpoint{4.774342in}{1.566043in}}%
\pgfpathlineto{\pgfqpoint{4.775208in}{1.612853in}}%
\pgfpathlineto{\pgfqpoint{4.776073in}{1.585694in}}%
\pgfpathlineto{\pgfqpoint{4.776937in}{1.643811in}}%
\pgfpathlineto{\pgfqpoint{4.777802in}{1.559603in}}%
\pgfpathlineto{\pgfqpoint{4.778668in}{1.640249in}}%
\pgfpathlineto{\pgfqpoint{4.781267in}{1.552806in}}%
\pgfpathlineto{\pgfqpoint{4.782133in}{1.636895in}}%
\pgfpathlineto{\pgfqpoint{4.783865in}{1.561176in}}%
\pgfpathlineto{\pgfqpoint{4.784730in}{1.649036in}}%
\pgfpathlineto{\pgfqpoint{4.785596in}{1.624874in}}%
\pgfpathlineto{\pgfqpoint{4.786461in}{1.628198in}}%
\pgfpathlineto{\pgfqpoint{4.787327in}{1.585813in}}%
\pgfpathlineto{\pgfqpoint{4.788193in}{1.670940in}}%
\pgfpathlineto{\pgfqpoint{4.789925in}{1.574948in}}%
\pgfpathlineto{\pgfqpoint{4.790791in}{1.606115in}}%
\pgfpathlineto{\pgfqpoint{4.792520in}{1.527815in}}%
\pgfpathlineto{\pgfqpoint{4.793384in}{1.622855in}}%
\pgfpathlineto{\pgfqpoint{4.794249in}{1.544703in}}%
\pgfpathlineto{\pgfqpoint{4.795115in}{1.583794in}}%
\pgfpathlineto{\pgfqpoint{4.795980in}{1.704126in}}%
\pgfpathlineto{\pgfqpoint{4.796845in}{1.663225in}}%
\pgfpathlineto{\pgfqpoint{4.798574in}{1.786464in}}%
\pgfpathlineto{\pgfqpoint{4.801166in}{1.644585in}}%
\pgfpathlineto{\pgfqpoint{4.802893in}{1.697894in}}%
\pgfpathlineto{\pgfqpoint{4.803757in}{1.675365in}}%
\pgfpathlineto{\pgfqpoint{4.804623in}{1.705967in}}%
\pgfpathlineto{\pgfqpoint{4.806350in}{1.532742in}}%
\pgfpathlineto{\pgfqpoint{4.808078in}{1.650698in}}%
\pgfpathlineto{\pgfqpoint{4.808943in}{1.592997in}}%
\pgfpathlineto{\pgfqpoint{4.809809in}{1.595075in}}%
\pgfpathlineto{\pgfqpoint{4.812399in}{1.733452in}}%
\pgfpathlineto{\pgfqpoint{4.813264in}{1.684032in}}%
\pgfpathlineto{\pgfqpoint{4.814129in}{1.690651in}}%
\pgfpathlineto{\pgfqpoint{4.814993in}{1.743841in}}%
\pgfpathlineto{\pgfqpoint{4.815858in}{1.669814in}}%
\pgfpathlineto{\pgfqpoint{4.816722in}{1.745147in}}%
\pgfpathlineto{\pgfqpoint{4.818454in}{1.616207in}}%
\pgfpathlineto{\pgfqpoint{4.820184in}{1.693945in}}%
\pgfpathlineto{\pgfqpoint{4.821048in}{1.661146in}}%
\pgfpathlineto{\pgfqpoint{4.821914in}{1.690710in}}%
\pgfpathlineto{\pgfqpoint{4.822781in}{1.690562in}}%
\pgfpathlineto{\pgfqpoint{4.824513in}{1.620840in}}%
\pgfpathlineto{\pgfqpoint{4.825379in}{1.630782in}}%
\pgfpathlineto{\pgfqpoint{4.826244in}{1.629269in}}%
\pgfpathlineto{\pgfqpoint{4.827110in}{1.714575in}}%
\pgfpathlineto{\pgfqpoint{4.827976in}{1.582846in}}%
\pgfpathlineto{\pgfqpoint{4.830573in}{1.694154in}}%
\pgfpathlineto{\pgfqpoint{4.831440in}{1.735115in}}%
\pgfpathlineto{\pgfqpoint{4.832305in}{1.532385in}}%
\pgfpathlineto{\pgfqpoint{4.833170in}{1.534523in}}%
\pgfpathlineto{\pgfqpoint{4.834033in}{1.589375in}}%
\pgfpathlineto{\pgfqpoint{4.834897in}{1.574000in}}%
\pgfpathlineto{\pgfqpoint{4.835762in}{1.569014in}}%
\pgfpathlineto{\pgfqpoint{4.836629in}{1.612882in}}%
\pgfpathlineto{\pgfqpoint{4.838359in}{1.565273in}}%
\pgfpathlineto{\pgfqpoint{4.839223in}{1.508937in}}%
\pgfpathlineto{\pgfqpoint{4.840953in}{1.639004in}}%
\pgfpathlineto{\pgfqpoint{4.842682in}{1.551917in}}%
\pgfpathlineto{\pgfqpoint{4.842682in}{1.551917in}}%
\pgfusepath{stroke}%
\end{pgfscope}%
\begin{pgfscope}%
\pgfsetrectcap%
\pgfsetmiterjoin%
\pgfsetlinewidth{0.803000pt}%
\definecolor{currentstroke}{rgb}{0.000000,0.000000,0.000000}%
\pgfsetstrokecolor{currentstroke}%
\pgfsetdash{}{0pt}%
\pgfpathmoveto{\pgfqpoint{0.483776in}{1.444834in}}%
\pgfpathlineto{\pgfqpoint{0.483776in}{2.029715in}}%
\pgfusepath{stroke}%
\end{pgfscope}%
\begin{pgfscope}%
\pgfsetrectcap%
\pgfsetmiterjoin%
\pgfsetlinewidth{0.803000pt}%
\definecolor{currentstroke}{rgb}{0.000000,0.000000,0.000000}%
\pgfsetstrokecolor{currentstroke}%
\pgfsetdash{}{0pt}%
\pgfpathmoveto{\pgfqpoint{5.050249in}{1.444834in}}%
\pgfpathlineto{\pgfqpoint{5.050249in}{2.029715in}}%
\pgfusepath{stroke}%
\end{pgfscope}%
\begin{pgfscope}%
\pgfsetrectcap%
\pgfsetmiterjoin%
\pgfsetlinewidth{0.803000pt}%
\definecolor{currentstroke}{rgb}{0.000000,0.000000,0.000000}%
\pgfsetstrokecolor{currentstroke}%
\pgfsetdash{}{0pt}%
\pgfpathmoveto{\pgfqpoint{0.483776in}{1.444834in}}%
\pgfpathlineto{\pgfqpoint{5.050249in}{1.444834in}}%
\pgfusepath{stroke}%
\end{pgfscope}%
\begin{pgfscope}%
\pgfsetrectcap%
\pgfsetmiterjoin%
\pgfsetlinewidth{0.803000pt}%
\definecolor{currentstroke}{rgb}{0.000000,0.000000,0.000000}%
\pgfsetstrokecolor{currentstroke}%
\pgfsetdash{}{0pt}%
\pgfpathmoveto{\pgfqpoint{0.483776in}{2.029715in}}%
\pgfpathlineto{\pgfqpoint{5.050249in}{2.029715in}}%
\pgfusepath{stroke}%
\end{pgfscope}%
\begin{pgfscope}%
\pgfsetbuttcap%
\pgfsetmiterjoin%
\definecolor{currentfill}{rgb}{1.000000,1.000000,1.000000}%
\pgfsetfillcolor{currentfill}%
\pgfsetlinewidth{0.000000pt}%
\definecolor{currentstroke}{rgb}{0.000000,0.000000,0.000000}%
\pgfsetstrokecolor{currentstroke}%
\pgfsetstrokeopacity{0.000000}%
\pgfsetdash{}{0pt}%
\pgfpathmoveto{\pgfqpoint{0.483776in}{0.538014in}}%
\pgfpathlineto{\pgfqpoint{5.050249in}{0.538014in}}%
\pgfpathlineto{\pgfqpoint{5.050249in}{1.122895in}}%
\pgfpathlineto{\pgfqpoint{0.483776in}{1.122895in}}%
\pgfpathlineto{\pgfqpoint{0.483776in}{0.538014in}}%
\pgfpathclose%
\pgfusepath{fill}%
\end{pgfscope}%
\begin{pgfscope}%
\pgfsetbuttcap%
\pgfsetroundjoin%
\definecolor{currentfill}{rgb}{0.000000,0.000000,0.000000}%
\pgfsetfillcolor{currentfill}%
\pgfsetlinewidth{0.803000pt}%
\definecolor{currentstroke}{rgb}{0.000000,0.000000,0.000000}%
\pgfsetstrokecolor{currentstroke}%
\pgfsetdash{}{0pt}%
\pgfsys@defobject{currentmarker}{\pgfqpoint{0.000000in}{-0.048611in}}{\pgfqpoint{0.000000in}{0.000000in}}{%
\pgfpathmoveto{\pgfqpoint{0.000000in}{0.000000in}}%
\pgfpathlineto{\pgfqpoint{0.000000in}{-0.048611in}}%
\pgfusepath{stroke,fill}%
}%
\begin{pgfscope}%
\pgfsys@transformshift{0.691021in}{0.538014in}%
\pgfsys@useobject{currentmarker}{}%
\end{pgfscope}%
\end{pgfscope}%
\begin{pgfscope}%
\definecolor{textcolor}{rgb}{0.000000,0.000000,0.000000}%
\pgfsetstrokecolor{textcolor}%
\pgfsetfillcolor{textcolor}%
\pgftext[x=0.691021in,y=0.440792in,,top]{\color{textcolor}\rmfamily\fontsize{8.000000}{9.600000}\selectfont \(\displaystyle {06{:}00}\)}%
\end{pgfscope}%
\begin{pgfscope}%
\pgfsetbuttcap%
\pgfsetroundjoin%
\definecolor{currentfill}{rgb}{0.000000,0.000000,0.000000}%
\pgfsetfillcolor{currentfill}%
\pgfsetlinewidth{0.803000pt}%
\definecolor{currentstroke}{rgb}{0.000000,0.000000,0.000000}%
\pgfsetstrokecolor{currentstroke}%
\pgfsetdash{}{0pt}%
\pgfsys@defobject{currentmarker}{\pgfqpoint{0.000000in}{-0.048611in}}{\pgfqpoint{0.000000in}{0.000000in}}{%
\pgfpathmoveto{\pgfqpoint{0.000000in}{0.000000in}}%
\pgfpathlineto{\pgfqpoint{0.000000in}{-0.048611in}}%
\pgfusepath{stroke,fill}%
}%
\begin{pgfscope}%
\pgfsys@transformshift{1.210067in}{0.538014in}%
\pgfsys@useobject{currentmarker}{}%
\end{pgfscope}%
\end{pgfscope}%
\begin{pgfscope}%
\definecolor{textcolor}{rgb}{0.000000,0.000000,0.000000}%
\pgfsetstrokecolor{textcolor}%
\pgfsetfillcolor{textcolor}%
\pgftext[x=1.210067in,y=0.440792in,,top]{\color{textcolor}\rmfamily\fontsize{8.000000}{9.600000}\selectfont \(\displaystyle {09{:}00}\)}%
\end{pgfscope}%
\begin{pgfscope}%
\pgfsetbuttcap%
\pgfsetroundjoin%
\definecolor{currentfill}{rgb}{0.000000,0.000000,0.000000}%
\pgfsetfillcolor{currentfill}%
\pgfsetlinewidth{0.803000pt}%
\definecolor{currentstroke}{rgb}{0.000000,0.000000,0.000000}%
\pgfsetstrokecolor{currentstroke}%
\pgfsetdash{}{0pt}%
\pgfsys@defobject{currentmarker}{\pgfqpoint{0.000000in}{-0.048611in}}{\pgfqpoint{0.000000in}{0.000000in}}{%
\pgfpathmoveto{\pgfqpoint{0.000000in}{0.000000in}}%
\pgfpathlineto{\pgfqpoint{0.000000in}{-0.048611in}}%
\pgfusepath{stroke,fill}%
}%
\begin{pgfscope}%
\pgfsys@transformshift{1.729114in}{0.538014in}%
\pgfsys@useobject{currentmarker}{}%
\end{pgfscope}%
\end{pgfscope}%
\begin{pgfscope}%
\definecolor{textcolor}{rgb}{0.000000,0.000000,0.000000}%
\pgfsetstrokecolor{textcolor}%
\pgfsetfillcolor{textcolor}%
\pgftext[x=1.729114in,y=0.440792in,,top]{\color{textcolor}\rmfamily\fontsize{8.000000}{9.600000}\selectfont \(\displaystyle {12{:}00}\)}%
\end{pgfscope}%
\begin{pgfscope}%
\pgfsetbuttcap%
\pgfsetroundjoin%
\definecolor{currentfill}{rgb}{0.000000,0.000000,0.000000}%
\pgfsetfillcolor{currentfill}%
\pgfsetlinewidth{0.803000pt}%
\definecolor{currentstroke}{rgb}{0.000000,0.000000,0.000000}%
\pgfsetstrokecolor{currentstroke}%
\pgfsetdash{}{0pt}%
\pgfsys@defobject{currentmarker}{\pgfqpoint{0.000000in}{-0.048611in}}{\pgfqpoint{0.000000in}{0.000000in}}{%
\pgfpathmoveto{\pgfqpoint{0.000000in}{0.000000in}}%
\pgfpathlineto{\pgfqpoint{0.000000in}{-0.048611in}}%
\pgfusepath{stroke,fill}%
}%
\begin{pgfscope}%
\pgfsys@transformshift{2.248160in}{0.538014in}%
\pgfsys@useobject{currentmarker}{}%
\end{pgfscope}%
\end{pgfscope}%
\begin{pgfscope}%
\definecolor{textcolor}{rgb}{0.000000,0.000000,0.000000}%
\pgfsetstrokecolor{textcolor}%
\pgfsetfillcolor{textcolor}%
\pgftext[x=2.248160in,y=0.440792in,,top]{\color{textcolor}\rmfamily\fontsize{8.000000}{9.600000}\selectfont \(\displaystyle {15{:}00}\)}%
\end{pgfscope}%
\begin{pgfscope}%
\pgfsetbuttcap%
\pgfsetroundjoin%
\definecolor{currentfill}{rgb}{0.000000,0.000000,0.000000}%
\pgfsetfillcolor{currentfill}%
\pgfsetlinewidth{0.803000pt}%
\definecolor{currentstroke}{rgb}{0.000000,0.000000,0.000000}%
\pgfsetstrokecolor{currentstroke}%
\pgfsetdash{}{0pt}%
\pgfsys@defobject{currentmarker}{\pgfqpoint{0.000000in}{-0.048611in}}{\pgfqpoint{0.000000in}{0.000000in}}{%
\pgfpathmoveto{\pgfqpoint{0.000000in}{0.000000in}}%
\pgfpathlineto{\pgfqpoint{0.000000in}{-0.048611in}}%
\pgfusepath{stroke,fill}%
}%
\begin{pgfscope}%
\pgfsys@transformshift{2.767206in}{0.538014in}%
\pgfsys@useobject{currentmarker}{}%
\end{pgfscope}%
\end{pgfscope}%
\begin{pgfscope}%
\definecolor{textcolor}{rgb}{0.000000,0.000000,0.000000}%
\pgfsetstrokecolor{textcolor}%
\pgfsetfillcolor{textcolor}%
\pgftext[x=2.767206in,y=0.440792in,,top]{\color{textcolor}\rmfamily\fontsize{8.000000}{9.600000}\selectfont \(\displaystyle {18{:}00}\)}%
\end{pgfscope}%
\begin{pgfscope}%
\pgfsetbuttcap%
\pgfsetroundjoin%
\definecolor{currentfill}{rgb}{0.000000,0.000000,0.000000}%
\pgfsetfillcolor{currentfill}%
\pgfsetlinewidth{0.803000pt}%
\definecolor{currentstroke}{rgb}{0.000000,0.000000,0.000000}%
\pgfsetstrokecolor{currentstroke}%
\pgfsetdash{}{0pt}%
\pgfsys@defobject{currentmarker}{\pgfqpoint{0.000000in}{-0.048611in}}{\pgfqpoint{0.000000in}{0.000000in}}{%
\pgfpathmoveto{\pgfqpoint{0.000000in}{0.000000in}}%
\pgfpathlineto{\pgfqpoint{0.000000in}{-0.048611in}}%
\pgfusepath{stroke,fill}%
}%
\begin{pgfscope}%
\pgfsys@transformshift{3.286252in}{0.538014in}%
\pgfsys@useobject{currentmarker}{}%
\end{pgfscope}%
\end{pgfscope}%
\begin{pgfscope}%
\definecolor{textcolor}{rgb}{0.000000,0.000000,0.000000}%
\pgfsetstrokecolor{textcolor}%
\pgfsetfillcolor{textcolor}%
\pgftext[x=3.286252in,y=0.440792in,,top]{\color{textcolor}\rmfamily\fontsize{8.000000}{9.600000}\selectfont \(\displaystyle {21{:}00}\)}%
\end{pgfscope}%
\begin{pgfscope}%
\pgfsetbuttcap%
\pgfsetroundjoin%
\definecolor{currentfill}{rgb}{0.000000,0.000000,0.000000}%
\pgfsetfillcolor{currentfill}%
\pgfsetlinewidth{0.803000pt}%
\definecolor{currentstroke}{rgb}{0.000000,0.000000,0.000000}%
\pgfsetstrokecolor{currentstroke}%
\pgfsetdash{}{0pt}%
\pgfsys@defobject{currentmarker}{\pgfqpoint{0.000000in}{-0.048611in}}{\pgfqpoint{0.000000in}{0.000000in}}{%
\pgfpathmoveto{\pgfqpoint{0.000000in}{0.000000in}}%
\pgfpathlineto{\pgfqpoint{0.000000in}{-0.048611in}}%
\pgfusepath{stroke,fill}%
}%
\begin{pgfscope}%
\pgfsys@transformshift{3.805298in}{0.538014in}%
\pgfsys@useobject{currentmarker}{}%
\end{pgfscope}%
\end{pgfscope}%
\begin{pgfscope}%
\definecolor{textcolor}{rgb}{0.000000,0.000000,0.000000}%
\pgfsetstrokecolor{textcolor}%
\pgfsetfillcolor{textcolor}%
\pgftext[x=3.805298in,y=0.440792in,,top]{\color{textcolor}\rmfamily\fontsize{8.000000}{9.600000}\selectfont \(\displaystyle {00{:}00}\)}%
\end{pgfscope}%
\begin{pgfscope}%
\pgfsetbuttcap%
\pgfsetroundjoin%
\definecolor{currentfill}{rgb}{0.000000,0.000000,0.000000}%
\pgfsetfillcolor{currentfill}%
\pgfsetlinewidth{0.803000pt}%
\definecolor{currentstroke}{rgb}{0.000000,0.000000,0.000000}%
\pgfsetstrokecolor{currentstroke}%
\pgfsetdash{}{0pt}%
\pgfsys@defobject{currentmarker}{\pgfqpoint{0.000000in}{-0.048611in}}{\pgfqpoint{0.000000in}{0.000000in}}{%
\pgfpathmoveto{\pgfqpoint{0.000000in}{0.000000in}}%
\pgfpathlineto{\pgfqpoint{0.000000in}{-0.048611in}}%
\pgfusepath{stroke,fill}%
}%
\begin{pgfscope}%
\pgfsys@transformshift{4.324344in}{0.538014in}%
\pgfsys@useobject{currentmarker}{}%
\end{pgfscope}%
\end{pgfscope}%
\begin{pgfscope}%
\definecolor{textcolor}{rgb}{0.000000,0.000000,0.000000}%
\pgfsetstrokecolor{textcolor}%
\pgfsetfillcolor{textcolor}%
\pgftext[x=4.324344in,y=0.440792in,,top]{\color{textcolor}\rmfamily\fontsize{8.000000}{9.600000}\selectfont \(\displaystyle {03{:}00}\)}%
\end{pgfscope}%
\begin{pgfscope}%
\pgfsetbuttcap%
\pgfsetroundjoin%
\definecolor{currentfill}{rgb}{0.000000,0.000000,0.000000}%
\pgfsetfillcolor{currentfill}%
\pgfsetlinewidth{0.803000pt}%
\definecolor{currentstroke}{rgb}{0.000000,0.000000,0.000000}%
\pgfsetstrokecolor{currentstroke}%
\pgfsetdash{}{0pt}%
\pgfsys@defobject{currentmarker}{\pgfqpoint{0.000000in}{-0.048611in}}{\pgfqpoint{0.000000in}{0.000000in}}{%
\pgfpathmoveto{\pgfqpoint{0.000000in}{0.000000in}}%
\pgfpathlineto{\pgfqpoint{0.000000in}{-0.048611in}}%
\pgfusepath{stroke,fill}%
}%
\begin{pgfscope}%
\pgfsys@transformshift{4.843390in}{0.538014in}%
\pgfsys@useobject{currentmarker}{}%
\end{pgfscope}%
\end{pgfscope}%
\begin{pgfscope}%
\definecolor{textcolor}{rgb}{0.000000,0.000000,0.000000}%
\pgfsetstrokecolor{textcolor}%
\pgfsetfillcolor{textcolor}%
\pgftext[x=4.843390in,y=0.440792in,,top]{\color{textcolor}\rmfamily\fontsize{8.000000}{9.600000}\selectfont \(\displaystyle {06{:}00}\)}%
\end{pgfscope}%
\begin{pgfscope}%
\definecolor{textcolor}{rgb}{0.000000,0.000000,0.000000}%
\pgfsetstrokecolor{textcolor}%
\pgfsetfillcolor{textcolor}%
\pgftext[x=2.767012in,y=0.286570in,,top]{\color{textcolor}\rmfamily\fontsize{10.000000}{12.000000}\selectfont Time (UTC)}%
\end{pgfscope}%
\begin{pgfscope}%
\pgfsetbuttcap%
\pgfsetroundjoin%
\definecolor{currentfill}{rgb}{0.000000,0.000000,0.000000}%
\pgfsetfillcolor{currentfill}%
\pgfsetlinewidth{0.803000pt}%
\definecolor{currentstroke}{rgb}{0.000000,0.000000,0.000000}%
\pgfsetstrokecolor{currentstroke}%
\pgfsetdash{}{0pt}%
\pgfsys@defobject{currentmarker}{\pgfqpoint{-0.048611in}{0.000000in}}{\pgfqpoint{-0.000000in}{0.000000in}}{%
\pgfpathmoveto{\pgfqpoint{-0.000000in}{0.000000in}}%
\pgfpathlineto{\pgfqpoint{-0.048611in}{0.000000in}}%
\pgfusepath{stroke,fill}%
}%
\begin{pgfscope}%
\pgfsys@transformshift{0.483776in}{0.719191in}%
\pgfsys@useobject{currentmarker}{}%
\end{pgfscope}%
\end{pgfscope}%
\begin{pgfscope}%
\definecolor{textcolor}{rgb}{0.000000,0.000000,0.000000}%
\pgfsetstrokecolor{textcolor}%
\pgfsetfillcolor{textcolor}%
\pgftext[x=0.327525in, y=0.680636in, left, base]{\color{textcolor}\rmfamily\fontsize{8.000000}{9.600000}\selectfont \(\displaystyle {0}\)}%
\end{pgfscope}%
\begin{pgfscope}%
\pgfsetbuttcap%
\pgfsetroundjoin%
\definecolor{currentfill}{rgb}{0.000000,0.000000,0.000000}%
\pgfsetfillcolor{currentfill}%
\pgfsetlinewidth{0.803000pt}%
\definecolor{currentstroke}{rgb}{0.000000,0.000000,0.000000}%
\pgfsetstrokecolor{currentstroke}%
\pgfsetdash{}{0pt}%
\pgfsys@defobject{currentmarker}{\pgfqpoint{-0.048611in}{0.000000in}}{\pgfqpoint{-0.000000in}{0.000000in}}{%
\pgfpathmoveto{\pgfqpoint{-0.000000in}{0.000000in}}%
\pgfpathlineto{\pgfqpoint{-0.048611in}{0.000000in}}%
\pgfusepath{stroke,fill}%
}%
\begin{pgfscope}%
\pgfsys@transformshift{0.483776in}{0.926366in}%
\pgfsys@useobject{currentmarker}{}%
\end{pgfscope}%
\end{pgfscope}%
\begin{pgfscope}%
\definecolor{textcolor}{rgb}{0.000000,0.000000,0.000000}%
\pgfsetstrokecolor{textcolor}%
\pgfsetfillcolor{textcolor}%
\pgftext[x=0.327525in, y=0.887811in, left, base]{\color{textcolor}\rmfamily\fontsize{8.000000}{9.600000}\selectfont \(\displaystyle {5}\)}%
\end{pgfscope}%
\begin{pgfscope}%
\definecolor{textcolor}{rgb}{0.000000,0.000000,0.000000}%
\pgfsetstrokecolor{textcolor}%
\pgfsetfillcolor{textcolor}%
\pgftext[x=0.483776in,y=1.164562in,left,base]{\color{textcolor}\rmfamily\fontsize{8.000000}{9.600000}\selectfont \(\displaystyle \times{10^{\ensuremath{-}6}}{}\)}%
\end{pgfscope}%
\begin{pgfscope}%
\pgfpathrectangle{\pgfqpoint{0.483776in}{0.538014in}}{\pgfqpoint{4.566474in}{0.584881in}}%
\pgfusepath{clip}%
\pgfsetrectcap%
\pgfsetroundjoin%
\pgfsetlinewidth{0.501875pt}%
\definecolor{currentstroke}{rgb}{0.000000,0.419608,0.643137}%
\pgfsetstrokecolor{currentstroke}%
\pgfsetstrokeopacity{0.700000}%
\pgfsetdash{}{0pt}%
\pgfpathmoveto{\pgfqpoint{0.691343in}{0.723309in}}%
\pgfpathlineto{\pgfqpoint{0.692205in}{0.730232in}}%
\pgfpathlineto{\pgfqpoint{0.693935in}{0.701294in}}%
\pgfpathlineto{\pgfqpoint{0.694800in}{0.716533in}}%
\pgfpathlineto{\pgfqpoint{0.695666in}{0.686239in}}%
\pgfpathlineto{\pgfqpoint{0.696532in}{0.687741in}}%
\pgfpathlineto{\pgfqpoint{0.698263in}{0.732136in}}%
\pgfpathlineto{\pgfqpoint{0.699128in}{0.727776in}}%
\pgfpathlineto{\pgfqpoint{0.699993in}{0.774404in}}%
\pgfpathlineto{\pgfqpoint{0.701725in}{0.645250in}}%
\pgfpathlineto{\pgfqpoint{0.703453in}{0.750814in}}%
\pgfpathlineto{\pgfqpoint{0.704319in}{0.749056in}}%
\pgfpathlineto{\pgfqpoint{0.705185in}{0.769605in}}%
\pgfpathlineto{\pgfqpoint{0.709512in}{0.701108in}}%
\pgfpathlineto{\pgfqpoint{0.711241in}{0.778473in}}%
\pgfpathlineto{\pgfqpoint{0.712105in}{0.705834in}}%
\pgfpathlineto{\pgfqpoint{0.712971in}{0.751585in}}%
\pgfpathlineto{\pgfqpoint{0.713837in}{0.688691in}}%
\pgfpathlineto{\pgfqpoint{0.714702in}{0.689609in}}%
\pgfpathlineto{\pgfqpoint{0.715567in}{0.766787in}}%
\pgfpathlineto{\pgfqpoint{0.717297in}{0.689609in}}%
\pgfpathlineto{\pgfqpoint{0.718163in}{0.747850in}}%
\pgfpathlineto{\pgfqpoint{0.719030in}{0.739681in}}%
\pgfpathlineto{\pgfqpoint{0.719895in}{0.652319in}}%
\pgfpathlineto{\pgfqpoint{0.720762in}{0.737703in}}%
\pgfpathlineto{\pgfqpoint{0.721627in}{0.672612in}}%
\pgfpathlineto{\pgfqpoint{0.723356in}{0.735287in}}%
\pgfpathlineto{\pgfqpoint{0.725084in}{0.736384in}}%
\pgfpathlineto{\pgfqpoint{0.725947in}{0.694920in}}%
\pgfpathlineto{\pgfqpoint{0.727676in}{0.751951in}}%
\pgfpathlineto{\pgfqpoint{0.728541in}{0.646309in}}%
\pgfpathlineto{\pgfqpoint{0.729407in}{0.755760in}}%
\pgfpathlineto{\pgfqpoint{0.730271in}{0.732867in}}%
\pgfpathlineto{\pgfqpoint{0.731135in}{0.762134in}}%
\pgfpathlineto{\pgfqpoint{0.732865in}{0.715030in}}%
\pgfpathlineto{\pgfqpoint{0.733731in}{0.722026in}}%
\pgfpathlineto{\pgfqpoint{0.735460in}{0.682540in}}%
\pgfpathlineto{\pgfqpoint{0.736325in}{0.782501in}}%
\pgfpathlineto{\pgfqpoint{0.737190in}{0.721036in}}%
\pgfpathlineto{\pgfqpoint{0.738920in}{0.778546in}}%
\pgfpathlineto{\pgfqpoint{0.740649in}{0.685873in}}%
\pgfpathlineto{\pgfqpoint{0.741516in}{0.685906in}}%
\pgfpathlineto{\pgfqpoint{0.742381in}{0.661365in}}%
\pgfpathlineto{\pgfqpoint{0.744112in}{0.728434in}}%
\pgfpathlineto{\pgfqpoint{0.744977in}{0.676640in}}%
\pgfpathlineto{\pgfqpoint{0.745840in}{0.733452in}}%
\pgfpathlineto{\pgfqpoint{0.746705in}{0.708985in}}%
\pgfpathlineto{\pgfqpoint{0.748435in}{0.780925in}}%
\pgfpathlineto{\pgfqpoint{0.750163in}{0.711949in}}%
\pgfpathlineto{\pgfqpoint{0.751028in}{0.710739in}}%
\pgfpathlineto{\pgfqpoint{0.751891in}{0.701656in}}%
\pgfpathlineto{\pgfqpoint{0.752757in}{0.767518in}}%
\pgfpathlineto{\pgfqpoint{0.753622in}{0.661804in}}%
\pgfpathlineto{\pgfqpoint{0.754488in}{0.702391in}}%
\pgfpathlineto{\pgfqpoint{0.755354in}{0.695577in}}%
\pgfpathlineto{\pgfqpoint{0.756219in}{0.680376in}}%
\pgfpathlineto{\pgfqpoint{0.757950in}{0.793269in}}%
\pgfpathlineto{\pgfqpoint{0.758816in}{0.693965in}}%
\pgfpathlineto{\pgfqpoint{0.759679in}{0.695870in}}%
\pgfpathlineto{\pgfqpoint{0.760542in}{0.741402in}}%
\pgfpathlineto{\pgfqpoint{0.761407in}{0.661584in}}%
\pgfpathlineto{\pgfqpoint{0.763135in}{0.730374in}}%
\pgfpathlineto{\pgfqpoint{0.764000in}{0.670744in}}%
\pgfpathlineto{\pgfqpoint{0.764865in}{0.676494in}}%
\pgfpathlineto{\pgfqpoint{0.765730in}{0.703454in}}%
\pgfpathlineto{\pgfqpoint{0.766596in}{0.664150in}}%
\pgfpathlineto{\pgfqpoint{0.768326in}{0.763892in}}%
\pgfpathlineto{\pgfqpoint{0.770057in}{0.681366in}}%
\pgfpathlineto{\pgfqpoint{0.770922in}{0.749900in}}%
\pgfpathlineto{\pgfqpoint{0.771788in}{0.703674in}}%
\pgfpathlineto{\pgfqpoint{0.772652in}{0.790597in}}%
\pgfpathlineto{\pgfqpoint{0.774383in}{0.694554in}}%
\pgfpathlineto{\pgfqpoint{0.776115in}{0.752576in}}%
\pgfpathlineto{\pgfqpoint{0.776979in}{0.711185in}}%
\pgfpathlineto{\pgfqpoint{0.777845in}{0.739279in}}%
\pgfpathlineto{\pgfqpoint{0.778710in}{0.680708in}}%
\pgfpathlineto{\pgfqpoint{0.779575in}{0.777190in}}%
\pgfpathlineto{\pgfqpoint{0.780439in}{0.774884in}}%
\pgfpathlineto{\pgfqpoint{0.781305in}{0.700856in}}%
\pgfpathlineto{\pgfqpoint{0.782170in}{0.760892in}}%
\pgfpathlineto{\pgfqpoint{0.783036in}{0.713601in}}%
\pgfpathlineto{\pgfqpoint{0.783901in}{0.722099in}}%
\pgfpathlineto{\pgfqpoint{0.785630in}{0.682576in}}%
\pgfpathlineto{\pgfqpoint{0.786494in}{0.747484in}}%
\pgfpathlineto{\pgfqpoint{0.787360in}{0.605032in}}%
\pgfpathlineto{\pgfqpoint{0.788225in}{0.615727in}}%
\pgfpathlineto{\pgfqpoint{0.789089in}{0.747338in}}%
\pgfpathlineto{\pgfqpoint{0.789954in}{0.655104in}}%
\pgfpathlineto{\pgfqpoint{0.791680in}{0.787081in}}%
\pgfpathlineto{\pgfqpoint{0.793412in}{0.719022in}}%
\pgfpathlineto{\pgfqpoint{0.794274in}{0.721255in}}%
\pgfpathlineto{\pgfqpoint{0.795138in}{0.700961in}}%
\pgfpathlineto{\pgfqpoint{0.796004in}{0.820449in}}%
\pgfpathlineto{\pgfqpoint{0.796868in}{0.703637in}}%
\pgfpathlineto{\pgfqpoint{0.797734in}{0.729497in}}%
\pgfpathlineto{\pgfqpoint{0.798599in}{0.747152in}}%
\pgfpathlineto{\pgfqpoint{0.799464in}{0.743343in}}%
\pgfpathlineto{\pgfqpoint{0.800327in}{0.693856in}}%
\pgfpathlineto{\pgfqpoint{0.801192in}{0.755321in}}%
\pgfpathlineto{\pgfqpoint{0.802924in}{0.712022in}}%
\pgfpathlineto{\pgfqpoint{0.803787in}{0.714880in}}%
\pgfpathlineto{\pgfqpoint{0.804652in}{0.653196in}}%
\pgfpathlineto{\pgfqpoint{0.805515in}{0.665028in}}%
\pgfpathlineto{\pgfqpoint{0.808107in}{0.727483in}}%
\pgfpathlineto{\pgfqpoint{0.809837in}{0.608435in}}%
\pgfpathlineto{\pgfqpoint{0.810702in}{0.681439in}}%
\pgfpathlineto{\pgfqpoint{0.811567in}{0.658803in}}%
\pgfpathlineto{\pgfqpoint{0.812431in}{0.718510in}}%
\pgfpathlineto{\pgfqpoint{0.813297in}{0.682576in}}%
\pgfpathlineto{\pgfqpoint{0.814161in}{0.740671in}}%
\pgfpathlineto{\pgfqpoint{0.815026in}{0.689682in}}%
\pgfpathlineto{\pgfqpoint{0.817620in}{0.783272in}}%
\pgfpathlineto{\pgfqpoint{0.819349in}{0.683786in}}%
\pgfpathlineto{\pgfqpoint{0.820213in}{0.773381in}}%
\pgfpathlineto{\pgfqpoint{0.821078in}{0.738913in}}%
\pgfpathlineto{\pgfqpoint{0.821942in}{0.767559in}}%
\pgfpathlineto{\pgfqpoint{0.822807in}{0.760599in}}%
\pgfpathlineto{\pgfqpoint{0.823671in}{0.723821in}}%
\pgfpathlineto{\pgfqpoint{0.824536in}{0.777263in}}%
\pgfpathlineto{\pgfqpoint{0.826268in}{0.719095in}}%
\pgfpathlineto{\pgfqpoint{0.827133in}{0.746859in}}%
\pgfpathlineto{\pgfqpoint{0.828864in}{0.724990in}}%
\pgfpathlineto{\pgfqpoint{0.830596in}{0.855140in}}%
\pgfpathlineto{\pgfqpoint{0.831462in}{0.760266in}}%
\pgfpathlineto{\pgfqpoint{0.832328in}{0.803014in}}%
\pgfpathlineto{\pgfqpoint{0.833193in}{0.746494in}}%
\pgfpathlineto{\pgfqpoint{0.834058in}{0.759316in}}%
\pgfpathlineto{\pgfqpoint{0.834922in}{0.777336in}}%
\pgfpathlineto{\pgfqpoint{0.836650in}{0.714076in}}%
\pgfpathlineto{\pgfqpoint{0.838382in}{0.684590in}}%
\pgfpathlineto{\pgfqpoint{0.840114in}{0.723529in}}%
\pgfpathlineto{\pgfqpoint{0.840980in}{0.720378in}}%
\pgfpathlineto{\pgfqpoint{0.841845in}{0.730927in}}%
\pgfpathlineto{\pgfqpoint{0.843575in}{0.677265in}}%
\pgfpathlineto{\pgfqpoint{0.845305in}{0.697705in}}%
\pgfpathlineto{\pgfqpoint{0.846169in}{0.675836in}}%
\pgfpathlineto{\pgfqpoint{0.847897in}{0.724406in}}%
\pgfpathlineto{\pgfqpoint{0.848762in}{0.724406in}}%
\pgfpathlineto{\pgfqpoint{0.849627in}{0.695139in}}%
\pgfpathlineto{\pgfqpoint{0.851354in}{0.746421in}}%
\pgfpathlineto{\pgfqpoint{0.852218in}{0.710194in}}%
\pgfpathlineto{\pgfqpoint{0.853081in}{0.734077in}}%
\pgfpathlineto{\pgfqpoint{0.853945in}{0.705359in}}%
\pgfpathlineto{\pgfqpoint{0.854810in}{0.759316in}}%
\pgfpathlineto{\pgfqpoint{0.855672in}{0.754736in}}%
\pgfpathlineto{\pgfqpoint{0.856537in}{0.740854in}}%
\pgfpathlineto{\pgfqpoint{0.857402in}{0.761549in}}%
\pgfpathlineto{\pgfqpoint{0.858267in}{0.749279in}}%
\pgfpathlineto{\pgfqpoint{0.859131in}{0.670379in}}%
\pgfpathlineto{\pgfqpoint{0.861730in}{0.738292in}}%
\pgfpathlineto{\pgfqpoint{0.862597in}{0.751147in}}%
\pgfpathlineto{\pgfqpoint{0.864329in}{0.720012in}}%
\pgfpathlineto{\pgfqpoint{0.865195in}{0.678694in}}%
\pgfpathlineto{\pgfqpoint{0.866062in}{0.750745in}}%
\pgfpathlineto{\pgfqpoint{0.866929in}{0.734849in}}%
\pgfpathlineto{\pgfqpoint{0.867794in}{0.774737in}}%
\pgfpathlineto{\pgfqpoint{0.868659in}{0.702175in}}%
\pgfpathlineto{\pgfqpoint{0.870388in}{0.747850in}}%
\pgfpathlineto{\pgfqpoint{0.871253in}{0.663200in}}%
\pgfpathlineto{\pgfqpoint{0.872119in}{0.745178in}}%
\pgfpathlineto{\pgfqpoint{0.872983in}{0.674740in}}%
\pgfpathlineto{\pgfqpoint{0.873848in}{0.766901in}}%
\pgfpathlineto{\pgfqpoint{0.874712in}{0.721368in}}%
\pgfpathlineto{\pgfqpoint{0.875577in}{0.750270in}}%
\pgfpathlineto{\pgfqpoint{0.877309in}{0.707084in}}%
\pgfpathlineto{\pgfqpoint{0.878174in}{0.699024in}}%
\pgfpathlineto{\pgfqpoint{0.879035in}{0.658255in}}%
\pgfpathlineto{\pgfqpoint{0.879903in}{0.667853in}}%
\pgfpathlineto{\pgfqpoint{0.881633in}{0.731990in}}%
\pgfpathlineto{\pgfqpoint{0.882498in}{0.705911in}}%
\pgfpathlineto{\pgfqpoint{0.883364in}{0.746680in}}%
\pgfpathlineto{\pgfqpoint{0.884229in}{0.721661in}}%
\pgfpathlineto{\pgfqpoint{0.885096in}{0.727154in}}%
\pgfpathlineto{\pgfqpoint{0.885960in}{0.755508in}}%
\pgfpathlineto{\pgfqpoint{0.886826in}{0.653675in}}%
\pgfpathlineto{\pgfqpoint{0.888555in}{0.733566in}}%
\pgfpathlineto{\pgfqpoint{0.891148in}{0.786350in}}%
\pgfpathlineto{\pgfqpoint{0.892875in}{0.743530in}}%
\pgfpathlineto{\pgfqpoint{0.893741in}{0.754850in}}%
\pgfpathlineto{\pgfqpoint{0.894605in}{0.728588in}}%
\pgfpathlineto{\pgfqpoint{0.896337in}{0.756019in}}%
\pgfpathlineto{\pgfqpoint{0.898067in}{0.704372in}}%
\pgfpathlineto{\pgfqpoint{0.901525in}{0.744041in}}%
\pgfpathlineto{\pgfqpoint{0.903254in}{0.711697in}}%
\pgfpathlineto{\pgfqpoint{0.904119in}{0.713528in}}%
\pgfpathlineto{\pgfqpoint{0.904984in}{0.687010in}}%
\pgfpathlineto{\pgfqpoint{0.906714in}{0.753238in}}%
\pgfpathlineto{\pgfqpoint{0.907579in}{0.773308in}}%
\pgfpathlineto{\pgfqpoint{0.909309in}{0.708107in}}%
\pgfpathlineto{\pgfqpoint{0.910175in}{0.764700in}}%
\pgfpathlineto{\pgfqpoint{0.911039in}{0.744114in}}%
\pgfpathlineto{\pgfqpoint{0.911905in}{0.653382in}}%
\pgfpathlineto{\pgfqpoint{0.912770in}{0.718364in}}%
\pgfpathlineto{\pgfqpoint{0.913635in}{0.653127in}}%
\pgfpathlineto{\pgfqpoint{0.914501in}{0.730049in}}%
\pgfpathlineto{\pgfqpoint{0.915365in}{0.654446in}}%
\pgfpathlineto{\pgfqpoint{0.917094in}{0.728620in}}%
\pgfpathlineto{\pgfqpoint{0.917959in}{0.706386in}}%
\pgfpathlineto{\pgfqpoint{0.918824in}{0.650232in}}%
\pgfpathlineto{\pgfqpoint{0.919687in}{0.701806in}}%
\pgfpathlineto{\pgfqpoint{0.920552in}{0.664406in}}%
\pgfpathlineto{\pgfqpoint{0.922280in}{0.757229in}}%
\pgfpathlineto{\pgfqpoint{0.924010in}{0.651295in}}%
\pgfpathlineto{\pgfqpoint{0.924875in}{0.746201in}}%
\pgfpathlineto{\pgfqpoint{0.925740in}{0.730415in}}%
\pgfpathlineto{\pgfqpoint{0.926603in}{0.781916in}}%
\pgfpathlineto{\pgfqpoint{0.927464in}{0.683161in}}%
\pgfpathlineto{\pgfqpoint{0.928329in}{0.696860in}}%
\pgfpathlineto{\pgfqpoint{0.929193in}{0.704591in}}%
\pgfpathlineto{\pgfqpoint{0.930058in}{0.660086in}}%
\pgfpathlineto{\pgfqpoint{0.931787in}{0.724665in}}%
\pgfpathlineto{\pgfqpoint{0.933516in}{0.679612in}}%
\pgfpathlineto{\pgfqpoint{0.934381in}{0.716423in}}%
\pgfpathlineto{\pgfqpoint{0.935246in}{0.702066in}}%
\pgfpathlineto{\pgfqpoint{0.936974in}{0.782907in}}%
\pgfpathlineto{\pgfqpoint{0.939570in}{0.666607in}}%
\pgfpathlineto{\pgfqpoint{0.940435in}{0.745657in}}%
\pgfpathlineto{\pgfqpoint{0.942166in}{0.707194in}}%
\pgfpathlineto{\pgfqpoint{0.943032in}{0.768988in}}%
\pgfpathlineto{\pgfqpoint{0.943897in}{0.750416in}}%
\pgfpathlineto{\pgfqpoint{0.944763in}{0.775911in}}%
\pgfpathlineto{\pgfqpoint{0.945628in}{0.719428in}}%
\pgfpathlineto{\pgfqpoint{0.946490in}{0.730488in}}%
\pgfpathlineto{\pgfqpoint{0.947355in}{0.733712in}}%
\pgfpathlineto{\pgfqpoint{0.948218in}{0.769865in}}%
\pgfpathlineto{\pgfqpoint{0.949946in}{0.658547in}}%
\pgfpathlineto{\pgfqpoint{0.950811in}{0.714117in}}%
\pgfpathlineto{\pgfqpoint{0.951674in}{0.637194in}}%
\pgfpathlineto{\pgfqpoint{0.952539in}{0.719135in}}%
\pgfpathlineto{\pgfqpoint{0.953405in}{0.686425in}}%
\pgfpathlineto{\pgfqpoint{0.954269in}{0.714738in}}%
\pgfpathlineto{\pgfqpoint{0.955133in}{0.672100in}}%
\pgfpathlineto{\pgfqpoint{0.955998in}{0.761403in}}%
\pgfpathlineto{\pgfqpoint{0.956863in}{0.704372in}}%
\pgfpathlineto{\pgfqpoint{0.957728in}{0.761111in}}%
\pgfpathlineto{\pgfqpoint{0.958594in}{0.675507in}}%
\pgfpathlineto{\pgfqpoint{0.959459in}{0.765212in}}%
\pgfpathlineto{\pgfqpoint{0.960325in}{0.701587in}}%
\pgfpathlineto{\pgfqpoint{0.961191in}{0.744187in}}%
\pgfpathlineto{\pgfqpoint{0.962055in}{0.679572in}}%
\pgfpathlineto{\pgfqpoint{0.962920in}{0.699938in}}%
\pgfpathlineto{\pgfqpoint{0.963783in}{0.766568in}}%
\pgfpathlineto{\pgfqpoint{0.964647in}{0.713345in}}%
\pgfpathlineto{\pgfqpoint{0.965512in}{0.720012in}}%
\pgfpathlineto{\pgfqpoint{0.968105in}{0.654186in}}%
\pgfpathlineto{\pgfqpoint{0.970701in}{0.725798in}}%
\pgfpathlineto{\pgfqpoint{0.972432in}{0.677558in}}%
\pgfpathlineto{\pgfqpoint{0.973297in}{0.720268in}}%
\pgfpathlineto{\pgfqpoint{0.974162in}{0.684517in}}%
\pgfpathlineto{\pgfqpoint{0.975027in}{0.759316in}}%
\pgfpathlineto{\pgfqpoint{0.975889in}{0.702943in}}%
\pgfpathlineto{\pgfqpoint{0.976751in}{0.739388in}}%
\pgfpathlineto{\pgfqpoint{0.978481in}{0.704884in}}%
\pgfpathlineto{\pgfqpoint{0.979346in}{0.715067in}}%
\pgfpathlineto{\pgfqpoint{0.981076in}{0.809242in}}%
\pgfpathlineto{\pgfqpoint{0.981941in}{0.768143in}}%
\pgfpathlineto{\pgfqpoint{0.984535in}{0.715213in}}%
\pgfpathlineto{\pgfqpoint{0.985400in}{0.741037in}}%
\pgfpathlineto{\pgfqpoint{0.986264in}{0.740891in}}%
\pgfpathlineto{\pgfqpoint{0.987991in}{0.701367in}}%
\pgfpathlineto{\pgfqpoint{0.988856in}{0.761111in}}%
\pgfpathlineto{\pgfqpoint{0.989722in}{0.733821in}}%
\pgfpathlineto{\pgfqpoint{0.990585in}{0.663127in}}%
\pgfpathlineto{\pgfqpoint{0.991449in}{0.756019in}}%
\pgfpathlineto{\pgfqpoint{0.992314in}{0.699280in}}%
\pgfpathlineto{\pgfqpoint{0.993178in}{0.737740in}}%
\pgfpathlineto{\pgfqpoint{0.994043in}{0.669023in}}%
\pgfpathlineto{\pgfqpoint{0.996636in}{0.735251in}}%
\pgfpathlineto{\pgfqpoint{0.997501in}{0.713053in}}%
\pgfpathlineto{\pgfqpoint{1.000961in}{0.750562in}}%
\pgfpathlineto{\pgfqpoint{1.001826in}{0.732210in}}%
\pgfpathlineto{\pgfqpoint{1.002691in}{0.752503in}}%
\pgfpathlineto{\pgfqpoint{1.003553in}{0.714628in}}%
\pgfpathlineto{\pgfqpoint{1.004419in}{0.783491in}}%
\pgfpathlineto{\pgfqpoint{1.005284in}{0.710121in}}%
\pgfpathlineto{\pgfqpoint{1.006149in}{0.733529in}}%
\pgfpathlineto{\pgfqpoint{1.007014in}{0.695399in}}%
\pgfpathlineto{\pgfqpoint{1.007877in}{0.762686in}}%
\pgfpathlineto{\pgfqpoint{1.008742in}{0.689576in}}%
\pgfpathlineto{\pgfqpoint{1.009607in}{0.748548in}}%
\pgfpathlineto{\pgfqpoint{1.011338in}{0.687083in}}%
\pgfpathlineto{\pgfqpoint{1.012203in}{0.694042in}}%
\pgfpathlineto{\pgfqpoint{1.013067in}{0.735580in}}%
\pgfpathlineto{\pgfqpoint{1.013933in}{0.722099in}}%
\pgfpathlineto{\pgfqpoint{1.014798in}{0.685029in}}%
\pgfpathlineto{\pgfqpoint{1.015662in}{0.746421in}}%
\pgfpathlineto{\pgfqpoint{1.016526in}{0.731146in}}%
\pgfpathlineto{\pgfqpoint{1.017391in}{0.740891in}}%
\pgfpathlineto{\pgfqpoint{1.020851in}{0.668182in}}%
\pgfpathlineto{\pgfqpoint{1.022581in}{0.687595in}}%
\pgfpathlineto{\pgfqpoint{1.023445in}{0.749133in}}%
\pgfpathlineto{\pgfqpoint{1.026037in}{0.674407in}}%
\pgfpathlineto{\pgfqpoint{1.027767in}{0.768070in}}%
\pgfpathlineto{\pgfqpoint{1.029494in}{0.734516in}}%
\pgfpathlineto{\pgfqpoint{1.030357in}{0.725798in}}%
\pgfpathlineto{\pgfqpoint{1.031221in}{0.743928in}}%
\pgfpathlineto{\pgfqpoint{1.034679in}{0.669059in}}%
\pgfpathlineto{\pgfqpoint{1.035545in}{0.706897in}}%
\pgfpathlineto{\pgfqpoint{1.036409in}{0.705395in}}%
\pgfpathlineto{\pgfqpoint{1.037275in}{0.706934in}}%
\pgfpathlineto{\pgfqpoint{1.038139in}{0.687302in}}%
\pgfpathlineto{\pgfqpoint{1.040734in}{0.746973in}}%
\pgfpathlineto{\pgfqpoint{1.041600in}{0.711916in}}%
\pgfpathlineto{\pgfqpoint{1.042466in}{0.772025in}}%
\pgfpathlineto{\pgfqpoint{1.043331in}{0.735507in}}%
\pgfpathlineto{\pgfqpoint{1.044195in}{0.803200in}}%
\pgfpathlineto{\pgfqpoint{1.045927in}{0.680270in}}%
\pgfpathlineto{\pgfqpoint{1.046792in}{0.684188in}}%
\pgfpathlineto{\pgfqpoint{1.047657in}{0.731771in}}%
\pgfpathlineto{\pgfqpoint{1.049390in}{0.692759in}}%
\pgfpathlineto{\pgfqpoint{1.050256in}{0.744187in}}%
\pgfpathlineto{\pgfqpoint{1.051987in}{0.642980in}}%
\pgfpathlineto{\pgfqpoint{1.052852in}{0.681260in}}%
\pgfpathlineto{\pgfqpoint{1.053719in}{0.660565in}}%
\pgfpathlineto{\pgfqpoint{1.056313in}{0.771294in}}%
\pgfpathlineto{\pgfqpoint{1.057179in}{0.766422in}}%
\pgfpathlineto{\pgfqpoint{1.058042in}{0.690161in}}%
\pgfpathlineto{\pgfqpoint{1.058907in}{0.750562in}}%
\pgfpathlineto{\pgfqpoint{1.059773in}{0.680781in}}%
\pgfpathlineto{\pgfqpoint{1.060637in}{0.712614in}}%
\pgfpathlineto{\pgfqpoint{1.062369in}{0.669721in}}%
\pgfpathlineto{\pgfqpoint{1.063234in}{0.676936in}}%
\pgfpathlineto{\pgfqpoint{1.064962in}{0.771587in}}%
\pgfpathlineto{\pgfqpoint{1.065827in}{0.659830in}}%
\pgfpathlineto{\pgfqpoint{1.066692in}{0.682138in}}%
\pgfpathlineto{\pgfqpoint{1.067557in}{0.745105in}}%
\pgfpathlineto{\pgfqpoint{1.068422in}{0.636240in}}%
\pgfpathlineto{\pgfqpoint{1.070151in}{0.752763in}}%
\pgfpathlineto{\pgfqpoint{1.073608in}{0.646975in}}%
\pgfpathlineto{\pgfqpoint{1.074470in}{0.784994in}}%
\pgfpathlineto{\pgfqpoint{1.075335in}{0.659867in}}%
\pgfpathlineto{\pgfqpoint{1.076199in}{0.704664in}}%
\pgfpathlineto{\pgfqpoint{1.077064in}{0.680197in}}%
\pgfpathlineto{\pgfqpoint{1.079657in}{0.732136in}}%
\pgfpathlineto{\pgfqpoint{1.081384in}{0.673018in}}%
\pgfpathlineto{\pgfqpoint{1.083981in}{0.753859in}}%
\pgfpathlineto{\pgfqpoint{1.085711in}{0.664045in}}%
\pgfpathlineto{\pgfqpoint{1.087440in}{0.773970in}}%
\pgfpathlineto{\pgfqpoint{1.088306in}{0.790857in}}%
\pgfpathlineto{\pgfqpoint{1.089172in}{0.718770in}}%
\pgfpathlineto{\pgfqpoint{1.090037in}{0.751553in}}%
\pgfpathlineto{\pgfqpoint{1.091767in}{0.683457in}}%
\pgfpathlineto{\pgfqpoint{1.095225in}{0.738990in}}%
\pgfpathlineto{\pgfqpoint{1.096956in}{0.696426in}}%
\pgfpathlineto{\pgfqpoint{1.097822in}{0.734556in}}%
\pgfpathlineto{\pgfqpoint{1.098687in}{0.723163in}}%
\pgfpathlineto{\pgfqpoint{1.100416in}{0.743384in}}%
\pgfpathlineto{\pgfqpoint{1.101281in}{0.768659in}}%
\pgfpathlineto{\pgfqpoint{1.102145in}{0.675032in}}%
\pgfpathlineto{\pgfqpoint{1.103008in}{0.687156in}}%
\pgfpathlineto{\pgfqpoint{1.104736in}{0.725656in}}%
\pgfpathlineto{\pgfqpoint{1.105601in}{0.702066in}}%
\pgfpathlineto{\pgfqpoint{1.106466in}{0.712834in}}%
\pgfpathlineto{\pgfqpoint{1.108192in}{0.692394in}}%
\pgfpathlineto{\pgfqpoint{1.109055in}{0.796387in}}%
\pgfpathlineto{\pgfqpoint{1.110782in}{0.715363in}}%
\pgfpathlineto{\pgfqpoint{1.112512in}{0.694116in}}%
\pgfpathlineto{\pgfqpoint{1.114239in}{0.690599in}}%
\pgfpathlineto{\pgfqpoint{1.115102in}{0.721368in}}%
\pgfpathlineto{\pgfqpoint{1.115967in}{0.717008in}}%
\pgfpathlineto{\pgfqpoint{1.116831in}{0.740964in}}%
\pgfpathlineto{\pgfqpoint{1.117696in}{0.729497in}}%
\pgfpathlineto{\pgfqpoint{1.118561in}{0.642980in}}%
\pgfpathlineto{\pgfqpoint{1.119426in}{0.672685in}}%
\pgfpathlineto{\pgfqpoint{1.120289in}{0.645067in}}%
\pgfpathlineto{\pgfqpoint{1.122016in}{0.727337in}}%
\pgfpathlineto{\pgfqpoint{1.122881in}{0.703454in}}%
\pgfpathlineto{\pgfqpoint{1.124612in}{0.771806in}}%
\pgfpathlineto{\pgfqpoint{1.125476in}{0.692979in}}%
\pgfpathlineto{\pgfqpoint{1.126342in}{0.761330in}}%
\pgfpathlineto{\pgfqpoint{1.128072in}{0.700198in}}%
\pgfpathlineto{\pgfqpoint{1.128936in}{0.725396in}}%
\pgfpathlineto{\pgfqpoint{1.130665in}{0.657231in}}%
\pgfpathlineto{\pgfqpoint{1.132396in}{0.717816in}}%
\pgfpathlineto{\pgfqpoint{1.133261in}{0.681187in}}%
\pgfpathlineto{\pgfqpoint{1.134991in}{0.716058in}}%
\pgfpathlineto{\pgfqpoint{1.135856in}{0.729976in}}%
\pgfpathlineto{\pgfqpoint{1.136722in}{0.672466in}}%
\pgfpathlineto{\pgfqpoint{1.137586in}{0.683640in}}%
\pgfpathlineto{\pgfqpoint{1.140181in}{0.745544in}}%
\pgfpathlineto{\pgfqpoint{1.141044in}{0.741296in}}%
\pgfpathlineto{\pgfqpoint{1.141910in}{0.729794in}}%
\pgfpathlineto{\pgfqpoint{1.143641in}{0.656606in}}%
\pgfpathlineto{\pgfqpoint{1.144506in}{0.711112in}}%
\pgfpathlineto{\pgfqpoint{1.145371in}{0.676607in}}%
\pgfpathlineto{\pgfqpoint{1.146235in}{0.767120in}}%
\pgfpathlineto{\pgfqpoint{1.147100in}{0.765983in}}%
\pgfpathlineto{\pgfqpoint{1.148829in}{0.697778in}}%
\pgfpathlineto{\pgfqpoint{1.149692in}{0.749133in}}%
\pgfpathlineto{\pgfqpoint{1.151424in}{0.709135in}}%
\pgfpathlineto{\pgfqpoint{1.153155in}{0.748694in}}%
\pgfpathlineto{\pgfqpoint{1.154020in}{0.690892in}}%
\pgfpathlineto{\pgfqpoint{1.154885in}{0.743895in}}%
\pgfpathlineto{\pgfqpoint{1.155749in}{0.740817in}}%
\pgfpathlineto{\pgfqpoint{1.156614in}{0.771806in}}%
\pgfpathlineto{\pgfqpoint{1.159208in}{0.574742in}}%
\pgfpathlineto{\pgfqpoint{1.160937in}{0.763567in}}%
\pgfpathlineto{\pgfqpoint{1.162667in}{0.719135in}}%
\pgfpathlineto{\pgfqpoint{1.163531in}{0.779098in}}%
\pgfpathlineto{\pgfqpoint{1.165259in}{0.705801in}}%
\pgfpathlineto{\pgfqpoint{1.166990in}{0.769426in}}%
\pgfpathlineto{\pgfqpoint{1.167854in}{0.744849in}}%
\pgfpathlineto{\pgfqpoint{1.168720in}{0.741808in}}%
\pgfpathlineto{\pgfqpoint{1.169584in}{0.682430in}}%
\pgfpathlineto{\pgfqpoint{1.171315in}{0.720784in}}%
\pgfpathlineto{\pgfqpoint{1.172180in}{0.700417in}}%
\pgfpathlineto{\pgfqpoint{1.173912in}{0.728401in}}%
\pgfpathlineto{\pgfqpoint{1.174778in}{0.726972in}}%
\pgfpathlineto{\pgfqpoint{1.176508in}{0.668657in}}%
\pgfpathlineto{\pgfqpoint{1.178239in}{0.722026in}}%
\pgfpathlineto{\pgfqpoint{1.179103in}{0.707742in}}%
\pgfpathlineto{\pgfqpoint{1.181699in}{0.761330in}}%
\pgfpathlineto{\pgfqpoint{1.182565in}{0.723894in}}%
\pgfpathlineto{\pgfqpoint{1.183430in}{0.790378in}}%
\pgfpathlineto{\pgfqpoint{1.185160in}{0.677704in}}%
\pgfpathlineto{\pgfqpoint{1.186891in}{0.733968in}}%
\pgfpathlineto{\pgfqpoint{1.188623in}{0.779058in}}%
\pgfpathlineto{\pgfqpoint{1.189487in}{0.690197in}}%
\pgfpathlineto{\pgfqpoint{1.190352in}{0.744041in}}%
\pgfpathlineto{\pgfqpoint{1.191218in}{0.659465in}}%
\pgfpathlineto{\pgfqpoint{1.192083in}{0.837668in}}%
\pgfpathlineto{\pgfqpoint{1.192948in}{0.716277in}}%
\pgfpathlineto{\pgfqpoint{1.193813in}{0.736164in}}%
\pgfpathlineto{\pgfqpoint{1.194678in}{0.747484in}}%
\pgfpathlineto{\pgfqpoint{1.195541in}{0.708692in}}%
\pgfpathlineto{\pgfqpoint{1.196406in}{0.730378in}}%
\pgfpathlineto{\pgfqpoint{1.198137in}{0.703089in}}%
\pgfpathlineto{\pgfqpoint{1.199001in}{0.763088in}}%
\pgfpathlineto{\pgfqpoint{1.199865in}{0.700304in}}%
\pgfpathlineto{\pgfqpoint{1.200729in}{0.746348in}}%
\pgfpathlineto{\pgfqpoint{1.201595in}{0.691586in}}%
\pgfpathlineto{\pgfqpoint{1.202459in}{0.747923in}}%
\pgfpathlineto{\pgfqpoint{1.203324in}{0.673489in}}%
\pgfpathlineto{\pgfqpoint{1.204189in}{0.702683in}}%
\pgfpathlineto{\pgfqpoint{1.205054in}{0.696089in}}%
\pgfpathlineto{\pgfqpoint{1.205919in}{0.693710in}}%
\pgfpathlineto{\pgfqpoint{1.207651in}{0.749718in}}%
\pgfpathlineto{\pgfqpoint{1.208516in}{0.715140in}}%
\pgfpathlineto{\pgfqpoint{1.209379in}{0.720232in}}%
\pgfpathlineto{\pgfqpoint{1.211111in}{0.680891in}}%
\pgfpathlineto{\pgfqpoint{1.211977in}{0.735141in}}%
\pgfpathlineto{\pgfqpoint{1.212842in}{0.730049in}}%
\pgfpathlineto{\pgfqpoint{1.213706in}{0.732136in}}%
\pgfpathlineto{\pgfqpoint{1.214572in}{0.690599in}}%
\pgfpathlineto{\pgfqpoint{1.217169in}{0.817558in}}%
\pgfpathlineto{\pgfqpoint{1.218035in}{0.682174in}}%
\pgfpathlineto{\pgfqpoint{1.218902in}{0.796826in}}%
\pgfpathlineto{\pgfqpoint{1.220634in}{0.682942in}}%
\pgfpathlineto{\pgfqpoint{1.222366in}{0.742247in}}%
\pgfpathlineto{\pgfqpoint{1.223232in}{0.735507in}}%
\pgfpathlineto{\pgfqpoint{1.224098in}{0.737520in}}%
\pgfpathlineto{\pgfqpoint{1.224965in}{0.768399in}}%
\pgfpathlineto{\pgfqpoint{1.225831in}{0.674553in}}%
\pgfpathlineto{\pgfqpoint{1.228427in}{0.727045in}}%
\pgfpathlineto{\pgfqpoint{1.230156in}{0.686165in}}%
\pgfpathlineto{\pgfqpoint{1.231020in}{0.728913in}}%
\pgfpathlineto{\pgfqpoint{1.232751in}{0.696568in}}%
\pgfpathlineto{\pgfqpoint{1.233617in}{0.701952in}}%
\pgfpathlineto{\pgfqpoint{1.234482in}{0.668913in}}%
\pgfpathlineto{\pgfqpoint{1.236211in}{0.728182in}}%
\pgfpathlineto{\pgfqpoint{1.237076in}{0.737817in}}%
\pgfpathlineto{\pgfqpoint{1.237942in}{0.678548in}}%
\pgfpathlineto{\pgfqpoint{1.238807in}{0.765910in}}%
\pgfpathlineto{\pgfqpoint{1.239671in}{0.757339in}}%
\pgfpathlineto{\pgfqpoint{1.240537in}{0.761257in}}%
\pgfpathlineto{\pgfqpoint{1.241402in}{0.677558in}}%
\pgfpathlineto{\pgfqpoint{1.242268in}{0.686019in}}%
\pgfpathlineto{\pgfqpoint{1.243999in}{0.733895in}}%
\pgfpathlineto{\pgfqpoint{1.244864in}{0.687668in}}%
\pgfpathlineto{\pgfqpoint{1.245728in}{0.748069in}}%
\pgfpathlineto{\pgfqpoint{1.246593in}{0.706678in}}%
\pgfpathlineto{\pgfqpoint{1.247458in}{0.716788in}}%
\pgfpathlineto{\pgfqpoint{1.249187in}{0.708660in}}%
\pgfpathlineto{\pgfqpoint{1.250053in}{0.695179in}}%
\pgfpathlineto{\pgfqpoint{1.250916in}{0.756864in}}%
\pgfpathlineto{\pgfqpoint{1.251781in}{0.736643in}}%
\pgfpathlineto{\pgfqpoint{1.252647in}{0.747484in}}%
\pgfpathlineto{\pgfqpoint{1.253512in}{0.704445in}}%
\pgfpathlineto{\pgfqpoint{1.254377in}{0.776971in}}%
\pgfpathlineto{\pgfqpoint{1.255243in}{0.721807in}}%
\pgfpathlineto{\pgfqpoint{1.256109in}{0.762906in}}%
\pgfpathlineto{\pgfqpoint{1.256974in}{0.742100in}}%
\pgfpathlineto{\pgfqpoint{1.257840in}{0.781112in}}%
\pgfpathlineto{\pgfqpoint{1.258706in}{0.671370in}}%
\pgfpathlineto{\pgfqpoint{1.259571in}{0.697924in}}%
\pgfpathlineto{\pgfqpoint{1.260436in}{0.721734in}}%
\pgfpathlineto{\pgfqpoint{1.261299in}{0.698988in}}%
\pgfpathlineto{\pgfqpoint{1.262165in}{0.755946in}}%
\pgfpathlineto{\pgfqpoint{1.263030in}{0.675032in}}%
\pgfpathlineto{\pgfqpoint{1.264759in}{0.711404in}}%
\pgfpathlineto{\pgfqpoint{1.265625in}{0.719464in}}%
\pgfpathlineto{\pgfqpoint{1.266490in}{0.798328in}}%
\pgfpathlineto{\pgfqpoint{1.269081in}{0.714336in}}%
\pgfpathlineto{\pgfqpoint{1.269946in}{0.713970in}}%
\pgfpathlineto{\pgfqpoint{1.270811in}{0.757229in}}%
\pgfpathlineto{\pgfqpoint{1.271677in}{0.747558in}}%
\pgfpathlineto{\pgfqpoint{1.272543in}{0.735507in}}%
\pgfpathlineto{\pgfqpoint{1.274272in}{0.647852in}}%
\pgfpathlineto{\pgfqpoint{1.276003in}{0.772244in}}%
\pgfpathlineto{\pgfqpoint{1.276869in}{0.718510in}}%
\pgfpathlineto{\pgfqpoint{1.277734in}{0.799757in}}%
\pgfpathlineto{\pgfqpoint{1.279465in}{0.745471in}}%
\pgfpathlineto{\pgfqpoint{1.281196in}{0.705874in}}%
\pgfpathlineto{\pgfqpoint{1.282062in}{0.709244in}}%
\pgfpathlineto{\pgfqpoint{1.282927in}{0.805433in}}%
\pgfpathlineto{\pgfqpoint{1.283793in}{0.702797in}}%
\pgfpathlineto{\pgfqpoint{1.284658in}{0.785802in}}%
\pgfpathlineto{\pgfqpoint{1.285521in}{0.688732in}}%
\pgfpathlineto{\pgfqpoint{1.286386in}{0.704810in}}%
\pgfpathlineto{\pgfqpoint{1.287250in}{0.708254in}}%
\pgfpathlineto{\pgfqpoint{1.288114in}{0.726606in}}%
\pgfpathlineto{\pgfqpoint{1.288980in}{0.771002in}}%
\pgfpathlineto{\pgfqpoint{1.289846in}{0.705874in}}%
\pgfpathlineto{\pgfqpoint{1.290711in}{0.804223in}}%
\pgfpathlineto{\pgfqpoint{1.291576in}{0.756202in}}%
\pgfpathlineto{\pgfqpoint{1.292440in}{0.800342in}}%
\pgfpathlineto{\pgfqpoint{1.293306in}{0.799059in}}%
\pgfpathlineto{\pgfqpoint{1.295037in}{0.714628in}}%
\pgfpathlineto{\pgfqpoint{1.295904in}{0.815836in}}%
\pgfpathlineto{\pgfqpoint{1.297633in}{0.645652in}}%
\pgfpathlineto{\pgfqpoint{1.298498in}{0.659611in}}%
\pgfpathlineto{\pgfqpoint{1.299363in}{0.623092in}}%
\pgfpathlineto{\pgfqpoint{1.300229in}{0.707815in}}%
\pgfpathlineto{\pgfqpoint{1.301092in}{0.675324in}}%
\pgfpathlineto{\pgfqpoint{1.301958in}{0.697413in}}%
\pgfpathlineto{\pgfqpoint{1.302823in}{0.697120in}}%
\pgfpathlineto{\pgfqpoint{1.304552in}{0.714263in}}%
\pgfpathlineto{\pgfqpoint{1.305417in}{0.762979in}}%
\pgfpathlineto{\pgfqpoint{1.307147in}{0.655177in}}%
\pgfpathlineto{\pgfqpoint{1.308013in}{0.760892in}}%
\pgfpathlineto{\pgfqpoint{1.308879in}{0.757302in}}%
\pgfpathlineto{\pgfqpoint{1.309745in}{0.756019in}}%
\pgfpathlineto{\pgfqpoint{1.310610in}{0.707121in}}%
\pgfpathlineto{\pgfqpoint{1.311476in}{0.765764in}}%
\pgfpathlineto{\pgfqpoint{1.312342in}{0.714409in}}%
\pgfpathlineto{\pgfqpoint{1.313208in}{0.784153in}}%
\pgfpathlineto{\pgfqpoint{1.314074in}{0.733566in}}%
\pgfpathlineto{\pgfqpoint{1.315803in}{0.792834in}}%
\pgfpathlineto{\pgfqpoint{1.318395in}{0.699134in}}%
\pgfpathlineto{\pgfqpoint{1.320124in}{0.739429in}}%
\pgfpathlineto{\pgfqpoint{1.321854in}{0.700636in}}%
\pgfpathlineto{\pgfqpoint{1.322720in}{0.723163in}}%
\pgfpathlineto{\pgfqpoint{1.323583in}{0.709025in}}%
\pgfpathlineto{\pgfqpoint{1.324448in}{0.724592in}}%
\pgfpathlineto{\pgfqpoint{1.326179in}{0.681918in}}%
\pgfpathlineto{\pgfqpoint{1.329636in}{0.757704in}}%
\pgfpathlineto{\pgfqpoint{1.330500in}{0.768842in}}%
\pgfpathlineto{\pgfqpoint{1.332230in}{0.725583in}}%
\pgfpathlineto{\pgfqpoint{1.333092in}{0.738146in}}%
\pgfpathlineto{\pgfqpoint{1.333954in}{0.735214in}}%
\pgfpathlineto{\pgfqpoint{1.334819in}{0.674995in}}%
\pgfpathlineto{\pgfqpoint{1.336550in}{0.743457in}}%
\pgfpathlineto{\pgfqpoint{1.337413in}{0.754923in}}%
\pgfpathlineto{\pgfqpoint{1.338278in}{0.709317in}}%
\pgfpathlineto{\pgfqpoint{1.340007in}{0.745251in}}%
\pgfpathlineto{\pgfqpoint{1.340869in}{0.709390in}}%
\pgfpathlineto{\pgfqpoint{1.341734in}{0.729684in}}%
\pgfpathlineto{\pgfqpoint{1.342599in}{0.692394in}}%
\pgfpathlineto{\pgfqpoint{1.343462in}{0.752868in}}%
\pgfpathlineto{\pgfqpoint{1.344329in}{0.670415in}}%
\pgfpathlineto{\pgfqpoint{1.346061in}{0.761549in}}%
\pgfpathlineto{\pgfqpoint{1.347786in}{0.699792in}}%
\pgfpathlineto{\pgfqpoint{1.348652in}{0.755654in}}%
\pgfpathlineto{\pgfqpoint{1.349515in}{0.709902in}}%
\pgfpathlineto{\pgfqpoint{1.350380in}{0.747484in}}%
\pgfpathlineto{\pgfqpoint{1.352107in}{0.668365in}}%
\pgfpathlineto{\pgfqpoint{1.352972in}{0.706240in}}%
\pgfpathlineto{\pgfqpoint{1.353837in}{0.646752in}}%
\pgfpathlineto{\pgfqpoint{1.354702in}{0.781478in}}%
\pgfpathlineto{\pgfqpoint{1.356431in}{0.717227in}}%
\pgfpathlineto{\pgfqpoint{1.357296in}{0.790524in}}%
\pgfpathlineto{\pgfqpoint{1.359028in}{0.696495in}}%
\pgfpathlineto{\pgfqpoint{1.359893in}{0.784003in}}%
\pgfpathlineto{\pgfqpoint{1.360758in}{0.655141in}}%
\pgfpathlineto{\pgfqpoint{1.362488in}{0.752211in}}%
\pgfpathlineto{\pgfqpoint{1.363353in}{0.750672in}}%
\pgfpathlineto{\pgfqpoint{1.364217in}{0.719354in}}%
\pgfpathlineto{\pgfqpoint{1.365081in}{0.794885in}}%
\pgfpathlineto{\pgfqpoint{1.365947in}{0.732762in}}%
\pgfpathlineto{\pgfqpoint{1.366813in}{0.761955in}}%
\pgfpathlineto{\pgfqpoint{1.368543in}{0.666936in}}%
\pgfpathlineto{\pgfqpoint{1.369408in}{0.677964in}}%
\pgfpathlineto{\pgfqpoint{1.370273in}{0.704810in}}%
\pgfpathlineto{\pgfqpoint{1.371138in}{0.683859in}}%
\pgfpathlineto{\pgfqpoint{1.372002in}{0.747996in}}%
\pgfpathlineto{\pgfqpoint{1.372866in}{0.696458in}}%
\pgfpathlineto{\pgfqpoint{1.373729in}{0.736424in}}%
\pgfpathlineto{\pgfqpoint{1.374594in}{0.731698in}}%
\pgfpathlineto{\pgfqpoint{1.375459in}{0.692248in}}%
\pgfpathlineto{\pgfqpoint{1.376322in}{0.697924in}}%
\pgfpathlineto{\pgfqpoint{1.377187in}{0.726570in}}%
\pgfpathlineto{\pgfqpoint{1.378916in}{0.625837in}}%
\pgfpathlineto{\pgfqpoint{1.379781in}{0.713089in}}%
\pgfpathlineto{\pgfqpoint{1.380646in}{0.681074in}}%
\pgfpathlineto{\pgfqpoint{1.381509in}{0.740817in}}%
\pgfpathlineto{\pgfqpoint{1.382373in}{0.684152in}}%
\pgfpathlineto{\pgfqpoint{1.383238in}{0.710966in}}%
\pgfpathlineto{\pgfqpoint{1.384968in}{0.682211in}}%
\pgfpathlineto{\pgfqpoint{1.385831in}{0.717008in}}%
\pgfpathlineto{\pgfqpoint{1.386696in}{0.699244in}}%
\pgfpathlineto{\pgfqpoint{1.387562in}{0.786569in}}%
\pgfpathlineto{\pgfqpoint{1.389291in}{0.703089in}}%
\pgfpathlineto{\pgfqpoint{1.390156in}{0.704884in}}%
\pgfpathlineto{\pgfqpoint{1.391021in}{0.743237in}}%
\pgfpathlineto{\pgfqpoint{1.391886in}{0.715213in}}%
\pgfpathlineto{\pgfqpoint{1.393613in}{0.731990in}}%
\pgfpathlineto{\pgfqpoint{1.395342in}{0.699426in}}%
\pgfpathlineto{\pgfqpoint{1.396208in}{0.669794in}}%
\pgfpathlineto{\pgfqpoint{1.397073in}{0.716131in}}%
\pgfpathlineto{\pgfqpoint{1.397938in}{0.664922in}}%
\pgfpathlineto{\pgfqpoint{1.399668in}{0.738438in}}%
\pgfpathlineto{\pgfqpoint{1.400534in}{0.716058in}}%
\pgfpathlineto{\pgfqpoint{1.401400in}{0.721478in}}%
\pgfpathlineto{\pgfqpoint{1.402265in}{0.776605in}}%
\pgfpathlineto{\pgfqpoint{1.403131in}{0.747119in}}%
\pgfpathlineto{\pgfqpoint{1.403997in}{0.760015in}}%
\pgfpathlineto{\pgfqpoint{1.404863in}{0.751845in}}%
\pgfpathlineto{\pgfqpoint{1.406594in}{0.716788in}}%
\pgfpathlineto{\pgfqpoint{1.408324in}{0.750635in}}%
\pgfpathlineto{\pgfqpoint{1.409190in}{0.735580in}}%
\pgfpathlineto{\pgfqpoint{1.410055in}{0.800196in}}%
\pgfpathlineto{\pgfqpoint{1.411785in}{0.757375in}}%
\pgfpathlineto{\pgfqpoint{1.412651in}{0.760965in}}%
\pgfpathlineto{\pgfqpoint{1.413515in}{0.709390in}}%
\pgfpathlineto{\pgfqpoint{1.414379in}{0.782322in}}%
\pgfpathlineto{\pgfqpoint{1.415244in}{0.766312in}}%
\pgfpathlineto{\pgfqpoint{1.416108in}{0.701075in}}%
\pgfpathlineto{\pgfqpoint{1.416972in}{0.785798in}}%
\pgfpathlineto{\pgfqpoint{1.418701in}{0.702723in}}%
\pgfpathlineto{\pgfqpoint{1.419566in}{0.747558in}}%
\pgfpathlineto{\pgfqpoint{1.420430in}{0.717596in}}%
\pgfpathlineto{\pgfqpoint{1.421294in}{0.727743in}}%
\pgfpathlineto{\pgfqpoint{1.422159in}{0.709025in}}%
\pgfpathlineto{\pgfqpoint{1.423024in}{0.618732in}}%
\pgfpathlineto{\pgfqpoint{1.423884in}{0.720638in}}%
\pgfpathlineto{\pgfqpoint{1.424749in}{0.649610in}}%
\pgfpathlineto{\pgfqpoint{1.426480in}{0.770417in}}%
\pgfpathlineto{\pgfqpoint{1.427345in}{0.746790in}}%
\pgfpathlineto{\pgfqpoint{1.428206in}{0.736936in}}%
\pgfpathlineto{\pgfqpoint{1.429937in}{0.632102in}}%
\pgfpathlineto{\pgfqpoint{1.431667in}{0.753932in}}%
\pgfpathlineto{\pgfqpoint{1.434265in}{0.685288in}}%
\pgfpathlineto{\pgfqpoint{1.435996in}{0.743164in}}%
\pgfpathlineto{\pgfqpoint{1.436861in}{0.699317in}}%
\pgfpathlineto{\pgfqpoint{1.437726in}{0.711039in}}%
\pgfpathlineto{\pgfqpoint{1.438592in}{0.734703in}}%
\pgfpathlineto{\pgfqpoint{1.439455in}{0.680124in}}%
\pgfpathlineto{\pgfqpoint{1.440320in}{0.717706in}}%
\pgfpathlineto{\pgfqpoint{1.441185in}{0.698988in}}%
\pgfpathlineto{\pgfqpoint{1.442050in}{0.707815in}}%
\pgfpathlineto{\pgfqpoint{1.443778in}{0.689353in}}%
\pgfpathlineto{\pgfqpoint{1.447237in}{0.758472in}}%
\pgfpathlineto{\pgfqpoint{1.448103in}{0.729351in}}%
\pgfpathlineto{\pgfqpoint{1.448967in}{0.658949in}}%
\pgfpathlineto{\pgfqpoint{1.449831in}{0.719095in}}%
\pgfpathlineto{\pgfqpoint{1.451559in}{0.667155in}}%
\pgfpathlineto{\pgfqpoint{1.452425in}{0.738398in}}%
\pgfpathlineto{\pgfqpoint{1.453289in}{0.696129in}}%
\pgfpathlineto{\pgfqpoint{1.454154in}{0.802356in}}%
\pgfpathlineto{\pgfqpoint{1.455019in}{0.748142in}}%
\pgfpathlineto{\pgfqpoint{1.455884in}{0.775797in}}%
\pgfpathlineto{\pgfqpoint{1.457614in}{0.724698in}}%
\pgfpathlineto{\pgfqpoint{1.458480in}{0.639021in}}%
\pgfpathlineto{\pgfqpoint{1.459345in}{0.733160in}}%
\pgfpathlineto{\pgfqpoint{1.460209in}{0.719095in}}%
\pgfpathlineto{\pgfqpoint{1.461941in}{0.665433in}}%
\pgfpathlineto{\pgfqpoint{1.462806in}{0.686092in}}%
\pgfpathlineto{\pgfqpoint{1.463672in}{0.749279in}}%
\pgfpathlineto{\pgfqpoint{1.464538in}{0.745138in}}%
\pgfpathlineto{\pgfqpoint{1.466268in}{0.663639in}}%
\pgfpathlineto{\pgfqpoint{1.467999in}{0.720049in}}%
\pgfpathlineto{\pgfqpoint{1.468864in}{0.744553in}}%
\pgfpathlineto{\pgfqpoint{1.470594in}{0.690193in}}%
\pgfpathlineto{\pgfqpoint{1.471459in}{0.759349in}}%
\pgfpathlineto{\pgfqpoint{1.472326in}{0.608544in}}%
\pgfpathlineto{\pgfqpoint{1.473192in}{0.706971in}}%
\pgfpathlineto{\pgfqpoint{1.474056in}{0.696860in}}%
\pgfpathlineto{\pgfqpoint{1.474922in}{0.674699in}}%
\pgfpathlineto{\pgfqpoint{1.475788in}{0.686312in}}%
\pgfpathlineto{\pgfqpoint{1.478386in}{0.755873in}}%
\pgfpathlineto{\pgfqpoint{1.480117in}{0.723675in}}%
\pgfpathlineto{\pgfqpoint{1.481847in}{0.744955in}}%
\pgfpathlineto{\pgfqpoint{1.483575in}{0.674261in}}%
\pgfpathlineto{\pgfqpoint{1.484441in}{0.652172in}}%
\pgfpathlineto{\pgfqpoint{1.485307in}{0.731438in}}%
\pgfpathlineto{\pgfqpoint{1.486174in}{0.719387in}}%
\pgfpathlineto{\pgfqpoint{1.487040in}{0.634847in}}%
\pgfpathlineto{\pgfqpoint{1.489635in}{0.749060in}}%
\pgfpathlineto{\pgfqpoint{1.492231in}{0.682795in}}%
\pgfpathlineto{\pgfqpoint{1.493097in}{0.779902in}}%
\pgfpathlineto{\pgfqpoint{1.494827in}{0.691769in}}%
\pgfpathlineto{\pgfqpoint{1.497424in}{0.778546in}}%
\pgfpathlineto{\pgfqpoint{1.499153in}{0.691696in}}%
\pgfpathlineto{\pgfqpoint{1.500017in}{0.726493in}}%
\pgfpathlineto{\pgfqpoint{1.500882in}{0.724808in}}%
\pgfpathlineto{\pgfqpoint{1.501745in}{0.636678in}}%
\pgfpathlineto{\pgfqpoint{1.504339in}{0.758691in}}%
\pgfpathlineto{\pgfqpoint{1.506068in}{0.727958in}}%
\pgfpathlineto{\pgfqpoint{1.506933in}{0.764002in}}%
\pgfpathlineto{\pgfqpoint{1.507798in}{0.699240in}}%
\pgfpathlineto{\pgfqpoint{1.508662in}{0.782355in}}%
\pgfpathlineto{\pgfqpoint{1.509524in}{0.764440in}}%
\pgfpathlineto{\pgfqpoint{1.510389in}{0.697482in}}%
\pgfpathlineto{\pgfqpoint{1.511254in}{0.786675in}}%
\pgfpathlineto{\pgfqpoint{1.512984in}{0.732096in}}%
\pgfpathlineto{\pgfqpoint{1.513848in}{0.779789in}}%
\pgfpathlineto{\pgfqpoint{1.514713in}{0.653342in}}%
\pgfpathlineto{\pgfqpoint{1.515579in}{0.688508in}}%
\pgfpathlineto{\pgfqpoint{1.516443in}{0.733160in}}%
\pgfpathlineto{\pgfqpoint{1.517308in}{0.701513in}}%
\pgfpathlineto{\pgfqpoint{1.518172in}{0.706532in}}%
\pgfpathlineto{\pgfqpoint{1.519904in}{0.720853in}}%
\pgfpathlineto{\pgfqpoint{1.520767in}{0.726387in}}%
\pgfpathlineto{\pgfqpoint{1.522496in}{0.689609in}}%
\pgfpathlineto{\pgfqpoint{1.524226in}{0.764554in}}%
\pgfpathlineto{\pgfqpoint{1.525091in}{0.718839in}}%
\pgfpathlineto{\pgfqpoint{1.525956in}{0.758472in}}%
\pgfpathlineto{\pgfqpoint{1.527685in}{0.715213in}}%
\pgfpathlineto{\pgfqpoint{1.529411in}{0.771038in}}%
\pgfpathlineto{\pgfqpoint{1.531141in}{0.738803in}}%
\pgfpathlineto{\pgfqpoint{1.532005in}{0.744005in}}%
\pgfpathlineto{\pgfqpoint{1.532870in}{0.778034in}}%
\pgfpathlineto{\pgfqpoint{1.533734in}{0.770783in}}%
\pgfpathlineto{\pgfqpoint{1.534599in}{0.715176in}}%
\pgfpathlineto{\pgfqpoint{1.535463in}{0.724592in}}%
\pgfpathlineto{\pgfqpoint{1.537190in}{0.696056in}}%
\pgfpathlineto{\pgfqpoint{1.538055in}{0.649866in}}%
\pgfpathlineto{\pgfqpoint{1.539786in}{0.731990in}}%
\pgfpathlineto{\pgfqpoint{1.540652in}{0.695691in}}%
\pgfpathlineto{\pgfqpoint{1.541516in}{0.748621in}}%
\pgfpathlineto{\pgfqpoint{1.543246in}{0.706313in}}%
\pgfpathlineto{\pgfqpoint{1.544975in}{0.729538in}}%
\pgfpathlineto{\pgfqpoint{1.545841in}{0.719866in}}%
\pgfpathlineto{\pgfqpoint{1.546705in}{0.685288in}}%
\pgfpathlineto{\pgfqpoint{1.547570in}{0.710893in}}%
\pgfpathlineto{\pgfqpoint{1.548435in}{0.671735in}}%
\pgfpathlineto{\pgfqpoint{1.549300in}{0.751001in}}%
\pgfpathlineto{\pgfqpoint{1.550163in}{0.738803in}}%
\pgfpathlineto{\pgfqpoint{1.551028in}{0.725104in}}%
\pgfpathlineto{\pgfqpoint{1.551893in}{0.729205in}}%
\pgfpathlineto{\pgfqpoint{1.552758in}{0.682576in}}%
\pgfpathlineto{\pgfqpoint{1.553624in}{0.693088in}}%
\pgfpathlineto{\pgfqpoint{1.554488in}{0.768801in}}%
\pgfpathlineto{\pgfqpoint{1.555353in}{0.730269in}}%
\pgfpathlineto{\pgfqpoint{1.556218in}{0.745251in}}%
\pgfpathlineto{\pgfqpoint{1.557083in}{0.677119in}}%
\pgfpathlineto{\pgfqpoint{1.557945in}{0.706313in}}%
\pgfpathlineto{\pgfqpoint{1.558810in}{0.692394in}}%
\pgfpathlineto{\pgfqpoint{1.559676in}{0.745324in}}%
\pgfpathlineto{\pgfqpoint{1.561408in}{0.660123in}}%
\pgfpathlineto{\pgfqpoint{1.562273in}{0.776459in}}%
\pgfpathlineto{\pgfqpoint{1.563139in}{0.704262in}}%
\pgfpathlineto{\pgfqpoint{1.564005in}{0.744959in}}%
\pgfpathlineto{\pgfqpoint{1.564872in}{0.654739in}}%
\pgfpathlineto{\pgfqpoint{1.565738in}{0.687302in}}%
\pgfpathlineto{\pgfqpoint{1.567467in}{0.605069in}}%
\pgfpathlineto{\pgfqpoint{1.569196in}{0.751260in}}%
\pgfpathlineto{\pgfqpoint{1.570061in}{0.715582in}}%
\pgfpathlineto{\pgfqpoint{1.570924in}{0.717121in}}%
\pgfpathlineto{\pgfqpoint{1.571789in}{0.723163in}}%
\pgfpathlineto{\pgfqpoint{1.573519in}{0.692175in}}%
\pgfpathlineto{\pgfqpoint{1.574383in}{0.751001in}}%
\pgfpathlineto{\pgfqpoint{1.576114in}{0.698363in}}%
\pgfpathlineto{\pgfqpoint{1.576978in}{0.755873in}}%
\pgfpathlineto{\pgfqpoint{1.577842in}{0.649720in}}%
\pgfpathlineto{\pgfqpoint{1.578706in}{0.665214in}}%
\pgfpathlineto{\pgfqpoint{1.580437in}{0.688585in}}%
\pgfpathlineto{\pgfqpoint{1.581303in}{0.742100in}}%
\pgfpathlineto{\pgfqpoint{1.583036in}{0.642980in}}%
\pgfpathlineto{\pgfqpoint{1.583903in}{0.704591in}}%
\pgfpathlineto{\pgfqpoint{1.584769in}{0.624042in}}%
\pgfpathlineto{\pgfqpoint{1.585634in}{0.732648in}}%
\pgfpathlineto{\pgfqpoint{1.587366in}{0.672247in}}%
\pgfpathlineto{\pgfqpoint{1.588232in}{0.772943in}}%
\pgfpathlineto{\pgfqpoint{1.589962in}{0.721222in}}%
\pgfpathlineto{\pgfqpoint{1.590828in}{0.732356in}}%
\pgfpathlineto{\pgfqpoint{1.591692in}{0.720378in}}%
\pgfpathlineto{\pgfqpoint{1.592558in}{0.663346in}}%
\pgfpathlineto{\pgfqpoint{1.595155in}{0.723163in}}%
\pgfpathlineto{\pgfqpoint{1.596021in}{0.694335in}}%
\pgfpathlineto{\pgfqpoint{1.596886in}{0.612357in}}%
\pgfpathlineto{\pgfqpoint{1.597751in}{0.751845in}}%
\pgfpathlineto{\pgfqpoint{1.598615in}{0.740233in}}%
\pgfpathlineto{\pgfqpoint{1.601207in}{0.700417in}}%
\pgfpathlineto{\pgfqpoint{1.602071in}{0.719647in}}%
\pgfpathlineto{\pgfqpoint{1.602937in}{0.703714in}}%
\pgfpathlineto{\pgfqpoint{1.603803in}{0.710747in}}%
\pgfpathlineto{\pgfqpoint{1.604669in}{0.728035in}}%
\pgfpathlineto{\pgfqpoint{1.606401in}{0.687010in}}%
\pgfpathlineto{\pgfqpoint{1.608130in}{0.716167in}}%
\pgfpathlineto{\pgfqpoint{1.608995in}{0.690745in}}%
\pgfpathlineto{\pgfqpoint{1.609860in}{0.740598in}}%
\pgfpathlineto{\pgfqpoint{1.610725in}{0.651332in}}%
\pgfpathlineto{\pgfqpoint{1.614186in}{0.758001in}}%
\pgfpathlineto{\pgfqpoint{1.615051in}{0.702175in}}%
\pgfpathlineto{\pgfqpoint{1.615917in}{0.756718in}}%
\pgfpathlineto{\pgfqpoint{1.616782in}{0.744886in}}%
\pgfpathlineto{\pgfqpoint{1.617645in}{0.751626in}}%
\pgfpathlineto{\pgfqpoint{1.619375in}{0.653935in}}%
\pgfpathlineto{\pgfqpoint{1.621104in}{0.709025in}}%
\pgfpathlineto{\pgfqpoint{1.621969in}{0.681334in}}%
\pgfpathlineto{\pgfqpoint{1.622835in}{0.792505in}}%
\pgfpathlineto{\pgfqpoint{1.624564in}{0.702431in}}%
\pgfpathlineto{\pgfqpoint{1.625428in}{0.734922in}}%
\pgfpathlineto{\pgfqpoint{1.627159in}{0.688951in}}%
\pgfpathlineto{\pgfqpoint{1.628022in}{0.718218in}}%
\pgfpathlineto{\pgfqpoint{1.628886in}{0.810598in}}%
\pgfpathlineto{\pgfqpoint{1.629752in}{0.703199in}}%
\pgfpathlineto{\pgfqpoint{1.630617in}{0.730049in}}%
\pgfpathlineto{\pgfqpoint{1.631481in}{0.778327in}}%
\pgfpathlineto{\pgfqpoint{1.632344in}{0.706313in}}%
\pgfpathlineto{\pgfqpoint{1.633210in}{0.769865in}}%
\pgfpathlineto{\pgfqpoint{1.634073in}{0.699573in}}%
\pgfpathlineto{\pgfqpoint{1.634936in}{0.700856in}}%
\pgfpathlineto{\pgfqpoint{1.636668in}{0.751476in}}%
\pgfpathlineto{\pgfqpoint{1.637533in}{0.732429in}}%
\pgfpathlineto{\pgfqpoint{1.638397in}{0.779975in}}%
\pgfpathlineto{\pgfqpoint{1.639263in}{0.726350in}}%
\pgfpathlineto{\pgfqpoint{1.640129in}{0.786277in}}%
\pgfpathlineto{\pgfqpoint{1.641860in}{0.644409in}}%
\pgfpathlineto{\pgfqpoint{1.642725in}{0.689682in}}%
\pgfpathlineto{\pgfqpoint{1.643590in}{0.751037in}}%
\pgfpathlineto{\pgfqpoint{1.644456in}{0.699426in}}%
\pgfpathlineto{\pgfqpoint{1.645321in}{0.719793in}}%
\pgfpathlineto{\pgfqpoint{1.646187in}{0.779683in}}%
\pgfpathlineto{\pgfqpoint{1.647053in}{0.772212in}}%
\pgfpathlineto{\pgfqpoint{1.647918in}{0.705468in}}%
\pgfpathlineto{\pgfqpoint{1.648784in}{0.710012in}}%
\pgfpathlineto{\pgfqpoint{1.649645in}{0.681220in}}%
\pgfpathlineto{\pgfqpoint{1.650509in}{0.730488in}}%
\pgfpathlineto{\pgfqpoint{1.651372in}{0.706240in}}%
\pgfpathlineto{\pgfqpoint{1.654830in}{0.790378in}}%
\pgfpathlineto{\pgfqpoint{1.655694in}{0.722578in}}%
\pgfpathlineto{\pgfqpoint{1.657422in}{0.783784in}}%
\pgfpathlineto{\pgfqpoint{1.660016in}{0.683713in}}%
\pgfpathlineto{\pgfqpoint{1.660879in}{0.763604in}}%
\pgfpathlineto{\pgfqpoint{1.662610in}{0.699280in}}%
\pgfpathlineto{\pgfqpoint{1.663476in}{0.724738in}}%
\pgfpathlineto{\pgfqpoint{1.664342in}{0.654665in}}%
\pgfpathlineto{\pgfqpoint{1.666070in}{0.722505in}}%
\pgfpathlineto{\pgfqpoint{1.667800in}{0.731442in}}%
\pgfpathlineto{\pgfqpoint{1.668666in}{0.685142in}}%
\pgfpathlineto{\pgfqpoint{1.669531in}{0.704372in}}%
\pgfpathlineto{\pgfqpoint{1.670393in}{0.692613in}}%
\pgfpathlineto{\pgfqpoint{1.671259in}{0.707669in}}%
\pgfpathlineto{\pgfqpoint{1.672123in}{0.695764in}}%
\pgfpathlineto{\pgfqpoint{1.672989in}{0.732136in}}%
\pgfpathlineto{\pgfqpoint{1.673852in}{0.661698in}}%
\pgfpathlineto{\pgfqpoint{1.675581in}{0.724519in}}%
\pgfpathlineto{\pgfqpoint{1.676445in}{0.739502in}}%
\pgfpathlineto{\pgfqpoint{1.677310in}{0.725985in}}%
\pgfpathlineto{\pgfqpoint{1.678175in}{0.694042in}}%
\pgfpathlineto{\pgfqpoint{1.679039in}{0.708034in}}%
\pgfpathlineto{\pgfqpoint{1.679905in}{0.689170in}}%
\pgfpathlineto{\pgfqpoint{1.681635in}{0.700490in}}%
\pgfpathlineto{\pgfqpoint{1.682499in}{0.749352in}}%
\pgfpathlineto{\pgfqpoint{1.683364in}{0.702943in}}%
\pgfpathlineto{\pgfqpoint{1.684229in}{0.812429in}}%
\pgfpathlineto{\pgfqpoint{1.685959in}{0.707742in}}%
\pgfpathlineto{\pgfqpoint{1.686825in}{0.678987in}}%
\pgfpathlineto{\pgfqpoint{1.687689in}{0.681626in}}%
\pgfpathlineto{\pgfqpoint{1.688554in}{0.678037in}}%
\pgfpathlineto{\pgfqpoint{1.689420in}{0.659172in}}%
\pgfpathlineto{\pgfqpoint{1.690285in}{0.696349in}}%
\pgfpathlineto{\pgfqpoint{1.691149in}{0.682174in}}%
\pgfpathlineto{\pgfqpoint{1.692015in}{0.637450in}}%
\pgfpathlineto{\pgfqpoint{1.693742in}{0.756316in}}%
\pgfpathlineto{\pgfqpoint{1.695472in}{0.683348in}}%
\pgfpathlineto{\pgfqpoint{1.696337in}{0.754740in}}%
\pgfpathlineto{\pgfqpoint{1.697201in}{0.730123in}}%
\pgfpathlineto{\pgfqpoint{1.698064in}{0.756864in}}%
\pgfpathlineto{\pgfqpoint{1.698929in}{0.662136in}}%
\pgfpathlineto{\pgfqpoint{1.699794in}{0.726972in}}%
\pgfpathlineto{\pgfqpoint{1.700660in}{0.653455in}}%
\pgfpathlineto{\pgfqpoint{1.701524in}{0.715286in}}%
\pgfpathlineto{\pgfqpoint{1.702389in}{0.686165in}}%
\pgfpathlineto{\pgfqpoint{1.703252in}{0.697595in}}%
\pgfpathlineto{\pgfqpoint{1.704117in}{0.630563in}}%
\pgfpathlineto{\pgfqpoint{1.705846in}{0.700563in}}%
\pgfpathlineto{\pgfqpoint{1.706709in}{0.651222in}}%
\pgfpathlineto{\pgfqpoint{1.707574in}{0.750124in}}%
\pgfpathlineto{\pgfqpoint{1.709304in}{0.664629in}}%
\pgfpathlineto{\pgfqpoint{1.712763in}{0.783089in}}%
\pgfpathlineto{\pgfqpoint{1.713627in}{0.703089in}}%
\pgfpathlineto{\pgfqpoint{1.714491in}{0.709464in}}%
\pgfpathlineto{\pgfqpoint{1.715355in}{0.695837in}}%
\pgfpathlineto{\pgfqpoint{1.716220in}{0.733639in}}%
\pgfpathlineto{\pgfqpoint{1.717949in}{0.639244in}}%
\pgfpathlineto{\pgfqpoint{1.719680in}{0.706020in}}%
\pgfpathlineto{\pgfqpoint{1.720545in}{0.714409in}}%
\pgfpathlineto{\pgfqpoint{1.721409in}{0.706971in}}%
\pgfpathlineto{\pgfqpoint{1.723138in}{0.630344in}}%
\pgfpathlineto{\pgfqpoint{1.724003in}{0.754005in}}%
\pgfpathlineto{\pgfqpoint{1.724868in}{0.683676in}}%
\pgfpathlineto{\pgfqpoint{1.725732in}{0.787779in}}%
\pgfpathlineto{\pgfqpoint{1.726596in}{0.667594in}}%
\pgfpathlineto{\pgfqpoint{1.728326in}{0.740452in}}%
\pgfpathlineto{\pgfqpoint{1.729191in}{0.678329in}}%
\pgfpathlineto{\pgfqpoint{1.730922in}{0.769353in}}%
\pgfpathlineto{\pgfqpoint{1.731786in}{0.722834in}}%
\pgfpathlineto{\pgfqpoint{1.732650in}{0.794885in}}%
\pgfpathlineto{\pgfqpoint{1.733516in}{0.735653in}}%
\pgfpathlineto{\pgfqpoint{1.734382in}{0.737520in}}%
\pgfpathlineto{\pgfqpoint{1.735246in}{0.771952in}}%
\pgfpathlineto{\pgfqpoint{1.736974in}{0.735872in}}%
\pgfpathlineto{\pgfqpoint{1.737840in}{0.764002in}}%
\pgfpathlineto{\pgfqpoint{1.739570in}{0.683088in}}%
\pgfpathlineto{\pgfqpoint{1.741299in}{0.716127in}}%
\pgfpathlineto{\pgfqpoint{1.743026in}{0.685800in}}%
\pgfpathlineto{\pgfqpoint{1.743891in}{0.725835in}}%
\pgfpathlineto{\pgfqpoint{1.744756in}{0.686019in}}%
\pgfpathlineto{\pgfqpoint{1.745620in}{0.722611in}}%
\pgfpathlineto{\pgfqpoint{1.746486in}{0.683786in}}%
\pgfpathlineto{\pgfqpoint{1.747352in}{0.722465in}}%
\pgfpathlineto{\pgfqpoint{1.748217in}{0.717519in}}%
\pgfpathlineto{\pgfqpoint{1.749082in}{0.732210in}}%
\pgfpathlineto{\pgfqpoint{1.749948in}{0.779829in}}%
\pgfpathlineto{\pgfqpoint{1.750814in}{0.704884in}}%
\pgfpathlineto{\pgfqpoint{1.751679in}{0.711843in}}%
\pgfpathlineto{\pgfqpoint{1.752545in}{0.727045in}}%
\pgfpathlineto{\pgfqpoint{1.753409in}{0.702577in}}%
\pgfpathlineto{\pgfqpoint{1.754272in}{0.737374in}}%
\pgfpathlineto{\pgfqpoint{1.756004in}{0.693677in}}%
\pgfpathlineto{\pgfqpoint{1.756870in}{0.774299in}}%
\pgfpathlineto{\pgfqpoint{1.758602in}{0.690526in}}%
\pgfpathlineto{\pgfqpoint{1.759468in}{0.692394in}}%
\pgfpathlineto{\pgfqpoint{1.761200in}{0.736936in}}%
\pgfpathlineto{\pgfqpoint{1.762066in}{0.726826in}}%
\pgfpathlineto{\pgfqpoint{1.762931in}{0.730305in}}%
\pgfpathlineto{\pgfqpoint{1.763797in}{0.713491in}}%
\pgfpathlineto{\pgfqpoint{1.764663in}{0.716350in}}%
\pgfpathlineto{\pgfqpoint{1.765529in}{0.709537in}}%
\pgfpathlineto{\pgfqpoint{1.766394in}{0.666059in}}%
\pgfpathlineto{\pgfqpoint{1.768988in}{0.766933in}}%
\pgfpathlineto{\pgfqpoint{1.770717in}{0.695066in}}%
\pgfpathlineto{\pgfqpoint{1.771582in}{0.734516in}}%
\pgfpathlineto{\pgfqpoint{1.773309in}{0.658401in}}%
\pgfpathlineto{\pgfqpoint{1.774173in}{0.747338in}}%
\pgfpathlineto{\pgfqpoint{1.775039in}{0.719939in}}%
\pgfpathlineto{\pgfqpoint{1.775903in}{0.728839in}}%
\pgfpathlineto{\pgfqpoint{1.776768in}{0.723017in}}%
\pgfpathlineto{\pgfqpoint{1.777633in}{0.707888in}}%
\pgfpathlineto{\pgfqpoint{1.778496in}{0.726460in}}%
\pgfpathlineto{\pgfqpoint{1.779358in}{0.709317in}}%
\pgfpathlineto{\pgfqpoint{1.780222in}{0.660196in}}%
\pgfpathlineto{\pgfqpoint{1.781951in}{0.721405in}}%
\pgfpathlineto{\pgfqpoint{1.782816in}{0.733127in}}%
\pgfpathlineto{\pgfqpoint{1.783681in}{0.694554in}}%
\pgfpathlineto{\pgfqpoint{1.784545in}{0.707523in}}%
\pgfpathlineto{\pgfqpoint{1.785410in}{0.660342in}}%
\pgfpathlineto{\pgfqpoint{1.788003in}{0.764262in}}%
\pgfpathlineto{\pgfqpoint{1.788868in}{0.626312in}}%
\pgfpathlineto{\pgfqpoint{1.789732in}{0.705176in}}%
\pgfpathlineto{\pgfqpoint{1.790596in}{0.703527in}}%
\pgfpathlineto{\pgfqpoint{1.792326in}{0.780268in}}%
\pgfpathlineto{\pgfqpoint{1.794054in}{0.641697in}}%
\pgfpathlineto{\pgfqpoint{1.794918in}{0.714701in}}%
\pgfpathlineto{\pgfqpoint{1.795782in}{0.685471in}}%
\pgfpathlineto{\pgfqpoint{1.796649in}{0.737082in}}%
\pgfpathlineto{\pgfqpoint{1.797512in}{0.726826in}}%
\pgfpathlineto{\pgfqpoint{1.798377in}{0.672356in}}%
\pgfpathlineto{\pgfqpoint{1.800107in}{0.738219in}}%
\pgfpathlineto{\pgfqpoint{1.800973in}{0.700746in}}%
\pgfpathlineto{\pgfqpoint{1.801837in}{0.724738in}}%
\pgfpathlineto{\pgfqpoint{1.802704in}{0.713970in}}%
\pgfpathlineto{\pgfqpoint{1.803570in}{0.679685in}}%
\pgfpathlineto{\pgfqpoint{1.806165in}{0.762979in}}%
\pgfpathlineto{\pgfqpoint{1.807031in}{0.784848in}}%
\pgfpathlineto{\pgfqpoint{1.807897in}{0.683567in}}%
\pgfpathlineto{\pgfqpoint{1.808763in}{0.776020in}}%
\pgfpathlineto{\pgfqpoint{1.810493in}{0.695874in}}%
\pgfpathlineto{\pgfqpoint{1.811358in}{0.700157in}}%
\pgfpathlineto{\pgfqpoint{1.813954in}{0.746055in}}%
\pgfpathlineto{\pgfqpoint{1.815683in}{0.700304in}}%
\pgfpathlineto{\pgfqpoint{1.816549in}{0.700523in}}%
\pgfpathlineto{\pgfqpoint{1.817413in}{0.749791in}}%
\pgfpathlineto{\pgfqpoint{1.819140in}{0.687814in}}%
\pgfpathlineto{\pgfqpoint{1.820870in}{0.744187in}}%
\pgfpathlineto{\pgfqpoint{1.821736in}{0.680562in}}%
\pgfpathlineto{\pgfqpoint{1.822603in}{0.817265in}}%
\pgfpathlineto{\pgfqpoint{1.823469in}{0.697413in}}%
\pgfpathlineto{\pgfqpoint{1.825198in}{0.781258in}}%
\pgfpathlineto{\pgfqpoint{1.826928in}{0.759170in}}%
\pgfpathlineto{\pgfqpoint{1.827794in}{0.763823in}}%
\pgfpathlineto{\pgfqpoint{1.828657in}{0.697120in}}%
\pgfpathlineto{\pgfqpoint{1.830388in}{0.765837in}}%
\pgfpathlineto{\pgfqpoint{1.831253in}{0.703089in}}%
\pgfpathlineto{\pgfqpoint{1.832979in}{0.790524in}}%
\pgfpathlineto{\pgfqpoint{1.834712in}{0.686677in}}%
\pgfpathlineto{\pgfqpoint{1.835577in}{0.741695in}}%
\pgfpathlineto{\pgfqpoint{1.837307in}{0.676494in}}%
\pgfpathlineto{\pgfqpoint{1.838171in}{0.736603in}}%
\pgfpathlineto{\pgfqpoint{1.839035in}{0.731182in}}%
\pgfpathlineto{\pgfqpoint{1.839899in}{0.755102in}}%
\pgfpathlineto{\pgfqpoint{1.840763in}{0.749937in}}%
\pgfpathlineto{\pgfqpoint{1.842494in}{0.693637in}}%
\pgfpathlineto{\pgfqpoint{1.843359in}{0.756275in}}%
\pgfpathlineto{\pgfqpoint{1.844224in}{0.753234in}}%
\pgfpathlineto{\pgfqpoint{1.845089in}{0.787008in}}%
\pgfpathlineto{\pgfqpoint{1.845955in}{0.751037in}}%
\pgfpathlineto{\pgfqpoint{1.846820in}{0.763417in}}%
\pgfpathlineto{\pgfqpoint{1.849412in}{0.708619in}}%
\pgfpathlineto{\pgfqpoint{1.850276in}{0.715286in}}%
\pgfpathlineto{\pgfqpoint{1.851141in}{0.676754in}}%
\pgfpathlineto{\pgfqpoint{1.852007in}{0.684371in}}%
\pgfpathlineto{\pgfqpoint{1.852870in}{0.725652in}}%
\pgfpathlineto{\pgfqpoint{1.853735in}{0.689170in}}%
\pgfpathlineto{\pgfqpoint{1.854601in}{0.770011in}}%
\pgfpathlineto{\pgfqpoint{1.855465in}{0.744736in}}%
\pgfpathlineto{\pgfqpoint{1.856331in}{0.780121in}}%
\pgfpathlineto{\pgfqpoint{1.857196in}{0.642687in}}%
\pgfpathlineto{\pgfqpoint{1.858062in}{0.670342in}}%
\pgfpathlineto{\pgfqpoint{1.858925in}{0.719574in}}%
\pgfpathlineto{\pgfqpoint{1.859790in}{0.691403in}}%
\pgfpathlineto{\pgfqpoint{1.861521in}{0.761184in}}%
\pgfpathlineto{\pgfqpoint{1.862388in}{0.707779in}}%
\pgfpathlineto{\pgfqpoint{1.864982in}{0.755544in}}%
\pgfpathlineto{\pgfqpoint{1.865845in}{0.765399in}}%
\pgfpathlineto{\pgfqpoint{1.867573in}{0.688549in}}%
\pgfpathlineto{\pgfqpoint{1.868440in}{0.757010in}}%
\pgfpathlineto{\pgfqpoint{1.871032in}{0.642468in}}%
\pgfpathlineto{\pgfqpoint{1.871897in}{0.742908in}}%
\pgfpathlineto{\pgfqpoint{1.872762in}{0.681845in}}%
\pgfpathlineto{\pgfqpoint{1.873626in}{0.710600in}}%
\pgfpathlineto{\pgfqpoint{1.874493in}{0.632431in}}%
\pgfpathlineto{\pgfqpoint{1.876224in}{0.680781in}}%
\pgfpathlineto{\pgfqpoint{1.877089in}{0.678512in}}%
\pgfpathlineto{\pgfqpoint{1.877953in}{0.658401in}}%
\pgfpathlineto{\pgfqpoint{1.878816in}{0.703016in}}%
\pgfpathlineto{\pgfqpoint{1.879681in}{0.676534in}}%
\pgfpathlineto{\pgfqpoint{1.882278in}{0.776459in}}%
\pgfpathlineto{\pgfqpoint{1.884008in}{0.741954in}}%
\pgfpathlineto{\pgfqpoint{1.884871in}{0.777815in}}%
\pgfpathlineto{\pgfqpoint{1.885735in}{0.774664in}}%
\pgfpathlineto{\pgfqpoint{1.886600in}{0.774518in}}%
\pgfpathlineto{\pgfqpoint{1.887463in}{0.838728in}}%
\pgfpathlineto{\pgfqpoint{1.888327in}{0.769020in}}%
\pgfpathlineto{\pgfqpoint{1.889190in}{0.777336in}}%
\pgfpathlineto{\pgfqpoint{1.892650in}{0.694335in}}%
\pgfpathlineto{\pgfqpoint{1.893515in}{0.682138in}}%
\pgfpathlineto{\pgfqpoint{1.894381in}{0.762540in}}%
\pgfpathlineto{\pgfqpoint{1.895246in}{0.659392in}}%
\pgfpathlineto{\pgfqpoint{1.896975in}{0.721368in}}%
\pgfpathlineto{\pgfqpoint{1.897841in}{0.720451in}}%
\pgfpathlineto{\pgfqpoint{1.898707in}{0.711697in}}%
\pgfpathlineto{\pgfqpoint{1.899572in}{0.790524in}}%
\pgfpathlineto{\pgfqpoint{1.901303in}{0.668438in}}%
\pgfpathlineto{\pgfqpoint{1.902169in}{0.747265in}}%
\pgfpathlineto{\pgfqpoint{1.903034in}{0.740452in}}%
\pgfpathlineto{\pgfqpoint{1.903899in}{0.632285in}}%
\pgfpathlineto{\pgfqpoint{1.905629in}{0.783199in}}%
\pgfpathlineto{\pgfqpoint{1.906494in}{0.716642in}}%
\pgfpathlineto{\pgfqpoint{1.907360in}{0.791003in}}%
\pgfpathlineto{\pgfqpoint{1.908225in}{0.690526in}}%
\pgfpathlineto{\pgfqpoint{1.909089in}{0.739721in}}%
\pgfpathlineto{\pgfqpoint{1.909953in}{0.715948in}}%
\pgfpathlineto{\pgfqpoint{1.910818in}{0.649354in}}%
\pgfpathlineto{\pgfqpoint{1.911684in}{0.783824in}}%
\pgfpathlineto{\pgfqpoint{1.912550in}{0.700344in}}%
\pgfpathlineto{\pgfqpoint{1.913415in}{0.774226in}}%
\pgfpathlineto{\pgfqpoint{1.916011in}{0.673384in}}%
\pgfpathlineto{\pgfqpoint{1.916877in}{0.737561in}}%
\pgfpathlineto{\pgfqpoint{1.917742in}{0.733127in}}%
\pgfpathlineto{\pgfqpoint{1.918607in}{0.726095in}}%
\pgfpathlineto{\pgfqpoint{1.919471in}{0.729867in}}%
\pgfpathlineto{\pgfqpoint{1.920335in}{0.676534in}}%
\pgfpathlineto{\pgfqpoint{1.921200in}{0.719574in}}%
\pgfpathlineto{\pgfqpoint{1.922066in}{0.699573in}}%
\pgfpathlineto{\pgfqpoint{1.922931in}{0.728328in}}%
\pgfpathlineto{\pgfqpoint{1.923796in}{0.701554in}}%
\pgfpathlineto{\pgfqpoint{1.924661in}{0.710783in}}%
\pgfpathlineto{\pgfqpoint{1.926393in}{0.662981in}}%
\pgfpathlineto{\pgfqpoint{1.927259in}{0.671881in}}%
\pgfpathlineto{\pgfqpoint{1.928122in}{0.732908in}}%
\pgfpathlineto{\pgfqpoint{1.928985in}{0.684594in}}%
\pgfpathlineto{\pgfqpoint{1.929850in}{0.704006in}}%
\pgfpathlineto{\pgfqpoint{1.930715in}{0.774006in}}%
\pgfpathlineto{\pgfqpoint{1.932446in}{0.701075in}}%
\pgfpathlineto{\pgfqpoint{1.933312in}{0.733931in}}%
\pgfpathlineto{\pgfqpoint{1.934179in}{0.697778in}}%
\pgfpathlineto{\pgfqpoint{1.935043in}{0.733493in}}%
\pgfpathlineto{\pgfqpoint{1.936775in}{0.697559in}}%
\pgfpathlineto{\pgfqpoint{1.938507in}{0.754992in}}%
\pgfpathlineto{\pgfqpoint{1.939373in}{0.754517in}}%
\pgfpathlineto{\pgfqpoint{1.941103in}{0.712318in}}%
\pgfpathlineto{\pgfqpoint{1.941967in}{0.727703in}}%
\pgfpathlineto{\pgfqpoint{1.943696in}{0.712428in}}%
\pgfpathlineto{\pgfqpoint{1.945425in}{0.745357in}}%
\pgfpathlineto{\pgfqpoint{1.946290in}{0.697153in}}%
\pgfpathlineto{\pgfqpoint{1.947154in}{0.734808in}}%
\pgfpathlineto{\pgfqpoint{1.948018in}{0.685211in}}%
\pgfpathlineto{\pgfqpoint{1.948883in}{0.698363in}}%
\pgfpathlineto{\pgfqpoint{1.949748in}{0.736603in}}%
\pgfpathlineto{\pgfqpoint{1.950610in}{0.728068in}}%
\pgfpathlineto{\pgfqpoint{1.951475in}{0.722684in}}%
\pgfpathlineto{\pgfqpoint{1.952341in}{0.707190in}}%
\pgfpathlineto{\pgfqpoint{1.953207in}{0.658620in}}%
\pgfpathlineto{\pgfqpoint{1.954936in}{0.720816in}}%
\pgfpathlineto{\pgfqpoint{1.955801in}{0.723529in}}%
\pgfpathlineto{\pgfqpoint{1.956666in}{0.755175in}}%
\pgfpathlineto{\pgfqpoint{1.957531in}{0.707555in}}%
\pgfpathlineto{\pgfqpoint{1.958396in}{0.784807in}}%
\pgfpathlineto{\pgfqpoint{1.961857in}{0.699865in}}%
\pgfpathlineto{\pgfqpoint{1.963588in}{0.742027in}}%
\pgfpathlineto{\pgfqpoint{1.964454in}{0.805433in}}%
\pgfpathlineto{\pgfqpoint{1.965319in}{0.735141in}}%
\pgfpathlineto{\pgfqpoint{1.966183in}{0.738146in}}%
\pgfpathlineto{\pgfqpoint{1.967048in}{0.697705in}}%
\pgfpathlineto{\pgfqpoint{1.967912in}{0.748694in}}%
\pgfpathlineto{\pgfqpoint{1.968778in}{0.694335in}}%
\pgfpathlineto{\pgfqpoint{1.969644in}{0.744443in}}%
\pgfpathlineto{\pgfqpoint{1.970509in}{0.738657in}}%
\pgfpathlineto{\pgfqpoint{1.971375in}{0.725323in}}%
\pgfpathlineto{\pgfqpoint{1.972240in}{0.766531in}}%
\pgfpathlineto{\pgfqpoint{1.973972in}{0.689755in}}%
\pgfpathlineto{\pgfqpoint{1.975702in}{0.737886in}}%
\pgfpathlineto{\pgfqpoint{1.976567in}{0.727081in}}%
\pgfpathlineto{\pgfqpoint{1.978298in}{0.699426in}}%
\pgfpathlineto{\pgfqpoint{1.980028in}{0.783491in}}%
\pgfpathlineto{\pgfqpoint{1.980894in}{0.677631in}}%
\pgfpathlineto{\pgfqpoint{1.981760in}{0.720597in}}%
\pgfpathlineto{\pgfqpoint{1.982625in}{0.706751in}}%
\pgfpathlineto{\pgfqpoint{1.983491in}{0.765066in}}%
\pgfpathlineto{\pgfqpoint{1.986949in}{0.629500in}}%
\pgfpathlineto{\pgfqpoint{1.987814in}{0.730488in}}%
\pgfpathlineto{\pgfqpoint{1.988674in}{0.716898in}}%
\pgfpathlineto{\pgfqpoint{1.989540in}{0.793821in}}%
\pgfpathlineto{\pgfqpoint{1.990405in}{0.717300in}}%
\pgfpathlineto{\pgfqpoint{1.992132in}{0.785067in}}%
\pgfpathlineto{\pgfqpoint{1.992997in}{0.756385in}}%
\pgfpathlineto{\pgfqpoint{1.993859in}{0.760193in}}%
\pgfpathlineto{\pgfqpoint{1.996456in}{0.693271in}}%
\pgfpathlineto{\pgfqpoint{1.997321in}{0.727118in}}%
\pgfpathlineto{\pgfqpoint{1.998186in}{0.686239in}}%
\pgfpathlineto{\pgfqpoint{1.999051in}{0.700669in}}%
\pgfpathlineto{\pgfqpoint{1.999915in}{0.760705in}}%
\pgfpathlineto{\pgfqpoint{2.001642in}{0.717885in}}%
\pgfpathlineto{\pgfqpoint{2.002508in}{0.766235in}}%
\pgfpathlineto{\pgfqpoint{2.003373in}{0.714295in}}%
\pgfpathlineto{\pgfqpoint{2.004238in}{0.730155in}}%
\pgfpathlineto{\pgfqpoint{2.005103in}{0.678325in}}%
\pgfpathlineto{\pgfqpoint{2.006833in}{0.728986in}}%
\pgfpathlineto{\pgfqpoint{2.007697in}{0.699719in}}%
\pgfpathlineto{\pgfqpoint{2.008563in}{0.757668in}}%
\pgfpathlineto{\pgfqpoint{2.009429in}{0.738032in}}%
\pgfpathlineto{\pgfqpoint{2.011158in}{0.758252in}}%
\pgfpathlineto{\pgfqpoint{2.012023in}{0.720962in}}%
\pgfpathlineto{\pgfqpoint{2.013752in}{0.785944in}}%
\pgfpathlineto{\pgfqpoint{2.014617in}{0.675138in}}%
\pgfpathlineto{\pgfqpoint{2.015482in}{0.704039in}}%
\pgfpathlineto{\pgfqpoint{2.016347in}{0.737740in}}%
\pgfpathlineto{\pgfqpoint{2.017211in}{0.700669in}}%
\pgfpathlineto{\pgfqpoint{2.018077in}{0.722757in}}%
\pgfpathlineto{\pgfqpoint{2.019804in}{0.633235in}}%
\pgfpathlineto{\pgfqpoint{2.020669in}{0.785652in}}%
\pgfpathlineto{\pgfqpoint{2.022400in}{0.676201in}}%
\pgfpathlineto{\pgfqpoint{2.024131in}{0.789574in}}%
\pgfpathlineto{\pgfqpoint{2.024995in}{0.756531in}}%
\pgfpathlineto{\pgfqpoint{2.026727in}{0.700267in}}%
\pgfpathlineto{\pgfqpoint{2.028455in}{0.734808in}}%
\pgfpathlineto{\pgfqpoint{2.029320in}{0.695248in}}%
\pgfpathlineto{\pgfqpoint{2.031052in}{0.766568in}}%
\pgfpathlineto{\pgfqpoint{2.031917in}{0.696203in}}%
\pgfpathlineto{\pgfqpoint{2.032782in}{0.711843in}}%
\pgfpathlineto{\pgfqpoint{2.033646in}{0.715067in}}%
\pgfpathlineto{\pgfqpoint{2.034510in}{0.730415in}}%
\pgfpathlineto{\pgfqpoint{2.035374in}{0.832979in}}%
\pgfpathlineto{\pgfqpoint{2.037104in}{0.754663in}}%
\pgfpathlineto{\pgfqpoint{2.037969in}{0.749279in}}%
\pgfpathlineto{\pgfqpoint{2.038834in}{0.658657in}}%
\pgfpathlineto{\pgfqpoint{2.039700in}{0.775030in}}%
\pgfpathlineto{\pgfqpoint{2.040565in}{0.694189in}}%
\pgfpathlineto{\pgfqpoint{2.041430in}{0.702870in}}%
\pgfpathlineto{\pgfqpoint{2.042296in}{0.704737in}}%
\pgfpathlineto{\pgfqpoint{2.043162in}{0.712468in}}%
\pgfpathlineto{\pgfqpoint{2.044026in}{0.787998in}}%
\pgfpathlineto{\pgfqpoint{2.044891in}{0.787487in}}%
\pgfpathlineto{\pgfqpoint{2.045756in}{0.806461in}}%
\pgfpathlineto{\pgfqpoint{2.048349in}{0.703933in}}%
\pgfpathlineto{\pgfqpoint{2.049214in}{0.818329in}}%
\pgfpathlineto{\pgfqpoint{2.050080in}{0.762321in}}%
\pgfpathlineto{\pgfqpoint{2.051810in}{0.803273in}}%
\pgfpathlineto{\pgfqpoint{2.052674in}{0.736022in}}%
\pgfpathlineto{\pgfqpoint{2.054402in}{0.801552in}}%
\pgfpathlineto{\pgfqpoint{2.055268in}{0.716460in}}%
\pgfpathlineto{\pgfqpoint{2.056131in}{0.729976in}}%
\pgfpathlineto{\pgfqpoint{2.056996in}{0.800488in}}%
\pgfpathlineto{\pgfqpoint{2.057861in}{0.770669in}}%
\pgfpathlineto{\pgfqpoint{2.058726in}{0.686239in}}%
\pgfpathlineto{\pgfqpoint{2.059591in}{0.760965in}}%
\pgfpathlineto{\pgfqpoint{2.060456in}{0.744480in}}%
\pgfpathlineto{\pgfqpoint{2.061322in}{0.739461in}}%
\pgfpathlineto{\pgfqpoint{2.063052in}{0.667667in}}%
\pgfpathlineto{\pgfqpoint{2.063916in}{0.684444in}}%
\pgfpathlineto{\pgfqpoint{2.064780in}{0.767595in}}%
\pgfpathlineto{\pgfqpoint{2.065646in}{0.755508in}}%
\pgfpathlineto{\pgfqpoint{2.067377in}{0.711441in}}%
\pgfpathlineto{\pgfqpoint{2.068243in}{0.762686in}}%
\pgfpathlineto{\pgfqpoint{2.070838in}{0.644482in}}%
\pgfpathlineto{\pgfqpoint{2.073429in}{0.735580in}}%
\pgfpathlineto{\pgfqpoint{2.074294in}{0.744809in}}%
\pgfpathlineto{\pgfqpoint{2.075160in}{0.724406in}}%
\pgfpathlineto{\pgfqpoint{2.076025in}{0.739023in}}%
\pgfpathlineto{\pgfqpoint{2.076890in}{0.681183in}}%
\pgfpathlineto{\pgfqpoint{2.080348in}{0.739023in}}%
\pgfpathlineto{\pgfqpoint{2.081213in}{0.754298in}}%
\pgfpathlineto{\pgfqpoint{2.082078in}{0.690745in}}%
\pgfpathlineto{\pgfqpoint{2.083807in}{0.747229in}}%
\pgfpathlineto{\pgfqpoint{2.085536in}{0.691549in}}%
\pgfpathlineto{\pgfqpoint{2.086401in}{0.752686in}}%
\pgfpathlineto{\pgfqpoint{2.087265in}{0.721880in}}%
\pgfpathlineto{\pgfqpoint{2.089859in}{0.768582in}}%
\pgfpathlineto{\pgfqpoint{2.090724in}{0.757485in}}%
\pgfpathlineto{\pgfqpoint{2.091589in}{0.697299in}}%
\pgfpathlineto{\pgfqpoint{2.094184in}{0.754590in}}%
\pgfpathlineto{\pgfqpoint{2.095914in}{0.704185in}}%
\pgfpathlineto{\pgfqpoint{2.096780in}{0.803639in}}%
\pgfpathlineto{\pgfqpoint{2.099373in}{0.724519in}}%
\pgfpathlineto{\pgfqpoint{2.100239in}{0.764335in}}%
\pgfpathlineto{\pgfqpoint{2.101101in}{0.762321in}}%
\pgfpathlineto{\pgfqpoint{2.101963in}{0.760819in}}%
\pgfpathlineto{\pgfqpoint{2.102828in}{0.716131in}}%
\pgfpathlineto{\pgfqpoint{2.103692in}{0.782761in}}%
\pgfpathlineto{\pgfqpoint{2.104558in}{0.765837in}}%
\pgfpathlineto{\pgfqpoint{2.107151in}{0.699938in}}%
\pgfpathlineto{\pgfqpoint{2.108016in}{0.718510in}}%
\pgfpathlineto{\pgfqpoint{2.108880in}{0.795250in}}%
\pgfpathlineto{\pgfqpoint{2.109746in}{0.790670in}}%
\pgfpathlineto{\pgfqpoint{2.110611in}{0.729278in}}%
\pgfpathlineto{\pgfqpoint{2.111477in}{0.749864in}}%
\pgfpathlineto{\pgfqpoint{2.112342in}{0.689426in}}%
\pgfpathlineto{\pgfqpoint{2.114071in}{0.753786in}}%
\pgfpathlineto{\pgfqpoint{2.114938in}{0.677521in}}%
\pgfpathlineto{\pgfqpoint{2.115804in}{0.755508in}}%
\pgfpathlineto{\pgfqpoint{2.116670in}{0.691111in}}%
\pgfpathlineto{\pgfqpoint{2.117537in}{0.702577in}}%
\pgfpathlineto{\pgfqpoint{2.120134in}{0.820562in}}%
\pgfpathlineto{\pgfqpoint{2.121865in}{0.775947in}}%
\pgfpathlineto{\pgfqpoint{2.122731in}{0.764189in}}%
\pgfpathlineto{\pgfqpoint{2.124463in}{0.681260in}}%
\pgfpathlineto{\pgfqpoint{2.125328in}{0.710162in}}%
\pgfpathlineto{\pgfqpoint{2.126193in}{0.743164in}}%
\pgfpathlineto{\pgfqpoint{2.127924in}{0.686060in}}%
\pgfpathlineto{\pgfqpoint{2.128790in}{0.744740in}}%
\pgfpathlineto{\pgfqpoint{2.129654in}{0.710820in}}%
\pgfpathlineto{\pgfqpoint{2.130520in}{0.721588in}}%
\pgfpathlineto{\pgfqpoint{2.131385in}{0.663529in}}%
\pgfpathlineto{\pgfqpoint{2.132251in}{0.820197in}}%
\pgfpathlineto{\pgfqpoint{2.133116in}{0.812027in}}%
\pgfpathlineto{\pgfqpoint{2.134847in}{0.769061in}}%
\pgfpathlineto{\pgfqpoint{2.135713in}{0.691517in}}%
\pgfpathlineto{\pgfqpoint{2.137440in}{0.770783in}}%
\pgfpathlineto{\pgfqpoint{2.138303in}{0.731442in}}%
\pgfpathlineto{\pgfqpoint{2.139169in}{0.798255in}}%
\pgfpathlineto{\pgfqpoint{2.140031in}{0.724885in}}%
\pgfpathlineto{\pgfqpoint{2.140897in}{0.749096in}}%
\pgfpathlineto{\pgfqpoint{2.141763in}{0.694481in}}%
\pgfpathlineto{\pgfqpoint{2.142629in}{0.742100in}}%
\pgfpathlineto{\pgfqpoint{2.143495in}{0.711002in}}%
\pgfpathlineto{\pgfqpoint{2.144360in}{0.733346in}}%
\pgfpathlineto{\pgfqpoint{2.145225in}{0.711770in}}%
\pgfpathlineto{\pgfqpoint{2.146091in}{0.742612in}}%
\pgfpathlineto{\pgfqpoint{2.146956in}{0.709829in}}%
\pgfpathlineto{\pgfqpoint{2.147822in}{0.723748in}}%
\pgfpathlineto{\pgfqpoint{2.148687in}{0.696974in}}%
\pgfpathlineto{\pgfqpoint{2.149554in}{0.747484in}}%
\pgfpathlineto{\pgfqpoint{2.150418in}{0.743822in}}%
\pgfpathlineto{\pgfqpoint{2.151284in}{0.766495in}}%
\pgfpathlineto{\pgfqpoint{2.152149in}{0.736237in}}%
\pgfpathlineto{\pgfqpoint{2.153015in}{0.650414in}}%
\pgfpathlineto{\pgfqpoint{2.153879in}{0.776897in}}%
\pgfpathlineto{\pgfqpoint{2.154743in}{0.746275in}}%
\pgfpathlineto{\pgfqpoint{2.155607in}{0.699426in}}%
\pgfpathlineto{\pgfqpoint{2.156472in}{0.709975in}}%
\pgfpathlineto{\pgfqpoint{2.157337in}{0.840965in}}%
\pgfpathlineto{\pgfqpoint{2.159066in}{0.705436in}}%
\pgfpathlineto{\pgfqpoint{2.159928in}{0.724410in}}%
\pgfpathlineto{\pgfqpoint{2.160794in}{0.698842in}}%
\pgfpathlineto{\pgfqpoint{2.161658in}{0.745584in}}%
\pgfpathlineto{\pgfqpoint{2.162520in}{0.742580in}}%
\pgfpathlineto{\pgfqpoint{2.163384in}{0.673237in}}%
\pgfpathlineto{\pgfqpoint{2.164251in}{0.830307in}}%
\pgfpathlineto{\pgfqpoint{2.165115in}{0.738073in}}%
\pgfpathlineto{\pgfqpoint{2.165980in}{0.754078in}}%
\pgfpathlineto{\pgfqpoint{2.166844in}{0.735360in}}%
\pgfpathlineto{\pgfqpoint{2.167707in}{0.673530in}}%
\pgfpathlineto{\pgfqpoint{2.169437in}{0.806059in}}%
\pgfpathlineto{\pgfqpoint{2.171166in}{0.651076in}}%
\pgfpathlineto{\pgfqpoint{2.172029in}{0.709866in}}%
\pgfpathlineto{\pgfqpoint{2.172893in}{0.731771in}}%
\pgfpathlineto{\pgfqpoint{2.174620in}{0.668073in}}%
\pgfpathlineto{\pgfqpoint{2.175484in}{0.683201in}}%
\pgfpathlineto{\pgfqpoint{2.176349in}{0.739356in}}%
\pgfpathlineto{\pgfqpoint{2.178080in}{0.691809in}}%
\pgfpathlineto{\pgfqpoint{2.178945in}{0.712614in}}%
\pgfpathlineto{\pgfqpoint{2.179810in}{0.633202in}}%
\pgfpathlineto{\pgfqpoint{2.180676in}{0.652140in}}%
\pgfpathlineto{\pgfqpoint{2.181542in}{0.731588in}}%
\pgfpathlineto{\pgfqpoint{2.182405in}{0.721661in}}%
\pgfpathlineto{\pgfqpoint{2.183270in}{0.739940in}}%
\pgfpathlineto{\pgfqpoint{2.184136in}{0.709244in}}%
\pgfpathlineto{\pgfqpoint{2.185000in}{0.739940in}}%
\pgfpathlineto{\pgfqpoint{2.185866in}{0.737963in}}%
\pgfpathlineto{\pgfqpoint{2.186731in}{0.752503in}}%
\pgfpathlineto{\pgfqpoint{2.187597in}{0.668073in}}%
\pgfpathlineto{\pgfqpoint{2.189326in}{0.859793in}}%
\pgfpathlineto{\pgfqpoint{2.191056in}{0.733237in}}%
\pgfpathlineto{\pgfqpoint{2.192788in}{0.767339in}}%
\pgfpathlineto{\pgfqpoint{2.195384in}{0.717048in}}%
\pgfpathlineto{\pgfqpoint{2.196249in}{0.744082in}}%
\pgfpathlineto{\pgfqpoint{2.197115in}{0.712103in}}%
\pgfpathlineto{\pgfqpoint{2.197982in}{0.752909in}}%
\pgfpathlineto{\pgfqpoint{2.198846in}{0.686571in}}%
\pgfpathlineto{\pgfqpoint{2.199710in}{0.755069in}}%
\pgfpathlineto{\pgfqpoint{2.200575in}{0.665876in}}%
\pgfpathlineto{\pgfqpoint{2.202305in}{0.760161in}}%
\pgfpathlineto{\pgfqpoint{2.204035in}{0.674886in}}%
\pgfpathlineto{\pgfqpoint{2.204900in}{0.740598in}}%
\pgfpathlineto{\pgfqpoint{2.205766in}{0.727743in}}%
\pgfpathlineto{\pgfqpoint{2.206632in}{0.767486in}}%
\pgfpathlineto{\pgfqpoint{2.207495in}{0.764335in}}%
\pgfpathlineto{\pgfqpoint{2.208361in}{0.753713in}}%
\pgfpathlineto{\pgfqpoint{2.209225in}{0.690599in}}%
\pgfpathlineto{\pgfqpoint{2.210090in}{0.693750in}}%
\pgfpathlineto{\pgfqpoint{2.210954in}{0.788437in}}%
\pgfpathlineto{\pgfqpoint{2.211819in}{0.748329in}}%
\pgfpathlineto{\pgfqpoint{2.212685in}{0.771919in}}%
\pgfpathlineto{\pgfqpoint{2.213551in}{0.763823in}}%
\pgfpathlineto{\pgfqpoint{2.215281in}{0.733200in}}%
\pgfpathlineto{\pgfqpoint{2.216146in}{0.743895in}}%
\pgfpathlineto{\pgfqpoint{2.217875in}{0.688732in}}%
\pgfpathlineto{\pgfqpoint{2.218741in}{0.729319in}}%
\pgfpathlineto{\pgfqpoint{2.219607in}{0.681882in}}%
\pgfpathlineto{\pgfqpoint{2.220471in}{0.755142in}}%
\pgfpathlineto{\pgfqpoint{2.221337in}{0.696755in}}%
\pgfpathlineto{\pgfqpoint{2.222202in}{0.734410in}}%
\pgfpathlineto{\pgfqpoint{2.223931in}{0.679100in}}%
\pgfpathlineto{\pgfqpoint{2.224796in}{0.703714in}}%
\pgfpathlineto{\pgfqpoint{2.225661in}{0.763750in}}%
\pgfpathlineto{\pgfqpoint{2.226526in}{0.712505in}}%
\pgfpathlineto{\pgfqpoint{2.227389in}{0.764887in}}%
\pgfpathlineto{\pgfqpoint{2.228254in}{0.731186in}}%
\pgfpathlineto{\pgfqpoint{2.229120in}{0.732835in}}%
\pgfpathlineto{\pgfqpoint{2.229986in}{0.789720in}}%
\pgfpathlineto{\pgfqpoint{2.232580in}{0.727524in}}%
\pgfpathlineto{\pgfqpoint{2.234309in}{0.774518in}}%
\pgfpathlineto{\pgfqpoint{2.235174in}{0.783126in}}%
\pgfpathlineto{\pgfqpoint{2.236039in}{0.672137in}}%
\pgfpathlineto{\pgfqpoint{2.236904in}{0.728255in}}%
\pgfpathlineto{\pgfqpoint{2.237768in}{0.717925in}}%
\pgfpathlineto{\pgfqpoint{2.238633in}{0.670379in}}%
\pgfpathlineto{\pgfqpoint{2.239497in}{0.691330in}}%
\pgfpathlineto{\pgfqpoint{2.240361in}{0.667886in}}%
\pgfpathlineto{\pgfqpoint{2.242090in}{0.752576in}}%
\pgfpathlineto{\pgfqpoint{2.242955in}{0.747192in}}%
\pgfpathlineto{\pgfqpoint{2.244685in}{0.714921in}}%
\pgfpathlineto{\pgfqpoint{2.245550in}{0.757193in}}%
\pgfpathlineto{\pgfqpoint{2.246416in}{0.718437in}}%
\pgfpathlineto{\pgfqpoint{2.247281in}{0.734297in}}%
\pgfpathlineto{\pgfqpoint{2.248147in}{0.658072in}}%
\pgfpathlineto{\pgfqpoint{2.249874in}{0.769500in}}%
\pgfpathlineto{\pgfqpoint{2.251606in}{0.657889in}}%
\pgfpathlineto{\pgfqpoint{2.253336in}{0.727962in}}%
\pgfpathlineto{\pgfqpoint{2.254201in}{0.710600in}}%
\pgfpathlineto{\pgfqpoint{2.255930in}{0.733054in}}%
\pgfpathlineto{\pgfqpoint{2.257658in}{0.667963in}}%
\pgfpathlineto{\pgfqpoint{2.258523in}{0.766203in}}%
\pgfpathlineto{\pgfqpoint{2.259389in}{0.683567in}}%
\pgfpathlineto{\pgfqpoint{2.260254in}{0.693969in}}%
\pgfpathlineto{\pgfqpoint{2.261118in}{0.803493in}}%
\pgfpathlineto{\pgfqpoint{2.262848in}{0.730821in}}%
\pgfpathlineto{\pgfqpoint{2.263712in}{0.773787in}}%
\pgfpathlineto{\pgfqpoint{2.264577in}{0.730711in}}%
\pgfpathlineto{\pgfqpoint{2.265441in}{0.750676in}}%
\pgfpathlineto{\pgfqpoint{2.266306in}{0.711591in}}%
\pgfpathlineto{\pgfqpoint{2.267171in}{0.712399in}}%
\pgfpathlineto{\pgfqpoint{2.268898in}{0.720345in}}%
\pgfpathlineto{\pgfqpoint{2.269763in}{0.677525in}}%
\pgfpathlineto{\pgfqpoint{2.271490in}{0.769792in}}%
\pgfpathlineto{\pgfqpoint{2.272353in}{0.762321in}}%
\pgfpathlineto{\pgfqpoint{2.273219in}{0.779683in}}%
\pgfpathlineto{\pgfqpoint{2.274084in}{0.770015in}}%
\pgfpathlineto{\pgfqpoint{2.274949in}{0.802689in}}%
\pgfpathlineto{\pgfqpoint{2.275812in}{0.715619in}}%
\pgfpathlineto{\pgfqpoint{2.276676in}{0.735949in}}%
\pgfpathlineto{\pgfqpoint{2.277541in}{0.743643in}}%
\pgfpathlineto{\pgfqpoint{2.280137in}{0.664191in}}%
\pgfpathlineto{\pgfqpoint{2.281003in}{0.773568in}}%
\pgfpathlineto{\pgfqpoint{2.281868in}{0.652651in}}%
\pgfpathlineto{\pgfqpoint{2.283598in}{0.714446in}}%
\pgfpathlineto{\pgfqpoint{2.284462in}{0.717998in}}%
\pgfpathlineto{\pgfqpoint{2.286192in}{0.664629in}}%
\pgfpathlineto{\pgfqpoint{2.287056in}{0.671662in}}%
\pgfpathlineto{\pgfqpoint{2.287921in}{0.670013in}}%
\pgfpathlineto{\pgfqpoint{2.290518in}{0.750233in}}%
\pgfpathlineto{\pgfqpoint{2.292249in}{0.625983in}}%
\pgfpathlineto{\pgfqpoint{2.293114in}{0.734849in}}%
\pgfpathlineto{\pgfqpoint{2.293978in}{0.702650in}}%
\pgfpathlineto{\pgfqpoint{2.294842in}{0.746607in}}%
\pgfpathlineto{\pgfqpoint{2.296571in}{0.656095in}}%
\pgfpathlineto{\pgfqpoint{2.297435in}{0.777011in}}%
\pgfpathlineto{\pgfqpoint{2.298302in}{0.676900in}}%
\pgfpathlineto{\pgfqpoint{2.299167in}{0.794227in}}%
\pgfpathlineto{\pgfqpoint{2.300033in}{0.722542in}}%
\pgfpathlineto{\pgfqpoint{2.300900in}{0.779171in}}%
\pgfpathlineto{\pgfqpoint{2.301766in}{0.766129in}}%
\pgfpathlineto{\pgfqpoint{2.302632in}{0.712834in}}%
\pgfpathlineto{\pgfqpoint{2.303497in}{0.716277in}}%
\pgfpathlineto{\pgfqpoint{2.304361in}{0.735580in}}%
\pgfpathlineto{\pgfqpoint{2.305227in}{0.697522in}}%
\pgfpathlineto{\pgfqpoint{2.306092in}{0.729684in}}%
\pgfpathlineto{\pgfqpoint{2.306957in}{0.673895in}}%
\pgfpathlineto{\pgfqpoint{2.308688in}{0.733493in}}%
\pgfpathlineto{\pgfqpoint{2.310420in}{0.632650in}}%
\pgfpathlineto{\pgfqpoint{2.311285in}{0.741370in}}%
\pgfpathlineto{\pgfqpoint{2.312152in}{0.739685in}}%
\pgfpathlineto{\pgfqpoint{2.313017in}{0.690088in}}%
\pgfpathlineto{\pgfqpoint{2.313883in}{0.798401in}}%
\pgfpathlineto{\pgfqpoint{2.315613in}{0.681553in}}%
\pgfpathlineto{\pgfqpoint{2.316478in}{0.707230in}}%
\pgfpathlineto{\pgfqpoint{2.317342in}{0.736278in}}%
\pgfpathlineto{\pgfqpoint{2.318206in}{0.707669in}}%
\pgfpathlineto{\pgfqpoint{2.319072in}{0.714738in}}%
\pgfpathlineto{\pgfqpoint{2.319937in}{0.696129in}}%
\pgfpathlineto{\pgfqpoint{2.321668in}{0.776130in}}%
\pgfpathlineto{\pgfqpoint{2.323399in}{0.712907in}}%
\pgfpathlineto{\pgfqpoint{2.324264in}{0.706276in}}%
\pgfpathlineto{\pgfqpoint{2.325993in}{0.728328in}}%
\pgfpathlineto{\pgfqpoint{2.326858in}{0.719647in}}%
\pgfpathlineto{\pgfqpoint{2.327723in}{0.697559in}}%
\pgfpathlineto{\pgfqpoint{2.328589in}{0.844482in}}%
\pgfpathlineto{\pgfqpoint{2.329452in}{0.694481in}}%
\pgfpathlineto{\pgfqpoint{2.330317in}{0.737520in}}%
\pgfpathlineto{\pgfqpoint{2.332043in}{0.662429in}}%
\pgfpathlineto{\pgfqpoint{2.332908in}{0.681439in}}%
\pgfpathlineto{\pgfqpoint{2.333775in}{0.656826in}}%
\pgfpathlineto{\pgfqpoint{2.334640in}{0.790451in}}%
\pgfpathlineto{\pgfqpoint{2.335506in}{0.719720in}}%
\pgfpathlineto{\pgfqpoint{2.336370in}{0.768769in}}%
\pgfpathlineto{\pgfqpoint{2.337235in}{0.629280in}}%
\pgfpathlineto{\pgfqpoint{2.338965in}{0.738146in}}%
\pgfpathlineto{\pgfqpoint{2.339831in}{0.675178in}}%
\pgfpathlineto{\pgfqpoint{2.340692in}{0.712176in}}%
\pgfpathlineto{\pgfqpoint{2.341556in}{0.651222in}}%
\pgfpathlineto{\pgfqpoint{2.342422in}{0.697486in}}%
\pgfpathlineto{\pgfqpoint{2.343286in}{0.632285in}}%
\pgfpathlineto{\pgfqpoint{2.345015in}{0.692394in}}%
\pgfpathlineto{\pgfqpoint{2.345881in}{0.654885in}}%
\pgfpathlineto{\pgfqpoint{2.346747in}{0.718802in}}%
\pgfpathlineto{\pgfqpoint{2.347612in}{0.666241in}}%
\pgfpathlineto{\pgfqpoint{2.348476in}{0.691225in}}%
\pgfpathlineto{\pgfqpoint{2.349341in}{0.756279in}}%
\pgfpathlineto{\pgfqpoint{2.350207in}{0.651921in}}%
\pgfpathlineto{\pgfqpoint{2.351073in}{0.746827in}}%
\pgfpathlineto{\pgfqpoint{2.353668in}{0.631993in}}%
\pgfpathlineto{\pgfqpoint{2.355397in}{0.712834in}}%
\pgfpathlineto{\pgfqpoint{2.356262in}{0.712907in}}%
\pgfpathlineto{\pgfqpoint{2.357127in}{0.649720in}}%
\pgfpathlineto{\pgfqpoint{2.357993in}{0.742100in}}%
\pgfpathlineto{\pgfqpoint{2.358859in}{0.723236in}}%
\pgfpathlineto{\pgfqpoint{2.359724in}{0.686275in}}%
\pgfpathlineto{\pgfqpoint{2.361451in}{0.805653in}}%
\pgfpathlineto{\pgfqpoint{2.362316in}{0.692723in}}%
\pgfpathlineto{\pgfqpoint{2.363181in}{0.694116in}}%
\pgfpathlineto{\pgfqpoint{2.364045in}{0.783784in}}%
\pgfpathlineto{\pgfqpoint{2.364909in}{0.697413in}}%
\pgfpathlineto{\pgfqpoint{2.365771in}{0.700636in}}%
\pgfpathlineto{\pgfqpoint{2.366638in}{0.652323in}}%
\pgfpathlineto{\pgfqpoint{2.368366in}{0.777523in}}%
\pgfpathlineto{\pgfqpoint{2.369231in}{0.691846in}}%
\pgfpathlineto{\pgfqpoint{2.370096in}{0.734045in}}%
\pgfpathlineto{\pgfqpoint{2.370961in}{0.692248in}}%
\pgfpathlineto{\pgfqpoint{2.371826in}{0.761074in}}%
\pgfpathlineto{\pgfqpoint{2.373557in}{0.726533in}}%
\pgfpathlineto{\pgfqpoint{2.375287in}{0.746754in}}%
\pgfpathlineto{\pgfqpoint{2.376151in}{0.726022in}}%
\pgfpathlineto{\pgfqpoint{2.377017in}{0.751260in}}%
\pgfpathlineto{\pgfqpoint{2.377884in}{0.688951in}}%
\pgfpathlineto{\pgfqpoint{2.378749in}{0.711697in}}%
\pgfpathlineto{\pgfqpoint{2.379616in}{0.772025in}}%
\pgfpathlineto{\pgfqpoint{2.380481in}{0.728913in}}%
\pgfpathlineto{\pgfqpoint{2.381346in}{0.747960in}}%
\pgfpathlineto{\pgfqpoint{2.382212in}{0.734297in}}%
\pgfpathlineto{\pgfqpoint{2.383076in}{0.765212in}}%
\pgfpathlineto{\pgfqpoint{2.384807in}{0.669608in}}%
\pgfpathlineto{\pgfqpoint{2.385670in}{0.773783in}}%
\pgfpathlineto{\pgfqpoint{2.386535in}{0.749425in}}%
\pgfpathlineto{\pgfqpoint{2.388265in}{0.809059in}}%
\pgfpathlineto{\pgfqpoint{2.389995in}{0.712208in}}%
\pgfpathlineto{\pgfqpoint{2.390860in}{0.724369in}}%
\pgfpathlineto{\pgfqpoint{2.391725in}{0.714076in}}%
\pgfpathlineto{\pgfqpoint{2.392590in}{0.790158in}}%
\pgfpathlineto{\pgfqpoint{2.393455in}{0.706313in}}%
\pgfpathlineto{\pgfqpoint{2.394320in}{0.725762in}}%
\pgfpathlineto{\pgfqpoint{2.395187in}{0.689389in}}%
\pgfpathlineto{\pgfqpoint{2.396051in}{0.692467in}}%
\pgfpathlineto{\pgfqpoint{2.396915in}{0.772870in}}%
\pgfpathlineto{\pgfqpoint{2.397781in}{0.709317in}}%
\pgfpathlineto{\pgfqpoint{2.398647in}{0.716496in}}%
\pgfpathlineto{\pgfqpoint{2.399511in}{0.707303in}}%
\pgfpathlineto{\pgfqpoint{2.401240in}{0.748694in}}%
\pgfpathlineto{\pgfqpoint{2.402968in}{0.735766in}}%
\pgfpathlineto{\pgfqpoint{2.403834in}{0.712395in}}%
\pgfpathlineto{\pgfqpoint{2.404699in}{0.733456in}}%
\pgfpathlineto{\pgfqpoint{2.405564in}{0.651442in}}%
\pgfpathlineto{\pgfqpoint{2.406429in}{0.752357in}}%
\pgfpathlineto{\pgfqpoint{2.408157in}{0.658182in}}%
\pgfpathlineto{\pgfqpoint{2.411618in}{0.774737in}}%
\pgfpathlineto{\pgfqpoint{2.413348in}{0.695837in}}%
\pgfpathlineto{\pgfqpoint{2.414213in}{0.719501in}}%
\pgfpathlineto{\pgfqpoint{2.415079in}{0.668584in}}%
\pgfpathlineto{\pgfqpoint{2.415943in}{0.709317in}}%
\pgfpathlineto{\pgfqpoint{2.416808in}{0.669465in}}%
\pgfpathlineto{\pgfqpoint{2.417672in}{0.749758in}}%
\pgfpathlineto{\pgfqpoint{2.418538in}{0.732835in}}%
\pgfpathlineto{\pgfqpoint{2.419402in}{0.702321in}}%
\pgfpathlineto{\pgfqpoint{2.421131in}{0.754225in}}%
\pgfpathlineto{\pgfqpoint{2.421995in}{0.701733in}}%
\pgfpathlineto{\pgfqpoint{2.422861in}{0.723529in}}%
\pgfpathlineto{\pgfqpoint{2.423725in}{0.691111in}}%
\pgfpathlineto{\pgfqpoint{2.424590in}{0.692979in}}%
\pgfpathlineto{\pgfqpoint{2.425455in}{0.737301in}}%
\pgfpathlineto{\pgfqpoint{2.426320in}{0.671918in}}%
\pgfpathlineto{\pgfqpoint{2.427184in}{0.672393in}}%
\pgfpathlineto{\pgfqpoint{2.428912in}{0.754480in}}%
\pgfpathlineto{\pgfqpoint{2.430641in}{0.692394in}}%
\pgfpathlineto{\pgfqpoint{2.431505in}{0.680562in}}%
\pgfpathlineto{\pgfqpoint{2.432370in}{0.699646in}}%
\pgfpathlineto{\pgfqpoint{2.434097in}{0.681220in}}%
\pgfpathlineto{\pgfqpoint{2.435826in}{0.777117in}}%
\pgfpathlineto{\pgfqpoint{2.436690in}{0.648729in}}%
\pgfpathlineto{\pgfqpoint{2.437555in}{0.739169in}}%
\pgfpathlineto{\pgfqpoint{2.438420in}{0.685946in}}%
\pgfpathlineto{\pgfqpoint{2.439284in}{0.731844in}}%
\pgfpathlineto{\pgfqpoint{2.441015in}{0.690672in}}%
\pgfpathlineto{\pgfqpoint{2.442744in}{0.747704in}}%
\pgfpathlineto{\pgfqpoint{2.443608in}{0.671881in}}%
\pgfpathlineto{\pgfqpoint{2.446201in}{0.787560in}}%
\pgfpathlineto{\pgfqpoint{2.447066in}{0.666059in}}%
\pgfpathlineto{\pgfqpoint{2.448795in}{0.732908in}}%
\pgfpathlineto{\pgfqpoint{2.449658in}{0.737926in}}%
\pgfpathlineto{\pgfqpoint{2.450524in}{0.618293in}}%
\pgfpathlineto{\pgfqpoint{2.451388in}{0.749060in}}%
\pgfpathlineto{\pgfqpoint{2.452253in}{0.682397in}}%
\pgfpathlineto{\pgfqpoint{2.453117in}{0.733419in}}%
\pgfpathlineto{\pgfqpoint{2.453982in}{0.724812in}}%
\pgfpathlineto{\pgfqpoint{2.454848in}{0.734958in}}%
\pgfpathlineto{\pgfqpoint{2.455713in}{0.726314in}}%
\pgfpathlineto{\pgfqpoint{2.456577in}{0.793455in}}%
\pgfpathlineto{\pgfqpoint{2.457441in}{0.724994in}}%
\pgfpathlineto{\pgfqpoint{2.458306in}{0.733273in}}%
\pgfpathlineto{\pgfqpoint{2.459170in}{0.752503in}}%
\pgfpathlineto{\pgfqpoint{2.460035in}{0.716642in}}%
\pgfpathlineto{\pgfqpoint{2.462629in}{0.783784in}}%
\pgfpathlineto{\pgfqpoint{2.463494in}{0.785944in}}%
\pgfpathlineto{\pgfqpoint{2.464360in}{0.713199in}}%
\pgfpathlineto{\pgfqpoint{2.465225in}{0.733346in}}%
\pgfpathlineto{\pgfqpoint{2.466090in}{0.737594in}}%
\pgfpathlineto{\pgfqpoint{2.466956in}{0.689097in}}%
\pgfpathlineto{\pgfqpoint{2.467821in}{0.735580in}}%
\pgfpathlineto{\pgfqpoint{2.468685in}{0.728986in}}%
\pgfpathlineto{\pgfqpoint{2.469550in}{0.747631in}}%
\pgfpathlineto{\pgfqpoint{2.471277in}{0.729278in}}%
\pgfpathlineto{\pgfqpoint{2.472140in}{0.642907in}}%
\pgfpathlineto{\pgfqpoint{2.473871in}{0.723200in}}%
\pgfpathlineto{\pgfqpoint{2.474737in}{0.695764in}}%
\pgfpathlineto{\pgfqpoint{2.475602in}{0.705801in}}%
\pgfpathlineto{\pgfqpoint{2.476468in}{0.656350in}}%
\pgfpathlineto{\pgfqpoint{2.478200in}{0.775874in}}%
\pgfpathlineto{\pgfqpoint{2.479066in}{0.734995in}}%
\pgfpathlineto{\pgfqpoint{2.479931in}{0.760965in}}%
\pgfpathlineto{\pgfqpoint{2.480797in}{0.731990in}}%
\pgfpathlineto{\pgfqpoint{2.481662in}{0.746827in}}%
\pgfpathlineto{\pgfqpoint{2.482527in}{0.707815in}}%
\pgfpathlineto{\pgfqpoint{2.483391in}{0.750306in}}%
\pgfpathlineto{\pgfqpoint{2.485119in}{0.703162in}}%
\pgfpathlineto{\pgfqpoint{2.486849in}{0.783053in}}%
\pgfpathlineto{\pgfqpoint{2.487715in}{0.774957in}}%
\pgfpathlineto{\pgfqpoint{2.489447in}{0.708400in}}%
\pgfpathlineto{\pgfqpoint{2.490310in}{0.743310in}}%
\pgfpathlineto{\pgfqpoint{2.492039in}{0.702537in}}%
\pgfpathlineto{\pgfqpoint{2.492905in}{0.727374in}}%
\pgfpathlineto{\pgfqpoint{2.493769in}{0.677119in}}%
\pgfpathlineto{\pgfqpoint{2.495501in}{0.749096in}}%
\pgfpathlineto{\pgfqpoint{2.496365in}{0.682430in}}%
\pgfpathlineto{\pgfqpoint{2.497231in}{0.769719in}}%
\pgfpathlineto{\pgfqpoint{2.498096in}{0.708692in}}%
\pgfpathlineto{\pgfqpoint{2.498962in}{0.752868in}}%
\pgfpathlineto{\pgfqpoint{2.499826in}{0.671512in}}%
\pgfpathlineto{\pgfqpoint{2.500692in}{0.715213in}}%
\pgfpathlineto{\pgfqpoint{2.501555in}{0.705176in}}%
\pgfpathlineto{\pgfqpoint{2.502421in}{0.707998in}}%
\pgfpathlineto{\pgfqpoint{2.503285in}{0.639390in}}%
\pgfpathlineto{\pgfqpoint{2.504149in}{0.747704in}}%
\pgfpathlineto{\pgfqpoint{2.505014in}{0.716825in}}%
\pgfpathlineto{\pgfqpoint{2.505880in}{0.707888in}}%
\pgfpathlineto{\pgfqpoint{2.506746in}{0.684225in}}%
\pgfpathlineto{\pgfqpoint{2.508476in}{0.751878in}}%
\pgfpathlineto{\pgfqpoint{2.509340in}{0.737886in}}%
\pgfpathlineto{\pgfqpoint{2.510205in}{0.741621in}}%
\pgfpathlineto{\pgfqpoint{2.511069in}{0.635103in}}%
\pgfpathlineto{\pgfqpoint{2.511934in}{0.710487in}}%
\pgfpathlineto{\pgfqpoint{2.512798in}{0.695358in}}%
\pgfpathlineto{\pgfqpoint{2.513664in}{0.745138in}}%
\pgfpathlineto{\pgfqpoint{2.514530in}{0.725652in}}%
\pgfpathlineto{\pgfqpoint{2.515396in}{0.746348in}}%
\pgfpathlineto{\pgfqpoint{2.516260in}{0.725981in}}%
\pgfpathlineto{\pgfqpoint{2.517125in}{0.756385in}}%
\pgfpathlineto{\pgfqpoint{2.518855in}{0.660927in}}%
\pgfpathlineto{\pgfqpoint{2.519720in}{0.737520in}}%
\pgfpathlineto{\pgfqpoint{2.520585in}{0.715140in}}%
\pgfpathlineto{\pgfqpoint{2.521451in}{0.722246in}}%
\pgfpathlineto{\pgfqpoint{2.522317in}{0.714076in}}%
\pgfpathlineto{\pgfqpoint{2.524046in}{0.762540in}}%
\pgfpathlineto{\pgfqpoint{2.524912in}{0.761403in}}%
\pgfpathlineto{\pgfqpoint{2.525777in}{0.691403in}}%
\pgfpathlineto{\pgfqpoint{2.527509in}{0.753453in}}%
\pgfpathlineto{\pgfqpoint{2.528374in}{0.696605in}}%
\pgfpathlineto{\pgfqpoint{2.529236in}{0.712208in}}%
\pgfpathlineto{\pgfqpoint{2.530102in}{0.696787in}}%
\pgfpathlineto{\pgfqpoint{2.530966in}{0.734333in}}%
\pgfpathlineto{\pgfqpoint{2.531831in}{0.680010in}}%
\pgfpathlineto{\pgfqpoint{2.533562in}{0.804772in}}%
\pgfpathlineto{\pgfqpoint{2.535290in}{0.705541in}}%
\pgfpathlineto{\pgfqpoint{2.536155in}{0.667959in}}%
\pgfpathlineto{\pgfqpoint{2.537886in}{0.759426in}}%
\pgfpathlineto{\pgfqpoint{2.540482in}{0.650378in}}%
\pgfpathlineto{\pgfqpoint{2.542211in}{0.750156in}}%
\pgfpathlineto{\pgfqpoint{2.543077in}{0.679572in}}%
\pgfpathlineto{\pgfqpoint{2.543942in}{0.773893in}}%
\pgfpathlineto{\pgfqpoint{2.544808in}{0.736164in}}%
\pgfpathlineto{\pgfqpoint{2.545672in}{0.759974in}}%
\pgfpathlineto{\pgfqpoint{2.546537in}{0.684956in}}%
\pgfpathlineto{\pgfqpoint{2.547403in}{0.725579in}}%
\pgfpathlineto{\pgfqpoint{2.548266in}{0.713565in}}%
\pgfpathlineto{\pgfqpoint{2.549131in}{0.731438in}}%
\pgfpathlineto{\pgfqpoint{2.549993in}{0.666237in}}%
\pgfpathlineto{\pgfqpoint{2.551724in}{0.733744in}}%
\pgfpathlineto{\pgfqpoint{2.552590in}{0.675576in}}%
\pgfpathlineto{\pgfqpoint{2.554322in}{0.749791in}}%
\pgfpathlineto{\pgfqpoint{2.555188in}{0.710560in}}%
\pgfpathlineto{\pgfqpoint{2.556051in}{0.749718in}}%
\pgfpathlineto{\pgfqpoint{2.556916in}{0.698217in}}%
\pgfpathlineto{\pgfqpoint{2.557782in}{0.726972in}}%
\pgfpathlineto{\pgfqpoint{2.558645in}{0.688326in}}%
\pgfpathlineto{\pgfqpoint{2.559511in}{0.742576in}}%
\pgfpathlineto{\pgfqpoint{2.560377in}{0.676786in}}%
\pgfpathlineto{\pgfqpoint{2.561241in}{0.677850in}}%
\pgfpathlineto{\pgfqpoint{2.562106in}{0.698728in}}%
\pgfpathlineto{\pgfqpoint{2.562970in}{0.764920in}}%
\pgfpathlineto{\pgfqpoint{2.563836in}{0.763417in}}%
\pgfpathlineto{\pgfqpoint{2.564700in}{0.655469in}}%
\pgfpathlineto{\pgfqpoint{2.566428in}{0.736972in}}%
\pgfpathlineto{\pgfqpoint{2.568157in}{0.782501in}}%
\pgfpathlineto{\pgfqpoint{2.569022in}{0.768472in}}%
\pgfpathlineto{\pgfqpoint{2.569888in}{0.701660in}}%
\pgfpathlineto{\pgfqpoint{2.570753in}{0.761842in}}%
\pgfpathlineto{\pgfqpoint{2.571618in}{0.701733in}}%
\pgfpathlineto{\pgfqpoint{2.573346in}{0.730780in}}%
\pgfpathlineto{\pgfqpoint{2.574211in}{0.685727in}}%
\pgfpathlineto{\pgfqpoint{2.575076in}{0.748841in}}%
\pgfpathlineto{\pgfqpoint{2.575939in}{0.721734in}}%
\pgfpathlineto{\pgfqpoint{2.578530in}{0.771952in}}%
\pgfpathlineto{\pgfqpoint{2.580257in}{0.715140in}}%
\pgfpathlineto{\pgfqpoint{2.581123in}{0.763527in}}%
\pgfpathlineto{\pgfqpoint{2.582854in}{0.641916in}}%
\pgfpathlineto{\pgfqpoint{2.583719in}{0.758033in}}%
\pgfpathlineto{\pgfqpoint{2.586313in}{0.684152in}}%
\pgfpathlineto{\pgfqpoint{2.587177in}{0.670306in}}%
\pgfpathlineto{\pgfqpoint{2.589773in}{0.703162in}}%
\pgfpathlineto{\pgfqpoint{2.590637in}{0.698399in}}%
\pgfpathlineto{\pgfqpoint{2.591500in}{0.769167in}}%
\pgfpathlineto{\pgfqpoint{2.592365in}{0.748069in}}%
\pgfpathlineto{\pgfqpoint{2.593230in}{0.673603in}}%
\pgfpathlineto{\pgfqpoint{2.594095in}{0.706532in}}%
\pgfpathlineto{\pgfqpoint{2.594961in}{0.696129in}}%
\pgfpathlineto{\pgfqpoint{2.595827in}{0.711039in}}%
\pgfpathlineto{\pgfqpoint{2.596692in}{0.747923in}}%
\pgfpathlineto{\pgfqpoint{2.598421in}{0.715652in}}%
\pgfpathlineto{\pgfqpoint{2.599286in}{0.771660in}}%
\pgfpathlineto{\pgfqpoint{2.600150in}{0.758435in}}%
\pgfpathlineto{\pgfqpoint{2.601015in}{0.755540in}}%
\pgfpathlineto{\pgfqpoint{2.601880in}{0.742978in}}%
\pgfpathlineto{\pgfqpoint{2.602742in}{0.698655in}}%
\pgfpathlineto{\pgfqpoint{2.604469in}{0.733050in}}%
\pgfpathlineto{\pgfqpoint{2.605334in}{0.713638in}}%
\pgfpathlineto{\pgfqpoint{2.606197in}{0.726566in}}%
\pgfpathlineto{\pgfqpoint{2.607063in}{0.664333in}}%
\pgfpathlineto{\pgfqpoint{2.607928in}{0.767080in}}%
\pgfpathlineto{\pgfqpoint{2.608793in}{0.745576in}}%
\pgfpathlineto{\pgfqpoint{2.609657in}{0.672502in}}%
\pgfpathlineto{\pgfqpoint{2.611387in}{0.771952in}}%
\pgfpathlineto{\pgfqpoint{2.614846in}{0.717227in}}%
\pgfpathlineto{\pgfqpoint{2.615710in}{0.714994in}}%
\pgfpathlineto{\pgfqpoint{2.616574in}{0.680306in}}%
\pgfpathlineto{\pgfqpoint{2.617439in}{0.754736in}}%
\pgfpathlineto{\pgfqpoint{2.618301in}{0.737447in}}%
\pgfpathlineto{\pgfqpoint{2.619166in}{0.680855in}}%
\pgfpathlineto{\pgfqpoint{2.620894in}{0.754663in}}%
\pgfpathlineto{\pgfqpoint{2.621760in}{0.756019in}}%
\pgfpathlineto{\pgfqpoint{2.624354in}{0.678873in}}%
\pgfpathlineto{\pgfqpoint{2.625219in}{0.688910in}}%
\pgfpathlineto{\pgfqpoint{2.626950in}{0.725981in}}%
\pgfpathlineto{\pgfqpoint{2.627816in}{0.753088in}}%
\pgfpathlineto{\pgfqpoint{2.628682in}{0.720816in}}%
\pgfpathlineto{\pgfqpoint{2.629548in}{0.734370in}}%
\pgfpathlineto{\pgfqpoint{2.630413in}{0.729643in}}%
\pgfpathlineto{\pgfqpoint{2.632145in}{0.653967in}}%
\pgfpathlineto{\pgfqpoint{2.634739in}{0.764846in}}%
\pgfpathlineto{\pgfqpoint{2.635603in}{0.695139in}}%
\pgfpathlineto{\pgfqpoint{2.636470in}{0.709975in}}%
\pgfpathlineto{\pgfqpoint{2.637336in}{0.827960in}}%
\pgfpathlineto{\pgfqpoint{2.639064in}{0.727264in}}%
\pgfpathlineto{\pgfqpoint{2.639929in}{0.722319in}}%
\pgfpathlineto{\pgfqpoint{2.641660in}{0.789574in}}%
\pgfpathlineto{\pgfqpoint{2.642524in}{0.721551in}}%
\pgfpathlineto{\pgfqpoint{2.643388in}{0.757595in}}%
\pgfpathlineto{\pgfqpoint{2.644252in}{0.673310in}}%
\pgfpathlineto{\pgfqpoint{2.645115in}{0.675397in}}%
\pgfpathlineto{\pgfqpoint{2.646844in}{0.731479in}}%
\pgfpathlineto{\pgfqpoint{2.647710in}{0.699244in}}%
\pgfpathlineto{\pgfqpoint{2.648575in}{0.748548in}}%
\pgfpathlineto{\pgfqpoint{2.649441in}{0.701294in}}%
\pgfpathlineto{\pgfqpoint{2.650307in}{0.772723in}}%
\pgfpathlineto{\pgfqpoint{2.651172in}{0.682430in}}%
\pgfpathlineto{\pgfqpoint{2.652902in}{0.749060in}}%
\pgfpathlineto{\pgfqpoint{2.653767in}{0.723748in}}%
\pgfpathlineto{\pgfqpoint{2.654633in}{0.663968in}}%
\pgfpathlineto{\pgfqpoint{2.655499in}{0.762906in}}%
\pgfpathlineto{\pgfqpoint{2.656365in}{0.732283in}}%
\pgfpathlineto{\pgfqpoint{2.657231in}{0.780085in}}%
\pgfpathlineto{\pgfqpoint{2.658097in}{0.710048in}}%
\pgfpathlineto{\pgfqpoint{2.658961in}{0.824371in}}%
\pgfpathlineto{\pgfqpoint{2.660689in}{0.687302in}}%
\pgfpathlineto{\pgfqpoint{2.661555in}{0.704116in}}%
\pgfpathlineto{\pgfqpoint{2.662417in}{0.765691in}}%
\pgfpathlineto{\pgfqpoint{2.664149in}{0.706240in}}%
\pgfpathlineto{\pgfqpoint{2.665878in}{0.746275in}}%
\pgfpathlineto{\pgfqpoint{2.667607in}{0.699938in}}%
\pgfpathlineto{\pgfqpoint{2.668473in}{0.736716in}}%
\pgfpathlineto{\pgfqpoint{2.669338in}{0.670087in}}%
\pgfpathlineto{\pgfqpoint{2.670204in}{0.762906in}}%
\pgfpathlineto{\pgfqpoint{2.671069in}{0.685398in}}%
\pgfpathlineto{\pgfqpoint{2.671934in}{0.763271in}}%
\pgfpathlineto{\pgfqpoint{2.672800in}{0.713638in}}%
\pgfpathlineto{\pgfqpoint{2.674531in}{0.775834in}}%
\pgfpathlineto{\pgfqpoint{2.676261in}{0.712537in}}%
\pgfpathlineto{\pgfqpoint{2.677125in}{0.812685in}}%
\pgfpathlineto{\pgfqpoint{2.678857in}{0.706751in}}%
\pgfpathlineto{\pgfqpoint{2.681452in}{0.808877in}}%
\pgfpathlineto{\pgfqpoint{2.682316in}{0.720378in}}%
\pgfpathlineto{\pgfqpoint{2.683182in}{0.731954in}}%
\pgfpathlineto{\pgfqpoint{2.684047in}{0.753672in}}%
\pgfpathlineto{\pgfqpoint{2.684912in}{0.732502in}}%
\pgfpathlineto{\pgfqpoint{2.685777in}{0.739132in}}%
\pgfpathlineto{\pgfqpoint{2.686642in}{0.711551in}}%
\pgfpathlineto{\pgfqpoint{2.687507in}{0.779829in}}%
\pgfpathlineto{\pgfqpoint{2.688372in}{0.721807in}}%
\pgfpathlineto{\pgfqpoint{2.689237in}{0.730780in}}%
\pgfpathlineto{\pgfqpoint{2.690102in}{0.735616in}}%
\pgfpathlineto{\pgfqpoint{2.690968in}{0.717227in}}%
\pgfpathlineto{\pgfqpoint{2.691834in}{0.810671in}}%
\pgfpathlineto{\pgfqpoint{2.694429in}{0.727483in}}%
\pgfpathlineto{\pgfqpoint{2.695292in}{0.684773in}}%
\pgfpathlineto{\pgfqpoint{2.696157in}{0.690892in}}%
\pgfpathlineto{\pgfqpoint{2.697023in}{0.739315in}}%
\pgfpathlineto{\pgfqpoint{2.697889in}{0.671037in}}%
\pgfpathlineto{\pgfqpoint{2.698752in}{0.735872in}}%
\pgfpathlineto{\pgfqpoint{2.699618in}{0.718766in}}%
\pgfpathlineto{\pgfqpoint{2.700483in}{0.686385in}}%
\pgfpathlineto{\pgfqpoint{2.701346in}{0.691842in}}%
\pgfpathlineto{\pgfqpoint{2.702209in}{0.705359in}}%
\pgfpathlineto{\pgfqpoint{2.703074in}{0.689755in}}%
\pgfpathlineto{\pgfqpoint{2.703940in}{0.739681in}}%
\pgfpathlineto{\pgfqpoint{2.704806in}{0.697413in}}%
\pgfpathlineto{\pgfqpoint{2.705670in}{0.768655in}}%
\pgfpathlineto{\pgfqpoint{2.706536in}{0.754517in}}%
\pgfpathlineto{\pgfqpoint{2.707401in}{0.726972in}}%
\pgfpathlineto{\pgfqpoint{2.709134in}{0.764518in}}%
\pgfpathlineto{\pgfqpoint{2.709995in}{0.737374in}}%
\pgfpathlineto{\pgfqpoint{2.710861in}{0.746128in}}%
\pgfpathlineto{\pgfqpoint{2.712592in}{0.687814in}}%
\pgfpathlineto{\pgfqpoint{2.713458in}{0.782355in}}%
\pgfpathlineto{\pgfqpoint{2.714324in}{0.713857in}}%
\pgfpathlineto{\pgfqpoint{2.715189in}{0.756969in}}%
\pgfpathlineto{\pgfqpoint{2.716054in}{0.738146in}}%
\pgfpathlineto{\pgfqpoint{2.716920in}{0.796826in}}%
\pgfpathlineto{\pgfqpoint{2.717785in}{0.729059in}}%
\pgfpathlineto{\pgfqpoint{2.718650in}{0.752357in}}%
\pgfpathlineto{\pgfqpoint{2.719515in}{0.662502in}}%
\pgfpathlineto{\pgfqpoint{2.722110in}{0.768363in}}%
\pgfpathlineto{\pgfqpoint{2.722975in}{0.736164in}}%
\pgfpathlineto{\pgfqpoint{2.723839in}{0.753896in}}%
\pgfpathlineto{\pgfqpoint{2.724705in}{0.713857in}}%
\pgfpathlineto{\pgfqpoint{2.726436in}{0.753672in}}%
\pgfpathlineto{\pgfqpoint{2.728166in}{0.700121in}}%
\pgfpathlineto{\pgfqpoint{2.729031in}{0.701587in}}%
\pgfpathlineto{\pgfqpoint{2.729896in}{0.778765in}}%
\pgfpathlineto{\pgfqpoint{2.730761in}{0.679060in}}%
\pgfpathlineto{\pgfqpoint{2.732492in}{0.727191in}}%
\pgfpathlineto{\pgfqpoint{2.733357in}{0.733858in}}%
\pgfpathlineto{\pgfqpoint{2.735086in}{0.794406in}}%
\pgfpathlineto{\pgfqpoint{2.735950in}{0.715359in}}%
\pgfpathlineto{\pgfqpoint{2.736813in}{0.748768in}}%
\pgfpathlineto{\pgfqpoint{2.737677in}{0.691038in}}%
\pgfpathlineto{\pgfqpoint{2.739404in}{0.782687in}}%
\pgfpathlineto{\pgfqpoint{2.740269in}{0.688183in}}%
\pgfpathlineto{\pgfqpoint{2.741132in}{0.697632in}}%
\pgfpathlineto{\pgfqpoint{2.741997in}{0.692248in}}%
\pgfpathlineto{\pgfqpoint{2.742860in}{0.707669in}}%
\pgfpathlineto{\pgfqpoint{2.743725in}{0.818215in}}%
\pgfpathlineto{\pgfqpoint{2.745455in}{0.693896in}}%
\pgfpathlineto{\pgfqpoint{2.746322in}{0.723967in}}%
\pgfpathlineto{\pgfqpoint{2.747187in}{0.717154in}}%
\pgfpathlineto{\pgfqpoint{2.748918in}{0.766568in}}%
\pgfpathlineto{\pgfqpoint{2.749784in}{0.746936in}}%
\pgfpathlineto{\pgfqpoint{2.750649in}{0.734077in}}%
\pgfpathlineto{\pgfqpoint{2.752380in}{0.766495in}}%
\pgfpathlineto{\pgfqpoint{2.753245in}{0.758179in}}%
\pgfpathlineto{\pgfqpoint{2.754976in}{0.682576in}}%
\pgfpathlineto{\pgfqpoint{2.755840in}{0.732283in}}%
\pgfpathlineto{\pgfqpoint{2.756706in}{0.696422in}}%
\pgfpathlineto{\pgfqpoint{2.758437in}{0.800853in}}%
\pgfpathlineto{\pgfqpoint{2.761898in}{0.648729in}}%
\pgfpathlineto{\pgfqpoint{2.763629in}{0.736790in}}%
\pgfpathlineto{\pgfqpoint{2.764491in}{0.693385in}}%
\pgfpathlineto{\pgfqpoint{2.765356in}{0.731479in}}%
\pgfpathlineto{\pgfqpoint{2.766220in}{0.680051in}}%
\pgfpathlineto{\pgfqpoint{2.767951in}{0.769134in}}%
\pgfpathlineto{\pgfqpoint{2.768815in}{0.769426in}}%
\pgfpathlineto{\pgfqpoint{2.769681in}{0.728109in}}%
\pgfpathlineto{\pgfqpoint{2.770546in}{0.774006in}}%
\pgfpathlineto{\pgfqpoint{2.772275in}{0.705143in}}%
\pgfpathlineto{\pgfqpoint{2.773139in}{0.684078in}}%
\pgfpathlineto{\pgfqpoint{2.774004in}{0.708546in}}%
\pgfpathlineto{\pgfqpoint{2.774866in}{0.678914in}}%
\pgfpathlineto{\pgfqpoint{2.775728in}{0.709866in}}%
\pgfpathlineto{\pgfqpoint{2.776593in}{0.700450in}}%
\pgfpathlineto{\pgfqpoint{2.777458in}{0.749645in}}%
\pgfpathlineto{\pgfqpoint{2.778324in}{0.727191in}}%
\pgfpathlineto{\pgfqpoint{2.779188in}{0.763344in}}%
\pgfpathlineto{\pgfqpoint{2.781780in}{0.717519in}}%
\pgfpathlineto{\pgfqpoint{2.782645in}{0.790962in}}%
\pgfpathlineto{\pgfqpoint{2.783507in}{0.720597in}}%
\pgfpathlineto{\pgfqpoint{2.784371in}{0.735945in}}%
\pgfpathlineto{\pgfqpoint{2.785235in}{0.701111in}}%
\pgfpathlineto{\pgfqpoint{2.786101in}{0.753932in}}%
\pgfpathlineto{\pgfqpoint{2.786965in}{0.732867in}}%
\pgfpathlineto{\pgfqpoint{2.787828in}{0.756312in}}%
\pgfpathlineto{\pgfqpoint{2.788692in}{0.741841in}}%
\pgfpathlineto{\pgfqpoint{2.789556in}{0.786017in}}%
\pgfpathlineto{\pgfqpoint{2.790421in}{0.648949in}}%
\pgfpathlineto{\pgfqpoint{2.793017in}{0.756092in}}%
\pgfpathlineto{\pgfqpoint{2.793883in}{0.701879in}}%
\pgfpathlineto{\pgfqpoint{2.794748in}{0.736311in}}%
\pgfpathlineto{\pgfqpoint{2.795613in}{0.702391in}}%
\pgfpathlineto{\pgfqpoint{2.796478in}{0.710487in}}%
\pgfpathlineto{\pgfqpoint{2.797343in}{0.779240in}}%
\pgfpathlineto{\pgfqpoint{2.798208in}{0.726200in}}%
\pgfpathlineto{\pgfqpoint{2.799072in}{0.728766in}}%
\pgfpathlineto{\pgfqpoint{2.799938in}{0.705286in}}%
\pgfpathlineto{\pgfqpoint{2.800803in}{0.718802in}}%
\pgfpathlineto{\pgfqpoint{2.802530in}{0.681585in}}%
\pgfpathlineto{\pgfqpoint{2.803395in}{0.789972in}}%
\pgfpathlineto{\pgfqpoint{2.804259in}{0.694920in}}%
\pgfpathlineto{\pgfqpoint{2.805124in}{0.743895in}}%
\pgfpathlineto{\pgfqpoint{2.805989in}{0.737740in}}%
\pgfpathlineto{\pgfqpoint{2.806856in}{0.701185in}}%
\pgfpathlineto{\pgfqpoint{2.808586in}{0.747079in}}%
\pgfpathlineto{\pgfqpoint{2.809451in}{0.683344in}}%
\pgfpathlineto{\pgfqpoint{2.810316in}{0.734370in}}%
\pgfpathlineto{\pgfqpoint{2.811181in}{0.731877in}}%
\pgfpathlineto{\pgfqpoint{2.812047in}{0.708985in}}%
\pgfpathlineto{\pgfqpoint{2.812912in}{0.710341in}}%
\pgfpathlineto{\pgfqpoint{2.813775in}{0.738142in}}%
\pgfpathlineto{\pgfqpoint{2.814641in}{0.624335in}}%
\pgfpathlineto{\pgfqpoint{2.815503in}{0.754371in}}%
\pgfpathlineto{\pgfqpoint{2.816368in}{0.751220in}}%
\pgfpathlineto{\pgfqpoint{2.818096in}{0.706167in}}%
\pgfpathlineto{\pgfqpoint{2.818961in}{0.714336in}}%
\pgfpathlineto{\pgfqpoint{2.819827in}{0.703787in}}%
\pgfpathlineto{\pgfqpoint{2.822421in}{0.765910in}}%
\pgfpathlineto{\pgfqpoint{2.823283in}{0.712208in}}%
\pgfpathlineto{\pgfqpoint{2.824146in}{0.798839in}}%
\pgfpathlineto{\pgfqpoint{2.825873in}{0.702212in}}%
\pgfpathlineto{\pgfqpoint{2.827603in}{0.724738in}}%
\pgfpathlineto{\pgfqpoint{2.828468in}{0.691773in}}%
\pgfpathlineto{\pgfqpoint{2.829334in}{0.739940in}}%
\pgfpathlineto{\pgfqpoint{2.830199in}{0.739169in}}%
\pgfpathlineto{\pgfqpoint{2.831064in}{0.672539in}}%
\pgfpathlineto{\pgfqpoint{2.831927in}{0.783857in}}%
\pgfpathlineto{\pgfqpoint{2.833654in}{0.657045in}}%
\pgfpathlineto{\pgfqpoint{2.834521in}{0.778400in}}%
\pgfpathlineto{\pgfqpoint{2.836252in}{0.708692in}}%
\pgfpathlineto{\pgfqpoint{2.837117in}{0.764002in}}%
\pgfpathlineto{\pgfqpoint{2.838847in}{0.692207in}}%
\pgfpathlineto{\pgfqpoint{2.839712in}{0.747558in}}%
\pgfpathlineto{\pgfqpoint{2.841442in}{0.702577in}}%
\pgfpathlineto{\pgfqpoint{2.842307in}{0.675215in}}%
\pgfpathlineto{\pgfqpoint{2.843172in}{0.697266in}}%
\pgfpathlineto{\pgfqpoint{2.844037in}{0.663127in}}%
\pgfpathlineto{\pgfqpoint{2.845764in}{0.738584in}}%
\pgfpathlineto{\pgfqpoint{2.847492in}{0.682247in}}%
\pgfpathlineto{\pgfqpoint{2.848356in}{0.735653in}}%
\pgfpathlineto{\pgfqpoint{2.849221in}{0.733931in}}%
\pgfpathlineto{\pgfqpoint{2.850086in}{0.717154in}}%
\pgfpathlineto{\pgfqpoint{2.850950in}{0.665287in}}%
\pgfpathlineto{\pgfqpoint{2.851815in}{0.688841in}}%
\pgfpathlineto{\pgfqpoint{2.852680in}{0.653455in}}%
\pgfpathlineto{\pgfqpoint{2.855272in}{0.716642in}}%
\pgfpathlineto{\pgfqpoint{2.856137in}{0.698217in}}%
\pgfpathlineto{\pgfqpoint{2.857003in}{0.655506in}}%
\pgfpathlineto{\pgfqpoint{2.857868in}{0.747484in}}%
\pgfpathlineto{\pgfqpoint{2.858734in}{0.683234in}}%
\pgfpathlineto{\pgfqpoint{2.859599in}{0.757156in}}%
\pgfpathlineto{\pgfqpoint{2.860465in}{0.711624in}}%
\pgfpathlineto{\pgfqpoint{2.861329in}{0.765764in}}%
\pgfpathlineto{\pgfqpoint{2.862193in}{0.705103in}}%
\pgfpathlineto{\pgfqpoint{2.863056in}{0.730488in}}%
\pgfpathlineto{\pgfqpoint{2.863921in}{0.680672in}}%
\pgfpathlineto{\pgfqpoint{2.864787in}{0.763417in}}%
\pgfpathlineto{\pgfqpoint{2.865652in}{0.744553in}}%
\pgfpathlineto{\pgfqpoint{2.866517in}{0.744261in}}%
\pgfpathlineto{\pgfqpoint{2.867382in}{0.706605in}}%
\pgfpathlineto{\pgfqpoint{2.868246in}{0.739315in}}%
\pgfpathlineto{\pgfqpoint{2.869111in}{0.732210in}}%
\pgfpathlineto{\pgfqpoint{2.870840in}{0.679827in}}%
\pgfpathlineto{\pgfqpoint{2.872568in}{0.753015in}}%
\pgfpathlineto{\pgfqpoint{2.873433in}{0.691001in}}%
\pgfpathlineto{\pgfqpoint{2.875161in}{0.760819in}}%
\pgfpathlineto{\pgfqpoint{2.876025in}{0.683051in}}%
\pgfpathlineto{\pgfqpoint{2.876888in}{0.706751in}}%
\pgfpathlineto{\pgfqpoint{2.877753in}{0.763198in}}%
\pgfpathlineto{\pgfqpoint{2.879484in}{0.687375in}}%
\pgfpathlineto{\pgfqpoint{2.881212in}{0.775103in}}%
\pgfpathlineto{\pgfqpoint{2.882941in}{0.711953in}}%
\pgfpathlineto{\pgfqpoint{2.883803in}{0.740525in}}%
\pgfpathlineto{\pgfqpoint{2.884668in}{0.735872in}}%
\pgfpathlineto{\pgfqpoint{2.885532in}{0.649903in}}%
\pgfpathlineto{\pgfqpoint{2.886397in}{0.786277in}}%
\pgfpathlineto{\pgfqpoint{2.888127in}{0.703820in}}%
\pgfpathlineto{\pgfqpoint{2.888993in}{0.687116in}}%
\pgfpathlineto{\pgfqpoint{2.889857in}{0.699605in}}%
\pgfpathlineto{\pgfqpoint{2.891587in}{0.779569in}}%
\pgfpathlineto{\pgfqpoint{2.893318in}{0.731584in}}%
\pgfpathlineto{\pgfqpoint{2.894183in}{0.735799in}}%
\pgfpathlineto{\pgfqpoint{2.895047in}{0.732356in}}%
\pgfpathlineto{\pgfqpoint{2.895913in}{0.740452in}}%
\pgfpathlineto{\pgfqpoint{2.898505in}{0.658328in}}%
\pgfpathlineto{\pgfqpoint{2.899368in}{0.756019in}}%
\pgfpathlineto{\pgfqpoint{2.900234in}{0.630783in}}%
\pgfpathlineto{\pgfqpoint{2.901099in}{0.740013in}}%
\pgfpathlineto{\pgfqpoint{2.901964in}{0.727780in}}%
\pgfpathlineto{\pgfqpoint{2.902827in}{0.685215in}}%
\pgfpathlineto{\pgfqpoint{2.905420in}{0.759828in}}%
\pgfpathlineto{\pgfqpoint{2.906286in}{0.675836in}}%
\pgfpathlineto{\pgfqpoint{2.907152in}{0.694042in}}%
\pgfpathlineto{\pgfqpoint{2.908881in}{0.756791in}}%
\pgfpathlineto{\pgfqpoint{2.909746in}{0.702650in}}%
\pgfpathlineto{\pgfqpoint{2.911476in}{0.776861in}}%
\pgfpathlineto{\pgfqpoint{2.914071in}{0.695874in}}%
\pgfpathlineto{\pgfqpoint{2.914936in}{0.712687in}}%
\pgfpathlineto{\pgfqpoint{2.915801in}{0.741881in}}%
\pgfpathlineto{\pgfqpoint{2.916667in}{0.714336in}}%
\pgfpathlineto{\pgfqpoint{2.917532in}{0.791003in}}%
\pgfpathlineto{\pgfqpoint{2.918398in}{0.698951in}}%
\pgfpathlineto{\pgfqpoint{2.919262in}{0.726022in}}%
\pgfpathlineto{\pgfqpoint{2.920993in}{0.709281in}}%
\pgfpathlineto{\pgfqpoint{2.921858in}{0.743384in}}%
\pgfpathlineto{\pgfqpoint{2.922722in}{0.696129in}}%
\pgfpathlineto{\pgfqpoint{2.923588in}{0.798766in}}%
\pgfpathlineto{\pgfqpoint{2.924453in}{0.775322in}}%
\pgfpathlineto{\pgfqpoint{2.926183in}{0.675836in}}%
\pgfpathlineto{\pgfqpoint{2.927046in}{0.792246in}}%
\pgfpathlineto{\pgfqpoint{2.927911in}{0.750562in}}%
\pgfpathlineto{\pgfqpoint{2.928776in}{0.769207in}}%
\pgfpathlineto{\pgfqpoint{2.929642in}{0.721880in}}%
\pgfpathlineto{\pgfqpoint{2.931370in}{0.798035in}}%
\pgfpathlineto{\pgfqpoint{2.932235in}{0.770709in}}%
\pgfpathlineto{\pgfqpoint{2.933966in}{0.707669in}}%
\pgfpathlineto{\pgfqpoint{2.935694in}{0.745836in}}%
\pgfpathlineto{\pgfqpoint{2.936558in}{0.754225in}}%
\pgfpathlineto{\pgfqpoint{2.937423in}{0.733383in}}%
\pgfpathlineto{\pgfqpoint{2.939153in}{0.754371in}}%
\pgfpathlineto{\pgfqpoint{2.940017in}{0.721697in}}%
\pgfpathlineto{\pgfqpoint{2.941744in}{0.775322in}}%
\pgfpathlineto{\pgfqpoint{2.942610in}{0.718766in}}%
\pgfpathlineto{\pgfqpoint{2.943475in}{0.728839in}}%
\pgfpathlineto{\pgfqpoint{2.944340in}{0.707596in}}%
\pgfpathlineto{\pgfqpoint{2.945203in}{0.778985in}}%
\pgfpathlineto{\pgfqpoint{2.946068in}{0.774551in}}%
\pgfpathlineto{\pgfqpoint{2.946935in}{0.742868in}}%
\pgfpathlineto{\pgfqpoint{2.947799in}{0.774551in}}%
\pgfpathlineto{\pgfqpoint{2.948660in}{0.692865in}}%
\pgfpathlineto{\pgfqpoint{2.950389in}{0.740192in}}%
\pgfpathlineto{\pgfqpoint{2.951255in}{0.702756in}}%
\pgfpathlineto{\pgfqpoint{2.952121in}{0.717921in}}%
\pgfpathlineto{\pgfqpoint{2.952984in}{0.708838in}}%
\pgfpathlineto{\pgfqpoint{2.953848in}{0.665945in}}%
\pgfpathlineto{\pgfqpoint{2.955575in}{0.740265in}}%
\pgfpathlineto{\pgfqpoint{2.956440in}{0.738178in}}%
\pgfpathlineto{\pgfqpoint{2.957306in}{0.687116in}}%
\pgfpathlineto{\pgfqpoint{2.958170in}{0.762719in}}%
\pgfpathlineto{\pgfqpoint{2.959036in}{0.659534in}}%
\pgfpathlineto{\pgfqpoint{2.961632in}{0.744809in}}%
\pgfpathlineto{\pgfqpoint{2.963362in}{0.695724in}}%
\pgfpathlineto{\pgfqpoint{2.965093in}{0.763856in}}%
\pgfpathlineto{\pgfqpoint{2.966821in}{0.705761in}}%
\pgfpathlineto{\pgfqpoint{2.967687in}{0.725981in}}%
\pgfpathlineto{\pgfqpoint{2.968551in}{0.719351in}}%
\pgfpathlineto{\pgfqpoint{2.969413in}{0.686604in}}%
\pgfpathlineto{\pgfqpoint{2.970278in}{0.714588in}}%
\pgfpathlineto{\pgfqpoint{2.971143in}{0.781108in}}%
\pgfpathlineto{\pgfqpoint{2.972009in}{0.760340in}}%
\pgfpathlineto{\pgfqpoint{2.972874in}{0.760851in}}%
\pgfpathlineto{\pgfqpoint{2.973738in}{0.789972in}}%
\pgfpathlineto{\pgfqpoint{2.975468in}{0.698765in}}%
\pgfpathlineto{\pgfqpoint{2.976333in}{0.751805in}}%
\pgfpathlineto{\pgfqpoint{2.977198in}{0.731730in}}%
\pgfpathlineto{\pgfqpoint{2.978062in}{0.761988in}}%
\pgfpathlineto{\pgfqpoint{2.979791in}{0.692759in}}%
\pgfpathlineto{\pgfqpoint{2.980655in}{0.696495in}}%
\pgfpathlineto{\pgfqpoint{2.982384in}{0.762353in}}%
\pgfpathlineto{\pgfqpoint{2.983249in}{0.658839in}}%
\pgfpathlineto{\pgfqpoint{2.984114in}{0.780341in}}%
\pgfpathlineto{\pgfqpoint{2.984979in}{0.689938in}}%
\pgfpathlineto{\pgfqpoint{2.985844in}{0.742978in}}%
\pgfpathlineto{\pgfqpoint{2.986710in}{0.606607in}}%
\pgfpathlineto{\pgfqpoint{2.989305in}{0.744114in}}%
\pgfpathlineto{\pgfqpoint{2.990170in}{0.744224in}}%
\pgfpathlineto{\pgfqpoint{2.991034in}{0.794479in}}%
\pgfpathlineto{\pgfqpoint{2.991898in}{0.698436in}}%
\pgfpathlineto{\pgfqpoint{2.992763in}{0.753234in}}%
\pgfpathlineto{\pgfqpoint{2.993628in}{0.747631in}}%
\pgfpathlineto{\pgfqpoint{2.994494in}{0.741296in}}%
\pgfpathlineto{\pgfqpoint{2.995360in}{0.846751in}}%
\pgfpathlineto{\pgfqpoint{2.997090in}{0.709646in}}%
\pgfpathlineto{\pgfqpoint{2.997954in}{0.775030in}}%
\pgfpathlineto{\pgfqpoint{2.998820in}{0.763052in}}%
\pgfpathlineto{\pgfqpoint{2.999686in}{0.708436in}}%
\pgfpathlineto{\pgfqpoint{3.000551in}{0.734150in}}%
\pgfpathlineto{\pgfqpoint{3.001416in}{0.716642in}}%
\pgfpathlineto{\pgfqpoint{3.002282in}{0.719939in}}%
\pgfpathlineto{\pgfqpoint{3.003146in}{0.726899in}}%
\pgfpathlineto{\pgfqpoint{3.004010in}{0.702431in}}%
\pgfpathlineto{\pgfqpoint{3.005740in}{0.766568in}}%
\pgfpathlineto{\pgfqpoint{3.008335in}{0.709975in}}%
\pgfpathlineto{\pgfqpoint{3.009201in}{0.738475in}}%
\pgfpathlineto{\pgfqpoint{3.010065in}{0.698363in}}%
\pgfpathlineto{\pgfqpoint{3.010930in}{0.732429in}}%
\pgfpathlineto{\pgfqpoint{3.011793in}{0.691330in}}%
\pgfpathlineto{\pgfqpoint{3.013525in}{0.718181in}}%
\pgfpathlineto{\pgfqpoint{3.014390in}{0.708254in}}%
\pgfpathlineto{\pgfqpoint{3.015255in}{0.752868in}}%
\pgfpathlineto{\pgfqpoint{3.016120in}{0.699938in}}%
\pgfpathlineto{\pgfqpoint{3.016985in}{0.705176in}}%
\pgfpathlineto{\pgfqpoint{3.019578in}{0.795762in}}%
\pgfpathlineto{\pgfqpoint{3.021309in}{0.701075in}}%
\pgfpathlineto{\pgfqpoint{3.023036in}{0.764664in}}%
\pgfpathlineto{\pgfqpoint{3.023900in}{0.721515in}}%
\pgfpathlineto{\pgfqpoint{3.024765in}{0.726826in}}%
\pgfpathlineto{\pgfqpoint{3.025630in}{0.729574in}}%
\pgfpathlineto{\pgfqpoint{3.026495in}{0.787852in}}%
\pgfpathlineto{\pgfqpoint{3.027361in}{0.786350in}}%
\pgfpathlineto{\pgfqpoint{3.028227in}{0.753859in}}%
\pgfpathlineto{\pgfqpoint{3.029093in}{0.776824in}}%
\pgfpathlineto{\pgfqpoint{3.029959in}{0.702504in}}%
\pgfpathlineto{\pgfqpoint{3.030824in}{0.740525in}}%
\pgfpathlineto{\pgfqpoint{3.031687in}{0.733785in}}%
\pgfpathlineto{\pgfqpoint{3.032552in}{0.726899in}}%
\pgfpathlineto{\pgfqpoint{3.033415in}{0.696349in}}%
\pgfpathlineto{\pgfqpoint{3.034281in}{0.818694in}}%
\pgfpathlineto{\pgfqpoint{3.035146in}{0.714299in}}%
\pgfpathlineto{\pgfqpoint{3.036011in}{0.853491in}}%
\pgfpathlineto{\pgfqpoint{3.038605in}{0.687814in}}%
\pgfpathlineto{\pgfqpoint{3.039470in}{0.753640in}}%
\pgfpathlineto{\pgfqpoint{3.040334in}{0.700271in}}%
\pgfpathlineto{\pgfqpoint{3.041199in}{0.777669in}}%
\pgfpathlineto{\pgfqpoint{3.042064in}{0.729538in}}%
\pgfpathlineto{\pgfqpoint{3.042930in}{0.782541in}}%
\pgfpathlineto{\pgfqpoint{3.044659in}{0.740415in}}%
\pgfpathlineto{\pgfqpoint{3.046388in}{0.727889in}}%
\pgfpathlineto{\pgfqpoint{3.047252in}{0.795214in}}%
\pgfpathlineto{\pgfqpoint{3.048983in}{0.719208in}}%
\pgfpathlineto{\pgfqpoint{3.049848in}{0.713970in}}%
\pgfpathlineto{\pgfqpoint{3.050713in}{0.796168in}}%
\pgfpathlineto{\pgfqpoint{3.051578in}{0.781295in}}%
\pgfpathlineto{\pgfqpoint{3.052443in}{0.726022in}}%
\pgfpathlineto{\pgfqpoint{3.053308in}{0.777303in}}%
\pgfpathlineto{\pgfqpoint{3.055037in}{0.686644in}}%
\pgfpathlineto{\pgfqpoint{3.055901in}{0.746827in}}%
\pgfpathlineto{\pgfqpoint{3.056766in}{0.612470in}}%
\pgfpathlineto{\pgfqpoint{3.058495in}{0.827229in}}%
\pgfpathlineto{\pgfqpoint{3.059359in}{0.681845in}}%
\pgfpathlineto{\pgfqpoint{3.060224in}{0.810817in}}%
\pgfpathlineto{\pgfqpoint{3.061088in}{0.732502in}}%
\pgfpathlineto{\pgfqpoint{3.061953in}{0.744041in}}%
\pgfpathlineto{\pgfqpoint{3.062818in}{0.760453in}}%
\pgfpathlineto{\pgfqpoint{3.063682in}{0.740525in}}%
\pgfpathlineto{\pgfqpoint{3.064545in}{0.743603in}}%
\pgfpathlineto{\pgfqpoint{3.065410in}{0.728255in}}%
\pgfpathlineto{\pgfqpoint{3.066275in}{0.678366in}}%
\pgfpathlineto{\pgfqpoint{3.067139in}{0.809940in}}%
\pgfpathlineto{\pgfqpoint{3.068870in}{0.702797in}}%
\pgfpathlineto{\pgfqpoint{3.069735in}{0.733054in}}%
\pgfpathlineto{\pgfqpoint{3.070600in}{0.717816in}}%
\pgfpathlineto{\pgfqpoint{3.072330in}{0.745397in}}%
\pgfpathlineto{\pgfqpoint{3.073195in}{0.712103in}}%
\pgfpathlineto{\pgfqpoint{3.074926in}{0.733566in}}%
\pgfpathlineto{\pgfqpoint{3.075790in}{0.709866in}}%
\pgfpathlineto{\pgfqpoint{3.076655in}{0.723821in}}%
\pgfpathlineto{\pgfqpoint{3.077520in}{0.708473in}}%
\pgfpathlineto{\pgfqpoint{3.078385in}{0.663200in}}%
\pgfpathlineto{\pgfqpoint{3.079251in}{0.664849in}}%
\pgfpathlineto{\pgfqpoint{3.080982in}{0.740452in}}%
\pgfpathlineto{\pgfqpoint{3.081847in}{0.815763in}}%
\pgfpathlineto{\pgfqpoint{3.082712in}{0.750416in}}%
\pgfpathlineto{\pgfqpoint{3.083578in}{0.754590in}}%
\pgfpathlineto{\pgfqpoint{3.084443in}{0.780487in}}%
\pgfpathlineto{\pgfqpoint{3.086170in}{0.735872in}}%
\pgfpathlineto{\pgfqpoint{3.087036in}{0.812174in}}%
\pgfpathlineto{\pgfqpoint{3.090498in}{0.680051in}}%
\pgfpathlineto{\pgfqpoint{3.093092in}{0.750051in}}%
\pgfpathlineto{\pgfqpoint{3.094822in}{0.685142in}}%
\pgfpathlineto{\pgfqpoint{3.095687in}{0.693677in}}%
\pgfpathlineto{\pgfqpoint{3.096552in}{0.734995in}}%
\pgfpathlineto{\pgfqpoint{3.097417in}{0.725177in}}%
\pgfpathlineto{\pgfqpoint{3.099147in}{0.756641in}}%
\pgfpathlineto{\pgfqpoint{3.100012in}{0.682722in}}%
\pgfpathlineto{\pgfqpoint{3.100878in}{0.725177in}}%
\pgfpathlineto{\pgfqpoint{3.101742in}{0.685435in}}%
\pgfpathlineto{\pgfqpoint{3.103470in}{0.738292in}}%
\pgfpathlineto{\pgfqpoint{3.104334in}{0.731588in}}%
\pgfpathlineto{\pgfqpoint{3.105199in}{0.695764in}}%
\pgfpathlineto{\pgfqpoint{3.106065in}{0.713272in}}%
\pgfpathlineto{\pgfqpoint{3.106930in}{0.694554in}}%
\pgfpathlineto{\pgfqpoint{3.107795in}{0.759682in}}%
\pgfpathlineto{\pgfqpoint{3.108661in}{0.742100in}}%
\pgfpathlineto{\pgfqpoint{3.109526in}{0.693823in}}%
\pgfpathlineto{\pgfqpoint{3.110391in}{0.705801in}}%
\pgfpathlineto{\pgfqpoint{3.111255in}{0.688878in}}%
\pgfpathlineto{\pgfqpoint{3.112119in}{0.751512in}}%
\pgfpathlineto{\pgfqpoint{3.112985in}{0.665068in}}%
\pgfpathlineto{\pgfqpoint{3.113850in}{0.777121in}}%
\pgfpathlineto{\pgfqpoint{3.114714in}{0.748914in}}%
\pgfpathlineto{\pgfqpoint{3.115579in}{0.749206in}}%
\pgfpathlineto{\pgfqpoint{3.117306in}{0.717925in}}%
\pgfpathlineto{\pgfqpoint{3.118172in}{0.723711in}}%
\pgfpathlineto{\pgfqpoint{3.119038in}{0.707961in}}%
\pgfpathlineto{\pgfqpoint{3.119902in}{0.749206in}}%
\pgfpathlineto{\pgfqpoint{3.120768in}{0.720122in}}%
\pgfpathlineto{\pgfqpoint{3.122496in}{0.767559in}}%
\pgfpathlineto{\pgfqpoint{3.123362in}{0.732173in}}%
\pgfpathlineto{\pgfqpoint{3.124228in}{0.744918in}}%
\pgfpathlineto{\pgfqpoint{3.125956in}{0.698290in}}%
\pgfpathlineto{\pgfqpoint{3.126821in}{0.730853in}}%
\pgfpathlineto{\pgfqpoint{3.127686in}{0.658507in}}%
\pgfpathlineto{\pgfqpoint{3.128550in}{0.684184in}}%
\pgfpathlineto{\pgfqpoint{3.129414in}{0.751512in}}%
\pgfpathlineto{\pgfqpoint{3.130279in}{0.678179in}}%
\pgfpathlineto{\pgfqpoint{3.132007in}{0.750229in}}%
\pgfpathlineto{\pgfqpoint{3.133740in}{0.713491in}}%
\pgfpathlineto{\pgfqpoint{3.134605in}{0.763271in}}%
\pgfpathlineto{\pgfqpoint{3.135471in}{0.698582in}}%
\pgfpathlineto{\pgfqpoint{3.136337in}{0.731292in}}%
\pgfpathlineto{\pgfqpoint{3.137201in}{0.717081in}}%
\pgfpathlineto{\pgfqpoint{3.138930in}{0.758983in}}%
\pgfpathlineto{\pgfqpoint{3.140660in}{0.678288in}}%
\pgfpathlineto{\pgfqpoint{3.141527in}{0.714076in}}%
\pgfpathlineto{\pgfqpoint{3.142392in}{0.676348in}}%
\pgfpathlineto{\pgfqpoint{3.143259in}{0.751001in}}%
\pgfpathlineto{\pgfqpoint{3.144124in}{0.746348in}}%
\pgfpathlineto{\pgfqpoint{3.144990in}{0.696276in}}%
\pgfpathlineto{\pgfqpoint{3.145857in}{0.770450in}}%
\pgfpathlineto{\pgfqpoint{3.147589in}{0.718949in}}%
\pgfpathlineto{\pgfqpoint{3.148455in}{0.724040in}}%
\pgfpathlineto{\pgfqpoint{3.149320in}{0.720195in}}%
\pgfpathlineto{\pgfqpoint{3.151050in}{0.758033in}}%
\pgfpathlineto{\pgfqpoint{3.151916in}{0.716935in}}%
\pgfpathlineto{\pgfqpoint{3.152782in}{0.761549in}}%
\pgfpathlineto{\pgfqpoint{3.155373in}{0.632650in}}%
\pgfpathlineto{\pgfqpoint{3.156236in}{0.735507in}}%
\pgfpathlineto{\pgfqpoint{3.157101in}{0.723821in}}%
\pgfpathlineto{\pgfqpoint{3.158829in}{0.714263in}}%
\pgfpathlineto{\pgfqpoint{3.159695in}{0.737155in}}%
\pgfpathlineto{\pgfqpoint{3.160561in}{0.685873in}}%
\pgfpathlineto{\pgfqpoint{3.161425in}{0.692540in}}%
\pgfpathlineto{\pgfqpoint{3.162292in}{0.696056in}}%
\pgfpathlineto{\pgfqpoint{3.163156in}{0.712761in}}%
\pgfpathlineto{\pgfqpoint{3.164021in}{0.808146in}}%
\pgfpathlineto{\pgfqpoint{3.165751in}{0.718437in}}%
\pgfpathlineto{\pgfqpoint{3.166615in}{0.767559in}}%
\pgfpathlineto{\pgfqpoint{3.167480in}{0.672320in}}%
\pgfpathlineto{\pgfqpoint{3.168345in}{0.734004in}}%
\pgfpathlineto{\pgfqpoint{3.169210in}{0.732136in}}%
\pgfpathlineto{\pgfqpoint{3.170075in}{0.722246in}}%
\pgfpathlineto{\pgfqpoint{3.170938in}{0.686202in}}%
\pgfpathlineto{\pgfqpoint{3.171802in}{0.743968in}}%
\pgfpathlineto{\pgfqpoint{3.173530in}{0.650085in}}%
\pgfpathlineto{\pgfqpoint{3.174393in}{0.762467in}}%
\pgfpathlineto{\pgfqpoint{3.175258in}{0.686352in}}%
\pgfpathlineto{\pgfqpoint{3.176124in}{0.712103in}}%
\pgfpathlineto{\pgfqpoint{3.176989in}{0.695399in}}%
\pgfpathlineto{\pgfqpoint{3.177853in}{0.707523in}}%
\pgfpathlineto{\pgfqpoint{3.178719in}{0.664922in}}%
\pgfpathlineto{\pgfqpoint{3.180447in}{0.750891in}}%
\pgfpathlineto{\pgfqpoint{3.181311in}{0.711989in}}%
\pgfpathlineto{\pgfqpoint{3.183039in}{0.768509in}}%
\pgfpathlineto{\pgfqpoint{3.183904in}{0.646569in}}%
\pgfpathlineto{\pgfqpoint{3.184769in}{0.655835in}}%
\pgfpathlineto{\pgfqpoint{3.185635in}{0.720232in}}%
\pgfpathlineto{\pgfqpoint{3.186500in}{0.711331in}}%
\pgfpathlineto{\pgfqpoint{3.188230in}{0.715838in}}%
\pgfpathlineto{\pgfqpoint{3.189095in}{0.682138in}}%
\pgfpathlineto{\pgfqpoint{3.189960in}{0.697815in}}%
\pgfpathlineto{\pgfqpoint{3.190825in}{0.769832in}}%
\pgfpathlineto{\pgfqpoint{3.192553in}{0.722871in}}%
\pgfpathlineto{\pgfqpoint{3.193417in}{0.743968in}}%
\pgfpathlineto{\pgfqpoint{3.194278in}{0.743603in}}%
\pgfpathlineto{\pgfqpoint{3.195143in}{0.733712in}}%
\pgfpathlineto{\pgfqpoint{3.196006in}{0.778254in}}%
\pgfpathlineto{\pgfqpoint{3.196869in}{0.728511in}}%
\pgfpathlineto{\pgfqpoint{3.197734in}{0.735507in}}%
\pgfpathlineto{\pgfqpoint{3.198598in}{0.755873in}}%
\pgfpathlineto{\pgfqpoint{3.199463in}{0.644701in}}%
\pgfpathlineto{\pgfqpoint{3.200327in}{0.658035in}}%
\pgfpathlineto{\pgfqpoint{3.201193in}{0.714628in}}%
\pgfpathlineto{\pgfqpoint{3.202922in}{0.685581in}}%
\pgfpathlineto{\pgfqpoint{3.205515in}{0.784555in}}%
\pgfpathlineto{\pgfqpoint{3.206381in}{0.685361in}}%
\pgfpathlineto{\pgfqpoint{3.207245in}{0.758472in}}%
\pgfpathlineto{\pgfqpoint{3.208975in}{0.718802in}}%
\pgfpathlineto{\pgfqpoint{3.209840in}{0.762906in}}%
\pgfpathlineto{\pgfqpoint{3.210707in}{0.706093in}}%
\pgfpathlineto{\pgfqpoint{3.211573in}{0.757960in}}%
\pgfpathlineto{\pgfqpoint{3.212439in}{0.757595in}}%
\pgfpathlineto{\pgfqpoint{3.213304in}{0.758106in}}%
\pgfpathlineto{\pgfqpoint{3.214169in}{0.788071in}}%
\pgfpathlineto{\pgfqpoint{3.215896in}{0.721734in}}%
\pgfpathlineto{\pgfqpoint{3.216762in}{0.742320in}}%
\pgfpathlineto{\pgfqpoint{3.217627in}{0.734849in}}%
\pgfpathlineto{\pgfqpoint{3.218492in}{0.694079in}}%
\pgfpathlineto{\pgfqpoint{3.220222in}{0.742174in}}%
\pgfpathlineto{\pgfqpoint{3.221087in}{0.750928in}}%
\pgfpathlineto{\pgfqpoint{3.222816in}{0.716131in}}%
\pgfpathlineto{\pgfqpoint{3.224547in}{0.774080in}}%
\pgfpathlineto{\pgfqpoint{3.225412in}{0.685508in}}%
\pgfpathlineto{\pgfqpoint{3.226276in}{0.686239in}}%
\pgfpathlineto{\pgfqpoint{3.227141in}{0.760745in}}%
\pgfpathlineto{\pgfqpoint{3.228006in}{0.745361in}}%
\pgfpathlineto{\pgfqpoint{3.228872in}{0.702723in}}%
\pgfpathlineto{\pgfqpoint{3.229739in}{0.794665in}}%
\pgfpathlineto{\pgfqpoint{3.230605in}{0.742320in}}%
\pgfpathlineto{\pgfqpoint{3.231470in}{0.787081in}}%
\pgfpathlineto{\pgfqpoint{3.232335in}{0.677594in}}%
\pgfpathlineto{\pgfqpoint{3.233200in}{0.690892in}}%
\pgfpathlineto{\pgfqpoint{3.234064in}{0.687887in}}%
\pgfpathlineto{\pgfqpoint{3.235794in}{0.796460in}}%
\pgfpathlineto{\pgfqpoint{3.237525in}{0.702833in}}%
\pgfpathlineto{\pgfqpoint{3.239254in}{0.740379in}}%
\pgfpathlineto{\pgfqpoint{3.242713in}{0.594410in}}%
\pgfpathlineto{\pgfqpoint{3.243578in}{0.725396in}}%
\pgfpathlineto{\pgfqpoint{3.244443in}{0.713345in}}%
\pgfpathlineto{\pgfqpoint{3.245308in}{0.691184in}}%
\pgfpathlineto{\pgfqpoint{3.246173in}{0.750197in}}%
\pgfpathlineto{\pgfqpoint{3.247039in}{0.653200in}}%
\pgfpathlineto{\pgfqpoint{3.247901in}{0.758366in}}%
\pgfpathlineto{\pgfqpoint{3.248767in}{0.711185in}}%
\pgfpathlineto{\pgfqpoint{3.250497in}{0.743895in}}%
\pgfpathlineto{\pgfqpoint{3.251361in}{0.714628in}}%
\pgfpathlineto{\pgfqpoint{3.252226in}{0.646277in}}%
\pgfpathlineto{\pgfqpoint{3.253089in}{0.803785in}}%
\pgfpathlineto{\pgfqpoint{3.253954in}{0.686681in}}%
\pgfpathlineto{\pgfqpoint{3.254819in}{0.703641in}}%
\pgfpathlineto{\pgfqpoint{3.255685in}{0.635655in}}%
\pgfpathlineto{\pgfqpoint{3.256550in}{0.726277in}}%
\pgfpathlineto{\pgfqpoint{3.257415in}{0.691078in}}%
\pgfpathlineto{\pgfqpoint{3.258280in}{0.732981in}}%
\pgfpathlineto{\pgfqpoint{3.259145in}{0.717852in}}%
\pgfpathlineto{\pgfqpoint{3.260010in}{0.746461in}}%
\pgfpathlineto{\pgfqpoint{3.260876in}{0.689207in}}%
\pgfpathlineto{\pgfqpoint{3.262605in}{0.721734in}}%
\pgfpathlineto{\pgfqpoint{3.263469in}{0.692613in}}%
\pgfpathlineto{\pgfqpoint{3.264334in}{0.772577in}}%
\pgfpathlineto{\pgfqpoint{3.266929in}{0.679133in}}%
\pgfpathlineto{\pgfqpoint{3.267794in}{0.784263in}}%
\pgfpathlineto{\pgfqpoint{3.268657in}{0.713678in}}%
\pgfpathlineto{\pgfqpoint{3.270385in}{0.745361in}}%
\pgfpathlineto{\pgfqpoint{3.272980in}{0.695070in}}%
\pgfpathlineto{\pgfqpoint{3.273846in}{0.794592in}}%
\pgfpathlineto{\pgfqpoint{3.275573in}{0.709317in}}%
\pgfpathlineto{\pgfqpoint{3.276438in}{0.769280in}}%
\pgfpathlineto{\pgfqpoint{3.277300in}{0.764042in}}%
\pgfpathlineto{\pgfqpoint{3.278164in}{0.707815in}}%
\pgfpathlineto{\pgfqpoint{3.279029in}{0.778765in}}%
\pgfpathlineto{\pgfqpoint{3.280759in}{0.668950in}}%
\pgfpathlineto{\pgfqpoint{3.281623in}{0.760819in}}%
\pgfpathlineto{\pgfqpoint{3.282489in}{0.637486in}}%
\pgfpathlineto{\pgfqpoint{3.284218in}{0.734556in}}%
\pgfpathlineto{\pgfqpoint{3.285084in}{0.722249in}}%
\pgfpathlineto{\pgfqpoint{3.286815in}{0.765910in}}%
\pgfpathlineto{\pgfqpoint{3.287681in}{0.700490in}}%
\pgfpathlineto{\pgfqpoint{3.288546in}{0.768988in}}%
\pgfpathlineto{\pgfqpoint{3.289412in}{0.754777in}}%
\pgfpathlineto{\pgfqpoint{3.290277in}{0.711039in}}%
\pgfpathlineto{\pgfqpoint{3.291141in}{0.750855in}}%
\pgfpathlineto{\pgfqpoint{3.292872in}{0.701440in}}%
\pgfpathlineto{\pgfqpoint{3.293738in}{0.676827in}}%
\pgfpathlineto{\pgfqpoint{3.294599in}{0.730013in}}%
\pgfpathlineto{\pgfqpoint{3.295464in}{0.692394in}}%
\pgfpathlineto{\pgfqpoint{3.296325in}{0.819498in}}%
\pgfpathlineto{\pgfqpoint{3.298058in}{0.687595in}}%
\pgfpathlineto{\pgfqpoint{3.298922in}{0.666643in}}%
\pgfpathlineto{\pgfqpoint{3.300649in}{0.746534in}}%
\pgfpathlineto{\pgfqpoint{3.302379in}{0.691809in}}%
\pgfpathlineto{\pgfqpoint{3.303244in}{0.711185in}}%
\pgfpathlineto{\pgfqpoint{3.304108in}{0.665287in}}%
\pgfpathlineto{\pgfqpoint{3.304975in}{0.786569in}}%
\pgfpathlineto{\pgfqpoint{3.305841in}{0.779537in}}%
\pgfpathlineto{\pgfqpoint{3.306706in}{0.656241in}}%
\pgfpathlineto{\pgfqpoint{3.307572in}{0.701627in}}%
\pgfpathlineto{\pgfqpoint{3.308436in}{0.676096in}}%
\pgfpathlineto{\pgfqpoint{3.309302in}{0.717267in}}%
\pgfpathlineto{\pgfqpoint{3.310168in}{0.672433in}}%
\pgfpathlineto{\pgfqpoint{3.313628in}{0.786354in}}%
\pgfpathlineto{\pgfqpoint{3.314493in}{0.671150in}}%
\pgfpathlineto{\pgfqpoint{3.317088in}{0.832540in}}%
\pgfpathlineto{\pgfqpoint{3.317953in}{0.673128in}}%
\pgfpathlineto{\pgfqpoint{3.319683in}{0.762102in}}%
\pgfpathlineto{\pgfqpoint{3.320547in}{0.730598in}}%
\pgfpathlineto{\pgfqpoint{3.321413in}{0.743237in}}%
\pgfpathlineto{\pgfqpoint{3.322275in}{0.786350in}}%
\pgfpathlineto{\pgfqpoint{3.323139in}{0.657926in}}%
\pgfpathlineto{\pgfqpoint{3.324870in}{0.774006in}}%
\pgfpathlineto{\pgfqpoint{3.325734in}{0.750270in}}%
\pgfpathlineto{\pgfqpoint{3.326597in}{0.768070in}}%
\pgfpathlineto{\pgfqpoint{3.327461in}{0.671772in}}%
\pgfpathlineto{\pgfqpoint{3.328325in}{0.675617in}}%
\pgfpathlineto{\pgfqpoint{3.329190in}{0.688074in}}%
\pgfpathlineto{\pgfqpoint{3.330054in}{0.675548in}}%
\pgfpathlineto{\pgfqpoint{3.330917in}{0.686279in}}%
\pgfpathlineto{\pgfqpoint{3.331782in}{0.717779in}}%
\pgfpathlineto{\pgfqpoint{3.332647in}{0.676534in}}%
\pgfpathlineto{\pgfqpoint{3.333513in}{0.706532in}}%
\pgfpathlineto{\pgfqpoint{3.334376in}{0.682284in}}%
\pgfpathlineto{\pgfqpoint{3.336971in}{0.763714in}}%
\pgfpathlineto{\pgfqpoint{3.341293in}{0.672653in}}%
\pgfpathlineto{\pgfqpoint{3.342154in}{0.785254in}}%
\pgfpathlineto{\pgfqpoint{3.343883in}{0.636646in}}%
\pgfpathlineto{\pgfqpoint{3.344746in}{0.704299in}}%
\pgfpathlineto{\pgfqpoint{3.345611in}{0.648291in}}%
\pgfpathlineto{\pgfqpoint{3.346476in}{0.765691in}}%
\pgfpathlineto{\pgfqpoint{3.347340in}{0.738584in}}%
\pgfpathlineto{\pgfqpoint{3.348206in}{0.727191in}}%
\pgfpathlineto{\pgfqpoint{3.349071in}{0.728364in}}%
\pgfpathlineto{\pgfqpoint{3.349936in}{0.685581in}}%
\pgfpathlineto{\pgfqpoint{3.350801in}{0.721661in}}%
\pgfpathlineto{\pgfqpoint{3.351667in}{0.666716in}}%
\pgfpathlineto{\pgfqpoint{3.352533in}{0.743676in}}%
\pgfpathlineto{\pgfqpoint{3.353399in}{0.714921in}}%
\pgfpathlineto{\pgfqpoint{3.354265in}{0.789135in}}%
\pgfpathlineto{\pgfqpoint{3.355130in}{0.662210in}}%
\pgfpathlineto{\pgfqpoint{3.355995in}{0.687522in}}%
\pgfpathlineto{\pgfqpoint{3.356859in}{0.684631in}}%
\pgfpathlineto{\pgfqpoint{3.357724in}{0.675105in}}%
\pgfpathlineto{\pgfqpoint{3.359454in}{0.629280in}}%
\pgfpathlineto{\pgfqpoint{3.360319in}{0.711404in}}%
\pgfpathlineto{\pgfqpoint{3.361184in}{0.678402in}}%
\pgfpathlineto{\pgfqpoint{3.362913in}{0.773235in}}%
\pgfpathlineto{\pgfqpoint{3.365508in}{0.709646in}}%
\pgfpathlineto{\pgfqpoint{3.366374in}{0.702723in}}%
\pgfpathlineto{\pgfqpoint{3.367238in}{0.705216in}}%
\pgfpathlineto{\pgfqpoint{3.368103in}{0.713897in}}%
\pgfpathlineto{\pgfqpoint{3.368968in}{0.698696in}}%
\pgfpathlineto{\pgfqpoint{3.369832in}{0.790085in}}%
\pgfpathlineto{\pgfqpoint{3.370697in}{0.635290in}}%
\pgfpathlineto{\pgfqpoint{3.372429in}{0.748621in}}%
\pgfpathlineto{\pgfqpoint{3.373294in}{0.658255in}}%
\pgfpathlineto{\pgfqpoint{3.375025in}{0.741589in}}%
\pgfpathlineto{\pgfqpoint{3.375889in}{0.705947in}}%
\pgfpathlineto{\pgfqpoint{3.376753in}{0.775947in}}%
\pgfpathlineto{\pgfqpoint{3.377619in}{0.696203in}}%
\pgfpathlineto{\pgfqpoint{3.378486in}{0.697413in}}%
\pgfpathlineto{\pgfqpoint{3.379350in}{0.690819in}}%
\pgfpathlineto{\pgfqpoint{3.380214in}{0.666680in}}%
\pgfpathlineto{\pgfqpoint{3.381080in}{0.732323in}}%
\pgfpathlineto{\pgfqpoint{3.382810in}{0.656939in}}%
\pgfpathlineto{\pgfqpoint{3.383674in}{0.704006in}}%
\pgfpathlineto{\pgfqpoint{3.384537in}{0.680124in}}%
\pgfpathlineto{\pgfqpoint{3.387132in}{0.784044in}}%
\pgfpathlineto{\pgfqpoint{3.388863in}{0.724665in}}%
\pgfpathlineto{\pgfqpoint{3.389728in}{0.756498in}}%
\pgfpathlineto{\pgfqpoint{3.390594in}{0.691005in}}%
\pgfpathlineto{\pgfqpoint{3.391460in}{0.718916in}}%
\pgfpathlineto{\pgfqpoint{3.392325in}{0.713386in}}%
\pgfpathlineto{\pgfqpoint{3.393191in}{0.728295in}}%
\pgfpathlineto{\pgfqpoint{3.394056in}{0.682397in}}%
\pgfpathlineto{\pgfqpoint{3.394921in}{0.737634in}}%
\pgfpathlineto{\pgfqpoint{3.395786in}{0.706020in}}%
\pgfpathlineto{\pgfqpoint{3.396649in}{0.717304in}}%
\pgfpathlineto{\pgfqpoint{3.397513in}{0.751114in}}%
\pgfpathlineto{\pgfqpoint{3.398379in}{0.684484in}}%
\pgfpathlineto{\pgfqpoint{3.399244in}{0.821334in}}%
\pgfpathlineto{\pgfqpoint{3.402704in}{0.681991in}}%
\pgfpathlineto{\pgfqpoint{3.404434in}{0.715400in}}%
\pgfpathlineto{\pgfqpoint{3.406164in}{0.666022in}}%
\pgfpathlineto{\pgfqpoint{3.407893in}{0.698696in}}%
\pgfpathlineto{\pgfqpoint{3.409621in}{0.727670in}}%
\pgfpathlineto{\pgfqpoint{3.410487in}{0.631554in}}%
\pgfpathlineto{\pgfqpoint{3.411351in}{0.747306in}}%
\pgfpathlineto{\pgfqpoint{3.412214in}{0.729172in}}%
\pgfpathlineto{\pgfqpoint{3.413080in}{0.665730in}}%
\pgfpathlineto{\pgfqpoint{3.413944in}{0.752544in}}%
\pgfpathlineto{\pgfqpoint{3.415676in}{0.686279in}}%
\pgfpathlineto{\pgfqpoint{3.418269in}{0.778075in}}%
\pgfpathlineto{\pgfqpoint{3.419134in}{0.752982in}}%
\pgfpathlineto{\pgfqpoint{3.419999in}{0.778294in}}%
\pgfpathlineto{\pgfqpoint{3.420865in}{0.658515in}}%
\pgfpathlineto{\pgfqpoint{3.421731in}{0.664045in}}%
\pgfpathlineto{\pgfqpoint{3.422597in}{0.763421in}}%
\pgfpathlineto{\pgfqpoint{3.423462in}{0.671296in}}%
\pgfpathlineto{\pgfqpoint{3.424327in}{0.692873in}}%
\pgfpathlineto{\pgfqpoint{3.425191in}{0.667195in}}%
\pgfpathlineto{\pgfqpoint{3.426054in}{0.728807in}}%
\pgfpathlineto{\pgfqpoint{3.427782in}{0.671004in}}%
\pgfpathlineto{\pgfqpoint{3.428646in}{0.804191in}}%
\pgfpathlineto{\pgfqpoint{3.430376in}{0.671516in}}%
\pgfpathlineto{\pgfqpoint{3.432107in}{0.768915in}}%
\pgfpathlineto{\pgfqpoint{3.433836in}{0.742360in}}%
\pgfpathlineto{\pgfqpoint{3.434701in}{0.673972in}}%
\pgfpathlineto{\pgfqpoint{3.435566in}{0.822763in}}%
\pgfpathlineto{\pgfqpoint{3.437295in}{0.672653in}}%
\pgfpathlineto{\pgfqpoint{3.438159in}{0.756425in}}%
\pgfpathlineto{\pgfqpoint{3.439025in}{0.677160in}}%
\pgfpathlineto{\pgfqpoint{3.439889in}{0.691444in}}%
\pgfpathlineto{\pgfqpoint{3.440753in}{0.739136in}}%
\pgfpathlineto{\pgfqpoint{3.441619in}{0.718952in}}%
\pgfpathlineto{\pgfqpoint{3.442483in}{0.744740in}}%
\pgfpathlineto{\pgfqpoint{3.443347in}{0.720711in}}%
\pgfpathlineto{\pgfqpoint{3.444212in}{0.736351in}}%
\pgfpathlineto{\pgfqpoint{3.445076in}{0.706719in}}%
\pgfpathlineto{\pgfqpoint{3.445941in}{0.724958in}}%
\pgfpathlineto{\pgfqpoint{3.446805in}{0.655766in}}%
\pgfpathlineto{\pgfqpoint{3.447669in}{0.742726in}}%
\pgfpathlineto{\pgfqpoint{3.448533in}{0.669063in}}%
\pgfpathlineto{\pgfqpoint{3.450259in}{0.761298in}}%
\pgfpathlineto{\pgfqpoint{3.451988in}{0.702910in}}%
\pgfpathlineto{\pgfqpoint{3.452853in}{0.699979in}}%
\pgfpathlineto{\pgfqpoint{3.453718in}{0.682580in}}%
\pgfpathlineto{\pgfqpoint{3.454582in}{0.696462in}}%
\pgfpathlineto{\pgfqpoint{3.455444in}{0.691078in}}%
\pgfpathlineto{\pgfqpoint{3.456310in}{0.718331in}}%
\pgfpathlineto{\pgfqpoint{3.457175in}{0.718039in}}%
\pgfpathlineto{\pgfqpoint{3.458905in}{0.745584in}}%
\pgfpathlineto{\pgfqpoint{3.460636in}{0.675292in}}%
\pgfpathlineto{\pgfqpoint{3.462364in}{0.800528in}}%
\pgfpathlineto{\pgfqpoint{3.463229in}{0.731629in}}%
\pgfpathlineto{\pgfqpoint{3.464094in}{0.740639in}}%
\pgfpathlineto{\pgfqpoint{3.464958in}{0.737634in}}%
\pgfpathlineto{\pgfqpoint{3.465821in}{0.583679in}}%
\pgfpathlineto{\pgfqpoint{3.467552in}{0.726574in}}%
\pgfpathlineto{\pgfqpoint{3.468416in}{0.718624in}}%
\pgfpathlineto{\pgfqpoint{3.469282in}{0.756206in}}%
\pgfpathlineto{\pgfqpoint{3.471877in}{0.711445in}}%
\pgfpathlineto{\pgfqpoint{3.472743in}{0.718404in}}%
\pgfpathlineto{\pgfqpoint{3.473609in}{0.696097in}}%
\pgfpathlineto{\pgfqpoint{3.474471in}{0.711445in}}%
\pgfpathlineto{\pgfqpoint{3.475336in}{0.685804in}}%
\pgfpathlineto{\pgfqpoint{3.476201in}{0.758626in}}%
\pgfpathlineto{\pgfqpoint{3.477067in}{0.743570in}}%
\pgfpathlineto{\pgfqpoint{3.477933in}{0.741995in}}%
\pgfpathlineto{\pgfqpoint{3.478798in}{0.750676in}}%
\pgfpathlineto{\pgfqpoint{3.480525in}{0.711445in}}%
\pgfpathlineto{\pgfqpoint{3.481389in}{0.767161in}}%
\pgfpathlineto{\pgfqpoint{3.482252in}{0.697416in}}%
\pgfpathlineto{\pgfqpoint{3.483118in}{0.809136in}}%
\pgfpathlineto{\pgfqpoint{3.484847in}{0.722323in}}%
\pgfpathlineto{\pgfqpoint{3.486575in}{0.760453in}}%
\pgfpathlineto{\pgfqpoint{3.488304in}{0.711185in}}%
\pgfpathlineto{\pgfqpoint{3.489170in}{0.717706in}}%
\pgfpathlineto{\pgfqpoint{3.490036in}{0.782176in}}%
\pgfpathlineto{\pgfqpoint{3.490901in}{0.762029in}}%
\pgfpathlineto{\pgfqpoint{3.491766in}{0.707340in}}%
\pgfpathlineto{\pgfqpoint{3.492631in}{0.712687in}}%
\pgfpathlineto{\pgfqpoint{3.493495in}{0.747411in}}%
\pgfpathlineto{\pgfqpoint{3.494360in}{0.621001in}}%
\pgfpathlineto{\pgfqpoint{3.496091in}{0.683348in}}%
\pgfpathlineto{\pgfqpoint{3.497819in}{0.676680in}}%
\pgfpathlineto{\pgfqpoint{3.498684in}{0.750599in}}%
\pgfpathlineto{\pgfqpoint{3.499548in}{0.718145in}}%
\pgfpathlineto{\pgfqpoint{3.500413in}{0.742174in}}%
\pgfpathlineto{\pgfqpoint{3.502142in}{0.680051in}}%
\pgfpathlineto{\pgfqpoint{3.503006in}{0.777011in}}%
\pgfpathlineto{\pgfqpoint{3.503870in}{0.715692in}}%
\pgfpathlineto{\pgfqpoint{3.504734in}{0.765033in}}%
\pgfpathlineto{\pgfqpoint{3.505598in}{0.711551in}}%
\pgfpathlineto{\pgfqpoint{3.507329in}{0.743457in}}%
\pgfpathlineto{\pgfqpoint{3.508192in}{0.682320in}}%
\pgfpathlineto{\pgfqpoint{3.509057in}{0.774737in}}%
\pgfpathlineto{\pgfqpoint{3.509923in}{0.769280in}}%
\pgfpathlineto{\pgfqpoint{3.511653in}{0.706240in}}%
\pgfpathlineto{\pgfqpoint{3.513384in}{0.741589in}}%
\pgfpathlineto{\pgfqpoint{3.515979in}{0.659392in}}%
\pgfpathlineto{\pgfqpoint{3.516844in}{0.662835in}}%
\pgfpathlineto{\pgfqpoint{3.517709in}{0.657633in}}%
\pgfpathlineto{\pgfqpoint{3.519437in}{0.724081in}}%
\pgfpathlineto{\pgfqpoint{3.521166in}{0.678037in}}%
\pgfpathlineto{\pgfqpoint{3.522897in}{0.765216in}}%
\pgfpathlineto{\pgfqpoint{3.523762in}{0.713532in}}%
\pgfpathlineto{\pgfqpoint{3.524627in}{0.743091in}}%
\pgfpathlineto{\pgfqpoint{3.526355in}{0.672141in}}%
\pgfpathlineto{\pgfqpoint{3.527220in}{0.743607in}}%
\pgfpathlineto{\pgfqpoint{3.528086in}{0.684411in}}%
\pgfpathlineto{\pgfqpoint{3.528952in}{0.706865in}}%
\pgfpathlineto{\pgfqpoint{3.529817in}{0.702431in}}%
\pgfpathlineto{\pgfqpoint{3.530682in}{0.688220in}}%
\pgfpathlineto{\pgfqpoint{3.531546in}{0.740013in}}%
\pgfpathlineto{\pgfqpoint{3.532412in}{0.679722in}}%
\pgfpathlineto{\pgfqpoint{3.534143in}{0.740598in}}%
\pgfpathlineto{\pgfqpoint{3.535870in}{0.696129in}}%
\pgfpathlineto{\pgfqpoint{3.536735in}{0.596278in}}%
\pgfpathlineto{\pgfqpoint{3.540196in}{0.778473in}}%
\pgfpathlineto{\pgfqpoint{3.541923in}{0.655141in}}%
\pgfpathlineto{\pgfqpoint{3.542787in}{0.754956in}}%
\pgfpathlineto{\pgfqpoint{3.543652in}{0.694408in}}%
\pgfpathlineto{\pgfqpoint{3.545379in}{0.762832in}}%
\pgfpathlineto{\pgfqpoint{3.546245in}{0.714336in}}%
\pgfpathlineto{\pgfqpoint{3.547110in}{0.796899in}}%
\pgfpathlineto{\pgfqpoint{3.547975in}{0.792319in}}%
\pgfpathlineto{\pgfqpoint{3.550568in}{0.670087in}}%
\pgfpathlineto{\pgfqpoint{3.552299in}{0.739315in}}%
\pgfpathlineto{\pgfqpoint{3.553165in}{0.692613in}}%
\pgfpathlineto{\pgfqpoint{3.554030in}{0.737447in}}%
\pgfpathlineto{\pgfqpoint{3.554896in}{0.691842in}}%
\pgfpathlineto{\pgfqpoint{3.555761in}{0.738621in}}%
\pgfpathlineto{\pgfqpoint{3.556626in}{0.702212in}}%
\pgfpathlineto{\pgfqpoint{3.557492in}{0.718364in}}%
\pgfpathlineto{\pgfqpoint{3.558357in}{0.714921in}}%
\pgfpathlineto{\pgfqpoint{3.559222in}{0.714774in}}%
\pgfpathlineto{\pgfqpoint{3.560951in}{0.657597in}}%
\pgfpathlineto{\pgfqpoint{3.561816in}{0.724738in}}%
\pgfpathlineto{\pgfqpoint{3.562682in}{0.707669in}}%
\pgfpathlineto{\pgfqpoint{3.563548in}{0.630929in}}%
\pgfpathlineto{\pgfqpoint{3.564412in}{0.682942in}}%
\pgfpathlineto{\pgfqpoint{3.565276in}{0.665178in}}%
\pgfpathlineto{\pgfqpoint{3.567003in}{0.745836in}}%
\pgfpathlineto{\pgfqpoint{3.567868in}{0.694664in}}%
\pgfpathlineto{\pgfqpoint{3.568733in}{0.715067in}}%
\pgfpathlineto{\pgfqpoint{3.569599in}{0.671662in}}%
\pgfpathlineto{\pgfqpoint{3.571329in}{0.721734in}}%
\pgfpathlineto{\pgfqpoint{3.572193in}{0.633860in}}%
\pgfpathlineto{\pgfqpoint{3.573058in}{0.780820in}}%
\pgfpathlineto{\pgfqpoint{3.573923in}{0.636678in}}%
\pgfpathlineto{\pgfqpoint{3.575651in}{0.744082in}}%
\pgfpathlineto{\pgfqpoint{3.576516in}{0.640674in}}%
\pgfpathlineto{\pgfqpoint{3.577381in}{0.732615in}}%
\pgfpathlineto{\pgfqpoint{3.579112in}{0.691882in}}%
\pgfpathlineto{\pgfqpoint{3.579978in}{0.683567in}}%
\pgfpathlineto{\pgfqpoint{3.580842in}{0.687635in}}%
\pgfpathlineto{\pgfqpoint{3.581708in}{0.746461in}}%
\pgfpathlineto{\pgfqpoint{3.582573in}{0.659392in}}%
\pgfpathlineto{\pgfqpoint{3.583437in}{0.740087in}}%
\pgfpathlineto{\pgfqpoint{3.584300in}{0.665145in}}%
\pgfpathlineto{\pgfqpoint{3.585164in}{0.781957in}}%
\pgfpathlineto{\pgfqpoint{3.586028in}{0.679393in}}%
\pgfpathlineto{\pgfqpoint{3.586892in}{0.753973in}}%
\pgfpathlineto{\pgfqpoint{3.587756in}{0.687343in}}%
\pgfpathlineto{\pgfqpoint{3.588620in}{0.774047in}}%
\pgfpathlineto{\pgfqpoint{3.589485in}{0.762252in}}%
\pgfpathlineto{\pgfqpoint{3.591214in}{0.760786in}}%
\pgfpathlineto{\pgfqpoint{3.592942in}{0.710349in}}%
\pgfpathlineto{\pgfqpoint{3.593804in}{0.750237in}}%
\pgfpathlineto{\pgfqpoint{3.594668in}{0.703828in}}%
\pgfpathlineto{\pgfqpoint{3.595532in}{0.750822in}}%
\pgfpathlineto{\pgfqpoint{3.596397in}{0.736355in}}%
\pgfpathlineto{\pgfqpoint{3.597262in}{0.693864in}}%
\pgfpathlineto{\pgfqpoint{3.598992in}{0.714852in}}%
\pgfpathlineto{\pgfqpoint{3.599857in}{0.711664in}}%
\pgfpathlineto{\pgfqpoint{3.600722in}{0.683900in}}%
\pgfpathlineto{\pgfqpoint{3.603316in}{0.811187in}}%
\pgfpathlineto{\pgfqpoint{3.605045in}{0.732835in}}%
\pgfpathlineto{\pgfqpoint{3.607639in}{0.697047in}}%
\pgfpathlineto{\pgfqpoint{3.610234in}{0.760380in}}%
\pgfpathlineto{\pgfqpoint{3.611099in}{0.741808in}}%
\pgfpathlineto{\pgfqpoint{3.612827in}{0.707779in}}%
\pgfpathlineto{\pgfqpoint{3.613692in}{0.808584in}}%
\pgfpathlineto{\pgfqpoint{3.614557in}{0.801698in}}%
\pgfpathlineto{\pgfqpoint{3.615422in}{0.682028in}}%
\pgfpathlineto{\pgfqpoint{3.616289in}{0.743310in}}%
\pgfpathlineto{\pgfqpoint{3.618017in}{0.677890in}}%
\pgfpathlineto{\pgfqpoint{3.618881in}{0.786496in}}%
\pgfpathlineto{\pgfqpoint{3.619745in}{0.755654in}}%
\pgfpathlineto{\pgfqpoint{3.621474in}{0.686206in}}%
\pgfpathlineto{\pgfqpoint{3.622340in}{0.692617in}}%
\pgfpathlineto{\pgfqpoint{3.623205in}{0.754631in}}%
\pgfpathlineto{\pgfqpoint{3.624067in}{0.702066in}}%
\pgfpathlineto{\pgfqpoint{3.624932in}{0.713751in}}%
\pgfpathlineto{\pgfqpoint{3.625798in}{0.673863in}}%
\pgfpathlineto{\pgfqpoint{3.626663in}{0.680603in}}%
\pgfpathlineto{\pgfqpoint{3.628394in}{0.729724in}}%
\pgfpathlineto{\pgfqpoint{3.629258in}{0.814228in}}%
\pgfpathlineto{\pgfqpoint{3.630123in}{0.763644in}}%
\pgfpathlineto{\pgfqpoint{3.630987in}{0.792432in}}%
\pgfpathlineto{\pgfqpoint{3.632716in}{0.698696in}}%
\pgfpathlineto{\pgfqpoint{3.633580in}{0.701627in}}%
\pgfpathlineto{\pgfqpoint{3.634446in}{0.722359in}}%
\pgfpathlineto{\pgfqpoint{3.635312in}{0.714961in}}%
\pgfpathlineto{\pgfqpoint{3.637043in}{0.756243in}}%
\pgfpathlineto{\pgfqpoint{3.638773in}{0.704997in}}%
\pgfpathlineto{\pgfqpoint{3.639640in}{0.610091in}}%
\pgfpathlineto{\pgfqpoint{3.641372in}{0.751334in}}%
\pgfpathlineto{\pgfqpoint{3.642235in}{0.720345in}}%
\pgfpathlineto{\pgfqpoint{3.643962in}{0.768001in}}%
\pgfpathlineto{\pgfqpoint{3.644829in}{0.767892in}}%
\pgfpathlineto{\pgfqpoint{3.645693in}{0.691078in}}%
\pgfpathlineto{\pgfqpoint{3.647418in}{0.752690in}}%
\pgfpathlineto{\pgfqpoint{3.648283in}{0.739794in}}%
\pgfpathlineto{\pgfqpoint{3.649149in}{0.706093in}}%
\pgfpathlineto{\pgfqpoint{3.650016in}{0.735986in}}%
\pgfpathlineto{\pgfqpoint{3.651747in}{0.719354in}}%
\pgfpathlineto{\pgfqpoint{3.652611in}{0.739867in}}%
\pgfpathlineto{\pgfqpoint{3.653476in}{0.726204in}}%
\pgfpathlineto{\pgfqpoint{3.654343in}{0.755215in}}%
\pgfpathlineto{\pgfqpoint{3.655208in}{0.742100in}}%
\pgfpathlineto{\pgfqpoint{3.656074in}{0.680599in}}%
\pgfpathlineto{\pgfqpoint{3.656940in}{0.681553in}}%
\pgfpathlineto{\pgfqpoint{3.657803in}{0.746754in}}%
\pgfpathlineto{\pgfqpoint{3.658669in}{0.741516in}}%
\pgfpathlineto{\pgfqpoint{3.659533in}{0.738146in}}%
\pgfpathlineto{\pgfqpoint{3.661264in}{0.702650in}}%
\pgfpathlineto{\pgfqpoint{3.662127in}{0.777596in}}%
\pgfpathlineto{\pgfqpoint{3.662989in}{0.696020in}}%
\pgfpathlineto{\pgfqpoint{3.663855in}{0.716496in}}%
\pgfpathlineto{\pgfqpoint{3.664721in}{0.722651in}}%
\pgfpathlineto{\pgfqpoint{3.665587in}{0.745032in}}%
\pgfpathlineto{\pgfqpoint{3.666450in}{0.682284in}}%
\pgfpathlineto{\pgfqpoint{3.667313in}{0.714555in}}%
\pgfpathlineto{\pgfqpoint{3.668179in}{0.649135in}}%
\pgfpathlineto{\pgfqpoint{3.669045in}{0.734703in}}%
\pgfpathlineto{\pgfqpoint{3.670775in}{0.637303in}}%
\pgfpathlineto{\pgfqpoint{3.671641in}{0.681187in}}%
\pgfpathlineto{\pgfqpoint{3.672504in}{0.665730in}}%
\pgfpathlineto{\pgfqpoint{3.674235in}{0.727962in}}%
\pgfpathlineto{\pgfqpoint{3.675101in}{0.611845in}}%
\pgfpathlineto{\pgfqpoint{3.676831in}{0.750822in}}%
\pgfpathlineto{\pgfqpoint{3.678563in}{0.745032in}}%
\pgfpathlineto{\pgfqpoint{3.679428in}{0.729538in}}%
\pgfpathlineto{\pgfqpoint{3.680294in}{0.732063in}}%
\pgfpathlineto{\pgfqpoint{3.681160in}{0.661479in}}%
\pgfpathlineto{\pgfqpoint{3.682891in}{0.746973in}}%
\pgfpathlineto{\pgfqpoint{3.683756in}{0.688732in}}%
\pgfpathlineto{\pgfqpoint{3.685485in}{0.742872in}}%
\pgfpathlineto{\pgfqpoint{3.687215in}{0.628622in}}%
\pgfpathlineto{\pgfqpoint{3.688079in}{0.735766in}}%
\pgfpathlineto{\pgfqpoint{3.688943in}{0.689649in}}%
\pgfpathlineto{\pgfqpoint{3.689808in}{0.695179in}}%
\pgfpathlineto{\pgfqpoint{3.692401in}{0.763677in}}%
\pgfpathlineto{\pgfqpoint{3.693266in}{0.671516in}}%
\pgfpathlineto{\pgfqpoint{3.694131in}{0.710710in}}%
\pgfpathlineto{\pgfqpoint{3.694996in}{0.703349in}}%
\pgfpathlineto{\pgfqpoint{3.695860in}{0.710308in}}%
\pgfpathlineto{\pgfqpoint{3.697589in}{0.782761in}}%
\pgfpathlineto{\pgfqpoint{3.698452in}{0.692650in}}%
\pgfpathlineto{\pgfqpoint{3.700181in}{0.762540in}}%
\pgfpathlineto{\pgfqpoint{3.701046in}{0.725839in}}%
\pgfpathlineto{\pgfqpoint{3.701911in}{0.785546in}}%
\pgfpathlineto{\pgfqpoint{3.704508in}{0.676242in}}%
\pgfpathlineto{\pgfqpoint{3.705373in}{0.763823in}}%
\pgfpathlineto{\pgfqpoint{3.706238in}{0.757083in}}%
\pgfpathlineto{\pgfqpoint{3.707967in}{0.666790in}}%
\pgfpathlineto{\pgfqpoint{3.708833in}{0.669794in}}%
\pgfpathlineto{\pgfqpoint{3.709698in}{0.704080in}}%
\pgfpathlineto{\pgfqpoint{3.710563in}{0.693677in}}%
\pgfpathlineto{\pgfqpoint{3.711428in}{0.737269in}}%
\pgfpathlineto{\pgfqpoint{3.712293in}{0.715911in}}%
\pgfpathlineto{\pgfqpoint{3.713157in}{0.788733in}}%
\pgfpathlineto{\pgfqpoint{3.714886in}{0.671589in}}%
\pgfpathlineto{\pgfqpoint{3.716611in}{0.706459in}}%
\pgfpathlineto{\pgfqpoint{3.717477in}{0.684261in}}%
\pgfpathlineto{\pgfqpoint{3.718343in}{0.749352in}}%
\pgfpathlineto{\pgfqpoint{3.719209in}{0.732356in}}%
\pgfpathlineto{\pgfqpoint{3.720938in}{0.760672in}}%
\pgfpathlineto{\pgfqpoint{3.721802in}{0.690380in}}%
\pgfpathlineto{\pgfqpoint{3.724396in}{0.774591in}}%
\pgfpathlineto{\pgfqpoint{3.726126in}{0.730561in}}%
\pgfpathlineto{\pgfqpoint{3.727855in}{0.733566in}}%
\pgfpathlineto{\pgfqpoint{3.728721in}{0.716350in}}%
\pgfpathlineto{\pgfqpoint{3.729587in}{0.766531in}}%
\pgfpathlineto{\pgfqpoint{3.730452in}{0.703162in}}%
\pgfpathlineto{\pgfqpoint{3.731315in}{0.721442in}}%
\pgfpathlineto{\pgfqpoint{3.732179in}{0.719391in}}%
\pgfpathlineto{\pgfqpoint{3.733044in}{0.678841in}}%
\pgfpathlineto{\pgfqpoint{3.734774in}{0.741735in}}%
\pgfpathlineto{\pgfqpoint{3.735640in}{0.721880in}}%
\pgfpathlineto{\pgfqpoint{3.736506in}{0.646167in}}%
\pgfpathlineto{\pgfqpoint{3.738238in}{0.721222in}}%
\pgfpathlineto{\pgfqpoint{3.739103in}{0.657305in}}%
\pgfpathlineto{\pgfqpoint{3.739967in}{0.762394in}}%
\pgfpathlineto{\pgfqpoint{3.742560in}{0.678402in}}%
\pgfpathlineto{\pgfqpoint{3.744291in}{0.723675in}}%
\pgfpathlineto{\pgfqpoint{3.745156in}{0.654885in}}%
\pgfpathlineto{\pgfqpoint{3.746022in}{0.757266in}}%
\pgfpathlineto{\pgfqpoint{3.746887in}{0.731406in}}%
\pgfpathlineto{\pgfqpoint{3.747753in}{0.687229in}}%
\pgfpathlineto{\pgfqpoint{3.748618in}{0.785729in}}%
\pgfpathlineto{\pgfqpoint{3.750349in}{0.722286in}}%
\pgfpathlineto{\pgfqpoint{3.751213in}{0.730455in}}%
\pgfpathlineto{\pgfqpoint{3.752078in}{0.723788in}}%
\pgfpathlineto{\pgfqpoint{3.752944in}{0.682909in}}%
\pgfpathlineto{\pgfqpoint{3.754676in}{0.739648in}}%
\pgfpathlineto{\pgfqpoint{3.755541in}{0.674411in}}%
\pgfpathlineto{\pgfqpoint{3.756406in}{0.759576in}}%
\pgfpathlineto{\pgfqpoint{3.757272in}{0.695106in}}%
\pgfpathlineto{\pgfqpoint{3.759004in}{0.728109in}}%
\pgfpathlineto{\pgfqpoint{3.760734in}{0.722542in}}%
\pgfpathlineto{\pgfqpoint{3.762463in}{0.678914in}}%
\pgfpathlineto{\pgfqpoint{3.764191in}{0.779317in}}%
\pgfpathlineto{\pgfqpoint{3.765055in}{0.682836in}}%
\pgfpathlineto{\pgfqpoint{3.765921in}{0.795802in}}%
\pgfpathlineto{\pgfqpoint{3.766786in}{0.677525in}}%
\pgfpathlineto{\pgfqpoint{3.768513in}{0.749685in}}%
\pgfpathlineto{\pgfqpoint{3.769378in}{0.710674in}}%
\pgfpathlineto{\pgfqpoint{3.771970in}{0.766203in}}%
\pgfpathlineto{\pgfqpoint{3.772835in}{0.712541in}}%
\pgfpathlineto{\pgfqpoint{3.773700in}{0.799830in}}%
\pgfpathlineto{\pgfqpoint{3.775427in}{0.719866in}}%
\pgfpathlineto{\pgfqpoint{3.776290in}{0.778806in}}%
\pgfpathlineto{\pgfqpoint{3.778019in}{0.728328in}}%
\pgfpathlineto{\pgfqpoint{3.778884in}{0.757156in}}%
\pgfpathlineto{\pgfqpoint{3.780615in}{0.685215in}}%
\pgfpathlineto{\pgfqpoint{3.781479in}{0.736132in}}%
\pgfpathlineto{\pgfqpoint{3.782344in}{0.695399in}}%
\pgfpathlineto{\pgfqpoint{3.783208in}{0.743676in}}%
\pgfpathlineto{\pgfqpoint{3.784074in}{0.682690in}}%
\pgfpathlineto{\pgfqpoint{3.785802in}{0.768549in}}%
\pgfpathlineto{\pgfqpoint{3.786665in}{0.686352in}}%
\pgfpathlineto{\pgfqpoint{3.787530in}{0.783386in}}%
\pgfpathlineto{\pgfqpoint{3.788396in}{0.711957in}}%
\pgfpathlineto{\pgfqpoint{3.789261in}{0.727012in}}%
\pgfpathlineto{\pgfqpoint{3.791853in}{0.658588in}}%
\pgfpathlineto{\pgfqpoint{3.792718in}{0.758220in}}%
\pgfpathlineto{\pgfqpoint{3.793584in}{0.694668in}}%
\pgfpathlineto{\pgfqpoint{3.794449in}{0.740160in}}%
\pgfpathlineto{\pgfqpoint{3.795314in}{0.654154in}}%
\pgfpathlineto{\pgfqpoint{3.797911in}{0.751845in}}%
\pgfpathlineto{\pgfqpoint{3.798777in}{0.648952in}}%
\pgfpathlineto{\pgfqpoint{3.799640in}{0.698330in}}%
\pgfpathlineto{\pgfqpoint{3.800503in}{0.668844in}}%
\pgfpathlineto{\pgfqpoint{3.801369in}{0.674740in}}%
\pgfpathlineto{\pgfqpoint{3.802234in}{0.695472in}}%
\pgfpathlineto{\pgfqpoint{3.803099in}{0.751037in}}%
\pgfpathlineto{\pgfqpoint{3.803964in}{0.696755in}}%
\pgfpathlineto{\pgfqpoint{3.804830in}{0.760234in}}%
\pgfpathlineto{\pgfqpoint{3.806562in}{0.672287in}}%
\pgfpathlineto{\pgfqpoint{3.808293in}{0.744411in}}%
\pgfpathlineto{\pgfqpoint{3.809157in}{0.695618in}}%
\pgfpathlineto{\pgfqpoint{3.810020in}{0.737009in}}%
\pgfpathlineto{\pgfqpoint{3.810885in}{0.642103in}}%
\pgfpathlineto{\pgfqpoint{3.811751in}{0.706240in}}%
\pgfpathlineto{\pgfqpoint{3.812616in}{0.694444in}}%
\pgfpathlineto{\pgfqpoint{3.813481in}{0.611626in}}%
\pgfpathlineto{\pgfqpoint{3.815209in}{0.726350in}}%
\pgfpathlineto{\pgfqpoint{3.816076in}{0.716423in}}%
\pgfpathlineto{\pgfqpoint{3.816939in}{0.634445in}}%
\pgfpathlineto{\pgfqpoint{3.817803in}{0.722725in}}%
\pgfpathlineto{\pgfqpoint{3.818667in}{0.705728in}}%
\pgfpathlineto{\pgfqpoint{3.819532in}{0.726314in}}%
\pgfpathlineto{\pgfqpoint{3.820398in}{0.705655in}}%
\pgfpathlineto{\pgfqpoint{3.821264in}{0.757229in}}%
\pgfpathlineto{\pgfqpoint{3.822129in}{0.708660in}}%
\pgfpathlineto{\pgfqpoint{3.822992in}{0.719907in}}%
\pgfpathlineto{\pgfqpoint{3.823858in}{0.717048in}}%
\pgfpathlineto{\pgfqpoint{3.824723in}{0.751187in}}%
\pgfpathlineto{\pgfqpoint{3.825589in}{0.740858in}}%
\pgfpathlineto{\pgfqpoint{3.826455in}{0.692873in}}%
\pgfpathlineto{\pgfqpoint{3.828185in}{0.771408in}}%
\pgfpathlineto{\pgfqpoint{3.829917in}{0.672214in}}%
\pgfpathlineto{\pgfqpoint{3.830781in}{0.742580in}}%
\pgfpathlineto{\pgfqpoint{3.831644in}{0.698001in}}%
\pgfpathlineto{\pgfqpoint{3.832509in}{0.723861in}}%
\pgfpathlineto{\pgfqpoint{3.833375in}{0.790930in}}%
\pgfpathlineto{\pgfqpoint{3.834240in}{0.740013in}}%
\pgfpathlineto{\pgfqpoint{3.835971in}{0.769905in}}%
\pgfpathlineto{\pgfqpoint{3.836836in}{0.739575in}}%
\pgfpathlineto{\pgfqpoint{3.837698in}{0.789793in}}%
\pgfpathlineto{\pgfqpoint{3.838562in}{0.652286in}}%
\pgfpathlineto{\pgfqpoint{3.839427in}{0.740379in}}%
\pgfpathlineto{\pgfqpoint{3.840290in}{0.715838in}}%
\pgfpathlineto{\pgfqpoint{3.841155in}{0.744520in}}%
\pgfpathlineto{\pgfqpoint{3.843749in}{0.710272in}}%
\pgfpathlineto{\pgfqpoint{3.844612in}{0.730748in}}%
\pgfpathlineto{\pgfqpoint{3.845478in}{0.672360in}}%
\pgfpathlineto{\pgfqpoint{3.846342in}{0.726976in}}%
\pgfpathlineto{\pgfqpoint{3.848071in}{0.701919in}}%
\pgfpathlineto{\pgfqpoint{3.848933in}{0.746607in}}%
\pgfpathlineto{\pgfqpoint{3.849796in}{0.714263in}}%
\pgfpathlineto{\pgfqpoint{3.850662in}{0.715327in}}%
\pgfpathlineto{\pgfqpoint{3.851528in}{0.691444in}}%
\pgfpathlineto{\pgfqpoint{3.852393in}{0.767599in}}%
\pgfpathlineto{\pgfqpoint{3.853258in}{0.725218in}}%
\pgfpathlineto{\pgfqpoint{3.854122in}{0.758147in}}%
\pgfpathlineto{\pgfqpoint{3.855851in}{0.661519in}}%
\pgfpathlineto{\pgfqpoint{3.856715in}{0.666976in}}%
\pgfpathlineto{\pgfqpoint{3.857580in}{0.747964in}}%
\pgfpathlineto{\pgfqpoint{3.858443in}{0.660090in}}%
\pgfpathlineto{\pgfqpoint{3.859307in}{0.677452in}}%
\pgfpathlineto{\pgfqpoint{3.860171in}{0.734410in}}%
\pgfpathlineto{\pgfqpoint{3.861901in}{0.673863in}}%
\pgfpathlineto{\pgfqpoint{3.862767in}{0.690603in}}%
\pgfpathlineto{\pgfqpoint{3.863631in}{0.756279in}}%
\pgfpathlineto{\pgfqpoint{3.866223in}{0.659319in}}%
\pgfpathlineto{\pgfqpoint{3.867087in}{0.693092in}}%
\pgfpathlineto{\pgfqpoint{3.867951in}{0.676461in}}%
\pgfpathlineto{\pgfqpoint{3.870542in}{0.756425in}}%
\pgfpathlineto{\pgfqpoint{3.872272in}{0.675876in}}%
\pgfpathlineto{\pgfqpoint{3.873137in}{0.732835in}}%
\pgfpathlineto{\pgfqpoint{3.874002in}{0.654300in}}%
\pgfpathlineto{\pgfqpoint{3.874867in}{0.719172in}}%
\pgfpathlineto{\pgfqpoint{3.875730in}{0.694595in}}%
\pgfpathlineto{\pgfqpoint{3.877459in}{0.715400in}}%
\pgfpathlineto{\pgfqpoint{3.878324in}{0.783751in}}%
\pgfpathlineto{\pgfqpoint{3.879189in}{0.689174in}}%
\pgfpathlineto{\pgfqpoint{3.880052in}{0.764887in}}%
\pgfpathlineto{\pgfqpoint{3.880914in}{0.716058in}}%
\pgfpathlineto{\pgfqpoint{3.881778in}{0.720272in}}%
\pgfpathlineto{\pgfqpoint{3.882643in}{0.720491in}}%
\pgfpathlineto{\pgfqpoint{3.883507in}{0.748329in}}%
\pgfpathlineto{\pgfqpoint{3.884372in}{0.647158in}}%
\pgfpathlineto{\pgfqpoint{3.885238in}{0.715619in}}%
\pgfpathlineto{\pgfqpoint{3.886103in}{0.696828in}}%
\pgfpathlineto{\pgfqpoint{3.886970in}{0.709870in}}%
\pgfpathlineto{\pgfqpoint{3.887836in}{0.751114in}}%
\pgfpathlineto{\pgfqpoint{3.888702in}{0.701042in}}%
\pgfpathlineto{\pgfqpoint{3.890434in}{0.728109in}}%
\pgfpathlineto{\pgfqpoint{3.891300in}{0.685325in}}%
\pgfpathlineto{\pgfqpoint{3.892166in}{0.690088in}}%
\pgfpathlineto{\pgfqpoint{3.893031in}{0.736205in}}%
\pgfpathlineto{\pgfqpoint{3.893896in}{0.653057in}}%
\pgfpathlineto{\pgfqpoint{3.895627in}{0.760161in}}%
\pgfpathlineto{\pgfqpoint{3.896493in}{0.736863in}}%
\pgfpathlineto{\pgfqpoint{3.897360in}{0.619576in}}%
\pgfpathlineto{\pgfqpoint{3.899091in}{0.711737in}}%
\pgfpathlineto{\pgfqpoint{3.899958in}{0.711591in}}%
\pgfpathlineto{\pgfqpoint{3.900822in}{0.707084in}}%
\pgfpathlineto{\pgfqpoint{3.901689in}{0.684777in}}%
\pgfpathlineto{\pgfqpoint{3.902555in}{0.739648in}}%
\pgfpathlineto{\pgfqpoint{3.903419in}{0.676315in}}%
\pgfpathlineto{\pgfqpoint{3.904283in}{0.739502in}}%
\pgfpathlineto{\pgfqpoint{3.905147in}{0.688732in}}%
\pgfpathlineto{\pgfqpoint{3.906011in}{0.692873in}}%
\pgfpathlineto{\pgfqpoint{3.906875in}{0.684777in}}%
\pgfpathlineto{\pgfqpoint{3.907739in}{0.750456in}}%
\pgfpathlineto{\pgfqpoint{3.910332in}{0.654925in}}%
\pgfpathlineto{\pgfqpoint{3.911197in}{0.658515in}}%
\pgfpathlineto{\pgfqpoint{3.912061in}{0.709870in}}%
\pgfpathlineto{\pgfqpoint{3.912926in}{0.682324in}}%
\pgfpathlineto{\pgfqpoint{3.913792in}{0.701115in}}%
\pgfpathlineto{\pgfqpoint{3.914657in}{0.697599in}}%
\pgfpathlineto{\pgfqpoint{3.916387in}{0.701846in}}%
\pgfpathlineto{\pgfqpoint{3.917253in}{0.667232in}}%
\pgfpathlineto{\pgfqpoint{3.918984in}{0.733054in}}%
\pgfpathlineto{\pgfqpoint{3.919849in}{0.727487in}}%
\pgfpathlineto{\pgfqpoint{3.920712in}{0.705363in}}%
\pgfpathlineto{\pgfqpoint{3.921578in}{0.753201in}}%
\pgfpathlineto{\pgfqpoint{3.922443in}{0.659172in}}%
\pgfpathlineto{\pgfqpoint{3.923309in}{0.706865in}}%
\pgfpathlineto{\pgfqpoint{3.925036in}{0.660675in}}%
\pgfpathlineto{\pgfqpoint{3.926765in}{0.747086in}}%
\pgfpathlineto{\pgfqpoint{3.927629in}{0.702983in}}%
\pgfpathlineto{\pgfqpoint{3.928493in}{0.766389in}}%
\pgfpathlineto{\pgfqpoint{3.929358in}{0.647673in}}%
\pgfpathlineto{\pgfqpoint{3.930224in}{0.720418in}}%
\pgfpathlineto{\pgfqpoint{3.931951in}{0.663716in}}%
\pgfpathlineto{\pgfqpoint{3.932817in}{0.759503in}}%
\pgfpathlineto{\pgfqpoint{3.933682in}{0.737195in}}%
\pgfpathlineto{\pgfqpoint{3.934547in}{0.738625in}}%
\pgfpathlineto{\pgfqpoint{3.935411in}{0.737342in}}%
\pgfpathlineto{\pgfqpoint{3.936274in}{0.752434in}}%
\pgfpathlineto{\pgfqpoint{3.937139in}{0.744082in}}%
\pgfpathlineto{\pgfqpoint{3.938004in}{0.699613in}}%
\pgfpathlineto{\pgfqpoint{3.938870in}{0.740237in}}%
\pgfpathlineto{\pgfqpoint{3.939731in}{0.705915in}}%
\pgfpathlineto{\pgfqpoint{3.940595in}{0.738990in}}%
\pgfpathlineto{\pgfqpoint{3.941461in}{0.717820in}}%
\pgfpathlineto{\pgfqpoint{3.943191in}{0.788039in}}%
\pgfpathlineto{\pgfqpoint{3.944057in}{0.787747in}}%
\pgfpathlineto{\pgfqpoint{3.944922in}{0.764010in}}%
\pgfpathlineto{\pgfqpoint{3.945788in}{0.659944in}}%
\pgfpathlineto{\pgfqpoint{3.946653in}{0.673351in}}%
\pgfpathlineto{\pgfqpoint{3.947519in}{0.701335in}}%
\pgfpathlineto{\pgfqpoint{3.948385in}{0.700421in}}%
\pgfpathlineto{\pgfqpoint{3.949249in}{0.711079in}}%
\pgfpathlineto{\pgfqpoint{3.950114in}{0.744926in}}%
\pgfpathlineto{\pgfqpoint{3.950979in}{0.709102in}}%
\pgfpathlineto{\pgfqpoint{3.951843in}{0.714523in}}%
\pgfpathlineto{\pgfqpoint{3.952708in}{0.760640in}}%
\pgfpathlineto{\pgfqpoint{3.953573in}{0.731665in}}%
\pgfpathlineto{\pgfqpoint{3.954436in}{0.777709in}}%
\pgfpathlineto{\pgfqpoint{3.955301in}{0.736538in}}%
\pgfpathlineto{\pgfqpoint{3.956166in}{0.764156in}}%
\pgfpathlineto{\pgfqpoint{3.957031in}{0.751666in}}%
\pgfpathlineto{\pgfqpoint{3.957894in}{0.719614in}}%
\pgfpathlineto{\pgfqpoint{3.958759in}{0.723131in}}%
\pgfpathlineto{\pgfqpoint{3.959623in}{0.741995in}}%
\pgfpathlineto{\pgfqpoint{3.960486in}{0.694119in}}%
\pgfpathlineto{\pgfqpoint{3.962216in}{0.772910in}}%
\pgfpathlineto{\pgfqpoint{3.963081in}{0.748954in}}%
\pgfpathlineto{\pgfqpoint{3.963945in}{0.764083in}}%
\pgfpathlineto{\pgfqpoint{3.964811in}{0.726464in}}%
\pgfpathlineto{\pgfqpoint{3.965676in}{0.790418in}}%
\pgfpathlineto{\pgfqpoint{3.966542in}{0.776426in}}%
\pgfpathlineto{\pgfqpoint{3.967407in}{0.649322in}}%
\pgfpathlineto{\pgfqpoint{3.968272in}{0.698590in}}%
\pgfpathlineto{\pgfqpoint{3.969137in}{0.697599in}}%
\pgfpathlineto{\pgfqpoint{3.970867in}{0.745365in}}%
\pgfpathlineto{\pgfqpoint{3.971732in}{0.770969in}}%
\pgfpathlineto{\pgfqpoint{3.972596in}{0.739615in}}%
\pgfpathlineto{\pgfqpoint{3.973460in}{0.799871in}}%
\pgfpathlineto{\pgfqpoint{3.974325in}{0.745036in}}%
\pgfpathlineto{\pgfqpoint{3.975191in}{0.762434in}}%
\pgfpathlineto{\pgfqpoint{3.976055in}{0.755037in}}%
\pgfpathlineto{\pgfqpoint{3.976921in}{0.709468in}}%
\pgfpathlineto{\pgfqpoint{3.977787in}{0.735035in}}%
\pgfpathlineto{\pgfqpoint{3.978650in}{0.733606in}}%
\pgfpathlineto{\pgfqpoint{3.979516in}{0.649212in}}%
\pgfpathlineto{\pgfqpoint{3.981249in}{0.748589in}}%
\pgfpathlineto{\pgfqpoint{3.982981in}{0.690932in}}%
\pgfpathlineto{\pgfqpoint{3.983846in}{0.699394in}}%
\pgfpathlineto{\pgfqpoint{3.984712in}{0.709870in}}%
\pgfpathlineto{\pgfqpoint{3.985575in}{0.630823in}}%
\pgfpathlineto{\pgfqpoint{3.986438in}{0.682397in}}%
\pgfpathlineto{\pgfqpoint{3.987304in}{0.640308in}}%
\pgfpathlineto{\pgfqpoint{3.988169in}{0.702837in}}%
\pgfpathlineto{\pgfqpoint{3.989902in}{0.672872in}}%
\pgfpathlineto{\pgfqpoint{3.990765in}{0.729245in}}%
\pgfpathlineto{\pgfqpoint{3.991629in}{0.640235in}}%
\pgfpathlineto{\pgfqpoint{3.992494in}{0.726501in}}%
\pgfpathlineto{\pgfqpoint{3.994223in}{0.664889in}}%
\pgfpathlineto{\pgfqpoint{3.995089in}{0.769979in}}%
\pgfpathlineto{\pgfqpoint{3.995954in}{0.684704in}}%
\pgfpathlineto{\pgfqpoint{3.996820in}{0.773860in}}%
\pgfpathlineto{\pgfqpoint{3.997684in}{0.730528in}}%
\pgfpathlineto{\pgfqpoint{3.998549in}{0.755694in}}%
\pgfpathlineto{\pgfqpoint{3.999414in}{0.750822in}}%
\pgfpathlineto{\pgfqpoint{4.000279in}{0.665547in}}%
\pgfpathlineto{\pgfqpoint{4.002007in}{0.725071in}}%
\pgfpathlineto{\pgfqpoint{4.002874in}{0.698294in}}%
\pgfpathlineto{\pgfqpoint{4.004603in}{0.719281in}}%
\pgfpathlineto{\pgfqpoint{4.005467in}{0.714153in}}%
\pgfpathlineto{\pgfqpoint{4.007191in}{0.667122in}}%
\pgfpathlineto{\pgfqpoint{4.008923in}{0.738917in}}%
\pgfpathlineto{\pgfqpoint{4.009788in}{0.703495in}}%
\pgfpathlineto{\pgfqpoint{4.010652in}{0.706426in}}%
\pgfpathlineto{\pgfqpoint{4.011517in}{0.742100in}}%
\pgfpathlineto{\pgfqpoint{4.013246in}{0.697559in}}%
\pgfpathlineto{\pgfqpoint{4.014108in}{0.692175in}}%
\pgfpathlineto{\pgfqpoint{4.014974in}{0.734154in}}%
\pgfpathlineto{\pgfqpoint{4.015839in}{0.708879in}}%
\pgfpathlineto{\pgfqpoint{4.016704in}{0.773276in}}%
\pgfpathlineto{\pgfqpoint{4.017569in}{0.725181in}}%
\pgfpathlineto{\pgfqpoint{4.018434in}{0.751772in}}%
\pgfpathlineto{\pgfqpoint{4.019301in}{0.700636in}}%
\pgfpathlineto{\pgfqpoint{4.020166in}{0.736132in}}%
\pgfpathlineto{\pgfqpoint{4.021032in}{0.732689in}}%
\pgfpathlineto{\pgfqpoint{4.021898in}{0.737122in}}%
\pgfpathlineto{\pgfqpoint{4.023629in}{0.705915in}}%
\pgfpathlineto{\pgfqpoint{4.024494in}{0.740492in}}%
\pgfpathlineto{\pgfqpoint{4.025359in}{0.730821in}}%
\pgfpathlineto{\pgfqpoint{4.026224in}{0.710527in}}%
\pgfpathlineto{\pgfqpoint{4.027953in}{0.750603in}}%
\pgfpathlineto{\pgfqpoint{4.028818in}{0.692727in}}%
\pgfpathlineto{\pgfqpoint{4.029684in}{0.777271in}}%
\pgfpathlineto{\pgfqpoint{4.031412in}{0.680566in}}%
\pgfpathlineto{\pgfqpoint{4.032277in}{0.682105in}}%
\pgfpathlineto{\pgfqpoint{4.034007in}{0.771517in}}%
\pgfpathlineto{\pgfqpoint{4.036604in}{0.708477in}}%
\pgfpathlineto{\pgfqpoint{4.037469in}{0.749320in}}%
\pgfpathlineto{\pgfqpoint{4.038334in}{0.705915in}}%
\pgfpathlineto{\pgfqpoint{4.039199in}{0.709577in}}%
\pgfpathlineto{\pgfqpoint{4.040065in}{0.696682in}}%
\pgfpathlineto{\pgfqpoint{4.040931in}{0.738734in}}%
\pgfpathlineto{\pgfqpoint{4.042662in}{0.685110in}}%
\pgfpathlineto{\pgfqpoint{4.045257in}{0.753348in}}%
\pgfpathlineto{\pgfqpoint{4.046121in}{0.736497in}}%
\pgfpathlineto{\pgfqpoint{4.046986in}{0.737926in}}%
\pgfpathlineto{\pgfqpoint{4.047850in}{0.756718in}}%
\pgfpathlineto{\pgfqpoint{4.048715in}{0.691517in}}%
\pgfpathlineto{\pgfqpoint{4.049579in}{0.785546in}}%
\pgfpathlineto{\pgfqpoint{4.050444in}{0.706792in}}%
\pgfpathlineto{\pgfqpoint{4.052176in}{0.787706in}}%
\pgfpathlineto{\pgfqpoint{4.053040in}{0.681443in}}%
\pgfpathlineto{\pgfqpoint{4.053906in}{0.740087in}}%
\pgfpathlineto{\pgfqpoint{4.054771in}{0.711478in}}%
\pgfpathlineto{\pgfqpoint{4.055636in}{0.645948in}}%
\pgfpathlineto{\pgfqpoint{4.056502in}{0.775363in}}%
\pgfpathlineto{\pgfqpoint{4.058228in}{0.714409in}}%
\pgfpathlineto{\pgfqpoint{4.059092in}{0.730382in}}%
\pgfpathlineto{\pgfqpoint{4.060822in}{0.690380in}}%
\pgfpathlineto{\pgfqpoint{4.061687in}{0.740306in}}%
\pgfpathlineto{\pgfqpoint{4.062552in}{0.685910in}}%
\pgfpathlineto{\pgfqpoint{4.063417in}{0.733493in}}%
\pgfpathlineto{\pgfqpoint{4.064282in}{0.721953in}}%
\pgfpathlineto{\pgfqpoint{4.065147in}{0.712432in}}%
\pgfpathlineto{\pgfqpoint{4.066012in}{0.764408in}}%
\pgfpathlineto{\pgfqpoint{4.067741in}{0.705874in}}%
\pgfpathlineto{\pgfqpoint{4.068604in}{0.738438in}}%
\pgfpathlineto{\pgfqpoint{4.069469in}{0.672762in}}%
\pgfpathlineto{\pgfqpoint{4.071198in}{0.700344in}}%
\pgfpathlineto{\pgfqpoint{4.072928in}{0.676315in}}%
\pgfpathlineto{\pgfqpoint{4.073792in}{0.688585in}}%
\pgfpathlineto{\pgfqpoint{4.074657in}{0.729136in}}%
\pgfpathlineto{\pgfqpoint{4.075522in}{0.720638in}}%
\pgfpathlineto{\pgfqpoint{4.076388in}{0.705582in}}%
\pgfpathlineto{\pgfqpoint{4.078118in}{0.761663in}}%
\pgfpathlineto{\pgfqpoint{4.079849in}{0.695252in}}%
\pgfpathlineto{\pgfqpoint{4.080713in}{0.698622in}}%
\pgfpathlineto{\pgfqpoint{4.081580in}{0.700381in}}%
\pgfpathlineto{\pgfqpoint{4.082444in}{0.782468in}}%
\pgfpathlineto{\pgfqpoint{4.084175in}{0.686608in}}%
\pgfpathlineto{\pgfqpoint{4.085039in}{0.725802in}}%
\pgfpathlineto{\pgfqpoint{4.085903in}{0.695106in}}%
\pgfpathlineto{\pgfqpoint{4.086768in}{0.739063in}}%
\pgfpathlineto{\pgfqpoint{4.087632in}{0.728035in}}%
\pgfpathlineto{\pgfqpoint{4.088498in}{0.652798in}}%
\pgfpathlineto{\pgfqpoint{4.090229in}{0.740858in}}%
\pgfpathlineto{\pgfqpoint{4.091095in}{0.708915in}}%
\pgfpathlineto{\pgfqpoint{4.093691in}{0.825029in}}%
\pgfpathlineto{\pgfqpoint{4.094556in}{0.743895in}}%
\pgfpathlineto{\pgfqpoint{4.095420in}{0.805653in}}%
\pgfpathlineto{\pgfqpoint{4.097151in}{0.707450in}}%
\pgfpathlineto{\pgfqpoint{4.098016in}{0.758439in}}%
\pgfpathlineto{\pgfqpoint{4.098881in}{0.741589in}}%
\pgfpathlineto{\pgfqpoint{4.099746in}{0.688732in}}%
\pgfpathlineto{\pgfqpoint{4.100611in}{0.695216in}}%
\pgfpathlineto{\pgfqpoint{4.101476in}{0.725437in}}%
\pgfpathlineto{\pgfqpoint{4.102341in}{0.687781in}}%
\pgfpathlineto{\pgfqpoint{4.103206in}{0.764923in}}%
\pgfpathlineto{\pgfqpoint{4.104071in}{0.738073in}}%
\pgfpathlineto{\pgfqpoint{4.104936in}{0.753128in}}%
\pgfpathlineto{\pgfqpoint{4.107530in}{0.608771in}}%
\pgfpathlineto{\pgfqpoint{4.108395in}{0.726939in}}%
\pgfpathlineto{\pgfqpoint{4.110125in}{0.670310in}}%
\pgfpathlineto{\pgfqpoint{4.110990in}{0.689722in}}%
\pgfpathlineto{\pgfqpoint{4.111854in}{0.645473in}}%
\pgfpathlineto{\pgfqpoint{4.113584in}{0.713386in}}%
\pgfpathlineto{\pgfqpoint{4.114449in}{0.703422in}}%
\pgfpathlineto{\pgfqpoint{4.115313in}{0.678037in}}%
\pgfpathlineto{\pgfqpoint{4.116177in}{0.733972in}}%
\pgfpathlineto{\pgfqpoint{4.117040in}{0.677562in}}%
\pgfpathlineto{\pgfqpoint{4.117905in}{0.691371in}}%
\pgfpathlineto{\pgfqpoint{4.119636in}{0.766243in}}%
\pgfpathlineto{\pgfqpoint{4.120501in}{0.755475in}}%
\pgfpathlineto{\pgfqpoint{4.122231in}{0.690859in}}%
\pgfpathlineto{\pgfqpoint{4.123096in}{0.727012in}}%
\pgfpathlineto{\pgfqpoint{4.123961in}{0.683900in}}%
\pgfpathlineto{\pgfqpoint{4.125692in}{0.767745in}}%
\pgfpathlineto{\pgfqpoint{4.126557in}{0.697563in}}%
\pgfpathlineto{\pgfqpoint{4.127422in}{0.716244in}}%
\pgfpathlineto{\pgfqpoint{4.129154in}{0.692617in}}%
\pgfpathlineto{\pgfqpoint{4.130019in}{0.702983in}}%
\pgfpathlineto{\pgfqpoint{4.130884in}{0.667195in}}%
\pgfpathlineto{\pgfqpoint{4.132610in}{0.711810in}}%
\pgfpathlineto{\pgfqpoint{4.133474in}{0.753348in}}%
\pgfpathlineto{\pgfqpoint{4.134339in}{0.635143in}}%
\pgfpathlineto{\pgfqpoint{4.135203in}{0.724373in}}%
\pgfpathlineto{\pgfqpoint{4.136068in}{0.722067in}}%
\pgfpathlineto{\pgfqpoint{4.137798in}{0.785180in}}%
\pgfpathlineto{\pgfqpoint{4.139529in}{0.708952in}}%
\pgfpathlineto{\pgfqpoint{4.141259in}{0.722761in}}%
\pgfpathlineto{\pgfqpoint{4.142124in}{0.721661in}}%
\pgfpathlineto{\pgfqpoint{4.143853in}{0.747484in}}%
\pgfpathlineto{\pgfqpoint{4.144718in}{0.714482in}}%
\pgfpathlineto{\pgfqpoint{4.145583in}{0.772577in}}%
\pgfpathlineto{\pgfqpoint{4.148177in}{0.673310in}}%
\pgfpathlineto{\pgfqpoint{4.149043in}{0.710820in}}%
\pgfpathlineto{\pgfqpoint{4.150774in}{0.667195in}}%
\pgfpathlineto{\pgfqpoint{4.151640in}{0.725656in}}%
\pgfpathlineto{\pgfqpoint{4.152507in}{0.711883in}}%
\pgfpathlineto{\pgfqpoint{4.153374in}{0.688147in}}%
\pgfpathlineto{\pgfqpoint{4.154234in}{0.744667in}}%
\pgfpathlineto{\pgfqpoint{4.155100in}{0.669100in}}%
\pgfpathlineto{\pgfqpoint{4.156830in}{0.737780in}}%
\pgfpathlineto{\pgfqpoint{4.157692in}{0.674703in}}%
\pgfpathlineto{\pgfqpoint{4.158557in}{0.725583in}}%
\pgfpathlineto{\pgfqpoint{4.159420in}{0.702618in}}%
\pgfpathlineto{\pgfqpoint{4.160285in}{0.716354in}}%
\pgfpathlineto{\pgfqpoint{4.161148in}{0.709943in}}%
\pgfpathlineto{\pgfqpoint{4.162013in}{0.726793in}}%
\pgfpathlineto{\pgfqpoint{4.162876in}{0.658149in}}%
\pgfpathlineto{\pgfqpoint{4.164607in}{0.693937in}}%
\pgfpathlineto{\pgfqpoint{4.165472in}{0.658441in}}%
\pgfpathlineto{\pgfqpoint{4.166337in}{0.766535in}}%
\pgfpathlineto{\pgfqpoint{4.167203in}{0.637490in}}%
\pgfpathlineto{\pgfqpoint{4.168067in}{0.728515in}}%
\pgfpathlineto{\pgfqpoint{4.169798in}{0.666318in}}%
\pgfpathlineto{\pgfqpoint{4.170663in}{0.693717in}}%
\pgfpathlineto{\pgfqpoint{4.171528in}{0.661811in}}%
\pgfpathlineto{\pgfqpoint{4.173258in}{0.693937in}}%
\pgfpathlineto{\pgfqpoint{4.174125in}{0.646719in}}%
\pgfpathlineto{\pgfqpoint{4.174991in}{0.695951in}}%
\pgfpathlineto{\pgfqpoint{4.175858in}{0.625805in}}%
\pgfpathlineto{\pgfqpoint{4.177588in}{0.662396in}}%
\pgfpathlineto{\pgfqpoint{4.178452in}{0.646610in}}%
\pgfpathlineto{\pgfqpoint{4.180184in}{0.713166in}}%
\pgfpathlineto{\pgfqpoint{4.181050in}{0.677086in}}%
\pgfpathlineto{\pgfqpoint{4.181915in}{0.735693in}}%
\pgfpathlineto{\pgfqpoint{4.183646in}{0.668077in}}%
\pgfpathlineto{\pgfqpoint{4.185377in}{0.736059in}}%
\pgfpathlineto{\pgfqpoint{4.186240in}{0.682909in}}%
\pgfpathlineto{\pgfqpoint{4.187102in}{0.729099in}}%
\pgfpathlineto{\pgfqpoint{4.187968in}{0.684265in}}%
\pgfpathlineto{\pgfqpoint{4.188833in}{0.707271in}}%
\pgfpathlineto{\pgfqpoint{4.189700in}{0.697965in}}%
\pgfpathlineto{\pgfqpoint{4.190565in}{0.717048in}}%
\pgfpathlineto{\pgfqpoint{4.191432in}{0.680822in}}%
\pgfpathlineto{\pgfqpoint{4.192297in}{0.724779in}}%
\pgfpathlineto{\pgfqpoint{4.193163in}{0.678589in}}%
\pgfpathlineto{\pgfqpoint{4.194030in}{0.692581in}}%
\pgfpathlineto{\pgfqpoint{4.194895in}{0.680237in}}%
\pgfpathlineto{\pgfqpoint{4.195760in}{0.702289in}}%
\pgfpathlineto{\pgfqpoint{4.196623in}{0.637344in}}%
\pgfpathlineto{\pgfqpoint{4.197486in}{0.645513in}}%
\pgfpathlineto{\pgfqpoint{4.199218in}{0.775330in}}%
\pgfpathlineto{\pgfqpoint{4.200083in}{0.675072in}}%
\pgfpathlineto{\pgfqpoint{4.201815in}{0.744155in}}%
\pgfpathlineto{\pgfqpoint{4.202680in}{0.661081in}}%
\pgfpathlineto{\pgfqpoint{4.204411in}{0.779870in}}%
\pgfpathlineto{\pgfqpoint{4.206142in}{0.638367in}}%
\pgfpathlineto{\pgfqpoint{4.208736in}{0.766974in}}%
\pgfpathlineto{\pgfqpoint{4.210467in}{0.695585in}}%
\pgfpathlineto{\pgfqpoint{4.212198in}{0.659871in}}%
\pgfpathlineto{\pgfqpoint{4.213925in}{0.786975in}}%
\pgfpathlineto{\pgfqpoint{4.215654in}{0.664191in}}%
\pgfpathlineto{\pgfqpoint{4.217385in}{0.721774in}}%
\pgfpathlineto{\pgfqpoint{4.218250in}{0.698184in}}%
\pgfpathlineto{\pgfqpoint{4.219976in}{0.707157in}}%
\pgfpathlineto{\pgfqpoint{4.220842in}{0.701700in}}%
\pgfpathlineto{\pgfqpoint{4.221705in}{0.686681in}}%
\pgfpathlineto{\pgfqpoint{4.223436in}{0.756937in}}%
\pgfpathlineto{\pgfqpoint{4.224302in}{0.775582in}}%
\pgfpathlineto{\pgfqpoint{4.226031in}{0.682544in}}%
\pgfpathlineto{\pgfqpoint{4.226894in}{0.705951in}}%
\pgfpathlineto{\pgfqpoint{4.227760in}{0.634924in}}%
\pgfpathlineto{\pgfqpoint{4.229490in}{0.710308in}}%
\pgfpathlineto{\pgfqpoint{4.230356in}{0.718697in}}%
\pgfpathlineto{\pgfqpoint{4.232084in}{0.643313in}}%
\pgfpathlineto{\pgfqpoint{4.232949in}{0.732469in}}%
\pgfpathlineto{\pgfqpoint{4.233814in}{0.688110in}}%
\pgfpathlineto{\pgfqpoint{4.234678in}{0.758074in}}%
\pgfpathlineto{\pgfqpoint{4.236409in}{0.658186in}}%
\pgfpathlineto{\pgfqpoint{4.238138in}{0.724998in}}%
\pgfpathlineto{\pgfqpoint{4.239003in}{0.711006in}}%
\pgfpathlineto{\pgfqpoint{4.240732in}{0.740419in}}%
\pgfpathlineto{\pgfqpoint{4.243325in}{0.711372in}}%
\pgfpathlineto{\pgfqpoint{4.244190in}{0.734191in}}%
\pgfpathlineto{\pgfqpoint{4.245054in}{0.733825in}}%
\pgfpathlineto{\pgfqpoint{4.245919in}{0.739502in}}%
\pgfpathlineto{\pgfqpoint{4.246784in}{0.710235in}}%
\pgfpathlineto{\pgfqpoint{4.248514in}{0.754631in}}%
\pgfpathlineto{\pgfqpoint{4.249380in}{0.720126in}}%
\pgfpathlineto{\pgfqpoint{4.250245in}{0.638331in}}%
\pgfpathlineto{\pgfqpoint{4.251107in}{0.766682in}}%
\pgfpathlineto{\pgfqpoint{4.251972in}{0.749320in}}%
\pgfpathlineto{\pgfqpoint{4.252837in}{0.679356in}}%
\pgfpathlineto{\pgfqpoint{4.253702in}{0.727085in}}%
\pgfpathlineto{\pgfqpoint{4.255431in}{0.650418in}}%
\pgfpathlineto{\pgfqpoint{4.257163in}{0.720126in}}%
\pgfpathlineto{\pgfqpoint{4.258027in}{0.719281in}}%
\pgfpathlineto{\pgfqpoint{4.258891in}{0.724300in}}%
\pgfpathlineto{\pgfqpoint{4.259758in}{0.652578in}}%
\pgfpathlineto{\pgfqpoint{4.262349in}{0.770015in}}%
\pgfpathlineto{\pgfqpoint{4.263215in}{0.700198in}}%
\pgfpathlineto{\pgfqpoint{4.265810in}{0.789720in}}%
\pgfpathlineto{\pgfqpoint{4.267541in}{0.704080in}}%
\pgfpathlineto{\pgfqpoint{4.268406in}{0.742872in}}%
\pgfpathlineto{\pgfqpoint{4.269270in}{0.675511in}}%
\pgfpathlineto{\pgfqpoint{4.271000in}{0.745876in}}%
\pgfpathlineto{\pgfqpoint{4.272729in}{0.645327in}}%
\pgfpathlineto{\pgfqpoint{4.273593in}{0.719647in}}%
\pgfpathlineto{\pgfqpoint{4.274458in}{0.694335in}}%
\pgfpathlineto{\pgfqpoint{4.276188in}{0.757887in}}%
\pgfpathlineto{\pgfqpoint{4.277054in}{0.625325in}}%
\pgfpathlineto{\pgfqpoint{4.277919in}{0.687010in}}%
\pgfpathlineto{\pgfqpoint{4.278785in}{0.624335in}}%
\pgfpathlineto{\pgfqpoint{4.280515in}{0.760819in}}%
\pgfpathlineto{\pgfqpoint{4.281379in}{0.653053in}}%
\pgfpathlineto{\pgfqpoint{4.282243in}{0.750416in}}%
\pgfpathlineto{\pgfqpoint{4.284839in}{0.667561in}}%
\pgfpathlineto{\pgfqpoint{4.285704in}{0.681626in}}%
\pgfpathlineto{\pgfqpoint{4.286569in}{0.667707in}}%
\pgfpathlineto{\pgfqpoint{4.287434in}{0.614411in}}%
\pgfpathlineto{\pgfqpoint{4.289164in}{0.701993in}}%
\pgfpathlineto{\pgfqpoint{4.290029in}{0.691882in}}%
\pgfpathlineto{\pgfqpoint{4.290894in}{0.705436in}}%
\pgfpathlineto{\pgfqpoint{4.291759in}{0.663899in}}%
\pgfpathlineto{\pgfqpoint{4.292625in}{0.695252in}}%
\pgfpathlineto{\pgfqpoint{4.293491in}{0.658953in}}%
\pgfpathlineto{\pgfqpoint{4.296948in}{0.707377in}}%
\pgfpathlineto{\pgfqpoint{4.298678in}{0.697193in}}%
\pgfpathlineto{\pgfqpoint{4.300410in}{0.599871in}}%
\pgfpathlineto{\pgfqpoint{4.301276in}{0.681886in}}%
\pgfpathlineto{\pgfqpoint{4.302140in}{0.681699in}}%
\pgfpathlineto{\pgfqpoint{4.303005in}{0.682653in}}%
\pgfpathlineto{\pgfqpoint{4.303869in}{0.638148in}}%
\pgfpathlineto{\pgfqpoint{4.304734in}{0.725802in}}%
\pgfpathlineto{\pgfqpoint{4.305599in}{0.684119in}}%
\pgfpathlineto{\pgfqpoint{4.306465in}{0.762581in}}%
\pgfpathlineto{\pgfqpoint{4.307331in}{0.709139in}}%
\pgfpathlineto{\pgfqpoint{4.308196in}{0.714742in}}%
\pgfpathlineto{\pgfqpoint{4.309061in}{0.709358in}}%
\pgfpathlineto{\pgfqpoint{4.309926in}{0.658039in}}%
\pgfpathlineto{\pgfqpoint{4.311656in}{0.724446in}}%
\pgfpathlineto{\pgfqpoint{4.312522in}{0.695658in}}%
\pgfpathlineto{\pgfqpoint{4.313387in}{0.731885in}}%
\pgfpathlineto{\pgfqpoint{4.314253in}{0.728515in}}%
\pgfpathlineto{\pgfqpoint{4.315118in}{0.677598in}}%
\pgfpathlineto{\pgfqpoint{4.316847in}{0.735730in}}%
\pgfpathlineto{\pgfqpoint{4.318576in}{0.713605in}}%
\pgfpathlineto{\pgfqpoint{4.319442in}{0.723752in}}%
\pgfpathlineto{\pgfqpoint{4.320307in}{0.720711in}}%
\pgfpathlineto{\pgfqpoint{4.321170in}{0.696901in}}%
\pgfpathlineto{\pgfqpoint{4.322035in}{0.744045in}}%
\pgfpathlineto{\pgfqpoint{4.323763in}{0.708002in}}%
\pgfpathlineto{\pgfqpoint{4.324629in}{0.750895in}}%
\pgfpathlineto{\pgfqpoint{4.326357in}{0.678991in}}%
\pgfpathlineto{\pgfqpoint{4.328083in}{0.741954in}}%
\pgfpathlineto{\pgfqpoint{4.328948in}{0.672982in}}%
\pgfpathlineto{\pgfqpoint{4.329813in}{0.756718in}}%
\pgfpathlineto{\pgfqpoint{4.330678in}{0.735986in}}%
\pgfpathlineto{\pgfqpoint{4.331544in}{0.705326in}}%
\pgfpathlineto{\pgfqpoint{4.332409in}{0.762102in}}%
\pgfpathlineto{\pgfqpoint{4.333276in}{0.701554in}}%
\pgfpathlineto{\pgfqpoint{4.334141in}{0.717194in}}%
\pgfpathlineto{\pgfqpoint{4.335005in}{0.662835in}}%
\pgfpathlineto{\pgfqpoint{4.335868in}{0.766353in}}%
\pgfpathlineto{\pgfqpoint{4.336733in}{0.654852in}}%
\pgfpathlineto{\pgfqpoint{4.337598in}{0.677744in}}%
\pgfpathlineto{\pgfqpoint{4.338461in}{0.707230in}}%
\pgfpathlineto{\pgfqpoint{4.339326in}{0.787560in}}%
\pgfpathlineto{\pgfqpoint{4.341058in}{0.672835in}}%
\pgfpathlineto{\pgfqpoint{4.341924in}{0.684484in}}%
\pgfpathlineto{\pgfqpoint{4.342786in}{0.693165in}}%
\pgfpathlineto{\pgfqpoint{4.344511in}{0.732104in}}%
\pgfpathlineto{\pgfqpoint{4.345375in}{0.739794in}}%
\pgfpathlineto{\pgfqpoint{4.347106in}{0.702285in}}%
\pgfpathlineto{\pgfqpoint{4.347971in}{0.723898in}}%
\pgfpathlineto{\pgfqpoint{4.348837in}{0.809063in}}%
\pgfpathlineto{\pgfqpoint{4.350566in}{0.684594in}}%
\pgfpathlineto{\pgfqpoint{4.352298in}{0.736424in}}%
\pgfpathlineto{\pgfqpoint{4.353162in}{0.686279in}}%
\pgfpathlineto{\pgfqpoint{4.355757in}{0.770677in}}%
\pgfpathlineto{\pgfqpoint{4.356622in}{0.718550in}}%
\pgfpathlineto{\pgfqpoint{4.358349in}{0.793642in}}%
\pgfpathlineto{\pgfqpoint{4.359212in}{0.772325in}}%
\pgfpathlineto{\pgfqpoint{4.360077in}{0.803314in}}%
\pgfpathlineto{\pgfqpoint{4.363537in}{0.683315in}}%
\pgfpathlineto{\pgfqpoint{4.365264in}{0.719834in}}%
\pgfpathlineto{\pgfqpoint{4.366126in}{0.656501in}}%
\pgfpathlineto{\pgfqpoint{4.366989in}{0.728441in}}%
\pgfpathlineto{\pgfqpoint{4.367853in}{0.694375in}}%
\pgfpathlineto{\pgfqpoint{4.368718in}{0.755914in}}%
\pgfpathlineto{\pgfqpoint{4.370449in}{0.698330in}}%
\pgfpathlineto{\pgfqpoint{4.371313in}{0.758878in}}%
\pgfpathlineto{\pgfqpoint{4.373044in}{0.663314in}}%
\pgfpathlineto{\pgfqpoint{4.373909in}{0.665510in}}%
\pgfpathlineto{\pgfqpoint{4.374774in}{0.745438in}}%
\pgfpathlineto{\pgfqpoint{4.375640in}{0.682617in}}%
\pgfpathlineto{\pgfqpoint{4.376505in}{0.738588in}}%
\pgfpathlineto{\pgfqpoint{4.378235in}{0.665108in}}%
\pgfpathlineto{\pgfqpoint{4.380827in}{0.736392in}}%
\pgfpathlineto{\pgfqpoint{4.383424in}{0.708846in}}%
\pgfpathlineto{\pgfqpoint{4.384290in}{0.673278in}}%
\pgfpathlineto{\pgfqpoint{4.386020in}{0.743132in}}%
\pgfpathlineto{\pgfqpoint{4.387751in}{0.693238in}}%
\pgfpathlineto{\pgfqpoint{4.388616in}{0.692325in}}%
\pgfpathlineto{\pgfqpoint{4.389481in}{0.706426in}}%
\pgfpathlineto{\pgfqpoint{4.390346in}{0.695512in}}%
\pgfpathlineto{\pgfqpoint{4.391212in}{0.719395in}}%
\pgfpathlineto{\pgfqpoint{4.392077in}{0.709285in}}%
\pgfpathlineto{\pgfqpoint{4.392939in}{0.660346in}}%
\pgfpathlineto{\pgfqpoint{4.394670in}{0.722213in}}%
\pgfpathlineto{\pgfqpoint{4.395533in}{0.719834in}}%
\pgfpathlineto{\pgfqpoint{4.397262in}{0.662729in}}%
\pgfpathlineto{\pgfqpoint{4.398127in}{0.756575in}}%
\pgfpathlineto{\pgfqpoint{4.398990in}{0.703681in}}%
\pgfpathlineto{\pgfqpoint{4.401586in}{0.745730in}}%
\pgfpathlineto{\pgfqpoint{4.403318in}{0.685987in}}%
\pgfpathlineto{\pgfqpoint{4.405047in}{0.725693in}}%
\pgfpathlineto{\pgfqpoint{4.405913in}{0.703056in}}%
\pgfpathlineto{\pgfqpoint{4.406779in}{0.716829in}}%
\pgfpathlineto{\pgfqpoint{4.407644in}{0.632581in}}%
\pgfpathlineto{\pgfqpoint{4.408510in}{0.805620in}}%
\pgfpathlineto{\pgfqpoint{4.409376in}{0.718404in}}%
\pgfpathlineto{\pgfqpoint{4.410240in}{0.723642in}}%
\pgfpathlineto{\pgfqpoint{4.411970in}{0.701152in}}%
\pgfpathlineto{\pgfqpoint{4.412836in}{0.691225in}}%
\pgfpathlineto{\pgfqpoint{4.413701in}{0.771408in}}%
\pgfpathlineto{\pgfqpoint{4.415430in}{0.682836in}}%
\pgfpathlineto{\pgfqpoint{4.418891in}{0.762581in}}%
\pgfpathlineto{\pgfqpoint{4.419756in}{0.660382in}}%
\pgfpathlineto{\pgfqpoint{4.420621in}{0.677744in}}%
\pgfpathlineto{\pgfqpoint{4.422347in}{0.735474in}}%
\pgfpathlineto{\pgfqpoint{4.423211in}{0.718624in}}%
\pgfpathlineto{\pgfqpoint{4.424076in}{0.748073in}}%
\pgfpathlineto{\pgfqpoint{4.424939in}{0.680051in}}%
\pgfpathlineto{\pgfqpoint{4.425802in}{0.693092in}}%
\pgfpathlineto{\pgfqpoint{4.426667in}{0.704339in}}%
\pgfpathlineto{\pgfqpoint{4.428398in}{0.771408in}}%
\pgfpathlineto{\pgfqpoint{4.430129in}{0.725583in}}%
\pgfpathlineto{\pgfqpoint{4.430994in}{0.772764in}}%
\pgfpathlineto{\pgfqpoint{4.431859in}{0.711299in}}%
\pgfpathlineto{\pgfqpoint{4.432725in}{0.781445in}}%
\pgfpathlineto{\pgfqpoint{4.433588in}{0.777307in}}%
\pgfpathlineto{\pgfqpoint{4.436184in}{0.663095in}}%
\pgfpathlineto{\pgfqpoint{4.437050in}{0.757781in}}%
\pgfpathlineto{\pgfqpoint{4.437915in}{0.750456in}}%
\pgfpathlineto{\pgfqpoint{4.438780in}{0.726866in}}%
\pgfpathlineto{\pgfqpoint{4.439645in}{0.665108in}}%
\pgfpathlineto{\pgfqpoint{4.442235in}{0.746940in}}%
\pgfpathlineto{\pgfqpoint{4.443964in}{0.699759in}}%
\pgfpathlineto{\pgfqpoint{4.444829in}{0.723642in}}%
\pgfpathlineto{\pgfqpoint{4.445695in}{0.680676in}}%
\pgfpathlineto{\pgfqpoint{4.446561in}{0.699101in}}%
\pgfpathlineto{\pgfqpoint{4.447427in}{0.698257in}}%
\pgfpathlineto{\pgfqpoint{4.449156in}{0.629028in}}%
\pgfpathlineto{\pgfqpoint{4.450020in}{0.698846in}}%
\pgfpathlineto{\pgfqpoint{4.450885in}{0.683753in}}%
\pgfpathlineto{\pgfqpoint{4.451751in}{0.715546in}}%
\pgfpathlineto{\pgfqpoint{4.452617in}{0.702947in}}%
\pgfpathlineto{\pgfqpoint{4.453484in}{0.717706in}}%
\pgfpathlineto{\pgfqpoint{4.454349in}{0.684411in}}%
\pgfpathlineto{\pgfqpoint{4.455213in}{0.740675in}}%
\pgfpathlineto{\pgfqpoint{4.456077in}{0.674082in}}%
\pgfpathlineto{\pgfqpoint{4.456943in}{0.756133in}}%
\pgfpathlineto{\pgfqpoint{4.457807in}{0.742872in}}%
\pgfpathlineto{\pgfqpoint{4.458672in}{0.730894in}}%
\pgfpathlineto{\pgfqpoint{4.459537in}{0.672762in}}%
\pgfpathlineto{\pgfqpoint{4.461266in}{0.715765in}}%
\pgfpathlineto{\pgfqpoint{4.462131in}{0.700417in}}%
\pgfpathlineto{\pgfqpoint{4.462995in}{0.737342in}}%
\pgfpathlineto{\pgfqpoint{4.464723in}{0.702581in}}%
\pgfpathlineto{\pgfqpoint{4.465588in}{0.691371in}}%
\pgfpathlineto{\pgfqpoint{4.466453in}{0.749758in}}%
\pgfpathlineto{\pgfqpoint{4.467318in}{0.693604in}}%
\pgfpathlineto{\pgfqpoint{4.468183in}{0.706792in}}%
\pgfpathlineto{\pgfqpoint{4.469047in}{0.671995in}}%
\pgfpathlineto{\pgfqpoint{4.470777in}{0.742068in}}%
\pgfpathlineto{\pgfqpoint{4.471642in}{0.702800in}}%
\pgfpathlineto{\pgfqpoint{4.472507in}{0.785180in}}%
\pgfpathlineto{\pgfqpoint{4.473373in}{0.651701in}}%
\pgfpathlineto{\pgfqpoint{4.474235in}{0.675840in}}%
\pgfpathlineto{\pgfqpoint{4.476831in}{0.794665in}}%
\pgfpathlineto{\pgfqpoint{4.478560in}{0.700783in}}%
\pgfpathlineto{\pgfqpoint{4.479425in}{0.709244in}}%
\pgfpathlineto{\pgfqpoint{4.480290in}{0.653788in}}%
\pgfpathlineto{\pgfqpoint{4.482023in}{0.761078in}}%
\pgfpathlineto{\pgfqpoint{4.482888in}{0.653569in}}%
\pgfpathlineto{\pgfqpoint{4.483754in}{0.749685in}}%
\pgfpathlineto{\pgfqpoint{4.484620in}{0.703349in}}%
\pgfpathlineto{\pgfqpoint{4.485486in}{0.770052in}}%
\pgfpathlineto{\pgfqpoint{4.486350in}{0.730382in}}%
\pgfpathlineto{\pgfqpoint{4.487216in}{0.800675in}}%
\pgfpathlineto{\pgfqpoint{4.488081in}{0.680124in}}%
\pgfpathlineto{\pgfqpoint{4.488945in}{0.695033in}}%
\pgfpathlineto{\pgfqpoint{4.489808in}{0.657451in}}%
\pgfpathlineto{\pgfqpoint{4.492402in}{0.737853in}}%
\pgfpathlineto{\pgfqpoint{4.493266in}{0.654889in}}%
\pgfpathlineto{\pgfqpoint{4.496727in}{0.757343in}}%
\pgfpathlineto{\pgfqpoint{4.497591in}{0.682178in}}%
\pgfpathlineto{\pgfqpoint{4.499321in}{0.776499in}}%
\pgfpathlineto{\pgfqpoint{4.501049in}{0.699686in}}%
\pgfpathlineto{\pgfqpoint{4.501914in}{0.670200in}}%
\pgfpathlineto{\pgfqpoint{4.504511in}{0.759211in}}%
\pgfpathlineto{\pgfqpoint{4.505375in}{0.690421in}}%
\pgfpathlineto{\pgfqpoint{4.506240in}{0.716171in}}%
\pgfpathlineto{\pgfqpoint{4.507105in}{0.689799in}}%
\pgfpathlineto{\pgfqpoint{4.507970in}{0.727711in}}%
\pgfpathlineto{\pgfqpoint{4.508835in}{0.682617in}}%
\pgfpathlineto{\pgfqpoint{4.509700in}{0.716573in}}%
\pgfpathlineto{\pgfqpoint{4.510564in}{0.654998in}}%
\pgfpathlineto{\pgfqpoint{4.512294in}{0.728186in}}%
\pgfpathlineto{\pgfqpoint{4.513159in}{0.657085in}}%
\pgfpathlineto{\pgfqpoint{4.514025in}{0.677890in}}%
\pgfpathlineto{\pgfqpoint{4.514890in}{0.639577in}}%
\pgfpathlineto{\pgfqpoint{4.515753in}{0.670346in}}%
\pgfpathlineto{\pgfqpoint{4.516619in}{0.612032in}}%
\pgfpathlineto{\pgfqpoint{4.518346in}{0.712322in}}%
\pgfpathlineto{\pgfqpoint{4.519212in}{0.686060in}}%
\pgfpathlineto{\pgfqpoint{4.520076in}{0.750822in}}%
\pgfpathlineto{\pgfqpoint{4.521806in}{0.682434in}}%
\pgfpathlineto{\pgfqpoint{4.522670in}{0.685767in}}%
\pgfpathlineto{\pgfqpoint{4.523535in}{0.694741in}}%
\pgfpathlineto{\pgfqpoint{4.524400in}{0.728368in}}%
\pgfpathlineto{\pgfqpoint{4.525263in}{0.716463in}}%
\pgfpathlineto{\pgfqpoint{4.526128in}{0.669100in}}%
\pgfpathlineto{\pgfqpoint{4.527857in}{0.701042in}}%
\pgfpathlineto{\pgfqpoint{4.528722in}{0.646719in}}%
\pgfpathlineto{\pgfqpoint{4.529586in}{0.654852in}}%
\pgfpathlineto{\pgfqpoint{4.530452in}{0.663387in}}%
\pgfpathlineto{\pgfqpoint{4.532179in}{0.753900in}}%
\pgfpathlineto{\pgfqpoint{4.533043in}{0.689211in}}%
\pgfpathlineto{\pgfqpoint{4.533909in}{0.730236in}}%
\pgfpathlineto{\pgfqpoint{4.535638in}{0.664962in}}%
\pgfpathlineto{\pgfqpoint{4.537370in}{0.696170in}}%
\pgfpathlineto{\pgfqpoint{4.538236in}{0.695878in}}%
\pgfpathlineto{\pgfqpoint{4.539100in}{0.697892in}}%
\pgfpathlineto{\pgfqpoint{4.539965in}{0.734962in}}%
\pgfpathlineto{\pgfqpoint{4.541691in}{0.688845in}}%
\pgfpathlineto{\pgfqpoint{4.542555in}{0.743205in}}%
\pgfpathlineto{\pgfqpoint{4.544282in}{0.709285in}}%
\pgfpathlineto{\pgfqpoint{4.545145in}{0.723683in}}%
\pgfpathlineto{\pgfqpoint{4.546010in}{0.706467in}}%
\pgfpathlineto{\pgfqpoint{4.546874in}{0.768663in}}%
\pgfpathlineto{\pgfqpoint{4.548603in}{0.714376in}}%
\pgfpathlineto{\pgfqpoint{4.549468in}{0.682324in}}%
\pgfpathlineto{\pgfqpoint{4.551196in}{0.726354in}}%
\pgfpathlineto{\pgfqpoint{4.552060in}{0.722400in}}%
\pgfpathlineto{\pgfqpoint{4.552925in}{0.725770in}}%
\pgfpathlineto{\pgfqpoint{4.553786in}{0.744301in}}%
\pgfpathlineto{\pgfqpoint{4.554650in}{0.721299in}}%
\pgfpathlineto{\pgfqpoint{4.555516in}{0.728368in}}%
\pgfpathlineto{\pgfqpoint{4.556379in}{0.747598in}}%
\pgfpathlineto{\pgfqpoint{4.558975in}{0.665182in}}%
\pgfpathlineto{\pgfqpoint{4.559839in}{0.678881in}}%
\pgfpathlineto{\pgfqpoint{4.560705in}{0.674999in}}%
\pgfpathlineto{\pgfqpoint{4.561571in}{0.682470in}}%
\pgfpathlineto{\pgfqpoint{4.562436in}{0.721921in}}%
\pgfpathlineto{\pgfqpoint{4.563300in}{0.705289in}}%
\pgfpathlineto{\pgfqpoint{4.564162in}{0.657195in}}%
\pgfpathlineto{\pgfqpoint{4.565028in}{0.667195in}}%
\pgfpathlineto{\pgfqpoint{4.566758in}{0.696426in}}%
\pgfpathlineto{\pgfqpoint{4.567622in}{0.674301in}}%
\pgfpathlineto{\pgfqpoint{4.568487in}{0.741077in}}%
\pgfpathlineto{\pgfqpoint{4.569351in}{0.617010in}}%
\pgfpathlineto{\pgfqpoint{4.571948in}{0.751480in}}%
\pgfpathlineto{\pgfqpoint{4.573680in}{0.686023in}}%
\pgfpathlineto{\pgfqpoint{4.574545in}{0.723788in}}%
\pgfpathlineto{\pgfqpoint{4.575410in}{0.644815in}}%
\pgfpathlineto{\pgfqpoint{4.577141in}{0.746867in}}%
\pgfpathlineto{\pgfqpoint{4.578005in}{0.711226in}}%
\pgfpathlineto{\pgfqpoint{4.578869in}{0.583057in}}%
\pgfpathlineto{\pgfqpoint{4.579732in}{0.765293in}}%
\pgfpathlineto{\pgfqpoint{4.583189in}{0.584852in}}%
\pgfpathlineto{\pgfqpoint{4.584916in}{0.663825in}}%
\pgfpathlineto{\pgfqpoint{4.586645in}{0.674374in}}%
\pgfpathlineto{\pgfqpoint{4.587511in}{0.767266in}}%
\pgfpathlineto{\pgfqpoint{4.588377in}{0.662031in}}%
\pgfpathlineto{\pgfqpoint{4.590109in}{0.758001in}}%
\pgfpathlineto{\pgfqpoint{4.591841in}{0.690786in}}%
\pgfpathlineto{\pgfqpoint{4.592706in}{0.751041in}}%
\pgfpathlineto{\pgfqpoint{4.593571in}{0.670127in}}%
\pgfpathlineto{\pgfqpoint{4.594436in}{0.730821in}}%
\pgfpathlineto{\pgfqpoint{4.595300in}{0.667195in}}%
\pgfpathlineto{\pgfqpoint{4.596165in}{0.673716in}}%
\pgfpathlineto{\pgfqpoint{4.597894in}{0.684927in}}%
\pgfpathlineto{\pgfqpoint{4.598760in}{0.634266in}}%
\pgfpathlineto{\pgfqpoint{4.599625in}{0.689357in}}%
\pgfpathlineto{\pgfqpoint{4.600489in}{0.647194in}}%
\pgfpathlineto{\pgfqpoint{4.603082in}{0.732689in}}%
\pgfpathlineto{\pgfqpoint{4.603947in}{0.667301in}}%
\pgfpathlineto{\pgfqpoint{4.604812in}{0.711514in}}%
\pgfpathlineto{\pgfqpoint{4.607407in}{0.643089in}}%
\pgfpathlineto{\pgfqpoint{4.609137in}{0.657816in}}%
\pgfpathlineto{\pgfqpoint{4.610002in}{0.688585in}}%
\pgfpathlineto{\pgfqpoint{4.610868in}{0.675803in}}%
\pgfpathlineto{\pgfqpoint{4.613459in}{0.739429in}}%
\pgfpathlineto{\pgfqpoint{4.615186in}{0.612251in}}%
\pgfpathlineto{\pgfqpoint{4.616918in}{0.717231in}}%
\pgfpathlineto{\pgfqpoint{4.617784in}{0.652286in}}%
\pgfpathlineto{\pgfqpoint{4.619515in}{0.716131in}}%
\pgfpathlineto{\pgfqpoint{4.620380in}{0.676607in}}%
\pgfpathlineto{\pgfqpoint{4.621246in}{0.683238in}}%
\pgfpathlineto{\pgfqpoint{4.622110in}{0.681845in}}%
\pgfpathlineto{\pgfqpoint{4.622974in}{0.716756in}}%
\pgfpathlineto{\pgfqpoint{4.623839in}{0.596684in}}%
\pgfpathlineto{\pgfqpoint{4.626436in}{0.691627in}}%
\pgfpathlineto{\pgfqpoint{4.627301in}{0.696243in}}%
\pgfpathlineto{\pgfqpoint{4.629028in}{0.756571in}}%
\pgfpathlineto{\pgfqpoint{4.629894in}{0.735141in}}%
\pgfpathlineto{\pgfqpoint{4.630756in}{0.748991in}}%
\pgfpathlineto{\pgfqpoint{4.631622in}{0.712103in}}%
\pgfpathlineto{\pgfqpoint{4.632489in}{0.776313in}}%
\pgfpathlineto{\pgfqpoint{4.633355in}{0.633312in}}%
\pgfpathlineto{\pgfqpoint{4.634220in}{0.728547in}}%
\pgfpathlineto{\pgfqpoint{4.635081in}{0.724812in}}%
\pgfpathlineto{\pgfqpoint{4.635945in}{0.712505in}}%
\pgfpathlineto{\pgfqpoint{4.637675in}{0.805141in}}%
\pgfpathlineto{\pgfqpoint{4.640272in}{0.662798in}}%
\pgfpathlineto{\pgfqpoint{4.642003in}{0.771002in}}%
\pgfpathlineto{\pgfqpoint{4.642868in}{0.724227in}}%
\pgfpathlineto{\pgfqpoint{4.643733in}{0.748256in}}%
\pgfpathlineto{\pgfqpoint{4.645460in}{0.692357in}}%
\pgfpathlineto{\pgfqpoint{4.646326in}{0.722651in}}%
\pgfpathlineto{\pgfqpoint{4.648057in}{0.678914in}}%
\pgfpathlineto{\pgfqpoint{4.649788in}{0.705322in}}%
\pgfpathlineto{\pgfqpoint{4.650652in}{0.654812in}}%
\pgfpathlineto{\pgfqpoint{4.651516in}{0.706020in}}%
\pgfpathlineto{\pgfqpoint{4.652381in}{0.671918in}}%
\pgfpathlineto{\pgfqpoint{4.653247in}{0.699207in}}%
\pgfpathlineto{\pgfqpoint{4.654112in}{0.685435in}}%
\pgfpathlineto{\pgfqpoint{4.654976in}{0.698915in}}%
\pgfpathlineto{\pgfqpoint{4.655843in}{0.794300in}}%
\pgfpathlineto{\pgfqpoint{4.657572in}{0.703787in}}%
\pgfpathlineto{\pgfqpoint{4.658437in}{0.734264in}}%
\pgfpathlineto{\pgfqpoint{4.659302in}{0.727707in}}%
\pgfpathlineto{\pgfqpoint{4.660168in}{0.661040in}}%
\pgfpathlineto{\pgfqpoint{4.662761in}{0.792944in}}%
\pgfpathlineto{\pgfqpoint{4.664490in}{0.709281in}}%
\pgfpathlineto{\pgfqpoint{4.665354in}{0.700929in}}%
\pgfpathlineto{\pgfqpoint{4.666218in}{0.665693in}}%
\pgfpathlineto{\pgfqpoint{4.667082in}{0.722578in}}%
\pgfpathlineto{\pgfqpoint{4.667948in}{0.684923in}}%
\pgfpathlineto{\pgfqpoint{4.668812in}{0.759686in}}%
\pgfpathlineto{\pgfqpoint{4.669677in}{0.605438in}}%
\pgfpathlineto{\pgfqpoint{4.670542in}{0.711957in}}%
\pgfpathlineto{\pgfqpoint{4.672272in}{0.620234in}}%
\pgfpathlineto{\pgfqpoint{4.673137in}{0.728182in}}%
\pgfpathlineto{\pgfqpoint{4.674000in}{0.705801in}}%
\pgfpathlineto{\pgfqpoint{4.674866in}{0.702504in}}%
\pgfpathlineto{\pgfqpoint{4.675730in}{0.720930in}}%
\pgfpathlineto{\pgfqpoint{4.676595in}{0.765764in}}%
\pgfpathlineto{\pgfqpoint{4.677460in}{0.705070in}}%
\pgfpathlineto{\pgfqpoint{4.678325in}{0.723017in}}%
\pgfpathlineto{\pgfqpoint{4.680055in}{0.690672in}}%
\pgfpathlineto{\pgfqpoint{4.681784in}{0.726533in}}%
\pgfpathlineto{\pgfqpoint{4.682650in}{0.791368in}}%
\pgfpathlineto{\pgfqpoint{4.683516in}{0.663310in}}%
\pgfpathlineto{\pgfqpoint{4.685244in}{0.745324in}}%
\pgfpathlineto{\pgfqpoint{4.686110in}{0.715984in}}%
\pgfpathlineto{\pgfqpoint{4.686974in}{0.747890in}}%
\pgfpathlineto{\pgfqpoint{4.688705in}{0.684192in}}%
\pgfpathlineto{\pgfqpoint{4.689568in}{0.704153in}}%
\pgfpathlineto{\pgfqpoint{4.690432in}{0.772139in}}%
\pgfpathlineto{\pgfqpoint{4.691297in}{0.755361in}}%
\pgfpathlineto{\pgfqpoint{4.692162in}{0.719062in}}%
\pgfpathlineto{\pgfqpoint{4.693027in}{0.756425in}}%
\pgfpathlineto{\pgfqpoint{4.693889in}{0.706280in}}%
\pgfpathlineto{\pgfqpoint{4.695621in}{0.745584in}}%
\pgfpathlineto{\pgfqpoint{4.697352in}{0.727597in}}%
\pgfpathlineto{\pgfqpoint{4.699083in}{0.671077in}}%
\pgfpathlineto{\pgfqpoint{4.699949in}{0.739356in}}%
\pgfpathlineto{\pgfqpoint{4.701675in}{0.685621in}}%
\pgfpathlineto{\pgfqpoint{4.702540in}{0.710235in}}%
\pgfpathlineto{\pgfqpoint{4.703406in}{0.618845in}}%
\pgfpathlineto{\pgfqpoint{4.704271in}{0.634632in}}%
\pgfpathlineto{\pgfqpoint{4.705135in}{0.649687in}}%
\pgfpathlineto{\pgfqpoint{4.706864in}{0.724633in}}%
\pgfpathlineto{\pgfqpoint{4.707730in}{0.707490in}}%
\pgfpathlineto{\pgfqpoint{4.708594in}{0.714596in}}%
\pgfpathlineto{\pgfqpoint{4.709461in}{0.694818in}}%
\pgfpathlineto{\pgfqpoint{4.710325in}{0.733095in}}%
\pgfpathlineto{\pgfqpoint{4.712056in}{0.652838in}}%
\pgfpathlineto{\pgfqpoint{4.713785in}{0.750164in}}%
\pgfpathlineto{\pgfqpoint{4.715513in}{0.657012in}}%
\pgfpathlineto{\pgfqpoint{4.716380in}{0.707234in}}%
\pgfpathlineto{\pgfqpoint{4.717245in}{0.613096in}}%
\pgfpathlineto{\pgfqpoint{4.718111in}{0.691557in}}%
\pgfpathlineto{\pgfqpoint{4.719839in}{0.655071in}}%
\pgfpathlineto{\pgfqpoint{4.720705in}{0.722400in}}%
\pgfpathlineto{\pgfqpoint{4.721569in}{0.652655in}}%
\pgfpathlineto{\pgfqpoint{4.723300in}{0.752032in}}%
\pgfpathlineto{\pgfqpoint{4.725030in}{0.713240in}}%
\pgfpathlineto{\pgfqpoint{4.725894in}{0.719030in}}%
\pgfpathlineto{\pgfqpoint{4.726759in}{0.718185in}}%
\pgfpathlineto{\pgfqpoint{4.729353in}{0.606867in}}%
\pgfpathlineto{\pgfqpoint{4.730215in}{0.724706in}}%
\pgfpathlineto{\pgfqpoint{4.731080in}{0.682434in}}%
\pgfpathlineto{\pgfqpoint{4.731947in}{0.692873in}}%
\pgfpathlineto{\pgfqpoint{4.732812in}{0.756133in}}%
\pgfpathlineto{\pgfqpoint{4.733678in}{0.739063in}}%
\pgfpathlineto{\pgfqpoint{4.734543in}{0.691663in}}%
\pgfpathlineto{\pgfqpoint{4.735409in}{0.703641in}}%
\pgfpathlineto{\pgfqpoint{4.736271in}{0.729976in}}%
\pgfpathlineto{\pgfqpoint{4.737133in}{0.720784in}}%
\pgfpathlineto{\pgfqpoint{4.737997in}{0.696462in}}%
\pgfpathlineto{\pgfqpoint{4.739729in}{0.740858in}}%
\pgfpathlineto{\pgfqpoint{4.741460in}{0.698111in}}%
\pgfpathlineto{\pgfqpoint{4.742327in}{0.779870in}}%
\pgfpathlineto{\pgfqpoint{4.743193in}{0.719208in}}%
\pgfpathlineto{\pgfqpoint{4.744060in}{0.754923in}}%
\pgfpathlineto{\pgfqpoint{4.746655in}{0.660049in}}%
\pgfpathlineto{\pgfqpoint{4.747518in}{0.667451in}}%
\pgfpathlineto{\pgfqpoint{4.748384in}{0.718291in}}%
\pgfpathlineto{\pgfqpoint{4.749246in}{0.715911in}}%
\pgfpathlineto{\pgfqpoint{4.750112in}{0.716902in}}%
\pgfpathlineto{\pgfqpoint{4.752708in}{0.762800in}}%
\pgfpathlineto{\pgfqpoint{4.754440in}{0.694814in}}%
\pgfpathlineto{\pgfqpoint{4.755305in}{0.745803in}}%
\pgfpathlineto{\pgfqpoint{4.757035in}{0.659505in}}%
\pgfpathlineto{\pgfqpoint{4.758765in}{0.757343in}}%
\pgfpathlineto{\pgfqpoint{4.760492in}{0.780747in}}%
\pgfpathlineto{\pgfqpoint{4.761358in}{0.704778in}}%
\pgfpathlineto{\pgfqpoint{4.763954in}{0.829978in}}%
\pgfpathlineto{\pgfqpoint{4.766549in}{0.688037in}}%
\pgfpathlineto{\pgfqpoint{4.767415in}{0.744593in}}%
\pgfpathlineto{\pgfqpoint{4.770012in}{0.682617in}}%
\pgfpathlineto{\pgfqpoint{4.770878in}{0.769134in}}%
\pgfpathlineto{\pgfqpoint{4.771744in}{0.715034in}}%
\pgfpathlineto{\pgfqpoint{4.772610in}{0.717048in}}%
\pgfpathlineto{\pgfqpoint{4.773477in}{0.729063in}}%
\pgfpathlineto{\pgfqpoint{4.774342in}{0.690640in}}%
\pgfpathlineto{\pgfqpoint{4.775208in}{0.744447in}}%
\pgfpathlineto{\pgfqpoint{4.776073in}{0.691444in}}%
\pgfpathlineto{\pgfqpoint{4.777802in}{0.765106in}}%
\pgfpathlineto{\pgfqpoint{4.779534in}{0.682690in}}%
\pgfpathlineto{\pgfqpoint{4.780401in}{0.721847in}}%
\pgfpathlineto{\pgfqpoint{4.781267in}{0.637929in}}%
\pgfpathlineto{\pgfqpoint{4.782998in}{0.800967in}}%
\pgfpathlineto{\pgfqpoint{4.783865in}{0.735766in}}%
\pgfpathlineto{\pgfqpoint{4.784730in}{0.735986in}}%
\pgfpathlineto{\pgfqpoint{4.785596in}{0.732360in}}%
\pgfpathlineto{\pgfqpoint{4.786461in}{0.795875in}}%
\pgfpathlineto{\pgfqpoint{4.787327in}{0.751187in}}%
\pgfpathlineto{\pgfqpoint{4.788193in}{0.795583in}}%
\pgfpathlineto{\pgfqpoint{4.789059in}{0.704778in}}%
\pgfpathlineto{\pgfqpoint{4.789925in}{0.742799in}}%
\pgfpathlineto{\pgfqpoint{4.792520in}{0.696974in}}%
\pgfpathlineto{\pgfqpoint{4.793384in}{0.776719in}}%
\pgfpathlineto{\pgfqpoint{4.795115in}{0.689503in}}%
\pgfpathlineto{\pgfqpoint{4.795980in}{0.753055in}}%
\pgfpathlineto{\pgfqpoint{4.796845in}{0.721076in}}%
\pgfpathlineto{\pgfqpoint{4.798574in}{0.788843in}}%
\pgfpathlineto{\pgfqpoint{4.801166in}{0.717633in}}%
\pgfpathlineto{\pgfqpoint{4.802893in}{0.681553in}}%
\pgfpathlineto{\pgfqpoint{4.804623in}{0.715692in}}%
\pgfpathlineto{\pgfqpoint{4.805486in}{0.705801in}}%
\pgfpathlineto{\pgfqpoint{4.806350in}{0.681845in}}%
\pgfpathlineto{\pgfqpoint{4.808078in}{0.821845in}}%
\pgfpathlineto{\pgfqpoint{4.809809in}{0.668365in}}%
\pgfpathlineto{\pgfqpoint{4.810674in}{0.799392in}}%
\pgfpathlineto{\pgfqpoint{4.811537in}{0.732944in}}%
\pgfpathlineto{\pgfqpoint{4.812399in}{0.772066in}}%
\pgfpathlineto{\pgfqpoint{4.814129in}{0.665693in}}%
\pgfpathlineto{\pgfqpoint{4.815858in}{0.760088in}}%
\pgfpathlineto{\pgfqpoint{4.816722in}{0.747562in}}%
\pgfpathlineto{\pgfqpoint{4.817588in}{0.734264in}}%
\pgfpathlineto{\pgfqpoint{4.818454in}{0.675219in}}%
\pgfpathlineto{\pgfqpoint{4.821048in}{0.732506in}}%
\pgfpathlineto{\pgfqpoint{4.821914in}{0.672580in}}%
\pgfpathlineto{\pgfqpoint{4.823647in}{0.715984in}}%
\pgfpathlineto{\pgfqpoint{4.824513in}{0.682064in}}%
\pgfpathlineto{\pgfqpoint{4.825379in}{0.698915in}}%
\pgfpathlineto{\pgfqpoint{4.826244in}{0.750672in}}%
\pgfpathlineto{\pgfqpoint{4.827110in}{0.723935in}}%
\pgfpathlineto{\pgfqpoint{4.827976in}{0.640747in}}%
\pgfpathlineto{\pgfqpoint{4.828842in}{0.669063in}}%
\pgfpathlineto{\pgfqpoint{4.829709in}{0.649760in}}%
\pgfpathlineto{\pgfqpoint{4.831440in}{0.691078in}}%
\pgfpathlineto{\pgfqpoint{4.832305in}{0.693312in}}%
\pgfpathlineto{\pgfqpoint{4.833170in}{0.643897in}}%
\pgfpathlineto{\pgfqpoint{4.834033in}{0.695252in}}%
\pgfpathlineto{\pgfqpoint{4.834897in}{0.650491in}}%
\pgfpathlineto{\pgfqpoint{4.835762in}{0.656022in}}%
\pgfpathlineto{\pgfqpoint{4.837494in}{0.745105in}}%
\pgfpathlineto{\pgfqpoint{4.839223in}{0.638733in}}%
\pgfpathlineto{\pgfqpoint{4.840088in}{0.720930in}}%
\pgfpathlineto{\pgfqpoint{4.840953in}{0.710381in}}%
\pgfpathlineto{\pgfqpoint{4.841818in}{0.762321in}}%
\pgfpathlineto{\pgfqpoint{4.842682in}{0.699098in}}%
\pgfpathlineto{\pgfqpoint{4.842682in}{0.699098in}}%
\pgfusepath{stroke}%
\end{pgfscope}%
\begin{pgfscope}%
\pgfsetrectcap%
\pgfsetmiterjoin%
\pgfsetlinewidth{0.803000pt}%
\definecolor{currentstroke}{rgb}{0.000000,0.000000,0.000000}%
\pgfsetstrokecolor{currentstroke}%
\pgfsetdash{}{0pt}%
\pgfpathmoveto{\pgfqpoint{0.483776in}{0.538014in}}%
\pgfpathlineto{\pgfqpoint{0.483776in}{1.122895in}}%
\pgfusepath{stroke}%
\end{pgfscope}%
\begin{pgfscope}%
\pgfsetrectcap%
\pgfsetmiterjoin%
\pgfsetlinewidth{0.803000pt}%
\definecolor{currentstroke}{rgb}{0.000000,0.000000,0.000000}%
\pgfsetstrokecolor{currentstroke}%
\pgfsetdash{}{0pt}%
\pgfpathmoveto{\pgfqpoint{5.050249in}{0.538014in}}%
\pgfpathlineto{\pgfqpoint{5.050249in}{1.122895in}}%
\pgfusepath{stroke}%
\end{pgfscope}%
\begin{pgfscope}%
\pgfsetrectcap%
\pgfsetmiterjoin%
\pgfsetlinewidth{0.803000pt}%
\definecolor{currentstroke}{rgb}{0.000000,0.000000,0.000000}%
\pgfsetstrokecolor{currentstroke}%
\pgfsetdash{}{0pt}%
\pgfpathmoveto{\pgfqpoint{0.483776in}{0.538014in}}%
\pgfpathlineto{\pgfqpoint{5.050249in}{0.538014in}}%
\pgfusepath{stroke}%
\end{pgfscope}%
\begin{pgfscope}%
\pgfsetrectcap%
\pgfsetmiterjoin%
\pgfsetlinewidth{0.803000pt}%
\definecolor{currentstroke}{rgb}{0.000000,0.000000,0.000000}%
\pgfsetstrokecolor{currentstroke}%
\pgfsetdash{}{0pt}%
\pgfpathmoveto{\pgfqpoint{0.483776in}{1.122895in}}%
\pgfpathlineto{\pgfqpoint{5.050249in}{1.122895in}}%
\pgfusepath{stroke}%
\end{pgfscope}%
\end{pgfpicture}%
\makeatother%
\endgroup%

    \includegraphics[width=0.75\textwidth]{example-image-golden}
    \caption{Voltage noise of an LM399, measured with a bandwidth of \qtyrange{0.1}{10}{\Hz}.}
    \label{fig:noise_lm399}
\end{figure}

Measuring two references against each other would then result in around \qty{2.1}{\micro \volt} of noise. This make distinguishing the jumps possible, but challenging.

A third option is to use a high-pass filter and an amplifier. Additionally, the signal can be low-pass filtered to remove any excess high frequency noise. This approach also requires less resolution than directly measuring the voltage, because the signal-to-noise-ratio is improved du the amplifier. It is therefore possible to use an off-the-shelf analog-to-digital converter (ADC). One such circuit, along with some examples, is demonstrated in \cite{technote_ti_popcorn_noise,kay2012operational}. It must be noted, that due to the high-pass filtering, it not possible to measure slow voltage drifts using this method.

The forth and final option presented here, is approaching the problem in the frequency domain and requires a low-noise amplifier with a low frequency cutoff. As it was already discussed in section \ref{sec:theory_burst_noise}, popcorn noise is found to have a frequency dependence of $1/f^2$. This can be used to distinguish it from other random noise processes that show a frequency dependence of $1/f$. A good example of an op-amp, that has excessive burst in comparison to a good sample is given in \textit{The Art of Electronics} \citep[p. 478]{horowitz1989}. Going to frequencies below \qty{10}{\Hz}, one can sort the references by their noise spectrum.

In this work only options one and two were were tested, as it was said above, with options three and four there is a chicken and egg problem. One needs a number on known good devices to compare other DUTs to. At the start of the evaluation, most of the data available about the LM399 was from the data sheet. Compiling a dataset of the performance of dozens of LM399 is expensive and time consuming and companies typically treat such data as a closely guarded secret.

The next section deals with the choice of multimeter to satisfy the requirements test the Zener diodes according to options one and two, so either directly measuring the output voltage or difference of a known good sample against the DUT.

\subsection{Choosing a Multimeter for Testing Zener Diodes}
The DMM used plays an important role for the test setup. In this section, some of the challenges, that can be encountered will be discussed. The expected amplitude of the popcorn noise is around \qty[per-mode=symbol]{0.5}{\micro\volt \per \volt} or \qty{3.5}{\micro\volt} of the output voltage, when considering the \qty{7}{\volt} Zener voltage of the LM399 diode.

The \qty{7}{\volt} will typically be measured on the \qty{10}{\volt} range. It is not a trivial task, because a signal-to-noise-ratio of \qty[per-mode=symbol]{0.35}{\micro\volt \per \volt} or more than \qty{130}{\decibel} is required. This calls for a device, that not only has the required resolution, but also the stability over time and temperature to ensure the measurement will not be distorted by the DMM.

Therefore, a voltmeter with lower noise and a more stable reference, than the DUT is mandatory. This only leaves the class of very low noise \num{7.5} or \num{8.5} digit multimeters. These multimeters feature a different type of voltage reference, because the LM399 is not suitable due to its noise. The only Zener diodes that meet those requirements are the Analog Devices LTZ1000 \cite{datasheet_LTZ1000}, the Motorola SZA263 (out of production) and the Linear Technology (LT) LTFLU-1, a proprietary design by Fluke and LT. The LTZ1000, for example, is specified for a typical noise of \qty{1.2}{\micro\volt_{pp}} in a frequency range of \qtyrange{0.1}{10}{\Hz} \cite{datasheet_LTZ1000}. Additionally, in comparison to the LM399, those Zener diodes do not suffer from the popcorn noise issue.

The equipment manufacturers typically have a preference for one of those diodes. Keysight utilizes the LTZ1000, Fluke uses the SZA263 (in older devices) or the LTFLU-1 in newer model, while Keithley employs the LTZ1000 in their \device{Model 2002} and the LTFLU-1 in the newer \device{DMM7510}, because they were bought by Fortive, the same company that owns Fluke. To sum it it up, Keysight uses the LTZ1000 and Fluke/Keithley the LTFLU-1 in their top end meters.

Comparing only \num{7.5} and \num{8.5} digit voltmeters, narrows down the choice of multimeters considerably. The market for high-end \num{8.5} digit DMMs is limited and therefore every device on the market caters for a certain niche. It is therefore prudent to look at their specifications to choose the correct device for this purpose. In table \ref{tab:list_of_dmms} a list of popular \num{8.5} DMMs can be found. Several models included in the table, are already discontinued, but these DMMs can still be acquired on the second-hand market.

\begin{table}[h]
    \centering
    \begin{tabular}{ |l|l|l| }
        \hline
        Manufacturer & Model & Remarks \\
        \hline
        Advantest & \device{R6581} & Discontinued. Scanner cards available. \\
        Datron/Wavetek & \device{1812} & Discontinued. Wavetek was bought by Fluke. \\
        Fluke & \device{8508A} & Discontinued. \qty{20}{\volt} range. \\
        Fluke & \device{8588A} & In production. \\
        Keithley/Tektronix & \device{2002} & In production. Scanner card available. \qty{20}{\volt} range. \\
        Keysight & \device{3458A} & In Production. \\
        Solartron & \device{7081} & Discontinued. Slow. \\
        Transmille & \device{8104} & In Production. External scanner available. Slow. \\
        \hline
    \end{tabular}
    \caption{Overview of \num{8.5} digit multimeters.}
    \label{tab:list_of_dmms}
\end{table}

While the author has not tested every multimeter in table \ref{tab:list_of_dmms}, it is possible to judge some of them apriori by their specifications. The \device{Solartron 7081} (also sold as \device{Guildline 9578}) is a less optimal choice, because a conversion takes \qty{52}{\s} for \num{8.5} digits. The discontinued \device{Fluke 8508A} and the \device{Wavetek 1812} multimeter are very similar devices, because Fluke bought Wavetek in 2000 and as a result, the \device{Fluke 8508A} is more of an update to the \device{Wavetek 1812} than a new device. They are both in included in the list, because it is very rare to see one of the Fluke devices on the second hand market, while the \device{Wavetek 1812} can be found with a bit of patience. Again they are fairly slow, taking \qty{25}{\second} for a conversion at \num{8.5} digits.

The other multimeters are still in production and similar in price, but their field of use is slightly is different. The \device{Fluke 8588A} excels at stability and features a modern user interface, whereas the \device{Keysight 3458A} is unbeaten in linearity and noise. A detailed comparison of those two meters can be found in the work of \citeauthor*{article_fluke_8588A_noise} \cite{article_fluke_8588A_noise}. The \device{Keithley Model 2002} focuses on its scanning capability and the \device{Transmille 8104} does have electrometer functions. Unfortunately, the \device{8104} is also fairly slow at \num{8.5} digit with conversions taking \qty{4}{\s} at its fastest setting \cite{datasheet_transmille8104}, so it will not be considered.

To narrow it down even further, several \num{7.5} and \num{8.5} digit multimeters were tested. The results of those tests will be discussed here to give an impression of the performance of these devices. The tested multimeters are the \device{Keysight 3458A}, the \device{Keithley Model 2002}, the \device{Keysight 34470A} and a \device{Keithley DMM6500}. The \device{3458A} was chosen, because it is very fast and already used in section \ref{} of this work. The \device{Model 2002} was chosen for its internal scanning unit. The \device{34470A} was chosen as a lower-end and cheaper alternative and because it is a fairly low noise device. Finally the \device{DMM6500} is on the list to compare a DMM with an LM399 reference. A \device{Fluke 8588A} was not tested, because it was not released at the time of testing and the older model \device{8508A} is considered too slow as mentioned above.

\minisec{The tests}

Two test were run on this selection of devices. The first one was done using a \device{Fluke 5440B} calibrator supplying \qty{10}{\volt} to all mulimeters and taking readings over the course of a week. This data was used to estimate the noise and the stability of the multimeters, including burst noise. The noise of the DMM at \qty{10}{\volt} is typically not found in the datasheet, because the noise performance is usually quoted for shorted inputs, which does not include the internal reference noise. This test allows to check for popcorn noise of the internal reference. The calibrator has a specified output noise of \qty{< 1.5}{\micro \volt} within a bandwidth of \qtyrange{0.1}{10}{\Hz} at \qty{1}{\volt} and is stable to within \qty{5}{\micro \volt_{rms}} over \qty{30}{\day}, a specification far superior to the LM399.

The second test was done using a known bad LM399 voltage reference instead of the calibrator. This test was done to see how well a DMM can make out the popcorn noise.

Based on these two tests, a multimeter was chosen for an automated test setup to bin the LM399s.

\minisec{Test Setup}

The tests were done in a stable and monitored lab environment, with a temperature deviation of at most $\Delta T = \qty{\pm 0.2}{\kelvin}$. All multimeters were connected to the same DUT. Although this might potentially cause interference between the multimeters due to the pump out current spikes caused by the switching interals, no ill effects, like voltage offsets or increased noise, were observed during the setup of the tests. A more detailed discussion of the pump out current of the \device{3458A} can be found in \cite{article_3458A_input_mpedance}.

The three \num{8.5} and \num{7.5} digit multimeters were connected using shielded cables, either Pomona 1167-60 or self-made cables. See section \ref{} for details on the self-made cables. The GUARD terminal of the calibrator was connected to chassis GROUND at the calibrator and then connected to the cable shield. On the \device{3458A}, the shield was connected to the GUARD terminal and the GUARD switch was set to open according to the manual \cite{manual_keysight3458a}. For the other multimeters, that do not have a GUARD terminal, the shield was left floating at the DMM side. Additionally the \device{Fluke 5440B}, the \device{HP 3458A} and the \device{Keysight 34470A} have an autocalibration routine, which was run once prior to the measurement. The detailed settings used for the DMMs can be found in the appendix \ref{appendix:dmm_test} on page \pageref{appendix:dmm_test}, a summary ist given in table \ref{tab:dmm_settings_concise} to show the important differences.

\begin{table}[ht]
    \centering
    \begin{tabular}{lcc}
        \toprule
        DMM& Integration time in \unit{NPLC}& Conversion time in \unit{\s}\\
        \midrule
        \device{HP 3458A}& 100 & \qty{0}{\s}\\
        \device{Keithley Model 2002} & 40& \qty{0}{\s}\\
        \device{Keysight 34470A}& 100    & \qty{0}{\s}\\
        \device{Keithley DMM6500}& 90& \qty{0}{\s}\\
        \bottomrule
    \end{tabular}
    \caption{Concise list of differences in the settings used for comparing the DMMs.}
    \label{tab:dmm_settings_concise}
\end{table}

All DMMs were configured to have a similar conversion time. This lead to different integration times, which are given in power line cycles at \qty{50}{\Hz}. The \device{Model 2002} takes considerable longer for a measurement than the Keysight multimeters. The reason is the auto-zero function, which is shown in figure \ref{fig:dmm_autozero_comparison}. The \device{Model 2002} does three steps when doing auto-zeroing, it measures the signal, the zero point for an offset compensation and also the reference voltage for a gain correction. In comparison, the \device{3458A} only corrects for the offset drift. The gain is adjusted when using the ACAL function. The former auto-zero routine, therefore takes longer by one half, but results in more stable measurements.

These measurements were done by measuring the output voltage of a pre-production version of the reference PCB for the digital current driver. The reference board was kept at \qty{23}{\celsius} in a custom thermal chamber. The chamber is detailed in section \ref{}. Additionally, a \qty{500}{\g}  bag of Bentonite desiccant was added to keep the references at a low humidity of around \qty{20}{\percent} relative humidity. The reference board inserted into a motherboard holding up to 4 reference modules. The motherboard, also called LM399 breakout board, provides the voltage regulators and the operational amplifier for the kelvin sensed pins of the reference. The multimeter was directly connected to the reference via a DB9 connector, without an other components in between the reference and the DMM like buffers, multiplexers or filters. The DMM itself was exposed to the ambient temperature of the lab. The setup is shown in figure \ref{fig:lm399_vs_34470a_setup}.

\begin{figure}[ht]
    \centering
    \resizebox {0.8\textwidth} {!} {
        \import{figures/}{34470A_vs_LM399.tex}
    } % resizebox
    \caption{Measurement setup for tesing an LM399 reference board with the \device{Keysight 34470A}}
    \label{fig:lm399_vs_34470a_setup}
\end{figure}

The reference boards amplify the Zener voltage to \qty{10}{\volt}, which improves the signal to noise ratio, because it makes use of the full DMM range. The \qty{10}{\volt} range is typically the lowest (relative) noise and lowest drift range those multimeter because no internal pre-amplifiers of attenuators are required. It is important to keep the temperature drift of the DMM low or at least predictable, because the device is exposed to the ambient laboratory and not in a temperature controlled environment like the references.

\newpage
The reference is a negative \qty{10}{\volt} reference that uses a self-biasing technique to derive its \qty{1}{\mA} Zener current from its own \qty{-10}{\volt} output. The details of this circuit are discussed in section \ref{}.

% \begin{figure}[h]
%     \centering
%     \scalebox{0.7}{%
%         \import{figures/}{lm399_reference_circuit.tex}
%     } % scalebox
%     \caption{Self-biased LM399 negative voltage reference.}
%     \label{fig:lm399_negative_10V}
% \end{figure}

With the amplified output the expected burst noise step size of about \qty[per-mode=symbol]{0.5}{\micro\volt \per \volt}, becomes \qty{5}{\micro\volt}. The resolution of the \qty{10}{\volt} range of the \device{34470A} is \qty{100}{\nano \volt}, but the measurement is not limited by quantization. See section \ref{} of this work for a detailed characterization.

\begin{figure}[ht]
    \centering
    %% Creator: Matplotlib, PGF backend
%%
%% To include the figure in your LaTeX document, write
%%   \input{<filename>.pgf}
%%
%% Make sure the required packages are loaded in your preamble
%%   \usepackage{pgf}
%%
%% Also ensure that all the required font packages are loaded; for instance,
%% the lmodern package is sometimes necessary when using math font.
%%   \usepackage{lmodern}
%%
%% Figures using additional raster images can only be included by \input if
%% they are in the same directory as the main LaTeX file. For loading figures
%% from other directories you can use the `import` package
%%   \usepackage{import}
%%
%% and then include the figures with
%%   \import{<path to file>}{<filename>.pgf}
%%
%% Matplotlib used the following preamble
%%   \usepackage{siunitx}
%%   \usepackage{fontspec}
%%
\begingroup%
\makeatletter%
\begin{pgfpicture}%
\pgfpathrectangle{\pgfpointorigin}{\pgfqpoint{5.208662in}{3.219130in}}%
\pgfusepath{use as bounding box, clip}%
\begin{pgfscope}%
\pgfsetbuttcap%
\pgfsetmiterjoin%
\definecolor{currentfill}{rgb}{1.000000,1.000000,1.000000}%
\pgfsetfillcolor{currentfill}%
\pgfsetlinewidth{0.000000pt}%
\definecolor{currentstroke}{rgb}{1.000000,1.000000,1.000000}%
\pgfsetstrokecolor{currentstroke}%
\pgfsetdash{}{0pt}%
\pgfpathmoveto{\pgfqpoint{0.000000in}{0.000000in}}%
\pgfpathlineto{\pgfqpoint{5.208662in}{0.000000in}}%
\pgfpathlineto{\pgfqpoint{5.208662in}{3.219130in}}%
\pgfpathlineto{\pgfqpoint{0.000000in}{3.219130in}}%
\pgfpathlineto{\pgfqpoint{0.000000in}{0.000000in}}%
\pgfpathclose%
\pgfusepath{fill}%
\end{pgfscope}%
\begin{pgfscope}%
\pgfsetbuttcap%
\pgfsetmiterjoin%
\definecolor{currentfill}{rgb}{1.000000,1.000000,1.000000}%
\pgfsetfillcolor{currentfill}%
\pgfsetlinewidth{0.000000pt}%
\definecolor{currentstroke}{rgb}{0.000000,0.000000,0.000000}%
\pgfsetstrokecolor{currentstroke}%
\pgfsetstrokeopacity{0.000000}%
\pgfsetdash{}{0pt}%
\pgfpathmoveto{\pgfqpoint{0.634869in}{0.539544in}}%
\pgfpathlineto{\pgfqpoint{4.514985in}{0.539544in}}%
\pgfpathlineto{\pgfqpoint{4.514985in}{2.944887in}}%
\pgfpathlineto{\pgfqpoint{0.634869in}{2.944887in}}%
\pgfpathlineto{\pgfqpoint{0.634869in}{0.539544in}}%
\pgfpathclose%
\pgfusepath{fill}%
\end{pgfscope}%
\begin{pgfscope}%
\pgfsetbuttcap%
\pgfsetroundjoin%
\definecolor{currentfill}{rgb}{0.000000,0.000000,0.000000}%
\pgfsetfillcolor{currentfill}%
\pgfsetlinewidth{0.803000pt}%
\definecolor{currentstroke}{rgb}{0.000000,0.000000,0.000000}%
\pgfsetstrokecolor{currentstroke}%
\pgfsetdash{}{0pt}%
\pgfsys@defobject{currentmarker}{\pgfqpoint{0.000000in}{-0.048611in}}{\pgfqpoint{0.000000in}{0.000000in}}{%
\pgfpathmoveto{\pgfqpoint{0.000000in}{0.000000in}}%
\pgfpathlineto{\pgfqpoint{0.000000in}{-0.048611in}}%
\pgfusepath{stroke,fill}%
}%
\begin{pgfscope}%
\pgfsys@transformshift{0.811157in}{0.539544in}%
\pgfsys@useobject{currentmarker}{}%
\end{pgfscope}%
\end{pgfscope}%
\begin{pgfscope}%
\definecolor{textcolor}{rgb}{0.000000,0.000000,0.000000}%
\pgfsetstrokecolor{textcolor}%
\pgfsetfillcolor{textcolor}%
\pgftext[x=0.811157in,y=0.442322in,,top]{\color{textcolor}\rmfamily\fontsize{8.000000}{9.600000}\selectfont \(\displaystyle {00{:}00}\)}%
\end{pgfscope}%
\begin{pgfscope}%
\pgfsetbuttcap%
\pgfsetroundjoin%
\definecolor{currentfill}{rgb}{0.000000,0.000000,0.000000}%
\pgfsetfillcolor{currentfill}%
\pgfsetlinewidth{0.803000pt}%
\definecolor{currentstroke}{rgb}{0.000000,0.000000,0.000000}%
\pgfsetstrokecolor{currentstroke}%
\pgfsetdash{}{0pt}%
\pgfsys@defobject{currentmarker}{\pgfqpoint{0.000000in}{-0.048611in}}{\pgfqpoint{0.000000in}{0.000000in}}{%
\pgfpathmoveto{\pgfqpoint{0.000000in}{0.000000in}}%
\pgfpathlineto{\pgfqpoint{0.000000in}{-0.048611in}}%
\pgfusepath{stroke,fill}%
}%
\begin{pgfscope}%
\pgfsys@transformshift{1.252105in}{0.539544in}%
\pgfsys@useobject{currentmarker}{}%
\end{pgfscope}%
\end{pgfscope}%
\begin{pgfscope}%
\definecolor{textcolor}{rgb}{0.000000,0.000000,0.000000}%
\pgfsetstrokecolor{textcolor}%
\pgfsetfillcolor{textcolor}%
\pgftext[x=1.252105in,y=0.442322in,,top]{\color{textcolor}\rmfamily\fontsize{8.000000}{9.600000}\selectfont \(\displaystyle {03{:}00}\)}%
\end{pgfscope}%
\begin{pgfscope}%
\pgfsetbuttcap%
\pgfsetroundjoin%
\definecolor{currentfill}{rgb}{0.000000,0.000000,0.000000}%
\pgfsetfillcolor{currentfill}%
\pgfsetlinewidth{0.803000pt}%
\definecolor{currentstroke}{rgb}{0.000000,0.000000,0.000000}%
\pgfsetstrokecolor{currentstroke}%
\pgfsetdash{}{0pt}%
\pgfsys@defobject{currentmarker}{\pgfqpoint{0.000000in}{-0.048611in}}{\pgfqpoint{0.000000in}{0.000000in}}{%
\pgfpathmoveto{\pgfqpoint{0.000000in}{0.000000in}}%
\pgfpathlineto{\pgfqpoint{0.000000in}{-0.048611in}}%
\pgfusepath{stroke,fill}%
}%
\begin{pgfscope}%
\pgfsys@transformshift{1.693053in}{0.539544in}%
\pgfsys@useobject{currentmarker}{}%
\end{pgfscope}%
\end{pgfscope}%
\begin{pgfscope}%
\definecolor{textcolor}{rgb}{0.000000,0.000000,0.000000}%
\pgfsetstrokecolor{textcolor}%
\pgfsetfillcolor{textcolor}%
\pgftext[x=1.693053in,y=0.442322in,,top]{\color{textcolor}\rmfamily\fontsize{8.000000}{9.600000}\selectfont \(\displaystyle {06{:}00}\)}%
\end{pgfscope}%
\begin{pgfscope}%
\pgfsetbuttcap%
\pgfsetroundjoin%
\definecolor{currentfill}{rgb}{0.000000,0.000000,0.000000}%
\pgfsetfillcolor{currentfill}%
\pgfsetlinewidth{0.803000pt}%
\definecolor{currentstroke}{rgb}{0.000000,0.000000,0.000000}%
\pgfsetstrokecolor{currentstroke}%
\pgfsetdash{}{0pt}%
\pgfsys@defobject{currentmarker}{\pgfqpoint{0.000000in}{-0.048611in}}{\pgfqpoint{0.000000in}{0.000000in}}{%
\pgfpathmoveto{\pgfqpoint{0.000000in}{0.000000in}}%
\pgfpathlineto{\pgfqpoint{0.000000in}{-0.048611in}}%
\pgfusepath{stroke,fill}%
}%
\begin{pgfscope}%
\pgfsys@transformshift{2.134001in}{0.539544in}%
\pgfsys@useobject{currentmarker}{}%
\end{pgfscope}%
\end{pgfscope}%
\begin{pgfscope}%
\definecolor{textcolor}{rgb}{0.000000,0.000000,0.000000}%
\pgfsetstrokecolor{textcolor}%
\pgfsetfillcolor{textcolor}%
\pgftext[x=2.134001in,y=0.442322in,,top]{\color{textcolor}\rmfamily\fontsize{8.000000}{9.600000}\selectfont \(\displaystyle {09{:}00}\)}%
\end{pgfscope}%
\begin{pgfscope}%
\pgfsetbuttcap%
\pgfsetroundjoin%
\definecolor{currentfill}{rgb}{0.000000,0.000000,0.000000}%
\pgfsetfillcolor{currentfill}%
\pgfsetlinewidth{0.803000pt}%
\definecolor{currentstroke}{rgb}{0.000000,0.000000,0.000000}%
\pgfsetstrokecolor{currentstroke}%
\pgfsetdash{}{0pt}%
\pgfsys@defobject{currentmarker}{\pgfqpoint{0.000000in}{-0.048611in}}{\pgfqpoint{0.000000in}{0.000000in}}{%
\pgfpathmoveto{\pgfqpoint{0.000000in}{0.000000in}}%
\pgfpathlineto{\pgfqpoint{0.000000in}{-0.048611in}}%
\pgfusepath{stroke,fill}%
}%
\begin{pgfscope}%
\pgfsys@transformshift{2.574949in}{0.539544in}%
\pgfsys@useobject{currentmarker}{}%
\end{pgfscope}%
\end{pgfscope}%
\begin{pgfscope}%
\definecolor{textcolor}{rgb}{0.000000,0.000000,0.000000}%
\pgfsetstrokecolor{textcolor}%
\pgfsetfillcolor{textcolor}%
\pgftext[x=2.574949in,y=0.442322in,,top]{\color{textcolor}\rmfamily\fontsize{8.000000}{9.600000}\selectfont \(\displaystyle {12{:}00}\)}%
\end{pgfscope}%
\begin{pgfscope}%
\pgfsetbuttcap%
\pgfsetroundjoin%
\definecolor{currentfill}{rgb}{0.000000,0.000000,0.000000}%
\pgfsetfillcolor{currentfill}%
\pgfsetlinewidth{0.803000pt}%
\definecolor{currentstroke}{rgb}{0.000000,0.000000,0.000000}%
\pgfsetstrokecolor{currentstroke}%
\pgfsetdash{}{0pt}%
\pgfsys@defobject{currentmarker}{\pgfqpoint{0.000000in}{-0.048611in}}{\pgfqpoint{0.000000in}{0.000000in}}{%
\pgfpathmoveto{\pgfqpoint{0.000000in}{0.000000in}}%
\pgfpathlineto{\pgfqpoint{0.000000in}{-0.048611in}}%
\pgfusepath{stroke,fill}%
}%
\begin{pgfscope}%
\pgfsys@transformshift{3.015896in}{0.539544in}%
\pgfsys@useobject{currentmarker}{}%
\end{pgfscope}%
\end{pgfscope}%
\begin{pgfscope}%
\definecolor{textcolor}{rgb}{0.000000,0.000000,0.000000}%
\pgfsetstrokecolor{textcolor}%
\pgfsetfillcolor{textcolor}%
\pgftext[x=3.015896in,y=0.442322in,,top]{\color{textcolor}\rmfamily\fontsize{8.000000}{9.600000}\selectfont \(\displaystyle {15{:}00}\)}%
\end{pgfscope}%
\begin{pgfscope}%
\pgfsetbuttcap%
\pgfsetroundjoin%
\definecolor{currentfill}{rgb}{0.000000,0.000000,0.000000}%
\pgfsetfillcolor{currentfill}%
\pgfsetlinewidth{0.803000pt}%
\definecolor{currentstroke}{rgb}{0.000000,0.000000,0.000000}%
\pgfsetstrokecolor{currentstroke}%
\pgfsetdash{}{0pt}%
\pgfsys@defobject{currentmarker}{\pgfqpoint{0.000000in}{-0.048611in}}{\pgfqpoint{0.000000in}{0.000000in}}{%
\pgfpathmoveto{\pgfqpoint{0.000000in}{0.000000in}}%
\pgfpathlineto{\pgfqpoint{0.000000in}{-0.048611in}}%
\pgfusepath{stroke,fill}%
}%
\begin{pgfscope}%
\pgfsys@transformshift{3.456844in}{0.539544in}%
\pgfsys@useobject{currentmarker}{}%
\end{pgfscope}%
\end{pgfscope}%
\begin{pgfscope}%
\definecolor{textcolor}{rgb}{0.000000,0.000000,0.000000}%
\pgfsetstrokecolor{textcolor}%
\pgfsetfillcolor{textcolor}%
\pgftext[x=3.456844in,y=0.442322in,,top]{\color{textcolor}\rmfamily\fontsize{8.000000}{9.600000}\selectfont \(\displaystyle {18{:}00}\)}%
\end{pgfscope}%
\begin{pgfscope}%
\pgfsetbuttcap%
\pgfsetroundjoin%
\definecolor{currentfill}{rgb}{0.000000,0.000000,0.000000}%
\pgfsetfillcolor{currentfill}%
\pgfsetlinewidth{0.803000pt}%
\definecolor{currentstroke}{rgb}{0.000000,0.000000,0.000000}%
\pgfsetstrokecolor{currentstroke}%
\pgfsetdash{}{0pt}%
\pgfsys@defobject{currentmarker}{\pgfqpoint{0.000000in}{-0.048611in}}{\pgfqpoint{0.000000in}{0.000000in}}{%
\pgfpathmoveto{\pgfqpoint{0.000000in}{0.000000in}}%
\pgfpathlineto{\pgfqpoint{0.000000in}{-0.048611in}}%
\pgfusepath{stroke,fill}%
}%
\begin{pgfscope}%
\pgfsys@transformshift{3.897792in}{0.539544in}%
\pgfsys@useobject{currentmarker}{}%
\end{pgfscope}%
\end{pgfscope}%
\begin{pgfscope}%
\definecolor{textcolor}{rgb}{0.000000,0.000000,0.000000}%
\pgfsetstrokecolor{textcolor}%
\pgfsetfillcolor{textcolor}%
\pgftext[x=3.897792in,y=0.442322in,,top]{\color{textcolor}\rmfamily\fontsize{8.000000}{9.600000}\selectfont \(\displaystyle {21{:}00}\)}%
\end{pgfscope}%
\begin{pgfscope}%
\pgfsetbuttcap%
\pgfsetroundjoin%
\definecolor{currentfill}{rgb}{0.000000,0.000000,0.000000}%
\pgfsetfillcolor{currentfill}%
\pgfsetlinewidth{0.803000pt}%
\definecolor{currentstroke}{rgb}{0.000000,0.000000,0.000000}%
\pgfsetstrokecolor{currentstroke}%
\pgfsetdash{}{0pt}%
\pgfsys@defobject{currentmarker}{\pgfqpoint{0.000000in}{-0.048611in}}{\pgfqpoint{0.000000in}{0.000000in}}{%
\pgfpathmoveto{\pgfqpoint{0.000000in}{0.000000in}}%
\pgfpathlineto{\pgfqpoint{0.000000in}{-0.048611in}}%
\pgfusepath{stroke,fill}%
}%
\begin{pgfscope}%
\pgfsys@transformshift{4.338740in}{0.539544in}%
\pgfsys@useobject{currentmarker}{}%
\end{pgfscope}%
\end{pgfscope}%
\begin{pgfscope}%
\definecolor{textcolor}{rgb}{0.000000,0.000000,0.000000}%
\pgfsetstrokecolor{textcolor}%
\pgfsetfillcolor{textcolor}%
\pgftext[x=4.338740in,y=0.442322in,,top]{\color{textcolor}\rmfamily\fontsize{8.000000}{9.600000}\selectfont \(\displaystyle {00{:}00}\)}%
\end{pgfscope}%
\begin{pgfscope}%
\definecolor{textcolor}{rgb}{0.000000,0.000000,0.000000}%
\pgfsetstrokecolor{textcolor}%
\pgfsetfillcolor{textcolor}%
\pgftext[x=2.574927in,y=0.288100in,,top]{\color{textcolor}\rmfamily\fontsize{10.000000}{12.000000}\selectfont Time (UTC)}%
\end{pgfscope}%
\begin{pgfscope}%
\pgfsetbuttcap%
\pgfsetroundjoin%
\definecolor{currentfill}{rgb}{0.000000,0.000000,0.000000}%
\pgfsetfillcolor{currentfill}%
\pgfsetlinewidth{0.803000pt}%
\definecolor{currentstroke}{rgb}{0.000000,0.000000,0.000000}%
\pgfsetstrokecolor{currentstroke}%
\pgfsetdash{}{0pt}%
\pgfsys@defobject{currentmarker}{\pgfqpoint{-0.048611in}{0.000000in}}{\pgfqpoint{-0.000000in}{0.000000in}}{%
\pgfpathmoveto{\pgfqpoint{-0.000000in}{0.000000in}}%
\pgfpathlineto{\pgfqpoint{-0.048611in}{0.000000in}}%
\pgfusepath{stroke,fill}%
}%
\begin{pgfscope}%
\pgfsys@transformshift{0.634869in}{0.746268in}%
\pgfsys@useobject{currentmarker}{}%
\end{pgfscope}%
\end{pgfscope}%
\begin{pgfscope}%
\definecolor{textcolor}{rgb}{0.000000,0.000000,0.000000}%
\pgfsetstrokecolor{textcolor}%
\pgfsetfillcolor{textcolor}%
\pgftext[x=0.327767in, y=0.707713in, left, base]{\color{textcolor}\rmfamily\fontsize{8.000000}{9.600000}\selectfont \(\displaystyle {\ensuremath{-}15}\)}%
\end{pgfscope}%
\begin{pgfscope}%
\pgfsetbuttcap%
\pgfsetroundjoin%
\definecolor{currentfill}{rgb}{0.000000,0.000000,0.000000}%
\pgfsetfillcolor{currentfill}%
\pgfsetlinewidth{0.803000pt}%
\definecolor{currentstroke}{rgb}{0.000000,0.000000,0.000000}%
\pgfsetstrokecolor{currentstroke}%
\pgfsetdash{}{0pt}%
\pgfsys@defobject{currentmarker}{\pgfqpoint{-0.048611in}{0.000000in}}{\pgfqpoint{-0.000000in}{0.000000in}}{%
\pgfpathmoveto{\pgfqpoint{-0.000000in}{0.000000in}}%
\pgfpathlineto{\pgfqpoint{-0.048611in}{0.000000in}}%
\pgfusepath{stroke,fill}%
}%
\begin{pgfscope}%
\pgfsys@transformshift{0.634869in}{1.158849in}%
\pgfsys@useobject{currentmarker}{}%
\end{pgfscope}%
\end{pgfscope}%
\begin{pgfscope}%
\definecolor{textcolor}{rgb}{0.000000,0.000000,0.000000}%
\pgfsetstrokecolor{textcolor}%
\pgfsetfillcolor{textcolor}%
\pgftext[x=0.327767in, y=1.120293in, left, base]{\color{textcolor}\rmfamily\fontsize{8.000000}{9.600000}\selectfont \(\displaystyle {\ensuremath{-}10}\)}%
\end{pgfscope}%
\begin{pgfscope}%
\pgfsetbuttcap%
\pgfsetroundjoin%
\definecolor{currentfill}{rgb}{0.000000,0.000000,0.000000}%
\pgfsetfillcolor{currentfill}%
\pgfsetlinewidth{0.803000pt}%
\definecolor{currentstroke}{rgb}{0.000000,0.000000,0.000000}%
\pgfsetstrokecolor{currentstroke}%
\pgfsetdash{}{0pt}%
\pgfsys@defobject{currentmarker}{\pgfqpoint{-0.048611in}{0.000000in}}{\pgfqpoint{-0.000000in}{0.000000in}}{%
\pgfpathmoveto{\pgfqpoint{-0.000000in}{0.000000in}}%
\pgfpathlineto{\pgfqpoint{-0.048611in}{0.000000in}}%
\pgfusepath{stroke,fill}%
}%
\begin{pgfscope}%
\pgfsys@transformshift{0.634869in}{1.571429in}%
\pgfsys@useobject{currentmarker}{}%
\end{pgfscope}%
\end{pgfscope}%
\begin{pgfscope}%
\definecolor{textcolor}{rgb}{0.000000,0.000000,0.000000}%
\pgfsetstrokecolor{textcolor}%
\pgfsetfillcolor{textcolor}%
\pgftext[x=0.386796in, y=1.532873in, left, base]{\color{textcolor}\rmfamily\fontsize{8.000000}{9.600000}\selectfont \(\displaystyle {\ensuremath{-}5}\)}%
\end{pgfscope}%
\begin{pgfscope}%
\pgfsetbuttcap%
\pgfsetroundjoin%
\definecolor{currentfill}{rgb}{0.000000,0.000000,0.000000}%
\pgfsetfillcolor{currentfill}%
\pgfsetlinewidth{0.803000pt}%
\definecolor{currentstroke}{rgb}{0.000000,0.000000,0.000000}%
\pgfsetstrokecolor{currentstroke}%
\pgfsetdash{}{0pt}%
\pgfsys@defobject{currentmarker}{\pgfqpoint{-0.048611in}{0.000000in}}{\pgfqpoint{-0.000000in}{0.000000in}}{%
\pgfpathmoveto{\pgfqpoint{-0.000000in}{0.000000in}}%
\pgfpathlineto{\pgfqpoint{-0.048611in}{0.000000in}}%
\pgfusepath{stroke,fill}%
}%
\begin{pgfscope}%
\pgfsys@transformshift{0.634869in}{1.984009in}%
\pgfsys@useobject{currentmarker}{}%
\end{pgfscope}%
\end{pgfscope}%
\begin{pgfscope}%
\definecolor{textcolor}{rgb}{0.000000,0.000000,0.000000}%
\pgfsetstrokecolor{textcolor}%
\pgfsetfillcolor{textcolor}%
\pgftext[x=0.478618in, y=1.945454in, left, base]{\color{textcolor}\rmfamily\fontsize{8.000000}{9.600000}\selectfont \(\displaystyle {0}\)}%
\end{pgfscope}%
\begin{pgfscope}%
\pgfsetbuttcap%
\pgfsetroundjoin%
\definecolor{currentfill}{rgb}{0.000000,0.000000,0.000000}%
\pgfsetfillcolor{currentfill}%
\pgfsetlinewidth{0.803000pt}%
\definecolor{currentstroke}{rgb}{0.000000,0.000000,0.000000}%
\pgfsetstrokecolor{currentstroke}%
\pgfsetdash{}{0pt}%
\pgfsys@defobject{currentmarker}{\pgfqpoint{-0.048611in}{0.000000in}}{\pgfqpoint{-0.000000in}{0.000000in}}{%
\pgfpathmoveto{\pgfqpoint{-0.000000in}{0.000000in}}%
\pgfpathlineto{\pgfqpoint{-0.048611in}{0.000000in}}%
\pgfusepath{stroke,fill}%
}%
\begin{pgfscope}%
\pgfsys@transformshift{0.634869in}{2.396589in}%
\pgfsys@useobject{currentmarker}{}%
\end{pgfscope}%
\end{pgfscope}%
\begin{pgfscope}%
\definecolor{textcolor}{rgb}{0.000000,0.000000,0.000000}%
\pgfsetstrokecolor{textcolor}%
\pgfsetfillcolor{textcolor}%
\pgftext[x=0.478618in, y=2.358034in, left, base]{\color{textcolor}\rmfamily\fontsize{8.000000}{9.600000}\selectfont \(\displaystyle {5}\)}%
\end{pgfscope}%
\begin{pgfscope}%
\pgfsetbuttcap%
\pgfsetroundjoin%
\definecolor{currentfill}{rgb}{0.000000,0.000000,0.000000}%
\pgfsetfillcolor{currentfill}%
\pgfsetlinewidth{0.803000pt}%
\definecolor{currentstroke}{rgb}{0.000000,0.000000,0.000000}%
\pgfsetstrokecolor{currentstroke}%
\pgfsetdash{}{0pt}%
\pgfsys@defobject{currentmarker}{\pgfqpoint{-0.048611in}{0.000000in}}{\pgfqpoint{-0.000000in}{0.000000in}}{%
\pgfpathmoveto{\pgfqpoint{-0.000000in}{0.000000in}}%
\pgfpathlineto{\pgfqpoint{-0.048611in}{0.000000in}}%
\pgfusepath{stroke,fill}%
}%
\begin{pgfscope}%
\pgfsys@transformshift{0.634869in}{2.809170in}%
\pgfsys@useobject{currentmarker}{}%
\end{pgfscope}%
\end{pgfscope}%
\begin{pgfscope}%
\definecolor{textcolor}{rgb}{0.000000,0.000000,0.000000}%
\pgfsetstrokecolor{textcolor}%
\pgfsetfillcolor{textcolor}%
\pgftext[x=0.419589in, y=2.770614in, left, base]{\color{textcolor}\rmfamily\fontsize{8.000000}{9.600000}\selectfont \(\displaystyle {10}\)}%
\end{pgfscope}%
\begin{pgfscope}%
\definecolor{textcolor}{rgb}{0.000000,0.000000,0.000000}%
\pgfsetstrokecolor{textcolor}%
\pgfsetfillcolor{textcolor}%
\pgftext[x=0.272211in,y=1.742216in,,bottom,rotate=90.000000]{\color{textcolor}\rmfamily\fontsize{10.000000}{12.000000}\selectfont Voltage deviation in V}%
\end{pgfscope}%
\begin{pgfscope}%
\definecolor{textcolor}{rgb}{0.000000,0.000000,0.000000}%
\pgfsetstrokecolor{textcolor}%
\pgfsetfillcolor{textcolor}%
\pgftext[x=0.634869in,y=2.986554in,left,base]{\color{textcolor}\rmfamily\fontsize{8.000000}{9.600000}\selectfont \(\displaystyle \times{10^{\ensuremath{-}6}}{}\)}%
\end{pgfscope}%
\begin{pgfscope}%
\pgfpathrectangle{\pgfqpoint{0.634869in}{0.539544in}}{\pgfqpoint{3.880116in}{2.405343in}}%
\pgfusepath{clip}%
\pgfsetrectcap%
\pgfsetroundjoin%
\pgfsetlinewidth{0.501875pt}%
\definecolor{currentstroke}{rgb}{0.121569,0.466667,0.705882}%
\pgfsetstrokecolor{currentstroke}%
\pgfsetstrokeopacity{0.700000}%
\pgfsetdash{}{0pt}%
\pgfpathmoveto{\pgfqpoint{0.811238in}{1.927877in}}%
\pgfpathlineto{\pgfqpoint{0.811442in}{1.919625in}}%
\pgfpathlineto{\pgfqpoint{0.812667in}{2.233186in}}%
\pgfpathlineto{\pgfqpoint{0.814096in}{2.142418in}}%
\pgfpathlineto{\pgfqpoint{0.814912in}{2.092909in}}%
\pgfpathlineto{\pgfqpoint{0.815321in}{2.200180in}}%
\pgfpathlineto{\pgfqpoint{0.816341in}{1.969135in}}%
\pgfpathlineto{\pgfqpoint{0.815729in}{2.241438in}}%
\pgfpathlineto{\pgfqpoint{0.816750in}{2.117664in}}%
\pgfpathlineto{\pgfqpoint{0.817362in}{2.208431in}}%
\pgfpathlineto{\pgfqpoint{0.817770in}{2.167173in}}%
\pgfpathlineto{\pgfqpoint{0.818179in}{2.084657in}}%
\pgfpathlineto{\pgfqpoint{0.818587in}{2.200180in}}%
\pgfpathlineto{\pgfqpoint{0.819403in}{2.241438in}}%
\pgfpathlineto{\pgfqpoint{0.819812in}{2.233186in}}%
\pgfpathlineto{\pgfqpoint{0.820016in}{2.224934in}}%
\pgfpathlineto{\pgfqpoint{0.820220in}{2.249689in}}%
\pgfpathlineto{\pgfqpoint{0.820628in}{2.290947in}}%
\pgfpathlineto{\pgfqpoint{0.821241in}{2.257941in}}%
\pgfpathlineto{\pgfqpoint{0.821853in}{2.224934in}}%
\pgfpathlineto{\pgfqpoint{0.822057in}{2.257941in}}%
\pgfpathlineto{\pgfqpoint{0.822261in}{2.274444in}}%
\pgfpathlineto{\pgfqpoint{0.822465in}{2.224934in}}%
\pgfpathlineto{\pgfqpoint{0.822874in}{2.257941in}}%
\pgfpathlineto{\pgfqpoint{0.823282in}{2.224934in}}%
\pgfpathlineto{\pgfqpoint{0.823486in}{2.249689in}}%
\pgfpathlineto{\pgfqpoint{0.824507in}{2.307450in}}%
\pgfpathlineto{\pgfqpoint{0.824711in}{2.266192in}}%
\pgfpathlineto{\pgfqpoint{0.825119in}{2.299199in}}%
\pgfpathlineto{\pgfqpoint{0.825732in}{2.348709in}}%
\pgfpathlineto{\pgfqpoint{0.826140in}{2.290947in}}%
\pgfpathlineto{\pgfqpoint{0.826548in}{2.340457in}}%
\pgfpathlineto{\pgfqpoint{0.827161in}{2.175425in}}%
\pgfpathlineto{\pgfqpoint{0.827569in}{2.282696in}}%
\pgfpathlineto{\pgfqpoint{0.827773in}{2.282696in}}%
\pgfpathlineto{\pgfqpoint{0.828181in}{2.365212in}}%
\pgfpathlineto{\pgfqpoint{0.828794in}{2.274444in}}%
\pgfpathlineto{\pgfqpoint{0.830019in}{2.365212in}}%
\pgfpathlineto{\pgfqpoint{0.831039in}{2.290947in}}%
\pgfpathlineto{\pgfqpoint{0.831244in}{2.348709in}}%
\pgfpathlineto{\pgfqpoint{0.831652in}{2.274444in}}%
\pgfpathlineto{\pgfqpoint{0.831856in}{2.134167in}}%
\pgfpathlineto{\pgfqpoint{0.832468in}{2.340457in}}%
\pgfpathlineto{\pgfqpoint{0.832673in}{2.389967in}}%
\pgfpathlineto{\pgfqpoint{0.832877in}{2.340457in}}%
\pgfpathlineto{\pgfqpoint{0.833081in}{2.191928in}}%
\pgfpathlineto{\pgfqpoint{0.833897in}{2.282696in}}%
\pgfpathlineto{\pgfqpoint{0.834102in}{2.290947in}}%
\pgfpathlineto{\pgfqpoint{0.834306in}{2.109412in}}%
\pgfpathlineto{\pgfqpoint{0.834918in}{2.307450in}}%
\pgfpathlineto{\pgfqpoint{0.835122in}{2.274444in}}%
\pgfpathlineto{\pgfqpoint{0.835735in}{2.348709in}}%
\pgfpathlineto{\pgfqpoint{0.835939in}{2.266192in}}%
\pgfpathlineto{\pgfqpoint{0.836143in}{2.241438in}}%
\pgfpathlineto{\pgfqpoint{0.836755in}{2.282696in}}%
\pgfpathlineto{\pgfqpoint{0.836960in}{2.249689in}}%
\pgfpathlineto{\pgfqpoint{0.837572in}{2.241438in}}%
\pgfpathlineto{\pgfqpoint{0.837980in}{2.307450in}}%
\pgfpathlineto{\pgfqpoint{0.838184in}{2.274444in}}%
\pgfpathlineto{\pgfqpoint{0.838797in}{2.290947in}}%
\pgfpathlineto{\pgfqpoint{0.839409in}{2.406470in}}%
\pgfpathlineto{\pgfqpoint{0.839818in}{2.274444in}}%
\pgfpathlineto{\pgfqpoint{0.840634in}{2.340457in}}%
\pgfpathlineto{\pgfqpoint{0.841042in}{2.332205in}}%
\pgfpathlineto{\pgfqpoint{0.841451in}{2.249689in}}%
\pgfpathlineto{\pgfqpoint{0.841655in}{2.299199in}}%
\pgfpathlineto{\pgfqpoint{0.841859in}{2.406470in}}%
\pgfpathlineto{\pgfqpoint{0.842880in}{2.381715in}}%
\pgfpathlineto{\pgfqpoint{0.843900in}{2.464231in}}%
\pgfpathlineto{\pgfqpoint{0.844105in}{2.422973in}}%
\pgfpathlineto{\pgfqpoint{0.844309in}{2.365212in}}%
\pgfpathlineto{\pgfqpoint{0.844717in}{2.447728in}}%
\pgfpathlineto{\pgfqpoint{0.844921in}{2.439476in}}%
\pgfpathlineto{\pgfqpoint{0.845534in}{2.505489in}}%
\pgfpathlineto{\pgfqpoint{0.845738in}{2.464231in}}%
\pgfpathlineto{\pgfqpoint{0.845942in}{2.373463in}}%
\pgfpathlineto{\pgfqpoint{0.846758in}{2.447728in}}%
\pgfpathlineto{\pgfqpoint{0.847371in}{2.356960in}}%
\pgfpathlineto{\pgfqpoint{0.847779in}{2.455979in}}%
\pgfpathlineto{\pgfqpoint{0.847983in}{2.455979in}}%
\pgfpathlineto{\pgfqpoint{0.848187in}{2.480734in}}%
\pgfpathlineto{\pgfqpoint{0.848596in}{2.455979in}}%
\pgfpathlineto{\pgfqpoint{0.849821in}{2.315702in}}%
\pgfpathlineto{\pgfqpoint{0.850025in}{2.365212in}}%
\pgfpathlineto{\pgfqpoint{0.850637in}{2.323954in}}%
\pgfpathlineto{\pgfqpoint{0.850841in}{2.356960in}}%
\pgfpathlineto{\pgfqpoint{0.853291in}{2.497237in}}%
\pgfpathlineto{\pgfqpoint{0.853495in}{2.488986in}}%
\pgfpathlineto{\pgfqpoint{0.854720in}{2.406470in}}%
\pgfpathlineto{\pgfqpoint{0.855128in}{2.422973in}}%
\pgfpathlineto{\pgfqpoint{0.855537in}{2.439476in}}%
\pgfpathlineto{\pgfqpoint{0.856149in}{2.398218in}}%
\pgfpathlineto{\pgfqpoint{0.856353in}{2.464231in}}%
\pgfpathlineto{\pgfqpoint{0.857170in}{2.398218in}}%
\pgfpathlineto{\pgfqpoint{0.857782in}{2.348709in}}%
\pgfpathlineto{\pgfqpoint{0.857578in}{2.431225in}}%
\pgfpathlineto{\pgfqpoint{0.857986in}{2.422973in}}%
\pgfpathlineto{\pgfqpoint{0.858395in}{2.422973in}}%
\pgfpathlineto{\pgfqpoint{0.859007in}{2.373463in}}%
\pgfpathlineto{\pgfqpoint{0.859415in}{2.398218in}}%
\pgfpathlineto{\pgfqpoint{0.860436in}{2.422973in}}%
\pgfpathlineto{\pgfqpoint{0.860640in}{2.381715in}}%
\pgfpathlineto{\pgfqpoint{0.860844in}{2.455979in}}%
\pgfpathlineto{\pgfqpoint{0.861661in}{2.398218in}}%
\pgfpathlineto{\pgfqpoint{0.861865in}{2.398218in}}%
\pgfpathlineto{\pgfqpoint{0.862273in}{2.365212in}}%
\pgfpathlineto{\pgfqpoint{0.863090in}{2.439476in}}%
\pgfpathlineto{\pgfqpoint{0.863702in}{2.398218in}}%
\pgfpathlineto{\pgfqpoint{0.863498in}{2.447728in}}%
\pgfpathlineto{\pgfqpoint{0.863906in}{2.422973in}}%
\pgfpathlineto{\pgfqpoint{0.864315in}{2.406470in}}%
\pgfpathlineto{\pgfqpoint{0.864723in}{2.455979in}}%
\pgfpathlineto{\pgfqpoint{0.864927in}{2.315702in}}%
\pgfpathlineto{\pgfqpoint{0.865131in}{2.521992in}}%
\pgfpathlineto{\pgfqpoint{0.865744in}{2.480734in}}%
\pgfpathlineto{\pgfqpoint{0.865948in}{2.455979in}}%
\pgfpathlineto{\pgfqpoint{0.866152in}{2.249689in}}%
\pgfpathlineto{\pgfqpoint{0.866356in}{2.480734in}}%
\pgfpathlineto{\pgfqpoint{0.866969in}{2.455979in}}%
\pgfpathlineto{\pgfqpoint{0.868602in}{2.554999in}}%
\pgfpathlineto{\pgfqpoint{0.869214in}{2.439476in}}%
\pgfpathlineto{\pgfqpoint{0.869622in}{2.488986in}}%
\pgfpathlineto{\pgfqpoint{0.870847in}{2.521992in}}%
\pgfpathlineto{\pgfqpoint{0.871051in}{2.513741in}}%
\pgfpathlineto{\pgfqpoint{0.871256in}{2.596257in}}%
\pgfpathlineto{\pgfqpoint{0.871664in}{2.488986in}}%
\pgfpathlineto{\pgfqpoint{0.872072in}{2.579753in}}%
\pgfpathlineto{\pgfqpoint{0.873093in}{2.480734in}}%
\pgfpathlineto{\pgfqpoint{0.873297in}{2.554999in}}%
\pgfpathlineto{\pgfqpoint{0.874114in}{2.513741in}}%
\pgfpathlineto{\pgfqpoint{0.874930in}{2.472483in}}%
\pgfpathlineto{\pgfqpoint{0.875543in}{2.315702in}}%
\pgfpathlineto{\pgfqpoint{0.876155in}{2.579753in}}%
\pgfpathlineto{\pgfqpoint{0.876563in}{2.398218in}}%
\pgfpathlineto{\pgfqpoint{0.877380in}{2.538495in}}%
\pgfpathlineto{\pgfqpoint{0.877788in}{2.497237in}}%
\pgfpathlineto{\pgfqpoint{0.878196in}{2.191928in}}%
\pgfpathlineto{\pgfqpoint{0.878809in}{2.439476in}}%
\pgfpathlineto{\pgfqpoint{0.879013in}{2.439476in}}%
\pgfpathlineto{\pgfqpoint{0.879217in}{2.455979in}}%
\pgfpathlineto{\pgfqpoint{0.879830in}{2.472483in}}%
\pgfpathlineto{\pgfqpoint{0.880442in}{2.200180in}}%
\pgfpathlineto{\pgfqpoint{0.880646in}{2.497237in}}%
\pgfpathlineto{\pgfqpoint{0.881667in}{2.431225in}}%
\pgfpathlineto{\pgfqpoint{0.881871in}{2.431225in}}%
\pgfpathlineto{\pgfqpoint{0.883096in}{2.596257in}}%
\pgfpathlineto{\pgfqpoint{0.883300in}{2.538495in}}%
\pgfpathlineto{\pgfqpoint{0.884117in}{2.505489in}}%
\pgfpathlineto{\pgfqpoint{0.883912in}{2.563250in}}%
\pgfpathlineto{\pgfqpoint{0.884321in}{2.513741in}}%
\pgfpathlineto{\pgfqpoint{0.885137in}{2.596257in}}%
\pgfpathlineto{\pgfqpoint{0.885546in}{2.554999in}}%
\pgfpathlineto{\pgfqpoint{0.885750in}{2.554999in}}%
\pgfpathlineto{\pgfqpoint{0.886362in}{2.612760in}}%
\pgfpathlineto{\pgfqpoint{0.886975in}{2.596257in}}%
\pgfpathlineto{\pgfqpoint{0.887179in}{2.488986in}}%
\pgfpathlineto{\pgfqpoint{0.887995in}{2.579753in}}%
\pgfpathlineto{\pgfqpoint{0.888404in}{2.588005in}}%
\pgfpathlineto{\pgfqpoint{0.888812in}{2.530244in}}%
\pgfpathlineto{\pgfqpoint{0.889833in}{2.637515in}}%
\pgfpathlineto{\pgfqpoint{0.890037in}{2.621011in}}%
\pgfpathlineto{\pgfqpoint{0.890649in}{2.579753in}}%
\pgfpathlineto{\pgfqpoint{0.891057in}{2.612760in}}%
\pgfpathlineto{\pgfqpoint{0.891262in}{2.629263in}}%
\pgfpathlineto{\pgfqpoint{0.891466in}{2.579753in}}%
\pgfpathlineto{\pgfqpoint{0.893099in}{2.472483in}}%
\pgfpathlineto{\pgfqpoint{0.893303in}{2.464231in}}%
\pgfpathlineto{\pgfqpoint{0.893507in}{2.480734in}}%
\pgfpathlineto{\pgfqpoint{0.894936in}{2.654018in}}%
\pgfpathlineto{\pgfqpoint{0.895140in}{2.662269in}}%
\pgfpathlineto{\pgfqpoint{0.896569in}{2.455979in}}%
\pgfpathlineto{\pgfqpoint{0.896773in}{2.505489in}}%
\pgfpathlineto{\pgfqpoint{0.897386in}{2.546747in}}%
\pgfpathlineto{\pgfqpoint{0.897590in}{2.521992in}}%
\pgfpathlineto{\pgfqpoint{0.898611in}{2.480734in}}%
\pgfpathlineto{\pgfqpoint{0.899019in}{2.546747in}}%
\pgfpathlineto{\pgfqpoint{0.899836in}{2.538495in}}%
\pgfpathlineto{\pgfqpoint{0.900040in}{2.530244in}}%
\pgfpathlineto{\pgfqpoint{0.901265in}{2.579753in}}%
\pgfpathlineto{\pgfqpoint{0.901469in}{2.554999in}}%
\pgfpathlineto{\pgfqpoint{0.901673in}{2.612760in}}%
\pgfpathlineto{\pgfqpoint{0.902285in}{2.571502in}}%
\pgfpathlineto{\pgfqpoint{0.902489in}{2.654018in}}%
\pgfpathlineto{\pgfqpoint{0.903306in}{2.637515in}}%
\pgfpathlineto{\pgfqpoint{0.903510in}{2.604508in}}%
\pgfpathlineto{\pgfqpoint{0.904327in}{2.629263in}}%
\pgfpathlineto{\pgfqpoint{0.904531in}{2.670521in}}%
\pgfpathlineto{\pgfqpoint{0.905143in}{2.596257in}}%
\pgfpathlineto{\pgfqpoint{0.905347in}{2.637515in}}%
\pgfpathlineto{\pgfqpoint{0.905960in}{2.645766in}}%
\pgfpathlineto{\pgfqpoint{0.906572in}{2.563250in}}%
\pgfpathlineto{\pgfqpoint{0.908001in}{2.447728in}}%
\pgfpathlineto{\pgfqpoint{0.908205in}{2.505489in}}%
\pgfpathlineto{\pgfqpoint{0.909022in}{2.464231in}}%
\pgfpathlineto{\pgfqpoint{0.909226in}{2.464231in}}%
\pgfpathlineto{\pgfqpoint{0.909838in}{2.422973in}}%
\pgfpathlineto{\pgfqpoint{0.910043in}{2.472483in}}%
\pgfpathlineto{\pgfqpoint{0.910247in}{2.455979in}}%
\pgfpathlineto{\pgfqpoint{0.911880in}{2.579753in}}%
\pgfpathlineto{\pgfqpoint{0.910655in}{2.447728in}}%
\pgfpathlineto{\pgfqpoint{0.912288in}{2.554999in}}%
\pgfpathlineto{\pgfqpoint{0.912492in}{2.521992in}}%
\pgfpathlineto{\pgfqpoint{0.913105in}{2.596257in}}%
\pgfpathlineto{\pgfqpoint{0.913717in}{2.563250in}}%
\pgfpathlineto{\pgfqpoint{0.913921in}{2.629263in}}%
\pgfpathlineto{\pgfqpoint{0.916371in}{2.497237in}}%
\pgfpathlineto{\pgfqpoint{0.917188in}{2.554999in}}%
\pgfpathlineto{\pgfqpoint{0.917392in}{2.530244in}}%
\pgfpathlineto{\pgfqpoint{0.917596in}{2.480734in}}%
\pgfpathlineto{\pgfqpoint{0.918004in}{2.571502in}}%
\pgfpathlineto{\pgfqpoint{0.918412in}{2.538495in}}%
\pgfpathlineto{\pgfqpoint{0.919025in}{2.563250in}}%
\pgfpathlineto{\pgfqpoint{0.919229in}{2.546747in}}%
\pgfpathlineto{\pgfqpoint{0.919433in}{2.513741in}}%
\pgfpathlineto{\pgfqpoint{0.919841in}{2.612760in}}%
\pgfpathlineto{\pgfqpoint{0.920046in}{2.579753in}}%
\pgfpathlineto{\pgfqpoint{0.920862in}{2.554999in}}%
\pgfpathlineto{\pgfqpoint{0.920454in}{2.596257in}}%
\pgfpathlineto{\pgfqpoint{0.921066in}{2.579753in}}%
\pgfpathlineto{\pgfqpoint{0.921475in}{2.670521in}}%
\pgfpathlineto{\pgfqpoint{0.922291in}{2.662269in}}%
\pgfpathlineto{\pgfqpoint{0.923108in}{2.588005in}}%
\pgfpathlineto{\pgfqpoint{0.923720in}{2.604508in}}%
\pgfpathlineto{\pgfqpoint{0.923924in}{2.612760in}}%
\pgfpathlineto{\pgfqpoint{0.924537in}{2.497237in}}%
\pgfpathlineto{\pgfqpoint{0.925149in}{2.538495in}}%
\pgfpathlineto{\pgfqpoint{0.925966in}{2.497237in}}%
\pgfpathlineto{\pgfqpoint{0.926782in}{2.505489in}}%
\pgfpathlineto{\pgfqpoint{0.928415in}{2.596257in}}%
\pgfpathlineto{\pgfqpoint{0.929028in}{2.612760in}}%
\pgfpathlineto{\pgfqpoint{0.929640in}{2.571502in}}%
\pgfpathlineto{\pgfqpoint{0.930457in}{2.654018in}}%
\pgfpathlineto{\pgfqpoint{0.930865in}{2.596257in}}%
\pgfpathlineto{\pgfqpoint{0.931069in}{2.596257in}}%
\pgfpathlineto{\pgfqpoint{0.932090in}{2.505489in}}%
\pgfpathlineto{\pgfqpoint{0.931682in}{2.637515in}}%
\pgfpathlineto{\pgfqpoint{0.932498in}{2.521992in}}%
\pgfpathlineto{\pgfqpoint{0.932702in}{2.563250in}}%
\pgfpathlineto{\pgfqpoint{0.933315in}{2.505489in}}%
\pgfpathlineto{\pgfqpoint{0.933519in}{2.505489in}}%
\pgfpathlineto{\pgfqpoint{0.934540in}{2.621011in}}%
\pgfpathlineto{\pgfqpoint{0.933927in}{2.497237in}}%
\pgfpathlineto{\pgfqpoint{0.934948in}{2.563250in}}%
\pgfpathlineto{\pgfqpoint{0.935152in}{2.546747in}}%
\pgfpathlineto{\pgfqpoint{0.935765in}{2.579753in}}%
\pgfpathlineto{\pgfqpoint{0.935969in}{2.579753in}}%
\pgfpathlineto{\pgfqpoint{0.936173in}{2.307450in}}%
\pgfpathlineto{\pgfqpoint{0.936989in}{2.389967in}}%
\pgfpathlineto{\pgfqpoint{0.937806in}{2.612760in}}%
\pgfpathlineto{\pgfqpoint{0.938214in}{2.571502in}}%
\pgfpathlineto{\pgfqpoint{0.939847in}{2.505489in}}%
\pgfpathlineto{\pgfqpoint{0.940052in}{2.546747in}}%
\pgfpathlineto{\pgfqpoint{0.940664in}{2.505489in}}%
\pgfpathlineto{\pgfqpoint{0.940868in}{2.389967in}}%
\pgfpathlineto{\pgfqpoint{0.941481in}{2.604508in}}%
\pgfpathlineto{\pgfqpoint{0.941889in}{2.422973in}}%
\pgfpathlineto{\pgfqpoint{0.942297in}{2.554999in}}%
\pgfpathlineto{\pgfqpoint{0.943114in}{2.505489in}}%
\pgfpathlineto{\pgfqpoint{0.943318in}{2.513741in}}%
\pgfpathlineto{\pgfqpoint{0.943726in}{2.488986in}}%
\pgfpathlineto{\pgfqpoint{0.943930in}{2.439476in}}%
\pgfpathlineto{\pgfqpoint{0.944339in}{2.554999in}}%
\pgfpathlineto{\pgfqpoint{0.944543in}{2.521992in}}%
\pgfpathlineto{\pgfqpoint{0.945768in}{2.662269in}}%
\pgfpathlineto{\pgfqpoint{0.946176in}{2.637515in}}%
\pgfpathlineto{\pgfqpoint{0.946584in}{2.414721in}}%
\pgfpathlineto{\pgfqpoint{0.947197in}{2.588005in}}%
\pgfpathlineto{\pgfqpoint{0.947605in}{2.687024in}}%
\pgfpathlineto{\pgfqpoint{0.948421in}{2.645766in}}%
\pgfpathlineto{\pgfqpoint{0.949646in}{2.563250in}}%
\pgfpathlineto{\pgfqpoint{0.950463in}{2.637515in}}%
\pgfpathlineto{\pgfqpoint{0.950667in}{2.629263in}}%
\pgfpathlineto{\pgfqpoint{0.950871in}{2.546747in}}%
\pgfpathlineto{\pgfqpoint{0.951279in}{2.637515in}}%
\pgfpathlineto{\pgfqpoint{0.951688in}{2.596257in}}%
\pgfpathlineto{\pgfqpoint{0.951892in}{2.637515in}}%
\pgfpathlineto{\pgfqpoint{0.952096in}{2.588005in}}%
\pgfpathlineto{\pgfqpoint{0.952300in}{2.621011in}}%
\pgfpathlineto{\pgfqpoint{0.952504in}{2.447728in}}%
\pgfpathlineto{\pgfqpoint{0.952913in}{2.728282in}}%
\pgfpathlineto{\pgfqpoint{0.953321in}{2.678773in}}%
\pgfpathlineto{\pgfqpoint{0.954546in}{2.769540in}}%
\pgfpathlineto{\pgfqpoint{0.954750in}{2.720031in}}%
\pgfpathlineto{\pgfqpoint{0.955771in}{2.480734in}}%
\pgfpathlineto{\pgfqpoint{0.955975in}{2.645766in}}%
\pgfpathlineto{\pgfqpoint{0.956383in}{2.835553in}}%
\pgfpathlineto{\pgfqpoint{0.956995in}{2.753037in}}%
\pgfpathlineto{\pgfqpoint{0.957200in}{2.711779in}}%
\pgfpathlineto{\pgfqpoint{0.958016in}{2.744786in}}%
\pgfpathlineto{\pgfqpoint{0.958424in}{2.769540in}}%
\pgfpathlineto{\pgfqpoint{0.958833in}{2.720031in}}%
\pgfpathlineto{\pgfqpoint{0.959853in}{2.670521in}}%
\pgfpathlineto{\pgfqpoint{0.960058in}{2.678773in}}%
\pgfpathlineto{\pgfqpoint{0.960262in}{2.695276in}}%
\pgfpathlineto{\pgfqpoint{0.960670in}{2.670521in}}%
\pgfpathlineto{\pgfqpoint{0.962099in}{2.571502in}}%
\pgfpathlineto{\pgfqpoint{0.962711in}{2.703527in}}%
\pgfpathlineto{\pgfqpoint{0.962507in}{2.431225in}}%
\pgfpathlineto{\pgfqpoint{0.963120in}{2.678773in}}%
\pgfpathlineto{\pgfqpoint{0.963528in}{2.604508in}}%
\pgfpathlineto{\pgfqpoint{0.964345in}{2.621011in}}%
\pgfpathlineto{\pgfqpoint{0.964549in}{2.629263in}}%
\pgfpathlineto{\pgfqpoint{0.965365in}{2.406470in}}%
\pgfpathlineto{\pgfqpoint{0.965569in}{2.579753in}}%
\pgfpathlineto{\pgfqpoint{0.966998in}{2.678773in}}%
\pgfpathlineto{\pgfqpoint{0.968223in}{2.406470in}}%
\pgfpathlineto{\pgfqpoint{0.968427in}{2.521992in}}%
\pgfpathlineto{\pgfqpoint{0.969448in}{2.621011in}}%
\pgfpathlineto{\pgfqpoint{0.969652in}{2.612760in}}%
\pgfpathlineto{\pgfqpoint{0.969856in}{2.596257in}}%
\pgfpathlineto{\pgfqpoint{0.970061in}{2.621011in}}%
\pgfpathlineto{\pgfqpoint{0.970265in}{2.604508in}}%
\pgfpathlineto{\pgfqpoint{0.970673in}{2.662269in}}%
\pgfpathlineto{\pgfqpoint{0.971081in}{2.554999in}}%
\pgfpathlineto{\pgfqpoint{0.971285in}{2.621011in}}%
\pgfpathlineto{\pgfqpoint{0.972510in}{2.282696in}}%
\pgfpathlineto{\pgfqpoint{0.973531in}{2.662269in}}%
\pgfpathlineto{\pgfqpoint{0.973735in}{2.645766in}}%
\pgfpathlineto{\pgfqpoint{0.973939in}{2.654018in}}%
\pgfpathlineto{\pgfqpoint{0.974143in}{2.637515in}}%
\pgfpathlineto{\pgfqpoint{0.975368in}{2.480734in}}%
\pgfpathlineto{\pgfqpoint{0.974756in}{2.678773in}}%
\pgfpathlineto{\pgfqpoint{0.975572in}{2.588005in}}%
\pgfpathlineto{\pgfqpoint{0.975777in}{2.621011in}}%
\pgfpathlineto{\pgfqpoint{0.976389in}{2.604508in}}%
\pgfpathlineto{\pgfqpoint{0.976593in}{2.554999in}}%
\pgfpathlineto{\pgfqpoint{0.977206in}{2.621011in}}%
\pgfpathlineto{\pgfqpoint{0.977614in}{2.563250in}}%
\pgfpathlineto{\pgfqpoint{0.979043in}{2.662269in}}%
\pgfpathlineto{\pgfqpoint{0.979247in}{2.488986in}}%
\pgfpathlineto{\pgfqpoint{0.980064in}{2.662269in}}%
\pgfpathlineto{\pgfqpoint{0.981493in}{2.480734in}}%
\pgfpathlineto{\pgfqpoint{0.980472in}{2.687024in}}%
\pgfpathlineto{\pgfqpoint{0.981697in}{2.513741in}}%
\pgfpathlineto{\pgfqpoint{0.981901in}{2.554999in}}%
\pgfpathlineto{\pgfqpoint{0.982309in}{2.414721in}}%
\pgfpathlineto{\pgfqpoint{0.982922in}{2.546747in}}%
\pgfpathlineto{\pgfqpoint{0.983126in}{2.538495in}}%
\pgfpathlineto{\pgfqpoint{0.983534in}{2.488986in}}%
\pgfpathlineto{\pgfqpoint{0.984555in}{2.654018in}}%
\pgfpathlineto{\pgfqpoint{0.985984in}{2.554999in}}%
\pgfpathlineto{\pgfqpoint{0.987821in}{2.720031in}}%
\pgfpathlineto{\pgfqpoint{0.988842in}{2.365212in}}%
\pgfpathlineto{\pgfqpoint{0.989250in}{2.554999in}}%
\pgfpathlineto{\pgfqpoint{0.989658in}{2.571502in}}%
\pgfpathlineto{\pgfqpoint{0.990679in}{2.488986in}}%
\pgfpathlineto{\pgfqpoint{0.990883in}{2.505489in}}%
\pgfpathlineto{\pgfqpoint{0.991700in}{2.654018in}}%
\pgfpathlineto{\pgfqpoint{0.991904in}{2.406470in}}%
\pgfpathlineto{\pgfqpoint{0.992720in}{2.645766in}}%
\pgfpathlineto{\pgfqpoint{0.993333in}{2.687024in}}%
\pgfpathlineto{\pgfqpoint{0.993537in}{2.621011in}}%
\pgfpathlineto{\pgfqpoint{0.994149in}{2.554999in}}%
\pgfpathlineto{\pgfqpoint{0.994558in}{2.662269in}}%
\pgfpathlineto{\pgfqpoint{0.994966in}{2.645766in}}%
\pgfpathlineto{\pgfqpoint{0.995170in}{2.365212in}}%
\pgfpathlineto{\pgfqpoint{0.995987in}{2.711779in}}%
\pgfpathlineto{\pgfqpoint{0.997007in}{2.612760in}}%
\pgfpathlineto{\pgfqpoint{0.997416in}{2.621011in}}%
\pgfpathlineto{\pgfqpoint{0.998028in}{2.596257in}}%
\pgfpathlineto{\pgfqpoint{0.998640in}{2.645766in}}%
\pgfpathlineto{\pgfqpoint{0.999253in}{2.521992in}}%
\pgfpathlineto{\pgfqpoint{0.999865in}{2.554999in}}%
\pgfpathlineto{\pgfqpoint{1.000069in}{2.554999in}}%
\pgfpathlineto{\pgfqpoint{1.000274in}{2.546747in}}%
\pgfpathlineto{\pgfqpoint{1.000478in}{2.398218in}}%
\pgfpathlineto{\pgfqpoint{1.001294in}{2.579753in}}%
\pgfpathlineto{\pgfqpoint{1.001498in}{2.538495in}}%
\pgfpathlineto{\pgfqpoint{1.002111in}{2.612760in}}%
\pgfpathlineto{\pgfqpoint{1.003336in}{2.703527in}}%
\pgfpathlineto{\pgfqpoint{1.004561in}{2.554999in}}%
\pgfpathlineto{\pgfqpoint{1.005990in}{2.645766in}}%
\pgfpathlineto{\pgfqpoint{1.006602in}{2.546747in}}%
\pgfpathlineto{\pgfqpoint{1.007010in}{2.604508in}}%
\pgfpathlineto{\pgfqpoint{1.007827in}{2.662269in}}%
\pgfpathlineto{\pgfqpoint{1.008031in}{2.629263in}}%
\pgfpathlineto{\pgfqpoint{1.008439in}{2.604508in}}%
\pgfpathlineto{\pgfqpoint{1.008848in}{2.662269in}}%
\pgfpathlineto{\pgfqpoint{1.009052in}{2.654018in}}%
\pgfpathlineto{\pgfqpoint{1.009256in}{2.703527in}}%
\pgfpathlineto{\pgfqpoint{1.010277in}{2.695276in}}%
\pgfpathlineto{\pgfqpoint{1.010889in}{2.720031in}}%
\pgfpathlineto{\pgfqpoint{1.011501in}{2.645766in}}%
\pgfpathlineto{\pgfqpoint{1.011706in}{2.720031in}}%
\pgfpathlineto{\pgfqpoint{1.012522in}{2.678773in}}%
\pgfpathlineto{\pgfqpoint{1.013543in}{2.554999in}}%
\pgfpathlineto{\pgfqpoint{1.013951in}{2.645766in}}%
\pgfpathlineto{\pgfqpoint{1.014564in}{2.662269in}}%
\pgfpathlineto{\pgfqpoint{1.014359in}{2.637515in}}%
\pgfpathlineto{\pgfqpoint{1.014972in}{2.645766in}}%
\pgfpathlineto{\pgfqpoint{1.015993in}{2.596257in}}%
\pgfpathlineto{\pgfqpoint{1.015584in}{2.687024in}}%
\pgfpathlineto{\pgfqpoint{1.016197in}{2.604508in}}%
\pgfpathlineto{\pgfqpoint{1.016401in}{2.645766in}}%
\pgfpathlineto{\pgfqpoint{1.017013in}{2.571502in}}%
\pgfpathlineto{\pgfqpoint{1.017217in}{2.604508in}}%
\pgfpathlineto{\pgfqpoint{1.017422in}{2.612760in}}%
\pgfpathlineto{\pgfqpoint{1.018851in}{2.422973in}}%
\pgfpathlineto{\pgfqpoint{1.019259in}{2.439476in}}%
\pgfpathlineto{\pgfqpoint{1.020484in}{2.307450in}}%
\pgfpathlineto{\pgfqpoint{1.021300in}{2.373463in}}%
\pgfpathlineto{\pgfqpoint{1.021096in}{2.290947in}}%
\pgfpathlineto{\pgfqpoint{1.021504in}{2.299199in}}%
\pgfpathlineto{\pgfqpoint{1.021709in}{2.315702in}}%
\pgfpathlineto{\pgfqpoint{1.022117in}{2.266192in}}%
\pgfpathlineto{\pgfqpoint{1.022525in}{2.299199in}}%
\pgfpathlineto{\pgfqpoint{1.023546in}{2.257941in}}%
\pgfpathlineto{\pgfqpoint{1.022933in}{2.340457in}}%
\pgfpathlineto{\pgfqpoint{1.023750in}{2.274444in}}%
\pgfpathlineto{\pgfqpoint{1.024975in}{2.373463in}}%
\pgfpathlineto{\pgfqpoint{1.026812in}{2.150670in}}%
\pgfpathlineto{\pgfqpoint{1.027629in}{2.315702in}}%
\pgfpathlineto{\pgfqpoint{1.028241in}{2.282696in}}%
\pgfpathlineto{\pgfqpoint{1.028445in}{2.290947in}}%
\pgfpathlineto{\pgfqpoint{1.028649in}{2.282696in}}%
\pgfpathlineto{\pgfqpoint{1.029466in}{2.109412in}}%
\pgfpathlineto{\pgfqpoint{1.030078in}{2.183676in}}%
\pgfpathlineto{\pgfqpoint{1.030283in}{2.183676in}}%
\pgfpathlineto{\pgfqpoint{1.030487in}{2.059902in}}%
\pgfpathlineto{\pgfqpoint{1.031507in}{2.101160in}}%
\pgfpathlineto{\pgfqpoint{1.032120in}{2.092909in}}%
\pgfpathlineto{\pgfqpoint{1.032324in}{2.134167in}}%
\pgfpathlineto{\pgfqpoint{1.032528in}{2.241438in}}%
\pgfpathlineto{\pgfqpoint{1.033141in}{2.076406in}}%
\pgfpathlineto{\pgfqpoint{1.034161in}{2.010393in}}%
\pgfpathlineto{\pgfqpoint{1.034570in}{2.026896in}}%
\pgfpathlineto{\pgfqpoint{1.035794in}{2.200180in}}%
\pgfpathlineto{\pgfqpoint{1.034978in}{1.969135in}}%
\pgfpathlineto{\pgfqpoint{1.035999in}{2.158922in}}%
\pgfpathlineto{\pgfqpoint{1.036203in}{1.993890in}}%
\pgfpathlineto{\pgfqpoint{1.036611in}{2.200180in}}%
\pgfpathlineto{\pgfqpoint{1.037019in}{2.084657in}}%
\pgfpathlineto{\pgfqpoint{1.038244in}{2.224934in}}%
\pgfpathlineto{\pgfqpoint{1.038448in}{2.191928in}}%
\pgfpathlineto{\pgfqpoint{1.038652in}{1.919625in}}%
\pgfpathlineto{\pgfqpoint{1.039469in}{2.092909in}}%
\pgfpathlineto{\pgfqpoint{1.039877in}{2.134167in}}%
\pgfpathlineto{\pgfqpoint{1.040286in}{2.084657in}}%
\pgfpathlineto{\pgfqpoint{1.040490in}{2.059902in}}%
\pgfpathlineto{\pgfqpoint{1.040694in}{2.109412in}}%
\pgfpathlineto{\pgfqpoint{1.041306in}{2.084657in}}%
\pgfpathlineto{\pgfqpoint{1.041715in}{2.018644in}}%
\pgfpathlineto{\pgfqpoint{1.042531in}{2.167173in}}%
\pgfpathlineto{\pgfqpoint{1.043348in}{2.026896in}}%
\pgfpathlineto{\pgfqpoint{1.043960in}{2.084657in}}%
\pgfpathlineto{\pgfqpoint{1.044368in}{2.109412in}}%
\pgfpathlineto{\pgfqpoint{1.044777in}{1.853612in}}%
\pgfpathlineto{\pgfqpoint{1.045593in}{1.993890in}}%
\pgfpathlineto{\pgfqpoint{1.046002in}{2.035148in}}%
\pgfpathlineto{\pgfqpoint{1.046206in}{1.977386in}}%
\pgfpathlineto{\pgfqpoint{1.046614in}{1.993890in}}%
\pgfpathlineto{\pgfqpoint{1.047431in}{1.903122in}}%
\pgfpathlineto{\pgfqpoint{1.047635in}{1.969135in}}%
\pgfpathlineto{\pgfqpoint{1.048451in}{1.894870in}}%
\pgfpathlineto{\pgfqpoint{1.048043in}{1.985638in}}%
\pgfpathlineto{\pgfqpoint{1.049064in}{1.936128in}}%
\pgfpathlineto{\pgfqpoint{1.049880in}{1.977386in}}%
\pgfpathlineto{\pgfqpoint{1.050084in}{1.969135in}}%
\pgfpathlineto{\pgfqpoint{1.050493in}{1.762845in}}%
\pgfpathlineto{\pgfqpoint{1.051105in}{1.927877in}}%
\pgfpathlineto{\pgfqpoint{1.051718in}{1.985638in}}%
\pgfpathlineto{\pgfqpoint{1.051513in}{1.870115in}}%
\pgfpathlineto{\pgfqpoint{1.051922in}{1.936128in}}%
\pgfpathlineto{\pgfqpoint{1.052126in}{1.828857in}}%
\pgfpathlineto{\pgfqpoint{1.052942in}{1.944380in}}%
\pgfpathlineto{\pgfqpoint{1.053351in}{2.002141in}}%
\pgfpathlineto{\pgfqpoint{1.053963in}{1.977386in}}%
\pgfpathlineto{\pgfqpoint{1.054780in}{1.878367in}}%
\pgfpathlineto{\pgfqpoint{1.054984in}{1.919625in}}%
\pgfpathlineto{\pgfqpoint{1.055392in}{2.010393in}}%
\pgfpathlineto{\pgfqpoint{1.056209in}{1.960883in}}%
\pgfpathlineto{\pgfqpoint{1.057842in}{2.101160in}}%
\pgfpathlineto{\pgfqpoint{1.058250in}{2.018644in}}%
\pgfpathlineto{\pgfqpoint{1.058863in}{2.059902in}}%
\pgfpathlineto{\pgfqpoint{1.059067in}{2.076406in}}%
\pgfpathlineto{\pgfqpoint{1.059271in}{2.010393in}}%
\pgfpathlineto{\pgfqpoint{1.059475in}{2.010393in}}%
\pgfpathlineto{\pgfqpoint{1.060700in}{2.134167in}}%
\pgfpathlineto{\pgfqpoint{1.061516in}{1.886619in}}%
\pgfpathlineto{\pgfqpoint{1.061721in}{2.035148in}}%
\pgfpathlineto{\pgfqpoint{1.061925in}{2.142418in}}%
\pgfpathlineto{\pgfqpoint{1.062129in}{1.936128in}}%
\pgfpathlineto{\pgfqpoint{1.062741in}{2.117664in}}%
\pgfpathlineto{\pgfqpoint{1.063966in}{1.993890in}}%
\pgfpathlineto{\pgfqpoint{1.064783in}{2.158922in}}%
\pgfpathlineto{\pgfqpoint{1.065191in}{2.117664in}}%
\pgfpathlineto{\pgfqpoint{1.065599in}{2.216683in}}%
\pgfpathlineto{\pgfqpoint{1.066212in}{2.109412in}}%
\pgfpathlineto{\pgfqpoint{1.067641in}{1.977386in}}%
\pgfpathlineto{\pgfqpoint{1.066620in}{2.117664in}}%
\pgfpathlineto{\pgfqpoint{1.068049in}{1.985638in}}%
\pgfpathlineto{\pgfqpoint{1.068661in}{1.969135in}}%
\pgfpathlineto{\pgfqpoint{1.069070in}{2.010393in}}%
\pgfpathlineto{\pgfqpoint{1.069478in}{2.051651in}}%
\pgfpathlineto{\pgfqpoint{1.070295in}{1.944380in}}%
\pgfpathlineto{\pgfqpoint{1.070907in}{2.035148in}}%
\pgfpathlineto{\pgfqpoint{1.071315in}{1.911373in}}%
\pgfpathlineto{\pgfqpoint{1.073153in}{1.771096in}}%
\pgfpathlineto{\pgfqpoint{1.074377in}{1.878367in}}%
\pgfpathlineto{\pgfqpoint{1.074582in}{1.878367in}}%
\pgfpathlineto{\pgfqpoint{1.074786in}{1.861864in}}%
\pgfpathlineto{\pgfqpoint{1.075194in}{1.960883in}}%
\pgfpathlineto{\pgfqpoint{1.075806in}{1.944380in}}%
\pgfpathlineto{\pgfqpoint{1.077031in}{1.853612in}}%
\pgfpathlineto{\pgfqpoint{1.078256in}{1.977386in}}%
\pgfpathlineto{\pgfqpoint{1.078460in}{1.993890in}}%
\pgfpathlineto{\pgfqpoint{1.078664in}{1.960883in}}%
\pgfpathlineto{\pgfqpoint{1.078868in}{1.894870in}}%
\pgfpathlineto{\pgfqpoint{1.079073in}{2.002141in}}%
\pgfpathlineto{\pgfqpoint{1.079685in}{1.927877in}}%
\pgfpathlineto{\pgfqpoint{1.080093in}{1.870115in}}%
\pgfpathlineto{\pgfqpoint{1.080706in}{1.969135in}}%
\pgfpathlineto{\pgfqpoint{1.080910in}{1.746341in}}%
\pgfpathlineto{\pgfqpoint{1.081114in}{1.985638in}}%
\pgfpathlineto{\pgfqpoint{1.081726in}{1.894870in}}%
\pgfpathlineto{\pgfqpoint{1.082339in}{1.960883in}}%
\pgfpathlineto{\pgfqpoint{1.082135in}{1.886619in}}%
\pgfpathlineto{\pgfqpoint{1.082747in}{1.936128in}}%
\pgfpathlineto{\pgfqpoint{1.083155in}{1.878367in}}%
\pgfpathlineto{\pgfqpoint{1.083360in}{2.010393in}}%
\pgfpathlineto{\pgfqpoint{1.083564in}{2.059902in}}%
\pgfpathlineto{\pgfqpoint{1.084176in}{2.010393in}}%
\pgfpathlineto{\pgfqpoint{1.085605in}{1.878367in}}%
\pgfpathlineto{\pgfqpoint{1.085809in}{1.878367in}}%
\pgfpathlineto{\pgfqpoint{1.086830in}{1.977386in}}%
\pgfpathlineto{\pgfqpoint{1.087238in}{1.936128in}}%
\pgfpathlineto{\pgfqpoint{1.088463in}{1.820606in}}%
\pgfpathlineto{\pgfqpoint{1.089892in}{2.026896in}}%
\pgfpathlineto{\pgfqpoint{1.091117in}{1.762845in}}%
\pgfpathlineto{\pgfqpoint{1.091934in}{2.043399in}}%
\pgfpathlineto{\pgfqpoint{1.092342in}{1.985638in}}%
\pgfpathlineto{\pgfqpoint{1.092750in}{1.729838in}}%
\pgfpathlineto{\pgfqpoint{1.092954in}{2.002141in}}%
\pgfpathlineto{\pgfqpoint{1.093363in}{2.002141in}}%
\pgfpathlineto{\pgfqpoint{1.093975in}{2.051651in}}%
\pgfpathlineto{\pgfqpoint{1.094383in}{2.043399in}}%
\pgfpathlineto{\pgfqpoint{1.096016in}{1.870115in}}%
\pgfpathlineto{\pgfqpoint{1.096629in}{1.936128in}}%
\pgfpathlineto{\pgfqpoint{1.097037in}{1.878367in}}%
\pgfpathlineto{\pgfqpoint{1.097445in}{1.878367in}}%
\pgfpathlineto{\pgfqpoint{1.097650in}{1.845361in}}%
\pgfpathlineto{\pgfqpoint{1.097854in}{1.911373in}}%
\pgfpathlineto{\pgfqpoint{1.098058in}{1.903122in}}%
\pgfpathlineto{\pgfqpoint{1.098874in}{2.092909in}}%
\pgfpathlineto{\pgfqpoint{1.099283in}{1.969135in}}%
\pgfpathlineto{\pgfqpoint{1.099691in}{1.911373in}}%
\pgfpathlineto{\pgfqpoint{1.100099in}{1.927877in}}%
\pgfpathlineto{\pgfqpoint{1.100712in}{2.084657in}}%
\pgfpathlineto{\pgfqpoint{1.101120in}{1.977386in}}%
\pgfpathlineto{\pgfqpoint{1.101937in}{1.911373in}}%
\pgfpathlineto{\pgfqpoint{1.102345in}{1.919625in}}%
\pgfpathlineto{\pgfqpoint{1.103161in}{1.993890in}}%
\pgfpathlineto{\pgfqpoint{1.103570in}{1.944380in}}%
\pgfpathlineto{\pgfqpoint{1.104795in}{2.018644in}}%
\pgfpathlineto{\pgfqpoint{1.105203in}{1.936128in}}%
\pgfpathlineto{\pgfqpoint{1.105815in}{2.010393in}}%
\pgfpathlineto{\pgfqpoint{1.106224in}{2.002141in}}%
\pgfpathlineto{\pgfqpoint{1.106428in}{2.026896in}}%
\pgfpathlineto{\pgfqpoint{1.106632in}{2.018644in}}%
\pgfpathlineto{\pgfqpoint{1.107244in}{2.092909in}}%
\pgfpathlineto{\pgfqpoint{1.107448in}{2.010393in}}%
\pgfpathlineto{\pgfqpoint{1.108877in}{1.853612in}}%
\pgfpathlineto{\pgfqpoint{1.108061in}{2.035148in}}%
\pgfpathlineto{\pgfqpoint{1.109490in}{1.894870in}}%
\pgfpathlineto{\pgfqpoint{1.109694in}{1.952632in}}%
\pgfpathlineto{\pgfqpoint{1.109898in}{1.886619in}}%
\pgfpathlineto{\pgfqpoint{1.110306in}{1.919625in}}%
\pgfpathlineto{\pgfqpoint{1.111327in}{1.787599in}}%
\pgfpathlineto{\pgfqpoint{1.111531in}{1.853612in}}%
\pgfpathlineto{\pgfqpoint{1.112960in}{1.680329in}}%
\pgfpathlineto{\pgfqpoint{1.113573in}{1.771096in}}%
\pgfpathlineto{\pgfqpoint{1.113981in}{1.696832in}}%
\pgfpathlineto{\pgfqpoint{1.114185in}{1.705083in}}%
\pgfpathlineto{\pgfqpoint{1.115002in}{1.696832in}}%
\pgfpathlineto{\pgfqpoint{1.115410in}{1.787599in}}%
\pgfpathlineto{\pgfqpoint{1.116022in}{1.837109in}}%
\pgfpathlineto{\pgfqpoint{1.116227in}{1.787599in}}%
\pgfpathlineto{\pgfqpoint{1.116431in}{1.771096in}}%
\pgfpathlineto{\pgfqpoint{1.116839in}{1.820606in}}%
\pgfpathlineto{\pgfqpoint{1.117043in}{1.870115in}}%
\pgfpathlineto{\pgfqpoint{1.117656in}{1.754593in}}%
\pgfpathlineto{\pgfqpoint{1.118472in}{1.556554in}}%
\pgfpathlineto{\pgfqpoint{1.118676in}{1.705083in}}%
\pgfpathlineto{\pgfqpoint{1.119697in}{1.581309in}}%
\pgfpathlineto{\pgfqpoint{1.120105in}{1.614316in}}%
\pgfpathlineto{\pgfqpoint{1.121534in}{1.771096in}}%
\pgfpathlineto{\pgfqpoint{1.122351in}{1.507045in}}%
\pgfpathlineto{\pgfqpoint{1.122963in}{1.639071in}}%
\pgfpathlineto{\pgfqpoint{1.123167in}{1.639071in}}%
\pgfpathlineto{\pgfqpoint{1.125005in}{1.779348in}}%
\pgfpathlineto{\pgfqpoint{1.125413in}{1.820606in}}%
\pgfpathlineto{\pgfqpoint{1.125617in}{1.787599in}}%
\pgfpathlineto{\pgfqpoint{1.126434in}{1.705083in}}%
\pgfpathlineto{\pgfqpoint{1.127046in}{1.738090in}}%
\pgfpathlineto{\pgfqpoint{1.127659in}{1.696832in}}%
\pgfpathlineto{\pgfqpoint{1.128067in}{1.647322in}}%
\pgfpathlineto{\pgfqpoint{1.128679in}{1.705083in}}%
\pgfpathlineto{\pgfqpoint{1.129496in}{1.680329in}}%
\pgfpathlineto{\pgfqpoint{1.130108in}{1.779348in}}%
\pgfpathlineto{\pgfqpoint{1.131946in}{1.647322in}}%
\pgfpathlineto{\pgfqpoint{1.132558in}{1.729838in}}%
\pgfpathlineto{\pgfqpoint{1.132966in}{1.639071in}}%
\pgfpathlineto{\pgfqpoint{1.133170in}{1.705083in}}%
\pgfpathlineto{\pgfqpoint{1.133579in}{1.540051in}}%
\pgfpathlineto{\pgfqpoint{1.134395in}{1.672077in}}%
\pgfpathlineto{\pgfqpoint{1.134599in}{1.663825in}}%
\pgfpathlineto{\pgfqpoint{1.134804in}{1.738090in}}%
\pgfpathlineto{\pgfqpoint{1.135008in}{1.622567in}}%
\pgfpathlineto{\pgfqpoint{1.135620in}{1.663825in}}%
\pgfpathlineto{\pgfqpoint{1.135824in}{1.647322in}}%
\pgfpathlineto{\pgfqpoint{1.136028in}{1.721587in}}%
\pgfpathlineto{\pgfqpoint{1.136641in}{1.639071in}}%
\pgfpathlineto{\pgfqpoint{1.137866in}{1.556554in}}%
\pgfpathlineto{\pgfqpoint{1.138070in}{1.556554in}}%
\pgfpathlineto{\pgfqpoint{1.138478in}{1.465787in}}%
\pgfpathlineto{\pgfqpoint{1.139091in}{1.490542in}}%
\pgfpathlineto{\pgfqpoint{1.139295in}{1.597813in}}%
\pgfpathlineto{\pgfqpoint{1.140111in}{1.531800in}}%
\pgfpathlineto{\pgfqpoint{1.140315in}{1.498793in}}%
\pgfpathlineto{\pgfqpoint{1.140520in}{1.589561in}}%
\pgfpathlineto{\pgfqpoint{1.140928in}{1.523548in}}%
\pgfpathlineto{\pgfqpoint{1.142153in}{1.672077in}}%
\pgfpathlineto{\pgfqpoint{1.143582in}{1.540051in}}%
\pgfpathlineto{\pgfqpoint{1.143990in}{1.531800in}}%
\pgfpathlineto{\pgfqpoint{1.144807in}{1.581309in}}%
\pgfpathlineto{\pgfqpoint{1.145623in}{1.383271in}}%
\pgfpathlineto{\pgfqpoint{1.146031in}{1.498793in}}%
\pgfpathlineto{\pgfqpoint{1.146644in}{1.399774in}}%
\pgfpathlineto{\pgfqpoint{1.147052in}{1.465787in}}%
\pgfpathlineto{\pgfqpoint{1.147460in}{1.449284in}}%
\pgfpathlineto{\pgfqpoint{1.147665in}{1.490542in}}%
\pgfpathlineto{\pgfqpoint{1.148073in}{1.375019in}}%
\pgfpathlineto{\pgfqpoint{1.148481in}{1.457535in}}%
\pgfpathlineto{\pgfqpoint{1.149706in}{1.383271in}}%
\pgfpathlineto{\pgfqpoint{1.150318in}{1.424529in}}%
\pgfpathlineto{\pgfqpoint{1.150931in}{1.399774in}}%
\pgfpathlineto{\pgfqpoint{1.151135in}{1.408026in}}%
\pgfpathlineto{\pgfqpoint{1.151952in}{1.540051in}}%
\pgfpathlineto{\pgfqpoint{1.152564in}{1.507045in}}%
\pgfpathlineto{\pgfqpoint{1.153176in}{1.531800in}}%
\pgfpathlineto{\pgfqpoint{1.153789in}{1.432780in}}%
\pgfpathlineto{\pgfqpoint{1.153993in}{1.432780in}}%
\pgfpathlineto{\pgfqpoint{1.154401in}{1.523548in}}%
\pgfpathlineto{\pgfqpoint{1.154605in}{1.267748in}}%
\pgfpathlineto{\pgfqpoint{1.155014in}{1.606064in}}%
\pgfpathlineto{\pgfqpoint{1.155422in}{1.498793in}}%
\pgfpathlineto{\pgfqpoint{1.156239in}{1.589561in}}%
\pgfpathlineto{\pgfqpoint{1.156647in}{1.540051in}}%
\pgfpathlineto{\pgfqpoint{1.156851in}{1.490542in}}%
\pgfpathlineto{\pgfqpoint{1.157055in}{1.564806in}}%
\pgfpathlineto{\pgfqpoint{1.157259in}{1.523548in}}%
\pgfpathlineto{\pgfqpoint{1.157463in}{1.630819in}}%
\pgfpathlineto{\pgfqpoint{1.158280in}{1.540051in}}%
\pgfpathlineto{\pgfqpoint{1.159097in}{1.556554in}}%
\pgfpathlineto{\pgfqpoint{1.160321in}{1.408026in}}%
\pgfpathlineto{\pgfqpoint{1.160525in}{1.449284in}}%
\pgfpathlineto{\pgfqpoint{1.160934in}{1.457535in}}%
\pgfpathlineto{\pgfqpoint{1.162159in}{1.581309in}}%
\pgfpathlineto{\pgfqpoint{1.162363in}{1.573058in}}%
\pgfpathlineto{\pgfqpoint{1.162771in}{1.523548in}}%
\pgfpathlineto{\pgfqpoint{1.163179in}{1.597813in}}%
\pgfpathlineto{\pgfqpoint{1.163588in}{1.581309in}}%
\pgfpathlineto{\pgfqpoint{1.165629in}{1.804103in}}%
\pgfpathlineto{\pgfqpoint{1.166446in}{1.630819in}}%
\pgfpathlineto{\pgfqpoint{1.166854in}{1.738090in}}%
\pgfpathlineto{\pgfqpoint{1.167058in}{1.696832in}}%
\pgfpathlineto{\pgfqpoint{1.167670in}{1.416277in}}%
\pgfpathlineto{\pgfqpoint{1.168079in}{1.663825in}}%
\pgfpathlineto{\pgfqpoint{1.168487in}{1.688580in}}%
\pgfpathlineto{\pgfqpoint{1.169304in}{1.589561in}}%
\pgfpathlineto{\pgfqpoint{1.170120in}{1.630819in}}%
\pgfpathlineto{\pgfqpoint{1.170528in}{1.606064in}}%
\pgfpathlineto{\pgfqpoint{1.170733in}{1.366768in}}%
\pgfpathlineto{\pgfqpoint{1.171345in}{1.771096in}}%
\pgfpathlineto{\pgfqpoint{1.172774in}{1.647322in}}%
\pgfpathlineto{\pgfqpoint{1.172978in}{1.696832in}}%
\pgfpathlineto{\pgfqpoint{1.173182in}{1.672077in}}%
\pgfpathlineto{\pgfqpoint{1.173386in}{1.432780in}}%
\pgfpathlineto{\pgfqpoint{1.174203in}{1.630819in}}%
\pgfpathlineto{\pgfqpoint{1.174407in}{1.639071in}}%
\pgfpathlineto{\pgfqpoint{1.175020in}{1.531800in}}%
\pgfpathlineto{\pgfqpoint{1.175428in}{1.589561in}}%
\pgfpathlineto{\pgfqpoint{1.175836in}{1.655574in}}%
\pgfpathlineto{\pgfqpoint{1.176040in}{1.606064in}}%
\pgfpathlineto{\pgfqpoint{1.177878in}{1.457535in}}%
\pgfpathlineto{\pgfqpoint{1.179307in}{1.589561in}}%
\pgfpathlineto{\pgfqpoint{1.179919in}{1.581309in}}%
\pgfpathlineto{\pgfqpoint{1.180940in}{1.333761in}}%
\pgfpathlineto{\pgfqpoint{1.181144in}{1.366768in}}%
\pgfpathlineto{\pgfqpoint{1.181348in}{1.606064in}}%
\pgfpathlineto{\pgfqpoint{1.182369in}{1.548303in}}%
\pgfpathlineto{\pgfqpoint{1.183389in}{1.639071in}}%
\pgfpathlineto{\pgfqpoint{1.182981in}{1.540051in}}%
\pgfpathlineto{\pgfqpoint{1.183594in}{1.573058in}}%
\pgfpathlineto{\pgfqpoint{1.184410in}{1.548303in}}%
\pgfpathlineto{\pgfqpoint{1.185431in}{1.622567in}}%
\pgfpathlineto{\pgfqpoint{1.185635in}{1.589561in}}%
\pgfpathlineto{\pgfqpoint{1.186452in}{1.515296in}}%
\pgfpathlineto{\pgfqpoint{1.186656in}{1.581309in}}%
\pgfpathlineto{\pgfqpoint{1.187064in}{1.523548in}}%
\pgfpathlineto{\pgfqpoint{1.187268in}{1.432780in}}%
\pgfpathlineto{\pgfqpoint{1.188085in}{1.523548in}}%
\pgfpathlineto{\pgfqpoint{1.188289in}{1.531800in}}%
\pgfpathlineto{\pgfqpoint{1.188493in}{1.366768in}}%
\pgfpathlineto{\pgfqpoint{1.188697in}{1.564806in}}%
\pgfpathlineto{\pgfqpoint{1.189310in}{1.556554in}}%
\pgfpathlineto{\pgfqpoint{1.189514in}{1.474038in}}%
\pgfpathlineto{\pgfqpoint{1.189718in}{1.606064in}}%
\pgfpathlineto{\pgfqpoint{1.190330in}{1.581309in}}%
\pgfpathlineto{\pgfqpoint{1.191759in}{1.498793in}}%
\pgfpathlineto{\pgfqpoint{1.192576in}{1.597813in}}%
\pgfpathlineto{\pgfqpoint{1.192168in}{1.482290in}}%
\pgfpathlineto{\pgfqpoint{1.192780in}{1.564806in}}%
\pgfpathlineto{\pgfqpoint{1.192984in}{1.474038in}}%
\pgfpathlineto{\pgfqpoint{1.194005in}{1.498793in}}%
\pgfpathlineto{\pgfqpoint{1.194209in}{1.498793in}}%
\pgfpathlineto{\pgfqpoint{1.194617in}{1.531800in}}%
\pgfpathlineto{\pgfqpoint{1.194821in}{1.474038in}}%
\pgfpathlineto{\pgfqpoint{1.195026in}{1.474038in}}%
\pgfpathlineto{\pgfqpoint{1.195230in}{1.474038in}}%
\pgfpathlineto{\pgfqpoint{1.195434in}{1.457535in}}%
\pgfpathlineto{\pgfqpoint{1.195638in}{1.284252in}}%
\pgfpathlineto{\pgfqpoint{1.196046in}{1.606064in}}%
\pgfpathlineto{\pgfqpoint{1.196455in}{1.490542in}}%
\pgfpathlineto{\pgfqpoint{1.197475in}{1.573058in}}%
\pgfpathlineto{\pgfqpoint{1.198496in}{1.556554in}}%
\pgfpathlineto{\pgfqpoint{1.198700in}{1.647322in}}%
\pgfpathlineto{\pgfqpoint{1.199721in}{1.630819in}}%
\pgfpathlineto{\pgfqpoint{1.199925in}{1.622567in}}%
\pgfpathlineto{\pgfqpoint{1.200129in}{1.515296in}}%
\pgfpathlineto{\pgfqpoint{1.200946in}{1.573058in}}%
\pgfpathlineto{\pgfqpoint{1.201150in}{1.573058in}}%
\pgfpathlineto{\pgfqpoint{1.201762in}{1.498793in}}%
\pgfpathlineto{\pgfqpoint{1.202171in}{1.540051in}}%
\pgfpathlineto{\pgfqpoint{1.203804in}{1.688580in}}%
\pgfpathlineto{\pgfqpoint{1.204008in}{1.647322in}}%
\pgfpathlineto{\pgfqpoint{1.205029in}{1.474038in}}%
\pgfpathlineto{\pgfqpoint{1.205233in}{1.556554in}}%
\pgfpathlineto{\pgfqpoint{1.205641in}{1.581309in}}%
\pgfpathlineto{\pgfqpoint{1.206049in}{1.540051in}}%
\pgfpathlineto{\pgfqpoint{1.206866in}{1.606064in}}%
\pgfpathlineto{\pgfqpoint{1.207478in}{1.366768in}}%
\pgfpathlineto{\pgfqpoint{1.207682in}{1.300755in}}%
\pgfpathlineto{\pgfqpoint{1.207887in}{1.564806in}}%
\pgfpathlineto{\pgfqpoint{1.208295in}{1.432780in}}%
\pgfpathlineto{\pgfqpoint{1.208499in}{1.416277in}}%
\pgfpathlineto{\pgfqpoint{1.208907in}{1.176981in}}%
\pgfpathlineto{\pgfqpoint{1.209520in}{1.416277in}}%
\pgfpathlineto{\pgfqpoint{1.209724in}{1.441032in}}%
\pgfpathlineto{\pgfqpoint{1.211153in}{1.218239in}}%
\pgfpathlineto{\pgfqpoint{1.211561in}{1.102716in}}%
\pgfpathlineto{\pgfqpoint{1.212174in}{1.143974in}}%
\pgfpathlineto{\pgfqpoint{1.212990in}{1.086213in}}%
\pgfpathlineto{\pgfqpoint{1.213194in}{1.259497in}}%
\pgfpathlineto{\pgfqpoint{1.213603in}{1.086213in}}%
\pgfpathlineto{\pgfqpoint{1.214419in}{1.127471in}}%
\pgfpathlineto{\pgfqpoint{1.216052in}{1.284252in}}%
\pgfpathlineto{\pgfqpoint{1.217481in}{1.135723in}}%
\pgfpathlineto{\pgfqpoint{1.217685in}{1.193484in}}%
\pgfpathlineto{\pgfqpoint{1.218298in}{1.077961in}}%
\pgfpathlineto{\pgfqpoint{1.219319in}{1.127471in}}%
\pgfpathlineto{\pgfqpoint{1.218706in}{1.053207in}}%
\pgfpathlineto{\pgfqpoint{1.219727in}{1.119219in}}%
\pgfpathlineto{\pgfqpoint{1.219931in}{1.110968in}}%
\pgfpathlineto{\pgfqpoint{1.221360in}{0.822162in}}%
\pgfpathlineto{\pgfqpoint{1.222177in}{1.053207in}}%
\pgfpathlineto{\pgfqpoint{1.222585in}{1.036703in}}%
\pgfpathlineto{\pgfqpoint{1.222789in}{1.036703in}}%
\pgfpathlineto{\pgfqpoint{1.222993in}{1.011949in}}%
\pgfpathlineto{\pgfqpoint{1.223401in}{1.168729in}}%
\pgfpathlineto{\pgfqpoint{1.224014in}{1.061458in}}%
\pgfpathlineto{\pgfqpoint{1.225239in}{0.978942in}}%
\pgfpathlineto{\pgfqpoint{1.226055in}{1.276000in}}%
\pgfpathlineto{\pgfqpoint{1.226464in}{1.168729in}}%
\pgfpathlineto{\pgfqpoint{1.227688in}{1.011949in}}%
\pgfpathlineto{\pgfqpoint{1.227893in}{1.036703in}}%
\pgfpathlineto{\pgfqpoint{1.228709in}{0.838665in}}%
\pgfpathlineto{\pgfqpoint{1.229117in}{0.912929in}}%
\pgfpathlineto{\pgfqpoint{1.229526in}{0.863420in}}%
\pgfpathlineto{\pgfqpoint{1.230342in}{0.879923in}}%
\pgfpathlineto{\pgfqpoint{1.230546in}{0.945936in}}%
\pgfpathlineto{\pgfqpoint{1.231567in}{0.937684in}}%
\pgfpathlineto{\pgfqpoint{1.231771in}{0.896426in}}%
\pgfpathlineto{\pgfqpoint{1.232384in}{0.987194in}}%
\pgfpathlineto{\pgfqpoint{1.232588in}{0.954187in}}%
\pgfpathlineto{\pgfqpoint{1.233609in}{1.069710in}}%
\pgfpathlineto{\pgfqpoint{1.234221in}{1.044955in}}%
\pgfpathlineto{\pgfqpoint{1.234629in}{0.937684in}}%
\pgfpathlineto{\pgfqpoint{1.235242in}{1.053207in}}%
\pgfpathlineto{\pgfqpoint{1.235854in}{1.020200in}}%
\pgfpathlineto{\pgfqpoint{1.236058in}{1.036703in}}%
\pgfpathlineto{\pgfqpoint{1.236262in}{1.102716in}}%
\pgfpathlineto{\pgfqpoint{1.236875in}{1.020200in}}%
\pgfpathlineto{\pgfqpoint{1.237079in}{1.020200in}}%
\pgfpathlineto{\pgfqpoint{1.237487in}{1.061458in}}%
\pgfpathlineto{\pgfqpoint{1.237691in}{1.011949in}}%
\pgfpathlineto{\pgfqpoint{1.239120in}{0.888175in}}%
\pgfpathlineto{\pgfqpoint{1.239529in}{1.028452in}}%
\pgfpathlineto{\pgfqpoint{1.240754in}{1.011949in}}%
\pgfpathlineto{\pgfqpoint{1.241570in}{0.912929in}}%
\pgfpathlineto{\pgfqpoint{1.241162in}{1.044955in}}%
\pgfpathlineto{\pgfqpoint{1.242183in}{0.945936in}}%
\pgfpathlineto{\pgfqpoint{1.243816in}{1.069710in}}%
\pgfpathlineto{\pgfqpoint{1.242591in}{0.937684in}}%
\pgfpathlineto{\pgfqpoint{1.244020in}{1.011949in}}%
\pgfpathlineto{\pgfqpoint{1.244224in}{1.020200in}}%
\pgfpathlineto{\pgfqpoint{1.244428in}{0.995445in}}%
\pgfpathlineto{\pgfqpoint{1.245653in}{1.176981in}}%
\pgfpathlineto{\pgfqpoint{1.245857in}{1.102716in}}%
\pgfpathlineto{\pgfqpoint{1.246265in}{1.284252in}}%
\pgfpathlineto{\pgfqpoint{1.246469in}{1.259497in}}%
\pgfpathlineto{\pgfqpoint{1.248715in}{1.515296in}}%
\pgfpathlineto{\pgfqpoint{1.250348in}{1.391522in}}%
\pgfpathlineto{\pgfqpoint{1.250552in}{1.449284in}}%
\pgfpathlineto{\pgfqpoint{1.250756in}{1.218239in}}%
\pgfpathlineto{\pgfqpoint{1.251573in}{1.507045in}}%
\pgfpathlineto{\pgfqpoint{1.252185in}{1.540051in}}%
\pgfpathlineto{\pgfqpoint{1.252594in}{1.515296in}}%
\pgfpathlineto{\pgfqpoint{1.253206in}{1.490542in}}%
\pgfpathlineto{\pgfqpoint{1.253410in}{1.523548in}}%
\pgfpathlineto{\pgfqpoint{1.253614in}{1.507045in}}%
\pgfpathlineto{\pgfqpoint{1.253819in}{1.548303in}}%
\pgfpathlineto{\pgfqpoint{1.254431in}{1.498793in}}%
\pgfpathlineto{\pgfqpoint{1.254839in}{1.531800in}}%
\pgfpathlineto{\pgfqpoint{1.255043in}{1.507045in}}%
\pgfpathlineto{\pgfqpoint{1.255248in}{1.606064in}}%
\pgfpathlineto{\pgfqpoint{1.255452in}{1.564806in}}%
\pgfpathlineto{\pgfqpoint{1.256268in}{1.738090in}}%
\pgfpathlineto{\pgfqpoint{1.257085in}{1.713335in}}%
\pgfpathlineto{\pgfqpoint{1.257289in}{1.688580in}}%
\pgfpathlineto{\pgfqpoint{1.257697in}{1.746341in}}%
\pgfpathlineto{\pgfqpoint{1.258310in}{1.853612in}}%
\pgfpathlineto{\pgfqpoint{1.258718in}{1.820606in}}%
\pgfpathlineto{\pgfqpoint{1.259330in}{1.738090in}}%
\pgfpathlineto{\pgfqpoint{1.259739in}{1.762845in}}%
\pgfpathlineto{\pgfqpoint{1.261168in}{1.894870in}}%
\pgfpathlineto{\pgfqpoint{1.262188in}{1.804103in}}%
\pgfpathlineto{\pgfqpoint{1.262393in}{1.812354in}}%
\pgfpathlineto{\pgfqpoint{1.263209in}{1.919625in}}%
\pgfpathlineto{\pgfqpoint{1.262801in}{1.804103in}}%
\pgfpathlineto{\pgfqpoint{1.263413in}{1.903122in}}%
\pgfpathlineto{\pgfqpoint{1.264638in}{1.820606in}}%
\pgfpathlineto{\pgfqpoint{1.265659in}{1.927877in}}%
\pgfpathlineto{\pgfqpoint{1.265251in}{1.804103in}}%
\pgfpathlineto{\pgfqpoint{1.265863in}{1.886619in}}%
\pgfpathlineto{\pgfqpoint{1.266067in}{1.861864in}}%
\pgfpathlineto{\pgfqpoint{1.266271in}{1.911373in}}%
\pgfpathlineto{\pgfqpoint{1.266680in}{1.911373in}}%
\pgfpathlineto{\pgfqpoint{1.266884in}{1.969135in}}%
\pgfpathlineto{\pgfqpoint{1.267292in}{1.886619in}}%
\pgfpathlineto{\pgfqpoint{1.267700in}{1.903122in}}%
\pgfpathlineto{\pgfqpoint{1.267904in}{1.911373in}}%
\pgfpathlineto{\pgfqpoint{1.268109in}{1.886619in}}%
\pgfpathlineto{\pgfqpoint{1.268313in}{1.894870in}}%
\pgfpathlineto{\pgfqpoint{1.268721in}{1.845361in}}%
\pgfpathlineto{\pgfqpoint{1.268925in}{1.903122in}}%
\pgfpathlineto{\pgfqpoint{1.269129in}{1.903122in}}%
\pgfpathlineto{\pgfqpoint{1.269742in}{2.051651in}}%
\pgfpathlineto{\pgfqpoint{1.270558in}{2.002141in}}%
\pgfpathlineto{\pgfqpoint{1.270762in}{2.018644in}}%
\pgfpathlineto{\pgfqpoint{1.270967in}{1.993890in}}%
\pgfpathlineto{\pgfqpoint{1.271171in}{1.927877in}}%
\pgfpathlineto{\pgfqpoint{1.271987in}{2.018644in}}%
\pgfpathlineto{\pgfqpoint{1.272396in}{1.977386in}}%
\pgfpathlineto{\pgfqpoint{1.273008in}{2.035148in}}%
\pgfpathlineto{\pgfqpoint{1.273212in}{2.051651in}}%
\pgfpathlineto{\pgfqpoint{1.274029in}{1.944380in}}%
\pgfpathlineto{\pgfqpoint{1.274233in}{2.002141in}}%
\pgfpathlineto{\pgfqpoint{1.275049in}{2.101160in}}%
\pgfpathlineto{\pgfqpoint{1.275458in}{2.084657in}}%
\pgfpathlineto{\pgfqpoint{1.276070in}{2.109412in}}%
\pgfpathlineto{\pgfqpoint{1.276683in}{2.018644in}}%
\pgfpathlineto{\pgfqpoint{1.276887in}{2.084657in}}%
\pgfpathlineto{\pgfqpoint{1.277499in}{1.977386in}}%
\pgfpathlineto{\pgfqpoint{1.277703in}{2.035148in}}%
\pgfpathlineto{\pgfqpoint{1.277907in}{2.018644in}}%
\pgfpathlineto{\pgfqpoint{1.278112in}{2.043399in}}%
\pgfpathlineto{\pgfqpoint{1.279336in}{2.109412in}}%
\pgfpathlineto{\pgfqpoint{1.279541in}{2.109412in}}%
\pgfpathlineto{\pgfqpoint{1.279745in}{2.183676in}}%
\pgfpathlineto{\pgfqpoint{1.280561in}{2.150670in}}%
\pgfpathlineto{\pgfqpoint{1.280765in}{2.150670in}}%
\pgfpathlineto{\pgfqpoint{1.280970in}{2.191928in}}%
\pgfpathlineto{\pgfqpoint{1.281378in}{2.142418in}}%
\pgfpathlineto{\pgfqpoint{1.282603in}{1.886619in}}%
\pgfpathlineto{\pgfqpoint{1.283828in}{2.158922in}}%
\pgfpathlineto{\pgfqpoint{1.284644in}{1.927877in}}%
\pgfpathlineto{\pgfqpoint{1.285052in}{2.117664in}}%
\pgfpathlineto{\pgfqpoint{1.286073in}{2.191928in}}%
\pgfpathlineto{\pgfqpoint{1.286277in}{2.142418in}}%
\pgfpathlineto{\pgfqpoint{1.286890in}{2.241438in}}%
\pgfpathlineto{\pgfqpoint{1.287094in}{2.274444in}}%
\pgfpathlineto{\pgfqpoint{1.287298in}{2.208431in}}%
\pgfpathlineto{\pgfqpoint{1.287910in}{2.241438in}}%
\pgfpathlineto{\pgfqpoint{1.288115in}{2.241438in}}%
\pgfpathlineto{\pgfqpoint{1.288931in}{2.101160in}}%
\pgfpathlineto{\pgfqpoint{1.289544in}{2.175425in}}%
\pgfpathlineto{\pgfqpoint{1.289748in}{2.183676in}}%
\pgfpathlineto{\pgfqpoint{1.289952in}{2.158922in}}%
\pgfpathlineto{\pgfqpoint{1.290360in}{2.043399in}}%
\pgfpathlineto{\pgfqpoint{1.291177in}{2.076406in}}%
\pgfpathlineto{\pgfqpoint{1.291381in}{2.076406in}}%
\pgfpathlineto{\pgfqpoint{1.291585in}{2.068154in}}%
\pgfpathlineto{\pgfqpoint{1.291789in}{2.092909in}}%
\pgfpathlineto{\pgfqpoint{1.292197in}{2.158922in}}%
\pgfpathlineto{\pgfqpoint{1.292402in}{2.101160in}}%
\pgfpathlineto{\pgfqpoint{1.292810in}{2.018644in}}%
\pgfpathlineto{\pgfqpoint{1.293422in}{2.125915in}}%
\pgfpathlineto{\pgfqpoint{1.294035in}{2.068154in}}%
\pgfpathlineto{\pgfqpoint{1.294239in}{2.142418in}}%
\pgfpathlineto{\pgfqpoint{1.294443in}{2.109412in}}%
\pgfpathlineto{\pgfqpoint{1.295055in}{2.142418in}}%
\pgfpathlineto{\pgfqpoint{1.295260in}{2.092909in}}%
\pgfpathlineto{\pgfqpoint{1.295464in}{2.101160in}}%
\pgfpathlineto{\pgfqpoint{1.295668in}{2.068154in}}%
\pgfpathlineto{\pgfqpoint{1.296484in}{2.043399in}}%
\pgfpathlineto{\pgfqpoint{1.297301in}{2.109412in}}%
\pgfpathlineto{\pgfqpoint{1.297709in}{1.985638in}}%
\pgfpathlineto{\pgfqpoint{1.298526in}{2.051651in}}%
\pgfpathlineto{\pgfqpoint{1.299138in}{2.002141in}}%
\pgfpathlineto{\pgfqpoint{1.299342in}{2.084657in}}%
\pgfpathlineto{\pgfqpoint{1.299751in}{2.117664in}}%
\pgfpathlineto{\pgfqpoint{1.300159in}{2.084657in}}%
\pgfpathlineto{\pgfqpoint{1.300976in}{2.101160in}}%
\pgfpathlineto{\pgfqpoint{1.301384in}{1.993890in}}%
\pgfpathlineto{\pgfqpoint{1.302405in}{2.142418in}}%
\pgfpathlineto{\pgfqpoint{1.302609in}{2.117664in}}%
\pgfpathlineto{\pgfqpoint{1.302813in}{2.043399in}}%
\pgfpathlineto{\pgfqpoint{1.303425in}{2.150670in}}%
\pgfpathlineto{\pgfqpoint{1.303834in}{2.068154in}}%
\pgfpathlineto{\pgfqpoint{1.304242in}{2.167173in}}%
\pgfpathlineto{\pgfqpoint{1.304854in}{2.125915in}}%
\pgfpathlineto{\pgfqpoint{1.306079in}{2.051651in}}%
\pgfpathlineto{\pgfqpoint{1.306487in}{2.068154in}}%
\pgfpathlineto{\pgfqpoint{1.306692in}{2.035148in}}%
\pgfpathlineto{\pgfqpoint{1.306896in}{2.002141in}}%
\pgfpathlineto{\pgfqpoint{1.307100in}{2.117664in}}%
\pgfpathlineto{\pgfqpoint{1.307304in}{2.092909in}}%
\pgfpathlineto{\pgfqpoint{1.307916in}{2.150670in}}%
\pgfpathlineto{\pgfqpoint{1.308121in}{2.125915in}}%
\pgfpathlineto{\pgfqpoint{1.309141in}{2.043399in}}%
\pgfpathlineto{\pgfqpoint{1.309345in}{2.051651in}}%
\pgfpathlineto{\pgfqpoint{1.309550in}{2.043399in}}%
\pgfpathlineto{\pgfqpoint{1.309754in}{2.059902in}}%
\pgfpathlineto{\pgfqpoint{1.309958in}{2.117664in}}%
\pgfpathlineto{\pgfqpoint{1.310570in}{2.026896in}}%
\pgfpathlineto{\pgfqpoint{1.310979in}{1.993890in}}%
\pgfpathlineto{\pgfqpoint{1.311387in}{2.035148in}}%
\pgfpathlineto{\pgfqpoint{1.311591in}{2.043399in}}%
\pgfpathlineto{\pgfqpoint{1.312612in}{2.142418in}}%
\pgfpathlineto{\pgfqpoint{1.312816in}{2.117664in}}%
\pgfpathlineto{\pgfqpoint{1.313020in}{2.117664in}}%
\pgfpathlineto{\pgfqpoint{1.314857in}{1.985638in}}%
\pgfpathlineto{\pgfqpoint{1.316286in}{2.109412in}}%
\pgfpathlineto{\pgfqpoint{1.316490in}{2.092909in}}%
\pgfpathlineto{\pgfqpoint{1.316695in}{2.125915in}}%
\pgfpathlineto{\pgfqpoint{1.317307in}{2.109412in}}%
\pgfpathlineto{\pgfqpoint{1.317715in}{2.200180in}}%
\pgfpathlineto{\pgfqpoint{1.318124in}{2.257941in}}%
\pgfpathlineto{\pgfqpoint{1.318940in}{2.101160in}}%
\pgfpathlineto{\pgfqpoint{1.319144in}{2.158922in}}%
\pgfpathlineto{\pgfqpoint{1.320165in}{2.134167in}}%
\pgfpathlineto{\pgfqpoint{1.320982in}{2.051651in}}%
\pgfpathlineto{\pgfqpoint{1.320777in}{2.142418in}}%
\pgfpathlineto{\pgfqpoint{1.321390in}{2.084657in}}%
\pgfpathlineto{\pgfqpoint{1.321594in}{2.084657in}}%
\pgfpathlineto{\pgfqpoint{1.323023in}{2.241438in}}%
\pgfpathlineto{\pgfqpoint{1.323635in}{2.282696in}}%
\pgfpathlineto{\pgfqpoint{1.324452in}{2.101160in}}%
\pgfpathlineto{\pgfqpoint{1.324656in}{2.076406in}}%
\pgfpathlineto{\pgfqpoint{1.324860in}{2.142418in}}%
\pgfpathlineto{\pgfqpoint{1.325269in}{2.125915in}}%
\pgfpathlineto{\pgfqpoint{1.327718in}{2.332205in}}%
\pgfpathlineto{\pgfqpoint{1.327922in}{2.282696in}}%
\pgfpathlineto{\pgfqpoint{1.329556in}{2.158922in}}%
\pgfpathlineto{\pgfqpoint{1.330168in}{2.233186in}}%
\pgfpathlineto{\pgfqpoint{1.330576in}{2.150670in}}%
\pgfpathlineto{\pgfqpoint{1.330780in}{2.200180in}}%
\pgfpathlineto{\pgfqpoint{1.332413in}{1.977386in}}%
\pgfpathlineto{\pgfqpoint{1.333026in}{2.233186in}}%
\pgfpathlineto{\pgfqpoint{1.333638in}{2.150670in}}%
\pgfpathlineto{\pgfqpoint{1.334251in}{2.076406in}}%
\pgfpathlineto{\pgfqpoint{1.334455in}{2.167173in}}%
\pgfpathlineto{\pgfqpoint{1.334659in}{2.167173in}}%
\pgfpathlineto{\pgfqpoint{1.335884in}{2.266192in}}%
\pgfpathlineto{\pgfqpoint{1.337109in}{2.167173in}}%
\pgfpathlineto{\pgfqpoint{1.337313in}{2.175425in}}%
\pgfpathlineto{\pgfqpoint{1.337517in}{2.200180in}}%
\pgfpathlineto{\pgfqpoint{1.337721in}{1.927877in}}%
\pgfpathlineto{\pgfqpoint{1.338538in}{2.150670in}}%
\pgfpathlineto{\pgfqpoint{1.339150in}{2.092909in}}%
\pgfpathlineto{\pgfqpoint{1.339763in}{2.101160in}}%
\pgfpathlineto{\pgfqpoint{1.340783in}{2.158922in}}%
\pgfpathlineto{\pgfqpoint{1.340987in}{2.142418in}}%
\pgfpathlineto{\pgfqpoint{1.343029in}{2.274444in}}%
\pgfpathlineto{\pgfqpoint{1.344050in}{2.158922in}}%
\pgfpathlineto{\pgfqpoint{1.344254in}{2.233186in}}%
\pgfpathlineto{\pgfqpoint{1.344866in}{2.340457in}}%
\pgfpathlineto{\pgfqpoint{1.345070in}{2.290947in}}%
\pgfpathlineto{\pgfqpoint{1.346499in}{2.101160in}}%
\pgfpathlineto{\pgfqpoint{1.346703in}{2.092909in}}%
\pgfpathlineto{\pgfqpoint{1.346908in}{2.109412in}}%
\pgfpathlineto{\pgfqpoint{1.347112in}{2.101160in}}%
\pgfpathlineto{\pgfqpoint{1.348337in}{2.200180in}}%
\pgfpathlineto{\pgfqpoint{1.347928in}{2.076406in}}%
\pgfpathlineto{\pgfqpoint{1.348745in}{2.191928in}}%
\pgfpathlineto{\pgfqpoint{1.349153in}{2.134167in}}%
\pgfpathlineto{\pgfqpoint{1.349766in}{2.018644in}}%
\pgfpathlineto{\pgfqpoint{1.349970in}{2.084657in}}%
\pgfpathlineto{\pgfqpoint{1.351195in}{2.183676in}}%
\pgfpathlineto{\pgfqpoint{1.352419in}{2.117664in}}%
\pgfpathlineto{\pgfqpoint{1.353032in}{2.158922in}}%
\pgfpathlineto{\pgfqpoint{1.353440in}{2.150670in}}%
\pgfpathlineto{\pgfqpoint{1.353644in}{2.117664in}}%
\pgfpathlineto{\pgfqpoint{1.353848in}{2.200180in}}%
\pgfpathlineto{\pgfqpoint{1.354053in}{2.183676in}}%
\pgfpathlineto{\pgfqpoint{1.354257in}{2.200180in}}%
\pgfpathlineto{\pgfqpoint{1.354461in}{2.183676in}}%
\pgfpathlineto{\pgfqpoint{1.355277in}{2.101160in}}%
\pgfpathlineto{\pgfqpoint{1.355686in}{2.125915in}}%
\pgfpathlineto{\pgfqpoint{1.355890in}{2.134167in}}%
\pgfpathlineto{\pgfqpoint{1.356094in}{2.117664in}}%
\pgfpathlineto{\pgfqpoint{1.356298in}{2.092909in}}%
\pgfpathlineto{\pgfqpoint{1.356502in}{2.134167in}}%
\pgfpathlineto{\pgfqpoint{1.357115in}{2.109412in}}%
\pgfpathlineto{\pgfqpoint{1.357319in}{2.150670in}}%
\pgfpathlineto{\pgfqpoint{1.357931in}{2.092909in}}%
\pgfpathlineto{\pgfqpoint{1.358135in}{2.125915in}}%
\pgfpathlineto{\pgfqpoint{1.358544in}{2.043399in}}%
\pgfpathlineto{\pgfqpoint{1.358748in}{2.092909in}}%
\pgfpathlineto{\pgfqpoint{1.359769in}{2.175425in}}%
\pgfpathlineto{\pgfqpoint{1.360381in}{2.109412in}}%
\pgfpathlineto{\pgfqpoint{1.360789in}{2.150670in}}%
\pgfpathlineto{\pgfqpoint{1.361606in}{2.183676in}}%
\pgfpathlineto{\pgfqpoint{1.362831in}{2.084657in}}%
\pgfpathlineto{\pgfqpoint{1.363443in}{2.142418in}}%
\pgfpathlineto{\pgfqpoint{1.363239in}{2.068154in}}%
\pgfpathlineto{\pgfqpoint{1.363647in}{2.068154in}}%
\pgfpathlineto{\pgfqpoint{1.363851in}{1.960883in}}%
\pgfpathlineto{\pgfqpoint{1.364056in}{2.142418in}}%
\pgfpathlineto{\pgfqpoint{1.364464in}{2.101160in}}%
\pgfpathlineto{\pgfqpoint{1.365485in}{2.266192in}}%
\pgfpathlineto{\pgfqpoint{1.366301in}{2.191928in}}%
\pgfpathlineto{\pgfqpoint{1.366709in}{2.224934in}}%
\pgfpathlineto{\pgfqpoint{1.367730in}{1.952632in}}%
\pgfpathlineto{\pgfqpoint{1.368751in}{2.241438in}}%
\pgfpathlineto{\pgfqpoint{1.368955in}{2.191928in}}%
\pgfpathlineto{\pgfqpoint{1.369159in}{2.158922in}}%
\pgfpathlineto{\pgfqpoint{1.369567in}{2.208431in}}%
\pgfpathlineto{\pgfqpoint{1.369772in}{2.191928in}}%
\pgfpathlineto{\pgfqpoint{1.369976in}{2.249689in}}%
\pgfpathlineto{\pgfqpoint{1.370792in}{2.224934in}}%
\pgfpathlineto{\pgfqpoint{1.371405in}{2.142418in}}%
\pgfpathlineto{\pgfqpoint{1.371201in}{2.241438in}}%
\pgfpathlineto{\pgfqpoint{1.371813in}{2.167173in}}%
\pgfpathlineto{\pgfqpoint{1.372425in}{2.282696in}}%
\pgfpathlineto{\pgfqpoint{1.372834in}{2.158922in}}%
\pgfpathlineto{\pgfqpoint{1.373446in}{2.059902in}}%
\pgfpathlineto{\pgfqpoint{1.374263in}{2.068154in}}%
\pgfpathlineto{\pgfqpoint{1.374671in}{2.035148in}}%
\pgfpathlineto{\pgfqpoint{1.375692in}{2.134167in}}%
\pgfpathlineto{\pgfqpoint{1.377121in}{2.059902in}}%
\pgfpathlineto{\pgfqpoint{1.377325in}{1.870115in}}%
\pgfpathlineto{\pgfqpoint{1.377937in}{2.117664in}}%
\pgfpathlineto{\pgfqpoint{1.378141in}{2.068154in}}%
\pgfpathlineto{\pgfqpoint{1.378754in}{2.175425in}}%
\pgfpathlineto{\pgfqpoint{1.379162in}{2.117664in}}%
\pgfpathlineto{\pgfqpoint{1.379775in}{2.092909in}}%
\pgfpathlineto{\pgfqpoint{1.379570in}{2.125915in}}%
\pgfpathlineto{\pgfqpoint{1.380183in}{2.109412in}}%
\pgfpathlineto{\pgfqpoint{1.380795in}{2.274444in}}%
\pgfpathlineto{\pgfqpoint{1.381204in}{2.175425in}}%
\pgfpathlineto{\pgfqpoint{1.382020in}{2.092909in}}%
\pgfpathlineto{\pgfqpoint{1.382224in}{2.158922in}}%
\pgfpathlineto{\pgfqpoint{1.382428in}{2.183676in}}%
\pgfpathlineto{\pgfqpoint{1.382633in}{2.084657in}}%
\pgfpathlineto{\pgfqpoint{1.382837in}{2.076406in}}%
\pgfpathlineto{\pgfqpoint{1.383653in}{2.010393in}}%
\pgfpathlineto{\pgfqpoint{1.383857in}{2.076406in}}%
\pgfpathlineto{\pgfqpoint{1.384062in}{2.101160in}}%
\pgfpathlineto{\pgfqpoint{1.384470in}{2.092909in}}%
\pgfpathlineto{\pgfqpoint{1.384674in}{2.010393in}}%
\pgfpathlineto{\pgfqpoint{1.384878in}{2.101160in}}%
\pgfpathlineto{\pgfqpoint{1.385491in}{2.101160in}}%
\pgfpathlineto{\pgfqpoint{1.386511in}{2.026896in}}%
\pgfpathlineto{\pgfqpoint{1.386920in}{2.035148in}}%
\pgfpathlineto{\pgfqpoint{1.387736in}{1.969135in}}%
\pgfpathlineto{\pgfqpoint{1.387940in}{2.002141in}}%
\pgfpathlineto{\pgfqpoint{1.388144in}{2.051651in}}%
\pgfpathlineto{\pgfqpoint{1.388553in}{1.894870in}}%
\pgfpathlineto{\pgfqpoint{1.391002in}{2.142418in}}%
\pgfpathlineto{\pgfqpoint{1.391615in}{2.059902in}}%
\pgfpathlineto{\pgfqpoint{1.392023in}{2.068154in}}%
\pgfpathlineto{\pgfqpoint{1.392636in}{2.142418in}}%
\pgfpathlineto{\pgfqpoint{1.393248in}{2.109412in}}%
\pgfpathlineto{\pgfqpoint{1.393656in}{2.150670in}}%
\pgfpathlineto{\pgfqpoint{1.393860in}{2.117664in}}%
\pgfpathlineto{\pgfqpoint{1.395085in}{2.043399in}}%
\pgfpathlineto{\pgfqpoint{1.396106in}{2.241438in}}%
\pgfpathlineto{\pgfqpoint{1.396718in}{2.191928in}}%
\pgfpathlineto{\pgfqpoint{1.396923in}{2.150670in}}%
\pgfpathlineto{\pgfqpoint{1.397535in}{2.158922in}}%
\pgfpathlineto{\pgfqpoint{1.398352in}{2.266192in}}%
\pgfpathlineto{\pgfqpoint{1.398760in}{2.257941in}}%
\pgfpathlineto{\pgfqpoint{1.399985in}{2.076406in}}%
\pgfpathlineto{\pgfqpoint{1.399168in}{2.266192in}}%
\pgfpathlineto{\pgfqpoint{1.400393in}{2.125915in}}%
\pgfpathlineto{\pgfqpoint{1.401414in}{2.257941in}}%
\pgfpathlineto{\pgfqpoint{1.401618in}{2.233186in}}%
\pgfpathlineto{\pgfqpoint{1.401822in}{2.249689in}}%
\pgfpathlineto{\pgfqpoint{1.402026in}{2.191928in}}%
\pgfpathlineto{\pgfqpoint{1.402230in}{2.191928in}}%
\pgfpathlineto{\pgfqpoint{1.403455in}{2.125915in}}%
\pgfpathlineto{\pgfqpoint{1.404272in}{2.241438in}}%
\pgfpathlineto{\pgfqpoint{1.404884in}{2.216683in}}%
\pgfpathlineto{\pgfqpoint{1.405292in}{2.158922in}}%
\pgfpathlineto{\pgfqpoint{1.405905in}{2.175425in}}%
\pgfpathlineto{\pgfqpoint{1.406926in}{2.241438in}}%
\pgfpathlineto{\pgfqpoint{1.407334in}{2.233186in}}%
\pgfpathlineto{\pgfqpoint{1.408150in}{2.158922in}}%
\pgfpathlineto{\pgfqpoint{1.408355in}{2.175425in}}%
\pgfpathlineto{\pgfqpoint{1.408559in}{2.241438in}}%
\pgfpathlineto{\pgfqpoint{1.408967in}{2.167173in}}%
\pgfpathlineto{\pgfqpoint{1.409171in}{2.167173in}}%
\pgfpathlineto{\pgfqpoint{1.409375in}{2.142418in}}%
\pgfpathlineto{\pgfqpoint{1.409579in}{2.183676in}}%
\pgfpathlineto{\pgfqpoint{1.409784in}{2.167173in}}%
\pgfpathlineto{\pgfqpoint{1.410600in}{2.224934in}}%
\pgfpathlineto{\pgfqpoint{1.411008in}{2.134167in}}%
\pgfpathlineto{\pgfqpoint{1.411417in}{2.158922in}}%
\pgfpathlineto{\pgfqpoint{1.411825in}{2.274444in}}%
\pgfpathlineto{\pgfqpoint{1.412437in}{2.257941in}}%
\pgfpathlineto{\pgfqpoint{1.412642in}{2.208431in}}%
\pgfpathlineto{\pgfqpoint{1.413050in}{2.274444in}}%
\pgfpathlineto{\pgfqpoint{1.413458in}{2.249689in}}%
\pgfpathlineto{\pgfqpoint{1.414887in}{2.150670in}}%
\pgfpathlineto{\pgfqpoint{1.415091in}{2.224934in}}%
\pgfpathlineto{\pgfqpoint{1.415908in}{2.200180in}}%
\pgfpathlineto{\pgfqpoint{1.416724in}{2.150670in}}%
\pgfpathlineto{\pgfqpoint{1.417337in}{2.224934in}}%
\pgfpathlineto{\pgfqpoint{1.417541in}{2.142418in}}%
\pgfpathlineto{\pgfqpoint{1.417745in}{2.183676in}}%
\pgfpathlineto{\pgfqpoint{1.417949in}{2.175425in}}%
\pgfpathlineto{\pgfqpoint{1.419582in}{2.299199in}}%
\pgfpathlineto{\pgfqpoint{1.419991in}{2.191928in}}%
\pgfpathlineto{\pgfqpoint{1.420603in}{2.224934in}}%
\pgfpathlineto{\pgfqpoint{1.421420in}{2.249689in}}%
\pgfpathlineto{\pgfqpoint{1.422440in}{2.175425in}}%
\pgfpathlineto{\pgfqpoint{1.422644in}{2.233186in}}%
\pgfpathlineto{\pgfqpoint{1.423257in}{2.158922in}}%
\pgfpathlineto{\pgfqpoint{1.423461in}{2.167173in}}%
\pgfpathlineto{\pgfqpoint{1.423869in}{2.200180in}}%
\pgfpathlineto{\pgfqpoint{1.424686in}{2.183676in}}%
\pgfpathlineto{\pgfqpoint{1.425094in}{2.125915in}}%
\pgfpathlineto{\pgfqpoint{1.425298in}{2.208431in}}%
\pgfpathlineto{\pgfqpoint{1.425707in}{2.200180in}}%
\pgfpathlineto{\pgfqpoint{1.426931in}{2.257941in}}%
\pgfpathlineto{\pgfqpoint{1.426523in}{2.183676in}}%
\pgfpathlineto{\pgfqpoint{1.427136in}{2.233186in}}%
\pgfpathlineto{\pgfqpoint{1.428769in}{2.142418in}}%
\pgfpathlineto{\pgfqpoint{1.429381in}{2.101160in}}%
\pgfpathlineto{\pgfqpoint{1.429994in}{2.216683in}}%
\pgfpathlineto{\pgfqpoint{1.431218in}{2.134167in}}%
\pgfpathlineto{\pgfqpoint{1.432035in}{2.233186in}}%
\pgfpathlineto{\pgfqpoint{1.432647in}{2.224934in}}%
\pgfpathlineto{\pgfqpoint{1.432852in}{2.191928in}}%
\pgfpathlineto{\pgfqpoint{1.433260in}{2.249689in}}%
\pgfpathlineto{\pgfqpoint{1.433668in}{2.282696in}}%
\pgfpathlineto{\pgfqpoint{1.434076in}{2.224934in}}%
\pgfpathlineto{\pgfqpoint{1.434281in}{2.241438in}}%
\pgfpathlineto{\pgfqpoint{1.434485in}{2.249689in}}%
\pgfpathlineto{\pgfqpoint{1.434689in}{2.233186in}}%
\pgfpathlineto{\pgfqpoint{1.434893in}{2.200180in}}%
\pgfpathlineto{\pgfqpoint{1.435505in}{2.257941in}}%
\pgfpathlineto{\pgfqpoint{1.435710in}{2.241438in}}%
\pgfpathlineto{\pgfqpoint{1.436118in}{2.290947in}}%
\pgfpathlineto{\pgfqpoint{1.436322in}{2.274444in}}%
\pgfpathlineto{\pgfqpoint{1.436526in}{2.216683in}}%
\pgfpathlineto{\pgfqpoint{1.437343in}{2.299199in}}%
\pgfpathlineto{\pgfqpoint{1.437547in}{2.307450in}}%
\pgfpathlineto{\pgfqpoint{1.437751in}{2.365212in}}%
\pgfpathlineto{\pgfqpoint{1.438363in}{2.257941in}}%
\pgfpathlineto{\pgfqpoint{1.438568in}{2.315702in}}%
\pgfpathlineto{\pgfqpoint{1.438772in}{2.307450in}}%
\pgfpathlineto{\pgfqpoint{1.438976in}{2.315702in}}%
\pgfpathlineto{\pgfqpoint{1.439384in}{2.348709in}}%
\pgfpathlineto{\pgfqpoint{1.439997in}{2.315702in}}%
\pgfpathlineto{\pgfqpoint{1.440201in}{2.266192in}}%
\pgfpathlineto{\pgfqpoint{1.440609in}{2.381715in}}%
\pgfpathlineto{\pgfqpoint{1.440813in}{2.389967in}}%
\pgfpathlineto{\pgfqpoint{1.441017in}{2.282696in}}%
\pgfpathlineto{\pgfqpoint{1.442038in}{2.307450in}}%
\pgfpathlineto{\pgfqpoint{1.443263in}{2.381715in}}%
\pgfpathlineto{\pgfqpoint{1.444692in}{2.282696in}}%
\pgfpathlineto{\pgfqpoint{1.445304in}{2.249689in}}%
\pgfpathlineto{\pgfqpoint{1.445713in}{2.299199in}}%
\pgfpathlineto{\pgfqpoint{1.446121in}{2.241438in}}%
\pgfpathlineto{\pgfqpoint{1.446937in}{2.257941in}}%
\pgfpathlineto{\pgfqpoint{1.447142in}{2.249689in}}%
\pgfpathlineto{\pgfqpoint{1.447958in}{2.299199in}}%
\pgfpathlineto{\pgfqpoint{1.448162in}{2.266192in}}%
\pgfpathlineto{\pgfqpoint{1.448366in}{2.249689in}}%
\pgfpathlineto{\pgfqpoint{1.448571in}{2.282696in}}%
\pgfpathlineto{\pgfqpoint{1.448775in}{2.274444in}}%
\pgfpathlineto{\pgfqpoint{1.448979in}{2.307450in}}%
\pgfpathlineto{\pgfqpoint{1.449387in}{2.249689in}}%
\pgfpathlineto{\pgfqpoint{1.449795in}{2.274444in}}%
\pgfpathlineto{\pgfqpoint{1.450408in}{2.216683in}}%
\pgfpathlineto{\pgfqpoint{1.451020in}{2.249689in}}%
\pgfpathlineto{\pgfqpoint{1.451633in}{2.101160in}}%
\pgfpathlineto{\pgfqpoint{1.452449in}{2.191928in}}%
\pgfpathlineto{\pgfqpoint{1.452653in}{2.241438in}}%
\pgfpathlineto{\pgfqpoint{1.452858in}{2.183676in}}%
\pgfpathlineto{\pgfqpoint{1.453470in}{2.191928in}}%
\pgfpathlineto{\pgfqpoint{1.454695in}{2.299199in}}%
\pgfpathlineto{\pgfqpoint{1.455103in}{2.266192in}}%
\pgfpathlineto{\pgfqpoint{1.455511in}{2.274444in}}%
\pgfpathlineto{\pgfqpoint{1.456124in}{2.142418in}}%
\pgfpathlineto{\pgfqpoint{1.456736in}{2.340457in}}%
\pgfpathlineto{\pgfqpoint{1.457349in}{2.266192in}}%
\pgfpathlineto{\pgfqpoint{1.457553in}{2.266192in}}%
\pgfpathlineto{\pgfqpoint{1.457757in}{2.183676in}}%
\pgfpathlineto{\pgfqpoint{1.457961in}{2.307450in}}%
\pgfpathlineto{\pgfqpoint{1.458574in}{2.266192in}}%
\pgfpathlineto{\pgfqpoint{1.459594in}{2.356960in}}%
\pgfpathlineto{\pgfqpoint{1.460207in}{2.233186in}}%
\pgfpathlineto{\pgfqpoint{1.460819in}{2.290947in}}%
\pgfpathlineto{\pgfqpoint{1.461227in}{2.224934in}}%
\pgfpathlineto{\pgfqpoint{1.461636in}{2.299199in}}%
\pgfpathlineto{\pgfqpoint{1.462044in}{2.241438in}}%
\pgfpathlineto{\pgfqpoint{1.462656in}{2.183676in}}%
\pgfpathlineto{\pgfqpoint{1.462452in}{2.274444in}}%
\pgfpathlineto{\pgfqpoint{1.463065in}{2.191928in}}%
\pgfpathlineto{\pgfqpoint{1.464494in}{2.348709in}}%
\pgfpathlineto{\pgfqpoint{1.464698in}{2.348709in}}%
\pgfpathlineto{\pgfqpoint{1.465923in}{2.224934in}}%
\pgfpathlineto{\pgfqpoint{1.466127in}{2.241438in}}%
\pgfpathlineto{\pgfqpoint{1.466331in}{2.266192in}}%
\pgfpathlineto{\pgfqpoint{1.466943in}{2.208431in}}%
\pgfpathlineto{\pgfqpoint{1.467148in}{2.249689in}}%
\pgfpathlineto{\pgfqpoint{1.467352in}{1.985638in}}%
\pgfpathlineto{\pgfqpoint{1.467964in}{2.299199in}}%
\pgfpathlineto{\pgfqpoint{1.468168in}{2.208431in}}%
\pgfpathlineto{\pgfqpoint{1.468372in}{2.266192in}}%
\pgfpathlineto{\pgfqpoint{1.469189in}{2.191928in}}%
\pgfpathlineto{\pgfqpoint{1.469801in}{2.290947in}}%
\pgfpathlineto{\pgfqpoint{1.470414in}{2.241438in}}%
\pgfpathlineto{\pgfqpoint{1.470618in}{2.208431in}}%
\pgfpathlineto{\pgfqpoint{1.470822in}{2.249689in}}%
\pgfpathlineto{\pgfqpoint{1.471230in}{2.249689in}}%
\pgfpathlineto{\pgfqpoint{1.471435in}{2.323954in}}%
\pgfpathlineto{\pgfqpoint{1.472251in}{2.249689in}}%
\pgfpathlineto{\pgfqpoint{1.472864in}{2.200180in}}%
\pgfpathlineto{\pgfqpoint{1.473068in}{2.257941in}}%
\pgfpathlineto{\pgfqpoint{1.473476in}{2.224934in}}%
\pgfpathlineto{\pgfqpoint{1.473680in}{2.216683in}}%
\pgfpathlineto{\pgfqpoint{1.473884in}{2.241438in}}%
\pgfpathlineto{\pgfqpoint{1.474088in}{2.233186in}}%
\pgfpathlineto{\pgfqpoint{1.474497in}{2.315702in}}%
\pgfpathlineto{\pgfqpoint{1.474905in}{2.299199in}}%
\pgfpathlineto{\pgfqpoint{1.476130in}{2.142418in}}%
\pgfpathlineto{\pgfqpoint{1.477355in}{2.356960in}}%
\pgfpathlineto{\pgfqpoint{1.477559in}{2.315702in}}%
\pgfpathlineto{\pgfqpoint{1.477763in}{2.332205in}}%
\pgfpathlineto{\pgfqpoint{1.478171in}{2.117664in}}%
\pgfpathlineto{\pgfqpoint{1.478784in}{2.282696in}}%
\pgfpathlineto{\pgfqpoint{1.480009in}{2.224934in}}%
\pgfpathlineto{\pgfqpoint{1.480417in}{2.266192in}}%
\pgfpathlineto{\pgfqpoint{1.481029in}{2.216683in}}%
\pgfpathlineto{\pgfqpoint{1.481233in}{2.216683in}}%
\pgfpathlineto{\pgfqpoint{1.481642in}{2.323954in}}%
\pgfpathlineto{\pgfqpoint{1.482458in}{2.257941in}}%
\pgfpathlineto{\pgfqpoint{1.483275in}{2.299199in}}%
\pgfpathlineto{\pgfqpoint{1.483479in}{2.249689in}}%
\pgfpathlineto{\pgfqpoint{1.483887in}{2.307450in}}%
\pgfpathlineto{\pgfqpoint{1.484296in}{2.257941in}}%
\pgfpathlineto{\pgfqpoint{1.484500in}{2.274444in}}%
\pgfpathlineto{\pgfqpoint{1.484908in}{2.241438in}}%
\pgfpathlineto{\pgfqpoint{1.485112in}{2.233186in}}%
\pgfpathlineto{\pgfqpoint{1.485929in}{2.299199in}}%
\pgfpathlineto{\pgfqpoint{1.486133in}{2.109412in}}%
\pgfpathlineto{\pgfqpoint{1.486541in}{2.356960in}}%
\pgfpathlineto{\pgfqpoint{1.486949in}{2.340457in}}%
\pgfpathlineto{\pgfqpoint{1.487154in}{2.398218in}}%
\pgfpathlineto{\pgfqpoint{1.488174in}{2.381715in}}%
\pgfpathlineto{\pgfqpoint{1.490216in}{2.233186in}}%
\pgfpathlineto{\pgfqpoint{1.490828in}{2.241438in}}%
\pgfpathlineto{\pgfqpoint{1.491441in}{2.274444in}}%
\pgfpathlineto{\pgfqpoint{1.491849in}{2.002141in}}%
\pgfpathlineto{\pgfqpoint{1.492461in}{2.249689in}}%
\pgfpathlineto{\pgfqpoint{1.493890in}{2.340457in}}%
\pgfpathlineto{\pgfqpoint{1.494094in}{2.356960in}}%
\pgfpathlineto{\pgfqpoint{1.494299in}{2.315702in}}%
\pgfpathlineto{\pgfqpoint{1.494707in}{2.340457in}}%
\pgfpathlineto{\pgfqpoint{1.497157in}{1.812354in}}%
\pgfpathlineto{\pgfqpoint{1.497565in}{1.878367in}}%
\pgfpathlineto{\pgfqpoint{1.497973in}{1.919625in}}%
\pgfpathlineto{\pgfqpoint{1.498177in}{1.828857in}}%
\pgfpathlineto{\pgfqpoint{1.498586in}{1.688580in}}%
\pgfpathlineto{\pgfqpoint{1.499402in}{1.993890in}}%
\pgfpathlineto{\pgfqpoint{1.499810in}{1.828857in}}%
\pgfpathlineto{\pgfqpoint{1.500014in}{1.903122in}}%
\pgfpathlineto{\pgfqpoint{1.500627in}{1.812354in}}%
\pgfpathlineto{\pgfqpoint{1.500831in}{1.820606in}}%
\pgfpathlineto{\pgfqpoint{1.501035in}{1.837109in}}%
\pgfpathlineto{\pgfqpoint{1.501443in}{1.804103in}}%
\pgfpathlineto{\pgfqpoint{1.501648in}{1.804103in}}%
\pgfpathlineto{\pgfqpoint{1.502260in}{1.680329in}}%
\pgfpathlineto{\pgfqpoint{1.502668in}{1.787599in}}%
\pgfpathlineto{\pgfqpoint{1.502872in}{1.903122in}}%
\pgfpathlineto{\pgfqpoint{1.503893in}{1.870115in}}%
\pgfpathlineto{\pgfqpoint{1.504914in}{1.878367in}}%
\pgfpathlineto{\pgfqpoint{1.505322in}{1.754593in}}%
\pgfpathlineto{\pgfqpoint{1.506955in}{1.919625in}}%
\pgfpathlineto{\pgfqpoint{1.507568in}{1.870115in}}%
\pgfpathlineto{\pgfqpoint{1.507364in}{1.936128in}}%
\pgfpathlineto{\pgfqpoint{1.507772in}{1.927877in}}%
\pgfpathlineto{\pgfqpoint{1.508384in}{1.911373in}}%
\pgfpathlineto{\pgfqpoint{1.508997in}{1.977386in}}%
\pgfpathlineto{\pgfqpoint{1.510426in}{1.861864in}}%
\pgfpathlineto{\pgfqpoint{1.510630in}{1.927877in}}%
\pgfpathlineto{\pgfqpoint{1.511242in}{1.837109in}}%
\pgfpathlineto{\pgfqpoint{1.511446in}{1.845361in}}%
\pgfpathlineto{\pgfqpoint{1.512059in}{1.952632in}}%
\pgfpathlineto{\pgfqpoint{1.512467in}{1.936128in}}%
\pgfpathlineto{\pgfqpoint{1.513692in}{1.738090in}}%
\pgfpathlineto{\pgfqpoint{1.514509in}{1.870115in}}%
\pgfpathlineto{\pgfqpoint{1.514917in}{1.828857in}}%
\pgfpathlineto{\pgfqpoint{1.515733in}{1.944380in}}%
\pgfpathlineto{\pgfqpoint{1.516142in}{1.903122in}}%
\pgfpathlineto{\pgfqpoint{1.516754in}{1.754593in}}%
\pgfpathlineto{\pgfqpoint{1.516958in}{1.886619in}}%
\pgfpathlineto{\pgfqpoint{1.517162in}{1.927877in}}%
\pgfpathlineto{\pgfqpoint{1.517367in}{1.680329in}}%
\pgfpathlineto{\pgfqpoint{1.518183in}{1.944380in}}%
\pgfpathlineto{\pgfqpoint{1.518796in}{1.828857in}}%
\pgfpathlineto{\pgfqpoint{1.519816in}{1.861864in}}%
\pgfpathlineto{\pgfqpoint{1.520633in}{1.960883in}}%
\pgfpathlineto{\pgfqpoint{1.521041in}{1.894870in}}%
\pgfpathlineto{\pgfqpoint{1.521858in}{1.795851in}}%
\pgfpathlineto{\pgfqpoint{1.522674in}{1.845361in}}%
\pgfpathlineto{\pgfqpoint{1.523083in}{1.969135in}}%
\pgfpathlineto{\pgfqpoint{1.524103in}{1.936128in}}%
\pgfpathlineto{\pgfqpoint{1.524512in}{1.853612in}}%
\pgfpathlineto{\pgfqpoint{1.524920in}{1.919625in}}%
\pgfpathlineto{\pgfqpoint{1.525941in}{2.051651in}}%
\pgfpathlineto{\pgfqpoint{1.526349in}{1.969135in}}%
\pgfpathlineto{\pgfqpoint{1.526757in}{1.985638in}}%
\pgfpathlineto{\pgfqpoint{1.527574in}{1.894870in}}%
\pgfpathlineto{\pgfqpoint{1.528186in}{1.886619in}}%
\pgfpathlineto{\pgfqpoint{1.528799in}{1.985638in}}%
\pgfpathlineto{\pgfqpoint{1.530023in}{1.787599in}}%
\pgfpathlineto{\pgfqpoint{1.530636in}{2.076406in}}%
\pgfpathlineto{\pgfqpoint{1.531452in}{2.035148in}}%
\pgfpathlineto{\pgfqpoint{1.532269in}{2.043399in}}%
\pgfpathlineto{\pgfqpoint{1.532677in}{1.952632in}}%
\pgfpathlineto{\pgfqpoint{1.533494in}{2.068154in}}%
\pgfpathlineto{\pgfqpoint{1.533902in}{2.051651in}}%
\pgfpathlineto{\pgfqpoint{1.534106in}{2.051651in}}%
\pgfpathlineto{\pgfqpoint{1.534515in}{2.035148in}}%
\pgfpathlineto{\pgfqpoint{1.534923in}{1.952632in}}%
\pgfpathlineto{\pgfqpoint{1.535535in}{2.018644in}}%
\pgfpathlineto{\pgfqpoint{1.535739in}{2.026896in}}%
\pgfpathlineto{\pgfqpoint{1.536352in}{1.936128in}}%
\pgfpathlineto{\pgfqpoint{1.536556in}{2.035148in}}%
\pgfpathlineto{\pgfqpoint{1.536760in}{2.018644in}}%
\pgfpathlineto{\pgfqpoint{1.538802in}{1.837109in}}%
\pgfpathlineto{\pgfqpoint{1.539006in}{1.903122in}}%
\pgfpathlineto{\pgfqpoint{1.539618in}{1.795851in}}%
\pgfpathlineto{\pgfqpoint{1.539822in}{1.820606in}}%
\pgfpathlineto{\pgfqpoint{1.541251in}{1.672077in}}%
\pgfpathlineto{\pgfqpoint{1.541455in}{1.672077in}}%
\pgfpathlineto{\pgfqpoint{1.542476in}{1.738090in}}%
\pgfpathlineto{\pgfqpoint{1.542680in}{1.705083in}}%
\pgfpathlineto{\pgfqpoint{1.543497in}{1.655574in}}%
\pgfpathlineto{\pgfqpoint{1.543701in}{1.705083in}}%
\pgfpathlineto{\pgfqpoint{1.543905in}{1.738090in}}%
\pgfpathlineto{\pgfqpoint{1.544313in}{1.639071in}}%
\pgfpathlineto{\pgfqpoint{1.544926in}{1.597813in}}%
\pgfpathlineto{\pgfqpoint{1.545130in}{1.738090in}}%
\pgfpathlineto{\pgfqpoint{1.546151in}{1.705083in}}%
\pgfpathlineto{\pgfqpoint{1.546355in}{1.705083in}}%
\pgfpathlineto{\pgfqpoint{1.546559in}{1.688580in}}%
\pgfpathlineto{\pgfqpoint{1.546967in}{1.738090in}}%
\pgfpathlineto{\pgfqpoint{1.547171in}{1.845361in}}%
\pgfpathlineto{\pgfqpoint{1.547988in}{1.721587in}}%
\pgfpathlineto{\pgfqpoint{1.548396in}{1.795851in}}%
\pgfpathlineto{\pgfqpoint{1.549213in}{1.754593in}}%
\pgfpathlineto{\pgfqpoint{1.549825in}{1.705083in}}%
\pgfpathlineto{\pgfqpoint{1.550642in}{1.688580in}}%
\pgfpathlineto{\pgfqpoint{1.550846in}{1.812354in}}%
\pgfpathlineto{\pgfqpoint{1.551458in}{1.672077in}}%
\pgfpathlineto{\pgfqpoint{1.552071in}{1.729838in}}%
\pgfpathlineto{\pgfqpoint{1.553092in}{1.779348in}}%
\pgfpathlineto{\pgfqpoint{1.553500in}{1.771096in}}%
\pgfpathlineto{\pgfqpoint{1.554725in}{1.729838in}}%
\pgfpathlineto{\pgfqpoint{1.554929in}{1.787599in}}%
\pgfpathlineto{\pgfqpoint{1.555950in}{1.779348in}}%
\pgfpathlineto{\pgfqpoint{1.556358in}{1.779348in}}%
\pgfpathlineto{\pgfqpoint{1.556766in}{1.705083in}}%
\pgfpathlineto{\pgfqpoint{1.556970in}{1.779348in}}%
\pgfpathlineto{\pgfqpoint{1.557174in}{1.837109in}}%
\pgfpathlineto{\pgfqpoint{1.557583in}{1.713335in}}%
\pgfpathlineto{\pgfqpoint{1.557991in}{1.779348in}}%
\pgfpathlineto{\pgfqpoint{1.558603in}{1.713335in}}%
\pgfpathlineto{\pgfqpoint{1.559012in}{1.721587in}}%
\pgfpathlineto{\pgfqpoint{1.559216in}{1.754593in}}%
\pgfpathlineto{\pgfqpoint{1.559420in}{1.581309in}}%
\pgfpathlineto{\pgfqpoint{1.560032in}{1.812354in}}%
\pgfpathlineto{\pgfqpoint{1.560237in}{1.804103in}}%
\pgfpathlineto{\pgfqpoint{1.560441in}{1.828857in}}%
\pgfpathlineto{\pgfqpoint{1.560849in}{1.812354in}}%
\pgfpathlineto{\pgfqpoint{1.561053in}{1.721587in}}%
\pgfpathlineto{\pgfqpoint{1.561666in}{1.861864in}}%
\pgfpathlineto{\pgfqpoint{1.561870in}{1.828857in}}%
\pgfpathlineto{\pgfqpoint{1.562074in}{1.853612in}}%
\pgfpathlineto{\pgfqpoint{1.562278in}{1.779348in}}%
\pgfpathlineto{\pgfqpoint{1.562890in}{1.828857in}}%
\pgfpathlineto{\pgfqpoint{1.563095in}{1.663825in}}%
\pgfpathlineto{\pgfqpoint{1.563911in}{1.721587in}}%
\pgfpathlineto{\pgfqpoint{1.564319in}{1.754593in}}%
\pgfpathlineto{\pgfqpoint{1.564728in}{1.696832in}}%
\pgfpathlineto{\pgfqpoint{1.564932in}{1.688580in}}%
\pgfpathlineto{\pgfqpoint{1.565136in}{1.713335in}}%
\pgfpathlineto{\pgfqpoint{1.565340in}{1.771096in}}%
\pgfpathlineto{\pgfqpoint{1.566157in}{1.696832in}}%
\pgfpathlineto{\pgfqpoint{1.566565in}{1.705083in}}%
\pgfpathlineto{\pgfqpoint{1.567382in}{1.639071in}}%
\pgfpathlineto{\pgfqpoint{1.567790in}{1.663825in}}%
\pgfpathlineto{\pgfqpoint{1.567994in}{1.738090in}}%
\pgfpathlineto{\pgfqpoint{1.568606in}{1.655574in}}%
\pgfpathlineto{\pgfqpoint{1.568811in}{1.696832in}}%
\pgfpathlineto{\pgfqpoint{1.570035in}{1.622567in}}%
\pgfpathlineto{\pgfqpoint{1.570240in}{1.622567in}}%
\pgfpathlineto{\pgfqpoint{1.570444in}{1.630819in}}%
\pgfpathlineto{\pgfqpoint{1.571056in}{1.540051in}}%
\pgfpathlineto{\pgfqpoint{1.571873in}{1.548303in}}%
\pgfpathlineto{\pgfqpoint{1.573506in}{1.655574in}}%
\pgfpathlineto{\pgfqpoint{1.573914in}{1.647322in}}%
\pgfpathlineto{\pgfqpoint{1.574118in}{1.622567in}}%
\pgfpathlineto{\pgfqpoint{1.574322in}{1.729838in}}%
\pgfpathlineto{\pgfqpoint{1.574527in}{1.680329in}}%
\pgfpathlineto{\pgfqpoint{1.575343in}{1.639071in}}%
\pgfpathlineto{\pgfqpoint{1.575956in}{1.787599in}}%
\pgfpathlineto{\pgfqpoint{1.577589in}{1.581309in}}%
\pgfpathlineto{\pgfqpoint{1.578201in}{1.655574in}}%
\pgfpathlineto{\pgfqpoint{1.578609in}{1.614316in}}%
\pgfpathlineto{\pgfqpoint{1.578814in}{2.142418in}}%
\pgfpathlineto{\pgfqpoint{1.579630in}{1.614316in}}%
\pgfpathlineto{\pgfqpoint{1.579834in}{1.531800in}}%
\pgfpathlineto{\pgfqpoint{1.580243in}{1.680329in}}%
\pgfpathlineto{\pgfqpoint{1.580447in}{1.655574in}}%
\pgfpathlineto{\pgfqpoint{1.580651in}{1.663825in}}%
\pgfpathlineto{\pgfqpoint{1.580855in}{1.630819in}}%
\pgfpathlineto{\pgfqpoint{1.581059in}{1.606064in}}%
\pgfpathlineto{\pgfqpoint{1.581263in}{1.696832in}}%
\pgfpathlineto{\pgfqpoint{1.581467in}{1.672077in}}%
\pgfpathlineto{\pgfqpoint{1.581672in}{1.688580in}}%
\pgfpathlineto{\pgfqpoint{1.582080in}{1.449284in}}%
\pgfpathlineto{\pgfqpoint{1.582692in}{1.655574in}}%
\pgfpathlineto{\pgfqpoint{1.582896in}{1.647322in}}%
\pgfpathlineto{\pgfqpoint{1.583101in}{1.663825in}}%
\pgfpathlineto{\pgfqpoint{1.583509in}{1.754593in}}%
\pgfpathlineto{\pgfqpoint{1.583917in}{1.556554in}}%
\pgfpathlineto{\pgfqpoint{1.584121in}{1.672077in}}%
\pgfpathlineto{\pgfqpoint{1.584325in}{1.680329in}}%
\pgfpathlineto{\pgfqpoint{1.584529in}{1.672077in}}%
\pgfpathlineto{\pgfqpoint{1.584734in}{1.771096in}}%
\pgfpathlineto{\pgfqpoint{1.585550in}{1.672077in}}%
\pgfpathlineto{\pgfqpoint{1.585958in}{1.622567in}}%
\pgfpathlineto{\pgfqpoint{1.586571in}{1.672077in}}%
\pgfpathlineto{\pgfqpoint{1.587183in}{1.779348in}}%
\pgfpathlineto{\pgfqpoint{1.587387in}{1.746341in}}%
\pgfpathlineto{\pgfqpoint{1.588000in}{1.432780in}}%
\pgfpathlineto{\pgfqpoint{1.588612in}{1.663825in}}%
\pgfpathlineto{\pgfqpoint{1.588816in}{1.647322in}}%
\pgfpathlineto{\pgfqpoint{1.589021in}{1.713335in}}%
\pgfpathlineto{\pgfqpoint{1.589633in}{1.663825in}}%
\pgfpathlineto{\pgfqpoint{1.589837in}{1.688580in}}%
\pgfpathlineto{\pgfqpoint{1.590654in}{1.754593in}}%
\pgfpathlineto{\pgfqpoint{1.590858in}{1.688580in}}%
\pgfpathlineto{\pgfqpoint{1.591062in}{1.630819in}}%
\pgfpathlineto{\pgfqpoint{1.591674in}{1.713335in}}%
\pgfpathlineto{\pgfqpoint{1.591879in}{1.738090in}}%
\pgfpathlineto{\pgfqpoint{1.592287in}{1.540051in}}%
\pgfpathlineto{\pgfqpoint{1.592695in}{1.779348in}}%
\pgfpathlineto{\pgfqpoint{1.592899in}{1.795851in}}%
\pgfpathlineto{\pgfqpoint{1.593512in}{1.762845in}}%
\pgfpathlineto{\pgfqpoint{1.594532in}{1.647322in}}%
\pgfpathlineto{\pgfqpoint{1.594941in}{1.655574in}}%
\pgfpathlineto{\pgfqpoint{1.595145in}{1.688580in}}%
\pgfpathlineto{\pgfqpoint{1.595553in}{1.729838in}}%
\pgfpathlineto{\pgfqpoint{1.596166in}{1.449284in}}%
\pgfpathlineto{\pgfqpoint{1.597186in}{1.713335in}}%
\pgfpathlineto{\pgfqpoint{1.597390in}{1.696832in}}%
\pgfpathlineto{\pgfqpoint{1.597595in}{1.639071in}}%
\pgfpathlineto{\pgfqpoint{1.598615in}{1.647322in}}%
\pgfpathlineto{\pgfqpoint{1.598819in}{1.647322in}}%
\pgfpathlineto{\pgfqpoint{1.599228in}{1.622567in}}%
\pgfpathlineto{\pgfqpoint{1.599432in}{1.663825in}}%
\pgfpathlineto{\pgfqpoint{1.599636in}{1.639071in}}%
\pgfpathlineto{\pgfqpoint{1.600657in}{1.680329in}}%
\pgfpathlineto{\pgfqpoint{1.601065in}{1.713335in}}%
\pgfpathlineto{\pgfqpoint{1.601882in}{1.564806in}}%
\pgfpathlineto{\pgfqpoint{1.602290in}{1.630819in}}%
\pgfpathlineto{\pgfqpoint{1.602494in}{1.366768in}}%
\pgfpathlineto{\pgfqpoint{1.603106in}{1.663825in}}%
\pgfpathlineto{\pgfqpoint{1.603311in}{1.639071in}}%
\pgfpathlineto{\pgfqpoint{1.603719in}{1.614316in}}%
\pgfpathlineto{\pgfqpoint{1.604535in}{1.721587in}}%
\pgfpathlineto{\pgfqpoint{1.605760in}{1.342013in}}%
\pgfpathlineto{\pgfqpoint{1.605964in}{1.350264in}}%
\pgfpathlineto{\pgfqpoint{1.606169in}{1.647322in}}%
\pgfpathlineto{\pgfqpoint{1.607189in}{1.606064in}}%
\pgfpathlineto{\pgfqpoint{1.608822in}{1.795851in}}%
\pgfpathlineto{\pgfqpoint{1.609027in}{1.787599in}}%
\pgfpathlineto{\pgfqpoint{1.609435in}{1.787599in}}%
\pgfpathlineto{\pgfqpoint{1.609843in}{1.886619in}}%
\pgfpathlineto{\pgfqpoint{1.610251in}{1.812354in}}%
\pgfpathlineto{\pgfqpoint{1.610456in}{1.713335in}}%
\pgfpathlineto{\pgfqpoint{1.611476in}{1.721587in}}%
\pgfpathlineto{\pgfqpoint{1.612293in}{1.795851in}}%
\pgfpathlineto{\pgfqpoint{1.612701in}{1.771096in}}%
\pgfpathlineto{\pgfqpoint{1.613109in}{1.779348in}}%
\pgfpathlineto{\pgfqpoint{1.613722in}{1.597813in}}%
\pgfpathlineto{\pgfqpoint{1.613926in}{1.828857in}}%
\pgfpathlineto{\pgfqpoint{1.614743in}{1.820606in}}%
\pgfpathlineto{\pgfqpoint{1.615967in}{1.399774in}}%
\pgfpathlineto{\pgfqpoint{1.616172in}{1.432780in}}%
\pgfpathlineto{\pgfqpoint{1.616784in}{1.639071in}}%
\pgfpathlineto{\pgfqpoint{1.617396in}{1.581309in}}%
\pgfpathlineto{\pgfqpoint{1.617601in}{1.589561in}}%
\pgfpathlineto{\pgfqpoint{1.618417in}{1.597813in}}%
\pgfpathlineto{\pgfqpoint{1.618621in}{1.523548in}}%
\pgfpathlineto{\pgfqpoint{1.619234in}{1.614316in}}%
\pgfpathlineto{\pgfqpoint{1.619642in}{1.523548in}}%
\pgfpathlineto{\pgfqpoint{1.621479in}{1.630819in}}%
\pgfpathlineto{\pgfqpoint{1.621683in}{1.630819in}}%
\pgfpathlineto{\pgfqpoint{1.622704in}{1.548303in}}%
\pgfpathlineto{\pgfqpoint{1.623112in}{1.564806in}}%
\pgfpathlineto{\pgfqpoint{1.623929in}{1.622567in}}%
\pgfpathlineto{\pgfqpoint{1.624133in}{1.614316in}}%
\pgfpathlineto{\pgfqpoint{1.624746in}{1.556554in}}%
\pgfpathlineto{\pgfqpoint{1.624541in}{1.622567in}}%
\pgfpathlineto{\pgfqpoint{1.624950in}{1.589561in}}%
\pgfpathlineto{\pgfqpoint{1.625154in}{1.655574in}}%
\pgfpathlineto{\pgfqpoint{1.625766in}{1.498793in}}%
\pgfpathlineto{\pgfqpoint{1.625970in}{1.498793in}}%
\pgfpathlineto{\pgfqpoint{1.626379in}{1.490542in}}%
\pgfpathlineto{\pgfqpoint{1.627195in}{1.540051in}}%
\pgfpathlineto{\pgfqpoint{1.628216in}{1.449284in}}%
\pgfpathlineto{\pgfqpoint{1.628420in}{1.482290in}}%
\pgfpathlineto{\pgfqpoint{1.628624in}{1.531800in}}%
\pgfpathlineto{\pgfqpoint{1.628828in}{1.457535in}}%
\pgfpathlineto{\pgfqpoint{1.629237in}{1.474038in}}%
\pgfpathlineto{\pgfqpoint{1.630257in}{1.432780in}}%
\pgfpathlineto{\pgfqpoint{1.630462in}{1.457535in}}%
\pgfpathlineto{\pgfqpoint{1.630870in}{1.548303in}}%
\pgfpathlineto{\pgfqpoint{1.631482in}{1.498793in}}%
\pgfpathlineto{\pgfqpoint{1.632503in}{1.276000in}}%
\pgfpathlineto{\pgfqpoint{1.632911in}{1.317258in}}%
\pgfpathlineto{\pgfqpoint{1.633524in}{1.523548in}}%
\pgfpathlineto{\pgfqpoint{1.634136in}{1.474038in}}%
\pgfpathlineto{\pgfqpoint{1.635157in}{1.441032in}}%
\pgfpathlineto{\pgfqpoint{1.636382in}{1.531800in}}%
\pgfpathlineto{\pgfqpoint{1.635769in}{1.416277in}}%
\pgfpathlineto{\pgfqpoint{1.636586in}{1.507045in}}%
\pgfpathlineto{\pgfqpoint{1.637198in}{1.474038in}}%
\pgfpathlineto{\pgfqpoint{1.637402in}{1.490542in}}%
\pgfpathlineto{\pgfqpoint{1.637811in}{1.482290in}}%
\pgfpathlineto{\pgfqpoint{1.638831in}{1.589561in}}%
\pgfpathlineto{\pgfqpoint{1.639444in}{1.474038in}}%
\pgfpathlineto{\pgfqpoint{1.640056in}{1.507045in}}%
\pgfpathlineto{\pgfqpoint{1.640260in}{1.457535in}}%
\pgfpathlineto{\pgfqpoint{1.640465in}{1.556554in}}%
\pgfpathlineto{\pgfqpoint{1.641077in}{1.498793in}}%
\pgfpathlineto{\pgfqpoint{1.641281in}{1.523548in}}%
\pgfpathlineto{\pgfqpoint{1.641485in}{1.465787in}}%
\pgfpathlineto{\pgfqpoint{1.641689in}{1.391522in}}%
\pgfpathlineto{\pgfqpoint{1.642506in}{1.432780in}}%
\pgfpathlineto{\pgfqpoint{1.642710in}{1.449284in}}%
\pgfpathlineto{\pgfqpoint{1.642914in}{1.441032in}}%
\pgfpathlineto{\pgfqpoint{1.643527in}{1.201736in}}%
\pgfpathlineto{\pgfqpoint{1.643731in}{1.309006in}}%
\pgfpathlineto{\pgfqpoint{1.644343in}{1.515296in}}%
\pgfpathlineto{\pgfqpoint{1.644956in}{1.441032in}}%
\pgfpathlineto{\pgfqpoint{1.645160in}{1.482290in}}%
\pgfpathlineto{\pgfqpoint{1.645364in}{1.399774in}}%
\pgfpathlineto{\pgfqpoint{1.645772in}{1.416277in}}%
\pgfpathlineto{\pgfqpoint{1.645976in}{1.399774in}}%
\pgfpathlineto{\pgfqpoint{1.646385in}{1.449284in}}%
\pgfpathlineto{\pgfqpoint{1.646589in}{1.465787in}}%
\pgfpathlineto{\pgfqpoint{1.646997in}{1.432780in}}%
\pgfpathlineto{\pgfqpoint{1.647814in}{1.325510in}}%
\pgfpathlineto{\pgfqpoint{1.648222in}{1.366768in}}%
\pgfpathlineto{\pgfqpoint{1.648630in}{1.399774in}}%
\pgfpathlineto{\pgfqpoint{1.648834in}{1.242994in}}%
\pgfpathlineto{\pgfqpoint{1.649651in}{1.465787in}}%
\pgfpathlineto{\pgfqpoint{1.650059in}{1.531800in}}%
\pgfpathlineto{\pgfqpoint{1.650468in}{1.457535in}}%
\pgfpathlineto{\pgfqpoint{1.650672in}{1.457535in}}%
\pgfpathlineto{\pgfqpoint{1.650876in}{1.350264in}}%
\pgfpathlineto{\pgfqpoint{1.651692in}{1.408026in}}%
\pgfpathlineto{\pgfqpoint{1.652101in}{1.515296in}}%
\pgfpathlineto{\pgfqpoint{1.652917in}{1.482290in}}%
\pgfpathlineto{\pgfqpoint{1.654142in}{1.416277in}}%
\pgfpathlineto{\pgfqpoint{1.654346in}{1.474038in}}%
\pgfpathlineto{\pgfqpoint{1.655163in}{1.457535in}}%
\pgfpathlineto{\pgfqpoint{1.655979in}{1.424529in}}%
\pgfpathlineto{\pgfqpoint{1.655571in}{1.498793in}}%
\pgfpathlineto{\pgfqpoint{1.656184in}{1.457535in}}%
\pgfpathlineto{\pgfqpoint{1.656388in}{1.457535in}}%
\pgfpathlineto{\pgfqpoint{1.656592in}{1.548303in}}%
\pgfpathlineto{\pgfqpoint{1.657000in}{1.432780in}}%
\pgfpathlineto{\pgfqpoint{1.657204in}{1.449284in}}%
\pgfpathlineto{\pgfqpoint{1.657613in}{1.185232in}}%
\pgfpathlineto{\pgfqpoint{1.658429in}{1.391522in}}%
\pgfpathlineto{\pgfqpoint{1.658633in}{1.482290in}}%
\pgfpathlineto{\pgfqpoint{1.659246in}{1.383271in}}%
\pgfpathlineto{\pgfqpoint{1.659450in}{1.408026in}}%
\pgfpathlineto{\pgfqpoint{1.660266in}{1.300755in}}%
\pgfpathlineto{\pgfqpoint{1.660471in}{1.375019in}}%
\pgfpathlineto{\pgfqpoint{1.661083in}{1.474038in}}%
\pgfpathlineto{\pgfqpoint{1.661491in}{1.375019in}}%
\pgfpathlineto{\pgfqpoint{1.662104in}{1.209987in}}%
\pgfpathlineto{\pgfqpoint{1.662308in}{1.061458in}}%
\pgfpathlineto{\pgfqpoint{1.662716in}{1.498793in}}%
\pgfpathlineto{\pgfqpoint{1.662920in}{1.391522in}}%
\pgfpathlineto{\pgfqpoint{1.664553in}{1.201736in}}%
\pgfpathlineto{\pgfqpoint{1.665982in}{1.399774in}}%
\pgfpathlineto{\pgfqpoint{1.667003in}{1.276000in}}%
\pgfpathlineto{\pgfqpoint{1.666799in}{1.416277in}}%
\pgfpathlineto{\pgfqpoint{1.667207in}{1.284252in}}%
\pgfpathlineto{\pgfqpoint{1.667411in}{1.342013in}}%
\pgfpathlineto{\pgfqpoint{1.668024in}{1.251245in}}%
\pgfpathlineto{\pgfqpoint{1.668228in}{1.077961in}}%
\pgfpathlineto{\pgfqpoint{1.668636in}{1.350264in}}%
\pgfpathlineto{\pgfqpoint{1.669044in}{1.267748in}}%
\pgfpathlineto{\pgfqpoint{1.669453in}{1.251245in}}%
\pgfpathlineto{\pgfqpoint{1.669657in}{1.276000in}}%
\pgfpathlineto{\pgfqpoint{1.670065in}{1.325510in}}%
\pgfpathlineto{\pgfqpoint{1.670473in}{1.242994in}}%
\pgfpathlineto{\pgfqpoint{1.671698in}{1.383271in}}%
\pgfpathlineto{\pgfqpoint{1.671902in}{1.350264in}}%
\pgfpathlineto{\pgfqpoint{1.672719in}{1.234742in}}%
\pgfpathlineto{\pgfqpoint{1.673127in}{1.242994in}}%
\pgfpathlineto{\pgfqpoint{1.674148in}{1.317258in}}%
\pgfpathlineto{\pgfqpoint{1.673740in}{1.226490in}}%
\pgfpathlineto{\pgfqpoint{1.674352in}{1.284252in}}%
\pgfpathlineto{\pgfqpoint{1.674965in}{1.391522in}}%
\pgfpathlineto{\pgfqpoint{1.674760in}{1.276000in}}%
\pgfpathlineto{\pgfqpoint{1.675169in}{1.300755in}}%
\pgfpathlineto{\pgfqpoint{1.675577in}{1.325510in}}%
\pgfpathlineto{\pgfqpoint{1.675985in}{1.135723in}}%
\pgfpathlineto{\pgfqpoint{1.677210in}{1.399774in}}%
\pgfpathlineto{\pgfqpoint{1.677823in}{1.342013in}}%
\pgfpathlineto{\pgfqpoint{1.678027in}{1.391522in}}%
\pgfpathlineto{\pgfqpoint{1.678231in}{1.408026in}}%
\pgfpathlineto{\pgfqpoint{1.678435in}{1.366768in}}%
\pgfpathlineto{\pgfqpoint{1.678843in}{1.391522in}}%
\pgfpathlineto{\pgfqpoint{1.679047in}{1.317258in}}%
\pgfpathlineto{\pgfqpoint{1.679252in}{1.408026in}}%
\pgfpathlineto{\pgfqpoint{1.679864in}{1.399774in}}%
\pgfpathlineto{\pgfqpoint{1.680068in}{1.408026in}}%
\pgfpathlineto{\pgfqpoint{1.680272in}{1.399774in}}%
\pgfpathlineto{\pgfqpoint{1.681497in}{1.119219in}}%
\pgfpathlineto{\pgfqpoint{1.681701in}{1.292503in}}%
\pgfpathlineto{\pgfqpoint{1.681905in}{1.300755in}}%
\pgfpathlineto{\pgfqpoint{1.682110in}{1.556554in}}%
\pgfpathlineto{\pgfqpoint{1.682926in}{1.399774in}}%
\pgfpathlineto{\pgfqpoint{1.683130in}{1.399774in}}%
\pgfpathlineto{\pgfqpoint{1.683334in}{1.416277in}}%
\pgfpathlineto{\pgfqpoint{1.683947in}{1.168729in}}%
\pgfpathlineto{\pgfqpoint{1.684355in}{1.449284in}}%
\pgfpathlineto{\pgfqpoint{1.684968in}{1.540051in}}%
\pgfpathlineto{\pgfqpoint{1.685376in}{1.482290in}}%
\pgfpathlineto{\pgfqpoint{1.686805in}{1.383271in}}%
\pgfpathlineto{\pgfqpoint{1.686192in}{1.507045in}}%
\pgfpathlineto{\pgfqpoint{1.687009in}{1.399774in}}%
\pgfpathlineto{\pgfqpoint{1.687213in}{1.399774in}}%
\pgfpathlineto{\pgfqpoint{1.687621in}{1.176981in}}%
\pgfpathlineto{\pgfqpoint{1.688234in}{1.408026in}}%
\pgfpathlineto{\pgfqpoint{1.688438in}{1.333761in}}%
\pgfpathlineto{\pgfqpoint{1.688642in}{1.441032in}}%
\pgfpathlineto{\pgfqpoint{1.689459in}{1.383271in}}%
\pgfpathlineto{\pgfqpoint{1.689867in}{1.284252in}}%
\pgfpathlineto{\pgfqpoint{1.690275in}{1.457535in}}%
\pgfpathlineto{\pgfqpoint{1.691092in}{1.317258in}}%
\pgfpathlineto{\pgfqpoint{1.693950in}{0.970691in}}%
\pgfpathlineto{\pgfqpoint{1.691704in}{1.342013in}}%
\pgfpathlineto{\pgfqpoint{1.694154in}{0.978942in}}%
\pgfpathlineto{\pgfqpoint{1.695175in}{0.970691in}}%
\pgfpathlineto{\pgfqpoint{1.695787in}{1.366768in}}%
\pgfpathlineto{\pgfqpoint{1.696808in}{1.061458in}}%
\pgfpathlineto{\pgfqpoint{1.697012in}{1.242994in}}%
\pgfpathlineto{\pgfqpoint{1.697420in}{1.193484in}}%
\pgfpathlineto{\pgfqpoint{1.697624in}{1.251245in}}%
\pgfpathlineto{\pgfqpoint{1.697829in}{1.209987in}}%
\pgfpathlineto{\pgfqpoint{1.698033in}{1.284252in}}%
\pgfpathlineto{\pgfqpoint{1.698441in}{1.193484in}}%
\pgfpathlineto{\pgfqpoint{1.698645in}{1.193484in}}%
\pgfpathlineto{\pgfqpoint{1.699258in}{1.218239in}}%
\pgfpathlineto{\pgfqpoint{1.699666in}{1.110968in}}%
\pgfpathlineto{\pgfqpoint{1.699870in}{1.267748in}}%
\pgfpathlineto{\pgfqpoint{1.700687in}{1.209987in}}%
\pgfpathlineto{\pgfqpoint{1.701095in}{1.168729in}}%
\pgfpathlineto{\pgfqpoint{1.701299in}{1.185232in}}%
\pgfpathlineto{\pgfqpoint{1.701911in}{1.317258in}}%
\pgfpathlineto{\pgfqpoint{1.702116in}{1.276000in}}%
\pgfpathlineto{\pgfqpoint{1.703136in}{0.970691in}}%
\pgfpathlineto{\pgfqpoint{1.703340in}{1.110968in}}%
\pgfpathlineto{\pgfqpoint{1.704157in}{1.135723in}}%
\pgfpathlineto{\pgfqpoint{1.704769in}{1.036703in}}%
\pgfpathlineto{\pgfqpoint{1.705994in}{1.218239in}}%
\pgfpathlineto{\pgfqpoint{1.706198in}{1.185232in}}%
\pgfpathlineto{\pgfqpoint{1.707832in}{0.978942in}}%
\pgfpathlineto{\pgfqpoint{1.708036in}{1.011949in}}%
\pgfpathlineto{\pgfqpoint{1.708240in}{0.838665in}}%
\pgfpathlineto{\pgfqpoint{1.708444in}{1.119219in}}%
\pgfpathlineto{\pgfqpoint{1.709056in}{1.011949in}}%
\pgfpathlineto{\pgfqpoint{1.709669in}{1.094465in}}%
\pgfpathlineto{\pgfqpoint{1.710281in}{1.069710in}}%
\pgfpathlineto{\pgfqpoint{1.710485in}{1.003697in}}%
\pgfpathlineto{\pgfqpoint{1.711098in}{1.135723in}}%
\pgfpathlineto{\pgfqpoint{1.711302in}{1.036703in}}%
\pgfpathlineto{\pgfqpoint{1.711914in}{1.234742in}}%
\pgfpathlineto{\pgfqpoint{1.712527in}{1.102716in}}%
\pgfpathlineto{\pgfqpoint{1.712731in}{1.069710in}}%
\pgfpathlineto{\pgfqpoint{1.713139in}{1.102716in}}%
\pgfpathlineto{\pgfqpoint{1.713343in}{1.160477in}}%
\pgfpathlineto{\pgfqpoint{1.713548in}{1.053207in}}%
\pgfpathlineto{\pgfqpoint{1.713956in}{1.119219in}}%
\pgfpathlineto{\pgfqpoint{1.714364in}{0.838665in}}%
\pgfpathlineto{\pgfqpoint{1.714977in}{0.962439in}}%
\pgfpathlineto{\pgfqpoint{1.715793in}{1.069710in}}%
\pgfpathlineto{\pgfqpoint{1.715589in}{0.921181in}}%
\pgfpathlineto{\pgfqpoint{1.715997in}{1.011949in}}%
\pgfpathlineto{\pgfqpoint{1.717426in}{0.904678in}}%
\pgfpathlineto{\pgfqpoint{1.718447in}{1.119219in}}%
\pgfpathlineto{\pgfqpoint{1.718651in}{1.086213in}}%
\pgfpathlineto{\pgfqpoint{1.719672in}{1.020200in}}%
\pgfpathlineto{\pgfqpoint{1.719876in}{1.028452in}}%
\pgfpathlineto{\pgfqpoint{1.721509in}{1.259497in}}%
\pgfpathlineto{\pgfqpoint{1.722938in}{1.424529in}}%
\pgfpathlineto{\pgfqpoint{1.723755in}{1.408026in}}%
\pgfpathlineto{\pgfqpoint{1.723959in}{1.391522in}}%
\pgfpathlineto{\pgfqpoint{1.724163in}{1.432780in}}%
\pgfpathlineto{\pgfqpoint{1.724571in}{1.424529in}}%
\pgfpathlineto{\pgfqpoint{1.724775in}{1.432780in}}%
\pgfpathlineto{\pgfqpoint{1.724980in}{1.391522in}}%
\pgfpathlineto{\pgfqpoint{1.725592in}{1.490542in}}%
\pgfpathlineto{\pgfqpoint{1.725796in}{1.449284in}}%
\pgfpathlineto{\pgfqpoint{1.726000in}{1.449284in}}%
\pgfpathlineto{\pgfqpoint{1.726409in}{1.441032in}}%
\pgfpathlineto{\pgfqpoint{1.726817in}{1.523548in}}%
\pgfpathlineto{\pgfqpoint{1.727633in}{1.647322in}}%
\pgfpathlineto{\pgfqpoint{1.728042in}{1.614316in}}%
\pgfpathlineto{\pgfqpoint{1.728246in}{1.490542in}}%
\pgfpathlineto{\pgfqpoint{1.729062in}{1.655574in}}%
\pgfpathlineto{\pgfqpoint{1.729879in}{1.573058in}}%
\pgfpathlineto{\pgfqpoint{1.730287in}{1.630819in}}%
\pgfpathlineto{\pgfqpoint{1.731716in}{1.746341in}}%
\pgfpathlineto{\pgfqpoint{1.731104in}{1.622567in}}%
\pgfpathlineto{\pgfqpoint{1.732125in}{1.721587in}}%
\pgfpathlineto{\pgfqpoint{1.733145in}{1.342013in}}%
\pgfpathlineto{\pgfqpoint{1.733554in}{1.531800in}}%
\pgfpathlineto{\pgfqpoint{1.733758in}{1.540051in}}%
\pgfpathlineto{\pgfqpoint{1.734370in}{1.457535in}}%
\pgfpathlineto{\pgfqpoint{1.734778in}{1.482290in}}%
\pgfpathlineto{\pgfqpoint{1.736207in}{1.696832in}}%
\pgfpathlineto{\pgfqpoint{1.736616in}{1.614316in}}%
\pgfpathlineto{\pgfqpoint{1.737024in}{1.630819in}}%
\pgfpathlineto{\pgfqpoint{1.738249in}{1.861864in}}%
\pgfpathlineto{\pgfqpoint{1.738453in}{1.837109in}}%
\pgfpathlineto{\pgfqpoint{1.739065in}{1.771096in}}%
\pgfpathlineto{\pgfqpoint{1.739678in}{1.787599in}}%
\pgfpathlineto{\pgfqpoint{1.739882in}{1.795851in}}%
\pgfpathlineto{\pgfqpoint{1.740903in}{1.952632in}}%
\pgfpathlineto{\pgfqpoint{1.741107in}{1.936128in}}%
\pgfpathlineto{\pgfqpoint{1.741311in}{1.738090in}}%
\pgfpathlineto{\pgfqpoint{1.741515in}{2.018644in}}%
\pgfpathlineto{\pgfqpoint{1.741923in}{2.010393in}}%
\pgfpathlineto{\pgfqpoint{1.742128in}{2.051651in}}%
\pgfpathlineto{\pgfqpoint{1.743148in}{2.043399in}}%
\pgfpathlineto{\pgfqpoint{1.743352in}{2.035148in}}%
\pgfpathlineto{\pgfqpoint{1.743965in}{1.861864in}}%
\pgfpathlineto{\pgfqpoint{1.744373in}{1.944380in}}%
\pgfpathlineto{\pgfqpoint{1.745802in}{2.051651in}}%
\pgfpathlineto{\pgfqpoint{1.746006in}{2.051651in}}%
\pgfpathlineto{\pgfqpoint{1.746619in}{2.084657in}}%
\pgfpathlineto{\pgfqpoint{1.746823in}{1.944380in}}%
\pgfpathlineto{\pgfqpoint{1.747639in}{2.117664in}}%
\pgfpathlineto{\pgfqpoint{1.747844in}{2.117664in}}%
\pgfpathlineto{\pgfqpoint{1.748048in}{2.084657in}}%
\pgfpathlineto{\pgfqpoint{1.748456in}{2.150670in}}%
\pgfpathlineto{\pgfqpoint{1.748660in}{2.224934in}}%
\pgfpathlineto{\pgfqpoint{1.749477in}{2.142418in}}%
\pgfpathlineto{\pgfqpoint{1.750293in}{2.084657in}}%
\pgfpathlineto{\pgfqpoint{1.750702in}{2.109412in}}%
\pgfpathlineto{\pgfqpoint{1.751314in}{2.175425in}}%
\pgfpathlineto{\pgfqpoint{1.751722in}{2.117664in}}%
\pgfpathlineto{\pgfqpoint{1.752743in}{2.043399in}}%
\pgfpathlineto{\pgfqpoint{1.752335in}{2.134167in}}%
\pgfpathlineto{\pgfqpoint{1.753355in}{2.092909in}}%
\pgfpathlineto{\pgfqpoint{1.754784in}{2.208431in}}%
\pgfpathlineto{\pgfqpoint{1.754988in}{2.026896in}}%
\pgfpathlineto{\pgfqpoint{1.755805in}{2.282696in}}%
\pgfpathlineto{\pgfqpoint{1.756213in}{2.183676in}}%
\pgfpathlineto{\pgfqpoint{1.756622in}{2.224934in}}%
\pgfpathlineto{\pgfqpoint{1.756826in}{2.051651in}}%
\pgfpathlineto{\pgfqpoint{1.757642in}{2.200180in}}%
\pgfpathlineto{\pgfqpoint{1.757846in}{2.191928in}}%
\pgfpathlineto{\pgfqpoint{1.758459in}{2.299199in}}%
\pgfpathlineto{\pgfqpoint{1.759071in}{2.266192in}}%
\pgfpathlineto{\pgfqpoint{1.759275in}{2.233186in}}%
\pgfpathlineto{\pgfqpoint{1.759888in}{2.290947in}}%
\pgfpathlineto{\pgfqpoint{1.760500in}{2.332205in}}%
\pgfpathlineto{\pgfqpoint{1.760704in}{2.282696in}}%
\pgfpathlineto{\pgfqpoint{1.760909in}{2.299199in}}%
\pgfpathlineto{\pgfqpoint{1.761929in}{2.134167in}}%
\pgfpathlineto{\pgfqpoint{1.762542in}{2.167173in}}%
\pgfpathlineto{\pgfqpoint{1.762746in}{2.224934in}}%
\pgfpathlineto{\pgfqpoint{1.763562in}{2.200180in}}%
\pgfpathlineto{\pgfqpoint{1.763767in}{2.142418in}}%
\pgfpathlineto{\pgfqpoint{1.764175in}{2.233186in}}%
\pgfpathlineto{\pgfqpoint{1.764379in}{2.183676in}}%
\pgfpathlineto{\pgfqpoint{1.765808in}{2.299199in}}%
\pgfpathlineto{\pgfqpoint{1.766625in}{2.233186in}}%
\pgfpathlineto{\pgfqpoint{1.767033in}{2.282696in}}%
\pgfpathlineto{\pgfqpoint{1.767645in}{2.332205in}}%
\pgfpathlineto{\pgfqpoint{1.768054in}{2.290947in}}%
\pgfpathlineto{\pgfqpoint{1.768258in}{2.257941in}}%
\pgfpathlineto{\pgfqpoint{1.768666in}{2.356960in}}%
\pgfpathlineto{\pgfqpoint{1.769074in}{2.299199in}}%
\pgfpathlineto{\pgfqpoint{1.769483in}{2.348709in}}%
\pgfpathlineto{\pgfqpoint{1.769687in}{2.233186in}}%
\pgfpathlineto{\pgfqpoint{1.770503in}{2.332205in}}%
\pgfpathlineto{\pgfqpoint{1.770707in}{2.323954in}}%
\pgfpathlineto{\pgfqpoint{1.770912in}{2.365212in}}%
\pgfpathlineto{\pgfqpoint{1.771320in}{2.315702in}}%
\pgfpathlineto{\pgfqpoint{1.771728in}{2.340457in}}%
\pgfpathlineto{\pgfqpoint{1.772341in}{2.282696in}}%
\pgfpathlineto{\pgfqpoint{1.772749in}{2.348709in}}%
\pgfpathlineto{\pgfqpoint{1.772953in}{2.356960in}}%
\pgfpathlineto{\pgfqpoint{1.774178in}{2.266192in}}%
\pgfpathlineto{\pgfqpoint{1.775199in}{2.373463in}}%
\pgfpathlineto{\pgfqpoint{1.775607in}{2.365212in}}%
\pgfpathlineto{\pgfqpoint{1.776423in}{2.282696in}}%
\pgfpathlineto{\pgfqpoint{1.776628in}{2.315702in}}%
\pgfpathlineto{\pgfqpoint{1.776832in}{2.373463in}}%
\pgfpathlineto{\pgfqpoint{1.777240in}{2.266192in}}%
\pgfpathlineto{\pgfqpoint{1.777648in}{2.315702in}}%
\pgfpathlineto{\pgfqpoint{1.777852in}{2.290947in}}%
\pgfpathlineto{\pgfqpoint{1.778465in}{2.340457in}}%
\pgfpathlineto{\pgfqpoint{1.778669in}{2.348709in}}%
\pgfpathlineto{\pgfqpoint{1.779486in}{2.233186in}}%
\pgfpathlineto{\pgfqpoint{1.779894in}{2.307450in}}%
\pgfpathlineto{\pgfqpoint{1.780098in}{2.340457in}}%
\pgfpathlineto{\pgfqpoint{1.780506in}{2.249689in}}%
\pgfpathlineto{\pgfqpoint{1.780710in}{2.299199in}}%
\pgfpathlineto{\pgfqpoint{1.780915in}{2.282696in}}%
\pgfpathlineto{\pgfqpoint{1.781323in}{2.315702in}}%
\pgfpathlineto{\pgfqpoint{1.781527in}{2.340457in}}%
\pgfpathlineto{\pgfqpoint{1.781935in}{2.274444in}}%
\pgfpathlineto{\pgfqpoint{1.782139in}{2.282696in}}%
\pgfpathlineto{\pgfqpoint{1.783160in}{2.224934in}}%
\pgfpathlineto{\pgfqpoint{1.783364in}{2.257941in}}%
\pgfpathlineto{\pgfqpoint{1.784793in}{2.431225in}}%
\pgfpathlineto{\pgfqpoint{1.784997in}{2.431225in}}%
\pgfpathlineto{\pgfqpoint{1.785202in}{2.497237in}}%
\pgfpathlineto{\pgfqpoint{1.786222in}{2.472483in}}%
\pgfpathlineto{\pgfqpoint{1.786631in}{2.521992in}}%
\pgfpathlineto{\pgfqpoint{1.787855in}{2.389967in}}%
\pgfpathlineto{\pgfqpoint{1.788672in}{2.538495in}}%
\pgfpathlineto{\pgfqpoint{1.788876in}{2.530244in}}%
\pgfpathlineto{\pgfqpoint{1.789080in}{2.290947in}}%
\pgfpathlineto{\pgfqpoint{1.789897in}{2.604508in}}%
\pgfpathlineto{\pgfqpoint{1.790101in}{2.629263in}}%
\pgfpathlineto{\pgfqpoint{1.790305in}{2.521992in}}%
\pgfpathlineto{\pgfqpoint{1.790509in}{2.538495in}}%
\pgfpathlineto{\pgfqpoint{1.791122in}{2.422973in}}%
\pgfpathlineto{\pgfqpoint{1.791734in}{2.464231in}}%
\pgfpathlineto{\pgfqpoint{1.792142in}{2.464231in}}%
\pgfpathlineto{\pgfqpoint{1.792347in}{2.472483in}}%
\pgfpathlineto{\pgfqpoint{1.792551in}{2.414721in}}%
\pgfpathlineto{\pgfqpoint{1.792755in}{2.538495in}}%
\pgfpathlineto{\pgfqpoint{1.793571in}{2.431225in}}%
\pgfpathlineto{\pgfqpoint{1.793776in}{2.431225in}}%
\pgfpathlineto{\pgfqpoint{1.794388in}{2.422973in}}%
\pgfpathlineto{\pgfqpoint{1.794796in}{2.480734in}}%
\pgfpathlineto{\pgfqpoint{1.795409in}{2.439476in}}%
\pgfpathlineto{\pgfqpoint{1.795613in}{2.431225in}}%
\pgfpathlineto{\pgfqpoint{1.795817in}{2.480734in}}%
\pgfpathlineto{\pgfqpoint{1.796634in}{2.472483in}}%
\pgfpathlineto{\pgfqpoint{1.797042in}{2.389967in}}%
\pgfpathlineto{\pgfqpoint{1.797654in}{2.472483in}}%
\pgfpathlineto{\pgfqpoint{1.799492in}{2.563250in}}%
\pgfpathlineto{\pgfqpoint{1.800104in}{2.513741in}}%
\pgfpathlineto{\pgfqpoint{1.800308in}{2.546747in}}%
\pgfpathlineto{\pgfqpoint{1.801125in}{2.629263in}}%
\pgfpathlineto{\pgfqpoint{1.801329in}{2.554999in}}%
\pgfpathlineto{\pgfqpoint{1.801533in}{2.546747in}}%
\pgfpathlineto{\pgfqpoint{1.802554in}{2.406470in}}%
\pgfpathlineto{\pgfqpoint{1.803574in}{2.521992in}}%
\pgfpathlineto{\pgfqpoint{1.803779in}{2.513741in}}%
\pgfpathlineto{\pgfqpoint{1.804187in}{2.472483in}}%
\pgfpathlineto{\pgfqpoint{1.804595in}{2.554999in}}%
\pgfpathlineto{\pgfqpoint{1.804799in}{2.505489in}}%
\pgfpathlineto{\pgfqpoint{1.806024in}{2.571502in}}%
\pgfpathlineto{\pgfqpoint{1.807045in}{2.513741in}}%
\pgfpathlineto{\pgfqpoint{1.807249in}{2.571502in}}%
\pgfpathlineto{\pgfqpoint{1.807861in}{2.472483in}}%
\pgfpathlineto{\pgfqpoint{1.808066in}{2.505489in}}%
\pgfpathlineto{\pgfqpoint{1.809086in}{2.249689in}}%
\pgfpathlineto{\pgfqpoint{1.809699in}{2.340457in}}%
\pgfpathlineto{\pgfqpoint{1.811740in}{2.604508in}}%
\pgfpathlineto{\pgfqpoint{1.812148in}{2.629263in}}%
\pgfpathlineto{\pgfqpoint{1.812353in}{2.596257in}}%
\pgfpathlineto{\pgfqpoint{1.812965in}{2.299199in}}%
\pgfpathlineto{\pgfqpoint{1.813373in}{2.414721in}}%
\pgfpathlineto{\pgfqpoint{1.813577in}{2.654018in}}%
\pgfpathlineto{\pgfqpoint{1.814598in}{2.579753in}}%
\pgfpathlineto{\pgfqpoint{1.815619in}{2.365212in}}%
\pgfpathlineto{\pgfqpoint{1.815823in}{2.596257in}}%
\pgfpathlineto{\pgfqpoint{1.816435in}{2.521992in}}%
\pgfpathlineto{\pgfqpoint{1.816640in}{2.257941in}}%
\pgfpathlineto{\pgfqpoint{1.817456in}{2.621011in}}%
\pgfpathlineto{\pgfqpoint{1.818273in}{2.373463in}}%
\pgfpathlineto{\pgfqpoint{1.819089in}{2.464231in}}%
\pgfpathlineto{\pgfqpoint{1.819498in}{2.480734in}}%
\pgfpathlineto{\pgfqpoint{1.819702in}{2.447728in}}%
\pgfpathlineto{\pgfqpoint{1.820314in}{2.480734in}}%
\pgfpathlineto{\pgfqpoint{1.820927in}{2.340457in}}%
\pgfpathlineto{\pgfqpoint{1.821539in}{2.497237in}}%
\pgfpathlineto{\pgfqpoint{1.822151in}{2.480734in}}%
\pgfpathlineto{\pgfqpoint{1.822356in}{2.538495in}}%
\pgfpathlineto{\pgfqpoint{1.822968in}{2.414721in}}%
\pgfpathlineto{\pgfqpoint{1.824601in}{2.538495in}}%
\pgfpathlineto{\pgfqpoint{1.825009in}{2.216683in}}%
\pgfpathlineto{\pgfqpoint{1.825622in}{2.414721in}}%
\pgfpathlineto{\pgfqpoint{1.826030in}{2.538495in}}%
\pgfpathlineto{\pgfqpoint{1.826438in}{2.406470in}}%
\pgfpathlineto{\pgfqpoint{1.826643in}{2.414721in}}%
\pgfpathlineto{\pgfqpoint{1.827459in}{2.315702in}}%
\pgfpathlineto{\pgfqpoint{1.827051in}{2.472483in}}%
\pgfpathlineto{\pgfqpoint{1.827867in}{2.373463in}}%
\pgfpathlineto{\pgfqpoint{1.828072in}{2.373463in}}%
\pgfpathlineto{\pgfqpoint{1.828480in}{2.398218in}}%
\pgfpathlineto{\pgfqpoint{1.829705in}{2.257941in}}%
\pgfpathlineto{\pgfqpoint{1.829909in}{2.266192in}}%
\pgfpathlineto{\pgfqpoint{1.830113in}{2.307450in}}%
\pgfpathlineto{\pgfqpoint{1.830317in}{2.233186in}}%
\pgfpathlineto{\pgfqpoint{1.830930in}{2.257941in}}%
\pgfpathlineto{\pgfqpoint{1.831950in}{2.208431in}}%
\pgfpathlineto{\pgfqpoint{1.833175in}{2.274444in}}%
\pgfpathlineto{\pgfqpoint{1.833583in}{2.158922in}}%
\pgfpathlineto{\pgfqpoint{1.834196in}{2.233186in}}%
\pgfpathlineto{\pgfqpoint{1.834400in}{2.257941in}}%
\pgfpathlineto{\pgfqpoint{1.834604in}{2.216683in}}%
\pgfpathlineto{\pgfqpoint{1.835217in}{2.224934in}}%
\pgfpathlineto{\pgfqpoint{1.835421in}{2.224934in}}%
\pgfpathlineto{\pgfqpoint{1.835625in}{2.274444in}}%
\pgfpathlineto{\pgfqpoint{1.836033in}{2.208431in}}%
\pgfpathlineto{\pgfqpoint{1.836441in}{2.208431in}}%
\pgfpathlineto{\pgfqpoint{1.836850in}{2.134167in}}%
\pgfpathlineto{\pgfqpoint{1.837054in}{2.208431in}}%
\pgfpathlineto{\pgfqpoint{1.837258in}{2.290947in}}%
\pgfpathlineto{\pgfqpoint{1.837870in}{2.183676in}}%
\pgfpathlineto{\pgfqpoint{1.838074in}{2.183676in}}%
\pgfpathlineto{\pgfqpoint{1.838483in}{2.200180in}}%
\pgfpathlineto{\pgfqpoint{1.839095in}{2.158922in}}%
\pgfpathlineto{\pgfqpoint{1.839299in}{1.985638in}}%
\pgfpathlineto{\pgfqpoint{1.839912in}{2.241438in}}%
\pgfpathlineto{\pgfqpoint{1.840116in}{2.249689in}}%
\pgfpathlineto{\pgfqpoint{1.840320in}{2.241438in}}%
\pgfpathlineto{\pgfqpoint{1.840932in}{1.944380in}}%
\pgfpathlineto{\pgfqpoint{1.841341in}{2.208431in}}%
\pgfpathlineto{\pgfqpoint{1.842566in}{2.266192in}}%
\pgfpathlineto{\pgfqpoint{1.842974in}{2.233186in}}%
\pgfpathlineto{\pgfqpoint{1.843178in}{2.224934in}}%
\pgfpathlineto{\pgfqpoint{1.843382in}{2.257941in}}%
\pgfpathlineto{\pgfqpoint{1.843586in}{2.257941in}}%
\pgfpathlineto{\pgfqpoint{1.843790in}{2.282696in}}%
\pgfpathlineto{\pgfqpoint{1.844199in}{2.241438in}}%
\pgfpathlineto{\pgfqpoint{1.844403in}{2.257941in}}%
\pgfpathlineto{\pgfqpoint{1.844607in}{2.200180in}}%
\pgfpathlineto{\pgfqpoint{1.845015in}{2.274444in}}%
\pgfpathlineto{\pgfqpoint{1.845832in}{2.389967in}}%
\pgfpathlineto{\pgfqpoint{1.846036in}{2.323954in}}%
\pgfpathlineto{\pgfqpoint{1.846240in}{2.315702in}}%
\pgfpathlineto{\pgfqpoint{1.846444in}{2.323954in}}%
\pgfpathlineto{\pgfqpoint{1.847057in}{2.365212in}}%
\pgfpathlineto{\pgfqpoint{1.846853in}{2.299199in}}%
\pgfpathlineto{\pgfqpoint{1.847465in}{2.323954in}}%
\pgfpathlineto{\pgfqpoint{1.847873in}{2.299199in}}%
\pgfpathlineto{\pgfqpoint{1.848486in}{2.340457in}}%
\pgfpathlineto{\pgfqpoint{1.849302in}{2.323954in}}%
\pgfpathlineto{\pgfqpoint{1.849506in}{2.365212in}}%
\pgfpathlineto{\pgfqpoint{1.849915in}{2.340457in}}%
\pgfpathlineto{\pgfqpoint{1.850119in}{2.241438in}}%
\pgfpathlineto{\pgfqpoint{1.850935in}{2.365212in}}%
\pgfpathlineto{\pgfqpoint{1.851140in}{2.389967in}}%
\pgfpathlineto{\pgfqpoint{1.851344in}{2.348709in}}%
\pgfpathlineto{\pgfqpoint{1.851548in}{2.266192in}}%
\pgfpathlineto{\pgfqpoint{1.852364in}{2.315702in}}%
\pgfpathlineto{\pgfqpoint{1.853589in}{2.224934in}}%
\pgfpathlineto{\pgfqpoint{1.853793in}{2.208431in}}%
\pgfpathlineto{\pgfqpoint{1.853998in}{2.233186in}}%
\pgfpathlineto{\pgfqpoint{1.854202in}{2.266192in}}%
\pgfpathlineto{\pgfqpoint{1.854406in}{2.208431in}}%
\pgfpathlineto{\pgfqpoint{1.855018in}{2.224934in}}%
\pgfpathlineto{\pgfqpoint{1.855222in}{2.233186in}}%
\pgfpathlineto{\pgfqpoint{1.855427in}{2.158922in}}%
\pgfpathlineto{\pgfqpoint{1.856243in}{2.183676in}}%
\pgfpathlineto{\pgfqpoint{1.857264in}{2.249689in}}%
\pgfpathlineto{\pgfqpoint{1.857468in}{2.183676in}}%
\pgfpathlineto{\pgfqpoint{1.858080in}{2.323954in}}%
\pgfpathlineto{\pgfqpoint{1.858285in}{2.323954in}}%
\pgfpathlineto{\pgfqpoint{1.858693in}{2.290947in}}%
\pgfpathlineto{\pgfqpoint{1.859305in}{2.307450in}}%
\pgfpathlineto{\pgfqpoint{1.859509in}{2.315702in}}%
\pgfpathlineto{\pgfqpoint{1.860122in}{2.373463in}}%
\pgfpathlineto{\pgfqpoint{1.860734in}{2.233186in}}%
\pgfpathlineto{\pgfqpoint{1.861959in}{2.348709in}}%
\pgfpathlineto{\pgfqpoint{1.862163in}{2.323954in}}%
\pgfpathlineto{\pgfqpoint{1.862980in}{2.059902in}}%
\pgfpathlineto{\pgfqpoint{1.863388in}{2.274444in}}%
\pgfpathlineto{\pgfqpoint{1.864205in}{2.257941in}}%
\pgfpathlineto{\pgfqpoint{1.864409in}{2.315702in}}%
\pgfpathlineto{\pgfqpoint{1.865021in}{2.084657in}}%
\pgfpathlineto{\pgfqpoint{1.865838in}{2.233186in}}%
\pgfpathlineto{\pgfqpoint{1.866042in}{2.274444in}}%
\pgfpathlineto{\pgfqpoint{1.866654in}{2.175425in}}%
\pgfpathlineto{\pgfqpoint{1.868492in}{2.348709in}}%
\pgfpathlineto{\pgfqpoint{1.869512in}{2.307450in}}%
\pgfpathlineto{\pgfqpoint{1.869717in}{2.365212in}}%
\pgfpathlineto{\pgfqpoint{1.870533in}{2.307450in}}%
\pgfpathlineto{\pgfqpoint{1.870737in}{2.274444in}}%
\pgfpathlineto{\pgfqpoint{1.871350in}{2.348709in}}%
\pgfpathlineto{\pgfqpoint{1.871554in}{2.307450in}}%
\pgfpathlineto{\pgfqpoint{1.871962in}{2.257941in}}%
\pgfpathlineto{\pgfqpoint{1.872575in}{2.389967in}}%
\pgfpathlineto{\pgfqpoint{1.873799in}{2.290947in}}%
\pgfpathlineto{\pgfqpoint{1.874208in}{2.282696in}}%
\pgfpathlineto{\pgfqpoint{1.875024in}{2.373463in}}%
\pgfpathlineto{\pgfqpoint{1.875433in}{2.315702in}}%
\pgfpathlineto{\pgfqpoint{1.875841in}{2.323954in}}%
\pgfpathlineto{\pgfqpoint{1.876249in}{2.373463in}}%
\pgfpathlineto{\pgfqpoint{1.877066in}{2.257941in}}%
\pgfpathlineto{\pgfqpoint{1.877678in}{2.315702in}}%
\pgfpathlineto{\pgfqpoint{1.878291in}{2.307450in}}%
\pgfpathlineto{\pgfqpoint{1.878903in}{2.299199in}}%
\pgfpathlineto{\pgfqpoint{1.879107in}{2.340457in}}%
\pgfpathlineto{\pgfqpoint{1.879515in}{2.274444in}}%
\pgfpathlineto{\pgfqpoint{1.879924in}{2.356960in}}%
\pgfpathlineto{\pgfqpoint{1.880128in}{2.373463in}}%
\pgfpathlineto{\pgfqpoint{1.880332in}{2.365212in}}%
\pgfpathlineto{\pgfqpoint{1.881353in}{2.059902in}}%
\pgfpathlineto{\pgfqpoint{1.881557in}{2.101160in}}%
\pgfpathlineto{\pgfqpoint{1.881965in}{2.431225in}}%
\pgfpathlineto{\pgfqpoint{1.882782in}{2.389967in}}%
\pgfpathlineto{\pgfqpoint{1.884007in}{2.084657in}}%
\pgfpathlineto{\pgfqpoint{1.885027in}{2.348709in}}%
\pgfpathlineto{\pgfqpoint{1.885231in}{2.332205in}}%
\pgfpathlineto{\pgfqpoint{1.886048in}{2.414721in}}%
\pgfpathlineto{\pgfqpoint{1.886252in}{2.373463in}}%
\pgfpathlineto{\pgfqpoint{1.887273in}{2.117664in}}%
\pgfpathlineto{\pgfqpoint{1.886865in}{2.422973in}}%
\pgfpathlineto{\pgfqpoint{1.887477in}{2.224934in}}%
\pgfpathlineto{\pgfqpoint{1.888702in}{2.414721in}}%
\pgfpathlineto{\pgfqpoint{1.888906in}{2.406470in}}%
\pgfpathlineto{\pgfqpoint{1.889723in}{2.455979in}}%
\pgfpathlineto{\pgfqpoint{1.890947in}{2.290947in}}%
\pgfpathlineto{\pgfqpoint{1.891968in}{2.464231in}}%
\pgfpathlineto{\pgfqpoint{1.891560in}{2.183676in}}%
\pgfpathlineto{\pgfqpoint{1.892172in}{2.414721in}}%
\pgfpathlineto{\pgfqpoint{1.892376in}{2.150670in}}%
\pgfpathlineto{\pgfqpoint{1.893193in}{2.299199in}}%
\pgfpathlineto{\pgfqpoint{1.893397in}{2.373463in}}%
\pgfpathlineto{\pgfqpoint{1.893805in}{2.134167in}}%
\pgfpathlineto{\pgfqpoint{1.894010in}{2.068154in}}%
\pgfpathlineto{\pgfqpoint{1.894214in}{2.323954in}}%
\pgfpathlineto{\pgfqpoint{1.894418in}{2.323954in}}%
\pgfpathlineto{\pgfqpoint{1.894826in}{2.340457in}}%
\pgfpathlineto{\pgfqpoint{1.895030in}{2.414721in}}%
\pgfpathlineto{\pgfqpoint{1.895439in}{2.365212in}}%
\pgfpathlineto{\pgfqpoint{1.895643in}{2.233186in}}%
\pgfpathlineto{\pgfqpoint{1.896663in}{2.282696in}}%
\pgfpathlineto{\pgfqpoint{1.897480in}{2.464231in}}%
\pgfpathlineto{\pgfqpoint{1.897888in}{2.414721in}}%
\pgfpathlineto{\pgfqpoint{1.898705in}{2.422973in}}%
\pgfpathlineto{\pgfqpoint{1.899317in}{2.348709in}}%
\pgfpathlineto{\pgfqpoint{1.900134in}{2.340457in}}%
\pgfpathlineto{\pgfqpoint{1.900746in}{2.431225in}}%
\pgfpathlineto{\pgfqpoint{1.901563in}{2.299199in}}%
\pgfpathlineto{\pgfqpoint{1.902175in}{2.365212in}}%
\pgfpathlineto{\pgfqpoint{1.902379in}{2.365212in}}%
\pgfpathlineto{\pgfqpoint{1.903604in}{2.084657in}}%
\pgfpathlineto{\pgfqpoint{1.902992in}{2.381715in}}%
\pgfpathlineto{\pgfqpoint{1.903808in}{2.299199in}}%
\pgfpathlineto{\pgfqpoint{1.905033in}{2.381715in}}%
\pgfpathlineto{\pgfqpoint{1.905442in}{2.365212in}}%
\pgfpathlineto{\pgfqpoint{1.906666in}{2.274444in}}%
\pgfpathlineto{\pgfqpoint{1.905850in}{2.389967in}}%
\pgfpathlineto{\pgfqpoint{1.907075in}{2.299199in}}%
\pgfpathlineto{\pgfqpoint{1.908095in}{2.414721in}}%
\pgfpathlineto{\pgfqpoint{1.908300in}{2.348709in}}%
\pgfpathlineto{\pgfqpoint{1.908504in}{2.356960in}}%
\pgfpathlineto{\pgfqpoint{1.908708in}{2.332205in}}%
\pgfpathlineto{\pgfqpoint{1.908912in}{2.332205in}}%
\pgfpathlineto{\pgfqpoint{1.909116in}{2.323954in}}%
\pgfpathlineto{\pgfqpoint{1.909524in}{2.307450in}}%
\pgfpathlineto{\pgfqpoint{1.910749in}{2.439476in}}%
\pgfpathlineto{\pgfqpoint{1.912791in}{2.340457in}}%
\pgfpathlineto{\pgfqpoint{1.912995in}{2.406470in}}%
\pgfpathlineto{\pgfqpoint{1.913811in}{2.323954in}}%
\pgfpathlineto{\pgfqpoint{1.914016in}{2.381715in}}%
\pgfpathlineto{\pgfqpoint{1.914220in}{2.282696in}}%
\pgfpathlineto{\pgfqpoint{1.915036in}{2.406470in}}%
\pgfpathlineto{\pgfqpoint{1.915853in}{2.307450in}}%
\pgfpathlineto{\pgfqpoint{1.915445in}{2.422973in}}%
\pgfpathlineto{\pgfqpoint{1.916057in}{2.356960in}}%
\pgfpathlineto{\pgfqpoint{1.916261in}{2.414721in}}%
\pgfpathlineto{\pgfqpoint{1.917078in}{2.340457in}}%
\pgfpathlineto{\pgfqpoint{1.917282in}{2.332205in}}%
\pgfpathlineto{\pgfqpoint{1.917486in}{2.365212in}}%
\pgfpathlineto{\pgfqpoint{1.917690in}{2.373463in}}%
\pgfpathlineto{\pgfqpoint{1.918711in}{2.241438in}}%
\pgfpathlineto{\pgfqpoint{1.918915in}{2.340457in}}%
\pgfpathlineto{\pgfqpoint{1.919323in}{2.332205in}}%
\pgfpathlineto{\pgfqpoint{1.919527in}{2.348709in}}%
\pgfpathlineto{\pgfqpoint{1.919936in}{2.315702in}}%
\pgfpathlineto{\pgfqpoint{1.920956in}{2.422973in}}%
\pgfpathlineto{\pgfqpoint{1.922589in}{2.356960in}}%
\pgfpathlineto{\pgfqpoint{1.922794in}{2.439476in}}%
\pgfpathlineto{\pgfqpoint{1.923406in}{2.323954in}}%
\pgfpathlineto{\pgfqpoint{1.924223in}{2.142418in}}%
\pgfpathlineto{\pgfqpoint{1.923814in}{2.381715in}}%
\pgfpathlineto{\pgfqpoint{1.924427in}{2.167173in}}%
\pgfpathlineto{\pgfqpoint{1.925447in}{2.389967in}}%
\pgfpathlineto{\pgfqpoint{1.925652in}{2.365212in}}%
\pgfpathlineto{\pgfqpoint{1.926264in}{2.084657in}}%
\pgfpathlineto{\pgfqpoint{1.926672in}{2.381715in}}%
\pgfpathlineto{\pgfqpoint{1.926876in}{2.365212in}}%
\pgfpathlineto{\pgfqpoint{1.927081in}{2.406470in}}%
\pgfpathlineto{\pgfqpoint{1.927489in}{2.464231in}}%
\pgfpathlineto{\pgfqpoint{1.927693in}{2.414721in}}%
\pgfpathlineto{\pgfqpoint{1.928510in}{2.340457in}}%
\pgfpathlineto{\pgfqpoint{1.928714in}{2.389967in}}%
\pgfpathlineto{\pgfqpoint{1.929326in}{2.414721in}}%
\pgfpathlineto{\pgfqpoint{1.930755in}{2.299199in}}%
\pgfpathlineto{\pgfqpoint{1.930959in}{2.274444in}}%
\pgfpathlineto{\pgfqpoint{1.931163in}{2.356960in}}%
\pgfpathlineto{\pgfqpoint{1.931776in}{2.348709in}}%
\pgfpathlineto{\pgfqpoint{1.932388in}{2.373463in}}%
\pgfpathlineto{\pgfqpoint{1.932592in}{2.340457in}}%
\pgfpathlineto{\pgfqpoint{1.932797in}{2.365212in}}%
\pgfpathlineto{\pgfqpoint{1.933001in}{2.092909in}}%
\pgfpathlineto{\pgfqpoint{1.933817in}{2.233186in}}%
\pgfpathlineto{\pgfqpoint{1.935450in}{2.299199in}}%
\pgfpathlineto{\pgfqpoint{1.935655in}{2.290947in}}%
\pgfpathlineto{\pgfqpoint{1.936471in}{2.257941in}}%
\pgfpathlineto{\pgfqpoint{1.936879in}{2.340457in}}%
\pgfpathlineto{\pgfqpoint{1.938104in}{2.274444in}}%
\pgfpathlineto{\pgfqpoint{1.937288in}{2.373463in}}%
\pgfpathlineto{\pgfqpoint{1.938308in}{2.299199in}}%
\pgfpathlineto{\pgfqpoint{1.939125in}{2.340457in}}%
\pgfpathlineto{\pgfqpoint{1.940554in}{2.216683in}}%
\pgfpathlineto{\pgfqpoint{1.941779in}{2.348709in}}%
\pgfpathlineto{\pgfqpoint{1.942187in}{2.356960in}}%
\pgfpathlineto{\pgfqpoint{1.942391in}{2.323954in}}%
\pgfpathlineto{\pgfqpoint{1.942595in}{2.381715in}}%
\pgfpathlineto{\pgfqpoint{1.943004in}{2.373463in}}%
\pgfpathlineto{\pgfqpoint{1.943208in}{2.381715in}}%
\pgfpathlineto{\pgfqpoint{1.943412in}{2.373463in}}%
\pgfpathlineto{\pgfqpoint{1.944024in}{2.282696in}}%
\pgfpathlineto{\pgfqpoint{1.944637in}{2.340457in}}%
\pgfpathlineto{\pgfqpoint{1.944841in}{2.332205in}}%
\pgfpathlineto{\pgfqpoint{1.945045in}{2.340457in}}%
\pgfpathlineto{\pgfqpoint{1.945658in}{2.505489in}}%
\pgfpathlineto{\pgfqpoint{1.946066in}{2.398218in}}%
\pgfpathlineto{\pgfqpoint{1.947087in}{2.315702in}}%
\pgfpathlineto{\pgfqpoint{1.947291in}{2.340457in}}%
\pgfpathlineto{\pgfqpoint{1.947495in}{2.315702in}}%
\pgfpathlineto{\pgfqpoint{1.947699in}{2.365212in}}%
\pgfpathlineto{\pgfqpoint{1.948924in}{2.431225in}}%
\pgfpathlineto{\pgfqpoint{1.949536in}{2.381715in}}%
\pgfpathlineto{\pgfqpoint{1.949740in}{2.332205in}}%
\pgfpathlineto{\pgfqpoint{1.950353in}{2.422973in}}%
\pgfpathlineto{\pgfqpoint{1.950761in}{2.398218in}}%
\pgfpathlineto{\pgfqpoint{1.950965in}{2.406470in}}%
\pgfpathlineto{\pgfqpoint{1.951169in}{2.455979in}}%
\pgfpathlineto{\pgfqpoint{1.951782in}{2.414721in}}%
\pgfpathlineto{\pgfqpoint{1.952394in}{2.447728in}}%
\pgfpathlineto{\pgfqpoint{1.953415in}{2.175425in}}%
\pgfpathlineto{\pgfqpoint{1.954640in}{2.431225in}}%
\pgfpathlineto{\pgfqpoint{1.954844in}{2.414721in}}%
\pgfpathlineto{\pgfqpoint{1.955048in}{2.431225in}}%
\pgfpathlineto{\pgfqpoint{1.955252in}{2.365212in}}%
\pgfpathlineto{\pgfqpoint{1.955661in}{2.373463in}}%
\pgfpathlineto{\pgfqpoint{1.955865in}{2.472483in}}%
\pgfpathlineto{\pgfqpoint{1.956681in}{2.389967in}}%
\pgfpathlineto{\pgfqpoint{1.957498in}{2.472483in}}%
\pgfpathlineto{\pgfqpoint{1.957702in}{2.414721in}}%
\pgfpathlineto{\pgfqpoint{1.958110in}{2.315702in}}%
\pgfpathlineto{\pgfqpoint{1.959131in}{2.373463in}}%
\pgfpathlineto{\pgfqpoint{1.959335in}{2.480734in}}%
\pgfpathlineto{\pgfqpoint{1.960152in}{2.365212in}}%
\pgfpathlineto{\pgfqpoint{1.960356in}{2.224934in}}%
\pgfpathlineto{\pgfqpoint{1.960764in}{2.472483in}}%
\pgfpathlineto{\pgfqpoint{1.961172in}{2.356960in}}%
\pgfpathlineto{\pgfqpoint{1.962397in}{2.546747in}}%
\pgfpathlineto{\pgfqpoint{1.962806in}{2.332205in}}%
\pgfpathlineto{\pgfqpoint{1.963010in}{2.588005in}}%
\pgfpathlineto{\pgfqpoint{1.963418in}{2.546747in}}%
\pgfpathlineto{\pgfqpoint{1.964030in}{2.497237in}}%
\pgfpathlineto{\pgfqpoint{1.964235in}{2.365212in}}%
\pgfpathlineto{\pgfqpoint{1.965051in}{2.505489in}}%
\pgfpathlineto{\pgfqpoint{1.965664in}{2.257941in}}%
\pgfpathlineto{\pgfqpoint{1.966072in}{2.513741in}}%
\pgfpathlineto{\pgfqpoint{1.966480in}{2.464231in}}%
\pgfpathlineto{\pgfqpoint{1.966684in}{2.216683in}}%
\pgfpathlineto{\pgfqpoint{1.967093in}{2.497237in}}%
\pgfpathlineto{\pgfqpoint{1.967501in}{2.455979in}}%
\pgfpathlineto{\pgfqpoint{1.968317in}{2.497237in}}%
\pgfpathlineto{\pgfqpoint{1.968113in}{2.439476in}}%
\pgfpathlineto{\pgfqpoint{1.968726in}{2.488986in}}%
\pgfpathlineto{\pgfqpoint{1.968930in}{2.505489in}}%
\pgfpathlineto{\pgfqpoint{1.969338in}{2.497237in}}%
\pgfpathlineto{\pgfqpoint{1.970155in}{2.356960in}}%
\pgfpathlineto{\pgfqpoint{1.969746in}{2.505489in}}%
\pgfpathlineto{\pgfqpoint{1.970359in}{2.431225in}}%
\pgfpathlineto{\pgfqpoint{1.971380in}{2.530244in}}%
\pgfpathlineto{\pgfqpoint{1.970971in}{2.389967in}}%
\pgfpathlineto{\pgfqpoint{1.971584in}{2.521992in}}%
\pgfpathlineto{\pgfqpoint{1.971992in}{2.431225in}}%
\pgfpathlineto{\pgfqpoint{1.972604in}{2.538495in}}%
\pgfpathlineto{\pgfqpoint{1.973013in}{2.488986in}}%
\pgfpathlineto{\pgfqpoint{1.973217in}{2.530244in}}%
\pgfpathlineto{\pgfqpoint{1.973421in}{2.389967in}}%
\pgfpathlineto{\pgfqpoint{1.974238in}{2.505489in}}%
\pgfpathlineto{\pgfqpoint{1.974850in}{2.472483in}}%
\pgfpathlineto{\pgfqpoint{1.975667in}{2.480734in}}%
\pgfpathlineto{\pgfqpoint{1.975871in}{2.497237in}}%
\pgfpathlineto{\pgfqpoint{1.976279in}{2.480734in}}%
\pgfpathlineto{\pgfqpoint{1.976483in}{2.439476in}}%
\pgfpathlineto{\pgfqpoint{1.976687in}{2.488986in}}%
\pgfpathlineto{\pgfqpoint{1.977300in}{2.472483in}}%
\pgfpathlineto{\pgfqpoint{1.978320in}{2.406470in}}%
\pgfpathlineto{\pgfqpoint{1.978933in}{2.332205in}}%
\pgfpathlineto{\pgfqpoint{1.979341in}{2.389967in}}%
\pgfpathlineto{\pgfqpoint{1.979545in}{2.414721in}}%
\pgfpathlineto{\pgfqpoint{1.979954in}{2.348709in}}%
\pgfpathlineto{\pgfqpoint{1.980158in}{2.365212in}}%
\pgfpathlineto{\pgfqpoint{1.981178in}{2.422973in}}%
\pgfpathlineto{\pgfqpoint{1.981383in}{2.406470in}}%
\pgfpathlineto{\pgfqpoint{1.982812in}{2.117664in}}%
\pgfpathlineto{\pgfqpoint{1.984036in}{2.233186in}}%
\pgfpathlineto{\pgfqpoint{1.985465in}{1.952632in}}%
\pgfpathlineto{\pgfqpoint{1.985874in}{2.059902in}}%
\pgfpathlineto{\pgfqpoint{1.986486in}{1.960883in}}%
\pgfpathlineto{\pgfqpoint{1.987507in}{2.018644in}}%
\pgfpathlineto{\pgfqpoint{1.988732in}{1.870115in}}%
\pgfpathlineto{\pgfqpoint{1.988936in}{1.886619in}}%
\pgfpathlineto{\pgfqpoint{1.989140in}{1.886619in}}%
\pgfpathlineto{\pgfqpoint{1.989548in}{1.680329in}}%
\pgfpathlineto{\pgfqpoint{1.989957in}{1.985638in}}%
\pgfpathlineto{\pgfqpoint{1.990365in}{2.035148in}}%
\pgfpathlineto{\pgfqpoint{1.990977in}{1.969135in}}%
\pgfpathlineto{\pgfqpoint{1.992202in}{2.150670in}}%
\pgfpathlineto{\pgfqpoint{1.992610in}{2.076406in}}%
\pgfpathlineto{\pgfqpoint{1.992815in}{2.002141in}}%
\pgfpathlineto{\pgfqpoint{1.993427in}{2.142418in}}%
\pgfpathlineto{\pgfqpoint{1.994652in}{2.076406in}}%
\pgfpathlineto{\pgfqpoint{1.995060in}{2.101160in}}%
\pgfpathlineto{\pgfqpoint{1.996081in}{2.142418in}}%
\pgfpathlineto{\pgfqpoint{1.997510in}{2.002141in}}%
\pgfpathlineto{\pgfqpoint{1.998122in}{2.051651in}}%
\pgfpathlineto{\pgfqpoint{1.998326in}{1.969135in}}%
\pgfpathlineto{\pgfqpoint{1.998531in}{1.886619in}}%
\pgfpathlineto{\pgfqpoint{1.999143in}{2.076406in}}%
\pgfpathlineto{\pgfqpoint{1.999551in}{2.010393in}}%
\pgfpathlineto{\pgfqpoint{1.999960in}{2.150670in}}%
\pgfpathlineto{\pgfqpoint{2.000980in}{1.985638in}}%
\pgfpathlineto{\pgfqpoint{2.001389in}{1.993890in}}%
\pgfpathlineto{\pgfqpoint{2.002818in}{1.845361in}}%
\pgfpathlineto{\pgfqpoint{2.003838in}{1.927877in}}%
\pgfpathlineto{\pgfqpoint{2.004042in}{1.870115in}}%
\pgfpathlineto{\pgfqpoint{2.004247in}{1.639071in}}%
\pgfpathlineto{\pgfqpoint{2.005063in}{1.960883in}}%
\pgfpathlineto{\pgfqpoint{2.005267in}{1.919625in}}%
\pgfpathlineto{\pgfqpoint{2.005880in}{1.993890in}}%
\pgfpathlineto{\pgfqpoint{2.006084in}{1.960883in}}%
\pgfpathlineto{\pgfqpoint{2.006288in}{1.952632in}}%
\pgfpathlineto{\pgfqpoint{2.006696in}{1.977386in}}%
\pgfpathlineto{\pgfqpoint{2.007309in}{2.051651in}}%
\pgfpathlineto{\pgfqpoint{2.007513in}{2.002141in}}%
\pgfpathlineto{\pgfqpoint{2.008738in}{1.762845in}}%
\pgfpathlineto{\pgfqpoint{2.009758in}{1.993890in}}%
\pgfpathlineto{\pgfqpoint{2.009962in}{1.944380in}}%
\pgfpathlineto{\pgfqpoint{2.010371in}{2.002141in}}%
\pgfpathlineto{\pgfqpoint{2.010779in}{1.911373in}}%
\pgfpathlineto{\pgfqpoint{2.010983in}{1.944380in}}%
\pgfpathlineto{\pgfqpoint{2.011391in}{1.886619in}}%
\pgfpathlineto{\pgfqpoint{2.012004in}{1.927877in}}%
\pgfpathlineto{\pgfqpoint{2.012616in}{1.911373in}}%
\pgfpathlineto{\pgfqpoint{2.013637in}{2.018644in}}%
\pgfpathlineto{\pgfqpoint{2.013841in}{1.911373in}}%
\pgfpathlineto{\pgfqpoint{2.014658in}{1.985638in}}%
\pgfpathlineto{\pgfqpoint{2.015883in}{1.919625in}}%
\pgfpathlineto{\pgfqpoint{2.015270in}{2.002141in}}%
\pgfpathlineto{\pgfqpoint{2.016087in}{1.944380in}}%
\pgfpathlineto{\pgfqpoint{2.016291in}{1.952632in}}%
\pgfpathlineto{\pgfqpoint{2.016495in}{1.911373in}}%
\pgfpathlineto{\pgfqpoint{2.016903in}{1.985638in}}%
\pgfpathlineto{\pgfqpoint{2.017312in}{1.952632in}}%
\pgfpathlineto{\pgfqpoint{2.018332in}{1.903122in}}%
\pgfpathlineto{\pgfqpoint{2.019353in}{1.977386in}}%
\pgfpathlineto{\pgfqpoint{2.019557in}{1.952632in}}%
\pgfpathlineto{\pgfqpoint{2.020374in}{1.894870in}}%
\pgfpathlineto{\pgfqpoint{2.019965in}{2.002141in}}%
\pgfpathlineto{\pgfqpoint{2.020578in}{1.936128in}}%
\pgfpathlineto{\pgfqpoint{2.021394in}{2.051651in}}%
\pgfpathlineto{\pgfqpoint{2.021599in}{1.969135in}}%
\pgfpathlineto{\pgfqpoint{2.022415in}{1.870115in}}%
\pgfpathlineto{\pgfqpoint{2.022823in}{1.919625in}}%
\pgfpathlineto{\pgfqpoint{2.023640in}{2.043399in}}%
\pgfpathlineto{\pgfqpoint{2.024252in}{2.002141in}}%
\pgfpathlineto{\pgfqpoint{2.025681in}{1.804103in}}%
\pgfpathlineto{\pgfqpoint{2.026702in}{1.886619in}}%
\pgfpathlineto{\pgfqpoint{2.026906in}{1.837109in}}%
\pgfpathlineto{\pgfqpoint{2.027519in}{1.746341in}}%
\pgfpathlineto{\pgfqpoint{2.028335in}{1.762845in}}%
\pgfpathlineto{\pgfqpoint{2.030581in}{1.993890in}}%
\pgfpathlineto{\pgfqpoint{2.030785in}{1.985638in}}%
\pgfpathlineto{\pgfqpoint{2.031602in}{2.002141in}}%
\pgfpathlineto{\pgfqpoint{2.031193in}{1.977386in}}%
\pgfpathlineto{\pgfqpoint{2.031806in}{1.985638in}}%
\pgfpathlineto{\pgfqpoint{2.032010in}{1.985638in}}%
\pgfpathlineto{\pgfqpoint{2.032214in}{1.977386in}}%
\pgfpathlineto{\pgfqpoint{2.032418in}{2.010393in}}%
\pgfpathlineto{\pgfqpoint{2.032622in}{2.010393in}}%
\pgfpathlineto{\pgfqpoint{2.033643in}{1.911373in}}%
\pgfpathlineto{\pgfqpoint{2.033847in}{1.919625in}}%
\pgfpathlineto{\pgfqpoint{2.034664in}{1.993890in}}%
\pgfpathlineto{\pgfqpoint{2.035072in}{1.977386in}}%
\pgfpathlineto{\pgfqpoint{2.036297in}{1.894870in}}%
\pgfpathlineto{\pgfqpoint{2.036501in}{1.919625in}}%
\pgfpathlineto{\pgfqpoint{2.036705in}{1.993890in}}%
\pgfpathlineto{\pgfqpoint{2.037318in}{1.837109in}}%
\pgfpathlineto{\pgfqpoint{2.037522in}{1.960883in}}%
\pgfpathlineto{\pgfqpoint{2.037930in}{1.639071in}}%
\pgfpathlineto{\pgfqpoint{2.038747in}{1.845361in}}%
\pgfpathlineto{\pgfqpoint{2.039767in}{1.688580in}}%
\pgfpathlineto{\pgfqpoint{2.040176in}{1.696832in}}%
\pgfpathlineto{\pgfqpoint{2.040584in}{1.762845in}}%
\pgfpathlineto{\pgfqpoint{2.041400in}{1.754593in}}%
\pgfpathlineto{\pgfqpoint{2.041605in}{1.713335in}}%
\pgfpathlineto{\pgfqpoint{2.042013in}{1.820606in}}%
\pgfpathlineto{\pgfqpoint{2.042421in}{1.762845in}}%
\pgfpathlineto{\pgfqpoint{2.042829in}{1.738090in}}%
\pgfpathlineto{\pgfqpoint{2.043238in}{1.820606in}}%
\pgfpathlineto{\pgfqpoint{2.044054in}{1.696832in}}%
\pgfpathlineto{\pgfqpoint{2.044463in}{1.754593in}}%
\pgfpathlineto{\pgfqpoint{2.045279in}{1.729838in}}%
\pgfpathlineto{\pgfqpoint{2.045483in}{1.738090in}}%
\pgfpathlineto{\pgfqpoint{2.046504in}{1.795851in}}%
\pgfpathlineto{\pgfqpoint{2.046708in}{1.614316in}}%
\pgfpathlineto{\pgfqpoint{2.047525in}{1.861864in}}%
\pgfpathlineto{\pgfqpoint{2.047729in}{1.853612in}}%
\pgfpathlineto{\pgfqpoint{2.047933in}{1.870115in}}%
\pgfpathlineto{\pgfqpoint{2.048341in}{1.944380in}}%
\pgfpathlineto{\pgfqpoint{2.048750in}{1.828857in}}%
\pgfpathlineto{\pgfqpoint{2.048954in}{1.911373in}}%
\pgfpathlineto{\pgfqpoint{2.050383in}{1.820606in}}%
\pgfpathlineto{\pgfqpoint{2.050791in}{1.919625in}}%
\pgfpathlineto{\pgfqpoint{2.050995in}{1.795851in}}%
\pgfpathlineto{\pgfqpoint{2.051199in}{1.861864in}}%
\pgfpathlineto{\pgfqpoint{2.052832in}{1.581309in}}%
\pgfpathlineto{\pgfqpoint{2.053241in}{1.787599in}}%
\pgfpathlineto{\pgfqpoint{2.054057in}{1.746341in}}%
\pgfpathlineto{\pgfqpoint{2.054874in}{1.853612in}}%
\pgfpathlineto{\pgfqpoint{2.055486in}{1.845361in}}%
\pgfpathlineto{\pgfqpoint{2.056507in}{1.738090in}}%
\pgfpathlineto{\pgfqpoint{2.056711in}{1.754593in}}%
\pgfpathlineto{\pgfqpoint{2.056915in}{1.746341in}}%
\pgfpathlineto{\pgfqpoint{2.057324in}{1.828857in}}%
\pgfpathlineto{\pgfqpoint{2.057936in}{1.729838in}}%
\pgfpathlineto{\pgfqpoint{2.058344in}{1.837109in}}%
\pgfpathlineto{\pgfqpoint{2.058753in}{1.828857in}}%
\pgfpathlineto{\pgfqpoint{2.058957in}{1.647322in}}%
\pgfpathlineto{\pgfqpoint{2.059161in}{1.870115in}}%
\pgfpathlineto{\pgfqpoint{2.059773in}{1.771096in}}%
\pgfpathlineto{\pgfqpoint{2.060182in}{1.894870in}}%
\pgfpathlineto{\pgfqpoint{2.060794in}{1.787599in}}%
\pgfpathlineto{\pgfqpoint{2.060998in}{1.771096in}}%
\pgfpathlineto{\pgfqpoint{2.061202in}{1.779348in}}%
\pgfpathlineto{\pgfqpoint{2.061406in}{1.837109in}}%
\pgfpathlineto{\pgfqpoint{2.061611in}{1.771096in}}%
\pgfpathlineto{\pgfqpoint{2.062223in}{1.795851in}}%
\pgfpathlineto{\pgfqpoint{2.062427in}{1.771096in}}%
\pgfpathlineto{\pgfqpoint{2.062631in}{1.861864in}}%
\pgfpathlineto{\pgfqpoint{2.062835in}{1.903122in}}%
\pgfpathlineto{\pgfqpoint{2.063244in}{1.845361in}}%
\pgfpathlineto{\pgfqpoint{2.063448in}{1.853612in}}%
\pgfpathlineto{\pgfqpoint{2.063652in}{1.622567in}}%
\pgfpathlineto{\pgfqpoint{2.064264in}{1.886619in}}%
\pgfpathlineto{\pgfqpoint{2.064469in}{1.878367in}}%
\pgfpathlineto{\pgfqpoint{2.064673in}{1.903122in}}%
\pgfpathlineto{\pgfqpoint{2.064877in}{1.828857in}}%
\pgfpathlineto{\pgfqpoint{2.065285in}{1.861864in}}%
\pgfpathlineto{\pgfqpoint{2.066510in}{1.696832in}}%
\pgfpathlineto{\pgfqpoint{2.066918in}{1.787599in}}%
\pgfpathlineto{\pgfqpoint{2.067122in}{1.886619in}}%
\pgfpathlineto{\pgfqpoint{2.067327in}{1.672077in}}%
\pgfpathlineto{\pgfqpoint{2.067939in}{1.746341in}}%
\pgfpathlineto{\pgfqpoint{2.068551in}{1.713335in}}%
\pgfpathlineto{\pgfqpoint{2.068347in}{1.762845in}}%
\pgfpathlineto{\pgfqpoint{2.068960in}{1.746341in}}%
\pgfpathlineto{\pgfqpoint{2.069368in}{1.779348in}}%
\pgfpathlineto{\pgfqpoint{2.069980in}{1.762845in}}%
\pgfpathlineto{\pgfqpoint{2.070185in}{1.705083in}}%
\pgfpathlineto{\pgfqpoint{2.070389in}{1.787599in}}%
\pgfpathlineto{\pgfqpoint{2.071205in}{1.721587in}}%
\pgfpathlineto{\pgfqpoint{2.072022in}{1.837109in}}%
\pgfpathlineto{\pgfqpoint{2.073247in}{1.606064in}}%
\pgfpathlineto{\pgfqpoint{2.074676in}{1.762845in}}%
\pgfpathlineto{\pgfqpoint{2.075696in}{1.663825in}}%
\pgfpathlineto{\pgfqpoint{2.075901in}{1.696832in}}%
\pgfpathlineto{\pgfqpoint{2.076309in}{1.729838in}}%
\pgfpathlineto{\pgfqpoint{2.076717in}{1.672077in}}%
\pgfpathlineto{\pgfqpoint{2.076921in}{1.647322in}}%
\pgfpathlineto{\pgfqpoint{2.077738in}{1.672077in}}%
\pgfpathlineto{\pgfqpoint{2.079167in}{1.795851in}}%
\pgfpathlineto{\pgfqpoint{2.079371in}{1.779348in}}%
\pgfpathlineto{\pgfqpoint{2.079575in}{1.746341in}}%
\pgfpathlineto{\pgfqpoint{2.079779in}{1.804103in}}%
\pgfpathlineto{\pgfqpoint{2.080188in}{1.762845in}}%
\pgfpathlineto{\pgfqpoint{2.081004in}{1.911373in}}%
\pgfpathlineto{\pgfqpoint{2.081208in}{1.870115in}}%
\pgfpathlineto{\pgfqpoint{2.082433in}{1.729838in}}%
\pgfpathlineto{\pgfqpoint{2.083454in}{1.812354in}}%
\pgfpathlineto{\pgfqpoint{2.083658in}{1.771096in}}%
\pgfpathlineto{\pgfqpoint{2.084270in}{1.828857in}}%
\pgfpathlineto{\pgfqpoint{2.084679in}{1.771096in}}%
\pgfpathlineto{\pgfqpoint{2.085087in}{1.672077in}}%
\pgfpathlineto{\pgfqpoint{2.085495in}{1.779348in}}%
\pgfpathlineto{\pgfqpoint{2.085699in}{1.795851in}}%
\pgfpathlineto{\pgfqpoint{2.085904in}{1.771096in}}%
\pgfpathlineto{\pgfqpoint{2.086516in}{1.474038in}}%
\pgfpathlineto{\pgfqpoint{2.086924in}{1.746341in}}%
\pgfpathlineto{\pgfqpoint{2.087128in}{1.762845in}}%
\pgfpathlineto{\pgfqpoint{2.088353in}{1.597813in}}%
\pgfpathlineto{\pgfqpoint{2.089374in}{1.672077in}}%
\pgfpathlineto{\pgfqpoint{2.089986in}{1.721587in}}%
\pgfpathlineto{\pgfqpoint{2.090191in}{1.663825in}}%
\pgfpathlineto{\pgfqpoint{2.091007in}{1.531800in}}%
\pgfpathlineto{\pgfqpoint{2.091211in}{1.663825in}}%
\pgfpathlineto{\pgfqpoint{2.091620in}{1.639071in}}%
\pgfpathlineto{\pgfqpoint{2.091824in}{1.738090in}}%
\pgfpathlineto{\pgfqpoint{2.092640in}{1.630819in}}%
\pgfpathlineto{\pgfqpoint{2.093457in}{1.705083in}}%
\pgfpathlineto{\pgfqpoint{2.093661in}{1.655574in}}%
\pgfpathlineto{\pgfqpoint{2.094273in}{1.540051in}}%
\pgfpathlineto{\pgfqpoint{2.095090in}{1.573058in}}%
\pgfpathlineto{\pgfqpoint{2.095906in}{1.655574in}}%
\pgfpathlineto{\pgfqpoint{2.096315in}{1.606064in}}%
\pgfpathlineto{\pgfqpoint{2.096723in}{1.614316in}}%
\pgfpathlineto{\pgfqpoint{2.097335in}{1.564806in}}%
\pgfpathlineto{\pgfqpoint{2.097540in}{1.589561in}}%
\pgfpathlineto{\pgfqpoint{2.097744in}{1.515296in}}%
\pgfpathlineto{\pgfqpoint{2.098356in}{1.573058in}}%
\pgfpathlineto{\pgfqpoint{2.098560in}{1.573058in}}%
\pgfpathlineto{\pgfqpoint{2.098764in}{1.564806in}}%
\pgfpathlineto{\pgfqpoint{2.098969in}{1.597813in}}%
\pgfpathlineto{\pgfqpoint{2.099173in}{1.581309in}}%
\pgfpathlineto{\pgfqpoint{2.100806in}{1.705083in}}%
\pgfpathlineto{\pgfqpoint{2.101010in}{1.622567in}}%
\pgfpathlineto{\pgfqpoint{2.101827in}{1.738090in}}%
\pgfpathlineto{\pgfqpoint{2.103256in}{1.449284in}}%
\pgfpathlineto{\pgfqpoint{2.103460in}{1.564806in}}%
\pgfpathlineto{\pgfqpoint{2.104685in}{1.680329in}}%
\pgfpathlineto{\pgfqpoint{2.105909in}{1.597813in}}%
\pgfpathlineto{\pgfqpoint{2.106114in}{1.647322in}}%
\pgfpathlineto{\pgfqpoint{2.106318in}{1.498793in}}%
\pgfpathlineto{\pgfqpoint{2.107134in}{1.630819in}}%
\pgfpathlineto{\pgfqpoint{2.107747in}{1.375019in}}%
\pgfpathlineto{\pgfqpoint{2.107543in}{1.655574in}}%
\pgfpathlineto{\pgfqpoint{2.108155in}{1.589561in}}%
\pgfpathlineto{\pgfqpoint{2.108359in}{1.614316in}}%
\pgfpathlineto{\pgfqpoint{2.108767in}{1.556554in}}%
\pgfpathlineto{\pgfqpoint{2.109176in}{1.597813in}}%
\pgfpathlineto{\pgfqpoint{2.109788in}{1.556554in}}%
\pgfpathlineto{\pgfqpoint{2.109992in}{1.639071in}}%
\pgfpathlineto{\pgfqpoint{2.110196in}{1.606064in}}%
\pgfpathlineto{\pgfqpoint{2.111013in}{1.556554in}}%
\pgfpathlineto{\pgfqpoint{2.110605in}{1.622567in}}%
\pgfpathlineto{\pgfqpoint{2.111217in}{1.581309in}}%
\pgfpathlineto{\pgfqpoint{2.111421in}{1.639071in}}%
\pgfpathlineto{\pgfqpoint{2.111830in}{1.556554in}}%
\pgfpathlineto{\pgfqpoint{2.112238in}{1.606064in}}%
\pgfpathlineto{\pgfqpoint{2.112442in}{1.597813in}}%
\pgfpathlineto{\pgfqpoint{2.112850in}{1.614316in}}%
\pgfpathlineto{\pgfqpoint{2.113463in}{1.639071in}}%
\pgfpathlineto{\pgfqpoint{2.113667in}{1.589561in}}%
\pgfpathlineto{\pgfqpoint{2.114075in}{1.630819in}}%
\pgfpathlineto{\pgfqpoint{2.114279in}{1.663825in}}%
\pgfpathlineto{\pgfqpoint{2.114688in}{1.581309in}}%
\pgfpathlineto{\pgfqpoint{2.114892in}{1.589561in}}%
\pgfpathlineto{\pgfqpoint{2.115912in}{1.515296in}}%
\pgfpathlineto{\pgfqpoint{2.115504in}{1.614316in}}%
\pgfpathlineto{\pgfqpoint{2.116117in}{1.548303in}}%
\pgfpathlineto{\pgfqpoint{2.116525in}{1.515296in}}%
\pgfpathlineto{\pgfqpoint{2.116729in}{1.523548in}}%
\pgfpathlineto{\pgfqpoint{2.117750in}{1.515296in}}%
\pgfpathlineto{\pgfqpoint{2.117954in}{1.589561in}}%
\pgfpathlineto{\pgfqpoint{2.118566in}{1.490542in}}%
\pgfpathlineto{\pgfqpoint{2.118975in}{1.581309in}}%
\pgfpathlineto{\pgfqpoint{2.119383in}{1.573058in}}%
\pgfpathlineto{\pgfqpoint{2.119587in}{1.622567in}}%
\pgfpathlineto{\pgfqpoint{2.119995in}{1.515296in}}%
\pgfpathlineto{\pgfqpoint{2.120608in}{1.432780in}}%
\pgfpathlineto{\pgfqpoint{2.121016in}{1.490542in}}%
\pgfpathlineto{\pgfqpoint{2.121833in}{1.548303in}}%
\pgfpathlineto{\pgfqpoint{2.121424in}{1.457535in}}%
\pgfpathlineto{\pgfqpoint{2.122241in}{1.507045in}}%
\pgfpathlineto{\pgfqpoint{2.122445in}{1.498793in}}%
\pgfpathlineto{\pgfqpoint{2.122853in}{1.556554in}}%
\pgfpathlineto{\pgfqpoint{2.123262in}{1.449284in}}%
\pgfpathlineto{\pgfqpoint{2.123466in}{1.490542in}}%
\pgfpathlineto{\pgfqpoint{2.123670in}{1.474038in}}%
\pgfpathlineto{\pgfqpoint{2.123874in}{1.507045in}}%
\pgfpathlineto{\pgfqpoint{2.125303in}{1.655574in}}%
\pgfpathlineto{\pgfqpoint{2.126732in}{1.498793in}}%
\pgfpathlineto{\pgfqpoint{2.128161in}{1.663825in}}%
\pgfpathlineto{\pgfqpoint{2.128773in}{1.655574in}}%
\pgfpathlineto{\pgfqpoint{2.129590in}{1.672077in}}%
\pgfpathlineto{\pgfqpoint{2.129794in}{1.630819in}}%
\pgfpathlineto{\pgfqpoint{2.130407in}{1.738090in}}%
\pgfpathlineto{\pgfqpoint{2.130815in}{1.655574in}}%
\pgfpathlineto{\pgfqpoint{2.131019in}{1.647322in}}%
\pgfpathlineto{\pgfqpoint{2.131427in}{1.721587in}}%
\pgfpathlineto{\pgfqpoint{2.131631in}{1.639071in}}%
\pgfpathlineto{\pgfqpoint{2.132040in}{1.696832in}}%
\pgfpathlineto{\pgfqpoint{2.132448in}{1.531800in}}%
\pgfpathlineto{\pgfqpoint{2.133469in}{1.556554in}}%
\pgfpathlineto{\pgfqpoint{2.133673in}{1.630819in}}%
\pgfpathlineto{\pgfqpoint{2.134489in}{1.556554in}}%
\pgfpathlineto{\pgfqpoint{2.135102in}{1.581309in}}%
\pgfpathlineto{\pgfqpoint{2.134898in}{1.548303in}}%
\pgfpathlineto{\pgfqpoint{2.135306in}{1.556554in}}%
\pgfpathlineto{\pgfqpoint{2.135510in}{1.515296in}}%
\pgfpathlineto{\pgfqpoint{2.135918in}{1.589561in}}%
\pgfpathlineto{\pgfqpoint{2.136123in}{1.622567in}}%
\pgfpathlineto{\pgfqpoint{2.136327in}{1.507045in}}%
\pgfpathlineto{\pgfqpoint{2.136531in}{1.523548in}}%
\pgfpathlineto{\pgfqpoint{2.136735in}{1.523548in}}%
\pgfpathlineto{\pgfqpoint{2.136939in}{1.515296in}}%
\pgfpathlineto{\pgfqpoint{2.137143in}{1.531800in}}%
\pgfpathlineto{\pgfqpoint{2.137347in}{1.531800in}}%
\pgfpathlineto{\pgfqpoint{2.137756in}{1.564806in}}%
\pgfpathlineto{\pgfqpoint{2.137960in}{1.531800in}}%
\pgfpathlineto{\pgfqpoint{2.138572in}{1.474038in}}%
\pgfpathlineto{\pgfqpoint{2.138776in}{1.548303in}}%
\pgfpathlineto{\pgfqpoint{2.138981in}{1.622567in}}%
\pgfpathlineto{\pgfqpoint{2.139797in}{1.531800in}}%
\pgfpathlineto{\pgfqpoint{2.140614in}{1.548303in}}%
\pgfpathlineto{\pgfqpoint{2.141226in}{1.424529in}}%
\pgfpathlineto{\pgfqpoint{2.142451in}{1.507045in}}%
\pgfpathlineto{\pgfqpoint{2.142655in}{1.441032in}}%
\pgfpathlineto{\pgfqpoint{2.143063in}{1.548303in}}%
\pgfpathlineto{\pgfqpoint{2.143472in}{1.498793in}}%
\pgfpathlineto{\pgfqpoint{2.143676in}{1.548303in}}%
\pgfpathlineto{\pgfqpoint{2.143880in}{1.325510in}}%
\pgfpathlineto{\pgfqpoint{2.144084in}{1.573058in}}%
\pgfpathlineto{\pgfqpoint{2.144697in}{1.490542in}}%
\pgfpathlineto{\pgfqpoint{2.144901in}{1.482290in}}%
\pgfpathlineto{\pgfqpoint{2.145105in}{1.531800in}}%
\pgfpathlineto{\pgfqpoint{2.145513in}{1.465787in}}%
\pgfpathlineto{\pgfqpoint{2.145717in}{1.490542in}}%
\pgfpathlineto{\pgfqpoint{2.146126in}{1.358516in}}%
\pgfpathlineto{\pgfqpoint{2.146738in}{1.391522in}}%
\pgfpathlineto{\pgfqpoint{2.147759in}{1.523548in}}%
\pgfpathlineto{\pgfqpoint{2.147146in}{1.366768in}}%
\pgfpathlineto{\pgfqpoint{2.147963in}{1.515296in}}%
\pgfpathlineto{\pgfqpoint{2.148371in}{1.168729in}}%
\pgfpathlineto{\pgfqpoint{2.148984in}{1.218239in}}%
\pgfpathlineto{\pgfqpoint{2.150004in}{1.482290in}}%
\pgfpathlineto{\pgfqpoint{2.150208in}{1.416277in}}%
\pgfpathlineto{\pgfqpoint{2.151229in}{1.515296in}}%
\pgfpathlineto{\pgfqpoint{2.151637in}{1.490542in}}%
\pgfpathlineto{\pgfqpoint{2.152454in}{1.457535in}}%
\pgfpathlineto{\pgfqpoint{2.152658in}{1.515296in}}%
\pgfpathlineto{\pgfqpoint{2.153475in}{1.498793in}}%
\pgfpathlineto{\pgfqpoint{2.154291in}{1.408026in}}%
\pgfpathlineto{\pgfqpoint{2.154495in}{1.474038in}}%
\pgfpathlineto{\pgfqpoint{2.154904in}{1.573058in}}%
\pgfpathlineto{\pgfqpoint{2.155516in}{1.474038in}}%
\pgfpathlineto{\pgfqpoint{2.155720in}{1.242994in}}%
\pgfpathlineto{\pgfqpoint{2.156333in}{1.573058in}}%
\pgfpathlineto{\pgfqpoint{2.156537in}{1.531800in}}%
\pgfpathlineto{\pgfqpoint{2.156741in}{1.531800in}}%
\pgfpathlineto{\pgfqpoint{2.156945in}{1.515296in}}%
\pgfpathlineto{\pgfqpoint{2.157149in}{1.564806in}}%
\pgfpathlineto{\pgfqpoint{2.157353in}{1.540051in}}%
\pgfpathlineto{\pgfqpoint{2.158374in}{1.655574in}}%
\pgfpathlineto{\pgfqpoint{2.158782in}{1.614316in}}%
\pgfpathlineto{\pgfqpoint{2.159803in}{1.531800in}}%
\pgfpathlineto{\pgfqpoint{2.160007in}{1.556554in}}%
\pgfpathlineto{\pgfqpoint{2.160211in}{1.564806in}}%
\pgfpathlineto{\pgfqpoint{2.161232in}{1.779348in}}%
\pgfpathlineto{\pgfqpoint{2.161436in}{1.705083in}}%
\pgfpathlineto{\pgfqpoint{2.161640in}{1.449284in}}%
\pgfpathlineto{\pgfqpoint{2.162457in}{1.597813in}}%
\pgfpathlineto{\pgfqpoint{2.163886in}{1.482290in}}%
\pgfpathlineto{\pgfqpoint{2.164090in}{1.490542in}}%
\pgfpathlineto{\pgfqpoint{2.164294in}{1.556554in}}%
\pgfpathlineto{\pgfqpoint{2.164498in}{1.474038in}}%
\pgfpathlineto{\pgfqpoint{2.165111in}{1.531800in}}%
\pgfpathlineto{\pgfqpoint{2.165927in}{1.432780in}}%
\pgfpathlineto{\pgfqpoint{2.166336in}{1.474038in}}%
\pgfpathlineto{\pgfqpoint{2.166744in}{1.515296in}}%
\pgfpathlineto{\pgfqpoint{2.166948in}{1.465787in}}%
\pgfpathlineto{\pgfqpoint{2.167356in}{1.573058in}}%
\pgfpathlineto{\pgfqpoint{2.167969in}{1.482290in}}%
\pgfpathlineto{\pgfqpoint{2.168785in}{1.383271in}}%
\pgfpathlineto{\pgfqpoint{2.169194in}{1.416277in}}%
\pgfpathlineto{\pgfqpoint{2.169806in}{1.465787in}}%
\pgfpathlineto{\pgfqpoint{2.171031in}{1.325510in}}%
\pgfpathlineto{\pgfqpoint{2.172052in}{1.399774in}}%
\pgfpathlineto{\pgfqpoint{2.171439in}{1.292503in}}%
\pgfpathlineto{\pgfqpoint{2.172460in}{1.391522in}}%
\pgfpathlineto{\pgfqpoint{2.173072in}{1.176981in}}%
\pgfpathlineto{\pgfqpoint{2.173277in}{1.399774in}}%
\pgfpathlineto{\pgfqpoint{2.174297in}{1.548303in}}%
\pgfpathlineto{\pgfqpoint{2.174706in}{1.507045in}}%
\pgfpathlineto{\pgfqpoint{2.174910in}{1.292503in}}%
\pgfpathlineto{\pgfqpoint{2.175522in}{1.573058in}}%
\pgfpathlineto{\pgfqpoint{2.175726in}{1.548303in}}%
\pgfpathlineto{\pgfqpoint{2.176747in}{1.482290in}}%
\pgfpathlineto{\pgfqpoint{2.176951in}{1.490542in}}%
\pgfpathlineto{\pgfqpoint{2.177155in}{1.523548in}}%
\pgfpathlineto{\pgfqpoint{2.177768in}{1.597813in}}%
\pgfpathlineto{\pgfqpoint{2.178176in}{1.333761in}}%
\pgfpathlineto{\pgfqpoint{2.178584in}{1.564806in}}%
\pgfpathlineto{\pgfqpoint{2.179401in}{1.482290in}}%
\pgfpathlineto{\pgfqpoint{2.180217in}{1.424529in}}%
\pgfpathlineto{\pgfqpoint{2.180421in}{1.482290in}}%
\pgfpathlineto{\pgfqpoint{2.181238in}{1.408026in}}%
\pgfpathlineto{\pgfqpoint{2.181850in}{1.135723in}}%
\pgfpathlineto{\pgfqpoint{2.182055in}{1.432780in}}%
\pgfpathlineto{\pgfqpoint{2.182871in}{1.498793in}}%
\pgfpathlineto{\pgfqpoint{2.183075in}{1.482290in}}%
\pgfpathlineto{\pgfqpoint{2.185321in}{1.143974in}}%
\pgfpathlineto{\pgfqpoint{2.185525in}{0.888175in}}%
\pgfpathlineto{\pgfqpoint{2.185933in}{1.160477in}}%
\pgfpathlineto{\pgfqpoint{2.186342in}{1.102716in}}%
\pgfpathlineto{\pgfqpoint{2.187158in}{1.176981in}}%
\pgfpathlineto{\pgfqpoint{2.186750in}{1.094465in}}%
\pgfpathlineto{\pgfqpoint{2.187362in}{1.160477in}}%
\pgfpathlineto{\pgfqpoint{2.188179in}{0.871671in}}%
\pgfpathlineto{\pgfqpoint{2.188383in}{1.044955in}}%
\pgfpathlineto{\pgfqpoint{2.188587in}{1.168729in}}%
\pgfpathlineto{\pgfqpoint{2.189404in}{1.044955in}}%
\pgfpathlineto{\pgfqpoint{2.189812in}{1.102716in}}%
\pgfpathlineto{\pgfqpoint{2.190220in}{1.020200in}}%
\pgfpathlineto{\pgfqpoint{2.190424in}{1.061458in}}%
\pgfpathlineto{\pgfqpoint{2.190629in}{0.756149in}}%
\pgfpathlineto{\pgfqpoint{2.191037in}{1.069710in}}%
\pgfpathlineto{\pgfqpoint{2.191445in}{0.838665in}}%
\pgfpathlineto{\pgfqpoint{2.192058in}{1.028452in}}%
\pgfpathlineto{\pgfqpoint{2.192670in}{0.945936in}}%
\pgfpathlineto{\pgfqpoint{2.192874in}{0.690136in}}%
\pgfpathlineto{\pgfqpoint{2.193282in}{1.011949in}}%
\pgfpathlineto{\pgfqpoint{2.193691in}{0.921181in}}%
\pgfpathlineto{\pgfqpoint{2.193895in}{0.855168in}}%
\pgfpathlineto{\pgfqpoint{2.194507in}{0.970691in}}%
\pgfpathlineto{\pgfqpoint{2.194711in}{1.061458in}}%
\pgfpathlineto{\pgfqpoint{2.194916in}{0.937684in}}%
\pgfpathlineto{\pgfqpoint{2.195120in}{0.723142in}}%
\pgfpathlineto{\pgfqpoint{2.195936in}{0.888175in}}%
\pgfpathlineto{\pgfqpoint{2.197365in}{0.995445in}}%
\pgfpathlineto{\pgfqpoint{2.196345in}{0.855168in}}%
\pgfpathlineto{\pgfqpoint{2.197569in}{0.954187in}}%
\pgfpathlineto{\pgfqpoint{2.197978in}{0.690136in}}%
\pgfpathlineto{\pgfqpoint{2.198794in}{0.863420in}}%
\pgfpathlineto{\pgfqpoint{2.199815in}{1.086213in}}%
\pgfpathlineto{\pgfqpoint{2.200223in}{1.011949in}}%
\pgfpathlineto{\pgfqpoint{2.200427in}{0.912929in}}%
\pgfpathlineto{\pgfqpoint{2.201244in}{1.003697in}}%
\pgfpathlineto{\pgfqpoint{2.202673in}{1.152226in}}%
\pgfpathlineto{\pgfqpoint{2.203081in}{1.185232in}}%
\pgfpathlineto{\pgfqpoint{2.203490in}{1.110968in}}%
\pgfpathlineto{\pgfqpoint{2.203694in}{1.168729in}}%
\pgfpathlineto{\pgfqpoint{2.204306in}{1.069710in}}%
\pgfpathlineto{\pgfqpoint{2.204714in}{1.003697in}}%
\pgfpathlineto{\pgfqpoint{2.205123in}{1.069710in}}%
\pgfpathlineto{\pgfqpoint{2.205735in}{1.152226in}}%
\pgfpathlineto{\pgfqpoint{2.205939in}{1.044955in}}%
\pgfpathlineto{\pgfqpoint{2.206348in}{1.061458in}}%
\pgfpathlineto{\pgfqpoint{2.206756in}{0.912929in}}%
\pgfpathlineto{\pgfqpoint{2.206960in}{0.698388in}}%
\pgfpathlineto{\pgfqpoint{2.207368in}{1.028452in}}%
\pgfpathlineto{\pgfqpoint{2.207777in}{0.929433in}}%
\pgfpathlineto{\pgfqpoint{2.208389in}{0.970691in}}%
\pgfpathlineto{\pgfqpoint{2.208593in}{0.954187in}}%
\pgfpathlineto{\pgfqpoint{2.209001in}{0.805659in}}%
\pgfpathlineto{\pgfqpoint{2.209614in}{0.896426in}}%
\pgfpathlineto{\pgfqpoint{2.211043in}{1.086213in}}%
\pgfpathlineto{\pgfqpoint{2.211247in}{1.053207in}}%
\pgfpathlineto{\pgfqpoint{2.211655in}{1.119219in}}%
\pgfpathlineto{\pgfqpoint{2.211859in}{1.077961in}}%
\pgfpathlineto{\pgfqpoint{2.212268in}{1.135723in}}%
\pgfpathlineto{\pgfqpoint{2.212676in}{1.011949in}}%
\pgfpathlineto{\pgfqpoint{2.213084in}{1.044955in}}%
\pgfpathlineto{\pgfqpoint{2.213493in}{1.020200in}}%
\pgfpathlineto{\pgfqpoint{2.213697in}{0.987194in}}%
\pgfpathlineto{\pgfqpoint{2.214309in}{1.061458in}}%
\pgfpathlineto{\pgfqpoint{2.214513in}{1.028452in}}%
\pgfpathlineto{\pgfqpoint{2.215942in}{1.201736in}}%
\pgfpathlineto{\pgfqpoint{2.216146in}{1.284252in}}%
\pgfpathlineto{\pgfqpoint{2.216759in}{1.160477in}}%
\pgfpathlineto{\pgfqpoint{2.216963in}{1.234742in}}%
\pgfpathlineto{\pgfqpoint{2.217167in}{1.185232in}}%
\pgfpathlineto{\pgfqpoint{2.217984in}{1.226490in}}%
\pgfpathlineto{\pgfqpoint{2.218188in}{1.242994in}}%
\pgfpathlineto{\pgfqpoint{2.218596in}{1.193484in}}%
\pgfpathlineto{\pgfqpoint{2.218800in}{1.193484in}}%
\pgfpathlineto{\pgfqpoint{2.219004in}{1.185232in}}%
\pgfpathlineto{\pgfqpoint{2.219413in}{1.309006in}}%
\pgfpathlineto{\pgfqpoint{2.219821in}{1.251245in}}%
\pgfpathlineto{\pgfqpoint{2.220025in}{1.152226in}}%
\pgfpathlineto{\pgfqpoint{2.220229in}{1.342013in}}%
\pgfpathlineto{\pgfqpoint{2.220842in}{1.292503in}}%
\pgfpathlineto{\pgfqpoint{2.221454in}{1.375019in}}%
\pgfpathlineto{\pgfqpoint{2.221862in}{1.209987in}}%
\pgfpathlineto{\pgfqpoint{2.222067in}{1.531800in}}%
\pgfpathlineto{\pgfqpoint{2.222475in}{1.226490in}}%
\pgfpathlineto{\pgfqpoint{2.222679in}{1.317258in}}%
\pgfpathlineto{\pgfqpoint{2.223291in}{1.176981in}}%
\pgfpathlineto{\pgfqpoint{2.223496in}{1.193484in}}%
\pgfpathlineto{\pgfqpoint{2.224108in}{1.077961in}}%
\pgfpathlineto{\pgfqpoint{2.224720in}{1.160477in}}%
\pgfpathlineto{\pgfqpoint{2.225537in}{1.069710in}}%
\pgfpathlineto{\pgfqpoint{2.225741in}{1.119219in}}%
\pgfpathlineto{\pgfqpoint{2.227170in}{1.226490in}}%
\pgfpathlineto{\pgfqpoint{2.227374in}{1.209987in}}%
\pgfpathlineto{\pgfqpoint{2.228191in}{1.267748in}}%
\pgfpathlineto{\pgfqpoint{2.227987in}{1.201736in}}%
\pgfpathlineto{\pgfqpoint{2.228803in}{1.251245in}}%
\pgfpathlineto{\pgfqpoint{2.229007in}{1.226490in}}%
\pgfpathlineto{\pgfqpoint{2.229416in}{1.284252in}}%
\pgfpathlineto{\pgfqpoint{2.229620in}{1.259497in}}%
\pgfpathlineto{\pgfqpoint{2.229824in}{1.350264in}}%
\pgfpathlineto{\pgfqpoint{2.230232in}{1.226490in}}%
\pgfpathlineto{\pgfqpoint{2.230641in}{1.276000in}}%
\pgfpathlineto{\pgfqpoint{2.232274in}{1.383271in}}%
\pgfpathlineto{\pgfqpoint{2.232682in}{1.333761in}}%
\pgfpathlineto{\pgfqpoint{2.232886in}{1.292503in}}%
\pgfpathlineto{\pgfqpoint{2.233294in}{1.416277in}}%
\pgfpathlineto{\pgfqpoint{2.234111in}{1.482290in}}%
\pgfpathlineto{\pgfqpoint{2.234928in}{1.556554in}}%
\pgfpathlineto{\pgfqpoint{2.234723in}{1.465787in}}%
\pgfpathlineto{\pgfqpoint{2.235132in}{1.548303in}}%
\pgfpathlineto{\pgfqpoint{2.235744in}{1.474038in}}%
\pgfpathlineto{\pgfqpoint{2.236152in}{1.523548in}}%
\pgfpathlineto{\pgfqpoint{2.237377in}{1.606064in}}%
\pgfpathlineto{\pgfqpoint{2.237581in}{1.564806in}}%
\pgfpathlineto{\pgfqpoint{2.237990in}{1.655574in}}%
\pgfpathlineto{\pgfqpoint{2.238194in}{1.630819in}}%
\pgfpathlineto{\pgfqpoint{2.238602in}{1.663825in}}%
\pgfpathlineto{\pgfqpoint{2.238806in}{1.482290in}}%
\pgfpathlineto{\pgfqpoint{2.239623in}{1.738090in}}%
\pgfpathlineto{\pgfqpoint{2.239827in}{1.713335in}}%
\pgfpathlineto{\pgfqpoint{2.240031in}{1.771096in}}%
\pgfpathlineto{\pgfqpoint{2.241052in}{1.870115in}}%
\pgfpathlineto{\pgfqpoint{2.240848in}{1.705083in}}%
\pgfpathlineto{\pgfqpoint{2.241256in}{1.853612in}}%
\pgfpathlineto{\pgfqpoint{2.241868in}{1.861864in}}%
\pgfpathlineto{\pgfqpoint{2.242277in}{1.804103in}}%
\pgfpathlineto{\pgfqpoint{2.243502in}{1.861864in}}%
\pgfpathlineto{\pgfqpoint{2.243706in}{1.853612in}}%
\pgfpathlineto{\pgfqpoint{2.244114in}{1.845361in}}%
\pgfpathlineto{\pgfqpoint{2.244726in}{1.894870in}}%
\pgfpathlineto{\pgfqpoint{2.244931in}{1.853612in}}%
\pgfpathlineto{\pgfqpoint{2.245339in}{1.919625in}}%
\pgfpathlineto{\pgfqpoint{2.245747in}{1.894870in}}%
\pgfpathlineto{\pgfqpoint{2.247584in}{2.043399in}}%
\pgfpathlineto{\pgfqpoint{2.247789in}{2.059902in}}%
\pgfpathlineto{\pgfqpoint{2.248605in}{2.068154in}}%
\pgfpathlineto{\pgfqpoint{2.249013in}{1.878367in}}%
\pgfpathlineto{\pgfqpoint{2.249830in}{2.068154in}}%
\pgfpathlineto{\pgfqpoint{2.250034in}{1.812354in}}%
\pgfpathlineto{\pgfqpoint{2.250442in}{2.117664in}}%
\pgfpathlineto{\pgfqpoint{2.250851in}{2.101160in}}%
\pgfpathlineto{\pgfqpoint{2.251259in}{2.051651in}}%
\pgfpathlineto{\pgfqpoint{2.251463in}{2.018644in}}%
\pgfpathlineto{\pgfqpoint{2.251871in}{2.059902in}}%
\pgfpathlineto{\pgfqpoint{2.252688in}{2.158922in}}%
\pgfpathlineto{\pgfqpoint{2.253096in}{2.092909in}}%
\pgfpathlineto{\pgfqpoint{2.254117in}{2.175425in}}%
\pgfpathlineto{\pgfqpoint{2.254729in}{2.158922in}}%
\pgfpathlineto{\pgfqpoint{2.254934in}{2.216683in}}%
\pgfpathlineto{\pgfqpoint{2.255750in}{2.134167in}}%
\pgfpathlineto{\pgfqpoint{2.256158in}{2.092909in}}%
\pgfpathlineto{\pgfqpoint{2.256567in}{2.167173in}}%
\pgfpathlineto{\pgfqpoint{2.256771in}{2.142418in}}%
\pgfpathlineto{\pgfqpoint{2.257587in}{2.208431in}}%
\pgfpathlineto{\pgfqpoint{2.257383in}{2.125915in}}%
\pgfpathlineto{\pgfqpoint{2.258200in}{2.183676in}}%
\pgfpathlineto{\pgfqpoint{2.259016in}{2.224934in}}%
\pgfpathlineto{\pgfqpoint{2.259425in}{2.109412in}}%
\pgfpathlineto{\pgfqpoint{2.259629in}{2.109412in}}%
\pgfpathlineto{\pgfqpoint{2.259833in}{1.927877in}}%
\pgfpathlineto{\pgfqpoint{2.260241in}{2.142418in}}%
\pgfpathlineto{\pgfqpoint{2.260650in}{2.092909in}}%
\pgfpathlineto{\pgfqpoint{2.261262in}{2.150670in}}%
\pgfpathlineto{\pgfqpoint{2.261466in}{2.059902in}}%
\pgfpathlineto{\pgfqpoint{2.261670in}{2.101160in}}%
\pgfpathlineto{\pgfqpoint{2.262487in}{2.158922in}}%
\pgfpathlineto{\pgfqpoint{2.262078in}{2.092909in}}%
\pgfpathlineto{\pgfqpoint{2.262691in}{2.150670in}}%
\pgfpathlineto{\pgfqpoint{2.263303in}{1.894870in}}%
\pgfpathlineto{\pgfqpoint{2.263712in}{2.101160in}}%
\pgfpathlineto{\pgfqpoint{2.263916in}{2.084657in}}%
\pgfpathlineto{\pgfqpoint{2.264120in}{2.142418in}}%
\pgfpathlineto{\pgfqpoint{2.264528in}{2.109412in}}%
\pgfpathlineto{\pgfqpoint{2.264936in}{2.134167in}}%
\pgfpathlineto{\pgfqpoint{2.265141in}{2.092909in}}%
\pgfpathlineto{\pgfqpoint{2.266161in}{2.010393in}}%
\pgfpathlineto{\pgfqpoint{2.266365in}{2.018644in}}%
\pgfpathlineto{\pgfqpoint{2.268203in}{2.109412in}}%
\pgfpathlineto{\pgfqpoint{2.269019in}{2.043399in}}%
\pgfpathlineto{\pgfqpoint{2.269428in}{2.051651in}}%
\pgfpathlineto{\pgfqpoint{2.269632in}{2.059902in}}%
\pgfpathlineto{\pgfqpoint{2.269836in}{2.035148in}}%
\pgfpathlineto{\pgfqpoint{2.270244in}{1.969135in}}%
\pgfpathlineto{\pgfqpoint{2.270857in}{2.035148in}}%
\pgfpathlineto{\pgfqpoint{2.271265in}{2.035148in}}%
\pgfpathlineto{\pgfqpoint{2.271673in}{2.117664in}}%
\pgfpathlineto{\pgfqpoint{2.272286in}{2.059902in}}%
\pgfpathlineto{\pgfqpoint{2.272898in}{2.026896in}}%
\pgfpathlineto{\pgfqpoint{2.273306in}{2.035148in}}%
\pgfpathlineto{\pgfqpoint{2.274531in}{2.101160in}}%
\pgfpathlineto{\pgfqpoint{2.274735in}{2.101160in}}%
\pgfpathlineto{\pgfqpoint{2.274939in}{2.117664in}}%
\pgfpathlineto{\pgfqpoint{2.275348in}{2.068154in}}%
\pgfpathlineto{\pgfqpoint{2.275552in}{2.059902in}}%
\pgfpathlineto{\pgfqpoint{2.275756in}{2.125915in}}%
\pgfpathlineto{\pgfqpoint{2.276368in}{2.051651in}}%
\pgfpathlineto{\pgfqpoint{2.276573in}{2.059902in}}%
\pgfpathlineto{\pgfqpoint{2.276777in}{2.068154in}}%
\pgfpathlineto{\pgfqpoint{2.276981in}{2.043399in}}%
\pgfpathlineto{\pgfqpoint{2.277185in}{2.035148in}}%
\pgfpathlineto{\pgfqpoint{2.277389in}{2.043399in}}%
\pgfpathlineto{\pgfqpoint{2.277593in}{2.092909in}}%
\pgfpathlineto{\pgfqpoint{2.278002in}{2.026896in}}%
\pgfpathlineto{\pgfqpoint{2.278206in}{2.026896in}}%
\pgfpathlineto{\pgfqpoint{2.278410in}{1.969135in}}%
\pgfpathlineto{\pgfqpoint{2.279431in}{1.993890in}}%
\pgfpathlineto{\pgfqpoint{2.279635in}{2.002141in}}%
\pgfpathlineto{\pgfqpoint{2.280043in}{1.894870in}}%
\pgfpathlineto{\pgfqpoint{2.280451in}{2.026896in}}%
\pgfpathlineto{\pgfqpoint{2.280655in}{1.985638in}}%
\pgfpathlineto{\pgfqpoint{2.281472in}{2.117664in}}%
\pgfpathlineto{\pgfqpoint{2.281880in}{2.092909in}}%
\pgfpathlineto{\pgfqpoint{2.282289in}{2.068154in}}%
\pgfpathlineto{\pgfqpoint{2.282697in}{2.101160in}}%
\pgfpathlineto{\pgfqpoint{2.282901in}{2.092909in}}%
\pgfpathlineto{\pgfqpoint{2.283105in}{2.101160in}}%
\pgfpathlineto{\pgfqpoint{2.283309in}{2.059902in}}%
\pgfpathlineto{\pgfqpoint{2.283922in}{2.150670in}}%
\pgfpathlineto{\pgfqpoint{2.284330in}{2.076406in}}%
\pgfpathlineto{\pgfqpoint{2.284942in}{2.043399in}}%
\pgfpathlineto{\pgfqpoint{2.285759in}{2.167173in}}%
\pgfpathlineto{\pgfqpoint{2.286167in}{2.125915in}}%
\pgfpathlineto{\pgfqpoint{2.286780in}{2.175425in}}%
\pgfpathlineto{\pgfqpoint{2.287188in}{2.068154in}}%
\pgfpathlineto{\pgfqpoint{2.288005in}{2.167173in}}%
\pgfpathlineto{\pgfqpoint{2.287800in}{2.018644in}}%
\pgfpathlineto{\pgfqpoint{2.288413in}{2.134167in}}%
\pgfpathlineto{\pgfqpoint{2.288617in}{2.142418in}}%
\pgfpathlineto{\pgfqpoint{2.290046in}{2.059902in}}%
\pgfpathlineto{\pgfqpoint{2.290250in}{2.084657in}}%
\pgfpathlineto{\pgfqpoint{2.290658in}{2.010393in}}%
\pgfpathlineto{\pgfqpoint{2.291067in}{2.076406in}}%
\pgfpathlineto{\pgfqpoint{2.291475in}{2.035148in}}%
\pgfpathlineto{\pgfqpoint{2.291883in}{2.084657in}}%
\pgfpathlineto{\pgfqpoint{2.292087in}{2.084657in}}%
\pgfpathlineto{\pgfqpoint{2.293721in}{2.241438in}}%
\pgfpathlineto{\pgfqpoint{2.293925in}{2.241438in}}%
\pgfpathlineto{\pgfqpoint{2.294537in}{2.010393in}}%
\pgfpathlineto{\pgfqpoint{2.295150in}{2.117664in}}%
\pgfpathlineto{\pgfqpoint{2.295354in}{2.002141in}}%
\pgfpathlineto{\pgfqpoint{2.295558in}{2.167173in}}%
\pgfpathlineto{\pgfqpoint{2.296170in}{2.142418in}}%
\pgfpathlineto{\pgfqpoint{2.296987in}{2.125915in}}%
\pgfpathlineto{\pgfqpoint{2.297395in}{2.183676in}}%
\pgfpathlineto{\pgfqpoint{2.297803in}{2.175425in}}%
\pgfpathlineto{\pgfqpoint{2.298212in}{1.878367in}}%
\pgfpathlineto{\pgfqpoint{2.298620in}{2.208431in}}%
\pgfpathlineto{\pgfqpoint{2.298824in}{2.183676in}}%
\pgfpathlineto{\pgfqpoint{2.299437in}{2.183676in}}%
\pgfpathlineto{\pgfqpoint{2.299845in}{2.142418in}}%
\pgfpathlineto{\pgfqpoint{2.300661in}{2.167173in}}%
\pgfpathlineto{\pgfqpoint{2.301070in}{2.142418in}}%
\pgfpathlineto{\pgfqpoint{2.302090in}{2.249689in}}%
\pgfpathlineto{\pgfqpoint{2.302703in}{2.043399in}}%
\pgfpathlineto{\pgfqpoint{2.302499in}{2.257941in}}%
\pgfpathlineto{\pgfqpoint{2.303315in}{2.150670in}}%
\pgfpathlineto{\pgfqpoint{2.304540in}{2.282696in}}%
\pgfpathlineto{\pgfqpoint{2.305153in}{2.208431in}}%
\pgfpathlineto{\pgfqpoint{2.305561in}{2.266192in}}%
\pgfpathlineto{\pgfqpoint{2.305765in}{2.266192in}}%
\pgfpathlineto{\pgfqpoint{2.305969in}{2.216683in}}%
\pgfpathlineto{\pgfqpoint{2.306582in}{2.299199in}}%
\pgfpathlineto{\pgfqpoint{2.306786in}{2.299199in}}%
\pgfpathlineto{\pgfqpoint{2.307194in}{2.356960in}}%
\pgfpathlineto{\pgfqpoint{2.307806in}{2.307450in}}%
\pgfpathlineto{\pgfqpoint{2.309440in}{2.018644in}}%
\pgfpathlineto{\pgfqpoint{2.309644in}{2.068154in}}%
\pgfpathlineto{\pgfqpoint{2.310052in}{2.274444in}}%
\pgfpathlineto{\pgfqpoint{2.310869in}{2.249689in}}%
\pgfpathlineto{\pgfqpoint{2.311073in}{2.200180in}}%
\pgfpathlineto{\pgfqpoint{2.311685in}{2.290947in}}%
\pgfpathlineto{\pgfqpoint{2.312502in}{2.257941in}}%
\pgfpathlineto{\pgfqpoint{2.312706in}{2.332205in}}%
\pgfpathlineto{\pgfqpoint{2.313318in}{2.200180in}}%
\pgfpathlineto{\pgfqpoint{2.313522in}{2.241438in}}%
\pgfpathlineto{\pgfqpoint{2.314339in}{2.175425in}}%
\pgfpathlineto{\pgfqpoint{2.314543in}{2.216683in}}%
\pgfpathlineto{\pgfqpoint{2.315156in}{2.200180in}}%
\pgfpathlineto{\pgfqpoint{2.315972in}{2.307450in}}%
\pgfpathlineto{\pgfqpoint{2.316585in}{1.993890in}}%
\pgfpathlineto{\pgfqpoint{2.316789in}{2.332205in}}%
\pgfpathlineto{\pgfqpoint{2.316993in}{2.290947in}}%
\pgfpathlineto{\pgfqpoint{2.317401in}{2.323954in}}%
\pgfpathlineto{\pgfqpoint{2.317605in}{2.282696in}}%
\pgfpathlineto{\pgfqpoint{2.317809in}{2.216683in}}%
\pgfpathlineto{\pgfqpoint{2.318422in}{2.290947in}}%
\pgfpathlineto{\pgfqpoint{2.318626in}{2.274444in}}%
\pgfpathlineto{\pgfqpoint{2.319238in}{2.315702in}}%
\pgfpathlineto{\pgfqpoint{2.320055in}{2.125915in}}%
\pgfpathlineto{\pgfqpoint{2.320463in}{2.191928in}}%
\pgfpathlineto{\pgfqpoint{2.320667in}{2.191928in}}%
\pgfpathlineto{\pgfqpoint{2.320872in}{2.224934in}}%
\pgfpathlineto{\pgfqpoint{2.321280in}{2.150670in}}%
\pgfpathlineto{\pgfqpoint{2.321484in}{2.183676in}}%
\pgfpathlineto{\pgfqpoint{2.321688in}{2.167173in}}%
\pgfpathlineto{\pgfqpoint{2.321892in}{2.216683in}}%
\pgfpathlineto{\pgfqpoint{2.322301in}{2.191928in}}%
\pgfpathlineto{\pgfqpoint{2.322505in}{2.216683in}}%
\pgfpathlineto{\pgfqpoint{2.322913in}{2.208431in}}%
\pgfpathlineto{\pgfqpoint{2.323117in}{2.002141in}}%
\pgfpathlineto{\pgfqpoint{2.323934in}{2.249689in}}%
\pgfpathlineto{\pgfqpoint{2.324342in}{2.018644in}}%
\pgfpathlineto{\pgfqpoint{2.324954in}{2.233186in}}%
\pgfpathlineto{\pgfqpoint{2.325975in}{2.158922in}}%
\pgfpathlineto{\pgfqpoint{2.326179in}{2.167173in}}%
\pgfpathlineto{\pgfqpoint{2.326383in}{2.125915in}}%
\pgfpathlineto{\pgfqpoint{2.326792in}{2.216683in}}%
\pgfpathlineto{\pgfqpoint{2.326996in}{2.216683in}}%
\pgfpathlineto{\pgfqpoint{2.328221in}{2.315702in}}%
\pgfpathlineto{\pgfqpoint{2.328629in}{2.257941in}}%
\pgfpathlineto{\pgfqpoint{2.328833in}{2.348709in}}%
\pgfpathlineto{\pgfqpoint{2.329241in}{2.282696in}}%
\pgfpathlineto{\pgfqpoint{2.329854in}{2.422973in}}%
\pgfpathlineto{\pgfqpoint{2.330670in}{2.381715in}}%
\pgfpathlineto{\pgfqpoint{2.332916in}{2.241438in}}%
\pgfpathlineto{\pgfqpoint{2.333120in}{2.224934in}}%
\pgfpathlineto{\pgfqpoint{2.333733in}{2.249689in}}%
\pgfpathlineto{\pgfqpoint{2.333937in}{2.241438in}}%
\pgfpathlineto{\pgfqpoint{2.335366in}{2.389967in}}%
\pgfpathlineto{\pgfqpoint{2.335774in}{2.299199in}}%
\pgfpathlineto{\pgfqpoint{2.336591in}{2.356960in}}%
\pgfpathlineto{\pgfqpoint{2.337407in}{2.373463in}}%
\pgfpathlineto{\pgfqpoint{2.337611in}{2.356960in}}%
\pgfpathlineto{\pgfqpoint{2.338224in}{2.299199in}}%
\pgfpathlineto{\pgfqpoint{2.339449in}{2.307450in}}%
\pgfpathlineto{\pgfqpoint{2.339653in}{2.340457in}}%
\pgfpathlineto{\pgfqpoint{2.340061in}{2.249689in}}%
\pgfpathlineto{\pgfqpoint{2.340265in}{2.323954in}}%
\pgfpathlineto{\pgfqpoint{2.340469in}{2.266192in}}%
\pgfpathlineto{\pgfqpoint{2.341286in}{2.307450in}}%
\pgfpathlineto{\pgfqpoint{2.342511in}{2.224934in}}%
\pgfpathlineto{\pgfqpoint{2.342715in}{2.274444in}}%
\pgfpathlineto{\pgfqpoint{2.343940in}{2.356960in}}%
\pgfpathlineto{\pgfqpoint{2.343123in}{2.216683in}}%
\pgfpathlineto{\pgfqpoint{2.344144in}{2.332205in}}%
\pgfpathlineto{\pgfqpoint{2.344348in}{2.332205in}}%
\pgfpathlineto{\pgfqpoint{2.344756in}{2.257941in}}%
\pgfpathlineto{\pgfqpoint{2.345165in}{2.340457in}}%
\pgfpathlineto{\pgfqpoint{2.345369in}{2.323954in}}%
\pgfpathlineto{\pgfqpoint{2.345777in}{2.398218in}}%
\pgfpathlineto{\pgfqpoint{2.346389in}{2.356960in}}%
\pgfpathlineto{\pgfqpoint{2.347002in}{2.389967in}}%
\pgfpathlineto{\pgfqpoint{2.347206in}{2.340457in}}%
\pgfpathlineto{\pgfqpoint{2.348431in}{2.480734in}}%
\pgfpathlineto{\pgfqpoint{2.350472in}{2.315702in}}%
\pgfpathlineto{\pgfqpoint{2.351901in}{2.546747in}}%
\pgfpathlineto{\pgfqpoint{2.352105in}{2.530244in}}%
\pgfpathlineto{\pgfqpoint{2.352309in}{2.530244in}}%
\pgfpathlineto{\pgfqpoint{2.352514in}{2.571502in}}%
\pgfpathlineto{\pgfqpoint{2.353126in}{2.513741in}}%
\pgfpathlineto{\pgfqpoint{2.353330in}{2.554999in}}%
\pgfpathlineto{\pgfqpoint{2.354555in}{2.472483in}}%
\pgfpathlineto{\pgfqpoint{2.355576in}{2.538495in}}%
\pgfpathlineto{\pgfqpoint{2.356392in}{2.447728in}}%
\pgfpathlineto{\pgfqpoint{2.356801in}{2.455979in}}%
\pgfpathlineto{\pgfqpoint{2.357005in}{2.439476in}}%
\pgfpathlineto{\pgfqpoint{2.357209in}{2.455979in}}%
\pgfpathlineto{\pgfqpoint{2.357413in}{2.513741in}}%
\pgfpathlineto{\pgfqpoint{2.358025in}{2.505489in}}%
\pgfpathlineto{\pgfqpoint{2.358230in}{2.406470in}}%
\pgfpathlineto{\pgfqpoint{2.359046in}{2.505489in}}%
\pgfpathlineto{\pgfqpoint{2.359250in}{2.455979in}}%
\pgfpathlineto{\pgfqpoint{2.359863in}{2.554999in}}%
\pgfpathlineto{\pgfqpoint{2.360271in}{2.464231in}}%
\pgfpathlineto{\pgfqpoint{2.360475in}{2.455979in}}%
\pgfpathlineto{\pgfqpoint{2.360679in}{2.464231in}}%
\pgfpathlineto{\pgfqpoint{2.361700in}{2.571502in}}%
\pgfpathlineto{\pgfqpoint{2.362108in}{2.538495in}}%
\pgfpathlineto{\pgfqpoint{2.362312in}{2.546747in}}%
\pgfpathlineto{\pgfqpoint{2.362517in}{2.521992in}}%
\pgfpathlineto{\pgfqpoint{2.363129in}{2.455979in}}%
\pgfpathlineto{\pgfqpoint{2.363537in}{2.530244in}}%
\pgfpathlineto{\pgfqpoint{2.363741in}{2.530244in}}%
\pgfpathlineto{\pgfqpoint{2.364966in}{2.447728in}}%
\pgfpathlineto{\pgfqpoint{2.365375in}{2.455979in}}%
\pgfpathlineto{\pgfqpoint{2.365783in}{2.538495in}}%
\pgfpathlineto{\pgfqpoint{2.366395in}{2.480734in}}%
\pgfpathlineto{\pgfqpoint{2.366599in}{2.480734in}}%
\pgfpathlineto{\pgfqpoint{2.367620in}{2.398218in}}%
\pgfpathlineto{\pgfqpoint{2.367824in}{2.472483in}}%
\pgfpathlineto{\pgfqpoint{2.368845in}{2.455979in}}%
\pgfpathlineto{\pgfqpoint{2.369866in}{2.414721in}}%
\pgfpathlineto{\pgfqpoint{2.370070in}{2.422973in}}%
\pgfpathlineto{\pgfqpoint{2.370886in}{2.398218in}}%
\pgfpathlineto{\pgfqpoint{2.371091in}{2.406470in}}%
\pgfpathlineto{\pgfqpoint{2.371907in}{2.398218in}}%
\pgfpathlineto{\pgfqpoint{2.372724in}{2.579753in}}%
\pgfpathlineto{\pgfqpoint{2.373540in}{2.472483in}}%
\pgfpathlineto{\pgfqpoint{2.373949in}{2.480734in}}%
\pgfpathlineto{\pgfqpoint{2.374153in}{2.546747in}}%
\pgfpathlineto{\pgfqpoint{2.374969in}{2.505489in}}%
\pgfpathlineto{\pgfqpoint{2.375173in}{2.488986in}}%
\pgfpathlineto{\pgfqpoint{2.375378in}{2.505489in}}%
\pgfpathlineto{\pgfqpoint{2.375582in}{2.332205in}}%
\pgfpathlineto{\pgfqpoint{2.375786in}{2.546747in}}%
\pgfpathlineto{\pgfqpoint{2.376398in}{2.521992in}}%
\pgfpathlineto{\pgfqpoint{2.376602in}{2.538495in}}%
\pgfpathlineto{\pgfqpoint{2.377215in}{2.521992in}}%
\pgfpathlineto{\pgfqpoint{2.378440in}{2.422973in}}%
\pgfpathlineto{\pgfqpoint{2.378644in}{2.431225in}}%
\pgfpathlineto{\pgfqpoint{2.379665in}{2.472483in}}%
\pgfpathlineto{\pgfqpoint{2.379869in}{2.406470in}}%
\pgfpathlineto{\pgfqpoint{2.380073in}{2.480734in}}%
\pgfpathlineto{\pgfqpoint{2.380685in}{2.464231in}}%
\pgfpathlineto{\pgfqpoint{2.380889in}{2.464231in}}%
\pgfpathlineto{\pgfqpoint{2.381502in}{2.431225in}}%
\pgfpathlineto{\pgfqpoint{2.382727in}{2.224934in}}%
\pgfpathlineto{\pgfqpoint{2.382114in}{2.464231in}}%
\pgfpathlineto{\pgfqpoint{2.382931in}{2.299199in}}%
\pgfpathlineto{\pgfqpoint{2.384564in}{2.464231in}}%
\pgfpathlineto{\pgfqpoint{2.385381in}{2.398218in}}%
\pgfpathlineto{\pgfqpoint{2.385585in}{2.431225in}}%
\pgfpathlineto{\pgfqpoint{2.385789in}{2.464231in}}%
\pgfpathlineto{\pgfqpoint{2.385993in}{2.422973in}}%
\pgfpathlineto{\pgfqpoint{2.386605in}{2.191928in}}%
\pgfpathlineto{\pgfqpoint{2.386810in}{2.488986in}}%
\pgfpathlineto{\pgfqpoint{2.387014in}{2.464231in}}%
\pgfpathlineto{\pgfqpoint{2.387218in}{2.579753in}}%
\pgfpathlineto{\pgfqpoint{2.387422in}{2.422973in}}%
\pgfpathlineto{\pgfqpoint{2.388034in}{2.513741in}}%
\pgfpathlineto{\pgfqpoint{2.389259in}{2.431225in}}%
\pgfpathlineto{\pgfqpoint{2.389463in}{2.439476in}}%
\pgfpathlineto{\pgfqpoint{2.389668in}{2.513741in}}%
\pgfpathlineto{\pgfqpoint{2.390076in}{2.389967in}}%
\pgfpathlineto{\pgfqpoint{2.390484in}{2.447728in}}%
\pgfpathlineto{\pgfqpoint{2.391301in}{2.340457in}}%
\pgfpathlineto{\pgfqpoint{2.391913in}{2.348709in}}%
\pgfpathlineto{\pgfqpoint{2.392730in}{2.389967in}}%
\pgfpathlineto{\pgfqpoint{2.392934in}{2.381715in}}%
\pgfpathlineto{\pgfqpoint{2.393750in}{2.183676in}}%
\pgfpathlineto{\pgfqpoint{2.393955in}{2.332205in}}%
\pgfpathlineto{\pgfqpoint{2.394771in}{2.422973in}}%
\pgfpathlineto{\pgfqpoint{2.394567in}{2.315702in}}%
\pgfpathlineto{\pgfqpoint{2.394975in}{2.365212in}}%
\pgfpathlineto{\pgfqpoint{2.395179in}{2.340457in}}%
\pgfpathlineto{\pgfqpoint{2.396608in}{2.554999in}}%
\pgfpathlineto{\pgfqpoint{2.396813in}{2.554999in}}%
\pgfpathlineto{\pgfqpoint{2.397017in}{2.563250in}}%
\pgfpathlineto{\pgfqpoint{2.397221in}{2.621011in}}%
\pgfpathlineto{\pgfqpoint{2.397629in}{2.464231in}}%
\pgfpathlineto{\pgfqpoint{2.398037in}{2.431225in}}%
\pgfpathlineto{\pgfqpoint{2.398242in}{2.389967in}}%
\pgfpathlineto{\pgfqpoint{2.398854in}{2.472483in}}%
\pgfpathlineto{\pgfqpoint{2.399058in}{2.480734in}}%
\pgfpathlineto{\pgfqpoint{2.399262in}{2.455979in}}%
\pgfpathlineto{\pgfqpoint{2.399671in}{2.398218in}}%
\pgfpathlineto{\pgfqpoint{2.399875in}{2.530244in}}%
\pgfpathlineto{\pgfqpoint{2.400895in}{2.488986in}}%
\pgfpathlineto{\pgfqpoint{2.401304in}{2.488986in}}%
\pgfpathlineto{\pgfqpoint{2.401712in}{2.521992in}}%
\pgfpathlineto{\pgfqpoint{2.402120in}{2.497237in}}%
\pgfpathlineto{\pgfqpoint{2.402529in}{2.398218in}}%
\pgfpathlineto{\pgfqpoint{2.403549in}{2.439476in}}%
\pgfpathlineto{\pgfqpoint{2.405182in}{2.546747in}}%
\pgfpathlineto{\pgfqpoint{2.405387in}{2.480734in}}%
\pgfpathlineto{\pgfqpoint{2.406203in}{2.571502in}}%
\pgfpathlineto{\pgfqpoint{2.407836in}{2.480734in}}%
\pgfpathlineto{\pgfqpoint{2.408245in}{2.455979in}}%
\pgfpathlineto{\pgfqpoint{2.409061in}{2.521992in}}%
\pgfpathlineto{\pgfqpoint{2.409265in}{2.521992in}}%
\pgfpathlineto{\pgfqpoint{2.410082in}{2.439476in}}%
\pgfpathlineto{\pgfqpoint{2.410286in}{2.464231in}}%
\pgfpathlineto{\pgfqpoint{2.410490in}{2.505489in}}%
\pgfpathlineto{\pgfqpoint{2.410898in}{2.439476in}}%
\pgfpathlineto{\pgfqpoint{2.411511in}{2.497237in}}%
\pgfpathlineto{\pgfqpoint{2.412736in}{2.439476in}}%
\pgfpathlineto{\pgfqpoint{2.413756in}{2.521992in}}%
\pgfpathlineto{\pgfqpoint{2.413348in}{2.422973in}}%
\pgfpathlineto{\pgfqpoint{2.413961in}{2.497237in}}%
\pgfpathlineto{\pgfqpoint{2.414369in}{2.521992in}}%
\pgfpathlineto{\pgfqpoint{2.415390in}{2.422973in}}%
\pgfpathlineto{\pgfqpoint{2.416002in}{2.488986in}}%
\pgfpathlineto{\pgfqpoint{2.416410in}{2.447728in}}%
\pgfpathlineto{\pgfqpoint{2.416614in}{2.299199in}}%
\pgfpathlineto{\pgfqpoint{2.416819in}{2.488986in}}%
\pgfpathlineto{\pgfqpoint{2.417431in}{2.480734in}}%
\pgfpathlineto{\pgfqpoint{2.417839in}{2.488986in}}%
\pgfpathlineto{\pgfqpoint{2.418248in}{2.604508in}}%
\pgfpathlineto{\pgfqpoint{2.418860in}{2.588005in}}%
\pgfpathlineto{\pgfqpoint{2.419268in}{2.464231in}}%
\pgfpathlineto{\pgfqpoint{2.419881in}{2.521992in}}%
\pgfpathlineto{\pgfqpoint{2.420493in}{2.546747in}}%
\pgfpathlineto{\pgfqpoint{2.421106in}{2.422973in}}%
\pgfpathlineto{\pgfqpoint{2.421514in}{2.455979in}}%
\pgfpathlineto{\pgfqpoint{2.421718in}{2.521992in}}%
\pgfpathlineto{\pgfqpoint{2.421922in}{2.431225in}}%
\pgfpathlineto{\pgfqpoint{2.422535in}{2.447728in}}%
\pgfpathlineto{\pgfqpoint{2.423351in}{2.505489in}}%
\pgfpathlineto{\pgfqpoint{2.423964in}{2.480734in}}%
\pgfpathlineto{\pgfqpoint{2.425188in}{2.422973in}}%
\pgfpathlineto{\pgfqpoint{2.426209in}{2.488986in}}%
\pgfpathlineto{\pgfqpoint{2.426413in}{2.307450in}}%
\pgfpathlineto{\pgfqpoint{2.427230in}{2.530244in}}%
\pgfpathlineto{\pgfqpoint{2.427434in}{2.554999in}}%
\pgfpathlineto{\pgfqpoint{2.427638in}{2.497237in}}%
\pgfpathlineto{\pgfqpoint{2.427842in}{2.513741in}}%
\pgfpathlineto{\pgfqpoint{2.429475in}{2.422973in}}%
\pgfpathlineto{\pgfqpoint{2.430496in}{2.554999in}}%
\pgfpathlineto{\pgfqpoint{2.430088in}{2.389967in}}%
\pgfpathlineto{\pgfqpoint{2.430904in}{2.488986in}}%
\pgfpathlineto{\pgfqpoint{2.431313in}{2.158922in}}%
\pgfpathlineto{\pgfqpoint{2.432129in}{2.323954in}}%
\pgfpathlineto{\pgfqpoint{2.433354in}{2.455979in}}%
\pgfpathlineto{\pgfqpoint{2.433558in}{2.323954in}}%
\pgfpathlineto{\pgfqpoint{2.434375in}{2.472483in}}%
\pgfpathlineto{\pgfqpoint{2.436008in}{2.175425in}}%
\pgfpathlineto{\pgfqpoint{2.437233in}{2.579753in}}%
\pgfpathlineto{\pgfqpoint{2.439274in}{2.422973in}}%
\pgfpathlineto{\pgfqpoint{2.437641in}{2.596257in}}%
\pgfpathlineto{\pgfqpoint{2.439478in}{2.455979in}}%
\pgfpathlineto{\pgfqpoint{2.440907in}{2.563250in}}%
\pgfpathlineto{\pgfqpoint{2.441316in}{2.563250in}}%
\pgfpathlineto{\pgfqpoint{2.441520in}{2.488986in}}%
\pgfpathlineto{\pgfqpoint{2.442132in}{2.579753in}}%
\pgfpathlineto{\pgfqpoint{2.442336in}{2.513741in}}%
\pgfpathlineto{\pgfqpoint{2.442949in}{2.497237in}}%
\pgfpathlineto{\pgfqpoint{2.443357in}{2.563250in}}%
\pgfpathlineto{\pgfqpoint{2.444174in}{2.480734in}}%
\pgfpathlineto{\pgfqpoint{2.444378in}{2.563250in}}%
\pgfpathlineto{\pgfqpoint{2.445194in}{2.538495in}}%
\pgfpathlineto{\pgfqpoint{2.445603in}{2.612760in}}%
\pgfpathlineto{\pgfqpoint{2.446011in}{2.505489in}}%
\pgfpathlineto{\pgfqpoint{2.446827in}{2.554999in}}%
\pgfpathlineto{\pgfqpoint{2.447032in}{2.579753in}}%
\pgfpathlineto{\pgfqpoint{2.447440in}{2.554999in}}%
\pgfpathlineto{\pgfqpoint{2.447848in}{2.257941in}}%
\pgfpathlineto{\pgfqpoint{2.448461in}{2.464231in}}%
\pgfpathlineto{\pgfqpoint{2.449481in}{2.579753in}}%
\pgfpathlineto{\pgfqpoint{2.449685in}{2.538495in}}%
\pgfpathlineto{\pgfqpoint{2.450094in}{2.513741in}}%
\pgfpathlineto{\pgfqpoint{2.450706in}{2.563250in}}%
\pgfpathlineto{\pgfqpoint{2.450910in}{2.299199in}}%
\pgfpathlineto{\pgfqpoint{2.451931in}{2.406470in}}%
\pgfpathlineto{\pgfqpoint{2.452339in}{2.389967in}}%
\pgfpathlineto{\pgfqpoint{2.453156in}{2.447728in}}%
\pgfpathlineto{\pgfqpoint{2.453360in}{2.398218in}}%
\pgfpathlineto{\pgfqpoint{2.453564in}{2.464231in}}%
\pgfpathlineto{\pgfqpoint{2.453972in}{2.439476in}}%
\pgfpathlineto{\pgfqpoint{2.454177in}{2.480734in}}%
\pgfpathlineto{\pgfqpoint{2.454789in}{2.389967in}}%
\pgfpathlineto{\pgfqpoint{2.454993in}{2.381715in}}%
\pgfpathlineto{\pgfqpoint{2.455197in}{2.158922in}}%
\pgfpathlineto{\pgfqpoint{2.456014in}{2.439476in}}%
\pgfpathlineto{\pgfqpoint{2.456218in}{2.406470in}}%
\pgfpathlineto{\pgfqpoint{2.456422in}{2.464231in}}%
\pgfpathlineto{\pgfqpoint{2.457239in}{2.604508in}}%
\pgfpathlineto{\pgfqpoint{2.457647in}{2.579753in}}%
\pgfpathlineto{\pgfqpoint{2.458668in}{2.596257in}}%
\pgfpathlineto{\pgfqpoint{2.458872in}{2.497237in}}%
\pgfpathlineto{\pgfqpoint{2.459688in}{2.621011in}}%
\pgfpathlineto{\pgfqpoint{2.459893in}{2.546747in}}%
\pgfpathlineto{\pgfqpoint{2.460097in}{2.340457in}}%
\pgfpathlineto{\pgfqpoint{2.461117in}{2.381715in}}%
\pgfpathlineto{\pgfqpoint{2.462955in}{2.142418in}}%
\pgfpathlineto{\pgfqpoint{2.463975in}{2.290947in}}%
\pgfpathlineto{\pgfqpoint{2.464180in}{2.241438in}}%
\pgfpathlineto{\pgfqpoint{2.464384in}{2.290947in}}%
\pgfpathlineto{\pgfqpoint{2.464588in}{2.158922in}}%
\pgfpathlineto{\pgfqpoint{2.464792in}{2.084657in}}%
\pgfpathlineto{\pgfqpoint{2.465200in}{2.274444in}}%
\pgfpathlineto{\pgfqpoint{2.465404in}{2.241438in}}%
\pgfpathlineto{\pgfqpoint{2.466425in}{2.018644in}}%
\pgfpathlineto{\pgfqpoint{2.466221in}{2.249689in}}%
\pgfpathlineto{\pgfqpoint{2.466629in}{2.134167in}}%
\pgfpathlineto{\pgfqpoint{2.467650in}{2.356960in}}%
\pgfpathlineto{\pgfqpoint{2.467854in}{2.257941in}}%
\pgfpathlineto{\pgfqpoint{2.468058in}{2.233186in}}%
\pgfpathlineto{\pgfqpoint{2.468467in}{2.315702in}}%
\pgfpathlineto{\pgfqpoint{2.468671in}{2.389967in}}%
\pgfpathlineto{\pgfqpoint{2.469079in}{2.266192in}}%
\pgfpathlineto{\pgfqpoint{2.469283in}{2.323954in}}%
\pgfpathlineto{\pgfqpoint{2.469487in}{2.257941in}}%
\pgfpathlineto{\pgfqpoint{2.469691in}{2.332205in}}%
\pgfpathlineto{\pgfqpoint{2.470304in}{2.282696in}}%
\pgfpathlineto{\pgfqpoint{2.471529in}{2.398218in}}%
\pgfpathlineto{\pgfqpoint{2.472141in}{2.389967in}}%
\pgfpathlineto{\pgfqpoint{2.473366in}{2.142418in}}%
\pgfpathlineto{\pgfqpoint{2.473570in}{2.356960in}}%
\pgfpathlineto{\pgfqpoint{2.474591in}{2.299199in}}%
\pgfpathlineto{\pgfqpoint{2.474795in}{2.332205in}}%
\pgfpathlineto{\pgfqpoint{2.474999in}{2.257941in}}%
\pgfpathlineto{\pgfqpoint{2.475203in}{2.200180in}}%
\pgfpathlineto{\pgfqpoint{2.475816in}{2.340457in}}%
\pgfpathlineto{\pgfqpoint{2.476020in}{2.348709in}}%
\pgfpathlineto{\pgfqpoint{2.476428in}{2.224934in}}%
\pgfpathlineto{\pgfqpoint{2.477041in}{2.299199in}}%
\pgfpathlineto{\pgfqpoint{2.477449in}{2.332205in}}%
\pgfpathlineto{\pgfqpoint{2.477653in}{2.323954in}}%
\pgfpathlineto{\pgfqpoint{2.477857in}{2.233186in}}%
\pgfpathlineto{\pgfqpoint{2.478061in}{2.365212in}}%
\pgfpathlineto{\pgfqpoint{2.478878in}{2.249689in}}%
\pgfpathlineto{\pgfqpoint{2.479082in}{2.299199in}}%
\pgfpathlineto{\pgfqpoint{2.479694in}{2.233186in}}%
\pgfpathlineto{\pgfqpoint{2.479899in}{2.266192in}}%
\pgfpathlineto{\pgfqpoint{2.480307in}{1.993890in}}%
\pgfpathlineto{\pgfqpoint{2.481123in}{2.200180in}}%
\pgfpathlineto{\pgfqpoint{2.482348in}{2.348709in}}%
\pgfpathlineto{\pgfqpoint{2.482552in}{2.266192in}}%
\pgfpathlineto{\pgfqpoint{2.483369in}{2.348709in}}%
\pgfpathlineto{\pgfqpoint{2.483777in}{2.439476in}}%
\pgfpathlineto{\pgfqpoint{2.483981in}{2.340457in}}%
\pgfpathlineto{\pgfqpoint{2.484390in}{2.422973in}}%
\pgfpathlineto{\pgfqpoint{2.485615in}{2.200180in}}%
\pgfpathlineto{\pgfqpoint{2.486023in}{2.249689in}}%
\pgfpathlineto{\pgfqpoint{2.486227in}{2.307450in}}%
\pgfpathlineto{\pgfqpoint{2.486839in}{2.241438in}}%
\pgfpathlineto{\pgfqpoint{2.487044in}{2.266192in}}%
\pgfpathlineto{\pgfqpoint{2.488064in}{2.216683in}}%
\pgfpathlineto{\pgfqpoint{2.488268in}{2.125915in}}%
\pgfpathlineto{\pgfqpoint{2.488677in}{2.249689in}}%
\pgfpathlineto{\pgfqpoint{2.489085in}{2.233186in}}%
\pgfpathlineto{\pgfqpoint{2.490106in}{2.340457in}}%
\pgfpathlineto{\pgfqpoint{2.489697in}{2.216683in}}%
\pgfpathlineto{\pgfqpoint{2.490514in}{2.266192in}}%
\pgfpathlineto{\pgfqpoint{2.491943in}{1.952632in}}%
\pgfpathlineto{\pgfqpoint{2.492147in}{2.002141in}}%
\pgfpathlineto{\pgfqpoint{2.492555in}{2.191928in}}%
\pgfpathlineto{\pgfqpoint{2.493372in}{2.183676in}}%
\pgfpathlineto{\pgfqpoint{2.494597in}{2.092909in}}%
\pgfpathlineto{\pgfqpoint{2.495413in}{2.150670in}}%
\pgfpathlineto{\pgfqpoint{2.496638in}{1.977386in}}%
\pgfpathlineto{\pgfqpoint{2.498067in}{2.191928in}}%
\pgfpathlineto{\pgfqpoint{2.499088in}{2.134167in}}%
\pgfpathlineto{\pgfqpoint{2.498884in}{2.224934in}}%
\pgfpathlineto{\pgfqpoint{2.499292in}{2.167173in}}%
\pgfpathlineto{\pgfqpoint{2.499496in}{2.233186in}}%
\pgfpathlineto{\pgfqpoint{2.499700in}{2.134167in}}%
\pgfpathlineto{\pgfqpoint{2.499905in}{2.191928in}}%
\pgfpathlineto{\pgfqpoint{2.501129in}{2.018644in}}%
\pgfpathlineto{\pgfqpoint{2.502150in}{2.158922in}}%
\pgfpathlineto{\pgfqpoint{2.502354in}{2.125915in}}%
\pgfpathlineto{\pgfqpoint{2.502763in}{2.150670in}}%
\pgfpathlineto{\pgfqpoint{2.502967in}{2.200180in}}%
\pgfpathlineto{\pgfqpoint{2.503783in}{2.134167in}}%
\pgfpathlineto{\pgfqpoint{2.505416in}{2.241438in}}%
\pgfpathlineto{\pgfqpoint{2.505825in}{2.150670in}}%
\pgfpathlineto{\pgfqpoint{2.506437in}{2.233186in}}%
\pgfpathlineto{\pgfqpoint{2.506641in}{2.249689in}}%
\pgfpathlineto{\pgfqpoint{2.506845in}{2.216683in}}%
\pgfpathlineto{\pgfqpoint{2.507662in}{2.167173in}}%
\pgfpathlineto{\pgfqpoint{2.507866in}{2.175425in}}%
\pgfpathlineto{\pgfqpoint{2.508070in}{2.208431in}}%
\pgfpathlineto{\pgfqpoint{2.508683in}{2.158922in}}%
\pgfpathlineto{\pgfqpoint{2.509908in}{2.026896in}}%
\pgfpathlineto{\pgfqpoint{2.510112in}{2.035148in}}%
\pgfpathlineto{\pgfqpoint{2.510520in}{2.150670in}}%
\pgfpathlineto{\pgfqpoint{2.510928in}{2.026896in}}%
\pgfpathlineto{\pgfqpoint{2.511132in}{2.051651in}}%
\pgfpathlineto{\pgfqpoint{2.511337in}{2.002141in}}%
\pgfpathlineto{\pgfqpoint{2.512153in}{2.068154in}}%
\pgfpathlineto{\pgfqpoint{2.512357in}{2.068154in}}%
\pgfpathlineto{\pgfqpoint{2.512561in}{2.026896in}}%
\pgfpathlineto{\pgfqpoint{2.512970in}{2.084657in}}%
\pgfpathlineto{\pgfqpoint{2.513174in}{2.175425in}}%
\pgfpathlineto{\pgfqpoint{2.513786in}{1.977386in}}%
\pgfpathlineto{\pgfqpoint{2.514399in}{2.142418in}}%
\pgfpathlineto{\pgfqpoint{2.514807in}{1.977386in}}%
\pgfpathlineto{\pgfqpoint{2.515011in}{2.002141in}}%
\pgfpathlineto{\pgfqpoint{2.515215in}{1.944380in}}%
\pgfpathlineto{\pgfqpoint{2.515828in}{1.977386in}}%
\pgfpathlineto{\pgfqpoint{2.517257in}{1.828857in}}%
\pgfpathlineto{\pgfqpoint{2.516236in}{1.985638in}}%
\pgfpathlineto{\pgfqpoint{2.517461in}{1.853612in}}%
\pgfpathlineto{\pgfqpoint{2.517665in}{1.903122in}}%
\pgfpathlineto{\pgfqpoint{2.518073in}{1.845361in}}%
\pgfpathlineto{\pgfqpoint{2.518277in}{1.639071in}}%
\pgfpathlineto{\pgfqpoint{2.519094in}{1.903122in}}%
\pgfpathlineto{\pgfqpoint{2.519706in}{1.944380in}}%
\pgfpathlineto{\pgfqpoint{2.519502in}{1.878367in}}%
\pgfpathlineto{\pgfqpoint{2.519910in}{1.894870in}}%
\pgfpathlineto{\pgfqpoint{2.521135in}{1.746341in}}%
\pgfpathlineto{\pgfqpoint{2.522156in}{1.936128in}}%
\pgfpathlineto{\pgfqpoint{2.522360in}{1.894870in}}%
\pgfpathlineto{\pgfqpoint{2.523177in}{1.746341in}}%
\pgfpathlineto{\pgfqpoint{2.523381in}{1.771096in}}%
\pgfpathlineto{\pgfqpoint{2.523993in}{1.927877in}}%
\pgfpathlineto{\pgfqpoint{2.524606in}{1.886619in}}%
\pgfpathlineto{\pgfqpoint{2.526035in}{1.993890in}}%
\pgfpathlineto{\pgfqpoint{2.526239in}{2.018644in}}%
\pgfpathlineto{\pgfqpoint{2.526647in}{1.977386in}}%
\pgfpathlineto{\pgfqpoint{2.527055in}{2.010393in}}%
\pgfpathlineto{\pgfqpoint{2.527668in}{1.919625in}}%
\pgfpathlineto{\pgfqpoint{2.528893in}{2.051651in}}%
\pgfpathlineto{\pgfqpoint{2.529097in}{2.043399in}}%
\pgfpathlineto{\pgfqpoint{2.529505in}{2.092909in}}%
\pgfpathlineto{\pgfqpoint{2.529913in}{2.035148in}}%
\pgfpathlineto{\pgfqpoint{2.530118in}{2.084657in}}%
\pgfpathlineto{\pgfqpoint{2.530526in}{2.035148in}}%
\pgfpathlineto{\pgfqpoint{2.530934in}{2.134167in}}%
\pgfpathlineto{\pgfqpoint{2.531955in}{2.076406in}}%
\pgfpathlineto{\pgfqpoint{2.532363in}{2.101160in}}%
\pgfpathlineto{\pgfqpoint{2.532567in}{2.125915in}}%
\pgfpathlineto{\pgfqpoint{2.532976in}{2.059902in}}%
\pgfpathlineto{\pgfqpoint{2.533180in}{2.068154in}}%
\pgfpathlineto{\pgfqpoint{2.533384in}{2.068154in}}%
\pgfpathlineto{\pgfqpoint{2.533792in}{2.117664in}}%
\pgfpathlineto{\pgfqpoint{2.534405in}{2.084657in}}%
\pgfpathlineto{\pgfqpoint{2.535017in}{2.002141in}}%
\pgfpathlineto{\pgfqpoint{2.535629in}{2.010393in}}%
\pgfpathlineto{\pgfqpoint{2.535834in}{2.068154in}}%
\pgfpathlineto{\pgfqpoint{2.536650in}{1.993890in}}%
\pgfpathlineto{\pgfqpoint{2.536854in}{2.043399in}}%
\pgfpathlineto{\pgfqpoint{2.537058in}{2.018644in}}%
\pgfpathlineto{\pgfqpoint{2.537467in}{2.084657in}}%
\pgfpathlineto{\pgfqpoint{2.537671in}{2.068154in}}%
\pgfpathlineto{\pgfqpoint{2.537875in}{2.076406in}}%
\pgfpathlineto{\pgfqpoint{2.538487in}{2.084657in}}%
\pgfpathlineto{\pgfqpoint{2.539304in}{1.977386in}}%
\pgfpathlineto{\pgfqpoint{2.539508in}{1.977386in}}%
\pgfpathlineto{\pgfqpoint{2.540121in}{1.936128in}}%
\pgfpathlineto{\pgfqpoint{2.540325in}{1.969135in}}%
\pgfpathlineto{\pgfqpoint{2.540529in}{2.010393in}}%
\pgfpathlineto{\pgfqpoint{2.540937in}{1.919625in}}%
\pgfpathlineto{\pgfqpoint{2.541345in}{1.960883in}}%
\pgfpathlineto{\pgfqpoint{2.542366in}{1.894870in}}%
\pgfpathlineto{\pgfqpoint{2.542162in}{1.969135in}}%
\pgfpathlineto{\pgfqpoint{2.542774in}{1.919625in}}%
\pgfpathlineto{\pgfqpoint{2.542979in}{1.919625in}}%
\pgfpathlineto{\pgfqpoint{2.544408in}{1.787599in}}%
\pgfpathlineto{\pgfqpoint{2.543387in}{1.936128in}}%
\pgfpathlineto{\pgfqpoint{2.544612in}{1.837109in}}%
\pgfpathlineto{\pgfqpoint{2.545020in}{1.870115in}}%
\pgfpathlineto{\pgfqpoint{2.545224in}{1.845361in}}%
\pgfpathlineto{\pgfqpoint{2.546245in}{1.729838in}}%
\pgfpathlineto{\pgfqpoint{2.546449in}{1.787599in}}%
\pgfpathlineto{\pgfqpoint{2.546653in}{1.787599in}}%
\pgfpathlineto{\pgfqpoint{2.548082in}{1.903122in}}%
\pgfpathlineto{\pgfqpoint{2.548286in}{1.812354in}}%
\pgfpathlineto{\pgfqpoint{2.548899in}{1.927877in}}%
\pgfpathlineto{\pgfqpoint{2.549103in}{1.870115in}}%
\pgfpathlineto{\pgfqpoint{2.549511in}{1.828857in}}%
\pgfpathlineto{\pgfqpoint{2.549919in}{1.853612in}}%
\pgfpathlineto{\pgfqpoint{2.550736in}{1.721587in}}%
\pgfpathlineto{\pgfqpoint{2.550940in}{1.828857in}}%
\pgfpathlineto{\pgfqpoint{2.551144in}{1.903122in}}%
\pgfpathlineto{\pgfqpoint{2.551757in}{1.746341in}}%
\pgfpathlineto{\pgfqpoint{2.552165in}{1.540051in}}%
\pgfpathlineto{\pgfqpoint{2.552573in}{1.606064in}}%
\pgfpathlineto{\pgfqpoint{2.552982in}{1.845361in}}%
\pgfpathlineto{\pgfqpoint{2.553798in}{1.804103in}}%
\pgfpathlineto{\pgfqpoint{2.554819in}{1.746341in}}%
\pgfpathlineto{\pgfqpoint{2.555431in}{1.853612in}}%
\pgfpathlineto{\pgfqpoint{2.556044in}{1.845361in}}%
\pgfpathlineto{\pgfqpoint{2.557881in}{1.688580in}}%
\pgfpathlineto{\pgfqpoint{2.556656in}{1.861864in}}%
\pgfpathlineto{\pgfqpoint{2.558085in}{1.729838in}}%
\pgfpathlineto{\pgfqpoint{2.558493in}{1.696832in}}%
\pgfpathlineto{\pgfqpoint{2.558902in}{1.746341in}}%
\pgfpathlineto{\pgfqpoint{2.559310in}{1.705083in}}%
\pgfpathlineto{\pgfqpoint{2.559514in}{1.729838in}}%
\pgfpathlineto{\pgfqpoint{2.560127in}{1.713335in}}%
\pgfpathlineto{\pgfqpoint{2.560331in}{1.655574in}}%
\pgfpathlineto{\pgfqpoint{2.560943in}{1.779348in}}%
\pgfpathlineto{\pgfqpoint{2.561147in}{1.738090in}}%
\pgfpathlineto{\pgfqpoint{2.561351in}{1.738090in}}%
\pgfpathlineto{\pgfqpoint{2.561556in}{1.771096in}}%
\pgfpathlineto{\pgfqpoint{2.562168in}{1.713335in}}%
\pgfpathlineto{\pgfqpoint{2.562372in}{1.738090in}}%
\pgfpathlineto{\pgfqpoint{2.563393in}{1.870115in}}%
\pgfpathlineto{\pgfqpoint{2.563597in}{1.795851in}}%
\pgfpathlineto{\pgfqpoint{2.563801in}{1.754593in}}%
\pgfpathlineto{\pgfqpoint{2.564414in}{1.812354in}}%
\pgfpathlineto{\pgfqpoint{2.564618in}{1.804103in}}%
\pgfpathlineto{\pgfqpoint{2.566455in}{1.993890in}}%
\pgfpathlineto{\pgfqpoint{2.566863in}{1.746341in}}%
\pgfpathlineto{\pgfqpoint{2.567476in}{2.002141in}}%
\pgfpathlineto{\pgfqpoint{2.567680in}{1.853612in}}%
\pgfpathlineto{\pgfqpoint{2.568088in}{1.985638in}}%
\pgfpathlineto{\pgfqpoint{2.568496in}{1.969135in}}%
\pgfpathlineto{\pgfqpoint{2.568701in}{1.746341in}}%
\pgfpathlineto{\pgfqpoint{2.569313in}{1.977386in}}%
\pgfpathlineto{\pgfqpoint{2.569721in}{1.795851in}}%
\pgfpathlineto{\pgfqpoint{2.570334in}{2.018644in}}%
\pgfpathlineto{\pgfqpoint{2.570946in}{1.936128in}}%
\pgfpathlineto{\pgfqpoint{2.571150in}{1.927877in}}%
\pgfpathlineto{\pgfqpoint{2.571354in}{1.944380in}}%
\pgfpathlineto{\pgfqpoint{2.571559in}{1.936128in}}%
\pgfpathlineto{\pgfqpoint{2.572171in}{2.018644in}}%
\pgfpathlineto{\pgfqpoint{2.572375in}{2.010393in}}%
\pgfpathlineto{\pgfqpoint{2.572988in}{1.853612in}}%
\pgfpathlineto{\pgfqpoint{2.573600in}{1.870115in}}%
\pgfpathlineto{\pgfqpoint{2.574212in}{1.886619in}}%
\pgfpathlineto{\pgfqpoint{2.574621in}{1.804103in}}%
\pgfpathlineto{\pgfqpoint{2.576050in}{1.944380in}}%
\pgfpathlineto{\pgfqpoint{2.577479in}{1.853612in}}%
\pgfpathlineto{\pgfqpoint{2.578091in}{1.903122in}}%
\pgfpathlineto{\pgfqpoint{2.578499in}{1.845361in}}%
\pgfpathlineto{\pgfqpoint{2.578704in}{1.630819in}}%
\pgfpathlineto{\pgfqpoint{2.579316in}{2.002141in}}%
\pgfpathlineto{\pgfqpoint{2.579724in}{1.705083in}}%
\pgfpathlineto{\pgfqpoint{2.580337in}{2.068154in}}%
\pgfpathlineto{\pgfqpoint{2.581153in}{1.977386in}}%
\pgfpathlineto{\pgfqpoint{2.581970in}{1.787599in}}%
\pgfpathlineto{\pgfqpoint{2.581766in}{2.035148in}}%
\pgfpathlineto{\pgfqpoint{2.582174in}{1.936128in}}%
\pgfpathlineto{\pgfqpoint{2.582378in}{1.960883in}}%
\pgfpathlineto{\pgfqpoint{2.582786in}{1.944380in}}%
\pgfpathlineto{\pgfqpoint{2.584215in}{1.771096in}}%
\pgfpathlineto{\pgfqpoint{2.585440in}{1.837109in}}%
\pgfpathlineto{\pgfqpoint{2.585644in}{1.853612in}}%
\pgfpathlineto{\pgfqpoint{2.585849in}{1.812354in}}%
\pgfpathlineto{\pgfqpoint{2.587073in}{1.606064in}}%
\pgfpathlineto{\pgfqpoint{2.587278in}{1.614316in}}%
\pgfpathlineto{\pgfqpoint{2.587686in}{1.878367in}}%
\pgfpathlineto{\pgfqpoint{2.588502in}{1.812354in}}%
\pgfpathlineto{\pgfqpoint{2.588911in}{1.837109in}}%
\pgfpathlineto{\pgfqpoint{2.589115in}{1.795851in}}%
\pgfpathlineto{\pgfqpoint{2.589523in}{1.812354in}}%
\pgfpathlineto{\pgfqpoint{2.589931in}{1.771096in}}%
\pgfpathlineto{\pgfqpoint{2.590136in}{1.820606in}}%
\pgfpathlineto{\pgfqpoint{2.590340in}{1.779348in}}%
\pgfpathlineto{\pgfqpoint{2.591360in}{1.886619in}}%
\pgfpathlineto{\pgfqpoint{2.591565in}{1.804103in}}%
\pgfpathlineto{\pgfqpoint{2.592585in}{1.820606in}}%
\pgfpathlineto{\pgfqpoint{2.592789in}{1.861864in}}%
\pgfpathlineto{\pgfqpoint{2.592994in}{1.804103in}}%
\pgfpathlineto{\pgfqpoint{2.593198in}{1.804103in}}%
\pgfpathlineto{\pgfqpoint{2.593606in}{1.696832in}}%
\pgfpathlineto{\pgfqpoint{2.594423in}{1.705083in}}%
\pgfpathlineto{\pgfqpoint{2.595239in}{1.911373in}}%
\pgfpathlineto{\pgfqpoint{2.595647in}{1.820606in}}%
\pgfpathlineto{\pgfqpoint{2.595852in}{1.804103in}}%
\pgfpathlineto{\pgfqpoint{2.596260in}{1.853612in}}%
\pgfpathlineto{\pgfqpoint{2.596464in}{1.845361in}}%
\pgfpathlineto{\pgfqpoint{2.596668in}{1.870115in}}%
\pgfpathlineto{\pgfqpoint{2.596872in}{1.886619in}}%
\pgfpathlineto{\pgfqpoint{2.597076in}{1.837109in}}%
\pgfpathlineto{\pgfqpoint{2.597689in}{1.614316in}}%
\pgfpathlineto{\pgfqpoint{2.597893in}{1.853612in}}%
\pgfpathlineto{\pgfqpoint{2.598097in}{1.837109in}}%
\pgfpathlineto{\pgfqpoint{2.598301in}{1.820606in}}%
\pgfpathlineto{\pgfqpoint{2.598505in}{1.861864in}}%
\pgfpathlineto{\pgfqpoint{2.599934in}{1.985638in}}%
\pgfpathlineto{\pgfqpoint{2.600751in}{1.919625in}}%
\pgfpathlineto{\pgfqpoint{2.601159in}{1.927877in}}%
\pgfpathlineto{\pgfqpoint{2.603201in}{1.754593in}}%
\pgfpathlineto{\pgfqpoint{2.603405in}{1.828857in}}%
\pgfpathlineto{\pgfqpoint{2.603609in}{1.639071in}}%
\pgfpathlineto{\pgfqpoint{2.604221in}{1.936128in}}%
\pgfpathlineto{\pgfqpoint{2.604425in}{1.911373in}}%
\pgfpathlineto{\pgfqpoint{2.604834in}{1.927877in}}%
\pgfpathlineto{\pgfqpoint{2.605242in}{1.861864in}}%
\pgfpathlineto{\pgfqpoint{2.605446in}{1.919625in}}%
\pgfpathlineto{\pgfqpoint{2.605854in}{1.820606in}}%
\pgfpathlineto{\pgfqpoint{2.606263in}{1.870115in}}%
\pgfpathlineto{\pgfqpoint{2.607488in}{1.779348in}}%
\pgfpathlineto{\pgfqpoint{2.607692in}{1.886619in}}%
\pgfpathlineto{\pgfqpoint{2.608712in}{1.853612in}}%
\pgfpathlineto{\pgfqpoint{2.608917in}{1.696832in}}%
\pgfpathlineto{\pgfqpoint{2.609529in}{1.820606in}}%
\pgfpathlineto{\pgfqpoint{2.609733in}{1.985638in}}%
\pgfpathlineto{\pgfqpoint{2.610550in}{1.870115in}}%
\pgfpathlineto{\pgfqpoint{2.611775in}{2.018644in}}%
\pgfpathlineto{\pgfqpoint{2.612183in}{1.960883in}}%
\pgfpathlineto{\pgfqpoint{2.613408in}{1.870115in}}%
\pgfpathlineto{\pgfqpoint{2.613816in}{1.861864in}}%
\pgfpathlineto{\pgfqpoint{2.614224in}{1.911373in}}%
\pgfpathlineto{\pgfqpoint{2.614428in}{1.795851in}}%
\pgfpathlineto{\pgfqpoint{2.615449in}{1.837109in}}%
\pgfpathlineto{\pgfqpoint{2.615653in}{1.936128in}}%
\pgfpathlineto{\pgfqpoint{2.616266in}{1.746341in}}%
\pgfpathlineto{\pgfqpoint{2.616470in}{1.696832in}}%
\pgfpathlineto{\pgfqpoint{2.617286in}{1.721587in}}%
\pgfpathlineto{\pgfqpoint{2.618103in}{1.762845in}}%
\pgfpathlineto{\pgfqpoint{2.618715in}{1.408026in}}%
\pgfpathlineto{\pgfqpoint{2.619328in}{1.606064in}}%
\pgfpathlineto{\pgfqpoint{2.620757in}{1.812354in}}%
\pgfpathlineto{\pgfqpoint{2.621165in}{1.870115in}}%
\pgfpathlineto{\pgfqpoint{2.621573in}{1.771096in}}%
\pgfpathlineto{\pgfqpoint{2.622594in}{1.713335in}}%
\pgfpathlineto{\pgfqpoint{2.622798in}{1.721587in}}%
\pgfpathlineto{\pgfqpoint{2.623002in}{1.564806in}}%
\pgfpathlineto{\pgfqpoint{2.623207in}{1.762845in}}%
\pgfpathlineto{\pgfqpoint{2.623819in}{1.696832in}}%
\pgfpathlineto{\pgfqpoint{2.624227in}{1.581309in}}%
\pgfpathlineto{\pgfqpoint{2.624636in}{1.630819in}}%
\pgfpathlineto{\pgfqpoint{2.624840in}{1.350264in}}%
\pgfpathlineto{\pgfqpoint{2.625452in}{1.713335in}}%
\pgfpathlineto{\pgfqpoint{2.625656in}{1.762845in}}%
\pgfpathlineto{\pgfqpoint{2.625860in}{1.663825in}}%
\pgfpathlineto{\pgfqpoint{2.626269in}{1.672077in}}%
\pgfpathlineto{\pgfqpoint{2.626473in}{1.680329in}}%
\pgfpathlineto{\pgfqpoint{2.626881in}{1.663825in}}%
\pgfpathlineto{\pgfqpoint{2.627085in}{1.655574in}}%
\pgfpathlineto{\pgfqpoint{2.627289in}{1.672077in}}%
\pgfpathlineto{\pgfqpoint{2.627494in}{1.663825in}}%
\pgfpathlineto{\pgfqpoint{2.628310in}{1.391522in}}%
\pgfpathlineto{\pgfqpoint{2.628514in}{1.457535in}}%
\pgfpathlineto{\pgfqpoint{2.629739in}{1.738090in}}%
\pgfpathlineto{\pgfqpoint{2.629943in}{1.729838in}}%
\pgfpathlineto{\pgfqpoint{2.630147in}{1.762845in}}%
\pgfpathlineto{\pgfqpoint{2.630352in}{1.762845in}}%
\pgfpathlineto{\pgfqpoint{2.630964in}{1.738090in}}%
\pgfpathlineto{\pgfqpoint{2.631372in}{1.762845in}}%
\pgfpathlineto{\pgfqpoint{2.632189in}{1.845361in}}%
\pgfpathlineto{\pgfqpoint{2.632393in}{1.721587in}}%
\pgfpathlineto{\pgfqpoint{2.633210in}{1.870115in}}%
\pgfpathlineto{\pgfqpoint{2.634230in}{1.878367in}}%
\pgfpathlineto{\pgfqpoint{2.634434in}{1.746341in}}%
\pgfpathlineto{\pgfqpoint{2.634843in}{1.936128in}}%
\pgfpathlineto{\pgfqpoint{2.635047in}{1.721587in}}%
\pgfpathlineto{\pgfqpoint{2.635455in}{1.828857in}}%
\pgfpathlineto{\pgfqpoint{2.636884in}{1.647322in}}%
\pgfpathlineto{\pgfqpoint{2.637088in}{1.696832in}}%
\pgfpathlineto{\pgfqpoint{2.637497in}{1.762845in}}%
\pgfpathlineto{\pgfqpoint{2.637905in}{1.680329in}}%
\pgfpathlineto{\pgfqpoint{2.638109in}{1.713335in}}%
\pgfpathlineto{\pgfqpoint{2.639130in}{1.597813in}}%
\pgfpathlineto{\pgfqpoint{2.639742in}{1.622567in}}%
\pgfpathlineto{\pgfqpoint{2.640763in}{1.589561in}}%
\pgfpathlineto{\pgfqpoint{2.640967in}{1.597813in}}%
\pgfpathlineto{\pgfqpoint{2.641784in}{1.556554in}}%
\pgfpathlineto{\pgfqpoint{2.642192in}{1.680329in}}%
\pgfpathlineto{\pgfqpoint{2.642396in}{1.597813in}}%
\pgfpathlineto{\pgfqpoint{2.643417in}{1.622567in}}%
\pgfpathlineto{\pgfqpoint{2.644029in}{1.606064in}}%
\pgfpathlineto{\pgfqpoint{2.644642in}{1.696832in}}%
\pgfpathlineto{\pgfqpoint{2.644846in}{1.663825in}}%
\pgfpathlineto{\pgfqpoint{2.645050in}{1.696832in}}%
\pgfpathlineto{\pgfqpoint{2.645254in}{1.787599in}}%
\pgfpathlineto{\pgfqpoint{2.645866in}{1.738090in}}%
\pgfpathlineto{\pgfqpoint{2.646275in}{1.639071in}}%
\pgfpathlineto{\pgfqpoint{2.647091in}{1.647322in}}%
\pgfpathlineto{\pgfqpoint{2.648520in}{1.540051in}}%
\pgfpathlineto{\pgfqpoint{2.649745in}{1.606064in}}%
\pgfpathlineto{\pgfqpoint{2.649949in}{1.606064in}}%
\pgfpathlineto{\pgfqpoint{2.650562in}{1.680329in}}%
\pgfpathlineto{\pgfqpoint{2.651174in}{1.639071in}}%
\pgfpathlineto{\pgfqpoint{2.651991in}{1.573058in}}%
\pgfpathlineto{\pgfqpoint{2.652195in}{1.647322in}}%
\pgfpathlineto{\pgfqpoint{2.652603in}{1.655574in}}%
\pgfpathlineto{\pgfqpoint{2.653011in}{1.573058in}}%
\pgfpathlineto{\pgfqpoint{2.654236in}{1.762845in}}%
\pgfpathlineto{\pgfqpoint{2.655869in}{1.209987in}}%
\pgfpathlineto{\pgfqpoint{2.656278in}{1.234742in}}%
\pgfpathlineto{\pgfqpoint{2.656890in}{1.457535in}}%
\pgfpathlineto{\pgfqpoint{2.657298in}{1.342013in}}%
\pgfpathlineto{\pgfqpoint{2.657707in}{1.284252in}}%
\pgfpathlineto{\pgfqpoint{2.658115in}{1.375019in}}%
\pgfpathlineto{\pgfqpoint{2.658932in}{1.507045in}}%
\pgfpathlineto{\pgfqpoint{2.659340in}{1.474038in}}%
\pgfpathlineto{\pgfqpoint{2.659748in}{1.309006in}}%
\pgfpathlineto{\pgfqpoint{2.660361in}{1.358516in}}%
\pgfpathlineto{\pgfqpoint{2.660565in}{1.465787in}}%
\pgfpathlineto{\pgfqpoint{2.661177in}{1.226490in}}%
\pgfpathlineto{\pgfqpoint{2.661994in}{1.259497in}}%
\pgfpathlineto{\pgfqpoint{2.662402in}{1.094465in}}%
\pgfpathlineto{\pgfqpoint{2.662810in}{1.251245in}}%
\pgfpathlineto{\pgfqpoint{2.663423in}{1.077961in}}%
\pgfpathlineto{\pgfqpoint{2.663627in}{1.176981in}}%
\pgfpathlineto{\pgfqpoint{2.664648in}{1.094465in}}%
\pgfpathlineto{\pgfqpoint{2.664852in}{1.135723in}}%
\pgfpathlineto{\pgfqpoint{2.665056in}{1.226490in}}%
\pgfpathlineto{\pgfqpoint{2.665872in}{1.168729in}}%
\pgfpathlineto{\pgfqpoint{2.666281in}{1.160477in}}%
\pgfpathlineto{\pgfqpoint{2.666485in}{1.193484in}}%
\pgfpathlineto{\pgfqpoint{2.667097in}{1.300755in}}%
\pgfpathlineto{\pgfqpoint{2.667301in}{1.242994in}}%
\pgfpathlineto{\pgfqpoint{2.668526in}{1.086213in}}%
\pgfpathlineto{\pgfqpoint{2.668730in}{1.094465in}}%
\pgfpathlineto{\pgfqpoint{2.669139in}{0.995445in}}%
\pgfpathlineto{\pgfqpoint{2.669343in}{1.069710in}}%
\pgfpathlineto{\pgfqpoint{2.669751in}{1.193484in}}%
\pgfpathlineto{\pgfqpoint{2.670159in}{1.036703in}}%
\pgfpathlineto{\pgfqpoint{2.670364in}{0.987194in}}%
\pgfpathlineto{\pgfqpoint{2.670772in}{1.077961in}}%
\pgfpathlineto{\pgfqpoint{2.670976in}{1.160477in}}%
\pgfpathlineto{\pgfqpoint{2.671384in}{1.061458in}}%
\pgfpathlineto{\pgfqpoint{2.671793in}{1.094465in}}%
\pgfpathlineto{\pgfqpoint{2.672201in}{1.044955in}}%
\pgfpathlineto{\pgfqpoint{2.672405in}{1.086213in}}%
\pgfpathlineto{\pgfqpoint{2.672609in}{1.011949in}}%
\pgfpathlineto{\pgfqpoint{2.673017in}{1.135723in}}%
\pgfpathlineto{\pgfqpoint{2.673222in}{1.110968in}}%
\pgfpathlineto{\pgfqpoint{2.673834in}{1.218239in}}%
\pgfpathlineto{\pgfqpoint{2.674446in}{1.160477in}}%
\pgfpathlineto{\pgfqpoint{2.674855in}{1.176981in}}%
\pgfpathlineto{\pgfqpoint{2.675263in}{1.135723in}}%
\pgfpathlineto{\pgfqpoint{2.675875in}{1.152226in}}%
\pgfpathlineto{\pgfqpoint{2.676080in}{1.160477in}}%
\pgfpathlineto{\pgfqpoint{2.676284in}{1.102716in}}%
\pgfpathlineto{\pgfqpoint{2.676896in}{1.168729in}}%
\pgfpathlineto{\pgfqpoint{2.677100in}{1.152226in}}%
\pgfpathlineto{\pgfqpoint{2.679958in}{1.441032in}}%
\pgfpathlineto{\pgfqpoint{2.680367in}{1.416277in}}%
\pgfpathlineto{\pgfqpoint{2.680775in}{1.284252in}}%
\pgfpathlineto{\pgfqpoint{2.681591in}{1.309006in}}%
\pgfpathlineto{\pgfqpoint{2.681796in}{1.375019in}}%
\pgfpathlineto{\pgfqpoint{2.682204in}{1.242994in}}%
\pgfpathlineto{\pgfqpoint{2.682408in}{1.242994in}}%
\pgfpathlineto{\pgfqpoint{2.684245in}{1.523548in}}%
\pgfpathlineto{\pgfqpoint{2.682816in}{1.226490in}}%
\pgfpathlineto{\pgfqpoint{2.684653in}{1.383271in}}%
\pgfpathlineto{\pgfqpoint{2.685878in}{1.251245in}}%
\pgfpathlineto{\pgfqpoint{2.687511in}{1.424529in}}%
\pgfpathlineto{\pgfqpoint{2.688940in}{1.135723in}}%
\pgfpathlineto{\pgfqpoint{2.689349in}{1.143974in}}%
\pgfpathlineto{\pgfqpoint{2.689757in}{1.110968in}}%
\pgfpathlineto{\pgfqpoint{2.690574in}{1.242994in}}%
\pgfpathlineto{\pgfqpoint{2.690778in}{1.185232in}}%
\pgfpathlineto{\pgfqpoint{2.690982in}{1.135723in}}%
\pgfpathlineto{\pgfqpoint{2.691186in}{1.300755in}}%
\pgfpathlineto{\pgfqpoint{2.691798in}{1.474038in}}%
\pgfpathlineto{\pgfqpoint{2.692207in}{1.284252in}}%
\pgfpathlineto{\pgfqpoint{2.692411in}{1.284252in}}%
\pgfpathlineto{\pgfqpoint{2.693023in}{1.358516in}}%
\pgfpathlineto{\pgfqpoint{2.693432in}{1.292503in}}%
\pgfpathlineto{\pgfqpoint{2.694248in}{1.325510in}}%
\pgfpathlineto{\pgfqpoint{2.694656in}{1.201736in}}%
\pgfpathlineto{\pgfqpoint{2.694861in}{1.201736in}}%
\pgfpathlineto{\pgfqpoint{2.695065in}{1.003697in}}%
\pgfpathlineto{\pgfqpoint{2.695269in}{1.209987in}}%
\pgfpathlineto{\pgfqpoint{2.696085in}{1.077961in}}%
\pgfpathlineto{\pgfqpoint{2.696698in}{1.317258in}}%
\pgfpathlineto{\pgfqpoint{2.697310in}{1.185232in}}%
\pgfpathlineto{\pgfqpoint{2.698331in}{1.110968in}}%
\pgfpathlineto{\pgfqpoint{2.699556in}{1.267748in}}%
\pgfpathlineto{\pgfqpoint{2.699760in}{1.251245in}}%
\pgfpathlineto{\pgfqpoint{2.700372in}{1.259497in}}%
\pgfpathlineto{\pgfqpoint{2.702414in}{1.589561in}}%
\pgfpathlineto{\pgfqpoint{2.702618in}{1.597813in}}%
\pgfpathlineto{\pgfqpoint{2.702822in}{1.564806in}}%
\pgfpathlineto{\pgfqpoint{2.704251in}{1.424529in}}%
\pgfpathlineto{\pgfqpoint{2.704455in}{1.432780in}}%
\pgfpathlineto{\pgfqpoint{2.706497in}{1.589561in}}%
\pgfpathlineto{\pgfqpoint{2.704864in}{1.424529in}}%
\pgfpathlineto{\pgfqpoint{2.706701in}{1.564806in}}%
\pgfpathlineto{\pgfqpoint{2.706905in}{1.540051in}}%
\pgfpathlineto{\pgfqpoint{2.707313in}{1.597813in}}%
\pgfpathlineto{\pgfqpoint{2.707517in}{1.581309in}}%
\pgfpathlineto{\pgfqpoint{2.707722in}{1.630819in}}%
\pgfpathlineto{\pgfqpoint{2.708130in}{1.540051in}}%
\pgfpathlineto{\pgfqpoint{2.708334in}{1.556554in}}%
\pgfpathlineto{\pgfqpoint{2.708538in}{1.531800in}}%
\pgfpathlineto{\pgfqpoint{2.708742in}{1.556554in}}%
\pgfpathlineto{\pgfqpoint{2.709355in}{1.663825in}}%
\pgfpathlineto{\pgfqpoint{2.709967in}{1.647322in}}%
\pgfpathlineto{\pgfqpoint{2.710171in}{1.655574in}}%
\pgfpathlineto{\pgfqpoint{2.710375in}{1.647322in}}%
\pgfpathlineto{\pgfqpoint{2.710784in}{1.606064in}}%
\pgfpathlineto{\pgfqpoint{2.710988in}{1.696832in}}%
\pgfpathlineto{\pgfqpoint{2.711192in}{1.746341in}}%
\pgfpathlineto{\pgfqpoint{2.711396in}{1.647322in}}%
\pgfpathlineto{\pgfqpoint{2.711600in}{1.647322in}}%
\pgfpathlineto{\pgfqpoint{2.712009in}{1.606064in}}%
\pgfpathlineto{\pgfqpoint{2.712417in}{1.647322in}}%
\pgfpathlineto{\pgfqpoint{2.713642in}{1.721587in}}%
\pgfpathlineto{\pgfqpoint{2.713846in}{1.680329in}}%
\pgfpathlineto{\pgfqpoint{2.714050in}{1.655574in}}%
\pgfpathlineto{\pgfqpoint{2.714662in}{1.713335in}}%
\pgfpathlineto{\pgfqpoint{2.715683in}{1.762845in}}%
\pgfpathlineto{\pgfqpoint{2.715887in}{1.754593in}}%
\pgfpathlineto{\pgfqpoint{2.716296in}{1.812354in}}%
\pgfpathlineto{\pgfqpoint{2.716704in}{1.787599in}}%
\pgfpathlineto{\pgfqpoint{2.716908in}{1.713335in}}%
\pgfpathlineto{\pgfqpoint{2.717520in}{1.886619in}}%
\pgfpathlineto{\pgfqpoint{2.717929in}{1.894870in}}%
\pgfpathlineto{\pgfqpoint{2.718133in}{1.861864in}}%
\pgfpathlineto{\pgfqpoint{2.718949in}{1.944380in}}%
\pgfpathlineto{\pgfqpoint{2.719154in}{1.894870in}}%
\pgfpathlineto{\pgfqpoint{2.719766in}{1.936128in}}%
\pgfpathlineto{\pgfqpoint{2.720174in}{1.870115in}}%
\pgfpathlineto{\pgfqpoint{2.721399in}{2.035148in}}%
\pgfpathlineto{\pgfqpoint{2.722216in}{1.985638in}}%
\pgfpathlineto{\pgfqpoint{2.721807in}{2.043399in}}%
\pgfpathlineto{\pgfqpoint{2.722420in}{2.010393in}}%
\pgfpathlineto{\pgfqpoint{2.724257in}{2.142418in}}%
\pgfpathlineto{\pgfqpoint{2.724461in}{2.134167in}}%
\pgfpathlineto{\pgfqpoint{2.725074in}{2.109412in}}%
\pgfpathlineto{\pgfqpoint{2.724870in}{2.191928in}}%
\pgfpathlineto{\pgfqpoint{2.725278in}{2.167173in}}%
\pgfpathlineto{\pgfqpoint{2.726094in}{2.117664in}}%
\pgfpathlineto{\pgfqpoint{2.727319in}{2.249689in}}%
\pgfpathlineto{\pgfqpoint{2.727728in}{2.216683in}}%
\pgfpathlineto{\pgfqpoint{2.728748in}{2.092909in}}%
\pgfpathlineto{\pgfqpoint{2.729769in}{2.191928in}}%
\pgfpathlineto{\pgfqpoint{2.729973in}{2.175425in}}%
\pgfpathlineto{\pgfqpoint{2.730381in}{2.142418in}}%
\pgfpathlineto{\pgfqpoint{2.730586in}{2.191928in}}%
\pgfpathlineto{\pgfqpoint{2.730994in}{2.175425in}}%
\pgfpathlineto{\pgfqpoint{2.731198in}{2.249689in}}%
\pgfpathlineto{\pgfqpoint{2.731810in}{2.208431in}}%
\pgfpathlineto{\pgfqpoint{2.732015in}{2.125915in}}%
\pgfpathlineto{\pgfqpoint{2.732219in}{2.274444in}}%
\pgfpathlineto{\pgfqpoint{2.733035in}{2.142418in}}%
\pgfpathlineto{\pgfqpoint{2.733239in}{2.150670in}}%
\pgfpathlineto{\pgfqpoint{2.733444in}{2.142418in}}%
\pgfpathlineto{\pgfqpoint{2.733648in}{2.117664in}}%
\pgfpathlineto{\pgfqpoint{2.734056in}{2.158922in}}%
\pgfpathlineto{\pgfqpoint{2.734464in}{2.150670in}}%
\pgfpathlineto{\pgfqpoint{2.734668in}{2.216683in}}%
\pgfpathlineto{\pgfqpoint{2.735485in}{2.142418in}}%
\pgfpathlineto{\pgfqpoint{2.735689in}{2.134167in}}%
\pgfpathlineto{\pgfqpoint{2.736710in}{2.216683in}}%
\pgfpathlineto{\pgfqpoint{2.736914in}{1.993890in}}%
\pgfpathlineto{\pgfqpoint{2.737731in}{2.158922in}}%
\pgfpathlineto{\pgfqpoint{2.737935in}{2.158922in}}%
\pgfpathlineto{\pgfqpoint{2.738139in}{2.125915in}}%
\pgfpathlineto{\pgfqpoint{2.738751in}{2.158922in}}%
\pgfpathlineto{\pgfqpoint{2.740180in}{2.323954in}}%
\pgfpathlineto{\pgfqpoint{2.740384in}{2.290947in}}%
\pgfpathlineto{\pgfqpoint{2.740997in}{2.257941in}}%
\pgfpathlineto{\pgfqpoint{2.741201in}{2.290947in}}%
\pgfpathlineto{\pgfqpoint{2.741609in}{2.323954in}}%
\pgfpathlineto{\pgfqpoint{2.741813in}{2.233186in}}%
\pgfpathlineto{\pgfqpoint{2.742630in}{2.332205in}}%
\pgfpathlineto{\pgfqpoint{2.742834in}{2.241438in}}%
\pgfpathlineto{\pgfqpoint{2.744467in}{2.365212in}}%
\pgfpathlineto{\pgfqpoint{2.744876in}{2.381715in}}%
\pgfpathlineto{\pgfqpoint{2.745080in}{2.373463in}}%
\pgfpathlineto{\pgfqpoint{2.745488in}{2.150670in}}%
\pgfpathlineto{\pgfqpoint{2.746305in}{2.282696in}}%
\pgfpathlineto{\pgfqpoint{2.746509in}{2.315702in}}%
\pgfpathlineto{\pgfqpoint{2.746917in}{2.274444in}}%
\pgfpathlineto{\pgfqpoint{2.747325in}{1.985638in}}%
\pgfpathlineto{\pgfqpoint{2.747938in}{2.216683in}}%
\pgfpathlineto{\pgfqpoint{2.748754in}{2.348709in}}%
\pgfpathlineto{\pgfqpoint{2.748958in}{2.274444in}}%
\pgfpathlineto{\pgfqpoint{2.749367in}{2.092909in}}%
\pgfpathlineto{\pgfqpoint{2.750183in}{2.224934in}}%
\pgfpathlineto{\pgfqpoint{2.751408in}{2.365212in}}%
\pgfpathlineto{\pgfqpoint{2.752633in}{2.233186in}}%
\pgfpathlineto{\pgfqpoint{2.753245in}{2.332205in}}%
\pgfpathlineto{\pgfqpoint{2.753858in}{2.315702in}}%
\pgfpathlineto{\pgfqpoint{2.754470in}{2.233186in}}%
\pgfpathlineto{\pgfqpoint{2.755083in}{2.249689in}}%
\pgfpathlineto{\pgfqpoint{2.755695in}{2.348709in}}%
\pgfpathlineto{\pgfqpoint{2.756103in}{2.233186in}}%
\pgfpathlineto{\pgfqpoint{2.756716in}{2.307450in}}%
\pgfpathlineto{\pgfqpoint{2.756920in}{2.175425in}}%
\pgfpathlineto{\pgfqpoint{2.757532in}{2.323954in}}%
\pgfpathlineto{\pgfqpoint{2.757737in}{2.299199in}}%
\pgfpathlineto{\pgfqpoint{2.758145in}{2.299199in}}%
\pgfpathlineto{\pgfqpoint{2.758961in}{2.365212in}}%
\pgfpathlineto{\pgfqpoint{2.759574in}{2.340457in}}%
\pgfpathlineto{\pgfqpoint{2.760186in}{2.282696in}}%
\pgfpathlineto{\pgfqpoint{2.760390in}{2.323954in}}%
\pgfpathlineto{\pgfqpoint{2.761003in}{2.307450in}}%
\pgfpathlineto{\pgfqpoint{2.761615in}{2.406470in}}%
\pgfpathlineto{\pgfqpoint{2.763248in}{2.266192in}}%
\pgfpathlineto{\pgfqpoint{2.764065in}{2.365212in}}%
\pgfpathlineto{\pgfqpoint{2.764269in}{2.266192in}}%
\pgfpathlineto{\pgfqpoint{2.764473in}{2.274444in}}%
\pgfpathlineto{\pgfqpoint{2.764677in}{2.266192in}}%
\pgfpathlineto{\pgfqpoint{2.764882in}{2.241438in}}%
\pgfpathlineto{\pgfqpoint{2.765290in}{2.315702in}}%
\pgfpathlineto{\pgfqpoint{2.765902in}{2.356960in}}%
\pgfpathlineto{\pgfqpoint{2.766106in}{2.299199in}}%
\pgfpathlineto{\pgfqpoint{2.766311in}{2.323954in}}%
\pgfpathlineto{\pgfqpoint{2.766515in}{2.282696in}}%
\pgfpathlineto{\pgfqpoint{2.766923in}{2.373463in}}%
\pgfpathlineto{\pgfqpoint{2.767331in}{2.340457in}}%
\pgfpathlineto{\pgfqpoint{2.767535in}{2.389967in}}%
\pgfpathlineto{\pgfqpoint{2.768148in}{2.332205in}}%
\pgfpathlineto{\pgfqpoint{2.768760in}{2.282696in}}%
\pgfpathlineto{\pgfqpoint{2.769168in}{2.323954in}}%
\pgfpathlineto{\pgfqpoint{2.769373in}{2.348709in}}%
\pgfpathlineto{\pgfqpoint{2.769577in}{2.299199in}}%
\pgfpathlineto{\pgfqpoint{2.769781in}{2.307450in}}%
\pgfpathlineto{\pgfqpoint{2.770802in}{2.208431in}}%
\pgfpathlineto{\pgfqpoint{2.771414in}{2.224934in}}%
\pgfpathlineto{\pgfqpoint{2.772026in}{2.373463in}}%
\pgfpathlineto{\pgfqpoint{2.772843in}{2.332205in}}%
\pgfpathlineto{\pgfqpoint{2.773864in}{2.299199in}}%
\pgfpathlineto{\pgfqpoint{2.774068in}{2.307450in}}%
\pgfpathlineto{\pgfqpoint{2.774680in}{2.381715in}}%
\pgfpathlineto{\pgfqpoint{2.774476in}{2.299199in}}%
\pgfpathlineto{\pgfqpoint{2.775497in}{2.365212in}}%
\pgfpathlineto{\pgfqpoint{2.775701in}{2.365212in}}%
\pgfpathlineto{\pgfqpoint{2.776109in}{2.373463in}}%
\pgfpathlineto{\pgfqpoint{2.777130in}{2.282696in}}%
\pgfpathlineto{\pgfqpoint{2.777538in}{2.373463in}}%
\pgfpathlineto{\pgfqpoint{2.778151in}{2.356960in}}%
\pgfpathlineto{\pgfqpoint{2.779580in}{2.200180in}}%
\pgfpathlineto{\pgfqpoint{2.779784in}{2.233186in}}%
\pgfpathlineto{\pgfqpoint{2.780396in}{2.282696in}}%
\pgfpathlineto{\pgfqpoint{2.780192in}{2.216683in}}%
\pgfpathlineto{\pgfqpoint{2.780600in}{2.257941in}}%
\pgfpathlineto{\pgfqpoint{2.781417in}{2.208431in}}%
\pgfpathlineto{\pgfqpoint{2.781621in}{2.266192in}}%
\pgfpathlineto{\pgfqpoint{2.782234in}{2.299199in}}%
\pgfpathlineto{\pgfqpoint{2.782029in}{2.257941in}}%
\pgfpathlineto{\pgfqpoint{2.782642in}{2.266192in}}%
\pgfpathlineto{\pgfqpoint{2.782846in}{2.266192in}}%
\pgfpathlineto{\pgfqpoint{2.783050in}{2.216683in}}%
\pgfpathlineto{\pgfqpoint{2.783867in}{2.282696in}}%
\pgfpathlineto{\pgfqpoint{2.784071in}{2.274444in}}%
\pgfpathlineto{\pgfqpoint{2.784275in}{2.200180in}}%
\pgfpathlineto{\pgfqpoint{2.784887in}{2.249689in}}%
\pgfpathlineto{\pgfqpoint{2.785092in}{2.323954in}}%
\pgfpathlineto{\pgfqpoint{2.785908in}{2.233186in}}%
\pgfpathlineto{\pgfqpoint{2.786112in}{2.290947in}}%
\pgfpathlineto{\pgfqpoint{2.786929in}{2.282696in}}%
\pgfpathlineto{\pgfqpoint{2.787133in}{2.233186in}}%
\pgfpathlineto{\pgfqpoint{2.787745in}{2.348709in}}%
\pgfpathlineto{\pgfqpoint{2.788562in}{2.406470in}}%
\pgfpathlineto{\pgfqpoint{2.788154in}{2.340457in}}%
\pgfpathlineto{\pgfqpoint{2.789174in}{2.389967in}}%
\pgfpathlineto{\pgfqpoint{2.789991in}{2.266192in}}%
\pgfpathlineto{\pgfqpoint{2.790195in}{2.307450in}}%
\pgfpathlineto{\pgfqpoint{2.790808in}{2.373463in}}%
\pgfpathlineto{\pgfqpoint{2.791012in}{2.299199in}}%
\pgfpathlineto{\pgfqpoint{2.791420in}{2.282696in}}%
\pgfpathlineto{\pgfqpoint{2.791624in}{2.323954in}}%
\pgfpathlineto{\pgfqpoint{2.791828in}{2.315702in}}%
\pgfpathlineto{\pgfqpoint{2.792645in}{2.406470in}}%
\pgfpathlineto{\pgfqpoint{2.792849in}{2.356960in}}%
\pgfpathlineto{\pgfqpoint{2.793461in}{2.282696in}}%
\pgfpathlineto{\pgfqpoint{2.793870in}{2.332205in}}%
\pgfpathlineto{\pgfqpoint{2.794890in}{2.373463in}}%
\pgfpathlineto{\pgfqpoint{2.796524in}{2.109412in}}%
\pgfpathlineto{\pgfqpoint{2.797748in}{2.414721in}}%
\pgfpathlineto{\pgfqpoint{2.798973in}{2.299199in}}%
\pgfpathlineto{\pgfqpoint{2.799177in}{2.389967in}}%
\pgfpathlineto{\pgfqpoint{2.799994in}{2.381715in}}%
\pgfpathlineto{\pgfqpoint{2.800198in}{2.332205in}}%
\pgfpathlineto{\pgfqpoint{2.801015in}{2.356960in}}%
\pgfpathlineto{\pgfqpoint{2.801627in}{2.406470in}}%
\pgfpathlineto{\pgfqpoint{2.802035in}{2.398218in}}%
\pgfpathlineto{\pgfqpoint{2.802852in}{2.282696in}}%
\pgfpathlineto{\pgfqpoint{2.802648in}{2.406470in}}%
\pgfpathlineto{\pgfqpoint{2.803056in}{2.398218in}}%
\pgfpathlineto{\pgfqpoint{2.803260in}{2.389967in}}%
\pgfpathlineto{\pgfqpoint{2.803464in}{2.455979in}}%
\pgfpathlineto{\pgfqpoint{2.803873in}{2.381715in}}%
\pgfpathlineto{\pgfqpoint{2.804077in}{2.381715in}}%
\pgfpathlineto{\pgfqpoint{2.804689in}{2.431225in}}%
\pgfpathlineto{\pgfqpoint{2.805302in}{2.332205in}}%
\pgfpathlineto{\pgfqpoint{2.806322in}{2.389967in}}%
\pgfpathlineto{\pgfqpoint{2.807547in}{2.125915in}}%
\pgfpathlineto{\pgfqpoint{2.806935in}{2.447728in}}%
\pgfpathlineto{\pgfqpoint{2.807751in}{2.323954in}}%
\pgfpathlineto{\pgfqpoint{2.808160in}{2.332205in}}%
\pgfpathlineto{\pgfqpoint{2.808364in}{2.307450in}}%
\pgfpathlineto{\pgfqpoint{2.808568in}{2.142418in}}%
\pgfpathlineto{\pgfqpoint{2.809385in}{2.332205in}}%
\pgfpathlineto{\pgfqpoint{2.809793in}{2.398218in}}%
\pgfpathlineto{\pgfqpoint{2.810201in}{2.315702in}}%
\pgfpathlineto{\pgfqpoint{2.810405in}{2.356960in}}%
\pgfpathlineto{\pgfqpoint{2.811834in}{2.257941in}}%
\pgfpathlineto{\pgfqpoint{2.812038in}{2.282696in}}%
\pgfpathlineto{\pgfqpoint{2.813263in}{2.447728in}}%
\pgfpathlineto{\pgfqpoint{2.814080in}{2.332205in}}%
\pgfpathlineto{\pgfqpoint{2.814488in}{2.356960in}}%
\pgfpathlineto{\pgfqpoint{2.814692in}{2.373463in}}%
\pgfpathlineto{\pgfqpoint{2.814896in}{2.332205in}}%
\pgfpathlineto{\pgfqpoint{2.815101in}{2.332205in}}%
\pgfpathlineto{\pgfqpoint{2.816325in}{2.101160in}}%
\pgfpathlineto{\pgfqpoint{2.815509in}{2.373463in}}%
\pgfpathlineto{\pgfqpoint{2.816530in}{2.233186in}}%
\pgfpathlineto{\pgfqpoint{2.816938in}{2.406470in}}%
\pgfpathlineto{\pgfqpoint{2.817550in}{2.257941in}}%
\pgfpathlineto{\pgfqpoint{2.818163in}{2.422973in}}%
\pgfpathlineto{\pgfqpoint{2.818775in}{2.332205in}}%
\pgfpathlineto{\pgfqpoint{2.818979in}{2.315702in}}%
\pgfpathlineto{\pgfqpoint{2.819183in}{2.365212in}}%
\pgfpathlineto{\pgfqpoint{2.819388in}{2.348709in}}%
\pgfpathlineto{\pgfqpoint{2.820000in}{2.323954in}}%
\pgfpathlineto{\pgfqpoint{2.820204in}{2.365212in}}%
\pgfpathlineto{\pgfqpoint{2.821633in}{2.208431in}}%
\pgfpathlineto{\pgfqpoint{2.822246in}{2.497237in}}%
\pgfpathlineto{\pgfqpoint{2.823062in}{2.356960in}}%
\pgfpathlineto{\pgfqpoint{2.823266in}{2.365212in}}%
\pgfpathlineto{\pgfqpoint{2.823470in}{2.134167in}}%
\pgfpathlineto{\pgfqpoint{2.823879in}{2.439476in}}%
\pgfpathlineto{\pgfqpoint{2.824287in}{2.406470in}}%
\pgfpathlineto{\pgfqpoint{2.824899in}{2.488986in}}%
\pgfpathlineto{\pgfqpoint{2.825308in}{2.455979in}}%
\pgfpathlineto{\pgfqpoint{2.825920in}{2.142418in}}%
\pgfpathlineto{\pgfqpoint{2.826328in}{2.340457in}}%
\pgfpathlineto{\pgfqpoint{2.826737in}{2.274444in}}%
\pgfpathlineto{\pgfqpoint{2.827145in}{2.315702in}}%
\pgfpathlineto{\pgfqpoint{2.827962in}{2.389967in}}%
\pgfpathlineto{\pgfqpoint{2.828370in}{2.381715in}}%
\pgfpathlineto{\pgfqpoint{2.828982in}{2.414721in}}%
\pgfpathlineto{\pgfqpoint{2.829186in}{2.266192in}}%
\pgfpathlineto{\pgfqpoint{2.830207in}{2.546747in}}%
\pgfpathlineto{\pgfqpoint{2.830411in}{2.455979in}}%
\pgfpathlineto{\pgfqpoint{2.831228in}{2.621011in}}%
\pgfpathlineto{\pgfqpoint{2.832044in}{2.579753in}}%
\pgfpathlineto{\pgfqpoint{2.832249in}{2.530244in}}%
\pgfpathlineto{\pgfqpoint{2.833269in}{2.538495in}}%
\pgfpathlineto{\pgfqpoint{2.833473in}{2.554999in}}%
\pgfpathlineto{\pgfqpoint{2.833678in}{2.521992in}}%
\pgfpathlineto{\pgfqpoint{2.833882in}{2.521992in}}%
\pgfpathlineto{\pgfqpoint{2.834086in}{2.447728in}}%
\pgfpathlineto{\pgfqpoint{2.834698in}{2.554999in}}%
\pgfpathlineto{\pgfqpoint{2.835107in}{2.455979in}}%
\pgfpathlineto{\pgfqpoint{2.835311in}{2.472483in}}%
\pgfpathlineto{\pgfqpoint{2.835515in}{2.398218in}}%
\pgfpathlineto{\pgfqpoint{2.835719in}{2.513741in}}%
\pgfpathlineto{\pgfqpoint{2.836331in}{2.455979in}}%
\pgfpathlineto{\pgfqpoint{2.836536in}{2.513741in}}%
\pgfpathlineto{\pgfqpoint{2.836740in}{2.282696in}}%
\pgfpathlineto{\pgfqpoint{2.837148in}{2.472483in}}%
\pgfpathlineto{\pgfqpoint{2.837352in}{2.406470in}}%
\pgfpathlineto{\pgfqpoint{2.837760in}{2.480734in}}%
\pgfpathlineto{\pgfqpoint{2.838373in}{2.414721in}}%
\pgfpathlineto{\pgfqpoint{2.839802in}{2.538495in}}%
\pgfpathlineto{\pgfqpoint{2.841027in}{2.414721in}}%
\pgfpathlineto{\pgfqpoint{2.841435in}{2.398218in}}%
\pgfpathlineto{\pgfqpoint{2.842252in}{2.464231in}}%
\pgfpathlineto{\pgfqpoint{2.842456in}{2.422973in}}%
\pgfpathlineto{\pgfqpoint{2.842864in}{2.497237in}}%
\pgfpathlineto{\pgfqpoint{2.843068in}{2.488986in}}%
\pgfpathlineto{\pgfqpoint{2.843272in}{2.497237in}}%
\pgfpathlineto{\pgfqpoint{2.843476in}{2.563250in}}%
\pgfpathlineto{\pgfqpoint{2.844089in}{2.455979in}}%
\pgfpathlineto{\pgfqpoint{2.844293in}{2.497237in}}%
\pgfpathlineto{\pgfqpoint{2.844497in}{2.431225in}}%
\pgfpathlineto{\pgfqpoint{2.845314in}{2.521992in}}%
\pgfpathlineto{\pgfqpoint{2.845518in}{2.579753in}}%
\pgfpathlineto{\pgfqpoint{2.846334in}{2.538495in}}%
\pgfpathlineto{\pgfqpoint{2.846743in}{2.588005in}}%
\pgfpathlineto{\pgfqpoint{2.846947in}{2.513741in}}%
\pgfpathlineto{\pgfqpoint{2.847763in}{2.563250in}}%
\pgfpathlineto{\pgfqpoint{2.847968in}{2.579753in}}%
\pgfpathlineto{\pgfqpoint{2.848172in}{2.530244in}}%
\pgfpathlineto{\pgfqpoint{2.848580in}{2.571502in}}%
\pgfpathlineto{\pgfqpoint{2.848784in}{2.530244in}}%
\pgfpathlineto{\pgfqpoint{2.849192in}{2.588005in}}%
\pgfpathlineto{\pgfqpoint{2.849601in}{2.563250in}}%
\pgfpathlineto{\pgfqpoint{2.849805in}{2.579753in}}%
\pgfpathlineto{\pgfqpoint{2.850009in}{2.546747in}}%
\pgfpathlineto{\pgfqpoint{2.850621in}{2.563250in}}%
\pgfpathlineto{\pgfqpoint{2.851234in}{2.488986in}}%
\pgfpathlineto{\pgfqpoint{2.851846in}{2.530244in}}%
\pgfpathlineto{\pgfqpoint{2.852867in}{2.480734in}}%
\pgfpathlineto{\pgfqpoint{2.853683in}{2.406470in}}%
\pgfpathlineto{\pgfqpoint{2.853888in}{2.455979in}}%
\pgfpathlineto{\pgfqpoint{2.854500in}{2.563250in}}%
\pgfpathlineto{\pgfqpoint{2.854704in}{2.447728in}}%
\pgfpathlineto{\pgfqpoint{2.855521in}{2.521992in}}%
\pgfpathlineto{\pgfqpoint{2.856337in}{2.447728in}}%
\pgfpathlineto{\pgfqpoint{2.855929in}{2.546747in}}%
\pgfpathlineto{\pgfqpoint{2.856746in}{2.464231in}}%
\pgfpathlineto{\pgfqpoint{2.857154in}{2.538495in}}%
\pgfpathlineto{\pgfqpoint{2.857358in}{2.431225in}}%
\pgfpathlineto{\pgfqpoint{2.857562in}{2.447728in}}%
\pgfpathlineto{\pgfqpoint{2.858175in}{2.480734in}}%
\pgfpathlineto{\pgfqpoint{2.858787in}{2.381715in}}%
\pgfpathlineto{\pgfqpoint{2.859604in}{2.414721in}}%
\pgfpathlineto{\pgfqpoint{2.859808in}{2.340457in}}%
\pgfpathlineto{\pgfqpoint{2.860420in}{2.505489in}}%
\pgfpathlineto{\pgfqpoint{2.860624in}{2.472483in}}%
\pgfpathlineto{\pgfqpoint{2.861033in}{2.538495in}}%
\pgfpathlineto{\pgfqpoint{2.861237in}{2.554999in}}%
\pgfpathlineto{\pgfqpoint{2.861645in}{2.282696in}}%
\pgfpathlineto{\pgfqpoint{2.862257in}{2.299199in}}%
\pgfpathlineto{\pgfqpoint{2.862462in}{2.546747in}}%
\pgfpathlineto{\pgfqpoint{2.863074in}{2.290947in}}%
\pgfpathlineto{\pgfqpoint{2.863278in}{2.290947in}}%
\pgfpathlineto{\pgfqpoint{2.864911in}{2.546747in}}%
\pgfpathlineto{\pgfqpoint{2.863686in}{2.257941in}}%
\pgfpathlineto{\pgfqpoint{2.865115in}{2.530244in}}%
\pgfpathlineto{\pgfqpoint{2.866136in}{2.422973in}}%
\pgfpathlineto{\pgfqpoint{2.866340in}{2.464231in}}%
\pgfpathlineto{\pgfqpoint{2.866953in}{2.480734in}}%
\pgfpathlineto{\pgfqpoint{2.867565in}{2.348709in}}%
\pgfpathlineto{\pgfqpoint{2.868790in}{2.521992in}}%
\pgfpathlineto{\pgfqpoint{2.870015in}{2.422973in}}%
\pgfpathlineto{\pgfqpoint{2.870831in}{2.513741in}}%
\pgfpathlineto{\pgfqpoint{2.871240in}{2.497237in}}%
\pgfpathlineto{\pgfqpoint{2.871852in}{2.480734in}}%
\pgfpathlineto{\pgfqpoint{2.872056in}{2.513741in}}%
\pgfpathlineto{\pgfqpoint{2.872260in}{2.414721in}}%
\pgfpathlineto{\pgfqpoint{2.873077in}{2.497237in}}%
\pgfpathlineto{\pgfqpoint{2.873894in}{2.530244in}}%
\pgfpathlineto{\pgfqpoint{2.874098in}{2.480734in}}%
\pgfpathlineto{\pgfqpoint{2.874302in}{2.571502in}}%
\pgfpathlineto{\pgfqpoint{2.874914in}{2.546747in}}%
\pgfpathlineto{\pgfqpoint{2.875118in}{2.571502in}}%
\pgfpathlineto{\pgfqpoint{2.875323in}{2.530244in}}%
\pgfpathlineto{\pgfqpoint{2.875935in}{2.282696in}}%
\pgfpathlineto{\pgfqpoint{2.876343in}{2.521992in}}%
\pgfpathlineto{\pgfqpoint{2.876752in}{2.480734in}}%
\pgfpathlineto{\pgfqpoint{2.876956in}{2.546747in}}%
\pgfpathlineto{\pgfqpoint{2.877160in}{2.521992in}}%
\pgfpathlineto{\pgfqpoint{2.877976in}{2.662269in}}%
\pgfpathlineto{\pgfqpoint{2.878181in}{2.629263in}}%
\pgfpathlineto{\pgfqpoint{2.878997in}{2.464231in}}%
\pgfpathlineto{\pgfqpoint{2.879201in}{2.505489in}}%
\pgfpathlineto{\pgfqpoint{2.879610in}{2.637515in}}%
\pgfpathlineto{\pgfqpoint{2.880426in}{2.596257in}}%
\pgfpathlineto{\pgfqpoint{2.881447in}{2.488986in}}%
\pgfpathlineto{\pgfqpoint{2.881651in}{2.521992in}}%
\pgfpathlineto{\pgfqpoint{2.881855in}{2.530244in}}%
\pgfpathlineto{\pgfqpoint{2.882059in}{2.521992in}}%
\pgfpathlineto{\pgfqpoint{2.882263in}{2.497237in}}%
\pgfpathlineto{\pgfqpoint{2.882672in}{2.554999in}}%
\pgfpathlineto{\pgfqpoint{2.882876in}{2.563250in}}%
\pgfpathlineto{\pgfqpoint{2.883897in}{2.455979in}}%
\pgfpathlineto{\pgfqpoint{2.884713in}{2.588005in}}%
\pgfpathlineto{\pgfqpoint{2.885121in}{2.530244in}}%
\pgfpathlineto{\pgfqpoint{2.885938in}{2.398218in}}%
\pgfpathlineto{\pgfqpoint{2.886755in}{2.439476in}}%
\pgfpathlineto{\pgfqpoint{2.886959in}{2.431225in}}%
\pgfpathlineto{\pgfqpoint{2.887163in}{2.455979in}}%
\pgfpathlineto{\pgfqpoint{2.887775in}{2.497237in}}%
\pgfpathlineto{\pgfqpoint{2.887571in}{2.439476in}}%
\pgfpathlineto{\pgfqpoint{2.888388in}{2.472483in}}%
\pgfpathlineto{\pgfqpoint{2.888592in}{2.431225in}}%
\pgfpathlineto{\pgfqpoint{2.888796in}{2.488986in}}%
\pgfpathlineto{\pgfqpoint{2.889204in}{2.472483in}}%
\pgfpathlineto{\pgfqpoint{2.889817in}{2.497237in}}%
\pgfpathlineto{\pgfqpoint{2.890021in}{2.472483in}}%
\pgfpathlineto{\pgfqpoint{2.890429in}{2.398218in}}%
\pgfpathlineto{\pgfqpoint{2.891042in}{2.447728in}}%
\pgfpathlineto{\pgfqpoint{2.891450in}{2.472483in}}%
\pgfpathlineto{\pgfqpoint{2.891654in}{2.455979in}}%
\pgfpathlineto{\pgfqpoint{2.892266in}{2.406470in}}%
\pgfpathlineto{\pgfqpoint{2.892675in}{2.431225in}}%
\pgfpathlineto{\pgfqpoint{2.892879in}{2.447728in}}%
\pgfpathlineto{\pgfqpoint{2.893083in}{2.398218in}}%
\pgfpathlineto{\pgfqpoint{2.893287in}{2.389967in}}%
\pgfpathlineto{\pgfqpoint{2.893491in}{2.422973in}}%
\pgfpathlineto{\pgfqpoint{2.893695in}{2.398218in}}%
\pgfpathlineto{\pgfqpoint{2.894716in}{2.497237in}}%
\pgfpathlineto{\pgfqpoint{2.894920in}{2.439476in}}%
\pgfpathlineto{\pgfqpoint{2.896553in}{2.546747in}}%
\pgfpathlineto{\pgfqpoint{2.897778in}{2.464231in}}%
\pgfpathlineto{\pgfqpoint{2.897574in}{2.563250in}}%
\pgfpathlineto{\pgfqpoint{2.897982in}{2.472483in}}%
\pgfpathlineto{\pgfqpoint{2.898187in}{2.472483in}}%
\pgfpathlineto{\pgfqpoint{2.899411in}{2.563250in}}%
\pgfpathlineto{\pgfqpoint{2.900636in}{2.365212in}}%
\pgfpathlineto{\pgfqpoint{2.901861in}{2.629263in}}%
\pgfpathlineto{\pgfqpoint{2.902065in}{2.546747in}}%
\pgfpathlineto{\pgfqpoint{2.902474in}{2.571502in}}%
\pgfpathlineto{\pgfqpoint{2.902678in}{2.538495in}}%
\pgfpathlineto{\pgfqpoint{2.903903in}{2.323954in}}%
\pgfpathlineto{\pgfqpoint{2.904107in}{2.332205in}}%
\pgfpathlineto{\pgfqpoint{2.904515in}{2.381715in}}%
\pgfpathlineto{\pgfqpoint{2.905127in}{2.348709in}}%
\pgfpathlineto{\pgfqpoint{2.906556in}{2.282696in}}%
\pgfpathlineto{\pgfqpoint{2.907169in}{2.406470in}}%
\pgfpathlineto{\pgfqpoint{2.907985in}{2.373463in}}%
\pgfpathlineto{\pgfqpoint{2.908598in}{2.332205in}}%
\pgfpathlineto{\pgfqpoint{2.909414in}{2.274444in}}%
\pgfpathlineto{\pgfqpoint{2.909619in}{2.299199in}}%
\pgfpathlineto{\pgfqpoint{2.910843in}{2.389967in}}%
\pgfpathlineto{\pgfqpoint{2.910027in}{2.282696in}}%
\pgfpathlineto{\pgfqpoint{2.911252in}{2.365212in}}%
\pgfpathlineto{\pgfqpoint{2.911456in}{2.290947in}}%
\pgfpathlineto{\pgfqpoint{2.912068in}{2.373463in}}%
\pgfpathlineto{\pgfqpoint{2.912272in}{2.373463in}}%
\pgfpathlineto{\pgfqpoint{2.912681in}{2.406470in}}%
\pgfpathlineto{\pgfqpoint{2.912885in}{2.340457in}}%
\pgfpathlineto{\pgfqpoint{2.913089in}{2.356960in}}%
\pgfpathlineto{\pgfqpoint{2.913701in}{2.373463in}}%
\pgfpathlineto{\pgfqpoint{2.914722in}{2.142418in}}%
\pgfpathlineto{\pgfqpoint{2.915130in}{2.389967in}}%
\pgfpathlineto{\pgfqpoint{2.915539in}{2.076406in}}%
\pgfpathlineto{\pgfqpoint{2.915947in}{2.290947in}}%
\pgfpathlineto{\pgfqpoint{2.916559in}{2.414721in}}%
\pgfpathlineto{\pgfqpoint{2.917172in}{2.332205in}}%
\pgfpathlineto{\pgfqpoint{2.917376in}{2.340457in}}%
\pgfpathlineto{\pgfqpoint{2.917580in}{2.323954in}}%
\pgfpathlineto{\pgfqpoint{2.917784in}{2.290947in}}%
\pgfpathlineto{\pgfqpoint{2.918193in}{2.348709in}}%
\pgfpathlineto{\pgfqpoint{2.918397in}{2.340457in}}%
\pgfpathlineto{\pgfqpoint{2.918601in}{2.398218in}}%
\pgfpathlineto{\pgfqpoint{2.919417in}{2.348709in}}%
\pgfpathlineto{\pgfqpoint{2.919622in}{2.348709in}}%
\pgfpathlineto{\pgfqpoint{2.920030in}{2.266192in}}%
\pgfpathlineto{\pgfqpoint{2.920846in}{2.290947in}}%
\pgfpathlineto{\pgfqpoint{2.921663in}{2.365212in}}%
\pgfpathlineto{\pgfqpoint{2.921255in}{2.282696in}}%
\pgfpathlineto{\pgfqpoint{2.921867in}{2.299199in}}%
\pgfpathlineto{\pgfqpoint{2.922275in}{2.323954in}}%
\pgfpathlineto{\pgfqpoint{2.923704in}{2.241438in}}%
\pgfpathlineto{\pgfqpoint{2.924521in}{2.323954in}}%
\pgfpathlineto{\pgfqpoint{2.924929in}{2.307450in}}%
\pgfpathlineto{\pgfqpoint{2.925338in}{2.373463in}}%
\pgfpathlineto{\pgfqpoint{2.925746in}{2.332205in}}%
\pgfpathlineto{\pgfqpoint{2.926358in}{2.150670in}}%
\pgfpathlineto{\pgfqpoint{2.926767in}{2.191928in}}%
\pgfpathlineto{\pgfqpoint{2.927379in}{2.414721in}}%
\pgfpathlineto{\pgfqpoint{2.927991in}{2.340457in}}%
\pgfpathlineto{\pgfqpoint{2.929012in}{2.414721in}}%
\pgfpathlineto{\pgfqpoint{2.928604in}{2.315702in}}%
\pgfpathlineto{\pgfqpoint{2.929216in}{2.381715in}}%
\pgfpathlineto{\pgfqpoint{2.929625in}{2.241438in}}%
\pgfpathlineto{\pgfqpoint{2.930237in}{2.356960in}}%
\pgfpathlineto{\pgfqpoint{2.930441in}{2.414721in}}%
\pgfpathlineto{\pgfqpoint{2.930849in}{2.348709in}}%
\pgfpathlineto{\pgfqpoint{2.931666in}{2.092909in}}%
\pgfpathlineto{\pgfqpoint{2.932278in}{2.150670in}}%
\pgfpathlineto{\pgfqpoint{2.933707in}{2.398218in}}%
\pgfpathlineto{\pgfqpoint{2.934320in}{2.389967in}}%
\pgfpathlineto{\pgfqpoint{2.935545in}{2.323954in}}%
\pgfpathlineto{\pgfqpoint{2.935136in}{2.398218in}}%
\pgfpathlineto{\pgfqpoint{2.935953in}{2.340457in}}%
\pgfpathlineto{\pgfqpoint{2.937178in}{2.414721in}}%
\pgfpathlineto{\pgfqpoint{2.937382in}{2.398218in}}%
\pgfpathlineto{\pgfqpoint{2.937994in}{2.447728in}}%
\pgfpathlineto{\pgfqpoint{2.938811in}{2.332205in}}%
\pgfpathlineto{\pgfqpoint{2.939423in}{2.356960in}}%
\pgfpathlineto{\pgfqpoint{2.939627in}{2.348709in}}%
\pgfpathlineto{\pgfqpoint{2.940444in}{2.257941in}}%
\pgfpathlineto{\pgfqpoint{2.940648in}{2.282696in}}%
\pgfpathlineto{\pgfqpoint{2.941261in}{2.356960in}}%
\pgfpathlineto{\pgfqpoint{2.941465in}{2.150670in}}%
\pgfpathlineto{\pgfqpoint{2.942077in}{2.389967in}}%
\pgfpathlineto{\pgfqpoint{2.942281in}{2.373463in}}%
\pgfpathlineto{\pgfqpoint{2.943506in}{2.101160in}}%
\pgfpathlineto{\pgfqpoint{2.943710in}{2.200180in}}%
\pgfpathlineto{\pgfqpoint{2.943914in}{2.348709in}}%
\pgfpathlineto{\pgfqpoint{2.944935in}{2.299199in}}%
\pgfpathlineto{\pgfqpoint{2.946364in}{2.398218in}}%
\pgfpathlineto{\pgfqpoint{2.948814in}{2.142418in}}%
\pgfpathlineto{\pgfqpoint{2.949222in}{2.200180in}}%
\pgfpathlineto{\pgfqpoint{2.950447in}{1.936128in}}%
\pgfpathlineto{\pgfqpoint{2.951059in}{2.175425in}}%
\pgfpathlineto{\pgfqpoint{2.951468in}{2.010393in}}%
\pgfpathlineto{\pgfqpoint{2.953305in}{1.853612in}}%
\pgfpathlineto{\pgfqpoint{2.954326in}{2.010393in}}%
\pgfpathlineto{\pgfqpoint{2.954530in}{2.002141in}}%
\pgfpathlineto{\pgfqpoint{2.954734in}{1.927877in}}%
\pgfpathlineto{\pgfqpoint{2.954938in}{2.035148in}}%
\pgfpathlineto{\pgfqpoint{2.955551in}{2.026896in}}%
\pgfpathlineto{\pgfqpoint{2.955755in}{2.035148in}}%
\pgfpathlineto{\pgfqpoint{2.956163in}{1.936128in}}%
\pgfpathlineto{\pgfqpoint{2.956980in}{1.944380in}}%
\pgfpathlineto{\pgfqpoint{2.957184in}{1.919625in}}%
\pgfpathlineto{\pgfqpoint{2.957592in}{1.985638in}}%
\pgfpathlineto{\pgfqpoint{2.957796in}{1.969135in}}%
\pgfpathlineto{\pgfqpoint{2.958000in}{2.043399in}}%
\pgfpathlineto{\pgfqpoint{2.958613in}{1.944380in}}%
\pgfpathlineto{\pgfqpoint{2.959021in}{1.861864in}}%
\pgfpathlineto{\pgfqpoint{2.959429in}{2.010393in}}%
\pgfpathlineto{\pgfqpoint{2.959838in}{1.969135in}}%
\pgfpathlineto{\pgfqpoint{2.960450in}{1.985638in}}%
\pgfpathlineto{\pgfqpoint{2.960654in}{2.043399in}}%
\pgfpathlineto{\pgfqpoint{2.961267in}{1.952632in}}%
\pgfpathlineto{\pgfqpoint{2.961471in}{2.002141in}}%
\pgfpathlineto{\pgfqpoint{2.962287in}{2.018644in}}%
\pgfpathlineto{\pgfqpoint{2.962900in}{1.927877in}}%
\pgfpathlineto{\pgfqpoint{2.964533in}{2.043399in}}%
\pgfpathlineto{\pgfqpoint{2.965962in}{1.911373in}}%
\pgfpathlineto{\pgfqpoint{2.966778in}{1.977386in}}%
\pgfpathlineto{\pgfqpoint{2.967187in}{1.952632in}}%
\pgfpathlineto{\pgfqpoint{2.967391in}{1.985638in}}%
\pgfpathlineto{\pgfqpoint{2.967595in}{1.936128in}}%
\pgfpathlineto{\pgfqpoint{2.968207in}{1.969135in}}%
\pgfpathlineto{\pgfqpoint{2.968412in}{1.894870in}}%
\pgfpathlineto{\pgfqpoint{2.968616in}{1.977386in}}%
\pgfpathlineto{\pgfqpoint{2.969228in}{1.936128in}}%
\pgfpathlineto{\pgfqpoint{2.970453in}{2.076406in}}%
\pgfpathlineto{\pgfqpoint{2.970861in}{2.035148in}}%
\pgfpathlineto{\pgfqpoint{2.971065in}{2.035148in}}%
\pgfpathlineto{\pgfqpoint{2.972699in}{2.101160in}}%
\pgfpathlineto{\pgfqpoint{2.972903in}{2.035148in}}%
\pgfpathlineto{\pgfqpoint{2.973515in}{2.175425in}}%
\pgfpathlineto{\pgfqpoint{2.973719in}{2.125915in}}%
\pgfpathlineto{\pgfqpoint{2.974536in}{2.175425in}}%
\pgfpathlineto{\pgfqpoint{2.975557in}{2.026896in}}%
\pgfpathlineto{\pgfqpoint{2.975761in}{2.051651in}}%
\pgfpathlineto{\pgfqpoint{2.975965in}{2.142418in}}%
\pgfpathlineto{\pgfqpoint{2.976373in}{2.026896in}}%
\pgfpathlineto{\pgfqpoint{2.976577in}{2.043399in}}%
\pgfpathlineto{\pgfqpoint{2.977394in}{1.936128in}}%
\pgfpathlineto{\pgfqpoint{2.977802in}{1.952632in}}%
\pgfpathlineto{\pgfqpoint{2.978823in}{2.059902in}}%
\pgfpathlineto{\pgfqpoint{2.979027in}{2.043399in}}%
\pgfpathlineto{\pgfqpoint{2.979435in}{1.812354in}}%
\pgfpathlineto{\pgfqpoint{2.980252in}{1.960883in}}%
\pgfpathlineto{\pgfqpoint{2.980456in}{1.985638in}}%
\pgfpathlineto{\pgfqpoint{2.980864in}{1.911373in}}%
\pgfpathlineto{\pgfqpoint{2.981068in}{1.927877in}}%
\pgfpathlineto{\pgfqpoint{2.981273in}{1.853612in}}%
\pgfpathlineto{\pgfqpoint{2.982089in}{1.894870in}}%
\pgfpathlineto{\pgfqpoint{2.982293in}{1.894870in}}%
\pgfpathlineto{\pgfqpoint{2.982702in}{1.886619in}}%
\pgfpathlineto{\pgfqpoint{2.982906in}{1.969135in}}%
\pgfpathlineto{\pgfqpoint{2.983926in}{1.936128in}}%
\pgfpathlineto{\pgfqpoint{2.985355in}{1.837109in}}%
\pgfpathlineto{\pgfqpoint{2.985560in}{1.936128in}}%
\pgfpathlineto{\pgfqpoint{2.986580in}{1.919625in}}%
\pgfpathlineto{\pgfqpoint{2.986989in}{1.903122in}}%
\pgfpathlineto{\pgfqpoint{2.987397in}{1.969135in}}%
\pgfpathlineto{\pgfqpoint{2.988213in}{1.837109in}}%
\pgfpathlineto{\pgfqpoint{2.988622in}{1.845361in}}%
\pgfpathlineto{\pgfqpoint{2.989847in}{1.927877in}}%
\pgfpathlineto{\pgfqpoint{2.990051in}{1.919625in}}%
\pgfpathlineto{\pgfqpoint{2.990255in}{1.894870in}}%
\pgfpathlineto{\pgfqpoint{2.990867in}{1.911373in}}%
\pgfpathlineto{\pgfqpoint{2.991071in}{1.960883in}}%
\pgfpathlineto{\pgfqpoint{2.991684in}{1.853612in}}%
\pgfpathlineto{\pgfqpoint{2.991888in}{1.721587in}}%
\pgfpathlineto{\pgfqpoint{2.992296in}{1.944380in}}%
\pgfpathlineto{\pgfqpoint{2.992705in}{1.878367in}}%
\pgfpathlineto{\pgfqpoint{2.993725in}{2.043399in}}%
\pgfpathlineto{\pgfqpoint{2.994134in}{1.960883in}}%
\pgfpathlineto{\pgfqpoint{2.995358in}{2.068154in}}%
\pgfpathlineto{\pgfqpoint{2.995563in}{2.026896in}}%
\pgfpathlineto{\pgfqpoint{2.996992in}{1.870115in}}%
\pgfpathlineto{\pgfqpoint{2.995971in}{2.043399in}}%
\pgfpathlineto{\pgfqpoint{2.997400in}{1.886619in}}%
\pgfpathlineto{\pgfqpoint{2.998829in}{2.084657in}}%
\pgfpathlineto{\pgfqpoint{2.998421in}{1.870115in}}%
\pgfpathlineto{\pgfqpoint{2.999033in}{2.035148in}}%
\pgfpathlineto{\pgfqpoint{3.000054in}{1.861864in}}%
\pgfpathlineto{\pgfqpoint{3.000258in}{1.878367in}}%
\pgfpathlineto{\pgfqpoint{3.001074in}{1.919625in}}%
\pgfpathlineto{\pgfqpoint{3.001483in}{1.894870in}}%
\pgfpathlineto{\pgfqpoint{3.002095in}{1.779348in}}%
\pgfpathlineto{\pgfqpoint{3.002299in}{1.622567in}}%
\pgfpathlineto{\pgfqpoint{3.002912in}{1.985638in}}%
\pgfpathlineto{\pgfqpoint{3.004749in}{1.589561in}}%
\pgfpathlineto{\pgfqpoint{3.005157in}{1.754593in}}%
\pgfpathlineto{\pgfqpoint{3.006586in}{1.944380in}}%
\pgfpathlineto{\pgfqpoint{3.006995in}{1.696832in}}%
\pgfpathlineto{\pgfqpoint{3.007403in}{1.960883in}}%
\pgfpathlineto{\pgfqpoint{3.007811in}{1.993890in}}%
\pgfpathlineto{\pgfqpoint{3.008015in}{1.960883in}}%
\pgfpathlineto{\pgfqpoint{3.008424in}{1.795851in}}%
\pgfpathlineto{\pgfqpoint{3.009036in}{2.010393in}}%
\pgfpathlineto{\pgfqpoint{3.009240in}{2.010393in}}%
\pgfpathlineto{\pgfqpoint{3.010669in}{1.853612in}}%
\pgfpathlineto{\pgfqpoint{3.009648in}{2.018644in}}%
\pgfpathlineto{\pgfqpoint{3.010873in}{1.919625in}}%
\pgfpathlineto{\pgfqpoint{3.011077in}{1.919625in}}%
\pgfpathlineto{\pgfqpoint{3.011282in}{1.837109in}}%
\pgfpathlineto{\pgfqpoint{3.012098in}{1.936128in}}%
\pgfpathlineto{\pgfqpoint{3.012302in}{1.936128in}}%
\pgfpathlineto{\pgfqpoint{3.013119in}{1.845361in}}%
\pgfpathlineto{\pgfqpoint{3.013527in}{1.886619in}}%
\pgfpathlineto{\pgfqpoint{3.014140in}{1.870115in}}%
\pgfpathlineto{\pgfqpoint{3.014548in}{1.952632in}}%
\pgfpathlineto{\pgfqpoint{3.014752in}{1.911373in}}%
\pgfpathlineto{\pgfqpoint{3.015569in}{1.960883in}}%
\pgfpathlineto{\pgfqpoint{3.016589in}{1.894870in}}%
\pgfpathlineto{\pgfqpoint{3.016793in}{1.911373in}}%
\pgfpathlineto{\pgfqpoint{3.017202in}{1.977386in}}%
\pgfpathlineto{\pgfqpoint{3.017406in}{1.886619in}}%
\pgfpathlineto{\pgfqpoint{3.017610in}{1.630819in}}%
\pgfpathlineto{\pgfqpoint{3.018222in}{2.035148in}}%
\pgfpathlineto{\pgfqpoint{3.018427in}{1.969135in}}%
\pgfpathlineto{\pgfqpoint{3.019039in}{1.771096in}}%
\pgfpathlineto{\pgfqpoint{3.020264in}{1.795851in}}%
\pgfpathlineto{\pgfqpoint{3.020672in}{1.787599in}}%
\pgfpathlineto{\pgfqpoint{3.021693in}{1.853612in}}%
\pgfpathlineto{\pgfqpoint{3.022714in}{1.812354in}}%
\pgfpathlineto{\pgfqpoint{3.022101in}{1.886619in}}%
\pgfpathlineto{\pgfqpoint{3.022918in}{1.820606in}}%
\pgfpathlineto{\pgfqpoint{3.023734in}{1.886619in}}%
\pgfpathlineto{\pgfqpoint{3.023938in}{1.828857in}}%
\pgfpathlineto{\pgfqpoint{3.024551in}{1.771096in}}%
\pgfpathlineto{\pgfqpoint{3.024755in}{1.804103in}}%
\pgfpathlineto{\pgfqpoint{3.026184in}{2.026896in}}%
\pgfpathlineto{\pgfqpoint{3.026592in}{1.861864in}}%
\pgfpathlineto{\pgfqpoint{3.027409in}{1.870115in}}%
\pgfpathlineto{\pgfqpoint{3.028838in}{1.523548in}}%
\pgfpathlineto{\pgfqpoint{3.029858in}{1.870115in}}%
\pgfpathlineto{\pgfqpoint{3.030063in}{1.804103in}}%
\pgfpathlineto{\pgfqpoint{3.030267in}{1.804103in}}%
\pgfpathlineto{\pgfqpoint{3.030879in}{1.754593in}}%
\pgfpathlineto{\pgfqpoint{3.031492in}{1.779348in}}%
\pgfpathlineto{\pgfqpoint{3.032104in}{1.721587in}}%
\pgfpathlineto{\pgfqpoint{3.032308in}{1.540051in}}%
\pgfpathlineto{\pgfqpoint{3.032716in}{1.812354in}}%
\pgfpathlineto{\pgfqpoint{3.032921in}{1.771096in}}%
\pgfpathlineto{\pgfqpoint{3.033125in}{1.903122in}}%
\pgfpathlineto{\pgfqpoint{3.033329in}{1.614316in}}%
\pgfpathlineto{\pgfqpoint{3.034145in}{1.878367in}}%
\pgfpathlineto{\pgfqpoint{3.034350in}{1.853612in}}%
\pgfpathlineto{\pgfqpoint{3.034758in}{1.936128in}}%
\pgfpathlineto{\pgfqpoint{3.037003in}{1.787599in}}%
\pgfpathlineto{\pgfqpoint{3.037820in}{1.853612in}}%
\pgfpathlineto{\pgfqpoint{3.038024in}{1.779348in}}%
\pgfpathlineto{\pgfqpoint{3.039657in}{1.919625in}}%
\pgfpathlineto{\pgfqpoint{3.040066in}{1.878367in}}%
\pgfpathlineto{\pgfqpoint{3.041086in}{1.705083in}}%
\pgfpathlineto{\pgfqpoint{3.041290in}{1.713335in}}%
\pgfpathlineto{\pgfqpoint{3.041495in}{1.738090in}}%
\pgfpathlineto{\pgfqpoint{3.041903in}{1.680329in}}%
\pgfpathlineto{\pgfqpoint{3.042107in}{1.647322in}}%
\pgfpathlineto{\pgfqpoint{3.042515in}{1.721587in}}%
\pgfpathlineto{\pgfqpoint{3.043536in}{1.795851in}}%
\pgfpathlineto{\pgfqpoint{3.043128in}{1.688580in}}%
\pgfpathlineto{\pgfqpoint{3.043740in}{1.762845in}}%
\pgfpathlineto{\pgfqpoint{3.044557in}{1.663825in}}%
\pgfpathlineto{\pgfqpoint{3.044965in}{1.713335in}}%
\pgfpathlineto{\pgfqpoint{3.045373in}{1.828857in}}%
\pgfpathlineto{\pgfqpoint{3.045577in}{1.705083in}}%
\pgfpathlineto{\pgfqpoint{3.045986in}{1.787599in}}%
\pgfpathlineto{\pgfqpoint{3.047006in}{1.729838in}}%
\pgfpathlineto{\pgfqpoint{3.046802in}{1.820606in}}%
\pgfpathlineto{\pgfqpoint{3.047211in}{1.738090in}}%
\pgfpathlineto{\pgfqpoint{3.048027in}{1.672077in}}%
\pgfpathlineto{\pgfqpoint{3.049456in}{1.837109in}}%
\pgfpathlineto{\pgfqpoint{3.049864in}{1.754593in}}%
\pgfpathlineto{\pgfqpoint{3.050273in}{1.804103in}}%
\pgfpathlineto{\pgfqpoint{3.050681in}{1.861864in}}%
\pgfpathlineto{\pgfqpoint{3.050885in}{1.787599in}}%
\pgfpathlineto{\pgfqpoint{3.051089in}{1.828857in}}%
\pgfpathlineto{\pgfqpoint{3.051498in}{1.713335in}}%
\pgfpathlineto{\pgfqpoint{3.052110in}{1.754593in}}%
\pgfpathlineto{\pgfqpoint{3.052518in}{1.861864in}}%
\pgfpathlineto{\pgfqpoint{3.052722in}{1.746341in}}%
\pgfpathlineto{\pgfqpoint{3.053131in}{1.837109in}}%
\pgfpathlineto{\pgfqpoint{3.054151in}{1.622567in}}%
\pgfpathlineto{\pgfqpoint{3.054560in}{1.663825in}}%
\pgfpathlineto{\pgfqpoint{3.055376in}{1.771096in}}%
\pgfpathlineto{\pgfqpoint{3.055785in}{1.754593in}}%
\pgfpathlineto{\pgfqpoint{3.056601in}{1.672077in}}%
\pgfpathlineto{\pgfqpoint{3.056805in}{1.680329in}}%
\pgfpathlineto{\pgfqpoint{3.057418in}{1.787599in}}%
\pgfpathlineto{\pgfqpoint{3.057826in}{1.721587in}}%
\pgfpathlineto{\pgfqpoint{3.058234in}{1.639071in}}%
\pgfpathlineto{\pgfqpoint{3.058847in}{1.696832in}}%
\pgfpathlineto{\pgfqpoint{3.059051in}{1.721587in}}%
\pgfpathlineto{\pgfqpoint{3.059255in}{1.647322in}}%
\pgfpathlineto{\pgfqpoint{3.059663in}{1.614316in}}%
\pgfpathlineto{\pgfqpoint{3.060072in}{1.688580in}}%
\pgfpathlineto{\pgfqpoint{3.060276in}{1.746341in}}%
\pgfpathlineto{\pgfqpoint{3.060684in}{1.680329in}}%
\pgfpathlineto{\pgfqpoint{3.061092in}{1.713335in}}%
\pgfpathlineto{\pgfqpoint{3.061296in}{1.713335in}}%
\pgfpathlineto{\pgfqpoint{3.062521in}{1.630819in}}%
\pgfpathlineto{\pgfqpoint{3.062725in}{1.639071in}}%
\pgfpathlineto{\pgfqpoint{3.062930in}{1.614316in}}%
\pgfpathlineto{\pgfqpoint{3.063338in}{1.564806in}}%
\pgfpathlineto{\pgfqpoint{3.063746in}{1.647322in}}%
\pgfpathlineto{\pgfqpoint{3.063950in}{1.581309in}}%
\pgfpathlineto{\pgfqpoint{3.064154in}{1.647322in}}%
\pgfpathlineto{\pgfqpoint{3.064767in}{1.531800in}}%
\pgfpathlineto{\pgfqpoint{3.064971in}{1.564806in}}%
\pgfpathlineto{\pgfqpoint{3.065379in}{1.597813in}}%
\pgfpathlineto{\pgfqpoint{3.065583in}{1.589561in}}%
\pgfpathlineto{\pgfqpoint{3.066400in}{1.383271in}}%
\pgfpathlineto{\pgfqpoint{3.066604in}{1.614316in}}%
\pgfpathlineto{\pgfqpoint{3.067421in}{1.721587in}}%
\pgfpathlineto{\pgfqpoint{3.067625in}{1.474038in}}%
\pgfpathlineto{\pgfqpoint{3.068441in}{1.738090in}}%
\pgfpathlineto{\pgfqpoint{3.068646in}{1.721587in}}%
\pgfpathlineto{\pgfqpoint{3.068850in}{1.754593in}}%
\pgfpathlineto{\pgfqpoint{3.069054in}{1.754593in}}%
\pgfpathlineto{\pgfqpoint{3.069462in}{1.630819in}}%
\pgfpathlineto{\pgfqpoint{3.070075in}{1.812354in}}%
\pgfpathlineto{\pgfqpoint{3.071095in}{1.771096in}}%
\pgfpathlineto{\pgfqpoint{3.071504in}{1.911373in}}%
\pgfpathlineto{\pgfqpoint{3.071912in}{1.804103in}}%
\pgfpathlineto{\pgfqpoint{3.073137in}{1.721587in}}%
\pgfpathlineto{\pgfqpoint{3.073749in}{1.754593in}}%
\pgfpathlineto{\pgfqpoint{3.074974in}{1.614316in}}%
\pgfpathlineto{\pgfqpoint{3.075178in}{1.622567in}}%
\pgfpathlineto{\pgfqpoint{3.077015in}{1.812354in}}%
\pgfpathlineto{\pgfqpoint{3.077220in}{1.746341in}}%
\pgfpathlineto{\pgfqpoint{3.077424in}{1.449284in}}%
\pgfpathlineto{\pgfqpoint{3.078240in}{1.828857in}}%
\pgfpathlineto{\pgfqpoint{3.079261in}{1.581309in}}%
\pgfpathlineto{\pgfqpoint{3.079669in}{1.630819in}}%
\pgfpathlineto{\pgfqpoint{3.080078in}{1.705083in}}%
\pgfpathlineto{\pgfqpoint{3.080690in}{1.672077in}}%
\pgfpathlineto{\pgfqpoint{3.080894in}{1.622567in}}%
\pgfpathlineto{\pgfqpoint{3.081507in}{1.680329in}}%
\pgfpathlineto{\pgfqpoint{3.082527in}{1.779348in}}%
\pgfpathlineto{\pgfqpoint{3.082936in}{1.729838in}}%
\pgfpathlineto{\pgfqpoint{3.083956in}{1.647322in}}%
\pgfpathlineto{\pgfqpoint{3.084160in}{1.688580in}}%
\pgfpathlineto{\pgfqpoint{3.084977in}{1.713335in}}%
\pgfpathlineto{\pgfqpoint{3.085181in}{1.688580in}}%
\pgfpathlineto{\pgfqpoint{3.085794in}{1.647322in}}%
\pgfpathlineto{\pgfqpoint{3.086202in}{1.696832in}}%
\pgfpathlineto{\pgfqpoint{3.086814in}{1.663825in}}%
\pgfpathlineto{\pgfqpoint{3.087427in}{1.746341in}}%
\pgfpathlineto{\pgfqpoint{3.089060in}{1.622567in}}%
\pgfpathlineto{\pgfqpoint{3.089468in}{1.630819in}}%
\pgfpathlineto{\pgfqpoint{3.091101in}{1.762845in}}%
\pgfpathlineto{\pgfqpoint{3.091510in}{1.771096in}}%
\pgfpathlineto{\pgfqpoint{3.092530in}{1.564806in}}%
\pgfpathlineto{\pgfqpoint{3.093143in}{1.787599in}}%
\pgfpathlineto{\pgfqpoint{3.093551in}{1.688580in}}%
\pgfpathlineto{\pgfqpoint{3.093755in}{1.556554in}}%
\pgfpathlineto{\pgfqpoint{3.093959in}{1.779348in}}%
\pgfpathlineto{\pgfqpoint{3.094572in}{1.688580in}}%
\pgfpathlineto{\pgfqpoint{3.095184in}{1.647322in}}%
\pgfpathlineto{\pgfqpoint{3.095797in}{1.754593in}}%
\pgfpathlineto{\pgfqpoint{3.097226in}{1.663825in}}%
\pgfpathlineto{\pgfqpoint{3.097430in}{1.696832in}}%
\pgfpathlineto{\pgfqpoint{3.097838in}{1.606064in}}%
\pgfpathlineto{\pgfqpoint{3.098042in}{1.647322in}}%
\pgfpathlineto{\pgfqpoint{3.098859in}{1.606064in}}%
\pgfpathlineto{\pgfqpoint{3.099675in}{1.729838in}}%
\pgfpathlineto{\pgfqpoint{3.100084in}{1.680329in}}%
\pgfpathlineto{\pgfqpoint{3.100696in}{1.540051in}}%
\pgfpathlineto{\pgfqpoint{3.101513in}{1.614316in}}%
\pgfpathlineto{\pgfqpoint{3.101921in}{1.581309in}}%
\pgfpathlineto{\pgfqpoint{3.102329in}{1.309006in}}%
\pgfpathlineto{\pgfqpoint{3.103350in}{1.358516in}}%
\pgfpathlineto{\pgfqpoint{3.103554in}{1.350264in}}%
\pgfpathlineto{\pgfqpoint{3.104575in}{1.721587in}}%
\pgfpathlineto{\pgfqpoint{3.104983in}{1.705083in}}%
\pgfpathlineto{\pgfqpoint{3.105391in}{1.721587in}}%
\pgfpathlineto{\pgfqpoint{3.106208in}{1.663825in}}%
\pgfpathlineto{\pgfqpoint{3.106616in}{1.705083in}}%
\pgfpathlineto{\pgfqpoint{3.106820in}{1.606064in}}%
\pgfpathlineto{\pgfqpoint{3.107637in}{1.713335in}}%
\pgfpathlineto{\pgfqpoint{3.108045in}{1.630819in}}%
\pgfpathlineto{\pgfqpoint{3.108453in}{1.573058in}}%
\pgfpathlineto{\pgfqpoint{3.108862in}{1.606064in}}%
\pgfpathlineto{\pgfqpoint{3.109678in}{1.738090in}}%
\pgfpathlineto{\pgfqpoint{3.109474in}{1.581309in}}%
\pgfpathlineto{\pgfqpoint{3.109882in}{1.597813in}}%
\pgfpathlineto{\pgfqpoint{3.110903in}{1.465787in}}%
\pgfpathlineto{\pgfqpoint{3.111107in}{1.482290in}}%
\pgfpathlineto{\pgfqpoint{3.111924in}{1.523548in}}%
\pgfpathlineto{\pgfqpoint{3.111720in}{1.432780in}}%
\pgfpathlineto{\pgfqpoint{3.112128in}{1.482290in}}%
\pgfpathlineto{\pgfqpoint{3.112332in}{1.465787in}}%
\pgfpathlineto{\pgfqpoint{3.112536in}{1.564806in}}%
\pgfpathlineto{\pgfqpoint{3.113353in}{1.498793in}}%
\pgfpathlineto{\pgfqpoint{3.113965in}{1.581309in}}%
\pgfpathlineto{\pgfqpoint{3.114169in}{1.556554in}}%
\pgfpathlineto{\pgfqpoint{3.114782in}{1.465787in}}%
\pgfpathlineto{\pgfqpoint{3.114986in}{1.515296in}}%
\pgfpathlineto{\pgfqpoint{3.115190in}{1.589561in}}%
\pgfpathlineto{\pgfqpoint{3.116007in}{1.490542in}}%
\pgfpathlineto{\pgfqpoint{3.116211in}{1.482290in}}%
\pgfpathlineto{\pgfqpoint{3.117436in}{1.639071in}}%
\pgfpathlineto{\pgfqpoint{3.117640in}{1.597813in}}%
\pgfpathlineto{\pgfqpoint{3.118252in}{1.647322in}}%
\pgfpathlineto{\pgfqpoint{3.118456in}{1.474038in}}%
\pgfpathlineto{\pgfqpoint{3.119273in}{1.688580in}}%
\pgfpathlineto{\pgfqpoint{3.120294in}{1.705083in}}%
\pgfpathlineto{\pgfqpoint{3.120702in}{1.564806in}}%
\pgfpathlineto{\pgfqpoint{3.120906in}{1.663825in}}%
\pgfpathlineto{\pgfqpoint{3.121518in}{1.465787in}}%
\pgfpathlineto{\pgfqpoint{3.121723in}{1.391522in}}%
\pgfpathlineto{\pgfqpoint{3.122539in}{1.465787in}}%
\pgfpathlineto{\pgfqpoint{3.122947in}{1.449284in}}%
\pgfpathlineto{\pgfqpoint{3.123764in}{1.507045in}}%
\pgfpathlineto{\pgfqpoint{3.124581in}{1.432780in}}%
\pgfpathlineto{\pgfqpoint{3.124785in}{1.457535in}}%
\pgfpathlineto{\pgfqpoint{3.125397in}{1.449284in}}%
\pgfpathlineto{\pgfqpoint{3.126010in}{1.515296in}}%
\pgfpathlineto{\pgfqpoint{3.127439in}{1.424529in}}%
\pgfpathlineto{\pgfqpoint{3.128663in}{1.606064in}}%
\pgfpathlineto{\pgfqpoint{3.129072in}{1.333761in}}%
\pgfpathlineto{\pgfqpoint{3.129684in}{1.523548in}}%
\pgfpathlineto{\pgfqpoint{3.132134in}{1.713335in}}%
\pgfpathlineto{\pgfqpoint{3.132338in}{1.688580in}}%
\pgfpathlineto{\pgfqpoint{3.132950in}{1.639071in}}%
\pgfpathlineto{\pgfqpoint{3.133563in}{1.663825in}}%
\pgfpathlineto{\pgfqpoint{3.134379in}{1.762845in}}%
\pgfpathlineto{\pgfqpoint{3.134584in}{1.647322in}}%
\pgfpathlineto{\pgfqpoint{3.135604in}{1.474038in}}%
\pgfpathlineto{\pgfqpoint{3.135808in}{1.498793in}}%
\pgfpathlineto{\pgfqpoint{3.136625in}{1.622567in}}%
\pgfpathlineto{\pgfqpoint{3.137033in}{1.548303in}}%
\pgfpathlineto{\pgfqpoint{3.137442in}{1.507045in}}%
\pgfpathlineto{\pgfqpoint{3.137646in}{1.523548in}}%
\pgfpathlineto{\pgfqpoint{3.137850in}{1.581309in}}%
\pgfpathlineto{\pgfqpoint{3.138258in}{1.498793in}}%
\pgfpathlineto{\pgfqpoint{3.138666in}{1.573058in}}%
\pgfpathlineto{\pgfqpoint{3.140300in}{1.457535in}}%
\pgfpathlineto{\pgfqpoint{3.140504in}{1.432780in}}%
\pgfpathlineto{\pgfqpoint{3.140708in}{1.267748in}}%
\pgfpathlineto{\pgfqpoint{3.141116in}{1.474038in}}%
\pgfpathlineto{\pgfqpoint{3.141524in}{1.449284in}}%
\pgfpathlineto{\pgfqpoint{3.141729in}{1.531800in}}%
\pgfpathlineto{\pgfqpoint{3.142749in}{1.507045in}}%
\pgfpathlineto{\pgfqpoint{3.143362in}{1.209987in}}%
\pgfpathlineto{\pgfqpoint{3.143770in}{1.449284in}}%
\pgfpathlineto{\pgfqpoint{3.144995in}{1.564806in}}%
\pgfpathlineto{\pgfqpoint{3.145199in}{1.531800in}}%
\pgfpathlineto{\pgfqpoint{3.146220in}{1.424529in}}%
\pgfpathlineto{\pgfqpoint{3.146424in}{1.432780in}}%
\pgfpathlineto{\pgfqpoint{3.147853in}{1.573058in}}%
\pgfpathlineto{\pgfqpoint{3.148057in}{1.507045in}}%
\pgfpathlineto{\pgfqpoint{3.148874in}{1.540051in}}%
\pgfpathlineto{\pgfqpoint{3.149486in}{1.581309in}}%
\pgfpathlineto{\pgfqpoint{3.149894in}{1.540051in}}%
\pgfpathlineto{\pgfqpoint{3.150098in}{1.540051in}}%
\pgfpathlineto{\pgfqpoint{3.150711in}{1.515296in}}%
\pgfpathlineto{\pgfqpoint{3.151323in}{1.589561in}}%
\pgfpathlineto{\pgfqpoint{3.152956in}{1.416277in}}%
\pgfpathlineto{\pgfqpoint{3.153773in}{1.474038in}}%
\pgfpathlineto{\pgfqpoint{3.153365in}{1.408026in}}%
\pgfpathlineto{\pgfqpoint{3.153977in}{1.441032in}}%
\pgfpathlineto{\pgfqpoint{3.154181in}{1.432780in}}%
\pgfpathlineto{\pgfqpoint{3.154385in}{1.465787in}}%
\pgfpathlineto{\pgfqpoint{3.154590in}{1.474038in}}%
\pgfpathlineto{\pgfqpoint{3.154794in}{1.449284in}}%
\pgfpathlineto{\pgfqpoint{3.155406in}{1.399774in}}%
\pgfpathlineto{\pgfqpoint{3.155202in}{1.457535in}}%
\pgfpathlineto{\pgfqpoint{3.155814in}{1.432780in}}%
\pgfpathlineto{\pgfqpoint{3.157039in}{1.531800in}}%
\pgfpathlineto{\pgfqpoint{3.156631in}{1.391522in}}%
\pgfpathlineto{\pgfqpoint{3.157243in}{1.523548in}}%
\pgfpathlineto{\pgfqpoint{3.157448in}{1.523548in}}%
\pgfpathlineto{\pgfqpoint{3.157652in}{1.465787in}}%
\pgfpathlineto{\pgfqpoint{3.158468in}{1.548303in}}%
\pgfpathlineto{\pgfqpoint{3.159081in}{1.515296in}}%
\pgfpathlineto{\pgfqpoint{3.159285in}{1.581309in}}%
\pgfpathlineto{\pgfqpoint{3.159489in}{1.548303in}}%
\pgfpathlineto{\pgfqpoint{3.159693in}{2.315702in}}%
\pgfpathlineto{\pgfqpoint{3.160101in}{1.540051in}}%
\pgfpathlineto{\pgfqpoint{3.160510in}{1.639071in}}%
\pgfpathlineto{\pgfqpoint{3.161326in}{1.465787in}}%
\pgfpathlineto{\pgfqpoint{3.161939in}{1.482290in}}%
\pgfpathlineto{\pgfqpoint{3.162551in}{1.548303in}}%
\pgfpathlineto{\pgfqpoint{3.163164in}{1.531800in}}%
\pgfpathlineto{\pgfqpoint{3.163572in}{1.482290in}}%
\pgfpathlineto{\pgfqpoint{3.164184in}{1.531800in}}%
\pgfpathlineto{\pgfqpoint{3.164797in}{1.573058in}}%
\pgfpathlineto{\pgfqpoint{3.165409in}{1.548303in}}%
\pgfpathlineto{\pgfqpoint{3.165613in}{1.548303in}}%
\pgfpathlineto{\pgfqpoint{3.165817in}{1.581309in}}%
\pgfpathlineto{\pgfqpoint{3.166226in}{1.474038in}}%
\pgfpathlineto{\pgfqpoint{3.166430in}{1.507045in}}%
\pgfpathlineto{\pgfqpoint{3.166838in}{1.383271in}}%
\pgfpathlineto{\pgfqpoint{3.167042in}{1.135723in}}%
\pgfpathlineto{\pgfqpoint{3.167655in}{1.416277in}}%
\pgfpathlineto{\pgfqpoint{3.167859in}{1.383271in}}%
\pgfpathlineto{\pgfqpoint{3.168267in}{1.309006in}}%
\pgfpathlineto{\pgfqpoint{3.168880in}{1.366768in}}%
\pgfpathlineto{\pgfqpoint{3.169084in}{1.383271in}}%
\pgfpathlineto{\pgfqpoint{3.169288in}{1.342013in}}%
\pgfpathlineto{\pgfqpoint{3.169492in}{1.309006in}}%
\pgfpathlineto{\pgfqpoint{3.169696in}{1.383271in}}%
\pgfpathlineto{\pgfqpoint{3.170309in}{1.358516in}}%
\pgfpathlineto{\pgfqpoint{3.170717in}{1.490542in}}%
\pgfpathlineto{\pgfqpoint{3.172350in}{1.086213in}}%
\pgfpathlineto{\pgfqpoint{3.173983in}{1.358516in}}%
\pgfpathlineto{\pgfqpoint{3.174596in}{0.879923in}}%
\pgfpathlineto{\pgfqpoint{3.175208in}{1.061458in}}%
\pgfpathlineto{\pgfqpoint{3.175412in}{1.020200in}}%
\pgfpathlineto{\pgfqpoint{3.175820in}{1.110968in}}%
\pgfpathlineto{\pgfqpoint{3.176025in}{1.077961in}}%
\pgfpathlineto{\pgfqpoint{3.177249in}{1.226490in}}%
\pgfpathlineto{\pgfqpoint{3.177454in}{1.143974in}}%
\pgfpathlineto{\pgfqpoint{3.178270in}{1.218239in}}%
\pgfpathlineto{\pgfqpoint{3.177862in}{1.053207in}}%
\pgfpathlineto{\pgfqpoint{3.178678in}{1.193484in}}%
\pgfpathlineto{\pgfqpoint{3.178883in}{1.185232in}}%
\pgfpathlineto{\pgfqpoint{3.179087in}{1.342013in}}%
\pgfpathlineto{\pgfqpoint{3.179903in}{1.209987in}}%
\pgfpathlineto{\pgfqpoint{3.180720in}{1.325510in}}%
\pgfpathlineto{\pgfqpoint{3.181536in}{1.317258in}}%
\pgfpathlineto{\pgfqpoint{3.182761in}{1.086213in}}%
\pgfpathlineto{\pgfqpoint{3.182965in}{1.251245in}}%
\pgfpathlineto{\pgfqpoint{3.183374in}{1.309006in}}%
\pgfpathlineto{\pgfqpoint{3.184190in}{1.201736in}}%
\pgfpathlineto{\pgfqpoint{3.184394in}{1.276000in}}%
\pgfpathlineto{\pgfqpoint{3.184803in}{1.193484in}}%
\pgfpathlineto{\pgfqpoint{3.185007in}{1.209987in}}%
\pgfpathlineto{\pgfqpoint{3.185211in}{1.152226in}}%
\pgfpathlineto{\pgfqpoint{3.185619in}{1.333761in}}%
\pgfpathlineto{\pgfqpoint{3.186028in}{1.218239in}}%
\pgfpathlineto{\pgfqpoint{3.186640in}{1.267748in}}%
\pgfpathlineto{\pgfqpoint{3.186844in}{1.242994in}}%
\pgfpathlineto{\pgfqpoint{3.187252in}{1.028452in}}%
\pgfpathlineto{\pgfqpoint{3.187865in}{1.209987in}}%
\pgfpathlineto{\pgfqpoint{3.188069in}{1.251245in}}%
\pgfpathlineto{\pgfqpoint{3.188273in}{1.242994in}}%
\pgfpathlineto{\pgfqpoint{3.188477in}{1.003697in}}%
\pgfpathlineto{\pgfqpoint{3.188886in}{1.300755in}}%
\pgfpathlineto{\pgfqpoint{3.189294in}{1.300755in}}%
\pgfpathlineto{\pgfqpoint{3.189906in}{1.325510in}}%
\pgfpathlineto{\pgfqpoint{3.190519in}{1.251245in}}%
\pgfpathlineto{\pgfqpoint{3.190927in}{1.317258in}}%
\pgfpathlineto{\pgfqpoint{3.191335in}{1.226490in}}%
\pgfpathlineto{\pgfqpoint{3.191539in}{1.201736in}}%
\pgfpathlineto{\pgfqpoint{3.191744in}{1.226490in}}%
\pgfpathlineto{\pgfqpoint{3.192968in}{1.399774in}}%
\pgfpathlineto{\pgfqpoint{3.193172in}{1.391522in}}%
\pgfpathlineto{\pgfqpoint{3.193377in}{1.391522in}}%
\pgfpathlineto{\pgfqpoint{3.193581in}{1.358516in}}%
\pgfpathlineto{\pgfqpoint{3.193989in}{1.441032in}}%
\pgfpathlineto{\pgfqpoint{3.194193in}{1.449284in}}%
\pgfpathlineto{\pgfqpoint{3.194397in}{1.416277in}}%
\pgfpathlineto{\pgfqpoint{3.194601in}{1.416277in}}%
\pgfpathlineto{\pgfqpoint{3.195010in}{1.391522in}}%
\pgfpathlineto{\pgfqpoint{3.196030in}{1.284252in}}%
\pgfpathlineto{\pgfqpoint{3.196235in}{1.309006in}}%
\pgfpathlineto{\pgfqpoint{3.197051in}{1.234742in}}%
\pgfpathlineto{\pgfqpoint{3.196643in}{1.317258in}}%
\pgfpathlineto{\pgfqpoint{3.197255in}{1.300755in}}%
\pgfpathlineto{\pgfqpoint{3.197459in}{1.366768in}}%
\pgfpathlineto{\pgfqpoint{3.197868in}{1.317258in}}%
\pgfpathlineto{\pgfqpoint{3.198072in}{1.185232in}}%
\pgfpathlineto{\pgfqpoint{3.199093in}{1.218239in}}%
\pgfpathlineto{\pgfqpoint{3.199297in}{1.209987in}}%
\pgfpathlineto{\pgfqpoint{3.199501in}{1.242994in}}%
\pgfpathlineto{\pgfqpoint{3.199705in}{1.226490in}}%
\pgfpathlineto{\pgfqpoint{3.200522in}{1.350264in}}%
\pgfpathlineto{\pgfqpoint{3.200317in}{1.218239in}}%
\pgfpathlineto{\pgfqpoint{3.200930in}{1.333761in}}%
\pgfpathlineto{\pgfqpoint{3.201746in}{1.259497in}}%
\pgfpathlineto{\pgfqpoint{3.202563in}{1.432780in}}%
\pgfpathlineto{\pgfqpoint{3.202971in}{1.416277in}}%
\pgfpathlineto{\pgfqpoint{3.203175in}{1.375019in}}%
\pgfpathlineto{\pgfqpoint{3.203788in}{1.465787in}}%
\pgfpathlineto{\pgfqpoint{3.203992in}{1.416277in}}%
\pgfpathlineto{\pgfqpoint{3.204196in}{1.424529in}}%
\pgfpathlineto{\pgfqpoint{3.204400in}{1.399774in}}%
\pgfpathlineto{\pgfqpoint{3.205013in}{1.226490in}}%
\pgfpathlineto{\pgfqpoint{3.205625in}{1.292503in}}%
\pgfpathlineto{\pgfqpoint{3.206033in}{1.350264in}}%
\pgfpathlineto{\pgfqpoint{3.206238in}{1.325510in}}%
\pgfpathlineto{\pgfqpoint{3.207258in}{1.474038in}}%
\pgfpathlineto{\pgfqpoint{3.207462in}{1.465787in}}%
\pgfpathlineto{\pgfqpoint{3.208891in}{1.209987in}}%
\pgfpathlineto{\pgfqpoint{3.210525in}{1.408026in}}%
\pgfpathlineto{\pgfqpoint{3.211137in}{1.300755in}}%
\pgfpathlineto{\pgfqpoint{3.211545in}{1.399774in}}%
\pgfpathlineto{\pgfqpoint{3.211749in}{1.432780in}}%
\pgfpathlineto{\pgfqpoint{3.212158in}{1.375019in}}%
\pgfpathlineto{\pgfqpoint{3.212566in}{1.391522in}}%
\pgfpathlineto{\pgfqpoint{3.213587in}{1.457535in}}%
\pgfpathlineto{\pgfqpoint{3.213995in}{1.408026in}}%
\pgfpathlineto{\pgfqpoint{3.214199in}{1.408026in}}%
\pgfpathlineto{\pgfqpoint{3.215016in}{1.300755in}}%
\pgfpathlineto{\pgfqpoint{3.215220in}{1.366768in}}%
\pgfpathlineto{\pgfqpoint{3.215424in}{1.383271in}}%
\pgfpathlineto{\pgfqpoint{3.215832in}{1.375019in}}%
\pgfpathlineto{\pgfqpoint{3.216036in}{1.292503in}}%
\pgfpathlineto{\pgfqpoint{3.216649in}{1.441032in}}%
\pgfpathlineto{\pgfqpoint{3.217057in}{1.465787in}}%
\pgfpathlineto{\pgfqpoint{3.217261in}{1.424529in}}%
\pgfpathlineto{\pgfqpoint{3.218486in}{1.606064in}}%
\pgfpathlineto{\pgfqpoint{3.218690in}{1.531800in}}%
\pgfpathlineto{\pgfqpoint{3.219507in}{1.589561in}}%
\pgfpathlineto{\pgfqpoint{3.219915in}{1.622567in}}%
\pgfpathlineto{\pgfqpoint{3.220119in}{1.696832in}}%
\pgfpathlineto{\pgfqpoint{3.220323in}{1.474038in}}%
\pgfpathlineto{\pgfqpoint{3.221140in}{1.680329in}}%
\pgfpathlineto{\pgfqpoint{3.222365in}{1.589561in}}%
\pgfpathlineto{\pgfqpoint{3.222569in}{1.597813in}}%
\pgfpathlineto{\pgfqpoint{3.223590in}{1.721587in}}%
\pgfpathlineto{\pgfqpoint{3.223794in}{1.573058in}}%
\pgfpathlineto{\pgfqpoint{3.224610in}{1.696832in}}%
\pgfpathlineto{\pgfqpoint{3.224815in}{1.672077in}}%
\pgfpathlineto{\pgfqpoint{3.225019in}{1.721587in}}%
\pgfpathlineto{\pgfqpoint{3.225223in}{1.713335in}}%
\pgfpathlineto{\pgfqpoint{3.226039in}{1.804103in}}%
\pgfpathlineto{\pgfqpoint{3.226244in}{1.523548in}}%
\pgfpathlineto{\pgfqpoint{3.227060in}{1.878367in}}%
\pgfpathlineto{\pgfqpoint{3.228285in}{2.035148in}}%
\pgfpathlineto{\pgfqpoint{3.228489in}{1.828857in}}%
\pgfpathlineto{\pgfqpoint{3.229306in}{2.092909in}}%
\pgfpathlineto{\pgfqpoint{3.229714in}{2.117664in}}%
\pgfpathlineto{\pgfqpoint{3.230735in}{1.787599in}}%
\pgfpathlineto{\pgfqpoint{3.230939in}{1.870115in}}%
\pgfpathlineto{\pgfqpoint{3.231755in}{1.713335in}}%
\pgfpathlineto{\pgfqpoint{3.232164in}{2.018644in}}%
\pgfpathlineto{\pgfqpoint{3.232980in}{1.845361in}}%
\pgfpathlineto{\pgfqpoint{3.232776in}{2.043399in}}%
\pgfpathlineto{\pgfqpoint{3.233184in}{2.026896in}}%
\pgfpathlineto{\pgfqpoint{3.233389in}{1.993890in}}%
\pgfpathlineto{\pgfqpoint{3.233593in}{2.076406in}}%
\pgfpathlineto{\pgfqpoint{3.234205in}{2.018644in}}%
\pgfpathlineto{\pgfqpoint{3.234613in}{2.010393in}}%
\pgfpathlineto{\pgfqpoint{3.235430in}{1.837109in}}%
\pgfpathlineto{\pgfqpoint{3.235634in}{2.026896in}}%
\pgfpathlineto{\pgfqpoint{3.236859in}{1.927877in}}%
\pgfpathlineto{\pgfqpoint{3.238696in}{2.150670in}}%
\pgfpathlineto{\pgfqpoint{3.238900in}{2.117664in}}%
\pgfpathlineto{\pgfqpoint{3.239105in}{2.092909in}}%
\pgfpathlineto{\pgfqpoint{3.239513in}{2.142418in}}%
\pgfpathlineto{\pgfqpoint{3.239717in}{2.142418in}}%
\pgfpathlineto{\pgfqpoint{3.240738in}{2.092909in}}%
\pgfpathlineto{\pgfqpoint{3.240942in}{2.109412in}}%
\pgfpathlineto{\pgfqpoint{3.241554in}{2.208431in}}%
\pgfpathlineto{\pgfqpoint{3.242371in}{2.167173in}}%
\pgfpathlineto{\pgfqpoint{3.242575in}{2.167173in}}%
\pgfpathlineto{\pgfqpoint{3.242983in}{2.216683in}}%
\pgfpathlineto{\pgfqpoint{3.243187in}{2.142418in}}%
\pgfpathlineto{\pgfqpoint{3.243392in}{2.142418in}}%
\pgfpathlineto{\pgfqpoint{3.245025in}{2.200180in}}%
\pgfpathlineto{\pgfqpoint{3.245229in}{2.142418in}}%
\pgfpathlineto{\pgfqpoint{3.245841in}{2.257941in}}%
\pgfpathlineto{\pgfqpoint{3.246250in}{2.241438in}}%
\pgfpathlineto{\pgfqpoint{3.246454in}{2.249689in}}%
\pgfpathlineto{\pgfqpoint{3.246658in}{2.167173in}}%
\pgfpathlineto{\pgfqpoint{3.247270in}{2.299199in}}%
\pgfpathlineto{\pgfqpoint{3.247474in}{2.257941in}}%
\pgfpathlineto{\pgfqpoint{3.247679in}{2.249689in}}%
\pgfpathlineto{\pgfqpoint{3.247883in}{2.290947in}}%
\pgfpathlineto{\pgfqpoint{3.248087in}{2.142418in}}%
\pgfpathlineto{\pgfqpoint{3.248903in}{2.026896in}}%
\pgfpathlineto{\pgfqpoint{3.249108in}{2.134167in}}%
\pgfpathlineto{\pgfqpoint{3.250128in}{2.274444in}}%
\pgfpathlineto{\pgfqpoint{3.250332in}{2.266192in}}%
\pgfpathlineto{\pgfqpoint{3.250537in}{2.299199in}}%
\pgfpathlineto{\pgfqpoint{3.250741in}{2.249689in}}%
\pgfpathlineto{\pgfqpoint{3.250945in}{2.257941in}}%
\pgfpathlineto{\pgfqpoint{3.252170in}{2.051651in}}%
\pgfpathlineto{\pgfqpoint{3.253395in}{2.183676in}}%
\pgfpathlineto{\pgfqpoint{3.253599in}{1.944380in}}%
\pgfpathlineto{\pgfqpoint{3.254415in}{2.125915in}}%
\pgfpathlineto{\pgfqpoint{3.254619in}{2.134167in}}%
\pgfpathlineto{\pgfqpoint{3.255640in}{1.977386in}}%
\pgfpathlineto{\pgfqpoint{3.256048in}{2.051651in}}%
\pgfpathlineto{\pgfqpoint{3.256253in}{2.010393in}}%
\pgfpathlineto{\pgfqpoint{3.256457in}{2.059902in}}%
\pgfpathlineto{\pgfqpoint{3.256865in}{2.051651in}}%
\pgfpathlineto{\pgfqpoint{3.258090in}{2.216683in}}%
\pgfpathlineto{\pgfqpoint{3.258294in}{1.985638in}}%
\pgfpathlineto{\pgfqpoint{3.258906in}{2.299199in}}%
\pgfpathlineto{\pgfqpoint{3.259111in}{2.299199in}}%
\pgfpathlineto{\pgfqpoint{3.260335in}{2.084657in}}%
\pgfpathlineto{\pgfqpoint{3.260744in}{2.134167in}}%
\pgfpathlineto{\pgfqpoint{3.260948in}{2.134167in}}%
\pgfpathlineto{\pgfqpoint{3.261152in}{2.125915in}}%
\pgfpathlineto{\pgfqpoint{3.262581in}{2.249689in}}%
\pgfpathlineto{\pgfqpoint{3.263602in}{2.307450in}}%
\pgfpathlineto{\pgfqpoint{3.265031in}{2.084657in}}%
\pgfpathlineto{\pgfqpoint{3.265643in}{2.323954in}}%
\pgfpathlineto{\pgfqpoint{3.266256in}{2.307450in}}%
\pgfpathlineto{\pgfqpoint{3.266460in}{2.249689in}}%
\pgfpathlineto{\pgfqpoint{3.267072in}{2.323954in}}%
\pgfpathlineto{\pgfqpoint{3.267276in}{2.290947in}}%
\pgfpathlineto{\pgfqpoint{3.267480in}{2.299199in}}%
\pgfpathlineto{\pgfqpoint{3.267685in}{2.282696in}}%
\pgfpathlineto{\pgfqpoint{3.268093in}{2.282696in}}%
\pgfpathlineto{\pgfqpoint{3.268297in}{2.241438in}}%
\pgfpathlineto{\pgfqpoint{3.268705in}{2.315702in}}%
\pgfpathlineto{\pgfqpoint{3.269114in}{2.299199in}}%
\pgfpathlineto{\pgfqpoint{3.270134in}{2.398218in}}%
\pgfpathlineto{\pgfqpoint{3.270747in}{2.249689in}}%
\pgfpathlineto{\pgfqpoint{3.271563in}{2.282696in}}%
\pgfpathlineto{\pgfqpoint{3.272380in}{2.406470in}}%
\pgfpathlineto{\pgfqpoint{3.272788in}{2.315702in}}%
\pgfpathlineto{\pgfqpoint{3.272992in}{2.332205in}}%
\pgfpathlineto{\pgfqpoint{3.273196in}{2.274444in}}%
\pgfpathlineto{\pgfqpoint{3.273605in}{2.307450in}}%
\pgfpathlineto{\pgfqpoint{3.274013in}{2.266192in}}%
\pgfpathlineto{\pgfqpoint{3.274625in}{2.191928in}}%
\pgfpathlineto{\pgfqpoint{3.275034in}{2.266192in}}%
\pgfpathlineto{\pgfqpoint{3.276054in}{2.323954in}}%
\pgfpathlineto{\pgfqpoint{3.275850in}{2.249689in}}%
\pgfpathlineto{\pgfqpoint{3.276259in}{2.307450in}}%
\pgfpathlineto{\pgfqpoint{3.276463in}{2.299199in}}%
\pgfpathlineto{\pgfqpoint{3.276667in}{2.332205in}}%
\pgfpathlineto{\pgfqpoint{3.276871in}{2.018644in}}%
\pgfpathlineto{\pgfqpoint{3.277687in}{2.373463in}}%
\pgfpathlineto{\pgfqpoint{3.278708in}{2.398218in}}%
\pgfpathlineto{\pgfqpoint{3.278912in}{2.389967in}}%
\pgfpathlineto{\pgfqpoint{3.279729in}{2.274444in}}%
\pgfpathlineto{\pgfqpoint{3.280137in}{2.332205in}}%
\pgfpathlineto{\pgfqpoint{3.280341in}{2.356960in}}%
\pgfpathlineto{\pgfqpoint{3.280750in}{2.142418in}}%
\pgfpathlineto{\pgfqpoint{3.280954in}{2.406470in}}%
\pgfpathlineto{\pgfqpoint{3.281362in}{2.340457in}}%
\pgfpathlineto{\pgfqpoint{3.281566in}{2.340457in}}%
\pgfpathlineto{\pgfqpoint{3.282791in}{2.488986in}}%
\pgfpathlineto{\pgfqpoint{3.282179in}{2.282696in}}%
\pgfpathlineto{\pgfqpoint{3.283199in}{2.439476in}}%
\pgfpathlineto{\pgfqpoint{3.283608in}{2.472483in}}%
\pgfpathlineto{\pgfqpoint{3.283812in}{2.398218in}}%
\pgfpathlineto{\pgfqpoint{3.284016in}{2.266192in}}%
\pgfpathlineto{\pgfqpoint{3.284628in}{2.439476in}}%
\pgfpathlineto{\pgfqpoint{3.285037in}{2.480734in}}%
\pgfpathlineto{\pgfqpoint{3.285241in}{2.431225in}}%
\pgfpathlineto{\pgfqpoint{3.285649in}{2.439476in}}%
\pgfpathlineto{\pgfqpoint{3.285853in}{2.439476in}}%
\pgfpathlineto{\pgfqpoint{3.287078in}{2.340457in}}%
\pgfpathlineto{\pgfqpoint{3.287282in}{2.340457in}}%
\pgfpathlineto{\pgfqpoint{3.287895in}{2.332205in}}%
\pgfpathlineto{\pgfqpoint{3.288507in}{2.422973in}}%
\pgfpathlineto{\pgfqpoint{3.288711in}{2.472483in}}%
\pgfpathlineto{\pgfqpoint{3.289324in}{2.414721in}}%
\pgfpathlineto{\pgfqpoint{3.290344in}{2.282696in}}%
\pgfpathlineto{\pgfqpoint{3.290753in}{2.323954in}}%
\pgfpathlineto{\pgfqpoint{3.290957in}{2.323954in}}%
\pgfpathlineto{\pgfqpoint{3.291365in}{2.290947in}}%
\pgfpathlineto{\pgfqpoint{3.291773in}{2.348709in}}%
\pgfpathlineto{\pgfqpoint{3.291977in}{2.373463in}}%
\pgfpathlineto{\pgfqpoint{3.292386in}{2.307450in}}%
\pgfpathlineto{\pgfqpoint{3.292590in}{2.315702in}}%
\pgfpathlineto{\pgfqpoint{3.294019in}{2.439476in}}%
\pgfpathlineto{\pgfqpoint{3.294835in}{2.414721in}}%
\pgfpathlineto{\pgfqpoint{3.295448in}{2.323954in}}%
\pgfpathlineto{\pgfqpoint{3.296060in}{2.348709in}}%
\pgfpathlineto{\pgfqpoint{3.296264in}{2.348709in}}%
\pgfpathlineto{\pgfqpoint{3.296673in}{2.282696in}}%
\pgfpathlineto{\pgfqpoint{3.297285in}{2.365212in}}%
\pgfpathlineto{\pgfqpoint{3.297693in}{2.414721in}}%
\pgfpathlineto{\pgfqpoint{3.298306in}{2.365212in}}%
\pgfpathlineto{\pgfqpoint{3.298510in}{2.365212in}}%
\pgfpathlineto{\pgfqpoint{3.299531in}{2.290947in}}%
\pgfpathlineto{\pgfqpoint{3.299939in}{2.299199in}}%
\pgfpathlineto{\pgfqpoint{3.300551in}{2.373463in}}%
\pgfpathlineto{\pgfqpoint{3.300756in}{2.315702in}}%
\pgfpathlineto{\pgfqpoint{3.301572in}{2.167173in}}%
\pgfpathlineto{\pgfqpoint{3.301776in}{2.224934in}}%
\pgfpathlineto{\pgfqpoint{3.302593in}{2.290947in}}%
\pgfpathlineto{\pgfqpoint{3.302797in}{2.233186in}}%
\pgfpathlineto{\pgfqpoint{3.303001in}{2.233186in}}%
\pgfpathlineto{\pgfqpoint{3.303409in}{2.282696in}}%
\pgfpathlineto{\pgfqpoint{3.304022in}{2.348709in}}%
\pgfpathlineto{\pgfqpoint{3.304430in}{2.299199in}}%
\pgfpathlineto{\pgfqpoint{3.305451in}{2.373463in}}%
\pgfpathlineto{\pgfqpoint{3.305859in}{2.340457in}}%
\pgfpathlineto{\pgfqpoint{3.306880in}{2.266192in}}%
\pgfpathlineto{\pgfqpoint{3.307084in}{2.299199in}}%
\pgfpathlineto{\pgfqpoint{3.307288in}{2.356960in}}%
\pgfpathlineto{\pgfqpoint{3.307696in}{2.249689in}}%
\pgfpathlineto{\pgfqpoint{3.308309in}{2.332205in}}%
\pgfpathlineto{\pgfqpoint{3.308921in}{2.422973in}}%
\pgfpathlineto{\pgfqpoint{3.309330in}{2.332205in}}%
\pgfpathlineto{\pgfqpoint{3.310146in}{2.323954in}}%
\pgfpathlineto{\pgfqpoint{3.310554in}{2.348709in}}%
\pgfpathlineto{\pgfqpoint{3.310963in}{2.282696in}}%
\pgfpathlineto{\pgfqpoint{3.311779in}{2.315702in}}%
\pgfpathlineto{\pgfqpoint{3.312188in}{2.340457in}}%
\pgfpathlineto{\pgfqpoint{3.312596in}{2.224934in}}%
\pgfpathlineto{\pgfqpoint{3.312800in}{2.010393in}}%
\pgfpathlineto{\pgfqpoint{3.313617in}{2.224934in}}%
\pgfpathlineto{\pgfqpoint{3.313821in}{2.249689in}}%
\pgfpathlineto{\pgfqpoint{3.314229in}{2.233186in}}%
\pgfpathlineto{\pgfqpoint{3.314433in}{2.183676in}}%
\pgfpathlineto{\pgfqpoint{3.314637in}{2.249689in}}%
\pgfpathlineto{\pgfqpoint{3.314841in}{2.233186in}}%
\pgfpathlineto{\pgfqpoint{3.315454in}{2.356960in}}%
\pgfpathlineto{\pgfqpoint{3.316066in}{2.307450in}}%
\pgfpathlineto{\pgfqpoint{3.316270in}{2.315702in}}%
\pgfpathlineto{\pgfqpoint{3.316475in}{2.307450in}}%
\pgfpathlineto{\pgfqpoint{3.316883in}{2.068154in}}%
\pgfpathlineto{\pgfqpoint{3.317495in}{2.373463in}}%
\pgfpathlineto{\pgfqpoint{3.317699in}{2.200180in}}%
\pgfpathlineto{\pgfqpoint{3.318720in}{2.389967in}}%
\pgfpathlineto{\pgfqpoint{3.319128in}{2.348709in}}%
\pgfpathlineto{\pgfqpoint{3.319333in}{2.282696in}}%
\pgfpathlineto{\pgfqpoint{3.320149in}{2.356960in}}%
\pgfpathlineto{\pgfqpoint{3.320762in}{2.274444in}}%
\pgfpathlineto{\pgfqpoint{3.320966in}{2.290947in}}%
\pgfpathlineto{\pgfqpoint{3.321170in}{2.068154in}}%
\pgfpathlineto{\pgfqpoint{3.321578in}{2.307450in}}%
\pgfpathlineto{\pgfqpoint{3.321986in}{2.257941in}}%
\pgfpathlineto{\pgfqpoint{3.322191in}{2.299199in}}%
\pgfpathlineto{\pgfqpoint{3.322395in}{2.092909in}}%
\pgfpathlineto{\pgfqpoint{3.323007in}{2.422973in}}%
\pgfpathlineto{\pgfqpoint{3.323211in}{2.381715in}}%
\pgfpathlineto{\pgfqpoint{3.324436in}{2.323954in}}%
\pgfpathlineto{\pgfqpoint{3.325253in}{2.406470in}}%
\pgfpathlineto{\pgfqpoint{3.325457in}{2.389967in}}%
\pgfpathlineto{\pgfqpoint{3.325865in}{2.422973in}}%
\pgfpathlineto{\pgfqpoint{3.326478in}{2.332205in}}%
\pgfpathlineto{\pgfqpoint{3.327090in}{2.315702in}}%
\pgfpathlineto{\pgfqpoint{3.327702in}{2.389967in}}%
\pgfpathlineto{\pgfqpoint{3.328723in}{2.340457in}}%
\pgfpathlineto{\pgfqpoint{3.329131in}{2.348709in}}%
\pgfpathlineto{\pgfqpoint{3.329336in}{2.266192in}}%
\pgfpathlineto{\pgfqpoint{3.330356in}{2.282696in}}%
\pgfpathlineto{\pgfqpoint{3.331173in}{2.348709in}}%
\pgfpathlineto{\pgfqpoint{3.331581in}{2.332205in}}%
\pgfpathlineto{\pgfqpoint{3.331785in}{2.315702in}}%
\pgfpathlineto{\pgfqpoint{3.331989in}{2.340457in}}%
\pgfpathlineto{\pgfqpoint{3.333010in}{2.414721in}}%
\pgfpathlineto{\pgfqpoint{3.333214in}{2.373463in}}%
\pgfpathlineto{\pgfqpoint{3.334031in}{2.381715in}}%
\pgfpathlineto{\pgfqpoint{3.334847in}{2.340457in}}%
\pgfpathlineto{\pgfqpoint{3.335664in}{2.406470in}}%
\pgfpathlineto{\pgfqpoint{3.335868in}{2.348709in}}%
\pgfpathlineto{\pgfqpoint{3.336276in}{2.340457in}}%
\pgfpathlineto{\pgfqpoint{3.336481in}{2.348709in}}%
\pgfpathlineto{\pgfqpoint{3.337297in}{2.431225in}}%
\pgfpathlineto{\pgfqpoint{3.337501in}{2.406470in}}%
\pgfpathlineto{\pgfqpoint{3.337910in}{2.208431in}}%
\pgfpathlineto{\pgfqpoint{3.338726in}{2.348709in}}%
\pgfpathlineto{\pgfqpoint{3.339339in}{2.307450in}}%
\pgfpathlineto{\pgfqpoint{3.339134in}{2.365212in}}%
\pgfpathlineto{\pgfqpoint{3.339543in}{2.356960in}}%
\pgfpathlineto{\pgfqpoint{3.340155in}{2.348709in}}%
\pgfpathlineto{\pgfqpoint{3.340972in}{2.422973in}}%
\pgfpathlineto{\pgfqpoint{3.341176in}{2.422973in}}%
\pgfpathlineto{\pgfqpoint{3.341584in}{2.356960in}}%
\pgfpathlineto{\pgfqpoint{3.342197in}{2.422973in}}%
\pgfpathlineto{\pgfqpoint{3.343217in}{2.348709in}}%
\pgfpathlineto{\pgfqpoint{3.342605in}{2.447728in}}%
\pgfpathlineto{\pgfqpoint{3.343421in}{2.389967in}}%
\pgfpathlineto{\pgfqpoint{3.343626in}{2.389967in}}%
\pgfpathlineto{\pgfqpoint{3.344442in}{2.431225in}}%
\pgfpathlineto{\pgfqpoint{3.344238in}{2.365212in}}%
\pgfpathlineto{\pgfqpoint{3.344646in}{2.414721in}}%
\pgfpathlineto{\pgfqpoint{3.345259in}{2.340457in}}%
\pgfpathlineto{\pgfqpoint{3.345871in}{2.348709in}}%
\pgfpathlineto{\pgfqpoint{3.346688in}{2.422973in}}%
\pgfpathlineto{\pgfqpoint{3.346892in}{2.406470in}}%
\pgfpathlineto{\pgfqpoint{3.347913in}{2.332205in}}%
\pgfpathlineto{\pgfqpoint{3.348117in}{2.365212in}}%
\pgfpathlineto{\pgfqpoint{3.348321in}{2.373463in}}%
\pgfpathlineto{\pgfqpoint{3.349137in}{2.266192in}}%
\pgfpathlineto{\pgfqpoint{3.349546in}{2.290947in}}%
\pgfpathlineto{\pgfqpoint{3.349750in}{2.315702in}}%
\pgfpathlineto{\pgfqpoint{3.350362in}{2.266192in}}%
\pgfpathlineto{\pgfqpoint{3.350566in}{2.249689in}}%
\pgfpathlineto{\pgfqpoint{3.350771in}{2.315702in}}%
\pgfpathlineto{\pgfqpoint{3.350975in}{2.299199in}}%
\pgfpathlineto{\pgfqpoint{3.351791in}{2.373463in}}%
\pgfpathlineto{\pgfqpoint{3.352200in}{2.348709in}}%
\pgfpathlineto{\pgfqpoint{3.352812in}{2.389967in}}%
\pgfpathlineto{\pgfqpoint{3.353220in}{2.356960in}}%
\pgfpathlineto{\pgfqpoint{3.353424in}{2.348709in}}%
\pgfpathlineto{\pgfqpoint{3.353629in}{2.381715in}}%
\pgfpathlineto{\pgfqpoint{3.353833in}{2.455979in}}%
\pgfpathlineto{\pgfqpoint{3.354445in}{2.389967in}}%
\pgfpathlineto{\pgfqpoint{3.354853in}{2.315702in}}%
\pgfpathlineto{\pgfqpoint{3.355058in}{2.398218in}}%
\pgfpathlineto{\pgfqpoint{3.355466in}{2.398218in}}%
\pgfpathlineto{\pgfqpoint{3.355670in}{2.422973in}}%
\pgfpathlineto{\pgfqpoint{3.356078in}{2.373463in}}%
\pgfpathlineto{\pgfqpoint{3.357507in}{2.290947in}}%
\pgfpathlineto{\pgfqpoint{3.357711in}{2.307450in}}%
\pgfpathlineto{\pgfqpoint{3.357916in}{2.307450in}}%
\pgfpathlineto{\pgfqpoint{3.358120in}{2.299199in}}%
\pgfpathlineto{\pgfqpoint{3.358324in}{2.332205in}}%
\pgfpathlineto{\pgfqpoint{3.358528in}{2.332205in}}%
\pgfpathlineto{\pgfqpoint{3.358732in}{2.290947in}}%
\pgfpathlineto{\pgfqpoint{3.359140in}{2.365212in}}%
\pgfpathlineto{\pgfqpoint{3.359345in}{2.431225in}}%
\pgfpathlineto{\pgfqpoint{3.360365in}{2.414721in}}%
\pgfpathlineto{\pgfqpoint{3.360569in}{2.447728in}}%
\pgfpathlineto{\pgfqpoint{3.360978in}{2.389967in}}%
\pgfpathlineto{\pgfqpoint{3.361386in}{2.414721in}}%
\pgfpathlineto{\pgfqpoint{3.361998in}{2.422973in}}%
\pgfpathlineto{\pgfqpoint{3.362202in}{2.381715in}}%
\pgfpathlineto{\pgfqpoint{3.362815in}{2.472483in}}%
\pgfpathlineto{\pgfqpoint{3.363019in}{2.439476in}}%
\pgfpathlineto{\pgfqpoint{3.363836in}{2.480734in}}%
\pgfpathlineto{\pgfqpoint{3.364040in}{2.414721in}}%
\pgfpathlineto{\pgfqpoint{3.364856in}{2.505489in}}%
\pgfpathlineto{\pgfqpoint{3.365060in}{2.464231in}}%
\pgfpathlineto{\pgfqpoint{3.365265in}{2.530244in}}%
\pgfpathlineto{\pgfqpoint{3.365877in}{2.488986in}}%
\pgfpathlineto{\pgfqpoint{3.366081in}{2.505489in}}%
\pgfpathlineto{\pgfqpoint{3.366489in}{2.472483in}}%
\pgfpathlineto{\pgfqpoint{3.367102in}{2.398218in}}%
\pgfpathlineto{\pgfqpoint{3.367510in}{2.439476in}}%
\pgfpathlineto{\pgfqpoint{3.368735in}{2.513741in}}%
\pgfpathlineto{\pgfqpoint{3.369552in}{2.406470in}}%
\pgfpathlineto{\pgfqpoint{3.370164in}{2.472483in}}%
\pgfpathlineto{\pgfqpoint{3.370368in}{2.513741in}}%
\pgfpathlineto{\pgfqpoint{3.371185in}{2.472483in}}%
\pgfpathlineto{\pgfqpoint{3.372614in}{2.406470in}}%
\pgfpathlineto{\pgfqpoint{3.372818in}{2.389967in}}%
\pgfpathlineto{\pgfqpoint{3.373022in}{2.414721in}}%
\pgfpathlineto{\pgfqpoint{3.373430in}{2.505489in}}%
\pgfpathlineto{\pgfqpoint{3.374043in}{2.398218in}}%
\pgfpathlineto{\pgfqpoint{3.374247in}{2.398218in}}%
\pgfpathlineto{\pgfqpoint{3.374451in}{2.389967in}}%
\pgfpathlineto{\pgfqpoint{3.375268in}{2.513741in}}%
\pgfpathlineto{\pgfqpoint{3.375880in}{2.480734in}}%
\pgfpathlineto{\pgfqpoint{3.376084in}{2.521992in}}%
\pgfpathlineto{\pgfqpoint{3.376492in}{2.406470in}}%
\pgfpathlineto{\pgfqpoint{3.376697in}{2.389967in}}%
\pgfpathlineto{\pgfqpoint{3.376901in}{2.431225in}}%
\pgfpathlineto{\pgfqpoint{3.377105in}{2.422973in}}%
\pgfpathlineto{\pgfqpoint{3.378330in}{2.497237in}}%
\pgfpathlineto{\pgfqpoint{3.380167in}{2.398218in}}%
\pgfpathlineto{\pgfqpoint{3.380371in}{2.406470in}}%
\pgfpathlineto{\pgfqpoint{3.381392in}{2.464231in}}%
\pgfpathlineto{\pgfqpoint{3.381596in}{2.439476in}}%
\pgfpathlineto{\pgfqpoint{3.382617in}{2.323954in}}%
\pgfpathlineto{\pgfqpoint{3.382004in}{2.472483in}}%
\pgfpathlineto{\pgfqpoint{3.383025in}{2.332205in}}%
\pgfpathlineto{\pgfqpoint{3.383229in}{2.356960in}}%
\pgfpathlineto{\pgfqpoint{3.383637in}{2.299199in}}%
\pgfpathlineto{\pgfqpoint{3.383842in}{2.323954in}}%
\pgfpathlineto{\pgfqpoint{3.385066in}{2.290947in}}%
\pgfpathlineto{\pgfqpoint{3.386087in}{2.480734in}}%
\pgfpathlineto{\pgfqpoint{3.386700in}{2.274444in}}%
\pgfpathlineto{\pgfqpoint{3.387312in}{2.323954in}}%
\pgfpathlineto{\pgfqpoint{3.388741in}{2.398218in}}%
\pgfpathlineto{\pgfqpoint{3.388945in}{2.381715in}}%
\pgfpathlineto{\pgfqpoint{3.390578in}{2.117664in}}%
\pgfpathlineto{\pgfqpoint{3.391803in}{2.455979in}}%
\pgfpathlineto{\pgfqpoint{3.392007in}{2.447728in}}%
\pgfpathlineto{\pgfqpoint{3.392211in}{2.455979in}}%
\pgfpathlineto{\pgfqpoint{3.393436in}{2.538495in}}%
\pgfpathlineto{\pgfqpoint{3.393640in}{2.315702in}}%
\pgfpathlineto{\pgfqpoint{3.394457in}{2.439476in}}%
\pgfpathlineto{\pgfqpoint{3.395069in}{2.381715in}}%
\pgfpathlineto{\pgfqpoint{3.395274in}{2.200180in}}%
\pgfpathlineto{\pgfqpoint{3.395886in}{2.406470in}}%
\pgfpathlineto{\pgfqpoint{3.396090in}{2.373463in}}%
\pgfpathlineto{\pgfqpoint{3.396294in}{2.365212in}}%
\pgfpathlineto{\pgfqpoint{3.396498in}{2.373463in}}%
\pgfpathlineto{\pgfqpoint{3.397519in}{2.488986in}}%
\pgfpathlineto{\pgfqpoint{3.396907in}{2.200180in}}%
\pgfpathlineto{\pgfqpoint{3.397723in}{2.455979in}}%
\pgfpathlineto{\pgfqpoint{3.398132in}{2.158922in}}%
\pgfpathlineto{\pgfqpoint{3.398540in}{2.389967in}}%
\pgfpathlineto{\pgfqpoint{3.399561in}{2.546747in}}%
\pgfpathlineto{\pgfqpoint{3.399969in}{2.266192in}}%
\pgfpathlineto{\pgfqpoint{3.400785in}{2.455979in}}%
\pgfpathlineto{\pgfqpoint{3.400990in}{2.530244in}}%
\pgfpathlineto{\pgfqpoint{3.401806in}{2.447728in}}%
\pgfpathlineto{\pgfqpoint{3.402827in}{2.150670in}}%
\pgfpathlineto{\pgfqpoint{3.403031in}{2.290947in}}%
\pgfpathlineto{\pgfqpoint{3.403235in}{2.472483in}}%
\pgfpathlineto{\pgfqpoint{3.404256in}{2.439476in}}%
\pgfpathlineto{\pgfqpoint{3.404868in}{2.480734in}}%
\pgfpathlineto{\pgfqpoint{3.405481in}{2.356960in}}%
\pgfpathlineto{\pgfqpoint{3.405685in}{2.447728in}}%
\pgfpathlineto{\pgfqpoint{3.406093in}{2.200180in}}%
\pgfpathlineto{\pgfqpoint{3.406706in}{2.422973in}}%
\pgfpathlineto{\pgfqpoint{3.407726in}{2.365212in}}%
\pgfpathlineto{\pgfqpoint{3.407930in}{2.406470in}}%
\pgfpathlineto{\pgfqpoint{3.408543in}{2.455979in}}%
\pgfpathlineto{\pgfqpoint{3.408747in}{2.365212in}}%
\pgfpathlineto{\pgfqpoint{3.408951in}{2.332205in}}%
\pgfpathlineto{\pgfqpoint{3.409564in}{2.406470in}}%
\pgfpathlineto{\pgfqpoint{3.410788in}{2.340457in}}%
\pgfpathlineto{\pgfqpoint{3.410993in}{2.356960in}}%
\pgfpathlineto{\pgfqpoint{3.411401in}{2.398218in}}%
\pgfpathlineto{\pgfqpoint{3.411605in}{2.348709in}}%
\pgfpathlineto{\pgfqpoint{3.411809in}{2.257941in}}%
\pgfpathlineto{\pgfqpoint{3.412422in}{2.431225in}}%
\pgfpathlineto{\pgfqpoint{3.412626in}{2.447728in}}%
\pgfpathlineto{\pgfqpoint{3.412830in}{2.406470in}}%
\pgfpathlineto{\pgfqpoint{3.413238in}{2.422973in}}%
\pgfpathlineto{\pgfqpoint{3.414259in}{2.323954in}}%
\pgfpathlineto{\pgfqpoint{3.414463in}{2.373463in}}%
\pgfpathlineto{\pgfqpoint{3.415075in}{2.323954in}}%
\pgfpathlineto{\pgfqpoint{3.415280in}{2.373463in}}%
\pgfpathlineto{\pgfqpoint{3.416096in}{2.422973in}}%
\pgfpathlineto{\pgfqpoint{3.416913in}{2.340457in}}%
\pgfpathlineto{\pgfqpoint{3.417117in}{2.365212in}}%
\pgfpathlineto{\pgfqpoint{3.417525in}{2.398218in}}%
\pgfpathlineto{\pgfqpoint{3.417729in}{2.348709in}}%
\pgfpathlineto{\pgfqpoint{3.417933in}{2.208431in}}%
\pgfpathlineto{\pgfqpoint{3.418546in}{2.365212in}}%
\pgfpathlineto{\pgfqpoint{3.418750in}{2.323954in}}%
\pgfpathlineto{\pgfqpoint{3.418954in}{2.348709in}}%
\pgfpathlineto{\pgfqpoint{3.419158in}{2.167173in}}%
\pgfpathlineto{\pgfqpoint{3.419975in}{2.282696in}}%
\pgfpathlineto{\pgfqpoint{3.421200in}{2.381715in}}%
\pgfpathlineto{\pgfqpoint{3.421404in}{2.373463in}}%
\pgfpathlineto{\pgfqpoint{3.421812in}{2.356960in}}%
\pgfpathlineto{\pgfqpoint{3.422220in}{2.373463in}}%
\pgfpathlineto{\pgfqpoint{3.423241in}{2.414721in}}%
\pgfpathlineto{\pgfqpoint{3.423445in}{2.381715in}}%
\pgfpathlineto{\pgfqpoint{3.423649in}{2.439476in}}%
\pgfpathlineto{\pgfqpoint{3.423854in}{2.406470in}}%
\pgfpathlineto{\pgfqpoint{3.424670in}{2.472483in}}%
\pgfpathlineto{\pgfqpoint{3.424874in}{2.406470in}}%
\pgfpathlineto{\pgfqpoint{3.425283in}{2.381715in}}%
\pgfpathlineto{\pgfqpoint{3.425487in}{2.480734in}}%
\pgfpathlineto{\pgfqpoint{3.426303in}{2.389967in}}%
\pgfpathlineto{\pgfqpoint{3.427120in}{2.398218in}}%
\pgfpathlineto{\pgfqpoint{3.427732in}{2.340457in}}%
\pgfpathlineto{\pgfqpoint{3.428753in}{2.431225in}}%
\pgfpathlineto{\pgfqpoint{3.428141in}{2.282696in}}%
\pgfpathlineto{\pgfqpoint{3.428957in}{2.398218in}}%
\pgfpathlineto{\pgfqpoint{3.429161in}{2.315702in}}%
\pgfpathlineto{\pgfqpoint{3.429978in}{2.406470in}}%
\pgfpathlineto{\pgfqpoint{3.430590in}{2.381715in}}%
\pgfpathlineto{\pgfqpoint{3.430999in}{2.447728in}}%
\pgfpathlineto{\pgfqpoint{3.431203in}{2.406470in}}%
\pgfpathlineto{\pgfqpoint{3.431611in}{2.422973in}}%
\pgfpathlineto{\pgfqpoint{3.432223in}{2.315702in}}%
\pgfpathlineto{\pgfqpoint{3.432836in}{2.546747in}}%
\pgfpathlineto{\pgfqpoint{3.434673in}{2.315702in}}%
\pgfpathlineto{\pgfqpoint{3.434877in}{2.348709in}}%
\pgfpathlineto{\pgfqpoint{3.435286in}{2.274444in}}%
\pgfpathlineto{\pgfqpoint{3.436102in}{2.183676in}}%
\pgfpathlineto{\pgfqpoint{3.436510in}{2.224934in}}%
\pgfpathlineto{\pgfqpoint{3.436715in}{2.290947in}}%
\pgfpathlineto{\pgfqpoint{3.436919in}{2.216683in}}%
\pgfpathlineto{\pgfqpoint{3.437327in}{2.249689in}}%
\pgfpathlineto{\pgfqpoint{3.437531in}{2.167173in}}%
\pgfpathlineto{\pgfqpoint{3.438144in}{2.266192in}}%
\pgfpathlineto{\pgfqpoint{3.438348in}{2.241438in}}%
\pgfpathlineto{\pgfqpoint{3.438960in}{1.969135in}}%
\pgfpathlineto{\pgfqpoint{3.439573in}{2.191928in}}%
\pgfpathlineto{\pgfqpoint{3.439981in}{2.208431in}}%
\pgfpathlineto{\pgfqpoint{3.440185in}{2.134167in}}%
\pgfpathlineto{\pgfqpoint{3.440797in}{2.216683in}}%
\pgfpathlineto{\pgfqpoint{3.441002in}{2.191928in}}%
\pgfpathlineto{\pgfqpoint{3.442226in}{2.315702in}}%
\pgfpathlineto{\pgfqpoint{3.442431in}{2.257941in}}%
\pgfpathlineto{\pgfqpoint{3.443655in}{2.340457in}}%
\pgfpathlineto{\pgfqpoint{3.444676in}{2.224934in}}%
\pgfpathlineto{\pgfqpoint{3.444880in}{2.282696in}}%
\pgfpathlineto{\pgfqpoint{3.446717in}{2.084657in}}%
\pgfpathlineto{\pgfqpoint{3.447330in}{2.035148in}}%
\pgfpathlineto{\pgfqpoint{3.447534in}{2.059902in}}%
\pgfpathlineto{\pgfqpoint{3.448759in}{2.158922in}}%
\pgfpathlineto{\pgfqpoint{3.449780in}{2.101160in}}%
\pgfpathlineto{\pgfqpoint{3.449984in}{2.142418in}}%
\pgfpathlineto{\pgfqpoint{3.450392in}{2.068154in}}%
\pgfpathlineto{\pgfqpoint{3.450800in}{2.092909in}}%
\pgfpathlineto{\pgfqpoint{3.451413in}{2.167173in}}%
\pgfpathlineto{\pgfqpoint{3.451821in}{2.109412in}}%
\pgfpathlineto{\pgfqpoint{3.452025in}{2.084657in}}%
\pgfpathlineto{\pgfqpoint{3.452229in}{2.117664in}}%
\pgfpathlineto{\pgfqpoint{3.452433in}{2.175425in}}%
\pgfpathlineto{\pgfqpoint{3.453046in}{2.092909in}}%
\pgfpathlineto{\pgfqpoint{3.453454in}{2.076406in}}%
\pgfpathlineto{\pgfqpoint{3.453658in}{2.092909in}}%
\pgfpathlineto{\pgfqpoint{3.454679in}{2.257941in}}%
\pgfpathlineto{\pgfqpoint{3.455087in}{2.216683in}}%
\pgfpathlineto{\pgfqpoint{3.455496in}{1.993890in}}%
\pgfpathlineto{\pgfqpoint{3.455904in}{2.315702in}}%
\pgfpathlineto{\pgfqpoint{3.456516in}{2.274444in}}%
\pgfpathlineto{\pgfqpoint{3.456720in}{2.340457in}}%
\pgfpathlineto{\pgfqpoint{3.457129in}{2.249689in}}%
\pgfpathlineto{\pgfqpoint{3.457537in}{2.290947in}}%
\pgfpathlineto{\pgfqpoint{3.457741in}{2.299199in}}%
\pgfpathlineto{\pgfqpoint{3.457945in}{2.266192in}}%
\pgfpathlineto{\pgfqpoint{3.458149in}{2.282696in}}%
\pgfpathlineto{\pgfqpoint{3.458354in}{2.241438in}}%
\pgfpathlineto{\pgfqpoint{3.459170in}{2.266192in}}%
\pgfpathlineto{\pgfqpoint{3.459783in}{2.332205in}}%
\pgfpathlineto{\pgfqpoint{3.459987in}{2.266192in}}%
\pgfpathlineto{\pgfqpoint{3.461416in}{2.200180in}}%
\pgfpathlineto{\pgfqpoint{3.462028in}{2.068154in}}%
\pgfpathlineto{\pgfqpoint{3.462232in}{1.969135in}}%
\pgfpathlineto{\pgfqpoint{3.462436in}{2.233186in}}%
\pgfpathlineto{\pgfqpoint{3.462641in}{2.183676in}}%
\pgfpathlineto{\pgfqpoint{3.463049in}{2.241438in}}%
\pgfpathlineto{\pgfqpoint{3.463457in}{2.158922in}}%
\pgfpathlineto{\pgfqpoint{3.463865in}{2.233186in}}%
\pgfpathlineto{\pgfqpoint{3.464070in}{2.233186in}}%
\pgfpathlineto{\pgfqpoint{3.464274in}{2.208431in}}%
\pgfpathlineto{\pgfqpoint{3.464886in}{2.249689in}}%
\pgfpathlineto{\pgfqpoint{3.465294in}{2.290947in}}%
\pgfpathlineto{\pgfqpoint{3.465499in}{2.216683in}}%
\pgfpathlineto{\pgfqpoint{3.466315in}{2.266192in}}%
\pgfpathlineto{\pgfqpoint{3.466723in}{2.249689in}}%
\pgfpathlineto{\pgfqpoint{3.466928in}{2.266192in}}%
\pgfpathlineto{\pgfqpoint{3.468152in}{2.307450in}}%
\pgfpathlineto{\pgfqpoint{3.468765in}{2.208431in}}%
\pgfpathlineto{\pgfqpoint{3.468969in}{2.249689in}}%
\pgfpathlineto{\pgfqpoint{3.469786in}{2.365212in}}%
\pgfpathlineto{\pgfqpoint{3.469377in}{2.208431in}}%
\pgfpathlineto{\pgfqpoint{3.469990in}{2.323954in}}%
\pgfpathlineto{\pgfqpoint{3.471419in}{2.092909in}}%
\pgfpathlineto{\pgfqpoint{3.472031in}{2.282696in}}%
\pgfpathlineto{\pgfqpoint{3.472644in}{2.200180in}}%
\pgfpathlineto{\pgfqpoint{3.472848in}{2.233186in}}%
\pgfpathlineto{\pgfqpoint{3.473664in}{1.903122in}}%
\pgfpathlineto{\pgfqpoint{3.473868in}{2.018644in}}%
\pgfpathlineto{\pgfqpoint{3.474889in}{2.167173in}}%
\pgfpathlineto{\pgfqpoint{3.475502in}{2.018644in}}%
\pgfpathlineto{\pgfqpoint{3.476114in}{2.068154in}}%
\pgfpathlineto{\pgfqpoint{3.476726in}{2.018644in}}%
\pgfpathlineto{\pgfqpoint{3.476522in}{2.084657in}}%
\pgfpathlineto{\pgfqpoint{3.476931in}{2.043399in}}%
\pgfpathlineto{\pgfqpoint{3.477135in}{2.076406in}}%
\pgfpathlineto{\pgfqpoint{3.477543in}{2.010393in}}%
\pgfpathlineto{\pgfqpoint{3.478155in}{2.059902in}}%
\pgfpathlineto{\pgfqpoint{3.478768in}{1.845361in}}%
\pgfpathlineto{\pgfqpoint{3.479380in}{2.076406in}}%
\pgfpathlineto{\pgfqpoint{3.479584in}{1.903122in}}%
\pgfpathlineto{\pgfqpoint{3.479789in}{1.746341in}}%
\pgfpathlineto{\pgfqpoint{3.480401in}{2.076406in}}%
\pgfpathlineto{\pgfqpoint{3.480605in}{2.010393in}}%
\pgfpathlineto{\pgfqpoint{3.481218in}{2.092909in}}%
\pgfpathlineto{\pgfqpoint{3.481422in}{2.068154in}}%
\pgfpathlineto{\pgfqpoint{3.482034in}{2.010393in}}%
\pgfpathlineto{\pgfqpoint{3.482238in}{2.018644in}}%
\pgfpathlineto{\pgfqpoint{3.482442in}{1.729838in}}%
\pgfpathlineto{\pgfqpoint{3.483259in}{2.035148in}}%
\pgfpathlineto{\pgfqpoint{3.483667in}{2.018644in}}%
\pgfpathlineto{\pgfqpoint{3.483871in}{2.051651in}}%
\pgfpathlineto{\pgfqpoint{3.484076in}{1.993890in}}%
\pgfpathlineto{\pgfqpoint{3.484688in}{2.125915in}}%
\pgfpathlineto{\pgfqpoint{3.484892in}{2.191928in}}%
\pgfpathlineto{\pgfqpoint{3.485300in}{2.043399in}}%
\pgfpathlineto{\pgfqpoint{3.485709in}{2.101160in}}%
\pgfpathlineto{\pgfqpoint{3.485913in}{2.109412in}}%
\pgfpathlineto{\pgfqpoint{3.486117in}{2.068154in}}%
\pgfpathlineto{\pgfqpoint{3.486525in}{2.134167in}}%
\pgfpathlineto{\pgfqpoint{3.486934in}{2.101160in}}%
\pgfpathlineto{\pgfqpoint{3.487138in}{2.101160in}}%
\pgfpathlineto{\pgfqpoint{3.487342in}{2.150670in}}%
\pgfpathlineto{\pgfqpoint{3.487750in}{2.051651in}}%
\pgfpathlineto{\pgfqpoint{3.488158in}{2.109412in}}%
\pgfpathlineto{\pgfqpoint{3.488363in}{2.117664in}}%
\pgfpathlineto{\pgfqpoint{3.489383in}{1.919625in}}%
\pgfpathlineto{\pgfqpoint{3.489996in}{1.960883in}}%
\pgfpathlineto{\pgfqpoint{3.490200in}{1.944380in}}%
\pgfpathlineto{\pgfqpoint{3.490404in}{1.985638in}}%
\pgfpathlineto{\pgfqpoint{3.490608in}{2.043399in}}%
\pgfpathlineto{\pgfqpoint{3.491425in}{1.960883in}}%
\pgfpathlineto{\pgfqpoint{3.492037in}{1.919625in}}%
\pgfpathlineto{\pgfqpoint{3.492854in}{2.059902in}}%
\pgfpathlineto{\pgfqpoint{3.493262in}{2.035148in}}%
\pgfpathlineto{\pgfqpoint{3.494487in}{1.960883in}}%
\pgfpathlineto{\pgfqpoint{3.494895in}{1.985638in}}%
\pgfpathlineto{\pgfqpoint{3.495099in}{1.985638in}}%
\pgfpathlineto{\pgfqpoint{3.495303in}{2.002141in}}%
\pgfpathlineto{\pgfqpoint{3.495508in}{1.977386in}}%
\pgfpathlineto{\pgfqpoint{3.496120in}{1.911373in}}%
\pgfpathlineto{\pgfqpoint{3.496324in}{1.985638in}}%
\pgfpathlineto{\pgfqpoint{3.496528in}{1.969135in}}%
\pgfpathlineto{\pgfqpoint{3.496937in}{2.010393in}}%
\pgfpathlineto{\pgfqpoint{3.497141in}{1.903122in}}%
\pgfpathlineto{\pgfqpoint{3.497345in}{1.919625in}}%
\pgfpathlineto{\pgfqpoint{3.497549in}{1.886619in}}%
\pgfpathlineto{\pgfqpoint{3.498161in}{1.911373in}}%
\pgfpathlineto{\pgfqpoint{3.498774in}{1.804103in}}%
\pgfpathlineto{\pgfqpoint{3.499182in}{1.853612in}}%
\pgfpathlineto{\pgfqpoint{3.499795in}{1.804103in}}%
\pgfpathlineto{\pgfqpoint{3.499999in}{1.804103in}}%
\pgfpathlineto{\pgfqpoint{3.501019in}{1.870115in}}%
\pgfpathlineto{\pgfqpoint{3.500407in}{1.779348in}}%
\pgfpathlineto{\pgfqpoint{3.501428in}{1.820606in}}%
\pgfpathlineto{\pgfqpoint{3.502448in}{1.779348in}}%
\pgfpathlineto{\pgfqpoint{3.502040in}{1.861864in}}%
\pgfpathlineto{\pgfqpoint{3.502653in}{1.804103in}}%
\pgfpathlineto{\pgfqpoint{3.504286in}{1.977386in}}%
\pgfpathlineto{\pgfqpoint{3.505511in}{1.663825in}}%
\pgfpathlineto{\pgfqpoint{3.505715in}{1.820606in}}%
\pgfpathlineto{\pgfqpoint{3.506940in}{2.035148in}}%
\pgfpathlineto{\pgfqpoint{3.507144in}{2.010393in}}%
\pgfpathlineto{\pgfqpoint{3.508164in}{1.886619in}}%
\pgfpathlineto{\pgfqpoint{3.508369in}{1.903122in}}%
\pgfpathlineto{\pgfqpoint{3.509389in}{1.861864in}}%
\pgfpathlineto{\pgfqpoint{3.508981in}{1.911373in}}%
\pgfpathlineto{\pgfqpoint{3.509798in}{1.878367in}}%
\pgfpathlineto{\pgfqpoint{3.510002in}{1.886619in}}%
\pgfpathlineto{\pgfqpoint{3.510206in}{1.870115in}}%
\pgfpathlineto{\pgfqpoint{3.511022in}{1.804103in}}%
\pgfpathlineto{\pgfqpoint{3.511227in}{1.878367in}}%
\pgfpathlineto{\pgfqpoint{3.512043in}{1.779348in}}%
\pgfpathlineto{\pgfqpoint{3.512451in}{1.795851in}}%
\pgfpathlineto{\pgfqpoint{3.512860in}{1.754593in}}%
\pgfpathlineto{\pgfqpoint{3.513268in}{1.886619in}}%
\pgfpathlineto{\pgfqpoint{3.513676in}{1.696832in}}%
\pgfpathlineto{\pgfqpoint{3.513880in}{1.762845in}}%
\pgfpathlineto{\pgfqpoint{3.515309in}{1.556554in}}%
\pgfpathlineto{\pgfqpoint{3.515514in}{1.630819in}}%
\pgfpathlineto{\pgfqpoint{3.516126in}{1.606064in}}%
\pgfpathlineto{\pgfqpoint{3.516330in}{1.729838in}}%
\pgfpathlineto{\pgfqpoint{3.517147in}{1.614316in}}%
\pgfpathlineto{\pgfqpoint{3.517351in}{1.581309in}}%
\pgfpathlineto{\pgfqpoint{3.518167in}{1.597813in}}%
\pgfpathlineto{\pgfqpoint{3.519188in}{1.639071in}}%
\pgfpathlineto{\pgfqpoint{3.519801in}{1.647322in}}%
\pgfpathlineto{\pgfqpoint{3.520005in}{1.581309in}}%
\pgfpathlineto{\pgfqpoint{3.521230in}{1.705083in}}%
\pgfpathlineto{\pgfqpoint{3.521842in}{1.630819in}}%
\pgfpathlineto{\pgfqpoint{3.522250in}{1.705083in}}%
\pgfpathlineto{\pgfqpoint{3.522454in}{1.705083in}}%
\pgfpathlineto{\pgfqpoint{3.523271in}{1.589561in}}%
\pgfpathlineto{\pgfqpoint{3.523883in}{1.630819in}}%
\pgfpathlineto{\pgfqpoint{3.524088in}{1.680329in}}%
\pgfpathlineto{\pgfqpoint{3.524292in}{1.630819in}}%
\pgfpathlineto{\pgfqpoint{3.524496in}{1.383271in}}%
\pgfpathlineto{\pgfqpoint{3.525312in}{1.564806in}}%
\pgfpathlineto{\pgfqpoint{3.525517in}{1.606064in}}%
\pgfpathlineto{\pgfqpoint{3.525721in}{1.490542in}}%
\pgfpathlineto{\pgfqpoint{3.526129in}{1.507045in}}%
\pgfpathlineto{\pgfqpoint{3.526741in}{1.540051in}}%
\pgfpathlineto{\pgfqpoint{3.526946in}{1.490542in}}%
\pgfpathlineto{\pgfqpoint{3.527150in}{1.515296in}}%
\pgfpathlineto{\pgfqpoint{3.528375in}{1.350264in}}%
\pgfpathlineto{\pgfqpoint{3.529599in}{1.515296in}}%
\pgfpathlineto{\pgfqpoint{3.530212in}{1.498793in}}%
\pgfpathlineto{\pgfqpoint{3.530416in}{1.531800in}}%
\pgfpathlineto{\pgfqpoint{3.530620in}{1.523548in}}%
\pgfpathlineto{\pgfqpoint{3.532049in}{1.647322in}}%
\pgfpathlineto{\pgfqpoint{3.532866in}{1.556554in}}%
\pgfpathlineto{\pgfqpoint{3.533886in}{1.705083in}}%
\pgfpathlineto{\pgfqpoint{3.534090in}{1.663825in}}%
\pgfpathlineto{\pgfqpoint{3.534295in}{1.729838in}}%
\pgfpathlineto{\pgfqpoint{3.534499in}{1.630819in}}%
\pgfpathlineto{\pgfqpoint{3.535111in}{1.647322in}}%
\pgfpathlineto{\pgfqpoint{3.536540in}{1.738090in}}%
\pgfpathlineto{\pgfqpoint{3.536744in}{1.705083in}}%
\pgfpathlineto{\pgfqpoint{3.537969in}{1.564806in}}%
\pgfpathlineto{\pgfqpoint{3.537153in}{1.713335in}}%
\pgfpathlineto{\pgfqpoint{3.538582in}{1.597813in}}%
\pgfpathlineto{\pgfqpoint{3.539398in}{1.696832in}}%
\pgfpathlineto{\pgfqpoint{3.539806in}{1.655574in}}%
\pgfpathlineto{\pgfqpoint{3.540215in}{1.655574in}}%
\pgfpathlineto{\pgfqpoint{3.541031in}{1.705083in}}%
\pgfpathlineto{\pgfqpoint{3.540623in}{1.639071in}}%
\pgfpathlineto{\pgfqpoint{3.541440in}{1.663825in}}%
\pgfpathlineto{\pgfqpoint{3.541644in}{1.663825in}}%
\pgfpathlineto{\pgfqpoint{3.541848in}{1.672077in}}%
\pgfpathlineto{\pgfqpoint{3.542052in}{1.663825in}}%
\pgfpathlineto{\pgfqpoint{3.542256in}{1.762845in}}%
\pgfpathlineto{\pgfqpoint{3.543073in}{1.672077in}}%
\pgfpathlineto{\pgfqpoint{3.543481in}{1.647322in}}%
\pgfpathlineto{\pgfqpoint{3.543685in}{1.672077in}}%
\pgfpathlineto{\pgfqpoint{3.544502in}{1.713335in}}%
\pgfpathlineto{\pgfqpoint{3.544706in}{1.696832in}}%
\pgfpathlineto{\pgfqpoint{3.545522in}{1.647322in}}%
\pgfpathlineto{\pgfqpoint{3.545727in}{1.672077in}}%
\pgfpathlineto{\pgfqpoint{3.546543in}{1.738090in}}%
\pgfpathlineto{\pgfqpoint{3.546135in}{1.622567in}}%
\pgfpathlineto{\pgfqpoint{3.546747in}{1.705083in}}%
\pgfpathlineto{\pgfqpoint{3.547564in}{1.663825in}}%
\pgfpathlineto{\pgfqpoint{3.547360in}{1.713335in}}%
\pgfpathlineto{\pgfqpoint{3.547768in}{1.680329in}}%
\pgfpathlineto{\pgfqpoint{3.549401in}{1.853612in}}%
\pgfpathlineto{\pgfqpoint{3.550626in}{1.754593in}}%
\pgfpathlineto{\pgfqpoint{3.553076in}{1.936128in}}%
\pgfpathlineto{\pgfqpoint{3.553688in}{1.820606in}}%
\pgfpathlineto{\pgfqpoint{3.554505in}{1.886619in}}%
\pgfpathlineto{\pgfqpoint{3.555117in}{1.853612in}}%
\pgfpathlineto{\pgfqpoint{3.554913in}{1.894870in}}%
\pgfpathlineto{\pgfqpoint{3.555321in}{1.878367in}}%
\pgfpathlineto{\pgfqpoint{3.555525in}{1.894870in}}%
\pgfpathlineto{\pgfqpoint{3.555730in}{1.853612in}}%
\pgfpathlineto{\pgfqpoint{3.556954in}{1.721587in}}%
\pgfpathlineto{\pgfqpoint{3.556138in}{1.919625in}}%
\pgfpathlineto{\pgfqpoint{3.557159in}{1.779348in}}%
\pgfpathlineto{\pgfqpoint{3.557567in}{1.820606in}}%
\pgfpathlineto{\pgfqpoint{3.557771in}{1.779348in}}%
\pgfpathlineto{\pgfqpoint{3.558996in}{1.647322in}}%
\pgfpathlineto{\pgfqpoint{3.559200in}{1.705083in}}%
\pgfpathlineto{\pgfqpoint{3.560017in}{1.655574in}}%
\pgfpathlineto{\pgfqpoint{3.560629in}{1.597813in}}%
\pgfpathlineto{\pgfqpoint{3.560833in}{1.688580in}}%
\pgfpathlineto{\pgfqpoint{3.561037in}{1.622567in}}%
\pgfpathlineto{\pgfqpoint{3.561446in}{1.713335in}}%
\pgfpathlineto{\pgfqpoint{3.561854in}{1.573058in}}%
\pgfpathlineto{\pgfqpoint{3.562670in}{1.597813in}}%
\pgfpathlineto{\pgfqpoint{3.563691in}{1.531800in}}%
\pgfpathlineto{\pgfqpoint{3.563283in}{1.622567in}}%
\pgfpathlineto{\pgfqpoint{3.563895in}{1.573058in}}%
\pgfpathlineto{\pgfqpoint{3.564304in}{1.663825in}}%
\pgfpathlineto{\pgfqpoint{3.565120in}{1.630819in}}%
\pgfpathlineto{\pgfqpoint{3.565324in}{1.630819in}}%
\pgfpathlineto{\pgfqpoint{3.566549in}{1.729838in}}%
\pgfpathlineto{\pgfqpoint{3.566753in}{1.721587in}}%
\pgfpathlineto{\pgfqpoint{3.566957in}{1.738090in}}%
\pgfpathlineto{\pgfqpoint{3.567162in}{1.696832in}}%
\pgfpathlineto{\pgfqpoint{3.567570in}{1.713335in}}%
\pgfpathlineto{\pgfqpoint{3.568386in}{1.647322in}}%
\pgfpathlineto{\pgfqpoint{3.568591in}{1.622567in}}%
\pgfpathlineto{\pgfqpoint{3.568795in}{1.672077in}}%
\pgfpathlineto{\pgfqpoint{3.568999in}{1.655574in}}%
\pgfpathlineto{\pgfqpoint{3.570428in}{1.812354in}}%
\pgfpathlineto{\pgfqpoint{3.571857in}{1.655574in}}%
\pgfpathlineto{\pgfqpoint{3.572061in}{1.663825in}}%
\pgfpathlineto{\pgfqpoint{3.572265in}{1.630819in}}%
\pgfpathlineto{\pgfqpoint{3.572469in}{1.622567in}}%
\pgfpathlineto{\pgfqpoint{3.573694in}{1.729838in}}%
\pgfpathlineto{\pgfqpoint{3.573898in}{1.688580in}}%
\pgfpathlineto{\pgfqpoint{3.574511in}{1.738090in}}%
\pgfpathlineto{\pgfqpoint{3.575327in}{1.812354in}}%
\pgfpathlineto{\pgfqpoint{3.575531in}{1.738090in}}%
\pgfpathlineto{\pgfqpoint{3.576960in}{1.655574in}}%
\pgfpathlineto{\pgfqpoint{3.577369in}{1.771096in}}%
\pgfpathlineto{\pgfqpoint{3.578594in}{1.729838in}}%
\pgfpathlineto{\pgfqpoint{3.580023in}{1.589561in}}%
\pgfpathlineto{\pgfqpoint{3.579002in}{1.738090in}}%
\pgfpathlineto{\pgfqpoint{3.580431in}{1.630819in}}%
\pgfpathlineto{\pgfqpoint{3.582268in}{1.820606in}}%
\pgfpathlineto{\pgfqpoint{3.583697in}{1.647322in}}%
\pgfpathlineto{\pgfqpoint{3.583901in}{1.680329in}}%
\pgfpathlineto{\pgfqpoint{3.584105in}{1.680329in}}%
\pgfpathlineto{\pgfqpoint{3.584514in}{1.663825in}}%
\pgfpathlineto{\pgfqpoint{3.585330in}{1.754593in}}%
\pgfpathlineto{\pgfqpoint{3.586351in}{1.614316in}}%
\pgfpathlineto{\pgfqpoint{3.586759in}{1.647322in}}%
\pgfpathlineto{\pgfqpoint{3.588188in}{1.738090in}}%
\pgfpathlineto{\pgfqpoint{3.589209in}{1.622567in}}%
\pgfpathlineto{\pgfqpoint{3.589413in}{1.655574in}}%
\pgfpathlineto{\pgfqpoint{3.590434in}{1.804103in}}%
\pgfpathlineto{\pgfqpoint{3.590842in}{1.787599in}}%
\pgfpathlineto{\pgfqpoint{3.591659in}{1.696832in}}%
\pgfpathlineto{\pgfqpoint{3.591863in}{1.738090in}}%
\pgfpathlineto{\pgfqpoint{3.592067in}{1.787599in}}%
\pgfpathlineto{\pgfqpoint{3.592271in}{1.663825in}}%
\pgfpathlineto{\pgfqpoint{3.592475in}{1.688580in}}%
\pgfpathlineto{\pgfqpoint{3.593088in}{1.581309in}}%
\pgfpathlineto{\pgfqpoint{3.593700in}{1.639071in}}%
\pgfpathlineto{\pgfqpoint{3.594108in}{1.680329in}}%
\pgfpathlineto{\pgfqpoint{3.594517in}{1.630819in}}%
\pgfpathlineto{\pgfqpoint{3.595537in}{1.597813in}}%
\pgfpathlineto{\pgfqpoint{3.595742in}{1.639071in}}%
\pgfpathlineto{\pgfqpoint{3.596150in}{1.556554in}}%
\pgfpathlineto{\pgfqpoint{3.596354in}{1.606064in}}%
\pgfpathlineto{\pgfqpoint{3.596558in}{1.573058in}}%
\pgfpathlineto{\pgfqpoint{3.596762in}{1.630819in}}%
\pgfpathlineto{\pgfqpoint{3.597375in}{1.581309in}}%
\pgfpathlineto{\pgfqpoint{3.597579in}{1.672077in}}%
\pgfpathlineto{\pgfqpoint{3.598600in}{1.647322in}}%
\pgfpathlineto{\pgfqpoint{3.598804in}{1.622567in}}%
\pgfpathlineto{\pgfqpoint{3.599008in}{1.696832in}}%
\pgfpathlineto{\pgfqpoint{3.599212in}{1.696832in}}%
\pgfpathlineto{\pgfqpoint{3.599416in}{1.771096in}}%
\pgfpathlineto{\pgfqpoint{3.600029in}{1.655574in}}%
\pgfpathlineto{\pgfqpoint{3.600233in}{1.721587in}}%
\pgfpathlineto{\pgfqpoint{3.601253in}{1.589561in}}%
\pgfpathlineto{\pgfqpoint{3.601458in}{1.639071in}}%
\pgfpathlineto{\pgfqpoint{3.601662in}{1.647322in}}%
\pgfpathlineto{\pgfqpoint{3.601866in}{1.482290in}}%
\pgfpathlineto{\pgfqpoint{3.602478in}{1.721587in}}%
\pgfpathlineto{\pgfqpoint{3.602682in}{1.729838in}}%
\pgfpathlineto{\pgfqpoint{3.602887in}{1.713335in}}%
\pgfpathlineto{\pgfqpoint{3.603499in}{1.630819in}}%
\pgfpathlineto{\pgfqpoint{3.604316in}{1.639071in}}%
\pgfpathlineto{\pgfqpoint{3.604724in}{1.696832in}}%
\pgfpathlineto{\pgfqpoint{3.604928in}{1.366768in}}%
\pgfpathlineto{\pgfqpoint{3.605745in}{1.639071in}}%
\pgfpathlineto{\pgfqpoint{3.606153in}{1.762845in}}%
\pgfpathlineto{\pgfqpoint{3.606357in}{1.556554in}}%
\pgfpathlineto{\pgfqpoint{3.606969in}{1.713335in}}%
\pgfpathlineto{\pgfqpoint{3.607582in}{1.771096in}}%
\pgfpathlineto{\pgfqpoint{3.607786in}{1.705083in}}%
\pgfpathlineto{\pgfqpoint{3.607990in}{1.729838in}}%
\pgfpathlineto{\pgfqpoint{3.608398in}{1.705083in}}%
\pgfpathlineto{\pgfqpoint{3.608807in}{1.754593in}}%
\pgfpathlineto{\pgfqpoint{3.609011in}{1.771096in}}%
\pgfpathlineto{\pgfqpoint{3.609215in}{1.713335in}}%
\pgfpathlineto{\pgfqpoint{3.609623in}{1.738090in}}%
\pgfpathlineto{\pgfqpoint{3.609827in}{1.564806in}}%
\pgfpathlineto{\pgfqpoint{3.610236in}{1.754593in}}%
\pgfpathlineto{\pgfqpoint{3.610644in}{1.672077in}}%
\pgfpathlineto{\pgfqpoint{3.611665in}{1.779348in}}%
\pgfpathlineto{\pgfqpoint{3.611869in}{1.729838in}}%
\pgfpathlineto{\pgfqpoint{3.612481in}{1.441032in}}%
\pgfpathlineto{\pgfqpoint{3.613094in}{1.672077in}}%
\pgfpathlineto{\pgfqpoint{3.613298in}{1.754593in}}%
\pgfpathlineto{\pgfqpoint{3.613910in}{1.663825in}}%
\pgfpathlineto{\pgfqpoint{3.614114in}{1.663825in}}%
\pgfpathlineto{\pgfqpoint{3.614319in}{1.432780in}}%
\pgfpathlineto{\pgfqpoint{3.614931in}{1.754593in}}%
\pgfpathlineto{\pgfqpoint{3.615135in}{1.705083in}}%
\pgfpathlineto{\pgfqpoint{3.615339in}{1.729838in}}%
\pgfpathlineto{\pgfqpoint{3.615952in}{1.688580in}}%
\pgfpathlineto{\pgfqpoint{3.616156in}{1.680329in}}%
\pgfpathlineto{\pgfqpoint{3.616564in}{1.696832in}}%
\pgfpathlineto{\pgfqpoint{3.617585in}{1.820606in}}%
\pgfpathlineto{\pgfqpoint{3.617993in}{1.779348in}}%
\pgfpathlineto{\pgfqpoint{3.618197in}{1.564806in}}%
\pgfpathlineto{\pgfqpoint{3.619014in}{1.721587in}}%
\pgfpathlineto{\pgfqpoint{3.619218in}{1.762845in}}%
\pgfpathlineto{\pgfqpoint{3.619422in}{1.705083in}}%
\pgfpathlineto{\pgfqpoint{3.619830in}{1.713335in}}%
\pgfpathlineto{\pgfqpoint{3.620443in}{1.729838in}}%
\pgfpathlineto{\pgfqpoint{3.621055in}{1.663825in}}%
\pgfpathlineto{\pgfqpoint{3.622892in}{1.845361in}}%
\pgfpathlineto{\pgfqpoint{3.621463in}{1.622567in}}%
\pgfpathlineto{\pgfqpoint{3.623097in}{1.837109in}}%
\pgfpathlineto{\pgfqpoint{3.624526in}{1.515296in}}%
\pgfpathlineto{\pgfqpoint{3.624730in}{1.498793in}}%
\pgfpathlineto{\pgfqpoint{3.624934in}{1.556554in}}%
\pgfpathlineto{\pgfqpoint{3.625138in}{1.564806in}}%
\pgfpathlineto{\pgfqpoint{3.625546in}{1.490542in}}%
\pgfpathlineto{\pgfqpoint{3.626159in}{1.540051in}}%
\pgfpathlineto{\pgfqpoint{3.626567in}{1.573058in}}%
\pgfpathlineto{\pgfqpoint{3.627588in}{1.408026in}}%
\pgfpathlineto{\pgfqpoint{3.627792in}{1.457535in}}%
\pgfpathlineto{\pgfqpoint{3.629425in}{1.309006in}}%
\pgfpathlineto{\pgfqpoint{3.628608in}{1.498793in}}%
\pgfpathlineto{\pgfqpoint{3.629629in}{1.333761in}}%
\pgfpathlineto{\pgfqpoint{3.629833in}{1.309006in}}%
\pgfpathlineto{\pgfqpoint{3.630037in}{1.366768in}}%
\pgfpathlineto{\pgfqpoint{3.630446in}{1.350264in}}%
\pgfpathlineto{\pgfqpoint{3.630650in}{1.375019in}}%
\pgfpathlineto{\pgfqpoint{3.630854in}{1.309006in}}%
\pgfpathlineto{\pgfqpoint{3.631875in}{1.102716in}}%
\pgfpathlineto{\pgfqpoint{3.632487in}{1.152226in}}%
\pgfpathlineto{\pgfqpoint{3.632691in}{1.383271in}}%
\pgfpathlineto{\pgfqpoint{3.633712in}{1.284252in}}%
\pgfpathlineto{\pgfqpoint{3.633916in}{1.408026in}}%
\pgfpathlineto{\pgfqpoint{3.634733in}{1.325510in}}%
\pgfpathlineto{\pgfqpoint{3.635753in}{1.110968in}}%
\pgfpathlineto{\pgfqpoint{3.636366in}{1.226490in}}%
\pgfpathlineto{\pgfqpoint{3.636570in}{1.209987in}}%
\pgfpathlineto{\pgfqpoint{3.636774in}{1.276000in}}%
\pgfpathlineto{\pgfqpoint{3.636978in}{1.317258in}}%
\pgfpathlineto{\pgfqpoint{3.637182in}{1.242994in}}%
\pgfpathlineto{\pgfqpoint{3.637591in}{1.044955in}}%
\pgfpathlineto{\pgfqpoint{3.638407in}{1.135723in}}%
\pgfpathlineto{\pgfqpoint{3.638816in}{1.176981in}}%
\pgfpathlineto{\pgfqpoint{3.640245in}{0.896426in}}%
\pgfpathlineto{\pgfqpoint{3.640449in}{0.921181in}}%
\pgfpathlineto{\pgfqpoint{3.640857in}{0.912929in}}%
\pgfpathlineto{\pgfqpoint{3.641674in}{0.731394in}}%
\pgfpathlineto{\pgfqpoint{3.641878in}{0.747897in}}%
\pgfpathlineto{\pgfqpoint{3.643103in}{1.003697in}}%
\pgfpathlineto{\pgfqpoint{3.643511in}{1.077961in}}%
\pgfpathlineto{\pgfqpoint{3.644327in}{0.896426in}}%
\pgfpathlineto{\pgfqpoint{3.645144in}{1.152226in}}%
\pgfpathlineto{\pgfqpoint{3.645552in}{0.995445in}}%
\pgfpathlineto{\pgfqpoint{3.645756in}{1.011949in}}%
\pgfpathlineto{\pgfqpoint{3.646165in}{0.648878in}}%
\pgfpathlineto{\pgfqpoint{3.646573in}{1.069710in}}%
\pgfpathlineto{\pgfqpoint{3.646777in}{1.028452in}}%
\pgfpathlineto{\pgfqpoint{3.647185in}{0.921181in}}%
\pgfpathlineto{\pgfqpoint{3.647594in}{1.094465in}}%
\pgfpathlineto{\pgfqpoint{3.647798in}{1.036703in}}%
\pgfpathlineto{\pgfqpoint{3.648206in}{1.127471in}}%
\pgfpathlineto{\pgfqpoint{3.648819in}{1.044955in}}%
\pgfpathlineto{\pgfqpoint{3.649431in}{0.987194in}}%
\pgfpathlineto{\pgfqpoint{3.649839in}{0.995445in}}%
\pgfpathlineto{\pgfqpoint{3.650043in}{1.028452in}}%
\pgfpathlineto{\pgfqpoint{3.650248in}{1.011949in}}%
\pgfpathlineto{\pgfqpoint{3.651472in}{0.813910in}}%
\pgfpathlineto{\pgfqpoint{3.651677in}{0.855168in}}%
\pgfpathlineto{\pgfqpoint{3.651881in}{0.797407in}}%
\pgfpathlineto{\pgfqpoint{3.652493in}{0.879923in}}%
\pgfpathlineto{\pgfqpoint{3.652697in}{0.871671in}}%
\pgfpathlineto{\pgfqpoint{3.653718in}{1.135723in}}%
\pgfpathlineto{\pgfqpoint{3.653922in}{1.094465in}}%
\pgfpathlineto{\pgfqpoint{3.655147in}{0.846917in}}%
\pgfpathlineto{\pgfqpoint{3.656372in}{1.218239in}}%
\pgfpathlineto{\pgfqpoint{3.656780in}{1.069710in}}%
\pgfpathlineto{\pgfqpoint{3.657393in}{1.143974in}}%
\pgfpathlineto{\pgfqpoint{3.658209in}{1.259497in}}%
\pgfpathlineto{\pgfqpoint{3.658413in}{1.069710in}}%
\pgfpathlineto{\pgfqpoint{3.659230in}{1.218239in}}%
\pgfpathlineto{\pgfqpoint{3.659842in}{1.317258in}}%
\pgfpathlineto{\pgfqpoint{3.660046in}{1.185232in}}%
\pgfpathlineto{\pgfqpoint{3.660251in}{1.185232in}}%
\pgfpathlineto{\pgfqpoint{3.660455in}{1.226490in}}%
\pgfpathlineto{\pgfqpoint{3.660863in}{1.127471in}}%
\pgfpathlineto{\pgfqpoint{3.661067in}{1.044955in}}%
\pgfpathlineto{\pgfqpoint{3.661884in}{1.160477in}}%
\pgfpathlineto{\pgfqpoint{3.662088in}{1.135723in}}%
\pgfpathlineto{\pgfqpoint{3.662496in}{1.201736in}}%
\pgfpathlineto{\pgfqpoint{3.662700in}{1.242994in}}%
\pgfpathlineto{\pgfqpoint{3.663109in}{1.135723in}}%
\pgfpathlineto{\pgfqpoint{3.663313in}{1.201736in}}%
\pgfpathlineto{\pgfqpoint{3.663517in}{1.044955in}}%
\pgfpathlineto{\pgfqpoint{3.664333in}{1.152226in}}%
\pgfpathlineto{\pgfqpoint{3.664538in}{1.259497in}}%
\pgfpathlineto{\pgfqpoint{3.664946in}{1.094465in}}%
\pgfpathlineto{\pgfqpoint{3.665354in}{1.185232in}}%
\pgfpathlineto{\pgfqpoint{3.666171in}{1.143974in}}%
\pgfpathlineto{\pgfqpoint{3.666579in}{1.168729in}}%
\pgfpathlineto{\pgfqpoint{3.667191in}{1.135723in}}%
\pgfpathlineto{\pgfqpoint{3.667396in}{1.176981in}}%
\pgfpathlineto{\pgfqpoint{3.668212in}{1.259497in}}%
\pgfpathlineto{\pgfqpoint{3.669029in}{1.234742in}}%
\pgfpathlineto{\pgfqpoint{3.669233in}{1.160477in}}%
\pgfpathlineto{\pgfqpoint{3.669845in}{1.276000in}}%
\pgfpathlineto{\pgfqpoint{3.670049in}{1.259497in}}%
\pgfpathlineto{\pgfqpoint{3.671478in}{1.358516in}}%
\pgfpathlineto{\pgfqpoint{3.672907in}{1.523548in}}%
\pgfpathlineto{\pgfqpoint{3.673316in}{1.482290in}}%
\pgfpathlineto{\pgfqpoint{3.673724in}{1.457535in}}%
\pgfpathlineto{\pgfqpoint{3.673928in}{1.490542in}}%
\pgfpathlineto{\pgfqpoint{3.675153in}{1.531800in}}%
\pgfpathlineto{\pgfqpoint{3.675357in}{1.490542in}}%
\pgfpathlineto{\pgfqpoint{3.675561in}{1.564806in}}%
\pgfpathlineto{\pgfqpoint{3.675765in}{1.556554in}}%
\pgfpathlineto{\pgfqpoint{3.676378in}{1.746341in}}%
\pgfpathlineto{\pgfqpoint{3.676786in}{1.556554in}}%
\pgfpathlineto{\pgfqpoint{3.677194in}{1.573058in}}%
\pgfpathlineto{\pgfqpoint{3.677399in}{1.548303in}}%
\pgfpathlineto{\pgfqpoint{3.677603in}{1.639071in}}%
\pgfpathlineto{\pgfqpoint{3.678419in}{1.589561in}}%
\pgfpathlineto{\pgfqpoint{3.679644in}{1.490542in}}%
\pgfpathlineto{\pgfqpoint{3.679848in}{1.498793in}}%
\pgfpathlineto{\pgfqpoint{3.680052in}{1.531800in}}%
\pgfpathlineto{\pgfqpoint{3.680461in}{1.482290in}}%
\pgfpathlineto{\pgfqpoint{3.680869in}{1.507045in}}%
\pgfpathlineto{\pgfqpoint{3.681277in}{1.515296in}}%
\pgfpathlineto{\pgfqpoint{3.681481in}{1.630819in}}%
\pgfpathlineto{\pgfqpoint{3.682094in}{1.457535in}}%
\pgfpathlineto{\pgfqpoint{3.682298in}{1.482290in}}%
\pgfpathlineto{\pgfqpoint{3.682502in}{1.457535in}}%
\pgfpathlineto{\pgfqpoint{3.682910in}{1.490542in}}%
\pgfpathlineto{\pgfqpoint{3.684339in}{1.647322in}}%
\pgfpathlineto{\pgfqpoint{3.684748in}{1.622567in}}%
\pgfpathlineto{\pgfqpoint{3.684952in}{1.581309in}}%
\pgfpathlineto{\pgfqpoint{3.685360in}{1.680329in}}%
\pgfpathlineto{\pgfqpoint{3.685564in}{1.639071in}}%
\pgfpathlineto{\pgfqpoint{3.686177in}{1.804103in}}%
\pgfpathlineto{\pgfqpoint{3.686789in}{1.762845in}}%
\pgfpathlineto{\pgfqpoint{3.690055in}{2.026896in}}%
\pgfpathlineto{\pgfqpoint{3.690260in}{2.035148in}}%
\pgfpathlineto{\pgfqpoint{3.690464in}{1.927877in}}%
\pgfpathlineto{\pgfqpoint{3.691280in}{2.026896in}}%
\pgfpathlineto{\pgfqpoint{3.691484in}{2.035148in}}%
\pgfpathlineto{\pgfqpoint{3.692301in}{2.167173in}}%
\pgfpathlineto{\pgfqpoint{3.692505in}{2.109412in}}%
\pgfpathlineto{\pgfqpoint{3.693526in}{2.117664in}}%
\pgfpathlineto{\pgfqpoint{3.693934in}{1.795851in}}%
\pgfpathlineto{\pgfqpoint{3.694547in}{2.092909in}}%
\pgfpathlineto{\pgfqpoint{3.695159in}{2.076406in}}%
\pgfpathlineto{\pgfqpoint{3.695771in}{2.010393in}}%
\pgfpathlineto{\pgfqpoint{3.696180in}{2.084657in}}%
\pgfpathlineto{\pgfqpoint{3.696384in}{2.076406in}}%
\pgfpathlineto{\pgfqpoint{3.696588in}{2.142418in}}%
\pgfpathlineto{\pgfqpoint{3.696996in}{2.018644in}}%
\pgfpathlineto{\pgfqpoint{3.697200in}{2.059902in}}%
\pgfpathlineto{\pgfqpoint{3.697405in}{1.795851in}}%
\pgfpathlineto{\pgfqpoint{3.698221in}{1.969135in}}%
\pgfpathlineto{\pgfqpoint{3.698425in}{1.969135in}}%
\pgfpathlineto{\pgfqpoint{3.700262in}{2.183676in}}%
\pgfpathlineto{\pgfqpoint{3.700875in}{2.059902in}}%
\pgfpathlineto{\pgfqpoint{3.701691in}{2.092909in}}%
\pgfpathlineto{\pgfqpoint{3.701896in}{2.092909in}}%
\pgfpathlineto{\pgfqpoint{3.702304in}{2.134167in}}%
\pgfpathlineto{\pgfqpoint{3.702916in}{2.092909in}}%
\pgfpathlineto{\pgfqpoint{3.703529in}{2.101160in}}%
\pgfpathlineto{\pgfqpoint{3.703937in}{2.059902in}}%
\pgfpathlineto{\pgfqpoint{3.705162in}{2.175425in}}%
\pgfpathlineto{\pgfqpoint{3.705570in}{2.142418in}}%
\pgfpathlineto{\pgfqpoint{3.705978in}{2.158922in}}%
\pgfpathlineto{\pgfqpoint{3.706183in}{2.208431in}}%
\pgfpathlineto{\pgfqpoint{3.706999in}{2.167173in}}%
\pgfpathlineto{\pgfqpoint{3.707203in}{2.167173in}}%
\pgfpathlineto{\pgfqpoint{3.707612in}{2.233186in}}%
\pgfpathlineto{\pgfqpoint{3.707816in}{2.158922in}}%
\pgfpathlineto{\pgfqpoint{3.708020in}{2.125915in}}%
\pgfpathlineto{\pgfqpoint{3.708224in}{2.200180in}}%
\pgfpathlineto{\pgfqpoint{3.708632in}{2.191928in}}%
\pgfpathlineto{\pgfqpoint{3.708836in}{2.191928in}}%
\pgfpathlineto{\pgfqpoint{3.709245in}{2.266192in}}%
\pgfpathlineto{\pgfqpoint{3.710061in}{2.249689in}}%
\pgfpathlineto{\pgfqpoint{3.710470in}{2.191928in}}%
\pgfpathlineto{\pgfqpoint{3.710674in}{2.175425in}}%
\pgfpathlineto{\pgfqpoint{3.710878in}{2.216683in}}%
\pgfpathlineto{\pgfqpoint{3.711082in}{2.200180in}}%
\pgfpathlineto{\pgfqpoint{3.711286in}{2.224934in}}%
\pgfpathlineto{\pgfqpoint{3.712103in}{2.216683in}}%
\pgfpathlineto{\pgfqpoint{3.712511in}{2.183676in}}%
\pgfpathlineto{\pgfqpoint{3.712919in}{2.200180in}}%
\pgfpathlineto{\pgfqpoint{3.713123in}{2.257941in}}%
\pgfpathlineto{\pgfqpoint{3.714144in}{2.249689in}}%
\pgfpathlineto{\pgfqpoint{3.715981in}{2.200180in}}%
\pgfpathlineto{\pgfqpoint{3.716186in}{2.208431in}}%
\pgfpathlineto{\pgfqpoint{3.718227in}{2.373463in}}%
\pgfpathlineto{\pgfqpoint{3.718635in}{2.323954in}}%
\pgfpathlineto{\pgfqpoint{3.719044in}{2.348709in}}%
\pgfpathlineto{\pgfqpoint{3.720064in}{2.191928in}}%
\pgfpathlineto{\pgfqpoint{3.720473in}{2.224934in}}%
\pgfpathlineto{\pgfqpoint{3.720881in}{2.167173in}}%
\pgfpathlineto{\pgfqpoint{3.721085in}{2.191928in}}%
\pgfpathlineto{\pgfqpoint{3.721697in}{2.175425in}}%
\pgfpathlineto{\pgfqpoint{3.721902in}{2.200180in}}%
\pgfpathlineto{\pgfqpoint{3.722514in}{2.282696in}}%
\pgfpathlineto{\pgfqpoint{3.723126in}{2.274444in}}%
\pgfpathlineto{\pgfqpoint{3.724555in}{2.208431in}}%
\pgfpathlineto{\pgfqpoint{3.725576in}{2.158922in}}%
\pgfpathlineto{\pgfqpoint{3.724964in}{2.216683in}}%
\pgfpathlineto{\pgfqpoint{3.725780in}{2.191928in}}%
\pgfpathlineto{\pgfqpoint{3.726189in}{2.249689in}}%
\pgfpathlineto{\pgfqpoint{3.726801in}{2.233186in}}%
\pgfpathlineto{\pgfqpoint{3.727005in}{2.191928in}}%
\pgfpathlineto{\pgfqpoint{3.727413in}{2.274444in}}%
\pgfpathlineto{\pgfqpoint{3.727618in}{2.249689in}}%
\pgfpathlineto{\pgfqpoint{3.727822in}{2.274444in}}%
\pgfpathlineto{\pgfqpoint{3.728230in}{2.224934in}}%
\pgfpathlineto{\pgfqpoint{3.728434in}{2.249689in}}%
\pgfpathlineto{\pgfqpoint{3.729047in}{1.903122in}}%
\pgfpathlineto{\pgfqpoint{3.729659in}{2.134167in}}%
\pgfpathlineto{\pgfqpoint{3.730067in}{2.208431in}}%
\pgfpathlineto{\pgfqpoint{3.730271in}{2.125915in}}%
\pgfpathlineto{\pgfqpoint{3.730476in}{1.886619in}}%
\pgfpathlineto{\pgfqpoint{3.731088in}{2.241438in}}%
\pgfpathlineto{\pgfqpoint{3.731292in}{2.158922in}}%
\pgfpathlineto{\pgfqpoint{3.731700in}{2.315702in}}%
\pgfpathlineto{\pgfqpoint{3.732517in}{2.290947in}}%
\pgfpathlineto{\pgfqpoint{3.735171in}{2.059902in}}%
\pgfpathlineto{\pgfqpoint{3.736192in}{2.307450in}}%
\pgfpathlineto{\pgfqpoint{3.736600in}{2.282696in}}%
\pgfpathlineto{\pgfqpoint{3.736804in}{2.282696in}}%
\pgfpathlineto{\pgfqpoint{3.737212in}{2.274444in}}%
\pgfpathlineto{\pgfqpoint{3.738029in}{2.332205in}}%
\pgfpathlineto{\pgfqpoint{3.738233in}{2.323954in}}%
\pgfpathlineto{\pgfqpoint{3.738437in}{2.381715in}}%
\pgfpathlineto{\pgfqpoint{3.738845in}{2.274444in}}%
\pgfpathlineto{\pgfqpoint{3.739050in}{2.274444in}}%
\pgfpathlineto{\pgfqpoint{3.739254in}{2.035148in}}%
\pgfpathlineto{\pgfqpoint{3.739662in}{2.307450in}}%
\pgfpathlineto{\pgfqpoint{3.740070in}{2.290947in}}%
\pgfpathlineto{\pgfqpoint{3.741295in}{2.249689in}}%
\pgfpathlineto{\pgfqpoint{3.741908in}{2.323954in}}%
\pgfpathlineto{\pgfqpoint{3.742316in}{2.266192in}}%
\pgfpathlineto{\pgfqpoint{3.742724in}{2.323954in}}%
\pgfpathlineto{\pgfqpoint{3.743541in}{2.307450in}}%
\pgfpathlineto{\pgfqpoint{3.743745in}{2.274444in}}%
\pgfpathlineto{\pgfqpoint{3.744153in}{2.365212in}}%
\pgfpathlineto{\pgfqpoint{3.744357in}{2.365212in}}%
\pgfpathlineto{\pgfqpoint{3.744561in}{2.431225in}}%
\pgfpathlineto{\pgfqpoint{3.745174in}{2.315702in}}%
\pgfpathlineto{\pgfqpoint{3.745378in}{2.356960in}}%
\pgfpathlineto{\pgfqpoint{3.746195in}{2.299199in}}%
\pgfpathlineto{\pgfqpoint{3.746807in}{2.373463in}}%
\pgfpathlineto{\pgfqpoint{3.747215in}{2.332205in}}%
\pgfpathlineto{\pgfqpoint{3.747419in}{2.315702in}}%
\pgfpathlineto{\pgfqpoint{3.747828in}{2.431225in}}%
\pgfpathlineto{\pgfqpoint{3.748440in}{2.389967in}}%
\pgfpathlineto{\pgfqpoint{3.749665in}{2.299199in}}%
\pgfpathlineto{\pgfqpoint{3.748848in}{2.406470in}}%
\pgfpathlineto{\pgfqpoint{3.749869in}{2.323954in}}%
\pgfpathlineto{\pgfqpoint{3.750073in}{2.340457in}}%
\pgfpathlineto{\pgfqpoint{3.750277in}{2.315702in}}%
\pgfpathlineto{\pgfqpoint{3.750482in}{2.257941in}}%
\pgfpathlineto{\pgfqpoint{3.750686in}{2.332205in}}%
\pgfpathlineto{\pgfqpoint{3.751094in}{2.332205in}}%
\pgfpathlineto{\pgfqpoint{3.751706in}{2.381715in}}%
\pgfpathlineto{\pgfqpoint{3.752115in}{2.348709in}}%
\pgfpathlineto{\pgfqpoint{3.752523in}{2.307450in}}%
\pgfpathlineto{\pgfqpoint{3.752727in}{2.398218in}}%
\pgfpathlineto{\pgfqpoint{3.753544in}{2.299199in}}%
\pgfpathlineto{\pgfqpoint{3.754769in}{2.406470in}}%
\pgfpathlineto{\pgfqpoint{3.755177in}{2.365212in}}%
\pgfpathlineto{\pgfqpoint{3.755789in}{2.332205in}}%
\pgfpathlineto{\pgfqpoint{3.756198in}{2.381715in}}%
\pgfpathlineto{\pgfqpoint{3.756810in}{2.356960in}}%
\pgfpathlineto{\pgfqpoint{3.758035in}{2.274444in}}%
\pgfpathlineto{\pgfqpoint{3.758443in}{2.282696in}}%
\pgfpathlineto{\pgfqpoint{3.758647in}{2.332205in}}%
\pgfpathlineto{\pgfqpoint{3.759464in}{2.290947in}}%
\pgfpathlineto{\pgfqpoint{3.759872in}{2.307450in}}%
\pgfpathlineto{\pgfqpoint{3.760076in}{2.257941in}}%
\pgfpathlineto{\pgfqpoint{3.760280in}{2.274444in}}%
\pgfpathlineto{\pgfqpoint{3.760485in}{2.233186in}}%
\pgfpathlineto{\pgfqpoint{3.760893in}{2.323954in}}%
\pgfpathlineto{\pgfqpoint{3.761301in}{2.282696in}}%
\pgfpathlineto{\pgfqpoint{3.762526in}{2.373463in}}%
\pgfpathlineto{\pgfqpoint{3.763343in}{2.455979in}}%
\pgfpathlineto{\pgfqpoint{3.763751in}{2.406470in}}%
\pgfpathlineto{\pgfqpoint{3.765180in}{2.340457in}}%
\pgfpathlineto{\pgfqpoint{3.765384in}{2.398218in}}%
\pgfpathlineto{\pgfqpoint{3.765588in}{2.315702in}}%
\pgfpathlineto{\pgfqpoint{3.765792in}{2.191928in}}%
\pgfpathlineto{\pgfqpoint{3.766405in}{2.381715in}}%
\pgfpathlineto{\pgfqpoint{3.766609in}{2.447728in}}%
\pgfpathlineto{\pgfqpoint{3.767425in}{2.365212in}}%
\pgfpathlineto{\pgfqpoint{3.767630in}{2.422973in}}%
\pgfpathlineto{\pgfqpoint{3.768038in}{2.406470in}}%
\pgfpathlineto{\pgfqpoint{3.768242in}{2.431225in}}%
\pgfpathlineto{\pgfqpoint{3.768854in}{2.117664in}}%
\pgfpathlineto{\pgfqpoint{3.769263in}{2.282696in}}%
\pgfpathlineto{\pgfqpoint{3.770079in}{2.455979in}}%
\pgfpathlineto{\pgfqpoint{3.770283in}{2.208431in}}%
\pgfpathlineto{\pgfqpoint{3.771100in}{2.414721in}}%
\pgfpathlineto{\pgfqpoint{3.771304in}{2.414721in}}%
\pgfpathlineto{\pgfqpoint{3.771712in}{2.381715in}}%
\pgfpathlineto{\pgfqpoint{3.772325in}{2.406470in}}%
\pgfpathlineto{\pgfqpoint{3.772733in}{2.422973in}}%
\pgfpathlineto{\pgfqpoint{3.772937in}{2.398218in}}%
\pgfpathlineto{\pgfqpoint{3.773141in}{2.406470in}}%
\pgfpathlineto{\pgfqpoint{3.773346in}{2.381715in}}%
\pgfpathlineto{\pgfqpoint{3.773550in}{2.109412in}}%
\pgfpathlineto{\pgfqpoint{3.774366in}{2.323954in}}%
\pgfpathlineto{\pgfqpoint{3.775387in}{2.389967in}}%
\pgfpathlineto{\pgfqpoint{3.775591in}{2.365212in}}%
\pgfpathlineto{\pgfqpoint{3.776408in}{2.290947in}}%
\pgfpathlineto{\pgfqpoint{3.776612in}{2.315702in}}%
\pgfpathlineto{\pgfqpoint{3.776816in}{2.389967in}}%
\pgfpathlineto{\pgfqpoint{3.777633in}{2.332205in}}%
\pgfpathlineto{\pgfqpoint{3.778857in}{2.257941in}}%
\pgfpathlineto{\pgfqpoint{3.779062in}{2.299199in}}%
\pgfpathlineto{\pgfqpoint{3.779878in}{2.266192in}}%
\pgfpathlineto{\pgfqpoint{3.780082in}{2.266192in}}%
\pgfpathlineto{\pgfqpoint{3.780286in}{2.307450in}}%
\pgfpathlineto{\pgfqpoint{3.780695in}{2.233186in}}%
\pgfpathlineto{\pgfqpoint{3.781103in}{2.249689in}}%
\pgfpathlineto{\pgfqpoint{3.782940in}{2.167173in}}%
\pgfpathlineto{\pgfqpoint{3.784165in}{2.249689in}}%
\pgfpathlineto{\pgfqpoint{3.784369in}{2.216683in}}%
\pgfpathlineto{\pgfqpoint{3.784573in}{2.175425in}}%
\pgfpathlineto{\pgfqpoint{3.785186in}{2.274444in}}%
\pgfpathlineto{\pgfqpoint{3.785798in}{2.282696in}}%
\pgfpathlineto{\pgfqpoint{3.786411in}{2.208431in}}%
\pgfpathlineto{\pgfqpoint{3.787840in}{2.315702in}}%
\pgfpathlineto{\pgfqpoint{3.788248in}{2.233186in}}%
\pgfpathlineto{\pgfqpoint{3.788452in}{2.125915in}}%
\pgfpathlineto{\pgfqpoint{3.789064in}{2.290947in}}%
\pgfpathlineto{\pgfqpoint{3.789269in}{2.241438in}}%
\pgfpathlineto{\pgfqpoint{3.789881in}{2.158922in}}%
\pgfpathlineto{\pgfqpoint{3.790085in}{2.175425in}}%
\pgfpathlineto{\pgfqpoint{3.790289in}{1.977386in}}%
\pgfpathlineto{\pgfqpoint{3.790698in}{2.266192in}}%
\pgfpathlineto{\pgfqpoint{3.791106in}{1.993890in}}%
\pgfpathlineto{\pgfqpoint{3.791310in}{2.249689in}}%
\pgfpathlineto{\pgfqpoint{3.791718in}{1.870115in}}%
\pgfpathlineto{\pgfqpoint{3.792331in}{2.167173in}}%
\pgfpathlineto{\pgfqpoint{3.792739in}{2.200180in}}%
\pgfpathlineto{\pgfqpoint{3.793147in}{2.142418in}}%
\pgfpathlineto{\pgfqpoint{3.793351in}{2.076406in}}%
\pgfpathlineto{\pgfqpoint{3.793556in}{2.241438in}}%
\pgfpathlineto{\pgfqpoint{3.793964in}{2.167173in}}%
\pgfpathlineto{\pgfqpoint{3.795801in}{2.356960in}}%
\pgfpathlineto{\pgfqpoint{3.796005in}{2.315702in}}%
\pgfpathlineto{\pgfqpoint{3.796618in}{2.389967in}}%
\pgfpathlineto{\pgfqpoint{3.796822in}{2.406470in}}%
\pgfpathlineto{\pgfqpoint{3.797843in}{2.241438in}}%
\pgfpathlineto{\pgfqpoint{3.798251in}{2.257941in}}%
\pgfpathlineto{\pgfqpoint{3.799272in}{2.249689in}}%
\pgfpathlineto{\pgfqpoint{3.800088in}{2.348709in}}%
\pgfpathlineto{\pgfqpoint{3.800292in}{2.315702in}}%
\pgfpathlineto{\pgfqpoint{3.800496in}{2.356960in}}%
\pgfpathlineto{\pgfqpoint{3.800905in}{2.332205in}}%
\pgfpathlineto{\pgfqpoint{3.801109in}{2.406470in}}%
\pgfpathlineto{\pgfqpoint{3.801925in}{2.340457in}}%
\pgfpathlineto{\pgfqpoint{3.803354in}{2.406470in}}%
\pgfpathlineto{\pgfqpoint{3.804171in}{2.274444in}}%
\pgfpathlineto{\pgfqpoint{3.804579in}{2.290947in}}%
\pgfpathlineto{\pgfqpoint{3.804988in}{2.274444in}}%
\pgfpathlineto{\pgfqpoint{3.805396in}{2.117664in}}%
\pgfpathlineto{\pgfqpoint{3.805600in}{2.356960in}}%
\pgfpathlineto{\pgfqpoint{3.805804in}{2.332205in}}%
\pgfpathlineto{\pgfqpoint{3.806621in}{2.389967in}}%
\pgfpathlineto{\pgfqpoint{3.807029in}{2.348709in}}%
\pgfpathlineto{\pgfqpoint{3.808458in}{2.249689in}}%
\pgfpathlineto{\pgfqpoint{3.807437in}{2.356960in}}%
\pgfpathlineto{\pgfqpoint{3.808662in}{2.274444in}}%
\pgfpathlineto{\pgfqpoint{3.809479in}{2.348709in}}%
\pgfpathlineto{\pgfqpoint{3.809887in}{2.068154in}}%
\pgfpathlineto{\pgfqpoint{3.810499in}{2.373463in}}%
\pgfpathlineto{\pgfqpoint{3.810908in}{2.092909in}}%
\pgfpathlineto{\pgfqpoint{3.811724in}{2.332205in}}%
\pgfpathlineto{\pgfqpoint{3.812745in}{2.208431in}}%
\pgfpathlineto{\pgfqpoint{3.813153in}{2.249689in}}%
\pgfpathlineto{\pgfqpoint{3.813357in}{2.323954in}}%
\pgfpathlineto{\pgfqpoint{3.814378in}{2.299199in}}%
\pgfpathlineto{\pgfqpoint{3.814582in}{2.299199in}}%
\pgfpathlineto{\pgfqpoint{3.814786in}{2.274444in}}%
\pgfpathlineto{\pgfqpoint{3.814991in}{2.340457in}}%
\pgfpathlineto{\pgfqpoint{3.815195in}{2.323954in}}%
\pgfpathlineto{\pgfqpoint{3.815807in}{2.307450in}}%
\pgfpathlineto{\pgfqpoint{3.816215in}{2.348709in}}%
\pgfpathlineto{\pgfqpoint{3.817032in}{2.216683in}}%
\pgfpathlineto{\pgfqpoint{3.817236in}{2.282696in}}%
\pgfpathlineto{\pgfqpoint{3.818665in}{2.373463in}}%
\pgfpathlineto{\pgfqpoint{3.820298in}{2.290947in}}%
\pgfpathlineto{\pgfqpoint{3.820502in}{2.332205in}}%
\pgfpathlineto{\pgfqpoint{3.820911in}{2.249689in}}%
\pgfpathlineto{\pgfqpoint{3.821319in}{2.290947in}}%
\pgfpathlineto{\pgfqpoint{3.821523in}{2.266192in}}%
\pgfpathlineto{\pgfqpoint{3.821727in}{2.323954in}}%
\pgfpathlineto{\pgfqpoint{3.821931in}{2.323954in}}%
\pgfpathlineto{\pgfqpoint{3.822340in}{2.373463in}}%
\pgfpathlineto{\pgfqpoint{3.822748in}{2.340457in}}%
\pgfpathlineto{\pgfqpoint{3.823769in}{2.233186in}}%
\pgfpathlineto{\pgfqpoint{3.823973in}{2.266192in}}%
\pgfpathlineto{\pgfqpoint{3.825402in}{2.373463in}}%
\pgfpathlineto{\pgfqpoint{3.825606in}{2.422973in}}%
\pgfpathlineto{\pgfqpoint{3.826218in}{2.332205in}}%
\pgfpathlineto{\pgfqpoint{3.826423in}{2.340457in}}%
\pgfpathlineto{\pgfqpoint{3.826627in}{2.332205in}}%
\pgfpathlineto{\pgfqpoint{3.826831in}{2.257941in}}%
\pgfpathlineto{\pgfqpoint{3.827647in}{2.299199in}}%
\pgfpathlineto{\pgfqpoint{3.828668in}{2.348709in}}%
\pgfpathlineto{\pgfqpoint{3.829281in}{2.224934in}}%
\pgfpathlineto{\pgfqpoint{3.829893in}{2.323954in}}%
\pgfpathlineto{\pgfqpoint{3.831934in}{2.480734in}}%
\pgfpathlineto{\pgfqpoint{3.832547in}{2.348709in}}%
\pgfpathlineto{\pgfqpoint{3.833159in}{2.422973in}}%
\pgfpathlineto{\pgfqpoint{3.833363in}{2.431225in}}%
\pgfpathlineto{\pgfqpoint{3.833568in}{2.398218in}}%
\pgfpathlineto{\pgfqpoint{3.834180in}{2.373463in}}%
\pgfpathlineto{\pgfqpoint{3.834792in}{2.439476in}}%
\pgfpathlineto{\pgfqpoint{3.835201in}{2.398218in}}%
\pgfpathlineto{\pgfqpoint{3.835405in}{2.365212in}}%
\pgfpathlineto{\pgfqpoint{3.835813in}{2.439476in}}%
\pgfpathlineto{\pgfqpoint{3.836017in}{2.447728in}}%
\pgfpathlineto{\pgfqpoint{3.836221in}{2.398218in}}%
\pgfpathlineto{\pgfqpoint{3.836834in}{2.480734in}}%
\pgfpathlineto{\pgfqpoint{3.837038in}{2.439476in}}%
\pgfpathlineto{\pgfqpoint{3.837242in}{2.472483in}}%
\pgfpathlineto{\pgfqpoint{3.837446in}{2.422973in}}%
\pgfpathlineto{\pgfqpoint{3.837855in}{2.422973in}}%
\pgfpathlineto{\pgfqpoint{3.838059in}{2.422973in}}%
\pgfpathlineto{\pgfqpoint{3.838671in}{2.398218in}}%
\pgfpathlineto{\pgfqpoint{3.838875in}{2.464231in}}%
\pgfpathlineto{\pgfqpoint{3.839284in}{2.389967in}}%
\pgfpathlineto{\pgfqpoint{3.839896in}{2.439476in}}%
\pgfpathlineto{\pgfqpoint{3.840304in}{2.389967in}}%
\pgfpathlineto{\pgfqpoint{3.840508in}{2.431225in}}%
\pgfpathlineto{\pgfqpoint{3.841529in}{2.480734in}}%
\pgfpathlineto{\pgfqpoint{3.841937in}{2.472483in}}%
\pgfpathlineto{\pgfqpoint{3.843162in}{2.389967in}}%
\pgfpathlineto{\pgfqpoint{3.843571in}{2.414721in}}%
\pgfpathlineto{\pgfqpoint{3.843775in}{2.472483in}}%
\pgfpathlineto{\pgfqpoint{3.844183in}{2.381715in}}%
\pgfpathlineto{\pgfqpoint{3.844591in}{2.422973in}}%
\pgfpathlineto{\pgfqpoint{3.846224in}{2.340457in}}%
\pgfpathlineto{\pgfqpoint{3.846633in}{2.257941in}}%
\pgfpathlineto{\pgfqpoint{3.847245in}{2.323954in}}%
\pgfpathlineto{\pgfqpoint{3.848470in}{2.398218in}}%
\pgfpathlineto{\pgfqpoint{3.848674in}{2.389967in}}%
\pgfpathlineto{\pgfqpoint{3.849287in}{2.373463in}}%
\pgfpathlineto{\pgfqpoint{3.849491in}{2.406470in}}%
\pgfpathlineto{\pgfqpoint{3.849899in}{2.381715in}}%
\pgfpathlineto{\pgfqpoint{3.850103in}{2.406470in}}%
\pgfpathlineto{\pgfqpoint{3.850307in}{2.439476in}}%
\pgfpathlineto{\pgfqpoint{3.850716in}{2.365212in}}%
\pgfpathlineto{\pgfqpoint{3.851736in}{2.266192in}}%
\pgfpathlineto{\pgfqpoint{3.852145in}{2.299199in}}%
\pgfpathlineto{\pgfqpoint{3.852553in}{2.257941in}}%
\pgfpathlineto{\pgfqpoint{3.852757in}{2.356960in}}%
\pgfpathlineto{\pgfqpoint{3.853574in}{2.340457in}}%
\pgfpathlineto{\pgfqpoint{3.853982in}{2.224934in}}%
\pgfpathlineto{\pgfqpoint{3.854798in}{2.241438in}}%
\pgfpathlineto{\pgfqpoint{3.856227in}{2.323954in}}%
\pgfpathlineto{\pgfqpoint{3.856432in}{2.323954in}}%
\pgfpathlineto{\pgfqpoint{3.857248in}{2.348709in}}%
\pgfpathlineto{\pgfqpoint{3.857656in}{2.249689in}}%
\pgfpathlineto{\pgfqpoint{3.858473in}{2.290947in}}%
\pgfpathlineto{\pgfqpoint{3.859290in}{2.249689in}}%
\pgfpathlineto{\pgfqpoint{3.859698in}{2.266192in}}%
\pgfpathlineto{\pgfqpoint{3.860719in}{2.431225in}}%
\pgfpathlineto{\pgfqpoint{3.860923in}{2.117664in}}%
\pgfpathlineto{\pgfqpoint{3.861739in}{2.356960in}}%
\pgfpathlineto{\pgfqpoint{3.861943in}{2.381715in}}%
\pgfpathlineto{\pgfqpoint{3.862352in}{2.233186in}}%
\pgfpathlineto{\pgfqpoint{3.862964in}{2.307450in}}%
\pgfpathlineto{\pgfqpoint{3.863372in}{2.340457in}}%
\pgfpathlineto{\pgfqpoint{3.863577in}{2.158922in}}%
\pgfpathlineto{\pgfqpoint{3.864393in}{2.389967in}}%
\pgfpathlineto{\pgfqpoint{3.864801in}{2.299199in}}%
\pgfpathlineto{\pgfqpoint{3.865414in}{2.381715in}}%
\pgfpathlineto{\pgfqpoint{3.865618in}{2.365212in}}%
\pgfpathlineto{\pgfqpoint{3.866026in}{2.150670in}}%
\pgfpathlineto{\pgfqpoint{3.866639in}{2.356960in}}%
\pgfpathlineto{\pgfqpoint{3.867251in}{2.381715in}}%
\pgfpathlineto{\pgfqpoint{3.868068in}{2.290947in}}%
\pgfpathlineto{\pgfqpoint{3.868272in}{2.290947in}}%
\pgfpathlineto{\pgfqpoint{3.868476in}{2.084657in}}%
\pgfpathlineto{\pgfqpoint{3.869088in}{2.414721in}}%
\pgfpathlineto{\pgfqpoint{3.869293in}{2.340457in}}%
\pgfpathlineto{\pgfqpoint{3.869701in}{2.365212in}}%
\pgfpathlineto{\pgfqpoint{3.869905in}{2.323954in}}%
\pgfpathlineto{\pgfqpoint{3.870517in}{2.431225in}}%
\pgfpathlineto{\pgfqpoint{3.870721in}{2.307450in}}%
\pgfpathlineto{\pgfqpoint{3.870926in}{2.307450in}}%
\pgfpathlineto{\pgfqpoint{3.871130in}{2.332205in}}%
\pgfpathlineto{\pgfqpoint{3.871742in}{2.315702in}}%
\pgfpathlineto{\pgfqpoint{3.872763in}{2.175425in}}%
\pgfpathlineto{\pgfqpoint{3.873579in}{2.216683in}}%
\pgfpathlineto{\pgfqpoint{3.875008in}{2.340457in}}%
\pgfpathlineto{\pgfqpoint{3.874396in}{2.208431in}}%
\pgfpathlineto{\pgfqpoint{3.875213in}{2.332205in}}%
\pgfpathlineto{\pgfqpoint{3.875417in}{2.348709in}}%
\pgfpathlineto{\pgfqpoint{3.875621in}{2.307450in}}%
\pgfpathlineto{\pgfqpoint{3.875825in}{2.266192in}}%
\pgfpathlineto{\pgfqpoint{3.876437in}{2.340457in}}%
\pgfpathlineto{\pgfqpoint{3.876642in}{2.356960in}}%
\pgfpathlineto{\pgfqpoint{3.877050in}{2.315702in}}%
\pgfpathlineto{\pgfqpoint{3.877866in}{2.249689in}}%
\pgfpathlineto{\pgfqpoint{3.877662in}{2.356960in}}%
\pgfpathlineto{\pgfqpoint{3.878071in}{2.257941in}}%
\pgfpathlineto{\pgfqpoint{3.878275in}{2.307450in}}%
\pgfpathlineto{\pgfqpoint{3.878683in}{2.241438in}}%
\pgfpathlineto{\pgfqpoint{3.878887in}{2.249689in}}%
\pgfpathlineto{\pgfqpoint{3.879500in}{2.167173in}}%
\pgfpathlineto{\pgfqpoint{3.879908in}{2.216683in}}%
\pgfpathlineto{\pgfqpoint{3.880724in}{2.282696in}}%
\pgfpathlineto{\pgfqpoint{3.880929in}{2.233186in}}%
\pgfpathlineto{\pgfqpoint{3.881541in}{2.208431in}}%
\pgfpathlineto{\pgfqpoint{3.881337in}{2.241438in}}%
\pgfpathlineto{\pgfqpoint{3.881745in}{2.233186in}}%
\pgfpathlineto{\pgfqpoint{3.882358in}{2.274444in}}%
\pgfpathlineto{\pgfqpoint{3.883174in}{2.266192in}}%
\pgfpathlineto{\pgfqpoint{3.884399in}{2.125915in}}%
\pgfpathlineto{\pgfqpoint{3.884807in}{2.183676in}}%
\pgfpathlineto{\pgfqpoint{3.885216in}{2.068154in}}%
\pgfpathlineto{\pgfqpoint{3.885828in}{2.101160in}}%
\pgfpathlineto{\pgfqpoint{3.886645in}{1.795851in}}%
\pgfpathlineto{\pgfqpoint{3.887665in}{2.026896in}}%
\pgfpathlineto{\pgfqpoint{3.887869in}{1.993890in}}%
\pgfpathlineto{\pgfqpoint{3.888074in}{2.018644in}}%
\pgfpathlineto{\pgfqpoint{3.888278in}{1.952632in}}%
\pgfpathlineto{\pgfqpoint{3.888482in}{1.985638in}}%
\pgfpathlineto{\pgfqpoint{3.889503in}{1.853612in}}%
\pgfpathlineto{\pgfqpoint{3.890115in}{1.861864in}}%
\pgfpathlineto{\pgfqpoint{3.890932in}{1.936128in}}%
\pgfpathlineto{\pgfqpoint{3.891340in}{2.076406in}}%
\pgfpathlineto{\pgfqpoint{3.891952in}{1.927877in}}%
\pgfpathlineto{\pgfqpoint{3.892156in}{1.936128in}}%
\pgfpathlineto{\pgfqpoint{3.892361in}{2.059902in}}%
\pgfpathlineto{\pgfqpoint{3.893177in}{1.985638in}}%
\pgfpathlineto{\pgfqpoint{3.893381in}{1.993890in}}%
\pgfpathlineto{\pgfqpoint{3.893585in}{1.977386in}}%
\pgfpathlineto{\pgfqpoint{3.893994in}{1.952632in}}%
\pgfpathlineto{\pgfqpoint{3.894402in}{1.985638in}}%
\pgfpathlineto{\pgfqpoint{3.895423in}{2.158922in}}%
\pgfpathlineto{\pgfqpoint{3.895831in}{2.109412in}}%
\pgfpathlineto{\pgfqpoint{3.896852in}{2.051651in}}%
\pgfpathlineto{\pgfqpoint{3.897668in}{2.101160in}}%
\pgfpathlineto{\pgfqpoint{3.897872in}{2.076406in}}%
\pgfpathlineto{\pgfqpoint{3.898485in}{2.101160in}}%
\pgfpathlineto{\pgfqpoint{3.898893in}{2.051651in}}%
\pgfpathlineto{\pgfqpoint{3.899710in}{2.101160in}}%
\pgfpathlineto{\pgfqpoint{3.899914in}{2.043399in}}%
\pgfpathlineto{\pgfqpoint{3.900118in}{2.035148in}}%
\pgfpathlineto{\pgfqpoint{3.900526in}{2.109412in}}%
\pgfpathlineto{\pgfqpoint{3.901139in}{2.018644in}}%
\pgfpathlineto{\pgfqpoint{3.901751in}{2.076406in}}%
\pgfpathlineto{\pgfqpoint{3.902364in}{2.051651in}}%
\pgfpathlineto{\pgfqpoint{3.902772in}{1.944380in}}%
\pgfpathlineto{\pgfqpoint{3.903384in}{2.051651in}}%
\pgfpathlineto{\pgfqpoint{3.903588in}{2.043399in}}%
\pgfpathlineto{\pgfqpoint{3.903793in}{2.092909in}}%
\pgfpathlineto{\pgfqpoint{3.903997in}{1.960883in}}%
\pgfpathlineto{\pgfqpoint{3.904405in}{1.977386in}}%
\pgfpathlineto{\pgfqpoint{3.904813in}{1.960883in}}%
\pgfpathlineto{\pgfqpoint{3.905017in}{2.051651in}}%
\pgfpathlineto{\pgfqpoint{3.905834in}{1.977386in}}%
\pgfpathlineto{\pgfqpoint{3.906446in}{1.985638in}}%
\pgfpathlineto{\pgfqpoint{3.907059in}{1.919625in}}%
\pgfpathlineto{\pgfqpoint{3.907263in}{1.911373in}}%
\pgfpathlineto{\pgfqpoint{3.907467in}{1.977386in}}%
\pgfpathlineto{\pgfqpoint{3.907875in}{1.960883in}}%
\pgfpathlineto{\pgfqpoint{3.908488in}{2.026896in}}%
\pgfpathlineto{\pgfqpoint{3.909100in}{1.787599in}}%
\pgfpathlineto{\pgfqpoint{3.909917in}{2.092909in}}%
\pgfpathlineto{\pgfqpoint{3.910325in}{2.018644in}}%
\pgfpathlineto{\pgfqpoint{3.910529in}{1.985638in}}%
\pgfpathlineto{\pgfqpoint{3.910733in}{2.051651in}}%
\pgfpathlineto{\pgfqpoint{3.910938in}{2.026896in}}%
\pgfpathlineto{\pgfqpoint{3.911754in}{2.092909in}}%
\pgfpathlineto{\pgfqpoint{3.912162in}{1.787599in}}%
\pgfpathlineto{\pgfqpoint{3.912775in}{2.059902in}}%
\pgfpathlineto{\pgfqpoint{3.913183in}{2.035148in}}%
\pgfpathlineto{\pgfqpoint{3.913591in}{1.936128in}}%
\pgfpathlineto{\pgfqpoint{3.914204in}{2.010393in}}%
\pgfpathlineto{\pgfqpoint{3.914612in}{2.051651in}}%
\pgfpathlineto{\pgfqpoint{3.914816in}{2.026896in}}%
\pgfpathlineto{\pgfqpoint{3.915020in}{1.960883in}}%
\pgfpathlineto{\pgfqpoint{3.915633in}{2.059902in}}%
\pgfpathlineto{\pgfqpoint{3.915837in}{2.059902in}}%
\pgfpathlineto{\pgfqpoint{3.916041in}{2.092909in}}%
\pgfpathlineto{\pgfqpoint{3.916449in}{1.985638in}}%
\pgfpathlineto{\pgfqpoint{3.916858in}{2.051651in}}%
\pgfpathlineto{\pgfqpoint{3.917062in}{2.068154in}}%
\pgfpathlineto{\pgfqpoint{3.917470in}{1.771096in}}%
\pgfpathlineto{\pgfqpoint{3.918287in}{1.960883in}}%
\pgfpathlineto{\pgfqpoint{3.919307in}{2.010393in}}%
\pgfpathlineto{\pgfqpoint{3.919512in}{1.985638in}}%
\pgfpathlineto{\pgfqpoint{3.919920in}{2.035148in}}%
\pgfpathlineto{\pgfqpoint{3.920124in}{2.051651in}}%
\pgfpathlineto{\pgfqpoint{3.920941in}{2.035148in}}%
\pgfpathlineto{\pgfqpoint{3.921145in}{2.026896in}}%
\pgfpathlineto{\pgfqpoint{3.921553in}{1.762845in}}%
\pgfpathlineto{\pgfqpoint{3.922370in}{1.779348in}}%
\pgfpathlineto{\pgfqpoint{3.923390in}{1.969135in}}%
\pgfpathlineto{\pgfqpoint{3.923594in}{1.944380in}}%
\pgfpathlineto{\pgfqpoint{3.924615in}{1.894870in}}%
\pgfpathlineto{\pgfqpoint{3.925432in}{2.051651in}}%
\pgfpathlineto{\pgfqpoint{3.926044in}{1.993890in}}%
\pgfpathlineto{\pgfqpoint{3.926657in}{1.861864in}}%
\pgfpathlineto{\pgfqpoint{3.927065in}{1.936128in}}%
\pgfpathlineto{\pgfqpoint{3.927677in}{2.002141in}}%
\pgfpathlineto{\pgfqpoint{3.928290in}{1.960883in}}%
\pgfpathlineto{\pgfqpoint{3.929719in}{1.903122in}}%
\pgfpathlineto{\pgfqpoint{3.930127in}{1.927877in}}%
\pgfpathlineto{\pgfqpoint{3.930331in}{1.952632in}}%
\pgfpathlineto{\pgfqpoint{3.930739in}{1.911373in}}%
\pgfpathlineto{\pgfqpoint{3.930944in}{1.870115in}}%
\pgfpathlineto{\pgfqpoint{3.931352in}{1.944380in}}%
\pgfpathlineto{\pgfqpoint{3.931556in}{1.936128in}}%
\pgfpathlineto{\pgfqpoint{3.931760in}{1.944380in}}%
\pgfpathlineto{\pgfqpoint{3.932168in}{1.705083in}}%
\pgfpathlineto{\pgfqpoint{3.932985in}{1.878367in}}%
\pgfpathlineto{\pgfqpoint{3.933802in}{1.927877in}}%
\pgfpathlineto{\pgfqpoint{3.934210in}{1.795851in}}%
\pgfpathlineto{\pgfqpoint{3.935026in}{1.837109in}}%
\pgfpathlineto{\pgfqpoint{3.935231in}{1.837109in}}%
\pgfpathlineto{\pgfqpoint{3.936455in}{1.960883in}}%
\pgfpathlineto{\pgfqpoint{3.936660in}{1.919625in}}%
\pgfpathlineto{\pgfqpoint{3.936864in}{1.919625in}}%
\pgfpathlineto{\pgfqpoint{3.937680in}{1.878367in}}%
\pgfpathlineto{\pgfqpoint{3.937476in}{1.927877in}}%
\pgfpathlineto{\pgfqpoint{3.937884in}{1.894870in}}%
\pgfpathlineto{\pgfqpoint{3.938293in}{1.688580in}}%
\pgfpathlineto{\pgfqpoint{3.938701in}{1.911373in}}%
\pgfpathlineto{\pgfqpoint{3.938905in}{1.886619in}}%
\pgfpathlineto{\pgfqpoint{3.939313in}{1.960883in}}%
\pgfpathlineto{\pgfqpoint{3.939926in}{1.944380in}}%
\pgfpathlineto{\pgfqpoint{3.940130in}{1.903122in}}%
\pgfpathlineto{\pgfqpoint{3.940538in}{1.952632in}}%
\pgfpathlineto{\pgfqpoint{3.940947in}{2.026896in}}%
\pgfpathlineto{\pgfqpoint{3.941355in}{1.936128in}}%
\pgfpathlineto{\pgfqpoint{3.941559in}{1.977386in}}%
\pgfpathlineto{\pgfqpoint{3.942580in}{1.894870in}}%
\pgfpathlineto{\pgfqpoint{3.942784in}{1.927877in}}%
\pgfpathlineto{\pgfqpoint{3.942988in}{1.969135in}}%
\pgfpathlineto{\pgfqpoint{3.943805in}{1.919625in}}%
\pgfpathlineto{\pgfqpoint{3.944213in}{2.018644in}}%
\pgfpathlineto{\pgfqpoint{3.944621in}{1.936128in}}%
\pgfpathlineto{\pgfqpoint{3.944825in}{1.911373in}}%
\pgfpathlineto{\pgfqpoint{3.945029in}{1.985638in}}%
\pgfpathlineto{\pgfqpoint{3.945234in}{1.952632in}}%
\pgfpathlineto{\pgfqpoint{3.945846in}{2.018644in}}%
\pgfpathlineto{\pgfqpoint{3.945642in}{1.936128in}}%
\pgfpathlineto{\pgfqpoint{3.946254in}{2.010393in}}%
\pgfpathlineto{\pgfqpoint{3.947479in}{1.886619in}}%
\pgfpathlineto{\pgfqpoint{3.947683in}{1.886619in}}%
\pgfpathlineto{\pgfqpoint{3.947887in}{1.647322in}}%
\pgfpathlineto{\pgfqpoint{3.948704in}{1.960883in}}%
\pgfpathlineto{\pgfqpoint{3.949112in}{1.894870in}}%
\pgfpathlineto{\pgfqpoint{3.949521in}{1.804103in}}%
\pgfpathlineto{\pgfqpoint{3.950133in}{1.903122in}}%
\pgfpathlineto{\pgfqpoint{3.950337in}{1.903122in}}%
\pgfpathlineto{\pgfqpoint{3.950745in}{1.853612in}}%
\pgfpathlineto{\pgfqpoint{3.951970in}{1.762845in}}%
\pgfpathlineto{\pgfqpoint{3.952174in}{1.812354in}}%
\pgfpathlineto{\pgfqpoint{3.952379in}{1.729838in}}%
\pgfpathlineto{\pgfqpoint{3.952991in}{1.787599in}}%
\pgfpathlineto{\pgfqpoint{3.954012in}{1.713335in}}%
\pgfpathlineto{\pgfqpoint{3.954216in}{1.738090in}}%
\pgfpathlineto{\pgfqpoint{3.955032in}{1.795851in}}%
\pgfpathlineto{\pgfqpoint{3.955236in}{1.787599in}}%
\pgfpathlineto{\pgfqpoint{3.955441in}{1.845361in}}%
\pgfpathlineto{\pgfqpoint{3.955849in}{1.771096in}}%
\pgfpathlineto{\pgfqpoint{3.956053in}{1.597813in}}%
\pgfpathlineto{\pgfqpoint{3.956257in}{1.853612in}}%
\pgfpathlineto{\pgfqpoint{3.956870in}{1.820606in}}%
\pgfpathlineto{\pgfqpoint{3.957890in}{1.960883in}}%
\pgfpathlineto{\pgfqpoint{3.958299in}{1.911373in}}%
\pgfpathlineto{\pgfqpoint{3.960544in}{1.688580in}}%
\pgfpathlineto{\pgfqpoint{3.960952in}{1.762845in}}%
\pgfpathlineto{\pgfqpoint{3.961157in}{1.507045in}}%
\pgfpathlineto{\pgfqpoint{3.961769in}{1.837109in}}%
\pgfpathlineto{\pgfqpoint{3.961973in}{1.812354in}}%
\pgfpathlineto{\pgfqpoint{3.962177in}{1.886619in}}%
\pgfpathlineto{\pgfqpoint{3.962586in}{1.746341in}}%
\pgfpathlineto{\pgfqpoint{3.962790in}{1.746341in}}%
\pgfpathlineto{\pgfqpoint{3.962994in}{1.729838in}}%
\pgfpathlineto{\pgfqpoint{3.963198in}{1.828857in}}%
\pgfpathlineto{\pgfqpoint{3.964219in}{1.795851in}}%
\pgfpathlineto{\pgfqpoint{3.964423in}{1.738090in}}%
\pgfpathlineto{\pgfqpoint{3.964831in}{1.820606in}}%
\pgfpathlineto{\pgfqpoint{3.965239in}{1.762845in}}%
\pgfpathlineto{\pgfqpoint{3.965444in}{1.787599in}}%
\pgfpathlineto{\pgfqpoint{3.965648in}{1.688580in}}%
\pgfpathlineto{\pgfqpoint{3.965852in}{1.721587in}}%
\pgfpathlineto{\pgfqpoint{3.966056in}{1.680329in}}%
\pgfpathlineto{\pgfqpoint{3.966260in}{1.837109in}}%
\pgfpathlineto{\pgfqpoint{3.966668in}{1.779348in}}%
\pgfpathlineto{\pgfqpoint{3.967893in}{1.845361in}}%
\pgfpathlineto{\pgfqpoint{3.967077in}{1.754593in}}%
\pgfpathlineto{\pgfqpoint{3.968097in}{1.828857in}}%
\pgfpathlineto{\pgfqpoint{3.968914in}{1.771096in}}%
\pgfpathlineto{\pgfqpoint{3.968506in}{1.837109in}}%
\pgfpathlineto{\pgfqpoint{3.969118in}{1.804103in}}%
\pgfpathlineto{\pgfqpoint{3.969526in}{1.894870in}}%
\pgfpathlineto{\pgfqpoint{3.969731in}{1.812354in}}%
\pgfpathlineto{\pgfqpoint{3.969935in}{1.762845in}}%
\pgfpathlineto{\pgfqpoint{3.970547in}{1.870115in}}%
\pgfpathlineto{\pgfqpoint{3.970751in}{1.870115in}}%
\pgfpathlineto{\pgfqpoint{3.971772in}{1.837109in}}%
\pgfpathlineto{\pgfqpoint{3.971976in}{1.878367in}}%
\pgfpathlineto{\pgfqpoint{3.972384in}{1.812354in}}%
\pgfpathlineto{\pgfqpoint{3.973813in}{1.573058in}}%
\pgfpathlineto{\pgfqpoint{3.975242in}{1.754593in}}%
\pgfpathlineto{\pgfqpoint{3.975447in}{1.713335in}}%
\pgfpathlineto{\pgfqpoint{3.977080in}{1.540051in}}%
\pgfpathlineto{\pgfqpoint{3.976059in}{1.738090in}}%
\pgfpathlineto{\pgfqpoint{3.977488in}{1.581309in}}%
\pgfpathlineto{\pgfqpoint{3.978305in}{1.738090in}}%
\pgfpathlineto{\pgfqpoint{3.979121in}{1.680329in}}%
\pgfpathlineto{\pgfqpoint{3.979325in}{1.630819in}}%
\pgfpathlineto{\pgfqpoint{3.979734in}{1.729838in}}%
\pgfpathlineto{\pgfqpoint{3.979938in}{1.705083in}}%
\pgfpathlineto{\pgfqpoint{3.981979in}{1.861864in}}%
\pgfpathlineto{\pgfqpoint{3.983000in}{1.721587in}}%
\pgfpathlineto{\pgfqpoint{3.983204in}{1.738090in}}%
\pgfpathlineto{\pgfqpoint{3.983408in}{1.754593in}}%
\pgfpathlineto{\pgfqpoint{3.983816in}{1.713335in}}%
\pgfpathlineto{\pgfqpoint{3.984021in}{1.721587in}}%
\pgfpathlineto{\pgfqpoint{3.984429in}{1.688580in}}%
\pgfpathlineto{\pgfqpoint{3.984633in}{1.779348in}}%
\pgfpathlineto{\pgfqpoint{3.985245in}{1.705083in}}%
\pgfpathlineto{\pgfqpoint{3.986674in}{1.597813in}}%
\pgfpathlineto{\pgfqpoint{3.987695in}{1.705083in}}%
\pgfpathlineto{\pgfqpoint{3.987899in}{1.655574in}}%
\pgfpathlineto{\pgfqpoint{3.988103in}{1.630819in}}%
\pgfpathlineto{\pgfqpoint{3.988512in}{1.713335in}}%
\pgfpathlineto{\pgfqpoint{3.989532in}{1.647322in}}%
\pgfpathlineto{\pgfqpoint{3.989737in}{1.721587in}}%
\pgfpathlineto{\pgfqpoint{3.990145in}{1.606064in}}%
\pgfpathlineto{\pgfqpoint{3.990553in}{1.680329in}}%
\pgfpathlineto{\pgfqpoint{3.990757in}{1.680329in}}%
\pgfpathlineto{\pgfqpoint{3.990961in}{1.663825in}}%
\pgfpathlineto{\pgfqpoint{3.991166in}{1.688580in}}%
\pgfpathlineto{\pgfqpoint{3.991370in}{1.688580in}}%
\pgfpathlineto{\pgfqpoint{3.992186in}{1.746341in}}%
\pgfpathlineto{\pgfqpoint{3.992595in}{1.721587in}}%
\pgfpathlineto{\pgfqpoint{3.993003in}{1.663825in}}%
\pgfpathlineto{\pgfqpoint{3.993207in}{1.746341in}}%
\pgfpathlineto{\pgfqpoint{3.993615in}{1.688580in}}%
\pgfpathlineto{\pgfqpoint{3.994024in}{1.680329in}}%
\pgfpathlineto{\pgfqpoint{3.994432in}{1.738090in}}%
\pgfpathlineto{\pgfqpoint{3.995657in}{1.630819in}}%
\pgfpathlineto{\pgfqpoint{3.996065in}{1.746341in}}%
\pgfpathlineto{\pgfqpoint{3.997086in}{1.688580in}}%
\pgfpathlineto{\pgfqpoint{3.997494in}{1.738090in}}%
\pgfpathlineto{\pgfqpoint{3.997902in}{1.680329in}}%
\pgfpathlineto{\pgfqpoint{3.998106in}{1.721587in}}%
\pgfpathlineto{\pgfqpoint{3.999331in}{1.606064in}}%
\pgfpathlineto{\pgfqpoint{3.999535in}{1.663825in}}%
\pgfpathlineto{\pgfqpoint{4.000352in}{1.622567in}}%
\pgfpathlineto{\pgfqpoint{4.000760in}{1.680329in}}%
\pgfpathlineto{\pgfqpoint{4.002189in}{1.795851in}}%
\pgfpathlineto{\pgfqpoint{4.002393in}{1.746341in}}%
\pgfpathlineto{\pgfqpoint{4.003006in}{1.779348in}}%
\pgfpathlineto{\pgfqpoint{4.003414in}{1.696832in}}%
\pgfpathlineto{\pgfqpoint{4.003618in}{1.738090in}}%
\pgfpathlineto{\pgfqpoint{4.004231in}{1.672077in}}%
\pgfpathlineto{\pgfqpoint{4.004639in}{1.630819in}}%
\pgfpathlineto{\pgfqpoint{4.005251in}{1.680329in}}%
\pgfpathlineto{\pgfqpoint{4.005456in}{1.688580in}}%
\pgfpathlineto{\pgfqpoint{4.006068in}{1.713335in}}%
\pgfpathlineto{\pgfqpoint{4.006476in}{1.622567in}}%
\pgfpathlineto{\pgfqpoint{4.007701in}{1.696832in}}%
\pgfpathlineto{\pgfqpoint{4.009334in}{1.540051in}}%
\pgfpathlineto{\pgfqpoint{4.009947in}{1.713335in}}%
\pgfpathlineto{\pgfqpoint{4.010559in}{1.647322in}}%
\pgfpathlineto{\pgfqpoint{4.010763in}{1.639071in}}%
\pgfpathlineto{\pgfqpoint{4.010967in}{1.655574in}}%
\pgfpathlineto{\pgfqpoint{4.011172in}{1.696832in}}%
\pgfpathlineto{\pgfqpoint{4.011580in}{1.614316in}}%
\pgfpathlineto{\pgfqpoint{4.011784in}{1.375019in}}%
\pgfpathlineto{\pgfqpoint{4.012601in}{1.581309in}}%
\pgfpathlineto{\pgfqpoint{4.012805in}{1.573058in}}%
\pgfpathlineto{\pgfqpoint{4.013009in}{1.606064in}}%
\pgfpathlineto{\pgfqpoint{4.013213in}{1.606064in}}%
\pgfpathlineto{\pgfqpoint{4.014234in}{1.713335in}}%
\pgfpathlineto{\pgfqpoint{4.014438in}{1.639071in}}%
\pgfpathlineto{\pgfqpoint{4.014642in}{1.424529in}}%
\pgfpathlineto{\pgfqpoint{4.014846in}{1.663825in}}%
\pgfpathlineto{\pgfqpoint{4.015459in}{1.581309in}}%
\pgfpathlineto{\pgfqpoint{4.015663in}{1.556554in}}%
\pgfpathlineto{\pgfqpoint{4.016071in}{1.614316in}}%
\pgfpathlineto{\pgfqpoint{4.016275in}{1.614316in}}%
\pgfpathlineto{\pgfqpoint{4.017092in}{1.688580in}}%
\pgfpathlineto{\pgfqpoint{4.017296in}{1.647322in}}%
\pgfpathlineto{\pgfqpoint{4.018317in}{1.548303in}}%
\pgfpathlineto{\pgfqpoint{4.017704in}{1.688580in}}%
\pgfpathlineto{\pgfqpoint{4.018521in}{1.597813in}}%
\pgfpathlineto{\pgfqpoint{4.019133in}{1.680329in}}%
\pgfpathlineto{\pgfqpoint{4.019337in}{1.564806in}}%
\pgfpathlineto{\pgfqpoint{4.019541in}{1.523548in}}%
\pgfpathlineto{\pgfqpoint{4.020154in}{1.614316in}}%
\pgfpathlineto{\pgfqpoint{4.020358in}{1.614316in}}%
\pgfpathlineto{\pgfqpoint{4.020970in}{1.606064in}}%
\pgfpathlineto{\pgfqpoint{4.021991in}{1.696832in}}%
\pgfpathlineto{\pgfqpoint{4.023012in}{1.531800in}}%
\pgfpathlineto{\pgfqpoint{4.023420in}{1.630819in}}%
\pgfpathlineto{\pgfqpoint{4.025053in}{1.713335in}}%
\pgfpathlineto{\pgfqpoint{4.024237in}{1.622567in}}%
\pgfpathlineto{\pgfqpoint{4.025257in}{1.680329in}}%
\pgfpathlineto{\pgfqpoint{4.025666in}{1.614316in}}%
\pgfpathlineto{\pgfqpoint{4.026074in}{1.713335in}}%
\pgfpathlineto{\pgfqpoint{4.026278in}{1.688580in}}%
\pgfpathlineto{\pgfqpoint{4.026891in}{1.655574in}}%
\pgfpathlineto{\pgfqpoint{4.027503in}{1.779348in}}%
\pgfpathlineto{\pgfqpoint{4.027911in}{1.672077in}}%
\pgfpathlineto{\pgfqpoint{4.028115in}{1.663825in}}%
\pgfpathlineto{\pgfqpoint{4.028320in}{1.672077in}}%
\pgfpathlineto{\pgfqpoint{4.028524in}{1.696832in}}%
\pgfpathlineto{\pgfqpoint{4.028932in}{1.647322in}}%
\pgfpathlineto{\pgfqpoint{4.029340in}{1.680329in}}%
\pgfpathlineto{\pgfqpoint{4.029749in}{1.614316in}}%
\pgfpathlineto{\pgfqpoint{4.030157in}{1.630819in}}%
\pgfpathlineto{\pgfqpoint{4.030361in}{1.416277in}}%
\pgfpathlineto{\pgfqpoint{4.031178in}{1.672077in}}%
\pgfpathlineto{\pgfqpoint{4.031994in}{1.655574in}}%
\pgfpathlineto{\pgfqpoint{4.032198in}{1.729838in}}%
\pgfpathlineto{\pgfqpoint{4.032402in}{1.482290in}}%
\pgfpathlineto{\pgfqpoint{4.033219in}{1.672077in}}%
\pgfpathlineto{\pgfqpoint{4.033423in}{1.655574in}}%
\pgfpathlineto{\pgfqpoint{4.033831in}{1.705083in}}%
\pgfpathlineto{\pgfqpoint{4.034036in}{1.713335in}}%
\pgfpathlineto{\pgfqpoint{4.034240in}{1.705083in}}%
\pgfpathlineto{\pgfqpoint{4.034444in}{1.465787in}}%
\pgfpathlineto{\pgfqpoint{4.034852in}{1.721587in}}%
\pgfpathlineto{\pgfqpoint{4.035260in}{1.672077in}}%
\pgfpathlineto{\pgfqpoint{4.035465in}{1.688580in}}%
\pgfpathlineto{\pgfqpoint{4.035669in}{1.622567in}}%
\pgfpathlineto{\pgfqpoint{4.036281in}{1.672077in}}%
\pgfpathlineto{\pgfqpoint{4.036485in}{1.597813in}}%
\pgfpathlineto{\pgfqpoint{4.037506in}{1.622567in}}%
\pgfpathlineto{\pgfqpoint{4.037710in}{1.639071in}}%
\pgfpathlineto{\pgfqpoint{4.037914in}{1.622567in}}%
\pgfpathlineto{\pgfqpoint{4.038118in}{1.564806in}}%
\pgfpathlineto{\pgfqpoint{4.038935in}{1.573058in}}%
\pgfpathlineto{\pgfqpoint{4.040364in}{1.655574in}}%
\pgfpathlineto{\pgfqpoint{4.041180in}{1.606064in}}%
\pgfpathlineto{\pgfqpoint{4.041589in}{1.713335in}}%
\pgfpathlineto{\pgfqpoint{4.042201in}{1.606064in}}%
\pgfpathlineto{\pgfqpoint{4.042405in}{1.606064in}}%
\pgfpathlineto{\pgfqpoint{4.043630in}{1.482290in}}%
\pgfpathlineto{\pgfqpoint{4.043834in}{1.523548in}}%
\pgfpathlineto{\pgfqpoint{4.044038in}{1.515296in}}%
\pgfpathlineto{\pgfqpoint{4.044651in}{1.622567in}}%
\pgfpathlineto{\pgfqpoint{4.045263in}{1.581309in}}%
\pgfpathlineto{\pgfqpoint{4.046284in}{1.474038in}}%
\pgfpathlineto{\pgfqpoint{4.046488in}{1.540051in}}%
\pgfpathlineto{\pgfqpoint{4.047101in}{1.457535in}}%
\pgfpathlineto{\pgfqpoint{4.047305in}{1.490542in}}%
\pgfpathlineto{\pgfqpoint{4.048121in}{1.375019in}}%
\pgfpathlineto{\pgfqpoint{4.048530in}{1.424529in}}%
\pgfpathlineto{\pgfqpoint{4.049754in}{1.523548in}}%
\pgfpathlineto{\pgfqpoint{4.049959in}{1.515296in}}%
\pgfpathlineto{\pgfqpoint{4.050163in}{1.465787in}}%
\pgfpathlineto{\pgfqpoint{4.050367in}{1.531800in}}%
\pgfpathlineto{\pgfqpoint{4.050979in}{1.531800in}}%
\pgfpathlineto{\pgfqpoint{4.051592in}{1.573058in}}%
\pgfpathlineto{\pgfqpoint{4.052817in}{1.424529in}}%
\pgfpathlineto{\pgfqpoint{4.053021in}{1.424529in}}%
\pgfpathlineto{\pgfqpoint{4.054450in}{1.556554in}}%
\pgfpathlineto{\pgfqpoint{4.055266in}{1.474038in}}%
\pgfpathlineto{\pgfqpoint{4.055675in}{1.482290in}}%
\pgfpathlineto{\pgfqpoint{4.056287in}{1.548303in}}%
\pgfpathlineto{\pgfqpoint{4.056695in}{1.490542in}}%
\pgfpathlineto{\pgfqpoint{4.056899in}{1.465787in}}%
\pgfpathlineto{\pgfqpoint{4.057308in}{1.540051in}}%
\pgfpathlineto{\pgfqpoint{4.058941in}{1.589561in}}%
\pgfpathlineto{\pgfqpoint{4.059349in}{1.564806in}}%
\pgfpathlineto{\pgfqpoint{4.059962in}{1.614316in}}%
\pgfpathlineto{\pgfqpoint{4.060982in}{1.531800in}}%
\pgfpathlineto{\pgfqpoint{4.060370in}{1.647322in}}%
\pgfpathlineto{\pgfqpoint{4.061391in}{1.548303in}}%
\pgfpathlineto{\pgfqpoint{4.062820in}{1.672077in}}%
\pgfpathlineto{\pgfqpoint{4.063432in}{1.663825in}}%
\pgfpathlineto{\pgfqpoint{4.063636in}{1.573058in}}%
\pgfpathlineto{\pgfqpoint{4.064453in}{1.630819in}}%
\pgfpathlineto{\pgfqpoint{4.064657in}{1.614316in}}%
\pgfpathlineto{\pgfqpoint{4.064861in}{1.655574in}}%
\pgfpathlineto{\pgfqpoint{4.065065in}{1.639071in}}%
\pgfpathlineto{\pgfqpoint{4.066086in}{1.754593in}}%
\pgfpathlineto{\pgfqpoint{4.066290in}{1.688580in}}%
\pgfpathlineto{\pgfqpoint{4.067719in}{1.556554in}}%
\pgfpathlineto{\pgfqpoint{4.067923in}{1.573058in}}%
\pgfpathlineto{\pgfqpoint{4.068127in}{1.507045in}}%
\pgfpathlineto{\pgfqpoint{4.068331in}{1.507045in}}%
\pgfpathlineto{\pgfqpoint{4.070577in}{1.622567in}}%
\pgfpathlineto{\pgfqpoint{4.070781in}{1.606064in}}%
\pgfpathlineto{\pgfqpoint{4.070985in}{1.663825in}}%
\pgfpathlineto{\pgfqpoint{4.071189in}{1.861864in}}%
\pgfpathlineto{\pgfqpoint{4.071394in}{1.589561in}}%
\pgfpathlineto{\pgfqpoint{4.072006in}{1.655574in}}%
\pgfpathlineto{\pgfqpoint{4.072210in}{1.457535in}}%
\pgfpathlineto{\pgfqpoint{4.072414in}{1.754593in}}%
\pgfpathlineto{\pgfqpoint{4.073027in}{1.688580in}}%
\pgfpathlineto{\pgfqpoint{4.074456in}{1.597813in}}%
\pgfpathlineto{\pgfqpoint{4.075681in}{1.713335in}}%
\pgfpathlineto{\pgfqpoint{4.076089in}{1.639071in}}%
\pgfpathlineto{\pgfqpoint{4.076293in}{1.672077in}}%
\pgfpathlineto{\pgfqpoint{4.076701in}{1.358516in}}%
\pgfpathlineto{\pgfqpoint{4.077314in}{1.573058in}}%
\pgfpathlineto{\pgfqpoint{4.078947in}{1.729838in}}%
\pgfpathlineto{\pgfqpoint{4.079968in}{1.531800in}}%
\pgfpathlineto{\pgfqpoint{4.080172in}{1.573058in}}%
\pgfpathlineto{\pgfqpoint{4.081601in}{1.441032in}}%
\pgfpathlineto{\pgfqpoint{4.082417in}{1.589561in}}%
\pgfpathlineto{\pgfqpoint{4.082621in}{1.457535in}}%
\pgfpathlineto{\pgfqpoint{4.083234in}{1.218239in}}%
\pgfpathlineto{\pgfqpoint{4.084050in}{1.284252in}}%
\pgfpathlineto{\pgfqpoint{4.084255in}{1.292503in}}%
\pgfpathlineto{\pgfqpoint{4.085071in}{1.020200in}}%
\pgfpathlineto{\pgfqpoint{4.085275in}{1.350264in}}%
\pgfpathlineto{\pgfqpoint{4.086296in}{1.251245in}}%
\pgfpathlineto{\pgfqpoint{4.087521in}{0.954187in}}%
\pgfpathlineto{\pgfqpoint{4.087725in}{1.069710in}}%
\pgfpathlineto{\pgfqpoint{4.089154in}{1.218239in}}%
\pgfpathlineto{\pgfqpoint{4.090583in}{0.822162in}}%
\pgfpathlineto{\pgfqpoint{4.090787in}{1.020200in}}%
\pgfpathlineto{\pgfqpoint{4.091400in}{0.937684in}}%
\pgfpathlineto{\pgfqpoint{4.092012in}{0.970691in}}%
\pgfpathlineto{\pgfqpoint{4.093237in}{1.201736in}}%
\pgfpathlineto{\pgfqpoint{4.093441in}{1.143974in}}%
\pgfpathlineto{\pgfqpoint{4.094462in}{0.764400in}}%
\pgfpathlineto{\pgfqpoint{4.094666in}{1.028452in}}%
\pgfpathlineto{\pgfqpoint{4.094870in}{1.028452in}}%
\pgfpathlineto{\pgfqpoint{4.095074in}{1.061458in}}%
\pgfpathlineto{\pgfqpoint{4.095278in}{1.003697in}}%
\pgfpathlineto{\pgfqpoint{4.095482in}{1.028452in}}%
\pgfpathlineto{\pgfqpoint{4.096503in}{0.789155in}}%
\pgfpathlineto{\pgfqpoint{4.097116in}{0.830413in}}%
\pgfpathlineto{\pgfqpoint{4.098545in}{1.160477in}}%
\pgfpathlineto{\pgfqpoint{4.099157in}{1.003697in}}%
\pgfpathlineto{\pgfqpoint{4.099361in}{0.970691in}}%
\pgfpathlineto{\pgfqpoint{4.099974in}{1.044955in}}%
\pgfpathlineto{\pgfqpoint{4.100790in}{1.185232in}}%
\pgfpathlineto{\pgfqpoint{4.101198in}{1.143974in}}%
\pgfpathlineto{\pgfqpoint{4.102015in}{0.805659in}}%
\pgfpathlineto{\pgfqpoint{4.102423in}{0.888175in}}%
\pgfpathlineto{\pgfqpoint{4.103852in}{1.218239in}}%
\pgfpathlineto{\pgfqpoint{4.104669in}{1.094465in}}%
\pgfpathlineto{\pgfqpoint{4.105077in}{1.160477in}}%
\pgfpathlineto{\pgfqpoint{4.105281in}{1.234742in}}%
\pgfpathlineto{\pgfqpoint{4.105894in}{1.127471in}}%
\pgfpathlineto{\pgfqpoint{4.106098in}{1.160477in}}%
\pgfpathlineto{\pgfqpoint{4.106710in}{1.086213in}}%
\pgfpathlineto{\pgfqpoint{4.107323in}{1.135723in}}%
\pgfpathlineto{\pgfqpoint{4.107527in}{1.168729in}}%
\pgfpathlineto{\pgfqpoint{4.107731in}{1.127471in}}%
\pgfpathlineto{\pgfqpoint{4.108139in}{1.044955in}}%
\pgfpathlineto{\pgfqpoint{4.108956in}{1.069710in}}%
\pgfpathlineto{\pgfqpoint{4.109568in}{1.119219in}}%
\pgfpathlineto{\pgfqpoint{4.109772in}{1.061458in}}%
\pgfpathlineto{\pgfqpoint{4.110181in}{1.028452in}}%
\pgfpathlineto{\pgfqpoint{4.110385in}{1.086213in}}%
\pgfpathlineto{\pgfqpoint{4.110589in}{1.086213in}}%
\pgfpathlineto{\pgfqpoint{4.110997in}{1.127471in}}%
\pgfpathlineto{\pgfqpoint{4.111406in}{1.119219in}}%
\pgfpathlineto{\pgfqpoint{4.111610in}{1.077961in}}%
\pgfpathlineto{\pgfqpoint{4.111814in}{1.176981in}}%
\pgfpathlineto{\pgfqpoint{4.112222in}{1.300755in}}%
\pgfpathlineto{\pgfqpoint{4.112426in}{1.201736in}}%
\pgfpathlineto{\pgfqpoint{4.112630in}{1.102716in}}%
\pgfpathlineto{\pgfqpoint{4.113243in}{1.317258in}}%
\pgfpathlineto{\pgfqpoint{4.114468in}{1.209987in}}%
\pgfpathlineto{\pgfqpoint{4.115080in}{1.366768in}}%
\pgfpathlineto{\pgfqpoint{4.115693in}{1.267748in}}%
\pgfpathlineto{\pgfqpoint{4.116509in}{1.358516in}}%
\pgfpathlineto{\pgfqpoint{4.116713in}{1.300755in}}%
\pgfpathlineto{\pgfqpoint{4.116917in}{1.209987in}}%
\pgfpathlineto{\pgfqpoint{4.117326in}{1.408026in}}%
\pgfpathlineto{\pgfqpoint{4.117530in}{1.432780in}}%
\pgfpathlineto{\pgfqpoint{4.117734in}{1.366768in}}%
\pgfpathlineto{\pgfqpoint{4.118346in}{1.309006in}}%
\pgfpathlineto{\pgfqpoint{4.118142in}{1.383271in}}%
\pgfpathlineto{\pgfqpoint{4.118551in}{1.350264in}}%
\pgfpathlineto{\pgfqpoint{4.118755in}{1.391522in}}%
\pgfpathlineto{\pgfqpoint{4.118959in}{1.300755in}}%
\pgfpathlineto{\pgfqpoint{4.119775in}{1.383271in}}%
\pgfpathlineto{\pgfqpoint{4.119980in}{1.383271in}}%
\pgfpathlineto{\pgfqpoint{4.121204in}{1.515296in}}%
\pgfpathlineto{\pgfqpoint{4.120388in}{1.366768in}}%
\pgfpathlineto{\pgfqpoint{4.121409in}{1.490542in}}%
\pgfpathlineto{\pgfqpoint{4.122225in}{1.507045in}}%
\pgfpathlineto{\pgfqpoint{4.122633in}{1.449284in}}%
\pgfpathlineto{\pgfqpoint{4.122838in}{1.515296in}}%
\pgfpathlineto{\pgfqpoint{4.123246in}{1.391522in}}%
\pgfpathlineto{\pgfqpoint{4.123858in}{1.490542in}}%
\pgfpathlineto{\pgfqpoint{4.125287in}{1.201736in}}%
\pgfpathlineto{\pgfqpoint{4.126512in}{1.523548in}}%
\pgfpathlineto{\pgfqpoint{4.127533in}{1.342013in}}%
\pgfpathlineto{\pgfqpoint{4.127737in}{1.441032in}}%
\pgfpathlineto{\pgfqpoint{4.128349in}{1.498793in}}%
\pgfpathlineto{\pgfqpoint{4.128962in}{1.482290in}}%
\pgfpathlineto{\pgfqpoint{4.129370in}{1.507045in}}%
\pgfpathlineto{\pgfqpoint{4.130187in}{1.408026in}}%
\pgfpathlineto{\pgfqpoint{4.131616in}{1.564806in}}%
\pgfpathlineto{\pgfqpoint{4.131820in}{1.556554in}}%
\pgfpathlineto{\pgfqpoint{4.132432in}{1.614316in}}%
\pgfpathlineto{\pgfqpoint{4.132636in}{1.573058in}}%
\pgfpathlineto{\pgfqpoint{4.134474in}{1.399774in}}%
\pgfpathlineto{\pgfqpoint{4.135494in}{1.540051in}}%
\pgfpathlineto{\pgfqpoint{4.135903in}{1.531800in}}%
\pgfpathlineto{\pgfqpoint{4.136719in}{1.498793in}}%
\pgfpathlineto{\pgfqpoint{4.138148in}{1.639071in}}%
\pgfpathlineto{\pgfqpoint{4.139169in}{1.573058in}}%
\pgfpathlineto{\pgfqpoint{4.139373in}{1.606064in}}%
\pgfpathlineto{\pgfqpoint{4.139781in}{1.606064in}}%
\pgfpathlineto{\pgfqpoint{4.140394in}{1.688580in}}%
\pgfpathlineto{\pgfqpoint{4.141210in}{1.663825in}}%
\pgfpathlineto{\pgfqpoint{4.141414in}{1.647322in}}%
\pgfpathlineto{\pgfqpoint{4.141619in}{1.663825in}}%
\pgfpathlineto{\pgfqpoint{4.142027in}{1.721587in}}%
\pgfpathlineto{\pgfqpoint{4.142639in}{1.705083in}}%
\pgfpathlineto{\pgfqpoint{4.142843in}{1.655574in}}%
\pgfpathlineto{\pgfqpoint{4.143456in}{1.771096in}}%
\pgfpathlineto{\pgfqpoint{4.143660in}{1.729838in}}%
\pgfpathlineto{\pgfqpoint{4.143864in}{1.779348in}}%
\pgfpathlineto{\pgfqpoint{4.144681in}{1.746341in}}%
\pgfpathlineto{\pgfqpoint{4.145701in}{1.828857in}}%
\pgfpathlineto{\pgfqpoint{4.145906in}{1.812354in}}%
\pgfpathlineto{\pgfqpoint{4.146110in}{1.771096in}}%
\pgfpathlineto{\pgfqpoint{4.146518in}{1.886619in}}%
\pgfpathlineto{\pgfqpoint{4.146722in}{1.837109in}}%
\pgfpathlineto{\pgfqpoint{4.146926in}{1.837109in}}%
\pgfpathlineto{\pgfqpoint{4.148355in}{1.952632in}}%
\pgfpathlineto{\pgfqpoint{4.148764in}{1.919625in}}%
\pgfpathlineto{\pgfqpoint{4.148968in}{1.894870in}}%
\pgfpathlineto{\pgfqpoint{4.149376in}{1.977386in}}%
\pgfpathlineto{\pgfqpoint{4.149580in}{1.969135in}}%
\pgfpathlineto{\pgfqpoint{4.150805in}{2.026896in}}%
\pgfpathlineto{\pgfqpoint{4.151417in}{2.018644in}}%
\pgfpathlineto{\pgfqpoint{4.151622in}{2.043399in}}%
\pgfpathlineto{\pgfqpoint{4.151826in}{1.853612in}}%
\pgfpathlineto{\pgfqpoint{4.152642in}{2.101160in}}%
\pgfpathlineto{\pgfqpoint{4.152846in}{2.076406in}}%
\pgfpathlineto{\pgfqpoint{4.153051in}{2.134167in}}%
\pgfpathlineto{\pgfqpoint{4.153663in}{2.109412in}}%
\pgfpathlineto{\pgfqpoint{4.153867in}{2.092909in}}%
\pgfpathlineto{\pgfqpoint{4.154071in}{1.977386in}}%
\pgfpathlineto{\pgfqpoint{4.154888in}{2.068154in}}%
\pgfpathlineto{\pgfqpoint{4.155500in}{1.894870in}}%
\pgfpathlineto{\pgfqpoint{4.156113in}{1.911373in}}%
\pgfpathlineto{\pgfqpoint{4.156929in}{2.134167in}}%
\pgfpathlineto{\pgfqpoint{4.157338in}{2.117664in}}%
\pgfpathlineto{\pgfqpoint{4.157746in}{2.158922in}}%
\pgfpathlineto{\pgfqpoint{4.157950in}{1.845361in}}%
\pgfpathlineto{\pgfqpoint{4.158767in}{2.175425in}}%
\pgfpathlineto{\pgfqpoint{4.158971in}{2.208431in}}%
\pgfpathlineto{\pgfqpoint{4.159583in}{2.142418in}}%
\pgfpathlineto{\pgfqpoint{4.159787in}{2.125915in}}%
\pgfpathlineto{\pgfqpoint{4.159991in}{2.158922in}}%
\pgfpathlineto{\pgfqpoint{4.160196in}{2.142418in}}%
\pgfpathlineto{\pgfqpoint{4.160400in}{2.200180in}}%
\pgfpathlineto{\pgfqpoint{4.161216in}{2.175425in}}%
\pgfpathlineto{\pgfqpoint{4.162645in}{2.051651in}}%
\pgfpathlineto{\pgfqpoint{4.163666in}{2.043399in}}%
\pgfpathlineto{\pgfqpoint{4.163870in}{2.150670in}}%
\pgfpathlineto{\pgfqpoint{4.164278in}{2.109412in}}%
\pgfpathlineto{\pgfqpoint{4.164687in}{2.183676in}}%
\pgfpathlineto{\pgfqpoint{4.164891in}{2.191928in}}%
\pgfpathlineto{\pgfqpoint{4.165707in}{2.216683in}}%
\pgfpathlineto{\pgfqpoint{4.166116in}{2.010393in}}%
\pgfpathlineto{\pgfqpoint{4.166320in}{2.026896in}}%
\pgfpathlineto{\pgfqpoint{4.166524in}{2.010393in}}%
\pgfpathlineto{\pgfqpoint{4.166728in}{1.969135in}}%
\pgfpathlineto{\pgfqpoint{4.167136in}{2.233186in}}%
\pgfpathlineto{\pgfqpoint{4.167953in}{2.142418in}}%
\pgfpathlineto{\pgfqpoint{4.168974in}{2.233186in}}%
\pgfpathlineto{\pgfqpoint{4.169382in}{2.208431in}}%
\pgfpathlineto{\pgfqpoint{4.170199in}{2.068154in}}%
\pgfpathlineto{\pgfqpoint{4.170607in}{2.125915in}}%
\pgfpathlineto{\pgfqpoint{4.171219in}{2.200180in}}%
\pgfpathlineto{\pgfqpoint{4.171628in}{2.109412in}}%
\pgfpathlineto{\pgfqpoint{4.171832in}{2.084657in}}%
\pgfpathlineto{\pgfqpoint{4.172036in}{2.142418in}}%
\pgfpathlineto{\pgfqpoint{4.172240in}{2.142418in}}%
\pgfpathlineto{\pgfqpoint{4.172444in}{2.241438in}}%
\pgfpathlineto{\pgfqpoint{4.173057in}{2.101160in}}%
\pgfpathlineto{\pgfqpoint{4.173465in}{2.208431in}}%
\pgfpathlineto{\pgfqpoint{4.173873in}{2.167173in}}%
\pgfpathlineto{\pgfqpoint{4.174690in}{2.233186in}}%
\pgfpathlineto{\pgfqpoint{4.175098in}{2.175425in}}%
\pgfpathlineto{\pgfqpoint{4.175506in}{2.233186in}}%
\pgfpathlineto{\pgfqpoint{4.175710in}{2.299199in}}%
\pgfpathlineto{\pgfqpoint{4.176527in}{2.208431in}}%
\pgfpathlineto{\pgfqpoint{4.176731in}{2.290947in}}%
\pgfpathlineto{\pgfqpoint{4.178160in}{2.208431in}}%
\pgfpathlineto{\pgfqpoint{4.179181in}{2.315702in}}%
\pgfpathlineto{\pgfqpoint{4.180202in}{2.241438in}}%
\pgfpathlineto{\pgfqpoint{4.181018in}{2.323954in}}%
\pgfpathlineto{\pgfqpoint{4.181426in}{2.307450in}}%
\pgfpathlineto{\pgfqpoint{4.181631in}{2.315702in}}%
\pgfpathlineto{\pgfqpoint{4.182039in}{2.340457in}}%
\pgfpathlineto{\pgfqpoint{4.183264in}{2.150670in}}%
\pgfpathlineto{\pgfqpoint{4.185305in}{2.266192in}}%
\pgfpathlineto{\pgfqpoint{4.185713in}{2.035148in}}%
\pgfpathlineto{\pgfqpoint{4.186122in}{2.356960in}}%
\pgfpathlineto{\pgfqpoint{4.186326in}{2.266192in}}%
\pgfpathlineto{\pgfqpoint{4.186530in}{2.257941in}}%
\pgfpathlineto{\pgfqpoint{4.187347in}{2.323954in}}%
\pgfpathlineto{\pgfqpoint{4.187551in}{2.299199in}}%
\pgfpathlineto{\pgfqpoint{4.188163in}{2.257941in}}%
\pgfpathlineto{\pgfqpoint{4.188776in}{2.274444in}}%
\pgfpathlineto{\pgfqpoint{4.188980in}{2.109412in}}%
\pgfpathlineto{\pgfqpoint{4.189796in}{2.323954in}}%
\pgfpathlineto{\pgfqpoint{4.190613in}{2.356960in}}%
\pgfpathlineto{\pgfqpoint{4.190409in}{2.299199in}}%
\pgfpathlineto{\pgfqpoint{4.190817in}{2.332205in}}%
\pgfpathlineto{\pgfqpoint{4.191838in}{2.076406in}}%
\pgfpathlineto{\pgfqpoint{4.191429in}{2.356960in}}%
\pgfpathlineto{\pgfqpoint{4.192246in}{2.290947in}}%
\pgfpathlineto{\pgfqpoint{4.192450in}{2.299199in}}%
\pgfpathlineto{\pgfqpoint{4.193063in}{2.233186in}}%
\pgfpathlineto{\pgfqpoint{4.193267in}{2.257941in}}%
\pgfpathlineto{\pgfqpoint{4.193471in}{2.323954in}}%
\pgfpathlineto{\pgfqpoint{4.194287in}{2.249689in}}%
\pgfpathlineto{\pgfqpoint{4.195104in}{2.233186in}}%
\pgfpathlineto{\pgfqpoint{4.196533in}{2.356960in}}%
\pgfpathlineto{\pgfqpoint{4.196737in}{2.365212in}}%
\pgfpathlineto{\pgfqpoint{4.197554in}{2.233186in}}%
\pgfpathlineto{\pgfqpoint{4.197962in}{2.299199in}}%
\pgfpathlineto{\pgfqpoint{4.198983in}{2.348709in}}%
\pgfpathlineto{\pgfqpoint{4.199187in}{2.282696in}}%
\pgfpathlineto{\pgfqpoint{4.199799in}{2.373463in}}%
\pgfpathlineto{\pgfqpoint{4.200003in}{2.348709in}}%
\pgfpathlineto{\pgfqpoint{4.200208in}{2.381715in}}%
\pgfpathlineto{\pgfqpoint{4.200616in}{2.323954in}}%
\pgfpathlineto{\pgfqpoint{4.200820in}{2.356960in}}%
\pgfpathlineto{\pgfqpoint{4.201637in}{2.117664in}}%
\pgfpathlineto{\pgfqpoint{4.201841in}{2.307450in}}%
\pgfpathlineto{\pgfqpoint{4.202861in}{2.439476in}}%
\pgfpathlineto{\pgfqpoint{4.203066in}{2.389967in}}%
\pgfpathlineto{\pgfqpoint{4.203678in}{2.447728in}}%
\pgfpathlineto{\pgfqpoint{4.204086in}{2.422973in}}%
\pgfpathlineto{\pgfqpoint{4.205719in}{2.076406in}}%
\pgfpathlineto{\pgfqpoint{4.206944in}{2.340457in}}%
\pgfpathlineto{\pgfqpoint{4.207965in}{2.373463in}}%
\pgfpathlineto{\pgfqpoint{4.208169in}{2.365212in}}%
\pgfpathlineto{\pgfqpoint{4.208986in}{2.274444in}}%
\pgfpathlineto{\pgfqpoint{4.209190in}{2.373463in}}%
\pgfpathlineto{\pgfqpoint{4.209598in}{2.315702in}}%
\pgfpathlineto{\pgfqpoint{4.210006in}{2.373463in}}%
\pgfpathlineto{\pgfqpoint{4.210210in}{2.398218in}}%
\pgfpathlineto{\pgfqpoint{4.210415in}{2.307450in}}%
\pgfpathlineto{\pgfqpoint{4.210619in}{2.307450in}}%
\pgfpathlineto{\pgfqpoint{4.211027in}{2.257941in}}%
\pgfpathlineto{\pgfqpoint{4.211231in}{2.323954in}}%
\pgfpathlineto{\pgfqpoint{4.211639in}{2.406470in}}%
\pgfpathlineto{\pgfqpoint{4.212252in}{2.381715in}}%
\pgfpathlineto{\pgfqpoint{4.213885in}{2.158922in}}%
\pgfpathlineto{\pgfqpoint{4.214293in}{2.340457in}}%
\pgfpathlineto{\pgfqpoint{4.215110in}{2.282696in}}%
\pgfpathlineto{\pgfqpoint{4.215314in}{2.274444in}}%
\pgfpathlineto{\pgfqpoint{4.216335in}{2.389967in}}%
\pgfpathlineto{\pgfqpoint{4.215722in}{2.266192in}}%
\pgfpathlineto{\pgfqpoint{4.216539in}{2.365212in}}%
\pgfpathlineto{\pgfqpoint{4.217764in}{2.290947in}}%
\pgfpathlineto{\pgfqpoint{4.218989in}{2.422973in}}%
\pgfpathlineto{\pgfqpoint{4.220009in}{2.323954in}}%
\pgfpathlineto{\pgfqpoint{4.220213in}{2.365212in}}%
\pgfpathlineto{\pgfqpoint{4.221030in}{2.332205in}}%
\pgfpathlineto{\pgfqpoint{4.220622in}{2.389967in}}%
\pgfpathlineto{\pgfqpoint{4.221234in}{2.365212in}}%
\pgfpathlineto{\pgfqpoint{4.221642in}{2.381715in}}%
\pgfpathlineto{\pgfqpoint{4.221847in}{2.348709in}}%
\pgfpathlineto{\pgfqpoint{4.222051in}{2.365212in}}%
\pgfpathlineto{\pgfqpoint{4.223071in}{2.315702in}}%
\pgfpathlineto{\pgfqpoint{4.223276in}{2.406470in}}%
\pgfpathlineto{\pgfqpoint{4.224092in}{2.381715in}}%
\pgfpathlineto{\pgfqpoint{4.224296in}{2.340457in}}%
\pgfpathlineto{\pgfqpoint{4.224705in}{2.398218in}}%
\pgfpathlineto{\pgfqpoint{4.225113in}{2.398218in}}%
\pgfpathlineto{\pgfqpoint{4.225317in}{2.414721in}}%
\pgfpathlineto{\pgfqpoint{4.225521in}{2.389967in}}%
\pgfpathlineto{\pgfqpoint{4.225929in}{2.389967in}}%
\pgfpathlineto{\pgfqpoint{4.227563in}{2.274444in}}%
\pgfpathlineto{\pgfqpoint{4.229400in}{2.373463in}}%
\pgfpathlineto{\pgfqpoint{4.228175in}{2.257941in}}%
\pgfpathlineto{\pgfqpoint{4.229604in}{2.356960in}}%
\pgfpathlineto{\pgfqpoint{4.229808in}{2.216683in}}%
\pgfpathlineto{\pgfqpoint{4.230421in}{2.389967in}}%
\pgfpathlineto{\pgfqpoint{4.230625in}{2.356960in}}%
\pgfpathlineto{\pgfqpoint{4.231237in}{2.323954in}}%
\pgfpathlineto{\pgfqpoint{4.232054in}{2.422973in}}%
\pgfpathlineto{\pgfqpoint{4.232666in}{2.340457in}}%
\pgfpathlineto{\pgfqpoint{4.232870in}{2.431225in}}%
\pgfpathlineto{\pgfqpoint{4.233074in}{2.505489in}}%
\pgfpathlineto{\pgfqpoint{4.233687in}{2.356960in}}%
\pgfpathlineto{\pgfqpoint{4.233891in}{2.422973in}}%
\pgfpathlineto{\pgfqpoint{4.234503in}{2.431225in}}%
\pgfpathlineto{\pgfqpoint{4.235320in}{2.249689in}}%
\pgfpathlineto{\pgfqpoint{4.236137in}{2.340457in}}%
\pgfpathlineto{\pgfqpoint{4.236749in}{2.365212in}}%
\pgfpathlineto{\pgfqpoint{4.237361in}{2.299199in}}%
\pgfpathlineto{\pgfqpoint{4.237566in}{2.348709in}}%
\pgfpathlineto{\pgfqpoint{4.238382in}{2.340457in}}%
\pgfpathlineto{\pgfqpoint{4.239607in}{2.290947in}}%
\pgfpathlineto{\pgfqpoint{4.240015in}{2.356960in}}%
\pgfpathlineto{\pgfqpoint{4.240424in}{2.307450in}}%
\pgfpathlineto{\pgfqpoint{4.240628in}{2.266192in}}%
\pgfpathlineto{\pgfqpoint{4.241240in}{2.315702in}}%
\pgfpathlineto{\pgfqpoint{4.241444in}{2.315702in}}%
\pgfpathlineto{\pgfqpoint{4.243282in}{2.142418in}}%
\pgfpathlineto{\pgfqpoint{4.243486in}{2.191928in}}%
\pgfpathlineto{\pgfqpoint{4.244302in}{2.307450in}}%
\pgfpathlineto{\pgfqpoint{4.244915in}{2.274444in}}%
\pgfpathlineto{\pgfqpoint{4.246548in}{2.183676in}}%
\pgfpathlineto{\pgfqpoint{4.246956in}{2.134167in}}%
\pgfpathlineto{\pgfqpoint{4.247364in}{2.233186in}}%
\pgfpathlineto{\pgfqpoint{4.247977in}{2.290947in}}%
\pgfpathlineto{\pgfqpoint{4.248385in}{2.076406in}}%
\pgfpathlineto{\pgfqpoint{4.248793in}{2.332205in}}%
\pgfpathlineto{\pgfqpoint{4.248998in}{2.315702in}}%
\pgfpathlineto{\pgfqpoint{4.250222in}{2.431225in}}%
\pgfpathlineto{\pgfqpoint{4.250427in}{2.398218in}}%
\pgfpathlineto{\pgfqpoint{4.251039in}{2.323954in}}%
\pgfpathlineto{\pgfqpoint{4.250835in}{2.431225in}}%
\pgfpathlineto{\pgfqpoint{4.251651in}{2.348709in}}%
\pgfpathlineto{\pgfqpoint{4.252264in}{2.381715in}}%
\pgfpathlineto{\pgfqpoint{4.252876in}{2.076406in}}%
\pgfpathlineto{\pgfqpoint{4.253285in}{2.274444in}}%
\pgfpathlineto{\pgfqpoint{4.254509in}{2.406470in}}%
\pgfpathlineto{\pgfqpoint{4.253693in}{2.257941in}}%
\pgfpathlineto{\pgfqpoint{4.254714in}{2.348709in}}%
\pgfpathlineto{\pgfqpoint{4.256551in}{2.282696in}}%
\pgfpathlineto{\pgfqpoint{4.257163in}{2.332205in}}%
\pgfpathlineto{\pgfqpoint{4.257776in}{2.323954in}}%
\pgfpathlineto{\pgfqpoint{4.258592in}{2.307450in}}%
\pgfpathlineto{\pgfqpoint{4.258796in}{2.381715in}}%
\pgfpathlineto{\pgfqpoint{4.260021in}{2.101160in}}%
\pgfpathlineto{\pgfqpoint{4.260225in}{2.290947in}}%
\pgfpathlineto{\pgfqpoint{4.260430in}{2.356960in}}%
\pgfpathlineto{\pgfqpoint{4.261246in}{2.266192in}}%
\pgfpathlineto{\pgfqpoint{4.261450in}{2.290947in}}%
\pgfpathlineto{\pgfqpoint{4.261654in}{2.282696in}}%
\pgfpathlineto{\pgfqpoint{4.261859in}{2.084657in}}%
\pgfpathlineto{\pgfqpoint{4.262267in}{2.381715in}}%
\pgfpathlineto{\pgfqpoint{4.262675in}{2.150670in}}%
\pgfpathlineto{\pgfqpoint{4.263696in}{2.381715in}}%
\pgfpathlineto{\pgfqpoint{4.263900in}{2.348709in}}%
\pgfpathlineto{\pgfqpoint{4.264104in}{2.340457in}}%
\pgfpathlineto{\pgfqpoint{4.264308in}{2.348709in}}%
\pgfpathlineto{\pgfqpoint{4.264921in}{2.389967in}}%
\pgfpathlineto{\pgfqpoint{4.266146in}{2.290947in}}%
\pgfpathlineto{\pgfqpoint{4.266350in}{2.299199in}}%
\pgfpathlineto{\pgfqpoint{4.266554in}{2.282696in}}%
\pgfpathlineto{\pgfqpoint{4.266962in}{2.158922in}}%
\pgfpathlineto{\pgfqpoint{4.267779in}{2.257941in}}%
\pgfpathlineto{\pgfqpoint{4.268187in}{2.241438in}}%
\pgfpathlineto{\pgfqpoint{4.268391in}{2.257941in}}%
\pgfpathlineto{\pgfqpoint{4.268595in}{2.299199in}}%
\pgfpathlineto{\pgfqpoint{4.268799in}{2.101160in}}%
\pgfpathlineto{\pgfqpoint{4.269004in}{2.332205in}}%
\pgfpathlineto{\pgfqpoint{4.269616in}{2.274444in}}%
\pgfpathlineto{\pgfqpoint{4.270228in}{2.365212in}}%
\pgfpathlineto{\pgfqpoint{4.270637in}{2.315702in}}%
\pgfpathlineto{\pgfqpoint{4.270841in}{2.282696in}}%
\pgfpathlineto{\pgfqpoint{4.271045in}{2.389967in}}%
\pgfpathlineto{\pgfqpoint{4.271249in}{2.389967in}}%
\pgfpathlineto{\pgfqpoint{4.271657in}{2.365212in}}%
\pgfpathlineto{\pgfqpoint{4.271862in}{2.422973in}}%
\pgfpathlineto{\pgfqpoint{4.273086in}{2.117664in}}%
\pgfpathlineto{\pgfqpoint{4.274107in}{2.381715in}}%
\pgfpathlineto{\pgfqpoint{4.274311in}{2.323954in}}%
\pgfpathlineto{\pgfqpoint{4.274515in}{2.249689in}}%
\pgfpathlineto{\pgfqpoint{4.274720in}{2.431225in}}%
\pgfpathlineto{\pgfqpoint{4.275128in}{2.365212in}}%
\pgfpathlineto{\pgfqpoint{4.275536in}{2.414721in}}%
\pgfpathlineto{\pgfqpoint{4.276149in}{2.373463in}}%
\pgfpathlineto{\pgfqpoint{4.276761in}{2.389967in}}%
\pgfpathlineto{\pgfqpoint{4.277169in}{2.340457in}}%
\pgfpathlineto{\pgfqpoint{4.277578in}{2.332205in}}%
\pgfpathlineto{\pgfqpoint{4.278394in}{2.381715in}}%
\pgfpathlineto{\pgfqpoint{4.278598in}{2.373463in}}%
\pgfpathlineto{\pgfqpoint{4.279007in}{2.422973in}}%
\pgfpathlineto{\pgfqpoint{4.279415in}{2.356960in}}%
\pgfpathlineto{\pgfqpoint{4.279619in}{2.356960in}}%
\pgfpathlineto{\pgfqpoint{4.279823in}{2.315702in}}%
\pgfpathlineto{\pgfqpoint{4.280231in}{2.373463in}}%
\pgfpathlineto{\pgfqpoint{4.280844in}{2.480734in}}%
\pgfpathlineto{\pgfqpoint{4.281456in}{2.464231in}}%
\pgfpathlineto{\pgfqpoint{4.282477in}{2.406470in}}%
\pgfpathlineto{\pgfqpoint{4.282681in}{2.414721in}}%
\pgfpathlineto{\pgfqpoint{4.282885in}{2.447728in}}%
\pgfpathlineto{\pgfqpoint{4.283294in}{2.381715in}}%
\pgfpathlineto{\pgfqpoint{4.284314in}{2.282696in}}%
\pgfpathlineto{\pgfqpoint{4.284518in}{2.340457in}}%
\pgfpathlineto{\pgfqpoint{4.284927in}{2.109412in}}%
\pgfpathlineto{\pgfqpoint{4.285539in}{2.299199in}}%
\pgfpathlineto{\pgfqpoint{4.286356in}{2.422973in}}%
\pgfpathlineto{\pgfqpoint{4.286764in}{2.381715in}}%
\pgfpathlineto{\pgfqpoint{4.286968in}{2.373463in}}%
\pgfpathlineto{\pgfqpoint{4.287376in}{2.497237in}}%
\pgfpathlineto{\pgfqpoint{4.287581in}{2.175425in}}%
\pgfpathlineto{\pgfqpoint{4.288397in}{2.422973in}}%
\pgfpathlineto{\pgfqpoint{4.288601in}{2.406470in}}%
\pgfpathlineto{\pgfqpoint{4.288805in}{2.472483in}}%
\pgfpathlineto{\pgfqpoint{4.289010in}{2.447728in}}%
\pgfpathlineto{\pgfqpoint{4.289418in}{2.505489in}}%
\pgfpathlineto{\pgfqpoint{4.289826in}{2.356960in}}%
\pgfpathlineto{\pgfqpoint{4.290643in}{2.389967in}}%
\pgfpathlineto{\pgfqpoint{4.290847in}{2.398218in}}%
\pgfpathlineto{\pgfqpoint{4.291051in}{2.373463in}}%
\pgfpathlineto{\pgfqpoint{4.291459in}{2.389967in}}%
\pgfpathlineto{\pgfqpoint{4.291663in}{2.340457in}}%
\pgfpathlineto{\pgfqpoint{4.292480in}{2.365212in}}%
\pgfpathlineto{\pgfqpoint{4.292684in}{2.431225in}}%
\pgfpathlineto{\pgfqpoint{4.293296in}{2.290947in}}%
\pgfpathlineto{\pgfqpoint{4.293501in}{2.282696in}}%
\pgfpathlineto{\pgfqpoint{4.293705in}{2.290947in}}%
\pgfpathlineto{\pgfqpoint{4.295338in}{2.365212in}}%
\pgfpathlineto{\pgfqpoint{4.296154in}{2.282696in}}%
\pgfpathlineto{\pgfqpoint{4.296563in}{2.332205in}}%
\pgfpathlineto{\pgfqpoint{4.297583in}{2.257941in}}%
\pgfpathlineto{\pgfqpoint{4.298808in}{2.389967in}}%
\pgfpathlineto{\pgfqpoint{4.299012in}{2.381715in}}%
\pgfpathlineto{\pgfqpoint{4.300850in}{2.480734in}}%
\pgfpathlineto{\pgfqpoint{4.302075in}{2.398218in}}%
\pgfpathlineto{\pgfqpoint{4.301258in}{2.488986in}}%
\pgfpathlineto{\pgfqpoint{4.302279in}{2.414721in}}%
\pgfpathlineto{\pgfqpoint{4.302687in}{2.480734in}}%
\pgfpathlineto{\pgfqpoint{4.303095in}{2.389967in}}%
\pgfpathlineto{\pgfqpoint{4.303504in}{2.365212in}}%
\pgfpathlineto{\pgfqpoint{4.303708in}{2.381715in}}%
\pgfpathlineto{\pgfqpoint{4.303912in}{2.431225in}}%
\pgfpathlineto{\pgfqpoint{4.304728in}{2.381715in}}%
\pgfpathlineto{\pgfqpoint{4.305137in}{2.365212in}}%
\pgfpathlineto{\pgfqpoint{4.305341in}{2.414721in}}%
\pgfpathlineto{\pgfqpoint{4.305545in}{2.381715in}}%
\pgfpathlineto{\pgfqpoint{4.306362in}{2.373463in}}%
\pgfpathlineto{\pgfqpoint{4.306770in}{2.472483in}}%
\pgfpathlineto{\pgfqpoint{4.307382in}{2.348709in}}%
\pgfpathlineto{\pgfqpoint{4.307791in}{2.414721in}}%
\pgfpathlineto{\pgfqpoint{4.307995in}{2.406470in}}%
\pgfpathlineto{\pgfqpoint{4.309220in}{2.455979in}}%
\pgfpathlineto{\pgfqpoint{4.309628in}{2.472483in}}%
\pgfpathlineto{\pgfqpoint{4.309832in}{2.439476in}}%
\pgfpathlineto{\pgfqpoint{4.311057in}{2.521992in}}%
\pgfpathlineto{\pgfqpoint{4.311873in}{2.414721in}}%
\pgfpathlineto{\pgfqpoint{4.312282in}{2.431225in}}%
\pgfpathlineto{\pgfqpoint{4.312486in}{2.488986in}}%
\pgfpathlineto{\pgfqpoint{4.313302in}{2.406470in}}%
\pgfpathlineto{\pgfqpoint{4.313915in}{2.447728in}}%
\pgfpathlineto{\pgfqpoint{4.314119in}{2.381715in}}%
\pgfpathlineto{\pgfqpoint{4.314323in}{2.422973in}}%
\pgfpathlineto{\pgfqpoint{4.314527in}{2.365212in}}%
\pgfpathlineto{\pgfqpoint{4.315344in}{2.414721in}}%
\pgfpathlineto{\pgfqpoint{4.315548in}{2.431225in}}%
\pgfpathlineto{\pgfqpoint{4.316160in}{2.340457in}}%
\pgfpathlineto{\pgfqpoint{4.316773in}{2.356960in}}%
\pgfpathlineto{\pgfqpoint{4.317589in}{2.505489in}}%
\pgfpathlineto{\pgfqpoint{4.318202in}{2.455979in}}%
\pgfpathlineto{\pgfqpoint{4.318406in}{2.315702in}}%
\pgfpathlineto{\pgfqpoint{4.319223in}{2.398218in}}%
\pgfpathlineto{\pgfqpoint{4.319427in}{2.439476in}}%
\pgfpathlineto{\pgfqpoint{4.320039in}{2.356960in}}%
\pgfpathlineto{\pgfqpoint{4.320447in}{2.406470in}}%
\pgfpathlineto{\pgfqpoint{4.320856in}{2.373463in}}%
\pgfpathlineto{\pgfqpoint{4.321672in}{2.464231in}}%
\pgfpathlineto{\pgfqpoint{4.321876in}{2.282696in}}%
\pgfpathlineto{\pgfqpoint{4.322693in}{2.373463in}}%
\pgfpathlineto{\pgfqpoint{4.323714in}{2.315702in}}%
\pgfpathlineto{\pgfqpoint{4.323918in}{2.084657in}}%
\pgfpathlineto{\pgfqpoint{4.324734in}{2.389967in}}%
\pgfpathlineto{\pgfqpoint{4.324939in}{2.447728in}}%
\pgfpathlineto{\pgfqpoint{4.325551in}{2.348709in}}%
\pgfpathlineto{\pgfqpoint{4.325959in}{2.422973in}}%
\pgfpathlineto{\pgfqpoint{4.326163in}{2.373463in}}%
\pgfpathlineto{\pgfqpoint{4.326368in}{2.431225in}}%
\pgfpathlineto{\pgfqpoint{4.326776in}{2.414721in}}%
\pgfpathlineto{\pgfqpoint{4.327592in}{2.505489in}}%
\pgfpathlineto{\pgfqpoint{4.327184in}{2.389967in}}%
\pgfpathlineto{\pgfqpoint{4.327797in}{2.480734in}}%
\pgfpathlineto{\pgfqpoint{4.328409in}{2.431225in}}%
\pgfpathlineto{\pgfqpoint{4.328817in}{2.439476in}}%
\pgfpathlineto{\pgfqpoint{4.329021in}{2.505489in}}%
\pgfpathlineto{\pgfqpoint{4.329226in}{2.414721in}}%
\pgfpathlineto{\pgfqpoint{4.329838in}{2.439476in}}%
\pgfpathlineto{\pgfqpoint{4.330042in}{2.455979in}}%
\pgfpathlineto{\pgfqpoint{4.330450in}{2.183676in}}%
\pgfpathlineto{\pgfqpoint{4.331063in}{2.389967in}}%
\pgfpathlineto{\pgfqpoint{4.331471in}{2.431225in}}%
\pgfpathlineto{\pgfqpoint{4.331879in}{2.356960in}}%
\pgfpathlineto{\pgfqpoint{4.332084in}{2.414721in}}%
\pgfpathlineto{\pgfqpoint{4.332492in}{2.365212in}}%
\pgfpathlineto{\pgfqpoint{4.332696in}{2.422973in}}%
\pgfpathlineto{\pgfqpoint{4.332900in}{2.422973in}}%
\pgfpathlineto{\pgfqpoint{4.335758in}{2.579753in}}%
\pgfpathlineto{\pgfqpoint{4.335962in}{2.612760in}}%
\pgfpathlineto{\pgfqpoint{4.336371in}{2.521992in}}%
\pgfpathlineto{\pgfqpoint{4.336575in}{2.588005in}}%
\pgfpathlineto{\pgfqpoint{4.337800in}{2.538495in}}%
\pgfpathlineto{\pgfqpoint{4.338004in}{2.604508in}}%
\pgfpathlineto{\pgfqpoint{4.338208in}{2.389967in}}%
\pgfpathlineto{\pgfqpoint{4.338616in}{2.323954in}}%
\pgfpathlineto{\pgfqpoint{4.338616in}{2.323954in}}%
\pgfusepath{stroke}%
\end{pgfscope}%
\begin{pgfscope}%
\pgfsetrectcap%
\pgfsetmiterjoin%
\pgfsetlinewidth{0.803000pt}%
\definecolor{currentstroke}{rgb}{0.000000,0.000000,0.000000}%
\pgfsetstrokecolor{currentstroke}%
\pgfsetdash{}{0pt}%
\pgfpathmoveto{\pgfqpoint{0.634869in}{0.539544in}}%
\pgfpathlineto{\pgfqpoint{0.634869in}{2.944887in}}%
\pgfusepath{stroke}%
\end{pgfscope}%
\begin{pgfscope}%
\pgfsetrectcap%
\pgfsetmiterjoin%
\pgfsetlinewidth{0.803000pt}%
\definecolor{currentstroke}{rgb}{0.000000,0.000000,0.000000}%
\pgfsetstrokecolor{currentstroke}%
\pgfsetdash{}{0pt}%
\pgfpathmoveto{\pgfqpoint{4.514985in}{0.539544in}}%
\pgfpathlineto{\pgfqpoint{4.514985in}{2.944887in}}%
\pgfusepath{stroke}%
\end{pgfscope}%
\begin{pgfscope}%
\pgfsetrectcap%
\pgfsetmiterjoin%
\pgfsetlinewidth{0.803000pt}%
\definecolor{currentstroke}{rgb}{0.000000,0.000000,0.000000}%
\pgfsetstrokecolor{currentstroke}%
\pgfsetdash{}{0pt}%
\pgfpathmoveto{\pgfqpoint{0.634869in}{0.539544in}}%
\pgfpathlineto{\pgfqpoint{4.514985in}{0.539544in}}%
\pgfusepath{stroke}%
\end{pgfscope}%
\begin{pgfscope}%
\pgfsetrectcap%
\pgfsetmiterjoin%
\pgfsetlinewidth{0.803000pt}%
\definecolor{currentstroke}{rgb}{0.000000,0.000000,0.000000}%
\pgfsetstrokecolor{currentstroke}%
\pgfsetdash{}{0pt}%
\pgfpathmoveto{\pgfqpoint{0.634869in}{2.944887in}}%
\pgfpathlineto{\pgfqpoint{4.514985in}{2.944887in}}%
\pgfusepath{stroke}%
\end{pgfscope}%
\begin{pgfscope}%
\pgfsetbuttcap%
\pgfsetroundjoin%
\definecolor{currentfill}{rgb}{0.000000,0.000000,0.000000}%
\pgfsetfillcolor{currentfill}%
\pgfsetlinewidth{0.803000pt}%
\definecolor{currentstroke}{rgb}{0.000000,0.000000,0.000000}%
\pgfsetstrokecolor{currentstroke}%
\pgfsetdash{}{0pt}%
\pgfsys@defobject{currentmarker}{\pgfqpoint{0.000000in}{0.000000in}}{\pgfqpoint{0.048611in}{0.000000in}}{%
\pgfpathmoveto{\pgfqpoint{0.000000in}{0.000000in}}%
\pgfpathlineto{\pgfqpoint{0.048611in}{0.000000in}}%
\pgfusepath{stroke,fill}%
}%
\begin{pgfscope}%
\pgfsys@transformshift{4.514985in}{0.725412in}%
\pgfsys@useobject{currentmarker}{}%
\end{pgfscope}%
\end{pgfscope}%
\begin{pgfscope}%
\definecolor{textcolor}{rgb}{0.000000,0.000000,0.000000}%
\pgfsetstrokecolor{textcolor}%
\pgfsetfillcolor{textcolor}%
\pgftext[x=4.612207in, y=0.686856in, left, base]{\color{textcolor}\rmfamily\fontsize{8.000000}{9.600000}\selectfont \(\displaystyle {20.00}\)}%
\end{pgfscope}%
\begin{pgfscope}%
\pgfsetbuttcap%
\pgfsetroundjoin%
\definecolor{currentfill}{rgb}{0.000000,0.000000,0.000000}%
\pgfsetfillcolor{currentfill}%
\pgfsetlinewidth{0.803000pt}%
\definecolor{currentstroke}{rgb}{0.000000,0.000000,0.000000}%
\pgfsetstrokecolor{currentstroke}%
\pgfsetdash{}{0pt}%
\pgfsys@defobject{currentmarker}{\pgfqpoint{0.000000in}{0.000000in}}{\pgfqpoint{0.048611in}{0.000000in}}{%
\pgfpathmoveto{\pgfqpoint{0.000000in}{0.000000in}}%
\pgfpathlineto{\pgfqpoint{0.048611in}{0.000000in}}%
\pgfusepath{stroke,fill}%
}%
\begin{pgfscope}%
\pgfsys@transformshift{4.514985in}{0.998746in}%
\pgfsys@useobject{currentmarker}{}%
\end{pgfscope}%
\end{pgfscope}%
\begin{pgfscope}%
\definecolor{textcolor}{rgb}{0.000000,0.000000,0.000000}%
\pgfsetstrokecolor{textcolor}%
\pgfsetfillcolor{textcolor}%
\pgftext[x=4.612207in, y=0.960190in, left, base]{\color{textcolor}\rmfamily\fontsize{8.000000}{9.600000}\selectfont \(\displaystyle {20.25}\)}%
\end{pgfscope}%
\begin{pgfscope}%
\pgfsetbuttcap%
\pgfsetroundjoin%
\definecolor{currentfill}{rgb}{0.000000,0.000000,0.000000}%
\pgfsetfillcolor{currentfill}%
\pgfsetlinewidth{0.803000pt}%
\definecolor{currentstroke}{rgb}{0.000000,0.000000,0.000000}%
\pgfsetstrokecolor{currentstroke}%
\pgfsetdash{}{0pt}%
\pgfsys@defobject{currentmarker}{\pgfqpoint{0.000000in}{0.000000in}}{\pgfqpoint{0.048611in}{0.000000in}}{%
\pgfpathmoveto{\pgfqpoint{0.000000in}{0.000000in}}%
\pgfpathlineto{\pgfqpoint{0.048611in}{0.000000in}}%
\pgfusepath{stroke,fill}%
}%
\begin{pgfscope}%
\pgfsys@transformshift{4.514985in}{1.272080in}%
\pgfsys@useobject{currentmarker}{}%
\end{pgfscope}%
\end{pgfscope}%
\begin{pgfscope}%
\definecolor{textcolor}{rgb}{0.000000,0.000000,0.000000}%
\pgfsetstrokecolor{textcolor}%
\pgfsetfillcolor{textcolor}%
\pgftext[x=4.612207in, y=1.233525in, left, base]{\color{textcolor}\rmfamily\fontsize{8.000000}{9.600000}\selectfont \(\displaystyle {20.50}\)}%
\end{pgfscope}%
\begin{pgfscope}%
\pgfsetbuttcap%
\pgfsetroundjoin%
\definecolor{currentfill}{rgb}{0.000000,0.000000,0.000000}%
\pgfsetfillcolor{currentfill}%
\pgfsetlinewidth{0.803000pt}%
\definecolor{currentstroke}{rgb}{0.000000,0.000000,0.000000}%
\pgfsetstrokecolor{currentstroke}%
\pgfsetdash{}{0pt}%
\pgfsys@defobject{currentmarker}{\pgfqpoint{0.000000in}{0.000000in}}{\pgfqpoint{0.048611in}{0.000000in}}{%
\pgfpathmoveto{\pgfqpoint{0.000000in}{0.000000in}}%
\pgfpathlineto{\pgfqpoint{0.048611in}{0.000000in}}%
\pgfusepath{stroke,fill}%
}%
\begin{pgfscope}%
\pgfsys@transformshift{4.514985in}{1.545415in}%
\pgfsys@useobject{currentmarker}{}%
\end{pgfscope}%
\end{pgfscope}%
\begin{pgfscope}%
\definecolor{textcolor}{rgb}{0.000000,0.000000,0.000000}%
\pgfsetstrokecolor{textcolor}%
\pgfsetfillcolor{textcolor}%
\pgftext[x=4.612207in, y=1.506859in, left, base]{\color{textcolor}\rmfamily\fontsize{8.000000}{9.600000}\selectfont \(\displaystyle {20.75}\)}%
\end{pgfscope}%
\begin{pgfscope}%
\pgfsetbuttcap%
\pgfsetroundjoin%
\definecolor{currentfill}{rgb}{0.000000,0.000000,0.000000}%
\pgfsetfillcolor{currentfill}%
\pgfsetlinewidth{0.803000pt}%
\definecolor{currentstroke}{rgb}{0.000000,0.000000,0.000000}%
\pgfsetstrokecolor{currentstroke}%
\pgfsetdash{}{0pt}%
\pgfsys@defobject{currentmarker}{\pgfqpoint{0.000000in}{0.000000in}}{\pgfqpoint{0.048611in}{0.000000in}}{%
\pgfpathmoveto{\pgfqpoint{0.000000in}{0.000000in}}%
\pgfpathlineto{\pgfqpoint{0.048611in}{0.000000in}}%
\pgfusepath{stroke,fill}%
}%
\begin{pgfscope}%
\pgfsys@transformshift{4.514985in}{1.818749in}%
\pgfsys@useobject{currentmarker}{}%
\end{pgfscope}%
\end{pgfscope}%
\begin{pgfscope}%
\definecolor{textcolor}{rgb}{0.000000,0.000000,0.000000}%
\pgfsetstrokecolor{textcolor}%
\pgfsetfillcolor{textcolor}%
\pgftext[x=4.612207in, y=1.780194in, left, base]{\color{textcolor}\rmfamily\fontsize{8.000000}{9.600000}\selectfont \(\displaystyle {21.00}\)}%
\end{pgfscope}%
\begin{pgfscope}%
\pgfsetbuttcap%
\pgfsetroundjoin%
\definecolor{currentfill}{rgb}{0.000000,0.000000,0.000000}%
\pgfsetfillcolor{currentfill}%
\pgfsetlinewidth{0.803000pt}%
\definecolor{currentstroke}{rgb}{0.000000,0.000000,0.000000}%
\pgfsetstrokecolor{currentstroke}%
\pgfsetdash{}{0pt}%
\pgfsys@defobject{currentmarker}{\pgfqpoint{0.000000in}{0.000000in}}{\pgfqpoint{0.048611in}{0.000000in}}{%
\pgfpathmoveto{\pgfqpoint{0.000000in}{0.000000in}}%
\pgfpathlineto{\pgfqpoint{0.048611in}{0.000000in}}%
\pgfusepath{stroke,fill}%
}%
\begin{pgfscope}%
\pgfsys@transformshift{4.514985in}{2.092084in}%
\pgfsys@useobject{currentmarker}{}%
\end{pgfscope}%
\end{pgfscope}%
\begin{pgfscope}%
\definecolor{textcolor}{rgb}{0.000000,0.000000,0.000000}%
\pgfsetstrokecolor{textcolor}%
\pgfsetfillcolor{textcolor}%
\pgftext[x=4.612207in, y=2.053528in, left, base]{\color{textcolor}\rmfamily\fontsize{8.000000}{9.600000}\selectfont \(\displaystyle {21.25}\)}%
\end{pgfscope}%
\begin{pgfscope}%
\pgfsetbuttcap%
\pgfsetroundjoin%
\definecolor{currentfill}{rgb}{0.000000,0.000000,0.000000}%
\pgfsetfillcolor{currentfill}%
\pgfsetlinewidth{0.803000pt}%
\definecolor{currentstroke}{rgb}{0.000000,0.000000,0.000000}%
\pgfsetstrokecolor{currentstroke}%
\pgfsetdash{}{0pt}%
\pgfsys@defobject{currentmarker}{\pgfqpoint{0.000000in}{0.000000in}}{\pgfqpoint{0.048611in}{0.000000in}}{%
\pgfpathmoveto{\pgfqpoint{0.000000in}{0.000000in}}%
\pgfpathlineto{\pgfqpoint{0.048611in}{0.000000in}}%
\pgfusepath{stroke,fill}%
}%
\begin{pgfscope}%
\pgfsys@transformshift{4.514985in}{2.365418in}%
\pgfsys@useobject{currentmarker}{}%
\end{pgfscope}%
\end{pgfscope}%
\begin{pgfscope}%
\definecolor{textcolor}{rgb}{0.000000,0.000000,0.000000}%
\pgfsetstrokecolor{textcolor}%
\pgfsetfillcolor{textcolor}%
\pgftext[x=4.612207in, y=2.326862in, left, base]{\color{textcolor}\rmfamily\fontsize{8.000000}{9.600000}\selectfont \(\displaystyle {21.50}\)}%
\end{pgfscope}%
\begin{pgfscope}%
\pgfsetbuttcap%
\pgfsetroundjoin%
\definecolor{currentfill}{rgb}{0.000000,0.000000,0.000000}%
\pgfsetfillcolor{currentfill}%
\pgfsetlinewidth{0.803000pt}%
\definecolor{currentstroke}{rgb}{0.000000,0.000000,0.000000}%
\pgfsetstrokecolor{currentstroke}%
\pgfsetdash{}{0pt}%
\pgfsys@defobject{currentmarker}{\pgfqpoint{0.000000in}{0.000000in}}{\pgfqpoint{0.048611in}{0.000000in}}{%
\pgfpathmoveto{\pgfqpoint{0.000000in}{0.000000in}}%
\pgfpathlineto{\pgfqpoint{0.048611in}{0.000000in}}%
\pgfusepath{stroke,fill}%
}%
\begin{pgfscope}%
\pgfsys@transformshift{4.514985in}{2.638752in}%
\pgfsys@useobject{currentmarker}{}%
\end{pgfscope}%
\end{pgfscope}%
\begin{pgfscope}%
\definecolor{textcolor}{rgb}{0.000000,0.000000,0.000000}%
\pgfsetstrokecolor{textcolor}%
\pgfsetfillcolor{textcolor}%
\pgftext[x=4.612207in, y=2.600197in, left, base]{\color{textcolor}\rmfamily\fontsize{8.000000}{9.600000}\selectfont \(\displaystyle {21.75}\)}%
\end{pgfscope}%
\begin{pgfscope}%
\pgfsetbuttcap%
\pgfsetroundjoin%
\definecolor{currentfill}{rgb}{0.000000,0.000000,0.000000}%
\pgfsetfillcolor{currentfill}%
\pgfsetlinewidth{0.803000pt}%
\definecolor{currentstroke}{rgb}{0.000000,0.000000,0.000000}%
\pgfsetstrokecolor{currentstroke}%
\pgfsetdash{}{0pt}%
\pgfsys@defobject{currentmarker}{\pgfqpoint{0.000000in}{0.000000in}}{\pgfqpoint{0.048611in}{0.000000in}}{%
\pgfpathmoveto{\pgfqpoint{0.000000in}{0.000000in}}%
\pgfpathlineto{\pgfqpoint{0.048611in}{0.000000in}}%
\pgfusepath{stroke,fill}%
}%
\begin{pgfscope}%
\pgfsys@transformshift{4.514985in}{2.912087in}%
\pgfsys@useobject{currentmarker}{}%
\end{pgfscope}%
\end{pgfscope}%
\begin{pgfscope}%
\definecolor{textcolor}{rgb}{0.000000,0.000000,0.000000}%
\pgfsetstrokecolor{textcolor}%
\pgfsetfillcolor{textcolor}%
\pgftext[x=4.612207in, y=2.873531in, left, base]{\color{textcolor}\rmfamily\fontsize{8.000000}{9.600000}\selectfont \(\displaystyle {22.00}\)}%
\end{pgfscope}%
\begin{pgfscope}%
\definecolor{textcolor}{rgb}{0.000000,0.000000,0.000000}%
\pgfsetstrokecolor{textcolor}%
\pgfsetfillcolor{textcolor}%
\pgftext[x=4.936671in,y=1.742216in,,top,rotate=90.000000]{\color{textcolor}\rmfamily\fontsize{10.000000}{12.000000}\selectfont Temperature in °C}%
\end{pgfscope}%
\begin{pgfscope}%
\pgfpathrectangle{\pgfqpoint{0.634869in}{0.539544in}}{\pgfqpoint{3.880116in}{2.405343in}}%
\pgfusepath{clip}%
\pgfsetrectcap%
\pgfsetroundjoin%
\pgfsetlinewidth{0.501875pt}%
\definecolor{currentstroke}{rgb}{0.698039,0.133333,0.133333}%
\pgfsetstrokecolor{currentstroke}%
\pgfsetstrokeopacity{0.700000}%
\pgfsetdash{}{0pt}%
\pgfpathmoveto{\pgfqpoint{0.811770in}{1.949950in}}%
\pgfpathlineto{\pgfqpoint{0.817935in}{2.092084in}}%
\pgfpathlineto{\pgfqpoint{0.820507in}{2.015550in}}%
\pgfpathlineto{\pgfqpoint{0.825774in}{2.157684in}}%
\pgfpathlineto{\pgfqpoint{0.828224in}{2.092084in}}%
\pgfpathlineto{\pgfqpoint{0.834185in}{2.223284in}}%
\pgfpathlineto{\pgfqpoint{0.836634in}{2.157684in}}%
\pgfpathlineto{\pgfqpoint{0.839084in}{2.223284in}}%
\pgfpathlineto{\pgfqpoint{0.844759in}{2.288884in}}%
\pgfpathlineto{\pgfqpoint{0.847209in}{2.223284in}}%
\pgfpathlineto{\pgfqpoint{0.849659in}{2.288884in}}%
\pgfpathlineto{\pgfqpoint{0.858192in}{2.365418in}}%
\pgfpathlineto{\pgfqpoint{0.860641in}{2.288884in}}%
\pgfpathlineto{\pgfqpoint{0.863091in}{2.365418in}}%
\pgfpathlineto{\pgfqpoint{0.869542in}{2.431018in}}%
\pgfpathlineto{\pgfqpoint{0.871992in}{2.365418in}}%
\pgfpathlineto{\pgfqpoint{0.874441in}{2.431018in}}%
\pgfpathlineto{\pgfqpoint{0.877299in}{2.365418in}}%
\pgfpathlineto{\pgfqpoint{0.879749in}{2.431018in}}%
\pgfpathlineto{\pgfqpoint{0.886282in}{2.496619in}}%
\pgfpathlineto{\pgfqpoint{0.888731in}{2.431018in}}%
\pgfpathlineto{\pgfqpoint{0.891181in}{2.496619in}}%
\pgfpathlineto{\pgfqpoint{0.903144in}{2.562219in}}%
\pgfpathlineto{\pgfqpoint{0.905594in}{2.496619in}}%
\pgfpathlineto{\pgfqpoint{0.908084in}{2.562219in}}%
\pgfpathlineto{\pgfqpoint{0.910738in}{2.496619in}}%
\pgfpathlineto{\pgfqpoint{0.913188in}{2.562219in}}%
\pgfpathlineto{\pgfqpoint{0.923966in}{2.638752in}}%
\pgfpathlineto{\pgfqpoint{0.926416in}{2.562219in}}%
\pgfpathlineto{\pgfqpoint{0.928866in}{2.638752in}}%
\pgfpathlineto{\pgfqpoint{0.931560in}{2.562219in}}%
\pgfpathlineto{\pgfqpoint{0.934010in}{2.638752in}}%
\pgfpathlineto{\pgfqpoint{0.946830in}{2.704353in}}%
\pgfpathlineto{\pgfqpoint{0.949280in}{2.638752in}}%
\pgfpathlineto{\pgfqpoint{0.951730in}{2.704353in}}%
\pgfpathlineto{\pgfqpoint{0.954179in}{2.638752in}}%
\pgfpathlineto{\pgfqpoint{0.956629in}{2.704353in}}%
\pgfpathlineto{\pgfqpoint{0.971205in}{2.769953in}}%
\pgfpathlineto{\pgfqpoint{0.973655in}{2.704353in}}%
\pgfpathlineto{\pgfqpoint{0.976145in}{2.769953in}}%
\pgfpathlineto{\pgfqpoint{0.978595in}{2.704353in}}%
\pgfpathlineto{\pgfqpoint{0.981045in}{2.769953in}}%
\pgfpathlineto{\pgfqpoint{0.983780in}{2.704353in}}%
\pgfpathlineto{\pgfqpoint{0.986230in}{2.769953in}}%
\pgfpathlineto{\pgfqpoint{0.998805in}{2.835553in}}%
\pgfpathlineto{\pgfqpoint{1.001255in}{2.769953in}}%
\pgfpathlineto{\pgfqpoint{1.003704in}{2.835553in}}%
\pgfpathlineto{\pgfqpoint{1.006154in}{2.769953in}}%
\pgfpathlineto{\pgfqpoint{1.008604in}{2.835553in}}%
\pgfpathlineto{\pgfqpoint{1.011176in}{2.769953in}}%
\pgfpathlineto{\pgfqpoint{1.013626in}{2.835553in}}%
\pgfpathlineto{\pgfqpoint{1.019342in}{2.704353in}}%
\pgfpathlineto{\pgfqpoint{1.021791in}{2.638752in}}%
\pgfpathlineto{\pgfqpoint{1.026691in}{2.365418in}}%
\pgfpathlineto{\pgfqpoint{1.034040in}{2.157684in}}%
\pgfpathlineto{\pgfqpoint{1.036490in}{2.015550in}}%
\pgfpathlineto{\pgfqpoint{1.052617in}{1.611015in}}%
\pgfpathlineto{\pgfqpoint{1.059150in}{1.468881in}}%
\pgfpathlineto{\pgfqpoint{1.068540in}{1.337681in}}%
\pgfpathlineto{\pgfqpoint{1.071031in}{1.403281in}}%
\pgfpathlineto{\pgfqpoint{1.076093in}{1.272080in}}%
\pgfpathlineto{\pgfqpoint{1.081483in}{1.195547in}}%
\pgfpathlineto{\pgfqpoint{1.089322in}{1.129947in}}%
\pgfpathlineto{\pgfqpoint{1.091853in}{1.195547in}}%
\pgfpathlineto{\pgfqpoint{1.097161in}{1.064346in}}%
\pgfpathlineto{\pgfqpoint{1.099611in}{1.129947in}}%
\pgfpathlineto{\pgfqpoint{1.102060in}{1.064346in}}%
\pgfpathlineto{\pgfqpoint{1.104592in}{1.129947in}}%
\pgfpathlineto{\pgfqpoint{1.107041in}{1.064346in}}%
\pgfpathlineto{\pgfqpoint{1.113206in}{0.998746in}}%
\pgfpathlineto{\pgfqpoint{1.115656in}{1.064346in}}%
\pgfpathlineto{\pgfqpoint{1.118147in}{0.998746in}}%
\pgfpathlineto{\pgfqpoint{1.120719in}{1.064346in}}%
\pgfpathlineto{\pgfqpoint{1.123169in}{0.998746in}}%
\pgfpathlineto{\pgfqpoint{1.127864in}{0.922212in}}%
\pgfpathlineto{\pgfqpoint{1.130354in}{0.998746in}}%
\pgfpathlineto{\pgfqpoint{1.132804in}{0.922212in}}%
\pgfpathlineto{\pgfqpoint{1.140112in}{0.856612in}}%
\pgfpathlineto{\pgfqpoint{1.142562in}{0.922212in}}%
\pgfpathlineto{\pgfqpoint{1.145012in}{0.856612in}}%
\pgfpathlineto{\pgfqpoint{1.148197in}{0.922212in}}%
\pgfpathlineto{\pgfqpoint{1.153259in}{0.791012in}}%
\pgfpathlineto{\pgfqpoint{1.155709in}{0.856612in}}%
\pgfpathlineto{\pgfqpoint{1.158200in}{0.791012in}}%
\pgfpathlineto{\pgfqpoint{1.160649in}{0.856612in}}%
\pgfpathlineto{\pgfqpoint{1.163099in}{0.791012in}}%
\pgfpathlineto{\pgfqpoint{1.173674in}{0.856612in}}%
\pgfpathlineto{\pgfqpoint{1.176123in}{0.791012in}}%
\pgfpathlineto{\pgfqpoint{1.178614in}{0.856612in}}%
\pgfpathlineto{\pgfqpoint{1.181063in}{0.791012in}}%
\pgfpathlineto{\pgfqpoint{1.183799in}{0.856612in}}%
\pgfpathlineto{\pgfqpoint{1.186249in}{0.791012in}}%
\pgfpathlineto{\pgfqpoint{1.204581in}{0.856612in}}%
\pgfpathlineto{\pgfqpoint{1.209194in}{0.922212in}}%
\pgfpathlineto{\pgfqpoint{1.226342in}{1.272080in}}%
\pgfpathlineto{\pgfqpoint{1.252473in}{1.676615in}}%
\pgfpathlineto{\pgfqpoint{1.276031in}{1.949950in}}%
\pgfpathlineto{\pgfqpoint{1.278603in}{1.884349in}}%
\pgfpathlineto{\pgfqpoint{1.283584in}{2.015550in}}%
\pgfpathlineto{\pgfqpoint{1.286850in}{1.949950in}}%
\pgfpathlineto{\pgfqpoint{1.292117in}{2.092084in}}%
\pgfpathlineto{\pgfqpoint{1.294812in}{2.015550in}}%
\pgfpathlineto{\pgfqpoint{1.297261in}{2.092084in}}%
\pgfpathlineto{\pgfqpoint{1.300650in}{2.157684in}}%
\pgfpathlineto{\pgfqpoint{1.303100in}{2.092084in}}%
\pgfpathlineto{\pgfqpoint{1.305550in}{2.157684in}}%
\pgfpathlineto{\pgfqpoint{1.310531in}{2.223284in}}%
\pgfpathlineto{\pgfqpoint{1.313021in}{2.157684in}}%
\pgfpathlineto{\pgfqpoint{1.315471in}{2.223284in}}%
\pgfpathlineto{\pgfqpoint{1.320901in}{2.288884in}}%
\pgfpathlineto{\pgfqpoint{1.323351in}{2.223284in}}%
\pgfpathlineto{\pgfqpoint{1.325801in}{2.288884in}}%
\pgfpathlineto{\pgfqpoint{1.334456in}{2.365418in}}%
\pgfpathlineto{\pgfqpoint{1.336947in}{2.288884in}}%
\pgfpathlineto{\pgfqpoint{1.339396in}{2.365418in}}%
\pgfpathlineto{\pgfqpoint{1.346419in}{2.431018in}}%
\pgfpathlineto{\pgfqpoint{1.348869in}{2.365418in}}%
\pgfpathlineto{\pgfqpoint{1.351318in}{2.431018in}}%
\pgfpathlineto{\pgfqpoint{1.354176in}{2.365418in}}%
\pgfpathlineto{\pgfqpoint{1.356626in}{2.431018in}}%
\pgfpathlineto{\pgfqpoint{1.361403in}{2.496619in}}%
\pgfpathlineto{\pgfqpoint{1.363853in}{2.431018in}}%
\pgfpathlineto{\pgfqpoint{1.366302in}{2.496619in}}%
\pgfpathlineto{\pgfqpoint{1.370957in}{2.431018in}}%
\pgfpathlineto{\pgfqpoint{1.373407in}{2.496619in}}%
\pgfpathlineto{\pgfqpoint{1.380511in}{2.562219in}}%
\pgfpathlineto{\pgfqpoint{1.382960in}{2.496619in}}%
\pgfpathlineto{\pgfqpoint{1.385410in}{2.562219in}}%
\pgfpathlineto{\pgfqpoint{1.388186in}{2.496619in}}%
\pgfpathlineto{\pgfqpoint{1.390636in}{2.562219in}}%
\pgfpathlineto{\pgfqpoint{1.399455in}{2.638752in}}%
\pgfpathlineto{\pgfqpoint{1.401905in}{2.562219in}}%
\pgfpathlineto{\pgfqpoint{1.404355in}{2.638752in}}%
\pgfpathlineto{\pgfqpoint{1.406845in}{2.562219in}}%
\pgfpathlineto{\pgfqpoint{1.409295in}{2.638752in}}%
\pgfpathlineto{\pgfqpoint{1.421258in}{2.704353in}}%
\pgfpathlineto{\pgfqpoint{1.423707in}{2.638752in}}%
\pgfpathlineto{\pgfqpoint{1.426198in}{2.704353in}}%
\pgfpathlineto{\pgfqpoint{1.428648in}{2.638752in}}%
\pgfpathlineto{\pgfqpoint{1.431097in}{2.704353in}}%
\pgfpathlineto{\pgfqpoint{1.434078in}{2.638752in}}%
\pgfpathlineto{\pgfqpoint{1.436527in}{2.704353in}}%
\pgfpathlineto{\pgfqpoint{1.444203in}{2.769953in}}%
\pgfpathlineto{\pgfqpoint{1.446653in}{2.704353in}}%
\pgfpathlineto{\pgfqpoint{1.449143in}{2.769953in}}%
\pgfpathlineto{\pgfqpoint{1.451593in}{2.704353in}}%
\pgfpathlineto{\pgfqpoint{1.454043in}{2.769953in}}%
\pgfpathlineto{\pgfqpoint{1.456493in}{2.704353in}}%
\pgfpathlineto{\pgfqpoint{1.458942in}{2.769953in}}%
\pgfpathlineto{\pgfqpoint{1.475886in}{2.835553in}}%
\pgfpathlineto{\pgfqpoint{1.478336in}{2.769953in}}%
\pgfpathlineto{\pgfqpoint{1.480908in}{2.835553in}}%
\pgfpathlineto{\pgfqpoint{1.483358in}{2.769953in}}%
\pgfpathlineto{\pgfqpoint{1.485889in}{2.835553in}}%
\pgfpathlineto{\pgfqpoint{1.488339in}{2.769953in}}%
\pgfpathlineto{\pgfqpoint{1.490788in}{2.835553in}}%
\pgfpathlineto{\pgfqpoint{1.496504in}{2.704353in}}%
\pgfpathlineto{\pgfqpoint{1.498954in}{2.638752in}}%
\pgfpathlineto{\pgfqpoint{1.501404in}{2.496619in}}%
\pgfpathlineto{\pgfqpoint{1.503854in}{2.431018in}}%
\pgfpathlineto{\pgfqpoint{1.506303in}{2.288884in}}%
\pgfpathlineto{\pgfqpoint{1.511203in}{2.157684in}}%
\pgfpathlineto{\pgfqpoint{1.513652in}{2.015550in}}%
\pgfpathlineto{\pgfqpoint{1.529821in}{1.611015in}}%
\pgfpathlineto{\pgfqpoint{1.538844in}{1.468881in}}%
\pgfpathlineto{\pgfqpoint{1.544355in}{1.403281in}}%
\pgfpathlineto{\pgfqpoint{1.546846in}{1.468881in}}%
\pgfpathlineto{\pgfqpoint{1.551745in}{1.337681in}}%
\pgfpathlineto{\pgfqpoint{1.558605in}{1.272080in}}%
\pgfpathlineto{\pgfqpoint{1.561054in}{1.337681in}}%
\pgfpathlineto{\pgfqpoint{1.566321in}{1.195547in}}%
\pgfpathlineto{\pgfqpoint{1.569098in}{1.272080in}}%
\pgfpathlineto{\pgfqpoint{1.571547in}{1.195547in}}%
\pgfpathlineto{\pgfqpoint{1.577672in}{1.129947in}}%
\pgfpathlineto{\pgfqpoint{1.580162in}{1.195547in}}%
\pgfpathlineto{\pgfqpoint{1.582612in}{1.129947in}}%
\pgfpathlineto{\pgfqpoint{1.586246in}{1.064346in}}%
\pgfpathlineto{\pgfqpoint{1.591962in}{0.998746in}}%
\pgfpathlineto{\pgfqpoint{1.594411in}{1.064346in}}%
\pgfpathlineto{\pgfqpoint{1.596861in}{0.998746in}}%
\pgfpathlineto{\pgfqpoint{1.599351in}{1.064346in}}%
\pgfpathlineto{\pgfqpoint{1.601801in}{0.998746in}}%
\pgfpathlineto{\pgfqpoint{1.605802in}{0.922212in}}%
\pgfpathlineto{\pgfqpoint{1.608252in}{0.998746in}}%
\pgfpathlineto{\pgfqpoint{1.610702in}{0.922212in}}%
\pgfpathlineto{\pgfqpoint{1.620419in}{0.856612in}}%
\pgfpathlineto{\pgfqpoint{1.622869in}{0.922212in}}%
\pgfpathlineto{\pgfqpoint{1.625318in}{0.856612in}}%
\pgfpathlineto{\pgfqpoint{1.627809in}{0.922212in}}%
\pgfpathlineto{\pgfqpoint{1.630259in}{0.856612in}}%
\pgfpathlineto{\pgfqpoint{1.635607in}{0.791012in}}%
\pgfpathlineto{\pgfqpoint{1.638057in}{0.856612in}}%
\pgfpathlineto{\pgfqpoint{1.640507in}{0.791012in}}%
\pgfpathlineto{\pgfqpoint{1.643120in}{0.856612in}}%
\pgfpathlineto{\pgfqpoint{1.645569in}{0.791012in}}%
\pgfpathlineto{\pgfqpoint{1.648060in}{0.856612in}}%
\pgfpathlineto{\pgfqpoint{1.650510in}{0.791012in}}%
\pgfpathlineto{\pgfqpoint{1.653408in}{0.856612in}}%
\pgfpathlineto{\pgfqpoint{1.655858in}{0.791012in}}%
\pgfpathlineto{\pgfqpoint{1.674353in}{0.725412in}}%
\pgfpathlineto{\pgfqpoint{1.676803in}{0.791012in}}%
\pgfpathlineto{\pgfqpoint{1.679253in}{0.725412in}}%
\pgfpathlineto{\pgfqpoint{1.681703in}{0.791012in}}%
\pgfpathlineto{\pgfqpoint{1.684152in}{0.725412in}}%
\pgfpathlineto{\pgfqpoint{1.692277in}{0.922212in}}%
\pgfpathlineto{\pgfqpoint{1.713916in}{1.337681in}}%
\pgfpathlineto{\pgfqpoint{1.732861in}{1.611015in}}%
\pgfpathlineto{\pgfqpoint{1.748498in}{1.818749in}}%
\pgfpathlineto{\pgfqpoint{1.760012in}{1.949950in}}%
\pgfpathlineto{\pgfqpoint{1.762461in}{1.884349in}}%
\pgfpathlineto{\pgfqpoint{1.768177in}{2.015550in}}%
\pgfpathlineto{\pgfqpoint{1.771117in}{1.949950in}}%
\pgfpathlineto{\pgfqpoint{1.776261in}{2.092084in}}%
\pgfpathlineto{\pgfqpoint{1.778997in}{2.015550in}}%
\pgfpathlineto{\pgfqpoint{1.781447in}{2.092084in}}%
\pgfpathlineto{\pgfqpoint{1.785734in}{2.157684in}}%
\pgfpathlineto{\pgfqpoint{1.788224in}{2.092084in}}%
\pgfpathlineto{\pgfqpoint{1.793858in}{2.223284in}}%
\pgfpathlineto{\pgfqpoint{1.796349in}{2.157684in}}%
\pgfpathlineto{\pgfqpoint{1.798799in}{2.223284in}}%
\pgfpathlineto{\pgfqpoint{1.804229in}{2.288884in}}%
\pgfpathlineto{\pgfqpoint{1.806679in}{2.223284in}}%
\pgfpathlineto{\pgfqpoint{1.809128in}{2.288884in}}%
\pgfpathlineto{\pgfqpoint{1.818192in}{2.365418in}}%
\pgfpathlineto{\pgfqpoint{1.820642in}{2.288884in}}%
\pgfpathlineto{\pgfqpoint{1.823092in}{2.365418in}}%
\pgfpathlineto{\pgfqpoint{1.829706in}{2.431018in}}%
\pgfpathlineto{\pgfqpoint{1.832156in}{2.365418in}}%
\pgfpathlineto{\pgfqpoint{1.834646in}{2.431018in}}%
\pgfpathlineto{\pgfqpoint{1.837953in}{2.365418in}}%
\pgfpathlineto{\pgfqpoint{1.840403in}{2.431018in}}%
\pgfpathlineto{\pgfqpoint{1.847303in}{2.496619in}}%
\pgfpathlineto{\pgfqpoint{1.849753in}{2.431018in}}%
\pgfpathlineto{\pgfqpoint{1.852202in}{2.496619in}}%
\pgfpathlineto{\pgfqpoint{1.865594in}{2.562219in}}%
\pgfpathlineto{\pgfqpoint{1.868044in}{2.496619in}}%
\pgfpathlineto{\pgfqpoint{1.870494in}{2.562219in}}%
\pgfpathlineto{\pgfqpoint{1.872943in}{2.496619in}}%
\pgfpathlineto{\pgfqpoint{1.875393in}{2.562219in}}%
\pgfpathlineto{\pgfqpoint{1.884988in}{2.638752in}}%
\pgfpathlineto{\pgfqpoint{1.887437in}{2.562219in}}%
\pgfpathlineto{\pgfqpoint{1.889887in}{2.638752in}}%
\pgfpathlineto{\pgfqpoint{1.892418in}{2.562219in}}%
\pgfpathlineto{\pgfqpoint{1.894868in}{2.638752in}}%
\pgfpathlineto{\pgfqpoint{1.908137in}{2.704353in}}%
\pgfpathlineto{\pgfqpoint{1.910587in}{2.638752in}}%
\pgfpathlineto{\pgfqpoint{1.913037in}{2.704353in}}%
\pgfpathlineto{\pgfqpoint{1.915527in}{2.638752in}}%
\pgfpathlineto{\pgfqpoint{1.917977in}{2.704353in}}%
\pgfpathlineto{\pgfqpoint{1.932635in}{2.769953in}}%
\pgfpathlineto{\pgfqpoint{1.935084in}{2.704353in}}%
\pgfpathlineto{\pgfqpoint{1.937983in}{2.769953in}}%
\pgfpathlineto{\pgfqpoint{1.940433in}{2.704353in}}%
\pgfpathlineto{\pgfqpoint{1.942923in}{2.769953in}}%
\pgfpathlineto{\pgfqpoint{1.945414in}{2.704353in}}%
\pgfpathlineto{\pgfqpoint{1.947864in}{2.769953in}}%
\pgfpathlineto{\pgfqpoint{1.962807in}{2.835553in}}%
\pgfpathlineto{\pgfqpoint{1.965257in}{2.769953in}}%
\pgfpathlineto{\pgfqpoint{1.968115in}{2.835553in}}%
\pgfpathlineto{\pgfqpoint{1.970564in}{2.769953in}}%
\pgfpathlineto{\pgfqpoint{1.973014in}{2.835553in}}%
\pgfpathlineto{\pgfqpoint{1.975504in}{2.769953in}}%
\pgfpathlineto{\pgfqpoint{1.977954in}{2.835553in}}%
\pgfpathlineto{\pgfqpoint{1.983221in}{2.704353in}}%
\pgfpathlineto{\pgfqpoint{1.985671in}{2.562219in}}%
\pgfpathlineto{\pgfqpoint{1.988120in}{2.496619in}}%
\pgfpathlineto{\pgfqpoint{1.993020in}{2.288884in}}%
\pgfpathlineto{\pgfqpoint{1.995470in}{2.223284in}}%
\pgfpathlineto{\pgfqpoint{2.000369in}{2.015550in}}%
\pgfpathlineto{\pgfqpoint{2.015884in}{1.611015in}}%
\pgfpathlineto{\pgfqpoint{2.024499in}{1.468881in}}%
\pgfpathlineto{\pgfqpoint{2.026989in}{1.545415in}}%
\pgfpathlineto{\pgfqpoint{2.031889in}{1.403281in}}%
\pgfpathlineto{\pgfqpoint{2.035604in}{1.337681in}}%
\pgfpathlineto{\pgfqpoint{2.038095in}{1.403281in}}%
\pgfpathlineto{\pgfqpoint{2.042994in}{1.272080in}}%
\pgfpathlineto{\pgfqpoint{2.047934in}{1.195547in}}%
\pgfpathlineto{\pgfqpoint{2.050425in}{1.272080in}}%
\pgfpathlineto{\pgfqpoint{2.055692in}{1.129947in}}%
\pgfpathlineto{\pgfqpoint{2.067491in}{0.998746in}}%
\pgfpathlineto{\pgfqpoint{2.069941in}{1.064346in}}%
\pgfpathlineto{\pgfqpoint{2.072391in}{0.998746in}}%
\pgfpathlineto{\pgfqpoint{2.081128in}{0.922212in}}%
\pgfpathlineto{\pgfqpoint{2.083578in}{0.998746in}}%
\pgfpathlineto{\pgfqpoint{2.086027in}{0.922212in}}%
\pgfpathlineto{\pgfqpoint{2.088599in}{0.998746in}}%
\pgfpathlineto{\pgfqpoint{2.091049in}{0.922212in}}%
\pgfpathlineto{\pgfqpoint{2.098807in}{0.856612in}}%
\pgfpathlineto{\pgfqpoint{2.101256in}{0.922212in}}%
\pgfpathlineto{\pgfqpoint{2.103706in}{0.856612in}}%
\pgfpathlineto{\pgfqpoint{2.106156in}{0.922212in}}%
\pgfpathlineto{\pgfqpoint{2.108605in}{0.856612in}}%
\pgfpathlineto{\pgfqpoint{2.111586in}{0.922212in}}%
\pgfpathlineto{\pgfqpoint{2.114036in}{0.856612in}}%
\pgfpathlineto{\pgfqpoint{2.119017in}{0.791012in}}%
\pgfpathlineto{\pgfqpoint{2.121466in}{0.856612in}}%
\pgfpathlineto{\pgfqpoint{2.123916in}{0.791012in}}%
\pgfpathlineto{\pgfqpoint{2.126447in}{0.856612in}}%
\pgfpathlineto{\pgfqpoint{2.128897in}{0.791012in}}%
\pgfpathlineto{\pgfqpoint{2.143065in}{0.725412in}}%
\pgfpathlineto{\pgfqpoint{2.145514in}{0.791012in}}%
\pgfpathlineto{\pgfqpoint{2.148005in}{0.725412in}}%
\pgfpathlineto{\pgfqpoint{2.150495in}{0.791012in}}%
\pgfpathlineto{\pgfqpoint{2.152945in}{0.725412in}}%
\pgfpathlineto{\pgfqpoint{2.158253in}{0.648878in}}%
\pgfpathlineto{\pgfqpoint{2.160703in}{0.725412in}}%
\pgfpathlineto{\pgfqpoint{2.163152in}{0.648878in}}%
\pgfpathlineto{\pgfqpoint{2.165602in}{0.725412in}}%
\pgfpathlineto{\pgfqpoint{2.168052in}{0.648878in}}%
\pgfpathlineto{\pgfqpoint{2.177973in}{0.725412in}}%
\pgfpathlineto{\pgfqpoint{2.201858in}{1.195547in}}%
\pgfpathlineto{\pgfqpoint{2.214310in}{1.403281in}}%
\pgfpathlineto{\pgfqpoint{2.249954in}{1.884349in}}%
\pgfpathlineto{\pgfqpoint{2.263917in}{2.015550in}}%
\pgfpathlineto{\pgfqpoint{2.266448in}{1.949950in}}%
\pgfpathlineto{\pgfqpoint{2.271756in}{2.092084in}}%
\pgfpathlineto{\pgfqpoint{2.274287in}{2.015550in}}%
\pgfpathlineto{\pgfqpoint{2.276737in}{2.092084in}}%
\pgfpathlineto{\pgfqpoint{2.280657in}{2.157684in}}%
\pgfpathlineto{\pgfqpoint{2.283229in}{2.092084in}}%
\pgfpathlineto{\pgfqpoint{2.288577in}{2.223284in}}%
\pgfpathlineto{\pgfqpoint{2.291027in}{2.157684in}}%
\pgfpathlineto{\pgfqpoint{2.293477in}{2.223284in}}%
\pgfpathlineto{\pgfqpoint{2.298948in}{2.288884in}}%
\pgfpathlineto{\pgfqpoint{2.301398in}{2.223284in}}%
\pgfpathlineto{\pgfqpoint{2.303847in}{2.288884in}}%
\pgfpathlineto{\pgfqpoint{2.310870in}{2.365418in}}%
\pgfpathlineto{\pgfqpoint{2.313319in}{2.288884in}}%
\pgfpathlineto{\pgfqpoint{2.315769in}{2.365418in}}%
\pgfpathlineto{\pgfqpoint{2.323567in}{2.431018in}}%
\pgfpathlineto{\pgfqpoint{2.326017in}{2.365418in}}%
\pgfpathlineto{\pgfqpoint{2.328467in}{2.431018in}}%
\pgfpathlineto{\pgfqpoint{2.338674in}{2.496619in}}%
\pgfpathlineto{\pgfqpoint{2.341124in}{2.431018in}}%
\pgfpathlineto{\pgfqpoint{2.343573in}{2.496619in}}%
\pgfpathlineto{\pgfqpoint{2.346023in}{2.431018in}}%
\pgfpathlineto{\pgfqpoint{2.348473in}{2.496619in}}%
\pgfpathlineto{\pgfqpoint{2.357700in}{2.562219in}}%
\pgfpathlineto{\pgfqpoint{2.360150in}{2.496619in}}%
\pgfpathlineto{\pgfqpoint{2.362640in}{2.562219in}}%
\pgfpathlineto{\pgfqpoint{2.365376in}{2.496619in}}%
\pgfpathlineto{\pgfqpoint{2.367826in}{2.562219in}}%
\pgfpathlineto{\pgfqpoint{2.378359in}{2.638752in}}%
\pgfpathlineto{\pgfqpoint{2.380809in}{2.562219in}}%
\pgfpathlineto{\pgfqpoint{2.383259in}{2.638752in}}%
\pgfpathlineto{\pgfqpoint{2.385708in}{2.562219in}}%
\pgfpathlineto{\pgfqpoint{2.388158in}{2.638752in}}%
\pgfpathlineto{\pgfqpoint{2.391465in}{2.562219in}}%
\pgfpathlineto{\pgfqpoint{2.393915in}{2.638752in}}%
\pgfpathlineto{\pgfqpoint{2.404326in}{2.704353in}}%
\pgfpathlineto{\pgfqpoint{2.406776in}{2.638752in}}%
\pgfpathlineto{\pgfqpoint{2.409226in}{2.704353in}}%
\pgfpathlineto{\pgfqpoint{2.411757in}{2.638752in}}%
\pgfpathlineto{\pgfqpoint{2.414207in}{2.704353in}}%
\pgfpathlineto{\pgfqpoint{2.417187in}{2.638752in}}%
\pgfpathlineto{\pgfqpoint{2.419637in}{2.704353in}}%
\pgfpathlineto{\pgfqpoint{2.430293in}{2.769953in}}%
\pgfpathlineto{\pgfqpoint{2.432743in}{2.704353in}}%
\pgfpathlineto{\pgfqpoint{2.435233in}{2.769953in}}%
\pgfpathlineto{\pgfqpoint{2.437683in}{2.704353in}}%
\pgfpathlineto{\pgfqpoint{2.440174in}{2.769953in}}%
\pgfpathlineto{\pgfqpoint{2.442705in}{2.704353in}}%
\pgfpathlineto{\pgfqpoint{2.445155in}{2.769953in}}%
\pgfpathlineto{\pgfqpoint{2.460506in}{2.704353in}}%
\pgfpathlineto{\pgfqpoint{2.462956in}{2.638752in}}%
\pgfpathlineto{\pgfqpoint{2.465406in}{2.496619in}}%
\pgfpathlineto{\pgfqpoint{2.467855in}{2.431018in}}%
\pgfpathlineto{\pgfqpoint{2.470305in}{2.288884in}}%
\pgfpathlineto{\pgfqpoint{2.472755in}{2.223284in}}%
\pgfpathlineto{\pgfqpoint{2.477654in}{2.015550in}}%
\pgfpathlineto{\pgfqpoint{2.492720in}{1.611015in}}%
\pgfpathlineto{\pgfqpoint{2.514971in}{1.272080in}}%
\pgfpathlineto{\pgfqpoint{2.517707in}{1.337681in}}%
\pgfpathlineto{\pgfqpoint{2.522974in}{1.195547in}}%
\pgfpathlineto{\pgfqpoint{2.525832in}{1.272080in}}%
\pgfpathlineto{\pgfqpoint{2.528282in}{1.195547in}}%
\pgfpathlineto{\pgfqpoint{2.533548in}{1.129947in}}%
\pgfpathlineto{\pgfqpoint{2.535998in}{1.195547in}}%
\pgfpathlineto{\pgfqpoint{2.538448in}{1.129947in}}%
\pgfpathlineto{\pgfqpoint{2.542817in}{1.064346in}}%
\pgfpathlineto{\pgfqpoint{2.545307in}{1.129947in}}%
\pgfpathlineto{\pgfqpoint{2.547757in}{1.064346in}}%
\pgfpathlineto{\pgfqpoint{2.552901in}{0.998746in}}%
\pgfpathlineto{\pgfqpoint{2.555392in}{1.064346in}}%
\pgfpathlineto{\pgfqpoint{2.557841in}{0.998746in}}%
\pgfpathlineto{\pgfqpoint{2.564619in}{0.922212in}}%
\pgfpathlineto{\pgfqpoint{2.567069in}{0.998746in}}%
\pgfpathlineto{\pgfqpoint{2.569518in}{0.922212in}}%
\pgfpathlineto{\pgfqpoint{2.572009in}{0.998746in}}%
\pgfpathlineto{\pgfqpoint{2.574459in}{0.922212in}}%
\pgfpathlineto{\pgfqpoint{2.578296in}{0.856612in}}%
\pgfpathlineto{\pgfqpoint{2.580828in}{0.922212in}}%
\pgfpathlineto{\pgfqpoint{2.583278in}{0.856612in}}%
\pgfpathlineto{\pgfqpoint{2.590953in}{0.791012in}}%
\pgfpathlineto{\pgfqpoint{2.593403in}{0.856612in}}%
\pgfpathlineto{\pgfqpoint{2.596506in}{0.791012in}}%
\pgfpathlineto{\pgfqpoint{2.598956in}{0.856612in}}%
\pgfpathlineto{\pgfqpoint{2.601405in}{0.791012in}}%
\pgfpathlineto{\pgfqpoint{2.609163in}{0.725412in}}%
\pgfpathlineto{\pgfqpoint{2.611613in}{0.791012in}}%
\pgfpathlineto{\pgfqpoint{2.614144in}{0.725412in}}%
\pgfpathlineto{\pgfqpoint{2.616594in}{0.791012in}}%
\pgfpathlineto{\pgfqpoint{2.619084in}{0.725412in}}%
\pgfpathlineto{\pgfqpoint{2.621534in}{0.791012in}}%
\pgfpathlineto{\pgfqpoint{2.623984in}{0.725412in}}%
\pgfpathlineto{\pgfqpoint{2.626433in}{0.791012in}}%
\pgfpathlineto{\pgfqpoint{2.628883in}{0.725412in}}%
\pgfpathlineto{\pgfqpoint{2.639988in}{0.648878in}}%
\pgfpathlineto{\pgfqpoint{2.642438in}{0.725412in}}%
\pgfpathlineto{\pgfqpoint{2.644888in}{0.648878in}}%
\pgfpathlineto{\pgfqpoint{2.647419in}{0.725412in}}%
\pgfpathlineto{\pgfqpoint{2.649869in}{0.648878in}}%
\pgfpathlineto{\pgfqpoint{2.652972in}{0.725412in}}%
\pgfpathlineto{\pgfqpoint{2.660893in}{0.856612in}}%
\pgfpathlineto{\pgfqpoint{2.664649in}{0.922212in}}%
\pgfpathlineto{\pgfqpoint{2.671222in}{1.064346in}}%
\pgfpathlineto{\pgfqpoint{2.691146in}{1.403281in}}%
\pgfpathlineto{\pgfqpoint{2.709928in}{1.676615in}}%
\pgfpathlineto{\pgfqpoint{2.712704in}{1.611015in}}%
\pgfpathlineto{\pgfqpoint{2.717603in}{1.742216in}}%
\pgfpathlineto{\pgfqpoint{2.722054in}{1.818749in}}%
\pgfpathlineto{\pgfqpoint{2.727565in}{1.884349in}}%
\pgfpathlineto{\pgfqpoint{2.730097in}{1.818749in}}%
\pgfpathlineto{\pgfqpoint{2.735078in}{1.949950in}}%
\pgfpathlineto{\pgfqpoint{2.742264in}{2.015550in}}%
\pgfpathlineto{\pgfqpoint{2.744795in}{1.949950in}}%
\pgfpathlineto{\pgfqpoint{2.750307in}{2.092084in}}%
\pgfpathlineto{\pgfqpoint{2.752879in}{2.015550in}}%
\pgfpathlineto{\pgfqpoint{2.755329in}{2.092084in}}%
\pgfpathlineto{\pgfqpoint{2.759208in}{2.157684in}}%
\pgfpathlineto{\pgfqpoint{2.761739in}{2.092084in}}%
\pgfpathlineto{\pgfqpoint{2.764189in}{2.157684in}}%
\pgfpathlineto{\pgfqpoint{2.767945in}{2.223284in}}%
\pgfpathlineto{\pgfqpoint{2.770395in}{2.157684in}}%
\pgfpathlineto{\pgfqpoint{2.772844in}{2.223284in}}%
\pgfpathlineto{\pgfqpoint{2.777948in}{2.288884in}}%
\pgfpathlineto{\pgfqpoint{2.780398in}{2.223284in}}%
\pgfpathlineto{\pgfqpoint{2.782847in}{2.288884in}}%
\pgfpathlineto{\pgfqpoint{2.790646in}{2.365418in}}%
\pgfpathlineto{\pgfqpoint{2.793095in}{2.288884in}}%
\pgfpathlineto{\pgfqpoint{2.795545in}{2.365418in}}%
\pgfpathlineto{\pgfqpoint{2.801833in}{2.431018in}}%
\pgfpathlineto{\pgfqpoint{2.804282in}{2.365418in}}%
\pgfpathlineto{\pgfqpoint{2.806732in}{2.431018in}}%
\pgfpathlineto{\pgfqpoint{2.809263in}{2.365418in}}%
\pgfpathlineto{\pgfqpoint{2.811713in}{2.431018in}}%
\pgfpathlineto{\pgfqpoint{2.820369in}{2.496619in}}%
\pgfpathlineto{\pgfqpoint{2.822818in}{2.431018in}}%
\pgfpathlineto{\pgfqpoint{2.825268in}{2.496619in}}%
\pgfpathlineto{\pgfqpoint{2.837639in}{2.562219in}}%
\pgfpathlineto{\pgfqpoint{2.840089in}{2.496619in}}%
\pgfpathlineto{\pgfqpoint{2.842539in}{2.562219in}}%
\pgfpathlineto{\pgfqpoint{2.845111in}{2.496619in}}%
\pgfpathlineto{\pgfqpoint{2.847560in}{2.562219in}}%
\pgfpathlineto{\pgfqpoint{2.857073in}{2.638752in}}%
\pgfpathlineto{\pgfqpoint{2.859523in}{2.562219in}}%
\pgfpathlineto{\pgfqpoint{2.862014in}{2.638752in}}%
\pgfpathlineto{\pgfqpoint{2.864463in}{2.562219in}}%
\pgfpathlineto{\pgfqpoint{2.866913in}{2.638752in}}%
\pgfpathlineto{\pgfqpoint{2.883694in}{2.704353in}}%
\pgfpathlineto{\pgfqpoint{2.886143in}{2.638752in}}%
\pgfpathlineto{\pgfqpoint{2.888675in}{2.704353in}}%
\pgfpathlineto{\pgfqpoint{2.891124in}{2.638752in}}%
\pgfpathlineto{\pgfqpoint{2.893574in}{2.704353in}}%
\pgfpathlineto{\pgfqpoint{2.896187in}{2.638752in}}%
\pgfpathlineto{\pgfqpoint{2.898637in}{2.704353in}}%
\pgfpathlineto{\pgfqpoint{2.909497in}{2.769953in}}%
\pgfpathlineto{\pgfqpoint{2.911947in}{2.704353in}}%
\pgfpathlineto{\pgfqpoint{2.914478in}{2.769953in}}%
\pgfpathlineto{\pgfqpoint{2.916928in}{2.704353in}}%
\pgfpathlineto{\pgfqpoint{2.919378in}{2.769953in}}%
\pgfpathlineto{\pgfqpoint{2.921868in}{2.704353in}}%
\pgfpathlineto{\pgfqpoint{2.924318in}{2.769953in}}%
\pgfpathlineto{\pgfqpoint{2.927870in}{2.704353in}}%
\pgfpathlineto{\pgfqpoint{2.930320in}{2.769953in}}%
\pgfpathlineto{\pgfqpoint{2.941139in}{2.835553in}}%
\pgfpathlineto{\pgfqpoint{2.943589in}{2.769953in}}%
\pgfpathlineto{\pgfqpoint{2.946039in}{2.835553in}}%
\pgfpathlineto{\pgfqpoint{2.958287in}{2.223284in}}%
\pgfpathlineto{\pgfqpoint{2.960737in}{2.157684in}}%
\pgfpathlineto{\pgfqpoint{2.963187in}{2.015550in}}%
\pgfpathlineto{\pgfqpoint{2.980702in}{1.545415in}}%
\pgfpathlineto{\pgfqpoint{2.989521in}{1.403281in}}%
\pgfpathlineto{\pgfqpoint{2.994380in}{1.337681in}}%
\pgfpathlineto{\pgfqpoint{2.996952in}{1.403281in}}%
\pgfpathlineto{\pgfqpoint{3.001851in}{1.272080in}}%
\pgfpathlineto{\pgfqpoint{3.007077in}{1.195547in}}%
\pgfpathlineto{\pgfqpoint{3.009568in}{1.272080in}}%
\pgfpathlineto{\pgfqpoint{3.012018in}{1.195547in}}%
\pgfpathlineto{\pgfqpoint{3.015529in}{1.129947in}}%
\pgfpathlineto{\pgfqpoint{3.017979in}{1.195547in}}%
\pgfpathlineto{\pgfqpoint{3.020428in}{1.129947in}}%
\pgfpathlineto{\pgfqpoint{3.025450in}{1.064346in}}%
\pgfpathlineto{\pgfqpoint{3.027900in}{1.129947in}}%
\pgfpathlineto{\pgfqpoint{3.030350in}{1.064346in}}%
\pgfpathlineto{\pgfqpoint{3.032799in}{1.129947in}}%
\pgfpathlineto{\pgfqpoint{3.035249in}{1.064346in}}%
\pgfpathlineto{\pgfqpoint{3.045905in}{0.998746in}}%
\pgfpathlineto{\pgfqpoint{3.048437in}{1.064346in}}%
\pgfpathlineto{\pgfqpoint{3.050886in}{0.998746in}}%
\pgfpathlineto{\pgfqpoint{3.057011in}{0.922212in}}%
\pgfpathlineto{\pgfqpoint{3.059460in}{0.998746in}}%
\pgfpathlineto{\pgfqpoint{3.061910in}{0.922212in}}%
\pgfpathlineto{\pgfqpoint{3.070362in}{0.856612in}}%
\pgfpathlineto{\pgfqpoint{3.072811in}{0.922212in}}%
\pgfpathlineto{\pgfqpoint{3.075261in}{0.856612in}}%
\pgfpathlineto{\pgfqpoint{3.077752in}{0.922212in}}%
\pgfpathlineto{\pgfqpoint{3.080201in}{0.856612in}}%
\pgfpathlineto{\pgfqpoint{3.095471in}{0.791012in}}%
\pgfpathlineto{\pgfqpoint{3.097921in}{0.856612in}}%
\pgfpathlineto{\pgfqpoint{3.100371in}{0.791012in}}%
\pgfpathlineto{\pgfqpoint{3.102902in}{0.856612in}}%
\pgfpathlineto{\pgfqpoint{3.105352in}{0.791012in}}%
\pgfpathlineto{\pgfqpoint{3.110496in}{0.856612in}}%
\pgfpathlineto{\pgfqpoint{3.112946in}{0.791012in}}%
\pgfpathlineto{\pgfqpoint{3.119519in}{0.725412in}}%
\pgfpathlineto{\pgfqpoint{3.121969in}{0.791012in}}%
\pgfpathlineto{\pgfqpoint{3.124541in}{0.725412in}}%
\pgfpathlineto{\pgfqpoint{3.127032in}{0.791012in}}%
\pgfpathlineto{\pgfqpoint{3.129481in}{0.725412in}}%
\pgfpathlineto{\pgfqpoint{3.131972in}{0.791012in}}%
\pgfpathlineto{\pgfqpoint{3.134422in}{0.725412in}}%
\pgfpathlineto{\pgfqpoint{3.136871in}{0.791012in}}%
\pgfpathlineto{\pgfqpoint{3.139321in}{0.725412in}}%
\pgfpathlineto{\pgfqpoint{3.154713in}{0.648878in}}%
\pgfpathlineto{\pgfqpoint{3.157163in}{0.725412in}}%
\pgfpathlineto{\pgfqpoint{3.159654in}{0.648878in}}%
\pgfpathlineto{\pgfqpoint{3.162103in}{0.725412in}}%
\pgfpathlineto{\pgfqpoint{3.192847in}{1.337681in}}%
\pgfpathlineto{\pgfqpoint{3.201993in}{1.468881in}}%
\pgfpathlineto{\pgfqpoint{3.206321in}{1.545415in}}%
\pgfpathlineto{\pgfqpoint{3.212118in}{1.611015in}}%
\pgfpathlineto{\pgfqpoint{3.215956in}{1.676615in}}%
\pgfpathlineto{\pgfqpoint{3.218487in}{1.611015in}}%
\pgfpathlineto{\pgfqpoint{3.223387in}{1.742216in}}%
\pgfpathlineto{\pgfqpoint{3.227592in}{1.818749in}}%
\pgfpathlineto{\pgfqpoint{3.230042in}{1.742216in}}%
\pgfpathlineto{\pgfqpoint{3.234982in}{1.884349in}}%
\pgfpathlineto{\pgfqpoint{3.248374in}{2.015550in}}%
\pgfpathlineto{\pgfqpoint{3.250905in}{1.949950in}}%
\pgfpathlineto{\pgfqpoint{3.255968in}{2.092084in}}%
\pgfpathlineto{\pgfqpoint{3.258418in}{2.015550in}}%
\pgfpathlineto{\pgfqpoint{3.263970in}{2.157684in}}%
\pgfpathlineto{\pgfqpoint{3.266461in}{2.092084in}}%
\pgfpathlineto{\pgfqpoint{3.268911in}{2.157684in}}%
\pgfpathlineto{\pgfqpoint{3.272667in}{2.223284in}}%
\pgfpathlineto{\pgfqpoint{3.275157in}{2.157684in}}%
\pgfpathlineto{\pgfqpoint{3.277607in}{2.223284in}}%
\pgfpathlineto{\pgfqpoint{3.282343in}{2.288884in}}%
\pgfpathlineto{\pgfqpoint{3.284793in}{2.223284in}}%
\pgfpathlineto{\pgfqpoint{3.287243in}{2.288884in}}%
\pgfpathlineto{\pgfqpoint{3.295490in}{2.365418in}}%
\pgfpathlineto{\pgfqpoint{3.297940in}{2.288884in}}%
\pgfpathlineto{\pgfqpoint{3.300389in}{2.365418in}}%
\pgfpathlineto{\pgfqpoint{3.307494in}{2.431018in}}%
\pgfpathlineto{\pgfqpoint{3.309943in}{2.365418in}}%
\pgfpathlineto{\pgfqpoint{3.312393in}{2.431018in}}%
\pgfpathlineto{\pgfqpoint{3.322723in}{2.496619in}}%
\pgfpathlineto{\pgfqpoint{3.325172in}{2.431018in}}%
\pgfpathlineto{\pgfqpoint{3.327622in}{2.496619in}}%
\pgfpathlineto{\pgfqpoint{3.330194in}{2.431018in}}%
\pgfpathlineto{\pgfqpoint{3.332644in}{2.496619in}}%
\pgfpathlineto{\pgfqpoint{3.340320in}{2.562219in}}%
\pgfpathlineto{\pgfqpoint{3.342769in}{2.496619in}}%
\pgfpathlineto{\pgfqpoint{3.345219in}{2.562219in}}%
\pgfpathlineto{\pgfqpoint{3.347669in}{2.496619in}}%
\pgfpathlineto{\pgfqpoint{3.350118in}{2.562219in}}%
\pgfpathlineto{\pgfqpoint{3.362326in}{2.638752in}}%
\pgfpathlineto{\pgfqpoint{3.364776in}{2.562219in}}%
\pgfpathlineto{\pgfqpoint{3.367226in}{2.638752in}}%
\pgfpathlineto{\pgfqpoint{3.369839in}{2.562219in}}%
\pgfpathlineto{\pgfqpoint{3.372288in}{2.638752in}}%
\pgfpathlineto{\pgfqpoint{3.383516in}{2.704353in}}%
\pgfpathlineto{\pgfqpoint{3.385966in}{2.638752in}}%
\pgfpathlineto{\pgfqpoint{3.388416in}{2.704353in}}%
\pgfpathlineto{\pgfqpoint{3.390906in}{2.638752in}}%
\pgfpathlineto{\pgfqpoint{3.393356in}{2.704353in}}%
\pgfpathlineto{\pgfqpoint{3.396377in}{2.638752in}}%
\pgfpathlineto{\pgfqpoint{3.398827in}{2.704353in}}%
\pgfpathlineto{\pgfqpoint{3.409483in}{2.769953in}}%
\pgfpathlineto{\pgfqpoint{3.411933in}{2.704353in}}%
\pgfpathlineto{\pgfqpoint{3.414464in}{2.769953in}}%
\pgfpathlineto{\pgfqpoint{3.416914in}{2.704353in}}%
\pgfpathlineto{\pgfqpoint{3.419364in}{2.769953in}}%
\pgfpathlineto{\pgfqpoint{3.421895in}{2.704353in}}%
\pgfpathlineto{\pgfqpoint{3.424345in}{2.769953in}}%
\pgfpathlineto{\pgfqpoint{3.434511in}{2.704353in}}%
\pgfpathlineto{\pgfqpoint{3.439410in}{2.562219in}}%
\pgfpathlineto{\pgfqpoint{3.446760in}{2.223284in}}%
\pgfpathlineto{\pgfqpoint{3.449209in}{2.157684in}}%
\pgfpathlineto{\pgfqpoint{3.451659in}{2.015550in}}%
\pgfpathlineto{\pgfqpoint{3.466806in}{1.611015in}}%
\pgfpathlineto{\pgfqpoint{3.486159in}{1.337681in}}%
\pgfpathlineto{\pgfqpoint{3.488609in}{1.403281in}}%
\pgfpathlineto{\pgfqpoint{3.493508in}{1.272080in}}%
\pgfpathlineto{\pgfqpoint{3.497999in}{1.195547in}}%
\pgfpathlineto{\pgfqpoint{3.500490in}{1.272080in}}%
\pgfpathlineto{\pgfqpoint{3.502940in}{1.195547in}}%
\pgfpathlineto{\pgfqpoint{3.506737in}{1.129947in}}%
\pgfpathlineto{\pgfqpoint{3.509227in}{1.195547in}}%
\pgfpathlineto{\pgfqpoint{3.514494in}{1.064346in}}%
\pgfpathlineto{\pgfqpoint{3.516944in}{1.129947in}}%
\pgfpathlineto{\pgfqpoint{3.522374in}{0.998746in}}%
\pgfpathlineto{\pgfqpoint{3.524824in}{1.064346in}}%
\pgfpathlineto{\pgfqpoint{3.527273in}{0.998746in}}%
\pgfpathlineto{\pgfqpoint{3.537031in}{0.922212in}}%
\pgfpathlineto{\pgfqpoint{3.539563in}{0.998746in}}%
\pgfpathlineto{\pgfqpoint{3.542012in}{0.922212in}}%
\pgfpathlineto{\pgfqpoint{3.548137in}{0.856612in}}%
\pgfpathlineto{\pgfqpoint{3.550586in}{0.922212in}}%
\pgfpathlineto{\pgfqpoint{3.553036in}{0.856612in}}%
\pgfpathlineto{\pgfqpoint{3.565611in}{0.791012in}}%
\pgfpathlineto{\pgfqpoint{3.568061in}{0.856612in}}%
\pgfpathlineto{\pgfqpoint{3.570511in}{0.791012in}}%
\pgfpathlineto{\pgfqpoint{3.579493in}{0.725412in}}%
\pgfpathlineto{\pgfqpoint{3.581943in}{0.791012in}}%
\pgfpathlineto{\pgfqpoint{3.584392in}{0.725412in}}%
\pgfpathlineto{\pgfqpoint{3.586924in}{0.791012in}}%
\pgfpathlineto{\pgfqpoint{3.589374in}{0.725412in}}%
\pgfpathlineto{\pgfqpoint{3.596396in}{0.648878in}}%
\pgfpathlineto{\pgfqpoint{3.598927in}{0.725412in}}%
\pgfpathlineto{\pgfqpoint{3.601377in}{0.648878in}}%
\pgfpathlineto{\pgfqpoint{3.603827in}{0.725412in}}%
\pgfpathlineto{\pgfqpoint{3.606277in}{0.648878in}}%
\pgfpathlineto{\pgfqpoint{3.621955in}{0.725412in}}%
\pgfpathlineto{\pgfqpoint{3.636612in}{0.998746in}}%
\pgfpathlineto{\pgfqpoint{3.651351in}{1.272080in}}%
\pgfpathlineto{\pgfqpoint{3.659272in}{1.403281in}}%
\pgfpathlineto{\pgfqpoint{3.664702in}{1.468881in}}%
\pgfpathlineto{\pgfqpoint{3.679523in}{1.676615in}}%
\pgfpathlineto{\pgfqpoint{3.685402in}{1.742216in}}%
\pgfpathlineto{\pgfqpoint{3.690832in}{1.818749in}}%
\pgfpathlineto{\pgfqpoint{3.693568in}{1.742216in}}%
\pgfpathlineto{\pgfqpoint{3.698467in}{1.884349in}}%
\pgfpathlineto{\pgfqpoint{3.703285in}{1.949950in}}%
\pgfpathlineto{\pgfqpoint{3.705816in}{1.884349in}}%
\pgfpathlineto{\pgfqpoint{3.711451in}{2.015550in}}%
\pgfpathlineto{\pgfqpoint{3.713982in}{1.949950in}}%
\pgfpathlineto{\pgfqpoint{3.718882in}{2.092084in}}%
\pgfpathlineto{\pgfqpoint{3.721372in}{2.015550in}}%
\pgfpathlineto{\pgfqpoint{3.726966in}{2.157684in}}%
\pgfpathlineto{\pgfqpoint{3.729415in}{2.092084in}}%
\pgfpathlineto{\pgfqpoint{3.731865in}{2.157684in}}%
\pgfpathlineto{\pgfqpoint{3.736315in}{2.223284in}}%
\pgfpathlineto{\pgfqpoint{3.738765in}{2.157684in}}%
\pgfpathlineto{\pgfqpoint{3.741215in}{2.223284in}}%
\pgfpathlineto{\pgfqpoint{3.746727in}{2.288884in}}%
\pgfpathlineto{\pgfqpoint{3.749176in}{2.223284in}}%
\pgfpathlineto{\pgfqpoint{3.751626in}{2.288884in}}%
\pgfpathlineto{\pgfqpoint{3.758281in}{2.365418in}}%
\pgfpathlineto{\pgfqpoint{3.760731in}{2.288884in}}%
\pgfpathlineto{\pgfqpoint{3.763180in}{2.365418in}}%
\pgfpathlineto{\pgfqpoint{3.770121in}{2.431018in}}%
\pgfpathlineto{\pgfqpoint{3.772571in}{2.365418in}}%
\pgfpathlineto{\pgfqpoint{3.775021in}{2.431018in}}%
\pgfpathlineto{\pgfqpoint{3.785391in}{2.496619in}}%
\pgfpathlineto{\pgfqpoint{3.787841in}{2.431018in}}%
\pgfpathlineto{\pgfqpoint{3.790291in}{2.496619in}}%
\pgfpathlineto{\pgfqpoint{3.792740in}{2.431018in}}%
\pgfpathlineto{\pgfqpoint{3.795190in}{2.496619in}}%
\pgfpathlineto{\pgfqpoint{3.803723in}{2.562219in}}%
\pgfpathlineto{\pgfqpoint{3.806173in}{2.496619in}}%
\pgfpathlineto{\pgfqpoint{3.808663in}{2.562219in}}%
\pgfpathlineto{\pgfqpoint{3.811113in}{2.496619in}}%
\pgfpathlineto{\pgfqpoint{3.813563in}{2.562219in}}%
\pgfpathlineto{\pgfqpoint{3.824913in}{2.638752in}}%
\pgfpathlineto{\pgfqpoint{3.827363in}{2.562219in}}%
\pgfpathlineto{\pgfqpoint{3.829813in}{2.638752in}}%
\pgfpathlineto{\pgfqpoint{3.832303in}{2.562219in}}%
\pgfpathlineto{\pgfqpoint{3.834753in}{2.638752in}}%
\pgfpathlineto{\pgfqpoint{3.847450in}{2.704353in}}%
\pgfpathlineto{\pgfqpoint{3.849900in}{2.638752in}}%
\pgfpathlineto{\pgfqpoint{3.852391in}{2.704353in}}%
\pgfpathlineto{\pgfqpoint{3.854840in}{2.638752in}}%
\pgfpathlineto{\pgfqpoint{3.857331in}{2.704353in}}%
\pgfpathlineto{\pgfqpoint{3.861495in}{2.638752in}}%
\pgfpathlineto{\pgfqpoint{3.863945in}{2.704353in}}%
\pgfpathlineto{\pgfqpoint{3.873417in}{2.769953in}}%
\pgfpathlineto{\pgfqpoint{3.875867in}{2.704353in}}%
\pgfpathlineto{\pgfqpoint{3.878358in}{2.769953in}}%
\pgfpathlineto{\pgfqpoint{3.880807in}{2.704353in}}%
\pgfpathlineto{\pgfqpoint{3.883257in}{2.769953in}}%
\pgfpathlineto{\pgfqpoint{3.890606in}{2.431018in}}%
\pgfpathlineto{\pgfqpoint{3.895506in}{2.288884in}}%
\pgfpathlineto{\pgfqpoint{3.897955in}{2.157684in}}%
\pgfpathlineto{\pgfqpoint{3.907754in}{1.884349in}}%
\pgfpathlineto{\pgfqpoint{3.910204in}{1.742216in}}%
\pgfpathlineto{\pgfqpoint{3.923922in}{1.468881in}}%
\pgfpathlineto{\pgfqpoint{3.934252in}{1.337681in}}%
\pgfpathlineto{\pgfqpoint{3.940580in}{1.272080in}}%
\pgfpathlineto{\pgfqpoint{3.943275in}{1.337681in}}%
\pgfpathlineto{\pgfqpoint{3.948174in}{1.195547in}}%
\pgfpathlineto{\pgfqpoint{3.953482in}{1.129947in}}%
\pgfpathlineto{\pgfqpoint{3.956013in}{1.195547in}}%
\pgfpathlineto{\pgfqpoint{3.958463in}{1.129947in}}%
\pgfpathlineto{\pgfqpoint{3.962546in}{1.064346in}}%
\pgfpathlineto{\pgfqpoint{3.964996in}{1.129947in}}%
\pgfpathlineto{\pgfqpoint{3.970508in}{0.998746in}}%
\pgfpathlineto{\pgfqpoint{3.972957in}{1.064346in}}%
\pgfpathlineto{\pgfqpoint{3.975407in}{0.998746in}}%
\pgfpathlineto{\pgfqpoint{3.982348in}{0.922212in}}%
\pgfpathlineto{\pgfqpoint{3.984798in}{0.998746in}}%
\pgfpathlineto{\pgfqpoint{3.987247in}{0.922212in}}%
\pgfpathlineto{\pgfqpoint{3.989738in}{0.998746in}}%
\pgfpathlineto{\pgfqpoint{3.992188in}{0.922212in}}%
\pgfpathlineto{\pgfqpoint{3.999904in}{0.856612in}}%
\pgfpathlineto{\pgfqpoint{4.002354in}{0.922212in}}%
\pgfpathlineto{\pgfqpoint{4.005661in}{0.856612in}}%
\pgfpathlineto{\pgfqpoint{4.008111in}{0.922212in}}%
\pgfpathlineto{\pgfqpoint{4.010560in}{0.856612in}}%
\pgfpathlineto{\pgfqpoint{4.013786in}{0.922212in}}%
\pgfpathlineto{\pgfqpoint{4.016236in}{0.856612in}}%
\pgfpathlineto{\pgfqpoint{4.021135in}{0.791012in}}%
\pgfpathlineto{\pgfqpoint{4.023585in}{0.856612in}}%
\pgfpathlineto{\pgfqpoint{4.026034in}{0.791012in}}%
\pgfpathlineto{\pgfqpoint{4.028607in}{0.856612in}}%
\pgfpathlineto{\pgfqpoint{4.031056in}{0.791012in}}%
\pgfpathlineto{\pgfqpoint{4.041263in}{0.725412in}}%
\pgfpathlineto{\pgfqpoint{4.043713in}{0.791012in}}%
\pgfpathlineto{\pgfqpoint{4.046163in}{0.725412in}}%
\pgfpathlineto{\pgfqpoint{4.048613in}{0.791012in}}%
\pgfpathlineto{\pgfqpoint{4.051062in}{0.725412in}}%
\pgfpathlineto{\pgfqpoint{4.058003in}{0.648878in}}%
\pgfpathlineto{\pgfqpoint{4.060453in}{0.725412in}}%
\pgfpathlineto{\pgfqpoint{4.062902in}{0.648878in}}%
\pgfpathlineto{\pgfqpoint{4.065393in}{0.725412in}}%
\pgfpathlineto{\pgfqpoint{4.067843in}{0.648878in}}%
\pgfpathlineto{\pgfqpoint{4.070333in}{0.725412in}}%
\pgfpathlineto{\pgfqpoint{4.072783in}{0.648878in}}%
\pgfpathlineto{\pgfqpoint{4.077764in}{0.791012in}}%
\pgfpathlineto{\pgfqpoint{4.092503in}{1.064346in}}%
\pgfpathlineto{\pgfqpoint{4.107895in}{1.337681in}}%
\pgfpathlineto{\pgfqpoint{4.132597in}{1.676615in}}%
\pgfpathlineto{\pgfqpoint{4.135210in}{1.611015in}}%
\pgfpathlineto{\pgfqpoint{4.140109in}{1.742216in}}%
\pgfpathlineto{\pgfqpoint{4.145009in}{1.818749in}}%
\pgfpathlineto{\pgfqpoint{4.150684in}{1.884349in}}%
\pgfpathlineto{\pgfqpoint{4.165423in}{2.015550in}}%
\pgfpathlineto{\pgfqpoint{4.167913in}{1.949950in}}%
\pgfpathlineto{\pgfqpoint{4.174201in}{2.092084in}}%
\pgfpathlineto{\pgfqpoint{4.176651in}{2.015550in}}%
\pgfpathlineto{\pgfqpoint{4.179100in}{2.092084in}}%
\pgfpathlineto{\pgfqpoint{4.183387in}{2.157684in}}%
\pgfpathlineto{\pgfqpoint{4.185837in}{2.092084in}}%
\pgfpathlineto{\pgfqpoint{4.188287in}{2.157684in}}%
\pgfpathlineto{\pgfqpoint{4.193023in}{2.223284in}}%
\pgfpathlineto{\pgfqpoint{4.195473in}{2.157684in}}%
\pgfpathlineto{\pgfqpoint{4.197922in}{2.223284in}}%
\pgfpathlineto{\pgfqpoint{4.204537in}{2.288884in}}%
\pgfpathlineto{\pgfqpoint{4.206986in}{2.223284in}}%
\pgfpathlineto{\pgfqpoint{4.209436in}{2.288884in}}%
\pgfpathlineto{\pgfqpoint{4.217398in}{2.365418in}}%
\pgfpathlineto{\pgfqpoint{4.219888in}{2.288884in}}%
\pgfpathlineto{\pgfqpoint{4.222338in}{2.365418in}}%
\pgfpathlineto{\pgfqpoint{4.231075in}{2.431018in}}%
\pgfpathlineto{\pgfqpoint{4.233525in}{2.365418in}}%
\pgfpathlineto{\pgfqpoint{4.235975in}{2.431018in}}%
\pgfpathlineto{\pgfqpoint{4.248141in}{2.496619in}}%
\pgfpathlineto{\pgfqpoint{4.250591in}{2.431018in}}%
\pgfpathlineto{\pgfqpoint{4.253082in}{2.496619in}}%
\pgfpathlineto{\pgfqpoint{4.255654in}{2.431018in}}%
\pgfpathlineto{\pgfqpoint{4.258104in}{2.496619in}}%
\pgfpathlineto{\pgfqpoint{4.266228in}{2.562219in}}%
\pgfpathlineto{\pgfqpoint{4.268678in}{2.496619in}}%
\pgfpathlineto{\pgfqpoint{4.271291in}{2.562219in}}%
\pgfpathlineto{\pgfqpoint{4.273741in}{2.496619in}}%
\pgfpathlineto{\pgfqpoint{4.276191in}{2.562219in}}%
\pgfpathlineto{\pgfqpoint{4.289460in}{2.638752in}}%
\pgfpathlineto{\pgfqpoint{4.291910in}{2.562219in}}%
\pgfpathlineto{\pgfqpoint{4.294359in}{2.638752in}}%
\pgfpathlineto{\pgfqpoint{4.296850in}{2.562219in}}%
\pgfpathlineto{\pgfqpoint{4.299300in}{2.638752in}}%
\pgfpathlineto{\pgfqpoint{4.302443in}{2.562219in}}%
\pgfpathlineto{\pgfqpoint{4.304893in}{2.638752in}}%
\pgfpathlineto{\pgfqpoint{4.318244in}{2.704353in}}%
\pgfpathlineto{\pgfqpoint{4.320694in}{2.638752in}}%
\pgfpathlineto{\pgfqpoint{4.323184in}{2.704353in}}%
\pgfpathlineto{\pgfqpoint{4.325634in}{2.638752in}}%
\pgfpathlineto{\pgfqpoint{4.328084in}{2.704353in}}%
\pgfpathlineto{\pgfqpoint{4.331146in}{2.638752in}}%
\pgfpathlineto{\pgfqpoint{4.333595in}{2.704353in}}%
\pgfpathlineto{\pgfqpoint{4.333595in}{2.704353in}}%
\pgfusepath{stroke}%
\end{pgfscope}%
\begin{pgfscope}%
\pgfsetrectcap%
\pgfsetmiterjoin%
\pgfsetlinewidth{0.803000pt}%
\definecolor{currentstroke}{rgb}{0.000000,0.000000,0.000000}%
\pgfsetstrokecolor{currentstroke}%
\pgfsetdash{}{0pt}%
\pgfpathmoveto{\pgfqpoint{0.634869in}{0.539544in}}%
\pgfpathlineto{\pgfqpoint{0.634869in}{2.944887in}}%
\pgfusepath{stroke}%
\end{pgfscope}%
\begin{pgfscope}%
\pgfsetrectcap%
\pgfsetmiterjoin%
\pgfsetlinewidth{0.803000pt}%
\definecolor{currentstroke}{rgb}{0.000000,0.000000,0.000000}%
\pgfsetstrokecolor{currentstroke}%
\pgfsetdash{}{0pt}%
\pgfpathmoveto{\pgfqpoint{4.514985in}{0.539544in}}%
\pgfpathlineto{\pgfqpoint{4.514985in}{2.944887in}}%
\pgfusepath{stroke}%
\end{pgfscope}%
\begin{pgfscope}%
\pgfsetrectcap%
\pgfsetmiterjoin%
\pgfsetlinewidth{0.803000pt}%
\definecolor{currentstroke}{rgb}{0.000000,0.000000,0.000000}%
\pgfsetstrokecolor{currentstroke}%
\pgfsetdash{}{0pt}%
\pgfpathmoveto{\pgfqpoint{0.634869in}{0.539544in}}%
\pgfpathlineto{\pgfqpoint{4.514985in}{0.539544in}}%
\pgfusepath{stroke}%
\end{pgfscope}%
\begin{pgfscope}%
\pgfsetrectcap%
\pgfsetmiterjoin%
\pgfsetlinewidth{0.803000pt}%
\definecolor{currentstroke}{rgb}{0.000000,0.000000,0.000000}%
\pgfsetstrokecolor{currentstroke}%
\pgfsetdash{}{0pt}%
\pgfpathmoveto{\pgfqpoint{0.634869in}{2.944887in}}%
\pgfpathlineto{\pgfqpoint{4.514985in}{2.944887in}}%
\pgfusepath{stroke}%
\end{pgfscope}%
\begin{pgfscope}%
\pgfsetbuttcap%
\pgfsetmiterjoin%
\definecolor{currentfill}{rgb}{1.000000,1.000000,1.000000}%
\pgfsetfillcolor{currentfill}%
\pgfsetfillopacity{0.800000}%
\pgfsetlinewidth{1.003750pt}%
\definecolor{currentstroke}{rgb}{0.800000,0.800000,0.800000}%
\pgfsetstrokecolor{currentstroke}%
\pgfsetstrokeopacity{0.800000}%
\pgfsetdash{}{0pt}%
\pgfpathmoveto{\pgfqpoint{0.712647in}{2.544665in}}%
\pgfpathlineto{\pgfqpoint{2.212202in}{2.544665in}}%
\pgfpathquadraticcurveto{\pgfqpoint{2.234424in}{2.544665in}}{\pgfqpoint{2.234424in}{2.566887in}}%
\pgfpathlineto{\pgfqpoint{2.234424in}{2.867109in}}%
\pgfpathquadraticcurveto{\pgfqpoint{2.234424in}{2.889331in}}{\pgfqpoint{2.212202in}{2.889331in}}%
\pgfpathlineto{\pgfqpoint{0.712647in}{2.889331in}}%
\pgfpathquadraticcurveto{\pgfqpoint{0.690424in}{2.889331in}}{\pgfqpoint{0.690424in}{2.867109in}}%
\pgfpathlineto{\pgfqpoint{0.690424in}{2.566887in}}%
\pgfpathquadraticcurveto{\pgfqpoint{0.690424in}{2.544665in}}{\pgfqpoint{0.712647in}{2.544665in}}%
\pgfpathlineto{\pgfqpoint{0.712647in}{2.544665in}}%
\pgfpathclose%
\pgfusepath{stroke,fill}%
\end{pgfscope}%
\begin{pgfscope}%
\pgfsetrectcap%
\pgfsetroundjoin%
\pgfsetlinewidth{0.501875pt}%
\definecolor{currentstroke}{rgb}{0.121569,0.466667,0.705882}%
\pgfsetstrokecolor{currentstroke}%
\pgfsetstrokeopacity{0.700000}%
\pgfsetdash{}{0pt}%
\pgfpathmoveto{\pgfqpoint{0.734869in}{2.805998in}}%
\pgfpathlineto{\pgfqpoint{0.845980in}{2.805998in}}%
\pgfpathlineto{\pgfqpoint{0.957091in}{2.805998in}}%
\pgfusepath{stroke}%
\end{pgfscope}%
\begin{pgfscope}%
\definecolor{textcolor}{rgb}{0.000000,0.000000,0.000000}%
\pgfsetstrokecolor{textcolor}%
\pgfsetfillcolor{textcolor}%
\pgftext[x=1.045980in,y=2.767109in,left,base]{\color{textcolor}\rmfamily\fontsize{8.000000}{9.600000}\selectfont DUT vs KS34470A}%
\end{pgfscope}%
\begin{pgfscope}%
\pgfsetrectcap%
\pgfsetroundjoin%
\pgfsetlinewidth{0.501875pt}%
\definecolor{currentstroke}{rgb}{0.698039,0.133333,0.133333}%
\pgfsetstrokecolor{currentstroke}%
\pgfsetstrokeopacity{0.700000}%
\pgfsetdash{}{0pt}%
\pgfpathmoveto{\pgfqpoint{0.734869in}{2.649554in}}%
\pgfpathlineto{\pgfqpoint{0.845980in}{2.649554in}}%
\pgfpathlineto{\pgfqpoint{0.957091in}{2.649554in}}%
\pgfusepath{stroke}%
\end{pgfscope}%
\begin{pgfscope}%
\definecolor{textcolor}{rgb}{0.000000,0.000000,0.000000}%
\pgfsetstrokecolor{textcolor}%
\pgfsetfillcolor{textcolor}%
\pgftext[x=1.045980in,y=2.610665in,left,base]{\color{textcolor}\rmfamily\fontsize{8.000000}{9.600000}\selectfont Ambient Temperature}%
\end{pgfscope}%
\end{pgfpicture}%
\makeatother%
\endgroup%

    \caption{Voltage deviation from the mean voltage of an LM399 negative \qty{-10}{\volt} reference measured with a Keysight 34470A at \qty{100}{\plc}.}
    \label{fig:lm399_vs_34470a}
\end{figure}

Figure \ref{fig:lm399_vs_34470a} shows an example of such measurement. This measurement highlights one the problems encountered during those measurements. From this measurement it is unclear whether the features seen in the graph are only a result of ambient temperature changes due to the cycling of the air conditioning or popcorn noise on top of that. These results hightlight the fact, that sub-\unit{ppm} measurements not only requires high-end gear, but also a very stable environment. From the data it follows, that the temperature coefficient of the DMM in linear approximation is:
\begin{equation}
    \alpha_\device{34470A} \approx \frac{\qty{6.08}{\micro\volt}-(-\qty{9.30}{\micro\volt})}{(\qty{21.85}{\celsius}-\qty{19.96}{\celsius})\qty{10}{\volt}} = \qty[per-mode=symbol]{0.86}{\micro\volt \per \volt \per \kelvin}
\end{equation}

While the temperature coeeficient is vastly better than the specified \qty[per-mode=symbol]{2}{\micro\volt \per \volt \per \kelvin} \cite{datasheet_keysight34470A}, it is not low enough for this type of measurement. The multimeter must therefore be kept in a temperature controlled environment. This issue was resolved by replacing the stock air conditioning controller with a custom PID controller as discussed in section \ref{}. Lastly, the noise floor of the measurement is \qty{1.5}{\micro\volt_{rms}}, resulting in an estimated signal-to-noise ratio (SNR) of about \qty{10}{\decibel}, which is suffient to detect the popcorn noise.

While the temperature issue was being worked on, testing of the Zener diodes continued. To work around the temperature drift of the DMM, the amplification of the reference voltage was increased to \qty{15}{\volt}, the same voltage required by the digital current driver, and a differential measurement was realized. This measurement was done against a primary \qty{15}{\volt} reference board. To ensure, that any popcorn noise found originates only in the DUT and not the primary reference used, several reference boards where tested against a \device{Fluke 5440B}. The \device{5440B} does not exhibit popcorn noise as it uses a different voltage reference ic, namely two Motorola SZA263 in series \cite{service_manual_fluke_5440b}. Finally, a board that did not show popcorn noise in a period of three days was selected. The serial number of this primary or golden reference is \textit{\#1}.

Using this differential technique, the results greatly improved

In order to test a large amount of Zener diodes, and considering the duration of the burn-in process, which can take anything between \qtyrange{100}{1000}{\hour}, it is necessary to have an automated setup. This consists of a digital multimeter (DMM) a scanner and test board, that holds the Zener diodes and provides the necessary infrastructure for the diodes.



To conclude, we need a high performance DMM, a scanner, and a test fixture. The choices will detailed in the following sections.



\subsection{A Scanner System for Testing Zener Diodes}
As discussed before the diodes need to be tested for \qty{1000}{\hour} and it is not be feasible to test them individually. So a minimum of 10 diodes must be tested at the same time. To keep the system compact, the test setup a scanner to multiplex a single multimeter input. Several commercial options currently available were considered for this project and are shown in table \ref{tab:list_of_daqs}.

\begin{table}[h]
    \centering
    \small
    \begin{tabular}{ |l|l|l|l|l|l|l|l| }
        \hline
        \multirow{2}{*}{} & \multicolumn{2}{l|}{Keysight} & \multicolumn{3}{l|}{Keithley} & Fluke & Rigol \\
        \cline{2-8}
        & DAQ973A & 34980A & DAQ6510 & 2750 & 3706 & 2680 & M300 \\
        \hline
        DMM & \num{6.5} & \num{6.5} & \num{6.5} & \num{6.5} & \num{7.5} & \qty{18}{\bit} & \num{6.5} \\
        \hline
        Channels & 3x20 & 8x40 & 2x10 & 5x20 & 6x60 & 6x20 & 5x32 \\
        \hline
        FET & \textcolor{green!60!black}{\checkmark} & \textcolor{green!60!black}{\checkmark} & \textcolor{green!60!black}{\checkmark} & \textcolor{green!60!black}{\checkmark} & \textcolor{green!60!black}{\checkmark} & \textcolor{red!80!black}{\ding{55}} & \textcolor{red!80!black}{\ding{55}} \\
        \hline
        Voltage & \qty{120}{\volt} & \qty{80}{\volt} & \qty{60}{\volt} & \qty{60}{\volt} & \qty{200}{\volt} & \qty{75}{\volt} & \qty{300}{\volt} \\
        \hline
        Card & DAQM900A & 34925A & 7710 & 7710 & 3724 & 2680A-PAI & MC3132 \\
        \hline
        USB & \textcolor{green!60!black}{\checkmark} & \textcolor{green!60!black}{\checkmark} & \textcolor{green!60!black}{\checkmark} & \textcolor{red!80!black}{\ding{55}} & \textcolor{green!60!black}{\checkmark} & \textcolor{red!80!black}{\ding{55}} & \textcolor{green!60!black}{\checkmark} \\
        \hline
        Ethernet & \textcolor{green!60!black}{\checkmark} & \textcolor{green!60!black}{\checkmark} & \textcolor{green!60!black}{\checkmark} & \textcolor{red!80!black}{\ding{55}} & \textcolor{green!60!black}{\checkmark} & \textcolor{green!60!black}{\checkmark} & \textcolor{green!60!black}{\checkmark} \\
        \hline
        GPIB & \textcolor{green!60!black}{\checkmark} & \textcolor{green!60!black}{\checkmark} & \textcolor{green!60!black}{\checkmark} & \textcolor{green!60!black}{\checkmark} & \textcolor{green!60!black}{\checkmark} & \textcolor{red!80!black}{\ding{55}} & \textcolor{green!60!black}{\checkmark} \\
        %DMM & & & & & & & \\
        \hline
    \end{tabular}
    \caption{Overview of scanner mainframes}
    \label{tab:list_of_daqs}
\end{table}

A recent trend to more compact devices has led major manufacturers to include multimeters in the scanner mainframe creating so called data acquisition units. Legacy devices, that only have switching capabilities are no available. For example Keithley replaced the small desktop switch mainframe \device{Model 7001} with the \device{DAQ6510} and Keysight is offering the \device{DAQ973A}, a scanning \num{6.5} digit DMM, that accepts extension cards. Unfortunately, for this project, as discussed above, the integrated \num{6.5} digit multimeter does not add any value.

The simplest option is to go with an \num{8.5} digit multimeter that already included a scanner option or buy a used \device{Keithley 7001} from a second-hand dealer. The author has tested both options and the simplicity of only having a single device to connect and program makes the integrated scanner card of the \device{Model 2002} very attractive.

\begin{figure}[ht]
    \centering
    \scalebox{0.7}{%
        \import{figures/}{simplified_scanner.tex}
    } % scalebox
    \caption{Simplified schematic of the scanner front-end with parasitic elements}
\end{figure}

The scanner card used to multiplex the DMM does have to meet several specifications. The most important aspects are the number channels and the lifetime of the relays. Other factors, such as channel to channel isolation, the contact potential, resistance and maximum voltage is not the limiting factor.

The reason is, that in this case, the voltage is low, there is no ac component involved and the the typical input impedance of high-end multimeters is far more than \qty{100}{\giga\ohm} \cite{datasheet_fluke8588A,article_3458A_input_mpedance_2,datasheet_keithley2002,article_3458A_input_mpedance}.

In this work the Keithley (now Tektronix) \device{Model 2002} was chosen for three reasons. It is a very compact system requiring only a half-sized 2U rack in comparison to the other DMMs, that are typically full-sized 2U rack devices. The other two advantages are the integrated scanner card slot, that allows to to fit a 10 channel scanner card and finally the \qty{20}{\volt} range. The latter is interesting for testing the final voltage reference boards, as these have a \qty{15}{\volt} output, which is too much for the \qty{10}{\volt} range of most DMMs, so that testing the voltage reference Printed circuit boards
(PCBs) one would have to switch to the \qty{100}{\volt} range and forgo an extra digit of resolution and add more noise.

The test setup consists of a mounting PCB, that holds up to 20 Zener diode. It provides power regulation and a minimal circuit required to support each diode. This circuit is given here:

\begin{figure}[ht]
    \centering
    \scalebox{0.7}{%
        \import{figures/}{zener_burnin.tex}
    } % scalebox
    \caption{Circuit used for burning in the Zener diodes}
\end{figure}

The compensation network is required when using the ADR1399, because of its very low dynamic impedance as recommended in the data sheet \cite{datasheet_ADR1399}. It is not strictly required for the LM399, but fitted nonetheless, because there are no downsides to it. This makes the board compatible with both types of references. Each Zener output is protected using an output buffer, which provides isolation and short circuit protection. Finally there is a common mode filter at the output to suppress high frequency noise via ground loops.

The two key metrics of concern, that need to measured are popcorn noise and drift.

digital multimeter and a scanner card


\begin{figure}[ht]
    \centering
    \import{figures/}{DIN_41612.tex}
    \caption{The extension connector used in several Keithley multimeters}
\end{figure}

\begin{table}[ht]
    \centering
    \begin{tabular}{llllll}
        \toprule
        Pin    & Function    & Cable Colour    & Pin    & Function    &  Cable Colour\\
        \midrule
        a1, b1    & \SIrange[print-implicit-plus=true]{6}{20}{\volt}    & brown    & \num{6}    & GND    & green/white\\
        a2, b2    & PD cathode    & red    & \num{7}    & LD Cathode    & blue/white\\
        a3, b3    & LD case (GND)    & red/white    & \num{8}    & LD Anode    & blue\\
        a4    & PD anode (GND)    & red/white    & \num{9}    & LD current    & green\\
        a5    & \SIrange{-6}{-20}{\volt}    & brown/white\\
        \bottomrule
    \end{tabular}
\end{table}

As a sidenote, for the pure entertainment of the author, several batches of LM399 Zener diodes were purchased from non-authorized dealers. Some were marked as refurbished, the others were not marked as such, but clearly were. These so-called refurbished diodes are not to be used in production devices. To entertain and warn the reader a small selection of examples are shown here in figure \ref{fig:fake_lm399}. All but one diode, which is shown for comparison, are refurbished.

\begin{figure}[h]
    \centering
    %\includegraphics[width=0.75\textwidth]{images/foo.png}
    \caption{Refurbished LM399 Zener diodes. From left to right: }
    \label{fig:fake_lm399}
\end{figure}

As it can be clearly seen, the sellers have gone to some effort to hide the fact, that these diodes have been used before. When a through-hole is soldered to the PCB, its legs will be trimmed to match the PCB thickness. In order to conceal this, the legs need to be extended to their original length. The legs of the LM399 are Kovar, because the LM399 is hermetically sealed with a glass seal and Kovar has the same coefficient of expansion as borosilicate glass. The forgers typically weld steel legs to the Kovar legs and then either gold-plate or tin them, as can be seen in fig. \ref{fig:fake_lm399_legs}.

\begin{figure}[h]
    \centering
    %\includegraphics[width=0.75\textwidth]{images/foo.png}
    \caption{Fake steel legs of a refurbished LM399.}
    \label{fig:fake_lm399_legs}
\end{figure}

Much to the delight of the author the refurbished diodes prove valuable for educational purposes. As the origin and method of extraction from the original circuit is unknown, but can be imagined to be rather savage, the diodes are typically faulty. They can therefore be used to validate the test setup and demonstrate the popcorn noise found in the LM399. A very good example in shown in fig. \ref{fig:fake_lm399_popcorn_noise}.

\begin{figure}[h]
    \centering
    %% Creator: Matplotlib, PGF backend
%%
%% To include the figure in your LaTeX document, write
%%   \input{<filename>.pgf}
%%
%% Make sure the required packages are loaded in your preamble
%%   \usepackage{pgf}
%%
%% Also ensure that all the required font packages are loaded; for instance,
%% the lmodern package is sometimes necessary when using math font.
%%   \usepackage{lmodern}
%%
%% Figures using additional raster images can only be included by \input if
%% they are in the same directory as the main LaTeX file. For loading figures
%% from other directories you can use the `import` package
%%   \usepackage{import}
%%
%% and then include the figures with
%%   \import{<path to file>}{<filename>.pgf}
%%
%% Matplotlib used the following preamble
%%   \usepackage{siunitx}
%%   \usepackage{fontspec}
%%
\begingroup%
\makeatletter%
\begin{pgfpicture}%
\pgfpathrectangle{\pgfpointorigin}{\pgfqpoint{5.208662in}{3.219130in}}%
\pgfusepath{use as bounding box, clip}%
\begin{pgfscope}%
\pgfsetbuttcap%
\pgfsetmiterjoin%
\definecolor{currentfill}{rgb}{1.000000,1.000000,1.000000}%
\pgfsetfillcolor{currentfill}%
\pgfsetlinewidth{0.000000pt}%
\definecolor{currentstroke}{rgb}{1.000000,1.000000,1.000000}%
\pgfsetstrokecolor{currentstroke}%
\pgfsetdash{}{0pt}%
\pgfpathmoveto{\pgfqpoint{0.000000in}{0.000000in}}%
\pgfpathlineto{\pgfqpoint{5.208662in}{0.000000in}}%
\pgfpathlineto{\pgfqpoint{5.208662in}{3.219130in}}%
\pgfpathlineto{\pgfqpoint{0.000000in}{3.219130in}}%
\pgfpathlineto{\pgfqpoint{0.000000in}{0.000000in}}%
\pgfpathclose%
\pgfusepath{fill}%
\end{pgfscope}%
\begin{pgfscope}%
\pgfsetbuttcap%
\pgfsetmiterjoin%
\definecolor{currentfill}{rgb}{1.000000,1.000000,1.000000}%
\pgfsetfillcolor{currentfill}%
\pgfsetlinewidth{0.000000pt}%
\definecolor{currentstroke}{rgb}{0.000000,0.000000,0.000000}%
\pgfsetstrokecolor{currentstroke}%
\pgfsetstrokeopacity{0.000000}%
\pgfsetdash{}{0pt}%
\pgfpathmoveto{\pgfqpoint{0.667540in}{0.539544in}}%
\pgfpathlineto{\pgfqpoint{5.058662in}{0.539544in}}%
\pgfpathlineto{\pgfqpoint{5.058662in}{2.944887in}}%
\pgfpathlineto{\pgfqpoint{0.667540in}{2.944887in}}%
\pgfpathlineto{\pgfqpoint{0.667540in}{0.539544in}}%
\pgfpathclose%
\pgfusepath{fill}%
\end{pgfscope}%
\begin{pgfscope}%
\pgfsetbuttcap%
\pgfsetroundjoin%
\definecolor{currentfill}{rgb}{0.000000,0.000000,0.000000}%
\pgfsetfillcolor{currentfill}%
\pgfsetlinewidth{0.803000pt}%
\definecolor{currentstroke}{rgb}{0.000000,0.000000,0.000000}%
\pgfsetstrokecolor{currentstroke}%
\pgfsetdash{}{0pt}%
\pgfsys@defobject{currentmarker}{\pgfqpoint{0.000000in}{-0.048611in}}{\pgfqpoint{0.000000in}{0.000000in}}{%
\pgfpathmoveto{\pgfqpoint{0.000000in}{0.000000in}}%
\pgfpathlineto{\pgfqpoint{0.000000in}{-0.048611in}}%
\pgfusepath{stroke,fill}%
}%
\begin{pgfscope}%
\pgfsys@transformshift{0.866046in}{0.539544in}%
\pgfsys@useobject{currentmarker}{}%
\end{pgfscope}%
\end{pgfscope}%
\begin{pgfscope}%
\definecolor{textcolor}{rgb}{0.000000,0.000000,0.000000}%
\pgfsetstrokecolor{textcolor}%
\pgfsetfillcolor{textcolor}%
\pgftext[x=0.866046in,y=0.442322in,,top]{\color{textcolor}\rmfamily\fontsize{8.000000}{9.600000}\selectfont \(\displaystyle {06{:}45}\)}%
\end{pgfscope}%
\begin{pgfscope}%
\pgfsetbuttcap%
\pgfsetroundjoin%
\definecolor{currentfill}{rgb}{0.000000,0.000000,0.000000}%
\pgfsetfillcolor{currentfill}%
\pgfsetlinewidth{0.803000pt}%
\definecolor{currentstroke}{rgb}{0.000000,0.000000,0.000000}%
\pgfsetstrokecolor{currentstroke}%
\pgfsetdash{}{0pt}%
\pgfsys@defobject{currentmarker}{\pgfqpoint{0.000000in}{-0.048611in}}{\pgfqpoint{0.000000in}{0.000000in}}{%
\pgfpathmoveto{\pgfqpoint{0.000000in}{0.000000in}}%
\pgfpathlineto{\pgfqpoint{0.000000in}{-0.048611in}}%
\pgfusepath{stroke,fill}%
}%
\begin{pgfscope}%
\pgfsys@transformshift{2.197480in}{0.539544in}%
\pgfsys@useobject{currentmarker}{}%
\end{pgfscope}%
\end{pgfscope}%
\begin{pgfscope}%
\definecolor{textcolor}{rgb}{0.000000,0.000000,0.000000}%
\pgfsetstrokecolor{textcolor}%
\pgfsetfillcolor{textcolor}%
\pgftext[x=2.197480in,y=0.442322in,,top]{\color{textcolor}\rmfamily\fontsize{8.000000}{9.600000}\selectfont \(\displaystyle {06{:}50}\)}%
\end{pgfscope}%
\begin{pgfscope}%
\pgfsetbuttcap%
\pgfsetroundjoin%
\definecolor{currentfill}{rgb}{0.000000,0.000000,0.000000}%
\pgfsetfillcolor{currentfill}%
\pgfsetlinewidth{0.803000pt}%
\definecolor{currentstroke}{rgb}{0.000000,0.000000,0.000000}%
\pgfsetstrokecolor{currentstroke}%
\pgfsetdash{}{0pt}%
\pgfsys@defobject{currentmarker}{\pgfqpoint{0.000000in}{-0.048611in}}{\pgfqpoint{0.000000in}{0.000000in}}{%
\pgfpathmoveto{\pgfqpoint{0.000000in}{0.000000in}}%
\pgfpathlineto{\pgfqpoint{0.000000in}{-0.048611in}}%
\pgfusepath{stroke,fill}%
}%
\begin{pgfscope}%
\pgfsys@transformshift{3.528915in}{0.539544in}%
\pgfsys@useobject{currentmarker}{}%
\end{pgfscope}%
\end{pgfscope}%
\begin{pgfscope}%
\definecolor{textcolor}{rgb}{0.000000,0.000000,0.000000}%
\pgfsetstrokecolor{textcolor}%
\pgfsetfillcolor{textcolor}%
\pgftext[x=3.528915in,y=0.442322in,,top]{\color{textcolor}\rmfamily\fontsize{8.000000}{9.600000}\selectfont \(\displaystyle {06{:}55}\)}%
\end{pgfscope}%
\begin{pgfscope}%
\pgfsetbuttcap%
\pgfsetroundjoin%
\definecolor{currentfill}{rgb}{0.000000,0.000000,0.000000}%
\pgfsetfillcolor{currentfill}%
\pgfsetlinewidth{0.803000pt}%
\definecolor{currentstroke}{rgb}{0.000000,0.000000,0.000000}%
\pgfsetstrokecolor{currentstroke}%
\pgfsetdash{}{0pt}%
\pgfsys@defobject{currentmarker}{\pgfqpoint{0.000000in}{-0.048611in}}{\pgfqpoint{0.000000in}{0.000000in}}{%
\pgfpathmoveto{\pgfqpoint{0.000000in}{0.000000in}}%
\pgfpathlineto{\pgfqpoint{0.000000in}{-0.048611in}}%
\pgfusepath{stroke,fill}%
}%
\begin{pgfscope}%
\pgfsys@transformshift{4.860349in}{0.539544in}%
\pgfsys@useobject{currentmarker}{}%
\end{pgfscope}%
\end{pgfscope}%
\begin{pgfscope}%
\definecolor{textcolor}{rgb}{0.000000,0.000000,0.000000}%
\pgfsetstrokecolor{textcolor}%
\pgfsetfillcolor{textcolor}%
\pgftext[x=4.860349in,y=0.442322in,,top]{\color{textcolor}\rmfamily\fontsize{8.000000}{9.600000}\selectfont \(\displaystyle {07{:}00}\)}%
\end{pgfscope}%
\begin{pgfscope}%
\definecolor{textcolor}{rgb}{0.000000,0.000000,0.000000}%
\pgfsetstrokecolor{textcolor}%
\pgfsetfillcolor{textcolor}%
\pgftext[x=2.863101in,y=0.288100in,,top]{\color{textcolor}\rmfamily\fontsize{10.000000}{12.000000}\selectfont Time (UTC)}%
\end{pgfscope}%
\begin{pgfscope}%
\pgfsetbuttcap%
\pgfsetroundjoin%
\definecolor{currentfill}{rgb}{0.000000,0.000000,0.000000}%
\pgfsetfillcolor{currentfill}%
\pgfsetlinewidth{0.803000pt}%
\definecolor{currentstroke}{rgb}{0.000000,0.000000,0.000000}%
\pgfsetstrokecolor{currentstroke}%
\pgfsetdash{}{0pt}%
\pgfsys@defobject{currentmarker}{\pgfqpoint{-0.048611in}{0.000000in}}{\pgfqpoint{-0.000000in}{0.000000in}}{%
\pgfpathmoveto{\pgfqpoint{-0.000000in}{0.000000in}}%
\pgfpathlineto{\pgfqpoint{-0.048611in}{0.000000in}}%
\pgfusepath{stroke,fill}%
}%
\begin{pgfscope}%
\pgfsys@transformshift{0.667540in}{0.611455in}%
\pgfsys@useobject{currentmarker}{}%
\end{pgfscope}%
\end{pgfscope}%
\begin{pgfscope}%
\definecolor{textcolor}{rgb}{0.000000,0.000000,0.000000}%
\pgfsetstrokecolor{textcolor}%
\pgfsetfillcolor{textcolor}%
\pgftext[x=0.327644in, y=0.572899in, left, base]{\color{textcolor}\rmfamily\fontsize{8.000000}{9.600000}\selectfont \(\displaystyle {\ensuremath{-}7.5}\)}%
\end{pgfscope}%
\begin{pgfscope}%
\pgfsetbuttcap%
\pgfsetroundjoin%
\definecolor{currentfill}{rgb}{0.000000,0.000000,0.000000}%
\pgfsetfillcolor{currentfill}%
\pgfsetlinewidth{0.803000pt}%
\definecolor{currentstroke}{rgb}{0.000000,0.000000,0.000000}%
\pgfsetstrokecolor{currentstroke}%
\pgfsetdash{}{0pt}%
\pgfsys@defobject{currentmarker}{\pgfqpoint{-0.048611in}{0.000000in}}{\pgfqpoint{-0.000000in}{0.000000in}}{%
\pgfpathmoveto{\pgfqpoint{-0.000000in}{0.000000in}}%
\pgfpathlineto{\pgfqpoint{-0.048611in}{0.000000in}}%
\pgfusepath{stroke,fill}%
}%
\begin{pgfscope}%
\pgfsys@transformshift{0.667540in}{0.925106in}%
\pgfsys@useobject{currentmarker}{}%
\end{pgfscope}%
\end{pgfscope}%
\begin{pgfscope}%
\definecolor{textcolor}{rgb}{0.000000,0.000000,0.000000}%
\pgfsetstrokecolor{textcolor}%
\pgfsetfillcolor{textcolor}%
\pgftext[x=0.327644in, y=0.886551in, left, base]{\color{textcolor}\rmfamily\fontsize{8.000000}{9.600000}\selectfont \(\displaystyle {\ensuremath{-}5.0}\)}%
\end{pgfscope}%
\begin{pgfscope}%
\pgfsetbuttcap%
\pgfsetroundjoin%
\definecolor{currentfill}{rgb}{0.000000,0.000000,0.000000}%
\pgfsetfillcolor{currentfill}%
\pgfsetlinewidth{0.803000pt}%
\definecolor{currentstroke}{rgb}{0.000000,0.000000,0.000000}%
\pgfsetstrokecolor{currentstroke}%
\pgfsetdash{}{0pt}%
\pgfsys@defobject{currentmarker}{\pgfqpoint{-0.048611in}{0.000000in}}{\pgfqpoint{-0.000000in}{0.000000in}}{%
\pgfpathmoveto{\pgfqpoint{-0.000000in}{0.000000in}}%
\pgfpathlineto{\pgfqpoint{-0.048611in}{0.000000in}}%
\pgfusepath{stroke,fill}%
}%
\begin{pgfscope}%
\pgfsys@transformshift{0.667540in}{1.238757in}%
\pgfsys@useobject{currentmarker}{}%
\end{pgfscope}%
\end{pgfscope}%
\begin{pgfscope}%
\definecolor{textcolor}{rgb}{0.000000,0.000000,0.000000}%
\pgfsetstrokecolor{textcolor}%
\pgfsetfillcolor{textcolor}%
\pgftext[x=0.327644in, y=1.200202in, left, base]{\color{textcolor}\rmfamily\fontsize{8.000000}{9.600000}\selectfont \(\displaystyle {\ensuremath{-}2.5}\)}%
\end{pgfscope}%
\begin{pgfscope}%
\pgfsetbuttcap%
\pgfsetroundjoin%
\definecolor{currentfill}{rgb}{0.000000,0.000000,0.000000}%
\pgfsetfillcolor{currentfill}%
\pgfsetlinewidth{0.803000pt}%
\definecolor{currentstroke}{rgb}{0.000000,0.000000,0.000000}%
\pgfsetstrokecolor{currentstroke}%
\pgfsetdash{}{0pt}%
\pgfsys@defobject{currentmarker}{\pgfqpoint{-0.048611in}{0.000000in}}{\pgfqpoint{-0.000000in}{0.000000in}}{%
\pgfpathmoveto{\pgfqpoint{-0.000000in}{0.000000in}}%
\pgfpathlineto{\pgfqpoint{-0.048611in}{0.000000in}}%
\pgfusepath{stroke,fill}%
}%
\begin{pgfscope}%
\pgfsys@transformshift{0.667540in}{1.552408in}%
\pgfsys@useobject{currentmarker}{}%
\end{pgfscope}%
\end{pgfscope}%
\begin{pgfscope}%
\definecolor{textcolor}{rgb}{0.000000,0.000000,0.000000}%
\pgfsetstrokecolor{textcolor}%
\pgfsetfillcolor{textcolor}%
\pgftext[x=0.419467in, y=1.513853in, left, base]{\color{textcolor}\rmfamily\fontsize{8.000000}{9.600000}\selectfont \(\displaystyle {0.0}\)}%
\end{pgfscope}%
\begin{pgfscope}%
\pgfsetbuttcap%
\pgfsetroundjoin%
\definecolor{currentfill}{rgb}{0.000000,0.000000,0.000000}%
\pgfsetfillcolor{currentfill}%
\pgfsetlinewidth{0.803000pt}%
\definecolor{currentstroke}{rgb}{0.000000,0.000000,0.000000}%
\pgfsetstrokecolor{currentstroke}%
\pgfsetdash{}{0pt}%
\pgfsys@defobject{currentmarker}{\pgfqpoint{-0.048611in}{0.000000in}}{\pgfqpoint{-0.000000in}{0.000000in}}{%
\pgfpathmoveto{\pgfqpoint{-0.000000in}{0.000000in}}%
\pgfpathlineto{\pgfqpoint{-0.048611in}{0.000000in}}%
\pgfusepath{stroke,fill}%
}%
\begin{pgfscope}%
\pgfsys@transformshift{0.667540in}{1.866059in}%
\pgfsys@useobject{currentmarker}{}%
\end{pgfscope}%
\end{pgfscope}%
\begin{pgfscope}%
\definecolor{textcolor}{rgb}{0.000000,0.000000,0.000000}%
\pgfsetstrokecolor{textcolor}%
\pgfsetfillcolor{textcolor}%
\pgftext[x=0.419467in, y=1.827504in, left, base]{\color{textcolor}\rmfamily\fontsize{8.000000}{9.600000}\selectfont \(\displaystyle {2.5}\)}%
\end{pgfscope}%
\begin{pgfscope}%
\pgfsetbuttcap%
\pgfsetroundjoin%
\definecolor{currentfill}{rgb}{0.000000,0.000000,0.000000}%
\pgfsetfillcolor{currentfill}%
\pgfsetlinewidth{0.803000pt}%
\definecolor{currentstroke}{rgb}{0.000000,0.000000,0.000000}%
\pgfsetstrokecolor{currentstroke}%
\pgfsetdash{}{0pt}%
\pgfsys@defobject{currentmarker}{\pgfqpoint{-0.048611in}{0.000000in}}{\pgfqpoint{-0.000000in}{0.000000in}}{%
\pgfpathmoveto{\pgfqpoint{-0.000000in}{0.000000in}}%
\pgfpathlineto{\pgfqpoint{-0.048611in}{0.000000in}}%
\pgfusepath{stroke,fill}%
}%
\begin{pgfscope}%
\pgfsys@transformshift{0.667540in}{2.179710in}%
\pgfsys@useobject{currentmarker}{}%
\end{pgfscope}%
\end{pgfscope}%
\begin{pgfscope}%
\definecolor{textcolor}{rgb}{0.000000,0.000000,0.000000}%
\pgfsetstrokecolor{textcolor}%
\pgfsetfillcolor{textcolor}%
\pgftext[x=0.419467in, y=2.141155in, left, base]{\color{textcolor}\rmfamily\fontsize{8.000000}{9.600000}\selectfont \(\displaystyle {5.0}\)}%
\end{pgfscope}%
\begin{pgfscope}%
\pgfsetbuttcap%
\pgfsetroundjoin%
\definecolor{currentfill}{rgb}{0.000000,0.000000,0.000000}%
\pgfsetfillcolor{currentfill}%
\pgfsetlinewidth{0.803000pt}%
\definecolor{currentstroke}{rgb}{0.000000,0.000000,0.000000}%
\pgfsetstrokecolor{currentstroke}%
\pgfsetdash{}{0pt}%
\pgfsys@defobject{currentmarker}{\pgfqpoint{-0.048611in}{0.000000in}}{\pgfqpoint{-0.000000in}{0.000000in}}{%
\pgfpathmoveto{\pgfqpoint{-0.000000in}{0.000000in}}%
\pgfpathlineto{\pgfqpoint{-0.048611in}{0.000000in}}%
\pgfusepath{stroke,fill}%
}%
\begin{pgfscope}%
\pgfsys@transformshift{0.667540in}{2.493362in}%
\pgfsys@useobject{currentmarker}{}%
\end{pgfscope}%
\end{pgfscope}%
\begin{pgfscope}%
\definecolor{textcolor}{rgb}{0.000000,0.000000,0.000000}%
\pgfsetstrokecolor{textcolor}%
\pgfsetfillcolor{textcolor}%
\pgftext[x=0.419467in, y=2.454806in, left, base]{\color{textcolor}\rmfamily\fontsize{8.000000}{9.600000}\selectfont \(\displaystyle {7.5}\)}%
\end{pgfscope}%
\begin{pgfscope}%
\pgfsetbuttcap%
\pgfsetroundjoin%
\definecolor{currentfill}{rgb}{0.000000,0.000000,0.000000}%
\pgfsetfillcolor{currentfill}%
\pgfsetlinewidth{0.803000pt}%
\definecolor{currentstroke}{rgb}{0.000000,0.000000,0.000000}%
\pgfsetstrokecolor{currentstroke}%
\pgfsetdash{}{0pt}%
\pgfsys@defobject{currentmarker}{\pgfqpoint{-0.048611in}{0.000000in}}{\pgfqpoint{-0.000000in}{0.000000in}}{%
\pgfpathmoveto{\pgfqpoint{-0.000000in}{0.000000in}}%
\pgfpathlineto{\pgfqpoint{-0.048611in}{0.000000in}}%
\pgfusepath{stroke,fill}%
}%
\begin{pgfscope}%
\pgfsys@transformshift{0.667540in}{2.807013in}%
\pgfsys@useobject{currentmarker}{}%
\end{pgfscope}%
\end{pgfscope}%
\begin{pgfscope}%
\definecolor{textcolor}{rgb}{0.000000,0.000000,0.000000}%
\pgfsetstrokecolor{textcolor}%
\pgfsetfillcolor{textcolor}%
\pgftext[x=0.360438in, y=2.768457in, left, base]{\color{textcolor}\rmfamily\fontsize{8.000000}{9.600000}\selectfont \(\displaystyle {10.0}\)}%
\end{pgfscope}%
\begin{pgfscope}%
\definecolor{textcolor}{rgb}{0.000000,0.000000,0.000000}%
\pgfsetstrokecolor{textcolor}%
\pgfsetfillcolor{textcolor}%
\pgftext[x=0.272089in,y=1.742216in,,bottom,rotate=90.000000]{\color{textcolor}\rmfamily\fontsize{10.000000}{12.000000}\selectfont Voltage deviation in V}%
\end{pgfscope}%
\begin{pgfscope}%
\definecolor{textcolor}{rgb}{0.000000,0.000000,0.000000}%
\pgfsetstrokecolor{textcolor}%
\pgfsetfillcolor{textcolor}%
\pgftext[x=0.667540in,y=2.986554in,left,base]{\color{textcolor}\rmfamily\fontsize{8.000000}{9.600000}\selectfont \(\displaystyle \times{10^{\ensuremath{-}6}}{}\)}%
\end{pgfscope}%
\begin{pgfscope}%
\pgfpathrectangle{\pgfqpoint{0.667540in}{0.539544in}}{\pgfqpoint{4.391122in}{2.405343in}}%
\pgfusepath{clip}%
\pgfsetrectcap%
\pgfsetroundjoin%
\pgfsetlinewidth{0.501875pt}%
\definecolor{currentstroke}{rgb}{0.121569,0.466667,0.705882}%
\pgfsetstrokecolor{currentstroke}%
\pgfsetstrokeopacity{0.700000}%
\pgfsetdash{}{0pt}%
\pgfpathmoveto{\pgfqpoint{0.867136in}{1.087387in}}%
\pgfpathlineto{\pgfqpoint{0.870808in}{1.118966in}}%
\pgfpathlineto{\pgfqpoint{0.872643in}{1.181144in}}%
\pgfpathlineto{\pgfqpoint{0.874481in}{0.958565in}}%
\pgfpathlineto{\pgfqpoint{0.876316in}{1.101426in}}%
\pgfpathlineto{\pgfqpoint{0.878153in}{0.951576in}}%
\pgfpathlineto{\pgfqpoint{0.881823in}{1.075920in}}%
\pgfpathlineto{\pgfqpoint{0.883659in}{1.252581in}}%
\pgfpathlineto{\pgfqpoint{0.885494in}{1.091590in}}%
\pgfpathlineto{\pgfqpoint{0.887330in}{1.123470in}}%
\pgfpathlineto{\pgfqpoint{0.889165in}{1.181997in}}%
\pgfpathlineto{\pgfqpoint{0.892838in}{1.123909in}}%
\pgfpathlineto{\pgfqpoint{0.894674in}{1.172450in}}%
\pgfpathlineto{\pgfqpoint{0.896510in}{1.289040in}}%
\pgfpathlineto{\pgfqpoint{0.898346in}{1.152940in}}%
\pgfpathlineto{\pgfqpoint{0.902018in}{1.215583in}}%
\pgfpathlineto{\pgfqpoint{0.905690in}{1.384741in}}%
\pgfpathlineto{\pgfqpoint{0.909363in}{1.223286in}}%
\pgfpathlineto{\pgfqpoint{0.911199in}{1.421614in}}%
\pgfpathlineto{\pgfqpoint{0.913035in}{1.319953in}}%
\pgfpathlineto{\pgfqpoint{0.914871in}{1.291248in}}%
\pgfpathlineto{\pgfqpoint{0.918543in}{1.115240in}}%
\pgfpathlineto{\pgfqpoint{0.920378in}{1.088742in}}%
\pgfpathlineto{\pgfqpoint{0.922213in}{1.270585in}}%
\pgfpathlineto{\pgfqpoint{0.924048in}{1.272203in}}%
\pgfpathlineto{\pgfqpoint{0.925883in}{1.289065in}}%
\pgfpathlineto{\pgfqpoint{0.929556in}{1.198307in}}%
\pgfpathlineto{\pgfqpoint{0.931392in}{1.190227in}}%
\pgfpathlineto{\pgfqpoint{0.933228in}{1.146442in}}%
\pgfpathlineto{\pgfqpoint{0.935065in}{1.283369in}}%
\pgfpathlineto{\pgfqpoint{0.938735in}{1.029588in}}%
\pgfpathlineto{\pgfqpoint{0.946079in}{1.626290in}}%
\pgfpathlineto{\pgfqpoint{0.949753in}{1.444096in}}%
\pgfpathlineto{\pgfqpoint{0.951589in}{1.616956in}}%
\pgfpathlineto{\pgfqpoint{0.953425in}{1.532446in}}%
\pgfpathlineto{\pgfqpoint{0.955260in}{1.652950in}}%
\pgfpathlineto{\pgfqpoint{0.957096in}{1.510703in}}%
\pgfpathlineto{\pgfqpoint{0.958931in}{1.617922in}}%
\pgfpathlineto{\pgfqpoint{0.960766in}{1.610194in}}%
\pgfpathlineto{\pgfqpoint{0.962600in}{1.313542in}}%
\pgfpathlineto{\pgfqpoint{0.964437in}{1.329589in}}%
\pgfpathlineto{\pgfqpoint{0.966272in}{1.506526in}}%
\pgfpathlineto{\pgfqpoint{0.968108in}{1.501068in}}%
\pgfpathlineto{\pgfqpoint{0.969944in}{1.597183in}}%
\pgfpathlineto{\pgfqpoint{0.971780in}{1.593921in}}%
\pgfpathlineto{\pgfqpoint{0.973616in}{1.657530in}}%
\pgfpathlineto{\pgfqpoint{0.975451in}{1.470130in}}%
\pgfpathlineto{\pgfqpoint{0.979123in}{1.384151in}}%
\pgfpathlineto{\pgfqpoint{0.980959in}{1.395556in}}%
\pgfpathlineto{\pgfqpoint{0.982794in}{1.581990in}}%
\pgfpathlineto{\pgfqpoint{0.984630in}{1.471886in}}%
\pgfpathlineto{\pgfqpoint{0.986466in}{1.237476in}}%
\pgfpathlineto{\pgfqpoint{0.990138in}{1.537000in}}%
\pgfpathlineto{\pgfqpoint{0.991973in}{1.372346in}}%
\pgfpathlineto{\pgfqpoint{0.995643in}{1.405605in}}%
\pgfpathlineto{\pgfqpoint{0.997478in}{1.710361in}}%
\pgfpathlineto{\pgfqpoint{0.999315in}{1.353740in}}%
\pgfpathlineto{\pgfqpoint{1.001150in}{1.559683in}}%
\pgfpathlineto{\pgfqpoint{1.002986in}{1.584449in}}%
\pgfpathlineto{\pgfqpoint{1.004822in}{1.532120in}}%
\pgfpathlineto{\pgfqpoint{1.006658in}{1.099607in}}%
\pgfpathlineto{\pgfqpoint{1.008497in}{1.116206in}}%
\pgfpathlineto{\pgfqpoint{1.010333in}{1.363701in}}%
\pgfpathlineto{\pgfqpoint{1.012168in}{1.297822in}}%
\pgfpathlineto{\pgfqpoint{1.014005in}{1.069810in}}%
\pgfpathlineto{\pgfqpoint{1.017676in}{1.366574in}}%
\pgfpathlineto{\pgfqpoint{1.019512in}{1.219133in}}%
\pgfpathlineto{\pgfqpoint{1.021348in}{1.249871in}}%
\pgfpathlineto{\pgfqpoint{1.023183in}{1.354882in}}%
\pgfpathlineto{\pgfqpoint{1.025021in}{1.274637in}}%
\pgfpathlineto{\pgfqpoint{1.026856in}{1.591236in}}%
\pgfpathlineto{\pgfqpoint{1.028692in}{1.362409in}}%
\pgfpathlineto{\pgfqpoint{1.030527in}{1.339613in}}%
\pgfpathlineto{\pgfqpoint{1.034199in}{1.115679in}}%
\pgfpathlineto{\pgfqpoint{1.036034in}{1.211944in}}%
\pgfpathlineto{\pgfqpoint{1.037869in}{1.103158in}}%
\pgfpathlineto{\pgfqpoint{1.039706in}{1.258302in}}%
\pgfpathlineto{\pgfqpoint{1.041541in}{1.234803in}}%
\pgfpathlineto{\pgfqpoint{1.043377in}{1.057089in}}%
\pgfpathlineto{\pgfqpoint{1.048886in}{2.234585in}}%
\pgfpathlineto{\pgfqpoint{1.050723in}{2.532253in}}%
\pgfpathlineto{\pgfqpoint{1.052559in}{2.137178in}}%
\pgfpathlineto{\pgfqpoint{1.054395in}{1.468800in}}%
\pgfpathlineto{\pgfqpoint{1.056245in}{1.494895in}}%
\pgfpathlineto{\pgfqpoint{1.058082in}{1.229421in}}%
\pgfpathlineto{\pgfqpoint{1.061752in}{1.352648in}}%
\pgfpathlineto{\pgfqpoint{1.063588in}{1.260711in}}%
\pgfpathlineto{\pgfqpoint{1.065424in}{1.255956in}}%
\pgfpathlineto{\pgfqpoint{1.067259in}{1.191256in}}%
\pgfpathlineto{\pgfqpoint{1.069094in}{1.201155in}}%
\pgfpathlineto{\pgfqpoint{1.070929in}{1.157846in}}%
\pgfpathlineto{\pgfqpoint{1.072765in}{1.334406in}}%
\pgfpathlineto{\pgfqpoint{1.074601in}{1.051305in}}%
\pgfpathlineto{\pgfqpoint{1.076437in}{1.210828in}}%
\pgfpathlineto{\pgfqpoint{1.078273in}{1.206675in}}%
\pgfpathlineto{\pgfqpoint{1.080109in}{1.457370in}}%
\pgfpathlineto{\pgfqpoint{1.081946in}{2.094308in}}%
\pgfpathlineto{\pgfqpoint{1.083781in}{2.144203in}}%
\pgfpathlineto{\pgfqpoint{1.085616in}{2.054763in}}%
\pgfpathlineto{\pgfqpoint{1.087454in}{2.151279in}}%
\pgfpathlineto{\pgfqpoint{1.089290in}{2.347499in}}%
\pgfpathlineto{\pgfqpoint{1.091125in}{2.151543in}}%
\pgfpathlineto{\pgfqpoint{1.092961in}{2.094546in}}%
\pgfpathlineto{\pgfqpoint{1.094797in}{2.186985in}}%
\pgfpathlineto{\pgfqpoint{1.096632in}{2.073795in}}%
\pgfpathlineto{\pgfqpoint{1.098468in}{2.131156in}}%
\pgfpathlineto{\pgfqpoint{1.100303in}{2.013587in}}%
\pgfpathlineto{\pgfqpoint{1.102138in}{2.176083in}}%
\pgfpathlineto{\pgfqpoint{1.105808in}{2.148871in}}%
\pgfpathlineto{\pgfqpoint{1.107646in}{2.056369in}}%
\pgfpathlineto{\pgfqpoint{1.111317in}{2.206846in}}%
\pgfpathlineto{\pgfqpoint{1.113154in}{2.051614in}}%
\pgfpathlineto{\pgfqpoint{1.114989in}{2.200761in}}%
\pgfpathlineto{\pgfqpoint{1.116824in}{2.146725in}}%
\pgfpathlineto{\pgfqpoint{1.118661in}{2.019144in}}%
\pgfpathlineto{\pgfqpoint{1.120497in}{2.141468in}}%
\pgfpathlineto{\pgfqpoint{1.124169in}{1.269732in}}%
\pgfpathlineto{\pgfqpoint{1.126004in}{1.341721in}}%
\pgfpathlineto{\pgfqpoint{1.127845in}{1.286606in}}%
\pgfpathlineto{\pgfqpoint{1.129681in}{1.146241in}}%
\pgfpathlineto{\pgfqpoint{1.133355in}{1.080562in}}%
\pgfpathlineto{\pgfqpoint{1.135191in}{1.274085in}}%
\pgfpathlineto{\pgfqpoint{1.137027in}{1.059849in}}%
\pgfpathlineto{\pgfqpoint{1.138862in}{1.087538in}}%
\pgfpathlineto{\pgfqpoint{1.140696in}{1.174469in}}%
\pgfpathlineto{\pgfqpoint{1.142533in}{1.206085in}}%
\pgfpathlineto{\pgfqpoint{1.144368in}{1.114060in}}%
\pgfpathlineto{\pgfqpoint{1.146205in}{1.310669in}}%
\pgfpathlineto{\pgfqpoint{1.148041in}{1.983175in}}%
\pgfpathlineto{\pgfqpoint{1.149878in}{2.057134in}}%
\pgfpathlineto{\pgfqpoint{1.151713in}{1.299027in}}%
\pgfpathlineto{\pgfqpoint{1.153550in}{2.024163in}}%
\pgfpathlineto{\pgfqpoint{1.155386in}{1.916505in}}%
\pgfpathlineto{\pgfqpoint{1.157223in}{1.936228in}}%
\pgfpathlineto{\pgfqpoint{1.159058in}{2.108121in}}%
\pgfpathlineto{\pgfqpoint{1.160894in}{2.131682in}}%
\pgfpathlineto{\pgfqpoint{1.162731in}{2.048352in}}%
\pgfpathlineto{\pgfqpoint{1.166405in}{2.197474in}}%
\pgfpathlineto{\pgfqpoint{1.168241in}{2.119337in}}%
\pgfpathlineto{\pgfqpoint{1.170076in}{2.177965in}}%
\pgfpathlineto{\pgfqpoint{1.171914in}{2.155921in}}%
\pgfpathlineto{\pgfqpoint{1.173748in}{2.047235in}}%
\pgfpathlineto{\pgfqpoint{1.175582in}{2.072151in}}%
\pgfpathlineto{\pgfqpoint{1.177418in}{2.185630in}}%
\pgfpathlineto{\pgfqpoint{1.181087in}{1.964230in}}%
\pgfpathlineto{\pgfqpoint{1.186596in}{2.183071in}}%
\pgfpathlineto{\pgfqpoint{1.190268in}{1.984066in}}%
\pgfpathlineto{\pgfqpoint{1.192105in}{1.972047in}}%
\pgfpathlineto{\pgfqpoint{1.193940in}{1.421413in}}%
\pgfpathlineto{\pgfqpoint{1.197612in}{1.164069in}}%
\pgfpathlineto{\pgfqpoint{1.199448in}{1.245317in}}%
\pgfpathlineto{\pgfqpoint{1.201285in}{1.047541in}}%
\pgfpathlineto{\pgfqpoint{1.203121in}{1.305262in}}%
\pgfpathlineto{\pgfqpoint{1.204957in}{1.405982in}}%
\pgfpathlineto{\pgfqpoint{1.208632in}{1.280609in}}%
\pgfpathlineto{\pgfqpoint{1.210467in}{1.242118in}}%
\pgfpathlineto{\pgfqpoint{1.212302in}{1.359611in}}%
\pgfpathlineto{\pgfqpoint{1.214137in}{1.275666in}}%
\pgfpathlineto{\pgfqpoint{1.215972in}{1.375043in}}%
\pgfpathlineto{\pgfqpoint{1.217807in}{1.207415in}}%
\pgfpathlineto{\pgfqpoint{1.221479in}{1.316428in}}%
\pgfpathlineto{\pgfqpoint{1.223316in}{1.354831in}}%
\pgfpathlineto{\pgfqpoint{1.225150in}{1.049398in}}%
\pgfpathlineto{\pgfqpoint{1.226988in}{1.149678in}}%
\pgfpathlineto{\pgfqpoint{1.228824in}{1.126845in}}%
\pgfpathlineto{\pgfqpoint{1.230659in}{1.174971in}}%
\pgfpathlineto{\pgfqpoint{1.232495in}{1.334820in}}%
\pgfpathlineto{\pgfqpoint{1.234331in}{1.203852in}}%
\pgfpathlineto{\pgfqpoint{1.236168in}{1.285715in}}%
\pgfpathlineto{\pgfqpoint{1.238004in}{1.260862in}}%
\pgfpathlineto{\pgfqpoint{1.239840in}{1.427523in}}%
\pgfpathlineto{\pgfqpoint{1.241677in}{1.373400in}}%
\pgfpathlineto{\pgfqpoint{1.243513in}{1.231328in}}%
\pgfpathlineto{\pgfqpoint{1.245348in}{1.389910in}}%
\pgfpathlineto{\pgfqpoint{1.249019in}{1.220689in}}%
\pgfpathlineto{\pgfqpoint{1.250854in}{1.366261in}}%
\pgfpathlineto{\pgfqpoint{1.252690in}{1.261564in}}%
\pgfpathlineto{\pgfqpoint{1.256359in}{1.400549in}}%
\pgfpathlineto{\pgfqpoint{1.258196in}{1.241904in}}%
\pgfpathlineto{\pgfqpoint{1.260032in}{1.239847in}}%
\pgfpathlineto{\pgfqpoint{1.261867in}{1.542972in}}%
\pgfpathlineto{\pgfqpoint{1.263704in}{2.158004in}}%
\pgfpathlineto{\pgfqpoint{1.265540in}{2.350322in}}%
\pgfpathlineto{\pgfqpoint{1.269213in}{2.139147in}}%
\pgfpathlineto{\pgfqpoint{1.271049in}{2.259288in}}%
\pgfpathlineto{\pgfqpoint{1.272886in}{2.184689in}}%
\pgfpathlineto{\pgfqpoint{1.274721in}{2.231022in}}%
\pgfpathlineto{\pgfqpoint{1.276557in}{2.170939in}}%
\pgfpathlineto{\pgfqpoint{1.278395in}{2.312145in}}%
\pgfpathlineto{\pgfqpoint{1.280229in}{2.275749in}}%
\pgfpathlineto{\pgfqpoint{1.283924in}{2.348616in}}%
\pgfpathlineto{\pgfqpoint{1.285760in}{2.339119in}}%
\pgfpathlineto{\pgfqpoint{1.287594in}{2.442611in}}%
\pgfpathlineto{\pgfqpoint{1.289430in}{2.232402in}}%
\pgfpathlineto{\pgfqpoint{1.291265in}{2.217159in}}%
\pgfpathlineto{\pgfqpoint{1.294938in}{2.254496in}}%
\pgfpathlineto{\pgfqpoint{1.296774in}{2.273478in}}%
\pgfpathlineto{\pgfqpoint{1.298608in}{2.227710in}}%
\pgfpathlineto{\pgfqpoint{1.300446in}{2.217096in}}%
\pgfpathlineto{\pgfqpoint{1.302282in}{2.156800in}}%
\pgfpathlineto{\pgfqpoint{1.304117in}{1.349863in}}%
\pgfpathlineto{\pgfqpoint{1.305953in}{1.111012in}}%
\pgfpathlineto{\pgfqpoint{1.307790in}{1.240211in}}%
\pgfpathlineto{\pgfqpoint{1.309627in}{1.209059in}}%
\pgfpathlineto{\pgfqpoint{1.311463in}{1.149503in}}%
\pgfpathlineto{\pgfqpoint{1.315135in}{1.233423in}}%
\pgfpathlineto{\pgfqpoint{1.316970in}{1.146203in}}%
\pgfpathlineto{\pgfqpoint{1.318807in}{1.237614in}}%
\pgfpathlineto{\pgfqpoint{1.320643in}{1.170279in}}%
\pgfpathlineto{\pgfqpoint{1.322477in}{1.275264in}}%
\pgfpathlineto{\pgfqpoint{1.326148in}{1.116733in}}%
\pgfpathlineto{\pgfqpoint{1.327983in}{1.249369in}}%
\pgfpathlineto{\pgfqpoint{1.329819in}{1.117410in}}%
\pgfpathlineto{\pgfqpoint{1.331656in}{1.345899in}}%
\pgfpathlineto{\pgfqpoint{1.333492in}{1.277786in}}%
\pgfpathlineto{\pgfqpoint{1.335329in}{1.164044in}}%
\pgfpathlineto{\pgfqpoint{1.337164in}{1.195396in}}%
\pgfpathlineto{\pgfqpoint{1.339000in}{1.120935in}}%
\pgfpathlineto{\pgfqpoint{1.342670in}{1.215344in}}%
\pgfpathlineto{\pgfqpoint{1.344507in}{1.183641in}}%
\pgfpathlineto{\pgfqpoint{1.347225in}{1.035560in}}%
\pgfpathlineto{\pgfqpoint{1.349060in}{1.134811in}}%
\pgfpathlineto{\pgfqpoint{1.350897in}{1.093146in}}%
\pgfpathlineto{\pgfqpoint{1.352733in}{1.092556in}}%
\pgfpathlineto{\pgfqpoint{1.354568in}{1.137546in}}%
\pgfpathlineto{\pgfqpoint{1.358240in}{0.961801in}}%
\pgfpathlineto{\pgfqpoint{1.360072in}{1.107900in}}%
\pgfpathlineto{\pgfqpoint{1.361907in}{1.141072in}}%
\pgfpathlineto{\pgfqpoint{1.363739in}{1.103434in}}%
\pgfpathlineto{\pgfqpoint{1.365572in}{1.188345in}}%
\pgfpathlineto{\pgfqpoint{1.367406in}{1.130960in}}%
\pgfpathlineto{\pgfqpoint{1.369240in}{1.462226in}}%
\pgfpathlineto{\pgfqpoint{1.372906in}{1.214930in}}%
\pgfpathlineto{\pgfqpoint{1.374739in}{1.296116in}}%
\pgfpathlineto{\pgfqpoint{1.376572in}{1.258653in}}%
\pgfpathlineto{\pgfqpoint{1.378404in}{1.542846in}}%
\pgfpathlineto{\pgfqpoint{1.383907in}{1.135514in}}%
\pgfpathlineto{\pgfqpoint{1.385743in}{0.977346in}}%
\pgfpathlineto{\pgfqpoint{1.387576in}{1.130307in}}%
\pgfpathlineto{\pgfqpoint{1.389409in}{1.049160in}}%
\pgfpathlineto{\pgfqpoint{1.391244in}{1.216725in}}%
\pgfpathlineto{\pgfqpoint{1.393077in}{1.149152in}}%
\pgfpathlineto{\pgfqpoint{1.396744in}{1.240123in}}%
\pgfpathlineto{\pgfqpoint{1.398578in}{1.367377in}}%
\pgfpathlineto{\pgfqpoint{1.402245in}{1.101690in}}%
\pgfpathlineto{\pgfqpoint{1.404078in}{1.141423in}}%
\pgfpathlineto{\pgfqpoint{1.405913in}{1.019538in}}%
\pgfpathlineto{\pgfqpoint{1.407746in}{1.120622in}}%
\pgfpathlineto{\pgfqpoint{1.409581in}{1.145764in}}%
\pgfpathlineto{\pgfqpoint{1.413248in}{0.967886in}}%
\pgfpathlineto{\pgfqpoint{1.415082in}{1.012161in}}%
\pgfpathlineto{\pgfqpoint{1.416914in}{0.978927in}}%
\pgfpathlineto{\pgfqpoint{1.420582in}{1.154847in}}%
\pgfpathlineto{\pgfqpoint{1.424249in}{1.255780in}}%
\pgfpathlineto{\pgfqpoint{1.426084in}{1.173892in}}%
\pgfpathlineto{\pgfqpoint{1.427917in}{1.306793in}}%
\pgfpathlineto{\pgfqpoint{1.429749in}{1.031294in}}%
\pgfpathlineto{\pgfqpoint{1.431585in}{1.137459in}}%
\pgfpathlineto{\pgfqpoint{1.433419in}{1.176301in}}%
\pgfpathlineto{\pgfqpoint{1.437087in}{1.023239in}}%
\pgfpathlineto{\pgfqpoint{1.438920in}{1.168046in}}%
\pgfpathlineto{\pgfqpoint{1.440752in}{1.202008in}}%
\pgfpathlineto{\pgfqpoint{1.442587in}{1.384064in}}%
\pgfpathlineto{\pgfqpoint{1.448088in}{1.099670in}}%
\pgfpathlineto{\pgfqpoint{1.449922in}{1.246283in}}%
\pgfpathlineto{\pgfqpoint{1.453590in}{1.226661in}}%
\pgfpathlineto{\pgfqpoint{1.455424in}{1.263183in}}%
\pgfpathlineto{\pgfqpoint{1.457277in}{1.175297in}}%
\pgfpathlineto{\pgfqpoint{1.459481in}{1.229948in}}%
\pgfpathlineto{\pgfqpoint{1.461314in}{1.053388in}}%
\pgfpathlineto{\pgfqpoint{1.463148in}{1.119819in}}%
\pgfpathlineto{\pgfqpoint{1.464981in}{0.999415in}}%
\pgfpathlineto{\pgfqpoint{1.466813in}{1.487932in}}%
\pgfpathlineto{\pgfqpoint{1.472313in}{1.085242in}}%
\pgfpathlineto{\pgfqpoint{1.474146in}{1.262216in}}%
\pgfpathlineto{\pgfqpoint{1.475980in}{1.228681in}}%
\pgfpathlineto{\pgfqpoint{1.477812in}{1.015599in}}%
\pgfpathlineto{\pgfqpoint{1.479648in}{2.006573in}}%
\pgfpathlineto{\pgfqpoint{1.481481in}{1.754410in}}%
\pgfpathlineto{\pgfqpoint{1.483314in}{0.936647in}}%
\pgfpathlineto{\pgfqpoint{1.486984in}{1.164922in}}%
\pgfpathlineto{\pgfqpoint{1.488817in}{1.159377in}}%
\pgfpathlineto{\pgfqpoint{1.490650in}{1.092908in}}%
\pgfpathlineto{\pgfqpoint{1.492484in}{1.070902in}}%
\pgfpathlineto{\pgfqpoint{1.494318in}{0.980257in}}%
\pgfpathlineto{\pgfqpoint{1.496151in}{1.178233in}}%
\pgfpathlineto{\pgfqpoint{1.497985in}{1.212873in}}%
\pgfpathlineto{\pgfqpoint{1.499818in}{1.077953in}}%
\pgfpathlineto{\pgfqpoint{1.503486in}{1.267034in}}%
\pgfpathlineto{\pgfqpoint{1.505321in}{1.082269in}}%
\pgfpathlineto{\pgfqpoint{1.507154in}{1.092494in}}%
\pgfpathlineto{\pgfqpoint{1.508989in}{1.132515in}}%
\pgfpathlineto{\pgfqpoint{1.510822in}{0.911943in}}%
\pgfpathlineto{\pgfqpoint{1.512654in}{1.099319in}}%
\pgfpathlineto{\pgfqpoint{1.514489in}{1.161133in}}%
\pgfpathlineto{\pgfqpoint{1.516322in}{1.102781in}}%
\pgfpathlineto{\pgfqpoint{1.518156in}{0.998825in}}%
\pgfpathlineto{\pgfqpoint{1.523657in}{1.369636in}}%
\pgfpathlineto{\pgfqpoint{1.525491in}{1.289366in}}%
\pgfpathlineto{\pgfqpoint{1.527325in}{1.364379in}}%
\pgfpathlineto{\pgfqpoint{1.529158in}{1.253961in}}%
\pgfpathlineto{\pgfqpoint{1.531015in}{1.482964in}}%
\pgfpathlineto{\pgfqpoint{1.532848in}{1.217289in}}%
\pgfpathlineto{\pgfqpoint{1.534681in}{1.226310in}}%
\pgfpathlineto{\pgfqpoint{1.536516in}{1.525370in}}%
\pgfpathlineto{\pgfqpoint{1.540182in}{1.090762in}}%
\pgfpathlineto{\pgfqpoint{1.542016in}{1.211468in}}%
\pgfpathlineto{\pgfqpoint{1.543849in}{1.068230in}}%
\pgfpathlineto{\pgfqpoint{1.547517in}{1.425528in}}%
\pgfpathlineto{\pgfqpoint{1.549349in}{1.303756in}}%
\pgfpathlineto{\pgfqpoint{1.551181in}{1.381981in}}%
\pgfpathlineto{\pgfqpoint{1.553015in}{1.334469in}}%
\pgfpathlineto{\pgfqpoint{1.554849in}{1.321509in}}%
\pgfpathlineto{\pgfqpoint{1.556681in}{1.274988in}}%
\pgfpathlineto{\pgfqpoint{1.558518in}{1.349713in}}%
\pgfpathlineto{\pgfqpoint{1.560351in}{1.174796in}}%
\pgfpathlineto{\pgfqpoint{1.562183in}{1.204392in}}%
\pgfpathlineto{\pgfqpoint{1.564017in}{1.295238in}}%
\pgfpathlineto{\pgfqpoint{1.565850in}{1.920482in}}%
\pgfpathlineto{\pgfqpoint{1.567684in}{2.088135in}}%
\pgfpathlineto{\pgfqpoint{1.569520in}{2.135358in}}%
\pgfpathlineto{\pgfqpoint{1.571352in}{2.117430in}}%
\pgfpathlineto{\pgfqpoint{1.575017in}{1.991731in}}%
\pgfpathlineto{\pgfqpoint{1.578712in}{2.385790in}}%
\pgfpathlineto{\pgfqpoint{1.580545in}{2.154278in}}%
\pgfpathlineto{\pgfqpoint{1.582379in}{2.229529in}}%
\pgfpathlineto{\pgfqpoint{1.584213in}{2.219003in}}%
\pgfpathlineto{\pgfqpoint{1.586048in}{2.173461in}}%
\pgfpathlineto{\pgfqpoint{1.589716in}{1.343314in}}%
\pgfpathlineto{\pgfqpoint{1.593382in}{1.132867in}}%
\pgfpathlineto{\pgfqpoint{1.595217in}{1.185999in}}%
\pgfpathlineto{\pgfqpoint{1.597050in}{1.140834in}}%
\pgfpathlineto{\pgfqpoint{1.598882in}{1.236234in}}%
\pgfpathlineto{\pgfqpoint{1.600716in}{2.220007in}}%
\pgfpathlineto{\pgfqpoint{1.602565in}{2.084020in}}%
\pgfpathlineto{\pgfqpoint{1.604399in}{2.099125in}}%
\pgfpathlineto{\pgfqpoint{1.606234in}{2.076555in}}%
\pgfpathlineto{\pgfqpoint{1.608066in}{2.084283in}}%
\pgfpathlineto{\pgfqpoint{1.609901in}{2.215302in}}%
\pgfpathlineto{\pgfqpoint{1.611736in}{2.072214in}}%
\pgfpathlineto{\pgfqpoint{1.613569in}{2.225351in}}%
\pgfpathlineto{\pgfqpoint{1.615402in}{2.118221in}}%
\pgfpathlineto{\pgfqpoint{1.617235in}{2.101007in}}%
\pgfpathlineto{\pgfqpoint{1.619070in}{2.177752in}}%
\pgfpathlineto{\pgfqpoint{1.620903in}{2.181954in}}%
\pgfpathlineto{\pgfqpoint{1.626403in}{1.094463in}}%
\pgfpathlineto{\pgfqpoint{1.628237in}{1.137283in}}%
\pgfpathlineto{\pgfqpoint{1.630070in}{1.216135in}}%
\pgfpathlineto{\pgfqpoint{1.631903in}{1.150996in}}%
\pgfpathlineto{\pgfqpoint{1.633737in}{1.274022in}}%
\pgfpathlineto{\pgfqpoint{1.635570in}{1.224716in}}%
\pgfpathlineto{\pgfqpoint{1.637403in}{1.308286in}}%
\pgfpathlineto{\pgfqpoint{1.639237in}{1.318925in}}%
\pgfpathlineto{\pgfqpoint{1.642905in}{0.961099in}}%
\pgfpathlineto{\pgfqpoint{1.644738in}{1.003404in}}%
\pgfpathlineto{\pgfqpoint{1.648404in}{1.246760in}}%
\pgfpathlineto{\pgfqpoint{1.650238in}{1.113270in}}%
\pgfpathlineto{\pgfqpoint{1.652073in}{1.205032in}}%
\pgfpathlineto{\pgfqpoint{1.653906in}{1.088203in}}%
\pgfpathlineto{\pgfqpoint{1.655738in}{1.330680in}}%
\pgfpathlineto{\pgfqpoint{1.659405in}{1.214930in}}%
\pgfpathlineto{\pgfqpoint{1.661238in}{1.358319in}}%
\pgfpathlineto{\pgfqpoint{1.663073in}{1.318448in}}%
\pgfpathlineto{\pgfqpoint{1.664906in}{1.330204in}}%
\pgfpathlineto{\pgfqpoint{1.666740in}{1.293419in}}%
\pgfpathlineto{\pgfqpoint{1.668574in}{1.187555in}}%
\pgfpathlineto{\pgfqpoint{1.670408in}{1.339111in}}%
\pgfpathlineto{\pgfqpoint{1.672240in}{1.212559in}}%
\pgfpathlineto{\pgfqpoint{1.674074in}{1.192486in}}%
\pgfpathlineto{\pgfqpoint{1.675908in}{1.292239in}}%
\pgfpathlineto{\pgfqpoint{1.677742in}{1.175862in}}%
\pgfpathlineto{\pgfqpoint{1.679575in}{1.524278in}}%
\pgfpathlineto{\pgfqpoint{1.681409in}{2.329019in}}%
\pgfpathlineto{\pgfqpoint{1.685075in}{1.986261in}}%
\pgfpathlineto{\pgfqpoint{1.686909in}{1.267536in}}%
\pgfpathlineto{\pgfqpoint{1.688743in}{1.353326in}}%
\pgfpathlineto{\pgfqpoint{1.690577in}{1.222195in}}%
\pgfpathlineto{\pgfqpoint{1.692411in}{1.172625in}}%
\pgfpathlineto{\pgfqpoint{1.694244in}{1.195020in}}%
\pgfpathlineto{\pgfqpoint{1.696076in}{1.164345in}}%
\pgfpathlineto{\pgfqpoint{1.697909in}{1.208507in}}%
\pgfpathlineto{\pgfqpoint{1.699744in}{1.574902in}}%
\pgfpathlineto{\pgfqpoint{1.701578in}{1.477419in}}%
\pgfpathlineto{\pgfqpoint{1.703412in}{1.219083in}}%
\pgfpathlineto{\pgfqpoint{1.705247in}{1.361268in}}%
\pgfpathlineto{\pgfqpoint{1.707080in}{1.268615in}}%
\pgfpathlineto{\pgfqpoint{1.710745in}{1.311786in}}%
\pgfpathlineto{\pgfqpoint{1.712579in}{1.307282in}}%
\pgfpathlineto{\pgfqpoint{1.714412in}{1.367729in}}%
\pgfpathlineto{\pgfqpoint{1.716246in}{1.267298in}}%
\pgfpathlineto{\pgfqpoint{1.718078in}{1.320154in}}%
\pgfpathlineto{\pgfqpoint{1.719912in}{1.241026in}}%
\pgfpathlineto{\pgfqpoint{1.721745in}{1.353326in}}%
\pgfpathlineto{\pgfqpoint{1.723579in}{1.252255in}}%
\pgfpathlineto{\pgfqpoint{1.725413in}{1.259682in}}%
\pgfpathlineto{\pgfqpoint{1.727245in}{1.202246in}}%
\pgfpathlineto{\pgfqpoint{1.729080in}{1.195396in}}%
\pgfpathlineto{\pgfqpoint{1.730913in}{1.052033in}}%
\pgfpathlineto{\pgfqpoint{1.732745in}{1.267624in}}%
\pgfpathlineto{\pgfqpoint{1.734578in}{1.143656in}}%
\pgfpathlineto{\pgfqpoint{1.738246in}{1.307784in}}%
\pgfpathlineto{\pgfqpoint{1.740080in}{1.190779in}}%
\pgfpathlineto{\pgfqpoint{1.741914in}{1.490718in}}%
\pgfpathlineto{\pgfqpoint{1.743747in}{1.342047in}}%
\pgfpathlineto{\pgfqpoint{1.745581in}{1.308900in}}%
\pgfpathlineto{\pgfqpoint{1.747414in}{1.156491in}}%
\pgfpathlineto{\pgfqpoint{1.749249in}{1.336138in}}%
\pgfpathlineto{\pgfqpoint{1.751084in}{1.295087in}}%
\pgfpathlineto{\pgfqpoint{1.752918in}{1.298650in}}%
\pgfpathlineto{\pgfqpoint{1.754750in}{1.439066in}}%
\pgfpathlineto{\pgfqpoint{1.756586in}{1.238856in}}%
\pgfpathlineto{\pgfqpoint{1.758418in}{1.359172in}}%
\pgfpathlineto{\pgfqpoint{1.762086in}{1.262304in}}%
\pgfpathlineto{\pgfqpoint{1.763919in}{1.389032in}}%
\pgfpathlineto{\pgfqpoint{1.765752in}{1.300645in}}%
\pgfpathlineto{\pgfqpoint{1.767584in}{1.365759in}}%
\pgfpathlineto{\pgfqpoint{1.769418in}{1.488547in}}%
\pgfpathlineto{\pgfqpoint{1.771251in}{1.353125in}}%
\pgfpathlineto{\pgfqpoint{1.773085in}{1.071454in}}%
\pgfpathlineto{\pgfqpoint{1.776754in}{1.204003in}}%
\pgfpathlineto{\pgfqpoint{1.778589in}{1.116595in}}%
\pgfpathlineto{\pgfqpoint{1.780423in}{1.329526in}}%
\pgfpathlineto{\pgfqpoint{1.784088in}{1.396785in}}%
\pgfpathlineto{\pgfqpoint{1.785923in}{1.105604in}}%
\pgfpathlineto{\pgfqpoint{1.791425in}{1.381128in}}%
\pgfpathlineto{\pgfqpoint{1.793259in}{1.368318in}}%
\pgfpathlineto{\pgfqpoint{1.795091in}{1.368958in}}%
\pgfpathlineto{\pgfqpoint{1.796923in}{1.396961in}}%
\pgfpathlineto{\pgfqpoint{1.804261in}{1.188759in}}%
\pgfpathlineto{\pgfqpoint{1.806094in}{1.199950in}}%
\pgfpathlineto{\pgfqpoint{1.807926in}{1.299616in}}%
\pgfpathlineto{\pgfqpoint{1.809760in}{1.281048in}}%
\pgfpathlineto{\pgfqpoint{1.811593in}{1.423697in}}%
\pgfpathlineto{\pgfqpoint{1.813426in}{1.365082in}}%
\pgfpathlineto{\pgfqpoint{1.815261in}{1.409156in}}%
\pgfpathlineto{\pgfqpoint{1.818926in}{1.198332in}}%
\pgfpathlineto{\pgfqpoint{1.820760in}{1.233950in}}%
\pgfpathlineto{\pgfqpoint{1.822594in}{1.157520in}}%
\pgfpathlineto{\pgfqpoint{1.824427in}{1.136016in}}%
\pgfpathlineto{\pgfqpoint{1.826262in}{1.382157in}}%
\pgfpathlineto{\pgfqpoint{1.828096in}{1.172475in}}%
\pgfpathlineto{\pgfqpoint{1.829929in}{1.186727in}}%
\pgfpathlineto{\pgfqpoint{1.831763in}{1.150055in}}%
\pgfpathlineto{\pgfqpoint{1.833596in}{1.177857in}}%
\pgfpathlineto{\pgfqpoint{1.835429in}{1.140005in}}%
\pgfpathlineto{\pgfqpoint{1.839098in}{1.297094in}}%
\pgfpathlineto{\pgfqpoint{1.840930in}{1.399458in}}%
\pgfpathlineto{\pgfqpoint{1.842762in}{1.623216in}}%
\pgfpathlineto{\pgfqpoint{1.844596in}{2.350285in}}%
\pgfpathlineto{\pgfqpoint{1.846431in}{2.251058in}}%
\pgfpathlineto{\pgfqpoint{1.848264in}{2.248950in}}%
\pgfpathlineto{\pgfqpoint{1.850099in}{2.485719in}}%
\pgfpathlineto{\pgfqpoint{1.853767in}{2.166297in}}%
\pgfpathlineto{\pgfqpoint{1.855603in}{2.253354in}}%
\pgfpathlineto{\pgfqpoint{1.857436in}{2.148080in}}%
\pgfpathlineto{\pgfqpoint{1.859269in}{2.115347in}}%
\pgfpathlineto{\pgfqpoint{1.861102in}{2.146311in}}%
\pgfpathlineto{\pgfqpoint{1.862937in}{2.134330in}}%
\pgfpathlineto{\pgfqpoint{1.864769in}{2.200284in}}%
\pgfpathlineto{\pgfqpoint{1.868437in}{1.999159in}}%
\pgfpathlineto{\pgfqpoint{1.870270in}{2.312383in}}%
\pgfpathlineto{\pgfqpoint{1.872104in}{1.429631in}}%
\pgfpathlineto{\pgfqpoint{1.873938in}{1.215495in}}%
\pgfpathlineto{\pgfqpoint{1.875771in}{1.158047in}}%
\pgfpathlineto{\pgfqpoint{1.877984in}{1.237212in}}%
\pgfpathlineto{\pgfqpoint{1.881649in}{1.528895in}}%
\pgfpathlineto{\pgfqpoint{1.883483in}{2.110969in}}%
\pgfpathlineto{\pgfqpoint{1.885316in}{2.210547in}}%
\pgfpathlineto{\pgfqpoint{1.887148in}{2.115573in}}%
\pgfpathlineto{\pgfqpoint{1.888982in}{2.242778in}}%
\pgfpathlineto{\pgfqpoint{1.890815in}{2.198967in}}%
\pgfpathlineto{\pgfqpoint{1.894481in}{2.279011in}}%
\pgfpathlineto{\pgfqpoint{1.896315in}{2.248536in}}%
\pgfpathlineto{\pgfqpoint{1.898148in}{2.142999in}}%
\pgfpathlineto{\pgfqpoint{1.899983in}{2.221060in}}%
\pgfpathlineto{\pgfqpoint{1.901816in}{2.358101in}}%
\pgfpathlineto{\pgfqpoint{1.903651in}{2.255035in}}%
\pgfpathlineto{\pgfqpoint{1.907317in}{2.443075in}}%
\pgfpathlineto{\pgfqpoint{1.909150in}{2.615270in}}%
\pgfpathlineto{\pgfqpoint{1.910984in}{2.267606in}}%
\pgfpathlineto{\pgfqpoint{1.912818in}{2.255060in}}%
\pgfpathlineto{\pgfqpoint{1.916485in}{2.465708in}}%
\pgfpathlineto{\pgfqpoint{1.920151in}{2.112261in}}%
\pgfpathlineto{\pgfqpoint{1.921984in}{2.204136in}}%
\pgfpathlineto{\pgfqpoint{1.923819in}{2.207134in}}%
\pgfpathlineto{\pgfqpoint{1.925653in}{2.334652in}}%
\pgfpathlineto{\pgfqpoint{1.927486in}{1.253836in}}%
\pgfpathlineto{\pgfqpoint{1.929321in}{1.344079in}}%
\pgfpathlineto{\pgfqpoint{1.931154in}{1.131248in}}%
\pgfpathlineto{\pgfqpoint{1.932987in}{1.597384in}}%
\pgfpathlineto{\pgfqpoint{1.934824in}{1.776378in}}%
\pgfpathlineto{\pgfqpoint{1.936657in}{1.695105in}}%
\pgfpathlineto{\pgfqpoint{1.938490in}{1.516613in}}%
\pgfpathlineto{\pgfqpoint{1.940325in}{1.617219in}}%
\pgfpathlineto{\pgfqpoint{1.942159in}{1.541855in}}%
\pgfpathlineto{\pgfqpoint{1.945827in}{1.306052in}}%
\pgfpathlineto{\pgfqpoint{1.947660in}{1.349625in}}%
\pgfpathlineto{\pgfqpoint{1.949493in}{1.056035in}}%
\pgfpathlineto{\pgfqpoint{1.951341in}{1.290269in}}%
\pgfpathlineto{\pgfqpoint{1.953174in}{1.252845in}}%
\pgfpathlineto{\pgfqpoint{1.955006in}{1.070965in}}%
\pgfpathlineto{\pgfqpoint{1.956840in}{1.311874in}}%
\pgfpathlineto{\pgfqpoint{1.960506in}{1.188935in}}%
\pgfpathlineto{\pgfqpoint{1.962341in}{1.290771in}}%
\pgfpathlineto{\pgfqpoint{1.964174in}{1.229798in}}%
\pgfpathlineto{\pgfqpoint{1.966007in}{1.209937in}}%
\pgfpathlineto{\pgfqpoint{1.967841in}{1.210916in}}%
\pgfpathlineto{\pgfqpoint{1.971508in}{2.044324in}}%
\pgfpathlineto{\pgfqpoint{1.973343in}{2.051413in}}%
\pgfpathlineto{\pgfqpoint{1.977011in}{2.339031in}}%
\pgfpathlineto{\pgfqpoint{1.978845in}{2.248361in}}%
\pgfpathlineto{\pgfqpoint{1.980679in}{1.942099in}}%
\pgfpathlineto{\pgfqpoint{1.982512in}{2.126012in}}%
\pgfpathlineto{\pgfqpoint{1.984346in}{2.094283in}}%
\pgfpathlineto{\pgfqpoint{1.986181in}{2.142372in}}%
\pgfpathlineto{\pgfqpoint{1.988015in}{2.394158in}}%
\pgfpathlineto{\pgfqpoint{1.989847in}{2.089189in}}%
\pgfpathlineto{\pgfqpoint{1.991680in}{2.035868in}}%
\pgfpathlineto{\pgfqpoint{1.995348in}{2.190862in}}%
\pgfpathlineto{\pgfqpoint{1.997182in}{2.178981in}}%
\pgfpathlineto{\pgfqpoint{1.999016in}{2.002132in}}%
\pgfpathlineto{\pgfqpoint{2.000873in}{2.293903in}}%
\pgfpathlineto{\pgfqpoint{2.002706in}{1.409507in}}%
\pgfpathlineto{\pgfqpoint{2.004539in}{1.377778in}}%
\pgfpathlineto{\pgfqpoint{2.006374in}{1.125051in}}%
\pgfpathlineto{\pgfqpoint{2.008208in}{1.186727in}}%
\pgfpathlineto{\pgfqpoint{2.010042in}{1.146391in}}%
\pgfpathlineto{\pgfqpoint{2.011876in}{1.228330in}}%
\pgfpathlineto{\pgfqpoint{2.013709in}{1.123733in}}%
\pgfpathlineto{\pgfqpoint{2.015542in}{1.282014in}}%
\pgfpathlineto{\pgfqpoint{2.017375in}{0.981963in}}%
\pgfpathlineto{\pgfqpoint{2.019208in}{1.294886in}}%
\pgfpathlineto{\pgfqpoint{2.022876in}{1.117849in}}%
\pgfpathlineto{\pgfqpoint{2.024711in}{1.052033in}}%
\pgfpathlineto{\pgfqpoint{2.026543in}{1.145149in}}%
\pgfpathlineto{\pgfqpoint{2.028377in}{1.134397in}}%
\pgfpathlineto{\pgfqpoint{2.030211in}{1.103923in}}%
\pgfpathlineto{\pgfqpoint{2.032044in}{1.309227in}}%
\pgfpathlineto{\pgfqpoint{2.033877in}{1.243849in}}%
\pgfpathlineto{\pgfqpoint{2.035712in}{1.300093in}}%
\pgfpathlineto{\pgfqpoint{2.039381in}{1.157143in}}%
\pgfpathlineto{\pgfqpoint{2.041214in}{1.375984in}}%
\pgfpathlineto{\pgfqpoint{2.043048in}{2.166347in}}%
\pgfpathlineto{\pgfqpoint{2.044882in}{1.992083in}}%
\pgfpathlineto{\pgfqpoint{2.046716in}{2.385577in}}%
\pgfpathlineto{\pgfqpoint{2.050384in}{2.229491in}}%
\pgfpathlineto{\pgfqpoint{2.054049in}{2.055252in}}%
\pgfpathlineto{\pgfqpoint{2.055883in}{2.069216in}}%
\pgfpathlineto{\pgfqpoint{2.057717in}{2.196671in}}%
\pgfpathlineto{\pgfqpoint{2.059552in}{1.349775in}}%
\pgfpathlineto{\pgfqpoint{2.061385in}{1.258980in}}%
\pgfpathlineto{\pgfqpoint{2.063219in}{1.267586in}}%
\pgfpathlineto{\pgfqpoint{2.065051in}{1.313580in}}%
\pgfpathlineto{\pgfqpoint{2.066885in}{1.268477in}}%
\pgfpathlineto{\pgfqpoint{2.068720in}{1.133569in}}%
\pgfpathlineto{\pgfqpoint{2.070554in}{1.164809in}}%
\pgfpathlineto{\pgfqpoint{2.072387in}{1.166453in}}%
\pgfpathlineto{\pgfqpoint{2.074222in}{1.187818in}}%
\pgfpathlineto{\pgfqpoint{2.076056in}{1.271827in}}%
\pgfpathlineto{\pgfqpoint{2.077889in}{1.118062in}}%
\pgfpathlineto{\pgfqpoint{2.079722in}{1.189725in}}%
\pgfpathlineto{\pgfqpoint{2.081556in}{0.964386in}}%
\pgfpathlineto{\pgfqpoint{2.083390in}{1.215257in}}%
\pgfpathlineto{\pgfqpoint{2.085224in}{1.270472in}}%
\pgfpathlineto{\pgfqpoint{2.087060in}{1.137484in}}%
\pgfpathlineto{\pgfqpoint{2.088893in}{1.109217in}}%
\pgfpathlineto{\pgfqpoint{2.090725in}{1.192511in}}%
\pgfpathlineto{\pgfqpoint{2.092559in}{1.181470in}}%
\pgfpathlineto{\pgfqpoint{2.094392in}{1.283131in}}%
\pgfpathlineto{\pgfqpoint{2.096225in}{1.178534in}}%
\pgfpathlineto{\pgfqpoint{2.098059in}{1.221894in}}%
\pgfpathlineto{\pgfqpoint{2.099907in}{1.421238in}}%
\pgfpathlineto{\pgfqpoint{2.101741in}{1.193903in}}%
\pgfpathlineto{\pgfqpoint{2.103575in}{1.272002in}}%
\pgfpathlineto{\pgfqpoint{2.105407in}{1.269945in}}%
\pgfpathlineto{\pgfqpoint{2.109075in}{1.055157in}}%
\pgfpathlineto{\pgfqpoint{2.110908in}{1.064315in}}%
\pgfpathlineto{\pgfqpoint{2.112743in}{1.028672in}}%
\pgfpathlineto{\pgfqpoint{2.114577in}{1.128701in}}%
\pgfpathlineto{\pgfqpoint{2.116411in}{1.030943in}}%
\pgfpathlineto{\pgfqpoint{2.118245in}{1.024005in}}%
\pgfpathlineto{\pgfqpoint{2.121912in}{1.177029in}}%
\pgfpathlineto{\pgfqpoint{2.125580in}{1.027530in}}%
\pgfpathlineto{\pgfqpoint{2.127413in}{1.272906in}}%
\pgfpathlineto{\pgfqpoint{2.131080in}{1.159577in}}%
\pgfpathlineto{\pgfqpoint{2.132913in}{1.205709in}}%
\pgfpathlineto{\pgfqpoint{2.136582in}{1.199599in}}%
\pgfpathlineto{\pgfqpoint{2.138416in}{1.144447in}}%
\pgfpathlineto{\pgfqpoint{2.140250in}{1.135514in}}%
\pgfpathlineto{\pgfqpoint{2.142083in}{1.154910in}}%
\pgfpathlineto{\pgfqpoint{2.143917in}{1.290219in}}%
\pgfpathlineto{\pgfqpoint{2.145750in}{1.320656in}}%
\pgfpathlineto{\pgfqpoint{2.147586in}{1.234891in}}%
\pgfpathlineto{\pgfqpoint{2.149421in}{1.246634in}}%
\pgfpathlineto{\pgfqpoint{2.153089in}{1.157821in}}%
\pgfpathlineto{\pgfqpoint{2.154922in}{1.071454in}}%
\pgfpathlineto{\pgfqpoint{2.156755in}{1.217753in}}%
\pgfpathlineto{\pgfqpoint{2.158588in}{1.094639in}}%
\pgfpathlineto{\pgfqpoint{2.160423in}{1.142239in}}%
\pgfpathlineto{\pgfqpoint{2.162254in}{1.275703in}}%
\pgfpathlineto{\pgfqpoint{2.164087in}{1.283432in}}%
\pgfpathlineto{\pgfqpoint{2.167753in}{1.059937in}}%
\pgfpathlineto{\pgfqpoint{2.169586in}{1.231240in}}%
\pgfpathlineto{\pgfqpoint{2.171421in}{1.087852in}}%
\pgfpathlineto{\pgfqpoint{2.173255in}{1.199775in}}%
\pgfpathlineto{\pgfqpoint{2.175088in}{1.118000in}}%
\pgfpathlineto{\pgfqpoint{2.178756in}{1.935839in}}%
\pgfpathlineto{\pgfqpoint{2.180589in}{2.126338in}}%
\pgfpathlineto{\pgfqpoint{2.182425in}{2.032305in}}%
\pgfpathlineto{\pgfqpoint{2.184258in}{2.207373in}}%
\pgfpathlineto{\pgfqpoint{2.186092in}{2.025167in}}%
\pgfpathlineto{\pgfqpoint{2.187926in}{1.065030in}}%
\pgfpathlineto{\pgfqpoint{2.189760in}{1.289805in}}%
\pgfpathlineto{\pgfqpoint{2.191595in}{1.217251in}}%
\pgfpathlineto{\pgfqpoint{2.193430in}{1.218544in}}%
\pgfpathlineto{\pgfqpoint{2.195262in}{1.127196in}}%
\pgfpathlineto{\pgfqpoint{2.197096in}{1.322387in}}%
\pgfpathlineto{\pgfqpoint{2.198950in}{1.382684in}}%
\pgfpathlineto{\pgfqpoint{2.200782in}{1.168071in}}%
\pgfpathlineto{\pgfqpoint{2.202617in}{1.245054in}}%
\pgfpathlineto{\pgfqpoint{2.204450in}{1.213086in}}%
\pgfpathlineto{\pgfqpoint{2.206283in}{1.372283in}}%
\pgfpathlineto{\pgfqpoint{2.208117in}{1.284661in}}%
\pgfpathlineto{\pgfqpoint{2.209951in}{1.447936in}}%
\pgfpathlineto{\pgfqpoint{2.211786in}{1.227778in}}%
\pgfpathlineto{\pgfqpoint{2.213619in}{1.462640in}}%
\pgfpathlineto{\pgfqpoint{2.219122in}{1.289391in}}%
\pgfpathlineto{\pgfqpoint{2.220957in}{1.236146in}}%
\pgfpathlineto{\pgfqpoint{2.222791in}{1.074164in}}%
\pgfpathlineto{\pgfqpoint{2.224623in}{1.046537in}}%
\pgfpathlineto{\pgfqpoint{2.226457in}{1.119242in}}%
\pgfpathlineto{\pgfqpoint{2.228291in}{0.930863in}}%
\pgfpathlineto{\pgfqpoint{2.230125in}{1.210201in}}%
\pgfpathlineto{\pgfqpoint{2.231958in}{1.108038in}}%
\pgfpathlineto{\pgfqpoint{2.233792in}{1.210966in}}%
\pgfpathlineto{\pgfqpoint{2.237460in}{1.083335in}}%
\pgfpathlineto{\pgfqpoint{2.239295in}{1.061643in}}%
\pgfpathlineto{\pgfqpoint{2.241128in}{1.072520in}}%
\pgfpathlineto{\pgfqpoint{2.242963in}{1.029174in}}%
\pgfpathlineto{\pgfqpoint{2.246630in}{1.215495in}}%
\pgfpathlineto{\pgfqpoint{2.248478in}{0.970145in}}%
\pgfpathlineto{\pgfqpoint{2.250312in}{1.126255in}}%
\pgfpathlineto{\pgfqpoint{2.252145in}{1.032147in}}%
\pgfpathlineto{\pgfqpoint{2.253979in}{1.223161in}}%
\pgfpathlineto{\pgfqpoint{2.255812in}{1.093911in}}%
\pgfpathlineto{\pgfqpoint{2.257648in}{1.120710in}}%
\pgfpathlineto{\pgfqpoint{2.261314in}{0.861283in}}%
\pgfpathlineto{\pgfqpoint{2.263148in}{0.931867in}}%
\pgfpathlineto{\pgfqpoint{2.264982in}{1.100548in}}%
\pgfpathlineto{\pgfqpoint{2.266816in}{1.111212in}}%
\pgfpathlineto{\pgfqpoint{2.268648in}{0.909271in}}%
\pgfpathlineto{\pgfqpoint{2.270482in}{0.943171in}}%
\pgfpathlineto{\pgfqpoint{2.272317in}{1.219108in}}%
\pgfpathlineto{\pgfqpoint{2.274151in}{1.189462in}}%
\pgfpathlineto{\pgfqpoint{2.275986in}{1.033640in}}%
\pgfpathlineto{\pgfqpoint{2.277818in}{1.310807in}}%
\pgfpathlineto{\pgfqpoint{2.279652in}{0.981549in}}%
\pgfpathlineto{\pgfqpoint{2.281486in}{0.945316in}}%
\pgfpathlineto{\pgfqpoint{2.285153in}{0.801024in}}%
\pgfpathlineto{\pgfqpoint{2.286987in}{0.983983in}}%
\pgfpathlineto{\pgfqpoint{2.288821in}{0.738821in}}%
\pgfpathlineto{\pgfqpoint{2.290654in}{0.790410in}}%
\pgfpathlineto{\pgfqpoint{2.292489in}{0.986342in}}%
\pgfpathlineto{\pgfqpoint{2.294323in}{1.005499in}}%
\pgfpathlineto{\pgfqpoint{2.296157in}{0.935743in}}%
\pgfpathlineto{\pgfqpoint{2.298007in}{0.963395in}}%
\pgfpathlineto{\pgfqpoint{2.299841in}{0.803746in}}%
\pgfpathlineto{\pgfqpoint{2.301675in}{0.839716in}}%
\pgfpathlineto{\pgfqpoint{2.305342in}{0.944613in}}%
\pgfpathlineto{\pgfqpoint{2.307176in}{0.989014in}}%
\pgfpathlineto{\pgfqpoint{2.309010in}{0.999000in}}%
\pgfpathlineto{\pgfqpoint{2.310843in}{1.024005in}}%
\pgfpathlineto{\pgfqpoint{2.312677in}{1.002288in}}%
\pgfpathlineto{\pgfqpoint{2.316343in}{0.874569in}}%
\pgfpathlineto{\pgfqpoint{2.320010in}{1.113445in}}%
\pgfpathlineto{\pgfqpoint{2.323678in}{1.022035in}}%
\pgfpathlineto{\pgfqpoint{2.325512in}{1.086973in}}%
\pgfpathlineto{\pgfqpoint{2.327345in}{1.044480in}}%
\pgfpathlineto{\pgfqpoint{2.329179in}{0.861345in}}%
\pgfpathlineto{\pgfqpoint{2.331014in}{0.815126in}}%
\pgfpathlineto{\pgfqpoint{2.332847in}{0.970822in}}%
\pgfpathlineto{\pgfqpoint{2.334681in}{0.995563in}}%
\pgfpathlineto{\pgfqpoint{2.336516in}{0.912408in}}%
\pgfpathlineto{\pgfqpoint{2.338350in}{1.038232in}}%
\pgfpathlineto{\pgfqpoint{2.340183in}{1.060288in}}%
\pgfpathlineto{\pgfqpoint{2.342017in}{0.985752in}}%
\pgfpathlineto{\pgfqpoint{2.343851in}{1.101602in}}%
\pgfpathlineto{\pgfqpoint{2.345685in}{1.035434in}}%
\pgfpathlineto{\pgfqpoint{2.347518in}{1.112944in}}%
\pgfpathlineto{\pgfqpoint{2.349353in}{1.019890in}}%
\pgfpathlineto{\pgfqpoint{2.351186in}{1.018422in}}%
\pgfpathlineto{\pgfqpoint{2.353019in}{0.900564in}}%
\pgfpathlineto{\pgfqpoint{2.354854in}{1.045132in}}%
\pgfpathlineto{\pgfqpoint{2.356687in}{1.070162in}}%
\pgfpathlineto{\pgfqpoint{2.358520in}{1.025623in}}%
\pgfpathlineto{\pgfqpoint{2.360354in}{1.040202in}}%
\pgfpathlineto{\pgfqpoint{2.362187in}{0.987308in}}%
\pgfpathlineto{\pgfqpoint{2.364021in}{0.991385in}}%
\pgfpathlineto{\pgfqpoint{2.365856in}{0.912056in}}%
\pgfpathlineto{\pgfqpoint{2.369522in}{0.871771in}}%
\pgfpathlineto{\pgfqpoint{2.371356in}{0.873389in}}%
\pgfpathlineto{\pgfqpoint{2.373192in}{1.075280in}}%
\pgfpathlineto{\pgfqpoint{2.375026in}{0.964913in}}%
\pgfpathlineto{\pgfqpoint{2.376861in}{0.979291in}}%
\pgfpathlineto{\pgfqpoint{2.378694in}{1.046011in}}%
\pgfpathlineto{\pgfqpoint{2.380528in}{0.849941in}}%
\pgfpathlineto{\pgfqpoint{2.382365in}{0.875648in}}%
\pgfpathlineto{\pgfqpoint{2.386030in}{0.971964in}}%
\pgfpathlineto{\pgfqpoint{2.389698in}{0.835137in}}%
\pgfpathlineto{\pgfqpoint{2.391531in}{0.648878in}}%
\pgfpathlineto{\pgfqpoint{2.393366in}{0.690744in}}%
\pgfpathlineto{\pgfqpoint{2.397033in}{1.258428in}}%
\pgfpathlineto{\pgfqpoint{2.400702in}{0.871595in}}%
\pgfpathlineto{\pgfqpoint{2.402535in}{1.079358in}}%
\pgfpathlineto{\pgfqpoint{2.404369in}{0.981925in}}%
\pgfpathlineto{\pgfqpoint{2.406204in}{1.019363in}}%
\pgfpathlineto{\pgfqpoint{2.408037in}{0.916962in}}%
\pgfpathlineto{\pgfqpoint{2.409871in}{0.967974in}}%
\pgfpathlineto{\pgfqpoint{2.411705in}{1.051242in}}%
\pgfpathlineto{\pgfqpoint{2.415371in}{0.880240in}}%
\pgfpathlineto{\pgfqpoint{2.417205in}{0.915645in}}%
\pgfpathlineto{\pgfqpoint{2.419039in}{1.045596in}}%
\pgfpathlineto{\pgfqpoint{2.420872in}{0.994622in}}%
\pgfpathlineto{\pgfqpoint{2.424540in}{1.062320in}}%
\pgfpathlineto{\pgfqpoint{2.426374in}{0.912822in}}%
\pgfpathlineto{\pgfqpoint{2.428209in}{1.008586in}}%
\pgfpathlineto{\pgfqpoint{2.430043in}{1.014457in}}%
\pgfpathlineto{\pgfqpoint{2.431877in}{1.037793in}}%
\pgfpathlineto{\pgfqpoint{2.433711in}{0.772983in}}%
\pgfpathlineto{\pgfqpoint{2.435544in}{0.991385in}}%
\pgfpathlineto{\pgfqpoint{2.437378in}{0.962630in}}%
\pgfpathlineto{\pgfqpoint{2.439213in}{1.210765in}}%
\pgfpathlineto{\pgfqpoint{2.441046in}{1.815610in}}%
\pgfpathlineto{\pgfqpoint{2.442879in}{1.656739in}}%
\pgfpathlineto{\pgfqpoint{2.444714in}{1.981004in}}%
\pgfpathlineto{\pgfqpoint{2.446550in}{1.928411in}}%
\pgfpathlineto{\pgfqpoint{2.448385in}{1.950154in}}%
\pgfpathlineto{\pgfqpoint{2.450220in}{1.290621in}}%
\pgfpathlineto{\pgfqpoint{2.452054in}{1.294096in}}%
\pgfpathlineto{\pgfqpoint{2.453888in}{1.511406in}}%
\pgfpathlineto{\pgfqpoint{2.455723in}{1.110773in}}%
\pgfpathlineto{\pgfqpoint{2.459389in}{1.204304in}}%
\pgfpathlineto{\pgfqpoint{2.461223in}{1.071078in}}%
\pgfpathlineto{\pgfqpoint{2.463057in}{1.193953in}}%
\pgfpathlineto{\pgfqpoint{2.464891in}{1.058619in}}%
\pgfpathlineto{\pgfqpoint{2.466724in}{1.107097in}}%
\pgfpathlineto{\pgfqpoint{2.468558in}{1.103522in}}%
\pgfpathlineto{\pgfqpoint{2.470392in}{1.147320in}}%
\pgfpathlineto{\pgfqpoint{2.474059in}{2.006511in}}%
\pgfpathlineto{\pgfqpoint{2.475893in}{2.000049in}}%
\pgfpathlineto{\pgfqpoint{2.477727in}{1.851692in}}%
\pgfpathlineto{\pgfqpoint{2.479561in}{1.029036in}}%
\pgfpathlineto{\pgfqpoint{2.481394in}{1.026715in}}%
\pgfpathlineto{\pgfqpoint{2.483228in}{0.866866in}}%
\pgfpathlineto{\pgfqpoint{2.485062in}{0.929897in}}%
\pgfpathlineto{\pgfqpoint{2.486896in}{1.048156in}}%
\pgfpathlineto{\pgfqpoint{2.488729in}{0.928918in}}%
\pgfpathlineto{\pgfqpoint{2.490563in}{0.961889in}}%
\pgfpathlineto{\pgfqpoint{2.492397in}{0.857995in}}%
\pgfpathlineto{\pgfqpoint{2.494230in}{0.893375in}}%
\pgfpathlineto{\pgfqpoint{2.496065in}{0.872009in}}%
\pgfpathlineto{\pgfqpoint{2.497897in}{1.098641in}}%
\pgfpathlineto{\pgfqpoint{2.499731in}{0.992953in}}%
\pgfpathlineto{\pgfqpoint{2.503403in}{1.168096in}}%
\pgfpathlineto{\pgfqpoint{2.505236in}{1.478849in}}%
\pgfpathlineto{\pgfqpoint{2.507071in}{1.334406in}}%
\pgfpathlineto{\pgfqpoint{2.508905in}{1.393523in}}%
\pgfpathlineto{\pgfqpoint{2.510738in}{1.348333in}}%
\pgfpathlineto{\pgfqpoint{2.512572in}{1.338898in}}%
\pgfpathlineto{\pgfqpoint{2.514406in}{1.346752in}}%
\pgfpathlineto{\pgfqpoint{2.518075in}{1.100398in}}%
\pgfpathlineto{\pgfqpoint{2.519909in}{1.250461in}}%
\pgfpathlineto{\pgfqpoint{2.521742in}{1.113245in}}%
\pgfpathlineto{\pgfqpoint{2.523576in}{1.221718in}}%
\pgfpathlineto{\pgfqpoint{2.527245in}{1.021332in}}%
\pgfpathlineto{\pgfqpoint{2.529079in}{1.287773in}}%
\pgfpathlineto{\pgfqpoint{2.530912in}{1.257775in}}%
\pgfpathlineto{\pgfqpoint{2.532745in}{1.174118in}}%
\pgfpathlineto{\pgfqpoint{2.534578in}{1.298738in}}%
\pgfpathlineto{\pgfqpoint{2.536412in}{1.293795in}}%
\pgfpathlineto{\pgfqpoint{2.538246in}{1.268301in}}%
\pgfpathlineto{\pgfqpoint{2.540080in}{1.260008in}}%
\pgfpathlineto{\pgfqpoint{2.541915in}{1.242444in}}%
\pgfpathlineto{\pgfqpoint{2.547418in}{2.259941in}}%
\pgfpathlineto{\pgfqpoint{2.549252in}{2.315319in}}%
\pgfpathlineto{\pgfqpoint{2.551087in}{2.307653in}}%
\pgfpathlineto{\pgfqpoint{2.552922in}{2.231060in}}%
\pgfpathlineto{\pgfqpoint{2.554755in}{2.348641in}}%
\pgfpathlineto{\pgfqpoint{2.556590in}{2.236868in}}%
\pgfpathlineto{\pgfqpoint{2.558425in}{2.250795in}}%
\pgfpathlineto{\pgfqpoint{2.560258in}{2.104909in}}%
\pgfpathlineto{\pgfqpoint{2.562091in}{2.130089in}}%
\pgfpathlineto{\pgfqpoint{2.563926in}{2.072854in}}%
\pgfpathlineto{\pgfqpoint{2.565761in}{2.329609in}}%
\pgfpathlineto{\pgfqpoint{2.567594in}{2.319472in}}%
\pgfpathlineto{\pgfqpoint{2.569431in}{2.393305in}}%
\pgfpathlineto{\pgfqpoint{2.571263in}{2.301079in}}%
\pgfpathlineto{\pgfqpoint{2.573096in}{2.430354in}}%
\pgfpathlineto{\pgfqpoint{2.574931in}{2.383406in}}%
\pgfpathlineto{\pgfqpoint{2.576766in}{2.444104in}}%
\pgfpathlineto{\pgfqpoint{2.578600in}{2.332043in}}%
\pgfpathlineto{\pgfqpoint{2.580434in}{2.401736in}}%
\pgfpathlineto{\pgfqpoint{2.584101in}{2.219856in}}%
\pgfpathlineto{\pgfqpoint{2.585936in}{2.234610in}}%
\pgfpathlineto{\pgfqpoint{2.589603in}{2.443878in}}%
\pgfpathlineto{\pgfqpoint{2.591437in}{2.428597in}}%
\pgfpathlineto{\pgfqpoint{2.595105in}{2.225941in}}%
\pgfpathlineto{\pgfqpoint{2.596940in}{2.246805in}}%
\pgfpathlineto{\pgfqpoint{2.598773in}{2.450189in}}%
\pgfpathlineto{\pgfqpoint{2.600607in}{2.379843in}}%
\pgfpathlineto{\pgfqpoint{2.602442in}{2.401799in}}%
\pgfpathlineto{\pgfqpoint{2.604276in}{2.479835in}}%
\pgfpathlineto{\pgfqpoint{2.606110in}{2.304981in}}%
\pgfpathlineto{\pgfqpoint{2.607943in}{2.478756in}}%
\pgfpathlineto{\pgfqpoint{2.611611in}{1.396873in}}%
\pgfpathlineto{\pgfqpoint{2.613445in}{1.390788in}}%
\pgfpathlineto{\pgfqpoint{2.615280in}{1.459189in}}%
\pgfpathlineto{\pgfqpoint{2.617113in}{1.454823in}}%
\pgfpathlineto{\pgfqpoint{2.618945in}{1.555894in}}%
\pgfpathlineto{\pgfqpoint{2.620781in}{1.540525in}}%
\pgfpathlineto{\pgfqpoint{2.622614in}{1.455702in}}%
\pgfpathlineto{\pgfqpoint{2.624446in}{1.550838in}}%
\pgfpathlineto{\pgfqpoint{2.626282in}{1.843588in}}%
\pgfpathlineto{\pgfqpoint{2.628117in}{1.812938in}}%
\pgfpathlineto{\pgfqpoint{2.629951in}{1.620895in}}%
\pgfpathlineto{\pgfqpoint{2.633620in}{1.527365in}}%
\pgfpathlineto{\pgfqpoint{2.637287in}{1.633027in}}%
\pgfpathlineto{\pgfqpoint{2.639121in}{1.552544in}}%
\pgfpathlineto{\pgfqpoint{2.640953in}{1.653917in}}%
\pgfpathlineto{\pgfqpoint{2.642788in}{1.496715in}}%
\pgfpathlineto{\pgfqpoint{2.646455in}{1.703511in}}%
\pgfpathlineto{\pgfqpoint{2.648288in}{1.722380in}}%
\pgfpathlineto{\pgfqpoint{2.650122in}{1.792488in}}%
\pgfpathlineto{\pgfqpoint{2.657458in}{1.602553in}}%
\pgfpathlineto{\pgfqpoint{2.659291in}{1.605614in}}%
\pgfpathlineto{\pgfqpoint{2.661125in}{1.775714in}}%
\pgfpathlineto{\pgfqpoint{2.662959in}{1.770858in}}%
\pgfpathlineto{\pgfqpoint{2.664792in}{1.387326in}}%
\pgfpathlineto{\pgfqpoint{2.666626in}{1.444498in}}%
\pgfpathlineto{\pgfqpoint{2.668460in}{1.272994in}}%
\pgfpathlineto{\pgfqpoint{2.670294in}{1.391378in}}%
\pgfpathlineto{\pgfqpoint{2.672129in}{1.301147in}}%
\pgfpathlineto{\pgfqpoint{2.675797in}{1.282077in}}%
\pgfpathlineto{\pgfqpoint{2.677631in}{1.327180in}}%
\pgfpathlineto{\pgfqpoint{2.679465in}{1.298587in}}%
\pgfpathlineto{\pgfqpoint{2.681298in}{1.296530in}}%
\pgfpathlineto{\pgfqpoint{2.683133in}{1.372433in}}%
\pgfpathlineto{\pgfqpoint{2.684967in}{1.294008in}}%
\pgfpathlineto{\pgfqpoint{2.686800in}{1.320305in}}%
\pgfpathlineto{\pgfqpoint{2.688636in}{1.226636in}}%
\pgfpathlineto{\pgfqpoint{2.690470in}{1.308173in}}%
\pgfpathlineto{\pgfqpoint{2.692303in}{1.294385in}}%
\pgfpathlineto{\pgfqpoint{2.694139in}{1.293481in}}%
\pgfpathlineto{\pgfqpoint{2.695974in}{1.375959in}}%
\pgfpathlineto{\pgfqpoint{2.697807in}{1.311610in}}%
\pgfpathlineto{\pgfqpoint{2.699642in}{1.287396in}}%
\pgfpathlineto{\pgfqpoint{2.701476in}{1.193226in}}%
\pgfpathlineto{\pgfqpoint{2.703309in}{1.280396in}}%
\pgfpathlineto{\pgfqpoint{2.705144in}{1.439856in}}%
\pgfpathlineto{\pgfqpoint{2.706978in}{1.375921in}}%
\pgfpathlineto{\pgfqpoint{2.708810in}{1.504656in}}%
\pgfpathlineto{\pgfqpoint{2.710645in}{1.347893in}}%
\pgfpathlineto{\pgfqpoint{2.712479in}{1.353301in}}%
\pgfpathlineto{\pgfqpoint{2.714312in}{1.232119in}}%
\pgfpathlineto{\pgfqpoint{2.716146in}{1.324244in}}%
\pgfpathlineto{\pgfqpoint{2.719814in}{1.198395in}}%
\pgfpathlineto{\pgfqpoint{2.721649in}{1.311221in}}%
\pgfpathlineto{\pgfqpoint{2.723486in}{1.338434in}}%
\pgfpathlineto{\pgfqpoint{2.725319in}{1.394088in}}%
\pgfpathlineto{\pgfqpoint{2.727153in}{1.236999in}}%
\pgfpathlineto{\pgfqpoint{2.728986in}{1.307934in}}%
\pgfpathlineto{\pgfqpoint{2.730821in}{1.280433in}}%
\pgfpathlineto{\pgfqpoint{2.732655in}{1.329940in}}%
\pgfpathlineto{\pgfqpoint{2.734489in}{1.281424in}}%
\pgfpathlineto{\pgfqpoint{2.736323in}{1.154584in}}%
\pgfpathlineto{\pgfqpoint{2.738156in}{1.132666in}}%
\pgfpathlineto{\pgfqpoint{2.739990in}{1.187254in}}%
\pgfpathlineto{\pgfqpoint{2.741824in}{1.146028in}}%
\pgfpathlineto{\pgfqpoint{2.743657in}{1.356763in}}%
\pgfpathlineto{\pgfqpoint{2.747324in}{2.163537in}}%
\pgfpathlineto{\pgfqpoint{2.749158in}{2.217748in}}%
\pgfpathlineto{\pgfqpoint{2.750992in}{2.141029in}}%
\pgfpathlineto{\pgfqpoint{2.754676in}{2.262789in}}%
\pgfpathlineto{\pgfqpoint{2.756510in}{2.219982in}}%
\pgfpathlineto{\pgfqpoint{2.758345in}{2.138558in}}%
\pgfpathlineto{\pgfqpoint{2.762014in}{2.354638in}}%
\pgfpathlineto{\pgfqpoint{2.763847in}{2.189005in}}%
\pgfpathlineto{\pgfqpoint{2.765683in}{2.189306in}}%
\pgfpathlineto{\pgfqpoint{2.767516in}{2.292171in}}%
\pgfpathlineto{\pgfqpoint{2.769350in}{2.295082in}}%
\pgfpathlineto{\pgfqpoint{2.771184in}{2.274657in}}%
\pgfpathlineto{\pgfqpoint{2.773018in}{2.351263in}}%
\pgfpathlineto{\pgfqpoint{2.776687in}{2.170262in}}%
\pgfpathlineto{\pgfqpoint{2.778521in}{2.590466in}}%
\pgfpathlineto{\pgfqpoint{2.780356in}{2.303626in}}%
\pgfpathlineto{\pgfqpoint{2.782191in}{2.256528in}}%
\pgfpathlineto{\pgfqpoint{2.784025in}{2.169383in}}%
\pgfpathlineto{\pgfqpoint{2.785858in}{1.276017in}}%
\pgfpathlineto{\pgfqpoint{2.787692in}{1.340253in}}%
\pgfpathlineto{\pgfqpoint{2.789527in}{1.261125in}}%
\pgfpathlineto{\pgfqpoint{2.791360in}{1.346338in}}%
\pgfpathlineto{\pgfqpoint{2.793195in}{1.252042in}}%
\pgfpathlineto{\pgfqpoint{2.795029in}{1.292829in}}%
\pgfpathlineto{\pgfqpoint{2.796863in}{1.220049in}}%
\pgfpathlineto{\pgfqpoint{2.798697in}{1.217640in}}%
\pgfpathlineto{\pgfqpoint{2.800531in}{1.317218in}}%
\pgfpathlineto{\pgfqpoint{2.802364in}{1.265968in}}%
\pgfpathlineto{\pgfqpoint{2.804199in}{1.392620in}}%
\pgfpathlineto{\pgfqpoint{2.806034in}{1.422881in}}%
\pgfpathlineto{\pgfqpoint{2.811535in}{1.242758in}}%
\pgfpathlineto{\pgfqpoint{2.813369in}{1.462539in}}%
\pgfpathlineto{\pgfqpoint{2.815202in}{1.126782in}}%
\pgfpathlineto{\pgfqpoint{2.817037in}{1.348546in}}%
\pgfpathlineto{\pgfqpoint{2.818870in}{1.245555in}}%
\pgfpathlineto{\pgfqpoint{2.820704in}{1.281286in}}%
\pgfpathlineto{\pgfqpoint{2.822539in}{1.267737in}}%
\pgfpathlineto{\pgfqpoint{2.826205in}{1.419092in}}%
\pgfpathlineto{\pgfqpoint{2.828040in}{1.313580in}}%
\pgfpathlineto{\pgfqpoint{2.831708in}{1.424638in}}%
\pgfpathlineto{\pgfqpoint{2.833542in}{1.194280in}}%
\pgfpathlineto{\pgfqpoint{2.835376in}{1.393197in}}%
\pgfpathlineto{\pgfqpoint{2.837209in}{1.312049in}}%
\pgfpathlineto{\pgfqpoint{2.839043in}{1.344983in}}%
\pgfpathlineto{\pgfqpoint{2.840878in}{1.414061in}}%
\pgfpathlineto{\pgfqpoint{2.842711in}{1.411100in}}%
\pgfpathlineto{\pgfqpoint{2.844547in}{1.444360in}}%
\pgfpathlineto{\pgfqpoint{2.846380in}{1.272755in}}%
\pgfpathlineto{\pgfqpoint{2.848214in}{1.269318in}}%
\pgfpathlineto{\pgfqpoint{2.851883in}{1.443181in}}%
\pgfpathlineto{\pgfqpoint{2.853718in}{1.241026in}}%
\pgfpathlineto{\pgfqpoint{2.857387in}{1.493666in}}%
\pgfpathlineto{\pgfqpoint{2.859250in}{1.399583in}}%
\pgfpathlineto{\pgfqpoint{2.861437in}{1.383323in}}%
\pgfpathlineto{\pgfqpoint{2.863271in}{1.552720in}}%
\pgfpathlineto{\pgfqpoint{2.866939in}{1.375871in}}%
\pgfpathlineto{\pgfqpoint{2.870607in}{1.266796in}}%
\pgfpathlineto{\pgfqpoint{2.874275in}{2.185242in}}%
\pgfpathlineto{\pgfqpoint{2.876108in}{1.586720in}}%
\pgfpathlineto{\pgfqpoint{2.877943in}{1.824217in}}%
\pgfpathlineto{\pgfqpoint{2.879777in}{1.612753in}}%
\pgfpathlineto{\pgfqpoint{2.881615in}{1.246935in}}%
\pgfpathlineto{\pgfqpoint{2.883449in}{1.504531in}}%
\pgfpathlineto{\pgfqpoint{2.885282in}{1.295978in}}%
\pgfpathlineto{\pgfqpoint{2.888951in}{1.228593in}}%
\pgfpathlineto{\pgfqpoint{2.892618in}{1.422116in}}%
\pgfpathlineto{\pgfqpoint{2.894452in}{1.428728in}}%
\pgfpathlineto{\pgfqpoint{2.896285in}{1.293117in}}%
\pgfpathlineto{\pgfqpoint{2.898120in}{1.445351in}}%
\pgfpathlineto{\pgfqpoint{2.899953in}{1.768688in}}%
\pgfpathlineto{\pgfqpoint{2.901787in}{1.678306in}}%
\pgfpathlineto{\pgfqpoint{2.903622in}{1.415153in}}%
\pgfpathlineto{\pgfqpoint{2.905456in}{1.437836in}}%
\pgfpathlineto{\pgfqpoint{2.907289in}{1.567324in}}%
\pgfpathlineto{\pgfqpoint{2.909124in}{1.300319in}}%
\pgfpathlineto{\pgfqpoint{2.910958in}{1.322149in}}%
\pgfpathlineto{\pgfqpoint{2.912791in}{1.422705in}}%
\pgfpathlineto{\pgfqpoint{2.914626in}{1.453581in}}%
\pgfpathlineto{\pgfqpoint{2.916459in}{1.364818in}}%
\pgfpathlineto{\pgfqpoint{2.918292in}{1.373186in}}%
\pgfpathlineto{\pgfqpoint{2.920127in}{1.415918in}}%
\pgfpathlineto{\pgfqpoint{2.921961in}{1.604322in}}%
\pgfpathlineto{\pgfqpoint{2.923795in}{1.547401in}}%
\pgfpathlineto{\pgfqpoint{2.925630in}{1.266821in}}%
\pgfpathlineto{\pgfqpoint{2.927464in}{1.285514in}}%
\pgfpathlineto{\pgfqpoint{2.929298in}{1.337581in}}%
\pgfpathlineto{\pgfqpoint{2.931134in}{1.538995in}}%
\pgfpathlineto{\pgfqpoint{2.932967in}{1.204153in}}%
\pgfpathlineto{\pgfqpoint{2.934799in}{1.228656in}}%
\pgfpathlineto{\pgfqpoint{2.936633in}{1.316215in}}%
\pgfpathlineto{\pgfqpoint{2.938468in}{1.276519in}}%
\pgfpathlineto{\pgfqpoint{2.940301in}{1.208118in}}%
\pgfpathlineto{\pgfqpoint{2.943970in}{1.382859in}}%
\pgfpathlineto{\pgfqpoint{2.945803in}{1.625036in}}%
\pgfpathlineto{\pgfqpoint{2.947636in}{1.333353in}}%
\pgfpathlineto{\pgfqpoint{2.949470in}{1.384980in}}%
\pgfpathlineto{\pgfqpoint{2.951303in}{1.399545in}}%
\pgfpathlineto{\pgfqpoint{2.954971in}{1.128727in}}%
\pgfpathlineto{\pgfqpoint{2.956806in}{1.183352in}}%
\pgfpathlineto{\pgfqpoint{2.958640in}{1.129166in}}%
\pgfpathlineto{\pgfqpoint{2.960474in}{1.364354in}}%
\pgfpathlineto{\pgfqpoint{2.962307in}{1.189048in}}%
\pgfpathlineto{\pgfqpoint{2.964141in}{1.322500in}}%
\pgfpathlineto{\pgfqpoint{2.965976in}{1.118150in}}%
\pgfpathlineto{\pgfqpoint{2.967811in}{1.460457in}}%
\pgfpathlineto{\pgfqpoint{2.969644in}{1.254187in}}%
\pgfpathlineto{\pgfqpoint{2.971478in}{1.235443in}}%
\pgfpathlineto{\pgfqpoint{2.973311in}{1.271764in}}%
\pgfpathlineto{\pgfqpoint{2.975146in}{1.344807in}}%
\pgfpathlineto{\pgfqpoint{2.976981in}{1.275026in}}%
\pgfpathlineto{\pgfqpoint{2.978814in}{1.300319in}}%
\pgfpathlineto{\pgfqpoint{2.980649in}{1.169740in}}%
\pgfpathlineto{\pgfqpoint{2.982484in}{1.399345in}}%
\pgfpathlineto{\pgfqpoint{2.984318in}{1.235506in}}%
\pgfpathlineto{\pgfqpoint{2.986151in}{1.231102in}}%
\pgfpathlineto{\pgfqpoint{2.987986in}{1.174796in}}%
\pgfpathlineto{\pgfqpoint{2.989819in}{1.318624in}}%
\pgfpathlineto{\pgfqpoint{2.991653in}{1.280960in}}%
\pgfpathlineto{\pgfqpoint{2.993487in}{1.282491in}}%
\pgfpathlineto{\pgfqpoint{2.995321in}{1.362346in}}%
\pgfpathlineto{\pgfqpoint{2.997154in}{1.356450in}}%
\pgfpathlineto{\pgfqpoint{2.998989in}{1.258867in}}%
\pgfpathlineto{\pgfqpoint{3.000823in}{1.441976in}}%
\pgfpathlineto{\pgfqpoint{3.002655in}{1.174093in}}%
\pgfpathlineto{\pgfqpoint{3.004488in}{1.242845in}}%
\pgfpathlineto{\pgfqpoint{3.006324in}{1.427021in}}%
\pgfpathlineto{\pgfqpoint{3.008156in}{1.451235in}}%
\pgfpathlineto{\pgfqpoint{3.015492in}{1.220689in}}%
\pgfpathlineto{\pgfqpoint{3.017327in}{1.163228in}}%
\pgfpathlineto{\pgfqpoint{3.019162in}{1.466955in}}%
\pgfpathlineto{\pgfqpoint{3.024663in}{1.341257in}}%
\pgfpathlineto{\pgfqpoint{3.028332in}{1.136455in}}%
\pgfpathlineto{\pgfqpoint{3.033835in}{1.361355in}}%
\pgfpathlineto{\pgfqpoint{3.035669in}{1.186024in}}%
\pgfpathlineto{\pgfqpoint{3.039337in}{2.118459in}}%
\pgfpathlineto{\pgfqpoint{3.041170in}{1.711001in}}%
\pgfpathlineto{\pgfqpoint{3.043003in}{1.868178in}}%
\pgfpathlineto{\pgfqpoint{3.046673in}{2.457654in}}%
\pgfpathlineto{\pgfqpoint{3.050343in}{1.904524in}}%
\pgfpathlineto{\pgfqpoint{3.052175in}{2.014766in}}%
\pgfpathlineto{\pgfqpoint{3.054009in}{2.029520in}}%
\pgfpathlineto{\pgfqpoint{3.055844in}{2.119400in}}%
\pgfpathlineto{\pgfqpoint{3.057678in}{2.118283in}}%
\pgfpathlineto{\pgfqpoint{3.059511in}{2.213131in}}%
\pgfpathlineto{\pgfqpoint{3.061345in}{2.242426in}}%
\pgfpathlineto{\pgfqpoint{3.063179in}{1.405191in}}%
\pgfpathlineto{\pgfqpoint{3.065013in}{1.188433in}}%
\pgfpathlineto{\pgfqpoint{3.066847in}{1.467896in}}%
\pgfpathlineto{\pgfqpoint{3.068682in}{1.312137in}}%
\pgfpathlineto{\pgfqpoint{3.070517in}{1.404363in}}%
\pgfpathlineto{\pgfqpoint{3.072352in}{1.429606in}}%
\pgfpathlineto{\pgfqpoint{3.076019in}{1.215909in}}%
\pgfpathlineto{\pgfqpoint{3.077854in}{1.211794in}}%
\pgfpathlineto{\pgfqpoint{3.079689in}{1.255454in}}%
\pgfpathlineto{\pgfqpoint{3.081521in}{1.115177in}}%
\pgfpathlineto{\pgfqpoint{3.085189in}{1.337693in}}%
\pgfpathlineto{\pgfqpoint{3.087023in}{1.201393in}}%
\pgfpathlineto{\pgfqpoint{3.088855in}{1.352448in}}%
\pgfpathlineto{\pgfqpoint{3.090689in}{1.256069in}}%
\pgfpathlineto{\pgfqpoint{3.092523in}{1.229772in}}%
\pgfpathlineto{\pgfqpoint{3.094357in}{1.420585in}}%
\pgfpathlineto{\pgfqpoint{3.096190in}{1.273671in}}%
\pgfpathlineto{\pgfqpoint{3.098024in}{1.340642in}}%
\pgfpathlineto{\pgfqpoint{3.099858in}{1.320455in}}%
\pgfpathlineto{\pgfqpoint{3.101693in}{1.257888in}}%
\pgfpathlineto{\pgfqpoint{3.103527in}{1.393937in}}%
\pgfpathlineto{\pgfqpoint{3.107195in}{2.219806in}}%
\pgfpathlineto{\pgfqpoint{3.109029in}{2.172871in}}%
\pgfpathlineto{\pgfqpoint{3.110862in}{2.395363in}}%
\pgfpathlineto{\pgfqpoint{3.112696in}{2.438998in}}%
\pgfpathlineto{\pgfqpoint{3.114530in}{2.311856in}}%
\pgfpathlineto{\pgfqpoint{3.116365in}{2.079905in}}%
\pgfpathlineto{\pgfqpoint{3.118199in}{2.026697in}}%
\pgfpathlineto{\pgfqpoint{3.120033in}{2.254408in}}%
\pgfpathlineto{\pgfqpoint{3.121867in}{2.274306in}}%
\pgfpathlineto{\pgfqpoint{3.123701in}{2.375615in}}%
\pgfpathlineto{\pgfqpoint{3.125535in}{2.372704in}}%
\pgfpathlineto{\pgfqpoint{3.127369in}{2.071474in}}%
\pgfpathlineto{\pgfqpoint{3.132873in}{2.440792in}}%
\pgfpathlineto{\pgfqpoint{3.134707in}{1.208708in}}%
\pgfpathlineto{\pgfqpoint{3.136542in}{1.386498in}}%
\pgfpathlineto{\pgfqpoint{3.138375in}{1.210878in}}%
\pgfpathlineto{\pgfqpoint{3.140209in}{1.424136in}}%
\pgfpathlineto{\pgfqpoint{3.142043in}{1.361932in}}%
\pgfpathlineto{\pgfqpoint{3.143877in}{1.405304in}}%
\pgfpathlineto{\pgfqpoint{3.145710in}{1.349926in}}%
\pgfpathlineto{\pgfqpoint{3.147544in}{1.442152in}}%
\pgfpathlineto{\pgfqpoint{3.149377in}{1.307959in}}%
\pgfpathlineto{\pgfqpoint{3.151211in}{1.360678in}}%
\pgfpathlineto{\pgfqpoint{3.153045in}{1.355509in}}%
\pgfpathlineto{\pgfqpoint{3.154878in}{1.408654in}}%
\pgfpathlineto{\pgfqpoint{3.156713in}{1.553373in}}%
\pgfpathlineto{\pgfqpoint{3.158548in}{1.235381in}}%
\pgfpathlineto{\pgfqpoint{3.164050in}{1.501507in}}%
\pgfpathlineto{\pgfqpoint{3.165884in}{1.799451in}}%
\pgfpathlineto{\pgfqpoint{3.167717in}{1.806150in}}%
\pgfpathlineto{\pgfqpoint{3.169551in}{1.397927in}}%
\pgfpathlineto{\pgfqpoint{3.171384in}{1.345485in}}%
\pgfpathlineto{\pgfqpoint{3.173218in}{1.433156in}}%
\pgfpathlineto{\pgfqpoint{3.175055in}{1.248755in}}%
\pgfpathlineto{\pgfqpoint{3.176889in}{1.566885in}}%
\pgfpathlineto{\pgfqpoint{3.178722in}{1.473504in}}%
\pgfpathlineto{\pgfqpoint{3.180569in}{1.252368in}}%
\pgfpathlineto{\pgfqpoint{3.182403in}{1.340604in}}%
\pgfpathlineto{\pgfqpoint{3.184237in}{1.269995in}}%
\pgfpathlineto{\pgfqpoint{3.186071in}{1.435603in}}%
\pgfpathlineto{\pgfqpoint{3.187905in}{2.051701in}}%
\pgfpathlineto{\pgfqpoint{3.189739in}{2.101421in}}%
\pgfpathlineto{\pgfqpoint{3.191573in}{2.276301in}}%
\pgfpathlineto{\pgfqpoint{3.193406in}{2.291431in}}%
\pgfpathlineto{\pgfqpoint{3.195241in}{2.229843in}}%
\pgfpathlineto{\pgfqpoint{3.197075in}{2.045504in}}%
\pgfpathlineto{\pgfqpoint{3.198909in}{2.223883in}}%
\pgfpathlineto{\pgfqpoint{3.202579in}{2.192067in}}%
\pgfpathlineto{\pgfqpoint{3.204428in}{2.160037in}}%
\pgfpathlineto{\pgfqpoint{3.206260in}{2.211990in}}%
\pgfpathlineto{\pgfqpoint{3.208094in}{2.057636in}}%
\pgfpathlineto{\pgfqpoint{3.209927in}{2.248210in}}%
\pgfpathlineto{\pgfqpoint{3.211761in}{1.515847in}}%
\pgfpathlineto{\pgfqpoint{3.213595in}{1.387526in}}%
\pgfpathlineto{\pgfqpoint{3.215429in}{1.343603in}}%
\pgfpathlineto{\pgfqpoint{3.217263in}{1.178710in}}%
\pgfpathlineto{\pgfqpoint{3.222765in}{1.463869in}}%
\pgfpathlineto{\pgfqpoint{3.224599in}{1.310230in}}%
\pgfpathlineto{\pgfqpoint{3.226432in}{1.311610in}}%
\pgfpathlineto{\pgfqpoint{3.228266in}{1.445765in}}%
\pgfpathlineto{\pgfqpoint{3.230099in}{1.192021in}}%
\pgfpathlineto{\pgfqpoint{3.231933in}{1.092619in}}%
\pgfpathlineto{\pgfqpoint{3.235603in}{1.318097in}}%
\pgfpathlineto{\pgfqpoint{3.237436in}{1.130169in}}%
\pgfpathlineto{\pgfqpoint{3.239271in}{1.210702in}}%
\pgfpathlineto{\pgfqpoint{3.241104in}{1.145237in}}%
\pgfpathlineto{\pgfqpoint{3.242938in}{1.394088in}}%
\pgfpathlineto{\pgfqpoint{3.244772in}{1.172562in}}%
\pgfpathlineto{\pgfqpoint{3.246606in}{1.132252in}}%
\pgfpathlineto{\pgfqpoint{3.248439in}{1.143267in}}%
\pgfpathlineto{\pgfqpoint{3.250273in}{1.583220in}}%
\pgfpathlineto{\pgfqpoint{3.252109in}{1.613481in}}%
\pgfpathlineto{\pgfqpoint{3.253942in}{1.321685in}}%
\pgfpathlineto{\pgfqpoint{3.255778in}{1.368870in}}%
\pgfpathlineto{\pgfqpoint{3.257612in}{1.342511in}}%
\pgfpathlineto{\pgfqpoint{3.259444in}{1.284047in}}%
\pgfpathlineto{\pgfqpoint{3.261278in}{1.122993in}}%
\pgfpathlineto{\pgfqpoint{3.263112in}{1.341219in}}%
\pgfpathlineto{\pgfqpoint{3.264945in}{2.062253in}}%
\pgfpathlineto{\pgfqpoint{3.266779in}{1.943743in}}%
\pgfpathlineto{\pgfqpoint{3.268613in}{2.065251in}}%
\pgfpathlineto{\pgfqpoint{3.270446in}{1.999108in}}%
\pgfpathlineto{\pgfqpoint{3.272279in}{2.039971in}}%
\pgfpathlineto{\pgfqpoint{3.274116in}{2.165946in}}%
\pgfpathlineto{\pgfqpoint{3.275950in}{2.197474in}}%
\pgfpathlineto{\pgfqpoint{3.277784in}{1.971633in}}%
\pgfpathlineto{\pgfqpoint{3.279618in}{1.219485in}}%
\pgfpathlineto{\pgfqpoint{3.283286in}{1.114951in}}%
\pgfpathlineto{\pgfqpoint{3.285121in}{1.140156in}}%
\pgfpathlineto{\pgfqpoint{3.286954in}{1.291449in}}%
\pgfpathlineto{\pgfqpoint{3.290622in}{1.146115in}}%
\pgfpathlineto{\pgfqpoint{3.292455in}{1.202359in}}%
\pgfpathlineto{\pgfqpoint{3.294288in}{1.189550in}}%
\pgfpathlineto{\pgfqpoint{3.296121in}{1.245204in}}%
\pgfpathlineto{\pgfqpoint{3.297955in}{1.365169in}}%
\pgfpathlineto{\pgfqpoint{3.299789in}{1.406948in}}%
\pgfpathlineto{\pgfqpoint{3.301622in}{1.333942in}}%
\pgfpathlineto{\pgfqpoint{3.303456in}{1.128074in}}%
\pgfpathlineto{\pgfqpoint{3.307123in}{1.334231in}}%
\pgfpathlineto{\pgfqpoint{3.308957in}{1.215344in}}%
\pgfpathlineto{\pgfqpoint{3.312625in}{1.255040in}}%
\pgfpathlineto{\pgfqpoint{3.314460in}{1.193427in}}%
\pgfpathlineto{\pgfqpoint{3.316293in}{1.323215in}}%
\pgfpathlineto{\pgfqpoint{3.318126in}{1.115302in}}%
\pgfpathlineto{\pgfqpoint{3.323628in}{1.260159in}}%
\pgfpathlineto{\pgfqpoint{3.325463in}{2.005369in}}%
\pgfpathlineto{\pgfqpoint{3.327297in}{2.055603in}}%
\pgfpathlineto{\pgfqpoint{3.329130in}{2.170763in}}%
\pgfpathlineto{\pgfqpoint{3.330965in}{2.090895in}}%
\pgfpathlineto{\pgfqpoint{3.332797in}{2.090657in}}%
\pgfpathlineto{\pgfqpoint{3.334631in}{2.024966in}}%
\pgfpathlineto{\pgfqpoint{3.336466in}{2.023372in}}%
\pgfpathlineto{\pgfqpoint{3.338299in}{2.009471in}}%
\pgfpathlineto{\pgfqpoint{3.340132in}{2.154215in}}%
\pgfpathlineto{\pgfqpoint{3.343803in}{2.003600in}}%
\pgfpathlineto{\pgfqpoint{3.345637in}{2.074146in}}%
\pgfpathlineto{\pgfqpoint{3.349319in}{2.099677in}}%
\pgfpathlineto{\pgfqpoint{3.351153in}{2.242803in}}%
\pgfpathlineto{\pgfqpoint{3.352986in}{2.098297in}}%
\pgfpathlineto{\pgfqpoint{3.354819in}{2.155984in}}%
\pgfpathlineto{\pgfqpoint{3.356653in}{1.976952in}}%
\pgfpathlineto{\pgfqpoint{3.358488in}{1.399784in}}%
\pgfpathlineto{\pgfqpoint{3.360321in}{1.198508in}}%
\pgfpathlineto{\pgfqpoint{3.362156in}{1.255655in}}%
\pgfpathlineto{\pgfqpoint{3.363989in}{1.180943in}}%
\pgfpathlineto{\pgfqpoint{3.365824in}{1.160581in}}%
\pgfpathlineto{\pgfqpoint{3.367658in}{1.330818in}}%
\pgfpathlineto{\pgfqpoint{3.369491in}{1.255780in}}%
\pgfpathlineto{\pgfqpoint{3.371324in}{1.307081in}}%
\pgfpathlineto{\pgfqpoint{3.373159in}{1.420610in}}%
\pgfpathlineto{\pgfqpoint{3.374993in}{1.362698in}}%
\pgfpathlineto{\pgfqpoint{3.376827in}{1.266859in}}%
\pgfpathlineto{\pgfqpoint{3.378660in}{1.417411in}}%
\pgfpathlineto{\pgfqpoint{3.380495in}{1.195748in}}%
\pgfpathlineto{\pgfqpoint{3.382330in}{1.266683in}}%
\pgfpathlineto{\pgfqpoint{3.384164in}{1.226071in}}%
\pgfpathlineto{\pgfqpoint{3.386001in}{1.245932in}}%
\pgfpathlineto{\pgfqpoint{3.387834in}{1.332788in}}%
\pgfpathlineto{\pgfqpoint{3.389668in}{1.263421in}}%
\pgfpathlineto{\pgfqpoint{3.391503in}{1.124825in}}%
\pgfpathlineto{\pgfqpoint{3.393338in}{1.261037in}}%
\pgfpathlineto{\pgfqpoint{3.395171in}{1.265391in}}%
\pgfpathlineto{\pgfqpoint{3.397006in}{1.099670in}}%
\pgfpathlineto{\pgfqpoint{3.398840in}{1.234327in}}%
\pgfpathlineto{\pgfqpoint{3.400674in}{1.287007in}}%
\pgfpathlineto{\pgfqpoint{3.402510in}{1.138952in}}%
\pgfpathlineto{\pgfqpoint{3.404343in}{1.092494in}}%
\pgfpathlineto{\pgfqpoint{3.406177in}{1.233599in}}%
\pgfpathlineto{\pgfqpoint{3.408012in}{1.111513in}}%
\pgfpathlineto{\pgfqpoint{3.409845in}{1.350741in}}%
\pgfpathlineto{\pgfqpoint{3.411678in}{1.161785in}}%
\pgfpathlineto{\pgfqpoint{3.413513in}{1.243197in}}%
\pgfpathlineto{\pgfqpoint{3.415347in}{1.195045in}}%
\pgfpathlineto{\pgfqpoint{3.417181in}{1.086647in}}%
\pgfpathlineto{\pgfqpoint{3.419015in}{1.252493in}}%
\pgfpathlineto{\pgfqpoint{3.420849in}{1.223073in}}%
\pgfpathlineto{\pgfqpoint{3.422686in}{1.018271in}}%
\pgfpathlineto{\pgfqpoint{3.424521in}{1.031031in}}%
\pgfpathlineto{\pgfqpoint{3.426354in}{1.276168in}}%
\pgfpathlineto{\pgfqpoint{3.428187in}{1.104488in}}%
\pgfpathlineto{\pgfqpoint{3.430022in}{1.161020in}}%
\pgfpathlineto{\pgfqpoint{3.431855in}{1.147972in}}%
\pgfpathlineto{\pgfqpoint{3.435523in}{1.417499in}}%
\pgfpathlineto{\pgfqpoint{3.437356in}{1.311723in}}%
\pgfpathlineto{\pgfqpoint{3.439189in}{1.349951in}}%
\pgfpathlineto{\pgfqpoint{3.441023in}{1.173089in}}%
\pgfpathlineto{\pgfqpoint{3.442856in}{1.293331in}}%
\pgfpathlineto{\pgfqpoint{3.444691in}{1.658797in}}%
\pgfpathlineto{\pgfqpoint{3.446524in}{2.249101in}}%
\pgfpathlineto{\pgfqpoint{3.448382in}{2.235049in}}%
\pgfpathlineto{\pgfqpoint{3.450217in}{2.271746in}}%
\pgfpathlineto{\pgfqpoint{3.452050in}{2.094069in}}%
\pgfpathlineto{\pgfqpoint{3.455718in}{2.261961in}}%
\pgfpathlineto{\pgfqpoint{3.457551in}{2.291318in}}%
\pgfpathlineto{\pgfqpoint{3.461218in}{2.217422in}}%
\pgfpathlineto{\pgfqpoint{3.463056in}{2.192267in}}%
\pgfpathlineto{\pgfqpoint{3.464890in}{2.139825in}}%
\pgfpathlineto{\pgfqpoint{3.466727in}{2.141054in}}%
\pgfpathlineto{\pgfqpoint{3.468561in}{2.050522in}}%
\pgfpathlineto{\pgfqpoint{3.470395in}{2.179872in}}%
\pgfpathlineto{\pgfqpoint{3.474066in}{2.010588in}}%
\pgfpathlineto{\pgfqpoint{3.475899in}{2.344062in}}%
\pgfpathlineto{\pgfqpoint{3.477733in}{2.143965in}}%
\pgfpathlineto{\pgfqpoint{3.479567in}{2.309623in}}%
\pgfpathlineto{\pgfqpoint{3.481400in}{2.241398in}}%
\pgfpathlineto{\pgfqpoint{3.483235in}{2.483097in}}%
\pgfpathlineto{\pgfqpoint{3.486903in}{2.232051in}}%
\pgfpathlineto{\pgfqpoint{3.488739in}{2.200937in}}%
\pgfpathlineto{\pgfqpoint{3.490574in}{2.377409in}}%
\pgfpathlineto{\pgfqpoint{3.492407in}{2.192418in}}%
\pgfpathlineto{\pgfqpoint{3.494242in}{2.152484in}}%
\pgfpathlineto{\pgfqpoint{3.496076in}{2.186421in}}%
\pgfpathlineto{\pgfqpoint{3.497910in}{2.158807in}}%
\pgfpathlineto{\pgfqpoint{3.499744in}{2.431533in}}%
\pgfpathlineto{\pgfqpoint{3.501578in}{2.351815in}}%
\pgfpathlineto{\pgfqpoint{3.503411in}{2.183046in}}%
\pgfpathlineto{\pgfqpoint{3.505250in}{2.229730in}}%
\pgfpathlineto{\pgfqpoint{3.507084in}{2.382290in}}%
\pgfpathlineto{\pgfqpoint{3.508918in}{2.074974in}}%
\pgfpathlineto{\pgfqpoint{3.510752in}{2.204751in}}%
\pgfpathlineto{\pgfqpoint{3.512586in}{2.241423in}}%
\pgfpathlineto{\pgfqpoint{3.516251in}{1.332085in}}%
\pgfpathlineto{\pgfqpoint{3.518086in}{1.356174in}}%
\pgfpathlineto{\pgfqpoint{3.519919in}{1.276469in}}%
\pgfpathlineto{\pgfqpoint{3.521752in}{1.257487in}}%
\pgfpathlineto{\pgfqpoint{3.523588in}{1.351394in}}%
\pgfpathlineto{\pgfqpoint{3.525421in}{1.387326in}}%
\pgfpathlineto{\pgfqpoint{3.527256in}{1.190102in}}%
\pgfpathlineto{\pgfqpoint{3.529091in}{1.186614in}}%
\pgfpathlineto{\pgfqpoint{3.530924in}{1.361707in}}%
\pgfpathlineto{\pgfqpoint{3.532759in}{1.217929in}}%
\pgfpathlineto{\pgfqpoint{3.534593in}{1.300557in}}%
\pgfpathlineto{\pgfqpoint{3.536425in}{1.128701in}}%
\pgfpathlineto{\pgfqpoint{3.538258in}{1.259067in}}%
\pgfpathlineto{\pgfqpoint{3.540092in}{1.187379in}}%
\pgfpathlineto{\pgfqpoint{3.541927in}{1.243761in}}%
\pgfpathlineto{\pgfqpoint{3.543760in}{1.079860in}}%
\pgfpathlineto{\pgfqpoint{3.545594in}{1.216173in}}%
\pgfpathlineto{\pgfqpoint{3.547428in}{1.206299in}}%
\pgfpathlineto{\pgfqpoint{3.549263in}{1.213262in}}%
\pgfpathlineto{\pgfqpoint{3.551099in}{1.177380in}}%
\pgfpathlineto{\pgfqpoint{3.554765in}{1.013893in}}%
\pgfpathlineto{\pgfqpoint{3.556600in}{1.177531in}}%
\pgfpathlineto{\pgfqpoint{3.558433in}{0.895195in}}%
\pgfpathlineto{\pgfqpoint{3.562100in}{1.285339in}}%
\pgfpathlineto{\pgfqpoint{3.563934in}{1.035459in}}%
\pgfpathlineto{\pgfqpoint{3.565768in}{1.001083in}}%
\pgfpathlineto{\pgfqpoint{3.567603in}{1.187204in}}%
\pgfpathlineto{\pgfqpoint{3.569436in}{0.962567in}}%
\pgfpathlineto{\pgfqpoint{3.573103in}{1.134335in}}%
\pgfpathlineto{\pgfqpoint{3.576774in}{1.013140in}}%
\pgfpathlineto{\pgfqpoint{3.578608in}{1.080186in}}%
\pgfpathlineto{\pgfqpoint{3.580443in}{1.216173in}}%
\pgfpathlineto{\pgfqpoint{3.584114in}{1.005324in}}%
\pgfpathlineto{\pgfqpoint{3.585947in}{1.240963in}}%
\pgfpathlineto{\pgfqpoint{3.587780in}{1.265880in}}%
\pgfpathlineto{\pgfqpoint{3.589616in}{1.274901in}}%
\pgfpathlineto{\pgfqpoint{3.591449in}{1.469063in}}%
\pgfpathlineto{\pgfqpoint{3.595117in}{1.342486in}}%
\pgfpathlineto{\pgfqpoint{3.602455in}{0.984284in}}%
\pgfpathlineto{\pgfqpoint{3.604289in}{1.122403in}}%
\pgfpathlineto{\pgfqpoint{3.606124in}{1.041456in}}%
\pgfpathlineto{\pgfqpoint{3.607958in}{1.108302in}}%
\pgfpathlineto{\pgfqpoint{3.609790in}{1.069396in}}%
\pgfpathlineto{\pgfqpoint{3.611625in}{1.107750in}}%
\pgfpathlineto{\pgfqpoint{3.613459in}{1.110685in}}%
\pgfpathlineto{\pgfqpoint{3.615293in}{1.064265in}}%
\pgfpathlineto{\pgfqpoint{3.617126in}{1.049486in}}%
\pgfpathlineto{\pgfqpoint{3.618961in}{1.066548in}}%
\pgfpathlineto{\pgfqpoint{3.620794in}{1.072633in}}%
\pgfpathlineto{\pgfqpoint{3.622628in}{1.136191in}}%
\pgfpathlineto{\pgfqpoint{3.624462in}{1.414588in}}%
\pgfpathlineto{\pgfqpoint{3.626295in}{1.142364in}}%
\pgfpathlineto{\pgfqpoint{3.628128in}{1.125314in}}%
\pgfpathlineto{\pgfqpoint{3.631796in}{1.388154in}}%
\pgfpathlineto{\pgfqpoint{3.633629in}{1.422053in}}%
\pgfpathlineto{\pgfqpoint{3.635464in}{1.361757in}}%
\pgfpathlineto{\pgfqpoint{3.637298in}{1.382420in}}%
\pgfpathlineto{\pgfqpoint{3.639130in}{1.091879in}}%
\pgfpathlineto{\pgfqpoint{3.640965in}{1.317507in}}%
\pgfpathlineto{\pgfqpoint{3.644632in}{1.169740in}}%
\pgfpathlineto{\pgfqpoint{3.646465in}{1.215232in}}%
\pgfpathlineto{\pgfqpoint{3.648299in}{1.142151in}}%
\pgfpathlineto{\pgfqpoint{3.650133in}{1.251552in}}%
\pgfpathlineto{\pgfqpoint{3.651968in}{1.235092in}}%
\pgfpathlineto{\pgfqpoint{3.653803in}{1.106784in}}%
\pgfpathlineto{\pgfqpoint{3.655636in}{1.222019in}}%
\pgfpathlineto{\pgfqpoint{3.657471in}{1.034293in}}%
\pgfpathlineto{\pgfqpoint{3.662969in}{1.721025in}}%
\pgfpathlineto{\pgfqpoint{3.664803in}{1.622213in}}%
\pgfpathlineto{\pgfqpoint{3.670303in}{2.439111in}}%
\pgfpathlineto{\pgfqpoint{3.672137in}{2.351728in}}%
\pgfpathlineto{\pgfqpoint{3.673971in}{2.581684in}}%
\pgfpathlineto{\pgfqpoint{3.675806in}{2.414985in}}%
\pgfpathlineto{\pgfqpoint{3.677640in}{2.567018in}}%
\pgfpathlineto{\pgfqpoint{3.679474in}{2.565612in}}%
\pgfpathlineto{\pgfqpoint{3.681309in}{2.517687in}}%
\pgfpathlineto{\pgfqpoint{3.683143in}{2.542565in}}%
\pgfpathlineto{\pgfqpoint{3.684977in}{2.494790in}}%
\pgfpathlineto{\pgfqpoint{3.686813in}{2.498090in}}%
\pgfpathlineto{\pgfqpoint{3.688646in}{2.614793in}}%
\pgfpathlineto{\pgfqpoint{3.690479in}{2.633926in}}%
\pgfpathlineto{\pgfqpoint{3.692315in}{2.509481in}}%
\pgfpathlineto{\pgfqpoint{3.694149in}{2.495969in}}%
\pgfpathlineto{\pgfqpoint{3.695982in}{2.643448in}}%
\pgfpathlineto{\pgfqpoint{3.697816in}{2.486359in}}%
\pgfpathlineto{\pgfqpoint{3.699648in}{2.614994in}}%
\pgfpathlineto{\pgfqpoint{3.701483in}{2.419689in}}%
\pgfpathlineto{\pgfqpoint{3.703318in}{2.683457in}}%
\pgfpathlineto{\pgfqpoint{3.705152in}{2.482545in}}%
\pgfpathlineto{\pgfqpoint{3.706986in}{2.576778in}}%
\pgfpathlineto{\pgfqpoint{3.708820in}{2.414320in}}%
\pgfpathlineto{\pgfqpoint{3.710655in}{2.377786in}}%
\pgfpathlineto{\pgfqpoint{3.712488in}{2.541361in}}%
\pgfpathlineto{\pgfqpoint{3.714321in}{2.452134in}}%
\pgfpathlineto{\pgfqpoint{3.716156in}{2.436740in}}%
\pgfpathlineto{\pgfqpoint{3.717990in}{2.389453in}}%
\pgfpathlineto{\pgfqpoint{3.719822in}{2.598069in}}%
\pgfpathlineto{\pgfqpoint{3.721657in}{2.423541in}}%
\pgfpathlineto{\pgfqpoint{3.723491in}{2.546593in}}%
\pgfpathlineto{\pgfqpoint{3.725324in}{2.422976in}}%
\pgfpathlineto{\pgfqpoint{3.727159in}{2.712878in}}%
\pgfpathlineto{\pgfqpoint{3.728993in}{2.478957in}}%
\pgfpathlineto{\pgfqpoint{3.732661in}{2.370534in}}%
\pgfpathlineto{\pgfqpoint{3.734494in}{2.388011in}}%
\pgfpathlineto{\pgfqpoint{3.736327in}{2.574219in}}%
\pgfpathlineto{\pgfqpoint{3.738163in}{2.532077in}}%
\pgfpathlineto{\pgfqpoint{3.739999in}{2.572337in}}%
\pgfpathlineto{\pgfqpoint{3.741832in}{2.541750in}}%
\pgfpathlineto{\pgfqpoint{3.745501in}{2.604192in}}%
\pgfpathlineto{\pgfqpoint{3.747335in}{2.498315in}}%
\pgfpathlineto{\pgfqpoint{3.749169in}{2.535866in}}%
\pgfpathlineto{\pgfqpoint{3.751003in}{2.519970in}}%
\pgfpathlineto{\pgfqpoint{3.752838in}{2.389717in}}%
\pgfpathlineto{\pgfqpoint{3.754673in}{2.579237in}}%
\pgfpathlineto{\pgfqpoint{3.758339in}{2.384899in}}%
\pgfpathlineto{\pgfqpoint{3.760174in}{2.417218in}}%
\pgfpathlineto{\pgfqpoint{3.762009in}{2.410430in}}%
\pgfpathlineto{\pgfqpoint{3.763842in}{2.305006in}}%
\pgfpathlineto{\pgfqpoint{3.765677in}{2.378902in}}%
\pgfpathlineto{\pgfqpoint{3.767511in}{2.337688in}}%
\pgfpathlineto{\pgfqpoint{3.769345in}{2.527585in}}%
\pgfpathlineto{\pgfqpoint{3.771179in}{2.469961in}}%
\pgfpathlineto{\pgfqpoint{3.773011in}{2.506433in}}%
\pgfpathlineto{\pgfqpoint{3.774845in}{2.394484in}}%
\pgfpathlineto{\pgfqpoint{3.776679in}{2.403267in}}%
\pgfpathlineto{\pgfqpoint{3.778514in}{2.543720in}}%
\pgfpathlineto{\pgfqpoint{3.780349in}{2.486974in}}%
\pgfpathlineto{\pgfqpoint{3.782186in}{2.637828in}}%
\pgfpathlineto{\pgfqpoint{3.784019in}{1.627557in}}%
\pgfpathlineto{\pgfqpoint{3.787687in}{1.566232in}}%
\pgfpathlineto{\pgfqpoint{3.789522in}{1.507027in}}%
\pgfpathlineto{\pgfqpoint{3.791355in}{1.594674in}}%
\pgfpathlineto{\pgfqpoint{3.793189in}{1.462138in}}%
\pgfpathlineto{\pgfqpoint{3.795023in}{1.574588in}}%
\pgfpathlineto{\pgfqpoint{3.796856in}{1.554803in}}%
\pgfpathlineto{\pgfqpoint{3.798692in}{1.590835in}}%
\pgfpathlineto{\pgfqpoint{3.802362in}{1.383035in}}%
\pgfpathlineto{\pgfqpoint{3.804196in}{1.489312in}}%
\pgfpathlineto{\pgfqpoint{3.806032in}{1.387827in}}%
\pgfpathlineto{\pgfqpoint{3.807865in}{1.401816in}}%
\pgfpathlineto{\pgfqpoint{3.809697in}{1.539760in}}%
\pgfpathlineto{\pgfqpoint{3.811531in}{1.429694in}}%
\pgfpathlineto{\pgfqpoint{3.813365in}{1.393611in}}%
\pgfpathlineto{\pgfqpoint{3.815198in}{1.600583in}}%
\pgfpathlineto{\pgfqpoint{3.817032in}{1.399081in}}%
\pgfpathlineto{\pgfqpoint{3.818866in}{1.481057in}}%
\pgfpathlineto{\pgfqpoint{3.820699in}{1.312049in}}%
\pgfpathlineto{\pgfqpoint{3.822534in}{1.432717in}}%
\pgfpathlineto{\pgfqpoint{3.824367in}{1.416621in}}%
\pgfpathlineto{\pgfqpoint{3.826201in}{1.426432in}}%
\pgfpathlineto{\pgfqpoint{3.828036in}{1.360640in}}%
\pgfpathlineto{\pgfqpoint{3.829870in}{1.483842in}}%
\pgfpathlineto{\pgfqpoint{3.831703in}{1.433922in}}%
\pgfpathlineto{\pgfqpoint{3.833538in}{1.441299in}}%
\pgfpathlineto{\pgfqpoint{3.835371in}{1.568792in}}%
\pgfpathlineto{\pgfqpoint{3.837205in}{1.524742in}}%
\pgfpathlineto{\pgfqpoint{3.839039in}{1.433608in}}%
\pgfpathlineto{\pgfqpoint{3.840875in}{1.514643in}}%
\pgfpathlineto{\pgfqpoint{3.842708in}{1.358470in}}%
\pgfpathlineto{\pgfqpoint{3.844542in}{1.567211in}}%
\pgfpathlineto{\pgfqpoint{3.848208in}{1.461636in}}%
\pgfpathlineto{\pgfqpoint{3.850044in}{1.552921in}}%
\pgfpathlineto{\pgfqpoint{3.851891in}{1.584838in}}%
\pgfpathlineto{\pgfqpoint{3.853727in}{1.500328in}}%
\pgfpathlineto{\pgfqpoint{3.855561in}{1.478560in}}%
\pgfpathlineto{\pgfqpoint{3.857394in}{1.340730in}}%
\pgfpathlineto{\pgfqpoint{3.861063in}{1.523450in}}%
\pgfpathlineto{\pgfqpoint{3.862896in}{1.620067in}}%
\pgfpathlineto{\pgfqpoint{3.866563in}{1.322036in}}%
\pgfpathlineto{\pgfqpoint{3.868397in}{1.460983in}}%
\pgfpathlineto{\pgfqpoint{3.870230in}{1.477996in}}%
\pgfpathlineto{\pgfqpoint{3.872064in}{1.483491in}}%
\pgfpathlineto{\pgfqpoint{3.873898in}{1.588125in}}%
\pgfpathlineto{\pgfqpoint{3.877566in}{1.393022in}}%
\pgfpathlineto{\pgfqpoint{3.879399in}{1.462903in}}%
\pgfpathlineto{\pgfqpoint{3.881233in}{1.443921in}}%
\pgfpathlineto{\pgfqpoint{3.883067in}{1.365194in}}%
\pgfpathlineto{\pgfqpoint{3.884901in}{1.440508in}}%
\pgfpathlineto{\pgfqpoint{3.886734in}{1.327330in}}%
\pgfpathlineto{\pgfqpoint{3.888568in}{1.367553in}}%
\pgfpathlineto{\pgfqpoint{3.890402in}{1.552921in}}%
\pgfpathlineto{\pgfqpoint{3.894071in}{1.421639in}}%
\pgfpathlineto{\pgfqpoint{3.895904in}{1.447007in}}%
\pgfpathlineto{\pgfqpoint{3.897738in}{1.542056in}}%
\pgfpathlineto{\pgfqpoint{3.899574in}{1.356676in}}%
\pgfpathlineto{\pgfqpoint{3.901409in}{1.340228in}}%
\pgfpathlineto{\pgfqpoint{3.905078in}{1.476440in}}%
\pgfpathlineto{\pgfqpoint{3.906913in}{1.320919in}}%
\pgfpathlineto{\pgfqpoint{3.908748in}{1.358382in}}%
\pgfpathlineto{\pgfqpoint{3.910582in}{1.442829in}}%
\pgfpathlineto{\pgfqpoint{3.912415in}{1.597146in}}%
\pgfpathlineto{\pgfqpoint{3.914249in}{1.323128in}}%
\pgfpathlineto{\pgfqpoint{3.916085in}{1.429016in}}%
\pgfpathlineto{\pgfqpoint{3.917917in}{1.452264in}}%
\pgfpathlineto{\pgfqpoint{3.919750in}{1.408955in}}%
\pgfpathlineto{\pgfqpoint{3.923418in}{1.530802in}}%
\pgfpathlineto{\pgfqpoint{3.925252in}{1.457696in}}%
\pgfpathlineto{\pgfqpoint{3.927086in}{1.433483in}}%
\pgfpathlineto{\pgfqpoint{3.928919in}{1.599090in}}%
\pgfpathlineto{\pgfqpoint{3.932587in}{1.333528in}}%
\pgfpathlineto{\pgfqpoint{3.934422in}{1.484733in}}%
\pgfpathlineto{\pgfqpoint{3.936255in}{1.262392in}}%
\pgfpathlineto{\pgfqpoint{3.938088in}{1.510051in}}%
\pgfpathlineto{\pgfqpoint{3.939922in}{1.434160in}}%
\pgfpathlineto{\pgfqpoint{3.941755in}{1.457232in}}%
\pgfpathlineto{\pgfqpoint{3.943588in}{1.444448in}}%
\pgfpathlineto{\pgfqpoint{3.945425in}{1.465538in}}%
\pgfpathlineto{\pgfqpoint{3.947259in}{1.359210in}}%
\pgfpathlineto{\pgfqpoint{3.950929in}{1.514467in}}%
\pgfpathlineto{\pgfqpoint{3.952762in}{1.422843in}}%
\pgfpathlineto{\pgfqpoint{3.954594in}{1.427172in}}%
\pgfpathlineto{\pgfqpoint{3.956429in}{1.598262in}}%
\pgfpathlineto{\pgfqpoint{3.958263in}{1.519486in}}%
\pgfpathlineto{\pgfqpoint{3.960096in}{1.546196in}}%
\pgfpathlineto{\pgfqpoint{3.961930in}{1.552281in}}%
\pgfpathlineto{\pgfqpoint{3.963764in}{1.442478in}}%
\pgfpathlineto{\pgfqpoint{3.965599in}{1.600671in}}%
\pgfpathlineto{\pgfqpoint{3.967434in}{1.558504in}}%
\pgfpathlineto{\pgfqpoint{3.969267in}{1.594235in}}%
\pgfpathlineto{\pgfqpoint{3.971101in}{1.607371in}}%
\pgfpathlineto{\pgfqpoint{3.972938in}{1.422994in}}%
\pgfpathlineto{\pgfqpoint{3.974772in}{1.639225in}}%
\pgfpathlineto{\pgfqpoint{3.976604in}{1.614572in}}%
\pgfpathlineto{\pgfqpoint{3.980273in}{1.810905in}}%
\pgfpathlineto{\pgfqpoint{3.982106in}{1.556358in}}%
\pgfpathlineto{\pgfqpoint{3.983941in}{1.637080in}}%
\pgfpathlineto{\pgfqpoint{3.987610in}{1.467721in}}%
\pgfpathlineto{\pgfqpoint{3.989444in}{1.530150in}}%
\pgfpathlineto{\pgfqpoint{3.991276in}{1.634282in}}%
\pgfpathlineto{\pgfqpoint{3.993109in}{1.554125in}}%
\pgfpathlineto{\pgfqpoint{3.994943in}{1.790756in}}%
\pgfpathlineto{\pgfqpoint{3.998610in}{1.493603in}}%
\pgfpathlineto{\pgfqpoint{4.002279in}{1.611988in}}%
\pgfpathlineto{\pgfqpoint{4.004112in}{1.550286in}}%
\pgfpathlineto{\pgfqpoint{4.005945in}{1.533562in}}%
\pgfpathlineto{\pgfqpoint{4.007780in}{1.588564in}}%
\pgfpathlineto{\pgfqpoint{4.009614in}{1.503414in}}%
\pgfpathlineto{\pgfqpoint{4.011448in}{1.614723in}}%
\pgfpathlineto{\pgfqpoint{4.015116in}{1.443206in}}%
\pgfpathlineto{\pgfqpoint{4.016949in}{1.511531in}}%
\pgfpathlineto{\pgfqpoint{4.020618in}{1.328384in}}%
\pgfpathlineto{\pgfqpoint{4.022452in}{1.415065in}}%
\pgfpathlineto{\pgfqpoint{4.024289in}{1.661319in}}%
\pgfpathlineto{\pgfqpoint{4.026122in}{1.615225in}}%
\pgfpathlineto{\pgfqpoint{4.029789in}{1.419820in}}%
\pgfpathlineto{\pgfqpoint{4.031623in}{1.363789in}}%
\pgfpathlineto{\pgfqpoint{4.033457in}{1.433520in}}%
\pgfpathlineto{\pgfqpoint{4.035291in}{1.647016in}}%
\pgfpathlineto{\pgfqpoint{4.037125in}{1.504832in}}%
\pgfpathlineto{\pgfqpoint{4.040794in}{1.569933in}}%
\pgfpathlineto{\pgfqpoint{4.042629in}{1.631033in}}%
\pgfpathlineto{\pgfqpoint{4.044463in}{1.576909in}}%
\pgfpathlineto{\pgfqpoint{4.046298in}{1.622501in}}%
\pgfpathlineto{\pgfqpoint{4.048133in}{1.695845in}}%
\pgfpathlineto{\pgfqpoint{4.049965in}{1.907347in}}%
\pgfpathlineto{\pgfqpoint{4.051799in}{1.966815in}}%
\pgfpathlineto{\pgfqpoint{4.055467in}{1.556835in}}%
\pgfpathlineto{\pgfqpoint{4.057301in}{1.620657in}}%
\pgfpathlineto{\pgfqpoint{4.059136in}{1.768035in}}%
\pgfpathlineto{\pgfqpoint{4.060970in}{1.552369in}}%
\pgfpathlineto{\pgfqpoint{4.062804in}{1.476352in}}%
\pgfpathlineto{\pgfqpoint{4.064637in}{1.482676in}}%
\pgfpathlineto{\pgfqpoint{4.066470in}{1.557713in}}%
\pgfpathlineto{\pgfqpoint{4.068303in}{1.501620in}}%
\pgfpathlineto{\pgfqpoint{4.070138in}{1.586858in}}%
\pgfpathlineto{\pgfqpoint{4.071971in}{1.576545in}}%
\pgfpathlineto{\pgfqpoint{4.073806in}{1.447321in}}%
\pgfpathlineto{\pgfqpoint{4.075640in}{1.622413in}}%
\pgfpathlineto{\pgfqpoint{4.077474in}{1.547488in}}%
\pgfpathlineto{\pgfqpoint{4.081142in}{1.553774in}}%
\pgfpathlineto{\pgfqpoint{4.082976in}{1.225544in}}%
\pgfpathlineto{\pgfqpoint{4.084809in}{1.531003in}}%
\pgfpathlineto{\pgfqpoint{4.086642in}{1.455840in}}%
\pgfpathlineto{\pgfqpoint{4.088476in}{1.546259in}}%
\pgfpathlineto{\pgfqpoint{4.090309in}{1.365470in}}%
\pgfpathlineto{\pgfqpoint{4.092143in}{1.611599in}}%
\pgfpathlineto{\pgfqpoint{4.095810in}{1.299641in}}%
\pgfpathlineto{\pgfqpoint{4.097644in}{0.944463in}}%
\pgfpathlineto{\pgfqpoint{4.099480in}{1.302351in}}%
\pgfpathlineto{\pgfqpoint{4.101313in}{1.390199in}}%
\pgfpathlineto{\pgfqpoint{4.103146in}{1.319213in}}%
\pgfpathlineto{\pgfqpoint{4.104981in}{1.164307in}}%
\pgfpathlineto{\pgfqpoint{4.106815in}{1.356036in}}%
\pgfpathlineto{\pgfqpoint{4.108648in}{1.170279in}}%
\pgfpathlineto{\pgfqpoint{4.110482in}{1.220225in}}%
\pgfpathlineto{\pgfqpoint{4.112316in}{1.342248in}}%
\pgfpathlineto{\pgfqpoint{4.114150in}{1.191984in}}%
\pgfpathlineto{\pgfqpoint{4.115985in}{1.172475in}}%
\pgfpathlineto{\pgfqpoint{4.117818in}{1.168485in}}%
\pgfpathlineto{\pgfqpoint{4.123319in}{1.273408in}}%
\pgfpathlineto{\pgfqpoint{4.125154in}{1.267825in}}%
\pgfpathlineto{\pgfqpoint{4.126989in}{1.296116in}}%
\pgfpathlineto{\pgfqpoint{4.128822in}{1.529861in}}%
\pgfpathlineto{\pgfqpoint{4.130656in}{1.581513in}}%
\pgfpathlineto{\pgfqpoint{4.136157in}{1.188960in}}%
\pgfpathlineto{\pgfqpoint{4.139824in}{1.967367in}}%
\pgfpathlineto{\pgfqpoint{4.141657in}{2.147955in}}%
\pgfpathlineto{\pgfqpoint{4.143492in}{2.087696in}}%
\pgfpathlineto{\pgfqpoint{4.147158in}{2.331779in}}%
\pgfpathlineto{\pgfqpoint{4.148992in}{2.327137in}}%
\pgfpathlineto{\pgfqpoint{4.150827in}{2.360723in}}%
\pgfpathlineto{\pgfqpoint{4.154496in}{2.265875in}}%
\pgfpathlineto{\pgfqpoint{4.158163in}{2.399478in}}%
\pgfpathlineto{\pgfqpoint{4.159995in}{2.169321in}}%
\pgfpathlineto{\pgfqpoint{4.161830in}{2.220534in}}%
\pgfpathlineto{\pgfqpoint{4.163663in}{2.185191in}}%
\pgfpathlineto{\pgfqpoint{4.165497in}{2.347826in}}%
\pgfpathlineto{\pgfqpoint{4.167331in}{2.116314in}}%
\pgfpathlineto{\pgfqpoint{4.169164in}{2.238286in}}%
\pgfpathlineto{\pgfqpoint{4.170998in}{2.243894in}}%
\pgfpathlineto{\pgfqpoint{4.172834in}{2.218100in}}%
\pgfpathlineto{\pgfqpoint{4.174667in}{2.249628in}}%
\pgfpathlineto{\pgfqpoint{4.178336in}{1.245468in}}%
\pgfpathlineto{\pgfqpoint{4.180169in}{1.347517in}}%
\pgfpathlineto{\pgfqpoint{4.182002in}{1.072282in}}%
\pgfpathlineto{\pgfqpoint{4.183837in}{1.124674in}}%
\pgfpathlineto{\pgfqpoint{4.185671in}{1.104751in}}%
\pgfpathlineto{\pgfqpoint{4.187504in}{1.125113in}}%
\pgfpathlineto{\pgfqpoint{4.189339in}{1.255655in}}%
\pgfpathlineto{\pgfqpoint{4.191172in}{0.986981in}}%
\pgfpathlineto{\pgfqpoint{4.193005in}{1.023503in}}%
\pgfpathlineto{\pgfqpoint{4.194838in}{1.104136in}}%
\pgfpathlineto{\pgfqpoint{4.196673in}{1.130583in}}%
\pgfpathlineto{\pgfqpoint{4.198506in}{0.996830in}}%
\pgfpathlineto{\pgfqpoint{4.202200in}{1.083297in}}%
\pgfpathlineto{\pgfqpoint{4.204035in}{0.929069in}}%
\pgfpathlineto{\pgfqpoint{4.205869in}{1.217490in}}%
\pgfpathlineto{\pgfqpoint{4.207702in}{1.169627in}}%
\pgfpathlineto{\pgfqpoint{4.209536in}{1.069585in}}%
\pgfpathlineto{\pgfqpoint{4.211368in}{1.271438in}}%
\pgfpathlineto{\pgfqpoint{4.213201in}{1.271501in}}%
\pgfpathlineto{\pgfqpoint{4.215035in}{1.348596in}}%
\pgfpathlineto{\pgfqpoint{4.216869in}{1.361644in}}%
\pgfpathlineto{\pgfqpoint{4.218702in}{1.327180in}}%
\pgfpathlineto{\pgfqpoint{4.220537in}{1.238241in}}%
\pgfpathlineto{\pgfqpoint{4.222371in}{1.343189in}}%
\pgfpathlineto{\pgfqpoint{4.226039in}{1.094589in}}%
\pgfpathlineto{\pgfqpoint{4.227874in}{1.070024in}}%
\pgfpathlineto{\pgfqpoint{4.229707in}{0.991360in}}%
\pgfpathlineto{\pgfqpoint{4.231541in}{1.040465in}}%
\pgfpathlineto{\pgfqpoint{4.233374in}{1.252782in}}%
\pgfpathlineto{\pgfqpoint{4.235208in}{1.261125in}}%
\pgfpathlineto{\pgfqpoint{4.237041in}{1.161459in}}%
\pgfpathlineto{\pgfqpoint{4.238876in}{1.176565in}}%
\pgfpathlineto{\pgfqpoint{4.240710in}{1.310431in}}%
\pgfpathlineto{\pgfqpoint{4.242543in}{1.298148in}}%
\pgfpathlineto{\pgfqpoint{4.244379in}{1.008962in}}%
\pgfpathlineto{\pgfqpoint{4.248044in}{1.158762in}}%
\pgfpathlineto{\pgfqpoint{4.249878in}{1.154496in}}%
\pgfpathlineto{\pgfqpoint{4.251713in}{1.218017in}}%
\pgfpathlineto{\pgfqpoint{4.253547in}{1.084038in}}%
\pgfpathlineto{\pgfqpoint{4.255380in}{1.117448in}}%
\pgfpathlineto{\pgfqpoint{4.257214in}{1.178534in}}%
\pgfpathlineto{\pgfqpoint{4.259047in}{1.055357in}}%
\pgfpathlineto{\pgfqpoint{4.260882in}{1.310042in}}%
\pgfpathlineto{\pgfqpoint{4.262717in}{1.153053in}}%
\pgfpathlineto{\pgfqpoint{4.266386in}{1.243347in}}%
\pgfpathlineto{\pgfqpoint{4.268220in}{1.212735in}}%
\pgfpathlineto{\pgfqpoint{4.273723in}{1.433043in}}%
\pgfpathlineto{\pgfqpoint{4.275557in}{2.301543in}}%
\pgfpathlineto{\pgfqpoint{4.277392in}{2.229968in}}%
\pgfpathlineto{\pgfqpoint{4.279227in}{2.207573in}}%
\pgfpathlineto{\pgfqpoint{4.281060in}{2.328317in}}%
\pgfpathlineto{\pgfqpoint{4.282894in}{2.276037in}}%
\pgfpathlineto{\pgfqpoint{4.286560in}{2.077647in}}%
\pgfpathlineto{\pgfqpoint{4.288394in}{2.307829in}}%
\pgfpathlineto{\pgfqpoint{4.292060in}{2.333536in}}%
\pgfpathlineto{\pgfqpoint{4.293894in}{2.300452in}}%
\pgfpathlineto{\pgfqpoint{4.295728in}{2.325632in}}%
\pgfpathlineto{\pgfqpoint{4.297561in}{2.213156in}}%
\pgfpathlineto{\pgfqpoint{4.299395in}{2.285208in}}%
\pgfpathlineto{\pgfqpoint{4.303066in}{1.192611in}}%
\pgfpathlineto{\pgfqpoint{4.304899in}{1.243285in}}%
\pgfpathlineto{\pgfqpoint{4.306733in}{1.435603in}}%
\pgfpathlineto{\pgfqpoint{4.308566in}{1.345246in}}%
\pgfpathlineto{\pgfqpoint{4.312234in}{2.107293in}}%
\pgfpathlineto{\pgfqpoint{4.314067in}{2.162910in}}%
\pgfpathlineto{\pgfqpoint{4.315900in}{2.180512in}}%
\pgfpathlineto{\pgfqpoint{4.317734in}{2.163499in}}%
\pgfpathlineto{\pgfqpoint{4.319568in}{2.412526in}}%
\pgfpathlineto{\pgfqpoint{4.321401in}{2.469083in}}%
\pgfpathlineto{\pgfqpoint{4.325069in}{2.297253in}}%
\pgfpathlineto{\pgfqpoint{4.326902in}{2.462911in}}%
\pgfpathlineto{\pgfqpoint{4.328736in}{2.214248in}}%
\pgfpathlineto{\pgfqpoint{4.330571in}{2.145282in}}%
\pgfpathlineto{\pgfqpoint{4.332404in}{2.262312in}}%
\pgfpathlineto{\pgfqpoint{4.337905in}{2.039532in}}%
\pgfpathlineto{\pgfqpoint{4.339740in}{2.080231in}}%
\pgfpathlineto{\pgfqpoint{4.341576in}{2.219179in}}%
\pgfpathlineto{\pgfqpoint{4.343408in}{2.173762in}}%
\pgfpathlineto{\pgfqpoint{4.345242in}{2.096102in}}%
\pgfpathlineto{\pgfqpoint{4.347076in}{2.188541in}}%
\pgfpathlineto{\pgfqpoint{4.348910in}{2.155834in}}%
\pgfpathlineto{\pgfqpoint{4.350743in}{2.199757in}}%
\pgfpathlineto{\pgfqpoint{4.352577in}{2.290641in}}%
\pgfpathlineto{\pgfqpoint{4.354411in}{1.228305in}}%
\pgfpathlineto{\pgfqpoint{4.356245in}{1.269995in}}%
\pgfpathlineto{\pgfqpoint{4.358079in}{1.269004in}}%
\pgfpathlineto{\pgfqpoint{4.359912in}{1.320091in}}%
\pgfpathlineto{\pgfqpoint{4.361745in}{1.236359in}}%
\pgfpathlineto{\pgfqpoint{4.363581in}{1.376448in}}%
\pgfpathlineto{\pgfqpoint{4.365414in}{1.386058in}}%
\pgfpathlineto{\pgfqpoint{4.367247in}{1.491282in}}%
\pgfpathlineto{\pgfqpoint{4.369083in}{1.437899in}}%
\pgfpathlineto{\pgfqpoint{4.370916in}{1.427674in}}%
\pgfpathlineto{\pgfqpoint{4.372750in}{1.120446in}}%
\pgfpathlineto{\pgfqpoint{4.374585in}{1.203212in}}%
\pgfpathlineto{\pgfqpoint{4.376419in}{1.429631in}}%
\pgfpathlineto{\pgfqpoint{4.378253in}{1.358031in}}%
\pgfpathlineto{\pgfqpoint{4.380088in}{1.207880in}}%
\pgfpathlineto{\pgfqpoint{4.381923in}{1.388354in}}%
\pgfpathlineto{\pgfqpoint{4.383756in}{1.398253in}}%
\pgfpathlineto{\pgfqpoint{4.385591in}{1.305965in}}%
\pgfpathlineto{\pgfqpoint{4.387426in}{1.318598in}}%
\pgfpathlineto{\pgfqpoint{4.389259in}{1.313166in}}%
\pgfpathlineto{\pgfqpoint{4.391093in}{1.476239in}}%
\pgfpathlineto{\pgfqpoint{4.392929in}{1.337468in}}%
\pgfpathlineto{\pgfqpoint{4.394762in}{1.340755in}}%
\pgfpathlineto{\pgfqpoint{4.396597in}{1.354442in}}%
\pgfpathlineto{\pgfqpoint{4.398431in}{1.386209in}}%
\pgfpathlineto{\pgfqpoint{4.400264in}{1.308022in}}%
\pgfpathlineto{\pgfqpoint{4.403933in}{1.481647in}}%
\pgfpathlineto{\pgfqpoint{4.405798in}{1.339375in}}%
\pgfpathlineto{\pgfqpoint{4.407631in}{1.347868in}}%
\pgfpathlineto{\pgfqpoint{4.409466in}{1.256420in}}%
\pgfpathlineto{\pgfqpoint{4.411300in}{1.295325in}}%
\pgfpathlineto{\pgfqpoint{4.413134in}{1.297446in}}%
\pgfpathlineto{\pgfqpoint{4.414969in}{1.222809in}}%
\pgfpathlineto{\pgfqpoint{4.416802in}{1.453167in}}%
\pgfpathlineto{\pgfqpoint{4.418637in}{1.297910in}}%
\pgfpathlineto{\pgfqpoint{4.420471in}{1.393937in}}%
\pgfpathlineto{\pgfqpoint{4.422305in}{1.321798in}}%
\pgfpathlineto{\pgfqpoint{4.424138in}{1.379974in}}%
\pgfpathlineto{\pgfqpoint{4.425971in}{1.298889in}}%
\pgfpathlineto{\pgfqpoint{4.427805in}{1.402569in}}%
\pgfpathlineto{\pgfqpoint{4.429639in}{1.381830in}}%
\pgfpathlineto{\pgfqpoint{4.431472in}{1.341570in}}%
\pgfpathlineto{\pgfqpoint{4.435139in}{1.513790in}}%
\pgfpathlineto{\pgfqpoint{4.436973in}{1.574048in}}%
\pgfpathlineto{\pgfqpoint{4.438807in}{1.573521in}}%
\pgfpathlineto{\pgfqpoint{4.442474in}{1.284398in}}%
\pgfpathlineto{\pgfqpoint{4.444309in}{1.479740in}}%
\pgfpathlineto{\pgfqpoint{4.446143in}{1.241503in}}%
\pgfpathlineto{\pgfqpoint{4.447977in}{1.303669in}}%
\pgfpathlineto{\pgfqpoint{4.449812in}{1.283607in}}%
\pgfpathlineto{\pgfqpoint{4.451645in}{1.410097in}}%
\pgfpathlineto{\pgfqpoint{4.453478in}{1.260598in}}%
\pgfpathlineto{\pgfqpoint{4.458998in}{1.494456in}}%
\pgfpathlineto{\pgfqpoint{4.462665in}{1.328999in}}%
\pgfpathlineto{\pgfqpoint{4.464499in}{1.633642in}}%
\pgfpathlineto{\pgfqpoint{4.466333in}{2.176221in}}%
\pgfpathlineto{\pgfqpoint{4.469999in}{2.438973in}}%
\pgfpathlineto{\pgfqpoint{4.473668in}{2.248248in}}%
\pgfpathlineto{\pgfqpoint{4.477335in}{2.515566in}}%
\pgfpathlineto{\pgfqpoint{4.479184in}{2.207573in}}%
\pgfpathlineto{\pgfqpoint{4.481019in}{2.481930in}}%
\pgfpathlineto{\pgfqpoint{4.482853in}{2.321441in}}%
\pgfpathlineto{\pgfqpoint{4.484685in}{2.323813in}}%
\pgfpathlineto{\pgfqpoint{4.486520in}{2.311505in}}%
\pgfpathlineto{\pgfqpoint{4.488354in}{2.491152in}}%
\pgfpathlineto{\pgfqpoint{4.490186in}{2.382616in}}%
\pgfpathlineto{\pgfqpoint{4.492020in}{2.628719in}}%
\pgfpathlineto{\pgfqpoint{4.493854in}{2.442021in}}%
\pgfpathlineto{\pgfqpoint{4.495689in}{2.515504in}}%
\pgfpathlineto{\pgfqpoint{4.499358in}{2.405914in}}%
\pgfpathlineto{\pgfqpoint{4.501192in}{2.585849in}}%
\pgfpathlineto{\pgfqpoint{4.503027in}{2.337914in}}%
\pgfpathlineto{\pgfqpoint{4.506693in}{2.643937in}}%
\pgfpathlineto{\pgfqpoint{4.508528in}{2.462208in}}%
\pgfpathlineto{\pgfqpoint{4.510362in}{2.442498in}}%
\pgfpathlineto{\pgfqpoint{4.512194in}{2.406679in}}%
\pgfpathlineto{\pgfqpoint{4.514031in}{2.275749in}}%
\pgfpathlineto{\pgfqpoint{4.517699in}{2.494025in}}%
\pgfpathlineto{\pgfqpoint{4.521370in}{2.389127in}}%
\pgfpathlineto{\pgfqpoint{4.523203in}{2.303802in}}%
\pgfpathlineto{\pgfqpoint{4.525039in}{2.448784in}}%
\pgfpathlineto{\pgfqpoint{4.528706in}{2.320137in}}%
\pgfpathlineto{\pgfqpoint{4.530541in}{2.235702in}}%
\pgfpathlineto{\pgfqpoint{4.532377in}{2.276426in}}%
\pgfpathlineto{\pgfqpoint{4.534210in}{2.194563in}}%
\pgfpathlineto{\pgfqpoint{4.537878in}{2.315996in}}%
\pgfpathlineto{\pgfqpoint{4.541544in}{2.282360in}}%
\pgfpathlineto{\pgfqpoint{4.543377in}{2.193672in}}%
\pgfpathlineto{\pgfqpoint{4.545211in}{2.222591in}}%
\pgfpathlineto{\pgfqpoint{4.547045in}{2.590554in}}%
\pgfpathlineto{\pgfqpoint{4.548878in}{2.601331in}}%
\pgfpathlineto{\pgfqpoint{4.550711in}{2.835553in}}%
\pgfpathlineto{\pgfqpoint{4.552574in}{2.192380in}}%
\pgfpathlineto{\pgfqpoint{4.554408in}{2.299925in}}%
\pgfpathlineto{\pgfqpoint{4.556243in}{2.255888in}}%
\pgfpathlineto{\pgfqpoint{4.558077in}{2.241460in}}%
\pgfpathlineto{\pgfqpoint{4.559910in}{2.245864in}}%
\pgfpathlineto{\pgfqpoint{4.561745in}{2.306298in}}%
\pgfpathlineto{\pgfqpoint{4.565411in}{2.137027in}}%
\pgfpathlineto{\pgfqpoint{4.567246in}{2.415110in}}%
\pgfpathlineto{\pgfqpoint{4.569079in}{2.262136in}}%
\pgfpathlineto{\pgfqpoint{4.572747in}{2.401498in}}%
\pgfpathlineto{\pgfqpoint{4.574580in}{2.298846in}}%
\pgfpathlineto{\pgfqpoint{4.576415in}{2.314830in}}%
\pgfpathlineto{\pgfqpoint{4.578250in}{2.207260in}}%
\pgfpathlineto{\pgfqpoint{4.580083in}{2.321705in}}%
\pgfpathlineto{\pgfqpoint{4.581917in}{2.319610in}}%
\pgfpathlineto{\pgfqpoint{4.583750in}{2.331629in}}%
\pgfpathlineto{\pgfqpoint{4.585584in}{1.448914in}}%
\pgfpathlineto{\pgfqpoint{4.587417in}{1.422492in}}%
\pgfpathlineto{\pgfqpoint{4.589252in}{1.508997in}}%
\pgfpathlineto{\pgfqpoint{4.591086in}{1.481521in}}%
\pgfpathlineto{\pgfqpoint{4.592920in}{1.490454in}}%
\pgfpathlineto{\pgfqpoint{4.596587in}{1.432805in}}%
\pgfpathlineto{\pgfqpoint{4.598421in}{1.491659in}}%
\pgfpathlineto{\pgfqpoint{4.600254in}{1.447496in}}%
\pgfpathlineto{\pgfqpoint{4.602089in}{1.472099in}}%
\pgfpathlineto{\pgfqpoint{4.605756in}{1.294623in}}%
\pgfpathlineto{\pgfqpoint{4.607590in}{1.417323in}}%
\pgfpathlineto{\pgfqpoint{4.609425in}{2.295760in}}%
\pgfpathlineto{\pgfqpoint{4.611259in}{2.346295in}}%
\pgfpathlineto{\pgfqpoint{4.614929in}{2.163825in}}%
\pgfpathlineto{\pgfqpoint{4.616761in}{1.949301in}}%
\pgfpathlineto{\pgfqpoint{4.618596in}{2.319321in}}%
\pgfpathlineto{\pgfqpoint{4.620430in}{2.292962in}}%
\pgfpathlineto{\pgfqpoint{4.622262in}{2.119011in}}%
\pgfpathlineto{\pgfqpoint{4.624096in}{2.287705in}}%
\pgfpathlineto{\pgfqpoint{4.625930in}{2.334163in}}%
\pgfpathlineto{\pgfqpoint{4.629613in}{2.104821in}}%
\pgfpathlineto{\pgfqpoint{4.631447in}{2.167790in}}%
\pgfpathlineto{\pgfqpoint{4.633281in}{2.148017in}}%
\pgfpathlineto{\pgfqpoint{4.635115in}{1.205069in}}%
\pgfpathlineto{\pgfqpoint{4.636949in}{1.250072in}}%
\pgfpathlineto{\pgfqpoint{4.638784in}{1.253510in}}%
\pgfpathlineto{\pgfqpoint{4.642452in}{1.396672in}}%
\pgfpathlineto{\pgfqpoint{4.644286in}{1.339312in}}%
\pgfpathlineto{\pgfqpoint{4.646122in}{1.344456in}}%
\pgfpathlineto{\pgfqpoint{4.647955in}{1.353765in}}%
\pgfpathlineto{\pgfqpoint{4.651625in}{2.255700in}}%
\pgfpathlineto{\pgfqpoint{4.653458in}{2.160714in}}%
\pgfpathlineto{\pgfqpoint{4.655293in}{1.984091in}}%
\pgfpathlineto{\pgfqpoint{4.657854in}{2.046407in}}%
\pgfpathlineto{\pgfqpoint{4.659687in}{2.042330in}}%
\pgfpathlineto{\pgfqpoint{4.661523in}{2.114407in}}%
\pgfpathlineto{\pgfqpoint{4.663356in}{1.999660in}}%
\pgfpathlineto{\pgfqpoint{4.665190in}{1.964958in}}%
\pgfpathlineto{\pgfqpoint{4.667025in}{1.806941in}}%
\pgfpathlineto{\pgfqpoint{4.668858in}{1.970842in}}%
\pgfpathlineto{\pgfqpoint{4.672527in}{1.864678in}}%
\pgfpathlineto{\pgfqpoint{4.674360in}{2.003901in}}%
\pgfpathlineto{\pgfqpoint{4.676194in}{2.272048in}}%
\pgfpathlineto{\pgfqpoint{4.679861in}{2.104232in}}%
\pgfpathlineto{\pgfqpoint{4.681693in}{2.218150in}}%
\pgfpathlineto{\pgfqpoint{4.683527in}{2.132410in}}%
\pgfpathlineto{\pgfqpoint{4.685359in}{2.311354in}}%
\pgfpathlineto{\pgfqpoint{4.687192in}{2.137002in}}%
\pgfpathlineto{\pgfqpoint{4.689027in}{2.097043in}}%
\pgfpathlineto{\pgfqpoint{4.690860in}{2.243744in}}%
\pgfpathlineto{\pgfqpoint{4.694526in}{2.149071in}}%
\pgfpathlineto{\pgfqpoint{4.696359in}{1.854478in}}%
\pgfpathlineto{\pgfqpoint{4.698192in}{1.233474in}}%
\pgfpathlineto{\pgfqpoint{4.700028in}{1.407625in}}%
\pgfpathlineto{\pgfqpoint{4.701864in}{1.122905in}}%
\pgfpathlineto{\pgfqpoint{4.703698in}{1.129931in}}%
\pgfpathlineto{\pgfqpoint{4.705532in}{1.118351in}}%
\pgfpathlineto{\pgfqpoint{4.707367in}{1.212120in}}%
\pgfpathlineto{\pgfqpoint{4.709200in}{0.858083in}}%
\pgfpathlineto{\pgfqpoint{4.712867in}{1.312601in}}%
\pgfpathlineto{\pgfqpoint{4.714701in}{1.254488in}}%
\pgfpathlineto{\pgfqpoint{4.716533in}{1.062471in}}%
\pgfpathlineto{\pgfqpoint{4.718366in}{1.239671in}}%
\pgfpathlineto{\pgfqpoint{4.720199in}{1.284247in}}%
\pgfpathlineto{\pgfqpoint{4.722032in}{1.356625in}}%
\pgfpathlineto{\pgfqpoint{4.723865in}{1.988996in}}%
\pgfpathlineto{\pgfqpoint{4.725698in}{2.017413in}}%
\pgfpathlineto{\pgfqpoint{4.727532in}{1.936378in}}%
\pgfpathlineto{\pgfqpoint{4.729366in}{1.976362in}}%
\pgfpathlineto{\pgfqpoint{4.731200in}{2.057109in}}%
\pgfpathlineto{\pgfqpoint{4.733032in}{2.232440in}}%
\pgfpathlineto{\pgfqpoint{4.734866in}{2.219505in}}%
\pgfpathlineto{\pgfqpoint{4.736700in}{2.248160in}}%
\pgfpathlineto{\pgfqpoint{4.738533in}{2.256854in}}%
\pgfpathlineto{\pgfqpoint{4.740366in}{2.181954in}}%
\pgfpathlineto{\pgfqpoint{4.742199in}{2.159735in}}%
\pgfpathlineto{\pgfqpoint{4.744033in}{2.093016in}}%
\pgfpathlineto{\pgfqpoint{4.745866in}{2.253203in}}%
\pgfpathlineto{\pgfqpoint{4.747700in}{2.146110in}}%
\pgfpathlineto{\pgfqpoint{4.749534in}{2.152747in}}%
\pgfpathlineto{\pgfqpoint{4.751367in}{2.187161in}}%
\pgfpathlineto{\pgfqpoint{4.753202in}{2.157778in}}%
\pgfpathlineto{\pgfqpoint{4.755035in}{2.196972in}}%
\pgfpathlineto{\pgfqpoint{4.756868in}{1.926241in}}%
\pgfpathlineto{\pgfqpoint{4.758701in}{2.092049in}}%
\pgfpathlineto{\pgfqpoint{4.760535in}{2.138056in}}%
\pgfpathlineto{\pgfqpoint{4.764201in}{1.274587in}}%
\pgfpathlineto{\pgfqpoint{4.766036in}{1.197479in}}%
\pgfpathlineto{\pgfqpoint{4.767868in}{1.317570in}}%
\pgfpathlineto{\pgfqpoint{4.771535in}{1.194543in}}%
\pgfpathlineto{\pgfqpoint{4.773368in}{2.037838in}}%
\pgfpathlineto{\pgfqpoint{4.777037in}{2.132824in}}%
\pgfpathlineto{\pgfqpoint{4.778871in}{2.149699in}}%
\pgfpathlineto{\pgfqpoint{4.780704in}{2.062604in}}%
\pgfpathlineto{\pgfqpoint{4.782543in}{2.207072in}}%
\pgfpathlineto{\pgfqpoint{4.784383in}{2.222528in}}%
\pgfpathlineto{\pgfqpoint{4.788065in}{1.263270in}}%
\pgfpathlineto{\pgfqpoint{4.789906in}{1.241352in}}%
\pgfpathlineto{\pgfqpoint{4.791746in}{1.111802in}}%
\pgfpathlineto{\pgfqpoint{4.795426in}{1.164395in}}%
\pgfpathlineto{\pgfqpoint{4.797636in}{1.171948in}}%
\pgfpathlineto{\pgfqpoint{4.799475in}{2.813949in}}%
\pgfpathlineto{\pgfqpoint{4.801317in}{1.208469in}}%
\pgfpathlineto{\pgfqpoint{4.803157in}{1.281286in}}%
\pgfpathlineto{\pgfqpoint{4.804997in}{2.003186in}}%
\pgfpathlineto{\pgfqpoint{4.806836in}{2.013674in}}%
\pgfpathlineto{\pgfqpoint{4.810517in}{1.872117in}}%
\pgfpathlineto{\pgfqpoint{4.812358in}{1.914247in}}%
\pgfpathlineto{\pgfqpoint{4.814200in}{2.164616in}}%
\pgfpathlineto{\pgfqpoint{4.816040in}{2.025907in}}%
\pgfpathlineto{\pgfqpoint{4.817881in}{2.000689in}}%
\pgfpathlineto{\pgfqpoint{4.819721in}{2.196295in}}%
\pgfpathlineto{\pgfqpoint{4.821560in}{1.897711in}}%
\pgfpathlineto{\pgfqpoint{4.823400in}{1.955122in}}%
\pgfpathlineto{\pgfqpoint{4.825241in}{2.069454in}}%
\pgfpathlineto{\pgfqpoint{4.827080in}{2.062892in}}%
\pgfpathlineto{\pgfqpoint{4.828920in}{2.021930in}}%
\pgfpathlineto{\pgfqpoint{4.830759in}{2.116163in}}%
\pgfpathlineto{\pgfqpoint{4.836650in}{2.209581in}}%
\pgfpathlineto{\pgfqpoint{4.838489in}{2.140352in}}%
\pgfpathlineto{\pgfqpoint{4.840330in}{2.135760in}}%
\pgfpathlineto{\pgfqpoint{4.842169in}{2.034488in}}%
\pgfpathlineto{\pgfqpoint{4.844009in}{2.063683in}}%
\pgfpathlineto{\pgfqpoint{4.846214in}{2.033045in}}%
\pgfpathlineto{\pgfqpoint{4.848056in}{2.254031in}}%
\pgfpathlineto{\pgfqpoint{4.849893in}{2.000338in}}%
\pgfpathlineto{\pgfqpoint{4.851730in}{1.916831in}}%
\pgfpathlineto{\pgfqpoint{4.853564in}{1.924008in}}%
\pgfpathlineto{\pgfqpoint{4.857232in}{2.171077in}}%
\pgfpathlineto{\pgfqpoint{4.859065in}{2.124456in}}%
\pgfpathlineto{\pgfqpoint{4.859065in}{2.124456in}}%
\pgfusepath{stroke}%
\end{pgfscope}%
\begin{pgfscope}%
\pgfsetrectcap%
\pgfsetmiterjoin%
\pgfsetlinewidth{0.803000pt}%
\definecolor{currentstroke}{rgb}{0.000000,0.000000,0.000000}%
\pgfsetstrokecolor{currentstroke}%
\pgfsetdash{}{0pt}%
\pgfpathmoveto{\pgfqpoint{0.667540in}{0.539544in}}%
\pgfpathlineto{\pgfqpoint{0.667540in}{2.944887in}}%
\pgfusepath{stroke}%
\end{pgfscope}%
\begin{pgfscope}%
\pgfsetrectcap%
\pgfsetmiterjoin%
\pgfsetlinewidth{0.803000pt}%
\definecolor{currentstroke}{rgb}{0.000000,0.000000,0.000000}%
\pgfsetstrokecolor{currentstroke}%
\pgfsetdash{}{0pt}%
\pgfpathmoveto{\pgfqpoint{5.058662in}{0.539544in}}%
\pgfpathlineto{\pgfqpoint{5.058662in}{2.944887in}}%
\pgfusepath{stroke}%
\end{pgfscope}%
\begin{pgfscope}%
\pgfsetrectcap%
\pgfsetmiterjoin%
\pgfsetlinewidth{0.803000pt}%
\definecolor{currentstroke}{rgb}{0.000000,0.000000,0.000000}%
\pgfsetstrokecolor{currentstroke}%
\pgfsetdash{}{0pt}%
\pgfpathmoveto{\pgfqpoint{0.667540in}{0.539544in}}%
\pgfpathlineto{\pgfqpoint{5.058662in}{0.539544in}}%
\pgfusepath{stroke}%
\end{pgfscope}%
\begin{pgfscope}%
\pgfsetrectcap%
\pgfsetmiterjoin%
\pgfsetlinewidth{0.803000pt}%
\definecolor{currentstroke}{rgb}{0.000000,0.000000,0.000000}%
\pgfsetstrokecolor{currentstroke}%
\pgfsetdash{}{0pt}%
\pgfpathmoveto{\pgfqpoint{0.667540in}{2.944887in}}%
\pgfpathlineto{\pgfqpoint{5.058662in}{2.944887in}}%
\pgfusepath{stroke}%
\end{pgfscope}%
\begin{pgfscope}%
\pgfsetbuttcap%
\pgfsetmiterjoin%
\definecolor{currentfill}{rgb}{1.000000,1.000000,1.000000}%
\pgfsetfillcolor{currentfill}%
\pgfsetfillopacity{0.800000}%
\pgfsetlinewidth{1.003750pt}%
\definecolor{currentstroke}{rgb}{0.800000,0.800000,0.800000}%
\pgfsetstrokecolor{currentstroke}%
\pgfsetstrokeopacity{0.800000}%
\pgfsetdash{}{0pt}%
\pgfpathmoveto{\pgfqpoint{0.745318in}{2.701109in}}%
\pgfpathlineto{\pgfqpoint{2.092873in}{2.701109in}}%
\pgfpathquadraticcurveto{\pgfqpoint{2.115096in}{2.701109in}}{\pgfqpoint{2.115096in}{2.723331in}}%
\pgfpathlineto{\pgfqpoint{2.115096in}{2.867109in}}%
\pgfpathquadraticcurveto{\pgfqpoint{2.115096in}{2.889331in}}{\pgfqpoint{2.092873in}{2.889331in}}%
\pgfpathlineto{\pgfqpoint{0.745318in}{2.889331in}}%
\pgfpathquadraticcurveto{\pgfqpoint{0.723095in}{2.889331in}}{\pgfqpoint{0.723095in}{2.867109in}}%
\pgfpathlineto{\pgfqpoint{0.723095in}{2.723331in}}%
\pgfpathquadraticcurveto{\pgfqpoint{0.723095in}{2.701109in}}{\pgfqpoint{0.745318in}{2.701109in}}%
\pgfpathlineto{\pgfqpoint{0.745318in}{2.701109in}}%
\pgfpathclose%
\pgfusepath{stroke,fill}%
\end{pgfscope}%
\begin{pgfscope}%
\pgfsetrectcap%
\pgfsetroundjoin%
\pgfsetlinewidth{0.501875pt}%
\definecolor{currentstroke}{rgb}{0.121569,0.466667,0.705882}%
\pgfsetstrokecolor{currentstroke}%
\pgfsetstrokeopacity{0.700000}%
\pgfsetdash{}{0pt}%
\pgfpathmoveto{\pgfqpoint{0.767540in}{2.805998in}}%
\pgfpathlineto{\pgfqpoint{0.878651in}{2.805998in}}%
\pgfpathlineto{\pgfqpoint{0.989762in}{2.805998in}}%
\pgfusepath{stroke}%
\end{pgfscope}%
\begin{pgfscope}%
\definecolor{textcolor}{rgb}{0.000000,0.000000,0.000000}%
\pgfsetstrokecolor{textcolor}%
\pgfsetfillcolor{textcolor}%
\pgftext[x=1.078651in,y=2.767109in,left,base]{\color{textcolor}\rmfamily\fontsize{8.000000}{9.600000}\selectfont DUT vs KS34470A}%
\end{pgfscope}%
\end{pgfpicture}%
\makeatother%
\endgroup%

    \caption{Popcorn noise of a refurbished LM399 (\#15) over a period of \qty{15}{\minute}.}
    \label{fig:fake_lm399_popcorn_noise}
\end{figure}

TODO: Chinese/Ebay Zeners. Welded legs. Photots. Decap one of those.

%\begin{figure}[h]
%    \centering
    %\import{figures/}{dgDrive_protocol.tex}
%\end{figure}

%\subsection{Current Sources}
%Discuss Op amp choice (AD797)

%\subsection{Temperature Coeeficient}
%Discuss each section (Reference, DAC, Buffer/Divider, Filter, CC)
%\subsubsection{Voltage Reference}
%\subsubsection{DAC}
%\subsubsection{Divider}
%\subsubsection{Filter}
%Choice of components. Leakage current, size of resistor (input bias current of AD797), size of capacitor

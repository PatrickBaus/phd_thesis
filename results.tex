\chapter{Results}
\section{Laser Current Driver}
\subsection{Zener Diode Selection}
Early tests of the LM399 Zener diode as a reference have confirmed, what the data sheet \cite{datasheet_LM399} already suggest in the 'Low Frequency Noise Voltage' plot. There are random bi-stable voltage step changes. This phenomenon is called burst noise or popcorn noise.

\begin{figure}[ht]
    \centering
    %% Creator: Matplotlib, PGF backend
%%
%% To include the figure in your LaTeX document, write
%%   \input{<filename>.pgf}
%%
%% Make sure the required packages are loaded in your preamble
%%   \usepackage{pgf}
%%
%% Also ensure that all the required font packages are loaded; for instance,
%% the lmodern package is sometimes necessary when using math font.
%%   \usepackage{lmodern}
%%
%% Figures using additional raster images can only be included by \input if
%% they are in the same directory as the main LaTeX file. For loading figures
%% from other directories you can use the `import` package
%%   \usepackage{import}
%%
%% and then include the figures with
%%   \import{<path to file>}{<filename>.pgf}
%%
%% Matplotlib used the following preamble
%%   \usepackage{siunitx}
%%   \sisetup{per-mode = symbol}%
%%   \usepackage{fontspec}
%%   \makeatletter\@ifpackageloaded{underscore}{}{\usepackage[strings]{underscore}}\makeatother
%%
\begingroup%
\makeatletter%
\begin{pgfpicture}%
\pgfpathrectangle{\pgfpointorigin}{\pgfqpoint{5.150788in}{3.183362in}}%
\pgfusepath{use as bounding box, clip}%
\begin{pgfscope}%
\pgfsetbuttcap%
\pgfsetmiterjoin%
\definecolor{currentfill}{rgb}{1.000000,1.000000,1.000000}%
\pgfsetfillcolor{currentfill}%
\pgfsetlinewidth{0.000000pt}%
\definecolor{currentstroke}{rgb}{1.000000,1.000000,1.000000}%
\pgfsetstrokecolor{currentstroke}%
\pgfsetdash{}{0pt}%
\pgfpathmoveto{\pgfqpoint{0.000000in}{0.000000in}}%
\pgfpathlineto{\pgfqpoint{5.150788in}{0.000000in}}%
\pgfpathlineto{\pgfqpoint{5.150788in}{3.183362in}}%
\pgfpathlineto{\pgfqpoint{0.000000in}{3.183362in}}%
\pgfpathlineto{\pgfqpoint{0.000000in}{0.000000in}}%
\pgfpathclose%
\pgfusepath{fill}%
\end{pgfscope}%
\begin{pgfscope}%
\pgfsetbuttcap%
\pgfsetmiterjoin%
\definecolor{currentfill}{rgb}{1.000000,1.000000,1.000000}%
\pgfsetfillcolor{currentfill}%
\pgfsetlinewidth{0.000000pt}%
\definecolor{currentstroke}{rgb}{0.000000,0.000000,0.000000}%
\pgfsetstrokecolor{currentstroke}%
\pgfsetstrokeopacity{0.000000}%
\pgfsetdash{}{0pt}%
\pgfpathmoveto{\pgfqpoint{0.484581in}{2.334497in}}%
\pgfpathlineto{\pgfqpoint{5.000788in}{2.334497in}}%
\pgfpathlineto{\pgfqpoint{5.000788in}{2.909119in}}%
\pgfpathlineto{\pgfqpoint{0.484581in}{2.909119in}}%
\pgfpathlineto{\pgfqpoint{0.484581in}{2.334497in}}%
\pgfpathclose%
\pgfusepath{fill}%
\end{pgfscope}%
\begin{pgfscope}%
\pgfsetbuttcap%
\pgfsetroundjoin%
\definecolor{currentfill}{rgb}{0.000000,0.000000,0.000000}%
\pgfsetfillcolor{currentfill}%
\pgfsetlinewidth{0.803000pt}%
\definecolor{currentstroke}{rgb}{0.000000,0.000000,0.000000}%
\pgfsetstrokecolor{currentstroke}%
\pgfsetdash{}{0pt}%
\pgfsys@defobject{currentmarker}{\pgfqpoint{0.000000in}{-0.048611in}}{\pgfqpoint{0.000000in}{0.000000in}}{%
\pgfpathmoveto{\pgfqpoint{0.000000in}{0.000000in}}%
\pgfpathlineto{\pgfqpoint{0.000000in}{-0.048611in}}%
\pgfusepath{stroke,fill}%
}%
\begin{pgfscope}%
\pgfsys@transformshift{0.689546in}{2.334497in}%
\pgfsys@useobject{currentmarker}{}%
\end{pgfscope}%
\end{pgfscope}%
\begin{pgfscope}%
\pgfsetbuttcap%
\pgfsetroundjoin%
\definecolor{currentfill}{rgb}{0.000000,0.000000,0.000000}%
\pgfsetfillcolor{currentfill}%
\pgfsetlinewidth{0.803000pt}%
\definecolor{currentstroke}{rgb}{0.000000,0.000000,0.000000}%
\pgfsetstrokecolor{currentstroke}%
\pgfsetdash{}{0pt}%
\pgfsys@defobject{currentmarker}{\pgfqpoint{0.000000in}{-0.048611in}}{\pgfqpoint{0.000000in}{0.000000in}}{%
\pgfpathmoveto{\pgfqpoint{0.000000in}{0.000000in}}%
\pgfpathlineto{\pgfqpoint{0.000000in}{-0.048611in}}%
\pgfusepath{stroke,fill}%
}%
\begin{pgfscope}%
\pgfsys@transformshift{1.202878in}{2.334497in}%
\pgfsys@useobject{currentmarker}{}%
\end{pgfscope}%
\end{pgfscope}%
\begin{pgfscope}%
\pgfsetbuttcap%
\pgfsetroundjoin%
\definecolor{currentfill}{rgb}{0.000000,0.000000,0.000000}%
\pgfsetfillcolor{currentfill}%
\pgfsetlinewidth{0.803000pt}%
\definecolor{currentstroke}{rgb}{0.000000,0.000000,0.000000}%
\pgfsetstrokecolor{currentstroke}%
\pgfsetdash{}{0pt}%
\pgfsys@defobject{currentmarker}{\pgfqpoint{0.000000in}{-0.048611in}}{\pgfqpoint{0.000000in}{0.000000in}}{%
\pgfpathmoveto{\pgfqpoint{0.000000in}{0.000000in}}%
\pgfpathlineto{\pgfqpoint{0.000000in}{-0.048611in}}%
\pgfusepath{stroke,fill}%
}%
\begin{pgfscope}%
\pgfsys@transformshift{1.716211in}{2.334497in}%
\pgfsys@useobject{currentmarker}{}%
\end{pgfscope}%
\end{pgfscope}%
\begin{pgfscope}%
\pgfsetbuttcap%
\pgfsetroundjoin%
\definecolor{currentfill}{rgb}{0.000000,0.000000,0.000000}%
\pgfsetfillcolor{currentfill}%
\pgfsetlinewidth{0.803000pt}%
\definecolor{currentstroke}{rgb}{0.000000,0.000000,0.000000}%
\pgfsetstrokecolor{currentstroke}%
\pgfsetdash{}{0pt}%
\pgfsys@defobject{currentmarker}{\pgfqpoint{0.000000in}{-0.048611in}}{\pgfqpoint{0.000000in}{0.000000in}}{%
\pgfpathmoveto{\pgfqpoint{0.000000in}{0.000000in}}%
\pgfpathlineto{\pgfqpoint{0.000000in}{-0.048611in}}%
\pgfusepath{stroke,fill}%
}%
\begin{pgfscope}%
\pgfsys@transformshift{2.229543in}{2.334497in}%
\pgfsys@useobject{currentmarker}{}%
\end{pgfscope}%
\end{pgfscope}%
\begin{pgfscope}%
\pgfsetbuttcap%
\pgfsetroundjoin%
\definecolor{currentfill}{rgb}{0.000000,0.000000,0.000000}%
\pgfsetfillcolor{currentfill}%
\pgfsetlinewidth{0.803000pt}%
\definecolor{currentstroke}{rgb}{0.000000,0.000000,0.000000}%
\pgfsetstrokecolor{currentstroke}%
\pgfsetdash{}{0pt}%
\pgfsys@defobject{currentmarker}{\pgfqpoint{0.000000in}{-0.048611in}}{\pgfqpoint{0.000000in}{0.000000in}}{%
\pgfpathmoveto{\pgfqpoint{0.000000in}{0.000000in}}%
\pgfpathlineto{\pgfqpoint{0.000000in}{-0.048611in}}%
\pgfusepath{stroke,fill}%
}%
\begin{pgfscope}%
\pgfsys@transformshift{2.742876in}{2.334497in}%
\pgfsys@useobject{currentmarker}{}%
\end{pgfscope}%
\end{pgfscope}%
\begin{pgfscope}%
\pgfsetbuttcap%
\pgfsetroundjoin%
\definecolor{currentfill}{rgb}{0.000000,0.000000,0.000000}%
\pgfsetfillcolor{currentfill}%
\pgfsetlinewidth{0.803000pt}%
\definecolor{currentstroke}{rgb}{0.000000,0.000000,0.000000}%
\pgfsetstrokecolor{currentstroke}%
\pgfsetdash{}{0pt}%
\pgfsys@defobject{currentmarker}{\pgfqpoint{0.000000in}{-0.048611in}}{\pgfqpoint{0.000000in}{0.000000in}}{%
\pgfpathmoveto{\pgfqpoint{0.000000in}{0.000000in}}%
\pgfpathlineto{\pgfqpoint{0.000000in}{-0.048611in}}%
\pgfusepath{stroke,fill}%
}%
\begin{pgfscope}%
\pgfsys@transformshift{3.256208in}{2.334497in}%
\pgfsys@useobject{currentmarker}{}%
\end{pgfscope}%
\end{pgfscope}%
\begin{pgfscope}%
\pgfsetbuttcap%
\pgfsetroundjoin%
\definecolor{currentfill}{rgb}{0.000000,0.000000,0.000000}%
\pgfsetfillcolor{currentfill}%
\pgfsetlinewidth{0.803000pt}%
\definecolor{currentstroke}{rgb}{0.000000,0.000000,0.000000}%
\pgfsetstrokecolor{currentstroke}%
\pgfsetdash{}{0pt}%
\pgfsys@defobject{currentmarker}{\pgfqpoint{0.000000in}{-0.048611in}}{\pgfqpoint{0.000000in}{0.000000in}}{%
\pgfpathmoveto{\pgfqpoint{0.000000in}{0.000000in}}%
\pgfpathlineto{\pgfqpoint{0.000000in}{-0.048611in}}%
\pgfusepath{stroke,fill}%
}%
\begin{pgfscope}%
\pgfsys@transformshift{3.769541in}{2.334497in}%
\pgfsys@useobject{currentmarker}{}%
\end{pgfscope}%
\end{pgfscope}%
\begin{pgfscope}%
\pgfsetbuttcap%
\pgfsetroundjoin%
\definecolor{currentfill}{rgb}{0.000000,0.000000,0.000000}%
\pgfsetfillcolor{currentfill}%
\pgfsetlinewidth{0.803000pt}%
\definecolor{currentstroke}{rgb}{0.000000,0.000000,0.000000}%
\pgfsetstrokecolor{currentstroke}%
\pgfsetdash{}{0pt}%
\pgfsys@defobject{currentmarker}{\pgfqpoint{0.000000in}{-0.048611in}}{\pgfqpoint{0.000000in}{0.000000in}}{%
\pgfpathmoveto{\pgfqpoint{0.000000in}{0.000000in}}%
\pgfpathlineto{\pgfqpoint{0.000000in}{-0.048611in}}%
\pgfusepath{stroke,fill}%
}%
\begin{pgfscope}%
\pgfsys@transformshift{4.282873in}{2.334497in}%
\pgfsys@useobject{currentmarker}{}%
\end{pgfscope}%
\end{pgfscope}%
\begin{pgfscope}%
\pgfsetbuttcap%
\pgfsetroundjoin%
\definecolor{currentfill}{rgb}{0.000000,0.000000,0.000000}%
\pgfsetfillcolor{currentfill}%
\pgfsetlinewidth{0.803000pt}%
\definecolor{currentstroke}{rgb}{0.000000,0.000000,0.000000}%
\pgfsetstrokecolor{currentstroke}%
\pgfsetdash{}{0pt}%
\pgfsys@defobject{currentmarker}{\pgfqpoint{0.000000in}{-0.048611in}}{\pgfqpoint{0.000000in}{0.000000in}}{%
\pgfpathmoveto{\pgfqpoint{0.000000in}{0.000000in}}%
\pgfpathlineto{\pgfqpoint{0.000000in}{-0.048611in}}%
\pgfusepath{stroke,fill}%
}%
\begin{pgfscope}%
\pgfsys@transformshift{4.796206in}{2.334497in}%
\pgfsys@useobject{currentmarker}{}%
\end{pgfscope}%
\end{pgfscope}%
\begin{pgfscope}%
\pgfsetbuttcap%
\pgfsetroundjoin%
\definecolor{currentfill}{rgb}{0.000000,0.000000,0.000000}%
\pgfsetfillcolor{currentfill}%
\pgfsetlinewidth{0.803000pt}%
\definecolor{currentstroke}{rgb}{0.000000,0.000000,0.000000}%
\pgfsetstrokecolor{currentstroke}%
\pgfsetdash{}{0pt}%
\pgfsys@defobject{currentmarker}{\pgfqpoint{-0.048611in}{0.000000in}}{\pgfqpoint{-0.000000in}{0.000000in}}{%
\pgfpathmoveto{\pgfqpoint{-0.000000in}{0.000000in}}%
\pgfpathlineto{\pgfqpoint{-0.048611in}{0.000000in}}%
\pgfusepath{stroke,fill}%
}%
\begin{pgfscope}%
\pgfsys@transformshift{0.484581in}{2.512496in}%
\pgfsys@useobject{currentmarker}{}%
\end{pgfscope}%
\end{pgfscope}%
\begin{pgfscope}%
\definecolor{textcolor}{rgb}{0.000000,0.000000,0.000000}%
\pgfsetstrokecolor{textcolor}%
\pgfsetfillcolor{textcolor}%
\pgftext[x=0.328331in, y=2.473941in, left, base]{\color{textcolor}\rmfamily\fontsize{8.000000}{9.600000}\selectfont \(\displaystyle {0}\)}%
\end{pgfscope}%
\begin{pgfscope}%
\pgfsetbuttcap%
\pgfsetroundjoin%
\definecolor{currentfill}{rgb}{0.000000,0.000000,0.000000}%
\pgfsetfillcolor{currentfill}%
\pgfsetlinewidth{0.803000pt}%
\definecolor{currentstroke}{rgb}{0.000000,0.000000,0.000000}%
\pgfsetstrokecolor{currentstroke}%
\pgfsetdash{}{0pt}%
\pgfsys@defobject{currentmarker}{\pgfqpoint{-0.048611in}{0.000000in}}{\pgfqpoint{-0.000000in}{0.000000in}}{%
\pgfpathmoveto{\pgfqpoint{-0.000000in}{0.000000in}}%
\pgfpathlineto{\pgfqpoint{-0.048611in}{0.000000in}}%
\pgfusepath{stroke,fill}%
}%
\begin{pgfscope}%
\pgfsys@transformshift{0.484581in}{2.716037in}%
\pgfsys@useobject{currentmarker}{}%
\end{pgfscope}%
\end{pgfscope}%
\begin{pgfscope}%
\definecolor{textcolor}{rgb}{0.000000,0.000000,0.000000}%
\pgfsetstrokecolor{textcolor}%
\pgfsetfillcolor{textcolor}%
\pgftext[x=0.328331in, y=2.677482in, left, base]{\color{textcolor}\rmfamily\fontsize{8.000000}{9.600000}\selectfont \(\displaystyle {5}\)}%
\end{pgfscope}%
\begin{pgfscope}%
\definecolor{textcolor}{rgb}{0.000000,0.000000,0.000000}%
\pgfsetstrokecolor{textcolor}%
\pgfsetfillcolor{textcolor}%
\pgftext[x=0.484581in,y=2.950785in,left,base]{\color{textcolor}\rmfamily\fontsize{8.000000}{9.600000}\selectfont \(\displaystyle \times{10^{\ensuremath{-}6}}{}\)}%
\end{pgfscope}%
\begin{pgfscope}%
\pgfpathrectangle{\pgfqpoint{0.484581in}{2.334497in}}{\pgfqpoint{4.516206in}{0.574622in}}%
\pgfusepath{clip}%
\pgfsetrectcap%
\pgfsetroundjoin%
\pgfsetlinewidth{0.501875pt}%
\definecolor{currentstroke}{rgb}{0.003922,0.450980,0.698039}%
\pgfsetstrokecolor{currentstroke}%
\pgfsetstrokeopacity{0.700000}%
\pgfsetdash{}{0pt}%
\pgfpathmoveto{\pgfqpoint{0.689863in}{2.509930in}}%
\pgfpathlineto{\pgfqpoint{0.691573in}{2.569128in}}%
\pgfpathlineto{\pgfqpoint{0.694995in}{2.478459in}}%
\pgfpathlineto{\pgfqpoint{0.699275in}{2.580305in}}%
\pgfpathlineto{\pgfqpoint{0.703553in}{2.518208in}}%
\pgfpathlineto{\pgfqpoint{0.710398in}{2.564054in}}%
\pgfpathlineto{\pgfqpoint{0.712966in}{2.460398in}}%
\pgfpathlineto{\pgfqpoint{0.718959in}{2.382413in}}%
\pgfpathlineto{\pgfqpoint{0.722378in}{2.526844in}}%
\pgfpathlineto{\pgfqpoint{0.728363in}{2.563512in}}%
\pgfpathlineto{\pgfqpoint{0.730929in}{2.446928in}}%
\pgfpathlineto{\pgfqpoint{0.734351in}{2.508420in}}%
\pgfpathlineto{\pgfqpoint{0.740340in}{2.462634in}}%
\pgfpathlineto{\pgfqpoint{0.742908in}{2.509569in}}%
\pgfpathlineto{\pgfqpoint{0.748037in}{2.531618in}}%
\pgfpathlineto{\pgfqpoint{0.752314in}{2.422041in}}%
\pgfpathlineto{\pgfqpoint{0.755738in}{2.514704in}}%
\pgfpathlineto{\pgfqpoint{0.760009in}{2.489757in}}%
\pgfpathlineto{\pgfqpoint{0.764288in}{2.546054in}}%
\pgfpathlineto{\pgfqpoint{0.769423in}{2.581573in}}%
\pgfpathlineto{\pgfqpoint{0.771990in}{2.483110in}}%
\pgfpathlineto{\pgfqpoint{0.777124in}{2.481781in}}%
\pgfpathlineto{\pgfqpoint{0.780547in}{2.572209in}}%
\pgfpathlineto{\pgfqpoint{0.786534in}{2.392199in}}%
\pgfpathlineto{\pgfqpoint{0.789953in}{2.555597in}}%
\pgfpathlineto{\pgfqpoint{0.795939in}{2.458526in}}%
\pgfpathlineto{\pgfqpoint{0.799360in}{2.547927in}}%
\pgfpathlineto{\pgfqpoint{0.803633in}{2.421859in}}%
\pgfpathlineto{\pgfqpoint{0.806199in}{2.515549in}}%
\pgfpathlineto{\pgfqpoint{0.810475in}{2.476526in}}%
\pgfpathlineto{\pgfqpoint{0.814751in}{2.453994in}}%
\pgfpathlineto{\pgfqpoint{0.820735in}{2.649770in}}%
\pgfpathlineto{\pgfqpoint{0.823303in}{2.515246in}}%
\pgfpathlineto{\pgfqpoint{0.828440in}{2.617574in}}%
\pgfpathlineto{\pgfqpoint{0.832717in}{2.490117in}}%
\pgfpathlineto{\pgfqpoint{0.839565in}{2.452968in}}%
\pgfpathlineto{\pgfqpoint{0.842131in}{2.529864in}}%
\pgfpathlineto{\pgfqpoint{0.848114in}{2.449857in}}%
\pgfpathlineto{\pgfqpoint{0.852384in}{2.551822in}}%
\pgfpathlineto{\pgfqpoint{0.856662in}{2.459794in}}%
\pgfpathlineto{\pgfqpoint{0.859233in}{2.558859in}}%
\pgfpathlineto{\pgfqpoint{0.862659in}{2.464323in}}%
\pgfpathlineto{\pgfqpoint{0.868650in}{2.535180in}}%
\pgfpathlineto{\pgfqpoint{0.872926in}{2.477250in}}%
\pgfpathlineto{\pgfqpoint{0.876348in}{2.467797in}}%
\pgfpathlineto{\pgfqpoint{0.880627in}{2.539679in}}%
\pgfpathlineto{\pgfqpoint{0.884050in}{2.404099in}}%
\pgfpathlineto{\pgfqpoint{0.887469in}{2.536629in}}%
\pgfpathlineto{\pgfqpoint{0.895167in}{2.454266in}}%
\pgfpathlineto{\pgfqpoint{0.896877in}{2.573658in}}%
\pgfpathlineto{\pgfqpoint{0.902009in}{2.524850in}}%
\pgfpathlineto{\pgfqpoint{0.906287in}{2.809305in}}%
\pgfpathlineto{\pgfqpoint{0.908853in}{2.673270in}}%
\pgfpathlineto{\pgfqpoint{0.915695in}{2.794808in}}%
\pgfpathlineto{\pgfqpoint{0.919115in}{2.504372in}}%
\pgfpathlineto{\pgfqpoint{0.921681in}{2.466317in}}%
\pgfpathlineto{\pgfqpoint{0.925951in}{2.535422in}}%
\pgfpathlineto{\pgfqpoint{0.935359in}{2.435328in}}%
\pgfpathlineto{\pgfqpoint{0.939638in}{2.544754in}}%
\pgfpathlineto{\pgfqpoint{0.943058in}{2.464444in}}%
\pgfpathlineto{\pgfqpoint{0.947330in}{2.450187in}}%
\pgfpathlineto{\pgfqpoint{0.955029in}{2.439316in}}%
\pgfpathlineto{\pgfqpoint{0.957596in}{2.572632in}}%
\pgfpathlineto{\pgfqpoint{0.961015in}{2.449373in}}%
\pgfpathlineto{\pgfqpoint{0.965292in}{2.526783in}}%
\pgfpathlineto{\pgfqpoint{0.971277in}{2.574021in}}%
\pgfpathlineto{\pgfqpoint{0.972986in}{2.461303in}}%
\pgfpathlineto{\pgfqpoint{0.978119in}{2.553906in}}%
\pgfpathlineto{\pgfqpoint{0.983246in}{2.411046in}}%
\pgfpathlineto{\pgfqpoint{0.985812in}{2.492956in}}%
\pgfpathlineto{\pgfqpoint{0.990941in}{2.449706in}}%
\pgfpathlineto{\pgfqpoint{0.996074in}{2.539710in}}%
\pgfpathlineto{\pgfqpoint{1.001204in}{2.486674in}}%
\pgfpathlineto{\pgfqpoint{1.004625in}{2.536085in}}%
\pgfpathlineto{\pgfqpoint{1.007191in}{2.418051in}}%
\pgfpathlineto{\pgfqpoint{1.013177in}{2.564296in}}%
\pgfpathlineto{\pgfqpoint{1.016601in}{2.472720in}}%
\pgfpathlineto{\pgfqpoint{1.023437in}{2.595949in}}%
\pgfpathlineto{\pgfqpoint{1.024292in}{2.493530in}}%
\pgfpathlineto{\pgfqpoint{1.030277in}{2.444511in}}%
\pgfpathlineto{\pgfqpoint{1.035409in}{2.533065in}}%
\pgfpathlineto{\pgfqpoint{1.039687in}{2.573356in}}%
\pgfpathlineto{\pgfqpoint{1.046538in}{2.426389in}}%
\pgfpathlineto{\pgfqpoint{1.051672in}{2.516514in}}%
\pgfpathlineto{\pgfqpoint{1.055092in}{2.427506in}}%
\pgfpathlineto{\pgfqpoint{1.059370in}{2.535843in}}%
\pgfpathlineto{\pgfqpoint{1.065358in}{2.549193in}}%
\pgfpathlineto{\pgfqpoint{1.067066in}{2.434965in}}%
\pgfpathlineto{\pgfqpoint{1.073048in}{2.538502in}}%
\pgfpathlineto{\pgfqpoint{1.076466in}{2.455443in}}%
\pgfpathlineto{\pgfqpoint{1.082458in}{2.574805in}}%
\pgfpathlineto{\pgfqpoint{1.087592in}{2.433697in}}%
\pgfpathlineto{\pgfqpoint{1.088447in}{2.497062in}}%
\pgfpathlineto{\pgfqpoint{1.093579in}{2.563873in}}%
\pgfpathlineto{\pgfqpoint{1.097851in}{2.467585in}}%
\pgfpathlineto{\pgfqpoint{1.103832in}{2.552878in}}%
\pgfpathlineto{\pgfqpoint{1.107250in}{2.428744in}}%
\pgfpathlineto{\pgfqpoint{1.112379in}{2.554692in}}%
\pgfpathlineto{\pgfqpoint{1.114088in}{2.448135in}}%
\pgfpathlineto{\pgfqpoint{1.120074in}{2.527086in}}%
\pgfpathlineto{\pgfqpoint{1.123495in}{2.440825in}}%
\pgfpathlineto{\pgfqpoint{1.126917in}{2.517964in}}%
\pgfpathlineto{\pgfqpoint{1.131195in}{2.419200in}}%
\pgfpathlineto{\pgfqpoint{1.136328in}{2.548529in}}%
\pgfpathlineto{\pgfqpoint{1.139749in}{2.450399in}}%
\pgfpathlineto{\pgfqpoint{1.145737in}{2.538139in}}%
\pgfpathlineto{\pgfqpoint{1.148303in}{2.449222in}}%
\pgfpathlineto{\pgfqpoint{1.155144in}{2.590209in}}%
\pgfpathlineto{\pgfqpoint{1.158563in}{2.477975in}}%
\pgfpathlineto{\pgfqpoint{1.161130in}{2.562000in}}%
\pgfpathlineto{\pgfqpoint{1.166264in}{2.553422in}}%
\pgfpathlineto{\pgfqpoint{1.172254in}{2.422371in}}%
\pgfpathlineto{\pgfqpoint{1.174823in}{2.567558in}}%
\pgfpathlineto{\pgfqpoint{1.179102in}{2.466620in}}%
\pgfpathlineto{\pgfqpoint{1.183380in}{2.571725in}}%
\pgfpathlineto{\pgfqpoint{1.189368in}{2.453088in}}%
\pgfpathlineto{\pgfqpoint{1.191079in}{2.535966in}}%
\pgfpathlineto{\pgfqpoint{1.198776in}{2.443908in}}%
\pgfpathlineto{\pgfqpoint{1.201344in}{2.534033in}}%
\pgfpathlineto{\pgfqpoint{1.203910in}{2.473566in}}%
\pgfpathlineto{\pgfqpoint{1.209045in}{2.570638in}}%
\pgfpathlineto{\pgfqpoint{1.213329in}{2.482868in}}%
\pgfpathlineto{\pgfqpoint{1.216755in}{2.540496in}}%
\pgfpathlineto{\pgfqpoint{1.222746in}{2.460459in}}%
\pgfpathlineto{\pgfqpoint{1.227880in}{2.440946in}}%
\pgfpathlineto{\pgfqpoint{1.229590in}{2.551912in}}%
\pgfpathlineto{\pgfqpoint{1.235580in}{2.469518in}}%
\pgfpathlineto{\pgfqpoint{1.239001in}{2.532402in}}%
\pgfpathlineto{\pgfqpoint{1.243278in}{2.475649in}}%
\pgfpathlineto{\pgfqpoint{1.248413in}{2.554027in}}%
\pgfpathlineto{\pgfqpoint{1.251837in}{2.455625in}}%
\pgfpathlineto{\pgfqpoint{1.257825in}{2.410199in}}%
\pgfpathlineto{\pgfqpoint{1.259533in}{2.581843in}}%
\pgfpathlineto{\pgfqpoint{1.266376in}{2.476586in}}%
\pgfpathlineto{\pgfqpoint{1.270656in}{2.598243in}}%
\pgfpathlineto{\pgfqpoint{1.274080in}{2.472478in}}%
\pgfpathlineto{\pgfqpoint{1.277502in}{2.571483in}}%
\pgfpathlineto{\pgfqpoint{1.280922in}{2.518175in}}%
\pgfpathlineto{\pgfqpoint{1.285201in}{2.558617in}}%
\pgfpathlineto{\pgfqpoint{1.290336in}{2.372416in}}%
\pgfpathlineto{\pgfqpoint{1.295468in}{2.494345in}}%
\pgfpathlineto{\pgfqpoint{1.299746in}{2.446111in}}%
\pgfpathlineto{\pgfqpoint{1.302314in}{2.523401in}}%
\pgfpathlineto{\pgfqpoint{1.310013in}{2.405790in}}%
\pgfpathlineto{\pgfqpoint{1.310869in}{2.502923in}}%
\pgfpathlineto{\pgfqpoint{1.318566in}{2.440190in}}%
\pgfpathlineto{\pgfqpoint{1.321131in}{2.519111in}}%
\pgfpathlineto{\pgfqpoint{1.325401in}{2.460457in}}%
\pgfpathlineto{\pgfqpoint{1.329678in}{2.563208in}}%
\pgfpathlineto{\pgfqpoint{1.333096in}{2.477612in}}%
\pgfpathlineto{\pgfqpoint{1.339081in}{2.527328in}}%
\pgfpathlineto{\pgfqpoint{1.341646in}{2.469458in}}%
\pgfpathlineto{\pgfqpoint{1.345066in}{2.432187in}}%
\pgfpathlineto{\pgfqpoint{1.351054in}{2.429470in}}%
\pgfpathlineto{\pgfqpoint{1.354477in}{2.550160in}}%
\pgfpathlineto{\pgfqpoint{1.362176in}{2.418898in}}%
\pgfpathlineto{\pgfqpoint{1.367303in}{2.491023in}}%
\pgfpathlineto{\pgfqpoint{1.370723in}{2.437685in}}%
\pgfpathlineto{\pgfqpoint{1.375853in}{2.521770in}}%
\pgfpathlineto{\pgfqpoint{1.381839in}{2.395158in}}%
\pgfpathlineto{\pgfqpoint{1.386113in}{2.528475in}}%
\pgfpathlineto{\pgfqpoint{1.391249in}{2.465501in}}%
\pgfpathlineto{\pgfqpoint{1.394673in}{2.531616in}}%
\pgfpathlineto{\pgfqpoint{1.398953in}{2.476223in}}%
\pgfpathlineto{\pgfqpoint{1.401521in}{2.563510in}}%
\pgfpathlineto{\pgfqpoint{1.408362in}{2.571183in}}%
\pgfpathlineto{\pgfqpoint{1.410925in}{2.466015in}}%
\pgfpathlineto{\pgfqpoint{1.418616in}{2.551791in}}%
\pgfpathlineto{\pgfqpoint{1.425464in}{2.426932in}}%
\pgfpathlineto{\pgfqpoint{1.427175in}{2.512587in}}%
\pgfpathlineto{\pgfqpoint{1.431452in}{2.541341in}}%
\pgfpathlineto{\pgfqpoint{1.434872in}{2.452726in}}%
\pgfpathlineto{\pgfqpoint{1.440002in}{2.437987in}}%
\pgfpathlineto{\pgfqpoint{1.444278in}{2.504856in}}%
\pgfpathlineto{\pgfqpoint{1.451122in}{2.436235in}}%
\pgfpathlineto{\pgfqpoint{1.453691in}{2.535906in}}%
\pgfpathlineto{\pgfqpoint{1.457970in}{2.449585in}}%
\pgfpathlineto{\pgfqpoint{1.463106in}{2.437203in}}%
\pgfpathlineto{\pgfqpoint{1.468244in}{2.536087in}}%
\pgfpathlineto{\pgfqpoint{1.470810in}{2.559283in}}%
\pgfpathlineto{\pgfqpoint{1.474232in}{2.453088in}}%
\pgfpathlineto{\pgfqpoint{1.480224in}{2.514581in}}%
\pgfpathlineto{\pgfqpoint{1.485359in}{2.441581in}}%
\pgfpathlineto{\pgfqpoint{1.486215in}{2.557773in}}%
\pgfpathlineto{\pgfqpoint{1.491345in}{2.500629in}}%
\pgfpathlineto{\pgfqpoint{1.496476in}{2.468432in}}%
\pgfpathlineto{\pgfqpoint{1.499039in}{2.574868in}}%
\pgfpathlineto{\pgfqpoint{1.505027in}{2.459372in}}%
\pgfpathlineto{\pgfqpoint{1.511013in}{2.539861in}}%
\pgfpathlineto{\pgfqpoint{1.512722in}{2.446262in}}%
\pgfpathlineto{\pgfqpoint{1.518707in}{2.430738in}}%
\pgfpathlineto{\pgfqpoint{1.520417in}{2.530257in}}%
\pgfpathlineto{\pgfqpoint{1.528112in}{2.462995in}}%
\pgfpathlineto{\pgfqpoint{1.531534in}{2.538200in}}%
\pgfpathlineto{\pgfqpoint{1.535810in}{2.445597in}}%
\pgfpathlineto{\pgfqpoint{1.540941in}{2.558859in}}%
\pgfpathlineto{\pgfqpoint{1.541797in}{2.461515in}}%
\pgfpathlineto{\pgfqpoint{1.548638in}{2.506668in}}%
\pgfpathlineto{\pgfqpoint{1.551207in}{2.432580in}}%
\pgfpathlineto{\pgfqpoint{1.556343in}{2.434120in}}%
\pgfpathlineto{\pgfqpoint{1.558909in}{2.535843in}}%
\pgfpathlineto{\pgfqpoint{1.566605in}{2.550281in}}%
\pgfpathlineto{\pgfqpoint{1.567459in}{2.488124in}}%
\pgfpathlineto{\pgfqpoint{1.575167in}{2.437745in}}%
\pgfpathlineto{\pgfqpoint{1.577737in}{2.556986in}}%
\pgfpathlineto{\pgfqpoint{1.582014in}{2.419079in}}%
\pgfpathlineto{\pgfqpoint{1.585439in}{2.525394in}}%
\pgfpathlineto{\pgfqpoint{1.591423in}{2.470817in}}%
\pgfpathlineto{\pgfqpoint{1.596559in}{2.534757in}}%
\pgfpathlineto{\pgfqpoint{1.599126in}{2.428744in}}%
\pgfpathlineto{\pgfqpoint{1.603404in}{2.571241in}}%
\pgfpathlineto{\pgfqpoint{1.607680in}{2.461936in}}%
\pgfpathlineto{\pgfqpoint{1.613666in}{2.533428in}}%
\pgfpathlineto{\pgfqpoint{1.617087in}{2.470786in}}%
\pgfpathlineto{\pgfqpoint{1.618798in}{2.531979in}}%
\pgfpathlineto{\pgfqpoint{1.625639in}{2.579700in}}%
\pgfpathlineto{\pgfqpoint{1.630773in}{2.459914in}}%
\pgfpathlineto{\pgfqpoint{1.635054in}{2.558436in}}%
\pgfpathlineto{\pgfqpoint{1.636766in}{2.448196in}}%
\pgfpathlineto{\pgfqpoint{1.641889in}{2.568282in}}%
\pgfpathlineto{\pgfqpoint{1.648727in}{2.450671in}}%
\pgfpathlineto{\pgfqpoint{1.656428in}{2.523219in}}%
\pgfpathlineto{\pgfqpoint{1.658993in}{2.424395in}}%
\pgfpathlineto{\pgfqpoint{1.664122in}{2.570336in}}%
\pgfpathlineto{\pgfqpoint{1.666687in}{2.455745in}}%
\pgfpathlineto{\pgfqpoint{1.672676in}{2.585740in}}%
\pgfpathlineto{\pgfqpoint{1.676098in}{2.398148in}}%
\pgfpathlineto{\pgfqpoint{1.681229in}{2.527810in}}%
\pgfpathlineto{\pgfqpoint{1.682939in}{2.447319in}}%
\pgfpathlineto{\pgfqpoint{1.687214in}{2.485225in}}%
\pgfpathlineto{\pgfqpoint{1.693199in}{2.430314in}}%
\pgfpathlineto{\pgfqpoint{1.695764in}{2.524487in}}%
\pgfpathlineto{\pgfqpoint{1.700041in}{2.445416in}}%
\pgfpathlineto{\pgfqpoint{1.704314in}{2.428774in}}%
\pgfpathlineto{\pgfqpoint{1.709447in}{2.508904in}}%
\pgfpathlineto{\pgfqpoint{1.714577in}{2.476949in}}%
\pgfpathlineto{\pgfqpoint{1.723132in}{2.594590in}}%
\pgfpathlineto{\pgfqpoint{1.725698in}{2.493863in}}%
\pgfpathlineto{\pgfqpoint{1.730826in}{2.435751in}}%
\pgfpathlineto{\pgfqpoint{1.734249in}{2.521407in}}%
\pgfpathlineto{\pgfqpoint{1.738528in}{2.478157in}}%
\pgfpathlineto{\pgfqpoint{1.746231in}{2.442184in}}%
\pgfpathlineto{\pgfqpoint{1.748801in}{2.550281in}}%
\pgfpathlineto{\pgfqpoint{1.753081in}{2.430435in}}%
\pgfpathlineto{\pgfqpoint{1.758211in}{2.542611in}}%
\pgfpathlineto{\pgfqpoint{1.761630in}{2.501352in}}%
\pgfpathlineto{\pgfqpoint{1.764196in}{2.531616in}}%
\pgfpathlineto{\pgfqpoint{1.770178in}{2.457104in}}%
\pgfpathlineto{\pgfqpoint{1.775307in}{2.529864in}}%
\pgfpathlineto{\pgfqpoint{1.780436in}{2.537234in}}%
\pgfpathlineto{\pgfqpoint{1.781291in}{2.419442in}}%
\pgfpathlineto{\pgfqpoint{1.787279in}{2.538139in}}%
\pgfpathlineto{\pgfqpoint{1.790704in}{2.489392in}}%
\pgfpathlineto{\pgfqpoint{1.794127in}{2.572088in}}%
\pgfpathlineto{\pgfqpoint{1.803538in}{2.429772in}}%
\pgfpathlineto{\pgfqpoint{1.808671in}{2.549013in}}%
\pgfpathlineto{\pgfqpoint{1.811238in}{2.482929in}}%
\pgfpathlineto{\pgfqpoint{1.817226in}{2.577283in}}%
\pgfpathlineto{\pgfqpoint{1.821502in}{2.465231in}}%
\pgfpathlineto{\pgfqpoint{1.824922in}{2.549800in}}%
\pgfpathlineto{\pgfqpoint{1.829199in}{2.450069in}}%
\pgfpathlineto{\pgfqpoint{1.835185in}{2.534093in}}%
\pgfpathlineto{\pgfqpoint{1.838605in}{2.457981in}}%
\pgfpathlineto{\pgfqpoint{1.842028in}{2.547745in}}%
\pgfpathlineto{\pgfqpoint{1.847161in}{2.469337in}}%
\pgfpathlineto{\pgfqpoint{1.849731in}{2.552033in}}%
\pgfpathlineto{\pgfqpoint{1.854859in}{2.546264in}}%
\pgfpathlineto{\pgfqpoint{1.861702in}{2.393316in}}%
\pgfpathlineto{\pgfqpoint{1.865977in}{2.534877in}}%
\pgfpathlineto{\pgfqpoint{1.869400in}{2.460578in}}%
\pgfpathlineto{\pgfqpoint{1.871963in}{2.585740in}}%
\pgfpathlineto{\pgfqpoint{1.875380in}{2.608635in}}%
\pgfpathlineto{\pgfqpoint{1.880514in}{2.452060in}}%
\pgfpathlineto{\pgfqpoint{1.886505in}{2.548832in}}%
\pgfpathlineto{\pgfqpoint{1.888217in}{2.450550in}}%
\pgfpathlineto{\pgfqpoint{1.893350in}{2.566409in}}%
\pgfpathlineto{\pgfqpoint{1.896771in}{2.475861in}}%
\pgfpathlineto{\pgfqpoint{1.903618in}{2.538200in}}%
\pgfpathlineto{\pgfqpoint{1.906183in}{2.459310in}}%
\pgfpathlineto{\pgfqpoint{1.909606in}{2.515728in}}%
\pgfpathlineto{\pgfqpoint{1.916449in}{2.575047in}}%
\pgfpathlineto{\pgfqpoint{1.921587in}{2.445960in}}%
\pgfpathlineto{\pgfqpoint{1.922443in}{2.555053in}}%
\pgfpathlineto{\pgfqpoint{1.927577in}{2.456653in}}%
\pgfpathlineto{\pgfqpoint{1.933561in}{2.495796in}}%
\pgfpathlineto{\pgfqpoint{1.937837in}{2.449887in}}%
\pgfpathlineto{\pgfqpoint{1.941258in}{2.577767in}}%
\pgfpathlineto{\pgfqpoint{1.947248in}{2.464747in}}%
\pgfpathlineto{\pgfqpoint{1.949816in}{2.543274in}}%
\pgfpathlineto{\pgfqpoint{1.955805in}{2.523280in}}%
\pgfpathlineto{\pgfqpoint{1.958374in}{2.408056in}}%
\pgfpathlineto{\pgfqpoint{1.960940in}{2.481781in}}%
\pgfpathlineto{\pgfqpoint{1.965220in}{2.559222in}}%
\pgfpathlineto{\pgfqpoint{1.972063in}{2.459975in}}%
\pgfpathlineto{\pgfqpoint{1.974626in}{2.559131in}}%
\pgfpathlineto{\pgfqpoint{1.980610in}{2.460459in}}%
\pgfpathlineto{\pgfqpoint{1.982321in}{2.542308in}}%
\pgfpathlineto{\pgfqpoint{1.986595in}{2.468644in}}%
\pgfpathlineto{\pgfqpoint{1.993441in}{2.555446in}}%
\pgfpathlineto{\pgfqpoint{1.995151in}{2.559041in}}%
\pgfpathlineto{\pgfqpoint{1.999427in}{2.460519in}}%
\pgfpathlineto{\pgfqpoint{2.003702in}{2.524308in}}%
\pgfpathlineto{\pgfqpoint{2.009691in}{2.444511in}}%
\pgfpathlineto{\pgfqpoint{2.013969in}{2.546054in}}%
\pgfpathlineto{\pgfqpoint{2.019956in}{2.592324in}}%
\pgfpathlineto{\pgfqpoint{2.024234in}{2.460699in}}%
\pgfpathlineto{\pgfqpoint{2.025946in}{2.528475in}}%
\pgfpathlineto{\pgfqpoint{2.031077in}{2.481086in}}%
\pgfpathlineto{\pgfqpoint{2.033644in}{2.576376in}}%
\pgfpathlineto{\pgfqpoint{2.039628in}{2.482929in}}%
\pgfpathlineto{\pgfqpoint{2.043906in}{2.568887in}}%
\pgfpathlineto{\pgfqpoint{2.048183in}{2.476647in}}%
\pgfpathlineto{\pgfqpoint{2.050751in}{2.582660in}}%
\pgfpathlineto{\pgfqpoint{2.055027in}{2.469579in}}%
\pgfpathlineto{\pgfqpoint{2.060159in}{2.570941in}}%
\pgfpathlineto{\pgfqpoint{2.064434in}{2.476223in}}%
\pgfpathlineto{\pgfqpoint{2.068710in}{2.417781in}}%
\pgfpathlineto{\pgfqpoint{2.072131in}{2.524066in}}%
\pgfpathlineto{\pgfqpoint{2.076406in}{2.474232in}}%
\pgfpathlineto{\pgfqpoint{2.080685in}{2.566711in}}%
\pgfpathlineto{\pgfqpoint{2.088377in}{2.561879in}}%
\pgfpathlineto{\pgfqpoint{2.090942in}{2.458042in}}%
\pgfpathlineto{\pgfqpoint{2.093509in}{2.572693in}}%
\pgfpathlineto{\pgfqpoint{2.100359in}{2.447712in}}%
\pgfpathlineto{\pgfqpoint{2.102070in}{2.543455in}}%
\pgfpathlineto{\pgfqpoint{2.109774in}{2.570094in}}%
\pgfpathlineto{\pgfqpoint{2.110631in}{2.483955in}}%
\pgfpathlineto{\pgfqpoint{2.116622in}{2.597156in}}%
\pgfpathlineto{\pgfqpoint{2.122605in}{2.489271in}}%
\pgfpathlineto{\pgfqpoint{2.123462in}{2.557107in}}%
\pgfpathlineto{\pgfqpoint{2.130310in}{2.440765in}}%
\pgfpathlineto{\pgfqpoint{2.133734in}{2.515125in}}%
\pgfpathlineto{\pgfqpoint{2.139720in}{2.481479in}}%
\pgfpathlineto{\pgfqpoint{2.140576in}{2.583686in}}%
\pgfpathlineto{\pgfqpoint{2.146558in}{2.435298in}}%
\pgfpathlineto{\pgfqpoint{2.149976in}{2.562302in}}%
\pgfpathlineto{\pgfqpoint{2.156813in}{2.425966in}}%
\pgfpathlineto{\pgfqpoint{2.159379in}{2.606821in}}%
\pgfpathlineto{\pgfqpoint{2.162803in}{2.477068in}}%
\pgfpathlineto{\pgfqpoint{2.169647in}{2.469428in}}%
\pgfpathlineto{\pgfqpoint{2.172214in}{2.608724in}}%
\pgfpathlineto{\pgfqpoint{2.174781in}{2.487610in}}%
\pgfpathlineto{\pgfqpoint{2.181627in}{2.463779in}}%
\pgfpathlineto{\pgfqpoint{2.184193in}{2.529199in}}%
\pgfpathlineto{\pgfqpoint{2.191037in}{2.446926in}}%
\pgfpathlineto{\pgfqpoint{2.195315in}{2.533005in}}%
\pgfpathlineto{\pgfqpoint{2.199592in}{2.489755in}}%
\pgfpathlineto{\pgfqpoint{2.200449in}{2.512980in}}%
\pgfpathlineto{\pgfqpoint{2.205581in}{2.465591in}}%
\pgfpathlineto{\pgfqpoint{2.209001in}{2.529803in}}%
\pgfpathlineto{\pgfqpoint{2.213282in}{2.559643in}}%
\pgfpathlineto{\pgfqpoint{2.220121in}{2.455564in}}%
\pgfpathlineto{\pgfqpoint{2.224396in}{2.577525in}}%
\pgfpathlineto{\pgfqpoint{2.226107in}{2.493772in}}%
\pgfpathlineto{\pgfqpoint{2.230384in}{2.550342in}}%
\pgfpathlineto{\pgfqpoint{2.236373in}{2.483955in}}%
\pgfpathlineto{\pgfqpoint{2.240649in}{2.580666in}}%
\pgfpathlineto{\pgfqpoint{2.245780in}{2.502136in}}%
\pgfpathlineto{\pgfqpoint{2.249197in}{2.599027in}}%
\pgfpathlineto{\pgfqpoint{2.252617in}{2.485767in}}%
\pgfpathlineto{\pgfqpoint{2.256038in}{2.598061in}}%
\pgfpathlineto{\pgfqpoint{2.261169in}{2.462088in}}%
\pgfpathlineto{\pgfqpoint{2.265446in}{2.457437in}}%
\pgfpathlineto{\pgfqpoint{2.270579in}{2.563691in}}%
\pgfpathlineto{\pgfqpoint{2.274003in}{2.425966in}}%
\pgfpathlineto{\pgfqpoint{2.279133in}{2.588215in}}%
\pgfpathlineto{\pgfqpoint{2.281703in}{2.524487in}}%
\pgfpathlineto{\pgfqpoint{2.288549in}{2.486614in}}%
\pgfpathlineto{\pgfqpoint{2.293687in}{2.559885in}}%
\pgfpathlineto{\pgfqpoint{2.294543in}{2.473988in}}%
\pgfpathlineto{\pgfqpoint{2.300531in}{2.545449in}}%
\pgfpathlineto{\pgfqpoint{2.306519in}{2.450490in}}%
\pgfpathlineto{\pgfqpoint{2.309941in}{2.603922in}}%
\pgfpathlineto{\pgfqpoint{2.311649in}{2.488245in}}%
\pgfpathlineto{\pgfqpoint{2.317638in}{2.565322in}}%
\pgfpathlineto{\pgfqpoint{2.322768in}{2.435540in}}%
\pgfpathlineto{\pgfqpoint{2.324477in}{2.481177in}}%
\pgfpathlineto{\pgfqpoint{2.332180in}{2.436687in}}%
\pgfpathlineto{\pgfqpoint{2.333891in}{2.510896in}}%
\pgfpathlineto{\pgfqpoint{2.339879in}{2.472599in}}%
\pgfpathlineto{\pgfqpoint{2.341587in}{2.550191in}}%
\pgfpathlineto{\pgfqpoint{2.347573in}{2.470484in}}%
\pgfpathlineto{\pgfqpoint{2.353560in}{2.472599in}}%
\pgfpathlineto{\pgfqpoint{2.356982in}{2.581934in}}%
\pgfpathlineto{\pgfqpoint{2.359552in}{2.517119in}}%
\pgfpathlineto{\pgfqpoint{2.365540in}{2.434907in}}%
\pgfpathlineto{\pgfqpoint{2.368963in}{2.546778in}}%
\pgfpathlineto{\pgfqpoint{2.373239in}{2.550221in}}%
\pgfpathlineto{\pgfqpoint{2.379228in}{2.445597in}}%
\pgfpathlineto{\pgfqpoint{2.380084in}{2.583867in}}%
\pgfpathlineto{\pgfqpoint{2.386070in}{2.464686in}}%
\pgfpathlineto{\pgfqpoint{2.388635in}{2.522042in}}%
\pgfpathlineto{\pgfqpoint{2.395480in}{2.489936in}}%
\pgfpathlineto{\pgfqpoint{2.398901in}{2.575168in}}%
\pgfpathlineto{\pgfqpoint{2.404887in}{2.495433in}}%
\pgfpathlineto{\pgfqpoint{2.406597in}{2.547322in}}%
\pgfpathlineto{\pgfqpoint{2.412580in}{2.451034in}}%
\pgfpathlineto{\pgfqpoint{2.415143in}{2.554087in}}%
\pgfpathlineto{\pgfqpoint{2.421985in}{2.462269in}}%
\pgfpathlineto{\pgfqpoint{2.424553in}{2.530166in}}%
\pgfpathlineto{\pgfqpoint{2.429680in}{2.556503in}}%
\pgfpathlineto{\pgfqpoint{2.433100in}{2.487761in}}%
\pgfpathlineto{\pgfqpoint{2.439086in}{2.555476in}}%
\pgfpathlineto{\pgfqpoint{2.440796in}{2.525092in}}%
\pgfpathlineto{\pgfqpoint{2.445075in}{2.474199in}}%
\pgfpathlineto{\pgfqpoint{2.449351in}{2.589546in}}%
\pgfpathlineto{\pgfqpoint{2.455339in}{2.462269in}}%
\pgfpathlineto{\pgfqpoint{2.460475in}{2.579216in}}%
\pgfpathlineto{\pgfqpoint{2.464751in}{2.518750in}}%
\pgfpathlineto{\pgfqpoint{2.468174in}{2.601265in}}%
\pgfpathlineto{\pgfqpoint{2.471595in}{2.492595in}}%
\pgfpathlineto{\pgfqpoint{2.474162in}{2.564478in}}%
\pgfpathlineto{\pgfqpoint{2.480149in}{2.466105in}}%
\pgfpathlineto{\pgfqpoint{2.484427in}{2.532523in}}%
\pgfpathlineto{\pgfqpoint{2.489559in}{2.451700in}}%
\pgfpathlineto{\pgfqpoint{2.492982in}{2.570036in}}%
\pgfpathlineto{\pgfqpoint{2.497259in}{2.440041in}}%
\pgfpathlineto{\pgfqpoint{2.500682in}{2.539107in}}%
\pgfpathlineto{\pgfqpoint{2.505817in}{2.560793in}}%
\pgfpathlineto{\pgfqpoint{2.508382in}{2.466680in}}%
\pgfpathlineto{\pgfqpoint{2.512659in}{2.535845in}}%
\pgfpathlineto{\pgfqpoint{2.519502in}{2.439739in}}%
\pgfpathlineto{\pgfqpoint{2.522926in}{2.534063in}}%
\pgfpathlineto{\pgfqpoint{2.526346in}{2.571062in}}%
\pgfpathlineto{\pgfqpoint{2.531478in}{2.477494in}}%
\pgfpathlineto{\pgfqpoint{2.535758in}{2.456532in}}%
\pgfpathlineto{\pgfqpoint{2.538324in}{2.556021in}}%
\pgfpathlineto{\pgfqpoint{2.544308in}{2.496460in}}%
\pgfpathlineto{\pgfqpoint{2.546873in}{2.564538in}}%
\pgfpathlineto{\pgfqpoint{2.551150in}{2.472208in}}%
\pgfpathlineto{\pgfqpoint{2.555422in}{2.535089in}}%
\pgfpathlineto{\pgfqpoint{2.563119in}{2.454477in}}%
\pgfpathlineto{\pgfqpoint{2.563974in}{2.538684in}}%
\pgfpathlineto{\pgfqpoint{2.569960in}{2.443182in}}%
\pgfpathlineto{\pgfqpoint{2.575949in}{2.584956in}}%
\pgfpathlineto{\pgfqpoint{2.576804in}{2.486704in}}%
\pgfpathlineto{\pgfqpoint{2.581931in}{2.436296in}}%
\pgfpathlineto{\pgfqpoint{2.587916in}{2.595044in}}%
\pgfpathlineto{\pgfqpoint{2.590484in}{2.451700in}}%
\pgfpathlineto{\pgfqpoint{2.595611in}{2.538986in}}%
\pgfpathlineto{\pgfqpoint{2.599886in}{2.557168in}}%
\pgfpathlineto{\pgfqpoint{2.603308in}{2.423369in}}%
\pgfpathlineto{\pgfqpoint{2.606733in}{2.529261in}}%
\pgfpathlineto{\pgfqpoint{2.611867in}{2.495071in}}%
\pgfpathlineto{\pgfqpoint{2.615290in}{2.565383in}}%
\pgfpathlineto{\pgfqpoint{2.622129in}{2.464142in}}%
\pgfpathlineto{\pgfqpoint{2.624695in}{2.553483in}}%
\pgfpathlineto{\pgfqpoint{2.631542in}{2.445960in}}%
\pgfpathlineto{\pgfqpoint{2.633255in}{2.531918in}}%
\pgfpathlineto{\pgfqpoint{2.636679in}{2.566953in}}%
\pgfpathlineto{\pgfqpoint{2.643518in}{2.439316in}}%
\pgfpathlineto{\pgfqpoint{2.647797in}{2.530166in}}%
\pgfpathlineto{\pgfqpoint{2.652077in}{2.487037in}}%
\pgfpathlineto{\pgfqpoint{2.655500in}{2.555839in}}%
\pgfpathlineto{\pgfqpoint{2.659777in}{2.486614in}}%
\pgfpathlineto{\pgfqpoint{2.663199in}{2.565685in}}%
\pgfpathlineto{\pgfqpoint{2.666621in}{2.493440in}}%
\pgfpathlineto{\pgfqpoint{2.670900in}{2.562968in}}%
\pgfpathlineto{\pgfqpoint{2.676032in}{2.478157in}}%
\pgfpathlineto{\pgfqpoint{2.681163in}{2.473385in}}%
\pgfpathlineto{\pgfqpoint{2.686295in}{2.579216in}}%
\pgfpathlineto{\pgfqpoint{2.689720in}{2.478308in}}%
\pgfpathlineto{\pgfqpoint{2.693143in}{2.560067in}}%
\pgfpathlineto{\pgfqpoint{2.696565in}{2.460035in}}%
\pgfpathlineto{\pgfqpoint{2.703410in}{2.537839in}}%
\pgfpathlineto{\pgfqpoint{2.708544in}{2.434090in}}%
\pgfpathlineto{\pgfqpoint{2.710256in}{2.562605in}}%
\pgfpathlineto{\pgfqpoint{2.714525in}{2.466196in}}%
\pgfpathlineto{\pgfqpoint{2.718798in}{2.525032in}}%
\pgfpathlineto{\pgfqpoint{2.723934in}{2.461485in}}%
\pgfpathlineto{\pgfqpoint{2.728213in}{2.562363in}}%
\pgfpathlineto{\pgfqpoint{2.731635in}{2.473325in}}%
\pgfpathlineto{\pgfqpoint{2.736769in}{2.521528in}}%
\pgfpathlineto{\pgfqpoint{2.741046in}{2.460396in}}%
\pgfpathlineto{\pgfqpoint{2.743612in}{2.553362in}}%
\pgfpathlineto{\pgfqpoint{2.751304in}{2.430677in}}%
\pgfpathlineto{\pgfqpoint{2.752160in}{2.513797in}}%
\pgfpathlineto{\pgfqpoint{2.758145in}{2.497276in}}%
\pgfpathlineto{\pgfqpoint{2.762417in}{2.573840in}}%
\pgfpathlineto{\pgfqpoint{2.764980in}{2.510777in}}%
\pgfpathlineto{\pgfqpoint{2.772681in}{2.478520in}}%
\pgfpathlineto{\pgfqpoint{2.775247in}{2.533791in}}%
\pgfpathlineto{\pgfqpoint{2.781232in}{2.562605in}}%
\pgfpathlineto{\pgfqpoint{2.782089in}{2.481661in}}%
\pgfpathlineto{\pgfqpoint{2.789789in}{2.534517in}}%
\pgfpathlineto{\pgfqpoint{2.790642in}{2.446504in}}%
\pgfpathlineto{\pgfqpoint{2.794918in}{2.511319in}}%
\pgfpathlineto{\pgfqpoint{2.799189in}{2.451518in}}%
\pgfpathlineto{\pgfqpoint{2.806031in}{2.567679in}}%
\pgfpathlineto{\pgfqpoint{2.810305in}{2.609661in}}%
\pgfpathlineto{\pgfqpoint{2.812017in}{2.505582in}}%
\pgfpathlineto{\pgfqpoint{2.818005in}{2.572390in}}%
\pgfpathlineto{\pgfqpoint{2.821424in}{2.496339in}}%
\pgfpathlineto{\pgfqpoint{2.827409in}{2.577525in}}%
\pgfpathlineto{\pgfqpoint{2.831684in}{2.442819in}}%
\pgfpathlineto{\pgfqpoint{2.836818in}{2.590814in}}%
\pgfpathlineto{\pgfqpoint{2.837671in}{2.480030in}}%
\pgfpathlineto{\pgfqpoint{2.844515in}{2.561034in}}%
\pgfpathlineto{\pgfqpoint{2.847078in}{2.478338in}}%
\pgfpathlineto{\pgfqpoint{2.853918in}{2.407542in}}%
\pgfpathlineto{\pgfqpoint{2.856482in}{2.546868in}}%
\pgfpathlineto{\pgfqpoint{2.860755in}{2.464505in}}%
\pgfpathlineto{\pgfqpoint{2.866743in}{2.592024in}}%
\pgfpathlineto{\pgfqpoint{2.868455in}{2.490601in}}%
\pgfpathlineto{\pgfqpoint{2.871877in}{2.547806in}}%
\pgfpathlineto{\pgfqpoint{2.877860in}{2.470726in}}%
\pgfpathlineto{\pgfqpoint{2.882135in}{2.544723in}}%
\pgfpathlineto{\pgfqpoint{2.886413in}{2.565685in}}%
\pgfpathlineto{\pgfqpoint{2.889835in}{2.430284in}}%
\pgfpathlineto{\pgfqpoint{2.896681in}{2.580424in}}%
\pgfpathlineto{\pgfqpoint{2.898392in}{2.476103in}}%
\pgfpathlineto{\pgfqpoint{2.905233in}{2.564115in}}%
\pgfpathlineto{\pgfqpoint{2.906089in}{2.480421in}}%
\pgfpathlineto{\pgfqpoint{2.913784in}{2.535301in}}%
\pgfpathlineto{\pgfqpoint{2.917204in}{2.429046in}}%
\pgfpathlineto{\pgfqpoint{2.918914in}{2.543516in}}%
\pgfpathlineto{\pgfqpoint{2.923190in}{2.415336in}}%
\pgfpathlineto{\pgfqpoint{2.929171in}{2.520925in}}%
\pgfpathlineto{\pgfqpoint{2.932595in}{2.487460in}}%
\pgfpathlineto{\pgfqpoint{2.938585in}{2.646872in}}%
\pgfpathlineto{\pgfqpoint{2.942004in}{2.551249in}}%
\pgfpathlineto{\pgfqpoint{2.945424in}{2.659316in}}%
\pgfpathlineto{\pgfqpoint{2.949701in}{2.570157in}}%
\pgfpathlineto{\pgfqpoint{2.956540in}{2.656296in}}%
\pgfpathlineto{\pgfqpoint{2.960820in}{2.538926in}}%
\pgfpathlineto{\pgfqpoint{2.963385in}{2.630079in}}%
\pgfpathlineto{\pgfqpoint{2.966806in}{2.655631in}}%
\pgfpathlineto{\pgfqpoint{2.970229in}{2.505943in}}%
\pgfpathlineto{\pgfqpoint{2.975364in}{2.545570in}}%
\pgfpathlineto{\pgfqpoint{2.980495in}{2.477975in}}%
\pgfpathlineto{\pgfqpoint{2.986484in}{2.516998in}}%
\pgfpathlineto{\pgfqpoint{2.990761in}{2.464898in}}%
\pgfpathlineto{\pgfqpoint{2.992470in}{2.561879in}}%
\pgfpathlineto{\pgfqpoint{2.998455in}{2.457497in}}%
\pgfpathlineto{\pgfqpoint{3.001024in}{2.575289in}}%
\pgfpathlineto{\pgfqpoint{3.007011in}{2.474472in}}%
\pgfpathlineto{\pgfqpoint{3.009578in}{2.558436in}}%
\pgfpathlineto{\pgfqpoint{3.012997in}{2.468732in}}%
\pgfpathlineto{\pgfqpoint{3.018130in}{2.465833in}}%
\pgfpathlineto{\pgfqpoint{3.024973in}{2.531253in}}%
\pgfpathlineto{\pgfqpoint{3.029249in}{2.446805in}}%
\pgfpathlineto{\pgfqpoint{3.031813in}{2.568161in}}%
\pgfpathlineto{\pgfqpoint{3.035233in}{2.516605in}}%
\pgfpathlineto{\pgfqpoint{3.040363in}{2.562875in}}%
\pgfpathlineto{\pgfqpoint{3.042930in}{2.468793in}}%
\pgfpathlineto{\pgfqpoint{3.048063in}{2.574082in}}%
\pgfpathlineto{\pgfqpoint{3.052341in}{2.462571in}}%
\pgfpathlineto{\pgfqpoint{3.055765in}{2.573961in}}%
\pgfpathlineto{\pgfqpoint{3.062609in}{2.496339in}}%
\pgfpathlineto{\pgfqpoint{3.065175in}{2.539589in}}%
\pgfpathlineto{\pgfqpoint{3.071163in}{2.529440in}}%
\pgfpathlineto{\pgfqpoint{3.073730in}{2.472780in}}%
\pgfpathlineto{\pgfqpoint{3.078860in}{2.542248in}}%
\pgfpathlineto{\pgfqpoint{3.084849in}{2.439406in}}%
\pgfpathlineto{\pgfqpoint{3.086559in}{2.569429in}}%
\pgfpathlineto{\pgfqpoint{3.090835in}{2.469760in}}%
\pgfpathlineto{\pgfqpoint{3.096823in}{2.421043in}}%
\pgfpathlineto{\pgfqpoint{3.099388in}{2.517180in}}%
\pgfpathlineto{\pgfqpoint{3.102807in}{2.397878in}}%
\pgfpathlineto{\pgfqpoint{3.107087in}{2.532160in}}%
\pgfpathlineto{\pgfqpoint{3.113076in}{2.470274in}}%
\pgfpathlineto{\pgfqpoint{3.119072in}{2.587008in}}%
\pgfpathlineto{\pgfqpoint{3.119928in}{2.510535in}}%
\pgfpathlineto{\pgfqpoint{3.125916in}{2.474714in}}%
\pgfpathlineto{\pgfqpoint{3.128480in}{2.538805in}}%
\pgfpathlineto{\pgfqpoint{3.136178in}{2.608996in}}%
\pgfpathlineto{\pgfqpoint{3.137034in}{2.481479in}}%
\pgfpathlineto{\pgfqpoint{3.144728in}{2.521467in}}%
\pgfpathlineto{\pgfqpoint{3.145581in}{2.452907in}}%
\pgfpathlineto{\pgfqpoint{3.153277in}{2.425936in}}%
\pgfpathlineto{\pgfqpoint{3.154131in}{2.542066in}}%
\pgfpathlineto{\pgfqpoint{3.161832in}{2.465773in}}%
\pgfpathlineto{\pgfqpoint{3.164395in}{2.531737in}}%
\pgfpathlineto{\pgfqpoint{3.168665in}{2.560672in}}%
\pgfpathlineto{\pgfqpoint{3.172086in}{2.418656in}}%
\pgfpathlineto{\pgfqpoint{3.175506in}{2.568705in}}%
\pgfpathlineto{\pgfqpoint{3.181495in}{2.499479in}}%
\pgfpathlineto{\pgfqpoint{3.184063in}{2.590935in}}%
\pgfpathlineto{\pgfqpoint{3.191760in}{2.513192in}}%
\pgfpathlineto{\pgfqpoint{3.196038in}{2.556503in}}%
\pgfpathlineto{\pgfqpoint{3.197748in}{2.434423in}}%
\pgfpathlineto{\pgfqpoint{3.202885in}{2.569128in}}%
\pgfpathlineto{\pgfqpoint{3.205451in}{2.454659in}}%
\pgfpathlineto{\pgfqpoint{3.211439in}{2.502862in}}%
\pgfpathlineto{\pgfqpoint{3.217427in}{2.522856in}}%
\pgfpathlineto{\pgfqpoint{3.221702in}{2.452302in}}%
\pgfpathlineto{\pgfqpoint{3.224266in}{2.528112in}}%
\pgfpathlineto{\pgfqpoint{3.226834in}{2.437261in}}%
\pgfpathlineto{\pgfqpoint{3.231112in}{2.519232in}}%
\pgfpathlineto{\pgfqpoint{3.237954in}{2.474804in}}%
\pgfpathlineto{\pgfqpoint{3.243083in}{2.541401in}}%
\pgfpathlineto{\pgfqpoint{3.246503in}{2.455685in}}%
\pgfpathlineto{\pgfqpoint{3.248209in}{2.549979in}}%
\pgfpathlineto{\pgfqpoint{3.255909in}{2.558496in}}%
\pgfpathlineto{\pgfqpoint{3.256765in}{2.447349in}}%
\pgfpathlineto{\pgfqpoint{3.264463in}{2.558920in}}%
\pgfpathlineto{\pgfqpoint{3.268739in}{2.479788in}}%
\pgfpathlineto{\pgfqpoint{3.273013in}{2.531283in}}%
\pgfpathlineto{\pgfqpoint{3.276437in}{2.544905in}}%
\pgfpathlineto{\pgfqpoint{3.281573in}{2.421011in}}%
\pgfpathlineto{\pgfqpoint{3.283283in}{2.564355in}}%
\pgfpathlineto{\pgfqpoint{3.287561in}{2.495552in}}%
\pgfpathlineto{\pgfqpoint{3.294401in}{2.571302in}}%
\pgfpathlineto{\pgfqpoint{3.298673in}{2.464505in}}%
\pgfpathlineto{\pgfqpoint{3.299528in}{2.549798in}}%
\pgfpathlineto{\pgfqpoint{3.303802in}{2.461182in}}%
\pgfpathlineto{\pgfqpoint{3.308077in}{2.500747in}}%
\pgfpathlineto{\pgfqpoint{3.315769in}{2.437745in}}%
\pgfpathlineto{\pgfqpoint{3.316624in}{2.560430in}}%
\pgfpathlineto{\pgfqpoint{3.324328in}{2.574866in}}%
\pgfpathlineto{\pgfqpoint{3.325184in}{2.471694in}}%
\pgfpathlineto{\pgfqpoint{3.329460in}{2.444450in}}%
\pgfpathlineto{\pgfqpoint{3.333737in}{2.563570in}}%
\pgfpathlineto{\pgfqpoint{3.338014in}{2.480451in}}%
\pgfpathlineto{\pgfqpoint{3.342293in}{2.560188in}}%
\pgfpathlineto{\pgfqpoint{3.349137in}{2.444087in}}%
\pgfpathlineto{\pgfqpoint{3.351703in}{2.533065in}}%
\pgfpathlineto{\pgfqpoint{3.355122in}{2.487216in}}%
\pgfpathlineto{\pgfqpoint{3.361114in}{2.566590in}}%
\pgfpathlineto{\pgfqpoint{3.365390in}{2.457376in}}%
\pgfpathlineto{\pgfqpoint{3.371379in}{2.566772in}}%
\pgfpathlineto{\pgfqpoint{3.372234in}{2.516635in}}%
\pgfpathlineto{\pgfqpoint{3.379931in}{2.375827in}}%
\pgfpathlineto{\pgfqpoint{3.380784in}{2.538803in}}%
\pgfpathlineto{\pgfqpoint{3.385062in}{2.470363in}}%
\pgfpathlineto{\pgfqpoint{3.389340in}{2.589544in}}%
\pgfpathlineto{\pgfqpoint{3.397035in}{2.443422in}}%
\pgfpathlineto{\pgfqpoint{3.397890in}{2.569126in}}%
\pgfpathlineto{\pgfqpoint{3.402168in}{2.581571in}}%
\pgfpathlineto{\pgfqpoint{3.406443in}{2.480330in}}%
\pgfpathlineto{\pgfqpoint{3.412430in}{2.553120in}}%
\pgfpathlineto{\pgfqpoint{3.414994in}{2.503044in}}%
\pgfpathlineto{\pgfqpoint{3.420976in}{2.540585in}}%
\pgfpathlineto{\pgfqpoint{3.423538in}{2.476887in}}%
\pgfpathlineto{\pgfqpoint{3.427816in}{2.569126in}}%
\pgfpathlineto{\pgfqpoint{3.434657in}{2.438227in}}%
\pgfpathlineto{\pgfqpoint{3.437223in}{2.529138in}}%
\pgfpathlineto{\pgfqpoint{3.442356in}{2.476735in}}%
\pgfpathlineto{\pgfqpoint{3.445779in}{2.570697in}}%
\pgfpathlineto{\pgfqpoint{3.449197in}{2.459277in}}%
\pgfpathlineto{\pgfqpoint{3.453471in}{2.546112in}}%
\pgfpathlineto{\pgfqpoint{3.458605in}{2.581389in}}%
\pgfpathlineto{\pgfqpoint{3.462882in}{2.402287in}}%
\pgfpathlineto{\pgfqpoint{3.467156in}{2.599271in}}%
\pgfpathlineto{\pgfqpoint{3.472285in}{2.457013in}}%
\pgfpathlineto{\pgfqpoint{3.475705in}{2.506064in}}%
\pgfpathlineto{\pgfqpoint{3.482550in}{2.439134in}}%
\pgfpathlineto{\pgfqpoint{3.485118in}{2.509688in}}%
\pgfpathlineto{\pgfqpoint{3.491104in}{2.444569in}}%
\pgfpathlineto{\pgfqpoint{3.495380in}{2.558978in}}%
\pgfpathlineto{\pgfqpoint{3.496237in}{2.416420in}}%
\pgfpathlineto{\pgfqpoint{3.501370in}{2.524971in}}%
\pgfpathlineto{\pgfqpoint{3.504790in}{2.422160in}}%
\pgfpathlineto{\pgfqpoint{3.512483in}{2.580121in}}%
\pgfpathlineto{\pgfqpoint{3.513340in}{2.479364in}}%
\pgfpathlineto{\pgfqpoint{3.519327in}{2.531797in}}%
\pgfpathlineto{\pgfqpoint{3.522750in}{2.450008in}}%
\pgfpathlineto{\pgfqpoint{3.529596in}{2.524699in}}%
\pgfpathlineto{\pgfqpoint{3.532161in}{2.415939in}}%
\pgfpathlineto{\pgfqpoint{3.534724in}{2.508299in}}%
\pgfpathlineto{\pgfqpoint{3.539002in}{2.417116in}}%
\pgfpathlineto{\pgfqpoint{3.543277in}{2.522161in}}%
\pgfpathlineto{\pgfqpoint{3.547556in}{2.433697in}}%
\pgfpathlineto{\pgfqpoint{3.555248in}{2.452784in}}%
\pgfpathlineto{\pgfqpoint{3.558668in}{2.548709in}}%
\pgfpathlineto{\pgfqpoint{3.565506in}{2.407721in}}%
\pgfpathlineto{\pgfqpoint{3.569782in}{2.538863in}}%
\pgfpathlineto{\pgfqpoint{3.574058in}{2.454145in}}%
\pgfpathlineto{\pgfqpoint{3.580899in}{2.554539in}}%
\pgfpathlineto{\pgfqpoint{3.583469in}{2.466980in}}%
\pgfpathlineto{\pgfqpoint{3.587741in}{2.497304in}}%
\pgfpathlineto{\pgfqpoint{3.591161in}{2.453570in}}%
\pgfpathlineto{\pgfqpoint{3.596294in}{2.572449in}}%
\pgfpathlineto{\pgfqpoint{3.599713in}{2.480421in}}%
\pgfpathlineto{\pgfqpoint{3.603993in}{2.539347in}}%
\pgfpathlineto{\pgfqpoint{3.608273in}{2.456590in}}%
\pgfpathlineto{\pgfqpoint{3.611692in}{2.569308in}}%
\pgfpathlineto{\pgfqpoint{3.619390in}{2.549918in}}%
\pgfpathlineto{\pgfqpoint{3.622816in}{2.439255in}}%
\pgfpathlineto{\pgfqpoint{3.625381in}{2.564657in}}%
\pgfpathlineto{\pgfqpoint{3.631368in}{2.495794in}}%
\pgfpathlineto{\pgfqpoint{3.635643in}{2.566832in}}%
\pgfpathlineto{\pgfqpoint{3.637355in}{2.491295in}}%
\pgfpathlineto{\pgfqpoint{3.643344in}{2.552394in}}%
\pgfpathlineto{\pgfqpoint{3.646770in}{2.435782in}}%
\pgfpathlineto{\pgfqpoint{3.651047in}{2.497728in}}%
\pgfpathlineto{\pgfqpoint{3.655322in}{2.448315in}}%
\pgfpathlineto{\pgfqpoint{3.658742in}{2.519322in}}%
\pgfpathlineto{\pgfqpoint{3.665581in}{2.462481in}}%
\pgfpathlineto{\pgfqpoint{3.667291in}{2.563026in}}%
\pgfpathlineto{\pgfqpoint{3.674993in}{2.462088in}}%
\pgfpathlineto{\pgfqpoint{3.677560in}{2.545901in}}%
\pgfpathlineto{\pgfqpoint{3.680976in}{2.430314in}}%
\pgfpathlineto{\pgfqpoint{3.686964in}{2.560309in}}%
\pgfpathlineto{\pgfqpoint{3.692094in}{2.437443in}}%
\pgfpathlineto{\pgfqpoint{3.692951in}{2.535240in}}%
\pgfpathlineto{\pgfqpoint{3.700650in}{2.561758in}}%
\pgfpathlineto{\pgfqpoint{3.701506in}{2.479667in}}%
\pgfpathlineto{\pgfqpoint{3.707493in}{2.440221in}}%
\pgfpathlineto{\pgfqpoint{3.710060in}{2.535301in}}%
\pgfpathlineto{\pgfqpoint{3.714341in}{2.529138in}}%
\pgfpathlineto{\pgfqpoint{3.721187in}{2.444992in}}%
\pgfpathlineto{\pgfqpoint{3.726321in}{2.539891in}}%
\pgfpathlineto{\pgfqpoint{3.727177in}{2.455262in}}%
\pgfpathlineto{\pgfqpoint{3.734016in}{2.463265in}}%
\pgfpathlineto{\pgfqpoint{3.735725in}{2.553964in}}%
\pgfpathlineto{\pgfqpoint{3.741708in}{2.601807in}}%
\pgfpathlineto{\pgfqpoint{3.745984in}{2.435267in}}%
\pgfpathlineto{\pgfqpoint{3.749406in}{2.520742in}}%
\pgfpathlineto{\pgfqpoint{3.756244in}{2.438408in}}%
\pgfpathlineto{\pgfqpoint{3.759666in}{2.578914in}}%
\pgfpathlineto{\pgfqpoint{3.762235in}{2.439948in}}%
\pgfpathlineto{\pgfqpoint{3.768221in}{2.521709in}}%
\pgfpathlineto{\pgfqpoint{3.771647in}{2.463265in}}%
\pgfpathlineto{\pgfqpoint{3.775066in}{2.555476in}}%
\pgfpathlineto{\pgfqpoint{3.778487in}{2.433969in}}%
\pgfpathlineto{\pgfqpoint{3.783618in}{2.415092in}}%
\pgfpathlineto{\pgfqpoint{3.787896in}{2.554055in}}%
\pgfpathlineto{\pgfqpoint{3.794743in}{2.510472in}}%
\pgfpathlineto{\pgfqpoint{3.799020in}{2.627782in}}%
\pgfpathlineto{\pgfqpoint{3.801584in}{2.512829in}}%
\pgfpathlineto{\pgfqpoint{3.805856in}{2.564839in}}%
\pgfpathlineto{\pgfqpoint{3.810133in}{2.473504in}}%
\pgfpathlineto{\pgfqpoint{3.814406in}{2.460457in}}%
\pgfpathlineto{\pgfqpoint{3.816973in}{2.611230in}}%
\pgfpathlineto{\pgfqpoint{3.821247in}{2.470030in}}%
\pgfpathlineto{\pgfqpoint{3.828940in}{2.564899in}}%
\pgfpathlineto{\pgfqpoint{3.830650in}{2.463900in}}%
\pgfpathlineto{\pgfqpoint{3.834922in}{2.556261in}}%
\pgfpathlineto{\pgfqpoint{3.839197in}{2.486914in}}%
\pgfpathlineto{\pgfqpoint{3.842619in}{2.588094in}}%
\pgfpathlineto{\pgfqpoint{3.848600in}{2.462027in}}%
\pgfpathlineto{\pgfqpoint{3.851170in}{2.560851in}}%
\pgfpathlineto{\pgfqpoint{3.856308in}{2.511107in}}%
\pgfpathlineto{\pgfqpoint{3.859732in}{2.569913in}}%
\pgfpathlineto{\pgfqpoint{3.864870in}{2.465470in}}%
\pgfpathlineto{\pgfqpoint{3.869999in}{2.542851in}}%
\pgfpathlineto{\pgfqpoint{3.875984in}{2.468127in}}%
\pgfpathlineto{\pgfqpoint{3.878551in}{2.545749in}}%
\pgfpathlineto{\pgfqpoint{3.882831in}{2.475558in}}%
\pgfpathlineto{\pgfqpoint{3.887107in}{2.455141in}}%
\pgfpathlineto{\pgfqpoint{3.890525in}{2.526902in}}%
\pgfpathlineto{\pgfqpoint{3.895656in}{2.486854in}}%
\pgfpathlineto{\pgfqpoint{3.899075in}{2.549253in}}%
\pgfpathlineto{\pgfqpoint{3.903348in}{2.491383in}}%
\pgfpathlineto{\pgfqpoint{3.908484in}{2.568824in}}%
\pgfpathlineto{\pgfqpoint{3.911052in}{2.441126in}}%
\pgfpathlineto{\pgfqpoint{3.917892in}{2.594134in}}%
\pgfpathlineto{\pgfqpoint{3.919603in}{2.514821in}}%
\pgfpathlineto{\pgfqpoint{3.925587in}{2.565683in}}%
\pgfpathlineto{\pgfqpoint{3.928153in}{2.522010in}}%
\pgfpathlineto{\pgfqpoint{3.934142in}{2.484315in}}%
\pgfpathlineto{\pgfqpoint{3.936707in}{2.602410in}}%
\pgfpathlineto{\pgfqpoint{3.943555in}{2.438860in}}%
\pgfpathlineto{\pgfqpoint{3.945268in}{2.556561in}}%
\pgfpathlineto{\pgfqpoint{3.949543in}{2.432911in}}%
\pgfpathlineto{\pgfqpoint{3.954676in}{2.439858in}}%
\pgfpathlineto{\pgfqpoint{3.961520in}{2.597698in}}%
\pgfpathlineto{\pgfqpoint{3.962375in}{2.493438in}}%
\pgfpathlineto{\pgfqpoint{3.970067in}{2.451455in}}%
\pgfpathlineto{\pgfqpoint{3.970924in}{2.575952in}}%
\pgfpathlineto{\pgfqpoint{3.976909in}{2.459007in}}%
\pgfpathlineto{\pgfqpoint{3.979475in}{2.533849in}}%
\pgfpathlineto{\pgfqpoint{3.984613in}{2.488363in}}%
\pgfpathlineto{\pgfqpoint{3.991456in}{2.464805in}}%
\pgfpathlineto{\pgfqpoint{3.992312in}{2.549372in}}%
\pgfpathlineto{\pgfqpoint{3.997445in}{2.464684in}}%
\pgfpathlineto{\pgfqpoint{4.000867in}{2.543816in}}%
\pgfpathlineto{\pgfqpoint{4.006003in}{2.477792in}}%
\pgfpathlineto{\pgfqpoint{4.011132in}{2.537776in}}%
\pgfpathlineto{\pgfqpoint{4.015412in}{2.440039in}}%
\pgfpathlineto{\pgfqpoint{4.018834in}{2.584772in}}%
\pgfpathlineto{\pgfqpoint{4.022252in}{2.480814in}}%
\pgfpathlineto{\pgfqpoint{4.029095in}{2.456953in}}%
\pgfpathlineto{\pgfqpoint{4.032514in}{2.549556in}}%
\pgfpathlineto{\pgfqpoint{4.035079in}{2.497486in}}%
\pgfpathlineto{\pgfqpoint{4.039357in}{2.552757in}}%
\pgfpathlineto{\pgfqpoint{4.046202in}{2.497365in}}%
\pgfpathlineto{\pgfqpoint{4.048767in}{2.579819in}}%
\pgfpathlineto{\pgfqpoint{4.054759in}{2.630681in}}%
\pgfpathlineto{\pgfqpoint{4.056470in}{2.518599in}}%
\pgfpathlineto{\pgfqpoint{4.063314in}{2.564687in}}%
\pgfpathlineto{\pgfqpoint{4.068446in}{2.392620in}}%
\pgfpathlineto{\pgfqpoint{4.070156in}{2.547138in}}%
\pgfpathlineto{\pgfqpoint{4.073578in}{2.460517in}}%
\pgfpathlineto{\pgfqpoint{4.079562in}{2.426085in}}%
\pgfpathlineto{\pgfqpoint{4.082985in}{2.543695in}}%
\pgfpathlineto{\pgfqpoint{4.093250in}{2.436052in}}%
\pgfpathlineto{\pgfqpoint{4.097525in}{2.569731in}}%
\pgfpathlineto{\pgfqpoint{4.102659in}{2.582115in}}%
\pgfpathlineto{\pgfqpoint{4.103512in}{2.442335in}}%
\pgfpathlineto{\pgfqpoint{4.108646in}{2.571785in}}%
\pgfpathlineto{\pgfqpoint{4.113785in}{2.439616in}}%
\pgfpathlineto{\pgfqpoint{4.119765in}{2.578007in}}%
\pgfpathlineto{\pgfqpoint{4.121474in}{2.520318in}}%
\pgfpathlineto{\pgfqpoint{4.128317in}{2.423970in}}%
\pgfpathlineto{\pgfqpoint{4.130028in}{2.476100in}}%
\pgfpathlineto{\pgfqpoint{4.136878in}{2.442999in}}%
\pgfpathlineto{\pgfqpoint{4.137733in}{2.517117in}}%
\pgfpathlineto{\pgfqpoint{4.142868in}{2.457919in}}%
\pgfpathlineto{\pgfqpoint{4.147142in}{2.471994in}}%
\pgfpathlineto{\pgfqpoint{4.149711in}{2.526600in}}%
\pgfpathlineto{\pgfqpoint{4.152280in}{2.462932in}}%
\pgfpathlineto{\pgfqpoint{4.158268in}{2.465892in}}%
\pgfpathlineto{\pgfqpoint{4.159124in}{2.569911in}}%
\pgfpathlineto{\pgfqpoint{4.162548in}{2.479846in}}%
\pgfpathlineto{\pgfqpoint{4.166829in}{2.477159in}}%
\pgfpathlineto{\pgfqpoint{4.170249in}{2.571483in}}%
\pgfpathlineto{\pgfqpoint{4.172814in}{2.495008in}}%
\pgfpathlineto{\pgfqpoint{4.177091in}{2.569610in}}%
\pgfpathlineto{\pgfqpoint{4.181364in}{2.490478in}}%
\pgfpathlineto{\pgfqpoint{4.184787in}{2.564718in}}%
\pgfpathlineto{\pgfqpoint{4.186496in}{2.471692in}}%
\pgfpathlineto{\pgfqpoint{4.191629in}{2.488847in}}%
\pgfpathlineto{\pgfqpoint{4.194194in}{2.563268in}}%
\pgfpathlineto{\pgfqpoint{4.196761in}{2.500626in}}%
\pgfpathlineto{\pgfqpoint{4.201034in}{2.584530in}}%
\pgfpathlineto{\pgfqpoint{4.204456in}{2.480118in}}%
\pgfpathlineto{\pgfqpoint{4.207878in}{2.577583in}}%
\pgfpathlineto{\pgfqpoint{4.210442in}{2.489271in}}%
\pgfpathlineto{\pgfqpoint{4.216431in}{2.421434in}}%
\pgfpathlineto{\pgfqpoint{4.218141in}{2.512708in}}%
\pgfpathlineto{\pgfqpoint{4.223272in}{2.484015in}}%
\pgfpathlineto{\pgfqpoint{4.224983in}{2.558678in}}%
\pgfpathlineto{\pgfqpoint{4.230116in}{2.475467in}}%
\pgfpathlineto{\pgfqpoint{4.230972in}{2.518748in}}%
\pgfpathlineto{\pgfqpoint{4.235248in}{2.489543in}}%
\pgfpathlineto{\pgfqpoint{4.238671in}{2.453028in}}%
\pgfpathlineto{\pgfqpoint{4.242092in}{2.528082in}}%
\pgfpathlineto{\pgfqpoint{4.246370in}{2.436898in}}%
\pgfpathlineto{\pgfqpoint{4.248080in}{2.504130in}}%
\pgfpathlineto{\pgfqpoint{4.254070in}{2.441700in}}%
\pgfpathlineto{\pgfqpoint{4.257490in}{2.516454in}}%
\pgfpathlineto{\pgfqpoint{4.259202in}{2.444871in}}%
\pgfpathlineto{\pgfqpoint{4.263479in}{2.465107in}}%
\pgfpathlineto{\pgfqpoint{4.266903in}{2.586040in}}%
\pgfpathlineto{\pgfqpoint{4.270325in}{2.460638in}}%
\pgfpathlineto{\pgfqpoint{4.273748in}{2.549072in}}%
\pgfpathlineto{\pgfqpoint{4.277169in}{2.475407in}}%
\pgfpathlineto{\pgfqpoint{4.278880in}{2.544421in}}%
\pgfpathlineto{\pgfqpoint{4.288282in}{2.462148in}}%
\pgfpathlineto{\pgfqpoint{4.289137in}{2.549556in}}%
\pgfpathlineto{\pgfqpoint{4.292562in}{2.605371in}}%
\pgfpathlineto{\pgfqpoint{4.295981in}{2.454899in}}%
\pgfpathlineto{\pgfqpoint{4.300259in}{2.535059in}}%
\pgfpathlineto{\pgfqpoint{4.304529in}{2.576981in}}%
\pgfpathlineto{\pgfqpoint{4.307096in}{2.496941in}}%
\pgfpathlineto{\pgfqpoint{4.311374in}{2.592203in}}%
\pgfpathlineto{\pgfqpoint{4.315651in}{2.502015in}}%
\pgfpathlineto{\pgfqpoint{4.318213in}{2.593953in}}%
\pgfpathlineto{\pgfqpoint{4.320779in}{2.505277in}}%
\pgfpathlineto{\pgfqpoint{4.324195in}{2.559039in}}%
\pgfpathlineto{\pgfqpoint{4.326759in}{2.478245in}}%
\pgfpathlineto{\pgfqpoint{4.330182in}{2.524608in}}%
\pgfpathlineto{\pgfqpoint{4.336171in}{2.486974in}}%
\pgfpathlineto{\pgfqpoint{4.338734in}{2.478878in}}%
\pgfpathlineto{\pgfqpoint{4.340445in}{2.564234in}}%
\pgfpathlineto{\pgfqpoint{4.346437in}{2.456772in}}%
\pgfpathlineto{\pgfqpoint{4.347293in}{2.526539in}}%
\pgfpathlineto{\pgfqpoint{4.353279in}{2.442999in}}%
\pgfpathlineto{\pgfqpoint{4.356698in}{2.571723in}}%
\pgfpathlineto{\pgfqpoint{4.357553in}{2.445444in}}%
\pgfpathlineto{\pgfqpoint{4.360978in}{2.529199in}}%
\pgfpathlineto{\pgfqpoint{4.366112in}{2.464684in}}%
\pgfpathlineto{\pgfqpoint{4.367823in}{2.549253in}}%
\pgfpathlineto{\pgfqpoint{4.371246in}{2.462993in}}%
\pgfpathlineto{\pgfqpoint{4.377235in}{2.553843in}}%
\pgfpathlineto{\pgfqpoint{4.378945in}{2.463537in}}%
\pgfpathlineto{\pgfqpoint{4.381508in}{2.556200in}}%
\pgfpathlineto{\pgfqpoint{4.385782in}{2.410741in}}%
\pgfpathlineto{\pgfqpoint{4.389205in}{2.602531in}}%
\pgfpathlineto{\pgfqpoint{4.392627in}{2.463507in}}%
\pgfpathlineto{\pgfqpoint{4.395194in}{2.533910in}}%
\pgfpathlineto{\pgfqpoint{4.399466in}{2.471117in}}%
\pgfpathlineto{\pgfqpoint{4.402888in}{2.523943in}}%
\pgfpathlineto{\pgfqpoint{4.407166in}{2.444509in}}%
\pgfpathlineto{\pgfqpoint{4.408878in}{2.512073in}}%
\pgfpathlineto{\pgfqpoint{4.414012in}{2.412644in}}%
\pgfpathlineto{\pgfqpoint{4.415722in}{2.495552in}}%
\pgfpathlineto{\pgfqpoint{4.419998in}{2.457074in}}%
\pgfpathlineto{\pgfqpoint{4.424273in}{2.573235in}}%
\pgfpathlineto{\pgfqpoint{4.425984in}{2.483652in}}%
\pgfpathlineto{\pgfqpoint{4.431970in}{2.490660in}}%
\pgfpathlineto{\pgfqpoint{4.435391in}{2.521105in}}%
\pgfpathlineto{\pgfqpoint{4.437960in}{2.470424in}}%
\pgfpathlineto{\pgfqpoint{4.439672in}{2.557347in}}%
\pgfpathlineto{\pgfqpoint{4.443953in}{2.582718in}}%
\pgfpathlineto{\pgfqpoint{4.447371in}{2.470605in}}%
\pgfpathlineto{\pgfqpoint{4.451649in}{2.531674in}}%
\pgfpathlineto{\pgfqpoint{4.455068in}{2.456227in}}%
\pgfpathlineto{\pgfqpoint{4.456777in}{2.520137in}}%
\pgfpathlineto{\pgfqpoint{4.461912in}{2.558978in}}%
\pgfpathlineto{\pgfqpoint{4.464478in}{2.502739in}}%
\pgfpathlineto{\pgfqpoint{4.469610in}{2.560851in}}%
\pgfpathlineto{\pgfqpoint{4.470466in}{2.476947in}}%
\pgfpathlineto{\pgfqpoint{4.473885in}{2.437743in}}%
\pgfpathlineto{\pgfqpoint{4.478161in}{2.547501in}}%
\pgfpathlineto{\pgfqpoint{4.482436in}{2.496639in}}%
\pgfpathlineto{\pgfqpoint{4.486712in}{2.530739in}}%
\pgfpathlineto{\pgfqpoint{4.490131in}{2.449522in}}%
\pgfpathlineto{\pgfqpoint{4.490987in}{2.556140in}}%
\pgfpathlineto{\pgfqpoint{4.494410in}{2.466496in}}%
\pgfpathlineto{\pgfqpoint{4.498682in}{2.553874in}}%
\pgfpathlineto{\pgfqpoint{4.501244in}{2.479483in}}%
\pgfpathlineto{\pgfqpoint{4.504663in}{2.543695in}}%
\pgfpathlineto{\pgfqpoint{4.508083in}{2.496306in}}%
\pgfpathlineto{\pgfqpoint{4.511500in}{2.557166in}}%
\pgfpathlineto{\pgfqpoint{4.514921in}{2.455231in}}%
\pgfpathlineto{\pgfqpoint{4.520052in}{2.524427in}}%
\pgfpathlineto{\pgfqpoint{4.524328in}{2.461725in}}%
\pgfpathlineto{\pgfqpoint{4.525183in}{2.554569in}}%
\pgfpathlineto{\pgfqpoint{4.531176in}{2.475377in}}%
\pgfpathlineto{\pgfqpoint{4.532887in}{2.546050in}}%
\pgfpathlineto{\pgfqpoint{4.535450in}{2.443120in}}%
\pgfpathlineto{\pgfqpoint{4.538868in}{2.737601in}}%
\pgfpathlineto{\pgfqpoint{4.542287in}{2.840776in}}%
\pgfpathlineto{\pgfqpoint{4.545713in}{2.776291in}}%
\pgfpathlineto{\pgfqpoint{4.549136in}{2.797947in}}%
\pgfpathlineto{\pgfqpoint{4.552557in}{2.735487in}}%
\pgfpathlineto{\pgfqpoint{4.557688in}{2.817338in}}%
\pgfpathlineto{\pgfqpoint{4.560254in}{2.727032in}}%
\pgfpathlineto{\pgfqpoint{4.564531in}{2.776775in}}%
\pgfpathlineto{\pgfqpoint{4.566243in}{2.692539in}}%
\pgfpathlineto{\pgfqpoint{4.569658in}{2.777892in}}%
\pgfpathlineto{\pgfqpoint{4.573939in}{2.714134in}}%
\pgfpathlineto{\pgfqpoint{4.577361in}{2.795350in}}%
\pgfpathlineto{\pgfqpoint{4.581640in}{2.666745in}}%
\pgfpathlineto{\pgfqpoint{4.583350in}{2.830053in}}%
\pgfpathlineto{\pgfqpoint{4.588482in}{2.836004in}}%
\pgfpathlineto{\pgfqpoint{4.590190in}{2.753731in}}%
\pgfpathlineto{\pgfqpoint{4.593612in}{2.883000in}}%
\pgfpathlineto{\pgfqpoint{4.597035in}{2.766718in}}%
\pgfpathlineto{\pgfqpoint{4.600455in}{2.817580in}}%
\pgfpathlineto{\pgfqpoint{4.605589in}{2.788042in}}%
\pgfpathlineto{\pgfqpoint{4.608155in}{2.491688in}}%
\pgfpathlineto{\pgfqpoint{4.612433in}{2.543032in}}%
\pgfpathlineto{\pgfqpoint{4.616710in}{2.440039in}}%
\pgfpathlineto{\pgfqpoint{4.617565in}{2.494284in}}%
\pgfpathlineto{\pgfqpoint{4.621838in}{2.524427in}}%
\pgfpathlineto{\pgfqpoint{4.624405in}{2.574684in}}%
\pgfpathlineto{\pgfqpoint{4.627827in}{2.454175in}}%
\pgfpathlineto{\pgfqpoint{4.632102in}{2.558133in}}%
\pgfpathlineto{\pgfqpoint{4.635525in}{2.447349in}}%
\pgfpathlineto{\pgfqpoint{4.638091in}{2.543032in}}%
\pgfpathlineto{\pgfqpoint{4.642368in}{2.500385in}}%
\pgfpathlineto{\pgfqpoint{4.645787in}{2.605432in}}%
\pgfpathlineto{\pgfqpoint{4.649207in}{2.515486in}}%
\pgfpathlineto{\pgfqpoint{4.652631in}{2.541280in}}%
\pgfpathlineto{\pgfqpoint{4.656051in}{2.473837in}}%
\pgfpathlineto{\pgfqpoint{4.661182in}{2.528231in}}%
\pgfpathlineto{\pgfqpoint{4.662894in}{2.462539in}}%
\pgfpathlineto{\pgfqpoint{4.666317in}{2.505217in}}%
\pgfpathlineto{\pgfqpoint{4.668881in}{2.453750in}}%
\pgfpathlineto{\pgfqpoint{4.674015in}{2.474893in}}%
\pgfpathlineto{\pgfqpoint{4.677437in}{2.553450in}}%
\pgfpathlineto{\pgfqpoint{4.679148in}{2.498996in}}%
\pgfpathlineto{\pgfqpoint{4.682567in}{2.439555in}}%
\pgfpathlineto{\pgfqpoint{4.687701in}{2.542367in}}%
\pgfpathlineto{\pgfqpoint{4.689413in}{2.551610in}}%
\pgfpathlineto{\pgfqpoint{4.693685in}{2.478669in}}%
\pgfpathlineto{\pgfqpoint{4.697112in}{2.528533in}}%
\pgfpathlineto{\pgfqpoint{4.700535in}{2.432399in}}%
\pgfpathlineto{\pgfqpoint{4.704809in}{2.540917in}}%
\pgfpathlineto{\pgfqpoint{4.706522in}{2.485102in}}%
\pgfpathlineto{\pgfqpoint{4.711657in}{2.554085in}}%
\pgfpathlineto{\pgfqpoint{4.715932in}{2.507090in}}%
\pgfpathlineto{\pgfqpoint{4.718499in}{2.584530in}}%
\pgfpathlineto{\pgfqpoint{4.721922in}{2.497213in}}%
\pgfpathlineto{\pgfqpoint{4.724492in}{2.554448in}}%
\pgfpathlineto{\pgfqpoint{4.727918in}{2.471873in}}%
\pgfpathlineto{\pgfqpoint{4.730484in}{2.520923in}}%
\pgfpathlineto{\pgfqpoint{4.736479in}{2.470000in}}%
\pgfpathlineto{\pgfqpoint{4.737335in}{2.574563in}}%
\pgfpathlineto{\pgfqpoint{4.742472in}{2.561214in}}%
\pgfpathlineto{\pgfqpoint{4.745895in}{2.440160in}}%
\pgfpathlineto{\pgfqpoint{4.747605in}{2.522161in}}%
\pgfpathlineto{\pgfqpoint{4.751882in}{2.609659in}}%
\pgfpathlineto{\pgfqpoint{4.755301in}{2.482384in}}%
\pgfpathlineto{\pgfqpoint{4.759573in}{2.538170in}}%
\pgfpathlineto{\pgfqpoint{4.763849in}{2.486372in}}%
\pgfpathlineto{\pgfqpoint{4.766411in}{2.593652in}}%
\pgfpathlineto{\pgfqpoint{4.769831in}{2.471752in}}%
\pgfpathlineto{\pgfqpoint{4.773255in}{2.532581in}}%
\pgfpathlineto{\pgfqpoint{4.776680in}{2.544844in}}%
\pgfpathlineto{\pgfqpoint{4.778393in}{2.447319in}}%
\pgfpathlineto{\pgfqpoint{4.782675in}{2.502015in}}%
\pgfpathlineto{\pgfqpoint{4.787806in}{2.465561in}}%
\pgfpathlineto{\pgfqpoint{4.791230in}{2.500929in}}%
\pgfpathlineto{\pgfqpoint{4.794650in}{2.482293in}}%
\pgfpathlineto{\pgfqpoint{4.795506in}{2.527991in}}%
\pgfpathlineto{\pgfqpoint{4.795506in}{2.527991in}}%
\pgfusepath{stroke}%
\end{pgfscope}%
\begin{pgfscope}%
\pgfsetrectcap%
\pgfsetmiterjoin%
\pgfsetlinewidth{0.803000pt}%
\definecolor{currentstroke}{rgb}{0.000000,0.000000,0.000000}%
\pgfsetstrokecolor{currentstroke}%
\pgfsetdash{}{0pt}%
\pgfpathmoveto{\pgfqpoint{0.484581in}{2.334497in}}%
\pgfpathlineto{\pgfqpoint{0.484581in}{2.909119in}}%
\pgfusepath{stroke}%
\end{pgfscope}%
\begin{pgfscope}%
\pgfsetrectcap%
\pgfsetmiterjoin%
\pgfsetlinewidth{0.803000pt}%
\definecolor{currentstroke}{rgb}{0.000000,0.000000,0.000000}%
\pgfsetstrokecolor{currentstroke}%
\pgfsetdash{}{0pt}%
\pgfpathmoveto{\pgfqpoint{5.000788in}{2.334497in}}%
\pgfpathlineto{\pgfqpoint{5.000788in}{2.909119in}}%
\pgfusepath{stroke}%
\end{pgfscope}%
\begin{pgfscope}%
\pgfsetrectcap%
\pgfsetmiterjoin%
\pgfsetlinewidth{0.803000pt}%
\definecolor{currentstroke}{rgb}{0.000000,0.000000,0.000000}%
\pgfsetstrokecolor{currentstroke}%
\pgfsetdash{}{0pt}%
\pgfpathmoveto{\pgfqpoint{0.484581in}{2.334497in}}%
\pgfpathlineto{\pgfqpoint{5.000788in}{2.334497in}}%
\pgfusepath{stroke}%
\end{pgfscope}%
\begin{pgfscope}%
\pgfsetrectcap%
\pgfsetmiterjoin%
\pgfsetlinewidth{0.803000pt}%
\definecolor{currentstroke}{rgb}{0.000000,0.000000,0.000000}%
\pgfsetstrokecolor{currentstroke}%
\pgfsetdash{}{0pt}%
\pgfpathmoveto{\pgfqpoint{0.484581in}{2.909119in}}%
\pgfpathlineto{\pgfqpoint{5.000788in}{2.909119in}}%
\pgfusepath{stroke}%
\end{pgfscope}%
\begin{pgfscope}%
\pgfsetbuttcap%
\pgfsetmiterjoin%
\definecolor{currentfill}{rgb}{1.000000,1.000000,1.000000}%
\pgfsetfillcolor{currentfill}%
\pgfsetlinewidth{0.000000pt}%
\definecolor{currentstroke}{rgb}{0.000000,0.000000,0.000000}%
\pgfsetstrokecolor{currentstroke}%
\pgfsetstrokeopacity{0.000000}%
\pgfsetdash{}{0pt}%
\pgfpathmoveto{\pgfqpoint{0.484581in}{1.437021in}}%
\pgfpathlineto{\pgfqpoint{5.000788in}{1.437021in}}%
\pgfpathlineto{\pgfqpoint{5.000788in}{2.011643in}}%
\pgfpathlineto{\pgfqpoint{0.484581in}{2.011643in}}%
\pgfpathlineto{\pgfqpoint{0.484581in}{1.437021in}}%
\pgfpathclose%
\pgfusepath{fill}%
\end{pgfscope}%
\begin{pgfscope}%
\pgfsetbuttcap%
\pgfsetroundjoin%
\definecolor{currentfill}{rgb}{0.000000,0.000000,0.000000}%
\pgfsetfillcolor{currentfill}%
\pgfsetlinewidth{0.803000pt}%
\definecolor{currentstroke}{rgb}{0.000000,0.000000,0.000000}%
\pgfsetstrokecolor{currentstroke}%
\pgfsetdash{}{0pt}%
\pgfsys@defobject{currentmarker}{\pgfqpoint{0.000000in}{-0.048611in}}{\pgfqpoint{0.000000in}{0.000000in}}{%
\pgfpathmoveto{\pgfqpoint{0.000000in}{0.000000in}}%
\pgfpathlineto{\pgfqpoint{0.000000in}{-0.048611in}}%
\pgfusepath{stroke,fill}%
}%
\begin{pgfscope}%
\pgfsys@transformshift{0.689546in}{1.437021in}%
\pgfsys@useobject{currentmarker}{}%
\end{pgfscope}%
\end{pgfscope}%
\begin{pgfscope}%
\pgfsetbuttcap%
\pgfsetroundjoin%
\definecolor{currentfill}{rgb}{0.000000,0.000000,0.000000}%
\pgfsetfillcolor{currentfill}%
\pgfsetlinewidth{0.803000pt}%
\definecolor{currentstroke}{rgb}{0.000000,0.000000,0.000000}%
\pgfsetstrokecolor{currentstroke}%
\pgfsetdash{}{0pt}%
\pgfsys@defobject{currentmarker}{\pgfqpoint{0.000000in}{-0.048611in}}{\pgfqpoint{0.000000in}{0.000000in}}{%
\pgfpathmoveto{\pgfqpoint{0.000000in}{0.000000in}}%
\pgfpathlineto{\pgfqpoint{0.000000in}{-0.048611in}}%
\pgfusepath{stroke,fill}%
}%
\begin{pgfscope}%
\pgfsys@transformshift{1.202878in}{1.437021in}%
\pgfsys@useobject{currentmarker}{}%
\end{pgfscope}%
\end{pgfscope}%
\begin{pgfscope}%
\pgfsetbuttcap%
\pgfsetroundjoin%
\definecolor{currentfill}{rgb}{0.000000,0.000000,0.000000}%
\pgfsetfillcolor{currentfill}%
\pgfsetlinewidth{0.803000pt}%
\definecolor{currentstroke}{rgb}{0.000000,0.000000,0.000000}%
\pgfsetstrokecolor{currentstroke}%
\pgfsetdash{}{0pt}%
\pgfsys@defobject{currentmarker}{\pgfqpoint{0.000000in}{-0.048611in}}{\pgfqpoint{0.000000in}{0.000000in}}{%
\pgfpathmoveto{\pgfqpoint{0.000000in}{0.000000in}}%
\pgfpathlineto{\pgfqpoint{0.000000in}{-0.048611in}}%
\pgfusepath{stroke,fill}%
}%
\begin{pgfscope}%
\pgfsys@transformshift{1.716211in}{1.437021in}%
\pgfsys@useobject{currentmarker}{}%
\end{pgfscope}%
\end{pgfscope}%
\begin{pgfscope}%
\pgfsetbuttcap%
\pgfsetroundjoin%
\definecolor{currentfill}{rgb}{0.000000,0.000000,0.000000}%
\pgfsetfillcolor{currentfill}%
\pgfsetlinewidth{0.803000pt}%
\definecolor{currentstroke}{rgb}{0.000000,0.000000,0.000000}%
\pgfsetstrokecolor{currentstroke}%
\pgfsetdash{}{0pt}%
\pgfsys@defobject{currentmarker}{\pgfqpoint{0.000000in}{-0.048611in}}{\pgfqpoint{0.000000in}{0.000000in}}{%
\pgfpathmoveto{\pgfqpoint{0.000000in}{0.000000in}}%
\pgfpathlineto{\pgfqpoint{0.000000in}{-0.048611in}}%
\pgfusepath{stroke,fill}%
}%
\begin{pgfscope}%
\pgfsys@transformshift{2.229543in}{1.437021in}%
\pgfsys@useobject{currentmarker}{}%
\end{pgfscope}%
\end{pgfscope}%
\begin{pgfscope}%
\pgfsetbuttcap%
\pgfsetroundjoin%
\definecolor{currentfill}{rgb}{0.000000,0.000000,0.000000}%
\pgfsetfillcolor{currentfill}%
\pgfsetlinewidth{0.803000pt}%
\definecolor{currentstroke}{rgb}{0.000000,0.000000,0.000000}%
\pgfsetstrokecolor{currentstroke}%
\pgfsetdash{}{0pt}%
\pgfsys@defobject{currentmarker}{\pgfqpoint{0.000000in}{-0.048611in}}{\pgfqpoint{0.000000in}{0.000000in}}{%
\pgfpathmoveto{\pgfqpoint{0.000000in}{0.000000in}}%
\pgfpathlineto{\pgfqpoint{0.000000in}{-0.048611in}}%
\pgfusepath{stroke,fill}%
}%
\begin{pgfscope}%
\pgfsys@transformshift{2.742876in}{1.437021in}%
\pgfsys@useobject{currentmarker}{}%
\end{pgfscope}%
\end{pgfscope}%
\begin{pgfscope}%
\pgfsetbuttcap%
\pgfsetroundjoin%
\definecolor{currentfill}{rgb}{0.000000,0.000000,0.000000}%
\pgfsetfillcolor{currentfill}%
\pgfsetlinewidth{0.803000pt}%
\definecolor{currentstroke}{rgb}{0.000000,0.000000,0.000000}%
\pgfsetstrokecolor{currentstroke}%
\pgfsetdash{}{0pt}%
\pgfsys@defobject{currentmarker}{\pgfqpoint{0.000000in}{-0.048611in}}{\pgfqpoint{0.000000in}{0.000000in}}{%
\pgfpathmoveto{\pgfqpoint{0.000000in}{0.000000in}}%
\pgfpathlineto{\pgfqpoint{0.000000in}{-0.048611in}}%
\pgfusepath{stroke,fill}%
}%
\begin{pgfscope}%
\pgfsys@transformshift{3.256208in}{1.437021in}%
\pgfsys@useobject{currentmarker}{}%
\end{pgfscope}%
\end{pgfscope}%
\begin{pgfscope}%
\pgfsetbuttcap%
\pgfsetroundjoin%
\definecolor{currentfill}{rgb}{0.000000,0.000000,0.000000}%
\pgfsetfillcolor{currentfill}%
\pgfsetlinewidth{0.803000pt}%
\definecolor{currentstroke}{rgb}{0.000000,0.000000,0.000000}%
\pgfsetstrokecolor{currentstroke}%
\pgfsetdash{}{0pt}%
\pgfsys@defobject{currentmarker}{\pgfqpoint{0.000000in}{-0.048611in}}{\pgfqpoint{0.000000in}{0.000000in}}{%
\pgfpathmoveto{\pgfqpoint{0.000000in}{0.000000in}}%
\pgfpathlineto{\pgfqpoint{0.000000in}{-0.048611in}}%
\pgfusepath{stroke,fill}%
}%
\begin{pgfscope}%
\pgfsys@transformshift{3.769541in}{1.437021in}%
\pgfsys@useobject{currentmarker}{}%
\end{pgfscope}%
\end{pgfscope}%
\begin{pgfscope}%
\pgfsetbuttcap%
\pgfsetroundjoin%
\definecolor{currentfill}{rgb}{0.000000,0.000000,0.000000}%
\pgfsetfillcolor{currentfill}%
\pgfsetlinewidth{0.803000pt}%
\definecolor{currentstroke}{rgb}{0.000000,0.000000,0.000000}%
\pgfsetstrokecolor{currentstroke}%
\pgfsetdash{}{0pt}%
\pgfsys@defobject{currentmarker}{\pgfqpoint{0.000000in}{-0.048611in}}{\pgfqpoint{0.000000in}{0.000000in}}{%
\pgfpathmoveto{\pgfqpoint{0.000000in}{0.000000in}}%
\pgfpathlineto{\pgfqpoint{0.000000in}{-0.048611in}}%
\pgfusepath{stroke,fill}%
}%
\begin{pgfscope}%
\pgfsys@transformshift{4.282873in}{1.437021in}%
\pgfsys@useobject{currentmarker}{}%
\end{pgfscope}%
\end{pgfscope}%
\begin{pgfscope}%
\pgfsetbuttcap%
\pgfsetroundjoin%
\definecolor{currentfill}{rgb}{0.000000,0.000000,0.000000}%
\pgfsetfillcolor{currentfill}%
\pgfsetlinewidth{0.803000pt}%
\definecolor{currentstroke}{rgb}{0.000000,0.000000,0.000000}%
\pgfsetstrokecolor{currentstroke}%
\pgfsetdash{}{0pt}%
\pgfsys@defobject{currentmarker}{\pgfqpoint{0.000000in}{-0.048611in}}{\pgfqpoint{0.000000in}{0.000000in}}{%
\pgfpathmoveto{\pgfqpoint{0.000000in}{0.000000in}}%
\pgfpathlineto{\pgfqpoint{0.000000in}{-0.048611in}}%
\pgfusepath{stroke,fill}%
}%
\begin{pgfscope}%
\pgfsys@transformshift{4.796206in}{1.437021in}%
\pgfsys@useobject{currentmarker}{}%
\end{pgfscope}%
\end{pgfscope}%
\begin{pgfscope}%
\pgfsetbuttcap%
\pgfsetroundjoin%
\definecolor{currentfill}{rgb}{0.000000,0.000000,0.000000}%
\pgfsetfillcolor{currentfill}%
\pgfsetlinewidth{0.803000pt}%
\definecolor{currentstroke}{rgb}{0.000000,0.000000,0.000000}%
\pgfsetstrokecolor{currentstroke}%
\pgfsetdash{}{0pt}%
\pgfsys@defobject{currentmarker}{\pgfqpoint{-0.048611in}{0.000000in}}{\pgfqpoint{-0.000000in}{0.000000in}}{%
\pgfpathmoveto{\pgfqpoint{-0.000000in}{0.000000in}}%
\pgfpathlineto{\pgfqpoint{-0.048611in}{0.000000in}}%
\pgfusepath{stroke,fill}%
}%
\begin{pgfscope}%
\pgfsys@transformshift{0.484581in}{1.615020in}%
\pgfsys@useobject{currentmarker}{}%
\end{pgfscope}%
\end{pgfscope}%
\begin{pgfscope}%
\definecolor{textcolor}{rgb}{0.000000,0.000000,0.000000}%
\pgfsetstrokecolor{textcolor}%
\pgfsetfillcolor{textcolor}%
\pgftext[x=0.328331in, y=1.576464in, left, base]{\color{textcolor}\rmfamily\fontsize{8.000000}{9.600000}\selectfont \(\displaystyle {0}\)}%
\end{pgfscope}%
\begin{pgfscope}%
\pgfsetbuttcap%
\pgfsetroundjoin%
\definecolor{currentfill}{rgb}{0.000000,0.000000,0.000000}%
\pgfsetfillcolor{currentfill}%
\pgfsetlinewidth{0.803000pt}%
\definecolor{currentstroke}{rgb}{0.000000,0.000000,0.000000}%
\pgfsetstrokecolor{currentstroke}%
\pgfsetdash{}{0pt}%
\pgfsys@defobject{currentmarker}{\pgfqpoint{-0.048611in}{0.000000in}}{\pgfqpoint{-0.000000in}{0.000000in}}{%
\pgfpathmoveto{\pgfqpoint{-0.000000in}{0.000000in}}%
\pgfpathlineto{\pgfqpoint{-0.048611in}{0.000000in}}%
\pgfusepath{stroke,fill}%
}%
\begin{pgfscope}%
\pgfsys@transformshift{0.484581in}{1.818561in}%
\pgfsys@useobject{currentmarker}{}%
\end{pgfscope}%
\end{pgfscope}%
\begin{pgfscope}%
\definecolor{textcolor}{rgb}{0.000000,0.000000,0.000000}%
\pgfsetstrokecolor{textcolor}%
\pgfsetfillcolor{textcolor}%
\pgftext[x=0.328331in, y=1.780005in, left, base]{\color{textcolor}\rmfamily\fontsize{8.000000}{9.600000}\selectfont \(\displaystyle {5}\)}%
\end{pgfscope}%
\begin{pgfscope}%
\definecolor{textcolor}{rgb}{0.000000,0.000000,0.000000}%
\pgfsetstrokecolor{textcolor}%
\pgfsetfillcolor{textcolor}%
\pgftext[x=0.272775in,y=1.724332in,,bottom,rotate=90.000000]{\color{textcolor}\rmfamily\fontsize{10.000000}{12.000000}\selectfont Voltage deviation in \unit{\V}}%
\end{pgfscope}%
\begin{pgfscope}%
\definecolor{textcolor}{rgb}{0.000000,0.000000,0.000000}%
\pgfsetstrokecolor{textcolor}%
\pgfsetfillcolor{textcolor}%
\pgftext[x=0.484581in,y=2.053309in,left,base]{\color{textcolor}\rmfamily\fontsize{8.000000}{9.600000}\selectfont \(\displaystyle \times{10^{\ensuremath{-}6}}{}\)}%
\end{pgfscope}%
\begin{pgfscope}%
\pgfpathrectangle{\pgfqpoint{0.484581in}{1.437021in}}{\pgfqpoint{4.516206in}{0.574622in}}%
\pgfusepath{clip}%
\pgfsetrectcap%
\pgfsetroundjoin%
\pgfsetlinewidth{0.501875pt}%
\definecolor{currentstroke}{rgb}{0.003922,0.450980,0.698039}%
\pgfsetstrokecolor{currentstroke}%
\pgfsetstrokeopacity{0.700000}%
\pgfsetdash{}{0pt}%
\pgfpathmoveto{\pgfqpoint{0.689863in}{1.586143in}}%
\pgfpathlineto{\pgfqpoint{0.691573in}{1.625337in}}%
\pgfpathlineto{\pgfqpoint{0.694995in}{1.554591in}}%
\pgfpathlineto{\pgfqpoint{0.700132in}{1.624113in}}%
\pgfpathlineto{\pgfqpoint{0.706120in}{1.553835in}}%
\pgfpathlineto{\pgfqpoint{0.707833in}{1.630006in}}%
\pgfpathlineto{\pgfqpoint{0.712110in}{1.550218in}}%
\pgfpathlineto{\pgfqpoint{0.719814in}{1.595418in}}%
\pgfpathlineto{\pgfqpoint{0.723233in}{1.529484in}}%
\pgfpathlineto{\pgfqpoint{0.725797in}{1.600142in}}%
\pgfpathlineto{\pgfqpoint{0.730072in}{1.521436in}}%
\pgfpathlineto{\pgfqpoint{0.734351in}{1.620555in}}%
\pgfpathlineto{\pgfqpoint{0.740340in}{1.509859in}}%
\pgfpathlineto{\pgfqpoint{0.742908in}{1.627964in}}%
\pgfpathlineto{\pgfqpoint{0.749745in}{1.515752in}}%
\pgfpathlineto{\pgfqpoint{0.750602in}{1.631112in}}%
\pgfpathlineto{\pgfqpoint{0.757448in}{1.584280in}}%
\pgfpathlineto{\pgfqpoint{0.761721in}{1.771029in}}%
\pgfpathlineto{\pgfqpoint{0.764288in}{1.661499in}}%
\pgfpathlineto{\pgfqpoint{0.770278in}{1.670975in}}%
\pgfpathlineto{\pgfqpoint{0.771990in}{1.588302in}}%
\pgfpathlineto{\pgfqpoint{0.777980in}{1.623412in}}%
\pgfpathlineto{\pgfqpoint{0.783113in}{1.520705in}}%
\pgfpathlineto{\pgfqpoint{0.784824in}{1.596583in}}%
\pgfpathlineto{\pgfqpoint{0.792516in}{1.565616in}}%
\pgfpathlineto{\pgfqpoint{0.793373in}{1.644234in}}%
\pgfpathlineto{\pgfqpoint{0.797648in}{1.563749in}}%
\pgfpathlineto{\pgfqpoint{0.802779in}{1.499769in}}%
\pgfpathlineto{\pgfqpoint{0.807054in}{1.491546in}}%
\pgfpathlineto{\pgfqpoint{0.811330in}{1.621837in}}%
\pgfpathlineto{\pgfqpoint{0.816461in}{1.552432in}}%
\pgfpathlineto{\pgfqpoint{0.819881in}{1.653214in}}%
\pgfpathlineto{\pgfqpoint{0.825871in}{1.546953in}}%
\pgfpathlineto{\pgfqpoint{0.827584in}{1.683834in}}%
\pgfpathlineto{\pgfqpoint{0.836141in}{1.522426in}}%
\pgfpathlineto{\pgfqpoint{0.841276in}{1.594483in}}%
\pgfpathlineto{\pgfqpoint{0.848114in}{1.547709in}}%
\pgfpathlineto{\pgfqpoint{0.851531in}{1.619912in}}%
\pgfpathlineto{\pgfqpoint{0.854095in}{1.645692in}}%
\pgfpathlineto{\pgfqpoint{0.857519in}{1.567596in}}%
\pgfpathlineto{\pgfqpoint{0.864372in}{1.555406in}}%
\pgfpathlineto{\pgfqpoint{0.867793in}{1.625220in}}%
\pgfpathlineto{\pgfqpoint{0.871215in}{1.560422in}}%
\pgfpathlineto{\pgfqpoint{0.874639in}{1.655431in}}%
\pgfpathlineto{\pgfqpoint{0.879771in}{1.700631in}}%
\pgfpathlineto{\pgfqpoint{0.883195in}{1.617114in}}%
\pgfpathlineto{\pgfqpoint{0.889177in}{1.721161in}}%
\pgfpathlineto{\pgfqpoint{0.893457in}{1.578095in}}%
\pgfpathlineto{\pgfqpoint{0.898587in}{1.665404in}}%
\pgfpathlineto{\pgfqpoint{0.901153in}{1.582702in}}%
\pgfpathlineto{\pgfqpoint{0.906287in}{1.633735in}}%
\pgfpathlineto{\pgfqpoint{0.908853in}{1.559285in}}%
\pgfpathlineto{\pgfqpoint{0.916550in}{1.533012in}}%
\pgfpathlineto{\pgfqpoint{0.918259in}{1.614516in}}%
\pgfpathlineto{\pgfqpoint{0.924241in}{1.548176in}}%
\pgfpathlineto{\pgfqpoint{0.925951in}{1.593201in}}%
\pgfpathlineto{\pgfqpoint{0.930227in}{1.522163in}}%
\pgfpathlineto{\pgfqpoint{0.936214in}{1.628775in}}%
\pgfpathlineto{\pgfqpoint{0.938782in}{1.509446in}}%
\pgfpathlineto{\pgfqpoint{0.943058in}{1.594717in}}%
\pgfpathlineto{\pgfqpoint{0.947330in}{1.479644in}}%
\pgfpathlineto{\pgfqpoint{0.952461in}{1.636943in}}%
\pgfpathlineto{\pgfqpoint{0.958451in}{1.527588in}}%
\pgfpathlineto{\pgfqpoint{0.961015in}{1.600025in}}%
\pgfpathlineto{\pgfqpoint{0.967858in}{1.482388in}}%
\pgfpathlineto{\pgfqpoint{0.970425in}{1.625045in}}%
\pgfpathlineto{\pgfqpoint{0.976408in}{1.643591in}}%
\pgfpathlineto{\pgfqpoint{0.977263in}{1.606586in}}%
\pgfpathlineto{\pgfqpoint{0.982392in}{1.522689in}}%
\pgfpathlineto{\pgfqpoint{0.989232in}{1.502626in}}%
\pgfpathlineto{\pgfqpoint{0.990941in}{1.614896in}}%
\pgfpathlineto{\pgfqpoint{1.000349in}{1.478625in}}%
\pgfpathlineto{\pgfqpoint{1.002914in}{1.594454in}}%
\pgfpathlineto{\pgfqpoint{1.009758in}{1.546573in}}%
\pgfpathlineto{\pgfqpoint{1.011468in}{1.620789in}}%
\pgfpathlineto{\pgfqpoint{1.017456in}{1.559958in}}%
\pgfpathlineto{\pgfqpoint{1.022584in}{1.708156in}}%
\pgfpathlineto{\pgfqpoint{1.024292in}{1.600609in}}%
\pgfpathlineto{\pgfqpoint{1.031988in}{1.634320in}}%
\pgfpathlineto{\pgfqpoint{1.032842in}{1.691184in}}%
\pgfpathlineto{\pgfqpoint{1.039687in}{1.748691in}}%
\pgfpathlineto{\pgfqpoint{1.043113in}{1.674416in}}%
\pgfpathlineto{\pgfqpoint{1.048251in}{1.722268in}}%
\pgfpathlineto{\pgfqpoint{1.049962in}{1.651522in}}%
\pgfpathlineto{\pgfqpoint{1.055949in}{1.713289in}}%
\pgfpathlineto{\pgfqpoint{1.060226in}{1.562756in}}%
\pgfpathlineto{\pgfqpoint{1.064502in}{1.663011in}}%
\pgfpathlineto{\pgfqpoint{1.068774in}{1.536859in}}%
\pgfpathlineto{\pgfqpoint{1.071339in}{1.603086in}}%
\pgfpathlineto{\pgfqpoint{1.076466in}{1.541641in}}%
\pgfpathlineto{\pgfqpoint{1.080746in}{1.617724in}}%
\pgfpathlineto{\pgfqpoint{1.086736in}{1.542109in}}%
\pgfpathlineto{\pgfqpoint{1.091868in}{1.631226in}}%
\pgfpathlineto{\pgfqpoint{1.096997in}{1.523620in}}%
\pgfpathlineto{\pgfqpoint{1.102978in}{1.566836in}}%
\pgfpathlineto{\pgfqpoint{1.108959in}{1.626415in}}%
\pgfpathlineto{\pgfqpoint{1.111524in}{1.539661in}}%
\pgfpathlineto{\pgfqpoint{1.115796in}{1.683048in}}%
\pgfpathlineto{\pgfqpoint{1.121786in}{1.726002in}}%
\pgfpathlineto{\pgfqpoint{1.122640in}{1.618659in}}%
\pgfpathlineto{\pgfqpoint{1.130340in}{1.593551in}}%
\pgfpathlineto{\pgfqpoint{1.132050in}{1.648929in}}%
\pgfpathlineto{\pgfqpoint{1.136328in}{1.567830in}}%
\pgfpathlineto{\pgfqpoint{1.140604in}{1.724602in}}%
\pgfpathlineto{\pgfqpoint{1.146593in}{1.623032in}}%
\pgfpathlineto{\pgfqpoint{1.154289in}{1.756154in}}%
\pgfpathlineto{\pgfqpoint{1.157708in}{1.579319in}}%
\pgfpathlineto{\pgfqpoint{1.163696in}{1.517322in}}%
\pgfpathlineto{\pgfqpoint{1.165408in}{1.614546in}}%
\pgfpathlineto{\pgfqpoint{1.171400in}{1.551033in}}%
\pgfpathlineto{\pgfqpoint{1.174823in}{1.608072in}}%
\pgfpathlineto{\pgfqpoint{1.179102in}{1.542781in}}%
\pgfpathlineto{\pgfqpoint{1.185947in}{1.665579in}}%
\pgfpathlineto{\pgfqpoint{1.188512in}{1.532749in}}%
\pgfpathlineto{\pgfqpoint{1.191934in}{1.637001in}}%
\pgfpathlineto{\pgfqpoint{1.196209in}{1.492595in}}%
\pgfpathlineto{\pgfqpoint{1.203055in}{1.490786in}}%
\pgfpathlineto{\pgfqpoint{1.203910in}{1.586611in}}%
\pgfpathlineto{\pgfqpoint{1.211615in}{1.672053in}}%
\pgfpathlineto{\pgfqpoint{1.212472in}{1.589643in}}%
\pgfpathlineto{\pgfqpoint{1.216755in}{1.605128in}}%
\pgfpathlineto{\pgfqpoint{1.221891in}{1.553397in}}%
\pgfpathlineto{\pgfqpoint{1.228734in}{1.607780in}}%
\pgfpathlineto{\pgfqpoint{1.230446in}{1.528694in}}%
\pgfpathlineto{\pgfqpoint{1.236437in}{1.577978in}}%
\pgfpathlineto{\pgfqpoint{1.239857in}{1.539135in}}%
\pgfpathlineto{\pgfqpoint{1.245845in}{1.615072in}}%
\pgfpathlineto{\pgfqpoint{1.247557in}{1.566080in}}%
\pgfpathlineto{\pgfqpoint{1.251837in}{1.608540in}}%
\pgfpathlineto{\pgfqpoint{1.256112in}{1.540943in}}%
\pgfpathlineto{\pgfqpoint{1.262954in}{1.599323in}}%
\pgfpathlineto{\pgfqpoint{1.264666in}{1.522163in}}%
\pgfpathlineto{\pgfqpoint{1.269800in}{1.730783in}}%
\pgfpathlineto{\pgfqpoint{1.274080in}{1.587542in}}%
\pgfpathlineto{\pgfqpoint{1.277502in}{1.548874in}}%
\pgfpathlineto{\pgfqpoint{1.284344in}{1.739766in}}%
\pgfpathlineto{\pgfqpoint{1.286057in}{1.610407in}}%
\pgfpathlineto{\pgfqpoint{1.289480in}{1.562058in}}%
\pgfpathlineto{\pgfqpoint{1.297179in}{1.694070in}}%
\pgfpathlineto{\pgfqpoint{1.298889in}{1.618922in}}%
\pgfpathlineto{\pgfqpoint{1.302314in}{1.726060in}}%
\pgfpathlineto{\pgfqpoint{1.309157in}{1.629070in}}%
\pgfpathlineto{\pgfqpoint{1.312579in}{1.717369in}}%
\pgfpathlineto{\pgfqpoint{1.317712in}{1.618279in}}%
\pgfpathlineto{\pgfqpoint{1.321985in}{1.715678in}}%
\pgfpathlineto{\pgfqpoint{1.324548in}{1.687158in}}%
\pgfpathlineto{\pgfqpoint{1.331388in}{1.692408in}}%
\pgfpathlineto{\pgfqpoint{1.333951in}{1.537210in}}%
\pgfpathlineto{\pgfqpoint{1.339938in}{1.619795in}}%
\pgfpathlineto{\pgfqpoint{1.343354in}{1.535577in}}%
\pgfpathlineto{\pgfqpoint{1.345921in}{1.609588in}}%
\pgfpathlineto{\pgfqpoint{1.350199in}{1.549575in}}%
\pgfpathlineto{\pgfqpoint{1.356186in}{1.629651in}}%
\pgfpathlineto{\pgfqpoint{1.358754in}{1.557331in}}%
\pgfpathlineto{\pgfqpoint{1.365593in}{1.632976in}}%
\pgfpathlineto{\pgfqpoint{1.369869in}{1.502100in}}%
\pgfpathlineto{\pgfqpoint{1.371579in}{1.568239in}}%
\pgfpathlineto{\pgfqpoint{1.378419in}{1.617987in}}%
\pgfpathlineto{\pgfqpoint{1.382696in}{1.510440in}}%
\pgfpathlineto{\pgfqpoint{1.383549in}{1.581361in}}%
\pgfpathlineto{\pgfqpoint{1.387825in}{1.689142in}}%
\pgfpathlineto{\pgfqpoint{1.392960in}{1.650415in}}%
\pgfpathlineto{\pgfqpoint{1.396386in}{1.710195in}}%
\pgfpathlineto{\pgfqpoint{1.400665in}{1.665404in}}%
\pgfpathlineto{\pgfqpoint{1.404941in}{1.722239in}}%
\pgfpathlineto{\pgfqpoint{1.410071in}{1.632044in}}%
\pgfpathlineto{\pgfqpoint{1.416052in}{1.616588in}}%
\pgfpathlineto{\pgfqpoint{1.417764in}{1.747931in}}%
\pgfpathlineto{\pgfqpoint{1.424607in}{1.633443in}}%
\pgfpathlineto{\pgfqpoint{1.426320in}{1.661758in}}%
\pgfpathlineto{\pgfqpoint{1.433162in}{1.599995in}}%
\pgfpathlineto{\pgfqpoint{1.435727in}{1.686314in}}%
\pgfpathlineto{\pgfqpoint{1.439148in}{1.617465in}}%
\pgfpathlineto{\pgfqpoint{1.445989in}{1.705124in}}%
\pgfpathlineto{\pgfqpoint{1.447700in}{1.624084in}}%
\pgfpathlineto{\pgfqpoint{1.454547in}{1.676575in}}%
\pgfpathlineto{\pgfqpoint{1.457115in}{1.610670in}}%
\pgfpathlineto{\pgfqpoint{1.460537in}{1.665057in}}%
\pgfpathlineto{\pgfqpoint{1.464817in}{1.622714in}}%
\pgfpathlineto{\pgfqpoint{1.469955in}{1.730261in}}%
\pgfpathlineto{\pgfqpoint{1.473375in}{1.618922in}}%
\pgfpathlineto{\pgfqpoint{1.478513in}{1.520822in}}%
\pgfpathlineto{\pgfqpoint{1.485359in}{1.527851in}}%
\pgfpathlineto{\pgfqpoint{1.489636in}{1.629567in}}%
\pgfpathlineto{\pgfqpoint{1.490491in}{1.509567in}}%
\pgfpathlineto{\pgfqpoint{1.498186in}{1.649192in}}%
\pgfpathlineto{\pgfqpoint{1.499039in}{1.531526in}}%
\pgfpathlineto{\pgfqpoint{1.503315in}{1.663424in}}%
\pgfpathlineto{\pgfqpoint{1.509305in}{1.524380in}}%
\pgfpathlineto{\pgfqpoint{1.515290in}{1.629070in}}%
\pgfpathlineto{\pgfqpoint{1.516145in}{1.549634in}}%
\pgfpathlineto{\pgfqpoint{1.521272in}{1.679110in}}%
\pgfpathlineto{\pgfqpoint{1.527256in}{1.500350in}}%
\pgfpathlineto{\pgfqpoint{1.528967in}{1.616705in}}%
\pgfpathlineto{\pgfqpoint{1.535810in}{1.601307in}}%
\pgfpathlineto{\pgfqpoint{1.539230in}{1.533652in}}%
\pgfpathlineto{\pgfqpoint{1.545218in}{1.634901in}}%
\pgfpathlineto{\pgfqpoint{1.546074in}{1.545754in}}%
\pgfpathlineto{\pgfqpoint{1.552920in}{1.627901in}}%
\pgfpathlineto{\pgfqpoint{1.554633in}{1.535577in}}%
\pgfpathlineto{\pgfqpoint{1.561475in}{1.633209in}}%
\pgfpathlineto{\pgfqpoint{1.563184in}{1.527178in}}%
\pgfpathlineto{\pgfqpoint{1.568314in}{1.665462in}}%
\pgfpathlineto{\pgfqpoint{1.573455in}{1.552841in}}%
\pgfpathlineto{\pgfqpoint{1.577737in}{1.621662in}}%
\pgfpathlineto{\pgfqpoint{1.582014in}{1.532019in}}%
\pgfpathlineto{\pgfqpoint{1.587148in}{1.626444in}}%
\pgfpathlineto{\pgfqpoint{1.592280in}{1.516446in}}%
\pgfpathlineto{\pgfqpoint{1.596559in}{1.616354in}}%
\pgfpathlineto{\pgfqpoint{1.598270in}{1.508310in}}%
\pgfpathlineto{\pgfqpoint{1.605969in}{1.651347in}}%
\pgfpathlineto{\pgfqpoint{1.611957in}{1.534817in}}%
\pgfpathlineto{\pgfqpoint{1.617087in}{1.646419in}}%
\pgfpathlineto{\pgfqpoint{1.620507in}{1.579611in}}%
\pgfpathlineto{\pgfqpoint{1.623070in}{1.604748in}}%
\pgfpathlineto{\pgfqpoint{1.630773in}{1.553046in}}%
\pgfpathlineto{\pgfqpoint{1.635054in}{1.646273in}}%
\pgfpathlineto{\pgfqpoint{1.635909in}{1.561298in}}%
\pgfpathlineto{\pgfqpoint{1.643599in}{1.653155in}}%
\pgfpathlineto{\pgfqpoint{1.645308in}{1.567684in}}%
\pgfpathlineto{\pgfqpoint{1.649584in}{1.596320in}}%
\pgfpathlineto{\pgfqpoint{1.655572in}{1.521637in}}%
\pgfpathlineto{\pgfqpoint{1.660704in}{1.610056in}}%
\pgfpathlineto{\pgfqpoint{1.663267in}{1.560422in}}%
\pgfpathlineto{\pgfqpoint{1.671820in}{1.683892in}}%
\pgfpathlineto{\pgfqpoint{1.674387in}{1.534119in}}%
\pgfpathlineto{\pgfqpoint{1.680376in}{1.484572in}}%
\pgfpathlineto{\pgfqpoint{1.683795in}{1.622535in}}%
\pgfpathlineto{\pgfqpoint{1.692344in}{1.489793in}}%
\pgfpathlineto{\pgfqpoint{1.695764in}{1.611280in}}%
\pgfpathlineto{\pgfqpoint{1.701749in}{1.772136in}}%
\pgfpathlineto{\pgfqpoint{1.705169in}{1.640559in}}%
\pgfpathlineto{\pgfqpoint{1.712012in}{1.694274in}}%
\pgfpathlineto{\pgfqpoint{1.716288in}{1.611514in}}%
\pgfpathlineto{\pgfqpoint{1.717143in}{1.675260in}}%
\pgfpathlineto{\pgfqpoint{1.722276in}{1.715097in}}%
\pgfpathlineto{\pgfqpoint{1.725698in}{1.600551in}}%
\pgfpathlineto{\pgfqpoint{1.730826in}{1.519131in}}%
\pgfpathlineto{\pgfqpoint{1.735959in}{1.623646in}}%
\pgfpathlineto{\pgfqpoint{1.738528in}{1.568064in}}%
\pgfpathlineto{\pgfqpoint{1.742805in}{1.624811in}}%
\pgfpathlineto{\pgfqpoint{1.747088in}{1.558964in}}%
\pgfpathlineto{\pgfqpoint{1.751369in}{1.518838in}}%
\pgfpathlineto{\pgfqpoint{1.756502in}{1.610290in}}%
\pgfpathlineto{\pgfqpoint{1.759920in}{1.539252in}}%
\pgfpathlineto{\pgfqpoint{1.765902in}{1.614663in}}%
\pgfpathlineto{\pgfqpoint{1.771888in}{1.532545in}}%
\pgfpathlineto{\pgfqpoint{1.776162in}{1.620672in}}%
\pgfpathlineto{\pgfqpoint{1.777874in}{1.527003in}}%
\pgfpathlineto{\pgfqpoint{1.783002in}{1.591276in}}%
\pgfpathlineto{\pgfqpoint{1.786423in}{1.524961in}}%
\pgfpathlineto{\pgfqpoint{1.789847in}{1.633092in}}%
\pgfpathlineto{\pgfqpoint{1.794127in}{1.538843in}}%
\pgfpathlineto{\pgfqpoint{1.800974in}{1.519014in}}%
\pgfpathlineto{\pgfqpoint{1.806101in}{1.614604in}}%
\pgfpathlineto{\pgfqpoint{1.807814in}{1.513764in}}%
\pgfpathlineto{\pgfqpoint{1.813805in}{1.675202in}}%
\pgfpathlineto{\pgfqpoint{1.817226in}{1.562697in}}%
\pgfpathlineto{\pgfqpoint{1.822358in}{1.686636in}}%
\pgfpathlineto{\pgfqpoint{1.824068in}{1.612712in}}%
\pgfpathlineto{\pgfqpoint{1.831766in}{1.648782in}}%
\pgfpathlineto{\pgfqpoint{1.833476in}{1.558179in}}%
\pgfpathlineto{\pgfqpoint{1.836895in}{1.640267in}}%
\pgfpathlineto{\pgfqpoint{1.842883in}{1.545725in}}%
\pgfpathlineto{\pgfqpoint{1.848875in}{1.685350in}}%
\pgfpathlineto{\pgfqpoint{1.849731in}{1.567567in}}%
\pgfpathlineto{\pgfqpoint{1.854859in}{1.619445in}}%
\pgfpathlineto{\pgfqpoint{1.859990in}{1.529162in}}%
\pgfpathlineto{\pgfqpoint{1.864266in}{1.490436in}}%
\pgfpathlineto{\pgfqpoint{1.868544in}{1.628194in}}%
\pgfpathlineto{\pgfqpoint{1.872817in}{1.640384in}}%
\pgfpathlineto{\pgfqpoint{1.877090in}{1.539486in}}%
\pgfpathlineto{\pgfqpoint{1.879659in}{1.594366in}}%
\pgfpathlineto{\pgfqpoint{1.883937in}{1.544268in}}%
\pgfpathlineto{\pgfqpoint{1.888217in}{1.594717in}}%
\pgfpathlineto{\pgfqpoint{1.895915in}{1.510148in}}%
\pgfpathlineto{\pgfqpoint{1.896771in}{1.590340in}}%
\pgfpathlineto{\pgfqpoint{1.901907in}{1.605913in}}%
\pgfpathlineto{\pgfqpoint{1.907894in}{1.530269in}}%
\pgfpathlineto{\pgfqpoint{1.912175in}{1.508573in}}%
\pgfpathlineto{\pgfqpoint{1.913882in}{1.607605in}}%
\pgfpathlineto{\pgfqpoint{1.918162in}{1.528548in}}%
\pgfpathlineto{\pgfqpoint{1.922443in}{1.636767in}}%
\pgfpathlineto{\pgfqpoint{1.926721in}{1.526072in}}%
\pgfpathlineto{\pgfqpoint{1.932706in}{1.500993in}}%
\pgfpathlineto{\pgfqpoint{1.935269in}{1.617581in}}%
\pgfpathlineto{\pgfqpoint{1.940403in}{1.716613in}}%
\pgfpathlineto{\pgfqpoint{1.947248in}{1.607024in}}%
\pgfpathlineto{\pgfqpoint{1.950671in}{1.719236in}}%
\pgfpathlineto{\pgfqpoint{1.955805in}{1.564184in}}%
\pgfpathlineto{\pgfqpoint{1.960085in}{1.546368in}}%
\pgfpathlineto{\pgfqpoint{1.961796in}{1.620321in}}%
\pgfpathlineto{\pgfqpoint{1.967788in}{1.663654in}}%
\pgfpathlineto{\pgfqpoint{1.970354in}{1.526188in}}%
\pgfpathlineto{\pgfqpoint{1.976334in}{1.645224in}}%
\pgfpathlineto{\pgfqpoint{1.979754in}{1.525023in}}%
\pgfpathlineto{\pgfqpoint{1.985740in}{1.629655in}}%
\pgfpathlineto{\pgfqpoint{1.986595in}{1.553777in}}%
\pgfpathlineto{\pgfqpoint{1.993441in}{1.537707in}}%
\pgfpathlineto{\pgfqpoint{1.995151in}{1.647500in}}%
\pgfpathlineto{\pgfqpoint{2.001137in}{1.518082in}}%
\pgfpathlineto{\pgfqpoint{2.004557in}{1.605099in}}%
\pgfpathlineto{\pgfqpoint{2.010548in}{1.566723in}}%
\pgfpathlineto{\pgfqpoint{2.014825in}{1.650708in}}%
\pgfpathlineto{\pgfqpoint{2.016537in}{1.566489in}}%
\pgfpathlineto{\pgfqpoint{2.021667in}{1.650445in}}%
\pgfpathlineto{\pgfqpoint{2.027657in}{1.671877in}}%
\pgfpathlineto{\pgfqpoint{2.031077in}{1.543537in}}%
\pgfpathlineto{\pgfqpoint{2.035355in}{1.654847in}}%
\pgfpathlineto{\pgfqpoint{2.039628in}{1.555581in}}%
\pgfpathlineto{\pgfqpoint{2.043906in}{1.635661in}}%
\pgfpathlineto{\pgfqpoint{2.047328in}{1.531058in}}%
\pgfpathlineto{\pgfqpoint{2.054173in}{1.525662in}}%
\pgfpathlineto{\pgfqpoint{2.055027in}{1.633735in}}%
\pgfpathlineto{\pgfqpoint{2.060159in}{1.507466in}}%
\pgfpathlineto{\pgfqpoint{2.065290in}{1.621837in}}%
\pgfpathlineto{\pgfqpoint{2.067854in}{1.578504in}}%
\pgfpathlineto{\pgfqpoint{2.075551in}{1.509918in}}%
\pgfpathlineto{\pgfqpoint{2.077262in}{1.664239in}}%
\pgfpathlineto{\pgfqpoint{2.081538in}{1.578095in}}%
\pgfpathlineto{\pgfqpoint{2.084956in}{1.607956in}}%
\pgfpathlineto{\pgfqpoint{2.090086in}{1.524029in}}%
\pgfpathlineto{\pgfqpoint{2.093509in}{1.598684in}}%
\pgfpathlineto{\pgfqpoint{2.097788in}{1.519131in}}%
\pgfpathlineto{\pgfqpoint{2.102927in}{1.624402in}}%
\pgfpathlineto{\pgfqpoint{2.106351in}{1.542109in}}%
\pgfpathlineto{\pgfqpoint{2.114054in}{1.512248in}}%
\pgfpathlineto{\pgfqpoint{2.114910in}{1.620727in}}%
\pgfpathlineto{\pgfqpoint{2.122605in}{1.663946in}}%
\pgfpathlineto{\pgfqpoint{2.124318in}{1.588243in}}%
\pgfpathlineto{\pgfqpoint{2.130310in}{1.557740in}}%
\pgfpathlineto{\pgfqpoint{2.132878in}{1.648373in}}%
\pgfpathlineto{\pgfqpoint{2.138864in}{1.515719in}}%
\pgfpathlineto{\pgfqpoint{2.140576in}{1.599031in}}%
\pgfpathlineto{\pgfqpoint{2.145701in}{1.552257in}}%
\pgfpathlineto{\pgfqpoint{2.151687in}{1.762452in}}%
\pgfpathlineto{\pgfqpoint{2.153397in}{1.633823in}}%
\pgfpathlineto{\pgfqpoint{2.158523in}{1.725709in}}%
\pgfpathlineto{\pgfqpoint{2.164513in}{1.642075in}}%
\pgfpathlineto{\pgfqpoint{2.169647in}{1.646945in}}%
\pgfpathlineto{\pgfqpoint{2.173068in}{1.753297in}}%
\pgfpathlineto{\pgfqpoint{2.176493in}{1.561587in}}%
\pgfpathlineto{\pgfqpoint{2.181627in}{1.683542in}}%
\pgfpathlineto{\pgfqpoint{2.184193in}{1.617812in}}%
\pgfpathlineto{\pgfqpoint{2.189327in}{1.690833in}}%
\pgfpathlineto{\pgfqpoint{2.193602in}{1.635368in}}%
\pgfpathlineto{\pgfqpoint{2.197026in}{1.685642in}}%
\pgfpathlineto{\pgfqpoint{2.203016in}{1.582526in}}%
\pgfpathlineto{\pgfqpoint{2.205581in}{1.681675in}}%
\pgfpathlineto{\pgfqpoint{2.212427in}{1.697190in}}%
\pgfpathlineto{\pgfqpoint{2.215846in}{1.582468in}}%
\pgfpathlineto{\pgfqpoint{2.218411in}{1.701040in}}%
\pgfpathlineto{\pgfqpoint{2.222686in}{1.598976in}}%
\pgfpathlineto{\pgfqpoint{2.226962in}{1.709146in}}%
\pgfpathlineto{\pgfqpoint{2.232095in}{1.635719in}}%
\pgfpathlineto{\pgfqpoint{2.234663in}{1.700806in}}%
\pgfpathlineto{\pgfqpoint{2.241504in}{1.618279in}}%
\pgfpathlineto{\pgfqpoint{2.243215in}{1.705413in}}%
\pgfpathlineto{\pgfqpoint{2.250053in}{1.649539in}}%
\pgfpathlineto{\pgfqpoint{2.255182in}{1.735741in}}%
\pgfpathlineto{\pgfqpoint{2.258601in}{1.748629in}}%
\pgfpathlineto{\pgfqpoint{2.261169in}{1.634813in}}%
\pgfpathlineto{\pgfqpoint{2.264591in}{1.668670in}}%
\pgfpathlineto{\pgfqpoint{2.268867in}{1.510557in}}%
\pgfpathlineto{\pgfqpoint{2.273147in}{1.530795in}}%
\pgfpathlineto{\pgfqpoint{2.278277in}{1.613669in}}%
\pgfpathlineto{\pgfqpoint{2.282560in}{1.540300in}}%
\pgfpathlineto{\pgfqpoint{2.286838in}{1.601015in}}%
\pgfpathlineto{\pgfqpoint{2.290262in}{1.534236in}}%
\pgfpathlineto{\pgfqpoint{2.295399in}{1.639507in}}%
\pgfpathlineto{\pgfqpoint{2.298819in}{1.553919in}}%
\pgfpathlineto{\pgfqpoint{2.309087in}{1.658580in}}%
\pgfpathlineto{\pgfqpoint{2.311649in}{1.592211in}}%
\pgfpathlineto{\pgfqpoint{2.318493in}{1.627960in}}%
\pgfpathlineto{\pgfqpoint{2.321912in}{1.493351in}}%
\pgfpathlineto{\pgfqpoint{2.326189in}{1.578095in}}%
\pgfpathlineto{\pgfqpoint{2.330467in}{1.587074in}}%
\pgfpathlineto{\pgfqpoint{2.333891in}{1.516796in}}%
\pgfpathlineto{\pgfqpoint{2.339879in}{1.502392in}}%
\pgfpathlineto{\pgfqpoint{2.341587in}{1.622798in}}%
\pgfpathlineto{\pgfqpoint{2.348427in}{1.558379in}}%
\pgfpathlineto{\pgfqpoint{2.350137in}{1.639273in}}%
\pgfpathlineto{\pgfqpoint{2.356126in}{1.589409in}}%
\pgfpathlineto{\pgfqpoint{2.363830in}{1.541586in}}%
\pgfpathlineto{\pgfqpoint{2.370673in}{1.723905in}}%
\pgfpathlineto{\pgfqpoint{2.374095in}{1.595330in}}%
\pgfpathlineto{\pgfqpoint{2.375807in}{1.664122in}}%
\pgfpathlineto{\pgfqpoint{2.383504in}{1.680363in}}%
\pgfpathlineto{\pgfqpoint{2.386924in}{1.621720in}}%
\pgfpathlineto{\pgfqpoint{2.392057in}{1.637235in}}%
\pgfpathlineto{\pgfqpoint{2.396335in}{1.513180in}}%
\pgfpathlineto{\pgfqpoint{2.397190in}{1.634375in}}%
\pgfpathlineto{\pgfqpoint{2.402321in}{1.647208in}}%
\pgfpathlineto{\pgfqpoint{2.407451in}{1.534324in}}%
\pgfpathlineto{\pgfqpoint{2.411726in}{1.522250in}}%
\pgfpathlineto{\pgfqpoint{2.417709in}{1.508807in}}%
\pgfpathlineto{\pgfqpoint{2.418564in}{1.608131in}}%
\pgfpathlineto{\pgfqpoint{2.423696in}{1.525370in}}%
\pgfpathlineto{\pgfqpoint{2.427115in}{1.595765in}}%
\pgfpathlineto{\pgfqpoint{2.433100in}{1.582526in}}%
\pgfpathlineto{\pgfqpoint{2.436521in}{1.643708in}}%
\pgfpathlineto{\pgfqpoint{2.441651in}{1.684477in}}%
\pgfpathlineto{\pgfqpoint{2.445075in}{1.554387in}}%
\pgfpathlineto{\pgfqpoint{2.451058in}{1.546017in}}%
\pgfpathlineto{\pgfqpoint{2.452770in}{1.618104in}}%
\pgfpathlineto{\pgfqpoint{2.457908in}{1.645107in}}%
\pgfpathlineto{\pgfqpoint{2.463895in}{1.600142in}}%
\pgfpathlineto{\pgfqpoint{2.465606in}{1.668670in}}%
\pgfpathlineto{\pgfqpoint{2.470738in}{1.560279in}}%
\pgfpathlineto{\pgfqpoint{2.474162in}{1.643708in}}%
\pgfpathlineto{\pgfqpoint{2.479295in}{1.568005in}}%
\pgfpathlineto{\pgfqpoint{2.486139in}{1.598626in}}%
\pgfpathlineto{\pgfqpoint{2.489559in}{1.507729in}}%
\pgfpathlineto{\pgfqpoint{2.494692in}{1.613906in}}%
\pgfpathlineto{\pgfqpoint{2.496403in}{1.504522in}}%
\pgfpathlineto{\pgfqpoint{2.502393in}{1.653564in}}%
\pgfpathlineto{\pgfqpoint{2.506673in}{1.647968in}}%
\pgfpathlineto{\pgfqpoint{2.508382in}{1.546485in}}%
\pgfpathlineto{\pgfqpoint{2.516081in}{1.623762in}}%
\pgfpathlineto{\pgfqpoint{2.518648in}{1.536337in}}%
\pgfpathlineto{\pgfqpoint{2.522070in}{1.603641in}}%
\pgfpathlineto{\pgfqpoint{2.528054in}{1.514583in}}%
\pgfpathlineto{\pgfqpoint{2.530622in}{1.584514in}}%
\pgfpathlineto{\pgfqpoint{2.535758in}{1.638810in}}%
\pgfpathlineto{\pgfqpoint{2.538324in}{1.531730in}}%
\pgfpathlineto{\pgfqpoint{2.543453in}{1.653798in}}%
\pgfpathlineto{\pgfqpoint{2.549441in}{1.526013in}}%
\pgfpathlineto{\pgfqpoint{2.553715in}{1.628018in}}%
\pgfpathlineto{\pgfqpoint{2.556277in}{1.547709in}}%
\pgfpathlineto{\pgfqpoint{2.563119in}{1.626795in}}%
\pgfpathlineto{\pgfqpoint{2.565684in}{1.517030in}}%
\pgfpathlineto{\pgfqpoint{2.569960in}{1.681269in}}%
\pgfpathlineto{\pgfqpoint{2.572527in}{1.615598in}}%
\pgfpathlineto{\pgfqpoint{2.579370in}{1.703667in}}%
\pgfpathlineto{\pgfqpoint{2.581931in}{1.607550in}}%
\pgfpathlineto{\pgfqpoint{2.587061in}{1.688152in}}%
\pgfpathlineto{\pgfqpoint{2.593048in}{1.638897in}}%
\pgfpathlineto{\pgfqpoint{2.595611in}{1.703082in}}%
\pgfpathlineto{\pgfqpoint{2.601596in}{1.594603in}}%
\pgfpathlineto{\pgfqpoint{2.604164in}{1.674913in}}%
\pgfpathlineto{\pgfqpoint{2.609301in}{1.617728in}}%
\pgfpathlineto{\pgfqpoint{2.614436in}{1.710954in}}%
\pgfpathlineto{\pgfqpoint{2.617000in}{1.622597in}}%
\pgfpathlineto{\pgfqpoint{2.619567in}{1.712938in}}%
\pgfpathlineto{\pgfqpoint{2.624695in}{1.682552in}}%
\pgfpathlineto{\pgfqpoint{2.628975in}{1.753473in}}%
\pgfpathlineto{\pgfqpoint{2.632399in}{1.667388in}}%
\pgfpathlineto{\pgfqpoint{2.639241in}{1.693924in}}%
\pgfpathlineto{\pgfqpoint{2.641809in}{1.618279in}}%
\pgfpathlineto{\pgfqpoint{2.646085in}{1.544034in}}%
\pgfpathlineto{\pgfqpoint{2.649509in}{1.532662in}}%
\pgfpathlineto{\pgfqpoint{2.653787in}{1.658814in}}%
\pgfpathlineto{\pgfqpoint{2.658066in}{1.709263in}}%
\pgfpathlineto{\pgfqpoint{2.662344in}{1.603232in}}%
\pgfpathlineto{\pgfqpoint{2.670044in}{1.563048in}}%
\pgfpathlineto{\pgfqpoint{2.672610in}{1.632102in}}%
\pgfpathlineto{\pgfqpoint{2.677741in}{1.523039in}}%
\pgfpathlineto{\pgfqpoint{2.681163in}{1.635427in}}%
\pgfpathlineto{\pgfqpoint{2.684586in}{1.527967in}}%
\pgfpathlineto{\pgfqpoint{2.693143in}{1.714337in}}%
\pgfpathlineto{\pgfqpoint{2.696565in}{1.561415in}}%
\pgfpathlineto{\pgfqpoint{2.702555in}{1.600609in}}%
\pgfpathlineto{\pgfqpoint{2.708544in}{1.530912in}}%
\pgfpathlineto{\pgfqpoint{2.709399in}{1.621720in}}%
\pgfpathlineto{\pgfqpoint{2.714525in}{1.535811in}}%
\pgfpathlineto{\pgfqpoint{2.721365in}{1.653740in}}%
\pgfpathlineto{\pgfqpoint{2.723934in}{1.549634in}}%
\pgfpathlineto{\pgfqpoint{2.726501in}{1.616997in}}%
\pgfpathlineto{\pgfqpoint{2.731635in}{1.622159in}}%
\pgfpathlineto{\pgfqpoint{2.737626in}{1.495919in}}%
\pgfpathlineto{\pgfqpoint{2.739338in}{1.585968in}}%
\pgfpathlineto{\pgfqpoint{2.745323in}{1.525019in}}%
\pgfpathlineto{\pgfqpoint{2.747889in}{1.617578in}}%
\pgfpathlineto{\pgfqpoint{2.755581in}{1.547738in}}%
\pgfpathlineto{\pgfqpoint{2.756436in}{1.615832in}}%
\pgfpathlineto{\pgfqpoint{2.762417in}{1.532749in}}%
\pgfpathlineto{\pgfqpoint{2.764980in}{1.610465in}}%
\pgfpathlineto{\pgfqpoint{2.770971in}{1.523098in}}%
\pgfpathlineto{\pgfqpoint{2.776103in}{1.612098in}}%
\pgfpathlineto{\pgfqpoint{2.777811in}{1.565266in}}%
\pgfpathlineto{\pgfqpoint{2.782945in}{1.630703in}}%
\pgfpathlineto{\pgfqpoint{2.789789in}{1.558149in}}%
\pgfpathlineto{\pgfqpoint{2.790642in}{1.599703in}}%
\pgfpathlineto{\pgfqpoint{2.794918in}{1.536684in}}%
\pgfpathlineto{\pgfqpoint{2.799189in}{1.635076in}}%
\pgfpathlineto{\pgfqpoint{2.803464in}{1.564097in}}%
\pgfpathlineto{\pgfqpoint{2.809450in}{1.615072in}}%
\pgfpathlineto{\pgfqpoint{2.812873in}{1.669839in}}%
\pgfpathlineto{\pgfqpoint{2.817150in}{1.508398in}}%
\pgfpathlineto{\pgfqpoint{2.820569in}{1.624928in}}%
\pgfpathlineto{\pgfqpoint{2.827409in}{1.610465in}}%
\pgfpathlineto{\pgfqpoint{2.829973in}{1.686168in}}%
\pgfpathlineto{\pgfqpoint{2.833396in}{1.617523in}}%
\pgfpathlineto{\pgfqpoint{2.839382in}{1.710081in}}%
\pgfpathlineto{\pgfqpoint{2.844515in}{1.647880in}}%
\pgfpathlineto{\pgfqpoint{2.849642in}{1.707630in}}%
\pgfpathlineto{\pgfqpoint{2.850497in}{1.536278in}}%
\pgfpathlineto{\pgfqpoint{2.855627in}{1.637060in}}%
\pgfpathlineto{\pgfqpoint{2.859900in}{1.463140in}}%
\pgfpathlineto{\pgfqpoint{2.865887in}{1.641348in}}%
\pgfpathlineto{\pgfqpoint{2.867599in}{1.521348in}}%
\pgfpathlineto{\pgfqpoint{2.874440in}{1.510440in}}%
\pgfpathlineto{\pgfqpoint{2.876150in}{1.688674in}}%
\pgfpathlineto{\pgfqpoint{2.880425in}{1.530561in}}%
\pgfpathlineto{\pgfqpoint{2.885558in}{1.672754in}}%
\pgfpathlineto{\pgfqpoint{2.889835in}{1.530269in}}%
\pgfpathlineto{\pgfqpoint{2.893259in}{1.609296in}}%
\pgfpathlineto{\pgfqpoint{2.900103in}{1.564798in}}%
\pgfpathlineto{\pgfqpoint{2.905233in}{1.724076in}}%
\pgfpathlineto{\pgfqpoint{2.907800in}{1.665579in}}%
\pgfpathlineto{\pgfqpoint{2.911219in}{1.714396in}}%
\pgfpathlineto{\pgfqpoint{2.918060in}{1.636943in}}%
\pgfpathlineto{\pgfqpoint{2.921481in}{1.690428in}}%
\pgfpathlineto{\pgfqpoint{2.924043in}{1.543336in}}%
\pgfpathlineto{\pgfqpoint{2.930027in}{1.644760in}}%
\pgfpathlineto{\pgfqpoint{2.932595in}{1.528289in}}%
\pgfpathlineto{\pgfqpoint{2.936017in}{1.621140in}}%
\pgfpathlineto{\pgfqpoint{2.941150in}{1.534558in}}%
\pgfpathlineto{\pgfqpoint{2.944568in}{1.630937in}}%
\pgfpathlineto{\pgfqpoint{2.948846in}{1.563282in}}%
\pgfpathlineto{\pgfqpoint{2.954831in}{1.526130in}}%
\pgfpathlineto{\pgfqpoint{2.963385in}{1.667388in}}%
\pgfpathlineto{\pgfqpoint{2.965950in}{1.608715in}}%
\pgfpathlineto{\pgfqpoint{2.972797in}{1.584978in}}%
\pgfpathlineto{\pgfqpoint{2.977928in}{1.638108in}}%
\pgfpathlineto{\pgfqpoint{2.979640in}{1.572846in}}%
\pgfpathlineto{\pgfqpoint{2.983917in}{1.681328in}}%
\pgfpathlineto{\pgfqpoint{2.989050in}{1.635953in}}%
\pgfpathlineto{\pgfqpoint{2.991615in}{1.694099in}}%
\pgfpathlineto{\pgfqpoint{2.997600in}{1.638722in}}%
\pgfpathlineto{\pgfqpoint{3.000167in}{1.715649in}}%
\pgfpathlineto{\pgfqpoint{3.004446in}{1.605099in}}%
\pgfpathlineto{\pgfqpoint{3.008722in}{1.759888in}}%
\pgfpathlineto{\pgfqpoint{3.012997in}{1.605972in}}%
\pgfpathlineto{\pgfqpoint{3.018130in}{1.702848in}}%
\pgfpathlineto{\pgfqpoint{3.024118in}{1.633151in}}%
\pgfpathlineto{\pgfqpoint{3.026684in}{1.713928in}}%
\pgfpathlineto{\pgfqpoint{3.030958in}{1.753820in}}%
\pgfpathlineto{\pgfqpoint{3.034379in}{1.635310in}}%
\pgfpathlineto{\pgfqpoint{3.041219in}{1.621837in}}%
\pgfpathlineto{\pgfqpoint{3.046352in}{1.752132in}}%
\pgfpathlineto{\pgfqpoint{3.048063in}{1.665872in}}%
\pgfpathlineto{\pgfqpoint{3.054053in}{1.716145in}}%
\pgfpathlineto{\pgfqpoint{3.058329in}{1.584919in}}%
\pgfpathlineto{\pgfqpoint{3.062609in}{1.538667in}}%
\pgfpathlineto{\pgfqpoint{3.064319in}{1.661554in}}%
\pgfpathlineto{\pgfqpoint{3.069452in}{1.569404in}}%
\pgfpathlineto{\pgfqpoint{3.076293in}{1.663654in}}%
\pgfpathlineto{\pgfqpoint{3.078004in}{1.572963in}}%
\pgfpathlineto{\pgfqpoint{3.083138in}{1.587893in}}%
\pgfpathlineto{\pgfqpoint{3.089122in}{1.509943in}}%
\pgfpathlineto{\pgfqpoint{3.092546in}{1.618396in}}%
\pgfpathlineto{\pgfqpoint{3.095968in}{1.544677in}}%
\pgfpathlineto{\pgfqpoint{3.101097in}{1.608862in}}%
\pgfpathlineto{\pgfqpoint{3.103662in}{1.521377in}}%
\pgfpathlineto{\pgfqpoint{3.109653in}{1.635602in}}%
\pgfpathlineto{\pgfqpoint{3.113076in}{1.541440in}}%
\pgfpathlineto{\pgfqpoint{3.118215in}{1.754755in}}%
\pgfpathlineto{\pgfqpoint{3.122495in}{1.625366in}}%
\pgfpathlineto{\pgfqpoint{3.124207in}{1.698125in}}%
\pgfpathlineto{\pgfqpoint{3.129335in}{1.650912in}}%
\pgfpathlineto{\pgfqpoint{3.135323in}{1.791556in}}%
\pgfpathlineto{\pgfqpoint{3.138744in}{1.664502in}}%
\pgfpathlineto{\pgfqpoint{3.143874in}{1.709263in}}%
\pgfpathlineto{\pgfqpoint{3.145581in}{1.587659in}}%
\pgfpathlineto{\pgfqpoint{3.149859in}{1.566022in}}%
\pgfpathlineto{\pgfqpoint{3.155843in}{1.541528in}}%
\pgfpathlineto{\pgfqpoint{3.159266in}{1.646624in}}%
\pgfpathlineto{\pgfqpoint{3.164395in}{1.533885in}}%
\pgfpathlineto{\pgfqpoint{3.168665in}{1.643825in}}%
\pgfpathlineto{\pgfqpoint{3.172941in}{1.553831in}}%
\pgfpathlineto{\pgfqpoint{3.178926in}{1.675815in}}%
\pgfpathlineto{\pgfqpoint{3.179782in}{1.551092in}}%
\pgfpathlineto{\pgfqpoint{3.186627in}{1.532194in}}%
\pgfpathlineto{\pgfqpoint{3.189194in}{1.619795in}}%
\pgfpathlineto{\pgfqpoint{3.196038in}{1.499766in}}%
\pgfpathlineto{\pgfqpoint{3.200317in}{1.646331in}}%
\pgfpathlineto{\pgfqpoint{3.204595in}{1.519043in}}%
\pgfpathlineto{\pgfqpoint{3.206306in}{1.654847in}}%
\pgfpathlineto{\pgfqpoint{3.209727in}{1.521929in}}%
\pgfpathlineto{\pgfqpoint{3.214860in}{1.640559in}}%
\pgfpathlineto{\pgfqpoint{3.218280in}{1.652980in}}%
\pgfpathlineto{\pgfqpoint{3.225977in}{1.532223in}}%
\pgfpathlineto{\pgfqpoint{3.230256in}{1.690249in}}%
\pgfpathlineto{\pgfqpoint{3.232821in}{1.584539in}}%
\pgfpathlineto{\pgfqpoint{3.235387in}{1.707513in}}%
\pgfpathlineto{\pgfqpoint{3.239662in}{1.655285in}}%
\pgfpathlineto{\pgfqpoint{3.247355in}{1.719236in}}%
\pgfpathlineto{\pgfqpoint{3.251630in}{1.647003in}}%
\pgfpathlineto{\pgfqpoint{3.255054in}{1.691882in}}%
\pgfpathlineto{\pgfqpoint{3.259333in}{1.701913in}}%
\pgfpathlineto{\pgfqpoint{3.264463in}{1.735218in}}%
\pgfpathlineto{\pgfqpoint{3.268739in}{1.542342in}}%
\pgfpathlineto{\pgfqpoint{3.269594in}{1.626268in}}%
\pgfpathlineto{\pgfqpoint{3.273868in}{1.549575in}}%
\pgfpathlineto{\pgfqpoint{3.281573in}{1.644172in}}%
\pgfpathlineto{\pgfqpoint{3.284995in}{1.577277in}}%
\pgfpathlineto{\pgfqpoint{3.286705in}{1.674909in}}%
\pgfpathlineto{\pgfqpoint{3.291835in}{1.540125in}}%
\pgfpathlineto{\pgfqpoint{3.296964in}{1.633443in}}%
\pgfpathlineto{\pgfqpoint{3.299528in}{1.585149in}}%
\pgfpathlineto{\pgfqpoint{3.306369in}{1.632859in}}%
\pgfpathlineto{\pgfqpoint{3.309787in}{1.521812in}}%
\pgfpathlineto{\pgfqpoint{3.312350in}{1.641140in}}%
\pgfpathlineto{\pgfqpoint{3.320047in}{1.567772in}}%
\pgfpathlineto{\pgfqpoint{3.323472in}{1.688324in}}%
\pgfpathlineto{\pgfqpoint{3.327748in}{1.566314in}}%
\pgfpathlineto{\pgfqpoint{3.330315in}{1.650240in}}%
\pgfpathlineto{\pgfqpoint{3.337159in}{1.644757in}}%
\pgfpathlineto{\pgfqpoint{3.339724in}{1.514988in}}%
\pgfpathlineto{\pgfqpoint{3.342293in}{1.608482in}}%
\pgfpathlineto{\pgfqpoint{3.348282in}{1.539193in}}%
\pgfpathlineto{\pgfqpoint{3.351703in}{1.615886in}}%
\pgfpathlineto{\pgfqpoint{3.357690in}{1.547417in}}%
\pgfpathlineto{\pgfqpoint{3.359402in}{1.642656in}}%
\pgfpathlineto{\pgfqpoint{3.364537in}{1.575965in}}%
\pgfpathlineto{\pgfqpoint{3.368813in}{1.669777in}}%
\pgfpathlineto{\pgfqpoint{3.372234in}{1.579319in}}%
\pgfpathlineto{\pgfqpoint{3.379076in}{1.509972in}}%
\pgfpathlineto{\pgfqpoint{3.380784in}{1.617344in}}%
\pgfpathlineto{\pgfqpoint{3.385917in}{1.571414in}}%
\pgfpathlineto{\pgfqpoint{3.391052in}{1.654613in}}%
\pgfpathlineto{\pgfqpoint{3.396180in}{1.556980in}}%
\pgfpathlineto{\pgfqpoint{3.400457in}{1.637757in}}%
\pgfpathlineto{\pgfqpoint{3.405589in}{1.592207in}}%
\pgfpathlineto{\pgfqpoint{3.406443in}{1.644348in}}%
\pgfpathlineto{\pgfqpoint{3.414139in}{1.562347in}}%
\pgfpathlineto{\pgfqpoint{3.418410in}{1.655866in}}%
\pgfpathlineto{\pgfqpoint{3.420976in}{1.581971in}}%
\pgfpathlineto{\pgfqpoint{3.425249in}{1.662544in}}%
\pgfpathlineto{\pgfqpoint{3.427816in}{1.568294in}}%
\pgfpathlineto{\pgfqpoint{3.432947in}{1.504021in}}%
\pgfpathlineto{\pgfqpoint{3.437223in}{1.637378in}}%
\pgfpathlineto{\pgfqpoint{3.442356in}{1.551439in}}%
\pgfpathlineto{\pgfqpoint{3.445779in}{1.613377in}}%
\pgfpathlineto{\pgfqpoint{3.450907in}{1.684415in}}%
\pgfpathlineto{\pgfqpoint{3.456036in}{1.559198in}}%
\pgfpathlineto{\pgfqpoint{3.458605in}{1.639273in}}%
\pgfpathlineto{\pgfqpoint{3.463737in}{1.521929in}}%
\pgfpathlineto{\pgfqpoint{3.466302in}{1.648549in}}%
\pgfpathlineto{\pgfqpoint{3.472285in}{1.662953in}}%
\pgfpathlineto{\pgfqpoint{3.475705in}{1.560831in}}%
\pgfpathlineto{\pgfqpoint{3.479984in}{1.540651in}}%
\pgfpathlineto{\pgfqpoint{3.484262in}{1.609676in}}%
\pgfpathlineto{\pgfqpoint{3.491104in}{1.676601in}}%
\pgfpathlineto{\pgfqpoint{3.492814in}{1.570278in}}%
\pgfpathlineto{\pgfqpoint{3.497093in}{1.535285in}}%
\pgfpathlineto{\pgfqpoint{3.500515in}{1.625278in}}%
\pgfpathlineto{\pgfqpoint{3.504790in}{1.574712in}}%
\pgfpathlineto{\pgfqpoint{3.512483in}{1.545521in}}%
\pgfpathlineto{\pgfqpoint{3.515905in}{1.714454in}}%
\pgfpathlineto{\pgfqpoint{3.517615in}{1.584656in}}%
\pgfpathlineto{\pgfqpoint{3.524462in}{1.674095in}}%
\pgfpathlineto{\pgfqpoint{3.527029in}{1.523796in}}%
\pgfpathlineto{\pgfqpoint{3.533013in}{1.699582in}}%
\pgfpathlineto{\pgfqpoint{3.534724in}{1.637410in}}%
\pgfpathlineto{\pgfqpoint{3.539857in}{1.759595in}}%
\pgfpathlineto{\pgfqpoint{3.543277in}{1.658814in}}%
\pgfpathlineto{\pgfqpoint{3.547556in}{1.728975in}}%
\pgfpathlineto{\pgfqpoint{3.554393in}{1.612036in}}%
\pgfpathlineto{\pgfqpoint{3.556960in}{1.793712in}}%
\pgfpathlineto{\pgfqpoint{3.560375in}{1.677357in}}%
\pgfpathlineto{\pgfqpoint{3.565506in}{1.727167in}}%
\pgfpathlineto{\pgfqpoint{3.574058in}{1.638605in}}%
\pgfpathlineto{\pgfqpoint{3.578334in}{1.726381in}}%
\pgfpathlineto{\pgfqpoint{3.583469in}{1.678672in}}%
\pgfpathlineto{\pgfqpoint{3.589452in}{1.745772in}}%
\pgfpathlineto{\pgfqpoint{3.590304in}{1.682084in}}%
\pgfpathlineto{\pgfqpoint{3.596294in}{1.787005in}}%
\pgfpathlineto{\pgfqpoint{3.599713in}{1.615828in}}%
\pgfpathlineto{\pgfqpoint{3.604849in}{1.739123in}}%
\pgfpathlineto{\pgfqpoint{3.609981in}{1.762452in}}%
\pgfpathlineto{\pgfqpoint{3.611692in}{1.669981in}}%
\pgfpathlineto{\pgfqpoint{3.618535in}{1.656626in}}%
\pgfpathlineto{\pgfqpoint{3.620247in}{1.766536in}}%
\pgfpathlineto{\pgfqpoint{3.624526in}{1.656655in}}%
\pgfpathlineto{\pgfqpoint{3.631368in}{1.744607in}}%
\pgfpathlineto{\pgfqpoint{3.633075in}{1.663946in}}%
\pgfpathlineto{\pgfqpoint{3.639063in}{1.637001in}}%
\pgfpathlineto{\pgfqpoint{3.645057in}{1.731017in}}%
\pgfpathlineto{\pgfqpoint{3.647625in}{1.638342in}}%
\pgfpathlineto{\pgfqpoint{3.651902in}{1.715795in}}%
\pgfpathlineto{\pgfqpoint{3.655322in}{1.656772in}}%
\pgfpathlineto{\pgfqpoint{3.662162in}{1.789573in}}%
\pgfpathlineto{\pgfqpoint{3.666437in}{1.655022in}}%
\pgfpathlineto{\pgfqpoint{3.667291in}{1.730491in}}%
\pgfpathlineto{\pgfqpoint{3.674993in}{1.633385in}}%
\pgfpathlineto{\pgfqpoint{3.676704in}{1.750729in}}%
\pgfpathlineto{\pgfqpoint{3.680124in}{1.559081in}}%
\pgfpathlineto{\pgfqpoint{3.685255in}{1.701040in}}%
\pgfpathlineto{\pgfqpoint{3.689529in}{1.634959in}}%
\pgfpathlineto{\pgfqpoint{3.694663in}{1.674212in}}%
\pgfpathlineto{\pgfqpoint{3.699794in}{1.675435in}}%
\pgfpathlineto{\pgfqpoint{3.701506in}{1.553017in}}%
\pgfpathlineto{\pgfqpoint{3.708349in}{1.651581in}}%
\pgfpathlineto{\pgfqpoint{3.710060in}{1.550945in}}%
\pgfpathlineto{\pgfqpoint{3.714341in}{1.638400in}}%
\pgfpathlineto{\pgfqpoint{3.721187in}{1.667793in}}%
\pgfpathlineto{\pgfqpoint{3.723756in}{1.552082in}}%
\pgfpathlineto{\pgfqpoint{3.729741in}{1.636358in}}%
\pgfpathlineto{\pgfqpoint{3.731453in}{1.564999in}}%
\pgfpathlineto{\pgfqpoint{3.735725in}{1.659161in}}%
\pgfpathlineto{\pgfqpoint{3.740853in}{1.685028in}}%
\pgfpathlineto{\pgfqpoint{3.746839in}{1.577452in}}%
\pgfpathlineto{\pgfqpoint{3.751968in}{1.677708in}}%
\pgfpathlineto{\pgfqpoint{3.752824in}{1.592295in}}%
\pgfpathlineto{\pgfqpoint{3.757956in}{1.511196in}}%
\pgfpathlineto{\pgfqpoint{3.764798in}{1.566778in}}%
\pgfpathlineto{\pgfqpoint{3.766511in}{1.685584in}}%
\pgfpathlineto{\pgfqpoint{3.769934in}{1.625162in}}%
\pgfpathlineto{\pgfqpoint{3.775066in}{1.733118in}}%
\pgfpathlineto{\pgfqpoint{3.778487in}{1.628632in}}%
\pgfpathlineto{\pgfqpoint{3.787040in}{1.740464in}}%
\pgfpathlineto{\pgfqpoint{3.791320in}{1.603638in}}%
\pgfpathlineto{\pgfqpoint{3.799020in}{1.741166in}}%
\pgfpathlineto{\pgfqpoint{3.802439in}{1.640092in}}%
\pgfpathlineto{\pgfqpoint{3.807568in}{1.715561in}}%
\pgfpathlineto{\pgfqpoint{3.809278in}{1.649743in}}%
\pgfpathlineto{\pgfqpoint{3.816117in}{1.714830in}}%
\pgfpathlineto{\pgfqpoint{3.820392in}{1.558146in}}%
\pgfpathlineto{\pgfqpoint{3.823810in}{1.558672in}}%
\pgfpathlineto{\pgfqpoint{3.828940in}{1.690190in}}%
\pgfpathlineto{\pgfqpoint{3.832358in}{1.735770in}}%
\pgfpathlineto{\pgfqpoint{3.837489in}{1.612942in}}%
\pgfpathlineto{\pgfqpoint{3.838344in}{1.685525in}}%
\pgfpathlineto{\pgfqpoint{3.844324in}{1.558876in}}%
\pgfpathlineto{\pgfqpoint{3.850314in}{1.651055in}}%
\pgfpathlineto{\pgfqpoint{3.853739in}{1.562259in}}%
\pgfpathlineto{\pgfqpoint{3.857163in}{1.641023in}}%
\pgfpathlineto{\pgfqpoint{3.860589in}{1.573894in}}%
\pgfpathlineto{\pgfqpoint{3.867436in}{1.661495in}}%
\pgfpathlineto{\pgfqpoint{3.869999in}{1.552140in}}%
\pgfpathlineto{\pgfqpoint{3.872563in}{1.611773in}}%
\pgfpathlineto{\pgfqpoint{3.877696in}{1.662719in}}%
\pgfpathlineto{\pgfqpoint{3.881975in}{1.570511in}}%
\pgfpathlineto{\pgfqpoint{3.887961in}{1.558613in}}%
\pgfpathlineto{\pgfqpoint{3.889671in}{1.613318in}}%
\pgfpathlineto{\pgfqpoint{3.896512in}{1.584740in}}%
\pgfpathlineto{\pgfqpoint{3.899075in}{1.679399in}}%
\pgfpathlineto{\pgfqpoint{3.902494in}{1.582990in}}%
\pgfpathlineto{\pgfqpoint{3.910196in}{1.517465in}}%
\pgfpathlineto{\pgfqpoint{3.911052in}{1.606377in}}%
\pgfpathlineto{\pgfqpoint{3.918748in}{1.665634in}}%
\pgfpathlineto{\pgfqpoint{3.922167in}{1.628745in}}%
\pgfpathlineto{\pgfqpoint{3.925587in}{1.686165in}}%
\pgfpathlineto{\pgfqpoint{3.931576in}{1.623613in}}%
\pgfpathlineto{\pgfqpoint{3.933287in}{1.696196in}}%
\pgfpathlineto{\pgfqpoint{3.938418in}{1.668462in}}%
\pgfpathlineto{\pgfqpoint{3.940985in}{1.571268in}}%
\pgfpathlineto{\pgfqpoint{3.947833in}{1.541232in}}%
\pgfpathlineto{\pgfqpoint{3.952966in}{1.614137in}}%
\pgfpathlineto{\pgfqpoint{3.955532in}{1.519185in}}%
\pgfpathlineto{\pgfqpoint{3.958954in}{1.664820in}}%
\pgfpathlineto{\pgfqpoint{3.962375in}{1.573777in}}%
\pgfpathlineto{\pgfqpoint{3.970067in}{1.554588in}}%
\pgfpathlineto{\pgfqpoint{3.970924in}{1.617052in}}%
\pgfpathlineto{\pgfqpoint{3.976053in}{1.542372in}}%
\pgfpathlineto{\pgfqpoint{3.979475in}{1.659161in}}%
\pgfpathlineto{\pgfqpoint{3.986324in}{1.660560in}}%
\pgfpathlineto{\pgfqpoint{3.988889in}{1.569752in}}%
\pgfpathlineto{\pgfqpoint{3.992312in}{1.621395in}}%
\pgfpathlineto{\pgfqpoint{3.997445in}{1.547676in}}%
\pgfpathlineto{\pgfqpoint{4.000867in}{1.667735in}}%
\pgfpathlineto{\pgfqpoint{4.005147in}{1.593080in}}%
\pgfpathlineto{\pgfqpoint{4.011132in}{1.640819in}}%
\pgfpathlineto{\pgfqpoint{4.017123in}{1.559198in}}%
\pgfpathlineto{\pgfqpoint{4.020541in}{1.713519in}}%
\pgfpathlineto{\pgfqpoint{4.023107in}{1.577569in}}%
\pgfpathlineto{\pgfqpoint{4.027385in}{1.653681in}}%
\pgfpathlineto{\pgfqpoint{4.030804in}{1.596174in}}%
\pgfpathlineto{\pgfqpoint{4.035079in}{1.562171in}}%
\pgfpathlineto{\pgfqpoint{4.040214in}{1.590282in}}%
\pgfpathlineto{\pgfqpoint{4.043636in}{1.746006in}}%
\pgfpathlineto{\pgfqpoint{4.049623in}{1.648899in}}%
\pgfpathlineto{\pgfqpoint{4.053904in}{1.744168in}}%
\pgfpathlineto{\pgfqpoint{4.058181in}{1.602881in}}%
\pgfpathlineto{\pgfqpoint{4.062458in}{1.671410in}}%
\pgfpathlineto{\pgfqpoint{4.067590in}{1.526535in}}%
\pgfpathlineto{\pgfqpoint{4.069301in}{1.591739in}}%
\pgfpathlineto{\pgfqpoint{4.076144in}{1.535051in}}%
\pgfpathlineto{\pgfqpoint{4.080418in}{1.616993in}}%
\pgfpathlineto{\pgfqpoint{4.085551in}{1.552721in}}%
\pgfpathlineto{\pgfqpoint{4.087264in}{1.728274in}}%
\pgfpathlineto{\pgfqpoint{4.091542in}{1.646360in}}%
\pgfpathlineto{\pgfqpoint{4.097525in}{1.776217in}}%
\pgfpathlineto{\pgfqpoint{4.099237in}{1.674909in}}%
\pgfpathlineto{\pgfqpoint{4.104369in}{1.718188in}}%
\pgfpathlineto{\pgfqpoint{4.110358in}{1.639858in}}%
\pgfpathlineto{\pgfqpoint{4.114636in}{1.697219in}}%
\pgfpathlineto{\pgfqpoint{4.119765in}{1.657674in}}%
\pgfpathlineto{\pgfqpoint{4.120620in}{1.730082in}}%
\pgfpathlineto{\pgfqpoint{4.127462in}{1.640965in}}%
\pgfpathlineto{\pgfqpoint{4.132595in}{1.601654in}}%
\pgfpathlineto{\pgfqpoint{4.135165in}{1.683074in}}%
\pgfpathlineto{\pgfqpoint{4.138587in}{1.610841in}}%
\pgfpathlineto{\pgfqpoint{4.143724in}{1.701504in}}%
\pgfpathlineto{\pgfqpoint{4.147142in}{1.642832in}}%
\pgfpathlineto{\pgfqpoint{4.149711in}{1.691765in}}%
\pgfpathlineto{\pgfqpoint{4.153137in}{1.632157in}}%
\pgfpathlineto{\pgfqpoint{4.157412in}{1.639040in}}%
\pgfpathlineto{\pgfqpoint{4.159124in}{1.729205in}}%
\pgfpathlineto{\pgfqpoint{4.165117in}{1.738597in}}%
\pgfpathlineto{\pgfqpoint{4.166829in}{1.644114in}}%
\pgfpathlineto{\pgfqpoint{4.170249in}{1.725066in}}%
\pgfpathlineto{\pgfqpoint{4.172814in}{1.684765in}}%
\pgfpathlineto{\pgfqpoint{4.177947in}{1.656421in}}%
\pgfpathlineto{\pgfqpoint{4.180510in}{1.745831in}}%
\pgfpathlineto{\pgfqpoint{4.184787in}{1.651347in}}%
\pgfpathlineto{\pgfqpoint{4.186496in}{1.689547in}}%
\pgfpathlineto{\pgfqpoint{4.190773in}{1.632976in}}%
\pgfpathlineto{\pgfqpoint{4.194194in}{1.722093in}}%
\pgfpathlineto{\pgfqpoint{4.197616in}{1.677941in}}%
\pgfpathlineto{\pgfqpoint{4.205311in}{1.739299in}}%
\pgfpathlineto{\pgfqpoint{4.208735in}{1.737520in}}%
\pgfpathlineto{\pgfqpoint{4.213009in}{1.622301in}}%
\pgfpathlineto{\pgfqpoint{4.214719in}{1.639741in}}%
\pgfpathlineto{\pgfqpoint{4.218141in}{1.614546in}}%
\pgfpathlineto{\pgfqpoint{4.221561in}{1.715093in}}%
\pgfpathlineto{\pgfqpoint{4.225839in}{1.656246in}}%
\pgfpathlineto{\pgfqpoint{4.227550in}{1.710604in}}%
\pgfpathlineto{\pgfqpoint{4.232681in}{1.634930in}}%
\pgfpathlineto{\pgfqpoint{4.234393in}{1.733235in}}%
\pgfpathlineto{\pgfqpoint{4.238671in}{1.731602in}}%
\pgfpathlineto{\pgfqpoint{4.244659in}{1.518079in}}%
\pgfpathlineto{\pgfqpoint{4.248080in}{1.675319in}}%
\pgfpathlineto{\pgfqpoint{4.252359in}{1.577218in}}%
\pgfpathlineto{\pgfqpoint{4.254925in}{1.676718in}}%
\pgfpathlineto{\pgfqpoint{4.260913in}{1.589292in}}%
\pgfpathlineto{\pgfqpoint{4.261769in}{1.642832in}}%
\pgfpathlineto{\pgfqpoint{4.265191in}{1.603521in}}%
\pgfpathlineto{\pgfqpoint{4.270325in}{1.592733in}}%
\pgfpathlineto{\pgfqpoint{4.274604in}{1.659629in}}%
\pgfpathlineto{\pgfqpoint{4.278880in}{1.570278in}}%
\pgfpathlineto{\pgfqpoint{4.282299in}{1.671293in}}%
\pgfpathlineto{\pgfqpoint{4.286570in}{1.594366in}}%
\pgfpathlineto{\pgfqpoint{4.290850in}{1.643065in}}%
\pgfpathlineto{\pgfqpoint{4.293417in}{1.564038in}}%
\pgfpathlineto{\pgfqpoint{4.297691in}{1.634316in}}%
\pgfpathlineto{\pgfqpoint{4.301964in}{1.564911in}}%
\pgfpathlineto{\pgfqpoint{4.303673in}{1.641257in}}%
\pgfpathlineto{\pgfqpoint{4.307096in}{1.726641in}}%
\pgfpathlineto{\pgfqpoint{4.311374in}{1.575352in}}%
\pgfpathlineto{\pgfqpoint{4.313941in}{1.672107in}}%
\pgfpathlineto{\pgfqpoint{4.319069in}{1.674325in}}%
\pgfpathlineto{\pgfqpoint{4.319924in}{1.593489in}}%
\pgfpathlineto{\pgfqpoint{4.325903in}{1.571560in}}%
\pgfpathlineto{\pgfqpoint{4.326759in}{1.621804in}}%
\pgfpathlineto{\pgfqpoint{4.331893in}{1.531084in}}%
\pgfpathlineto{\pgfqpoint{4.336171in}{1.524610in}}%
\pgfpathlineto{\pgfqpoint{4.337026in}{1.643938in}}%
\pgfpathlineto{\pgfqpoint{4.340445in}{1.576575in}}%
\pgfpathlineto{\pgfqpoint{4.344726in}{1.642013in}}%
\pgfpathlineto{\pgfqpoint{4.348149in}{1.589931in}}%
\pgfpathlineto{\pgfqpoint{4.353279in}{1.660969in}}%
\pgfpathlineto{\pgfqpoint{4.354134in}{1.547764in}}%
\pgfpathlineto{\pgfqpoint{4.358409in}{1.614601in}}%
\pgfpathlineto{\pgfqpoint{4.363544in}{1.578384in}}%
\pgfpathlineto{\pgfqpoint{4.366112in}{1.667442in}}%
\pgfpathlineto{\pgfqpoint{4.369535in}{1.579841in}}%
\pgfpathlineto{\pgfqpoint{4.371246in}{1.669338in}}%
\pgfpathlineto{\pgfqpoint{4.377235in}{1.578676in}}%
\pgfpathlineto{\pgfqpoint{4.379798in}{1.623817in}}%
\pgfpathlineto{\pgfqpoint{4.383215in}{1.504635in}}%
\pgfpathlineto{\pgfqpoint{4.386637in}{1.655837in}}%
\pgfpathlineto{\pgfqpoint{4.390915in}{1.665050in}}%
\pgfpathlineto{\pgfqpoint{4.392627in}{1.551526in}}%
\pgfpathlineto{\pgfqpoint{4.395194in}{1.631748in}}%
\pgfpathlineto{\pgfqpoint{4.399466in}{1.576049in}}%
\pgfpathlineto{\pgfqpoint{4.404602in}{1.608712in}}%
\pgfpathlineto{\pgfqpoint{4.408022in}{1.529041in}}%
\pgfpathlineto{\pgfqpoint{4.409735in}{1.608653in}}%
\pgfpathlineto{\pgfqpoint{4.416578in}{1.546481in}}%
\pgfpathlineto{\pgfqpoint{4.419143in}{1.606030in}}%
\pgfpathlineto{\pgfqpoint{4.428550in}{1.545955in}}%
\pgfpathlineto{\pgfqpoint{4.429406in}{1.630407in}}%
\pgfpathlineto{\pgfqpoint{4.432826in}{1.678526in}}%
\pgfpathlineto{\pgfqpoint{4.436247in}{1.553831in}}%
\pgfpathlineto{\pgfqpoint{4.440528in}{1.634433in}}%
\pgfpathlineto{\pgfqpoint{4.444808in}{1.572378in}}%
\pgfpathlineto{\pgfqpoint{4.448226in}{1.690132in}}%
\pgfpathlineto{\pgfqpoint{4.451649in}{1.591389in}}%
\pgfpathlineto{\pgfqpoint{4.453359in}{1.648808in}}%
\pgfpathlineto{\pgfqpoint{4.457633in}{1.583516in}}%
\pgfpathlineto{\pgfqpoint{4.461057in}{1.635306in}}%
\pgfpathlineto{\pgfqpoint{4.464478in}{1.667092in}}%
\pgfpathlineto{\pgfqpoint{4.468754in}{1.640380in}}%
\pgfpathlineto{\pgfqpoint{4.473032in}{1.485475in}}%
\pgfpathlineto{\pgfqpoint{4.473885in}{1.589931in}}%
\pgfpathlineto{\pgfqpoint{4.479871in}{1.507638in}}%
\pgfpathlineto{\pgfqpoint{4.480727in}{1.647321in}}%
\pgfpathlineto{\pgfqpoint{4.484146in}{1.578238in}}%
\pgfpathlineto{\pgfqpoint{4.490131in}{1.659278in}}%
\pgfpathlineto{\pgfqpoint{4.490987in}{1.568703in}}%
\pgfpathlineto{\pgfqpoint{4.496974in}{1.593255in}}%
\pgfpathlineto{\pgfqpoint{4.498682in}{1.656651in}}%
\pgfpathlineto{\pgfqpoint{4.502953in}{1.676919in}}%
\pgfpathlineto{\pgfqpoint{4.507228in}{1.587889in}}%
\pgfpathlineto{\pgfqpoint{4.508938in}{1.665225in}}%
\pgfpathlineto{\pgfqpoint{4.513210in}{1.569459in}}%
\pgfpathlineto{\pgfqpoint{4.515776in}{1.620727in}}%
\pgfpathlineto{\pgfqpoint{4.519199in}{1.575322in}}%
\pgfpathlineto{\pgfqpoint{4.524328in}{1.689606in}}%
\pgfpathlineto{\pgfqpoint{4.525183in}{1.604923in}}%
\pgfpathlineto{\pgfqpoint{4.531176in}{1.561675in}}%
\pgfpathlineto{\pgfqpoint{4.532031in}{1.656651in}}%
\pgfpathlineto{\pgfqpoint{4.535450in}{1.727748in}}%
\pgfpathlineto{\pgfqpoint{4.538868in}{1.555230in}}%
\pgfpathlineto{\pgfqpoint{4.544856in}{1.632625in}}%
\pgfpathlineto{\pgfqpoint{4.548281in}{1.573427in}}%
\pgfpathlineto{\pgfqpoint{4.549136in}{1.632333in}}%
\pgfpathlineto{\pgfqpoint{4.552557in}{1.512420in}}%
\pgfpathlineto{\pgfqpoint{4.558543in}{1.642948in}}%
\pgfpathlineto{\pgfqpoint{4.559399in}{1.546193in}}%
\pgfpathlineto{\pgfqpoint{4.563676in}{1.516621in}}%
\pgfpathlineto{\pgfqpoint{4.567950in}{1.637641in}}%
\pgfpathlineto{\pgfqpoint{4.569658in}{1.533301in}}%
\pgfpathlineto{\pgfqpoint{4.573939in}{1.523328in}}%
\pgfpathlineto{\pgfqpoint{4.576507in}{1.633092in}}%
\pgfpathlineto{\pgfqpoint{4.580783in}{1.585588in}}%
\pgfpathlineto{\pgfqpoint{4.583350in}{1.638108in}}%
\pgfpathlineto{\pgfqpoint{4.589338in}{1.564272in}}%
\pgfpathlineto{\pgfqpoint{4.591044in}{1.639858in}}%
\pgfpathlineto{\pgfqpoint{4.595324in}{1.509855in}}%
\pgfpathlineto{\pgfqpoint{4.598746in}{1.658229in}}%
\pgfpathlineto{\pgfqpoint{4.602166in}{1.517089in}}%
\pgfpathlineto{\pgfqpoint{4.606443in}{1.599791in}}%
\pgfpathlineto{\pgfqpoint{4.609866in}{1.554007in}}%
\pgfpathlineto{\pgfqpoint{4.610723in}{1.591915in}}%
\pgfpathlineto{\pgfqpoint{4.615000in}{1.552140in}}%
\pgfpathlineto{\pgfqpoint{4.617565in}{1.636183in}}%
\pgfpathlineto{\pgfqpoint{4.622695in}{1.523328in}}%
\pgfpathlineto{\pgfqpoint{4.625260in}{1.588795in}}%
\pgfpathlineto{\pgfqpoint{4.630391in}{1.556863in}}%
\pgfpathlineto{\pgfqpoint{4.632102in}{1.600635in}}%
\pgfpathlineto{\pgfqpoint{4.634669in}{1.559227in}}%
\pgfpathlineto{\pgfqpoint{4.638945in}{1.560597in}}%
\pgfpathlineto{\pgfqpoint{4.644077in}{1.643475in}}%
\pgfpathlineto{\pgfqpoint{4.647497in}{1.530908in}}%
\pgfpathlineto{\pgfqpoint{4.648350in}{1.620493in}}%
\pgfpathlineto{\pgfqpoint{4.654343in}{1.609004in}}%
\pgfpathlineto{\pgfqpoint{4.656051in}{1.536333in}}%
\pgfpathlineto{\pgfqpoint{4.659473in}{1.669718in}}%
\pgfpathlineto{\pgfqpoint{4.664605in}{1.714626in}}%
\pgfpathlineto{\pgfqpoint{4.665461in}{1.615185in}}%
\pgfpathlineto{\pgfqpoint{4.668881in}{1.674091in}}%
\pgfpathlineto{\pgfqpoint{4.673161in}{1.622184in}}%
\pgfpathlineto{\pgfqpoint{4.677437in}{1.731306in}}%
\pgfpathlineto{\pgfqpoint{4.679148in}{1.637494in}}%
\pgfpathlineto{\pgfqpoint{4.682567in}{1.688031in}}%
\pgfpathlineto{\pgfqpoint{4.685989in}{1.614020in}}%
\pgfpathlineto{\pgfqpoint{4.690266in}{1.707805in}}%
\pgfpathlineto{\pgfqpoint{4.694542in}{1.749330in}}%
\pgfpathlineto{\pgfqpoint{4.697112in}{1.626268in}}%
\pgfpathlineto{\pgfqpoint{4.700535in}{1.590023in}}%
\pgfpathlineto{\pgfqpoint{4.704809in}{1.733523in}}%
\pgfpathlineto{\pgfqpoint{4.706522in}{1.748804in}}%
\pgfpathlineto{\pgfqpoint{4.709945in}{1.652512in}}%
\pgfpathlineto{\pgfqpoint{4.713368in}{1.804035in}}%
\pgfpathlineto{\pgfqpoint{4.716788in}{1.592119in}}%
\pgfpathlineto{\pgfqpoint{4.721066in}{1.647906in}}%
\pgfpathlineto{\pgfqpoint{4.723635in}{1.548114in}}%
\pgfpathlineto{\pgfqpoint{4.727062in}{1.597223in}}%
\pgfpathlineto{\pgfqpoint{4.731339in}{1.549776in}}%
\pgfpathlineto{\pgfqpoint{4.735622in}{1.625713in}}%
\pgfpathlineto{\pgfqpoint{4.737335in}{1.551322in}}%
\pgfpathlineto{\pgfqpoint{4.741616in}{1.659161in}}%
\pgfpathlineto{\pgfqpoint{4.745895in}{1.518546in}}%
\pgfpathlineto{\pgfqpoint{4.751882in}{1.772659in}}%
\pgfpathlineto{\pgfqpoint{4.754446in}{1.633268in}}%
\pgfpathlineto{\pgfqpoint{4.759573in}{1.523386in}}%
\pgfpathlineto{\pgfqpoint{4.761282in}{1.639273in}}%
\pgfpathlineto{\pgfqpoint{4.764703in}{1.715561in}}%
\pgfpathlineto{\pgfqpoint{4.769831in}{1.732066in}}%
\pgfpathlineto{\pgfqpoint{4.771544in}{1.605387in}}%
\pgfpathlineto{\pgfqpoint{4.774966in}{1.678584in}}%
\pgfpathlineto{\pgfqpoint{4.780961in}{1.572612in}}%
\pgfpathlineto{\pgfqpoint{4.784387in}{1.722210in}}%
\pgfpathlineto{\pgfqpoint{4.785242in}{1.523036in}}%
\pgfpathlineto{\pgfqpoint{4.789519in}{1.602122in}}%
\pgfpathlineto{\pgfqpoint{4.792084in}{1.499999in}}%
\pgfpathlineto{\pgfqpoint{4.795506in}{1.542225in}}%
\pgfpathlineto{\pgfqpoint{4.795506in}{1.542225in}}%
\pgfusepath{stroke}%
\end{pgfscope}%
\begin{pgfscope}%
\pgfsetrectcap%
\pgfsetmiterjoin%
\pgfsetlinewidth{0.803000pt}%
\definecolor{currentstroke}{rgb}{0.000000,0.000000,0.000000}%
\pgfsetstrokecolor{currentstroke}%
\pgfsetdash{}{0pt}%
\pgfpathmoveto{\pgfqpoint{0.484581in}{1.437021in}}%
\pgfpathlineto{\pgfqpoint{0.484581in}{2.011643in}}%
\pgfusepath{stroke}%
\end{pgfscope}%
\begin{pgfscope}%
\pgfsetrectcap%
\pgfsetmiterjoin%
\pgfsetlinewidth{0.803000pt}%
\definecolor{currentstroke}{rgb}{0.000000,0.000000,0.000000}%
\pgfsetstrokecolor{currentstroke}%
\pgfsetdash{}{0pt}%
\pgfpathmoveto{\pgfqpoint{5.000788in}{1.437021in}}%
\pgfpathlineto{\pgfqpoint{5.000788in}{2.011643in}}%
\pgfusepath{stroke}%
\end{pgfscope}%
\begin{pgfscope}%
\pgfsetrectcap%
\pgfsetmiterjoin%
\pgfsetlinewidth{0.803000pt}%
\definecolor{currentstroke}{rgb}{0.000000,0.000000,0.000000}%
\pgfsetstrokecolor{currentstroke}%
\pgfsetdash{}{0pt}%
\pgfpathmoveto{\pgfqpoint{0.484581in}{1.437021in}}%
\pgfpathlineto{\pgfqpoint{5.000788in}{1.437021in}}%
\pgfusepath{stroke}%
\end{pgfscope}%
\begin{pgfscope}%
\pgfsetrectcap%
\pgfsetmiterjoin%
\pgfsetlinewidth{0.803000pt}%
\definecolor{currentstroke}{rgb}{0.000000,0.000000,0.000000}%
\pgfsetstrokecolor{currentstroke}%
\pgfsetdash{}{0pt}%
\pgfpathmoveto{\pgfqpoint{0.484581in}{2.011643in}}%
\pgfpathlineto{\pgfqpoint{5.000788in}{2.011643in}}%
\pgfusepath{stroke}%
\end{pgfscope}%
\begin{pgfscope}%
\pgfsetbuttcap%
\pgfsetmiterjoin%
\definecolor{currentfill}{rgb}{1.000000,1.000000,1.000000}%
\pgfsetfillcolor{currentfill}%
\pgfsetlinewidth{0.000000pt}%
\definecolor{currentstroke}{rgb}{0.000000,0.000000,0.000000}%
\pgfsetstrokecolor{currentstroke}%
\pgfsetstrokeopacity{0.000000}%
\pgfsetdash{}{0pt}%
\pgfpathmoveto{\pgfqpoint{0.484581in}{0.539544in}}%
\pgfpathlineto{\pgfqpoint{5.000788in}{0.539544in}}%
\pgfpathlineto{\pgfqpoint{5.000788in}{1.114166in}}%
\pgfpathlineto{\pgfqpoint{0.484581in}{1.114166in}}%
\pgfpathlineto{\pgfqpoint{0.484581in}{0.539544in}}%
\pgfpathclose%
\pgfusepath{fill}%
\end{pgfscope}%
\begin{pgfscope}%
\pgfsetbuttcap%
\pgfsetroundjoin%
\definecolor{currentfill}{rgb}{0.000000,0.000000,0.000000}%
\pgfsetfillcolor{currentfill}%
\pgfsetlinewidth{0.803000pt}%
\definecolor{currentstroke}{rgb}{0.000000,0.000000,0.000000}%
\pgfsetstrokecolor{currentstroke}%
\pgfsetdash{}{0pt}%
\pgfsys@defobject{currentmarker}{\pgfqpoint{0.000000in}{-0.048611in}}{\pgfqpoint{0.000000in}{0.000000in}}{%
\pgfpathmoveto{\pgfqpoint{0.000000in}{0.000000in}}%
\pgfpathlineto{\pgfqpoint{0.000000in}{-0.048611in}}%
\pgfusepath{stroke,fill}%
}%
\begin{pgfscope}%
\pgfsys@transformshift{0.689546in}{0.539544in}%
\pgfsys@useobject{currentmarker}{}%
\end{pgfscope}%
\end{pgfscope}%
\begin{pgfscope}%
\definecolor{textcolor}{rgb}{0.000000,0.000000,0.000000}%
\pgfsetstrokecolor{textcolor}%
\pgfsetfillcolor{textcolor}%
\pgftext[x=0.689546in,y=0.442322in,,top]{\color{textcolor}\rmfamily\fontsize{8.000000}{9.600000}\selectfont \(\displaystyle {06{:}00}\)}%
\end{pgfscope}%
\begin{pgfscope}%
\pgfsetbuttcap%
\pgfsetroundjoin%
\definecolor{currentfill}{rgb}{0.000000,0.000000,0.000000}%
\pgfsetfillcolor{currentfill}%
\pgfsetlinewidth{0.803000pt}%
\definecolor{currentstroke}{rgb}{0.000000,0.000000,0.000000}%
\pgfsetstrokecolor{currentstroke}%
\pgfsetdash{}{0pt}%
\pgfsys@defobject{currentmarker}{\pgfqpoint{0.000000in}{-0.048611in}}{\pgfqpoint{0.000000in}{0.000000in}}{%
\pgfpathmoveto{\pgfqpoint{0.000000in}{0.000000in}}%
\pgfpathlineto{\pgfqpoint{0.000000in}{-0.048611in}}%
\pgfusepath{stroke,fill}%
}%
\begin{pgfscope}%
\pgfsys@transformshift{1.202878in}{0.539544in}%
\pgfsys@useobject{currentmarker}{}%
\end{pgfscope}%
\end{pgfscope}%
\begin{pgfscope}%
\definecolor{textcolor}{rgb}{0.000000,0.000000,0.000000}%
\pgfsetstrokecolor{textcolor}%
\pgfsetfillcolor{textcolor}%
\pgftext[x=1.202878in,y=0.442322in,,top]{\color{textcolor}\rmfamily\fontsize{8.000000}{9.600000}\selectfont \(\displaystyle {09{:}00}\)}%
\end{pgfscope}%
\begin{pgfscope}%
\pgfsetbuttcap%
\pgfsetroundjoin%
\definecolor{currentfill}{rgb}{0.000000,0.000000,0.000000}%
\pgfsetfillcolor{currentfill}%
\pgfsetlinewidth{0.803000pt}%
\definecolor{currentstroke}{rgb}{0.000000,0.000000,0.000000}%
\pgfsetstrokecolor{currentstroke}%
\pgfsetdash{}{0pt}%
\pgfsys@defobject{currentmarker}{\pgfqpoint{0.000000in}{-0.048611in}}{\pgfqpoint{0.000000in}{0.000000in}}{%
\pgfpathmoveto{\pgfqpoint{0.000000in}{0.000000in}}%
\pgfpathlineto{\pgfqpoint{0.000000in}{-0.048611in}}%
\pgfusepath{stroke,fill}%
}%
\begin{pgfscope}%
\pgfsys@transformshift{1.716211in}{0.539544in}%
\pgfsys@useobject{currentmarker}{}%
\end{pgfscope}%
\end{pgfscope}%
\begin{pgfscope}%
\definecolor{textcolor}{rgb}{0.000000,0.000000,0.000000}%
\pgfsetstrokecolor{textcolor}%
\pgfsetfillcolor{textcolor}%
\pgftext[x=1.716211in,y=0.442322in,,top]{\color{textcolor}\rmfamily\fontsize{8.000000}{9.600000}\selectfont \(\displaystyle {12{:}00}\)}%
\end{pgfscope}%
\begin{pgfscope}%
\pgfsetbuttcap%
\pgfsetroundjoin%
\definecolor{currentfill}{rgb}{0.000000,0.000000,0.000000}%
\pgfsetfillcolor{currentfill}%
\pgfsetlinewidth{0.803000pt}%
\definecolor{currentstroke}{rgb}{0.000000,0.000000,0.000000}%
\pgfsetstrokecolor{currentstroke}%
\pgfsetdash{}{0pt}%
\pgfsys@defobject{currentmarker}{\pgfqpoint{0.000000in}{-0.048611in}}{\pgfqpoint{0.000000in}{0.000000in}}{%
\pgfpathmoveto{\pgfqpoint{0.000000in}{0.000000in}}%
\pgfpathlineto{\pgfqpoint{0.000000in}{-0.048611in}}%
\pgfusepath{stroke,fill}%
}%
\begin{pgfscope}%
\pgfsys@transformshift{2.229543in}{0.539544in}%
\pgfsys@useobject{currentmarker}{}%
\end{pgfscope}%
\end{pgfscope}%
\begin{pgfscope}%
\definecolor{textcolor}{rgb}{0.000000,0.000000,0.000000}%
\pgfsetstrokecolor{textcolor}%
\pgfsetfillcolor{textcolor}%
\pgftext[x=2.229543in,y=0.442322in,,top]{\color{textcolor}\rmfamily\fontsize{8.000000}{9.600000}\selectfont \(\displaystyle {15{:}00}\)}%
\end{pgfscope}%
\begin{pgfscope}%
\pgfsetbuttcap%
\pgfsetroundjoin%
\definecolor{currentfill}{rgb}{0.000000,0.000000,0.000000}%
\pgfsetfillcolor{currentfill}%
\pgfsetlinewidth{0.803000pt}%
\definecolor{currentstroke}{rgb}{0.000000,0.000000,0.000000}%
\pgfsetstrokecolor{currentstroke}%
\pgfsetdash{}{0pt}%
\pgfsys@defobject{currentmarker}{\pgfqpoint{0.000000in}{-0.048611in}}{\pgfqpoint{0.000000in}{0.000000in}}{%
\pgfpathmoveto{\pgfqpoint{0.000000in}{0.000000in}}%
\pgfpathlineto{\pgfqpoint{0.000000in}{-0.048611in}}%
\pgfusepath{stroke,fill}%
}%
\begin{pgfscope}%
\pgfsys@transformshift{2.742876in}{0.539544in}%
\pgfsys@useobject{currentmarker}{}%
\end{pgfscope}%
\end{pgfscope}%
\begin{pgfscope}%
\definecolor{textcolor}{rgb}{0.000000,0.000000,0.000000}%
\pgfsetstrokecolor{textcolor}%
\pgfsetfillcolor{textcolor}%
\pgftext[x=2.742876in,y=0.442322in,,top]{\color{textcolor}\rmfamily\fontsize{8.000000}{9.600000}\selectfont \(\displaystyle {18{:}00}\)}%
\end{pgfscope}%
\begin{pgfscope}%
\pgfsetbuttcap%
\pgfsetroundjoin%
\definecolor{currentfill}{rgb}{0.000000,0.000000,0.000000}%
\pgfsetfillcolor{currentfill}%
\pgfsetlinewidth{0.803000pt}%
\definecolor{currentstroke}{rgb}{0.000000,0.000000,0.000000}%
\pgfsetstrokecolor{currentstroke}%
\pgfsetdash{}{0pt}%
\pgfsys@defobject{currentmarker}{\pgfqpoint{0.000000in}{-0.048611in}}{\pgfqpoint{0.000000in}{0.000000in}}{%
\pgfpathmoveto{\pgfqpoint{0.000000in}{0.000000in}}%
\pgfpathlineto{\pgfqpoint{0.000000in}{-0.048611in}}%
\pgfusepath{stroke,fill}%
}%
\begin{pgfscope}%
\pgfsys@transformshift{3.256208in}{0.539544in}%
\pgfsys@useobject{currentmarker}{}%
\end{pgfscope}%
\end{pgfscope}%
\begin{pgfscope}%
\definecolor{textcolor}{rgb}{0.000000,0.000000,0.000000}%
\pgfsetstrokecolor{textcolor}%
\pgfsetfillcolor{textcolor}%
\pgftext[x=3.256208in,y=0.442322in,,top]{\color{textcolor}\rmfamily\fontsize{8.000000}{9.600000}\selectfont \(\displaystyle {21{:}00}\)}%
\end{pgfscope}%
\begin{pgfscope}%
\pgfsetbuttcap%
\pgfsetroundjoin%
\definecolor{currentfill}{rgb}{0.000000,0.000000,0.000000}%
\pgfsetfillcolor{currentfill}%
\pgfsetlinewidth{0.803000pt}%
\definecolor{currentstroke}{rgb}{0.000000,0.000000,0.000000}%
\pgfsetstrokecolor{currentstroke}%
\pgfsetdash{}{0pt}%
\pgfsys@defobject{currentmarker}{\pgfqpoint{0.000000in}{-0.048611in}}{\pgfqpoint{0.000000in}{0.000000in}}{%
\pgfpathmoveto{\pgfqpoint{0.000000in}{0.000000in}}%
\pgfpathlineto{\pgfqpoint{0.000000in}{-0.048611in}}%
\pgfusepath{stroke,fill}%
}%
\begin{pgfscope}%
\pgfsys@transformshift{3.769541in}{0.539544in}%
\pgfsys@useobject{currentmarker}{}%
\end{pgfscope}%
\end{pgfscope}%
\begin{pgfscope}%
\definecolor{textcolor}{rgb}{0.000000,0.000000,0.000000}%
\pgfsetstrokecolor{textcolor}%
\pgfsetfillcolor{textcolor}%
\pgftext[x=3.769541in,y=0.442322in,,top]{\color{textcolor}\rmfamily\fontsize{8.000000}{9.600000}\selectfont \(\displaystyle {00{:}00}\)}%
\end{pgfscope}%
\begin{pgfscope}%
\pgfsetbuttcap%
\pgfsetroundjoin%
\definecolor{currentfill}{rgb}{0.000000,0.000000,0.000000}%
\pgfsetfillcolor{currentfill}%
\pgfsetlinewidth{0.803000pt}%
\definecolor{currentstroke}{rgb}{0.000000,0.000000,0.000000}%
\pgfsetstrokecolor{currentstroke}%
\pgfsetdash{}{0pt}%
\pgfsys@defobject{currentmarker}{\pgfqpoint{0.000000in}{-0.048611in}}{\pgfqpoint{0.000000in}{0.000000in}}{%
\pgfpathmoveto{\pgfqpoint{0.000000in}{0.000000in}}%
\pgfpathlineto{\pgfqpoint{0.000000in}{-0.048611in}}%
\pgfusepath{stroke,fill}%
}%
\begin{pgfscope}%
\pgfsys@transformshift{4.282873in}{0.539544in}%
\pgfsys@useobject{currentmarker}{}%
\end{pgfscope}%
\end{pgfscope}%
\begin{pgfscope}%
\definecolor{textcolor}{rgb}{0.000000,0.000000,0.000000}%
\pgfsetstrokecolor{textcolor}%
\pgfsetfillcolor{textcolor}%
\pgftext[x=4.282873in,y=0.442322in,,top]{\color{textcolor}\rmfamily\fontsize{8.000000}{9.600000}\selectfont \(\displaystyle {03{:}00}\)}%
\end{pgfscope}%
\begin{pgfscope}%
\pgfsetbuttcap%
\pgfsetroundjoin%
\definecolor{currentfill}{rgb}{0.000000,0.000000,0.000000}%
\pgfsetfillcolor{currentfill}%
\pgfsetlinewidth{0.803000pt}%
\definecolor{currentstroke}{rgb}{0.000000,0.000000,0.000000}%
\pgfsetstrokecolor{currentstroke}%
\pgfsetdash{}{0pt}%
\pgfsys@defobject{currentmarker}{\pgfqpoint{0.000000in}{-0.048611in}}{\pgfqpoint{0.000000in}{0.000000in}}{%
\pgfpathmoveto{\pgfqpoint{0.000000in}{0.000000in}}%
\pgfpathlineto{\pgfqpoint{0.000000in}{-0.048611in}}%
\pgfusepath{stroke,fill}%
}%
\begin{pgfscope}%
\pgfsys@transformshift{4.796206in}{0.539544in}%
\pgfsys@useobject{currentmarker}{}%
\end{pgfscope}%
\end{pgfscope}%
\begin{pgfscope}%
\definecolor{textcolor}{rgb}{0.000000,0.000000,0.000000}%
\pgfsetstrokecolor{textcolor}%
\pgfsetfillcolor{textcolor}%
\pgftext[x=4.796206in,y=0.442322in,,top]{\color{textcolor}\rmfamily\fontsize{8.000000}{9.600000}\selectfont \(\displaystyle {06{:}00}\)}%
\end{pgfscope}%
\begin{pgfscope}%
\definecolor{textcolor}{rgb}{0.000000,0.000000,0.000000}%
\pgfsetstrokecolor{textcolor}%
\pgfsetfillcolor{textcolor}%
\pgftext[x=2.742685in,y=0.288100in,,top]{\color{textcolor}\rmfamily\fontsize{10.000000}{12.000000}\selectfont Time (UTC)}%
\end{pgfscope}%
\begin{pgfscope}%
\pgfsetbuttcap%
\pgfsetroundjoin%
\definecolor{currentfill}{rgb}{0.000000,0.000000,0.000000}%
\pgfsetfillcolor{currentfill}%
\pgfsetlinewidth{0.803000pt}%
\definecolor{currentstroke}{rgb}{0.000000,0.000000,0.000000}%
\pgfsetstrokecolor{currentstroke}%
\pgfsetdash{}{0pt}%
\pgfsys@defobject{currentmarker}{\pgfqpoint{-0.048611in}{0.000000in}}{\pgfqpoint{-0.000000in}{0.000000in}}{%
\pgfpathmoveto{\pgfqpoint{-0.000000in}{0.000000in}}%
\pgfpathlineto{\pgfqpoint{-0.048611in}{0.000000in}}%
\pgfusepath{stroke,fill}%
}%
\begin{pgfscope}%
\pgfsys@transformshift{0.484581in}{0.717544in}%
\pgfsys@useobject{currentmarker}{}%
\end{pgfscope}%
\end{pgfscope}%
\begin{pgfscope}%
\definecolor{textcolor}{rgb}{0.000000,0.000000,0.000000}%
\pgfsetstrokecolor{textcolor}%
\pgfsetfillcolor{textcolor}%
\pgftext[x=0.328331in, y=0.678988in, left, base]{\color{textcolor}\rmfamily\fontsize{8.000000}{9.600000}\selectfont \(\displaystyle {0}\)}%
\end{pgfscope}%
\begin{pgfscope}%
\pgfsetbuttcap%
\pgfsetroundjoin%
\definecolor{currentfill}{rgb}{0.000000,0.000000,0.000000}%
\pgfsetfillcolor{currentfill}%
\pgfsetlinewidth{0.803000pt}%
\definecolor{currentstroke}{rgb}{0.000000,0.000000,0.000000}%
\pgfsetstrokecolor{currentstroke}%
\pgfsetdash{}{0pt}%
\pgfsys@defobject{currentmarker}{\pgfqpoint{-0.048611in}{0.000000in}}{\pgfqpoint{-0.000000in}{0.000000in}}{%
\pgfpathmoveto{\pgfqpoint{-0.000000in}{0.000000in}}%
\pgfpathlineto{\pgfqpoint{-0.048611in}{0.000000in}}%
\pgfusepath{stroke,fill}%
}%
\begin{pgfscope}%
\pgfsys@transformshift{0.484581in}{0.921085in}%
\pgfsys@useobject{currentmarker}{}%
\end{pgfscope}%
\end{pgfscope}%
\begin{pgfscope}%
\definecolor{textcolor}{rgb}{0.000000,0.000000,0.000000}%
\pgfsetstrokecolor{textcolor}%
\pgfsetfillcolor{textcolor}%
\pgftext[x=0.328331in, y=0.882529in, left, base]{\color{textcolor}\rmfamily\fontsize{8.000000}{9.600000}\selectfont \(\displaystyle {5}\)}%
\end{pgfscope}%
\begin{pgfscope}%
\definecolor{textcolor}{rgb}{0.000000,0.000000,0.000000}%
\pgfsetstrokecolor{textcolor}%
\pgfsetfillcolor{textcolor}%
\pgftext[x=0.484581in,y=1.155833in,left,base]{\color{textcolor}\rmfamily\fontsize{8.000000}{9.600000}\selectfont \(\displaystyle \times{10^{\ensuremath{-}6}}{}\)}%
\end{pgfscope}%
\begin{pgfscope}%
\pgfpathrectangle{\pgfqpoint{0.484581in}{0.539544in}}{\pgfqpoint{4.516206in}{0.574622in}}%
\pgfusepath{clip}%
\pgfsetrectcap%
\pgfsetroundjoin%
\pgfsetlinewidth{0.501875pt}%
\definecolor{currentstroke}{rgb}{0.003922,0.450980,0.698039}%
\pgfsetstrokecolor{currentstroke}%
\pgfsetstrokeopacity{0.700000}%
\pgfsetdash{}{0pt}%
\pgfpathmoveto{\pgfqpoint{0.689863in}{0.721589in}}%
\pgfpathlineto{\pgfqpoint{0.694140in}{0.685169in}}%
\pgfpathlineto{\pgfqpoint{0.698419in}{0.771788in}}%
\pgfpathlineto{\pgfqpoint{0.700132in}{0.644899in}}%
\pgfpathlineto{\pgfqpoint{0.703553in}{0.767073in}}%
\pgfpathlineto{\pgfqpoint{0.707833in}{0.699777in}}%
\pgfpathlineto{\pgfqpoint{0.713821in}{0.764305in}}%
\pgfpathlineto{\pgfqpoint{0.718102in}{0.651844in}}%
\pgfpathlineto{\pgfqpoint{0.721524in}{0.733357in}}%
\pgfpathlineto{\pgfqpoint{0.726652in}{0.645940in}}%
\pgfpathlineto{\pgfqpoint{0.729218in}{0.759733in}}%
\pgfpathlineto{\pgfqpoint{0.736917in}{0.775857in}}%
\pgfpathlineto{\pgfqpoint{0.740340in}{0.660732in}}%
\pgfpathlineto{\pgfqpoint{0.742052in}{0.726624in}}%
\pgfpathlineto{\pgfqpoint{0.746327in}{0.778195in}}%
\pgfpathlineto{\pgfqpoint{0.751457in}{0.661163in}}%
\pgfpathlineto{\pgfqpoint{0.755738in}{0.790322in}}%
\pgfpathlineto{\pgfqpoint{0.759157in}{0.660947in}}%
\pgfpathlineto{\pgfqpoint{0.765999in}{0.761461in}}%
\pgfpathlineto{\pgfqpoint{0.767711in}{0.680382in}}%
\pgfpathlineto{\pgfqpoint{0.773703in}{0.750343in}}%
\pgfpathlineto{\pgfqpoint{0.776269in}{0.679736in}}%
\pgfpathlineto{\pgfqpoint{0.783967in}{0.745340in}}%
\pgfpathlineto{\pgfqpoint{0.784824in}{0.605387in}}%
\pgfpathlineto{\pgfqpoint{0.789097in}{0.784242in}}%
\pgfpathlineto{\pgfqpoint{0.793373in}{0.817025in}}%
\pgfpathlineto{\pgfqpoint{0.797648in}{0.692652in}}%
\pgfpathlineto{\pgfqpoint{0.805343in}{0.725690in}}%
\pgfpathlineto{\pgfqpoint{0.807054in}{0.608730in}}%
\pgfpathlineto{\pgfqpoint{0.811330in}{0.738647in}}%
\pgfpathlineto{\pgfqpoint{0.816461in}{0.682759in}}%
\pgfpathlineto{\pgfqpoint{0.819025in}{0.765063in}}%
\pgfpathlineto{\pgfqpoint{0.823303in}{0.717449in}}%
\pgfpathlineto{\pgfqpoint{0.827584in}{0.851108in}}%
\pgfpathlineto{\pgfqpoint{0.834428in}{0.692078in}}%
\pgfpathlineto{\pgfqpoint{0.838709in}{0.729073in}}%
\pgfpathlineto{\pgfqpoint{0.840420in}{0.676353in}}%
\pgfpathlineto{\pgfqpoint{0.848114in}{0.744295in}}%
\pgfpathlineto{\pgfqpoint{0.848968in}{0.708705in}}%
\pgfpathlineto{\pgfqpoint{0.855805in}{0.669587in}}%
\pgfpathlineto{\pgfqpoint{0.859233in}{0.748939in}}%
\pgfpathlineto{\pgfqpoint{0.861803in}{0.677757in}}%
\pgfpathlineto{\pgfqpoint{0.867793in}{0.662535in}}%
\pgfpathlineto{\pgfqpoint{0.870360in}{0.764416in}}%
\pgfpathlineto{\pgfqpoint{0.875490in}{0.657676in}}%
\pgfpathlineto{\pgfqpoint{0.879771in}{0.744551in}}%
\pgfpathlineto{\pgfqpoint{0.883195in}{0.653176in}}%
\pgfpathlineto{\pgfqpoint{0.887469in}{0.783524in}}%
\pgfpathlineto{\pgfqpoint{0.893457in}{0.703128in}}%
\pgfpathlineto{\pgfqpoint{0.896021in}{0.741311in}}%
\pgfpathlineto{\pgfqpoint{0.903720in}{0.770711in}}%
\pgfpathlineto{\pgfqpoint{0.907998in}{0.652889in}}%
\pgfpathlineto{\pgfqpoint{0.910566in}{0.728211in}}%
\pgfpathlineto{\pgfqpoint{0.914841in}{0.649793in}}%
\pgfpathlineto{\pgfqpoint{0.918259in}{0.754914in}}%
\pgfpathlineto{\pgfqpoint{0.922534in}{0.779168in}}%
\pgfpathlineto{\pgfqpoint{0.925951in}{0.659475in}}%
\pgfpathlineto{\pgfqpoint{0.932791in}{0.780141in}}%
\pgfpathlineto{\pgfqpoint{0.935359in}{0.665881in}}%
\pgfpathlineto{\pgfqpoint{0.938782in}{0.766467in}}%
\pgfpathlineto{\pgfqpoint{0.947330in}{0.636984in}}%
\pgfpathlineto{\pgfqpoint{0.951605in}{0.759015in}}%
\pgfpathlineto{\pgfqpoint{0.959305in}{0.764089in}}%
\pgfpathlineto{\pgfqpoint{0.963579in}{0.653679in}}%
\pgfpathlineto{\pgfqpoint{0.966147in}{0.724035in}}%
\pgfpathlineto{\pgfqpoint{0.969570in}{0.683477in}}%
\pgfpathlineto{\pgfqpoint{0.976408in}{0.806015in}}%
\pgfpathlineto{\pgfqpoint{0.978974in}{0.720329in}}%
\pgfpathlineto{\pgfqpoint{0.984102in}{0.758728in}}%
\pgfpathlineto{\pgfqpoint{0.985812in}{0.662463in}}%
\pgfpathlineto{\pgfqpoint{0.991797in}{0.733321in}}%
\pgfpathlineto{\pgfqpoint{0.997785in}{0.750271in}}%
\pgfpathlineto{\pgfqpoint{0.999494in}{0.780716in}}%
\pgfpathlineto{\pgfqpoint{1.003769in}{0.688448in}}%
\pgfpathlineto{\pgfqpoint{1.010613in}{0.744295in}}%
\pgfpathlineto{\pgfqpoint{1.014889in}{0.673186in}}%
\pgfpathlineto{\pgfqpoint{1.018310in}{0.746960in}}%
\pgfpathlineto{\pgfqpoint{1.020874in}{0.673545in}}%
\pgfpathlineto{\pgfqpoint{1.026001in}{0.741846in}}%
\pgfpathlineto{\pgfqpoint{1.029421in}{0.668291in}}%
\pgfpathlineto{\pgfqpoint{1.038832in}{0.800079in}}%
\pgfpathlineto{\pgfqpoint{1.041400in}{0.683154in}}%
\pgfpathlineto{\pgfqpoint{1.046538in}{0.642669in}}%
\pgfpathlineto{\pgfqpoint{1.050817in}{0.768733in}}%
\pgfpathlineto{\pgfqpoint{1.054238in}{0.679807in}}%
\pgfpathlineto{\pgfqpoint{1.059370in}{0.769020in}}%
\pgfpathlineto{\pgfqpoint{1.062792in}{0.636047in}}%
\pgfpathlineto{\pgfqpoint{1.068774in}{0.782192in}}%
\pgfpathlineto{\pgfqpoint{1.071339in}{0.679233in}}%
\pgfpathlineto{\pgfqpoint{1.077323in}{0.749338in}}%
\pgfpathlineto{\pgfqpoint{1.079890in}{0.663364in}}%
\pgfpathlineto{\pgfqpoint{1.084169in}{0.749338in}}%
\pgfpathlineto{\pgfqpoint{1.091012in}{0.695177in}}%
\pgfpathlineto{\pgfqpoint{1.095289in}{0.766144in}}%
\pgfpathlineto{\pgfqpoint{1.096997in}{0.686070in}}%
\pgfpathlineto{\pgfqpoint{1.102978in}{0.793385in}}%
\pgfpathlineto{\pgfqpoint{1.105542in}{0.697336in}}%
\pgfpathlineto{\pgfqpoint{1.112379in}{0.642669in}}%
\pgfpathlineto{\pgfqpoint{1.118364in}{0.769235in}}%
\pgfpathlineto{\pgfqpoint{1.124349in}{0.656671in}}%
\pgfpathlineto{\pgfqpoint{1.129483in}{0.728139in}}%
\pgfpathlineto{\pgfqpoint{1.131195in}{0.682616in}}%
\pgfpathlineto{\pgfqpoint{1.137183in}{0.656056in}}%
\pgfpathlineto{\pgfqpoint{1.139749in}{0.764632in}}%
\pgfpathlineto{\pgfqpoint{1.147448in}{0.689740in}}%
\pgfpathlineto{\pgfqpoint{1.150013in}{0.769235in}}%
\pgfpathlineto{\pgfqpoint{1.152579in}{0.575628in}}%
\pgfpathlineto{\pgfqpoint{1.156855in}{0.776400in}}%
\pgfpathlineto{\pgfqpoint{1.162841in}{0.681427in}}%
\pgfpathlineto{\pgfqpoint{1.167121in}{0.726592in}}%
\pgfpathlineto{\pgfqpoint{1.169688in}{0.667896in}}%
\pgfpathlineto{\pgfqpoint{1.176534in}{0.787481in}}%
\pgfpathlineto{\pgfqpoint{1.178245in}{0.676784in}}%
\pgfpathlineto{\pgfqpoint{1.185091in}{0.833942in}}%
\pgfpathlineto{\pgfqpoint{1.188512in}{0.707229in}}%
\pgfpathlineto{\pgfqpoint{1.191934in}{0.760671in}}%
\pgfpathlineto{\pgfqpoint{1.196209in}{0.672643in}}%
\pgfpathlineto{\pgfqpoint{1.200488in}{0.747535in}}%
\pgfpathlineto{\pgfqpoint{1.203910in}{0.679915in}}%
\pgfpathlineto{\pgfqpoint{1.209902in}{0.814184in}}%
\pgfpathlineto{\pgfqpoint{1.212472in}{0.687834in}}%
\pgfpathlineto{\pgfqpoint{1.217612in}{0.765888in}}%
\pgfpathlineto{\pgfqpoint{1.222746in}{0.685097in}}%
\pgfpathlineto{\pgfqpoint{1.227024in}{0.668147in}}%
\pgfpathlineto{\pgfqpoint{1.231301in}{0.763443in}}%
\pgfpathlineto{\pgfqpoint{1.233868in}{0.676640in}}%
\pgfpathlineto{\pgfqpoint{1.238147in}{0.745915in}}%
\pgfpathlineto{\pgfqpoint{1.242423in}{0.693953in}}%
\pgfpathlineto{\pgfqpoint{1.246701in}{0.774309in}}%
\pgfpathlineto{\pgfqpoint{1.250981in}{0.670561in}}%
\pgfpathlineto{\pgfqpoint{1.258680in}{0.795292in}}%
\pgfpathlineto{\pgfqpoint{1.261242in}{0.712773in}}%
\pgfpathlineto{\pgfqpoint{1.266376in}{0.647456in}}%
\pgfpathlineto{\pgfqpoint{1.269800in}{0.796696in}}%
\pgfpathlineto{\pgfqpoint{1.273224in}{0.704460in}}%
\pgfpathlineto{\pgfqpoint{1.276647in}{0.782986in}}%
\pgfpathlineto{\pgfqpoint{1.281778in}{0.704460in}}%
\pgfpathlineto{\pgfqpoint{1.287770in}{0.812493in}}%
\pgfpathlineto{\pgfqpoint{1.291191in}{0.623130in}}%
\pgfpathlineto{\pgfqpoint{1.297179in}{0.760563in}}%
\pgfpathlineto{\pgfqpoint{1.298889in}{0.654652in}}%
\pgfpathlineto{\pgfqpoint{1.304883in}{0.781366in}}%
\pgfpathlineto{\pgfqpoint{1.310013in}{0.697838in}}%
\pgfpathlineto{\pgfqpoint{1.311723in}{0.737426in}}%
\pgfpathlineto{\pgfqpoint{1.317712in}{0.680924in}}%
\pgfpathlineto{\pgfqpoint{1.321985in}{0.766323in}}%
\pgfpathlineto{\pgfqpoint{1.326257in}{0.674123in}}%
\pgfpathlineto{\pgfqpoint{1.328822in}{0.752648in}}%
\pgfpathlineto{\pgfqpoint{1.335662in}{0.669623in}}%
\pgfpathlineto{\pgfqpoint{1.337375in}{0.759159in}}%
\pgfpathlineto{\pgfqpoint{1.343354in}{0.667609in}}%
\pgfpathlineto{\pgfqpoint{1.345921in}{0.778737in}}%
\pgfpathlineto{\pgfqpoint{1.351911in}{0.654616in}}%
\pgfpathlineto{\pgfqpoint{1.356186in}{0.791909in}}%
\pgfpathlineto{\pgfqpoint{1.359610in}{0.666205in}}%
\pgfpathlineto{\pgfqpoint{1.363031in}{0.745843in}}%
\pgfpathlineto{\pgfqpoint{1.369869in}{0.625827in}}%
\pgfpathlineto{\pgfqpoint{1.372432in}{0.738790in}}%
\pgfpathlineto{\pgfqpoint{1.378419in}{0.783740in}}%
\pgfpathlineto{\pgfqpoint{1.380129in}{0.701724in}}%
\pgfpathlineto{\pgfqpoint{1.386969in}{0.669013in}}%
\pgfpathlineto{\pgfqpoint{1.387825in}{0.714537in}}%
\pgfpathlineto{\pgfqpoint{1.392960in}{0.773950in}}%
\pgfpathlineto{\pgfqpoint{1.397242in}{0.715183in}}%
\pgfpathlineto{\pgfqpoint{1.400665in}{0.797127in}}%
\pgfpathlineto{\pgfqpoint{1.406651in}{0.699745in}}%
\pgfpathlineto{\pgfqpoint{1.410071in}{0.745412in}}%
\pgfpathlineto{\pgfqpoint{1.413491in}{0.618846in}}%
\pgfpathlineto{\pgfqpoint{1.417764in}{0.744658in}}%
\pgfpathlineto{\pgfqpoint{1.424607in}{0.684235in}}%
\pgfpathlineto{\pgfqpoint{1.426320in}{0.741096in}}%
\pgfpathlineto{\pgfqpoint{1.434016in}{0.688228in}}%
\pgfpathlineto{\pgfqpoint{1.437437in}{0.756135in}}%
\pgfpathlineto{\pgfqpoint{1.439148in}{0.658358in}}%
\pgfpathlineto{\pgfqpoint{1.444278in}{0.799249in}}%
\pgfpathlineto{\pgfqpoint{1.448556in}{0.638780in}}%
\pgfpathlineto{\pgfqpoint{1.453691in}{0.747104in}}%
\pgfpathlineto{\pgfqpoint{1.456259in}{0.662965in}}%
\pgfpathlineto{\pgfqpoint{1.462250in}{0.608837in}}%
\pgfpathlineto{\pgfqpoint{1.467386in}{0.743071in}}%
\pgfpathlineto{\pgfqpoint{1.471667in}{0.742855in}}%
\pgfpathlineto{\pgfqpoint{1.476802in}{0.634679in}}%
\pgfpathlineto{\pgfqpoint{1.479368in}{0.746888in}}%
\pgfpathlineto{\pgfqpoint{1.481936in}{0.681786in}}%
\pgfpathlineto{\pgfqpoint{1.487072in}{0.775857in}}%
\pgfpathlineto{\pgfqpoint{1.491345in}{0.636478in}}%
\pgfpathlineto{\pgfqpoint{1.496476in}{0.761568in}}%
\pgfpathlineto{\pgfqpoint{1.500749in}{0.783843in}}%
\pgfpathlineto{\pgfqpoint{1.504171in}{0.652849in}}%
\pgfpathlineto{\pgfqpoint{1.508448in}{0.717808in}}%
\pgfpathlineto{\pgfqpoint{1.511868in}{0.688480in}}%
\pgfpathlineto{\pgfqpoint{1.518707in}{0.768481in}}%
\pgfpathlineto{\pgfqpoint{1.522982in}{0.768230in}}%
\pgfpathlineto{\pgfqpoint{1.527256in}{0.649434in}}%
\pgfpathlineto{\pgfqpoint{1.530678in}{0.746457in}}%
\pgfpathlineto{\pgfqpoint{1.535810in}{0.684235in}}%
\pgfpathlineto{\pgfqpoint{1.538376in}{0.748795in}}%
\pgfpathlineto{\pgfqpoint{1.541797in}{0.681571in}}%
\pgfpathlineto{\pgfqpoint{1.548638in}{0.743218in}}%
\pgfpathlineto{\pgfqpoint{1.550351in}{0.659511in}}%
\pgfpathlineto{\pgfqpoint{1.556343in}{0.605423in}}%
\pgfpathlineto{\pgfqpoint{1.558909in}{0.713998in}}%
\pgfpathlineto{\pgfqpoint{1.565750in}{0.753582in}}%
\pgfpathlineto{\pgfqpoint{1.567459in}{0.664513in}}%
\pgfpathlineto{\pgfqpoint{1.573455in}{0.624064in}}%
\pgfpathlineto{\pgfqpoint{1.576880in}{0.770352in}}%
\pgfpathlineto{\pgfqpoint{1.581159in}{0.662678in}}%
\pgfpathlineto{\pgfqpoint{1.585439in}{0.612583in}}%
\pgfpathlineto{\pgfqpoint{1.588860in}{0.716228in}}%
\pgfpathlineto{\pgfqpoint{1.599126in}{0.650874in}}%
\pgfpathlineto{\pgfqpoint{1.602549in}{0.755672in}}%
\pgfpathlineto{\pgfqpoint{1.607680in}{0.653431in}}%
\pgfpathlineto{\pgfqpoint{1.611102in}{0.789572in}}%
\pgfpathlineto{\pgfqpoint{1.615378in}{0.687834in}}%
\pgfpathlineto{\pgfqpoint{1.619653in}{0.775642in}}%
\pgfpathlineto{\pgfqpoint{1.623070in}{0.699530in}}%
\pgfpathlineto{\pgfqpoint{1.629917in}{0.644073in}}%
\pgfpathlineto{\pgfqpoint{1.634197in}{0.776974in}}%
\pgfpathlineto{\pgfqpoint{1.637617in}{0.680238in}}%
\pgfpathlineto{\pgfqpoint{1.642745in}{0.787481in}}%
\pgfpathlineto{\pgfqpoint{1.647874in}{0.682688in}}%
\pgfpathlineto{\pgfqpoint{1.648727in}{0.761177in}}%
\pgfpathlineto{\pgfqpoint{1.656428in}{0.684092in}}%
\pgfpathlineto{\pgfqpoint{1.660704in}{0.730262in}}%
\pgfpathlineto{\pgfqpoint{1.661558in}{0.661059in}}%
\pgfpathlineto{\pgfqpoint{1.669255in}{0.699171in}}%
\pgfpathlineto{\pgfqpoint{1.671820in}{0.809146in}}%
\pgfpathlineto{\pgfqpoint{1.674387in}{0.678044in}}%
\pgfpathlineto{\pgfqpoint{1.679521in}{0.637236in}}%
\pgfpathlineto{\pgfqpoint{1.683795in}{0.752469in}}%
\pgfpathlineto{\pgfqpoint{1.688070in}{0.652961in}}%
\pgfpathlineto{\pgfqpoint{1.694908in}{0.747933in}}%
\pgfpathlineto{\pgfqpoint{1.696620in}{0.663939in}}%
\pgfpathlineto{\pgfqpoint{1.700041in}{0.780321in}}%
\pgfpathlineto{\pgfqpoint{1.705169in}{0.638999in}}%
\pgfpathlineto{\pgfqpoint{1.711157in}{0.751747in}}%
\pgfpathlineto{\pgfqpoint{1.713721in}{0.666851in}}%
\pgfpathlineto{\pgfqpoint{1.719709in}{0.791909in}}%
\pgfpathlineto{\pgfqpoint{1.721421in}{0.735551in}}%
\pgfpathlineto{\pgfqpoint{1.726553in}{0.682073in}}%
\pgfpathlineto{\pgfqpoint{1.733392in}{0.682759in}}%
\pgfpathlineto{\pgfqpoint{1.736816in}{0.777118in}}%
\pgfpathlineto{\pgfqpoint{1.738528in}{0.710324in}}%
\pgfpathlineto{\pgfqpoint{1.743662in}{0.771685in}}%
\pgfpathlineto{\pgfqpoint{1.747088in}{0.716228in}}%
\pgfpathlineto{\pgfqpoint{1.753081in}{0.665343in}}%
\pgfpathlineto{\pgfqpoint{1.755646in}{0.764448in}}%
\pgfpathlineto{\pgfqpoint{1.759920in}{0.657820in}}%
\pgfpathlineto{\pgfqpoint{1.765049in}{0.724685in}}%
\pgfpathlineto{\pgfqpoint{1.771888in}{0.659726in}}%
\pgfpathlineto{\pgfqpoint{1.774452in}{0.761823in}}%
\pgfpathlineto{\pgfqpoint{1.780436in}{0.641408in}}%
\pgfpathlineto{\pgfqpoint{1.783002in}{0.735120in}}%
\pgfpathlineto{\pgfqpoint{1.787279in}{0.699422in}}%
\pgfpathlineto{\pgfqpoint{1.793271in}{0.782048in}}%
\pgfpathlineto{\pgfqpoint{1.794127in}{0.682544in}}%
\pgfpathlineto{\pgfqpoint{1.800118in}{0.743936in}}%
\pgfpathlineto{\pgfqpoint{1.805246in}{0.686717in}}%
\pgfpathlineto{\pgfqpoint{1.808671in}{0.813897in}}%
\pgfpathlineto{\pgfqpoint{1.814659in}{0.695860in}}%
\pgfpathlineto{\pgfqpoint{1.818933in}{0.787625in}}%
\pgfpathlineto{\pgfqpoint{1.820647in}{0.685600in}}%
\pgfpathlineto{\pgfqpoint{1.825776in}{0.752824in}}%
\pgfpathlineto{\pgfqpoint{1.828343in}{0.692437in}}%
\pgfpathlineto{\pgfqpoint{1.832622in}{0.760994in}}%
\pgfpathlineto{\pgfqpoint{1.836895in}{0.675850in}}%
\pgfpathlineto{\pgfqpoint{1.842028in}{0.777405in}}%
\pgfpathlineto{\pgfqpoint{1.845449in}{0.690243in}}%
\pgfpathlineto{\pgfqpoint{1.851438in}{0.762940in}}%
\pgfpathlineto{\pgfqpoint{1.856567in}{0.642166in}}%
\pgfpathlineto{\pgfqpoint{1.859133in}{0.709103in}}%
\pgfpathlineto{\pgfqpoint{1.863412in}{0.657820in}}%
\pgfpathlineto{\pgfqpoint{1.867689in}{0.773807in}}%
\pgfpathlineto{\pgfqpoint{1.872817in}{0.834984in}}%
\pgfpathlineto{\pgfqpoint{1.877947in}{0.693123in}}%
\pgfpathlineto{\pgfqpoint{1.879659in}{0.760132in}}%
\pgfpathlineto{\pgfqpoint{1.886505in}{0.667681in}}%
\pgfpathlineto{\pgfqpoint{1.890783in}{0.780429in}}%
\pgfpathlineto{\pgfqpoint{1.895915in}{0.648932in}}%
\pgfpathlineto{\pgfqpoint{1.896771in}{0.781043in}}%
\pgfpathlineto{\pgfqpoint{1.901051in}{0.672539in}}%
\pgfpathlineto{\pgfqpoint{1.907894in}{0.726520in}}%
\pgfpathlineto{\pgfqpoint{1.911319in}{0.662319in}}%
\pgfpathlineto{\pgfqpoint{1.915593in}{0.771397in}}%
\pgfpathlineto{\pgfqpoint{1.919018in}{0.696506in}}%
\pgfpathlineto{\pgfqpoint{1.923300in}{0.752716in}}%
\pgfpathlineto{\pgfqpoint{1.930141in}{0.743250in}}%
\pgfpathlineto{\pgfqpoint{1.932706in}{0.684160in}}%
\pgfpathlineto{\pgfqpoint{1.937837in}{0.658035in}}%
\pgfpathlineto{\pgfqpoint{1.942970in}{0.782008in}}%
\pgfpathlineto{\pgfqpoint{1.946392in}{0.698556in}}%
\pgfpathlineto{\pgfqpoint{1.948960in}{0.802273in}}%
\pgfpathlineto{\pgfqpoint{1.953237in}{0.693123in}}%
\pgfpathlineto{\pgfqpoint{1.956661in}{0.764053in}}%
\pgfpathlineto{\pgfqpoint{1.964363in}{0.780716in}}%
\pgfpathlineto{\pgfqpoint{1.965220in}{0.676712in}}%
\pgfpathlineto{\pgfqpoint{1.971208in}{0.629425in}}%
\pgfpathlineto{\pgfqpoint{1.973771in}{0.790864in}}%
\pgfpathlineto{\pgfqpoint{1.980610in}{0.692078in}}%
\pgfpathlineto{\pgfqpoint{1.984031in}{0.758329in}}%
\pgfpathlineto{\pgfqpoint{1.989162in}{0.677394in}}%
\pgfpathlineto{\pgfqpoint{1.992584in}{0.755345in}}%
\pgfpathlineto{\pgfqpoint{1.997716in}{0.783125in}}%
\pgfpathlineto{\pgfqpoint{1.999427in}{0.702657in}}%
\pgfpathlineto{\pgfqpoint{2.003702in}{0.633095in}}%
\pgfpathlineto{\pgfqpoint{2.007981in}{0.786692in}}%
\pgfpathlineto{\pgfqpoint{2.013113in}{0.694021in}}%
\pgfpathlineto{\pgfqpoint{2.019100in}{0.829335in}}%
\pgfpathlineto{\pgfqpoint{2.022522in}{0.658071in}}%
\pgfpathlineto{\pgfqpoint{2.027657in}{0.785144in}}%
\pgfpathlineto{\pgfqpoint{2.031932in}{0.702553in}}%
\pgfpathlineto{\pgfqpoint{2.035355in}{0.800151in}}%
\pgfpathlineto{\pgfqpoint{2.038775in}{0.714860in}}%
\pgfpathlineto{\pgfqpoint{2.043051in}{0.758584in}}%
\pgfpathlineto{\pgfqpoint{2.046473in}{0.666923in}}%
\pgfpathlineto{\pgfqpoint{2.051607in}{0.760276in}}%
\pgfpathlineto{\pgfqpoint{2.055027in}{0.692908in}}%
\pgfpathlineto{\pgfqpoint{2.060159in}{0.680202in}}%
\pgfpathlineto{\pgfqpoint{2.064434in}{0.752034in}}%
\pgfpathlineto{\pgfqpoint{2.068710in}{0.690387in}}%
\pgfpathlineto{\pgfqpoint{2.072986in}{0.766068in}}%
\pgfpathlineto{\pgfqpoint{2.078974in}{0.702801in}}%
\pgfpathlineto{\pgfqpoint{2.080685in}{0.767113in}}%
\pgfpathlineto{\pgfqpoint{2.085812in}{0.714537in}}%
\pgfpathlineto{\pgfqpoint{2.091797in}{0.792268in}}%
\pgfpathlineto{\pgfqpoint{2.095221in}{0.688300in}}%
\pgfpathlineto{\pgfqpoint{2.099502in}{0.689956in}}%
\pgfpathlineto{\pgfqpoint{2.102927in}{0.817136in}}%
\pgfpathlineto{\pgfqpoint{2.107208in}{0.680278in}}%
\pgfpathlineto{\pgfqpoint{2.114054in}{0.662858in}}%
\pgfpathlineto{\pgfqpoint{2.114910in}{0.816777in}}%
\pgfpathlineto{\pgfqpoint{2.121752in}{0.795220in}}%
\pgfpathlineto{\pgfqpoint{2.124318in}{0.693267in}}%
\pgfpathlineto{\pgfqpoint{2.128598in}{0.740554in}}%
\pgfpathlineto{\pgfqpoint{2.135446in}{0.649973in}}%
\pgfpathlineto{\pgfqpoint{2.139720in}{0.837182in}}%
\pgfpathlineto{\pgfqpoint{2.141431in}{0.704029in}}%
\pgfpathlineto{\pgfqpoint{2.146558in}{0.826710in}}%
\pgfpathlineto{\pgfqpoint{2.149976in}{0.672683in}}%
\pgfpathlineto{\pgfqpoint{2.155105in}{0.729903in}}%
\pgfpathlineto{\pgfqpoint{2.157668in}{0.682185in}}%
\pgfpathlineto{\pgfqpoint{2.161947in}{0.633063in}}%
\pgfpathlineto{\pgfqpoint{2.169647in}{0.667322in}}%
\pgfpathlineto{\pgfqpoint{2.171357in}{0.855679in}}%
\pgfpathlineto{\pgfqpoint{2.177349in}{0.715438in}}%
\pgfpathlineto{\pgfqpoint{2.182483in}{0.665163in}}%
\pgfpathlineto{\pgfqpoint{2.184193in}{0.757794in}}%
\pgfpathlineto{\pgfqpoint{2.191037in}{0.689453in}}%
\pgfpathlineto{\pgfqpoint{2.192747in}{0.785575in}}%
\pgfpathlineto{\pgfqpoint{2.199592in}{0.687618in}}%
\pgfpathlineto{\pgfqpoint{2.202160in}{0.752864in}}%
\pgfpathlineto{\pgfqpoint{2.205581in}{0.678156in}}%
\pgfpathlineto{\pgfqpoint{2.211570in}{0.786835in}}%
\pgfpathlineto{\pgfqpoint{2.216701in}{0.780357in}}%
\pgfpathlineto{\pgfqpoint{2.217556in}{0.671315in}}%
\pgfpathlineto{\pgfqpoint{2.223541in}{0.750343in}}%
\pgfpathlineto{\pgfqpoint{2.229530in}{0.657496in}}%
\pgfpathlineto{\pgfqpoint{2.231239in}{0.766969in}}%
\pgfpathlineto{\pgfqpoint{2.238082in}{0.687762in}}%
\pgfpathlineto{\pgfqpoint{2.242359in}{0.800366in}}%
\pgfpathlineto{\pgfqpoint{2.243215in}{0.735914in}}%
\pgfpathlineto{\pgfqpoint{2.250909in}{0.676608in}}%
\pgfpathlineto{\pgfqpoint{2.252617in}{0.767257in}}%
\pgfpathlineto{\pgfqpoint{2.256038in}{0.799576in}}%
\pgfpathlineto{\pgfqpoint{2.261169in}{0.663508in}}%
\pgfpathlineto{\pgfqpoint{2.264591in}{0.712881in}}%
\pgfpathlineto{\pgfqpoint{2.271435in}{0.748041in}}%
\pgfpathlineto{\pgfqpoint{2.273147in}{0.625971in}}%
\pgfpathlineto{\pgfqpoint{2.278277in}{0.774349in}}%
\pgfpathlineto{\pgfqpoint{2.281703in}{0.776471in}}%
\pgfpathlineto{\pgfqpoint{2.291118in}{0.632521in}}%
\pgfpathlineto{\pgfqpoint{2.294543in}{0.795364in}}%
\pgfpathlineto{\pgfqpoint{2.298819in}{0.706223in}}%
\pgfpathlineto{\pgfqpoint{2.304810in}{0.704855in}}%
\pgfpathlineto{\pgfqpoint{2.309087in}{0.840636in}}%
\pgfpathlineto{\pgfqpoint{2.312504in}{0.661777in}}%
\pgfpathlineto{\pgfqpoint{2.316782in}{0.766251in}}%
\pgfpathlineto{\pgfqpoint{2.323623in}{0.632162in}}%
\pgfpathlineto{\pgfqpoint{2.327045in}{0.717161in}}%
\pgfpathlineto{\pgfqpoint{2.330467in}{0.651453in}}%
\pgfpathlineto{\pgfqpoint{2.333891in}{0.631874in}}%
\pgfpathlineto{\pgfqpoint{2.341587in}{0.802488in}}%
\pgfpathlineto{\pgfqpoint{2.346717in}{0.651848in}}%
\pgfpathlineto{\pgfqpoint{2.351849in}{0.758692in}}%
\pgfpathlineto{\pgfqpoint{2.357839in}{0.687834in}}%
\pgfpathlineto{\pgfqpoint{2.359552in}{0.769451in}}%
\pgfpathlineto{\pgfqpoint{2.364686in}{0.668830in}}%
\pgfpathlineto{\pgfqpoint{2.368106in}{0.805835in}}%
\pgfpathlineto{\pgfqpoint{2.371529in}{0.712518in}}%
\pgfpathlineto{\pgfqpoint{2.376661in}{0.770280in}}%
\pgfpathlineto{\pgfqpoint{2.380084in}{0.732172in}}%
\pgfpathlineto{\pgfqpoint{2.385215in}{0.650982in}}%
\pgfpathlineto{\pgfqpoint{2.391202in}{0.772115in}}%
\pgfpathlineto{\pgfqpoint{2.394625in}{0.667824in}}%
\pgfpathlineto{\pgfqpoint{2.397190in}{0.747574in}}%
\pgfpathlineto{\pgfqpoint{2.403176in}{0.689956in}}%
\pgfpathlineto{\pgfqpoint{2.408306in}{0.752214in}}%
\pgfpathlineto{\pgfqpoint{2.410871in}{0.679592in}}%
\pgfpathlineto{\pgfqpoint{2.415143in}{0.774453in}}%
\pgfpathlineto{\pgfqpoint{2.420276in}{0.689525in}}%
\pgfpathlineto{\pgfqpoint{2.425405in}{0.784713in}}%
\pgfpathlineto{\pgfqpoint{2.429680in}{0.618415in}}%
\pgfpathlineto{\pgfqpoint{2.432245in}{0.731522in}}%
\pgfpathlineto{\pgfqpoint{2.435666in}{0.790505in}}%
\pgfpathlineto{\pgfqpoint{2.442507in}{0.783125in}}%
\pgfpathlineto{\pgfqpoint{2.445931in}{0.687977in}}%
\pgfpathlineto{\pgfqpoint{2.451058in}{0.642597in}}%
\pgfpathlineto{\pgfqpoint{2.452770in}{0.721482in}}%
\pgfpathlineto{\pgfqpoint{2.457051in}{0.773232in}}%
\pgfpathlineto{\pgfqpoint{2.461330in}{0.706367in}}%
\pgfpathlineto{\pgfqpoint{2.465606in}{0.780285in}}%
\pgfpathlineto{\pgfqpoint{2.472449in}{0.676209in}}%
\pgfpathlineto{\pgfqpoint{2.475873in}{0.767185in}}%
\pgfpathlineto{\pgfqpoint{2.481860in}{0.639143in}}%
\pgfpathlineto{\pgfqpoint{2.482715in}{0.745556in}}%
\pgfpathlineto{\pgfqpoint{2.489559in}{0.634930in}}%
\pgfpathlineto{\pgfqpoint{2.492125in}{0.743035in}}%
\pgfpathlineto{\pgfqpoint{2.497259in}{0.660301in}}%
\pgfpathlineto{\pgfqpoint{2.502393in}{0.760132in}}%
\pgfpathlineto{\pgfqpoint{2.504105in}{0.690243in}}%
\pgfpathlineto{\pgfqpoint{2.511803in}{0.801623in}}%
\pgfpathlineto{\pgfqpoint{2.514369in}{0.667210in}}%
\pgfpathlineto{\pgfqpoint{2.518648in}{0.649937in}}%
\pgfpathlineto{\pgfqpoint{2.522070in}{0.771286in}}%
\pgfpathlineto{\pgfqpoint{2.528054in}{0.665519in}}%
\pgfpathlineto{\pgfqpoint{2.529766in}{0.731841in}}%
\pgfpathlineto{\pgfqpoint{2.536611in}{0.687219in}}%
\pgfpathlineto{\pgfqpoint{2.540888in}{0.762470in}}%
\pgfpathlineto{\pgfqpoint{2.542599in}{0.654939in}}%
\pgfpathlineto{\pgfqpoint{2.546873in}{0.765960in}}%
\pgfpathlineto{\pgfqpoint{2.552006in}{0.684666in}}%
\pgfpathlineto{\pgfqpoint{2.556277in}{0.769379in}}%
\pgfpathlineto{\pgfqpoint{2.560553in}{0.641624in}}%
\pgfpathlineto{\pgfqpoint{2.563974in}{0.683118in}}%
\pgfpathlineto{\pgfqpoint{2.569104in}{0.766642in}}%
\pgfpathlineto{\pgfqpoint{2.572527in}{0.694886in}}%
\pgfpathlineto{\pgfqpoint{2.576804in}{0.769092in}}%
\pgfpathlineto{\pgfqpoint{2.584495in}{0.663648in}}%
\pgfpathlineto{\pgfqpoint{2.585351in}{0.764592in}}%
\pgfpathlineto{\pgfqpoint{2.596466in}{0.679879in}}%
\pgfpathlineto{\pgfqpoint{2.598175in}{0.752393in}}%
\pgfpathlineto{\pgfqpoint{2.602452in}{0.687794in}}%
\pgfpathlineto{\pgfqpoint{2.609301in}{0.653463in}}%
\pgfpathlineto{\pgfqpoint{2.614436in}{0.824405in}}%
\pgfpathlineto{\pgfqpoint{2.615290in}{0.727094in}}%
\pgfpathlineto{\pgfqpoint{2.621275in}{0.672467in}}%
\pgfpathlineto{\pgfqpoint{2.627264in}{0.770137in}}%
\pgfpathlineto{\pgfqpoint{2.631542in}{0.663289in}}%
\pgfpathlineto{\pgfqpoint{2.635823in}{0.820878in}}%
\pgfpathlineto{\pgfqpoint{2.637532in}{0.686214in}}%
\pgfpathlineto{\pgfqpoint{2.642664in}{0.744152in}}%
\pgfpathlineto{\pgfqpoint{2.646085in}{0.669300in}}%
\pgfpathlineto{\pgfqpoint{2.651221in}{0.773192in}}%
\pgfpathlineto{\pgfqpoint{2.655500in}{0.705322in}}%
\pgfpathlineto{\pgfqpoint{2.658066in}{0.805656in}}%
\pgfpathlineto{\pgfqpoint{2.663199in}{0.710037in}}%
\pgfpathlineto{\pgfqpoint{2.668333in}{0.807419in}}%
\pgfpathlineto{\pgfqpoint{2.671754in}{0.683729in}}%
\pgfpathlineto{\pgfqpoint{2.675176in}{0.733932in}}%
\pgfpathlineto{\pgfqpoint{2.679450in}{0.688623in}}%
\pgfpathlineto{\pgfqpoint{2.685443in}{0.762075in}}%
\pgfpathlineto{\pgfqpoint{2.688863in}{0.686717in}}%
\pgfpathlineto{\pgfqpoint{2.693143in}{0.793816in}}%
\pgfpathlineto{\pgfqpoint{2.696565in}{0.716874in}}%
\pgfpathlineto{\pgfqpoint{2.702555in}{0.751420in}}%
\pgfpathlineto{\pgfqpoint{2.706832in}{0.678116in}}%
\pgfpathlineto{\pgfqpoint{2.711109in}{0.791439in}}%
\pgfpathlineto{\pgfqpoint{2.713672in}{0.689884in}}%
\pgfpathlineto{\pgfqpoint{2.719654in}{0.814831in}}%
\pgfpathlineto{\pgfqpoint{2.722221in}{0.722236in}}%
\pgfpathlineto{\pgfqpoint{2.728213in}{0.764017in}}%
\pgfpathlineto{\pgfqpoint{2.734203in}{0.797773in}}%
\pgfpathlineto{\pgfqpoint{2.737626in}{0.648318in}}%
\pgfpathlineto{\pgfqpoint{2.739338in}{0.734833in}}%
\pgfpathlineto{\pgfqpoint{2.746179in}{0.771397in}}%
\pgfpathlineto{\pgfqpoint{2.748744in}{0.683047in}}%
\pgfpathlineto{\pgfqpoint{2.754727in}{0.760922in}}%
\pgfpathlineto{\pgfqpoint{2.758145in}{0.788056in}}%
\pgfpathlineto{\pgfqpoint{2.760706in}{0.699781in}}%
\pgfpathlineto{\pgfqpoint{2.764980in}{0.783197in}}%
\pgfpathlineto{\pgfqpoint{2.769259in}{0.700535in}}%
\pgfpathlineto{\pgfqpoint{2.776955in}{0.686102in}}%
\pgfpathlineto{\pgfqpoint{2.778666in}{0.787083in}}%
\pgfpathlineto{\pgfqpoint{2.782089in}{0.699853in}}%
\pgfpathlineto{\pgfqpoint{2.789789in}{0.624351in}}%
\pgfpathlineto{\pgfqpoint{2.790642in}{0.752106in}}%
\pgfpathlineto{\pgfqpoint{2.794918in}{0.702410in}}%
\pgfpathlineto{\pgfqpoint{2.799189in}{0.795795in}}%
\pgfpathlineto{\pgfqpoint{2.803464in}{0.690606in}}%
\pgfpathlineto{\pgfqpoint{2.809450in}{0.775714in}}%
\pgfpathlineto{\pgfqpoint{2.813729in}{0.691033in}}%
\pgfpathlineto{\pgfqpoint{2.818861in}{0.662463in}}%
\pgfpathlineto{\pgfqpoint{2.820569in}{0.736596in}}%
\pgfpathlineto{\pgfqpoint{2.827409in}{0.652961in}}%
\pgfpathlineto{\pgfqpoint{2.832540in}{0.745340in}}%
\pgfpathlineto{\pgfqpoint{2.833396in}{0.682217in}}%
\pgfpathlineto{\pgfqpoint{2.839382in}{0.760994in}}%
\pgfpathlineto{\pgfqpoint{2.845369in}{0.678870in}}%
\pgfpathlineto{\pgfqpoint{2.847078in}{0.750774in}}%
\pgfpathlineto{\pgfqpoint{2.850497in}{0.682037in}}%
\pgfpathlineto{\pgfqpoint{2.855627in}{0.772474in}}%
\pgfpathlineto{\pgfqpoint{2.859900in}{0.649470in}}%
\pgfpathlineto{\pgfqpoint{2.865887in}{0.776862in}}%
\pgfpathlineto{\pgfqpoint{2.867599in}{0.729719in}}%
\pgfpathlineto{\pgfqpoint{2.874440in}{0.630686in}}%
\pgfpathlineto{\pgfqpoint{2.876150in}{0.725981in}}%
\pgfpathlineto{\pgfqpoint{2.880425in}{0.674949in}}%
\pgfpathlineto{\pgfqpoint{2.884702in}{0.770424in}}%
\pgfpathlineto{\pgfqpoint{2.891547in}{0.788096in}}%
\pgfpathlineto{\pgfqpoint{2.896681in}{0.694886in}}%
\pgfpathlineto{\pgfqpoint{2.897536in}{0.795723in}}%
\pgfpathlineto{\pgfqpoint{2.905233in}{0.795005in}}%
\pgfpathlineto{\pgfqpoint{2.906944in}{0.709750in}}%
\pgfpathlineto{\pgfqpoint{2.910364in}{0.751962in}}%
\pgfpathlineto{\pgfqpoint{2.918060in}{0.706151in}}%
\pgfpathlineto{\pgfqpoint{2.918914in}{0.776288in}}%
\pgfpathlineto{\pgfqpoint{2.927463in}{0.665231in}}%
\pgfpathlineto{\pgfqpoint{2.931738in}{0.760308in}}%
\pgfpathlineto{\pgfqpoint{2.936873in}{0.694487in}}%
\pgfpathlineto{\pgfqpoint{2.942857in}{0.685528in}}%
\pgfpathlineto{\pgfqpoint{2.944568in}{0.778374in}}%
\pgfpathlineto{\pgfqpoint{2.948846in}{0.697475in}}%
\pgfpathlineto{\pgfqpoint{2.955685in}{0.759949in}}%
\pgfpathlineto{\pgfqpoint{2.959964in}{0.606934in}}%
\pgfpathlineto{\pgfqpoint{2.964240in}{0.791510in}}%
\pgfpathlineto{\pgfqpoint{2.968519in}{0.842866in}}%
\pgfpathlineto{\pgfqpoint{2.970229in}{0.708166in}}%
\pgfpathlineto{\pgfqpoint{2.977074in}{0.701077in}}%
\pgfpathlineto{\pgfqpoint{2.978784in}{0.764089in}}%
\pgfpathlineto{\pgfqpoint{2.983062in}{0.697080in}}%
\pgfpathlineto{\pgfqpoint{2.988194in}{0.750630in}}%
\pgfpathlineto{\pgfqpoint{2.994182in}{0.699745in}}%
\pgfpathlineto{\pgfqpoint{2.999311in}{0.785000in}}%
\pgfpathlineto{\pgfqpoint{3.002736in}{0.701149in}}%
\pgfpathlineto{\pgfqpoint{3.007011in}{0.815301in}}%
\pgfpathlineto{\pgfqpoint{3.011287in}{0.686717in}}%
\pgfpathlineto{\pgfqpoint{3.013853in}{0.774995in}}%
\pgfpathlineto{\pgfqpoint{3.019839in}{0.792232in}}%
\pgfpathlineto{\pgfqpoint{3.021551in}{0.717560in}}%
\pgfpathlineto{\pgfqpoint{3.029249in}{0.612695in}}%
\pgfpathlineto{\pgfqpoint{3.030958in}{0.823686in}}%
\pgfpathlineto{\pgfqpoint{3.034379in}{0.741958in}}%
\pgfpathlineto{\pgfqpoint{3.039507in}{0.806701in}}%
\pgfpathlineto{\pgfqpoint{3.042930in}{0.716192in}}%
\pgfpathlineto{\pgfqpoint{3.050630in}{0.662535in}}%
\pgfpathlineto{\pgfqpoint{3.054053in}{0.812421in}}%
\pgfpathlineto{\pgfqpoint{3.059185in}{0.808895in}}%
\pgfpathlineto{\pgfqpoint{3.061754in}{0.684558in}}%
\pgfpathlineto{\pgfqpoint{3.064319in}{0.744371in}}%
\pgfpathlineto{\pgfqpoint{3.072018in}{0.681714in}}%
\pgfpathlineto{\pgfqpoint{3.072875in}{0.723424in}}%
\pgfpathlineto{\pgfqpoint{3.079716in}{0.757324in}}%
\pgfpathlineto{\pgfqpoint{3.084849in}{0.664370in}}%
\pgfpathlineto{\pgfqpoint{3.085704in}{0.774457in}}%
\pgfpathlineto{\pgfqpoint{3.090835in}{0.706511in}}%
\pgfpathlineto{\pgfqpoint{3.094255in}{0.765063in}}%
\pgfpathlineto{\pgfqpoint{3.099388in}{0.657923in}}%
\pgfpathlineto{\pgfqpoint{3.103662in}{0.748037in}}%
\pgfpathlineto{\pgfqpoint{3.110508in}{0.756638in}}%
\pgfpathlineto{\pgfqpoint{3.112219in}{0.677358in}}%
\pgfpathlineto{\pgfqpoint{3.117359in}{0.767903in}}%
\pgfpathlineto{\pgfqpoint{3.122495in}{0.755704in}}%
\pgfpathlineto{\pgfqpoint{3.126770in}{0.632521in}}%
\pgfpathlineto{\pgfqpoint{3.128480in}{0.722092in}}%
\pgfpathlineto{\pgfqpoint{3.135323in}{0.804938in}}%
\pgfpathlineto{\pgfqpoint{3.138744in}{0.671494in}}%
\pgfpathlineto{\pgfqpoint{3.143019in}{0.741886in}}%
\pgfpathlineto{\pgfqpoint{3.146436in}{0.685280in}}%
\pgfpathlineto{\pgfqpoint{3.151568in}{0.748687in}}%
\pgfpathlineto{\pgfqpoint{3.154987in}{0.646195in}}%
\pgfpathlineto{\pgfqpoint{3.161832in}{0.767296in}}%
\pgfpathlineto{\pgfqpoint{3.162686in}{0.725730in}}%
\pgfpathlineto{\pgfqpoint{3.166956in}{0.775570in}}%
\pgfpathlineto{\pgfqpoint{3.171229in}{0.657460in}}%
\pgfpathlineto{\pgfqpoint{3.176360in}{0.781761in}}%
\pgfpathlineto{\pgfqpoint{3.181495in}{0.704676in}}%
\pgfpathlineto{\pgfqpoint{3.184919in}{0.785216in}}%
\pgfpathlineto{\pgfqpoint{3.189194in}{0.692872in}}%
\pgfpathlineto{\pgfqpoint{3.195182in}{0.771469in}}%
\pgfpathlineto{\pgfqpoint{3.196893in}{0.685169in}}%
\pgfpathlineto{\pgfqpoint{3.202029in}{0.784242in}}%
\pgfpathlineto{\pgfqpoint{3.206306in}{0.793457in}}%
\pgfpathlineto{\pgfqpoint{3.213149in}{0.594951in}}%
\pgfpathlineto{\pgfqpoint{3.214004in}{0.723640in}}%
\pgfpathlineto{\pgfqpoint{3.218280in}{0.756031in}}%
\pgfpathlineto{\pgfqpoint{3.222557in}{0.645908in}}%
\pgfpathlineto{\pgfqpoint{3.226834in}{0.724505in}}%
\pgfpathlineto{\pgfqpoint{3.234531in}{0.769993in}}%
\pgfpathlineto{\pgfqpoint{3.237098in}{0.678188in}}%
\pgfpathlineto{\pgfqpoint{3.243083in}{0.693845in}}%
\pgfpathlineto{\pgfqpoint{3.243939in}{0.791622in}}%
\pgfpathlineto{\pgfqpoint{3.252486in}{0.637272in}}%
\pgfpathlineto{\pgfqpoint{3.256765in}{0.763443in}}%
\pgfpathlineto{\pgfqpoint{3.263612in}{0.675922in}}%
\pgfpathlineto{\pgfqpoint{3.266171in}{0.816091in}}%
\pgfpathlineto{\pgfqpoint{3.269594in}{0.710795in}}%
\pgfpathlineto{\pgfqpoint{3.274725in}{0.783740in}}%
\pgfpathlineto{\pgfqpoint{3.278148in}{0.675204in}}%
\pgfpathlineto{\pgfqpoint{3.283283in}{0.783528in}}%
\pgfpathlineto{\pgfqpoint{3.287561in}{0.672288in}}%
\pgfpathlineto{\pgfqpoint{3.291835in}{0.783524in}}%
\pgfpathlineto{\pgfqpoint{3.296964in}{0.670956in}}%
\pgfpathlineto{\pgfqpoint{3.302092in}{0.675635in}}%
\pgfpathlineto{\pgfqpoint{3.306369in}{0.761285in}}%
\pgfpathlineto{\pgfqpoint{3.311495in}{0.782447in}}%
\pgfpathlineto{\pgfqpoint{3.313205in}{0.636446in}}%
\pgfpathlineto{\pgfqpoint{3.316624in}{0.736596in}}%
\pgfpathlineto{\pgfqpoint{3.323472in}{0.786261in}}%
\pgfpathlineto{\pgfqpoint{3.327748in}{0.635688in}}%
\pgfpathlineto{\pgfqpoint{3.332026in}{0.770639in}}%
\pgfpathlineto{\pgfqpoint{3.333737in}{0.710938in}}%
\pgfpathlineto{\pgfqpoint{3.339724in}{0.635113in}}%
\pgfpathlineto{\pgfqpoint{3.345713in}{0.773304in}}%
\pgfpathlineto{\pgfqpoint{3.346570in}{0.694958in}}%
\pgfpathlineto{\pgfqpoint{3.351703in}{0.656383in}}%
\pgfpathlineto{\pgfqpoint{3.355978in}{0.781258in}}%
\pgfpathlineto{\pgfqpoint{3.359402in}{0.689852in}}%
\pgfpathlineto{\pgfqpoint{3.367101in}{0.683445in}}%
\pgfpathlineto{\pgfqpoint{3.367957in}{0.817894in}}%
\pgfpathlineto{\pgfqpoint{3.374800in}{0.665307in}}%
\pgfpathlineto{\pgfqpoint{3.379076in}{0.631443in}}%
\pgfpathlineto{\pgfqpoint{3.380784in}{0.727350in}}%
\pgfpathlineto{\pgfqpoint{3.388483in}{0.775610in}}%
\pgfpathlineto{\pgfqpoint{3.389340in}{0.657931in}}%
\pgfpathlineto{\pgfqpoint{3.397035in}{0.801052in}}%
\pgfpathlineto{\pgfqpoint{3.398746in}{0.670704in}}%
\pgfpathlineto{\pgfqpoint{3.403879in}{0.819298in}}%
\pgfpathlineto{\pgfqpoint{3.407299in}{0.676249in}}%
\pgfpathlineto{\pgfqpoint{3.410720in}{0.742644in}}%
\pgfpathlineto{\pgfqpoint{3.414994in}{0.655231in}}%
\pgfpathlineto{\pgfqpoint{3.419266in}{0.713962in}}%
\pgfpathlineto{\pgfqpoint{3.423538in}{0.689924in}}%
\pgfpathlineto{\pgfqpoint{3.430381in}{0.797454in}}%
\pgfpathlineto{\pgfqpoint{3.433801in}{0.584408in}}%
\pgfpathlineto{\pgfqpoint{3.437223in}{0.753909in}}%
\pgfpathlineto{\pgfqpoint{3.443211in}{0.684742in}}%
\pgfpathlineto{\pgfqpoint{3.444923in}{0.741495in}}%
\pgfpathlineto{\pgfqpoint{3.450907in}{0.805911in}}%
\pgfpathlineto{\pgfqpoint{3.456036in}{0.709678in}}%
\pgfpathlineto{\pgfqpoint{3.457749in}{0.779423in}}%
\pgfpathlineto{\pgfqpoint{3.462026in}{0.621076in}}%
\pgfpathlineto{\pgfqpoint{3.466302in}{0.748400in}}%
\pgfpathlineto{\pgfqpoint{3.470577in}{0.774349in}}%
\pgfpathlineto{\pgfqpoint{3.475705in}{0.681319in}}%
\pgfpathlineto{\pgfqpoint{3.480840in}{0.739548in}}%
\pgfpathlineto{\pgfqpoint{3.483407in}{0.658793in}}%
\pgfpathlineto{\pgfqpoint{3.490249in}{0.762761in}}%
\pgfpathlineto{\pgfqpoint{3.493668in}{0.671319in}}%
\pgfpathlineto{\pgfqpoint{3.498802in}{0.738001in}}%
\pgfpathlineto{\pgfqpoint{3.503934in}{0.596786in}}%
\pgfpathlineto{\pgfqpoint{3.507357in}{0.775785in}}%
\pgfpathlineto{\pgfqpoint{3.509065in}{0.654616in}}%
\pgfpathlineto{\pgfqpoint{3.514195in}{0.793888in}}%
\pgfpathlineto{\pgfqpoint{3.517615in}{0.669300in}}%
\pgfpathlineto{\pgfqpoint{3.522750in}{0.736632in}}%
\pgfpathlineto{\pgfqpoint{3.527884in}{0.657030in}}%
\pgfpathlineto{\pgfqpoint{3.533869in}{0.743721in}}%
\pgfpathlineto{\pgfqpoint{3.536436in}{0.670848in}}%
\pgfpathlineto{\pgfqpoint{3.539857in}{0.778091in}}%
\pgfpathlineto{\pgfqpoint{3.543277in}{0.640403in}}%
\pgfpathlineto{\pgfqpoint{3.548412in}{0.744335in}}%
\pgfpathlineto{\pgfqpoint{3.552684in}{0.678443in}}%
\pgfpathlineto{\pgfqpoint{3.556103in}{0.759849in}}%
\pgfpathlineto{\pgfqpoint{3.561230in}{0.702449in}}%
\pgfpathlineto{\pgfqpoint{3.567217in}{0.682871in}}%
\pgfpathlineto{\pgfqpoint{3.569782in}{0.807925in}}%
\pgfpathlineto{\pgfqpoint{3.574058in}{0.695788in}}%
\pgfpathlineto{\pgfqpoint{3.580044in}{0.805369in}}%
\pgfpathlineto{\pgfqpoint{3.581755in}{0.681032in}}%
\pgfpathlineto{\pgfqpoint{3.586031in}{0.753367in}}%
\pgfpathlineto{\pgfqpoint{3.592017in}{0.673010in}}%
\pgfpathlineto{\pgfqpoint{3.595438in}{0.810913in}}%
\pgfpathlineto{\pgfqpoint{3.598858in}{0.697407in}}%
\pgfpathlineto{\pgfqpoint{3.605706in}{0.610357in}}%
\pgfpathlineto{\pgfqpoint{3.607419in}{0.749122in}}%
\pgfpathlineto{\pgfqpoint{3.611692in}{0.689924in}}%
\pgfpathlineto{\pgfqpoint{3.615968in}{0.734043in}}%
\pgfpathlineto{\pgfqpoint{3.621959in}{0.679628in}}%
\pgfpathlineto{\pgfqpoint{3.627946in}{0.774924in}}%
\pgfpathlineto{\pgfqpoint{3.628798in}{0.694779in}}%
\pgfpathlineto{\pgfqpoint{3.634787in}{0.732783in}}%
\pgfpathlineto{\pgfqpoint{3.640777in}{0.612080in}}%
\pgfpathlineto{\pgfqpoint{3.641632in}{0.740809in}}%
\pgfpathlineto{\pgfqpoint{3.646770in}{0.660843in}}%
\pgfpathlineto{\pgfqpoint{3.652758in}{0.628563in}}%
\pgfpathlineto{\pgfqpoint{3.657887in}{0.761249in}}%
\pgfpathlineto{\pgfqpoint{3.658742in}{0.670704in}}%
\pgfpathlineto{\pgfqpoint{3.663017in}{0.779998in}}%
\pgfpathlineto{\pgfqpoint{3.669860in}{0.675348in}}%
\pgfpathlineto{\pgfqpoint{3.671571in}{0.754771in}}%
\pgfpathlineto{\pgfqpoint{3.678414in}{0.785866in}}%
\pgfpathlineto{\pgfqpoint{3.680124in}{0.670776in}}%
\pgfpathlineto{\pgfqpoint{3.686109in}{0.758297in}}%
\pgfpathlineto{\pgfqpoint{3.689529in}{0.771972in}}%
\pgfpathlineto{\pgfqpoint{3.695519in}{0.701795in}}%
\pgfpathlineto{\pgfqpoint{3.699794in}{0.739692in}}%
\pgfpathlineto{\pgfqpoint{3.701506in}{0.645800in}}%
\pgfpathlineto{\pgfqpoint{3.705784in}{0.725658in}}%
\pgfpathlineto{\pgfqpoint{3.710060in}{0.654365in}}%
\pgfpathlineto{\pgfqpoint{3.714341in}{0.739907in}}%
\pgfpathlineto{\pgfqpoint{3.720332in}{0.673548in}}%
\pgfpathlineto{\pgfqpoint{3.722900in}{0.724003in}}%
\pgfpathlineto{\pgfqpoint{3.730597in}{0.792811in}}%
\pgfpathlineto{\pgfqpoint{3.731453in}{0.676608in}}%
\pgfpathlineto{\pgfqpoint{3.738291in}{0.796768in}}%
\pgfpathlineto{\pgfqpoint{3.739998in}{0.718207in}}%
\pgfpathlineto{\pgfqpoint{3.745129in}{0.684164in}}%
\pgfpathlineto{\pgfqpoint{3.751968in}{0.780612in}}%
\pgfpathlineto{\pgfqpoint{3.756244in}{0.658003in}}%
\pgfpathlineto{\pgfqpoint{3.757100in}{0.755888in}}%
\pgfpathlineto{\pgfqpoint{3.763091in}{0.648537in}}%
\pgfpathlineto{\pgfqpoint{3.767366in}{0.748831in}}%
\pgfpathlineto{\pgfqpoint{3.769934in}{0.675958in}}%
\pgfpathlineto{\pgfqpoint{3.777634in}{0.611865in}}%
\pgfpathlineto{\pgfqpoint{3.779343in}{0.724577in}}%
\pgfpathlineto{\pgfqpoint{3.785331in}{0.754914in}}%
\pgfpathlineto{\pgfqpoint{3.790465in}{0.691687in}}%
\pgfpathlineto{\pgfqpoint{3.792176in}{0.768844in}}%
\pgfpathlineto{\pgfqpoint{3.795597in}{0.696725in}}%
\pgfpathlineto{\pgfqpoint{3.801584in}{0.786907in}}%
\pgfpathlineto{\pgfqpoint{3.804147in}{0.714249in}}%
\pgfpathlineto{\pgfqpoint{3.809278in}{0.671534in}}%
\pgfpathlineto{\pgfqpoint{3.816117in}{0.765102in}}%
\pgfpathlineto{\pgfqpoint{3.819537in}{0.660883in}}%
\pgfpathlineto{\pgfqpoint{3.821247in}{0.745811in}}%
\pgfpathlineto{\pgfqpoint{3.825520in}{0.673010in}}%
\pgfpathlineto{\pgfqpoint{3.829795in}{0.658721in}}%
\pgfpathlineto{\pgfqpoint{3.834067in}{0.754124in}}%
\pgfpathlineto{\pgfqpoint{3.841763in}{0.780971in}}%
\pgfpathlineto{\pgfqpoint{3.842619in}{0.688053in}}%
\pgfpathlineto{\pgfqpoint{3.847744in}{0.646774in}}%
\pgfpathlineto{\pgfqpoint{3.851170in}{0.748907in}}%
\pgfpathlineto{\pgfqpoint{3.857163in}{0.652570in}}%
\pgfpathlineto{\pgfqpoint{3.859732in}{0.734905in}}%
\pgfpathlineto{\pgfqpoint{3.866582in}{0.675419in}}%
\pgfpathlineto{\pgfqpoint{3.870855in}{0.748260in}}%
\pgfpathlineto{\pgfqpoint{3.873418in}{0.654405in}}%
\pgfpathlineto{\pgfqpoint{3.880263in}{0.666496in}}%
\pgfpathlineto{\pgfqpoint{3.884540in}{0.750957in}}%
\pgfpathlineto{\pgfqpoint{3.885396in}{0.658577in}}%
\pgfpathlineto{\pgfqpoint{3.891380in}{0.763914in}}%
\pgfpathlineto{\pgfqpoint{3.894800in}{0.663041in}}%
\pgfpathlineto{\pgfqpoint{3.899075in}{0.750203in}}%
\pgfpathlineto{\pgfqpoint{3.905916in}{0.785184in}}%
\pgfpathlineto{\pgfqpoint{3.908484in}{0.659335in}}%
\pgfpathlineto{\pgfqpoint{3.912763in}{0.742827in}}%
\pgfpathlineto{\pgfqpoint{3.917037in}{0.775035in}}%
\pgfpathlineto{\pgfqpoint{3.923020in}{0.692912in}}%
\pgfpathlineto{\pgfqpoint{3.924731in}{0.770320in}}%
\pgfpathlineto{\pgfqpoint{3.929865in}{0.648900in}}%
\pgfpathlineto{\pgfqpoint{3.935852in}{0.796808in}}%
\pgfpathlineto{\pgfqpoint{3.939274in}{0.707990in}}%
\pgfpathlineto{\pgfqpoint{3.943555in}{0.746425in}}%
\pgfpathlineto{\pgfqpoint{3.947833in}{0.630725in}}%
\pgfpathlineto{\pgfqpoint{3.952966in}{0.727421in}}%
\pgfpathlineto{\pgfqpoint{3.953820in}{0.639972in}}%
\pgfpathlineto{\pgfqpoint{3.958954in}{0.771254in}}%
\pgfpathlineto{\pgfqpoint{3.962375in}{0.664840in}}%
\pgfpathlineto{\pgfqpoint{3.969212in}{0.666388in}}%
\pgfpathlineto{\pgfqpoint{3.970924in}{0.736923in}}%
\pgfpathlineto{\pgfqpoint{3.976053in}{0.691001in}}%
\pgfpathlineto{\pgfqpoint{3.980331in}{0.749553in}}%
\pgfpathlineto{\pgfqpoint{3.985468in}{0.704500in}}%
\pgfpathlineto{\pgfqpoint{3.991456in}{0.774604in}}%
\pgfpathlineto{\pgfqpoint{3.993166in}{0.679596in}}%
\pgfpathlineto{\pgfqpoint{3.996588in}{0.753550in}}%
\pgfpathlineto{\pgfqpoint{4.004292in}{0.684060in}}%
\pgfpathlineto{\pgfqpoint{4.006858in}{0.751101in}}%
\pgfpathlineto{\pgfqpoint{4.011132in}{0.782734in}}%
\pgfpathlineto{\pgfqpoint{4.017123in}{0.645585in}}%
\pgfpathlineto{\pgfqpoint{4.017979in}{0.772730in}}%
\pgfpathlineto{\pgfqpoint{4.022252in}{0.689238in}}%
\pgfpathlineto{\pgfqpoint{4.027385in}{0.761967in}}%
\pgfpathlineto{\pgfqpoint{4.030804in}{0.671929in}}%
\pgfpathlineto{\pgfqpoint{4.039357in}{0.759270in}}%
\pgfpathlineto{\pgfqpoint{4.044492in}{0.688807in}}%
\pgfpathlineto{\pgfqpoint{4.049623in}{0.652315in}}%
\pgfpathlineto{\pgfqpoint{4.054759in}{0.821524in}}%
\pgfpathlineto{\pgfqpoint{4.058181in}{0.706008in}}%
\pgfpathlineto{\pgfqpoint{4.064169in}{0.762474in}}%
\pgfpathlineto{\pgfqpoint{4.068446in}{0.609061in}}%
\pgfpathlineto{\pgfqpoint{4.069301in}{0.725156in}}%
\pgfpathlineto{\pgfqpoint{4.073578in}{0.671534in}}%
\pgfpathlineto{\pgfqpoint{4.080418in}{0.763770in}}%
\pgfpathlineto{\pgfqpoint{4.082985in}{0.689708in}}%
\pgfpathlineto{\pgfqpoint{4.086408in}{0.765246in}}%
\pgfpathlineto{\pgfqpoint{4.091542in}{0.666460in}}%
\pgfpathlineto{\pgfqpoint{4.098381in}{0.782375in}}%
\pgfpathlineto{\pgfqpoint{4.100093in}{0.707484in}}%
\pgfpathlineto{\pgfqpoint{4.106080in}{0.769993in}}%
\pgfpathlineto{\pgfqpoint{4.108646in}{0.672467in}}%
\pgfpathlineto{\pgfqpoint{4.114636in}{0.742572in}}%
\pgfpathlineto{\pgfqpoint{4.118056in}{0.673836in}}%
\pgfpathlineto{\pgfqpoint{4.122329in}{0.725012in}}%
\pgfpathlineto{\pgfqpoint{4.127462in}{0.637275in}}%
\pgfpathlineto{\pgfqpoint{4.129172in}{0.697663in}}%
\pgfpathlineto{\pgfqpoint{4.136022in}{0.625795in}}%
\pgfpathlineto{\pgfqpoint{4.140300in}{0.711624in}}%
\pgfpathlineto{\pgfqpoint{4.143724in}{0.667325in}}%
\pgfpathlineto{\pgfqpoint{4.145436in}{0.734115in}}%
\pgfpathlineto{\pgfqpoint{4.151424in}{0.679847in}}%
\pgfpathlineto{\pgfqpoint{4.152280in}{0.723033in}}%
\pgfpathlineto{\pgfqpoint{4.156558in}{0.637132in}}%
\pgfpathlineto{\pgfqpoint{4.159124in}{0.772698in}}%
\pgfpathlineto{\pgfqpoint{4.162548in}{0.660452in}}%
\pgfpathlineto{\pgfqpoint{4.168538in}{0.764488in}}%
\pgfpathlineto{\pgfqpoint{4.171962in}{0.659263in}}%
\pgfpathlineto{\pgfqpoint{4.173670in}{0.784138in}}%
\pgfpathlineto{\pgfqpoint{4.177947in}{0.696905in}}%
\pgfpathlineto{\pgfqpoint{4.181364in}{0.685604in}}%
\pgfpathlineto{\pgfqpoint{4.183932in}{0.772945in}}%
\pgfpathlineto{\pgfqpoint{4.187352in}{0.634754in}}%
\pgfpathlineto{\pgfqpoint{4.189920in}{0.717058in}}%
\pgfpathlineto{\pgfqpoint{4.195906in}{0.657608in}}%
\pgfpathlineto{\pgfqpoint{4.196761in}{0.715223in}}%
\pgfpathlineto{\pgfqpoint{4.200182in}{0.738399in}}%
\pgfpathlineto{\pgfqpoint{4.205311in}{0.737498in}}%
\pgfpathlineto{\pgfqpoint{4.209590in}{0.638101in}}%
\pgfpathlineto{\pgfqpoint{4.210442in}{0.764201in}}%
\pgfpathlineto{\pgfqpoint{4.214719in}{0.649977in}}%
\pgfpathlineto{\pgfqpoint{4.218998in}{0.652099in}}%
\pgfpathlineto{\pgfqpoint{4.221561in}{0.767476in}}%
\pgfpathlineto{\pgfqpoint{4.224983in}{0.786835in}}%
\pgfpathlineto{\pgfqpoint{4.228405in}{0.674629in}}%
\pgfpathlineto{\pgfqpoint{4.231826in}{0.644975in}}%
\pgfpathlineto{\pgfqpoint{4.235248in}{0.755561in}}%
\pgfpathlineto{\pgfqpoint{4.237816in}{0.624351in}}%
\pgfpathlineto{\pgfqpoint{4.241236in}{0.748221in}}%
\pgfpathlineto{\pgfqpoint{4.246370in}{0.614601in}}%
\pgfpathlineto{\pgfqpoint{4.248080in}{0.700646in}}%
\pgfpathlineto{\pgfqpoint{4.252359in}{0.658362in}}%
\pgfpathlineto{\pgfqpoint{4.255778in}{0.705936in}}%
\pgfpathlineto{\pgfqpoint{4.259202in}{0.600316in}}%
\pgfpathlineto{\pgfqpoint{4.265191in}{0.760172in}}%
\pgfpathlineto{\pgfqpoint{4.268614in}{0.657464in}}%
\pgfpathlineto{\pgfqpoint{4.272037in}{0.730014in}}%
\pgfpathlineto{\pgfqpoint{4.275458in}{0.733792in}}%
\pgfpathlineto{\pgfqpoint{4.279734in}{0.695644in}}%
\pgfpathlineto{\pgfqpoint{4.284864in}{0.678048in}}%
\pgfpathlineto{\pgfqpoint{4.285716in}{0.739117in}}%
\pgfpathlineto{\pgfqpoint{4.290850in}{0.759701in}}%
\pgfpathlineto{\pgfqpoint{4.293417in}{0.662176in}}%
\pgfpathlineto{\pgfqpoint{4.297691in}{0.784713in}}%
\pgfpathlineto{\pgfqpoint{4.299404in}{0.672001in}}%
\pgfpathlineto{\pgfqpoint{4.307096in}{0.805839in}}%
\pgfpathlineto{\pgfqpoint{4.311374in}{0.685209in}}%
\pgfpathlineto{\pgfqpoint{4.313941in}{0.768126in}}%
\pgfpathlineto{\pgfqpoint{4.318213in}{0.800191in}}%
\pgfpathlineto{\pgfqpoint{4.319924in}{0.689708in}}%
\pgfpathlineto{\pgfqpoint{4.324195in}{0.655952in}}%
\pgfpathlineto{\pgfqpoint{4.326759in}{0.753622in}}%
\pgfpathlineto{\pgfqpoint{4.331037in}{0.662646in}}%
\pgfpathlineto{\pgfqpoint{4.334460in}{0.736600in}}%
\pgfpathlineto{\pgfqpoint{4.337026in}{0.701009in}}%
\pgfpathlineto{\pgfqpoint{4.342159in}{0.672435in}}%
\pgfpathlineto{\pgfqpoint{4.343870in}{0.741064in}}%
\pgfpathlineto{\pgfqpoint{4.350713in}{0.659730in}}%
\pgfpathlineto{\pgfqpoint{4.355844in}{0.754272in}}%
\pgfpathlineto{\pgfqpoint{4.360978in}{0.684921in}}%
\pgfpathlineto{\pgfqpoint{4.366112in}{0.802456in}}%
\pgfpathlineto{\pgfqpoint{4.368678in}{0.701189in}}%
\pgfpathlineto{\pgfqpoint{4.371246in}{0.768844in}}%
\pgfpathlineto{\pgfqpoint{4.377235in}{0.659766in}}%
\pgfpathlineto{\pgfqpoint{4.378945in}{0.731562in}}%
\pgfpathlineto{\pgfqpoint{4.382361in}{0.679089in}}%
\pgfpathlineto{\pgfqpoint{4.385782in}{0.768844in}}%
\pgfpathlineto{\pgfqpoint{4.390915in}{0.774640in}}%
\pgfpathlineto{\pgfqpoint{4.393482in}{0.662431in}}%
\pgfpathlineto{\pgfqpoint{4.395194in}{0.748260in}}%
\pgfpathlineto{\pgfqpoint{4.398612in}{0.696258in}}%
\pgfpathlineto{\pgfqpoint{4.402032in}{0.721916in}}%
\pgfpathlineto{\pgfqpoint{4.406311in}{0.628962in}}%
\pgfpathlineto{\pgfqpoint{4.408878in}{0.713962in}}%
\pgfpathlineto{\pgfqpoint{4.414012in}{0.753837in}}%
\pgfpathlineto{\pgfqpoint{4.416578in}{0.671929in}}%
\pgfpathlineto{\pgfqpoint{4.419998in}{0.735376in}}%
\pgfpathlineto{\pgfqpoint{4.422562in}{0.690211in}}%
\pgfpathlineto{\pgfqpoint{4.427694in}{0.740019in}}%
\pgfpathlineto{\pgfqpoint{4.430262in}{0.651237in}}%
\pgfpathlineto{\pgfqpoint{4.432826in}{0.784713in}}%
\pgfpathlineto{\pgfqpoint{4.437103in}{0.653288in}}%
\pgfpathlineto{\pgfqpoint{4.442241in}{0.767512in}}%
\pgfpathlineto{\pgfqpoint{4.444808in}{0.679161in}}%
\pgfpathlineto{\pgfqpoint{4.449081in}{0.735878in}}%
\pgfpathlineto{\pgfqpoint{4.449936in}{0.654369in}}%
\pgfpathlineto{\pgfqpoint{4.455924in}{0.773847in}}%
\pgfpathlineto{\pgfqpoint{4.458489in}{0.669412in}}%
\pgfpathlineto{\pgfqpoint{4.461057in}{0.756861in}}%
\pgfpathlineto{\pgfqpoint{4.463622in}{0.688667in}}%
\pgfpathlineto{\pgfqpoint{4.468754in}{0.726380in}}%
\pgfpathlineto{\pgfqpoint{4.473032in}{0.612264in}}%
\pgfpathlineto{\pgfqpoint{4.476450in}{0.748619in}}%
\pgfpathlineto{\pgfqpoint{4.478161in}{0.681431in}}%
\pgfpathlineto{\pgfqpoint{4.480727in}{0.726560in}}%
\pgfpathlineto{\pgfqpoint{4.485001in}{0.646343in}}%
\pgfpathlineto{\pgfqpoint{4.488421in}{0.751643in}}%
\pgfpathlineto{\pgfqpoint{4.491842in}{0.664266in}}%
\pgfpathlineto{\pgfqpoint{4.496121in}{0.733038in}}%
\pgfpathlineto{\pgfqpoint{4.497827in}{0.687730in}}%
\pgfpathlineto{\pgfqpoint{4.502953in}{0.766148in}}%
\pgfpathlineto{\pgfqpoint{4.505519in}{0.681323in}}%
\pgfpathlineto{\pgfqpoint{4.509790in}{0.742213in}}%
\pgfpathlineto{\pgfqpoint{4.512354in}{0.745452in}}%
\pgfpathlineto{\pgfqpoint{4.514921in}{0.664481in}}%
\pgfpathlineto{\pgfqpoint{4.518344in}{0.720225in}}%
\pgfpathlineto{\pgfqpoint{4.523473in}{0.673441in}}%
\pgfpathlineto{\pgfqpoint{4.526897in}{0.747646in}}%
\pgfpathlineto{\pgfqpoint{4.534596in}{0.583798in}}%
\pgfpathlineto{\pgfqpoint{4.535450in}{0.762837in}}%
\pgfpathlineto{\pgfqpoint{4.538868in}{0.585561in}}%
\pgfpathlineto{\pgfqpoint{4.543143in}{0.764775in}}%
\pgfpathlineto{\pgfqpoint{4.548281in}{0.748835in}}%
\pgfpathlineto{\pgfqpoint{4.549136in}{0.669340in}}%
\pgfpathlineto{\pgfqpoint{4.554268in}{0.634108in}}%
\pgfpathlineto{\pgfqpoint{4.558543in}{0.730804in}}%
\pgfpathlineto{\pgfqpoint{4.559399in}{0.666564in}}%
\pgfpathlineto{\pgfqpoint{4.562820in}{0.642777in}}%
\pgfpathlineto{\pgfqpoint{4.568805in}{0.737426in}}%
\pgfpathlineto{\pgfqpoint{4.570513in}{0.612479in}}%
\pgfpathlineto{\pgfqpoint{4.574795in}{0.714537in}}%
\pgfpathlineto{\pgfqpoint{4.579072in}{0.597185in}}%
\pgfpathlineto{\pgfqpoint{4.580783in}{0.682185in}}%
\pgfpathlineto{\pgfqpoint{4.584204in}{0.754268in}}%
\pgfpathlineto{\pgfqpoint{4.588482in}{0.633171in}}%
\pgfpathlineto{\pgfqpoint{4.592755in}{0.801986in}}%
\pgfpathlineto{\pgfqpoint{4.595324in}{0.662140in}}%
\pgfpathlineto{\pgfqpoint{4.597035in}{0.768445in}}%
\pgfpathlineto{\pgfqpoint{4.600455in}{0.691180in}}%
\pgfpathlineto{\pgfqpoint{4.605589in}{0.654293in}}%
\pgfpathlineto{\pgfqpoint{4.609011in}{0.684379in}}%
\pgfpathlineto{\pgfqpoint{4.610723in}{0.791335in}}%
\pgfpathlineto{\pgfqpoint{4.615000in}{0.660412in}}%
\pgfpathlineto{\pgfqpoint{4.617565in}{0.790002in}}%
\pgfpathlineto{\pgfqpoint{4.623549in}{0.757328in}}%
\pgfpathlineto{\pgfqpoint{4.624405in}{0.605786in}}%
\pgfpathlineto{\pgfqpoint{4.627827in}{0.726376in}}%
\pgfpathlineto{\pgfqpoint{4.631246in}{0.763299in}}%
\pgfpathlineto{\pgfqpoint{4.637235in}{0.788455in}}%
\pgfpathlineto{\pgfqpoint{4.638091in}{0.662642in}}%
\pgfpathlineto{\pgfqpoint{4.641512in}{0.745739in}}%
\pgfpathlineto{\pgfqpoint{4.644931in}{0.769562in}}%
\pgfpathlineto{\pgfqpoint{4.648350in}{0.704859in}}%
\pgfpathlineto{\pgfqpoint{4.654343in}{0.737354in}}%
\pgfpathlineto{\pgfqpoint{4.657762in}{0.618958in}}%
\pgfpathlineto{\pgfqpoint{4.660326in}{0.717600in}}%
\pgfpathlineto{\pgfqpoint{4.664605in}{0.731203in}}%
\pgfpathlineto{\pgfqpoint{4.666317in}{0.652354in}}%
\pgfpathlineto{\pgfqpoint{4.671449in}{0.613309in}}%
\pgfpathlineto{\pgfqpoint{4.672305in}{0.690394in}}%
\pgfpathlineto{\pgfqpoint{4.677437in}{0.749808in}}%
\pgfpathlineto{\pgfqpoint{4.681713in}{0.671390in}}%
\pgfpathlineto{\pgfqpoint{4.683423in}{0.607190in}}%
\pgfpathlineto{\pgfqpoint{4.686845in}{0.753837in}}%
\pgfpathlineto{\pgfqpoint{4.689413in}{0.702266in}}%
\pgfpathlineto{\pgfqpoint{4.695398in}{0.696833in}}%
\pgfpathlineto{\pgfqpoint{4.696255in}{0.777157in}}%
\pgfpathlineto{\pgfqpoint{4.700535in}{0.659439in}}%
\pgfpathlineto{\pgfqpoint{4.704809in}{0.740019in}}%
\pgfpathlineto{\pgfqpoint{4.710801in}{0.658904in}}%
\pgfpathlineto{\pgfqpoint{4.713368in}{0.768054in}}%
\pgfpathlineto{\pgfqpoint{4.717644in}{0.826387in}}%
\pgfpathlineto{\pgfqpoint{4.720210in}{0.686936in}}%
\pgfpathlineto{\pgfqpoint{4.724492in}{0.766610in}}%
\pgfpathlineto{\pgfqpoint{4.727918in}{0.689493in}}%
\pgfpathlineto{\pgfqpoint{4.731339in}{0.762653in}}%
\pgfpathlineto{\pgfqpoint{4.734766in}{0.637706in}}%
\pgfpathlineto{\pgfqpoint{4.739903in}{0.792883in}}%
\pgfpathlineto{\pgfqpoint{4.742472in}{0.703383in}}%
\pgfpathlineto{\pgfqpoint{4.746750in}{0.774062in}}%
\pgfpathlineto{\pgfqpoint{4.748462in}{0.688376in}}%
\pgfpathlineto{\pgfqpoint{4.751882in}{0.785973in}}%
\pgfpathlineto{\pgfqpoint{4.756154in}{0.680565in}}%
\pgfpathlineto{\pgfqpoint{4.760428in}{0.784857in}}%
\pgfpathlineto{\pgfqpoint{4.762995in}{0.667609in}}%
\pgfpathlineto{\pgfqpoint{4.765556in}{0.769490in}}%
\pgfpathlineto{\pgfqpoint{4.768121in}{0.710364in}}%
\pgfpathlineto{\pgfqpoint{4.774110in}{0.730625in}}%
\pgfpathlineto{\pgfqpoint{4.774966in}{0.671749in}}%
\pgfpathlineto{\pgfqpoint{4.779249in}{0.748472in}}%
\pgfpathlineto{\pgfqpoint{4.782675in}{0.649331in}}%
\pgfpathlineto{\pgfqpoint{4.787806in}{0.650049in}}%
\pgfpathlineto{\pgfqpoint{4.790374in}{0.743003in}}%
\pgfpathlineto{\pgfqpoint{4.792084in}{0.638496in}}%
\pgfpathlineto{\pgfqpoint{4.795506in}{0.697802in}}%
\pgfpathlineto{\pgfqpoint{4.795506in}{0.697802in}}%
\pgfusepath{stroke}%
\end{pgfscope}%
\begin{pgfscope}%
\pgfsetrectcap%
\pgfsetmiterjoin%
\pgfsetlinewidth{0.803000pt}%
\definecolor{currentstroke}{rgb}{0.000000,0.000000,0.000000}%
\pgfsetstrokecolor{currentstroke}%
\pgfsetdash{}{0pt}%
\pgfpathmoveto{\pgfqpoint{0.484581in}{0.539544in}}%
\pgfpathlineto{\pgfqpoint{0.484581in}{1.114166in}}%
\pgfusepath{stroke}%
\end{pgfscope}%
\begin{pgfscope}%
\pgfsetrectcap%
\pgfsetmiterjoin%
\pgfsetlinewidth{0.803000pt}%
\definecolor{currentstroke}{rgb}{0.000000,0.000000,0.000000}%
\pgfsetstrokecolor{currentstroke}%
\pgfsetdash{}{0pt}%
\pgfpathmoveto{\pgfqpoint{5.000788in}{0.539544in}}%
\pgfpathlineto{\pgfqpoint{5.000788in}{1.114166in}}%
\pgfusepath{stroke}%
\end{pgfscope}%
\begin{pgfscope}%
\pgfsetrectcap%
\pgfsetmiterjoin%
\pgfsetlinewidth{0.803000pt}%
\definecolor{currentstroke}{rgb}{0.000000,0.000000,0.000000}%
\pgfsetstrokecolor{currentstroke}%
\pgfsetdash{}{0pt}%
\pgfpathmoveto{\pgfqpoint{0.484581in}{0.539544in}}%
\pgfpathlineto{\pgfqpoint{5.000788in}{0.539544in}}%
\pgfusepath{stroke}%
\end{pgfscope}%
\begin{pgfscope}%
\pgfsetrectcap%
\pgfsetmiterjoin%
\pgfsetlinewidth{0.803000pt}%
\definecolor{currentstroke}{rgb}{0.000000,0.000000,0.000000}%
\pgfsetstrokecolor{currentstroke}%
\pgfsetdash{}{0pt}%
\pgfpathmoveto{\pgfqpoint{0.484581in}{1.114166in}}%
\pgfpathlineto{\pgfqpoint{5.000788in}{1.114166in}}%
\pgfusepath{stroke}%
\end{pgfscope}%
\end{pgfpicture}%
\makeatother%
\endgroup%

    \caption{Popcorn noise in different samples of the LM399 over a \qty{24}{\hour} period.}
    \label{fig:popcorn_noise_lm399}
\end{figure}

Figure \ref{fig:popcorn_noise_lm399} shows two samples of the LM399, that exhibit popcorn noise, while the last one does not.

The sources of popcorn noise in semiconductor devices are not yet fully understood, but some sources have been identified. Defects in the semiconductor crystal lattice and contamination of the semiconductor material have been linked to popcorn noise \cite{technote_ti_popcorn_noise}. This problem has improved over the years as manufacturing processes and wafer quality has evolved. Unfortunately the LM399 is built around a process from 1991, as can be seen etched into the die \cite{lm399_richi}.

The popcorn noise caused by defects and contamination can be reduced by lowering the strain on the lattice and removing surface contaminants on the die. This can be can be achieved using a high-temperature burn-in process. Manufacturers like Fluke and Keysight use similar techniques in their products. Fluke, for example, uses a period of \qty{60}{\day} burn-in for their references \cite{zener_popcorn_noise}.

Fortunately, the LM399 is a heated reference, which regulates its die to \qty{90}{\celsius} when turned on, so it is only required put the diodes in a simple test circuit and wait. The use of a separate test setup instead of the final circuit has both advantages and disadvantages. The disadvantage is, that the Zener diode will subjected to mechanical stress when soldered, this stress will not be removed by the burn-in process as it happens after the testing, when diode is soldered into the final circuit, but this mainly affects the voltage drift properties of the Zener diode and not the popcorn noise. The drift of the diode is also only of secondary concern in our setup, as the drift is mainly caused by the reference resistors used and are typically at least an order of magnitude worse than the the drift of the diode judging by the data sheet \cite{datasheet_LM399,datasheet_VPR}.

The advantages of testing the Zener diodes separately, on the other hand, are, that more diodes can be tested at the same time, as a special compact test fixture can be used. It is also simpler to remove the diodes from the test fixture, because they be socketed. Therefore for our application a separate test board was used. Building this test setup is detailed in the next sections.

\subsection{Building a Test Setup for Zener Diodes}
There are several ways to measure the popcorn noise of semiconductor devices. The most trivial one is to directly monitor the device in the time-domain. In this case, the Zener voltage can be monitored with a long-scale multimeter. It requires a low noise DMM, that can reliably distinguish between both voltage levels, which are about \qty{4}{\micro \volt} apart.
A related option is to use a second reference, whose voltage is similar to the device under test (DUT). Measuring the the voltage difference between the two references, less resolution is required. Directly comparing the difference of two references using a millivolt meter is commonly done when intercomparing primary voltage references. This method, however, increases the measurement noise by a factor of $\sqrt{2}$, if both references produce the same level of uncorrelated noise. The noise of the LM399 with a \qty{100}{\plc} integration time (\qty{2}{\second}) is about \qty{1.5}{\micro \volt_{pp}} as can be determined from the data in figure \ref{fig:popcorn_noise_lm399}.

\begin{figure}[ht]
    \centering
    %%% Creator: Matplotlib, PGF backend
%%
%% To include the figure in your LaTeX document, write
%%   \input{<filename>.pgf}
%%
%% Make sure the required packages are loaded in your preamble
%%   \usepackage{pgf}
%%
%% Also ensure that all the required font packages are loaded; for instance,
%% the lmodern package is sometimes necessary when using math font.
%%   \usepackage{lmodern}
%%
%% Figures using additional raster images can only be included by \input if
%% they are in the same directory as the main LaTeX file. For loading figures
%% from other directories you can use the `import` package
%%   \usepackage{import}
%%
%% and then include the figures with
%%   \import{<path to file>}{<filename>.pgf}
%%
%% Matplotlib used the following preamble
%%   \usepackage{siunitx}
%%   \sisetup{per-mode = symbol}%
%%   \usepackage{fontspec}
%%   \makeatletter\@ifpackageloaded{underscore}{}{\usepackage[strings]{underscore}}\makeatother
%%
\begingroup%
\makeatletter%
\begin{pgfpicture}%
\pgfpathrectangle{\pgfpointorigin}{\pgfqpoint{5.150788in}{3.183362in}}%
\pgfusepath{use as bounding box, clip}%
\begin{pgfscope}%
\pgfsetbuttcap%
\pgfsetmiterjoin%
\definecolor{currentfill}{rgb}{1.000000,1.000000,1.000000}%
\pgfsetfillcolor{currentfill}%
\pgfsetlinewidth{0.000000pt}%
\definecolor{currentstroke}{rgb}{1.000000,1.000000,1.000000}%
\pgfsetstrokecolor{currentstroke}%
\pgfsetdash{}{0pt}%
\pgfpathmoveto{\pgfqpoint{0.000000in}{0.000000in}}%
\pgfpathlineto{\pgfqpoint{5.150788in}{0.000000in}}%
\pgfpathlineto{\pgfqpoint{5.150788in}{3.183362in}}%
\pgfpathlineto{\pgfqpoint{0.000000in}{3.183362in}}%
\pgfpathlineto{\pgfqpoint{0.000000in}{0.000000in}}%
\pgfpathclose%
\pgfusepath{fill}%
\end{pgfscope}%
\begin{pgfscope}%
\pgfsetbuttcap%
\pgfsetmiterjoin%
\definecolor{currentfill}{rgb}{1.000000,1.000000,1.000000}%
\pgfsetfillcolor{currentfill}%
\pgfsetlinewidth{0.000000pt}%
\definecolor{currentstroke}{rgb}{0.000000,0.000000,0.000000}%
\pgfsetstrokecolor{currentstroke}%
\pgfsetstrokeopacity{0.000000}%
\pgfsetdash{}{0pt}%
\pgfpathmoveto{\pgfqpoint{0.484581in}{2.334497in}}%
\pgfpathlineto{\pgfqpoint{5.000788in}{2.334497in}}%
\pgfpathlineto{\pgfqpoint{5.000788in}{2.909119in}}%
\pgfpathlineto{\pgfqpoint{0.484581in}{2.909119in}}%
\pgfpathlineto{\pgfqpoint{0.484581in}{2.334497in}}%
\pgfpathclose%
\pgfusepath{fill}%
\end{pgfscope}%
\begin{pgfscope}%
\pgfsetbuttcap%
\pgfsetroundjoin%
\definecolor{currentfill}{rgb}{0.000000,0.000000,0.000000}%
\pgfsetfillcolor{currentfill}%
\pgfsetlinewidth{0.803000pt}%
\definecolor{currentstroke}{rgb}{0.000000,0.000000,0.000000}%
\pgfsetstrokecolor{currentstroke}%
\pgfsetdash{}{0pt}%
\pgfsys@defobject{currentmarker}{\pgfqpoint{0.000000in}{-0.048611in}}{\pgfqpoint{0.000000in}{0.000000in}}{%
\pgfpathmoveto{\pgfqpoint{0.000000in}{0.000000in}}%
\pgfpathlineto{\pgfqpoint{0.000000in}{-0.048611in}}%
\pgfusepath{stroke,fill}%
}%
\begin{pgfscope}%
\pgfsys@transformshift{0.689546in}{2.334497in}%
\pgfsys@useobject{currentmarker}{}%
\end{pgfscope}%
\end{pgfscope}%
\begin{pgfscope}%
\pgfsetbuttcap%
\pgfsetroundjoin%
\definecolor{currentfill}{rgb}{0.000000,0.000000,0.000000}%
\pgfsetfillcolor{currentfill}%
\pgfsetlinewidth{0.803000pt}%
\definecolor{currentstroke}{rgb}{0.000000,0.000000,0.000000}%
\pgfsetstrokecolor{currentstroke}%
\pgfsetdash{}{0pt}%
\pgfsys@defobject{currentmarker}{\pgfqpoint{0.000000in}{-0.048611in}}{\pgfqpoint{0.000000in}{0.000000in}}{%
\pgfpathmoveto{\pgfqpoint{0.000000in}{0.000000in}}%
\pgfpathlineto{\pgfqpoint{0.000000in}{-0.048611in}}%
\pgfusepath{stroke,fill}%
}%
\begin{pgfscope}%
\pgfsys@transformshift{1.202878in}{2.334497in}%
\pgfsys@useobject{currentmarker}{}%
\end{pgfscope}%
\end{pgfscope}%
\begin{pgfscope}%
\pgfsetbuttcap%
\pgfsetroundjoin%
\definecolor{currentfill}{rgb}{0.000000,0.000000,0.000000}%
\pgfsetfillcolor{currentfill}%
\pgfsetlinewidth{0.803000pt}%
\definecolor{currentstroke}{rgb}{0.000000,0.000000,0.000000}%
\pgfsetstrokecolor{currentstroke}%
\pgfsetdash{}{0pt}%
\pgfsys@defobject{currentmarker}{\pgfqpoint{0.000000in}{-0.048611in}}{\pgfqpoint{0.000000in}{0.000000in}}{%
\pgfpathmoveto{\pgfqpoint{0.000000in}{0.000000in}}%
\pgfpathlineto{\pgfqpoint{0.000000in}{-0.048611in}}%
\pgfusepath{stroke,fill}%
}%
\begin{pgfscope}%
\pgfsys@transformshift{1.716211in}{2.334497in}%
\pgfsys@useobject{currentmarker}{}%
\end{pgfscope}%
\end{pgfscope}%
\begin{pgfscope}%
\pgfsetbuttcap%
\pgfsetroundjoin%
\definecolor{currentfill}{rgb}{0.000000,0.000000,0.000000}%
\pgfsetfillcolor{currentfill}%
\pgfsetlinewidth{0.803000pt}%
\definecolor{currentstroke}{rgb}{0.000000,0.000000,0.000000}%
\pgfsetstrokecolor{currentstroke}%
\pgfsetdash{}{0pt}%
\pgfsys@defobject{currentmarker}{\pgfqpoint{0.000000in}{-0.048611in}}{\pgfqpoint{0.000000in}{0.000000in}}{%
\pgfpathmoveto{\pgfqpoint{0.000000in}{0.000000in}}%
\pgfpathlineto{\pgfqpoint{0.000000in}{-0.048611in}}%
\pgfusepath{stroke,fill}%
}%
\begin{pgfscope}%
\pgfsys@transformshift{2.229543in}{2.334497in}%
\pgfsys@useobject{currentmarker}{}%
\end{pgfscope}%
\end{pgfscope}%
\begin{pgfscope}%
\pgfsetbuttcap%
\pgfsetroundjoin%
\definecolor{currentfill}{rgb}{0.000000,0.000000,0.000000}%
\pgfsetfillcolor{currentfill}%
\pgfsetlinewidth{0.803000pt}%
\definecolor{currentstroke}{rgb}{0.000000,0.000000,0.000000}%
\pgfsetstrokecolor{currentstroke}%
\pgfsetdash{}{0pt}%
\pgfsys@defobject{currentmarker}{\pgfqpoint{0.000000in}{-0.048611in}}{\pgfqpoint{0.000000in}{0.000000in}}{%
\pgfpathmoveto{\pgfqpoint{0.000000in}{0.000000in}}%
\pgfpathlineto{\pgfqpoint{0.000000in}{-0.048611in}}%
\pgfusepath{stroke,fill}%
}%
\begin{pgfscope}%
\pgfsys@transformshift{2.742876in}{2.334497in}%
\pgfsys@useobject{currentmarker}{}%
\end{pgfscope}%
\end{pgfscope}%
\begin{pgfscope}%
\pgfsetbuttcap%
\pgfsetroundjoin%
\definecolor{currentfill}{rgb}{0.000000,0.000000,0.000000}%
\pgfsetfillcolor{currentfill}%
\pgfsetlinewidth{0.803000pt}%
\definecolor{currentstroke}{rgb}{0.000000,0.000000,0.000000}%
\pgfsetstrokecolor{currentstroke}%
\pgfsetdash{}{0pt}%
\pgfsys@defobject{currentmarker}{\pgfqpoint{0.000000in}{-0.048611in}}{\pgfqpoint{0.000000in}{0.000000in}}{%
\pgfpathmoveto{\pgfqpoint{0.000000in}{0.000000in}}%
\pgfpathlineto{\pgfqpoint{0.000000in}{-0.048611in}}%
\pgfusepath{stroke,fill}%
}%
\begin{pgfscope}%
\pgfsys@transformshift{3.256208in}{2.334497in}%
\pgfsys@useobject{currentmarker}{}%
\end{pgfscope}%
\end{pgfscope}%
\begin{pgfscope}%
\pgfsetbuttcap%
\pgfsetroundjoin%
\definecolor{currentfill}{rgb}{0.000000,0.000000,0.000000}%
\pgfsetfillcolor{currentfill}%
\pgfsetlinewidth{0.803000pt}%
\definecolor{currentstroke}{rgb}{0.000000,0.000000,0.000000}%
\pgfsetstrokecolor{currentstroke}%
\pgfsetdash{}{0pt}%
\pgfsys@defobject{currentmarker}{\pgfqpoint{0.000000in}{-0.048611in}}{\pgfqpoint{0.000000in}{0.000000in}}{%
\pgfpathmoveto{\pgfqpoint{0.000000in}{0.000000in}}%
\pgfpathlineto{\pgfqpoint{0.000000in}{-0.048611in}}%
\pgfusepath{stroke,fill}%
}%
\begin{pgfscope}%
\pgfsys@transformshift{3.769541in}{2.334497in}%
\pgfsys@useobject{currentmarker}{}%
\end{pgfscope}%
\end{pgfscope}%
\begin{pgfscope}%
\pgfsetbuttcap%
\pgfsetroundjoin%
\definecolor{currentfill}{rgb}{0.000000,0.000000,0.000000}%
\pgfsetfillcolor{currentfill}%
\pgfsetlinewidth{0.803000pt}%
\definecolor{currentstroke}{rgb}{0.000000,0.000000,0.000000}%
\pgfsetstrokecolor{currentstroke}%
\pgfsetdash{}{0pt}%
\pgfsys@defobject{currentmarker}{\pgfqpoint{0.000000in}{-0.048611in}}{\pgfqpoint{0.000000in}{0.000000in}}{%
\pgfpathmoveto{\pgfqpoint{0.000000in}{0.000000in}}%
\pgfpathlineto{\pgfqpoint{0.000000in}{-0.048611in}}%
\pgfusepath{stroke,fill}%
}%
\begin{pgfscope}%
\pgfsys@transformshift{4.282873in}{2.334497in}%
\pgfsys@useobject{currentmarker}{}%
\end{pgfscope}%
\end{pgfscope}%
\begin{pgfscope}%
\pgfsetbuttcap%
\pgfsetroundjoin%
\definecolor{currentfill}{rgb}{0.000000,0.000000,0.000000}%
\pgfsetfillcolor{currentfill}%
\pgfsetlinewidth{0.803000pt}%
\definecolor{currentstroke}{rgb}{0.000000,0.000000,0.000000}%
\pgfsetstrokecolor{currentstroke}%
\pgfsetdash{}{0pt}%
\pgfsys@defobject{currentmarker}{\pgfqpoint{0.000000in}{-0.048611in}}{\pgfqpoint{0.000000in}{0.000000in}}{%
\pgfpathmoveto{\pgfqpoint{0.000000in}{0.000000in}}%
\pgfpathlineto{\pgfqpoint{0.000000in}{-0.048611in}}%
\pgfusepath{stroke,fill}%
}%
\begin{pgfscope}%
\pgfsys@transformshift{4.796206in}{2.334497in}%
\pgfsys@useobject{currentmarker}{}%
\end{pgfscope}%
\end{pgfscope}%
\begin{pgfscope}%
\pgfsetbuttcap%
\pgfsetroundjoin%
\definecolor{currentfill}{rgb}{0.000000,0.000000,0.000000}%
\pgfsetfillcolor{currentfill}%
\pgfsetlinewidth{0.803000pt}%
\definecolor{currentstroke}{rgb}{0.000000,0.000000,0.000000}%
\pgfsetstrokecolor{currentstroke}%
\pgfsetdash{}{0pt}%
\pgfsys@defobject{currentmarker}{\pgfqpoint{-0.048611in}{0.000000in}}{\pgfqpoint{-0.000000in}{0.000000in}}{%
\pgfpathmoveto{\pgfqpoint{-0.000000in}{0.000000in}}%
\pgfpathlineto{\pgfqpoint{-0.048611in}{0.000000in}}%
\pgfusepath{stroke,fill}%
}%
\begin{pgfscope}%
\pgfsys@transformshift{0.484581in}{2.512496in}%
\pgfsys@useobject{currentmarker}{}%
\end{pgfscope}%
\end{pgfscope}%
\begin{pgfscope}%
\definecolor{textcolor}{rgb}{0.000000,0.000000,0.000000}%
\pgfsetstrokecolor{textcolor}%
\pgfsetfillcolor{textcolor}%
\pgftext[x=0.328331in, y=2.473941in, left, base]{\color{textcolor}\rmfamily\fontsize{8.000000}{9.600000}\selectfont \(\displaystyle {0}\)}%
\end{pgfscope}%
\begin{pgfscope}%
\pgfsetbuttcap%
\pgfsetroundjoin%
\definecolor{currentfill}{rgb}{0.000000,0.000000,0.000000}%
\pgfsetfillcolor{currentfill}%
\pgfsetlinewidth{0.803000pt}%
\definecolor{currentstroke}{rgb}{0.000000,0.000000,0.000000}%
\pgfsetstrokecolor{currentstroke}%
\pgfsetdash{}{0pt}%
\pgfsys@defobject{currentmarker}{\pgfqpoint{-0.048611in}{0.000000in}}{\pgfqpoint{-0.000000in}{0.000000in}}{%
\pgfpathmoveto{\pgfqpoint{-0.000000in}{0.000000in}}%
\pgfpathlineto{\pgfqpoint{-0.048611in}{0.000000in}}%
\pgfusepath{stroke,fill}%
}%
\begin{pgfscope}%
\pgfsys@transformshift{0.484581in}{2.716037in}%
\pgfsys@useobject{currentmarker}{}%
\end{pgfscope}%
\end{pgfscope}%
\begin{pgfscope}%
\definecolor{textcolor}{rgb}{0.000000,0.000000,0.000000}%
\pgfsetstrokecolor{textcolor}%
\pgfsetfillcolor{textcolor}%
\pgftext[x=0.328331in, y=2.677482in, left, base]{\color{textcolor}\rmfamily\fontsize{8.000000}{9.600000}\selectfont \(\displaystyle {5}\)}%
\end{pgfscope}%
\begin{pgfscope}%
\definecolor{textcolor}{rgb}{0.000000,0.000000,0.000000}%
\pgfsetstrokecolor{textcolor}%
\pgfsetfillcolor{textcolor}%
\pgftext[x=0.484581in,y=2.950785in,left,base]{\color{textcolor}\rmfamily\fontsize{8.000000}{9.600000}\selectfont \(\displaystyle \times{10^{\ensuremath{-}6}}{}\)}%
\end{pgfscope}%
\begin{pgfscope}%
\pgfpathrectangle{\pgfqpoint{0.484581in}{2.334497in}}{\pgfqpoint{4.516206in}{0.574622in}}%
\pgfusepath{clip}%
\pgfsetrectcap%
\pgfsetroundjoin%
\pgfsetlinewidth{0.501875pt}%
\definecolor{currentstroke}{rgb}{0.003922,0.450980,0.698039}%
\pgfsetstrokecolor{currentstroke}%
\pgfsetstrokeopacity{0.700000}%
\pgfsetdash{}{0pt}%
\pgfpathmoveto{\pgfqpoint{0.689863in}{2.509930in}}%
\pgfpathlineto{\pgfqpoint{0.691573in}{2.569128in}}%
\pgfpathlineto{\pgfqpoint{0.694995in}{2.478459in}}%
\pgfpathlineto{\pgfqpoint{0.699275in}{2.580305in}}%
\pgfpathlineto{\pgfqpoint{0.703553in}{2.518208in}}%
\pgfpathlineto{\pgfqpoint{0.710398in}{2.564054in}}%
\pgfpathlineto{\pgfqpoint{0.712966in}{2.460398in}}%
\pgfpathlineto{\pgfqpoint{0.718959in}{2.382413in}}%
\pgfpathlineto{\pgfqpoint{0.722378in}{2.526844in}}%
\pgfpathlineto{\pgfqpoint{0.728363in}{2.563512in}}%
\pgfpathlineto{\pgfqpoint{0.730929in}{2.446928in}}%
\pgfpathlineto{\pgfqpoint{0.734351in}{2.508420in}}%
\pgfpathlineto{\pgfqpoint{0.740340in}{2.462634in}}%
\pgfpathlineto{\pgfqpoint{0.742908in}{2.509569in}}%
\pgfpathlineto{\pgfqpoint{0.748037in}{2.531618in}}%
\pgfpathlineto{\pgfqpoint{0.752314in}{2.422041in}}%
\pgfpathlineto{\pgfqpoint{0.755738in}{2.514704in}}%
\pgfpathlineto{\pgfqpoint{0.760009in}{2.489757in}}%
\pgfpathlineto{\pgfqpoint{0.764288in}{2.546054in}}%
\pgfpathlineto{\pgfqpoint{0.769423in}{2.581573in}}%
\pgfpathlineto{\pgfqpoint{0.771990in}{2.483110in}}%
\pgfpathlineto{\pgfqpoint{0.777124in}{2.481781in}}%
\pgfpathlineto{\pgfqpoint{0.780547in}{2.572209in}}%
\pgfpathlineto{\pgfqpoint{0.786534in}{2.392199in}}%
\pgfpathlineto{\pgfqpoint{0.789953in}{2.555597in}}%
\pgfpathlineto{\pgfqpoint{0.795939in}{2.458526in}}%
\pgfpathlineto{\pgfqpoint{0.799360in}{2.547927in}}%
\pgfpathlineto{\pgfqpoint{0.803633in}{2.421859in}}%
\pgfpathlineto{\pgfqpoint{0.806199in}{2.515549in}}%
\pgfpathlineto{\pgfqpoint{0.810475in}{2.476526in}}%
\pgfpathlineto{\pgfqpoint{0.814751in}{2.453994in}}%
\pgfpathlineto{\pgfqpoint{0.820735in}{2.649770in}}%
\pgfpathlineto{\pgfqpoint{0.823303in}{2.515246in}}%
\pgfpathlineto{\pgfqpoint{0.828440in}{2.617574in}}%
\pgfpathlineto{\pgfqpoint{0.832717in}{2.490117in}}%
\pgfpathlineto{\pgfqpoint{0.839565in}{2.452968in}}%
\pgfpathlineto{\pgfqpoint{0.842131in}{2.529864in}}%
\pgfpathlineto{\pgfqpoint{0.848114in}{2.449857in}}%
\pgfpathlineto{\pgfqpoint{0.852384in}{2.551822in}}%
\pgfpathlineto{\pgfqpoint{0.856662in}{2.459794in}}%
\pgfpathlineto{\pgfqpoint{0.859233in}{2.558859in}}%
\pgfpathlineto{\pgfqpoint{0.862659in}{2.464323in}}%
\pgfpathlineto{\pgfqpoint{0.868650in}{2.535180in}}%
\pgfpathlineto{\pgfqpoint{0.872926in}{2.477250in}}%
\pgfpathlineto{\pgfqpoint{0.876348in}{2.467797in}}%
\pgfpathlineto{\pgfqpoint{0.880627in}{2.539679in}}%
\pgfpathlineto{\pgfqpoint{0.884050in}{2.404099in}}%
\pgfpathlineto{\pgfqpoint{0.887469in}{2.536629in}}%
\pgfpathlineto{\pgfqpoint{0.895167in}{2.454266in}}%
\pgfpathlineto{\pgfqpoint{0.896877in}{2.573658in}}%
\pgfpathlineto{\pgfqpoint{0.902009in}{2.524850in}}%
\pgfpathlineto{\pgfqpoint{0.906287in}{2.809305in}}%
\pgfpathlineto{\pgfqpoint{0.908853in}{2.673270in}}%
\pgfpathlineto{\pgfqpoint{0.915695in}{2.794808in}}%
\pgfpathlineto{\pgfqpoint{0.919115in}{2.504372in}}%
\pgfpathlineto{\pgfqpoint{0.921681in}{2.466317in}}%
\pgfpathlineto{\pgfqpoint{0.925951in}{2.535422in}}%
\pgfpathlineto{\pgfqpoint{0.935359in}{2.435328in}}%
\pgfpathlineto{\pgfqpoint{0.939638in}{2.544754in}}%
\pgfpathlineto{\pgfqpoint{0.943058in}{2.464444in}}%
\pgfpathlineto{\pgfqpoint{0.947330in}{2.450187in}}%
\pgfpathlineto{\pgfqpoint{0.955029in}{2.439316in}}%
\pgfpathlineto{\pgfqpoint{0.957596in}{2.572632in}}%
\pgfpathlineto{\pgfqpoint{0.961015in}{2.449373in}}%
\pgfpathlineto{\pgfqpoint{0.965292in}{2.526783in}}%
\pgfpathlineto{\pgfqpoint{0.971277in}{2.574021in}}%
\pgfpathlineto{\pgfqpoint{0.972986in}{2.461303in}}%
\pgfpathlineto{\pgfqpoint{0.978119in}{2.553906in}}%
\pgfpathlineto{\pgfqpoint{0.983246in}{2.411046in}}%
\pgfpathlineto{\pgfqpoint{0.985812in}{2.492956in}}%
\pgfpathlineto{\pgfqpoint{0.990941in}{2.449706in}}%
\pgfpathlineto{\pgfqpoint{0.996074in}{2.539710in}}%
\pgfpathlineto{\pgfqpoint{1.001204in}{2.486674in}}%
\pgfpathlineto{\pgfqpoint{1.004625in}{2.536085in}}%
\pgfpathlineto{\pgfqpoint{1.007191in}{2.418051in}}%
\pgfpathlineto{\pgfqpoint{1.013177in}{2.564296in}}%
\pgfpathlineto{\pgfqpoint{1.016601in}{2.472720in}}%
\pgfpathlineto{\pgfqpoint{1.023437in}{2.595949in}}%
\pgfpathlineto{\pgfqpoint{1.024292in}{2.493530in}}%
\pgfpathlineto{\pgfqpoint{1.030277in}{2.444511in}}%
\pgfpathlineto{\pgfqpoint{1.035409in}{2.533065in}}%
\pgfpathlineto{\pgfqpoint{1.039687in}{2.573356in}}%
\pgfpathlineto{\pgfqpoint{1.046538in}{2.426389in}}%
\pgfpathlineto{\pgfqpoint{1.051672in}{2.516514in}}%
\pgfpathlineto{\pgfqpoint{1.055092in}{2.427506in}}%
\pgfpathlineto{\pgfqpoint{1.059370in}{2.535843in}}%
\pgfpathlineto{\pgfqpoint{1.065358in}{2.549193in}}%
\pgfpathlineto{\pgfqpoint{1.067066in}{2.434965in}}%
\pgfpathlineto{\pgfqpoint{1.073048in}{2.538502in}}%
\pgfpathlineto{\pgfqpoint{1.076466in}{2.455443in}}%
\pgfpathlineto{\pgfqpoint{1.082458in}{2.574805in}}%
\pgfpathlineto{\pgfqpoint{1.087592in}{2.433697in}}%
\pgfpathlineto{\pgfqpoint{1.088447in}{2.497062in}}%
\pgfpathlineto{\pgfqpoint{1.093579in}{2.563873in}}%
\pgfpathlineto{\pgfqpoint{1.097851in}{2.467585in}}%
\pgfpathlineto{\pgfqpoint{1.103832in}{2.552878in}}%
\pgfpathlineto{\pgfqpoint{1.107250in}{2.428744in}}%
\pgfpathlineto{\pgfqpoint{1.112379in}{2.554692in}}%
\pgfpathlineto{\pgfqpoint{1.114088in}{2.448135in}}%
\pgfpathlineto{\pgfqpoint{1.120074in}{2.527086in}}%
\pgfpathlineto{\pgfqpoint{1.123495in}{2.440825in}}%
\pgfpathlineto{\pgfqpoint{1.126917in}{2.517964in}}%
\pgfpathlineto{\pgfqpoint{1.131195in}{2.419200in}}%
\pgfpathlineto{\pgfqpoint{1.136328in}{2.548529in}}%
\pgfpathlineto{\pgfqpoint{1.139749in}{2.450399in}}%
\pgfpathlineto{\pgfqpoint{1.145737in}{2.538139in}}%
\pgfpathlineto{\pgfqpoint{1.148303in}{2.449222in}}%
\pgfpathlineto{\pgfqpoint{1.155144in}{2.590209in}}%
\pgfpathlineto{\pgfqpoint{1.158563in}{2.477975in}}%
\pgfpathlineto{\pgfqpoint{1.161130in}{2.562000in}}%
\pgfpathlineto{\pgfqpoint{1.166264in}{2.553422in}}%
\pgfpathlineto{\pgfqpoint{1.172254in}{2.422371in}}%
\pgfpathlineto{\pgfqpoint{1.174823in}{2.567558in}}%
\pgfpathlineto{\pgfqpoint{1.179102in}{2.466620in}}%
\pgfpathlineto{\pgfqpoint{1.183380in}{2.571725in}}%
\pgfpathlineto{\pgfqpoint{1.189368in}{2.453088in}}%
\pgfpathlineto{\pgfqpoint{1.191079in}{2.535966in}}%
\pgfpathlineto{\pgfqpoint{1.198776in}{2.443908in}}%
\pgfpathlineto{\pgfqpoint{1.201344in}{2.534033in}}%
\pgfpathlineto{\pgfqpoint{1.203910in}{2.473566in}}%
\pgfpathlineto{\pgfqpoint{1.209045in}{2.570638in}}%
\pgfpathlineto{\pgfqpoint{1.213329in}{2.482868in}}%
\pgfpathlineto{\pgfqpoint{1.216755in}{2.540496in}}%
\pgfpathlineto{\pgfqpoint{1.222746in}{2.460459in}}%
\pgfpathlineto{\pgfqpoint{1.227880in}{2.440946in}}%
\pgfpathlineto{\pgfqpoint{1.229590in}{2.551912in}}%
\pgfpathlineto{\pgfqpoint{1.235580in}{2.469518in}}%
\pgfpathlineto{\pgfqpoint{1.239001in}{2.532402in}}%
\pgfpathlineto{\pgfqpoint{1.243278in}{2.475649in}}%
\pgfpathlineto{\pgfqpoint{1.248413in}{2.554027in}}%
\pgfpathlineto{\pgfqpoint{1.251837in}{2.455625in}}%
\pgfpathlineto{\pgfqpoint{1.257825in}{2.410199in}}%
\pgfpathlineto{\pgfqpoint{1.259533in}{2.581843in}}%
\pgfpathlineto{\pgfqpoint{1.266376in}{2.476586in}}%
\pgfpathlineto{\pgfqpoint{1.270656in}{2.598243in}}%
\pgfpathlineto{\pgfqpoint{1.274080in}{2.472478in}}%
\pgfpathlineto{\pgfqpoint{1.277502in}{2.571483in}}%
\pgfpathlineto{\pgfqpoint{1.280922in}{2.518175in}}%
\pgfpathlineto{\pgfqpoint{1.285201in}{2.558617in}}%
\pgfpathlineto{\pgfqpoint{1.290336in}{2.372416in}}%
\pgfpathlineto{\pgfqpoint{1.295468in}{2.494345in}}%
\pgfpathlineto{\pgfqpoint{1.299746in}{2.446111in}}%
\pgfpathlineto{\pgfqpoint{1.302314in}{2.523401in}}%
\pgfpathlineto{\pgfqpoint{1.310013in}{2.405790in}}%
\pgfpathlineto{\pgfqpoint{1.310869in}{2.502923in}}%
\pgfpathlineto{\pgfqpoint{1.318566in}{2.440190in}}%
\pgfpathlineto{\pgfqpoint{1.321131in}{2.519111in}}%
\pgfpathlineto{\pgfqpoint{1.325401in}{2.460457in}}%
\pgfpathlineto{\pgfqpoint{1.329678in}{2.563208in}}%
\pgfpathlineto{\pgfqpoint{1.333096in}{2.477612in}}%
\pgfpathlineto{\pgfqpoint{1.339081in}{2.527328in}}%
\pgfpathlineto{\pgfqpoint{1.341646in}{2.469458in}}%
\pgfpathlineto{\pgfqpoint{1.345066in}{2.432187in}}%
\pgfpathlineto{\pgfqpoint{1.351054in}{2.429470in}}%
\pgfpathlineto{\pgfqpoint{1.354477in}{2.550160in}}%
\pgfpathlineto{\pgfqpoint{1.362176in}{2.418898in}}%
\pgfpathlineto{\pgfqpoint{1.367303in}{2.491023in}}%
\pgfpathlineto{\pgfqpoint{1.370723in}{2.437685in}}%
\pgfpathlineto{\pgfqpoint{1.375853in}{2.521770in}}%
\pgfpathlineto{\pgfqpoint{1.381839in}{2.395158in}}%
\pgfpathlineto{\pgfqpoint{1.386113in}{2.528475in}}%
\pgfpathlineto{\pgfqpoint{1.391249in}{2.465501in}}%
\pgfpathlineto{\pgfqpoint{1.394673in}{2.531616in}}%
\pgfpathlineto{\pgfqpoint{1.398953in}{2.476223in}}%
\pgfpathlineto{\pgfqpoint{1.401521in}{2.563510in}}%
\pgfpathlineto{\pgfqpoint{1.408362in}{2.571183in}}%
\pgfpathlineto{\pgfqpoint{1.410925in}{2.466015in}}%
\pgfpathlineto{\pgfqpoint{1.418616in}{2.551791in}}%
\pgfpathlineto{\pgfqpoint{1.425464in}{2.426932in}}%
\pgfpathlineto{\pgfqpoint{1.427175in}{2.512587in}}%
\pgfpathlineto{\pgfqpoint{1.431452in}{2.541341in}}%
\pgfpathlineto{\pgfqpoint{1.434872in}{2.452726in}}%
\pgfpathlineto{\pgfqpoint{1.440002in}{2.437987in}}%
\pgfpathlineto{\pgfqpoint{1.444278in}{2.504856in}}%
\pgfpathlineto{\pgfqpoint{1.451122in}{2.436235in}}%
\pgfpathlineto{\pgfqpoint{1.453691in}{2.535906in}}%
\pgfpathlineto{\pgfqpoint{1.457970in}{2.449585in}}%
\pgfpathlineto{\pgfqpoint{1.463106in}{2.437203in}}%
\pgfpathlineto{\pgfqpoint{1.468244in}{2.536087in}}%
\pgfpathlineto{\pgfqpoint{1.470810in}{2.559283in}}%
\pgfpathlineto{\pgfqpoint{1.474232in}{2.453088in}}%
\pgfpathlineto{\pgfqpoint{1.480224in}{2.514581in}}%
\pgfpathlineto{\pgfqpoint{1.485359in}{2.441581in}}%
\pgfpathlineto{\pgfqpoint{1.486215in}{2.557773in}}%
\pgfpathlineto{\pgfqpoint{1.491345in}{2.500629in}}%
\pgfpathlineto{\pgfqpoint{1.496476in}{2.468432in}}%
\pgfpathlineto{\pgfqpoint{1.499039in}{2.574868in}}%
\pgfpathlineto{\pgfqpoint{1.505027in}{2.459372in}}%
\pgfpathlineto{\pgfqpoint{1.511013in}{2.539861in}}%
\pgfpathlineto{\pgfqpoint{1.512722in}{2.446262in}}%
\pgfpathlineto{\pgfqpoint{1.518707in}{2.430738in}}%
\pgfpathlineto{\pgfqpoint{1.520417in}{2.530257in}}%
\pgfpathlineto{\pgfqpoint{1.528112in}{2.462995in}}%
\pgfpathlineto{\pgfqpoint{1.531534in}{2.538200in}}%
\pgfpathlineto{\pgfqpoint{1.535810in}{2.445597in}}%
\pgfpathlineto{\pgfqpoint{1.540941in}{2.558859in}}%
\pgfpathlineto{\pgfqpoint{1.541797in}{2.461515in}}%
\pgfpathlineto{\pgfqpoint{1.548638in}{2.506668in}}%
\pgfpathlineto{\pgfqpoint{1.551207in}{2.432580in}}%
\pgfpathlineto{\pgfqpoint{1.556343in}{2.434120in}}%
\pgfpathlineto{\pgfqpoint{1.558909in}{2.535843in}}%
\pgfpathlineto{\pgfqpoint{1.566605in}{2.550281in}}%
\pgfpathlineto{\pgfqpoint{1.567459in}{2.488124in}}%
\pgfpathlineto{\pgfqpoint{1.575167in}{2.437745in}}%
\pgfpathlineto{\pgfqpoint{1.577737in}{2.556986in}}%
\pgfpathlineto{\pgfqpoint{1.582014in}{2.419079in}}%
\pgfpathlineto{\pgfqpoint{1.585439in}{2.525394in}}%
\pgfpathlineto{\pgfqpoint{1.591423in}{2.470817in}}%
\pgfpathlineto{\pgfqpoint{1.596559in}{2.534757in}}%
\pgfpathlineto{\pgfqpoint{1.599126in}{2.428744in}}%
\pgfpathlineto{\pgfqpoint{1.603404in}{2.571241in}}%
\pgfpathlineto{\pgfqpoint{1.607680in}{2.461936in}}%
\pgfpathlineto{\pgfqpoint{1.613666in}{2.533428in}}%
\pgfpathlineto{\pgfqpoint{1.617087in}{2.470786in}}%
\pgfpathlineto{\pgfqpoint{1.618798in}{2.531979in}}%
\pgfpathlineto{\pgfqpoint{1.625639in}{2.579700in}}%
\pgfpathlineto{\pgfqpoint{1.630773in}{2.459914in}}%
\pgfpathlineto{\pgfqpoint{1.635054in}{2.558436in}}%
\pgfpathlineto{\pgfqpoint{1.636766in}{2.448196in}}%
\pgfpathlineto{\pgfqpoint{1.641889in}{2.568282in}}%
\pgfpathlineto{\pgfqpoint{1.648727in}{2.450671in}}%
\pgfpathlineto{\pgfqpoint{1.656428in}{2.523219in}}%
\pgfpathlineto{\pgfqpoint{1.658993in}{2.424395in}}%
\pgfpathlineto{\pgfqpoint{1.664122in}{2.570336in}}%
\pgfpathlineto{\pgfqpoint{1.666687in}{2.455745in}}%
\pgfpathlineto{\pgfqpoint{1.672676in}{2.585740in}}%
\pgfpathlineto{\pgfqpoint{1.676098in}{2.398148in}}%
\pgfpathlineto{\pgfqpoint{1.681229in}{2.527810in}}%
\pgfpathlineto{\pgfqpoint{1.682939in}{2.447319in}}%
\pgfpathlineto{\pgfqpoint{1.687214in}{2.485225in}}%
\pgfpathlineto{\pgfqpoint{1.693199in}{2.430314in}}%
\pgfpathlineto{\pgfqpoint{1.695764in}{2.524487in}}%
\pgfpathlineto{\pgfqpoint{1.700041in}{2.445416in}}%
\pgfpathlineto{\pgfqpoint{1.704314in}{2.428774in}}%
\pgfpathlineto{\pgfqpoint{1.709447in}{2.508904in}}%
\pgfpathlineto{\pgfqpoint{1.714577in}{2.476949in}}%
\pgfpathlineto{\pgfqpoint{1.723132in}{2.594590in}}%
\pgfpathlineto{\pgfqpoint{1.725698in}{2.493863in}}%
\pgfpathlineto{\pgfqpoint{1.730826in}{2.435751in}}%
\pgfpathlineto{\pgfqpoint{1.734249in}{2.521407in}}%
\pgfpathlineto{\pgfqpoint{1.738528in}{2.478157in}}%
\pgfpathlineto{\pgfqpoint{1.746231in}{2.442184in}}%
\pgfpathlineto{\pgfqpoint{1.748801in}{2.550281in}}%
\pgfpathlineto{\pgfqpoint{1.753081in}{2.430435in}}%
\pgfpathlineto{\pgfqpoint{1.758211in}{2.542611in}}%
\pgfpathlineto{\pgfqpoint{1.761630in}{2.501352in}}%
\pgfpathlineto{\pgfqpoint{1.764196in}{2.531616in}}%
\pgfpathlineto{\pgfqpoint{1.770178in}{2.457104in}}%
\pgfpathlineto{\pgfqpoint{1.775307in}{2.529864in}}%
\pgfpathlineto{\pgfqpoint{1.780436in}{2.537234in}}%
\pgfpathlineto{\pgfqpoint{1.781291in}{2.419442in}}%
\pgfpathlineto{\pgfqpoint{1.787279in}{2.538139in}}%
\pgfpathlineto{\pgfqpoint{1.790704in}{2.489392in}}%
\pgfpathlineto{\pgfqpoint{1.794127in}{2.572088in}}%
\pgfpathlineto{\pgfqpoint{1.803538in}{2.429772in}}%
\pgfpathlineto{\pgfqpoint{1.808671in}{2.549013in}}%
\pgfpathlineto{\pgfqpoint{1.811238in}{2.482929in}}%
\pgfpathlineto{\pgfqpoint{1.817226in}{2.577283in}}%
\pgfpathlineto{\pgfqpoint{1.821502in}{2.465231in}}%
\pgfpathlineto{\pgfqpoint{1.824922in}{2.549800in}}%
\pgfpathlineto{\pgfqpoint{1.829199in}{2.450069in}}%
\pgfpathlineto{\pgfqpoint{1.835185in}{2.534093in}}%
\pgfpathlineto{\pgfqpoint{1.838605in}{2.457981in}}%
\pgfpathlineto{\pgfqpoint{1.842028in}{2.547745in}}%
\pgfpathlineto{\pgfqpoint{1.847161in}{2.469337in}}%
\pgfpathlineto{\pgfqpoint{1.849731in}{2.552033in}}%
\pgfpathlineto{\pgfqpoint{1.854859in}{2.546264in}}%
\pgfpathlineto{\pgfqpoint{1.861702in}{2.393316in}}%
\pgfpathlineto{\pgfqpoint{1.865977in}{2.534877in}}%
\pgfpathlineto{\pgfqpoint{1.869400in}{2.460578in}}%
\pgfpathlineto{\pgfqpoint{1.871963in}{2.585740in}}%
\pgfpathlineto{\pgfqpoint{1.875380in}{2.608635in}}%
\pgfpathlineto{\pgfqpoint{1.880514in}{2.452060in}}%
\pgfpathlineto{\pgfqpoint{1.886505in}{2.548832in}}%
\pgfpathlineto{\pgfqpoint{1.888217in}{2.450550in}}%
\pgfpathlineto{\pgfqpoint{1.893350in}{2.566409in}}%
\pgfpathlineto{\pgfqpoint{1.896771in}{2.475861in}}%
\pgfpathlineto{\pgfqpoint{1.903618in}{2.538200in}}%
\pgfpathlineto{\pgfqpoint{1.906183in}{2.459310in}}%
\pgfpathlineto{\pgfqpoint{1.909606in}{2.515728in}}%
\pgfpathlineto{\pgfqpoint{1.916449in}{2.575047in}}%
\pgfpathlineto{\pgfqpoint{1.921587in}{2.445960in}}%
\pgfpathlineto{\pgfqpoint{1.922443in}{2.555053in}}%
\pgfpathlineto{\pgfqpoint{1.927577in}{2.456653in}}%
\pgfpathlineto{\pgfqpoint{1.933561in}{2.495796in}}%
\pgfpathlineto{\pgfqpoint{1.937837in}{2.449887in}}%
\pgfpathlineto{\pgfqpoint{1.941258in}{2.577767in}}%
\pgfpathlineto{\pgfqpoint{1.947248in}{2.464747in}}%
\pgfpathlineto{\pgfqpoint{1.949816in}{2.543274in}}%
\pgfpathlineto{\pgfqpoint{1.955805in}{2.523280in}}%
\pgfpathlineto{\pgfqpoint{1.958374in}{2.408056in}}%
\pgfpathlineto{\pgfqpoint{1.960940in}{2.481781in}}%
\pgfpathlineto{\pgfqpoint{1.965220in}{2.559222in}}%
\pgfpathlineto{\pgfqpoint{1.972063in}{2.459975in}}%
\pgfpathlineto{\pgfqpoint{1.974626in}{2.559131in}}%
\pgfpathlineto{\pgfqpoint{1.980610in}{2.460459in}}%
\pgfpathlineto{\pgfqpoint{1.982321in}{2.542308in}}%
\pgfpathlineto{\pgfqpoint{1.986595in}{2.468644in}}%
\pgfpathlineto{\pgfqpoint{1.993441in}{2.555446in}}%
\pgfpathlineto{\pgfqpoint{1.995151in}{2.559041in}}%
\pgfpathlineto{\pgfqpoint{1.999427in}{2.460519in}}%
\pgfpathlineto{\pgfqpoint{2.003702in}{2.524308in}}%
\pgfpathlineto{\pgfqpoint{2.009691in}{2.444511in}}%
\pgfpathlineto{\pgfqpoint{2.013969in}{2.546054in}}%
\pgfpathlineto{\pgfqpoint{2.019956in}{2.592324in}}%
\pgfpathlineto{\pgfqpoint{2.024234in}{2.460699in}}%
\pgfpathlineto{\pgfqpoint{2.025946in}{2.528475in}}%
\pgfpathlineto{\pgfqpoint{2.031077in}{2.481086in}}%
\pgfpathlineto{\pgfqpoint{2.033644in}{2.576376in}}%
\pgfpathlineto{\pgfqpoint{2.039628in}{2.482929in}}%
\pgfpathlineto{\pgfqpoint{2.043906in}{2.568887in}}%
\pgfpathlineto{\pgfqpoint{2.048183in}{2.476647in}}%
\pgfpathlineto{\pgfqpoint{2.050751in}{2.582660in}}%
\pgfpathlineto{\pgfqpoint{2.055027in}{2.469579in}}%
\pgfpathlineto{\pgfqpoint{2.060159in}{2.570941in}}%
\pgfpathlineto{\pgfqpoint{2.064434in}{2.476223in}}%
\pgfpathlineto{\pgfqpoint{2.068710in}{2.417781in}}%
\pgfpathlineto{\pgfqpoint{2.072131in}{2.524066in}}%
\pgfpathlineto{\pgfqpoint{2.076406in}{2.474232in}}%
\pgfpathlineto{\pgfqpoint{2.080685in}{2.566711in}}%
\pgfpathlineto{\pgfqpoint{2.088377in}{2.561879in}}%
\pgfpathlineto{\pgfqpoint{2.090942in}{2.458042in}}%
\pgfpathlineto{\pgfqpoint{2.093509in}{2.572693in}}%
\pgfpathlineto{\pgfqpoint{2.100359in}{2.447712in}}%
\pgfpathlineto{\pgfqpoint{2.102070in}{2.543455in}}%
\pgfpathlineto{\pgfqpoint{2.109774in}{2.570094in}}%
\pgfpathlineto{\pgfqpoint{2.110631in}{2.483955in}}%
\pgfpathlineto{\pgfqpoint{2.116622in}{2.597156in}}%
\pgfpathlineto{\pgfqpoint{2.122605in}{2.489271in}}%
\pgfpathlineto{\pgfqpoint{2.123462in}{2.557107in}}%
\pgfpathlineto{\pgfqpoint{2.130310in}{2.440765in}}%
\pgfpathlineto{\pgfqpoint{2.133734in}{2.515125in}}%
\pgfpathlineto{\pgfqpoint{2.139720in}{2.481479in}}%
\pgfpathlineto{\pgfqpoint{2.140576in}{2.583686in}}%
\pgfpathlineto{\pgfqpoint{2.146558in}{2.435298in}}%
\pgfpathlineto{\pgfqpoint{2.149976in}{2.562302in}}%
\pgfpathlineto{\pgfqpoint{2.156813in}{2.425966in}}%
\pgfpathlineto{\pgfqpoint{2.159379in}{2.606821in}}%
\pgfpathlineto{\pgfqpoint{2.162803in}{2.477068in}}%
\pgfpathlineto{\pgfqpoint{2.169647in}{2.469428in}}%
\pgfpathlineto{\pgfqpoint{2.172214in}{2.608724in}}%
\pgfpathlineto{\pgfqpoint{2.174781in}{2.487610in}}%
\pgfpathlineto{\pgfqpoint{2.181627in}{2.463779in}}%
\pgfpathlineto{\pgfqpoint{2.184193in}{2.529199in}}%
\pgfpathlineto{\pgfqpoint{2.191037in}{2.446926in}}%
\pgfpathlineto{\pgfqpoint{2.195315in}{2.533005in}}%
\pgfpathlineto{\pgfqpoint{2.199592in}{2.489755in}}%
\pgfpathlineto{\pgfqpoint{2.200449in}{2.512980in}}%
\pgfpathlineto{\pgfqpoint{2.205581in}{2.465591in}}%
\pgfpathlineto{\pgfqpoint{2.209001in}{2.529803in}}%
\pgfpathlineto{\pgfqpoint{2.213282in}{2.559643in}}%
\pgfpathlineto{\pgfqpoint{2.220121in}{2.455564in}}%
\pgfpathlineto{\pgfqpoint{2.224396in}{2.577525in}}%
\pgfpathlineto{\pgfqpoint{2.226107in}{2.493772in}}%
\pgfpathlineto{\pgfqpoint{2.230384in}{2.550342in}}%
\pgfpathlineto{\pgfqpoint{2.236373in}{2.483955in}}%
\pgfpathlineto{\pgfqpoint{2.240649in}{2.580666in}}%
\pgfpathlineto{\pgfqpoint{2.245780in}{2.502136in}}%
\pgfpathlineto{\pgfqpoint{2.249197in}{2.599027in}}%
\pgfpathlineto{\pgfqpoint{2.252617in}{2.485767in}}%
\pgfpathlineto{\pgfqpoint{2.256038in}{2.598061in}}%
\pgfpathlineto{\pgfqpoint{2.261169in}{2.462088in}}%
\pgfpathlineto{\pgfqpoint{2.265446in}{2.457437in}}%
\pgfpathlineto{\pgfqpoint{2.270579in}{2.563691in}}%
\pgfpathlineto{\pgfqpoint{2.274003in}{2.425966in}}%
\pgfpathlineto{\pgfqpoint{2.279133in}{2.588215in}}%
\pgfpathlineto{\pgfqpoint{2.281703in}{2.524487in}}%
\pgfpathlineto{\pgfqpoint{2.288549in}{2.486614in}}%
\pgfpathlineto{\pgfqpoint{2.293687in}{2.559885in}}%
\pgfpathlineto{\pgfqpoint{2.294543in}{2.473988in}}%
\pgfpathlineto{\pgfqpoint{2.300531in}{2.545449in}}%
\pgfpathlineto{\pgfqpoint{2.306519in}{2.450490in}}%
\pgfpathlineto{\pgfqpoint{2.309941in}{2.603922in}}%
\pgfpathlineto{\pgfqpoint{2.311649in}{2.488245in}}%
\pgfpathlineto{\pgfqpoint{2.317638in}{2.565322in}}%
\pgfpathlineto{\pgfqpoint{2.322768in}{2.435540in}}%
\pgfpathlineto{\pgfqpoint{2.324477in}{2.481177in}}%
\pgfpathlineto{\pgfqpoint{2.332180in}{2.436687in}}%
\pgfpathlineto{\pgfqpoint{2.333891in}{2.510896in}}%
\pgfpathlineto{\pgfqpoint{2.339879in}{2.472599in}}%
\pgfpathlineto{\pgfqpoint{2.341587in}{2.550191in}}%
\pgfpathlineto{\pgfqpoint{2.347573in}{2.470484in}}%
\pgfpathlineto{\pgfqpoint{2.353560in}{2.472599in}}%
\pgfpathlineto{\pgfqpoint{2.356982in}{2.581934in}}%
\pgfpathlineto{\pgfqpoint{2.359552in}{2.517119in}}%
\pgfpathlineto{\pgfqpoint{2.365540in}{2.434907in}}%
\pgfpathlineto{\pgfqpoint{2.368963in}{2.546778in}}%
\pgfpathlineto{\pgfqpoint{2.373239in}{2.550221in}}%
\pgfpathlineto{\pgfqpoint{2.379228in}{2.445597in}}%
\pgfpathlineto{\pgfqpoint{2.380084in}{2.583867in}}%
\pgfpathlineto{\pgfqpoint{2.386070in}{2.464686in}}%
\pgfpathlineto{\pgfqpoint{2.388635in}{2.522042in}}%
\pgfpathlineto{\pgfqpoint{2.395480in}{2.489936in}}%
\pgfpathlineto{\pgfqpoint{2.398901in}{2.575168in}}%
\pgfpathlineto{\pgfqpoint{2.404887in}{2.495433in}}%
\pgfpathlineto{\pgfqpoint{2.406597in}{2.547322in}}%
\pgfpathlineto{\pgfqpoint{2.412580in}{2.451034in}}%
\pgfpathlineto{\pgfqpoint{2.415143in}{2.554087in}}%
\pgfpathlineto{\pgfqpoint{2.421985in}{2.462269in}}%
\pgfpathlineto{\pgfqpoint{2.424553in}{2.530166in}}%
\pgfpathlineto{\pgfqpoint{2.429680in}{2.556503in}}%
\pgfpathlineto{\pgfqpoint{2.433100in}{2.487761in}}%
\pgfpathlineto{\pgfqpoint{2.439086in}{2.555476in}}%
\pgfpathlineto{\pgfqpoint{2.440796in}{2.525092in}}%
\pgfpathlineto{\pgfqpoint{2.445075in}{2.474199in}}%
\pgfpathlineto{\pgfqpoint{2.449351in}{2.589546in}}%
\pgfpathlineto{\pgfqpoint{2.455339in}{2.462269in}}%
\pgfpathlineto{\pgfqpoint{2.460475in}{2.579216in}}%
\pgfpathlineto{\pgfqpoint{2.464751in}{2.518750in}}%
\pgfpathlineto{\pgfqpoint{2.468174in}{2.601265in}}%
\pgfpathlineto{\pgfqpoint{2.471595in}{2.492595in}}%
\pgfpathlineto{\pgfqpoint{2.474162in}{2.564478in}}%
\pgfpathlineto{\pgfqpoint{2.480149in}{2.466105in}}%
\pgfpathlineto{\pgfqpoint{2.484427in}{2.532523in}}%
\pgfpathlineto{\pgfqpoint{2.489559in}{2.451700in}}%
\pgfpathlineto{\pgfqpoint{2.492982in}{2.570036in}}%
\pgfpathlineto{\pgfqpoint{2.497259in}{2.440041in}}%
\pgfpathlineto{\pgfqpoint{2.500682in}{2.539107in}}%
\pgfpathlineto{\pgfqpoint{2.505817in}{2.560793in}}%
\pgfpathlineto{\pgfqpoint{2.508382in}{2.466680in}}%
\pgfpathlineto{\pgfqpoint{2.512659in}{2.535845in}}%
\pgfpathlineto{\pgfqpoint{2.519502in}{2.439739in}}%
\pgfpathlineto{\pgfqpoint{2.522926in}{2.534063in}}%
\pgfpathlineto{\pgfqpoint{2.526346in}{2.571062in}}%
\pgfpathlineto{\pgfqpoint{2.531478in}{2.477494in}}%
\pgfpathlineto{\pgfqpoint{2.535758in}{2.456532in}}%
\pgfpathlineto{\pgfqpoint{2.538324in}{2.556021in}}%
\pgfpathlineto{\pgfqpoint{2.544308in}{2.496460in}}%
\pgfpathlineto{\pgfqpoint{2.546873in}{2.564538in}}%
\pgfpathlineto{\pgfqpoint{2.551150in}{2.472208in}}%
\pgfpathlineto{\pgfqpoint{2.555422in}{2.535089in}}%
\pgfpathlineto{\pgfqpoint{2.563119in}{2.454477in}}%
\pgfpathlineto{\pgfqpoint{2.563974in}{2.538684in}}%
\pgfpathlineto{\pgfqpoint{2.569960in}{2.443182in}}%
\pgfpathlineto{\pgfqpoint{2.575949in}{2.584956in}}%
\pgfpathlineto{\pgfqpoint{2.576804in}{2.486704in}}%
\pgfpathlineto{\pgfqpoint{2.581931in}{2.436296in}}%
\pgfpathlineto{\pgfqpoint{2.587916in}{2.595044in}}%
\pgfpathlineto{\pgfqpoint{2.590484in}{2.451700in}}%
\pgfpathlineto{\pgfqpoint{2.595611in}{2.538986in}}%
\pgfpathlineto{\pgfqpoint{2.599886in}{2.557168in}}%
\pgfpathlineto{\pgfqpoint{2.603308in}{2.423369in}}%
\pgfpathlineto{\pgfqpoint{2.606733in}{2.529261in}}%
\pgfpathlineto{\pgfqpoint{2.611867in}{2.495071in}}%
\pgfpathlineto{\pgfqpoint{2.615290in}{2.565383in}}%
\pgfpathlineto{\pgfqpoint{2.622129in}{2.464142in}}%
\pgfpathlineto{\pgfqpoint{2.624695in}{2.553483in}}%
\pgfpathlineto{\pgfqpoint{2.631542in}{2.445960in}}%
\pgfpathlineto{\pgfqpoint{2.633255in}{2.531918in}}%
\pgfpathlineto{\pgfqpoint{2.636679in}{2.566953in}}%
\pgfpathlineto{\pgfqpoint{2.643518in}{2.439316in}}%
\pgfpathlineto{\pgfqpoint{2.647797in}{2.530166in}}%
\pgfpathlineto{\pgfqpoint{2.652077in}{2.487037in}}%
\pgfpathlineto{\pgfqpoint{2.655500in}{2.555839in}}%
\pgfpathlineto{\pgfqpoint{2.659777in}{2.486614in}}%
\pgfpathlineto{\pgfqpoint{2.663199in}{2.565685in}}%
\pgfpathlineto{\pgfqpoint{2.666621in}{2.493440in}}%
\pgfpathlineto{\pgfqpoint{2.670900in}{2.562968in}}%
\pgfpathlineto{\pgfqpoint{2.676032in}{2.478157in}}%
\pgfpathlineto{\pgfqpoint{2.681163in}{2.473385in}}%
\pgfpathlineto{\pgfqpoint{2.686295in}{2.579216in}}%
\pgfpathlineto{\pgfqpoint{2.689720in}{2.478308in}}%
\pgfpathlineto{\pgfqpoint{2.693143in}{2.560067in}}%
\pgfpathlineto{\pgfqpoint{2.696565in}{2.460035in}}%
\pgfpathlineto{\pgfqpoint{2.703410in}{2.537839in}}%
\pgfpathlineto{\pgfqpoint{2.708544in}{2.434090in}}%
\pgfpathlineto{\pgfqpoint{2.710256in}{2.562605in}}%
\pgfpathlineto{\pgfqpoint{2.714525in}{2.466196in}}%
\pgfpathlineto{\pgfqpoint{2.718798in}{2.525032in}}%
\pgfpathlineto{\pgfqpoint{2.723934in}{2.461485in}}%
\pgfpathlineto{\pgfqpoint{2.728213in}{2.562363in}}%
\pgfpathlineto{\pgfqpoint{2.731635in}{2.473325in}}%
\pgfpathlineto{\pgfqpoint{2.736769in}{2.521528in}}%
\pgfpathlineto{\pgfqpoint{2.741046in}{2.460396in}}%
\pgfpathlineto{\pgfqpoint{2.743612in}{2.553362in}}%
\pgfpathlineto{\pgfqpoint{2.751304in}{2.430677in}}%
\pgfpathlineto{\pgfqpoint{2.752160in}{2.513797in}}%
\pgfpathlineto{\pgfqpoint{2.758145in}{2.497276in}}%
\pgfpathlineto{\pgfqpoint{2.762417in}{2.573840in}}%
\pgfpathlineto{\pgfqpoint{2.764980in}{2.510777in}}%
\pgfpathlineto{\pgfqpoint{2.772681in}{2.478520in}}%
\pgfpathlineto{\pgfqpoint{2.775247in}{2.533791in}}%
\pgfpathlineto{\pgfqpoint{2.781232in}{2.562605in}}%
\pgfpathlineto{\pgfqpoint{2.782089in}{2.481661in}}%
\pgfpathlineto{\pgfqpoint{2.789789in}{2.534517in}}%
\pgfpathlineto{\pgfqpoint{2.790642in}{2.446504in}}%
\pgfpathlineto{\pgfqpoint{2.794918in}{2.511319in}}%
\pgfpathlineto{\pgfqpoint{2.799189in}{2.451518in}}%
\pgfpathlineto{\pgfqpoint{2.806031in}{2.567679in}}%
\pgfpathlineto{\pgfqpoint{2.810305in}{2.609661in}}%
\pgfpathlineto{\pgfqpoint{2.812017in}{2.505582in}}%
\pgfpathlineto{\pgfqpoint{2.818005in}{2.572390in}}%
\pgfpathlineto{\pgfqpoint{2.821424in}{2.496339in}}%
\pgfpathlineto{\pgfqpoint{2.827409in}{2.577525in}}%
\pgfpathlineto{\pgfqpoint{2.831684in}{2.442819in}}%
\pgfpathlineto{\pgfqpoint{2.836818in}{2.590814in}}%
\pgfpathlineto{\pgfqpoint{2.837671in}{2.480030in}}%
\pgfpathlineto{\pgfqpoint{2.844515in}{2.561034in}}%
\pgfpathlineto{\pgfqpoint{2.847078in}{2.478338in}}%
\pgfpathlineto{\pgfqpoint{2.853918in}{2.407542in}}%
\pgfpathlineto{\pgfqpoint{2.856482in}{2.546868in}}%
\pgfpathlineto{\pgfqpoint{2.860755in}{2.464505in}}%
\pgfpathlineto{\pgfqpoint{2.866743in}{2.592024in}}%
\pgfpathlineto{\pgfqpoint{2.868455in}{2.490601in}}%
\pgfpathlineto{\pgfqpoint{2.871877in}{2.547806in}}%
\pgfpathlineto{\pgfqpoint{2.877860in}{2.470726in}}%
\pgfpathlineto{\pgfqpoint{2.882135in}{2.544723in}}%
\pgfpathlineto{\pgfqpoint{2.886413in}{2.565685in}}%
\pgfpathlineto{\pgfqpoint{2.889835in}{2.430284in}}%
\pgfpathlineto{\pgfqpoint{2.896681in}{2.580424in}}%
\pgfpathlineto{\pgfqpoint{2.898392in}{2.476103in}}%
\pgfpathlineto{\pgfqpoint{2.905233in}{2.564115in}}%
\pgfpathlineto{\pgfqpoint{2.906089in}{2.480421in}}%
\pgfpathlineto{\pgfqpoint{2.913784in}{2.535301in}}%
\pgfpathlineto{\pgfqpoint{2.917204in}{2.429046in}}%
\pgfpathlineto{\pgfqpoint{2.918914in}{2.543516in}}%
\pgfpathlineto{\pgfqpoint{2.923190in}{2.415336in}}%
\pgfpathlineto{\pgfqpoint{2.929171in}{2.520925in}}%
\pgfpathlineto{\pgfqpoint{2.932595in}{2.487460in}}%
\pgfpathlineto{\pgfqpoint{2.938585in}{2.646872in}}%
\pgfpathlineto{\pgfqpoint{2.942004in}{2.551249in}}%
\pgfpathlineto{\pgfqpoint{2.945424in}{2.659316in}}%
\pgfpathlineto{\pgfqpoint{2.949701in}{2.570157in}}%
\pgfpathlineto{\pgfqpoint{2.956540in}{2.656296in}}%
\pgfpathlineto{\pgfqpoint{2.960820in}{2.538926in}}%
\pgfpathlineto{\pgfqpoint{2.963385in}{2.630079in}}%
\pgfpathlineto{\pgfqpoint{2.966806in}{2.655631in}}%
\pgfpathlineto{\pgfqpoint{2.970229in}{2.505943in}}%
\pgfpathlineto{\pgfqpoint{2.975364in}{2.545570in}}%
\pgfpathlineto{\pgfqpoint{2.980495in}{2.477975in}}%
\pgfpathlineto{\pgfqpoint{2.986484in}{2.516998in}}%
\pgfpathlineto{\pgfqpoint{2.990761in}{2.464898in}}%
\pgfpathlineto{\pgfqpoint{2.992470in}{2.561879in}}%
\pgfpathlineto{\pgfqpoint{2.998455in}{2.457497in}}%
\pgfpathlineto{\pgfqpoint{3.001024in}{2.575289in}}%
\pgfpathlineto{\pgfqpoint{3.007011in}{2.474472in}}%
\pgfpathlineto{\pgfqpoint{3.009578in}{2.558436in}}%
\pgfpathlineto{\pgfqpoint{3.012997in}{2.468732in}}%
\pgfpathlineto{\pgfqpoint{3.018130in}{2.465833in}}%
\pgfpathlineto{\pgfqpoint{3.024973in}{2.531253in}}%
\pgfpathlineto{\pgfqpoint{3.029249in}{2.446805in}}%
\pgfpathlineto{\pgfqpoint{3.031813in}{2.568161in}}%
\pgfpathlineto{\pgfqpoint{3.035233in}{2.516605in}}%
\pgfpathlineto{\pgfqpoint{3.040363in}{2.562875in}}%
\pgfpathlineto{\pgfqpoint{3.042930in}{2.468793in}}%
\pgfpathlineto{\pgfqpoint{3.048063in}{2.574082in}}%
\pgfpathlineto{\pgfqpoint{3.052341in}{2.462571in}}%
\pgfpathlineto{\pgfqpoint{3.055765in}{2.573961in}}%
\pgfpathlineto{\pgfqpoint{3.062609in}{2.496339in}}%
\pgfpathlineto{\pgfqpoint{3.065175in}{2.539589in}}%
\pgfpathlineto{\pgfqpoint{3.071163in}{2.529440in}}%
\pgfpathlineto{\pgfqpoint{3.073730in}{2.472780in}}%
\pgfpathlineto{\pgfqpoint{3.078860in}{2.542248in}}%
\pgfpathlineto{\pgfqpoint{3.084849in}{2.439406in}}%
\pgfpathlineto{\pgfqpoint{3.086559in}{2.569429in}}%
\pgfpathlineto{\pgfqpoint{3.090835in}{2.469760in}}%
\pgfpathlineto{\pgfqpoint{3.096823in}{2.421043in}}%
\pgfpathlineto{\pgfqpoint{3.099388in}{2.517180in}}%
\pgfpathlineto{\pgfqpoint{3.102807in}{2.397878in}}%
\pgfpathlineto{\pgfqpoint{3.107087in}{2.532160in}}%
\pgfpathlineto{\pgfqpoint{3.113076in}{2.470274in}}%
\pgfpathlineto{\pgfqpoint{3.119072in}{2.587008in}}%
\pgfpathlineto{\pgfqpoint{3.119928in}{2.510535in}}%
\pgfpathlineto{\pgfqpoint{3.125916in}{2.474714in}}%
\pgfpathlineto{\pgfqpoint{3.128480in}{2.538805in}}%
\pgfpathlineto{\pgfqpoint{3.136178in}{2.608996in}}%
\pgfpathlineto{\pgfqpoint{3.137034in}{2.481479in}}%
\pgfpathlineto{\pgfqpoint{3.144728in}{2.521467in}}%
\pgfpathlineto{\pgfqpoint{3.145581in}{2.452907in}}%
\pgfpathlineto{\pgfqpoint{3.153277in}{2.425936in}}%
\pgfpathlineto{\pgfqpoint{3.154131in}{2.542066in}}%
\pgfpathlineto{\pgfqpoint{3.161832in}{2.465773in}}%
\pgfpathlineto{\pgfqpoint{3.164395in}{2.531737in}}%
\pgfpathlineto{\pgfqpoint{3.168665in}{2.560672in}}%
\pgfpathlineto{\pgfqpoint{3.172086in}{2.418656in}}%
\pgfpathlineto{\pgfqpoint{3.175506in}{2.568705in}}%
\pgfpathlineto{\pgfqpoint{3.181495in}{2.499479in}}%
\pgfpathlineto{\pgfqpoint{3.184063in}{2.590935in}}%
\pgfpathlineto{\pgfqpoint{3.191760in}{2.513192in}}%
\pgfpathlineto{\pgfqpoint{3.196038in}{2.556503in}}%
\pgfpathlineto{\pgfqpoint{3.197748in}{2.434423in}}%
\pgfpathlineto{\pgfqpoint{3.202885in}{2.569128in}}%
\pgfpathlineto{\pgfqpoint{3.205451in}{2.454659in}}%
\pgfpathlineto{\pgfqpoint{3.211439in}{2.502862in}}%
\pgfpathlineto{\pgfqpoint{3.217427in}{2.522856in}}%
\pgfpathlineto{\pgfqpoint{3.221702in}{2.452302in}}%
\pgfpathlineto{\pgfqpoint{3.224266in}{2.528112in}}%
\pgfpathlineto{\pgfqpoint{3.226834in}{2.437261in}}%
\pgfpathlineto{\pgfqpoint{3.231112in}{2.519232in}}%
\pgfpathlineto{\pgfqpoint{3.237954in}{2.474804in}}%
\pgfpathlineto{\pgfqpoint{3.243083in}{2.541401in}}%
\pgfpathlineto{\pgfqpoint{3.246503in}{2.455685in}}%
\pgfpathlineto{\pgfqpoint{3.248209in}{2.549979in}}%
\pgfpathlineto{\pgfqpoint{3.255909in}{2.558496in}}%
\pgfpathlineto{\pgfqpoint{3.256765in}{2.447349in}}%
\pgfpathlineto{\pgfqpoint{3.264463in}{2.558920in}}%
\pgfpathlineto{\pgfqpoint{3.268739in}{2.479788in}}%
\pgfpathlineto{\pgfqpoint{3.273013in}{2.531283in}}%
\pgfpathlineto{\pgfqpoint{3.276437in}{2.544905in}}%
\pgfpathlineto{\pgfqpoint{3.281573in}{2.421011in}}%
\pgfpathlineto{\pgfqpoint{3.283283in}{2.564355in}}%
\pgfpathlineto{\pgfqpoint{3.287561in}{2.495552in}}%
\pgfpathlineto{\pgfqpoint{3.294401in}{2.571302in}}%
\pgfpathlineto{\pgfqpoint{3.298673in}{2.464505in}}%
\pgfpathlineto{\pgfqpoint{3.299528in}{2.549798in}}%
\pgfpathlineto{\pgfqpoint{3.303802in}{2.461182in}}%
\pgfpathlineto{\pgfqpoint{3.308077in}{2.500747in}}%
\pgfpathlineto{\pgfqpoint{3.315769in}{2.437745in}}%
\pgfpathlineto{\pgfqpoint{3.316624in}{2.560430in}}%
\pgfpathlineto{\pgfqpoint{3.324328in}{2.574866in}}%
\pgfpathlineto{\pgfqpoint{3.325184in}{2.471694in}}%
\pgfpathlineto{\pgfqpoint{3.329460in}{2.444450in}}%
\pgfpathlineto{\pgfqpoint{3.333737in}{2.563570in}}%
\pgfpathlineto{\pgfqpoint{3.338014in}{2.480451in}}%
\pgfpathlineto{\pgfqpoint{3.342293in}{2.560188in}}%
\pgfpathlineto{\pgfqpoint{3.349137in}{2.444087in}}%
\pgfpathlineto{\pgfqpoint{3.351703in}{2.533065in}}%
\pgfpathlineto{\pgfqpoint{3.355122in}{2.487216in}}%
\pgfpathlineto{\pgfqpoint{3.361114in}{2.566590in}}%
\pgfpathlineto{\pgfqpoint{3.365390in}{2.457376in}}%
\pgfpathlineto{\pgfqpoint{3.371379in}{2.566772in}}%
\pgfpathlineto{\pgfqpoint{3.372234in}{2.516635in}}%
\pgfpathlineto{\pgfqpoint{3.379931in}{2.375827in}}%
\pgfpathlineto{\pgfqpoint{3.380784in}{2.538803in}}%
\pgfpathlineto{\pgfqpoint{3.385062in}{2.470363in}}%
\pgfpathlineto{\pgfqpoint{3.389340in}{2.589544in}}%
\pgfpathlineto{\pgfqpoint{3.397035in}{2.443422in}}%
\pgfpathlineto{\pgfqpoint{3.397890in}{2.569126in}}%
\pgfpathlineto{\pgfqpoint{3.402168in}{2.581571in}}%
\pgfpathlineto{\pgfqpoint{3.406443in}{2.480330in}}%
\pgfpathlineto{\pgfqpoint{3.412430in}{2.553120in}}%
\pgfpathlineto{\pgfqpoint{3.414994in}{2.503044in}}%
\pgfpathlineto{\pgfqpoint{3.420976in}{2.540585in}}%
\pgfpathlineto{\pgfqpoint{3.423538in}{2.476887in}}%
\pgfpathlineto{\pgfqpoint{3.427816in}{2.569126in}}%
\pgfpathlineto{\pgfqpoint{3.434657in}{2.438227in}}%
\pgfpathlineto{\pgfqpoint{3.437223in}{2.529138in}}%
\pgfpathlineto{\pgfqpoint{3.442356in}{2.476735in}}%
\pgfpathlineto{\pgfqpoint{3.445779in}{2.570697in}}%
\pgfpathlineto{\pgfqpoint{3.449197in}{2.459277in}}%
\pgfpathlineto{\pgfqpoint{3.453471in}{2.546112in}}%
\pgfpathlineto{\pgfqpoint{3.458605in}{2.581389in}}%
\pgfpathlineto{\pgfqpoint{3.462882in}{2.402287in}}%
\pgfpathlineto{\pgfqpoint{3.467156in}{2.599271in}}%
\pgfpathlineto{\pgfqpoint{3.472285in}{2.457013in}}%
\pgfpathlineto{\pgfqpoint{3.475705in}{2.506064in}}%
\pgfpathlineto{\pgfqpoint{3.482550in}{2.439134in}}%
\pgfpathlineto{\pgfqpoint{3.485118in}{2.509688in}}%
\pgfpathlineto{\pgfqpoint{3.491104in}{2.444569in}}%
\pgfpathlineto{\pgfqpoint{3.495380in}{2.558978in}}%
\pgfpathlineto{\pgfqpoint{3.496237in}{2.416420in}}%
\pgfpathlineto{\pgfqpoint{3.501370in}{2.524971in}}%
\pgfpathlineto{\pgfqpoint{3.504790in}{2.422160in}}%
\pgfpathlineto{\pgfqpoint{3.512483in}{2.580121in}}%
\pgfpathlineto{\pgfqpoint{3.513340in}{2.479364in}}%
\pgfpathlineto{\pgfqpoint{3.519327in}{2.531797in}}%
\pgfpathlineto{\pgfqpoint{3.522750in}{2.450008in}}%
\pgfpathlineto{\pgfqpoint{3.529596in}{2.524699in}}%
\pgfpathlineto{\pgfqpoint{3.532161in}{2.415939in}}%
\pgfpathlineto{\pgfqpoint{3.534724in}{2.508299in}}%
\pgfpathlineto{\pgfqpoint{3.539002in}{2.417116in}}%
\pgfpathlineto{\pgfqpoint{3.543277in}{2.522161in}}%
\pgfpathlineto{\pgfqpoint{3.547556in}{2.433697in}}%
\pgfpathlineto{\pgfqpoint{3.555248in}{2.452784in}}%
\pgfpathlineto{\pgfqpoint{3.558668in}{2.548709in}}%
\pgfpathlineto{\pgfqpoint{3.565506in}{2.407721in}}%
\pgfpathlineto{\pgfqpoint{3.569782in}{2.538863in}}%
\pgfpathlineto{\pgfqpoint{3.574058in}{2.454145in}}%
\pgfpathlineto{\pgfqpoint{3.580899in}{2.554539in}}%
\pgfpathlineto{\pgfqpoint{3.583469in}{2.466980in}}%
\pgfpathlineto{\pgfqpoint{3.587741in}{2.497304in}}%
\pgfpathlineto{\pgfqpoint{3.591161in}{2.453570in}}%
\pgfpathlineto{\pgfqpoint{3.596294in}{2.572449in}}%
\pgfpathlineto{\pgfqpoint{3.599713in}{2.480421in}}%
\pgfpathlineto{\pgfqpoint{3.603993in}{2.539347in}}%
\pgfpathlineto{\pgfqpoint{3.608273in}{2.456590in}}%
\pgfpathlineto{\pgfqpoint{3.611692in}{2.569308in}}%
\pgfpathlineto{\pgfqpoint{3.619390in}{2.549918in}}%
\pgfpathlineto{\pgfqpoint{3.622816in}{2.439255in}}%
\pgfpathlineto{\pgfqpoint{3.625381in}{2.564657in}}%
\pgfpathlineto{\pgfqpoint{3.631368in}{2.495794in}}%
\pgfpathlineto{\pgfqpoint{3.635643in}{2.566832in}}%
\pgfpathlineto{\pgfqpoint{3.637355in}{2.491295in}}%
\pgfpathlineto{\pgfqpoint{3.643344in}{2.552394in}}%
\pgfpathlineto{\pgfqpoint{3.646770in}{2.435782in}}%
\pgfpathlineto{\pgfqpoint{3.651047in}{2.497728in}}%
\pgfpathlineto{\pgfqpoint{3.655322in}{2.448315in}}%
\pgfpathlineto{\pgfqpoint{3.658742in}{2.519322in}}%
\pgfpathlineto{\pgfqpoint{3.665581in}{2.462481in}}%
\pgfpathlineto{\pgfqpoint{3.667291in}{2.563026in}}%
\pgfpathlineto{\pgfqpoint{3.674993in}{2.462088in}}%
\pgfpathlineto{\pgfqpoint{3.677560in}{2.545901in}}%
\pgfpathlineto{\pgfqpoint{3.680976in}{2.430314in}}%
\pgfpathlineto{\pgfqpoint{3.686964in}{2.560309in}}%
\pgfpathlineto{\pgfqpoint{3.692094in}{2.437443in}}%
\pgfpathlineto{\pgfqpoint{3.692951in}{2.535240in}}%
\pgfpathlineto{\pgfqpoint{3.700650in}{2.561758in}}%
\pgfpathlineto{\pgfqpoint{3.701506in}{2.479667in}}%
\pgfpathlineto{\pgfqpoint{3.707493in}{2.440221in}}%
\pgfpathlineto{\pgfqpoint{3.710060in}{2.535301in}}%
\pgfpathlineto{\pgfqpoint{3.714341in}{2.529138in}}%
\pgfpathlineto{\pgfqpoint{3.721187in}{2.444992in}}%
\pgfpathlineto{\pgfqpoint{3.726321in}{2.539891in}}%
\pgfpathlineto{\pgfqpoint{3.727177in}{2.455262in}}%
\pgfpathlineto{\pgfqpoint{3.734016in}{2.463265in}}%
\pgfpathlineto{\pgfqpoint{3.735725in}{2.553964in}}%
\pgfpathlineto{\pgfqpoint{3.741708in}{2.601807in}}%
\pgfpathlineto{\pgfqpoint{3.745984in}{2.435267in}}%
\pgfpathlineto{\pgfqpoint{3.749406in}{2.520742in}}%
\pgfpathlineto{\pgfqpoint{3.756244in}{2.438408in}}%
\pgfpathlineto{\pgfqpoint{3.759666in}{2.578914in}}%
\pgfpathlineto{\pgfqpoint{3.762235in}{2.439948in}}%
\pgfpathlineto{\pgfqpoint{3.768221in}{2.521709in}}%
\pgfpathlineto{\pgfqpoint{3.771647in}{2.463265in}}%
\pgfpathlineto{\pgfqpoint{3.775066in}{2.555476in}}%
\pgfpathlineto{\pgfqpoint{3.778487in}{2.433969in}}%
\pgfpathlineto{\pgfqpoint{3.783618in}{2.415092in}}%
\pgfpathlineto{\pgfqpoint{3.787896in}{2.554055in}}%
\pgfpathlineto{\pgfqpoint{3.794743in}{2.510472in}}%
\pgfpathlineto{\pgfqpoint{3.799020in}{2.627782in}}%
\pgfpathlineto{\pgfqpoint{3.801584in}{2.512829in}}%
\pgfpathlineto{\pgfqpoint{3.805856in}{2.564839in}}%
\pgfpathlineto{\pgfqpoint{3.810133in}{2.473504in}}%
\pgfpathlineto{\pgfqpoint{3.814406in}{2.460457in}}%
\pgfpathlineto{\pgfqpoint{3.816973in}{2.611230in}}%
\pgfpathlineto{\pgfqpoint{3.821247in}{2.470030in}}%
\pgfpathlineto{\pgfqpoint{3.828940in}{2.564899in}}%
\pgfpathlineto{\pgfqpoint{3.830650in}{2.463900in}}%
\pgfpathlineto{\pgfqpoint{3.834922in}{2.556261in}}%
\pgfpathlineto{\pgfqpoint{3.839197in}{2.486914in}}%
\pgfpathlineto{\pgfqpoint{3.842619in}{2.588094in}}%
\pgfpathlineto{\pgfqpoint{3.848600in}{2.462027in}}%
\pgfpathlineto{\pgfqpoint{3.851170in}{2.560851in}}%
\pgfpathlineto{\pgfqpoint{3.856308in}{2.511107in}}%
\pgfpathlineto{\pgfqpoint{3.859732in}{2.569913in}}%
\pgfpathlineto{\pgfqpoint{3.864870in}{2.465470in}}%
\pgfpathlineto{\pgfqpoint{3.869999in}{2.542851in}}%
\pgfpathlineto{\pgfqpoint{3.875984in}{2.468127in}}%
\pgfpathlineto{\pgfqpoint{3.878551in}{2.545749in}}%
\pgfpathlineto{\pgfqpoint{3.882831in}{2.475558in}}%
\pgfpathlineto{\pgfqpoint{3.887107in}{2.455141in}}%
\pgfpathlineto{\pgfqpoint{3.890525in}{2.526902in}}%
\pgfpathlineto{\pgfqpoint{3.895656in}{2.486854in}}%
\pgfpathlineto{\pgfqpoint{3.899075in}{2.549253in}}%
\pgfpathlineto{\pgfqpoint{3.903348in}{2.491383in}}%
\pgfpathlineto{\pgfqpoint{3.908484in}{2.568824in}}%
\pgfpathlineto{\pgfqpoint{3.911052in}{2.441126in}}%
\pgfpathlineto{\pgfqpoint{3.917892in}{2.594134in}}%
\pgfpathlineto{\pgfqpoint{3.919603in}{2.514821in}}%
\pgfpathlineto{\pgfqpoint{3.925587in}{2.565683in}}%
\pgfpathlineto{\pgfqpoint{3.928153in}{2.522010in}}%
\pgfpathlineto{\pgfqpoint{3.934142in}{2.484315in}}%
\pgfpathlineto{\pgfqpoint{3.936707in}{2.602410in}}%
\pgfpathlineto{\pgfqpoint{3.943555in}{2.438860in}}%
\pgfpathlineto{\pgfqpoint{3.945268in}{2.556561in}}%
\pgfpathlineto{\pgfqpoint{3.949543in}{2.432911in}}%
\pgfpathlineto{\pgfqpoint{3.954676in}{2.439858in}}%
\pgfpathlineto{\pgfqpoint{3.961520in}{2.597698in}}%
\pgfpathlineto{\pgfqpoint{3.962375in}{2.493438in}}%
\pgfpathlineto{\pgfqpoint{3.970067in}{2.451455in}}%
\pgfpathlineto{\pgfqpoint{3.970924in}{2.575952in}}%
\pgfpathlineto{\pgfqpoint{3.976909in}{2.459007in}}%
\pgfpathlineto{\pgfqpoint{3.979475in}{2.533849in}}%
\pgfpathlineto{\pgfqpoint{3.984613in}{2.488363in}}%
\pgfpathlineto{\pgfqpoint{3.991456in}{2.464805in}}%
\pgfpathlineto{\pgfqpoint{3.992312in}{2.549372in}}%
\pgfpathlineto{\pgfqpoint{3.997445in}{2.464684in}}%
\pgfpathlineto{\pgfqpoint{4.000867in}{2.543816in}}%
\pgfpathlineto{\pgfqpoint{4.006003in}{2.477792in}}%
\pgfpathlineto{\pgfqpoint{4.011132in}{2.537776in}}%
\pgfpathlineto{\pgfqpoint{4.015412in}{2.440039in}}%
\pgfpathlineto{\pgfqpoint{4.018834in}{2.584772in}}%
\pgfpathlineto{\pgfqpoint{4.022252in}{2.480814in}}%
\pgfpathlineto{\pgfqpoint{4.029095in}{2.456953in}}%
\pgfpathlineto{\pgfqpoint{4.032514in}{2.549556in}}%
\pgfpathlineto{\pgfqpoint{4.035079in}{2.497486in}}%
\pgfpathlineto{\pgfqpoint{4.039357in}{2.552757in}}%
\pgfpathlineto{\pgfqpoint{4.046202in}{2.497365in}}%
\pgfpathlineto{\pgfqpoint{4.048767in}{2.579819in}}%
\pgfpathlineto{\pgfqpoint{4.054759in}{2.630681in}}%
\pgfpathlineto{\pgfqpoint{4.056470in}{2.518599in}}%
\pgfpathlineto{\pgfqpoint{4.063314in}{2.564687in}}%
\pgfpathlineto{\pgfqpoint{4.068446in}{2.392620in}}%
\pgfpathlineto{\pgfqpoint{4.070156in}{2.547138in}}%
\pgfpathlineto{\pgfqpoint{4.073578in}{2.460517in}}%
\pgfpathlineto{\pgfqpoint{4.079562in}{2.426085in}}%
\pgfpathlineto{\pgfqpoint{4.082985in}{2.543695in}}%
\pgfpathlineto{\pgfqpoint{4.093250in}{2.436052in}}%
\pgfpathlineto{\pgfqpoint{4.097525in}{2.569731in}}%
\pgfpathlineto{\pgfqpoint{4.102659in}{2.582115in}}%
\pgfpathlineto{\pgfqpoint{4.103512in}{2.442335in}}%
\pgfpathlineto{\pgfqpoint{4.108646in}{2.571785in}}%
\pgfpathlineto{\pgfqpoint{4.113785in}{2.439616in}}%
\pgfpathlineto{\pgfqpoint{4.119765in}{2.578007in}}%
\pgfpathlineto{\pgfqpoint{4.121474in}{2.520318in}}%
\pgfpathlineto{\pgfqpoint{4.128317in}{2.423970in}}%
\pgfpathlineto{\pgfqpoint{4.130028in}{2.476100in}}%
\pgfpathlineto{\pgfqpoint{4.136878in}{2.442999in}}%
\pgfpathlineto{\pgfqpoint{4.137733in}{2.517117in}}%
\pgfpathlineto{\pgfqpoint{4.142868in}{2.457919in}}%
\pgfpathlineto{\pgfqpoint{4.147142in}{2.471994in}}%
\pgfpathlineto{\pgfqpoint{4.149711in}{2.526600in}}%
\pgfpathlineto{\pgfqpoint{4.152280in}{2.462932in}}%
\pgfpathlineto{\pgfqpoint{4.158268in}{2.465892in}}%
\pgfpathlineto{\pgfqpoint{4.159124in}{2.569911in}}%
\pgfpathlineto{\pgfqpoint{4.162548in}{2.479846in}}%
\pgfpathlineto{\pgfqpoint{4.166829in}{2.477159in}}%
\pgfpathlineto{\pgfqpoint{4.170249in}{2.571483in}}%
\pgfpathlineto{\pgfqpoint{4.172814in}{2.495008in}}%
\pgfpathlineto{\pgfqpoint{4.177091in}{2.569610in}}%
\pgfpathlineto{\pgfqpoint{4.181364in}{2.490478in}}%
\pgfpathlineto{\pgfqpoint{4.184787in}{2.564718in}}%
\pgfpathlineto{\pgfqpoint{4.186496in}{2.471692in}}%
\pgfpathlineto{\pgfqpoint{4.191629in}{2.488847in}}%
\pgfpathlineto{\pgfqpoint{4.194194in}{2.563268in}}%
\pgfpathlineto{\pgfqpoint{4.196761in}{2.500626in}}%
\pgfpathlineto{\pgfqpoint{4.201034in}{2.584530in}}%
\pgfpathlineto{\pgfqpoint{4.204456in}{2.480118in}}%
\pgfpathlineto{\pgfqpoint{4.207878in}{2.577583in}}%
\pgfpathlineto{\pgfqpoint{4.210442in}{2.489271in}}%
\pgfpathlineto{\pgfqpoint{4.216431in}{2.421434in}}%
\pgfpathlineto{\pgfqpoint{4.218141in}{2.512708in}}%
\pgfpathlineto{\pgfqpoint{4.223272in}{2.484015in}}%
\pgfpathlineto{\pgfqpoint{4.224983in}{2.558678in}}%
\pgfpathlineto{\pgfqpoint{4.230116in}{2.475467in}}%
\pgfpathlineto{\pgfqpoint{4.230972in}{2.518748in}}%
\pgfpathlineto{\pgfqpoint{4.235248in}{2.489543in}}%
\pgfpathlineto{\pgfqpoint{4.238671in}{2.453028in}}%
\pgfpathlineto{\pgfqpoint{4.242092in}{2.528082in}}%
\pgfpathlineto{\pgfqpoint{4.246370in}{2.436898in}}%
\pgfpathlineto{\pgfqpoint{4.248080in}{2.504130in}}%
\pgfpathlineto{\pgfqpoint{4.254070in}{2.441700in}}%
\pgfpathlineto{\pgfqpoint{4.257490in}{2.516454in}}%
\pgfpathlineto{\pgfqpoint{4.259202in}{2.444871in}}%
\pgfpathlineto{\pgfqpoint{4.263479in}{2.465107in}}%
\pgfpathlineto{\pgfqpoint{4.266903in}{2.586040in}}%
\pgfpathlineto{\pgfqpoint{4.270325in}{2.460638in}}%
\pgfpathlineto{\pgfqpoint{4.273748in}{2.549072in}}%
\pgfpathlineto{\pgfqpoint{4.277169in}{2.475407in}}%
\pgfpathlineto{\pgfqpoint{4.278880in}{2.544421in}}%
\pgfpathlineto{\pgfqpoint{4.288282in}{2.462148in}}%
\pgfpathlineto{\pgfqpoint{4.289137in}{2.549556in}}%
\pgfpathlineto{\pgfqpoint{4.292562in}{2.605371in}}%
\pgfpathlineto{\pgfqpoint{4.295981in}{2.454899in}}%
\pgfpathlineto{\pgfqpoint{4.300259in}{2.535059in}}%
\pgfpathlineto{\pgfqpoint{4.304529in}{2.576981in}}%
\pgfpathlineto{\pgfqpoint{4.307096in}{2.496941in}}%
\pgfpathlineto{\pgfqpoint{4.311374in}{2.592203in}}%
\pgfpathlineto{\pgfqpoint{4.315651in}{2.502015in}}%
\pgfpathlineto{\pgfqpoint{4.318213in}{2.593953in}}%
\pgfpathlineto{\pgfqpoint{4.320779in}{2.505277in}}%
\pgfpathlineto{\pgfqpoint{4.324195in}{2.559039in}}%
\pgfpathlineto{\pgfqpoint{4.326759in}{2.478245in}}%
\pgfpathlineto{\pgfqpoint{4.330182in}{2.524608in}}%
\pgfpathlineto{\pgfqpoint{4.336171in}{2.486974in}}%
\pgfpathlineto{\pgfqpoint{4.338734in}{2.478878in}}%
\pgfpathlineto{\pgfqpoint{4.340445in}{2.564234in}}%
\pgfpathlineto{\pgfqpoint{4.346437in}{2.456772in}}%
\pgfpathlineto{\pgfqpoint{4.347293in}{2.526539in}}%
\pgfpathlineto{\pgfqpoint{4.353279in}{2.442999in}}%
\pgfpathlineto{\pgfqpoint{4.356698in}{2.571723in}}%
\pgfpathlineto{\pgfqpoint{4.357553in}{2.445444in}}%
\pgfpathlineto{\pgfqpoint{4.360978in}{2.529199in}}%
\pgfpathlineto{\pgfqpoint{4.366112in}{2.464684in}}%
\pgfpathlineto{\pgfqpoint{4.367823in}{2.549253in}}%
\pgfpathlineto{\pgfqpoint{4.371246in}{2.462993in}}%
\pgfpathlineto{\pgfqpoint{4.377235in}{2.553843in}}%
\pgfpathlineto{\pgfqpoint{4.378945in}{2.463537in}}%
\pgfpathlineto{\pgfqpoint{4.381508in}{2.556200in}}%
\pgfpathlineto{\pgfqpoint{4.385782in}{2.410741in}}%
\pgfpathlineto{\pgfqpoint{4.389205in}{2.602531in}}%
\pgfpathlineto{\pgfqpoint{4.392627in}{2.463507in}}%
\pgfpathlineto{\pgfqpoint{4.395194in}{2.533910in}}%
\pgfpathlineto{\pgfqpoint{4.399466in}{2.471117in}}%
\pgfpathlineto{\pgfqpoint{4.402888in}{2.523943in}}%
\pgfpathlineto{\pgfqpoint{4.407166in}{2.444509in}}%
\pgfpathlineto{\pgfqpoint{4.408878in}{2.512073in}}%
\pgfpathlineto{\pgfqpoint{4.414012in}{2.412644in}}%
\pgfpathlineto{\pgfqpoint{4.415722in}{2.495552in}}%
\pgfpathlineto{\pgfqpoint{4.419998in}{2.457074in}}%
\pgfpathlineto{\pgfqpoint{4.424273in}{2.573235in}}%
\pgfpathlineto{\pgfqpoint{4.425984in}{2.483652in}}%
\pgfpathlineto{\pgfqpoint{4.431970in}{2.490660in}}%
\pgfpathlineto{\pgfqpoint{4.435391in}{2.521105in}}%
\pgfpathlineto{\pgfqpoint{4.437960in}{2.470424in}}%
\pgfpathlineto{\pgfqpoint{4.439672in}{2.557347in}}%
\pgfpathlineto{\pgfqpoint{4.443953in}{2.582718in}}%
\pgfpathlineto{\pgfqpoint{4.447371in}{2.470605in}}%
\pgfpathlineto{\pgfqpoint{4.451649in}{2.531674in}}%
\pgfpathlineto{\pgfqpoint{4.455068in}{2.456227in}}%
\pgfpathlineto{\pgfqpoint{4.456777in}{2.520137in}}%
\pgfpathlineto{\pgfqpoint{4.461912in}{2.558978in}}%
\pgfpathlineto{\pgfqpoint{4.464478in}{2.502739in}}%
\pgfpathlineto{\pgfqpoint{4.469610in}{2.560851in}}%
\pgfpathlineto{\pgfqpoint{4.470466in}{2.476947in}}%
\pgfpathlineto{\pgfqpoint{4.473885in}{2.437743in}}%
\pgfpathlineto{\pgfqpoint{4.478161in}{2.547501in}}%
\pgfpathlineto{\pgfqpoint{4.482436in}{2.496639in}}%
\pgfpathlineto{\pgfqpoint{4.486712in}{2.530739in}}%
\pgfpathlineto{\pgfqpoint{4.490131in}{2.449522in}}%
\pgfpathlineto{\pgfqpoint{4.490987in}{2.556140in}}%
\pgfpathlineto{\pgfqpoint{4.494410in}{2.466496in}}%
\pgfpathlineto{\pgfqpoint{4.498682in}{2.553874in}}%
\pgfpathlineto{\pgfqpoint{4.501244in}{2.479483in}}%
\pgfpathlineto{\pgfqpoint{4.504663in}{2.543695in}}%
\pgfpathlineto{\pgfqpoint{4.508083in}{2.496306in}}%
\pgfpathlineto{\pgfqpoint{4.511500in}{2.557166in}}%
\pgfpathlineto{\pgfqpoint{4.514921in}{2.455231in}}%
\pgfpathlineto{\pgfqpoint{4.520052in}{2.524427in}}%
\pgfpathlineto{\pgfqpoint{4.524328in}{2.461725in}}%
\pgfpathlineto{\pgfqpoint{4.525183in}{2.554569in}}%
\pgfpathlineto{\pgfqpoint{4.531176in}{2.475377in}}%
\pgfpathlineto{\pgfqpoint{4.532887in}{2.546050in}}%
\pgfpathlineto{\pgfqpoint{4.535450in}{2.443120in}}%
\pgfpathlineto{\pgfqpoint{4.538868in}{2.737601in}}%
\pgfpathlineto{\pgfqpoint{4.542287in}{2.840776in}}%
\pgfpathlineto{\pgfqpoint{4.545713in}{2.776291in}}%
\pgfpathlineto{\pgfqpoint{4.549136in}{2.797947in}}%
\pgfpathlineto{\pgfqpoint{4.552557in}{2.735487in}}%
\pgfpathlineto{\pgfqpoint{4.557688in}{2.817338in}}%
\pgfpathlineto{\pgfqpoint{4.560254in}{2.727032in}}%
\pgfpathlineto{\pgfqpoint{4.564531in}{2.776775in}}%
\pgfpathlineto{\pgfqpoint{4.566243in}{2.692539in}}%
\pgfpathlineto{\pgfqpoint{4.569658in}{2.777892in}}%
\pgfpathlineto{\pgfqpoint{4.573939in}{2.714134in}}%
\pgfpathlineto{\pgfqpoint{4.577361in}{2.795350in}}%
\pgfpathlineto{\pgfqpoint{4.581640in}{2.666745in}}%
\pgfpathlineto{\pgfqpoint{4.583350in}{2.830053in}}%
\pgfpathlineto{\pgfqpoint{4.588482in}{2.836004in}}%
\pgfpathlineto{\pgfqpoint{4.590190in}{2.753731in}}%
\pgfpathlineto{\pgfqpoint{4.593612in}{2.883000in}}%
\pgfpathlineto{\pgfqpoint{4.597035in}{2.766718in}}%
\pgfpathlineto{\pgfqpoint{4.600455in}{2.817580in}}%
\pgfpathlineto{\pgfqpoint{4.605589in}{2.788042in}}%
\pgfpathlineto{\pgfqpoint{4.608155in}{2.491688in}}%
\pgfpathlineto{\pgfqpoint{4.612433in}{2.543032in}}%
\pgfpathlineto{\pgfqpoint{4.616710in}{2.440039in}}%
\pgfpathlineto{\pgfqpoint{4.617565in}{2.494284in}}%
\pgfpathlineto{\pgfqpoint{4.621838in}{2.524427in}}%
\pgfpathlineto{\pgfqpoint{4.624405in}{2.574684in}}%
\pgfpathlineto{\pgfqpoint{4.627827in}{2.454175in}}%
\pgfpathlineto{\pgfqpoint{4.632102in}{2.558133in}}%
\pgfpathlineto{\pgfqpoint{4.635525in}{2.447349in}}%
\pgfpathlineto{\pgfqpoint{4.638091in}{2.543032in}}%
\pgfpathlineto{\pgfqpoint{4.642368in}{2.500385in}}%
\pgfpathlineto{\pgfqpoint{4.645787in}{2.605432in}}%
\pgfpathlineto{\pgfqpoint{4.649207in}{2.515486in}}%
\pgfpathlineto{\pgfqpoint{4.652631in}{2.541280in}}%
\pgfpathlineto{\pgfqpoint{4.656051in}{2.473837in}}%
\pgfpathlineto{\pgfqpoint{4.661182in}{2.528231in}}%
\pgfpathlineto{\pgfqpoint{4.662894in}{2.462539in}}%
\pgfpathlineto{\pgfqpoint{4.666317in}{2.505217in}}%
\pgfpathlineto{\pgfqpoint{4.668881in}{2.453750in}}%
\pgfpathlineto{\pgfqpoint{4.674015in}{2.474893in}}%
\pgfpathlineto{\pgfqpoint{4.677437in}{2.553450in}}%
\pgfpathlineto{\pgfqpoint{4.679148in}{2.498996in}}%
\pgfpathlineto{\pgfqpoint{4.682567in}{2.439555in}}%
\pgfpathlineto{\pgfqpoint{4.687701in}{2.542367in}}%
\pgfpathlineto{\pgfqpoint{4.689413in}{2.551610in}}%
\pgfpathlineto{\pgfqpoint{4.693685in}{2.478669in}}%
\pgfpathlineto{\pgfqpoint{4.697112in}{2.528533in}}%
\pgfpathlineto{\pgfqpoint{4.700535in}{2.432399in}}%
\pgfpathlineto{\pgfqpoint{4.704809in}{2.540917in}}%
\pgfpathlineto{\pgfqpoint{4.706522in}{2.485102in}}%
\pgfpathlineto{\pgfqpoint{4.711657in}{2.554085in}}%
\pgfpathlineto{\pgfqpoint{4.715932in}{2.507090in}}%
\pgfpathlineto{\pgfqpoint{4.718499in}{2.584530in}}%
\pgfpathlineto{\pgfqpoint{4.721922in}{2.497213in}}%
\pgfpathlineto{\pgfqpoint{4.724492in}{2.554448in}}%
\pgfpathlineto{\pgfqpoint{4.727918in}{2.471873in}}%
\pgfpathlineto{\pgfqpoint{4.730484in}{2.520923in}}%
\pgfpathlineto{\pgfqpoint{4.736479in}{2.470000in}}%
\pgfpathlineto{\pgfqpoint{4.737335in}{2.574563in}}%
\pgfpathlineto{\pgfqpoint{4.742472in}{2.561214in}}%
\pgfpathlineto{\pgfqpoint{4.745895in}{2.440160in}}%
\pgfpathlineto{\pgfqpoint{4.747605in}{2.522161in}}%
\pgfpathlineto{\pgfqpoint{4.751882in}{2.609659in}}%
\pgfpathlineto{\pgfqpoint{4.755301in}{2.482384in}}%
\pgfpathlineto{\pgfqpoint{4.759573in}{2.538170in}}%
\pgfpathlineto{\pgfqpoint{4.763849in}{2.486372in}}%
\pgfpathlineto{\pgfqpoint{4.766411in}{2.593652in}}%
\pgfpathlineto{\pgfqpoint{4.769831in}{2.471752in}}%
\pgfpathlineto{\pgfqpoint{4.773255in}{2.532581in}}%
\pgfpathlineto{\pgfqpoint{4.776680in}{2.544844in}}%
\pgfpathlineto{\pgfqpoint{4.778393in}{2.447319in}}%
\pgfpathlineto{\pgfqpoint{4.782675in}{2.502015in}}%
\pgfpathlineto{\pgfqpoint{4.787806in}{2.465561in}}%
\pgfpathlineto{\pgfqpoint{4.791230in}{2.500929in}}%
\pgfpathlineto{\pgfqpoint{4.794650in}{2.482293in}}%
\pgfpathlineto{\pgfqpoint{4.795506in}{2.527991in}}%
\pgfpathlineto{\pgfqpoint{4.795506in}{2.527991in}}%
\pgfusepath{stroke}%
\end{pgfscope}%
\begin{pgfscope}%
\pgfsetrectcap%
\pgfsetmiterjoin%
\pgfsetlinewidth{0.803000pt}%
\definecolor{currentstroke}{rgb}{0.000000,0.000000,0.000000}%
\pgfsetstrokecolor{currentstroke}%
\pgfsetdash{}{0pt}%
\pgfpathmoveto{\pgfqpoint{0.484581in}{2.334497in}}%
\pgfpathlineto{\pgfqpoint{0.484581in}{2.909119in}}%
\pgfusepath{stroke}%
\end{pgfscope}%
\begin{pgfscope}%
\pgfsetrectcap%
\pgfsetmiterjoin%
\pgfsetlinewidth{0.803000pt}%
\definecolor{currentstroke}{rgb}{0.000000,0.000000,0.000000}%
\pgfsetstrokecolor{currentstroke}%
\pgfsetdash{}{0pt}%
\pgfpathmoveto{\pgfqpoint{5.000788in}{2.334497in}}%
\pgfpathlineto{\pgfqpoint{5.000788in}{2.909119in}}%
\pgfusepath{stroke}%
\end{pgfscope}%
\begin{pgfscope}%
\pgfsetrectcap%
\pgfsetmiterjoin%
\pgfsetlinewidth{0.803000pt}%
\definecolor{currentstroke}{rgb}{0.000000,0.000000,0.000000}%
\pgfsetstrokecolor{currentstroke}%
\pgfsetdash{}{0pt}%
\pgfpathmoveto{\pgfqpoint{0.484581in}{2.334497in}}%
\pgfpathlineto{\pgfqpoint{5.000788in}{2.334497in}}%
\pgfusepath{stroke}%
\end{pgfscope}%
\begin{pgfscope}%
\pgfsetrectcap%
\pgfsetmiterjoin%
\pgfsetlinewidth{0.803000pt}%
\definecolor{currentstroke}{rgb}{0.000000,0.000000,0.000000}%
\pgfsetstrokecolor{currentstroke}%
\pgfsetdash{}{0pt}%
\pgfpathmoveto{\pgfqpoint{0.484581in}{2.909119in}}%
\pgfpathlineto{\pgfqpoint{5.000788in}{2.909119in}}%
\pgfusepath{stroke}%
\end{pgfscope}%
\begin{pgfscope}%
\pgfsetbuttcap%
\pgfsetmiterjoin%
\definecolor{currentfill}{rgb}{1.000000,1.000000,1.000000}%
\pgfsetfillcolor{currentfill}%
\pgfsetlinewidth{0.000000pt}%
\definecolor{currentstroke}{rgb}{0.000000,0.000000,0.000000}%
\pgfsetstrokecolor{currentstroke}%
\pgfsetstrokeopacity{0.000000}%
\pgfsetdash{}{0pt}%
\pgfpathmoveto{\pgfqpoint{0.484581in}{1.437021in}}%
\pgfpathlineto{\pgfqpoint{5.000788in}{1.437021in}}%
\pgfpathlineto{\pgfqpoint{5.000788in}{2.011643in}}%
\pgfpathlineto{\pgfqpoint{0.484581in}{2.011643in}}%
\pgfpathlineto{\pgfqpoint{0.484581in}{1.437021in}}%
\pgfpathclose%
\pgfusepath{fill}%
\end{pgfscope}%
\begin{pgfscope}%
\pgfsetbuttcap%
\pgfsetroundjoin%
\definecolor{currentfill}{rgb}{0.000000,0.000000,0.000000}%
\pgfsetfillcolor{currentfill}%
\pgfsetlinewidth{0.803000pt}%
\definecolor{currentstroke}{rgb}{0.000000,0.000000,0.000000}%
\pgfsetstrokecolor{currentstroke}%
\pgfsetdash{}{0pt}%
\pgfsys@defobject{currentmarker}{\pgfqpoint{0.000000in}{-0.048611in}}{\pgfqpoint{0.000000in}{0.000000in}}{%
\pgfpathmoveto{\pgfqpoint{0.000000in}{0.000000in}}%
\pgfpathlineto{\pgfqpoint{0.000000in}{-0.048611in}}%
\pgfusepath{stroke,fill}%
}%
\begin{pgfscope}%
\pgfsys@transformshift{0.689546in}{1.437021in}%
\pgfsys@useobject{currentmarker}{}%
\end{pgfscope}%
\end{pgfscope}%
\begin{pgfscope}%
\pgfsetbuttcap%
\pgfsetroundjoin%
\definecolor{currentfill}{rgb}{0.000000,0.000000,0.000000}%
\pgfsetfillcolor{currentfill}%
\pgfsetlinewidth{0.803000pt}%
\definecolor{currentstroke}{rgb}{0.000000,0.000000,0.000000}%
\pgfsetstrokecolor{currentstroke}%
\pgfsetdash{}{0pt}%
\pgfsys@defobject{currentmarker}{\pgfqpoint{0.000000in}{-0.048611in}}{\pgfqpoint{0.000000in}{0.000000in}}{%
\pgfpathmoveto{\pgfqpoint{0.000000in}{0.000000in}}%
\pgfpathlineto{\pgfqpoint{0.000000in}{-0.048611in}}%
\pgfusepath{stroke,fill}%
}%
\begin{pgfscope}%
\pgfsys@transformshift{1.202878in}{1.437021in}%
\pgfsys@useobject{currentmarker}{}%
\end{pgfscope}%
\end{pgfscope}%
\begin{pgfscope}%
\pgfsetbuttcap%
\pgfsetroundjoin%
\definecolor{currentfill}{rgb}{0.000000,0.000000,0.000000}%
\pgfsetfillcolor{currentfill}%
\pgfsetlinewidth{0.803000pt}%
\definecolor{currentstroke}{rgb}{0.000000,0.000000,0.000000}%
\pgfsetstrokecolor{currentstroke}%
\pgfsetdash{}{0pt}%
\pgfsys@defobject{currentmarker}{\pgfqpoint{0.000000in}{-0.048611in}}{\pgfqpoint{0.000000in}{0.000000in}}{%
\pgfpathmoveto{\pgfqpoint{0.000000in}{0.000000in}}%
\pgfpathlineto{\pgfqpoint{0.000000in}{-0.048611in}}%
\pgfusepath{stroke,fill}%
}%
\begin{pgfscope}%
\pgfsys@transformshift{1.716211in}{1.437021in}%
\pgfsys@useobject{currentmarker}{}%
\end{pgfscope}%
\end{pgfscope}%
\begin{pgfscope}%
\pgfsetbuttcap%
\pgfsetroundjoin%
\definecolor{currentfill}{rgb}{0.000000,0.000000,0.000000}%
\pgfsetfillcolor{currentfill}%
\pgfsetlinewidth{0.803000pt}%
\definecolor{currentstroke}{rgb}{0.000000,0.000000,0.000000}%
\pgfsetstrokecolor{currentstroke}%
\pgfsetdash{}{0pt}%
\pgfsys@defobject{currentmarker}{\pgfqpoint{0.000000in}{-0.048611in}}{\pgfqpoint{0.000000in}{0.000000in}}{%
\pgfpathmoveto{\pgfqpoint{0.000000in}{0.000000in}}%
\pgfpathlineto{\pgfqpoint{0.000000in}{-0.048611in}}%
\pgfusepath{stroke,fill}%
}%
\begin{pgfscope}%
\pgfsys@transformshift{2.229543in}{1.437021in}%
\pgfsys@useobject{currentmarker}{}%
\end{pgfscope}%
\end{pgfscope}%
\begin{pgfscope}%
\pgfsetbuttcap%
\pgfsetroundjoin%
\definecolor{currentfill}{rgb}{0.000000,0.000000,0.000000}%
\pgfsetfillcolor{currentfill}%
\pgfsetlinewidth{0.803000pt}%
\definecolor{currentstroke}{rgb}{0.000000,0.000000,0.000000}%
\pgfsetstrokecolor{currentstroke}%
\pgfsetdash{}{0pt}%
\pgfsys@defobject{currentmarker}{\pgfqpoint{0.000000in}{-0.048611in}}{\pgfqpoint{0.000000in}{0.000000in}}{%
\pgfpathmoveto{\pgfqpoint{0.000000in}{0.000000in}}%
\pgfpathlineto{\pgfqpoint{0.000000in}{-0.048611in}}%
\pgfusepath{stroke,fill}%
}%
\begin{pgfscope}%
\pgfsys@transformshift{2.742876in}{1.437021in}%
\pgfsys@useobject{currentmarker}{}%
\end{pgfscope}%
\end{pgfscope}%
\begin{pgfscope}%
\pgfsetbuttcap%
\pgfsetroundjoin%
\definecolor{currentfill}{rgb}{0.000000,0.000000,0.000000}%
\pgfsetfillcolor{currentfill}%
\pgfsetlinewidth{0.803000pt}%
\definecolor{currentstroke}{rgb}{0.000000,0.000000,0.000000}%
\pgfsetstrokecolor{currentstroke}%
\pgfsetdash{}{0pt}%
\pgfsys@defobject{currentmarker}{\pgfqpoint{0.000000in}{-0.048611in}}{\pgfqpoint{0.000000in}{0.000000in}}{%
\pgfpathmoveto{\pgfqpoint{0.000000in}{0.000000in}}%
\pgfpathlineto{\pgfqpoint{0.000000in}{-0.048611in}}%
\pgfusepath{stroke,fill}%
}%
\begin{pgfscope}%
\pgfsys@transformshift{3.256208in}{1.437021in}%
\pgfsys@useobject{currentmarker}{}%
\end{pgfscope}%
\end{pgfscope}%
\begin{pgfscope}%
\pgfsetbuttcap%
\pgfsetroundjoin%
\definecolor{currentfill}{rgb}{0.000000,0.000000,0.000000}%
\pgfsetfillcolor{currentfill}%
\pgfsetlinewidth{0.803000pt}%
\definecolor{currentstroke}{rgb}{0.000000,0.000000,0.000000}%
\pgfsetstrokecolor{currentstroke}%
\pgfsetdash{}{0pt}%
\pgfsys@defobject{currentmarker}{\pgfqpoint{0.000000in}{-0.048611in}}{\pgfqpoint{0.000000in}{0.000000in}}{%
\pgfpathmoveto{\pgfqpoint{0.000000in}{0.000000in}}%
\pgfpathlineto{\pgfqpoint{0.000000in}{-0.048611in}}%
\pgfusepath{stroke,fill}%
}%
\begin{pgfscope}%
\pgfsys@transformshift{3.769541in}{1.437021in}%
\pgfsys@useobject{currentmarker}{}%
\end{pgfscope}%
\end{pgfscope}%
\begin{pgfscope}%
\pgfsetbuttcap%
\pgfsetroundjoin%
\definecolor{currentfill}{rgb}{0.000000,0.000000,0.000000}%
\pgfsetfillcolor{currentfill}%
\pgfsetlinewidth{0.803000pt}%
\definecolor{currentstroke}{rgb}{0.000000,0.000000,0.000000}%
\pgfsetstrokecolor{currentstroke}%
\pgfsetdash{}{0pt}%
\pgfsys@defobject{currentmarker}{\pgfqpoint{0.000000in}{-0.048611in}}{\pgfqpoint{0.000000in}{0.000000in}}{%
\pgfpathmoveto{\pgfqpoint{0.000000in}{0.000000in}}%
\pgfpathlineto{\pgfqpoint{0.000000in}{-0.048611in}}%
\pgfusepath{stroke,fill}%
}%
\begin{pgfscope}%
\pgfsys@transformshift{4.282873in}{1.437021in}%
\pgfsys@useobject{currentmarker}{}%
\end{pgfscope}%
\end{pgfscope}%
\begin{pgfscope}%
\pgfsetbuttcap%
\pgfsetroundjoin%
\definecolor{currentfill}{rgb}{0.000000,0.000000,0.000000}%
\pgfsetfillcolor{currentfill}%
\pgfsetlinewidth{0.803000pt}%
\definecolor{currentstroke}{rgb}{0.000000,0.000000,0.000000}%
\pgfsetstrokecolor{currentstroke}%
\pgfsetdash{}{0pt}%
\pgfsys@defobject{currentmarker}{\pgfqpoint{0.000000in}{-0.048611in}}{\pgfqpoint{0.000000in}{0.000000in}}{%
\pgfpathmoveto{\pgfqpoint{0.000000in}{0.000000in}}%
\pgfpathlineto{\pgfqpoint{0.000000in}{-0.048611in}}%
\pgfusepath{stroke,fill}%
}%
\begin{pgfscope}%
\pgfsys@transformshift{4.796206in}{1.437021in}%
\pgfsys@useobject{currentmarker}{}%
\end{pgfscope}%
\end{pgfscope}%
\begin{pgfscope}%
\pgfsetbuttcap%
\pgfsetroundjoin%
\definecolor{currentfill}{rgb}{0.000000,0.000000,0.000000}%
\pgfsetfillcolor{currentfill}%
\pgfsetlinewidth{0.803000pt}%
\definecolor{currentstroke}{rgb}{0.000000,0.000000,0.000000}%
\pgfsetstrokecolor{currentstroke}%
\pgfsetdash{}{0pt}%
\pgfsys@defobject{currentmarker}{\pgfqpoint{-0.048611in}{0.000000in}}{\pgfqpoint{-0.000000in}{0.000000in}}{%
\pgfpathmoveto{\pgfqpoint{-0.000000in}{0.000000in}}%
\pgfpathlineto{\pgfqpoint{-0.048611in}{0.000000in}}%
\pgfusepath{stroke,fill}%
}%
\begin{pgfscope}%
\pgfsys@transformshift{0.484581in}{1.615020in}%
\pgfsys@useobject{currentmarker}{}%
\end{pgfscope}%
\end{pgfscope}%
\begin{pgfscope}%
\definecolor{textcolor}{rgb}{0.000000,0.000000,0.000000}%
\pgfsetstrokecolor{textcolor}%
\pgfsetfillcolor{textcolor}%
\pgftext[x=0.328331in, y=1.576464in, left, base]{\color{textcolor}\rmfamily\fontsize{8.000000}{9.600000}\selectfont \(\displaystyle {0}\)}%
\end{pgfscope}%
\begin{pgfscope}%
\pgfsetbuttcap%
\pgfsetroundjoin%
\definecolor{currentfill}{rgb}{0.000000,0.000000,0.000000}%
\pgfsetfillcolor{currentfill}%
\pgfsetlinewidth{0.803000pt}%
\definecolor{currentstroke}{rgb}{0.000000,0.000000,0.000000}%
\pgfsetstrokecolor{currentstroke}%
\pgfsetdash{}{0pt}%
\pgfsys@defobject{currentmarker}{\pgfqpoint{-0.048611in}{0.000000in}}{\pgfqpoint{-0.000000in}{0.000000in}}{%
\pgfpathmoveto{\pgfqpoint{-0.000000in}{0.000000in}}%
\pgfpathlineto{\pgfqpoint{-0.048611in}{0.000000in}}%
\pgfusepath{stroke,fill}%
}%
\begin{pgfscope}%
\pgfsys@transformshift{0.484581in}{1.818561in}%
\pgfsys@useobject{currentmarker}{}%
\end{pgfscope}%
\end{pgfscope}%
\begin{pgfscope}%
\definecolor{textcolor}{rgb}{0.000000,0.000000,0.000000}%
\pgfsetstrokecolor{textcolor}%
\pgfsetfillcolor{textcolor}%
\pgftext[x=0.328331in, y=1.780005in, left, base]{\color{textcolor}\rmfamily\fontsize{8.000000}{9.600000}\selectfont \(\displaystyle {5}\)}%
\end{pgfscope}%
\begin{pgfscope}%
\definecolor{textcolor}{rgb}{0.000000,0.000000,0.000000}%
\pgfsetstrokecolor{textcolor}%
\pgfsetfillcolor{textcolor}%
\pgftext[x=0.272775in,y=1.724332in,,bottom,rotate=90.000000]{\color{textcolor}\rmfamily\fontsize{10.000000}{12.000000}\selectfont Voltage deviation in \unit{\V}}%
\end{pgfscope}%
\begin{pgfscope}%
\definecolor{textcolor}{rgb}{0.000000,0.000000,0.000000}%
\pgfsetstrokecolor{textcolor}%
\pgfsetfillcolor{textcolor}%
\pgftext[x=0.484581in,y=2.053309in,left,base]{\color{textcolor}\rmfamily\fontsize{8.000000}{9.600000}\selectfont \(\displaystyle \times{10^{\ensuremath{-}6}}{}\)}%
\end{pgfscope}%
\begin{pgfscope}%
\pgfpathrectangle{\pgfqpoint{0.484581in}{1.437021in}}{\pgfqpoint{4.516206in}{0.574622in}}%
\pgfusepath{clip}%
\pgfsetrectcap%
\pgfsetroundjoin%
\pgfsetlinewidth{0.501875pt}%
\definecolor{currentstroke}{rgb}{0.003922,0.450980,0.698039}%
\pgfsetstrokecolor{currentstroke}%
\pgfsetstrokeopacity{0.700000}%
\pgfsetdash{}{0pt}%
\pgfpathmoveto{\pgfqpoint{0.689863in}{1.586143in}}%
\pgfpathlineto{\pgfqpoint{0.691573in}{1.625337in}}%
\pgfpathlineto{\pgfqpoint{0.694995in}{1.554591in}}%
\pgfpathlineto{\pgfqpoint{0.700132in}{1.624113in}}%
\pgfpathlineto{\pgfqpoint{0.706120in}{1.553835in}}%
\pgfpathlineto{\pgfqpoint{0.707833in}{1.630006in}}%
\pgfpathlineto{\pgfqpoint{0.712110in}{1.550218in}}%
\pgfpathlineto{\pgfqpoint{0.719814in}{1.595418in}}%
\pgfpathlineto{\pgfqpoint{0.723233in}{1.529484in}}%
\pgfpathlineto{\pgfqpoint{0.725797in}{1.600142in}}%
\pgfpathlineto{\pgfqpoint{0.730072in}{1.521436in}}%
\pgfpathlineto{\pgfqpoint{0.734351in}{1.620555in}}%
\pgfpathlineto{\pgfqpoint{0.740340in}{1.509859in}}%
\pgfpathlineto{\pgfqpoint{0.742908in}{1.627964in}}%
\pgfpathlineto{\pgfqpoint{0.749745in}{1.515752in}}%
\pgfpathlineto{\pgfqpoint{0.750602in}{1.631112in}}%
\pgfpathlineto{\pgfqpoint{0.757448in}{1.584280in}}%
\pgfpathlineto{\pgfqpoint{0.761721in}{1.771029in}}%
\pgfpathlineto{\pgfqpoint{0.764288in}{1.661499in}}%
\pgfpathlineto{\pgfqpoint{0.770278in}{1.670975in}}%
\pgfpathlineto{\pgfqpoint{0.771990in}{1.588302in}}%
\pgfpathlineto{\pgfqpoint{0.777980in}{1.623412in}}%
\pgfpathlineto{\pgfqpoint{0.783113in}{1.520705in}}%
\pgfpathlineto{\pgfqpoint{0.784824in}{1.596583in}}%
\pgfpathlineto{\pgfqpoint{0.792516in}{1.565616in}}%
\pgfpathlineto{\pgfqpoint{0.793373in}{1.644234in}}%
\pgfpathlineto{\pgfqpoint{0.797648in}{1.563749in}}%
\pgfpathlineto{\pgfqpoint{0.802779in}{1.499769in}}%
\pgfpathlineto{\pgfqpoint{0.807054in}{1.491546in}}%
\pgfpathlineto{\pgfqpoint{0.811330in}{1.621837in}}%
\pgfpathlineto{\pgfqpoint{0.816461in}{1.552432in}}%
\pgfpathlineto{\pgfqpoint{0.819881in}{1.653214in}}%
\pgfpathlineto{\pgfqpoint{0.825871in}{1.546953in}}%
\pgfpathlineto{\pgfqpoint{0.827584in}{1.683834in}}%
\pgfpathlineto{\pgfqpoint{0.836141in}{1.522426in}}%
\pgfpathlineto{\pgfqpoint{0.841276in}{1.594483in}}%
\pgfpathlineto{\pgfqpoint{0.848114in}{1.547709in}}%
\pgfpathlineto{\pgfqpoint{0.851531in}{1.619912in}}%
\pgfpathlineto{\pgfqpoint{0.854095in}{1.645692in}}%
\pgfpathlineto{\pgfqpoint{0.857519in}{1.567596in}}%
\pgfpathlineto{\pgfqpoint{0.864372in}{1.555406in}}%
\pgfpathlineto{\pgfqpoint{0.867793in}{1.625220in}}%
\pgfpathlineto{\pgfqpoint{0.871215in}{1.560422in}}%
\pgfpathlineto{\pgfqpoint{0.874639in}{1.655431in}}%
\pgfpathlineto{\pgfqpoint{0.879771in}{1.700631in}}%
\pgfpathlineto{\pgfqpoint{0.883195in}{1.617114in}}%
\pgfpathlineto{\pgfqpoint{0.889177in}{1.721161in}}%
\pgfpathlineto{\pgfqpoint{0.893457in}{1.578095in}}%
\pgfpathlineto{\pgfqpoint{0.898587in}{1.665404in}}%
\pgfpathlineto{\pgfqpoint{0.901153in}{1.582702in}}%
\pgfpathlineto{\pgfqpoint{0.906287in}{1.633735in}}%
\pgfpathlineto{\pgfqpoint{0.908853in}{1.559285in}}%
\pgfpathlineto{\pgfqpoint{0.916550in}{1.533012in}}%
\pgfpathlineto{\pgfqpoint{0.918259in}{1.614516in}}%
\pgfpathlineto{\pgfqpoint{0.924241in}{1.548176in}}%
\pgfpathlineto{\pgfqpoint{0.925951in}{1.593201in}}%
\pgfpathlineto{\pgfqpoint{0.930227in}{1.522163in}}%
\pgfpathlineto{\pgfqpoint{0.936214in}{1.628775in}}%
\pgfpathlineto{\pgfqpoint{0.938782in}{1.509446in}}%
\pgfpathlineto{\pgfqpoint{0.943058in}{1.594717in}}%
\pgfpathlineto{\pgfqpoint{0.947330in}{1.479644in}}%
\pgfpathlineto{\pgfqpoint{0.952461in}{1.636943in}}%
\pgfpathlineto{\pgfqpoint{0.958451in}{1.527588in}}%
\pgfpathlineto{\pgfqpoint{0.961015in}{1.600025in}}%
\pgfpathlineto{\pgfqpoint{0.967858in}{1.482388in}}%
\pgfpathlineto{\pgfqpoint{0.970425in}{1.625045in}}%
\pgfpathlineto{\pgfqpoint{0.976408in}{1.643591in}}%
\pgfpathlineto{\pgfqpoint{0.977263in}{1.606586in}}%
\pgfpathlineto{\pgfqpoint{0.982392in}{1.522689in}}%
\pgfpathlineto{\pgfqpoint{0.989232in}{1.502626in}}%
\pgfpathlineto{\pgfqpoint{0.990941in}{1.614896in}}%
\pgfpathlineto{\pgfqpoint{1.000349in}{1.478625in}}%
\pgfpathlineto{\pgfqpoint{1.002914in}{1.594454in}}%
\pgfpathlineto{\pgfqpoint{1.009758in}{1.546573in}}%
\pgfpathlineto{\pgfqpoint{1.011468in}{1.620789in}}%
\pgfpathlineto{\pgfqpoint{1.017456in}{1.559958in}}%
\pgfpathlineto{\pgfqpoint{1.022584in}{1.708156in}}%
\pgfpathlineto{\pgfqpoint{1.024292in}{1.600609in}}%
\pgfpathlineto{\pgfqpoint{1.031988in}{1.634320in}}%
\pgfpathlineto{\pgfqpoint{1.032842in}{1.691184in}}%
\pgfpathlineto{\pgfqpoint{1.039687in}{1.748691in}}%
\pgfpathlineto{\pgfqpoint{1.043113in}{1.674416in}}%
\pgfpathlineto{\pgfqpoint{1.048251in}{1.722268in}}%
\pgfpathlineto{\pgfqpoint{1.049962in}{1.651522in}}%
\pgfpathlineto{\pgfqpoint{1.055949in}{1.713289in}}%
\pgfpathlineto{\pgfqpoint{1.060226in}{1.562756in}}%
\pgfpathlineto{\pgfqpoint{1.064502in}{1.663011in}}%
\pgfpathlineto{\pgfqpoint{1.068774in}{1.536859in}}%
\pgfpathlineto{\pgfqpoint{1.071339in}{1.603086in}}%
\pgfpathlineto{\pgfqpoint{1.076466in}{1.541641in}}%
\pgfpathlineto{\pgfqpoint{1.080746in}{1.617724in}}%
\pgfpathlineto{\pgfqpoint{1.086736in}{1.542109in}}%
\pgfpathlineto{\pgfqpoint{1.091868in}{1.631226in}}%
\pgfpathlineto{\pgfqpoint{1.096997in}{1.523620in}}%
\pgfpathlineto{\pgfqpoint{1.102978in}{1.566836in}}%
\pgfpathlineto{\pgfqpoint{1.108959in}{1.626415in}}%
\pgfpathlineto{\pgfqpoint{1.111524in}{1.539661in}}%
\pgfpathlineto{\pgfqpoint{1.115796in}{1.683048in}}%
\pgfpathlineto{\pgfqpoint{1.121786in}{1.726002in}}%
\pgfpathlineto{\pgfqpoint{1.122640in}{1.618659in}}%
\pgfpathlineto{\pgfqpoint{1.130340in}{1.593551in}}%
\pgfpathlineto{\pgfqpoint{1.132050in}{1.648929in}}%
\pgfpathlineto{\pgfqpoint{1.136328in}{1.567830in}}%
\pgfpathlineto{\pgfqpoint{1.140604in}{1.724602in}}%
\pgfpathlineto{\pgfqpoint{1.146593in}{1.623032in}}%
\pgfpathlineto{\pgfqpoint{1.154289in}{1.756154in}}%
\pgfpathlineto{\pgfqpoint{1.157708in}{1.579319in}}%
\pgfpathlineto{\pgfqpoint{1.163696in}{1.517322in}}%
\pgfpathlineto{\pgfqpoint{1.165408in}{1.614546in}}%
\pgfpathlineto{\pgfqpoint{1.171400in}{1.551033in}}%
\pgfpathlineto{\pgfqpoint{1.174823in}{1.608072in}}%
\pgfpathlineto{\pgfqpoint{1.179102in}{1.542781in}}%
\pgfpathlineto{\pgfqpoint{1.185947in}{1.665579in}}%
\pgfpathlineto{\pgfqpoint{1.188512in}{1.532749in}}%
\pgfpathlineto{\pgfqpoint{1.191934in}{1.637001in}}%
\pgfpathlineto{\pgfqpoint{1.196209in}{1.492595in}}%
\pgfpathlineto{\pgfqpoint{1.203055in}{1.490786in}}%
\pgfpathlineto{\pgfqpoint{1.203910in}{1.586611in}}%
\pgfpathlineto{\pgfqpoint{1.211615in}{1.672053in}}%
\pgfpathlineto{\pgfqpoint{1.212472in}{1.589643in}}%
\pgfpathlineto{\pgfqpoint{1.216755in}{1.605128in}}%
\pgfpathlineto{\pgfqpoint{1.221891in}{1.553397in}}%
\pgfpathlineto{\pgfqpoint{1.228734in}{1.607780in}}%
\pgfpathlineto{\pgfqpoint{1.230446in}{1.528694in}}%
\pgfpathlineto{\pgfqpoint{1.236437in}{1.577978in}}%
\pgfpathlineto{\pgfqpoint{1.239857in}{1.539135in}}%
\pgfpathlineto{\pgfqpoint{1.245845in}{1.615072in}}%
\pgfpathlineto{\pgfqpoint{1.247557in}{1.566080in}}%
\pgfpathlineto{\pgfqpoint{1.251837in}{1.608540in}}%
\pgfpathlineto{\pgfqpoint{1.256112in}{1.540943in}}%
\pgfpathlineto{\pgfqpoint{1.262954in}{1.599323in}}%
\pgfpathlineto{\pgfqpoint{1.264666in}{1.522163in}}%
\pgfpathlineto{\pgfqpoint{1.269800in}{1.730783in}}%
\pgfpathlineto{\pgfqpoint{1.274080in}{1.587542in}}%
\pgfpathlineto{\pgfqpoint{1.277502in}{1.548874in}}%
\pgfpathlineto{\pgfqpoint{1.284344in}{1.739766in}}%
\pgfpathlineto{\pgfqpoint{1.286057in}{1.610407in}}%
\pgfpathlineto{\pgfqpoint{1.289480in}{1.562058in}}%
\pgfpathlineto{\pgfqpoint{1.297179in}{1.694070in}}%
\pgfpathlineto{\pgfqpoint{1.298889in}{1.618922in}}%
\pgfpathlineto{\pgfqpoint{1.302314in}{1.726060in}}%
\pgfpathlineto{\pgfqpoint{1.309157in}{1.629070in}}%
\pgfpathlineto{\pgfqpoint{1.312579in}{1.717369in}}%
\pgfpathlineto{\pgfqpoint{1.317712in}{1.618279in}}%
\pgfpathlineto{\pgfqpoint{1.321985in}{1.715678in}}%
\pgfpathlineto{\pgfqpoint{1.324548in}{1.687158in}}%
\pgfpathlineto{\pgfqpoint{1.331388in}{1.692408in}}%
\pgfpathlineto{\pgfqpoint{1.333951in}{1.537210in}}%
\pgfpathlineto{\pgfqpoint{1.339938in}{1.619795in}}%
\pgfpathlineto{\pgfqpoint{1.343354in}{1.535577in}}%
\pgfpathlineto{\pgfqpoint{1.345921in}{1.609588in}}%
\pgfpathlineto{\pgfqpoint{1.350199in}{1.549575in}}%
\pgfpathlineto{\pgfqpoint{1.356186in}{1.629651in}}%
\pgfpathlineto{\pgfqpoint{1.358754in}{1.557331in}}%
\pgfpathlineto{\pgfqpoint{1.365593in}{1.632976in}}%
\pgfpathlineto{\pgfqpoint{1.369869in}{1.502100in}}%
\pgfpathlineto{\pgfqpoint{1.371579in}{1.568239in}}%
\pgfpathlineto{\pgfqpoint{1.378419in}{1.617987in}}%
\pgfpathlineto{\pgfqpoint{1.382696in}{1.510440in}}%
\pgfpathlineto{\pgfqpoint{1.383549in}{1.581361in}}%
\pgfpathlineto{\pgfqpoint{1.387825in}{1.689142in}}%
\pgfpathlineto{\pgfqpoint{1.392960in}{1.650415in}}%
\pgfpathlineto{\pgfqpoint{1.396386in}{1.710195in}}%
\pgfpathlineto{\pgfqpoint{1.400665in}{1.665404in}}%
\pgfpathlineto{\pgfqpoint{1.404941in}{1.722239in}}%
\pgfpathlineto{\pgfqpoint{1.410071in}{1.632044in}}%
\pgfpathlineto{\pgfqpoint{1.416052in}{1.616588in}}%
\pgfpathlineto{\pgfqpoint{1.417764in}{1.747931in}}%
\pgfpathlineto{\pgfqpoint{1.424607in}{1.633443in}}%
\pgfpathlineto{\pgfqpoint{1.426320in}{1.661758in}}%
\pgfpathlineto{\pgfqpoint{1.433162in}{1.599995in}}%
\pgfpathlineto{\pgfqpoint{1.435727in}{1.686314in}}%
\pgfpathlineto{\pgfqpoint{1.439148in}{1.617465in}}%
\pgfpathlineto{\pgfqpoint{1.445989in}{1.705124in}}%
\pgfpathlineto{\pgfqpoint{1.447700in}{1.624084in}}%
\pgfpathlineto{\pgfqpoint{1.454547in}{1.676575in}}%
\pgfpathlineto{\pgfqpoint{1.457115in}{1.610670in}}%
\pgfpathlineto{\pgfqpoint{1.460537in}{1.665057in}}%
\pgfpathlineto{\pgfqpoint{1.464817in}{1.622714in}}%
\pgfpathlineto{\pgfqpoint{1.469955in}{1.730261in}}%
\pgfpathlineto{\pgfqpoint{1.473375in}{1.618922in}}%
\pgfpathlineto{\pgfqpoint{1.478513in}{1.520822in}}%
\pgfpathlineto{\pgfqpoint{1.485359in}{1.527851in}}%
\pgfpathlineto{\pgfqpoint{1.489636in}{1.629567in}}%
\pgfpathlineto{\pgfqpoint{1.490491in}{1.509567in}}%
\pgfpathlineto{\pgfqpoint{1.498186in}{1.649192in}}%
\pgfpathlineto{\pgfqpoint{1.499039in}{1.531526in}}%
\pgfpathlineto{\pgfqpoint{1.503315in}{1.663424in}}%
\pgfpathlineto{\pgfqpoint{1.509305in}{1.524380in}}%
\pgfpathlineto{\pgfqpoint{1.515290in}{1.629070in}}%
\pgfpathlineto{\pgfqpoint{1.516145in}{1.549634in}}%
\pgfpathlineto{\pgfqpoint{1.521272in}{1.679110in}}%
\pgfpathlineto{\pgfqpoint{1.527256in}{1.500350in}}%
\pgfpathlineto{\pgfqpoint{1.528967in}{1.616705in}}%
\pgfpathlineto{\pgfqpoint{1.535810in}{1.601307in}}%
\pgfpathlineto{\pgfqpoint{1.539230in}{1.533652in}}%
\pgfpathlineto{\pgfqpoint{1.545218in}{1.634901in}}%
\pgfpathlineto{\pgfqpoint{1.546074in}{1.545754in}}%
\pgfpathlineto{\pgfqpoint{1.552920in}{1.627901in}}%
\pgfpathlineto{\pgfqpoint{1.554633in}{1.535577in}}%
\pgfpathlineto{\pgfqpoint{1.561475in}{1.633209in}}%
\pgfpathlineto{\pgfqpoint{1.563184in}{1.527178in}}%
\pgfpathlineto{\pgfqpoint{1.568314in}{1.665462in}}%
\pgfpathlineto{\pgfqpoint{1.573455in}{1.552841in}}%
\pgfpathlineto{\pgfqpoint{1.577737in}{1.621662in}}%
\pgfpathlineto{\pgfqpoint{1.582014in}{1.532019in}}%
\pgfpathlineto{\pgfqpoint{1.587148in}{1.626444in}}%
\pgfpathlineto{\pgfqpoint{1.592280in}{1.516446in}}%
\pgfpathlineto{\pgfqpoint{1.596559in}{1.616354in}}%
\pgfpathlineto{\pgfqpoint{1.598270in}{1.508310in}}%
\pgfpathlineto{\pgfqpoint{1.605969in}{1.651347in}}%
\pgfpathlineto{\pgfqpoint{1.611957in}{1.534817in}}%
\pgfpathlineto{\pgfqpoint{1.617087in}{1.646419in}}%
\pgfpathlineto{\pgfqpoint{1.620507in}{1.579611in}}%
\pgfpathlineto{\pgfqpoint{1.623070in}{1.604748in}}%
\pgfpathlineto{\pgfqpoint{1.630773in}{1.553046in}}%
\pgfpathlineto{\pgfqpoint{1.635054in}{1.646273in}}%
\pgfpathlineto{\pgfqpoint{1.635909in}{1.561298in}}%
\pgfpathlineto{\pgfqpoint{1.643599in}{1.653155in}}%
\pgfpathlineto{\pgfqpoint{1.645308in}{1.567684in}}%
\pgfpathlineto{\pgfqpoint{1.649584in}{1.596320in}}%
\pgfpathlineto{\pgfqpoint{1.655572in}{1.521637in}}%
\pgfpathlineto{\pgfqpoint{1.660704in}{1.610056in}}%
\pgfpathlineto{\pgfqpoint{1.663267in}{1.560422in}}%
\pgfpathlineto{\pgfqpoint{1.671820in}{1.683892in}}%
\pgfpathlineto{\pgfqpoint{1.674387in}{1.534119in}}%
\pgfpathlineto{\pgfqpoint{1.680376in}{1.484572in}}%
\pgfpathlineto{\pgfqpoint{1.683795in}{1.622535in}}%
\pgfpathlineto{\pgfqpoint{1.692344in}{1.489793in}}%
\pgfpathlineto{\pgfqpoint{1.695764in}{1.611280in}}%
\pgfpathlineto{\pgfqpoint{1.701749in}{1.772136in}}%
\pgfpathlineto{\pgfqpoint{1.705169in}{1.640559in}}%
\pgfpathlineto{\pgfqpoint{1.712012in}{1.694274in}}%
\pgfpathlineto{\pgfqpoint{1.716288in}{1.611514in}}%
\pgfpathlineto{\pgfqpoint{1.717143in}{1.675260in}}%
\pgfpathlineto{\pgfqpoint{1.722276in}{1.715097in}}%
\pgfpathlineto{\pgfqpoint{1.725698in}{1.600551in}}%
\pgfpathlineto{\pgfqpoint{1.730826in}{1.519131in}}%
\pgfpathlineto{\pgfqpoint{1.735959in}{1.623646in}}%
\pgfpathlineto{\pgfqpoint{1.738528in}{1.568064in}}%
\pgfpathlineto{\pgfqpoint{1.742805in}{1.624811in}}%
\pgfpathlineto{\pgfqpoint{1.747088in}{1.558964in}}%
\pgfpathlineto{\pgfqpoint{1.751369in}{1.518838in}}%
\pgfpathlineto{\pgfqpoint{1.756502in}{1.610290in}}%
\pgfpathlineto{\pgfqpoint{1.759920in}{1.539252in}}%
\pgfpathlineto{\pgfqpoint{1.765902in}{1.614663in}}%
\pgfpathlineto{\pgfqpoint{1.771888in}{1.532545in}}%
\pgfpathlineto{\pgfqpoint{1.776162in}{1.620672in}}%
\pgfpathlineto{\pgfqpoint{1.777874in}{1.527003in}}%
\pgfpathlineto{\pgfqpoint{1.783002in}{1.591276in}}%
\pgfpathlineto{\pgfqpoint{1.786423in}{1.524961in}}%
\pgfpathlineto{\pgfqpoint{1.789847in}{1.633092in}}%
\pgfpathlineto{\pgfqpoint{1.794127in}{1.538843in}}%
\pgfpathlineto{\pgfqpoint{1.800974in}{1.519014in}}%
\pgfpathlineto{\pgfqpoint{1.806101in}{1.614604in}}%
\pgfpathlineto{\pgfqpoint{1.807814in}{1.513764in}}%
\pgfpathlineto{\pgfqpoint{1.813805in}{1.675202in}}%
\pgfpathlineto{\pgfqpoint{1.817226in}{1.562697in}}%
\pgfpathlineto{\pgfqpoint{1.822358in}{1.686636in}}%
\pgfpathlineto{\pgfqpoint{1.824068in}{1.612712in}}%
\pgfpathlineto{\pgfqpoint{1.831766in}{1.648782in}}%
\pgfpathlineto{\pgfqpoint{1.833476in}{1.558179in}}%
\pgfpathlineto{\pgfqpoint{1.836895in}{1.640267in}}%
\pgfpathlineto{\pgfqpoint{1.842883in}{1.545725in}}%
\pgfpathlineto{\pgfqpoint{1.848875in}{1.685350in}}%
\pgfpathlineto{\pgfqpoint{1.849731in}{1.567567in}}%
\pgfpathlineto{\pgfqpoint{1.854859in}{1.619445in}}%
\pgfpathlineto{\pgfqpoint{1.859990in}{1.529162in}}%
\pgfpathlineto{\pgfqpoint{1.864266in}{1.490436in}}%
\pgfpathlineto{\pgfqpoint{1.868544in}{1.628194in}}%
\pgfpathlineto{\pgfqpoint{1.872817in}{1.640384in}}%
\pgfpathlineto{\pgfqpoint{1.877090in}{1.539486in}}%
\pgfpathlineto{\pgfqpoint{1.879659in}{1.594366in}}%
\pgfpathlineto{\pgfqpoint{1.883937in}{1.544268in}}%
\pgfpathlineto{\pgfqpoint{1.888217in}{1.594717in}}%
\pgfpathlineto{\pgfqpoint{1.895915in}{1.510148in}}%
\pgfpathlineto{\pgfqpoint{1.896771in}{1.590340in}}%
\pgfpathlineto{\pgfqpoint{1.901907in}{1.605913in}}%
\pgfpathlineto{\pgfqpoint{1.907894in}{1.530269in}}%
\pgfpathlineto{\pgfqpoint{1.912175in}{1.508573in}}%
\pgfpathlineto{\pgfqpoint{1.913882in}{1.607605in}}%
\pgfpathlineto{\pgfqpoint{1.918162in}{1.528548in}}%
\pgfpathlineto{\pgfqpoint{1.922443in}{1.636767in}}%
\pgfpathlineto{\pgfqpoint{1.926721in}{1.526072in}}%
\pgfpathlineto{\pgfqpoint{1.932706in}{1.500993in}}%
\pgfpathlineto{\pgfqpoint{1.935269in}{1.617581in}}%
\pgfpathlineto{\pgfqpoint{1.940403in}{1.716613in}}%
\pgfpathlineto{\pgfqpoint{1.947248in}{1.607024in}}%
\pgfpathlineto{\pgfqpoint{1.950671in}{1.719236in}}%
\pgfpathlineto{\pgfqpoint{1.955805in}{1.564184in}}%
\pgfpathlineto{\pgfqpoint{1.960085in}{1.546368in}}%
\pgfpathlineto{\pgfqpoint{1.961796in}{1.620321in}}%
\pgfpathlineto{\pgfqpoint{1.967788in}{1.663654in}}%
\pgfpathlineto{\pgfqpoint{1.970354in}{1.526188in}}%
\pgfpathlineto{\pgfqpoint{1.976334in}{1.645224in}}%
\pgfpathlineto{\pgfqpoint{1.979754in}{1.525023in}}%
\pgfpathlineto{\pgfqpoint{1.985740in}{1.629655in}}%
\pgfpathlineto{\pgfqpoint{1.986595in}{1.553777in}}%
\pgfpathlineto{\pgfqpoint{1.993441in}{1.537707in}}%
\pgfpathlineto{\pgfqpoint{1.995151in}{1.647500in}}%
\pgfpathlineto{\pgfqpoint{2.001137in}{1.518082in}}%
\pgfpathlineto{\pgfqpoint{2.004557in}{1.605099in}}%
\pgfpathlineto{\pgfqpoint{2.010548in}{1.566723in}}%
\pgfpathlineto{\pgfqpoint{2.014825in}{1.650708in}}%
\pgfpathlineto{\pgfqpoint{2.016537in}{1.566489in}}%
\pgfpathlineto{\pgfqpoint{2.021667in}{1.650445in}}%
\pgfpathlineto{\pgfqpoint{2.027657in}{1.671877in}}%
\pgfpathlineto{\pgfqpoint{2.031077in}{1.543537in}}%
\pgfpathlineto{\pgfqpoint{2.035355in}{1.654847in}}%
\pgfpathlineto{\pgfqpoint{2.039628in}{1.555581in}}%
\pgfpathlineto{\pgfqpoint{2.043906in}{1.635661in}}%
\pgfpathlineto{\pgfqpoint{2.047328in}{1.531058in}}%
\pgfpathlineto{\pgfqpoint{2.054173in}{1.525662in}}%
\pgfpathlineto{\pgfqpoint{2.055027in}{1.633735in}}%
\pgfpathlineto{\pgfqpoint{2.060159in}{1.507466in}}%
\pgfpathlineto{\pgfqpoint{2.065290in}{1.621837in}}%
\pgfpathlineto{\pgfqpoint{2.067854in}{1.578504in}}%
\pgfpathlineto{\pgfqpoint{2.075551in}{1.509918in}}%
\pgfpathlineto{\pgfqpoint{2.077262in}{1.664239in}}%
\pgfpathlineto{\pgfqpoint{2.081538in}{1.578095in}}%
\pgfpathlineto{\pgfqpoint{2.084956in}{1.607956in}}%
\pgfpathlineto{\pgfqpoint{2.090086in}{1.524029in}}%
\pgfpathlineto{\pgfqpoint{2.093509in}{1.598684in}}%
\pgfpathlineto{\pgfqpoint{2.097788in}{1.519131in}}%
\pgfpathlineto{\pgfqpoint{2.102927in}{1.624402in}}%
\pgfpathlineto{\pgfqpoint{2.106351in}{1.542109in}}%
\pgfpathlineto{\pgfqpoint{2.114054in}{1.512248in}}%
\pgfpathlineto{\pgfqpoint{2.114910in}{1.620727in}}%
\pgfpathlineto{\pgfqpoint{2.122605in}{1.663946in}}%
\pgfpathlineto{\pgfqpoint{2.124318in}{1.588243in}}%
\pgfpathlineto{\pgfqpoint{2.130310in}{1.557740in}}%
\pgfpathlineto{\pgfqpoint{2.132878in}{1.648373in}}%
\pgfpathlineto{\pgfqpoint{2.138864in}{1.515719in}}%
\pgfpathlineto{\pgfqpoint{2.140576in}{1.599031in}}%
\pgfpathlineto{\pgfqpoint{2.145701in}{1.552257in}}%
\pgfpathlineto{\pgfqpoint{2.151687in}{1.762452in}}%
\pgfpathlineto{\pgfqpoint{2.153397in}{1.633823in}}%
\pgfpathlineto{\pgfqpoint{2.158523in}{1.725709in}}%
\pgfpathlineto{\pgfqpoint{2.164513in}{1.642075in}}%
\pgfpathlineto{\pgfqpoint{2.169647in}{1.646945in}}%
\pgfpathlineto{\pgfqpoint{2.173068in}{1.753297in}}%
\pgfpathlineto{\pgfqpoint{2.176493in}{1.561587in}}%
\pgfpathlineto{\pgfqpoint{2.181627in}{1.683542in}}%
\pgfpathlineto{\pgfqpoint{2.184193in}{1.617812in}}%
\pgfpathlineto{\pgfqpoint{2.189327in}{1.690833in}}%
\pgfpathlineto{\pgfqpoint{2.193602in}{1.635368in}}%
\pgfpathlineto{\pgfqpoint{2.197026in}{1.685642in}}%
\pgfpathlineto{\pgfqpoint{2.203016in}{1.582526in}}%
\pgfpathlineto{\pgfqpoint{2.205581in}{1.681675in}}%
\pgfpathlineto{\pgfqpoint{2.212427in}{1.697190in}}%
\pgfpathlineto{\pgfqpoint{2.215846in}{1.582468in}}%
\pgfpathlineto{\pgfqpoint{2.218411in}{1.701040in}}%
\pgfpathlineto{\pgfqpoint{2.222686in}{1.598976in}}%
\pgfpathlineto{\pgfqpoint{2.226962in}{1.709146in}}%
\pgfpathlineto{\pgfqpoint{2.232095in}{1.635719in}}%
\pgfpathlineto{\pgfqpoint{2.234663in}{1.700806in}}%
\pgfpathlineto{\pgfqpoint{2.241504in}{1.618279in}}%
\pgfpathlineto{\pgfqpoint{2.243215in}{1.705413in}}%
\pgfpathlineto{\pgfqpoint{2.250053in}{1.649539in}}%
\pgfpathlineto{\pgfqpoint{2.255182in}{1.735741in}}%
\pgfpathlineto{\pgfqpoint{2.258601in}{1.748629in}}%
\pgfpathlineto{\pgfqpoint{2.261169in}{1.634813in}}%
\pgfpathlineto{\pgfqpoint{2.264591in}{1.668670in}}%
\pgfpathlineto{\pgfqpoint{2.268867in}{1.510557in}}%
\pgfpathlineto{\pgfqpoint{2.273147in}{1.530795in}}%
\pgfpathlineto{\pgfqpoint{2.278277in}{1.613669in}}%
\pgfpathlineto{\pgfqpoint{2.282560in}{1.540300in}}%
\pgfpathlineto{\pgfqpoint{2.286838in}{1.601015in}}%
\pgfpathlineto{\pgfqpoint{2.290262in}{1.534236in}}%
\pgfpathlineto{\pgfqpoint{2.295399in}{1.639507in}}%
\pgfpathlineto{\pgfqpoint{2.298819in}{1.553919in}}%
\pgfpathlineto{\pgfqpoint{2.309087in}{1.658580in}}%
\pgfpathlineto{\pgfqpoint{2.311649in}{1.592211in}}%
\pgfpathlineto{\pgfqpoint{2.318493in}{1.627960in}}%
\pgfpathlineto{\pgfqpoint{2.321912in}{1.493351in}}%
\pgfpathlineto{\pgfqpoint{2.326189in}{1.578095in}}%
\pgfpathlineto{\pgfqpoint{2.330467in}{1.587074in}}%
\pgfpathlineto{\pgfqpoint{2.333891in}{1.516796in}}%
\pgfpathlineto{\pgfqpoint{2.339879in}{1.502392in}}%
\pgfpathlineto{\pgfqpoint{2.341587in}{1.622798in}}%
\pgfpathlineto{\pgfqpoint{2.348427in}{1.558379in}}%
\pgfpathlineto{\pgfqpoint{2.350137in}{1.639273in}}%
\pgfpathlineto{\pgfqpoint{2.356126in}{1.589409in}}%
\pgfpathlineto{\pgfqpoint{2.363830in}{1.541586in}}%
\pgfpathlineto{\pgfqpoint{2.370673in}{1.723905in}}%
\pgfpathlineto{\pgfqpoint{2.374095in}{1.595330in}}%
\pgfpathlineto{\pgfqpoint{2.375807in}{1.664122in}}%
\pgfpathlineto{\pgfqpoint{2.383504in}{1.680363in}}%
\pgfpathlineto{\pgfqpoint{2.386924in}{1.621720in}}%
\pgfpathlineto{\pgfqpoint{2.392057in}{1.637235in}}%
\pgfpathlineto{\pgfqpoint{2.396335in}{1.513180in}}%
\pgfpathlineto{\pgfqpoint{2.397190in}{1.634375in}}%
\pgfpathlineto{\pgfqpoint{2.402321in}{1.647208in}}%
\pgfpathlineto{\pgfqpoint{2.407451in}{1.534324in}}%
\pgfpathlineto{\pgfqpoint{2.411726in}{1.522250in}}%
\pgfpathlineto{\pgfqpoint{2.417709in}{1.508807in}}%
\pgfpathlineto{\pgfqpoint{2.418564in}{1.608131in}}%
\pgfpathlineto{\pgfqpoint{2.423696in}{1.525370in}}%
\pgfpathlineto{\pgfqpoint{2.427115in}{1.595765in}}%
\pgfpathlineto{\pgfqpoint{2.433100in}{1.582526in}}%
\pgfpathlineto{\pgfqpoint{2.436521in}{1.643708in}}%
\pgfpathlineto{\pgfqpoint{2.441651in}{1.684477in}}%
\pgfpathlineto{\pgfqpoint{2.445075in}{1.554387in}}%
\pgfpathlineto{\pgfqpoint{2.451058in}{1.546017in}}%
\pgfpathlineto{\pgfqpoint{2.452770in}{1.618104in}}%
\pgfpathlineto{\pgfqpoint{2.457908in}{1.645107in}}%
\pgfpathlineto{\pgfqpoint{2.463895in}{1.600142in}}%
\pgfpathlineto{\pgfqpoint{2.465606in}{1.668670in}}%
\pgfpathlineto{\pgfqpoint{2.470738in}{1.560279in}}%
\pgfpathlineto{\pgfqpoint{2.474162in}{1.643708in}}%
\pgfpathlineto{\pgfqpoint{2.479295in}{1.568005in}}%
\pgfpathlineto{\pgfqpoint{2.486139in}{1.598626in}}%
\pgfpathlineto{\pgfqpoint{2.489559in}{1.507729in}}%
\pgfpathlineto{\pgfqpoint{2.494692in}{1.613906in}}%
\pgfpathlineto{\pgfqpoint{2.496403in}{1.504522in}}%
\pgfpathlineto{\pgfqpoint{2.502393in}{1.653564in}}%
\pgfpathlineto{\pgfqpoint{2.506673in}{1.647968in}}%
\pgfpathlineto{\pgfqpoint{2.508382in}{1.546485in}}%
\pgfpathlineto{\pgfqpoint{2.516081in}{1.623762in}}%
\pgfpathlineto{\pgfqpoint{2.518648in}{1.536337in}}%
\pgfpathlineto{\pgfqpoint{2.522070in}{1.603641in}}%
\pgfpathlineto{\pgfqpoint{2.528054in}{1.514583in}}%
\pgfpathlineto{\pgfqpoint{2.530622in}{1.584514in}}%
\pgfpathlineto{\pgfqpoint{2.535758in}{1.638810in}}%
\pgfpathlineto{\pgfqpoint{2.538324in}{1.531730in}}%
\pgfpathlineto{\pgfqpoint{2.543453in}{1.653798in}}%
\pgfpathlineto{\pgfqpoint{2.549441in}{1.526013in}}%
\pgfpathlineto{\pgfqpoint{2.553715in}{1.628018in}}%
\pgfpathlineto{\pgfqpoint{2.556277in}{1.547709in}}%
\pgfpathlineto{\pgfqpoint{2.563119in}{1.626795in}}%
\pgfpathlineto{\pgfqpoint{2.565684in}{1.517030in}}%
\pgfpathlineto{\pgfqpoint{2.569960in}{1.681269in}}%
\pgfpathlineto{\pgfqpoint{2.572527in}{1.615598in}}%
\pgfpathlineto{\pgfqpoint{2.579370in}{1.703667in}}%
\pgfpathlineto{\pgfqpoint{2.581931in}{1.607550in}}%
\pgfpathlineto{\pgfqpoint{2.587061in}{1.688152in}}%
\pgfpathlineto{\pgfqpoint{2.593048in}{1.638897in}}%
\pgfpathlineto{\pgfqpoint{2.595611in}{1.703082in}}%
\pgfpathlineto{\pgfqpoint{2.601596in}{1.594603in}}%
\pgfpathlineto{\pgfqpoint{2.604164in}{1.674913in}}%
\pgfpathlineto{\pgfqpoint{2.609301in}{1.617728in}}%
\pgfpathlineto{\pgfqpoint{2.614436in}{1.710954in}}%
\pgfpathlineto{\pgfqpoint{2.617000in}{1.622597in}}%
\pgfpathlineto{\pgfqpoint{2.619567in}{1.712938in}}%
\pgfpathlineto{\pgfqpoint{2.624695in}{1.682552in}}%
\pgfpathlineto{\pgfqpoint{2.628975in}{1.753473in}}%
\pgfpathlineto{\pgfqpoint{2.632399in}{1.667388in}}%
\pgfpathlineto{\pgfqpoint{2.639241in}{1.693924in}}%
\pgfpathlineto{\pgfqpoint{2.641809in}{1.618279in}}%
\pgfpathlineto{\pgfqpoint{2.646085in}{1.544034in}}%
\pgfpathlineto{\pgfqpoint{2.649509in}{1.532662in}}%
\pgfpathlineto{\pgfqpoint{2.653787in}{1.658814in}}%
\pgfpathlineto{\pgfqpoint{2.658066in}{1.709263in}}%
\pgfpathlineto{\pgfqpoint{2.662344in}{1.603232in}}%
\pgfpathlineto{\pgfqpoint{2.670044in}{1.563048in}}%
\pgfpathlineto{\pgfqpoint{2.672610in}{1.632102in}}%
\pgfpathlineto{\pgfqpoint{2.677741in}{1.523039in}}%
\pgfpathlineto{\pgfqpoint{2.681163in}{1.635427in}}%
\pgfpathlineto{\pgfqpoint{2.684586in}{1.527967in}}%
\pgfpathlineto{\pgfqpoint{2.693143in}{1.714337in}}%
\pgfpathlineto{\pgfqpoint{2.696565in}{1.561415in}}%
\pgfpathlineto{\pgfqpoint{2.702555in}{1.600609in}}%
\pgfpathlineto{\pgfqpoint{2.708544in}{1.530912in}}%
\pgfpathlineto{\pgfqpoint{2.709399in}{1.621720in}}%
\pgfpathlineto{\pgfqpoint{2.714525in}{1.535811in}}%
\pgfpathlineto{\pgfqpoint{2.721365in}{1.653740in}}%
\pgfpathlineto{\pgfqpoint{2.723934in}{1.549634in}}%
\pgfpathlineto{\pgfqpoint{2.726501in}{1.616997in}}%
\pgfpathlineto{\pgfqpoint{2.731635in}{1.622159in}}%
\pgfpathlineto{\pgfqpoint{2.737626in}{1.495919in}}%
\pgfpathlineto{\pgfqpoint{2.739338in}{1.585968in}}%
\pgfpathlineto{\pgfqpoint{2.745323in}{1.525019in}}%
\pgfpathlineto{\pgfqpoint{2.747889in}{1.617578in}}%
\pgfpathlineto{\pgfqpoint{2.755581in}{1.547738in}}%
\pgfpathlineto{\pgfqpoint{2.756436in}{1.615832in}}%
\pgfpathlineto{\pgfqpoint{2.762417in}{1.532749in}}%
\pgfpathlineto{\pgfqpoint{2.764980in}{1.610465in}}%
\pgfpathlineto{\pgfqpoint{2.770971in}{1.523098in}}%
\pgfpathlineto{\pgfqpoint{2.776103in}{1.612098in}}%
\pgfpathlineto{\pgfqpoint{2.777811in}{1.565266in}}%
\pgfpathlineto{\pgfqpoint{2.782945in}{1.630703in}}%
\pgfpathlineto{\pgfqpoint{2.789789in}{1.558149in}}%
\pgfpathlineto{\pgfqpoint{2.790642in}{1.599703in}}%
\pgfpathlineto{\pgfqpoint{2.794918in}{1.536684in}}%
\pgfpathlineto{\pgfqpoint{2.799189in}{1.635076in}}%
\pgfpathlineto{\pgfqpoint{2.803464in}{1.564097in}}%
\pgfpathlineto{\pgfqpoint{2.809450in}{1.615072in}}%
\pgfpathlineto{\pgfqpoint{2.812873in}{1.669839in}}%
\pgfpathlineto{\pgfqpoint{2.817150in}{1.508398in}}%
\pgfpathlineto{\pgfqpoint{2.820569in}{1.624928in}}%
\pgfpathlineto{\pgfqpoint{2.827409in}{1.610465in}}%
\pgfpathlineto{\pgfqpoint{2.829973in}{1.686168in}}%
\pgfpathlineto{\pgfqpoint{2.833396in}{1.617523in}}%
\pgfpathlineto{\pgfqpoint{2.839382in}{1.710081in}}%
\pgfpathlineto{\pgfqpoint{2.844515in}{1.647880in}}%
\pgfpathlineto{\pgfqpoint{2.849642in}{1.707630in}}%
\pgfpathlineto{\pgfqpoint{2.850497in}{1.536278in}}%
\pgfpathlineto{\pgfqpoint{2.855627in}{1.637060in}}%
\pgfpathlineto{\pgfqpoint{2.859900in}{1.463140in}}%
\pgfpathlineto{\pgfqpoint{2.865887in}{1.641348in}}%
\pgfpathlineto{\pgfqpoint{2.867599in}{1.521348in}}%
\pgfpathlineto{\pgfqpoint{2.874440in}{1.510440in}}%
\pgfpathlineto{\pgfqpoint{2.876150in}{1.688674in}}%
\pgfpathlineto{\pgfqpoint{2.880425in}{1.530561in}}%
\pgfpathlineto{\pgfqpoint{2.885558in}{1.672754in}}%
\pgfpathlineto{\pgfqpoint{2.889835in}{1.530269in}}%
\pgfpathlineto{\pgfqpoint{2.893259in}{1.609296in}}%
\pgfpathlineto{\pgfqpoint{2.900103in}{1.564798in}}%
\pgfpathlineto{\pgfqpoint{2.905233in}{1.724076in}}%
\pgfpathlineto{\pgfqpoint{2.907800in}{1.665579in}}%
\pgfpathlineto{\pgfqpoint{2.911219in}{1.714396in}}%
\pgfpathlineto{\pgfqpoint{2.918060in}{1.636943in}}%
\pgfpathlineto{\pgfqpoint{2.921481in}{1.690428in}}%
\pgfpathlineto{\pgfqpoint{2.924043in}{1.543336in}}%
\pgfpathlineto{\pgfqpoint{2.930027in}{1.644760in}}%
\pgfpathlineto{\pgfqpoint{2.932595in}{1.528289in}}%
\pgfpathlineto{\pgfqpoint{2.936017in}{1.621140in}}%
\pgfpathlineto{\pgfqpoint{2.941150in}{1.534558in}}%
\pgfpathlineto{\pgfqpoint{2.944568in}{1.630937in}}%
\pgfpathlineto{\pgfqpoint{2.948846in}{1.563282in}}%
\pgfpathlineto{\pgfqpoint{2.954831in}{1.526130in}}%
\pgfpathlineto{\pgfqpoint{2.963385in}{1.667388in}}%
\pgfpathlineto{\pgfqpoint{2.965950in}{1.608715in}}%
\pgfpathlineto{\pgfqpoint{2.972797in}{1.584978in}}%
\pgfpathlineto{\pgfqpoint{2.977928in}{1.638108in}}%
\pgfpathlineto{\pgfqpoint{2.979640in}{1.572846in}}%
\pgfpathlineto{\pgfqpoint{2.983917in}{1.681328in}}%
\pgfpathlineto{\pgfqpoint{2.989050in}{1.635953in}}%
\pgfpathlineto{\pgfqpoint{2.991615in}{1.694099in}}%
\pgfpathlineto{\pgfqpoint{2.997600in}{1.638722in}}%
\pgfpathlineto{\pgfqpoint{3.000167in}{1.715649in}}%
\pgfpathlineto{\pgfqpoint{3.004446in}{1.605099in}}%
\pgfpathlineto{\pgfqpoint{3.008722in}{1.759888in}}%
\pgfpathlineto{\pgfqpoint{3.012997in}{1.605972in}}%
\pgfpathlineto{\pgfqpoint{3.018130in}{1.702848in}}%
\pgfpathlineto{\pgfqpoint{3.024118in}{1.633151in}}%
\pgfpathlineto{\pgfqpoint{3.026684in}{1.713928in}}%
\pgfpathlineto{\pgfqpoint{3.030958in}{1.753820in}}%
\pgfpathlineto{\pgfqpoint{3.034379in}{1.635310in}}%
\pgfpathlineto{\pgfqpoint{3.041219in}{1.621837in}}%
\pgfpathlineto{\pgfqpoint{3.046352in}{1.752132in}}%
\pgfpathlineto{\pgfqpoint{3.048063in}{1.665872in}}%
\pgfpathlineto{\pgfqpoint{3.054053in}{1.716145in}}%
\pgfpathlineto{\pgfqpoint{3.058329in}{1.584919in}}%
\pgfpathlineto{\pgfqpoint{3.062609in}{1.538667in}}%
\pgfpathlineto{\pgfqpoint{3.064319in}{1.661554in}}%
\pgfpathlineto{\pgfqpoint{3.069452in}{1.569404in}}%
\pgfpathlineto{\pgfqpoint{3.076293in}{1.663654in}}%
\pgfpathlineto{\pgfqpoint{3.078004in}{1.572963in}}%
\pgfpathlineto{\pgfqpoint{3.083138in}{1.587893in}}%
\pgfpathlineto{\pgfqpoint{3.089122in}{1.509943in}}%
\pgfpathlineto{\pgfqpoint{3.092546in}{1.618396in}}%
\pgfpathlineto{\pgfqpoint{3.095968in}{1.544677in}}%
\pgfpathlineto{\pgfqpoint{3.101097in}{1.608862in}}%
\pgfpathlineto{\pgfqpoint{3.103662in}{1.521377in}}%
\pgfpathlineto{\pgfqpoint{3.109653in}{1.635602in}}%
\pgfpathlineto{\pgfqpoint{3.113076in}{1.541440in}}%
\pgfpathlineto{\pgfqpoint{3.118215in}{1.754755in}}%
\pgfpathlineto{\pgfqpoint{3.122495in}{1.625366in}}%
\pgfpathlineto{\pgfqpoint{3.124207in}{1.698125in}}%
\pgfpathlineto{\pgfqpoint{3.129335in}{1.650912in}}%
\pgfpathlineto{\pgfqpoint{3.135323in}{1.791556in}}%
\pgfpathlineto{\pgfqpoint{3.138744in}{1.664502in}}%
\pgfpathlineto{\pgfqpoint{3.143874in}{1.709263in}}%
\pgfpathlineto{\pgfqpoint{3.145581in}{1.587659in}}%
\pgfpathlineto{\pgfqpoint{3.149859in}{1.566022in}}%
\pgfpathlineto{\pgfqpoint{3.155843in}{1.541528in}}%
\pgfpathlineto{\pgfqpoint{3.159266in}{1.646624in}}%
\pgfpathlineto{\pgfqpoint{3.164395in}{1.533885in}}%
\pgfpathlineto{\pgfqpoint{3.168665in}{1.643825in}}%
\pgfpathlineto{\pgfqpoint{3.172941in}{1.553831in}}%
\pgfpathlineto{\pgfqpoint{3.178926in}{1.675815in}}%
\pgfpathlineto{\pgfqpoint{3.179782in}{1.551092in}}%
\pgfpathlineto{\pgfqpoint{3.186627in}{1.532194in}}%
\pgfpathlineto{\pgfqpoint{3.189194in}{1.619795in}}%
\pgfpathlineto{\pgfqpoint{3.196038in}{1.499766in}}%
\pgfpathlineto{\pgfqpoint{3.200317in}{1.646331in}}%
\pgfpathlineto{\pgfqpoint{3.204595in}{1.519043in}}%
\pgfpathlineto{\pgfqpoint{3.206306in}{1.654847in}}%
\pgfpathlineto{\pgfqpoint{3.209727in}{1.521929in}}%
\pgfpathlineto{\pgfqpoint{3.214860in}{1.640559in}}%
\pgfpathlineto{\pgfqpoint{3.218280in}{1.652980in}}%
\pgfpathlineto{\pgfqpoint{3.225977in}{1.532223in}}%
\pgfpathlineto{\pgfqpoint{3.230256in}{1.690249in}}%
\pgfpathlineto{\pgfqpoint{3.232821in}{1.584539in}}%
\pgfpathlineto{\pgfqpoint{3.235387in}{1.707513in}}%
\pgfpathlineto{\pgfqpoint{3.239662in}{1.655285in}}%
\pgfpathlineto{\pgfqpoint{3.247355in}{1.719236in}}%
\pgfpathlineto{\pgfqpoint{3.251630in}{1.647003in}}%
\pgfpathlineto{\pgfqpoint{3.255054in}{1.691882in}}%
\pgfpathlineto{\pgfqpoint{3.259333in}{1.701913in}}%
\pgfpathlineto{\pgfqpoint{3.264463in}{1.735218in}}%
\pgfpathlineto{\pgfqpoint{3.268739in}{1.542342in}}%
\pgfpathlineto{\pgfqpoint{3.269594in}{1.626268in}}%
\pgfpathlineto{\pgfqpoint{3.273868in}{1.549575in}}%
\pgfpathlineto{\pgfqpoint{3.281573in}{1.644172in}}%
\pgfpathlineto{\pgfqpoint{3.284995in}{1.577277in}}%
\pgfpathlineto{\pgfqpoint{3.286705in}{1.674909in}}%
\pgfpathlineto{\pgfqpoint{3.291835in}{1.540125in}}%
\pgfpathlineto{\pgfqpoint{3.296964in}{1.633443in}}%
\pgfpathlineto{\pgfqpoint{3.299528in}{1.585149in}}%
\pgfpathlineto{\pgfqpoint{3.306369in}{1.632859in}}%
\pgfpathlineto{\pgfqpoint{3.309787in}{1.521812in}}%
\pgfpathlineto{\pgfqpoint{3.312350in}{1.641140in}}%
\pgfpathlineto{\pgfqpoint{3.320047in}{1.567772in}}%
\pgfpathlineto{\pgfqpoint{3.323472in}{1.688324in}}%
\pgfpathlineto{\pgfqpoint{3.327748in}{1.566314in}}%
\pgfpathlineto{\pgfqpoint{3.330315in}{1.650240in}}%
\pgfpathlineto{\pgfqpoint{3.337159in}{1.644757in}}%
\pgfpathlineto{\pgfqpoint{3.339724in}{1.514988in}}%
\pgfpathlineto{\pgfqpoint{3.342293in}{1.608482in}}%
\pgfpathlineto{\pgfqpoint{3.348282in}{1.539193in}}%
\pgfpathlineto{\pgfqpoint{3.351703in}{1.615886in}}%
\pgfpathlineto{\pgfqpoint{3.357690in}{1.547417in}}%
\pgfpathlineto{\pgfqpoint{3.359402in}{1.642656in}}%
\pgfpathlineto{\pgfqpoint{3.364537in}{1.575965in}}%
\pgfpathlineto{\pgfqpoint{3.368813in}{1.669777in}}%
\pgfpathlineto{\pgfqpoint{3.372234in}{1.579319in}}%
\pgfpathlineto{\pgfqpoint{3.379076in}{1.509972in}}%
\pgfpathlineto{\pgfqpoint{3.380784in}{1.617344in}}%
\pgfpathlineto{\pgfqpoint{3.385917in}{1.571414in}}%
\pgfpathlineto{\pgfqpoint{3.391052in}{1.654613in}}%
\pgfpathlineto{\pgfqpoint{3.396180in}{1.556980in}}%
\pgfpathlineto{\pgfqpoint{3.400457in}{1.637757in}}%
\pgfpathlineto{\pgfqpoint{3.405589in}{1.592207in}}%
\pgfpathlineto{\pgfqpoint{3.406443in}{1.644348in}}%
\pgfpathlineto{\pgfqpoint{3.414139in}{1.562347in}}%
\pgfpathlineto{\pgfqpoint{3.418410in}{1.655866in}}%
\pgfpathlineto{\pgfqpoint{3.420976in}{1.581971in}}%
\pgfpathlineto{\pgfqpoint{3.425249in}{1.662544in}}%
\pgfpathlineto{\pgfqpoint{3.427816in}{1.568294in}}%
\pgfpathlineto{\pgfqpoint{3.432947in}{1.504021in}}%
\pgfpathlineto{\pgfqpoint{3.437223in}{1.637378in}}%
\pgfpathlineto{\pgfqpoint{3.442356in}{1.551439in}}%
\pgfpathlineto{\pgfqpoint{3.445779in}{1.613377in}}%
\pgfpathlineto{\pgfqpoint{3.450907in}{1.684415in}}%
\pgfpathlineto{\pgfqpoint{3.456036in}{1.559198in}}%
\pgfpathlineto{\pgfqpoint{3.458605in}{1.639273in}}%
\pgfpathlineto{\pgfqpoint{3.463737in}{1.521929in}}%
\pgfpathlineto{\pgfqpoint{3.466302in}{1.648549in}}%
\pgfpathlineto{\pgfqpoint{3.472285in}{1.662953in}}%
\pgfpathlineto{\pgfqpoint{3.475705in}{1.560831in}}%
\pgfpathlineto{\pgfqpoint{3.479984in}{1.540651in}}%
\pgfpathlineto{\pgfqpoint{3.484262in}{1.609676in}}%
\pgfpathlineto{\pgfqpoint{3.491104in}{1.676601in}}%
\pgfpathlineto{\pgfqpoint{3.492814in}{1.570278in}}%
\pgfpathlineto{\pgfqpoint{3.497093in}{1.535285in}}%
\pgfpathlineto{\pgfqpoint{3.500515in}{1.625278in}}%
\pgfpathlineto{\pgfqpoint{3.504790in}{1.574712in}}%
\pgfpathlineto{\pgfqpoint{3.512483in}{1.545521in}}%
\pgfpathlineto{\pgfqpoint{3.515905in}{1.714454in}}%
\pgfpathlineto{\pgfqpoint{3.517615in}{1.584656in}}%
\pgfpathlineto{\pgfqpoint{3.524462in}{1.674095in}}%
\pgfpathlineto{\pgfqpoint{3.527029in}{1.523796in}}%
\pgfpathlineto{\pgfqpoint{3.533013in}{1.699582in}}%
\pgfpathlineto{\pgfqpoint{3.534724in}{1.637410in}}%
\pgfpathlineto{\pgfqpoint{3.539857in}{1.759595in}}%
\pgfpathlineto{\pgfqpoint{3.543277in}{1.658814in}}%
\pgfpathlineto{\pgfqpoint{3.547556in}{1.728975in}}%
\pgfpathlineto{\pgfqpoint{3.554393in}{1.612036in}}%
\pgfpathlineto{\pgfqpoint{3.556960in}{1.793712in}}%
\pgfpathlineto{\pgfqpoint{3.560375in}{1.677357in}}%
\pgfpathlineto{\pgfqpoint{3.565506in}{1.727167in}}%
\pgfpathlineto{\pgfqpoint{3.574058in}{1.638605in}}%
\pgfpathlineto{\pgfqpoint{3.578334in}{1.726381in}}%
\pgfpathlineto{\pgfqpoint{3.583469in}{1.678672in}}%
\pgfpathlineto{\pgfqpoint{3.589452in}{1.745772in}}%
\pgfpathlineto{\pgfqpoint{3.590304in}{1.682084in}}%
\pgfpathlineto{\pgfqpoint{3.596294in}{1.787005in}}%
\pgfpathlineto{\pgfqpoint{3.599713in}{1.615828in}}%
\pgfpathlineto{\pgfqpoint{3.604849in}{1.739123in}}%
\pgfpathlineto{\pgfqpoint{3.609981in}{1.762452in}}%
\pgfpathlineto{\pgfqpoint{3.611692in}{1.669981in}}%
\pgfpathlineto{\pgfqpoint{3.618535in}{1.656626in}}%
\pgfpathlineto{\pgfqpoint{3.620247in}{1.766536in}}%
\pgfpathlineto{\pgfqpoint{3.624526in}{1.656655in}}%
\pgfpathlineto{\pgfqpoint{3.631368in}{1.744607in}}%
\pgfpathlineto{\pgfqpoint{3.633075in}{1.663946in}}%
\pgfpathlineto{\pgfqpoint{3.639063in}{1.637001in}}%
\pgfpathlineto{\pgfqpoint{3.645057in}{1.731017in}}%
\pgfpathlineto{\pgfqpoint{3.647625in}{1.638342in}}%
\pgfpathlineto{\pgfqpoint{3.651902in}{1.715795in}}%
\pgfpathlineto{\pgfqpoint{3.655322in}{1.656772in}}%
\pgfpathlineto{\pgfqpoint{3.662162in}{1.789573in}}%
\pgfpathlineto{\pgfqpoint{3.666437in}{1.655022in}}%
\pgfpathlineto{\pgfqpoint{3.667291in}{1.730491in}}%
\pgfpathlineto{\pgfqpoint{3.674993in}{1.633385in}}%
\pgfpathlineto{\pgfqpoint{3.676704in}{1.750729in}}%
\pgfpathlineto{\pgfqpoint{3.680124in}{1.559081in}}%
\pgfpathlineto{\pgfqpoint{3.685255in}{1.701040in}}%
\pgfpathlineto{\pgfqpoint{3.689529in}{1.634959in}}%
\pgfpathlineto{\pgfqpoint{3.694663in}{1.674212in}}%
\pgfpathlineto{\pgfqpoint{3.699794in}{1.675435in}}%
\pgfpathlineto{\pgfqpoint{3.701506in}{1.553017in}}%
\pgfpathlineto{\pgfqpoint{3.708349in}{1.651581in}}%
\pgfpathlineto{\pgfqpoint{3.710060in}{1.550945in}}%
\pgfpathlineto{\pgfqpoint{3.714341in}{1.638400in}}%
\pgfpathlineto{\pgfqpoint{3.721187in}{1.667793in}}%
\pgfpathlineto{\pgfqpoint{3.723756in}{1.552082in}}%
\pgfpathlineto{\pgfqpoint{3.729741in}{1.636358in}}%
\pgfpathlineto{\pgfqpoint{3.731453in}{1.564999in}}%
\pgfpathlineto{\pgfqpoint{3.735725in}{1.659161in}}%
\pgfpathlineto{\pgfqpoint{3.740853in}{1.685028in}}%
\pgfpathlineto{\pgfqpoint{3.746839in}{1.577452in}}%
\pgfpathlineto{\pgfqpoint{3.751968in}{1.677708in}}%
\pgfpathlineto{\pgfqpoint{3.752824in}{1.592295in}}%
\pgfpathlineto{\pgfqpoint{3.757956in}{1.511196in}}%
\pgfpathlineto{\pgfqpoint{3.764798in}{1.566778in}}%
\pgfpathlineto{\pgfqpoint{3.766511in}{1.685584in}}%
\pgfpathlineto{\pgfqpoint{3.769934in}{1.625162in}}%
\pgfpathlineto{\pgfqpoint{3.775066in}{1.733118in}}%
\pgfpathlineto{\pgfqpoint{3.778487in}{1.628632in}}%
\pgfpathlineto{\pgfqpoint{3.787040in}{1.740464in}}%
\pgfpathlineto{\pgfqpoint{3.791320in}{1.603638in}}%
\pgfpathlineto{\pgfqpoint{3.799020in}{1.741166in}}%
\pgfpathlineto{\pgfqpoint{3.802439in}{1.640092in}}%
\pgfpathlineto{\pgfqpoint{3.807568in}{1.715561in}}%
\pgfpathlineto{\pgfqpoint{3.809278in}{1.649743in}}%
\pgfpathlineto{\pgfqpoint{3.816117in}{1.714830in}}%
\pgfpathlineto{\pgfqpoint{3.820392in}{1.558146in}}%
\pgfpathlineto{\pgfqpoint{3.823810in}{1.558672in}}%
\pgfpathlineto{\pgfqpoint{3.828940in}{1.690190in}}%
\pgfpathlineto{\pgfqpoint{3.832358in}{1.735770in}}%
\pgfpathlineto{\pgfqpoint{3.837489in}{1.612942in}}%
\pgfpathlineto{\pgfqpoint{3.838344in}{1.685525in}}%
\pgfpathlineto{\pgfqpoint{3.844324in}{1.558876in}}%
\pgfpathlineto{\pgfqpoint{3.850314in}{1.651055in}}%
\pgfpathlineto{\pgfqpoint{3.853739in}{1.562259in}}%
\pgfpathlineto{\pgfqpoint{3.857163in}{1.641023in}}%
\pgfpathlineto{\pgfqpoint{3.860589in}{1.573894in}}%
\pgfpathlineto{\pgfqpoint{3.867436in}{1.661495in}}%
\pgfpathlineto{\pgfqpoint{3.869999in}{1.552140in}}%
\pgfpathlineto{\pgfqpoint{3.872563in}{1.611773in}}%
\pgfpathlineto{\pgfqpoint{3.877696in}{1.662719in}}%
\pgfpathlineto{\pgfqpoint{3.881975in}{1.570511in}}%
\pgfpathlineto{\pgfqpoint{3.887961in}{1.558613in}}%
\pgfpathlineto{\pgfqpoint{3.889671in}{1.613318in}}%
\pgfpathlineto{\pgfqpoint{3.896512in}{1.584740in}}%
\pgfpathlineto{\pgfqpoint{3.899075in}{1.679399in}}%
\pgfpathlineto{\pgfqpoint{3.902494in}{1.582990in}}%
\pgfpathlineto{\pgfqpoint{3.910196in}{1.517465in}}%
\pgfpathlineto{\pgfqpoint{3.911052in}{1.606377in}}%
\pgfpathlineto{\pgfqpoint{3.918748in}{1.665634in}}%
\pgfpathlineto{\pgfqpoint{3.922167in}{1.628745in}}%
\pgfpathlineto{\pgfqpoint{3.925587in}{1.686165in}}%
\pgfpathlineto{\pgfqpoint{3.931576in}{1.623613in}}%
\pgfpathlineto{\pgfqpoint{3.933287in}{1.696196in}}%
\pgfpathlineto{\pgfqpoint{3.938418in}{1.668462in}}%
\pgfpathlineto{\pgfqpoint{3.940985in}{1.571268in}}%
\pgfpathlineto{\pgfqpoint{3.947833in}{1.541232in}}%
\pgfpathlineto{\pgfqpoint{3.952966in}{1.614137in}}%
\pgfpathlineto{\pgfqpoint{3.955532in}{1.519185in}}%
\pgfpathlineto{\pgfqpoint{3.958954in}{1.664820in}}%
\pgfpathlineto{\pgfqpoint{3.962375in}{1.573777in}}%
\pgfpathlineto{\pgfqpoint{3.970067in}{1.554588in}}%
\pgfpathlineto{\pgfqpoint{3.970924in}{1.617052in}}%
\pgfpathlineto{\pgfqpoint{3.976053in}{1.542372in}}%
\pgfpathlineto{\pgfqpoint{3.979475in}{1.659161in}}%
\pgfpathlineto{\pgfqpoint{3.986324in}{1.660560in}}%
\pgfpathlineto{\pgfqpoint{3.988889in}{1.569752in}}%
\pgfpathlineto{\pgfqpoint{3.992312in}{1.621395in}}%
\pgfpathlineto{\pgfqpoint{3.997445in}{1.547676in}}%
\pgfpathlineto{\pgfqpoint{4.000867in}{1.667735in}}%
\pgfpathlineto{\pgfqpoint{4.005147in}{1.593080in}}%
\pgfpathlineto{\pgfqpoint{4.011132in}{1.640819in}}%
\pgfpathlineto{\pgfqpoint{4.017123in}{1.559198in}}%
\pgfpathlineto{\pgfqpoint{4.020541in}{1.713519in}}%
\pgfpathlineto{\pgfqpoint{4.023107in}{1.577569in}}%
\pgfpathlineto{\pgfqpoint{4.027385in}{1.653681in}}%
\pgfpathlineto{\pgfqpoint{4.030804in}{1.596174in}}%
\pgfpathlineto{\pgfqpoint{4.035079in}{1.562171in}}%
\pgfpathlineto{\pgfqpoint{4.040214in}{1.590282in}}%
\pgfpathlineto{\pgfqpoint{4.043636in}{1.746006in}}%
\pgfpathlineto{\pgfqpoint{4.049623in}{1.648899in}}%
\pgfpathlineto{\pgfqpoint{4.053904in}{1.744168in}}%
\pgfpathlineto{\pgfqpoint{4.058181in}{1.602881in}}%
\pgfpathlineto{\pgfqpoint{4.062458in}{1.671410in}}%
\pgfpathlineto{\pgfqpoint{4.067590in}{1.526535in}}%
\pgfpathlineto{\pgfqpoint{4.069301in}{1.591739in}}%
\pgfpathlineto{\pgfqpoint{4.076144in}{1.535051in}}%
\pgfpathlineto{\pgfqpoint{4.080418in}{1.616993in}}%
\pgfpathlineto{\pgfqpoint{4.085551in}{1.552721in}}%
\pgfpathlineto{\pgfqpoint{4.087264in}{1.728274in}}%
\pgfpathlineto{\pgfqpoint{4.091542in}{1.646360in}}%
\pgfpathlineto{\pgfqpoint{4.097525in}{1.776217in}}%
\pgfpathlineto{\pgfqpoint{4.099237in}{1.674909in}}%
\pgfpathlineto{\pgfqpoint{4.104369in}{1.718188in}}%
\pgfpathlineto{\pgfqpoint{4.110358in}{1.639858in}}%
\pgfpathlineto{\pgfqpoint{4.114636in}{1.697219in}}%
\pgfpathlineto{\pgfqpoint{4.119765in}{1.657674in}}%
\pgfpathlineto{\pgfqpoint{4.120620in}{1.730082in}}%
\pgfpathlineto{\pgfqpoint{4.127462in}{1.640965in}}%
\pgfpathlineto{\pgfqpoint{4.132595in}{1.601654in}}%
\pgfpathlineto{\pgfqpoint{4.135165in}{1.683074in}}%
\pgfpathlineto{\pgfqpoint{4.138587in}{1.610841in}}%
\pgfpathlineto{\pgfqpoint{4.143724in}{1.701504in}}%
\pgfpathlineto{\pgfqpoint{4.147142in}{1.642832in}}%
\pgfpathlineto{\pgfqpoint{4.149711in}{1.691765in}}%
\pgfpathlineto{\pgfqpoint{4.153137in}{1.632157in}}%
\pgfpathlineto{\pgfqpoint{4.157412in}{1.639040in}}%
\pgfpathlineto{\pgfqpoint{4.159124in}{1.729205in}}%
\pgfpathlineto{\pgfqpoint{4.165117in}{1.738597in}}%
\pgfpathlineto{\pgfqpoint{4.166829in}{1.644114in}}%
\pgfpathlineto{\pgfqpoint{4.170249in}{1.725066in}}%
\pgfpathlineto{\pgfqpoint{4.172814in}{1.684765in}}%
\pgfpathlineto{\pgfqpoint{4.177947in}{1.656421in}}%
\pgfpathlineto{\pgfqpoint{4.180510in}{1.745831in}}%
\pgfpathlineto{\pgfqpoint{4.184787in}{1.651347in}}%
\pgfpathlineto{\pgfqpoint{4.186496in}{1.689547in}}%
\pgfpathlineto{\pgfqpoint{4.190773in}{1.632976in}}%
\pgfpathlineto{\pgfqpoint{4.194194in}{1.722093in}}%
\pgfpathlineto{\pgfqpoint{4.197616in}{1.677941in}}%
\pgfpathlineto{\pgfqpoint{4.205311in}{1.739299in}}%
\pgfpathlineto{\pgfqpoint{4.208735in}{1.737520in}}%
\pgfpathlineto{\pgfqpoint{4.213009in}{1.622301in}}%
\pgfpathlineto{\pgfqpoint{4.214719in}{1.639741in}}%
\pgfpathlineto{\pgfqpoint{4.218141in}{1.614546in}}%
\pgfpathlineto{\pgfqpoint{4.221561in}{1.715093in}}%
\pgfpathlineto{\pgfqpoint{4.225839in}{1.656246in}}%
\pgfpathlineto{\pgfqpoint{4.227550in}{1.710604in}}%
\pgfpathlineto{\pgfqpoint{4.232681in}{1.634930in}}%
\pgfpathlineto{\pgfqpoint{4.234393in}{1.733235in}}%
\pgfpathlineto{\pgfqpoint{4.238671in}{1.731602in}}%
\pgfpathlineto{\pgfqpoint{4.244659in}{1.518079in}}%
\pgfpathlineto{\pgfqpoint{4.248080in}{1.675319in}}%
\pgfpathlineto{\pgfqpoint{4.252359in}{1.577218in}}%
\pgfpathlineto{\pgfqpoint{4.254925in}{1.676718in}}%
\pgfpathlineto{\pgfqpoint{4.260913in}{1.589292in}}%
\pgfpathlineto{\pgfqpoint{4.261769in}{1.642832in}}%
\pgfpathlineto{\pgfqpoint{4.265191in}{1.603521in}}%
\pgfpathlineto{\pgfqpoint{4.270325in}{1.592733in}}%
\pgfpathlineto{\pgfqpoint{4.274604in}{1.659629in}}%
\pgfpathlineto{\pgfqpoint{4.278880in}{1.570278in}}%
\pgfpathlineto{\pgfqpoint{4.282299in}{1.671293in}}%
\pgfpathlineto{\pgfqpoint{4.286570in}{1.594366in}}%
\pgfpathlineto{\pgfqpoint{4.290850in}{1.643065in}}%
\pgfpathlineto{\pgfqpoint{4.293417in}{1.564038in}}%
\pgfpathlineto{\pgfqpoint{4.297691in}{1.634316in}}%
\pgfpathlineto{\pgfqpoint{4.301964in}{1.564911in}}%
\pgfpathlineto{\pgfqpoint{4.303673in}{1.641257in}}%
\pgfpathlineto{\pgfqpoint{4.307096in}{1.726641in}}%
\pgfpathlineto{\pgfqpoint{4.311374in}{1.575352in}}%
\pgfpathlineto{\pgfqpoint{4.313941in}{1.672107in}}%
\pgfpathlineto{\pgfqpoint{4.319069in}{1.674325in}}%
\pgfpathlineto{\pgfqpoint{4.319924in}{1.593489in}}%
\pgfpathlineto{\pgfqpoint{4.325903in}{1.571560in}}%
\pgfpathlineto{\pgfqpoint{4.326759in}{1.621804in}}%
\pgfpathlineto{\pgfqpoint{4.331893in}{1.531084in}}%
\pgfpathlineto{\pgfqpoint{4.336171in}{1.524610in}}%
\pgfpathlineto{\pgfqpoint{4.337026in}{1.643938in}}%
\pgfpathlineto{\pgfqpoint{4.340445in}{1.576575in}}%
\pgfpathlineto{\pgfqpoint{4.344726in}{1.642013in}}%
\pgfpathlineto{\pgfqpoint{4.348149in}{1.589931in}}%
\pgfpathlineto{\pgfqpoint{4.353279in}{1.660969in}}%
\pgfpathlineto{\pgfqpoint{4.354134in}{1.547764in}}%
\pgfpathlineto{\pgfqpoint{4.358409in}{1.614601in}}%
\pgfpathlineto{\pgfqpoint{4.363544in}{1.578384in}}%
\pgfpathlineto{\pgfqpoint{4.366112in}{1.667442in}}%
\pgfpathlineto{\pgfqpoint{4.369535in}{1.579841in}}%
\pgfpathlineto{\pgfqpoint{4.371246in}{1.669338in}}%
\pgfpathlineto{\pgfqpoint{4.377235in}{1.578676in}}%
\pgfpathlineto{\pgfqpoint{4.379798in}{1.623817in}}%
\pgfpathlineto{\pgfqpoint{4.383215in}{1.504635in}}%
\pgfpathlineto{\pgfqpoint{4.386637in}{1.655837in}}%
\pgfpathlineto{\pgfqpoint{4.390915in}{1.665050in}}%
\pgfpathlineto{\pgfqpoint{4.392627in}{1.551526in}}%
\pgfpathlineto{\pgfqpoint{4.395194in}{1.631748in}}%
\pgfpathlineto{\pgfqpoint{4.399466in}{1.576049in}}%
\pgfpathlineto{\pgfqpoint{4.404602in}{1.608712in}}%
\pgfpathlineto{\pgfqpoint{4.408022in}{1.529041in}}%
\pgfpathlineto{\pgfqpoint{4.409735in}{1.608653in}}%
\pgfpathlineto{\pgfqpoint{4.416578in}{1.546481in}}%
\pgfpathlineto{\pgfqpoint{4.419143in}{1.606030in}}%
\pgfpathlineto{\pgfqpoint{4.428550in}{1.545955in}}%
\pgfpathlineto{\pgfqpoint{4.429406in}{1.630407in}}%
\pgfpathlineto{\pgfqpoint{4.432826in}{1.678526in}}%
\pgfpathlineto{\pgfqpoint{4.436247in}{1.553831in}}%
\pgfpathlineto{\pgfqpoint{4.440528in}{1.634433in}}%
\pgfpathlineto{\pgfqpoint{4.444808in}{1.572378in}}%
\pgfpathlineto{\pgfqpoint{4.448226in}{1.690132in}}%
\pgfpathlineto{\pgfqpoint{4.451649in}{1.591389in}}%
\pgfpathlineto{\pgfqpoint{4.453359in}{1.648808in}}%
\pgfpathlineto{\pgfqpoint{4.457633in}{1.583516in}}%
\pgfpathlineto{\pgfqpoint{4.461057in}{1.635306in}}%
\pgfpathlineto{\pgfqpoint{4.464478in}{1.667092in}}%
\pgfpathlineto{\pgfqpoint{4.468754in}{1.640380in}}%
\pgfpathlineto{\pgfqpoint{4.473032in}{1.485475in}}%
\pgfpathlineto{\pgfqpoint{4.473885in}{1.589931in}}%
\pgfpathlineto{\pgfqpoint{4.479871in}{1.507638in}}%
\pgfpathlineto{\pgfqpoint{4.480727in}{1.647321in}}%
\pgfpathlineto{\pgfqpoint{4.484146in}{1.578238in}}%
\pgfpathlineto{\pgfqpoint{4.490131in}{1.659278in}}%
\pgfpathlineto{\pgfqpoint{4.490987in}{1.568703in}}%
\pgfpathlineto{\pgfqpoint{4.496974in}{1.593255in}}%
\pgfpathlineto{\pgfqpoint{4.498682in}{1.656651in}}%
\pgfpathlineto{\pgfqpoint{4.502953in}{1.676919in}}%
\pgfpathlineto{\pgfqpoint{4.507228in}{1.587889in}}%
\pgfpathlineto{\pgfqpoint{4.508938in}{1.665225in}}%
\pgfpathlineto{\pgfqpoint{4.513210in}{1.569459in}}%
\pgfpathlineto{\pgfqpoint{4.515776in}{1.620727in}}%
\pgfpathlineto{\pgfqpoint{4.519199in}{1.575322in}}%
\pgfpathlineto{\pgfqpoint{4.524328in}{1.689606in}}%
\pgfpathlineto{\pgfqpoint{4.525183in}{1.604923in}}%
\pgfpathlineto{\pgfqpoint{4.531176in}{1.561675in}}%
\pgfpathlineto{\pgfqpoint{4.532031in}{1.656651in}}%
\pgfpathlineto{\pgfqpoint{4.535450in}{1.727748in}}%
\pgfpathlineto{\pgfqpoint{4.538868in}{1.555230in}}%
\pgfpathlineto{\pgfqpoint{4.544856in}{1.632625in}}%
\pgfpathlineto{\pgfqpoint{4.548281in}{1.573427in}}%
\pgfpathlineto{\pgfqpoint{4.549136in}{1.632333in}}%
\pgfpathlineto{\pgfqpoint{4.552557in}{1.512420in}}%
\pgfpathlineto{\pgfqpoint{4.558543in}{1.642948in}}%
\pgfpathlineto{\pgfqpoint{4.559399in}{1.546193in}}%
\pgfpathlineto{\pgfqpoint{4.563676in}{1.516621in}}%
\pgfpathlineto{\pgfqpoint{4.567950in}{1.637641in}}%
\pgfpathlineto{\pgfqpoint{4.569658in}{1.533301in}}%
\pgfpathlineto{\pgfqpoint{4.573939in}{1.523328in}}%
\pgfpathlineto{\pgfqpoint{4.576507in}{1.633092in}}%
\pgfpathlineto{\pgfqpoint{4.580783in}{1.585588in}}%
\pgfpathlineto{\pgfqpoint{4.583350in}{1.638108in}}%
\pgfpathlineto{\pgfqpoint{4.589338in}{1.564272in}}%
\pgfpathlineto{\pgfqpoint{4.591044in}{1.639858in}}%
\pgfpathlineto{\pgfqpoint{4.595324in}{1.509855in}}%
\pgfpathlineto{\pgfqpoint{4.598746in}{1.658229in}}%
\pgfpathlineto{\pgfqpoint{4.602166in}{1.517089in}}%
\pgfpathlineto{\pgfqpoint{4.606443in}{1.599791in}}%
\pgfpathlineto{\pgfqpoint{4.609866in}{1.554007in}}%
\pgfpathlineto{\pgfqpoint{4.610723in}{1.591915in}}%
\pgfpathlineto{\pgfqpoint{4.615000in}{1.552140in}}%
\pgfpathlineto{\pgfqpoint{4.617565in}{1.636183in}}%
\pgfpathlineto{\pgfqpoint{4.622695in}{1.523328in}}%
\pgfpathlineto{\pgfqpoint{4.625260in}{1.588795in}}%
\pgfpathlineto{\pgfqpoint{4.630391in}{1.556863in}}%
\pgfpathlineto{\pgfqpoint{4.632102in}{1.600635in}}%
\pgfpathlineto{\pgfqpoint{4.634669in}{1.559227in}}%
\pgfpathlineto{\pgfqpoint{4.638945in}{1.560597in}}%
\pgfpathlineto{\pgfqpoint{4.644077in}{1.643475in}}%
\pgfpathlineto{\pgfqpoint{4.647497in}{1.530908in}}%
\pgfpathlineto{\pgfqpoint{4.648350in}{1.620493in}}%
\pgfpathlineto{\pgfqpoint{4.654343in}{1.609004in}}%
\pgfpathlineto{\pgfqpoint{4.656051in}{1.536333in}}%
\pgfpathlineto{\pgfqpoint{4.659473in}{1.669718in}}%
\pgfpathlineto{\pgfqpoint{4.664605in}{1.714626in}}%
\pgfpathlineto{\pgfqpoint{4.665461in}{1.615185in}}%
\pgfpathlineto{\pgfqpoint{4.668881in}{1.674091in}}%
\pgfpathlineto{\pgfqpoint{4.673161in}{1.622184in}}%
\pgfpathlineto{\pgfqpoint{4.677437in}{1.731306in}}%
\pgfpathlineto{\pgfqpoint{4.679148in}{1.637494in}}%
\pgfpathlineto{\pgfqpoint{4.682567in}{1.688031in}}%
\pgfpathlineto{\pgfqpoint{4.685989in}{1.614020in}}%
\pgfpathlineto{\pgfqpoint{4.690266in}{1.707805in}}%
\pgfpathlineto{\pgfqpoint{4.694542in}{1.749330in}}%
\pgfpathlineto{\pgfqpoint{4.697112in}{1.626268in}}%
\pgfpathlineto{\pgfqpoint{4.700535in}{1.590023in}}%
\pgfpathlineto{\pgfqpoint{4.704809in}{1.733523in}}%
\pgfpathlineto{\pgfqpoint{4.706522in}{1.748804in}}%
\pgfpathlineto{\pgfqpoint{4.709945in}{1.652512in}}%
\pgfpathlineto{\pgfqpoint{4.713368in}{1.804035in}}%
\pgfpathlineto{\pgfqpoint{4.716788in}{1.592119in}}%
\pgfpathlineto{\pgfqpoint{4.721066in}{1.647906in}}%
\pgfpathlineto{\pgfqpoint{4.723635in}{1.548114in}}%
\pgfpathlineto{\pgfqpoint{4.727062in}{1.597223in}}%
\pgfpathlineto{\pgfqpoint{4.731339in}{1.549776in}}%
\pgfpathlineto{\pgfqpoint{4.735622in}{1.625713in}}%
\pgfpathlineto{\pgfqpoint{4.737335in}{1.551322in}}%
\pgfpathlineto{\pgfqpoint{4.741616in}{1.659161in}}%
\pgfpathlineto{\pgfqpoint{4.745895in}{1.518546in}}%
\pgfpathlineto{\pgfqpoint{4.751882in}{1.772659in}}%
\pgfpathlineto{\pgfqpoint{4.754446in}{1.633268in}}%
\pgfpathlineto{\pgfqpoint{4.759573in}{1.523386in}}%
\pgfpathlineto{\pgfqpoint{4.761282in}{1.639273in}}%
\pgfpathlineto{\pgfqpoint{4.764703in}{1.715561in}}%
\pgfpathlineto{\pgfqpoint{4.769831in}{1.732066in}}%
\pgfpathlineto{\pgfqpoint{4.771544in}{1.605387in}}%
\pgfpathlineto{\pgfqpoint{4.774966in}{1.678584in}}%
\pgfpathlineto{\pgfqpoint{4.780961in}{1.572612in}}%
\pgfpathlineto{\pgfqpoint{4.784387in}{1.722210in}}%
\pgfpathlineto{\pgfqpoint{4.785242in}{1.523036in}}%
\pgfpathlineto{\pgfqpoint{4.789519in}{1.602122in}}%
\pgfpathlineto{\pgfqpoint{4.792084in}{1.499999in}}%
\pgfpathlineto{\pgfqpoint{4.795506in}{1.542225in}}%
\pgfpathlineto{\pgfqpoint{4.795506in}{1.542225in}}%
\pgfusepath{stroke}%
\end{pgfscope}%
\begin{pgfscope}%
\pgfsetrectcap%
\pgfsetmiterjoin%
\pgfsetlinewidth{0.803000pt}%
\definecolor{currentstroke}{rgb}{0.000000,0.000000,0.000000}%
\pgfsetstrokecolor{currentstroke}%
\pgfsetdash{}{0pt}%
\pgfpathmoveto{\pgfqpoint{0.484581in}{1.437021in}}%
\pgfpathlineto{\pgfqpoint{0.484581in}{2.011643in}}%
\pgfusepath{stroke}%
\end{pgfscope}%
\begin{pgfscope}%
\pgfsetrectcap%
\pgfsetmiterjoin%
\pgfsetlinewidth{0.803000pt}%
\definecolor{currentstroke}{rgb}{0.000000,0.000000,0.000000}%
\pgfsetstrokecolor{currentstroke}%
\pgfsetdash{}{0pt}%
\pgfpathmoveto{\pgfqpoint{5.000788in}{1.437021in}}%
\pgfpathlineto{\pgfqpoint{5.000788in}{2.011643in}}%
\pgfusepath{stroke}%
\end{pgfscope}%
\begin{pgfscope}%
\pgfsetrectcap%
\pgfsetmiterjoin%
\pgfsetlinewidth{0.803000pt}%
\definecolor{currentstroke}{rgb}{0.000000,0.000000,0.000000}%
\pgfsetstrokecolor{currentstroke}%
\pgfsetdash{}{0pt}%
\pgfpathmoveto{\pgfqpoint{0.484581in}{1.437021in}}%
\pgfpathlineto{\pgfqpoint{5.000788in}{1.437021in}}%
\pgfusepath{stroke}%
\end{pgfscope}%
\begin{pgfscope}%
\pgfsetrectcap%
\pgfsetmiterjoin%
\pgfsetlinewidth{0.803000pt}%
\definecolor{currentstroke}{rgb}{0.000000,0.000000,0.000000}%
\pgfsetstrokecolor{currentstroke}%
\pgfsetdash{}{0pt}%
\pgfpathmoveto{\pgfqpoint{0.484581in}{2.011643in}}%
\pgfpathlineto{\pgfqpoint{5.000788in}{2.011643in}}%
\pgfusepath{stroke}%
\end{pgfscope}%
\begin{pgfscope}%
\pgfsetbuttcap%
\pgfsetmiterjoin%
\definecolor{currentfill}{rgb}{1.000000,1.000000,1.000000}%
\pgfsetfillcolor{currentfill}%
\pgfsetlinewidth{0.000000pt}%
\definecolor{currentstroke}{rgb}{0.000000,0.000000,0.000000}%
\pgfsetstrokecolor{currentstroke}%
\pgfsetstrokeopacity{0.000000}%
\pgfsetdash{}{0pt}%
\pgfpathmoveto{\pgfqpoint{0.484581in}{0.539544in}}%
\pgfpathlineto{\pgfqpoint{5.000788in}{0.539544in}}%
\pgfpathlineto{\pgfqpoint{5.000788in}{1.114166in}}%
\pgfpathlineto{\pgfqpoint{0.484581in}{1.114166in}}%
\pgfpathlineto{\pgfqpoint{0.484581in}{0.539544in}}%
\pgfpathclose%
\pgfusepath{fill}%
\end{pgfscope}%
\begin{pgfscope}%
\pgfsetbuttcap%
\pgfsetroundjoin%
\definecolor{currentfill}{rgb}{0.000000,0.000000,0.000000}%
\pgfsetfillcolor{currentfill}%
\pgfsetlinewidth{0.803000pt}%
\definecolor{currentstroke}{rgb}{0.000000,0.000000,0.000000}%
\pgfsetstrokecolor{currentstroke}%
\pgfsetdash{}{0pt}%
\pgfsys@defobject{currentmarker}{\pgfqpoint{0.000000in}{-0.048611in}}{\pgfqpoint{0.000000in}{0.000000in}}{%
\pgfpathmoveto{\pgfqpoint{0.000000in}{0.000000in}}%
\pgfpathlineto{\pgfqpoint{0.000000in}{-0.048611in}}%
\pgfusepath{stroke,fill}%
}%
\begin{pgfscope}%
\pgfsys@transformshift{0.689546in}{0.539544in}%
\pgfsys@useobject{currentmarker}{}%
\end{pgfscope}%
\end{pgfscope}%
\begin{pgfscope}%
\definecolor{textcolor}{rgb}{0.000000,0.000000,0.000000}%
\pgfsetstrokecolor{textcolor}%
\pgfsetfillcolor{textcolor}%
\pgftext[x=0.689546in,y=0.442322in,,top]{\color{textcolor}\rmfamily\fontsize{8.000000}{9.600000}\selectfont \(\displaystyle {06{:}00}\)}%
\end{pgfscope}%
\begin{pgfscope}%
\pgfsetbuttcap%
\pgfsetroundjoin%
\definecolor{currentfill}{rgb}{0.000000,0.000000,0.000000}%
\pgfsetfillcolor{currentfill}%
\pgfsetlinewidth{0.803000pt}%
\definecolor{currentstroke}{rgb}{0.000000,0.000000,0.000000}%
\pgfsetstrokecolor{currentstroke}%
\pgfsetdash{}{0pt}%
\pgfsys@defobject{currentmarker}{\pgfqpoint{0.000000in}{-0.048611in}}{\pgfqpoint{0.000000in}{0.000000in}}{%
\pgfpathmoveto{\pgfqpoint{0.000000in}{0.000000in}}%
\pgfpathlineto{\pgfqpoint{0.000000in}{-0.048611in}}%
\pgfusepath{stroke,fill}%
}%
\begin{pgfscope}%
\pgfsys@transformshift{1.202878in}{0.539544in}%
\pgfsys@useobject{currentmarker}{}%
\end{pgfscope}%
\end{pgfscope}%
\begin{pgfscope}%
\definecolor{textcolor}{rgb}{0.000000,0.000000,0.000000}%
\pgfsetstrokecolor{textcolor}%
\pgfsetfillcolor{textcolor}%
\pgftext[x=1.202878in,y=0.442322in,,top]{\color{textcolor}\rmfamily\fontsize{8.000000}{9.600000}\selectfont \(\displaystyle {09{:}00}\)}%
\end{pgfscope}%
\begin{pgfscope}%
\pgfsetbuttcap%
\pgfsetroundjoin%
\definecolor{currentfill}{rgb}{0.000000,0.000000,0.000000}%
\pgfsetfillcolor{currentfill}%
\pgfsetlinewidth{0.803000pt}%
\definecolor{currentstroke}{rgb}{0.000000,0.000000,0.000000}%
\pgfsetstrokecolor{currentstroke}%
\pgfsetdash{}{0pt}%
\pgfsys@defobject{currentmarker}{\pgfqpoint{0.000000in}{-0.048611in}}{\pgfqpoint{0.000000in}{0.000000in}}{%
\pgfpathmoveto{\pgfqpoint{0.000000in}{0.000000in}}%
\pgfpathlineto{\pgfqpoint{0.000000in}{-0.048611in}}%
\pgfusepath{stroke,fill}%
}%
\begin{pgfscope}%
\pgfsys@transformshift{1.716211in}{0.539544in}%
\pgfsys@useobject{currentmarker}{}%
\end{pgfscope}%
\end{pgfscope}%
\begin{pgfscope}%
\definecolor{textcolor}{rgb}{0.000000,0.000000,0.000000}%
\pgfsetstrokecolor{textcolor}%
\pgfsetfillcolor{textcolor}%
\pgftext[x=1.716211in,y=0.442322in,,top]{\color{textcolor}\rmfamily\fontsize{8.000000}{9.600000}\selectfont \(\displaystyle {12{:}00}\)}%
\end{pgfscope}%
\begin{pgfscope}%
\pgfsetbuttcap%
\pgfsetroundjoin%
\definecolor{currentfill}{rgb}{0.000000,0.000000,0.000000}%
\pgfsetfillcolor{currentfill}%
\pgfsetlinewidth{0.803000pt}%
\definecolor{currentstroke}{rgb}{0.000000,0.000000,0.000000}%
\pgfsetstrokecolor{currentstroke}%
\pgfsetdash{}{0pt}%
\pgfsys@defobject{currentmarker}{\pgfqpoint{0.000000in}{-0.048611in}}{\pgfqpoint{0.000000in}{0.000000in}}{%
\pgfpathmoveto{\pgfqpoint{0.000000in}{0.000000in}}%
\pgfpathlineto{\pgfqpoint{0.000000in}{-0.048611in}}%
\pgfusepath{stroke,fill}%
}%
\begin{pgfscope}%
\pgfsys@transformshift{2.229543in}{0.539544in}%
\pgfsys@useobject{currentmarker}{}%
\end{pgfscope}%
\end{pgfscope}%
\begin{pgfscope}%
\definecolor{textcolor}{rgb}{0.000000,0.000000,0.000000}%
\pgfsetstrokecolor{textcolor}%
\pgfsetfillcolor{textcolor}%
\pgftext[x=2.229543in,y=0.442322in,,top]{\color{textcolor}\rmfamily\fontsize{8.000000}{9.600000}\selectfont \(\displaystyle {15{:}00}\)}%
\end{pgfscope}%
\begin{pgfscope}%
\pgfsetbuttcap%
\pgfsetroundjoin%
\definecolor{currentfill}{rgb}{0.000000,0.000000,0.000000}%
\pgfsetfillcolor{currentfill}%
\pgfsetlinewidth{0.803000pt}%
\definecolor{currentstroke}{rgb}{0.000000,0.000000,0.000000}%
\pgfsetstrokecolor{currentstroke}%
\pgfsetdash{}{0pt}%
\pgfsys@defobject{currentmarker}{\pgfqpoint{0.000000in}{-0.048611in}}{\pgfqpoint{0.000000in}{0.000000in}}{%
\pgfpathmoveto{\pgfqpoint{0.000000in}{0.000000in}}%
\pgfpathlineto{\pgfqpoint{0.000000in}{-0.048611in}}%
\pgfusepath{stroke,fill}%
}%
\begin{pgfscope}%
\pgfsys@transformshift{2.742876in}{0.539544in}%
\pgfsys@useobject{currentmarker}{}%
\end{pgfscope}%
\end{pgfscope}%
\begin{pgfscope}%
\definecolor{textcolor}{rgb}{0.000000,0.000000,0.000000}%
\pgfsetstrokecolor{textcolor}%
\pgfsetfillcolor{textcolor}%
\pgftext[x=2.742876in,y=0.442322in,,top]{\color{textcolor}\rmfamily\fontsize{8.000000}{9.600000}\selectfont \(\displaystyle {18{:}00}\)}%
\end{pgfscope}%
\begin{pgfscope}%
\pgfsetbuttcap%
\pgfsetroundjoin%
\definecolor{currentfill}{rgb}{0.000000,0.000000,0.000000}%
\pgfsetfillcolor{currentfill}%
\pgfsetlinewidth{0.803000pt}%
\definecolor{currentstroke}{rgb}{0.000000,0.000000,0.000000}%
\pgfsetstrokecolor{currentstroke}%
\pgfsetdash{}{0pt}%
\pgfsys@defobject{currentmarker}{\pgfqpoint{0.000000in}{-0.048611in}}{\pgfqpoint{0.000000in}{0.000000in}}{%
\pgfpathmoveto{\pgfqpoint{0.000000in}{0.000000in}}%
\pgfpathlineto{\pgfqpoint{0.000000in}{-0.048611in}}%
\pgfusepath{stroke,fill}%
}%
\begin{pgfscope}%
\pgfsys@transformshift{3.256208in}{0.539544in}%
\pgfsys@useobject{currentmarker}{}%
\end{pgfscope}%
\end{pgfscope}%
\begin{pgfscope}%
\definecolor{textcolor}{rgb}{0.000000,0.000000,0.000000}%
\pgfsetstrokecolor{textcolor}%
\pgfsetfillcolor{textcolor}%
\pgftext[x=3.256208in,y=0.442322in,,top]{\color{textcolor}\rmfamily\fontsize{8.000000}{9.600000}\selectfont \(\displaystyle {21{:}00}\)}%
\end{pgfscope}%
\begin{pgfscope}%
\pgfsetbuttcap%
\pgfsetroundjoin%
\definecolor{currentfill}{rgb}{0.000000,0.000000,0.000000}%
\pgfsetfillcolor{currentfill}%
\pgfsetlinewidth{0.803000pt}%
\definecolor{currentstroke}{rgb}{0.000000,0.000000,0.000000}%
\pgfsetstrokecolor{currentstroke}%
\pgfsetdash{}{0pt}%
\pgfsys@defobject{currentmarker}{\pgfqpoint{0.000000in}{-0.048611in}}{\pgfqpoint{0.000000in}{0.000000in}}{%
\pgfpathmoveto{\pgfqpoint{0.000000in}{0.000000in}}%
\pgfpathlineto{\pgfqpoint{0.000000in}{-0.048611in}}%
\pgfusepath{stroke,fill}%
}%
\begin{pgfscope}%
\pgfsys@transformshift{3.769541in}{0.539544in}%
\pgfsys@useobject{currentmarker}{}%
\end{pgfscope}%
\end{pgfscope}%
\begin{pgfscope}%
\definecolor{textcolor}{rgb}{0.000000,0.000000,0.000000}%
\pgfsetstrokecolor{textcolor}%
\pgfsetfillcolor{textcolor}%
\pgftext[x=3.769541in,y=0.442322in,,top]{\color{textcolor}\rmfamily\fontsize{8.000000}{9.600000}\selectfont \(\displaystyle {00{:}00}\)}%
\end{pgfscope}%
\begin{pgfscope}%
\pgfsetbuttcap%
\pgfsetroundjoin%
\definecolor{currentfill}{rgb}{0.000000,0.000000,0.000000}%
\pgfsetfillcolor{currentfill}%
\pgfsetlinewidth{0.803000pt}%
\definecolor{currentstroke}{rgb}{0.000000,0.000000,0.000000}%
\pgfsetstrokecolor{currentstroke}%
\pgfsetdash{}{0pt}%
\pgfsys@defobject{currentmarker}{\pgfqpoint{0.000000in}{-0.048611in}}{\pgfqpoint{0.000000in}{0.000000in}}{%
\pgfpathmoveto{\pgfqpoint{0.000000in}{0.000000in}}%
\pgfpathlineto{\pgfqpoint{0.000000in}{-0.048611in}}%
\pgfusepath{stroke,fill}%
}%
\begin{pgfscope}%
\pgfsys@transformshift{4.282873in}{0.539544in}%
\pgfsys@useobject{currentmarker}{}%
\end{pgfscope}%
\end{pgfscope}%
\begin{pgfscope}%
\definecolor{textcolor}{rgb}{0.000000,0.000000,0.000000}%
\pgfsetstrokecolor{textcolor}%
\pgfsetfillcolor{textcolor}%
\pgftext[x=4.282873in,y=0.442322in,,top]{\color{textcolor}\rmfamily\fontsize{8.000000}{9.600000}\selectfont \(\displaystyle {03{:}00}\)}%
\end{pgfscope}%
\begin{pgfscope}%
\pgfsetbuttcap%
\pgfsetroundjoin%
\definecolor{currentfill}{rgb}{0.000000,0.000000,0.000000}%
\pgfsetfillcolor{currentfill}%
\pgfsetlinewidth{0.803000pt}%
\definecolor{currentstroke}{rgb}{0.000000,0.000000,0.000000}%
\pgfsetstrokecolor{currentstroke}%
\pgfsetdash{}{0pt}%
\pgfsys@defobject{currentmarker}{\pgfqpoint{0.000000in}{-0.048611in}}{\pgfqpoint{0.000000in}{0.000000in}}{%
\pgfpathmoveto{\pgfqpoint{0.000000in}{0.000000in}}%
\pgfpathlineto{\pgfqpoint{0.000000in}{-0.048611in}}%
\pgfusepath{stroke,fill}%
}%
\begin{pgfscope}%
\pgfsys@transformshift{4.796206in}{0.539544in}%
\pgfsys@useobject{currentmarker}{}%
\end{pgfscope}%
\end{pgfscope}%
\begin{pgfscope}%
\definecolor{textcolor}{rgb}{0.000000,0.000000,0.000000}%
\pgfsetstrokecolor{textcolor}%
\pgfsetfillcolor{textcolor}%
\pgftext[x=4.796206in,y=0.442322in,,top]{\color{textcolor}\rmfamily\fontsize{8.000000}{9.600000}\selectfont \(\displaystyle {06{:}00}\)}%
\end{pgfscope}%
\begin{pgfscope}%
\definecolor{textcolor}{rgb}{0.000000,0.000000,0.000000}%
\pgfsetstrokecolor{textcolor}%
\pgfsetfillcolor{textcolor}%
\pgftext[x=2.742685in,y=0.288100in,,top]{\color{textcolor}\rmfamily\fontsize{10.000000}{12.000000}\selectfont Time (UTC)}%
\end{pgfscope}%
\begin{pgfscope}%
\pgfsetbuttcap%
\pgfsetroundjoin%
\definecolor{currentfill}{rgb}{0.000000,0.000000,0.000000}%
\pgfsetfillcolor{currentfill}%
\pgfsetlinewidth{0.803000pt}%
\definecolor{currentstroke}{rgb}{0.000000,0.000000,0.000000}%
\pgfsetstrokecolor{currentstroke}%
\pgfsetdash{}{0pt}%
\pgfsys@defobject{currentmarker}{\pgfqpoint{-0.048611in}{0.000000in}}{\pgfqpoint{-0.000000in}{0.000000in}}{%
\pgfpathmoveto{\pgfqpoint{-0.000000in}{0.000000in}}%
\pgfpathlineto{\pgfqpoint{-0.048611in}{0.000000in}}%
\pgfusepath{stroke,fill}%
}%
\begin{pgfscope}%
\pgfsys@transformshift{0.484581in}{0.717544in}%
\pgfsys@useobject{currentmarker}{}%
\end{pgfscope}%
\end{pgfscope}%
\begin{pgfscope}%
\definecolor{textcolor}{rgb}{0.000000,0.000000,0.000000}%
\pgfsetstrokecolor{textcolor}%
\pgfsetfillcolor{textcolor}%
\pgftext[x=0.328331in, y=0.678988in, left, base]{\color{textcolor}\rmfamily\fontsize{8.000000}{9.600000}\selectfont \(\displaystyle {0}\)}%
\end{pgfscope}%
\begin{pgfscope}%
\pgfsetbuttcap%
\pgfsetroundjoin%
\definecolor{currentfill}{rgb}{0.000000,0.000000,0.000000}%
\pgfsetfillcolor{currentfill}%
\pgfsetlinewidth{0.803000pt}%
\definecolor{currentstroke}{rgb}{0.000000,0.000000,0.000000}%
\pgfsetstrokecolor{currentstroke}%
\pgfsetdash{}{0pt}%
\pgfsys@defobject{currentmarker}{\pgfqpoint{-0.048611in}{0.000000in}}{\pgfqpoint{-0.000000in}{0.000000in}}{%
\pgfpathmoveto{\pgfqpoint{-0.000000in}{0.000000in}}%
\pgfpathlineto{\pgfqpoint{-0.048611in}{0.000000in}}%
\pgfusepath{stroke,fill}%
}%
\begin{pgfscope}%
\pgfsys@transformshift{0.484581in}{0.921085in}%
\pgfsys@useobject{currentmarker}{}%
\end{pgfscope}%
\end{pgfscope}%
\begin{pgfscope}%
\definecolor{textcolor}{rgb}{0.000000,0.000000,0.000000}%
\pgfsetstrokecolor{textcolor}%
\pgfsetfillcolor{textcolor}%
\pgftext[x=0.328331in, y=0.882529in, left, base]{\color{textcolor}\rmfamily\fontsize{8.000000}{9.600000}\selectfont \(\displaystyle {5}\)}%
\end{pgfscope}%
\begin{pgfscope}%
\definecolor{textcolor}{rgb}{0.000000,0.000000,0.000000}%
\pgfsetstrokecolor{textcolor}%
\pgfsetfillcolor{textcolor}%
\pgftext[x=0.484581in,y=1.155833in,left,base]{\color{textcolor}\rmfamily\fontsize{8.000000}{9.600000}\selectfont \(\displaystyle \times{10^{\ensuremath{-}6}}{}\)}%
\end{pgfscope}%
\begin{pgfscope}%
\pgfpathrectangle{\pgfqpoint{0.484581in}{0.539544in}}{\pgfqpoint{4.516206in}{0.574622in}}%
\pgfusepath{clip}%
\pgfsetrectcap%
\pgfsetroundjoin%
\pgfsetlinewidth{0.501875pt}%
\definecolor{currentstroke}{rgb}{0.003922,0.450980,0.698039}%
\pgfsetstrokecolor{currentstroke}%
\pgfsetstrokeopacity{0.700000}%
\pgfsetdash{}{0pt}%
\pgfpathmoveto{\pgfqpoint{0.689863in}{0.721589in}}%
\pgfpathlineto{\pgfqpoint{0.694140in}{0.685169in}}%
\pgfpathlineto{\pgfqpoint{0.698419in}{0.771788in}}%
\pgfpathlineto{\pgfqpoint{0.700132in}{0.644899in}}%
\pgfpathlineto{\pgfqpoint{0.703553in}{0.767073in}}%
\pgfpathlineto{\pgfqpoint{0.707833in}{0.699777in}}%
\pgfpathlineto{\pgfqpoint{0.713821in}{0.764305in}}%
\pgfpathlineto{\pgfqpoint{0.718102in}{0.651844in}}%
\pgfpathlineto{\pgfqpoint{0.721524in}{0.733357in}}%
\pgfpathlineto{\pgfqpoint{0.726652in}{0.645940in}}%
\pgfpathlineto{\pgfqpoint{0.729218in}{0.759733in}}%
\pgfpathlineto{\pgfqpoint{0.736917in}{0.775857in}}%
\pgfpathlineto{\pgfqpoint{0.740340in}{0.660732in}}%
\pgfpathlineto{\pgfqpoint{0.742052in}{0.726624in}}%
\pgfpathlineto{\pgfqpoint{0.746327in}{0.778195in}}%
\pgfpathlineto{\pgfqpoint{0.751457in}{0.661163in}}%
\pgfpathlineto{\pgfqpoint{0.755738in}{0.790322in}}%
\pgfpathlineto{\pgfqpoint{0.759157in}{0.660947in}}%
\pgfpathlineto{\pgfqpoint{0.765999in}{0.761461in}}%
\pgfpathlineto{\pgfqpoint{0.767711in}{0.680382in}}%
\pgfpathlineto{\pgfqpoint{0.773703in}{0.750343in}}%
\pgfpathlineto{\pgfqpoint{0.776269in}{0.679736in}}%
\pgfpathlineto{\pgfqpoint{0.783967in}{0.745340in}}%
\pgfpathlineto{\pgfqpoint{0.784824in}{0.605387in}}%
\pgfpathlineto{\pgfqpoint{0.789097in}{0.784242in}}%
\pgfpathlineto{\pgfqpoint{0.793373in}{0.817025in}}%
\pgfpathlineto{\pgfqpoint{0.797648in}{0.692652in}}%
\pgfpathlineto{\pgfqpoint{0.805343in}{0.725690in}}%
\pgfpathlineto{\pgfqpoint{0.807054in}{0.608730in}}%
\pgfpathlineto{\pgfqpoint{0.811330in}{0.738647in}}%
\pgfpathlineto{\pgfqpoint{0.816461in}{0.682759in}}%
\pgfpathlineto{\pgfqpoint{0.819025in}{0.765063in}}%
\pgfpathlineto{\pgfqpoint{0.823303in}{0.717449in}}%
\pgfpathlineto{\pgfqpoint{0.827584in}{0.851108in}}%
\pgfpathlineto{\pgfqpoint{0.834428in}{0.692078in}}%
\pgfpathlineto{\pgfqpoint{0.838709in}{0.729073in}}%
\pgfpathlineto{\pgfqpoint{0.840420in}{0.676353in}}%
\pgfpathlineto{\pgfqpoint{0.848114in}{0.744295in}}%
\pgfpathlineto{\pgfqpoint{0.848968in}{0.708705in}}%
\pgfpathlineto{\pgfqpoint{0.855805in}{0.669587in}}%
\pgfpathlineto{\pgfqpoint{0.859233in}{0.748939in}}%
\pgfpathlineto{\pgfqpoint{0.861803in}{0.677757in}}%
\pgfpathlineto{\pgfqpoint{0.867793in}{0.662535in}}%
\pgfpathlineto{\pgfqpoint{0.870360in}{0.764416in}}%
\pgfpathlineto{\pgfqpoint{0.875490in}{0.657676in}}%
\pgfpathlineto{\pgfqpoint{0.879771in}{0.744551in}}%
\pgfpathlineto{\pgfqpoint{0.883195in}{0.653176in}}%
\pgfpathlineto{\pgfqpoint{0.887469in}{0.783524in}}%
\pgfpathlineto{\pgfqpoint{0.893457in}{0.703128in}}%
\pgfpathlineto{\pgfqpoint{0.896021in}{0.741311in}}%
\pgfpathlineto{\pgfqpoint{0.903720in}{0.770711in}}%
\pgfpathlineto{\pgfqpoint{0.907998in}{0.652889in}}%
\pgfpathlineto{\pgfqpoint{0.910566in}{0.728211in}}%
\pgfpathlineto{\pgfqpoint{0.914841in}{0.649793in}}%
\pgfpathlineto{\pgfqpoint{0.918259in}{0.754914in}}%
\pgfpathlineto{\pgfqpoint{0.922534in}{0.779168in}}%
\pgfpathlineto{\pgfqpoint{0.925951in}{0.659475in}}%
\pgfpathlineto{\pgfqpoint{0.932791in}{0.780141in}}%
\pgfpathlineto{\pgfqpoint{0.935359in}{0.665881in}}%
\pgfpathlineto{\pgfqpoint{0.938782in}{0.766467in}}%
\pgfpathlineto{\pgfqpoint{0.947330in}{0.636984in}}%
\pgfpathlineto{\pgfqpoint{0.951605in}{0.759015in}}%
\pgfpathlineto{\pgfqpoint{0.959305in}{0.764089in}}%
\pgfpathlineto{\pgfqpoint{0.963579in}{0.653679in}}%
\pgfpathlineto{\pgfqpoint{0.966147in}{0.724035in}}%
\pgfpathlineto{\pgfqpoint{0.969570in}{0.683477in}}%
\pgfpathlineto{\pgfqpoint{0.976408in}{0.806015in}}%
\pgfpathlineto{\pgfqpoint{0.978974in}{0.720329in}}%
\pgfpathlineto{\pgfqpoint{0.984102in}{0.758728in}}%
\pgfpathlineto{\pgfqpoint{0.985812in}{0.662463in}}%
\pgfpathlineto{\pgfqpoint{0.991797in}{0.733321in}}%
\pgfpathlineto{\pgfqpoint{0.997785in}{0.750271in}}%
\pgfpathlineto{\pgfqpoint{0.999494in}{0.780716in}}%
\pgfpathlineto{\pgfqpoint{1.003769in}{0.688448in}}%
\pgfpathlineto{\pgfqpoint{1.010613in}{0.744295in}}%
\pgfpathlineto{\pgfqpoint{1.014889in}{0.673186in}}%
\pgfpathlineto{\pgfqpoint{1.018310in}{0.746960in}}%
\pgfpathlineto{\pgfqpoint{1.020874in}{0.673545in}}%
\pgfpathlineto{\pgfqpoint{1.026001in}{0.741846in}}%
\pgfpathlineto{\pgfqpoint{1.029421in}{0.668291in}}%
\pgfpathlineto{\pgfqpoint{1.038832in}{0.800079in}}%
\pgfpathlineto{\pgfqpoint{1.041400in}{0.683154in}}%
\pgfpathlineto{\pgfqpoint{1.046538in}{0.642669in}}%
\pgfpathlineto{\pgfqpoint{1.050817in}{0.768733in}}%
\pgfpathlineto{\pgfqpoint{1.054238in}{0.679807in}}%
\pgfpathlineto{\pgfqpoint{1.059370in}{0.769020in}}%
\pgfpathlineto{\pgfqpoint{1.062792in}{0.636047in}}%
\pgfpathlineto{\pgfqpoint{1.068774in}{0.782192in}}%
\pgfpathlineto{\pgfqpoint{1.071339in}{0.679233in}}%
\pgfpathlineto{\pgfqpoint{1.077323in}{0.749338in}}%
\pgfpathlineto{\pgfqpoint{1.079890in}{0.663364in}}%
\pgfpathlineto{\pgfqpoint{1.084169in}{0.749338in}}%
\pgfpathlineto{\pgfqpoint{1.091012in}{0.695177in}}%
\pgfpathlineto{\pgfqpoint{1.095289in}{0.766144in}}%
\pgfpathlineto{\pgfqpoint{1.096997in}{0.686070in}}%
\pgfpathlineto{\pgfqpoint{1.102978in}{0.793385in}}%
\pgfpathlineto{\pgfqpoint{1.105542in}{0.697336in}}%
\pgfpathlineto{\pgfqpoint{1.112379in}{0.642669in}}%
\pgfpathlineto{\pgfqpoint{1.118364in}{0.769235in}}%
\pgfpathlineto{\pgfqpoint{1.124349in}{0.656671in}}%
\pgfpathlineto{\pgfqpoint{1.129483in}{0.728139in}}%
\pgfpathlineto{\pgfqpoint{1.131195in}{0.682616in}}%
\pgfpathlineto{\pgfqpoint{1.137183in}{0.656056in}}%
\pgfpathlineto{\pgfqpoint{1.139749in}{0.764632in}}%
\pgfpathlineto{\pgfqpoint{1.147448in}{0.689740in}}%
\pgfpathlineto{\pgfqpoint{1.150013in}{0.769235in}}%
\pgfpathlineto{\pgfqpoint{1.152579in}{0.575628in}}%
\pgfpathlineto{\pgfqpoint{1.156855in}{0.776400in}}%
\pgfpathlineto{\pgfqpoint{1.162841in}{0.681427in}}%
\pgfpathlineto{\pgfqpoint{1.167121in}{0.726592in}}%
\pgfpathlineto{\pgfqpoint{1.169688in}{0.667896in}}%
\pgfpathlineto{\pgfqpoint{1.176534in}{0.787481in}}%
\pgfpathlineto{\pgfqpoint{1.178245in}{0.676784in}}%
\pgfpathlineto{\pgfqpoint{1.185091in}{0.833942in}}%
\pgfpathlineto{\pgfqpoint{1.188512in}{0.707229in}}%
\pgfpathlineto{\pgfqpoint{1.191934in}{0.760671in}}%
\pgfpathlineto{\pgfqpoint{1.196209in}{0.672643in}}%
\pgfpathlineto{\pgfqpoint{1.200488in}{0.747535in}}%
\pgfpathlineto{\pgfqpoint{1.203910in}{0.679915in}}%
\pgfpathlineto{\pgfqpoint{1.209902in}{0.814184in}}%
\pgfpathlineto{\pgfqpoint{1.212472in}{0.687834in}}%
\pgfpathlineto{\pgfqpoint{1.217612in}{0.765888in}}%
\pgfpathlineto{\pgfqpoint{1.222746in}{0.685097in}}%
\pgfpathlineto{\pgfqpoint{1.227024in}{0.668147in}}%
\pgfpathlineto{\pgfqpoint{1.231301in}{0.763443in}}%
\pgfpathlineto{\pgfqpoint{1.233868in}{0.676640in}}%
\pgfpathlineto{\pgfqpoint{1.238147in}{0.745915in}}%
\pgfpathlineto{\pgfqpoint{1.242423in}{0.693953in}}%
\pgfpathlineto{\pgfqpoint{1.246701in}{0.774309in}}%
\pgfpathlineto{\pgfqpoint{1.250981in}{0.670561in}}%
\pgfpathlineto{\pgfqpoint{1.258680in}{0.795292in}}%
\pgfpathlineto{\pgfqpoint{1.261242in}{0.712773in}}%
\pgfpathlineto{\pgfqpoint{1.266376in}{0.647456in}}%
\pgfpathlineto{\pgfqpoint{1.269800in}{0.796696in}}%
\pgfpathlineto{\pgfqpoint{1.273224in}{0.704460in}}%
\pgfpathlineto{\pgfqpoint{1.276647in}{0.782986in}}%
\pgfpathlineto{\pgfqpoint{1.281778in}{0.704460in}}%
\pgfpathlineto{\pgfqpoint{1.287770in}{0.812493in}}%
\pgfpathlineto{\pgfqpoint{1.291191in}{0.623130in}}%
\pgfpathlineto{\pgfqpoint{1.297179in}{0.760563in}}%
\pgfpathlineto{\pgfqpoint{1.298889in}{0.654652in}}%
\pgfpathlineto{\pgfqpoint{1.304883in}{0.781366in}}%
\pgfpathlineto{\pgfqpoint{1.310013in}{0.697838in}}%
\pgfpathlineto{\pgfqpoint{1.311723in}{0.737426in}}%
\pgfpathlineto{\pgfqpoint{1.317712in}{0.680924in}}%
\pgfpathlineto{\pgfqpoint{1.321985in}{0.766323in}}%
\pgfpathlineto{\pgfqpoint{1.326257in}{0.674123in}}%
\pgfpathlineto{\pgfqpoint{1.328822in}{0.752648in}}%
\pgfpathlineto{\pgfqpoint{1.335662in}{0.669623in}}%
\pgfpathlineto{\pgfqpoint{1.337375in}{0.759159in}}%
\pgfpathlineto{\pgfqpoint{1.343354in}{0.667609in}}%
\pgfpathlineto{\pgfqpoint{1.345921in}{0.778737in}}%
\pgfpathlineto{\pgfqpoint{1.351911in}{0.654616in}}%
\pgfpathlineto{\pgfqpoint{1.356186in}{0.791909in}}%
\pgfpathlineto{\pgfqpoint{1.359610in}{0.666205in}}%
\pgfpathlineto{\pgfqpoint{1.363031in}{0.745843in}}%
\pgfpathlineto{\pgfqpoint{1.369869in}{0.625827in}}%
\pgfpathlineto{\pgfqpoint{1.372432in}{0.738790in}}%
\pgfpathlineto{\pgfqpoint{1.378419in}{0.783740in}}%
\pgfpathlineto{\pgfqpoint{1.380129in}{0.701724in}}%
\pgfpathlineto{\pgfqpoint{1.386969in}{0.669013in}}%
\pgfpathlineto{\pgfqpoint{1.387825in}{0.714537in}}%
\pgfpathlineto{\pgfqpoint{1.392960in}{0.773950in}}%
\pgfpathlineto{\pgfqpoint{1.397242in}{0.715183in}}%
\pgfpathlineto{\pgfqpoint{1.400665in}{0.797127in}}%
\pgfpathlineto{\pgfqpoint{1.406651in}{0.699745in}}%
\pgfpathlineto{\pgfqpoint{1.410071in}{0.745412in}}%
\pgfpathlineto{\pgfqpoint{1.413491in}{0.618846in}}%
\pgfpathlineto{\pgfqpoint{1.417764in}{0.744658in}}%
\pgfpathlineto{\pgfqpoint{1.424607in}{0.684235in}}%
\pgfpathlineto{\pgfqpoint{1.426320in}{0.741096in}}%
\pgfpathlineto{\pgfqpoint{1.434016in}{0.688228in}}%
\pgfpathlineto{\pgfqpoint{1.437437in}{0.756135in}}%
\pgfpathlineto{\pgfqpoint{1.439148in}{0.658358in}}%
\pgfpathlineto{\pgfqpoint{1.444278in}{0.799249in}}%
\pgfpathlineto{\pgfqpoint{1.448556in}{0.638780in}}%
\pgfpathlineto{\pgfqpoint{1.453691in}{0.747104in}}%
\pgfpathlineto{\pgfqpoint{1.456259in}{0.662965in}}%
\pgfpathlineto{\pgfqpoint{1.462250in}{0.608837in}}%
\pgfpathlineto{\pgfqpoint{1.467386in}{0.743071in}}%
\pgfpathlineto{\pgfqpoint{1.471667in}{0.742855in}}%
\pgfpathlineto{\pgfqpoint{1.476802in}{0.634679in}}%
\pgfpathlineto{\pgfqpoint{1.479368in}{0.746888in}}%
\pgfpathlineto{\pgfqpoint{1.481936in}{0.681786in}}%
\pgfpathlineto{\pgfqpoint{1.487072in}{0.775857in}}%
\pgfpathlineto{\pgfqpoint{1.491345in}{0.636478in}}%
\pgfpathlineto{\pgfqpoint{1.496476in}{0.761568in}}%
\pgfpathlineto{\pgfqpoint{1.500749in}{0.783843in}}%
\pgfpathlineto{\pgfqpoint{1.504171in}{0.652849in}}%
\pgfpathlineto{\pgfqpoint{1.508448in}{0.717808in}}%
\pgfpathlineto{\pgfqpoint{1.511868in}{0.688480in}}%
\pgfpathlineto{\pgfqpoint{1.518707in}{0.768481in}}%
\pgfpathlineto{\pgfqpoint{1.522982in}{0.768230in}}%
\pgfpathlineto{\pgfqpoint{1.527256in}{0.649434in}}%
\pgfpathlineto{\pgfqpoint{1.530678in}{0.746457in}}%
\pgfpathlineto{\pgfqpoint{1.535810in}{0.684235in}}%
\pgfpathlineto{\pgfqpoint{1.538376in}{0.748795in}}%
\pgfpathlineto{\pgfqpoint{1.541797in}{0.681571in}}%
\pgfpathlineto{\pgfqpoint{1.548638in}{0.743218in}}%
\pgfpathlineto{\pgfqpoint{1.550351in}{0.659511in}}%
\pgfpathlineto{\pgfqpoint{1.556343in}{0.605423in}}%
\pgfpathlineto{\pgfqpoint{1.558909in}{0.713998in}}%
\pgfpathlineto{\pgfqpoint{1.565750in}{0.753582in}}%
\pgfpathlineto{\pgfqpoint{1.567459in}{0.664513in}}%
\pgfpathlineto{\pgfqpoint{1.573455in}{0.624064in}}%
\pgfpathlineto{\pgfqpoint{1.576880in}{0.770352in}}%
\pgfpathlineto{\pgfqpoint{1.581159in}{0.662678in}}%
\pgfpathlineto{\pgfqpoint{1.585439in}{0.612583in}}%
\pgfpathlineto{\pgfqpoint{1.588860in}{0.716228in}}%
\pgfpathlineto{\pgfqpoint{1.599126in}{0.650874in}}%
\pgfpathlineto{\pgfqpoint{1.602549in}{0.755672in}}%
\pgfpathlineto{\pgfqpoint{1.607680in}{0.653431in}}%
\pgfpathlineto{\pgfqpoint{1.611102in}{0.789572in}}%
\pgfpathlineto{\pgfqpoint{1.615378in}{0.687834in}}%
\pgfpathlineto{\pgfqpoint{1.619653in}{0.775642in}}%
\pgfpathlineto{\pgfqpoint{1.623070in}{0.699530in}}%
\pgfpathlineto{\pgfqpoint{1.629917in}{0.644073in}}%
\pgfpathlineto{\pgfqpoint{1.634197in}{0.776974in}}%
\pgfpathlineto{\pgfqpoint{1.637617in}{0.680238in}}%
\pgfpathlineto{\pgfqpoint{1.642745in}{0.787481in}}%
\pgfpathlineto{\pgfqpoint{1.647874in}{0.682688in}}%
\pgfpathlineto{\pgfqpoint{1.648727in}{0.761177in}}%
\pgfpathlineto{\pgfqpoint{1.656428in}{0.684092in}}%
\pgfpathlineto{\pgfqpoint{1.660704in}{0.730262in}}%
\pgfpathlineto{\pgfqpoint{1.661558in}{0.661059in}}%
\pgfpathlineto{\pgfqpoint{1.669255in}{0.699171in}}%
\pgfpathlineto{\pgfqpoint{1.671820in}{0.809146in}}%
\pgfpathlineto{\pgfqpoint{1.674387in}{0.678044in}}%
\pgfpathlineto{\pgfqpoint{1.679521in}{0.637236in}}%
\pgfpathlineto{\pgfqpoint{1.683795in}{0.752469in}}%
\pgfpathlineto{\pgfqpoint{1.688070in}{0.652961in}}%
\pgfpathlineto{\pgfqpoint{1.694908in}{0.747933in}}%
\pgfpathlineto{\pgfqpoint{1.696620in}{0.663939in}}%
\pgfpathlineto{\pgfqpoint{1.700041in}{0.780321in}}%
\pgfpathlineto{\pgfqpoint{1.705169in}{0.638999in}}%
\pgfpathlineto{\pgfqpoint{1.711157in}{0.751747in}}%
\pgfpathlineto{\pgfqpoint{1.713721in}{0.666851in}}%
\pgfpathlineto{\pgfqpoint{1.719709in}{0.791909in}}%
\pgfpathlineto{\pgfqpoint{1.721421in}{0.735551in}}%
\pgfpathlineto{\pgfqpoint{1.726553in}{0.682073in}}%
\pgfpathlineto{\pgfqpoint{1.733392in}{0.682759in}}%
\pgfpathlineto{\pgfqpoint{1.736816in}{0.777118in}}%
\pgfpathlineto{\pgfqpoint{1.738528in}{0.710324in}}%
\pgfpathlineto{\pgfqpoint{1.743662in}{0.771685in}}%
\pgfpathlineto{\pgfqpoint{1.747088in}{0.716228in}}%
\pgfpathlineto{\pgfqpoint{1.753081in}{0.665343in}}%
\pgfpathlineto{\pgfqpoint{1.755646in}{0.764448in}}%
\pgfpathlineto{\pgfqpoint{1.759920in}{0.657820in}}%
\pgfpathlineto{\pgfqpoint{1.765049in}{0.724685in}}%
\pgfpathlineto{\pgfqpoint{1.771888in}{0.659726in}}%
\pgfpathlineto{\pgfqpoint{1.774452in}{0.761823in}}%
\pgfpathlineto{\pgfqpoint{1.780436in}{0.641408in}}%
\pgfpathlineto{\pgfqpoint{1.783002in}{0.735120in}}%
\pgfpathlineto{\pgfqpoint{1.787279in}{0.699422in}}%
\pgfpathlineto{\pgfqpoint{1.793271in}{0.782048in}}%
\pgfpathlineto{\pgfqpoint{1.794127in}{0.682544in}}%
\pgfpathlineto{\pgfqpoint{1.800118in}{0.743936in}}%
\pgfpathlineto{\pgfqpoint{1.805246in}{0.686717in}}%
\pgfpathlineto{\pgfqpoint{1.808671in}{0.813897in}}%
\pgfpathlineto{\pgfqpoint{1.814659in}{0.695860in}}%
\pgfpathlineto{\pgfqpoint{1.818933in}{0.787625in}}%
\pgfpathlineto{\pgfqpoint{1.820647in}{0.685600in}}%
\pgfpathlineto{\pgfqpoint{1.825776in}{0.752824in}}%
\pgfpathlineto{\pgfqpoint{1.828343in}{0.692437in}}%
\pgfpathlineto{\pgfqpoint{1.832622in}{0.760994in}}%
\pgfpathlineto{\pgfqpoint{1.836895in}{0.675850in}}%
\pgfpathlineto{\pgfqpoint{1.842028in}{0.777405in}}%
\pgfpathlineto{\pgfqpoint{1.845449in}{0.690243in}}%
\pgfpathlineto{\pgfqpoint{1.851438in}{0.762940in}}%
\pgfpathlineto{\pgfqpoint{1.856567in}{0.642166in}}%
\pgfpathlineto{\pgfqpoint{1.859133in}{0.709103in}}%
\pgfpathlineto{\pgfqpoint{1.863412in}{0.657820in}}%
\pgfpathlineto{\pgfqpoint{1.867689in}{0.773807in}}%
\pgfpathlineto{\pgfqpoint{1.872817in}{0.834984in}}%
\pgfpathlineto{\pgfqpoint{1.877947in}{0.693123in}}%
\pgfpathlineto{\pgfqpoint{1.879659in}{0.760132in}}%
\pgfpathlineto{\pgfqpoint{1.886505in}{0.667681in}}%
\pgfpathlineto{\pgfqpoint{1.890783in}{0.780429in}}%
\pgfpathlineto{\pgfqpoint{1.895915in}{0.648932in}}%
\pgfpathlineto{\pgfqpoint{1.896771in}{0.781043in}}%
\pgfpathlineto{\pgfqpoint{1.901051in}{0.672539in}}%
\pgfpathlineto{\pgfqpoint{1.907894in}{0.726520in}}%
\pgfpathlineto{\pgfqpoint{1.911319in}{0.662319in}}%
\pgfpathlineto{\pgfqpoint{1.915593in}{0.771397in}}%
\pgfpathlineto{\pgfqpoint{1.919018in}{0.696506in}}%
\pgfpathlineto{\pgfqpoint{1.923300in}{0.752716in}}%
\pgfpathlineto{\pgfqpoint{1.930141in}{0.743250in}}%
\pgfpathlineto{\pgfqpoint{1.932706in}{0.684160in}}%
\pgfpathlineto{\pgfqpoint{1.937837in}{0.658035in}}%
\pgfpathlineto{\pgfqpoint{1.942970in}{0.782008in}}%
\pgfpathlineto{\pgfqpoint{1.946392in}{0.698556in}}%
\pgfpathlineto{\pgfqpoint{1.948960in}{0.802273in}}%
\pgfpathlineto{\pgfqpoint{1.953237in}{0.693123in}}%
\pgfpathlineto{\pgfqpoint{1.956661in}{0.764053in}}%
\pgfpathlineto{\pgfqpoint{1.964363in}{0.780716in}}%
\pgfpathlineto{\pgfqpoint{1.965220in}{0.676712in}}%
\pgfpathlineto{\pgfqpoint{1.971208in}{0.629425in}}%
\pgfpathlineto{\pgfqpoint{1.973771in}{0.790864in}}%
\pgfpathlineto{\pgfqpoint{1.980610in}{0.692078in}}%
\pgfpathlineto{\pgfqpoint{1.984031in}{0.758329in}}%
\pgfpathlineto{\pgfqpoint{1.989162in}{0.677394in}}%
\pgfpathlineto{\pgfqpoint{1.992584in}{0.755345in}}%
\pgfpathlineto{\pgfqpoint{1.997716in}{0.783125in}}%
\pgfpathlineto{\pgfqpoint{1.999427in}{0.702657in}}%
\pgfpathlineto{\pgfqpoint{2.003702in}{0.633095in}}%
\pgfpathlineto{\pgfqpoint{2.007981in}{0.786692in}}%
\pgfpathlineto{\pgfqpoint{2.013113in}{0.694021in}}%
\pgfpathlineto{\pgfqpoint{2.019100in}{0.829335in}}%
\pgfpathlineto{\pgfqpoint{2.022522in}{0.658071in}}%
\pgfpathlineto{\pgfqpoint{2.027657in}{0.785144in}}%
\pgfpathlineto{\pgfqpoint{2.031932in}{0.702553in}}%
\pgfpathlineto{\pgfqpoint{2.035355in}{0.800151in}}%
\pgfpathlineto{\pgfqpoint{2.038775in}{0.714860in}}%
\pgfpathlineto{\pgfqpoint{2.043051in}{0.758584in}}%
\pgfpathlineto{\pgfqpoint{2.046473in}{0.666923in}}%
\pgfpathlineto{\pgfqpoint{2.051607in}{0.760276in}}%
\pgfpathlineto{\pgfqpoint{2.055027in}{0.692908in}}%
\pgfpathlineto{\pgfqpoint{2.060159in}{0.680202in}}%
\pgfpathlineto{\pgfqpoint{2.064434in}{0.752034in}}%
\pgfpathlineto{\pgfqpoint{2.068710in}{0.690387in}}%
\pgfpathlineto{\pgfqpoint{2.072986in}{0.766068in}}%
\pgfpathlineto{\pgfqpoint{2.078974in}{0.702801in}}%
\pgfpathlineto{\pgfqpoint{2.080685in}{0.767113in}}%
\pgfpathlineto{\pgfqpoint{2.085812in}{0.714537in}}%
\pgfpathlineto{\pgfqpoint{2.091797in}{0.792268in}}%
\pgfpathlineto{\pgfqpoint{2.095221in}{0.688300in}}%
\pgfpathlineto{\pgfqpoint{2.099502in}{0.689956in}}%
\pgfpathlineto{\pgfqpoint{2.102927in}{0.817136in}}%
\pgfpathlineto{\pgfqpoint{2.107208in}{0.680278in}}%
\pgfpathlineto{\pgfqpoint{2.114054in}{0.662858in}}%
\pgfpathlineto{\pgfqpoint{2.114910in}{0.816777in}}%
\pgfpathlineto{\pgfqpoint{2.121752in}{0.795220in}}%
\pgfpathlineto{\pgfqpoint{2.124318in}{0.693267in}}%
\pgfpathlineto{\pgfqpoint{2.128598in}{0.740554in}}%
\pgfpathlineto{\pgfqpoint{2.135446in}{0.649973in}}%
\pgfpathlineto{\pgfqpoint{2.139720in}{0.837182in}}%
\pgfpathlineto{\pgfqpoint{2.141431in}{0.704029in}}%
\pgfpathlineto{\pgfqpoint{2.146558in}{0.826710in}}%
\pgfpathlineto{\pgfqpoint{2.149976in}{0.672683in}}%
\pgfpathlineto{\pgfqpoint{2.155105in}{0.729903in}}%
\pgfpathlineto{\pgfqpoint{2.157668in}{0.682185in}}%
\pgfpathlineto{\pgfqpoint{2.161947in}{0.633063in}}%
\pgfpathlineto{\pgfqpoint{2.169647in}{0.667322in}}%
\pgfpathlineto{\pgfqpoint{2.171357in}{0.855679in}}%
\pgfpathlineto{\pgfqpoint{2.177349in}{0.715438in}}%
\pgfpathlineto{\pgfqpoint{2.182483in}{0.665163in}}%
\pgfpathlineto{\pgfqpoint{2.184193in}{0.757794in}}%
\pgfpathlineto{\pgfqpoint{2.191037in}{0.689453in}}%
\pgfpathlineto{\pgfqpoint{2.192747in}{0.785575in}}%
\pgfpathlineto{\pgfqpoint{2.199592in}{0.687618in}}%
\pgfpathlineto{\pgfqpoint{2.202160in}{0.752864in}}%
\pgfpathlineto{\pgfqpoint{2.205581in}{0.678156in}}%
\pgfpathlineto{\pgfqpoint{2.211570in}{0.786835in}}%
\pgfpathlineto{\pgfqpoint{2.216701in}{0.780357in}}%
\pgfpathlineto{\pgfqpoint{2.217556in}{0.671315in}}%
\pgfpathlineto{\pgfqpoint{2.223541in}{0.750343in}}%
\pgfpathlineto{\pgfqpoint{2.229530in}{0.657496in}}%
\pgfpathlineto{\pgfqpoint{2.231239in}{0.766969in}}%
\pgfpathlineto{\pgfqpoint{2.238082in}{0.687762in}}%
\pgfpathlineto{\pgfqpoint{2.242359in}{0.800366in}}%
\pgfpathlineto{\pgfqpoint{2.243215in}{0.735914in}}%
\pgfpathlineto{\pgfqpoint{2.250909in}{0.676608in}}%
\pgfpathlineto{\pgfqpoint{2.252617in}{0.767257in}}%
\pgfpathlineto{\pgfqpoint{2.256038in}{0.799576in}}%
\pgfpathlineto{\pgfqpoint{2.261169in}{0.663508in}}%
\pgfpathlineto{\pgfqpoint{2.264591in}{0.712881in}}%
\pgfpathlineto{\pgfqpoint{2.271435in}{0.748041in}}%
\pgfpathlineto{\pgfqpoint{2.273147in}{0.625971in}}%
\pgfpathlineto{\pgfqpoint{2.278277in}{0.774349in}}%
\pgfpathlineto{\pgfqpoint{2.281703in}{0.776471in}}%
\pgfpathlineto{\pgfqpoint{2.291118in}{0.632521in}}%
\pgfpathlineto{\pgfqpoint{2.294543in}{0.795364in}}%
\pgfpathlineto{\pgfqpoint{2.298819in}{0.706223in}}%
\pgfpathlineto{\pgfqpoint{2.304810in}{0.704855in}}%
\pgfpathlineto{\pgfqpoint{2.309087in}{0.840636in}}%
\pgfpathlineto{\pgfqpoint{2.312504in}{0.661777in}}%
\pgfpathlineto{\pgfqpoint{2.316782in}{0.766251in}}%
\pgfpathlineto{\pgfqpoint{2.323623in}{0.632162in}}%
\pgfpathlineto{\pgfqpoint{2.327045in}{0.717161in}}%
\pgfpathlineto{\pgfqpoint{2.330467in}{0.651453in}}%
\pgfpathlineto{\pgfqpoint{2.333891in}{0.631874in}}%
\pgfpathlineto{\pgfqpoint{2.341587in}{0.802488in}}%
\pgfpathlineto{\pgfqpoint{2.346717in}{0.651848in}}%
\pgfpathlineto{\pgfqpoint{2.351849in}{0.758692in}}%
\pgfpathlineto{\pgfqpoint{2.357839in}{0.687834in}}%
\pgfpathlineto{\pgfqpoint{2.359552in}{0.769451in}}%
\pgfpathlineto{\pgfqpoint{2.364686in}{0.668830in}}%
\pgfpathlineto{\pgfqpoint{2.368106in}{0.805835in}}%
\pgfpathlineto{\pgfqpoint{2.371529in}{0.712518in}}%
\pgfpathlineto{\pgfqpoint{2.376661in}{0.770280in}}%
\pgfpathlineto{\pgfqpoint{2.380084in}{0.732172in}}%
\pgfpathlineto{\pgfqpoint{2.385215in}{0.650982in}}%
\pgfpathlineto{\pgfqpoint{2.391202in}{0.772115in}}%
\pgfpathlineto{\pgfqpoint{2.394625in}{0.667824in}}%
\pgfpathlineto{\pgfqpoint{2.397190in}{0.747574in}}%
\pgfpathlineto{\pgfqpoint{2.403176in}{0.689956in}}%
\pgfpathlineto{\pgfqpoint{2.408306in}{0.752214in}}%
\pgfpathlineto{\pgfqpoint{2.410871in}{0.679592in}}%
\pgfpathlineto{\pgfqpoint{2.415143in}{0.774453in}}%
\pgfpathlineto{\pgfqpoint{2.420276in}{0.689525in}}%
\pgfpathlineto{\pgfqpoint{2.425405in}{0.784713in}}%
\pgfpathlineto{\pgfqpoint{2.429680in}{0.618415in}}%
\pgfpathlineto{\pgfqpoint{2.432245in}{0.731522in}}%
\pgfpathlineto{\pgfqpoint{2.435666in}{0.790505in}}%
\pgfpathlineto{\pgfqpoint{2.442507in}{0.783125in}}%
\pgfpathlineto{\pgfqpoint{2.445931in}{0.687977in}}%
\pgfpathlineto{\pgfqpoint{2.451058in}{0.642597in}}%
\pgfpathlineto{\pgfqpoint{2.452770in}{0.721482in}}%
\pgfpathlineto{\pgfqpoint{2.457051in}{0.773232in}}%
\pgfpathlineto{\pgfqpoint{2.461330in}{0.706367in}}%
\pgfpathlineto{\pgfqpoint{2.465606in}{0.780285in}}%
\pgfpathlineto{\pgfqpoint{2.472449in}{0.676209in}}%
\pgfpathlineto{\pgfqpoint{2.475873in}{0.767185in}}%
\pgfpathlineto{\pgfqpoint{2.481860in}{0.639143in}}%
\pgfpathlineto{\pgfqpoint{2.482715in}{0.745556in}}%
\pgfpathlineto{\pgfqpoint{2.489559in}{0.634930in}}%
\pgfpathlineto{\pgfqpoint{2.492125in}{0.743035in}}%
\pgfpathlineto{\pgfqpoint{2.497259in}{0.660301in}}%
\pgfpathlineto{\pgfqpoint{2.502393in}{0.760132in}}%
\pgfpathlineto{\pgfqpoint{2.504105in}{0.690243in}}%
\pgfpathlineto{\pgfqpoint{2.511803in}{0.801623in}}%
\pgfpathlineto{\pgfqpoint{2.514369in}{0.667210in}}%
\pgfpathlineto{\pgfqpoint{2.518648in}{0.649937in}}%
\pgfpathlineto{\pgfqpoint{2.522070in}{0.771286in}}%
\pgfpathlineto{\pgfqpoint{2.528054in}{0.665519in}}%
\pgfpathlineto{\pgfqpoint{2.529766in}{0.731841in}}%
\pgfpathlineto{\pgfqpoint{2.536611in}{0.687219in}}%
\pgfpathlineto{\pgfqpoint{2.540888in}{0.762470in}}%
\pgfpathlineto{\pgfqpoint{2.542599in}{0.654939in}}%
\pgfpathlineto{\pgfqpoint{2.546873in}{0.765960in}}%
\pgfpathlineto{\pgfqpoint{2.552006in}{0.684666in}}%
\pgfpathlineto{\pgfqpoint{2.556277in}{0.769379in}}%
\pgfpathlineto{\pgfqpoint{2.560553in}{0.641624in}}%
\pgfpathlineto{\pgfqpoint{2.563974in}{0.683118in}}%
\pgfpathlineto{\pgfqpoint{2.569104in}{0.766642in}}%
\pgfpathlineto{\pgfqpoint{2.572527in}{0.694886in}}%
\pgfpathlineto{\pgfqpoint{2.576804in}{0.769092in}}%
\pgfpathlineto{\pgfqpoint{2.584495in}{0.663648in}}%
\pgfpathlineto{\pgfqpoint{2.585351in}{0.764592in}}%
\pgfpathlineto{\pgfqpoint{2.596466in}{0.679879in}}%
\pgfpathlineto{\pgfqpoint{2.598175in}{0.752393in}}%
\pgfpathlineto{\pgfqpoint{2.602452in}{0.687794in}}%
\pgfpathlineto{\pgfqpoint{2.609301in}{0.653463in}}%
\pgfpathlineto{\pgfqpoint{2.614436in}{0.824405in}}%
\pgfpathlineto{\pgfqpoint{2.615290in}{0.727094in}}%
\pgfpathlineto{\pgfqpoint{2.621275in}{0.672467in}}%
\pgfpathlineto{\pgfqpoint{2.627264in}{0.770137in}}%
\pgfpathlineto{\pgfqpoint{2.631542in}{0.663289in}}%
\pgfpathlineto{\pgfqpoint{2.635823in}{0.820878in}}%
\pgfpathlineto{\pgfqpoint{2.637532in}{0.686214in}}%
\pgfpathlineto{\pgfqpoint{2.642664in}{0.744152in}}%
\pgfpathlineto{\pgfqpoint{2.646085in}{0.669300in}}%
\pgfpathlineto{\pgfqpoint{2.651221in}{0.773192in}}%
\pgfpathlineto{\pgfqpoint{2.655500in}{0.705322in}}%
\pgfpathlineto{\pgfqpoint{2.658066in}{0.805656in}}%
\pgfpathlineto{\pgfqpoint{2.663199in}{0.710037in}}%
\pgfpathlineto{\pgfqpoint{2.668333in}{0.807419in}}%
\pgfpathlineto{\pgfqpoint{2.671754in}{0.683729in}}%
\pgfpathlineto{\pgfqpoint{2.675176in}{0.733932in}}%
\pgfpathlineto{\pgfqpoint{2.679450in}{0.688623in}}%
\pgfpathlineto{\pgfqpoint{2.685443in}{0.762075in}}%
\pgfpathlineto{\pgfqpoint{2.688863in}{0.686717in}}%
\pgfpathlineto{\pgfqpoint{2.693143in}{0.793816in}}%
\pgfpathlineto{\pgfqpoint{2.696565in}{0.716874in}}%
\pgfpathlineto{\pgfqpoint{2.702555in}{0.751420in}}%
\pgfpathlineto{\pgfqpoint{2.706832in}{0.678116in}}%
\pgfpathlineto{\pgfqpoint{2.711109in}{0.791439in}}%
\pgfpathlineto{\pgfqpoint{2.713672in}{0.689884in}}%
\pgfpathlineto{\pgfqpoint{2.719654in}{0.814831in}}%
\pgfpathlineto{\pgfqpoint{2.722221in}{0.722236in}}%
\pgfpathlineto{\pgfqpoint{2.728213in}{0.764017in}}%
\pgfpathlineto{\pgfqpoint{2.734203in}{0.797773in}}%
\pgfpathlineto{\pgfqpoint{2.737626in}{0.648318in}}%
\pgfpathlineto{\pgfqpoint{2.739338in}{0.734833in}}%
\pgfpathlineto{\pgfqpoint{2.746179in}{0.771397in}}%
\pgfpathlineto{\pgfqpoint{2.748744in}{0.683047in}}%
\pgfpathlineto{\pgfqpoint{2.754727in}{0.760922in}}%
\pgfpathlineto{\pgfqpoint{2.758145in}{0.788056in}}%
\pgfpathlineto{\pgfqpoint{2.760706in}{0.699781in}}%
\pgfpathlineto{\pgfqpoint{2.764980in}{0.783197in}}%
\pgfpathlineto{\pgfqpoint{2.769259in}{0.700535in}}%
\pgfpathlineto{\pgfqpoint{2.776955in}{0.686102in}}%
\pgfpathlineto{\pgfqpoint{2.778666in}{0.787083in}}%
\pgfpathlineto{\pgfqpoint{2.782089in}{0.699853in}}%
\pgfpathlineto{\pgfqpoint{2.789789in}{0.624351in}}%
\pgfpathlineto{\pgfqpoint{2.790642in}{0.752106in}}%
\pgfpathlineto{\pgfqpoint{2.794918in}{0.702410in}}%
\pgfpathlineto{\pgfqpoint{2.799189in}{0.795795in}}%
\pgfpathlineto{\pgfqpoint{2.803464in}{0.690606in}}%
\pgfpathlineto{\pgfqpoint{2.809450in}{0.775714in}}%
\pgfpathlineto{\pgfqpoint{2.813729in}{0.691033in}}%
\pgfpathlineto{\pgfqpoint{2.818861in}{0.662463in}}%
\pgfpathlineto{\pgfqpoint{2.820569in}{0.736596in}}%
\pgfpathlineto{\pgfqpoint{2.827409in}{0.652961in}}%
\pgfpathlineto{\pgfqpoint{2.832540in}{0.745340in}}%
\pgfpathlineto{\pgfqpoint{2.833396in}{0.682217in}}%
\pgfpathlineto{\pgfqpoint{2.839382in}{0.760994in}}%
\pgfpathlineto{\pgfqpoint{2.845369in}{0.678870in}}%
\pgfpathlineto{\pgfqpoint{2.847078in}{0.750774in}}%
\pgfpathlineto{\pgfqpoint{2.850497in}{0.682037in}}%
\pgfpathlineto{\pgfqpoint{2.855627in}{0.772474in}}%
\pgfpathlineto{\pgfqpoint{2.859900in}{0.649470in}}%
\pgfpathlineto{\pgfqpoint{2.865887in}{0.776862in}}%
\pgfpathlineto{\pgfqpoint{2.867599in}{0.729719in}}%
\pgfpathlineto{\pgfqpoint{2.874440in}{0.630686in}}%
\pgfpathlineto{\pgfqpoint{2.876150in}{0.725981in}}%
\pgfpathlineto{\pgfqpoint{2.880425in}{0.674949in}}%
\pgfpathlineto{\pgfqpoint{2.884702in}{0.770424in}}%
\pgfpathlineto{\pgfqpoint{2.891547in}{0.788096in}}%
\pgfpathlineto{\pgfqpoint{2.896681in}{0.694886in}}%
\pgfpathlineto{\pgfqpoint{2.897536in}{0.795723in}}%
\pgfpathlineto{\pgfqpoint{2.905233in}{0.795005in}}%
\pgfpathlineto{\pgfqpoint{2.906944in}{0.709750in}}%
\pgfpathlineto{\pgfqpoint{2.910364in}{0.751962in}}%
\pgfpathlineto{\pgfqpoint{2.918060in}{0.706151in}}%
\pgfpathlineto{\pgfqpoint{2.918914in}{0.776288in}}%
\pgfpathlineto{\pgfqpoint{2.927463in}{0.665231in}}%
\pgfpathlineto{\pgfqpoint{2.931738in}{0.760308in}}%
\pgfpathlineto{\pgfqpoint{2.936873in}{0.694487in}}%
\pgfpathlineto{\pgfqpoint{2.942857in}{0.685528in}}%
\pgfpathlineto{\pgfqpoint{2.944568in}{0.778374in}}%
\pgfpathlineto{\pgfqpoint{2.948846in}{0.697475in}}%
\pgfpathlineto{\pgfqpoint{2.955685in}{0.759949in}}%
\pgfpathlineto{\pgfqpoint{2.959964in}{0.606934in}}%
\pgfpathlineto{\pgfqpoint{2.964240in}{0.791510in}}%
\pgfpathlineto{\pgfqpoint{2.968519in}{0.842866in}}%
\pgfpathlineto{\pgfqpoint{2.970229in}{0.708166in}}%
\pgfpathlineto{\pgfqpoint{2.977074in}{0.701077in}}%
\pgfpathlineto{\pgfqpoint{2.978784in}{0.764089in}}%
\pgfpathlineto{\pgfqpoint{2.983062in}{0.697080in}}%
\pgfpathlineto{\pgfqpoint{2.988194in}{0.750630in}}%
\pgfpathlineto{\pgfqpoint{2.994182in}{0.699745in}}%
\pgfpathlineto{\pgfqpoint{2.999311in}{0.785000in}}%
\pgfpathlineto{\pgfqpoint{3.002736in}{0.701149in}}%
\pgfpathlineto{\pgfqpoint{3.007011in}{0.815301in}}%
\pgfpathlineto{\pgfqpoint{3.011287in}{0.686717in}}%
\pgfpathlineto{\pgfqpoint{3.013853in}{0.774995in}}%
\pgfpathlineto{\pgfqpoint{3.019839in}{0.792232in}}%
\pgfpathlineto{\pgfqpoint{3.021551in}{0.717560in}}%
\pgfpathlineto{\pgfqpoint{3.029249in}{0.612695in}}%
\pgfpathlineto{\pgfqpoint{3.030958in}{0.823686in}}%
\pgfpathlineto{\pgfqpoint{3.034379in}{0.741958in}}%
\pgfpathlineto{\pgfqpoint{3.039507in}{0.806701in}}%
\pgfpathlineto{\pgfqpoint{3.042930in}{0.716192in}}%
\pgfpathlineto{\pgfqpoint{3.050630in}{0.662535in}}%
\pgfpathlineto{\pgfqpoint{3.054053in}{0.812421in}}%
\pgfpathlineto{\pgfqpoint{3.059185in}{0.808895in}}%
\pgfpathlineto{\pgfqpoint{3.061754in}{0.684558in}}%
\pgfpathlineto{\pgfqpoint{3.064319in}{0.744371in}}%
\pgfpathlineto{\pgfqpoint{3.072018in}{0.681714in}}%
\pgfpathlineto{\pgfqpoint{3.072875in}{0.723424in}}%
\pgfpathlineto{\pgfqpoint{3.079716in}{0.757324in}}%
\pgfpathlineto{\pgfqpoint{3.084849in}{0.664370in}}%
\pgfpathlineto{\pgfqpoint{3.085704in}{0.774457in}}%
\pgfpathlineto{\pgfqpoint{3.090835in}{0.706511in}}%
\pgfpathlineto{\pgfqpoint{3.094255in}{0.765063in}}%
\pgfpathlineto{\pgfqpoint{3.099388in}{0.657923in}}%
\pgfpathlineto{\pgfqpoint{3.103662in}{0.748037in}}%
\pgfpathlineto{\pgfqpoint{3.110508in}{0.756638in}}%
\pgfpathlineto{\pgfqpoint{3.112219in}{0.677358in}}%
\pgfpathlineto{\pgfqpoint{3.117359in}{0.767903in}}%
\pgfpathlineto{\pgfqpoint{3.122495in}{0.755704in}}%
\pgfpathlineto{\pgfqpoint{3.126770in}{0.632521in}}%
\pgfpathlineto{\pgfqpoint{3.128480in}{0.722092in}}%
\pgfpathlineto{\pgfqpoint{3.135323in}{0.804938in}}%
\pgfpathlineto{\pgfqpoint{3.138744in}{0.671494in}}%
\pgfpathlineto{\pgfqpoint{3.143019in}{0.741886in}}%
\pgfpathlineto{\pgfqpoint{3.146436in}{0.685280in}}%
\pgfpathlineto{\pgfqpoint{3.151568in}{0.748687in}}%
\pgfpathlineto{\pgfqpoint{3.154987in}{0.646195in}}%
\pgfpathlineto{\pgfqpoint{3.161832in}{0.767296in}}%
\pgfpathlineto{\pgfqpoint{3.162686in}{0.725730in}}%
\pgfpathlineto{\pgfqpoint{3.166956in}{0.775570in}}%
\pgfpathlineto{\pgfqpoint{3.171229in}{0.657460in}}%
\pgfpathlineto{\pgfqpoint{3.176360in}{0.781761in}}%
\pgfpathlineto{\pgfqpoint{3.181495in}{0.704676in}}%
\pgfpathlineto{\pgfqpoint{3.184919in}{0.785216in}}%
\pgfpathlineto{\pgfqpoint{3.189194in}{0.692872in}}%
\pgfpathlineto{\pgfqpoint{3.195182in}{0.771469in}}%
\pgfpathlineto{\pgfqpoint{3.196893in}{0.685169in}}%
\pgfpathlineto{\pgfqpoint{3.202029in}{0.784242in}}%
\pgfpathlineto{\pgfqpoint{3.206306in}{0.793457in}}%
\pgfpathlineto{\pgfqpoint{3.213149in}{0.594951in}}%
\pgfpathlineto{\pgfqpoint{3.214004in}{0.723640in}}%
\pgfpathlineto{\pgfqpoint{3.218280in}{0.756031in}}%
\pgfpathlineto{\pgfqpoint{3.222557in}{0.645908in}}%
\pgfpathlineto{\pgfqpoint{3.226834in}{0.724505in}}%
\pgfpathlineto{\pgfqpoint{3.234531in}{0.769993in}}%
\pgfpathlineto{\pgfqpoint{3.237098in}{0.678188in}}%
\pgfpathlineto{\pgfqpoint{3.243083in}{0.693845in}}%
\pgfpathlineto{\pgfqpoint{3.243939in}{0.791622in}}%
\pgfpathlineto{\pgfqpoint{3.252486in}{0.637272in}}%
\pgfpathlineto{\pgfqpoint{3.256765in}{0.763443in}}%
\pgfpathlineto{\pgfqpoint{3.263612in}{0.675922in}}%
\pgfpathlineto{\pgfqpoint{3.266171in}{0.816091in}}%
\pgfpathlineto{\pgfqpoint{3.269594in}{0.710795in}}%
\pgfpathlineto{\pgfqpoint{3.274725in}{0.783740in}}%
\pgfpathlineto{\pgfqpoint{3.278148in}{0.675204in}}%
\pgfpathlineto{\pgfqpoint{3.283283in}{0.783528in}}%
\pgfpathlineto{\pgfqpoint{3.287561in}{0.672288in}}%
\pgfpathlineto{\pgfqpoint{3.291835in}{0.783524in}}%
\pgfpathlineto{\pgfqpoint{3.296964in}{0.670956in}}%
\pgfpathlineto{\pgfqpoint{3.302092in}{0.675635in}}%
\pgfpathlineto{\pgfqpoint{3.306369in}{0.761285in}}%
\pgfpathlineto{\pgfqpoint{3.311495in}{0.782447in}}%
\pgfpathlineto{\pgfqpoint{3.313205in}{0.636446in}}%
\pgfpathlineto{\pgfqpoint{3.316624in}{0.736596in}}%
\pgfpathlineto{\pgfqpoint{3.323472in}{0.786261in}}%
\pgfpathlineto{\pgfqpoint{3.327748in}{0.635688in}}%
\pgfpathlineto{\pgfqpoint{3.332026in}{0.770639in}}%
\pgfpathlineto{\pgfqpoint{3.333737in}{0.710938in}}%
\pgfpathlineto{\pgfqpoint{3.339724in}{0.635113in}}%
\pgfpathlineto{\pgfqpoint{3.345713in}{0.773304in}}%
\pgfpathlineto{\pgfqpoint{3.346570in}{0.694958in}}%
\pgfpathlineto{\pgfqpoint{3.351703in}{0.656383in}}%
\pgfpathlineto{\pgfqpoint{3.355978in}{0.781258in}}%
\pgfpathlineto{\pgfqpoint{3.359402in}{0.689852in}}%
\pgfpathlineto{\pgfqpoint{3.367101in}{0.683445in}}%
\pgfpathlineto{\pgfqpoint{3.367957in}{0.817894in}}%
\pgfpathlineto{\pgfqpoint{3.374800in}{0.665307in}}%
\pgfpathlineto{\pgfqpoint{3.379076in}{0.631443in}}%
\pgfpathlineto{\pgfqpoint{3.380784in}{0.727350in}}%
\pgfpathlineto{\pgfqpoint{3.388483in}{0.775610in}}%
\pgfpathlineto{\pgfqpoint{3.389340in}{0.657931in}}%
\pgfpathlineto{\pgfqpoint{3.397035in}{0.801052in}}%
\pgfpathlineto{\pgfqpoint{3.398746in}{0.670704in}}%
\pgfpathlineto{\pgfqpoint{3.403879in}{0.819298in}}%
\pgfpathlineto{\pgfqpoint{3.407299in}{0.676249in}}%
\pgfpathlineto{\pgfqpoint{3.410720in}{0.742644in}}%
\pgfpathlineto{\pgfqpoint{3.414994in}{0.655231in}}%
\pgfpathlineto{\pgfqpoint{3.419266in}{0.713962in}}%
\pgfpathlineto{\pgfqpoint{3.423538in}{0.689924in}}%
\pgfpathlineto{\pgfqpoint{3.430381in}{0.797454in}}%
\pgfpathlineto{\pgfqpoint{3.433801in}{0.584408in}}%
\pgfpathlineto{\pgfqpoint{3.437223in}{0.753909in}}%
\pgfpathlineto{\pgfqpoint{3.443211in}{0.684742in}}%
\pgfpathlineto{\pgfqpoint{3.444923in}{0.741495in}}%
\pgfpathlineto{\pgfqpoint{3.450907in}{0.805911in}}%
\pgfpathlineto{\pgfqpoint{3.456036in}{0.709678in}}%
\pgfpathlineto{\pgfqpoint{3.457749in}{0.779423in}}%
\pgfpathlineto{\pgfqpoint{3.462026in}{0.621076in}}%
\pgfpathlineto{\pgfqpoint{3.466302in}{0.748400in}}%
\pgfpathlineto{\pgfqpoint{3.470577in}{0.774349in}}%
\pgfpathlineto{\pgfqpoint{3.475705in}{0.681319in}}%
\pgfpathlineto{\pgfqpoint{3.480840in}{0.739548in}}%
\pgfpathlineto{\pgfqpoint{3.483407in}{0.658793in}}%
\pgfpathlineto{\pgfqpoint{3.490249in}{0.762761in}}%
\pgfpathlineto{\pgfqpoint{3.493668in}{0.671319in}}%
\pgfpathlineto{\pgfqpoint{3.498802in}{0.738001in}}%
\pgfpathlineto{\pgfqpoint{3.503934in}{0.596786in}}%
\pgfpathlineto{\pgfqpoint{3.507357in}{0.775785in}}%
\pgfpathlineto{\pgfqpoint{3.509065in}{0.654616in}}%
\pgfpathlineto{\pgfqpoint{3.514195in}{0.793888in}}%
\pgfpathlineto{\pgfqpoint{3.517615in}{0.669300in}}%
\pgfpathlineto{\pgfqpoint{3.522750in}{0.736632in}}%
\pgfpathlineto{\pgfqpoint{3.527884in}{0.657030in}}%
\pgfpathlineto{\pgfqpoint{3.533869in}{0.743721in}}%
\pgfpathlineto{\pgfqpoint{3.536436in}{0.670848in}}%
\pgfpathlineto{\pgfqpoint{3.539857in}{0.778091in}}%
\pgfpathlineto{\pgfqpoint{3.543277in}{0.640403in}}%
\pgfpathlineto{\pgfqpoint{3.548412in}{0.744335in}}%
\pgfpathlineto{\pgfqpoint{3.552684in}{0.678443in}}%
\pgfpathlineto{\pgfqpoint{3.556103in}{0.759849in}}%
\pgfpathlineto{\pgfqpoint{3.561230in}{0.702449in}}%
\pgfpathlineto{\pgfqpoint{3.567217in}{0.682871in}}%
\pgfpathlineto{\pgfqpoint{3.569782in}{0.807925in}}%
\pgfpathlineto{\pgfqpoint{3.574058in}{0.695788in}}%
\pgfpathlineto{\pgfqpoint{3.580044in}{0.805369in}}%
\pgfpathlineto{\pgfqpoint{3.581755in}{0.681032in}}%
\pgfpathlineto{\pgfqpoint{3.586031in}{0.753367in}}%
\pgfpathlineto{\pgfqpoint{3.592017in}{0.673010in}}%
\pgfpathlineto{\pgfqpoint{3.595438in}{0.810913in}}%
\pgfpathlineto{\pgfqpoint{3.598858in}{0.697407in}}%
\pgfpathlineto{\pgfqpoint{3.605706in}{0.610357in}}%
\pgfpathlineto{\pgfqpoint{3.607419in}{0.749122in}}%
\pgfpathlineto{\pgfqpoint{3.611692in}{0.689924in}}%
\pgfpathlineto{\pgfqpoint{3.615968in}{0.734043in}}%
\pgfpathlineto{\pgfqpoint{3.621959in}{0.679628in}}%
\pgfpathlineto{\pgfqpoint{3.627946in}{0.774924in}}%
\pgfpathlineto{\pgfqpoint{3.628798in}{0.694779in}}%
\pgfpathlineto{\pgfqpoint{3.634787in}{0.732783in}}%
\pgfpathlineto{\pgfqpoint{3.640777in}{0.612080in}}%
\pgfpathlineto{\pgfqpoint{3.641632in}{0.740809in}}%
\pgfpathlineto{\pgfqpoint{3.646770in}{0.660843in}}%
\pgfpathlineto{\pgfqpoint{3.652758in}{0.628563in}}%
\pgfpathlineto{\pgfqpoint{3.657887in}{0.761249in}}%
\pgfpathlineto{\pgfqpoint{3.658742in}{0.670704in}}%
\pgfpathlineto{\pgfqpoint{3.663017in}{0.779998in}}%
\pgfpathlineto{\pgfqpoint{3.669860in}{0.675348in}}%
\pgfpathlineto{\pgfqpoint{3.671571in}{0.754771in}}%
\pgfpathlineto{\pgfqpoint{3.678414in}{0.785866in}}%
\pgfpathlineto{\pgfqpoint{3.680124in}{0.670776in}}%
\pgfpathlineto{\pgfqpoint{3.686109in}{0.758297in}}%
\pgfpathlineto{\pgfqpoint{3.689529in}{0.771972in}}%
\pgfpathlineto{\pgfqpoint{3.695519in}{0.701795in}}%
\pgfpathlineto{\pgfqpoint{3.699794in}{0.739692in}}%
\pgfpathlineto{\pgfqpoint{3.701506in}{0.645800in}}%
\pgfpathlineto{\pgfqpoint{3.705784in}{0.725658in}}%
\pgfpathlineto{\pgfqpoint{3.710060in}{0.654365in}}%
\pgfpathlineto{\pgfqpoint{3.714341in}{0.739907in}}%
\pgfpathlineto{\pgfqpoint{3.720332in}{0.673548in}}%
\pgfpathlineto{\pgfqpoint{3.722900in}{0.724003in}}%
\pgfpathlineto{\pgfqpoint{3.730597in}{0.792811in}}%
\pgfpathlineto{\pgfqpoint{3.731453in}{0.676608in}}%
\pgfpathlineto{\pgfqpoint{3.738291in}{0.796768in}}%
\pgfpathlineto{\pgfqpoint{3.739998in}{0.718207in}}%
\pgfpathlineto{\pgfqpoint{3.745129in}{0.684164in}}%
\pgfpathlineto{\pgfqpoint{3.751968in}{0.780612in}}%
\pgfpathlineto{\pgfqpoint{3.756244in}{0.658003in}}%
\pgfpathlineto{\pgfqpoint{3.757100in}{0.755888in}}%
\pgfpathlineto{\pgfqpoint{3.763091in}{0.648537in}}%
\pgfpathlineto{\pgfqpoint{3.767366in}{0.748831in}}%
\pgfpathlineto{\pgfqpoint{3.769934in}{0.675958in}}%
\pgfpathlineto{\pgfqpoint{3.777634in}{0.611865in}}%
\pgfpathlineto{\pgfqpoint{3.779343in}{0.724577in}}%
\pgfpathlineto{\pgfqpoint{3.785331in}{0.754914in}}%
\pgfpathlineto{\pgfqpoint{3.790465in}{0.691687in}}%
\pgfpathlineto{\pgfqpoint{3.792176in}{0.768844in}}%
\pgfpathlineto{\pgfqpoint{3.795597in}{0.696725in}}%
\pgfpathlineto{\pgfqpoint{3.801584in}{0.786907in}}%
\pgfpathlineto{\pgfqpoint{3.804147in}{0.714249in}}%
\pgfpathlineto{\pgfqpoint{3.809278in}{0.671534in}}%
\pgfpathlineto{\pgfqpoint{3.816117in}{0.765102in}}%
\pgfpathlineto{\pgfqpoint{3.819537in}{0.660883in}}%
\pgfpathlineto{\pgfqpoint{3.821247in}{0.745811in}}%
\pgfpathlineto{\pgfqpoint{3.825520in}{0.673010in}}%
\pgfpathlineto{\pgfqpoint{3.829795in}{0.658721in}}%
\pgfpathlineto{\pgfqpoint{3.834067in}{0.754124in}}%
\pgfpathlineto{\pgfqpoint{3.841763in}{0.780971in}}%
\pgfpathlineto{\pgfqpoint{3.842619in}{0.688053in}}%
\pgfpathlineto{\pgfqpoint{3.847744in}{0.646774in}}%
\pgfpathlineto{\pgfqpoint{3.851170in}{0.748907in}}%
\pgfpathlineto{\pgfqpoint{3.857163in}{0.652570in}}%
\pgfpathlineto{\pgfqpoint{3.859732in}{0.734905in}}%
\pgfpathlineto{\pgfqpoint{3.866582in}{0.675419in}}%
\pgfpathlineto{\pgfqpoint{3.870855in}{0.748260in}}%
\pgfpathlineto{\pgfqpoint{3.873418in}{0.654405in}}%
\pgfpathlineto{\pgfqpoint{3.880263in}{0.666496in}}%
\pgfpathlineto{\pgfqpoint{3.884540in}{0.750957in}}%
\pgfpathlineto{\pgfqpoint{3.885396in}{0.658577in}}%
\pgfpathlineto{\pgfqpoint{3.891380in}{0.763914in}}%
\pgfpathlineto{\pgfqpoint{3.894800in}{0.663041in}}%
\pgfpathlineto{\pgfqpoint{3.899075in}{0.750203in}}%
\pgfpathlineto{\pgfqpoint{3.905916in}{0.785184in}}%
\pgfpathlineto{\pgfqpoint{3.908484in}{0.659335in}}%
\pgfpathlineto{\pgfqpoint{3.912763in}{0.742827in}}%
\pgfpathlineto{\pgfqpoint{3.917037in}{0.775035in}}%
\pgfpathlineto{\pgfqpoint{3.923020in}{0.692912in}}%
\pgfpathlineto{\pgfqpoint{3.924731in}{0.770320in}}%
\pgfpathlineto{\pgfqpoint{3.929865in}{0.648900in}}%
\pgfpathlineto{\pgfqpoint{3.935852in}{0.796808in}}%
\pgfpathlineto{\pgfqpoint{3.939274in}{0.707990in}}%
\pgfpathlineto{\pgfqpoint{3.943555in}{0.746425in}}%
\pgfpathlineto{\pgfqpoint{3.947833in}{0.630725in}}%
\pgfpathlineto{\pgfqpoint{3.952966in}{0.727421in}}%
\pgfpathlineto{\pgfqpoint{3.953820in}{0.639972in}}%
\pgfpathlineto{\pgfqpoint{3.958954in}{0.771254in}}%
\pgfpathlineto{\pgfqpoint{3.962375in}{0.664840in}}%
\pgfpathlineto{\pgfqpoint{3.969212in}{0.666388in}}%
\pgfpathlineto{\pgfqpoint{3.970924in}{0.736923in}}%
\pgfpathlineto{\pgfqpoint{3.976053in}{0.691001in}}%
\pgfpathlineto{\pgfqpoint{3.980331in}{0.749553in}}%
\pgfpathlineto{\pgfqpoint{3.985468in}{0.704500in}}%
\pgfpathlineto{\pgfqpoint{3.991456in}{0.774604in}}%
\pgfpathlineto{\pgfqpoint{3.993166in}{0.679596in}}%
\pgfpathlineto{\pgfqpoint{3.996588in}{0.753550in}}%
\pgfpathlineto{\pgfqpoint{4.004292in}{0.684060in}}%
\pgfpathlineto{\pgfqpoint{4.006858in}{0.751101in}}%
\pgfpathlineto{\pgfqpoint{4.011132in}{0.782734in}}%
\pgfpathlineto{\pgfqpoint{4.017123in}{0.645585in}}%
\pgfpathlineto{\pgfqpoint{4.017979in}{0.772730in}}%
\pgfpathlineto{\pgfqpoint{4.022252in}{0.689238in}}%
\pgfpathlineto{\pgfqpoint{4.027385in}{0.761967in}}%
\pgfpathlineto{\pgfqpoint{4.030804in}{0.671929in}}%
\pgfpathlineto{\pgfqpoint{4.039357in}{0.759270in}}%
\pgfpathlineto{\pgfqpoint{4.044492in}{0.688807in}}%
\pgfpathlineto{\pgfqpoint{4.049623in}{0.652315in}}%
\pgfpathlineto{\pgfqpoint{4.054759in}{0.821524in}}%
\pgfpathlineto{\pgfqpoint{4.058181in}{0.706008in}}%
\pgfpathlineto{\pgfqpoint{4.064169in}{0.762474in}}%
\pgfpathlineto{\pgfqpoint{4.068446in}{0.609061in}}%
\pgfpathlineto{\pgfqpoint{4.069301in}{0.725156in}}%
\pgfpathlineto{\pgfqpoint{4.073578in}{0.671534in}}%
\pgfpathlineto{\pgfqpoint{4.080418in}{0.763770in}}%
\pgfpathlineto{\pgfqpoint{4.082985in}{0.689708in}}%
\pgfpathlineto{\pgfqpoint{4.086408in}{0.765246in}}%
\pgfpathlineto{\pgfqpoint{4.091542in}{0.666460in}}%
\pgfpathlineto{\pgfqpoint{4.098381in}{0.782375in}}%
\pgfpathlineto{\pgfqpoint{4.100093in}{0.707484in}}%
\pgfpathlineto{\pgfqpoint{4.106080in}{0.769993in}}%
\pgfpathlineto{\pgfqpoint{4.108646in}{0.672467in}}%
\pgfpathlineto{\pgfqpoint{4.114636in}{0.742572in}}%
\pgfpathlineto{\pgfqpoint{4.118056in}{0.673836in}}%
\pgfpathlineto{\pgfqpoint{4.122329in}{0.725012in}}%
\pgfpathlineto{\pgfqpoint{4.127462in}{0.637275in}}%
\pgfpathlineto{\pgfqpoint{4.129172in}{0.697663in}}%
\pgfpathlineto{\pgfqpoint{4.136022in}{0.625795in}}%
\pgfpathlineto{\pgfqpoint{4.140300in}{0.711624in}}%
\pgfpathlineto{\pgfqpoint{4.143724in}{0.667325in}}%
\pgfpathlineto{\pgfqpoint{4.145436in}{0.734115in}}%
\pgfpathlineto{\pgfqpoint{4.151424in}{0.679847in}}%
\pgfpathlineto{\pgfqpoint{4.152280in}{0.723033in}}%
\pgfpathlineto{\pgfqpoint{4.156558in}{0.637132in}}%
\pgfpathlineto{\pgfqpoint{4.159124in}{0.772698in}}%
\pgfpathlineto{\pgfqpoint{4.162548in}{0.660452in}}%
\pgfpathlineto{\pgfqpoint{4.168538in}{0.764488in}}%
\pgfpathlineto{\pgfqpoint{4.171962in}{0.659263in}}%
\pgfpathlineto{\pgfqpoint{4.173670in}{0.784138in}}%
\pgfpathlineto{\pgfqpoint{4.177947in}{0.696905in}}%
\pgfpathlineto{\pgfqpoint{4.181364in}{0.685604in}}%
\pgfpathlineto{\pgfqpoint{4.183932in}{0.772945in}}%
\pgfpathlineto{\pgfqpoint{4.187352in}{0.634754in}}%
\pgfpathlineto{\pgfqpoint{4.189920in}{0.717058in}}%
\pgfpathlineto{\pgfqpoint{4.195906in}{0.657608in}}%
\pgfpathlineto{\pgfqpoint{4.196761in}{0.715223in}}%
\pgfpathlineto{\pgfqpoint{4.200182in}{0.738399in}}%
\pgfpathlineto{\pgfqpoint{4.205311in}{0.737498in}}%
\pgfpathlineto{\pgfqpoint{4.209590in}{0.638101in}}%
\pgfpathlineto{\pgfqpoint{4.210442in}{0.764201in}}%
\pgfpathlineto{\pgfqpoint{4.214719in}{0.649977in}}%
\pgfpathlineto{\pgfqpoint{4.218998in}{0.652099in}}%
\pgfpathlineto{\pgfqpoint{4.221561in}{0.767476in}}%
\pgfpathlineto{\pgfqpoint{4.224983in}{0.786835in}}%
\pgfpathlineto{\pgfqpoint{4.228405in}{0.674629in}}%
\pgfpathlineto{\pgfqpoint{4.231826in}{0.644975in}}%
\pgfpathlineto{\pgfqpoint{4.235248in}{0.755561in}}%
\pgfpathlineto{\pgfqpoint{4.237816in}{0.624351in}}%
\pgfpathlineto{\pgfqpoint{4.241236in}{0.748221in}}%
\pgfpathlineto{\pgfqpoint{4.246370in}{0.614601in}}%
\pgfpathlineto{\pgfqpoint{4.248080in}{0.700646in}}%
\pgfpathlineto{\pgfqpoint{4.252359in}{0.658362in}}%
\pgfpathlineto{\pgfqpoint{4.255778in}{0.705936in}}%
\pgfpathlineto{\pgfqpoint{4.259202in}{0.600316in}}%
\pgfpathlineto{\pgfqpoint{4.265191in}{0.760172in}}%
\pgfpathlineto{\pgfqpoint{4.268614in}{0.657464in}}%
\pgfpathlineto{\pgfqpoint{4.272037in}{0.730014in}}%
\pgfpathlineto{\pgfqpoint{4.275458in}{0.733792in}}%
\pgfpathlineto{\pgfqpoint{4.279734in}{0.695644in}}%
\pgfpathlineto{\pgfqpoint{4.284864in}{0.678048in}}%
\pgfpathlineto{\pgfqpoint{4.285716in}{0.739117in}}%
\pgfpathlineto{\pgfqpoint{4.290850in}{0.759701in}}%
\pgfpathlineto{\pgfqpoint{4.293417in}{0.662176in}}%
\pgfpathlineto{\pgfqpoint{4.297691in}{0.784713in}}%
\pgfpathlineto{\pgfqpoint{4.299404in}{0.672001in}}%
\pgfpathlineto{\pgfqpoint{4.307096in}{0.805839in}}%
\pgfpathlineto{\pgfqpoint{4.311374in}{0.685209in}}%
\pgfpathlineto{\pgfqpoint{4.313941in}{0.768126in}}%
\pgfpathlineto{\pgfqpoint{4.318213in}{0.800191in}}%
\pgfpathlineto{\pgfqpoint{4.319924in}{0.689708in}}%
\pgfpathlineto{\pgfqpoint{4.324195in}{0.655952in}}%
\pgfpathlineto{\pgfqpoint{4.326759in}{0.753622in}}%
\pgfpathlineto{\pgfqpoint{4.331037in}{0.662646in}}%
\pgfpathlineto{\pgfqpoint{4.334460in}{0.736600in}}%
\pgfpathlineto{\pgfqpoint{4.337026in}{0.701009in}}%
\pgfpathlineto{\pgfqpoint{4.342159in}{0.672435in}}%
\pgfpathlineto{\pgfqpoint{4.343870in}{0.741064in}}%
\pgfpathlineto{\pgfqpoint{4.350713in}{0.659730in}}%
\pgfpathlineto{\pgfqpoint{4.355844in}{0.754272in}}%
\pgfpathlineto{\pgfqpoint{4.360978in}{0.684921in}}%
\pgfpathlineto{\pgfqpoint{4.366112in}{0.802456in}}%
\pgfpathlineto{\pgfqpoint{4.368678in}{0.701189in}}%
\pgfpathlineto{\pgfqpoint{4.371246in}{0.768844in}}%
\pgfpathlineto{\pgfqpoint{4.377235in}{0.659766in}}%
\pgfpathlineto{\pgfqpoint{4.378945in}{0.731562in}}%
\pgfpathlineto{\pgfqpoint{4.382361in}{0.679089in}}%
\pgfpathlineto{\pgfqpoint{4.385782in}{0.768844in}}%
\pgfpathlineto{\pgfqpoint{4.390915in}{0.774640in}}%
\pgfpathlineto{\pgfqpoint{4.393482in}{0.662431in}}%
\pgfpathlineto{\pgfqpoint{4.395194in}{0.748260in}}%
\pgfpathlineto{\pgfqpoint{4.398612in}{0.696258in}}%
\pgfpathlineto{\pgfqpoint{4.402032in}{0.721916in}}%
\pgfpathlineto{\pgfqpoint{4.406311in}{0.628962in}}%
\pgfpathlineto{\pgfqpoint{4.408878in}{0.713962in}}%
\pgfpathlineto{\pgfqpoint{4.414012in}{0.753837in}}%
\pgfpathlineto{\pgfqpoint{4.416578in}{0.671929in}}%
\pgfpathlineto{\pgfqpoint{4.419998in}{0.735376in}}%
\pgfpathlineto{\pgfqpoint{4.422562in}{0.690211in}}%
\pgfpathlineto{\pgfqpoint{4.427694in}{0.740019in}}%
\pgfpathlineto{\pgfqpoint{4.430262in}{0.651237in}}%
\pgfpathlineto{\pgfqpoint{4.432826in}{0.784713in}}%
\pgfpathlineto{\pgfqpoint{4.437103in}{0.653288in}}%
\pgfpathlineto{\pgfqpoint{4.442241in}{0.767512in}}%
\pgfpathlineto{\pgfqpoint{4.444808in}{0.679161in}}%
\pgfpathlineto{\pgfqpoint{4.449081in}{0.735878in}}%
\pgfpathlineto{\pgfqpoint{4.449936in}{0.654369in}}%
\pgfpathlineto{\pgfqpoint{4.455924in}{0.773847in}}%
\pgfpathlineto{\pgfqpoint{4.458489in}{0.669412in}}%
\pgfpathlineto{\pgfqpoint{4.461057in}{0.756861in}}%
\pgfpathlineto{\pgfqpoint{4.463622in}{0.688667in}}%
\pgfpathlineto{\pgfqpoint{4.468754in}{0.726380in}}%
\pgfpathlineto{\pgfqpoint{4.473032in}{0.612264in}}%
\pgfpathlineto{\pgfqpoint{4.476450in}{0.748619in}}%
\pgfpathlineto{\pgfqpoint{4.478161in}{0.681431in}}%
\pgfpathlineto{\pgfqpoint{4.480727in}{0.726560in}}%
\pgfpathlineto{\pgfqpoint{4.485001in}{0.646343in}}%
\pgfpathlineto{\pgfqpoint{4.488421in}{0.751643in}}%
\pgfpathlineto{\pgfqpoint{4.491842in}{0.664266in}}%
\pgfpathlineto{\pgfqpoint{4.496121in}{0.733038in}}%
\pgfpathlineto{\pgfqpoint{4.497827in}{0.687730in}}%
\pgfpathlineto{\pgfqpoint{4.502953in}{0.766148in}}%
\pgfpathlineto{\pgfqpoint{4.505519in}{0.681323in}}%
\pgfpathlineto{\pgfqpoint{4.509790in}{0.742213in}}%
\pgfpathlineto{\pgfqpoint{4.512354in}{0.745452in}}%
\pgfpathlineto{\pgfqpoint{4.514921in}{0.664481in}}%
\pgfpathlineto{\pgfqpoint{4.518344in}{0.720225in}}%
\pgfpathlineto{\pgfqpoint{4.523473in}{0.673441in}}%
\pgfpathlineto{\pgfqpoint{4.526897in}{0.747646in}}%
\pgfpathlineto{\pgfqpoint{4.534596in}{0.583798in}}%
\pgfpathlineto{\pgfqpoint{4.535450in}{0.762837in}}%
\pgfpathlineto{\pgfqpoint{4.538868in}{0.585561in}}%
\pgfpathlineto{\pgfqpoint{4.543143in}{0.764775in}}%
\pgfpathlineto{\pgfqpoint{4.548281in}{0.748835in}}%
\pgfpathlineto{\pgfqpoint{4.549136in}{0.669340in}}%
\pgfpathlineto{\pgfqpoint{4.554268in}{0.634108in}}%
\pgfpathlineto{\pgfqpoint{4.558543in}{0.730804in}}%
\pgfpathlineto{\pgfqpoint{4.559399in}{0.666564in}}%
\pgfpathlineto{\pgfqpoint{4.562820in}{0.642777in}}%
\pgfpathlineto{\pgfqpoint{4.568805in}{0.737426in}}%
\pgfpathlineto{\pgfqpoint{4.570513in}{0.612479in}}%
\pgfpathlineto{\pgfqpoint{4.574795in}{0.714537in}}%
\pgfpathlineto{\pgfqpoint{4.579072in}{0.597185in}}%
\pgfpathlineto{\pgfqpoint{4.580783in}{0.682185in}}%
\pgfpathlineto{\pgfqpoint{4.584204in}{0.754268in}}%
\pgfpathlineto{\pgfqpoint{4.588482in}{0.633171in}}%
\pgfpathlineto{\pgfqpoint{4.592755in}{0.801986in}}%
\pgfpathlineto{\pgfqpoint{4.595324in}{0.662140in}}%
\pgfpathlineto{\pgfqpoint{4.597035in}{0.768445in}}%
\pgfpathlineto{\pgfqpoint{4.600455in}{0.691180in}}%
\pgfpathlineto{\pgfqpoint{4.605589in}{0.654293in}}%
\pgfpathlineto{\pgfqpoint{4.609011in}{0.684379in}}%
\pgfpathlineto{\pgfqpoint{4.610723in}{0.791335in}}%
\pgfpathlineto{\pgfqpoint{4.615000in}{0.660412in}}%
\pgfpathlineto{\pgfqpoint{4.617565in}{0.790002in}}%
\pgfpathlineto{\pgfqpoint{4.623549in}{0.757328in}}%
\pgfpathlineto{\pgfqpoint{4.624405in}{0.605786in}}%
\pgfpathlineto{\pgfqpoint{4.627827in}{0.726376in}}%
\pgfpathlineto{\pgfqpoint{4.631246in}{0.763299in}}%
\pgfpathlineto{\pgfqpoint{4.637235in}{0.788455in}}%
\pgfpathlineto{\pgfqpoint{4.638091in}{0.662642in}}%
\pgfpathlineto{\pgfqpoint{4.641512in}{0.745739in}}%
\pgfpathlineto{\pgfqpoint{4.644931in}{0.769562in}}%
\pgfpathlineto{\pgfqpoint{4.648350in}{0.704859in}}%
\pgfpathlineto{\pgfqpoint{4.654343in}{0.737354in}}%
\pgfpathlineto{\pgfqpoint{4.657762in}{0.618958in}}%
\pgfpathlineto{\pgfqpoint{4.660326in}{0.717600in}}%
\pgfpathlineto{\pgfqpoint{4.664605in}{0.731203in}}%
\pgfpathlineto{\pgfqpoint{4.666317in}{0.652354in}}%
\pgfpathlineto{\pgfqpoint{4.671449in}{0.613309in}}%
\pgfpathlineto{\pgfqpoint{4.672305in}{0.690394in}}%
\pgfpathlineto{\pgfqpoint{4.677437in}{0.749808in}}%
\pgfpathlineto{\pgfqpoint{4.681713in}{0.671390in}}%
\pgfpathlineto{\pgfqpoint{4.683423in}{0.607190in}}%
\pgfpathlineto{\pgfqpoint{4.686845in}{0.753837in}}%
\pgfpathlineto{\pgfqpoint{4.689413in}{0.702266in}}%
\pgfpathlineto{\pgfqpoint{4.695398in}{0.696833in}}%
\pgfpathlineto{\pgfqpoint{4.696255in}{0.777157in}}%
\pgfpathlineto{\pgfqpoint{4.700535in}{0.659439in}}%
\pgfpathlineto{\pgfqpoint{4.704809in}{0.740019in}}%
\pgfpathlineto{\pgfqpoint{4.710801in}{0.658904in}}%
\pgfpathlineto{\pgfqpoint{4.713368in}{0.768054in}}%
\pgfpathlineto{\pgfqpoint{4.717644in}{0.826387in}}%
\pgfpathlineto{\pgfqpoint{4.720210in}{0.686936in}}%
\pgfpathlineto{\pgfqpoint{4.724492in}{0.766610in}}%
\pgfpathlineto{\pgfqpoint{4.727918in}{0.689493in}}%
\pgfpathlineto{\pgfqpoint{4.731339in}{0.762653in}}%
\pgfpathlineto{\pgfqpoint{4.734766in}{0.637706in}}%
\pgfpathlineto{\pgfqpoint{4.739903in}{0.792883in}}%
\pgfpathlineto{\pgfqpoint{4.742472in}{0.703383in}}%
\pgfpathlineto{\pgfqpoint{4.746750in}{0.774062in}}%
\pgfpathlineto{\pgfqpoint{4.748462in}{0.688376in}}%
\pgfpathlineto{\pgfqpoint{4.751882in}{0.785973in}}%
\pgfpathlineto{\pgfqpoint{4.756154in}{0.680565in}}%
\pgfpathlineto{\pgfqpoint{4.760428in}{0.784857in}}%
\pgfpathlineto{\pgfqpoint{4.762995in}{0.667609in}}%
\pgfpathlineto{\pgfqpoint{4.765556in}{0.769490in}}%
\pgfpathlineto{\pgfqpoint{4.768121in}{0.710364in}}%
\pgfpathlineto{\pgfqpoint{4.774110in}{0.730625in}}%
\pgfpathlineto{\pgfqpoint{4.774966in}{0.671749in}}%
\pgfpathlineto{\pgfqpoint{4.779249in}{0.748472in}}%
\pgfpathlineto{\pgfqpoint{4.782675in}{0.649331in}}%
\pgfpathlineto{\pgfqpoint{4.787806in}{0.650049in}}%
\pgfpathlineto{\pgfqpoint{4.790374in}{0.743003in}}%
\pgfpathlineto{\pgfqpoint{4.792084in}{0.638496in}}%
\pgfpathlineto{\pgfqpoint{4.795506in}{0.697802in}}%
\pgfpathlineto{\pgfqpoint{4.795506in}{0.697802in}}%
\pgfusepath{stroke}%
\end{pgfscope}%
\begin{pgfscope}%
\pgfsetrectcap%
\pgfsetmiterjoin%
\pgfsetlinewidth{0.803000pt}%
\definecolor{currentstroke}{rgb}{0.000000,0.000000,0.000000}%
\pgfsetstrokecolor{currentstroke}%
\pgfsetdash{}{0pt}%
\pgfpathmoveto{\pgfqpoint{0.484581in}{0.539544in}}%
\pgfpathlineto{\pgfqpoint{0.484581in}{1.114166in}}%
\pgfusepath{stroke}%
\end{pgfscope}%
\begin{pgfscope}%
\pgfsetrectcap%
\pgfsetmiterjoin%
\pgfsetlinewidth{0.803000pt}%
\definecolor{currentstroke}{rgb}{0.000000,0.000000,0.000000}%
\pgfsetstrokecolor{currentstroke}%
\pgfsetdash{}{0pt}%
\pgfpathmoveto{\pgfqpoint{5.000788in}{0.539544in}}%
\pgfpathlineto{\pgfqpoint{5.000788in}{1.114166in}}%
\pgfusepath{stroke}%
\end{pgfscope}%
\begin{pgfscope}%
\pgfsetrectcap%
\pgfsetmiterjoin%
\pgfsetlinewidth{0.803000pt}%
\definecolor{currentstroke}{rgb}{0.000000,0.000000,0.000000}%
\pgfsetstrokecolor{currentstroke}%
\pgfsetdash{}{0pt}%
\pgfpathmoveto{\pgfqpoint{0.484581in}{0.539544in}}%
\pgfpathlineto{\pgfqpoint{5.000788in}{0.539544in}}%
\pgfusepath{stroke}%
\end{pgfscope}%
\begin{pgfscope}%
\pgfsetrectcap%
\pgfsetmiterjoin%
\pgfsetlinewidth{0.803000pt}%
\definecolor{currentstroke}{rgb}{0.000000,0.000000,0.000000}%
\pgfsetstrokecolor{currentstroke}%
\pgfsetdash{}{0pt}%
\pgfpathmoveto{\pgfqpoint{0.484581in}{1.114166in}}%
\pgfpathlineto{\pgfqpoint{5.000788in}{1.114166in}}%
\pgfusepath{stroke}%
\end{pgfscope}%
\end{pgfpicture}%
\makeatother%
\endgroup%

    \includegraphics[width=0.75\textwidth]{example-image-golden}
    \caption{Voltage noise of an LM399, measured with a bandwidth of \qtyrange{0.1}{10}{\Hz}.}
    \label{fig:noise_lm399}
\end{figure}

Measuring two references against each other would then result in around \qty{2.1}{\micro \volt} of noise. This make distinguishing the jumps possible, but challenging.

A third option is to use a high-pass filter and an amplifier. Additionally, the signal can be low-pass filtered to remove any excess high frequency noise. This approach also requires less resolution than directly measuring the voltage, because the signal-to-noise-ratio is improved du the amplifier. It is therefore possible to use an off-the-shelf analog-to-digital converter (ADC). One such circuit, along with some examples, is demonstrated in \cite{technote_ti_popcorn_noise,kay2012operational}. It must be noted, that due to the high-pass filtering, it not possible to measure slow voltage drifts using this method.

The forth and final option presented here, is approaching the problem in the frequency domain and requires a low-noise amplifier with a low frequency cutoff. As it was already discussed in section \ref{}, popcorn noise is found to have a frequency dependence of $1/f^2$. This can be used to distinguish it from other random noise processes that show a frequency dependence of $1/f$. An good example is given in Horowitz and Hill in \textit{The Art of Electronics} on page 478 \cite{horowitz1989}. Going to frequencies below \qty{10}{\Hz}, one can sort the references by their noise spectrum.

In this work only options one and two were were tested, as it was said above, with options three and four there is a chicken and egg problem. One needs a number on known good devices to compare other DUTs to. At the start of the evaluation, most of the data available about the LM399 was from the data sheet. Compiling a dataset of the performance of dozens of LM399 is expensive and time consuming and companies typically treat such data as a closely guarded secret.

The next section deals with the choice of multimeter to satisfy the requirements test the Zener diodes according to options one and two, so either directly measuring the output voltage or difference of a known good sample against the DUT.

\subsection{Choosing a Multimeter for Testing Zener Diodes}
The DMM used plays an important role for the test setup. In this section, some of the challenges, that can be encountered will be discussed. The expected amplitude of the popcorn noise is around \qty[per-mode=symbol]{0.5}{\micro\volt \per \volt} or \qty{3.5}{\micro\volt} of the output voltage, when considering the \qty{7}{\volt} Zener voltage of the LM399 diode.

The \qty{7}{\volt} will typically be measured on the \qty{10}{\volt} range. It is not a trivial task, because a signal-to-noise-ratio of \qty[per-mode=symbol]{0.35}{\micro\volt \per \volt} or more than \qty{130}{\decibel} is required. This calls for a device, that not only has the required resolution, but also the stability over time and temperature to ensure the measurement will not be distorted by the DMM.

Therefore, a voltmeter with lower noise and a more stable reference, than the DUT is mandatory. This only leaves the class of very low noise \num{7.5} or \num{8.5} digit multimeters. These multimeters feature a different type of voltage reference, because the LM399 is not suitable due to its noise. The only Zener diodes that meet those requirements are the Analog Devices LTZ1000 \cite{datasheet_LTZ1000}, the Motorola SZA263 (out of production) and the Linear Technology (LT) LTFLU-1, a proprietary design by Fluke and LT. The LTZ1000, for example, is specified for a typical noise of \qty{1.2}{\micro\volt_{pp}} in a frequency range of \qtyrange{0.1}{10}{\Hz} \cite{datasheet_LTZ1000}. Additionally, in comparison to the LM399, those Zener diodes do not suffer from the popcorn noise issue.

The equipment manufacturers typically have a preference for one of those diodes. Keysight utilizes the LTZ1000, Fluke uses the SZA263 (in older devices) or the LTFLU-1 in newer model, while Keithley employs the LTZ1000 in their \device{Model 2002} and the LTFLU-1 in the newer \device{DMM7510}, because they were bought by Fortive, the same company that owns Fluke. To sum it it up, Keysight uses the LTZ1000 and Fluke/Keithley the LTFLU-1 in their top end meters.

Comparing only \num{7.5} and \num{8.5} digit voltmeters, narrows down the choice of multimeters considerably. The market for high-end \num{8.5} digit DMMs is limited and therefore every device on the market caters for a certain niche. It is therefore prudent to look at their specifications to choose the correct device for this purpose. In table \ref{tab:list_of_dmms} a list of popular \num{8.5} DMMs can be found. Several models included in the table, are already discontinued, but these DMMs can still be acquired on the second-hand market.

\begin{table}[h]
    \centering
    \begin{tabular}{ |l|l|l| }
        \hline
        Manufacturer & Model & Remarks \\
        \hline
        Advantest & \device{R6581} & Discontinued. Scanner cards available. \\
        Datron/Wavetek & \device{1812} & Discontinued. Wavetek was bought by Fluke. \\
        Fluke & \device{8508A} & Discontinued. \qty{20}{\volt} range. \\
        Fluke & \device{8588A} & In production. \\
        Keithley/Tektronix & \device{2002} & In production. Scanner card available. \qty{20}{\volt} range. \\
        Keysight & \device{3458A} & In Production. \\
        Solartron & \device{7081} & Discontinued. Slow. \\
        Transmille & \device{8104} & In Production. External scanner available. Slow. \\
        \hline
    \end{tabular}
    \caption{Overview of \num{8.5} digit multimeters.}
    \label{tab:list_of_dmms}
\end{table}

While the author has not tested every multimeter in table \ref{tab:list_of_dmms}, it is possible to judge some of them apriori by their specifications. The \device{Solartron 7081} (also sold as \device{Guildline 9578}) is a less optimal choice, because a conversion takes \qty{52}{\s} for \num{8.5} digits. The discontinued \device{Fluke 8508A} and the \device{Wavetek 1812} multimeter are very similar devices, because Fluke bought Wavetek in 2000 and as a result, the \device{Fluke 8508A} is more of an update to the \device{Wavetek 1812} than a new device. They are both in included in the list, because it is very rare to see one of the Fluke devices on the second hand market, while the \device{Wavetek 1812} can be found with a bit of patience. Again they are fairly slow, taking \qty{25}{\second} for a conversion at \num{8.5} digits.

The other multimeters are still in production and similar in price, but their field of use is slightly is different. The \device{Fluke 8588A} excels at stability and features a modern user interface, whereas the \device{Keysight 3458A} is unbeaten in linearity and noise. A detailed comparison of those two meters can be found in the work of \citeauthor*{article_fluke_8588A_noise} \cite{article_fluke_8588A_noise}. The \device{Keithley Model 2002} focuses on its scanning capability and the \device{Transmille 8104} does have electrometer functions. Unfortunately, the \device{8104} is also fairly slow at \num{8.5} digit with conversions taking \qty{4}{\s} at its fastest setting \cite{datasheet_transmille8104}, so it will not be considered.

To narrow it down even further, several \num{7.5} and \num{8.5} digit multimeters were tested. The results of those tests will be discussed here to give an impression of the performance of these devices. The tested multimeters are the \device{Keysight 3458A}, the \device{Keithley Model 2002}, the \device{Keysight 34470A} and a \device{Keithley DMM6500}. The \device{3458A} was chosen, because it is very fast and already used in section \ref{} of this work. The \device{Model 2002} was chosen for its internal scanning unit. The \device{34470A} was chosen as a lower-end and cheaper alternative and because it is a fairly low noise device. Finally the \device{DMM6500} is on the list to compare a DMM with an LM399 reference. A \device{Fluke 8588A} was not tested, because it was not released at the time of testing and the older model \device{8508A} is considered too slow as mentioned above.

\minisec{The tests}

Two test were run on this selection of devices. The first one was done using a \device{Fluke 5440B} calibrator supplying \qty{10}{\volt} to all mulimeters and taking readings over the course of a week. This data was used to estimate the noise and the stability of the multimeters, including burst noise. The noise of the DMM at \qty{10}{\volt} is typically not found in the datasheet, because the noise performance is usually quoted for shorted inputs, which does not include the internal reference noise. This test allows to check for popcorn noise of the internal reference. The calibrator has a specified output noise of \qty{< 1.5}{\micro \volt} within a bandwidth of \qtyrange{0.1}{10}{\Hz} at \qty{1}{\volt} and is stable to within \qty{5}{\micro \volt_{RMS}} over \qty{30}{\day}, a specification far superior to the LM399.

The second test was done using a known bad LM399 voltage reference instead of the calibrator. This test was done to see how well a DMM can make out the popcorn noise.

Based on these two tests, a multimeter was chosen for an automated test setup to bin the LM399s.

\minisec{Test Setup}

The tests were done in a stable and monitored lab environment, with a temperature deviation of at most $\Delta T = \qty{\pm 0.2}{\kelvin}$. All multimeters were connected to the same DUT. Although this might potentially cause interference between the multimeters due to the pump out current spikes caused by the switching interals, no ill effects, like voltage offsets or increased noise, were observed during the setup of the tests. A more detailed discussion of the pump out current of the \device{3458A} can be found in \cite{article_3458A_input_mpedance}.

The three \num{8.5} and \num{7.5} digit multimeters were connected using shielded cables, either Pomona 1167-60 or self-made cables. See section \ref{} for details on the self-made cables. The GUARD terminal of the calibrator was connected to chassis GROUND at the calibrator and then connected to the cable shield. On the \device{3458A}, the shield was connected to the GUARD terminal and the GUARD switch was set to open according to the manual \cite{manual_keysight3458a}. For the other multimeters, that do not have a GUARD terminal, the shield was left floating at the DMM side. Additionally the \device{Fluke 5440B}, the \device{HP 3458A} and the \device{Keysight 34470A} have an autocalibration routine, which was run once prior to the measurement. The detailed settings used for the DMMs can be found in the appendix \ref{appendix:dmm_test} on page \pageref{appendix:dmm_test}, a summary ist given in table \ref{tab:dmm_settings_concise} to show the important differences.

\begin{table}[ht]
    \centering
    \begin{tabular}{lcc}
        \toprule
        DMM& Integration time in \unit{NPLC}& Conversion time in \unit{\s}\\
        \midrule
        \device{HP 3458A}& 100 & \qty{0}{\s}\\
        \device{Keithley Model 2002} & 40& \qty{0}{\s}\\
        \device{Keysight 34470A}& 100    & \qty{0}{\s}\\
        \device{Keithley DMM6500}& 90& \qty{0}{\s}\\
        \bottomrule
    \end{tabular}
    \caption{Concise list of differences in the settings used for comparing the DMMs.}
    \label{tab:dmm_settings_concise}
\end{table}

All DMMs were configured to have a similar conversion time. This lead to different integration times, which are given in power line cycles at \qty{50}{\Hz}. The \device{Model 2002} takes considerable longer for a measurement than the Keysight multimeters. The reason is the auto-zero function, which is shown in figure \ref{dmm_autozero_comparison}. The \device{Model 2002} does three steps when doing auto-zeroing, it measures the signal, the zero point for an offset compensation and also the reference voltage for a gain correction. In comparison, the \device{3458A} only corrects for the offset drift. The gain is adjusted when using the ACAL function. The former auto-zero routine, therefore takes longer by one half, but results in more stable measurements.

\begin{figure}[ht]
    \centering
    %\resizebox {0.8\textwidth} {!} {
        \import{figures/}{dmm_autozero.tex}
    %} % resizebox
    \label{fig:dmm_autozero_comparison}
    \caption{Auto-zero phases of the \device{HP 3458A} and \device{Keithley Model 2002}.}
\end{figure}

These measurements were done by measuring the output voltage of a pre-production version of the reference PCB for the digital current driver. The reference board was kept at \qty{23}{\celsius} in a custom thermal chamber. The chamber is detailed in section \ref{}. Additionally, a \qty{500}{\g}  bag of Bentonite desiccant was added to keep the references at a low humidity of around \qty{20}{\percent} relative humidity. The reference board inserted into a motherboard holding up to 4 reference modules. The motherboard, also called LM399 breakout board, provides the voltage regulators and the operational amplifier for the kelvin sensed pins of the reference. The multimeter was directly connected to the reference via a DB9 connector, without an other components in between the reference and the DMM like buffers, multiplexers or filters. The DMM itself was exposed to the ambient temperature of the lab. The setup is shown in figure \ref{fig:lm399_vs_34470a_setup}.

\begin{figure}[ht]
    \centering
    \resizebox {0.8\textwidth} {!} {
        \import{figures/}{34470A_vs_LM399.tex}
    } % resizebox
    \caption{Measurement setup for tesing an LM399 reference board with the \device{Keysight 34470A}}
    \label{fig:lm399_vs_34470a_setup}
\end{figure}

The reference boards amplify the Zener voltage to \qty{10}{\volt}, which improves the signal to noise ratio, because it makes use of the full DMM range. The \qty{10}{\volt} range is typically the lowest (relative) noise and lowest drift range those multimeter because no internal pre-amplifiers of attenuators are required. It is important to keep the temperature drift of the DMM low or at least predictable, because the device is exposed to the ambient laboratory and not in a temperature controlled environment like the references.

\newpage
The reference is a negative \qty{10}{\volt} reference that uses a self-biasing technique to derive its \qty{1}{\mA} Zener current from its own \qty{-10}{\volt} output. The details of this circuit are discussed in section \ref{}.

% \begin{figure}[h]
%     \centering
%     \scalebox{0.7}{%
%         \import{figures/}{lm399_reference_circuit.tex}
%     } % scalebox
%     \caption{Self-biased LM399 negative voltage reference.}
%     \label{fig:lm399_negative_10V}
% \end{figure}

With the amplified output the expected burst noise step size of about \qty[per-mode=symbol]{0.5}{\micro\volt \per \volt}, becomes \qty{5}{\micro\volt}. The resolution of the \qty{10}{\volt} range of the \device{34470A} is \qty{100}{\nano \volt}, but the measurement is not limited by quantization. See section \ref{} of this work for a detailed characterization.

\begin{figure}[ht]
    \centering
    %% Creator: Matplotlib, PGF backend
%%
%% To include the figure in your LaTeX document, write
%%   \input{<filename>.pgf}
%%
%% Make sure the required packages are loaded in your preamble
%%   \usepackage{pgf}
%%
%% Also ensure that all the required font packages are loaded; for instance,
%% the lmodern package is sometimes necessary when using math font.
%%   \usepackage{lmodern}
%%
%% Figures using additional raster images can only be included by \input if
%% they are in the same directory as the main LaTeX file. For loading figures
%% from other directories you can use the `import` package
%%   \usepackage{import}
%%
%% and then include the figures with
%%   \import{<path to file>}{<filename>.pgf}
%%
%% Matplotlib used the following preamble
%%   \usepackage{fontspec}
%%
\begingroup%
\makeatletter%
\begin{pgfpicture}%
\pgfpathrectangle{\pgfpointorigin}{\pgfqpoint{5.200000in}{3.210000in}}%
\pgfusepath{use as bounding box, clip}%
\begin{pgfscope}%
\pgfsetbuttcap%
\pgfsetmiterjoin%
\definecolor{currentfill}{rgb}{1.000000,1.000000,1.000000}%
\pgfsetfillcolor{currentfill}%
\pgfsetlinewidth{0.000000pt}%
\definecolor{currentstroke}{rgb}{1.000000,1.000000,1.000000}%
\pgfsetstrokecolor{currentstroke}%
\pgfsetdash{}{0pt}%
\pgfpathmoveto{\pgfqpoint{0.000000in}{0.000000in}}%
\pgfpathlineto{\pgfqpoint{5.200000in}{0.000000in}}%
\pgfpathlineto{\pgfqpoint{5.200000in}{3.210000in}}%
\pgfpathlineto{\pgfqpoint{0.000000in}{3.210000in}}%
\pgfpathlineto{\pgfqpoint{0.000000in}{0.000000in}}%
\pgfpathclose%
\pgfusepath{fill}%
\end{pgfscope}%
\begin{pgfscope}%
\pgfsetbuttcap%
\pgfsetmiterjoin%
\definecolor{currentfill}{rgb}{1.000000,1.000000,1.000000}%
\pgfsetfillcolor{currentfill}%
\pgfsetlinewidth{0.000000pt}%
\definecolor{currentstroke}{rgb}{0.000000,0.000000,0.000000}%
\pgfsetstrokecolor{currentstroke}%
\pgfsetstrokeopacity{0.000000}%
\pgfsetdash{}{0pt}%
\pgfpathmoveto{\pgfqpoint{0.633813in}{0.538014in}}%
\pgfpathlineto{\pgfqpoint{4.507477in}{0.538014in}}%
\pgfpathlineto{\pgfqpoint{4.507477in}{2.936535in}}%
\pgfpathlineto{\pgfqpoint{0.633813in}{2.936535in}}%
\pgfpathlineto{\pgfqpoint{0.633813in}{0.538014in}}%
\pgfpathclose%
\pgfusepath{fill}%
\end{pgfscope}%
\begin{pgfscope}%
\pgfsetbuttcap%
\pgfsetroundjoin%
\definecolor{currentfill}{rgb}{0.000000,0.000000,0.000000}%
\pgfsetfillcolor{currentfill}%
\pgfsetlinewidth{0.803000pt}%
\definecolor{currentstroke}{rgb}{0.000000,0.000000,0.000000}%
\pgfsetstrokecolor{currentstroke}%
\pgfsetdash{}{0pt}%
\pgfsys@defobject{currentmarker}{\pgfqpoint{0.000000in}{-0.048611in}}{\pgfqpoint{0.000000in}{0.000000in}}{%
\pgfpathmoveto{\pgfqpoint{0.000000in}{0.000000in}}%
\pgfpathlineto{\pgfqpoint{0.000000in}{-0.048611in}}%
\pgfusepath{stroke,fill}%
}%
\begin{pgfscope}%
\pgfsys@transformshift{0.809808in}{0.538014in}%
\pgfsys@useobject{currentmarker}{}%
\end{pgfscope}%
\end{pgfscope}%
\begin{pgfscope}%
\definecolor{textcolor}{rgb}{0.000000,0.000000,0.000000}%
\pgfsetstrokecolor{textcolor}%
\pgfsetfillcolor{textcolor}%
\pgftext[x=0.809808in,y=0.440792in,,top]{\color{textcolor}\rmfamily\fontsize{8.000000}{9.600000}\selectfont \(\displaystyle {00{:}00}\)}%
\end{pgfscope}%
\begin{pgfscope}%
\pgfsetbuttcap%
\pgfsetroundjoin%
\definecolor{currentfill}{rgb}{0.000000,0.000000,0.000000}%
\pgfsetfillcolor{currentfill}%
\pgfsetlinewidth{0.803000pt}%
\definecolor{currentstroke}{rgb}{0.000000,0.000000,0.000000}%
\pgfsetstrokecolor{currentstroke}%
\pgfsetdash{}{0pt}%
\pgfsys@defobject{currentmarker}{\pgfqpoint{0.000000in}{-0.048611in}}{\pgfqpoint{0.000000in}{0.000000in}}{%
\pgfpathmoveto{\pgfqpoint{0.000000in}{0.000000in}}%
\pgfpathlineto{\pgfqpoint{0.000000in}{-0.048611in}}%
\pgfusepath{stroke,fill}%
}%
\begin{pgfscope}%
\pgfsys@transformshift{1.250023in}{0.538014in}%
\pgfsys@useobject{currentmarker}{}%
\end{pgfscope}%
\end{pgfscope}%
\begin{pgfscope}%
\definecolor{textcolor}{rgb}{0.000000,0.000000,0.000000}%
\pgfsetstrokecolor{textcolor}%
\pgfsetfillcolor{textcolor}%
\pgftext[x=1.250023in,y=0.440792in,,top]{\color{textcolor}\rmfamily\fontsize{8.000000}{9.600000}\selectfont \(\displaystyle {03{:}00}\)}%
\end{pgfscope}%
\begin{pgfscope}%
\pgfsetbuttcap%
\pgfsetroundjoin%
\definecolor{currentfill}{rgb}{0.000000,0.000000,0.000000}%
\pgfsetfillcolor{currentfill}%
\pgfsetlinewidth{0.803000pt}%
\definecolor{currentstroke}{rgb}{0.000000,0.000000,0.000000}%
\pgfsetstrokecolor{currentstroke}%
\pgfsetdash{}{0pt}%
\pgfsys@defobject{currentmarker}{\pgfqpoint{0.000000in}{-0.048611in}}{\pgfqpoint{0.000000in}{0.000000in}}{%
\pgfpathmoveto{\pgfqpoint{0.000000in}{0.000000in}}%
\pgfpathlineto{\pgfqpoint{0.000000in}{-0.048611in}}%
\pgfusepath{stroke,fill}%
}%
\begin{pgfscope}%
\pgfsys@transformshift{1.690237in}{0.538014in}%
\pgfsys@useobject{currentmarker}{}%
\end{pgfscope}%
\end{pgfscope}%
\begin{pgfscope}%
\definecolor{textcolor}{rgb}{0.000000,0.000000,0.000000}%
\pgfsetstrokecolor{textcolor}%
\pgfsetfillcolor{textcolor}%
\pgftext[x=1.690237in,y=0.440792in,,top]{\color{textcolor}\rmfamily\fontsize{8.000000}{9.600000}\selectfont \(\displaystyle {06{:}00}\)}%
\end{pgfscope}%
\begin{pgfscope}%
\pgfsetbuttcap%
\pgfsetroundjoin%
\definecolor{currentfill}{rgb}{0.000000,0.000000,0.000000}%
\pgfsetfillcolor{currentfill}%
\pgfsetlinewidth{0.803000pt}%
\definecolor{currentstroke}{rgb}{0.000000,0.000000,0.000000}%
\pgfsetstrokecolor{currentstroke}%
\pgfsetdash{}{0pt}%
\pgfsys@defobject{currentmarker}{\pgfqpoint{0.000000in}{-0.048611in}}{\pgfqpoint{0.000000in}{0.000000in}}{%
\pgfpathmoveto{\pgfqpoint{0.000000in}{0.000000in}}%
\pgfpathlineto{\pgfqpoint{0.000000in}{-0.048611in}}%
\pgfusepath{stroke,fill}%
}%
\begin{pgfscope}%
\pgfsys@transformshift{2.130452in}{0.538014in}%
\pgfsys@useobject{currentmarker}{}%
\end{pgfscope}%
\end{pgfscope}%
\begin{pgfscope}%
\definecolor{textcolor}{rgb}{0.000000,0.000000,0.000000}%
\pgfsetstrokecolor{textcolor}%
\pgfsetfillcolor{textcolor}%
\pgftext[x=2.130452in,y=0.440792in,,top]{\color{textcolor}\rmfamily\fontsize{8.000000}{9.600000}\selectfont \(\displaystyle {09{:}00}\)}%
\end{pgfscope}%
\begin{pgfscope}%
\pgfsetbuttcap%
\pgfsetroundjoin%
\definecolor{currentfill}{rgb}{0.000000,0.000000,0.000000}%
\pgfsetfillcolor{currentfill}%
\pgfsetlinewidth{0.803000pt}%
\definecolor{currentstroke}{rgb}{0.000000,0.000000,0.000000}%
\pgfsetstrokecolor{currentstroke}%
\pgfsetdash{}{0pt}%
\pgfsys@defobject{currentmarker}{\pgfqpoint{0.000000in}{-0.048611in}}{\pgfqpoint{0.000000in}{0.000000in}}{%
\pgfpathmoveto{\pgfqpoint{0.000000in}{0.000000in}}%
\pgfpathlineto{\pgfqpoint{0.000000in}{-0.048611in}}%
\pgfusepath{stroke,fill}%
}%
\begin{pgfscope}%
\pgfsys@transformshift{2.570666in}{0.538014in}%
\pgfsys@useobject{currentmarker}{}%
\end{pgfscope}%
\end{pgfscope}%
\begin{pgfscope}%
\definecolor{textcolor}{rgb}{0.000000,0.000000,0.000000}%
\pgfsetstrokecolor{textcolor}%
\pgfsetfillcolor{textcolor}%
\pgftext[x=2.570666in,y=0.440792in,,top]{\color{textcolor}\rmfamily\fontsize{8.000000}{9.600000}\selectfont \(\displaystyle {12{:}00}\)}%
\end{pgfscope}%
\begin{pgfscope}%
\pgfsetbuttcap%
\pgfsetroundjoin%
\definecolor{currentfill}{rgb}{0.000000,0.000000,0.000000}%
\pgfsetfillcolor{currentfill}%
\pgfsetlinewidth{0.803000pt}%
\definecolor{currentstroke}{rgb}{0.000000,0.000000,0.000000}%
\pgfsetstrokecolor{currentstroke}%
\pgfsetdash{}{0pt}%
\pgfsys@defobject{currentmarker}{\pgfqpoint{0.000000in}{-0.048611in}}{\pgfqpoint{0.000000in}{0.000000in}}{%
\pgfpathmoveto{\pgfqpoint{0.000000in}{0.000000in}}%
\pgfpathlineto{\pgfqpoint{0.000000in}{-0.048611in}}%
\pgfusepath{stroke,fill}%
}%
\begin{pgfscope}%
\pgfsys@transformshift{3.010881in}{0.538014in}%
\pgfsys@useobject{currentmarker}{}%
\end{pgfscope}%
\end{pgfscope}%
\begin{pgfscope}%
\definecolor{textcolor}{rgb}{0.000000,0.000000,0.000000}%
\pgfsetstrokecolor{textcolor}%
\pgfsetfillcolor{textcolor}%
\pgftext[x=3.010881in,y=0.440792in,,top]{\color{textcolor}\rmfamily\fontsize{8.000000}{9.600000}\selectfont \(\displaystyle {15{:}00}\)}%
\end{pgfscope}%
\begin{pgfscope}%
\pgfsetbuttcap%
\pgfsetroundjoin%
\definecolor{currentfill}{rgb}{0.000000,0.000000,0.000000}%
\pgfsetfillcolor{currentfill}%
\pgfsetlinewidth{0.803000pt}%
\definecolor{currentstroke}{rgb}{0.000000,0.000000,0.000000}%
\pgfsetstrokecolor{currentstroke}%
\pgfsetdash{}{0pt}%
\pgfsys@defobject{currentmarker}{\pgfqpoint{0.000000in}{-0.048611in}}{\pgfqpoint{0.000000in}{0.000000in}}{%
\pgfpathmoveto{\pgfqpoint{0.000000in}{0.000000in}}%
\pgfpathlineto{\pgfqpoint{0.000000in}{-0.048611in}}%
\pgfusepath{stroke,fill}%
}%
\begin{pgfscope}%
\pgfsys@transformshift{3.451096in}{0.538014in}%
\pgfsys@useobject{currentmarker}{}%
\end{pgfscope}%
\end{pgfscope}%
\begin{pgfscope}%
\definecolor{textcolor}{rgb}{0.000000,0.000000,0.000000}%
\pgfsetstrokecolor{textcolor}%
\pgfsetfillcolor{textcolor}%
\pgftext[x=3.451096in,y=0.440792in,,top]{\color{textcolor}\rmfamily\fontsize{8.000000}{9.600000}\selectfont \(\displaystyle {18{:}00}\)}%
\end{pgfscope}%
\begin{pgfscope}%
\pgfsetbuttcap%
\pgfsetroundjoin%
\definecolor{currentfill}{rgb}{0.000000,0.000000,0.000000}%
\pgfsetfillcolor{currentfill}%
\pgfsetlinewidth{0.803000pt}%
\definecolor{currentstroke}{rgb}{0.000000,0.000000,0.000000}%
\pgfsetstrokecolor{currentstroke}%
\pgfsetdash{}{0pt}%
\pgfsys@defobject{currentmarker}{\pgfqpoint{0.000000in}{-0.048611in}}{\pgfqpoint{0.000000in}{0.000000in}}{%
\pgfpathmoveto{\pgfqpoint{0.000000in}{0.000000in}}%
\pgfpathlineto{\pgfqpoint{0.000000in}{-0.048611in}}%
\pgfusepath{stroke,fill}%
}%
\begin{pgfscope}%
\pgfsys@transformshift{3.891310in}{0.538014in}%
\pgfsys@useobject{currentmarker}{}%
\end{pgfscope}%
\end{pgfscope}%
\begin{pgfscope}%
\definecolor{textcolor}{rgb}{0.000000,0.000000,0.000000}%
\pgfsetstrokecolor{textcolor}%
\pgfsetfillcolor{textcolor}%
\pgftext[x=3.891310in,y=0.440792in,,top]{\color{textcolor}\rmfamily\fontsize{8.000000}{9.600000}\selectfont \(\displaystyle {21{:}00}\)}%
\end{pgfscope}%
\begin{pgfscope}%
\pgfsetbuttcap%
\pgfsetroundjoin%
\definecolor{currentfill}{rgb}{0.000000,0.000000,0.000000}%
\pgfsetfillcolor{currentfill}%
\pgfsetlinewidth{0.803000pt}%
\definecolor{currentstroke}{rgb}{0.000000,0.000000,0.000000}%
\pgfsetstrokecolor{currentstroke}%
\pgfsetdash{}{0pt}%
\pgfsys@defobject{currentmarker}{\pgfqpoint{0.000000in}{-0.048611in}}{\pgfqpoint{0.000000in}{0.000000in}}{%
\pgfpathmoveto{\pgfqpoint{0.000000in}{0.000000in}}%
\pgfpathlineto{\pgfqpoint{0.000000in}{-0.048611in}}%
\pgfusepath{stroke,fill}%
}%
\begin{pgfscope}%
\pgfsys@transformshift{4.331525in}{0.538014in}%
\pgfsys@useobject{currentmarker}{}%
\end{pgfscope}%
\end{pgfscope}%
\begin{pgfscope}%
\definecolor{textcolor}{rgb}{0.000000,0.000000,0.000000}%
\pgfsetstrokecolor{textcolor}%
\pgfsetfillcolor{textcolor}%
\pgftext[x=4.331525in,y=0.440792in,,top]{\color{textcolor}\rmfamily\fontsize{8.000000}{9.600000}\selectfont \(\displaystyle {00{:}00}\)}%
\end{pgfscope}%
\begin{pgfscope}%
\definecolor{textcolor}{rgb}{0.000000,0.000000,0.000000}%
\pgfsetstrokecolor{textcolor}%
\pgfsetfillcolor{textcolor}%
\pgftext[x=2.570645in,y=0.286570in,,top]{\color{textcolor}\rmfamily\fontsize{10.000000}{12.000000}\selectfont Time (UTC)}%
\end{pgfscope}%
\begin{pgfscope}%
\pgfsetbuttcap%
\pgfsetroundjoin%
\definecolor{currentfill}{rgb}{0.000000,0.000000,0.000000}%
\pgfsetfillcolor{currentfill}%
\pgfsetlinewidth{0.803000pt}%
\definecolor{currentstroke}{rgb}{0.000000,0.000000,0.000000}%
\pgfsetstrokecolor{currentstroke}%
\pgfsetdash{}{0pt}%
\pgfsys@defobject{currentmarker}{\pgfqpoint{-0.048611in}{0.000000in}}{\pgfqpoint{-0.000000in}{0.000000in}}{%
\pgfpathmoveto{\pgfqpoint{-0.000000in}{0.000000in}}%
\pgfpathlineto{\pgfqpoint{-0.048611in}{0.000000in}}%
\pgfusepath{stroke,fill}%
}%
\begin{pgfscope}%
\pgfsys@transformshift{0.633813in}{0.744152in}%
\pgfsys@useobject{currentmarker}{}%
\end{pgfscope}%
\end{pgfscope}%
\begin{pgfscope}%
\definecolor{textcolor}{rgb}{0.000000,0.000000,0.000000}%
\pgfsetstrokecolor{textcolor}%
\pgfsetfillcolor{textcolor}%
\pgftext[x=0.326711in, y=0.705596in, left, base]{\color{textcolor}\rmfamily\fontsize{8.000000}{9.600000}\selectfont \(\displaystyle {\ensuremath{-}15}\)}%
\end{pgfscope}%
\begin{pgfscope}%
\pgfsetbuttcap%
\pgfsetroundjoin%
\definecolor{currentfill}{rgb}{0.000000,0.000000,0.000000}%
\pgfsetfillcolor{currentfill}%
\pgfsetlinewidth{0.803000pt}%
\definecolor{currentstroke}{rgb}{0.000000,0.000000,0.000000}%
\pgfsetstrokecolor{currentstroke}%
\pgfsetdash{}{0pt}%
\pgfsys@defobject{currentmarker}{\pgfqpoint{-0.048611in}{0.000000in}}{\pgfqpoint{-0.000000in}{0.000000in}}{%
\pgfpathmoveto{\pgfqpoint{-0.000000in}{0.000000in}}%
\pgfpathlineto{\pgfqpoint{-0.048611in}{0.000000in}}%
\pgfusepath{stroke,fill}%
}%
\begin{pgfscope}%
\pgfsys@transformshift{0.633813in}{1.155562in}%
\pgfsys@useobject{currentmarker}{}%
\end{pgfscope}%
\end{pgfscope}%
\begin{pgfscope}%
\definecolor{textcolor}{rgb}{0.000000,0.000000,0.000000}%
\pgfsetstrokecolor{textcolor}%
\pgfsetfillcolor{textcolor}%
\pgftext[x=0.326711in, y=1.117006in, left, base]{\color{textcolor}\rmfamily\fontsize{8.000000}{9.600000}\selectfont \(\displaystyle {\ensuremath{-}10}\)}%
\end{pgfscope}%
\begin{pgfscope}%
\pgfsetbuttcap%
\pgfsetroundjoin%
\definecolor{currentfill}{rgb}{0.000000,0.000000,0.000000}%
\pgfsetfillcolor{currentfill}%
\pgfsetlinewidth{0.803000pt}%
\definecolor{currentstroke}{rgb}{0.000000,0.000000,0.000000}%
\pgfsetstrokecolor{currentstroke}%
\pgfsetdash{}{0pt}%
\pgfsys@defobject{currentmarker}{\pgfqpoint{-0.048611in}{0.000000in}}{\pgfqpoint{-0.000000in}{0.000000in}}{%
\pgfpathmoveto{\pgfqpoint{-0.000000in}{0.000000in}}%
\pgfpathlineto{\pgfqpoint{-0.048611in}{0.000000in}}%
\pgfusepath{stroke,fill}%
}%
\begin{pgfscope}%
\pgfsys@transformshift{0.633813in}{1.566972in}%
\pgfsys@useobject{currentmarker}{}%
\end{pgfscope}%
\end{pgfscope}%
\begin{pgfscope}%
\definecolor{textcolor}{rgb}{0.000000,0.000000,0.000000}%
\pgfsetstrokecolor{textcolor}%
\pgfsetfillcolor{textcolor}%
\pgftext[x=0.385740in, y=1.528416in, left, base]{\color{textcolor}\rmfamily\fontsize{8.000000}{9.600000}\selectfont \(\displaystyle {\ensuremath{-}5}\)}%
\end{pgfscope}%
\begin{pgfscope}%
\pgfsetbuttcap%
\pgfsetroundjoin%
\definecolor{currentfill}{rgb}{0.000000,0.000000,0.000000}%
\pgfsetfillcolor{currentfill}%
\pgfsetlinewidth{0.803000pt}%
\definecolor{currentstroke}{rgb}{0.000000,0.000000,0.000000}%
\pgfsetstrokecolor{currentstroke}%
\pgfsetdash{}{0pt}%
\pgfsys@defobject{currentmarker}{\pgfqpoint{-0.048611in}{0.000000in}}{\pgfqpoint{-0.000000in}{0.000000in}}{%
\pgfpathmoveto{\pgfqpoint{-0.000000in}{0.000000in}}%
\pgfpathlineto{\pgfqpoint{-0.048611in}{0.000000in}}%
\pgfusepath{stroke,fill}%
}%
\begin{pgfscope}%
\pgfsys@transformshift{0.633813in}{1.978382in}%
\pgfsys@useobject{currentmarker}{}%
\end{pgfscope}%
\end{pgfscope}%
\begin{pgfscope}%
\definecolor{textcolor}{rgb}{0.000000,0.000000,0.000000}%
\pgfsetstrokecolor{textcolor}%
\pgfsetfillcolor{textcolor}%
\pgftext[x=0.477562in, y=1.939826in, left, base]{\color{textcolor}\rmfamily\fontsize{8.000000}{9.600000}\selectfont \(\displaystyle {0}\)}%
\end{pgfscope}%
\begin{pgfscope}%
\pgfsetbuttcap%
\pgfsetroundjoin%
\definecolor{currentfill}{rgb}{0.000000,0.000000,0.000000}%
\pgfsetfillcolor{currentfill}%
\pgfsetlinewidth{0.803000pt}%
\definecolor{currentstroke}{rgb}{0.000000,0.000000,0.000000}%
\pgfsetstrokecolor{currentstroke}%
\pgfsetdash{}{0pt}%
\pgfsys@defobject{currentmarker}{\pgfqpoint{-0.048611in}{0.000000in}}{\pgfqpoint{-0.000000in}{0.000000in}}{%
\pgfpathmoveto{\pgfqpoint{-0.000000in}{0.000000in}}%
\pgfpathlineto{\pgfqpoint{-0.048611in}{0.000000in}}%
\pgfusepath{stroke,fill}%
}%
\begin{pgfscope}%
\pgfsys@transformshift{0.633813in}{2.389792in}%
\pgfsys@useobject{currentmarker}{}%
\end{pgfscope}%
\end{pgfscope}%
\begin{pgfscope}%
\definecolor{textcolor}{rgb}{0.000000,0.000000,0.000000}%
\pgfsetstrokecolor{textcolor}%
\pgfsetfillcolor{textcolor}%
\pgftext[x=0.477562in, y=2.351237in, left, base]{\color{textcolor}\rmfamily\fontsize{8.000000}{9.600000}\selectfont \(\displaystyle {5}\)}%
\end{pgfscope}%
\begin{pgfscope}%
\pgfsetbuttcap%
\pgfsetroundjoin%
\definecolor{currentfill}{rgb}{0.000000,0.000000,0.000000}%
\pgfsetfillcolor{currentfill}%
\pgfsetlinewidth{0.803000pt}%
\definecolor{currentstroke}{rgb}{0.000000,0.000000,0.000000}%
\pgfsetstrokecolor{currentstroke}%
\pgfsetdash{}{0pt}%
\pgfsys@defobject{currentmarker}{\pgfqpoint{-0.048611in}{0.000000in}}{\pgfqpoint{-0.000000in}{0.000000in}}{%
\pgfpathmoveto{\pgfqpoint{-0.000000in}{0.000000in}}%
\pgfpathlineto{\pgfqpoint{-0.048611in}{0.000000in}}%
\pgfusepath{stroke,fill}%
}%
\begin{pgfscope}%
\pgfsys@transformshift{0.633813in}{2.801202in}%
\pgfsys@useobject{currentmarker}{}%
\end{pgfscope}%
\end{pgfscope}%
\begin{pgfscope}%
\definecolor{textcolor}{rgb}{0.000000,0.000000,0.000000}%
\pgfsetstrokecolor{textcolor}%
\pgfsetfillcolor{textcolor}%
\pgftext[x=0.418534in, y=2.762647in, left, base]{\color{textcolor}\rmfamily\fontsize{8.000000}{9.600000}\selectfont \(\displaystyle {10}\)}%
\end{pgfscope}%
\begin{pgfscope}%
\definecolor{textcolor}{rgb}{0.000000,0.000000,0.000000}%
\pgfsetstrokecolor{textcolor}%
\pgfsetfillcolor{textcolor}%
\pgftext[x=0.271156in,y=1.737274in,,bottom,rotate=90.000000]{\color{textcolor}\rmfamily\fontsize{10.000000}{12.000000}\selectfont Voltage deviation in V}%
\end{pgfscope}%
\begin{pgfscope}%
\definecolor{textcolor}{rgb}{0.000000,0.000000,0.000000}%
\pgfsetstrokecolor{textcolor}%
\pgfsetfillcolor{textcolor}%
\pgftext[x=0.633813in,y=2.978201in,left,base]{\color{textcolor}\rmfamily\fontsize{8.000000}{9.600000}\selectfont \(\displaystyle \times{10^{\ensuremath{-}6}}{}\)}%
\end{pgfscope}%
\begin{pgfscope}%
\pgfpathrectangle{\pgfqpoint{0.633813in}{0.538014in}}{\pgfqpoint{3.873664in}{2.398521in}}%
\pgfusepath{clip}%
\pgfsetrectcap%
\pgfsetroundjoin%
\pgfsetlinewidth{0.501875pt}%
\definecolor{currentstroke}{rgb}{0.121569,0.466667,0.705882}%
\pgfsetstrokecolor{currentstroke}%
\pgfsetstrokeopacity{0.700000}%
\pgfsetdash{}{0pt}%
\pgfpathmoveto{\pgfqpoint{0.809889in}{1.922409in}}%
\pgfpathlineto{\pgfqpoint{0.810092in}{1.914181in}}%
\pgfpathlineto{\pgfqpoint{0.811315in}{2.226852in}}%
\pgfpathlineto{\pgfqpoint{0.812742in}{2.136342in}}%
\pgfpathlineto{\pgfqpoint{0.813557in}{2.086973in}}%
\pgfpathlineto{\pgfqpoint{0.813965in}{2.193939in}}%
\pgfpathlineto{\pgfqpoint{0.814984in}{1.963550in}}%
\pgfpathlineto{\pgfqpoint{0.814372in}{2.235080in}}%
\pgfpathlineto{\pgfqpoint{0.815391in}{2.111657in}}%
\pgfpathlineto{\pgfqpoint{0.816003in}{2.202168in}}%
\pgfpathlineto{\pgfqpoint{0.816410in}{2.161027in}}%
\pgfpathlineto{\pgfqpoint{0.816818in}{2.078745in}}%
\pgfpathlineto{\pgfqpoint{0.817225in}{2.193939in}}%
\pgfpathlineto{\pgfqpoint{0.818041in}{2.235080in}}%
\pgfpathlineto{\pgfqpoint{0.818448in}{2.226852in}}%
\pgfpathlineto{\pgfqpoint{0.818652in}{2.218624in}}%
\pgfpathlineto{\pgfqpoint{0.818856in}{2.243309in}}%
\pgfpathlineto{\pgfqpoint{0.819264in}{2.284450in}}%
\pgfpathlineto{\pgfqpoint{0.819875in}{2.251537in}}%
\pgfpathlineto{\pgfqpoint{0.820486in}{2.218624in}}%
\pgfpathlineto{\pgfqpoint{0.820690in}{2.251537in}}%
\pgfpathlineto{\pgfqpoint{0.820894in}{2.267993in}}%
\pgfpathlineto{\pgfqpoint{0.821098in}{2.218624in}}%
\pgfpathlineto{\pgfqpoint{0.821505in}{2.251537in}}%
\pgfpathlineto{\pgfqpoint{0.821913in}{2.218624in}}%
\pgfpathlineto{\pgfqpoint{0.822117in}{2.243309in}}%
\pgfpathlineto{\pgfqpoint{0.823136in}{2.300906in}}%
\pgfpathlineto{\pgfqpoint{0.823340in}{2.259765in}}%
\pgfpathlineto{\pgfqpoint{0.823747in}{2.292678in}}%
\pgfpathlineto{\pgfqpoint{0.824359in}{2.342047in}}%
\pgfpathlineto{\pgfqpoint{0.824766in}{2.284450in}}%
\pgfpathlineto{\pgfqpoint{0.825174in}{2.333819in}}%
\pgfpathlineto{\pgfqpoint{0.825785in}{2.169255in}}%
\pgfpathlineto{\pgfqpoint{0.826193in}{2.276221in}}%
\pgfpathlineto{\pgfqpoint{0.826397in}{2.276221in}}%
\pgfpathlineto{\pgfqpoint{0.826804in}{2.358504in}}%
\pgfpathlineto{\pgfqpoint{0.827416in}{2.267993in}}%
\pgfpathlineto{\pgfqpoint{0.828638in}{2.358504in}}%
\pgfpathlineto{\pgfqpoint{0.829657in}{2.284450in}}%
\pgfpathlineto{\pgfqpoint{0.829861in}{2.342047in}}%
\pgfpathlineto{\pgfqpoint{0.830269in}{2.267993in}}%
\pgfpathlineto{\pgfqpoint{0.830473in}{2.128114in}}%
\pgfpathlineto{\pgfqpoint{0.831084in}{2.333819in}}%
\pgfpathlineto{\pgfqpoint{0.831288in}{2.383188in}}%
\pgfpathlineto{\pgfqpoint{0.831492in}{2.333819in}}%
\pgfpathlineto{\pgfqpoint{0.831696in}{2.185711in}}%
\pgfpathlineto{\pgfqpoint{0.832511in}{2.276221in}}%
\pgfpathlineto{\pgfqpoint{0.832715in}{2.284450in}}%
\pgfpathlineto{\pgfqpoint{0.832918in}{2.103429in}}%
\pgfpathlineto{\pgfqpoint{0.833530in}{2.300906in}}%
\pgfpathlineto{\pgfqpoint{0.833734in}{2.267993in}}%
\pgfpathlineto{\pgfqpoint{0.834345in}{2.342047in}}%
\pgfpathlineto{\pgfqpoint{0.834549in}{2.259765in}}%
\pgfpathlineto{\pgfqpoint{0.834753in}{2.235080in}}%
\pgfpathlineto{\pgfqpoint{0.835364in}{2.276221in}}%
\pgfpathlineto{\pgfqpoint{0.835568in}{2.243309in}}%
\pgfpathlineto{\pgfqpoint{0.836179in}{2.235080in}}%
\pgfpathlineto{\pgfqpoint{0.836587in}{2.300906in}}%
\pgfpathlineto{\pgfqpoint{0.836791in}{2.267993in}}%
\pgfpathlineto{\pgfqpoint{0.837402in}{2.284450in}}%
\pgfpathlineto{\pgfqpoint{0.838013in}{2.399645in}}%
\pgfpathlineto{\pgfqpoint{0.838421in}{2.267993in}}%
\pgfpathlineto{\pgfqpoint{0.839236in}{2.333819in}}%
\pgfpathlineto{\pgfqpoint{0.839644in}{2.325591in}}%
\pgfpathlineto{\pgfqpoint{0.840051in}{2.243309in}}%
\pgfpathlineto{\pgfqpoint{0.840255in}{2.292678in}}%
\pgfpathlineto{\pgfqpoint{0.840459in}{2.399645in}}%
\pgfpathlineto{\pgfqpoint{0.841478in}{2.374960in}}%
\pgfpathlineto{\pgfqpoint{0.842497in}{2.457242in}}%
\pgfpathlineto{\pgfqpoint{0.842701in}{2.416101in}}%
\pgfpathlineto{\pgfqpoint{0.842905in}{2.358504in}}%
\pgfpathlineto{\pgfqpoint{0.843312in}{2.440786in}}%
\pgfpathlineto{\pgfqpoint{0.843516in}{2.432557in}}%
\pgfpathlineto{\pgfqpoint{0.844127in}{2.498383in}}%
\pgfpathlineto{\pgfqpoint{0.844331in}{2.457242in}}%
\pgfpathlineto{\pgfqpoint{0.844535in}{2.366732in}}%
\pgfpathlineto{\pgfqpoint{0.845350in}{2.440786in}}%
\pgfpathlineto{\pgfqpoint{0.845962in}{2.350275in}}%
\pgfpathlineto{\pgfqpoint{0.846369in}{2.449014in}}%
\pgfpathlineto{\pgfqpoint{0.846573in}{2.449014in}}%
\pgfpathlineto{\pgfqpoint{0.846777in}{2.473698in}}%
\pgfpathlineto{\pgfqpoint{0.847185in}{2.449014in}}%
\pgfpathlineto{\pgfqpoint{0.848407in}{2.309134in}}%
\pgfpathlineto{\pgfqpoint{0.848611in}{2.358504in}}%
\pgfpathlineto{\pgfqpoint{0.849223in}{2.317363in}}%
\pgfpathlineto{\pgfqpoint{0.849426in}{2.350275in}}%
\pgfpathlineto{\pgfqpoint{0.851872in}{2.490155in}}%
\pgfpathlineto{\pgfqpoint{0.852076in}{2.481927in}}%
\pgfpathlineto{\pgfqpoint{0.853299in}{2.399645in}}%
\pgfpathlineto{\pgfqpoint{0.853706in}{2.416101in}}%
\pgfpathlineto{\pgfqpoint{0.854114in}{2.432557in}}%
\pgfpathlineto{\pgfqpoint{0.854725in}{2.391416in}}%
\pgfpathlineto{\pgfqpoint{0.854929in}{2.457242in}}%
\pgfpathlineto{\pgfqpoint{0.855744in}{2.391416in}}%
\pgfpathlineto{\pgfqpoint{0.856356in}{2.342047in}}%
\pgfpathlineto{\pgfqpoint{0.856152in}{2.424329in}}%
\pgfpathlineto{\pgfqpoint{0.856559in}{2.416101in}}%
\pgfpathlineto{\pgfqpoint{0.856967in}{2.416101in}}%
\pgfpathlineto{\pgfqpoint{0.857578in}{2.366732in}}%
\pgfpathlineto{\pgfqpoint{0.857986in}{2.391416in}}%
\pgfpathlineto{\pgfqpoint{0.859005in}{2.416101in}}%
\pgfpathlineto{\pgfqpoint{0.859209in}{2.374960in}}%
\pgfpathlineto{\pgfqpoint{0.859413in}{2.449014in}}%
\pgfpathlineto{\pgfqpoint{0.860228in}{2.391416in}}%
\pgfpathlineto{\pgfqpoint{0.860432in}{2.391416in}}%
\pgfpathlineto{\pgfqpoint{0.860839in}{2.358504in}}%
\pgfpathlineto{\pgfqpoint{0.861655in}{2.432557in}}%
\pgfpathlineto{\pgfqpoint{0.862266in}{2.391416in}}%
\pgfpathlineto{\pgfqpoint{0.862062in}{2.440786in}}%
\pgfpathlineto{\pgfqpoint{0.862470in}{2.416101in}}%
\pgfpathlineto{\pgfqpoint{0.862877in}{2.399645in}}%
\pgfpathlineto{\pgfqpoint{0.863285in}{2.449014in}}%
\pgfpathlineto{\pgfqpoint{0.863489in}{2.309134in}}%
\pgfpathlineto{\pgfqpoint{0.863693in}{2.514839in}}%
\pgfpathlineto{\pgfqpoint{0.864304in}{2.473698in}}%
\pgfpathlineto{\pgfqpoint{0.864508in}{2.449014in}}%
\pgfpathlineto{\pgfqpoint{0.864712in}{2.243309in}}%
\pgfpathlineto{\pgfqpoint{0.864915in}{2.473698in}}%
\pgfpathlineto{\pgfqpoint{0.865527in}{2.449014in}}%
\pgfpathlineto{\pgfqpoint{0.867157in}{2.547752in}}%
\pgfpathlineto{\pgfqpoint{0.867769in}{2.432557in}}%
\pgfpathlineto{\pgfqpoint{0.868176in}{2.481927in}}%
\pgfpathlineto{\pgfqpoint{0.869399in}{2.514839in}}%
\pgfpathlineto{\pgfqpoint{0.869603in}{2.506611in}}%
\pgfpathlineto{\pgfqpoint{0.869807in}{2.588893in}}%
\pgfpathlineto{\pgfqpoint{0.870214in}{2.481927in}}%
\pgfpathlineto{\pgfqpoint{0.870622in}{2.572437in}}%
\pgfpathlineto{\pgfqpoint{0.871641in}{2.473698in}}%
\pgfpathlineto{\pgfqpoint{0.871845in}{2.547752in}}%
\pgfpathlineto{\pgfqpoint{0.872660in}{2.506611in}}%
\pgfpathlineto{\pgfqpoint{0.873475in}{2.465470in}}%
\pgfpathlineto{\pgfqpoint{0.874087in}{2.309134in}}%
\pgfpathlineto{\pgfqpoint{0.874698in}{2.572437in}}%
\pgfpathlineto{\pgfqpoint{0.875106in}{2.391416in}}%
\pgfpathlineto{\pgfqpoint{0.875921in}{2.531296in}}%
\pgfpathlineto{\pgfqpoint{0.876328in}{2.490155in}}%
\pgfpathlineto{\pgfqpoint{0.876736in}{2.185711in}}%
\pgfpathlineto{\pgfqpoint{0.877347in}{2.432557in}}%
\pgfpathlineto{\pgfqpoint{0.877551in}{2.432557in}}%
\pgfpathlineto{\pgfqpoint{0.877755in}{2.449014in}}%
\pgfpathlineto{\pgfqpoint{0.878366in}{2.465470in}}%
\pgfpathlineto{\pgfqpoint{0.878978in}{2.193939in}}%
\pgfpathlineto{\pgfqpoint{0.879182in}{2.490155in}}%
\pgfpathlineto{\pgfqpoint{0.880201in}{2.424329in}}%
\pgfpathlineto{\pgfqpoint{0.880404in}{2.424329in}}%
\pgfpathlineto{\pgfqpoint{0.881627in}{2.588893in}}%
\pgfpathlineto{\pgfqpoint{0.881831in}{2.531296in}}%
\pgfpathlineto{\pgfqpoint{0.882646in}{2.498383in}}%
\pgfpathlineto{\pgfqpoint{0.882442in}{2.555980in}}%
\pgfpathlineto{\pgfqpoint{0.882850in}{2.506611in}}%
\pgfpathlineto{\pgfqpoint{0.883665in}{2.588893in}}%
\pgfpathlineto{\pgfqpoint{0.884073in}{2.547752in}}%
\pgfpathlineto{\pgfqpoint{0.884277in}{2.547752in}}%
\pgfpathlineto{\pgfqpoint{0.884888in}{2.605350in}}%
\pgfpathlineto{\pgfqpoint{0.885500in}{2.588893in}}%
\pgfpathlineto{\pgfqpoint{0.885703in}{2.481927in}}%
\pgfpathlineto{\pgfqpoint{0.886519in}{2.572437in}}%
\pgfpathlineto{\pgfqpoint{0.886926in}{2.580665in}}%
\pgfpathlineto{\pgfqpoint{0.887334in}{2.523068in}}%
\pgfpathlineto{\pgfqpoint{0.888353in}{2.630034in}}%
\pgfpathlineto{\pgfqpoint{0.888557in}{2.613578in}}%
\pgfpathlineto{\pgfqpoint{0.889168in}{2.572437in}}%
\pgfpathlineto{\pgfqpoint{0.889576in}{2.605350in}}%
\pgfpathlineto{\pgfqpoint{0.889779in}{2.621806in}}%
\pgfpathlineto{\pgfqpoint{0.889983in}{2.572437in}}%
\pgfpathlineto{\pgfqpoint{0.891614in}{2.465470in}}%
\pgfpathlineto{\pgfqpoint{0.891817in}{2.457242in}}%
\pgfpathlineto{\pgfqpoint{0.892021in}{2.473698in}}%
\pgfpathlineto{\pgfqpoint{0.893448in}{2.646491in}}%
\pgfpathlineto{\pgfqpoint{0.893652in}{2.654719in}}%
\pgfpathlineto{\pgfqpoint{0.895078in}{2.449014in}}%
\pgfpathlineto{\pgfqpoint{0.895282in}{2.498383in}}%
\pgfpathlineto{\pgfqpoint{0.895893in}{2.539524in}}%
\pgfpathlineto{\pgfqpoint{0.896097in}{2.514839in}}%
\pgfpathlineto{\pgfqpoint{0.897116in}{2.473698in}}%
\pgfpathlineto{\pgfqpoint{0.897524in}{2.539524in}}%
\pgfpathlineto{\pgfqpoint{0.898339in}{2.531296in}}%
\pgfpathlineto{\pgfqpoint{0.898543in}{2.523068in}}%
\pgfpathlineto{\pgfqpoint{0.899766in}{2.572437in}}%
\pgfpathlineto{\pgfqpoint{0.899970in}{2.547752in}}%
\pgfpathlineto{\pgfqpoint{0.900173in}{2.605350in}}%
\pgfpathlineto{\pgfqpoint{0.900785in}{2.564209in}}%
\pgfpathlineto{\pgfqpoint{0.900989in}{2.646491in}}%
\pgfpathlineto{\pgfqpoint{0.901804in}{2.630034in}}%
\pgfpathlineto{\pgfqpoint{0.902008in}{2.597121in}}%
\pgfpathlineto{\pgfqpoint{0.902823in}{2.621806in}}%
\pgfpathlineto{\pgfqpoint{0.903027in}{2.662947in}}%
\pgfpathlineto{\pgfqpoint{0.903638in}{2.588893in}}%
\pgfpathlineto{\pgfqpoint{0.903842in}{2.630034in}}%
\pgfpathlineto{\pgfqpoint{0.904453in}{2.638262in}}%
\pgfpathlineto{\pgfqpoint{0.905065in}{2.555980in}}%
\pgfpathlineto{\pgfqpoint{0.906491in}{2.440786in}}%
\pgfpathlineto{\pgfqpoint{0.906695in}{2.498383in}}%
\pgfpathlineto{\pgfqpoint{0.907510in}{2.457242in}}%
\pgfpathlineto{\pgfqpoint{0.907714in}{2.457242in}}%
\pgfpathlineto{\pgfqpoint{0.908325in}{2.416101in}}%
\pgfpathlineto{\pgfqpoint{0.908529in}{2.465470in}}%
\pgfpathlineto{\pgfqpoint{0.908733in}{2.449014in}}%
\pgfpathlineto{\pgfqpoint{0.910363in}{2.572437in}}%
\pgfpathlineto{\pgfqpoint{0.909141in}{2.440786in}}%
\pgfpathlineto{\pgfqpoint{0.910771in}{2.547752in}}%
\pgfpathlineto{\pgfqpoint{0.910975in}{2.514839in}}%
\pgfpathlineto{\pgfqpoint{0.911586in}{2.588893in}}%
\pgfpathlineto{\pgfqpoint{0.912198in}{2.555980in}}%
\pgfpathlineto{\pgfqpoint{0.912402in}{2.621806in}}%
\pgfpathlineto{\pgfqpoint{0.914847in}{2.490155in}}%
\pgfpathlineto{\pgfqpoint{0.915662in}{2.547752in}}%
\pgfpathlineto{\pgfqpoint{0.915866in}{2.523068in}}%
\pgfpathlineto{\pgfqpoint{0.916070in}{2.473698in}}%
\pgfpathlineto{\pgfqpoint{0.916478in}{2.564209in}}%
\pgfpathlineto{\pgfqpoint{0.916885in}{2.531296in}}%
\pgfpathlineto{\pgfqpoint{0.917497in}{2.555980in}}%
\pgfpathlineto{\pgfqpoint{0.917700in}{2.539524in}}%
\pgfpathlineto{\pgfqpoint{0.917904in}{2.506611in}}%
\pgfpathlineto{\pgfqpoint{0.918312in}{2.605350in}}%
\pgfpathlineto{\pgfqpoint{0.918516in}{2.572437in}}%
\pgfpathlineto{\pgfqpoint{0.919331in}{2.547752in}}%
\pgfpathlineto{\pgfqpoint{0.918923in}{2.588893in}}%
\pgfpathlineto{\pgfqpoint{0.919535in}{2.572437in}}%
\pgfpathlineto{\pgfqpoint{0.919942in}{2.662947in}}%
\pgfpathlineto{\pgfqpoint{0.920757in}{2.654719in}}%
\pgfpathlineto{\pgfqpoint{0.921573in}{2.580665in}}%
\pgfpathlineto{\pgfqpoint{0.922184in}{2.597121in}}%
\pgfpathlineto{\pgfqpoint{0.922388in}{2.605350in}}%
\pgfpathlineto{\pgfqpoint{0.922999in}{2.490155in}}%
\pgfpathlineto{\pgfqpoint{0.923611in}{2.531296in}}%
\pgfpathlineto{\pgfqpoint{0.924426in}{2.490155in}}%
\pgfpathlineto{\pgfqpoint{0.925241in}{2.498383in}}%
\pgfpathlineto{\pgfqpoint{0.926872in}{2.588893in}}%
\pgfpathlineto{\pgfqpoint{0.927483in}{2.605350in}}%
\pgfpathlineto{\pgfqpoint{0.928094in}{2.564209in}}%
\pgfpathlineto{\pgfqpoint{0.928910in}{2.646491in}}%
\pgfpathlineto{\pgfqpoint{0.929317in}{2.588893in}}%
\pgfpathlineto{\pgfqpoint{0.929521in}{2.588893in}}%
\pgfpathlineto{\pgfqpoint{0.930540in}{2.498383in}}%
\pgfpathlineto{\pgfqpoint{0.930132in}{2.630034in}}%
\pgfpathlineto{\pgfqpoint{0.930948in}{2.514839in}}%
\pgfpathlineto{\pgfqpoint{0.931151in}{2.555980in}}%
\pgfpathlineto{\pgfqpoint{0.931763in}{2.498383in}}%
\pgfpathlineto{\pgfqpoint{0.931967in}{2.498383in}}%
\pgfpathlineto{\pgfqpoint{0.932986in}{2.613578in}}%
\pgfpathlineto{\pgfqpoint{0.932374in}{2.490155in}}%
\pgfpathlineto{\pgfqpoint{0.933393in}{2.555980in}}%
\pgfpathlineto{\pgfqpoint{0.933597in}{2.539524in}}%
\pgfpathlineto{\pgfqpoint{0.934208in}{2.572437in}}%
\pgfpathlineto{\pgfqpoint{0.934412in}{2.572437in}}%
\pgfpathlineto{\pgfqpoint{0.934616in}{2.300906in}}%
\pgfpathlineto{\pgfqpoint{0.935431in}{2.383188in}}%
\pgfpathlineto{\pgfqpoint{0.936246in}{2.605350in}}%
\pgfpathlineto{\pgfqpoint{0.936654in}{2.564209in}}%
\pgfpathlineto{\pgfqpoint{0.938284in}{2.498383in}}%
\pgfpathlineto{\pgfqpoint{0.938488in}{2.539524in}}%
\pgfpathlineto{\pgfqpoint{0.939100in}{2.498383in}}%
\pgfpathlineto{\pgfqpoint{0.939304in}{2.383188in}}%
\pgfpathlineto{\pgfqpoint{0.939915in}{2.597121in}}%
\pgfpathlineto{\pgfqpoint{0.940323in}{2.416101in}}%
\pgfpathlineto{\pgfqpoint{0.940730in}{2.547752in}}%
\pgfpathlineto{\pgfqpoint{0.941545in}{2.498383in}}%
\pgfpathlineto{\pgfqpoint{0.941749in}{2.506611in}}%
\pgfpathlineto{\pgfqpoint{0.942157in}{2.481927in}}%
\pgfpathlineto{\pgfqpoint{0.942361in}{2.432557in}}%
\pgfpathlineto{\pgfqpoint{0.942768in}{2.547752in}}%
\pgfpathlineto{\pgfqpoint{0.942972in}{2.514839in}}%
\pgfpathlineto{\pgfqpoint{0.944195in}{2.654719in}}%
\pgfpathlineto{\pgfqpoint{0.944602in}{2.630034in}}%
\pgfpathlineto{\pgfqpoint{0.945010in}{2.407873in}}%
\pgfpathlineto{\pgfqpoint{0.945621in}{2.580665in}}%
\pgfpathlineto{\pgfqpoint{0.946029in}{2.679403in}}%
\pgfpathlineto{\pgfqpoint{0.946844in}{2.638262in}}%
\pgfpathlineto{\pgfqpoint{0.948067in}{2.555980in}}%
\pgfpathlineto{\pgfqpoint{0.948882in}{2.630034in}}%
\pgfpathlineto{\pgfqpoint{0.949086in}{2.621806in}}%
\pgfpathlineto{\pgfqpoint{0.949290in}{2.539524in}}%
\pgfpathlineto{\pgfqpoint{0.949697in}{2.630034in}}%
\pgfpathlineto{\pgfqpoint{0.950105in}{2.588893in}}%
\pgfpathlineto{\pgfqpoint{0.950309in}{2.630034in}}%
\pgfpathlineto{\pgfqpoint{0.950513in}{2.580665in}}%
\pgfpathlineto{\pgfqpoint{0.950716in}{2.613578in}}%
\pgfpathlineto{\pgfqpoint{0.950920in}{2.440786in}}%
\pgfpathlineto{\pgfqpoint{0.951328in}{2.720544in}}%
\pgfpathlineto{\pgfqpoint{0.951735in}{2.671175in}}%
\pgfpathlineto{\pgfqpoint{0.952958in}{2.761685in}}%
\pgfpathlineto{\pgfqpoint{0.953162in}{2.712316in}}%
\pgfpathlineto{\pgfqpoint{0.954181in}{2.473698in}}%
\pgfpathlineto{\pgfqpoint{0.954385in}{2.638262in}}%
\pgfpathlineto{\pgfqpoint{0.954793in}{2.827511in}}%
\pgfpathlineto{\pgfqpoint{0.955404in}{2.745229in}}%
\pgfpathlineto{\pgfqpoint{0.955608in}{2.704088in}}%
\pgfpathlineto{\pgfqpoint{0.956423in}{2.737001in}}%
\pgfpathlineto{\pgfqpoint{0.956831in}{2.761685in}}%
\pgfpathlineto{\pgfqpoint{0.957238in}{2.712316in}}%
\pgfpathlineto{\pgfqpoint{0.958257in}{2.662947in}}%
\pgfpathlineto{\pgfqpoint{0.958461in}{2.671175in}}%
\pgfpathlineto{\pgfqpoint{0.958665in}{2.687632in}}%
\pgfpathlineto{\pgfqpoint{0.959072in}{2.662947in}}%
\pgfpathlineto{\pgfqpoint{0.960499in}{2.564209in}}%
\pgfpathlineto{\pgfqpoint{0.961110in}{2.695860in}}%
\pgfpathlineto{\pgfqpoint{0.960907in}{2.424329in}}%
\pgfpathlineto{\pgfqpoint{0.961518in}{2.671175in}}%
\pgfpathlineto{\pgfqpoint{0.961926in}{2.597121in}}%
\pgfpathlineto{\pgfqpoint{0.962741in}{2.613578in}}%
\pgfpathlineto{\pgfqpoint{0.962945in}{2.621806in}}%
\pgfpathlineto{\pgfqpoint{0.963760in}{2.399645in}}%
\pgfpathlineto{\pgfqpoint{0.963964in}{2.572437in}}%
\pgfpathlineto{\pgfqpoint{0.965390in}{2.671175in}}%
\pgfpathlineto{\pgfqpoint{0.966613in}{2.399645in}}%
\pgfpathlineto{\pgfqpoint{0.966817in}{2.514839in}}%
\pgfpathlineto{\pgfqpoint{0.967836in}{2.613578in}}%
\pgfpathlineto{\pgfqpoint{0.968040in}{2.605350in}}%
\pgfpathlineto{\pgfqpoint{0.968244in}{2.588893in}}%
\pgfpathlineto{\pgfqpoint{0.968447in}{2.613578in}}%
\pgfpathlineto{\pgfqpoint{0.968651in}{2.597121in}}%
\pgfpathlineto{\pgfqpoint{0.969059in}{2.654719in}}%
\pgfpathlineto{\pgfqpoint{0.969466in}{2.547752in}}%
\pgfpathlineto{\pgfqpoint{0.969670in}{2.613578in}}%
\pgfpathlineto{\pgfqpoint{0.970893in}{2.276221in}}%
\pgfpathlineto{\pgfqpoint{0.971912in}{2.654719in}}%
\pgfpathlineto{\pgfqpoint{0.972116in}{2.638262in}}%
\pgfpathlineto{\pgfqpoint{0.972320in}{2.646491in}}%
\pgfpathlineto{\pgfqpoint{0.972523in}{2.630034in}}%
\pgfpathlineto{\pgfqpoint{0.973746in}{2.473698in}}%
\pgfpathlineto{\pgfqpoint{0.973135in}{2.671175in}}%
\pgfpathlineto{\pgfqpoint{0.973950in}{2.580665in}}%
\pgfpathlineto{\pgfqpoint{0.974154in}{2.613578in}}%
\pgfpathlineto{\pgfqpoint{0.974765in}{2.597121in}}%
\pgfpathlineto{\pgfqpoint{0.974969in}{2.547752in}}%
\pgfpathlineto{\pgfqpoint{0.975580in}{2.613578in}}%
\pgfpathlineto{\pgfqpoint{0.975988in}{2.555980in}}%
\pgfpathlineto{\pgfqpoint{0.977415in}{2.654719in}}%
\pgfpathlineto{\pgfqpoint{0.977618in}{2.481927in}}%
\pgfpathlineto{\pgfqpoint{0.978434in}{2.654719in}}%
\pgfpathlineto{\pgfqpoint{0.979860in}{2.473698in}}%
\pgfpathlineto{\pgfqpoint{0.978841in}{2.679403in}}%
\pgfpathlineto{\pgfqpoint{0.980064in}{2.506611in}}%
\pgfpathlineto{\pgfqpoint{0.980268in}{2.547752in}}%
\pgfpathlineto{\pgfqpoint{0.980676in}{2.407873in}}%
\pgfpathlineto{\pgfqpoint{0.981287in}{2.539524in}}%
\pgfpathlineto{\pgfqpoint{0.981491in}{2.531296in}}%
\pgfpathlineto{\pgfqpoint{0.981898in}{2.481927in}}%
\pgfpathlineto{\pgfqpoint{0.982917in}{2.646491in}}%
\pgfpathlineto{\pgfqpoint{0.984344in}{2.547752in}}%
\pgfpathlineto{\pgfqpoint{0.986178in}{2.712316in}}%
\pgfpathlineto{\pgfqpoint{0.987197in}{2.358504in}}%
\pgfpathlineto{\pgfqpoint{0.987605in}{2.547752in}}%
\pgfpathlineto{\pgfqpoint{0.988012in}{2.564209in}}%
\pgfpathlineto{\pgfqpoint{0.989031in}{2.481927in}}%
\pgfpathlineto{\pgfqpoint{0.989235in}{2.498383in}}%
\pgfpathlineto{\pgfqpoint{0.990050in}{2.646491in}}%
\pgfpathlineto{\pgfqpoint{0.990254in}{2.399645in}}%
\pgfpathlineto{\pgfqpoint{0.991069in}{2.638262in}}%
\pgfpathlineto{\pgfqpoint{0.991681in}{2.679403in}}%
\pgfpathlineto{\pgfqpoint{0.991885in}{2.613578in}}%
\pgfpathlineto{\pgfqpoint{0.992496in}{2.547752in}}%
\pgfpathlineto{\pgfqpoint{0.992904in}{2.654719in}}%
\pgfpathlineto{\pgfqpoint{0.993311in}{2.638262in}}%
\pgfpathlineto{\pgfqpoint{0.993515in}{2.358504in}}%
\pgfpathlineto{\pgfqpoint{0.994330in}{2.704088in}}%
\pgfpathlineto{\pgfqpoint{0.995349in}{2.605350in}}%
\pgfpathlineto{\pgfqpoint{0.995757in}{2.613578in}}%
\pgfpathlineto{\pgfqpoint{0.996368in}{2.588893in}}%
\pgfpathlineto{\pgfqpoint{0.996980in}{2.638262in}}%
\pgfpathlineto{\pgfqpoint{0.997591in}{2.514839in}}%
\pgfpathlineto{\pgfqpoint{0.998203in}{2.547752in}}%
\pgfpathlineto{\pgfqpoint{0.998406in}{2.547752in}}%
\pgfpathlineto{\pgfqpoint{0.998610in}{2.539524in}}%
\pgfpathlineto{\pgfqpoint{0.998814in}{2.391416in}}%
\pgfpathlineto{\pgfqpoint{0.999629in}{2.572437in}}%
\pgfpathlineto{\pgfqpoint{0.999833in}{2.531296in}}%
\pgfpathlineto{\pgfqpoint{1.000444in}{2.605350in}}%
\pgfpathlineto{\pgfqpoint{1.001667in}{2.695860in}}%
\pgfpathlineto{\pgfqpoint{1.002890in}{2.547752in}}%
\pgfpathlineto{\pgfqpoint{1.004317in}{2.638262in}}%
\pgfpathlineto{\pgfqpoint{1.004928in}{2.539524in}}%
\pgfpathlineto{\pgfqpoint{1.005336in}{2.597121in}}%
\pgfpathlineto{\pgfqpoint{1.006151in}{2.654719in}}%
\pgfpathlineto{\pgfqpoint{1.006355in}{2.621806in}}%
\pgfpathlineto{\pgfqpoint{1.006762in}{2.597121in}}%
\pgfpathlineto{\pgfqpoint{1.007170in}{2.654719in}}%
\pgfpathlineto{\pgfqpoint{1.007374in}{2.646491in}}%
\pgfpathlineto{\pgfqpoint{1.007578in}{2.695860in}}%
\pgfpathlineto{\pgfqpoint{1.008597in}{2.687632in}}%
\pgfpathlineto{\pgfqpoint{1.009208in}{2.712316in}}%
\pgfpathlineto{\pgfqpoint{1.009819in}{2.638262in}}%
\pgfpathlineto{\pgfqpoint{1.010023in}{2.712316in}}%
\pgfpathlineto{\pgfqpoint{1.010838in}{2.671175in}}%
\pgfpathlineto{\pgfqpoint{1.011857in}{2.547752in}}%
\pgfpathlineto{\pgfqpoint{1.012265in}{2.638262in}}%
\pgfpathlineto{\pgfqpoint{1.012876in}{2.654719in}}%
\pgfpathlineto{\pgfqpoint{1.012673in}{2.630034in}}%
\pgfpathlineto{\pgfqpoint{1.013284in}{2.638262in}}%
\pgfpathlineto{\pgfqpoint{1.014303in}{2.588893in}}%
\pgfpathlineto{\pgfqpoint{1.013895in}{2.679403in}}%
\pgfpathlineto{\pgfqpoint{1.014507in}{2.597121in}}%
\pgfpathlineto{\pgfqpoint{1.014711in}{2.638262in}}%
\pgfpathlineto{\pgfqpoint{1.015322in}{2.564209in}}%
\pgfpathlineto{\pgfqpoint{1.015526in}{2.597121in}}%
\pgfpathlineto{\pgfqpoint{1.015730in}{2.605350in}}%
\pgfpathlineto{\pgfqpoint{1.017156in}{2.416101in}}%
\pgfpathlineto{\pgfqpoint{1.017564in}{2.432557in}}%
\pgfpathlineto{\pgfqpoint{1.018787in}{2.300906in}}%
\pgfpathlineto{\pgfqpoint{1.019602in}{2.366732in}}%
\pgfpathlineto{\pgfqpoint{1.019398in}{2.284450in}}%
\pgfpathlineto{\pgfqpoint{1.019806in}{2.292678in}}%
\pgfpathlineto{\pgfqpoint{1.020010in}{2.309134in}}%
\pgfpathlineto{\pgfqpoint{1.020417in}{2.259765in}}%
\pgfpathlineto{\pgfqpoint{1.020825in}{2.292678in}}%
\pgfpathlineto{\pgfqpoint{1.021844in}{2.251537in}}%
\pgfpathlineto{\pgfqpoint{1.021232in}{2.333819in}}%
\pgfpathlineto{\pgfqpoint{1.022048in}{2.267993in}}%
\pgfpathlineto{\pgfqpoint{1.023270in}{2.366732in}}%
\pgfpathlineto{\pgfqpoint{1.025105in}{2.144570in}}%
\pgfpathlineto{\pgfqpoint{1.025920in}{2.309134in}}%
\pgfpathlineto{\pgfqpoint{1.026531in}{2.276221in}}%
\pgfpathlineto{\pgfqpoint{1.026735in}{2.284450in}}%
\pgfpathlineto{\pgfqpoint{1.026939in}{2.276221in}}%
\pgfpathlineto{\pgfqpoint{1.027754in}{2.103429in}}%
\pgfpathlineto{\pgfqpoint{1.028365in}{2.177483in}}%
\pgfpathlineto{\pgfqpoint{1.028569in}{2.177483in}}%
\pgfpathlineto{\pgfqpoint{1.028773in}{2.054060in}}%
\pgfpathlineto{\pgfqpoint{1.029792in}{2.095201in}}%
\pgfpathlineto{\pgfqpoint{1.030403in}{2.086973in}}%
\pgfpathlineto{\pgfqpoint{1.030607in}{2.128114in}}%
\pgfpathlineto{\pgfqpoint{1.030811in}{2.235080in}}%
\pgfpathlineto{\pgfqpoint{1.031422in}{2.070516in}}%
\pgfpathlineto{\pgfqpoint{1.032441in}{2.004691in}}%
\pgfpathlineto{\pgfqpoint{1.032849in}{2.021147in}}%
\pgfpathlineto{\pgfqpoint{1.034072in}{2.193939in}}%
\pgfpathlineto{\pgfqpoint{1.033257in}{1.963550in}}%
\pgfpathlineto{\pgfqpoint{1.034276in}{2.152798in}}%
\pgfpathlineto{\pgfqpoint{1.034480in}{1.988234in}}%
\pgfpathlineto{\pgfqpoint{1.034887in}{2.193939in}}%
\pgfpathlineto{\pgfqpoint{1.035295in}{2.078745in}}%
\pgfpathlineto{\pgfqpoint{1.036518in}{2.218624in}}%
\pgfpathlineto{\pgfqpoint{1.036721in}{2.185711in}}%
\pgfpathlineto{\pgfqpoint{1.036925in}{1.914181in}}%
\pgfpathlineto{\pgfqpoint{1.037740in}{2.086973in}}%
\pgfpathlineto{\pgfqpoint{1.038148in}{2.128114in}}%
\pgfpathlineto{\pgfqpoint{1.038556in}{2.078745in}}%
\pgfpathlineto{\pgfqpoint{1.038759in}{2.054060in}}%
\pgfpathlineto{\pgfqpoint{1.038963in}{2.103429in}}%
\pgfpathlineto{\pgfqpoint{1.039575in}{2.078745in}}%
\pgfpathlineto{\pgfqpoint{1.039982in}{2.012919in}}%
\pgfpathlineto{\pgfqpoint{1.040797in}{2.161027in}}%
\pgfpathlineto{\pgfqpoint{1.041613in}{2.021147in}}%
\pgfpathlineto{\pgfqpoint{1.042224in}{2.078745in}}%
\pgfpathlineto{\pgfqpoint{1.042632in}{2.103429in}}%
\pgfpathlineto{\pgfqpoint{1.043039in}{1.848355in}}%
\pgfpathlineto{\pgfqpoint{1.043854in}{1.988234in}}%
\pgfpathlineto{\pgfqpoint{1.044262in}{2.029375in}}%
\pgfpathlineto{\pgfqpoint{1.044466in}{1.971778in}}%
\pgfpathlineto{\pgfqpoint{1.044873in}{1.988234in}}%
\pgfpathlineto{\pgfqpoint{1.045689in}{1.897724in}}%
\pgfpathlineto{\pgfqpoint{1.046096in}{1.947093in}}%
\pgfpathlineto{\pgfqpoint{1.046300in}{1.980006in}}%
\pgfpathlineto{\pgfqpoint{1.046708in}{1.889496in}}%
\pgfpathlineto{\pgfqpoint{1.048134in}{1.971778in}}%
\pgfpathlineto{\pgfqpoint{1.048542in}{1.930637in}}%
\pgfpathlineto{\pgfqpoint{1.048746in}{1.757845in}}%
\pgfpathlineto{\pgfqpoint{1.049561in}{1.971778in}}%
\pgfpathlineto{\pgfqpoint{1.050376in}{1.823670in}}%
\pgfpathlineto{\pgfqpoint{1.049969in}{1.980006in}}%
\pgfpathlineto{\pgfqpoint{1.050784in}{1.873040in}}%
\pgfpathlineto{\pgfqpoint{1.050988in}{1.881268in}}%
\pgfpathlineto{\pgfqpoint{1.051599in}{1.996463in}}%
\pgfpathlineto{\pgfqpoint{1.052210in}{1.971778in}}%
\pgfpathlineto{\pgfqpoint{1.053026in}{1.873040in}}%
\pgfpathlineto{\pgfqpoint{1.053229in}{1.914181in}}%
\pgfpathlineto{\pgfqpoint{1.053637in}{2.004691in}}%
\pgfpathlineto{\pgfqpoint{1.054452in}{1.955322in}}%
\pgfpathlineto{\pgfqpoint{1.056083in}{2.095201in}}%
\pgfpathlineto{\pgfqpoint{1.056490in}{2.012919in}}%
\pgfpathlineto{\pgfqpoint{1.057102in}{2.054060in}}%
\pgfpathlineto{\pgfqpoint{1.057305in}{2.070516in}}%
\pgfpathlineto{\pgfqpoint{1.057509in}{2.004691in}}%
\pgfpathlineto{\pgfqpoint{1.057713in}{2.004691in}}%
\pgfpathlineto{\pgfqpoint{1.058936in}{2.128114in}}%
\pgfpathlineto{\pgfqpoint{1.059751in}{1.881268in}}%
\pgfpathlineto{\pgfqpoint{1.059955in}{2.029375in}}%
\pgfpathlineto{\pgfqpoint{1.060159in}{2.136342in}}%
\pgfpathlineto{\pgfqpoint{1.060363in}{1.930637in}}%
\pgfpathlineto{\pgfqpoint{1.060974in}{2.111657in}}%
\pgfpathlineto{\pgfqpoint{1.062197in}{1.988234in}}%
\pgfpathlineto{\pgfqpoint{1.063012in}{2.152798in}}%
\pgfpathlineto{\pgfqpoint{1.063420in}{2.111657in}}%
\pgfpathlineto{\pgfqpoint{1.063827in}{2.210396in}}%
\pgfpathlineto{\pgfqpoint{1.064439in}{2.103429in}}%
\pgfpathlineto{\pgfqpoint{1.065865in}{1.971778in}}%
\pgfpathlineto{\pgfqpoint{1.064846in}{2.111657in}}%
\pgfpathlineto{\pgfqpoint{1.066273in}{1.980006in}}%
\pgfpathlineto{\pgfqpoint{1.066884in}{1.963550in}}%
\pgfpathlineto{\pgfqpoint{1.067292in}{2.004691in}}%
\pgfpathlineto{\pgfqpoint{1.067699in}{2.045832in}}%
\pgfpathlineto{\pgfqpoint{1.068515in}{1.938865in}}%
\pgfpathlineto{\pgfqpoint{1.069126in}{2.029375in}}%
\pgfpathlineto{\pgfqpoint{1.069534in}{1.905952in}}%
\pgfpathlineto{\pgfqpoint{1.071368in}{1.766073in}}%
\pgfpathlineto{\pgfqpoint{1.072591in}{1.873040in}}%
\pgfpathlineto{\pgfqpoint{1.072794in}{1.873040in}}%
\pgfpathlineto{\pgfqpoint{1.072998in}{1.856583in}}%
\pgfpathlineto{\pgfqpoint{1.073406in}{1.955322in}}%
\pgfpathlineto{\pgfqpoint{1.074017in}{1.938865in}}%
\pgfpathlineto{\pgfqpoint{1.075240in}{1.848355in}}%
\pgfpathlineto{\pgfqpoint{1.076463in}{1.971778in}}%
\pgfpathlineto{\pgfqpoint{1.076667in}{1.988234in}}%
\pgfpathlineto{\pgfqpoint{1.076871in}{1.955322in}}%
\pgfpathlineto{\pgfqpoint{1.077074in}{1.889496in}}%
\pgfpathlineto{\pgfqpoint{1.077278in}{1.996463in}}%
\pgfpathlineto{\pgfqpoint{1.077890in}{1.922409in}}%
\pgfpathlineto{\pgfqpoint{1.078297in}{1.864811in}}%
\pgfpathlineto{\pgfqpoint{1.078909in}{1.963550in}}%
\pgfpathlineto{\pgfqpoint{1.079112in}{1.741388in}}%
\pgfpathlineto{\pgfqpoint{1.079316in}{1.980006in}}%
\pgfpathlineto{\pgfqpoint{1.079928in}{1.889496in}}%
\pgfpathlineto{\pgfqpoint{1.080539in}{1.955322in}}%
\pgfpathlineto{\pgfqpoint{1.080335in}{1.881268in}}%
\pgfpathlineto{\pgfqpoint{1.080947in}{1.930637in}}%
\pgfpathlineto{\pgfqpoint{1.081354in}{1.873040in}}%
\pgfpathlineto{\pgfqpoint{1.081558in}{2.004691in}}%
\pgfpathlineto{\pgfqpoint{1.081762in}{2.054060in}}%
\pgfpathlineto{\pgfqpoint{1.082373in}{2.004691in}}%
\pgfpathlineto{\pgfqpoint{1.083800in}{1.873040in}}%
\pgfpathlineto{\pgfqpoint{1.084004in}{1.873040in}}%
\pgfpathlineto{\pgfqpoint{1.085023in}{1.971778in}}%
\pgfpathlineto{\pgfqpoint{1.085430in}{1.930637in}}%
\pgfpathlineto{\pgfqpoint{1.086653in}{1.815442in}}%
\pgfpathlineto{\pgfqpoint{1.088080in}{2.021147in}}%
\pgfpathlineto{\pgfqpoint{1.089303in}{1.757845in}}%
\pgfpathlineto{\pgfqpoint{1.090118in}{2.037604in}}%
\pgfpathlineto{\pgfqpoint{1.090525in}{1.980006in}}%
\pgfpathlineto{\pgfqpoint{1.090933in}{1.724932in}}%
\pgfpathlineto{\pgfqpoint{1.091137in}{1.996463in}}%
\pgfpathlineto{\pgfqpoint{1.091544in}{1.996463in}}%
\pgfpathlineto{\pgfqpoint{1.092156in}{2.045832in}}%
\pgfpathlineto{\pgfqpoint{1.092563in}{2.037604in}}%
\pgfpathlineto{\pgfqpoint{1.094194in}{1.864811in}}%
\pgfpathlineto{\pgfqpoint{1.094805in}{1.930637in}}%
\pgfpathlineto{\pgfqpoint{1.095213in}{1.873040in}}%
\pgfpathlineto{\pgfqpoint{1.095620in}{1.873040in}}%
\pgfpathlineto{\pgfqpoint{1.095824in}{1.840127in}}%
\pgfpathlineto{\pgfqpoint{1.096028in}{1.905952in}}%
\pgfpathlineto{\pgfqpoint{1.096232in}{1.897724in}}%
\pgfpathlineto{\pgfqpoint{1.097047in}{2.086973in}}%
\pgfpathlineto{\pgfqpoint{1.097455in}{1.963550in}}%
\pgfpathlineto{\pgfqpoint{1.097862in}{1.905952in}}%
\pgfpathlineto{\pgfqpoint{1.098270in}{1.922409in}}%
\pgfpathlineto{\pgfqpoint{1.098881in}{2.078745in}}%
\pgfpathlineto{\pgfqpoint{1.099289in}{1.971778in}}%
\pgfpathlineto{\pgfqpoint{1.100104in}{1.905952in}}%
\pgfpathlineto{\pgfqpoint{1.100512in}{1.914181in}}%
\pgfpathlineto{\pgfqpoint{1.101327in}{1.988234in}}%
\pgfpathlineto{\pgfqpoint{1.101735in}{1.938865in}}%
\pgfpathlineto{\pgfqpoint{1.102957in}{2.012919in}}%
\pgfpathlineto{\pgfqpoint{1.103365in}{1.930637in}}%
\pgfpathlineto{\pgfqpoint{1.103976in}{2.004691in}}%
\pgfpathlineto{\pgfqpoint{1.104384in}{1.996463in}}%
\pgfpathlineto{\pgfqpoint{1.104588in}{2.021147in}}%
\pgfpathlineto{\pgfqpoint{1.104792in}{2.012919in}}%
\pgfpathlineto{\pgfqpoint{1.105403in}{2.086973in}}%
\pgfpathlineto{\pgfqpoint{1.105607in}{2.004691in}}%
\pgfpathlineto{\pgfqpoint{1.107033in}{1.848355in}}%
\pgfpathlineto{\pgfqpoint{1.106218in}{2.029375in}}%
\pgfpathlineto{\pgfqpoint{1.107645in}{1.889496in}}%
\pgfpathlineto{\pgfqpoint{1.107849in}{1.947093in}}%
\pgfpathlineto{\pgfqpoint{1.108052in}{1.881268in}}%
\pgfpathlineto{\pgfqpoint{1.108460in}{1.914181in}}%
\pgfpathlineto{\pgfqpoint{1.109479in}{1.782529in}}%
\pgfpathlineto{\pgfqpoint{1.109683in}{1.848355in}}%
\pgfpathlineto{\pgfqpoint{1.111109in}{1.675563in}}%
\pgfpathlineto{\pgfqpoint{1.111721in}{1.766073in}}%
\pgfpathlineto{\pgfqpoint{1.112128in}{1.692019in}}%
\pgfpathlineto{\pgfqpoint{1.112332in}{1.700247in}}%
\pgfpathlineto{\pgfqpoint{1.113147in}{1.692019in}}%
\pgfpathlineto{\pgfqpoint{1.113555in}{1.782529in}}%
\pgfpathlineto{\pgfqpoint{1.114167in}{1.831899in}}%
\pgfpathlineto{\pgfqpoint{1.114370in}{1.782529in}}%
\pgfpathlineto{\pgfqpoint{1.114574in}{1.766073in}}%
\pgfpathlineto{\pgfqpoint{1.114982in}{1.815442in}}%
\pgfpathlineto{\pgfqpoint{1.115186in}{1.864811in}}%
\pgfpathlineto{\pgfqpoint{1.115797in}{1.749617in}}%
\pgfpathlineto{\pgfqpoint{1.116612in}{1.552140in}}%
\pgfpathlineto{\pgfqpoint{1.116816in}{1.700247in}}%
\pgfpathlineto{\pgfqpoint{1.117835in}{1.576824in}}%
\pgfpathlineto{\pgfqpoint{1.118243in}{1.609737in}}%
\pgfpathlineto{\pgfqpoint{1.119669in}{1.766073in}}%
\pgfpathlineto{\pgfqpoint{1.120484in}{1.502771in}}%
\pgfpathlineto{\pgfqpoint{1.121096in}{1.634422in}}%
\pgfpathlineto{\pgfqpoint{1.121300in}{1.634422in}}%
\pgfpathlineto{\pgfqpoint{1.123134in}{1.774301in}}%
\pgfpathlineto{\pgfqpoint{1.123541in}{1.815442in}}%
\pgfpathlineto{\pgfqpoint{1.123745in}{1.782529in}}%
\pgfpathlineto{\pgfqpoint{1.124560in}{1.700247in}}%
\pgfpathlineto{\pgfqpoint{1.125172in}{1.733160in}}%
\pgfpathlineto{\pgfqpoint{1.125783in}{1.692019in}}%
\pgfpathlineto{\pgfqpoint{1.126191in}{1.642650in}}%
\pgfpathlineto{\pgfqpoint{1.126802in}{1.700247in}}%
\pgfpathlineto{\pgfqpoint{1.127618in}{1.675563in}}%
\pgfpathlineto{\pgfqpoint{1.128229in}{1.774301in}}%
\pgfpathlineto{\pgfqpoint{1.130063in}{1.642650in}}%
\pgfpathlineto{\pgfqpoint{1.130675in}{1.724932in}}%
\pgfpathlineto{\pgfqpoint{1.131082in}{1.634422in}}%
\pgfpathlineto{\pgfqpoint{1.131286in}{1.700247in}}%
\pgfpathlineto{\pgfqpoint{1.131694in}{1.535683in}}%
\pgfpathlineto{\pgfqpoint{1.132509in}{1.667335in}}%
\pgfpathlineto{\pgfqpoint{1.132713in}{1.659106in}}%
\pgfpathlineto{\pgfqpoint{1.132916in}{1.733160in}}%
\pgfpathlineto{\pgfqpoint{1.133120in}{1.617965in}}%
\pgfpathlineto{\pgfqpoint{1.133732in}{1.659106in}}%
\pgfpathlineto{\pgfqpoint{1.133935in}{1.642650in}}%
\pgfpathlineto{\pgfqpoint{1.134139in}{1.716704in}}%
\pgfpathlineto{\pgfqpoint{1.134751in}{1.634422in}}%
\pgfpathlineto{\pgfqpoint{1.135973in}{1.552140in}}%
\pgfpathlineto{\pgfqpoint{1.136177in}{1.552140in}}%
\pgfpathlineto{\pgfqpoint{1.136585in}{1.461630in}}%
\pgfpathlineto{\pgfqpoint{1.137196in}{1.486314in}}%
\pgfpathlineto{\pgfqpoint{1.137400in}{1.593281in}}%
\pgfpathlineto{\pgfqpoint{1.138215in}{1.527455in}}%
\pgfpathlineto{\pgfqpoint{1.138419in}{1.494542in}}%
\pgfpathlineto{\pgfqpoint{1.138623in}{1.585053in}}%
\pgfpathlineto{\pgfqpoint{1.139030in}{1.519227in}}%
\pgfpathlineto{\pgfqpoint{1.140253in}{1.667335in}}%
\pgfpathlineto{\pgfqpoint{1.141680in}{1.535683in}}%
\pgfpathlineto{\pgfqpoint{1.142088in}{1.527455in}}%
\pgfpathlineto{\pgfqpoint{1.142903in}{1.576824in}}%
\pgfpathlineto{\pgfqpoint{1.143718in}{1.379348in}}%
\pgfpathlineto{\pgfqpoint{1.144126in}{1.494542in}}%
\pgfpathlineto{\pgfqpoint{1.144737in}{1.395804in}}%
\pgfpathlineto{\pgfqpoint{1.145348in}{1.453401in}}%
\pgfpathlineto{\pgfqpoint{1.145552in}{1.445173in}}%
\pgfpathlineto{\pgfqpoint{1.145756in}{1.486314in}}%
\pgfpathlineto{\pgfqpoint{1.146164in}{1.371119in}}%
\pgfpathlineto{\pgfqpoint{1.146571in}{1.453401in}}%
\pgfpathlineto{\pgfqpoint{1.147794in}{1.379348in}}%
\pgfpathlineto{\pgfqpoint{1.148405in}{1.420489in}}%
\pgfpathlineto{\pgfqpoint{1.149017in}{1.395804in}}%
\pgfpathlineto{\pgfqpoint{1.149221in}{1.404032in}}%
\pgfpathlineto{\pgfqpoint{1.150036in}{1.535683in}}%
\pgfpathlineto{\pgfqpoint{1.150647in}{1.502771in}}%
\pgfpathlineto{\pgfqpoint{1.151259in}{1.527455in}}%
\pgfpathlineto{\pgfqpoint{1.151870in}{1.428717in}}%
\pgfpathlineto{\pgfqpoint{1.152074in}{1.428717in}}%
\pgfpathlineto{\pgfqpoint{1.152481in}{1.519227in}}%
\pgfpathlineto{\pgfqpoint{1.152685in}{1.264153in}}%
\pgfpathlineto{\pgfqpoint{1.153093in}{1.601509in}}%
\pgfpathlineto{\pgfqpoint{1.153500in}{1.494542in}}%
\pgfpathlineto{\pgfqpoint{1.154316in}{1.585053in}}%
\pgfpathlineto{\pgfqpoint{1.154723in}{1.535683in}}%
\pgfpathlineto{\pgfqpoint{1.154927in}{1.486314in}}%
\pgfpathlineto{\pgfqpoint{1.155131in}{1.560368in}}%
\pgfpathlineto{\pgfqpoint{1.155335in}{1.519227in}}%
\pgfpathlineto{\pgfqpoint{1.155539in}{1.626194in}}%
\pgfpathlineto{\pgfqpoint{1.156354in}{1.535683in}}%
\pgfpathlineto{\pgfqpoint{1.157169in}{1.552140in}}%
\pgfpathlineto{\pgfqpoint{1.158392in}{1.404032in}}%
\pgfpathlineto{\pgfqpoint{1.158596in}{1.445173in}}%
\pgfpathlineto{\pgfqpoint{1.159003in}{1.453401in}}%
\pgfpathlineto{\pgfqpoint{1.160226in}{1.576824in}}%
\pgfpathlineto{\pgfqpoint{1.160430in}{1.568596in}}%
\pgfpathlineto{\pgfqpoint{1.160837in}{1.519227in}}%
\pgfpathlineto{\pgfqpoint{1.161245in}{1.593281in}}%
\pgfpathlineto{\pgfqpoint{1.161653in}{1.576824in}}%
\pgfpathlineto{\pgfqpoint{1.163691in}{1.798986in}}%
\pgfpathlineto{\pgfqpoint{1.164506in}{1.626194in}}%
\pgfpathlineto{\pgfqpoint{1.164913in}{1.733160in}}%
\pgfpathlineto{\pgfqpoint{1.165117in}{1.692019in}}%
\pgfpathlineto{\pgfqpoint{1.165729in}{1.412260in}}%
\pgfpathlineto{\pgfqpoint{1.166136in}{1.659106in}}%
\pgfpathlineto{\pgfqpoint{1.166544in}{1.683791in}}%
\pgfpathlineto{\pgfqpoint{1.167359in}{1.585053in}}%
\pgfpathlineto{\pgfqpoint{1.168174in}{1.626194in}}%
\pgfpathlineto{\pgfqpoint{1.168582in}{1.601509in}}%
\pgfpathlineto{\pgfqpoint{1.168786in}{1.362891in}}%
\pgfpathlineto{\pgfqpoint{1.169397in}{1.766073in}}%
\pgfpathlineto{\pgfqpoint{1.170824in}{1.642650in}}%
\pgfpathlineto{\pgfqpoint{1.171028in}{1.692019in}}%
\pgfpathlineto{\pgfqpoint{1.171231in}{1.667335in}}%
\pgfpathlineto{\pgfqpoint{1.171435in}{1.428717in}}%
\pgfpathlineto{\pgfqpoint{1.172250in}{1.626194in}}%
\pgfpathlineto{\pgfqpoint{1.172454in}{1.634422in}}%
\pgfpathlineto{\pgfqpoint{1.173066in}{1.527455in}}%
\pgfpathlineto{\pgfqpoint{1.173473in}{1.585053in}}%
\pgfpathlineto{\pgfqpoint{1.173881in}{1.650878in}}%
\pgfpathlineto{\pgfqpoint{1.174085in}{1.601509in}}%
\pgfpathlineto{\pgfqpoint{1.175919in}{1.453401in}}%
\pgfpathlineto{\pgfqpoint{1.177345in}{1.585053in}}%
\pgfpathlineto{\pgfqpoint{1.177957in}{1.576824in}}%
\pgfpathlineto{\pgfqpoint{1.178976in}{1.329978in}}%
\pgfpathlineto{\pgfqpoint{1.179180in}{1.362891in}}%
\pgfpathlineto{\pgfqpoint{1.179383in}{1.601509in}}%
\pgfpathlineto{\pgfqpoint{1.180402in}{1.543912in}}%
\pgfpathlineto{\pgfqpoint{1.181422in}{1.634422in}}%
\pgfpathlineto{\pgfqpoint{1.181014in}{1.535683in}}%
\pgfpathlineto{\pgfqpoint{1.181625in}{1.568596in}}%
\pgfpathlineto{\pgfqpoint{1.182441in}{1.543912in}}%
\pgfpathlineto{\pgfqpoint{1.183460in}{1.617965in}}%
\pgfpathlineto{\pgfqpoint{1.183663in}{1.585053in}}%
\pgfpathlineto{\pgfqpoint{1.184479in}{1.510999in}}%
\pgfpathlineto{\pgfqpoint{1.184682in}{1.576824in}}%
\pgfpathlineto{\pgfqpoint{1.185090in}{1.519227in}}%
\pgfpathlineto{\pgfqpoint{1.185294in}{1.428717in}}%
\pgfpathlineto{\pgfqpoint{1.186109in}{1.519227in}}%
\pgfpathlineto{\pgfqpoint{1.186313in}{1.527455in}}%
\pgfpathlineto{\pgfqpoint{1.186517in}{1.362891in}}%
\pgfpathlineto{\pgfqpoint{1.186720in}{1.560368in}}%
\pgfpathlineto{\pgfqpoint{1.187332in}{1.552140in}}%
\pgfpathlineto{\pgfqpoint{1.187536in}{1.469858in}}%
\pgfpathlineto{\pgfqpoint{1.187739in}{1.601509in}}%
\pgfpathlineto{\pgfqpoint{1.188351in}{1.576824in}}%
\pgfpathlineto{\pgfqpoint{1.189777in}{1.494542in}}%
\pgfpathlineto{\pgfqpoint{1.190593in}{1.593281in}}%
\pgfpathlineto{\pgfqpoint{1.190185in}{1.478086in}}%
\pgfpathlineto{\pgfqpoint{1.190796in}{1.560368in}}%
\pgfpathlineto{\pgfqpoint{1.191000in}{1.469858in}}%
\pgfpathlineto{\pgfqpoint{1.192019in}{1.494542in}}%
\pgfpathlineto{\pgfqpoint{1.192223in}{1.494542in}}%
\pgfpathlineto{\pgfqpoint{1.192631in}{1.527455in}}%
\pgfpathlineto{\pgfqpoint{1.192834in}{1.469858in}}%
\pgfpathlineto{\pgfqpoint{1.193038in}{1.469858in}}%
\pgfpathlineto{\pgfqpoint{1.193242in}{1.469858in}}%
\pgfpathlineto{\pgfqpoint{1.193446in}{1.453401in}}%
\pgfpathlineto{\pgfqpoint{1.193650in}{1.280609in}}%
\pgfpathlineto{\pgfqpoint{1.194057in}{1.601509in}}%
\pgfpathlineto{\pgfqpoint{1.194465in}{1.486314in}}%
\pgfpathlineto{\pgfqpoint{1.195484in}{1.568596in}}%
\pgfpathlineto{\pgfqpoint{1.196503in}{1.552140in}}%
\pgfpathlineto{\pgfqpoint{1.196707in}{1.642650in}}%
\pgfpathlineto{\pgfqpoint{1.197726in}{1.626194in}}%
\pgfpathlineto{\pgfqpoint{1.197930in}{1.617965in}}%
\pgfpathlineto{\pgfqpoint{1.198133in}{1.510999in}}%
\pgfpathlineto{\pgfqpoint{1.198949in}{1.568596in}}%
\pgfpathlineto{\pgfqpoint{1.199152in}{1.568596in}}%
\pgfpathlineto{\pgfqpoint{1.199764in}{1.494542in}}%
\pgfpathlineto{\pgfqpoint{1.200171in}{1.535683in}}%
\pgfpathlineto{\pgfqpoint{1.201802in}{1.683791in}}%
\pgfpathlineto{\pgfqpoint{1.202006in}{1.642650in}}%
\pgfpathlineto{\pgfqpoint{1.203025in}{1.469858in}}%
\pgfpathlineto{\pgfqpoint{1.203228in}{1.552140in}}%
\pgfpathlineto{\pgfqpoint{1.203636in}{1.576824in}}%
\pgfpathlineto{\pgfqpoint{1.204044in}{1.535683in}}%
\pgfpathlineto{\pgfqpoint{1.204859in}{1.601509in}}%
\pgfpathlineto{\pgfqpoint{1.205470in}{1.362891in}}%
\pgfpathlineto{\pgfqpoint{1.205674in}{1.297066in}}%
\pgfpathlineto{\pgfqpoint{1.205878in}{1.560368in}}%
\pgfpathlineto{\pgfqpoint{1.206285in}{1.428717in}}%
\pgfpathlineto{\pgfqpoint{1.206489in}{1.412260in}}%
\pgfpathlineto{\pgfqpoint{1.206897in}{1.173643in}}%
\pgfpathlineto{\pgfqpoint{1.207508in}{1.412260in}}%
\pgfpathlineto{\pgfqpoint{1.207712in}{1.436945in}}%
\pgfpathlineto{\pgfqpoint{1.209139in}{1.214784in}}%
\pgfpathlineto{\pgfqpoint{1.209546in}{1.099589in}}%
\pgfpathlineto{\pgfqpoint{1.210158in}{1.140730in}}%
\pgfpathlineto{\pgfqpoint{1.210973in}{1.083132in}}%
\pgfpathlineto{\pgfqpoint{1.211177in}{1.255925in}}%
\pgfpathlineto{\pgfqpoint{1.211584in}{1.083132in}}%
\pgfpathlineto{\pgfqpoint{1.212400in}{1.124273in}}%
\pgfpathlineto{\pgfqpoint{1.214030in}{1.280609in}}%
\pgfpathlineto{\pgfqpoint{1.215457in}{1.132502in}}%
\pgfpathlineto{\pgfqpoint{1.215660in}{1.190099in}}%
\pgfpathlineto{\pgfqpoint{1.216272in}{1.074904in}}%
\pgfpathlineto{\pgfqpoint{1.217291in}{1.124273in}}%
\pgfpathlineto{\pgfqpoint{1.216679in}{1.050220in}}%
\pgfpathlineto{\pgfqpoint{1.217698in}{1.116045in}}%
\pgfpathlineto{\pgfqpoint{1.217902in}{1.107817in}}%
\pgfpathlineto{\pgfqpoint{1.219329in}{0.819830in}}%
\pgfpathlineto{\pgfqpoint{1.220144in}{1.050220in}}%
\pgfpathlineto{\pgfqpoint{1.220552in}{1.033763in}}%
\pgfpathlineto{\pgfqpoint{1.220755in}{1.033763in}}%
\pgfpathlineto{\pgfqpoint{1.220959in}{1.009079in}}%
\pgfpathlineto{\pgfqpoint{1.221367in}{1.165414in}}%
\pgfpathlineto{\pgfqpoint{1.221978in}{1.058448in}}%
\pgfpathlineto{\pgfqpoint{1.223201in}{0.976166in}}%
\pgfpathlineto{\pgfqpoint{1.224016in}{1.272381in}}%
\pgfpathlineto{\pgfqpoint{1.224424in}{1.165414in}}%
\pgfpathlineto{\pgfqpoint{1.225647in}{1.009079in}}%
\pgfpathlineto{\pgfqpoint{1.225851in}{1.033763in}}%
\pgfpathlineto{\pgfqpoint{1.226666in}{0.836286in}}%
\pgfpathlineto{\pgfqpoint{1.227073in}{0.910340in}}%
\pgfpathlineto{\pgfqpoint{1.227481in}{0.860971in}}%
\pgfpathlineto{\pgfqpoint{1.228296in}{0.877427in}}%
\pgfpathlineto{\pgfqpoint{1.228500in}{0.943253in}}%
\pgfpathlineto{\pgfqpoint{1.229519in}{0.935025in}}%
\pgfpathlineto{\pgfqpoint{1.229723in}{0.893884in}}%
\pgfpathlineto{\pgfqpoint{1.230334in}{0.984394in}}%
\pgfpathlineto{\pgfqpoint{1.230538in}{0.951481in}}%
\pgfpathlineto{\pgfqpoint{1.231557in}{1.066676in}}%
\pgfpathlineto{\pgfqpoint{1.232168in}{1.041991in}}%
\pgfpathlineto{\pgfqpoint{1.232576in}{0.935025in}}%
\pgfpathlineto{\pgfqpoint{1.233187in}{1.050220in}}%
\pgfpathlineto{\pgfqpoint{1.233799in}{1.017307in}}%
\pgfpathlineto{\pgfqpoint{1.234003in}{1.033763in}}%
\pgfpathlineto{\pgfqpoint{1.234206in}{1.099589in}}%
\pgfpathlineto{\pgfqpoint{1.234818in}{1.017307in}}%
\pgfpathlineto{\pgfqpoint{1.235022in}{1.017307in}}%
\pgfpathlineto{\pgfqpoint{1.235429in}{1.058448in}}%
\pgfpathlineto{\pgfqpoint{1.235633in}{1.009079in}}%
\pgfpathlineto{\pgfqpoint{1.237060in}{0.885656in}}%
\pgfpathlineto{\pgfqpoint{1.237467in}{1.025535in}}%
\pgfpathlineto{\pgfqpoint{1.238690in}{1.009079in}}%
\pgfpathlineto{\pgfqpoint{1.239505in}{0.910340in}}%
\pgfpathlineto{\pgfqpoint{1.239098in}{1.041991in}}%
\pgfpathlineto{\pgfqpoint{1.240117in}{0.943253in}}%
\pgfpathlineto{\pgfqpoint{1.241747in}{1.066676in}}%
\pgfpathlineto{\pgfqpoint{1.240524in}{0.935025in}}%
\pgfpathlineto{\pgfqpoint{1.241951in}{1.009079in}}%
\pgfpathlineto{\pgfqpoint{1.242155in}{1.017307in}}%
\pgfpathlineto{\pgfqpoint{1.242359in}{0.992622in}}%
\pgfpathlineto{\pgfqpoint{1.243581in}{1.173643in}}%
\pgfpathlineto{\pgfqpoint{1.243785in}{1.099589in}}%
\pgfpathlineto{\pgfqpoint{1.244193in}{1.280609in}}%
\pgfpathlineto{\pgfqpoint{1.244397in}{1.255925in}}%
\pgfpathlineto{\pgfqpoint{1.246638in}{1.510999in}}%
\pgfpathlineto{\pgfqpoint{1.248269in}{1.387576in}}%
\pgfpathlineto{\pgfqpoint{1.248473in}{1.445173in}}%
\pgfpathlineto{\pgfqpoint{1.248677in}{1.214784in}}%
\pgfpathlineto{\pgfqpoint{1.249492in}{1.502771in}}%
\pgfpathlineto{\pgfqpoint{1.250103in}{1.535683in}}%
\pgfpathlineto{\pgfqpoint{1.250511in}{1.510999in}}%
\pgfpathlineto{\pgfqpoint{1.251122in}{1.486314in}}%
\pgfpathlineto{\pgfqpoint{1.251326in}{1.519227in}}%
\pgfpathlineto{\pgfqpoint{1.251530in}{1.502771in}}%
\pgfpathlineto{\pgfqpoint{1.251734in}{1.543912in}}%
\pgfpathlineto{\pgfqpoint{1.252345in}{1.494542in}}%
\pgfpathlineto{\pgfqpoint{1.252753in}{1.527455in}}%
\pgfpathlineto{\pgfqpoint{1.252956in}{1.502771in}}%
\pgfpathlineto{\pgfqpoint{1.253160in}{1.601509in}}%
\pgfpathlineto{\pgfqpoint{1.253364in}{1.560368in}}%
\pgfpathlineto{\pgfqpoint{1.254179in}{1.733160in}}%
\pgfpathlineto{\pgfqpoint{1.254994in}{1.708476in}}%
\pgfpathlineto{\pgfqpoint{1.255198in}{1.683791in}}%
\pgfpathlineto{\pgfqpoint{1.255606in}{1.741388in}}%
\pgfpathlineto{\pgfqpoint{1.256217in}{1.848355in}}%
\pgfpathlineto{\pgfqpoint{1.256625in}{1.815442in}}%
\pgfpathlineto{\pgfqpoint{1.257236in}{1.733160in}}%
\pgfpathlineto{\pgfqpoint{1.257644in}{1.757845in}}%
\pgfpathlineto{\pgfqpoint{1.259070in}{1.889496in}}%
\pgfpathlineto{\pgfqpoint{1.260089in}{1.798986in}}%
\pgfpathlineto{\pgfqpoint{1.260293in}{1.807214in}}%
\pgfpathlineto{\pgfqpoint{1.261108in}{1.914181in}}%
\pgfpathlineto{\pgfqpoint{1.260701in}{1.798986in}}%
\pgfpathlineto{\pgfqpoint{1.261312in}{1.897724in}}%
\pgfpathlineto{\pgfqpoint{1.262535in}{1.815442in}}%
\pgfpathlineto{\pgfqpoint{1.263554in}{1.922409in}}%
\pgfpathlineto{\pgfqpoint{1.263147in}{1.798986in}}%
\pgfpathlineto{\pgfqpoint{1.263758in}{1.881268in}}%
\pgfpathlineto{\pgfqpoint{1.263962in}{1.856583in}}%
\pgfpathlineto{\pgfqpoint{1.264166in}{1.905952in}}%
\pgfpathlineto{\pgfqpoint{1.264573in}{1.905952in}}%
\pgfpathlineto{\pgfqpoint{1.264777in}{1.963550in}}%
\pgfpathlineto{\pgfqpoint{1.265185in}{1.881268in}}%
\pgfpathlineto{\pgfqpoint{1.265592in}{1.897724in}}%
\pgfpathlineto{\pgfqpoint{1.265796in}{1.905952in}}%
\pgfpathlineto{\pgfqpoint{1.266000in}{1.881268in}}%
\pgfpathlineto{\pgfqpoint{1.266204in}{1.889496in}}%
\pgfpathlineto{\pgfqpoint{1.266611in}{1.840127in}}%
\pgfpathlineto{\pgfqpoint{1.266815in}{1.897724in}}%
\pgfpathlineto{\pgfqpoint{1.267019in}{1.897724in}}%
\pgfpathlineto{\pgfqpoint{1.267630in}{2.045832in}}%
\pgfpathlineto{\pgfqpoint{1.268445in}{1.996463in}}%
\pgfpathlineto{\pgfqpoint{1.268649in}{2.012919in}}%
\pgfpathlineto{\pgfqpoint{1.268853in}{1.988234in}}%
\pgfpathlineto{\pgfqpoint{1.269057in}{1.922409in}}%
\pgfpathlineto{\pgfqpoint{1.269872in}{2.012919in}}%
\pgfpathlineto{\pgfqpoint{1.270280in}{1.971778in}}%
\pgfpathlineto{\pgfqpoint{1.270891in}{2.029375in}}%
\pgfpathlineto{\pgfqpoint{1.271095in}{2.045832in}}%
\pgfpathlineto{\pgfqpoint{1.271910in}{1.938865in}}%
\pgfpathlineto{\pgfqpoint{1.272114in}{1.996463in}}%
\pgfpathlineto{\pgfqpoint{1.272929in}{2.095201in}}%
\pgfpathlineto{\pgfqpoint{1.273337in}{2.078745in}}%
\pgfpathlineto{\pgfqpoint{1.273948in}{2.103429in}}%
\pgfpathlineto{\pgfqpoint{1.274559in}{2.012919in}}%
\pgfpathlineto{\pgfqpoint{1.274763in}{2.078745in}}%
\pgfpathlineto{\pgfqpoint{1.275375in}{1.971778in}}%
\pgfpathlineto{\pgfqpoint{1.275579in}{2.029375in}}%
\pgfpathlineto{\pgfqpoint{1.275782in}{2.012919in}}%
\pgfpathlineto{\pgfqpoint{1.275986in}{2.037604in}}%
\pgfpathlineto{\pgfqpoint{1.277209in}{2.103429in}}%
\pgfpathlineto{\pgfqpoint{1.277413in}{2.103429in}}%
\pgfpathlineto{\pgfqpoint{1.277617in}{2.177483in}}%
\pgfpathlineto{\pgfqpoint{1.278636in}{2.144570in}}%
\pgfpathlineto{\pgfqpoint{1.278839in}{2.185711in}}%
\pgfpathlineto{\pgfqpoint{1.279247in}{2.136342in}}%
\pgfpathlineto{\pgfqpoint{1.280470in}{1.881268in}}%
\pgfpathlineto{\pgfqpoint{1.281693in}{2.152798in}}%
\pgfpathlineto{\pgfqpoint{1.282508in}{1.922409in}}%
\pgfpathlineto{\pgfqpoint{1.282915in}{2.111657in}}%
\pgfpathlineto{\pgfqpoint{1.283934in}{2.185711in}}%
\pgfpathlineto{\pgfqpoint{1.284138in}{2.136342in}}%
\pgfpathlineto{\pgfqpoint{1.284750in}{2.235080in}}%
\pgfpathlineto{\pgfqpoint{1.284953in}{2.267993in}}%
\pgfpathlineto{\pgfqpoint{1.285157in}{2.202168in}}%
\pgfpathlineto{\pgfqpoint{1.285769in}{2.235080in}}%
\pgfpathlineto{\pgfqpoint{1.285972in}{2.235080in}}%
\pgfpathlineto{\pgfqpoint{1.286788in}{2.095201in}}%
\pgfpathlineto{\pgfqpoint{1.287399in}{2.169255in}}%
\pgfpathlineto{\pgfqpoint{1.287603in}{2.177483in}}%
\pgfpathlineto{\pgfqpoint{1.287807in}{2.152798in}}%
\pgfpathlineto{\pgfqpoint{1.288214in}{2.037604in}}%
\pgfpathlineto{\pgfqpoint{1.289030in}{2.070516in}}%
\pgfpathlineto{\pgfqpoint{1.289233in}{2.070516in}}%
\pgfpathlineto{\pgfqpoint{1.289437in}{2.062288in}}%
\pgfpathlineto{\pgfqpoint{1.289641in}{2.086973in}}%
\pgfpathlineto{\pgfqpoint{1.290049in}{2.152798in}}%
\pgfpathlineto{\pgfqpoint{1.290252in}{2.095201in}}%
\pgfpathlineto{\pgfqpoint{1.290660in}{2.012919in}}%
\pgfpathlineto{\pgfqpoint{1.291271in}{2.119886in}}%
\pgfpathlineto{\pgfqpoint{1.291883in}{2.062288in}}%
\pgfpathlineto{\pgfqpoint{1.292087in}{2.136342in}}%
\pgfpathlineto{\pgfqpoint{1.292290in}{2.103429in}}%
\pgfpathlineto{\pgfqpoint{1.292902in}{2.136342in}}%
\pgfpathlineto{\pgfqpoint{1.293106in}{2.086973in}}%
\pgfpathlineto{\pgfqpoint{1.293309in}{2.095201in}}%
\pgfpathlineto{\pgfqpoint{1.293513in}{2.062288in}}%
\pgfpathlineto{\pgfqpoint{1.294328in}{2.037604in}}%
\pgfpathlineto{\pgfqpoint{1.295144in}{2.103429in}}%
\pgfpathlineto{\pgfqpoint{1.295551in}{1.980006in}}%
\pgfpathlineto{\pgfqpoint{1.296366in}{2.045832in}}%
\pgfpathlineto{\pgfqpoint{1.296978in}{1.996463in}}%
\pgfpathlineto{\pgfqpoint{1.297182in}{2.078745in}}%
\pgfpathlineto{\pgfqpoint{1.297589in}{2.111657in}}%
\pgfpathlineto{\pgfqpoint{1.297997in}{2.078745in}}%
\pgfpathlineto{\pgfqpoint{1.298812in}{2.095201in}}%
\pgfpathlineto{\pgfqpoint{1.299220in}{1.988234in}}%
\pgfpathlineto{\pgfqpoint{1.300239in}{2.136342in}}%
\pgfpathlineto{\pgfqpoint{1.300442in}{2.111657in}}%
\pgfpathlineto{\pgfqpoint{1.300646in}{2.037604in}}%
\pgfpathlineto{\pgfqpoint{1.301258in}{2.144570in}}%
\pgfpathlineto{\pgfqpoint{1.301665in}{2.062288in}}%
\pgfpathlineto{\pgfqpoint{1.302073in}{2.161027in}}%
\pgfpathlineto{\pgfqpoint{1.302684in}{2.119886in}}%
\pgfpathlineto{\pgfqpoint{1.303907in}{2.045832in}}%
\pgfpathlineto{\pgfqpoint{1.304315in}{2.062288in}}%
\pgfpathlineto{\pgfqpoint{1.304519in}{2.029375in}}%
\pgfpathlineto{\pgfqpoint{1.304722in}{1.996463in}}%
\pgfpathlineto{\pgfqpoint{1.304926in}{2.111657in}}%
\pgfpathlineto{\pgfqpoint{1.305130in}{2.086973in}}%
\pgfpathlineto{\pgfqpoint{1.305741in}{2.144570in}}%
\pgfpathlineto{\pgfqpoint{1.305945in}{2.119886in}}%
\pgfpathlineto{\pgfqpoint{1.306964in}{2.037604in}}%
\pgfpathlineto{\pgfqpoint{1.307168in}{2.045832in}}%
\pgfpathlineto{\pgfqpoint{1.307372in}{2.037604in}}%
\pgfpathlineto{\pgfqpoint{1.307576in}{2.054060in}}%
\pgfpathlineto{\pgfqpoint{1.307779in}{2.111657in}}%
\pgfpathlineto{\pgfqpoint{1.308391in}{2.021147in}}%
\pgfpathlineto{\pgfqpoint{1.308798in}{1.988234in}}%
\pgfpathlineto{\pgfqpoint{1.309206in}{2.029375in}}%
\pgfpathlineto{\pgfqpoint{1.309410in}{2.037604in}}%
\pgfpathlineto{\pgfqpoint{1.310429in}{2.136342in}}%
\pgfpathlineto{\pgfqpoint{1.310633in}{2.111657in}}%
\pgfpathlineto{\pgfqpoint{1.310836in}{2.111657in}}%
\pgfpathlineto{\pgfqpoint{1.312671in}{1.980006in}}%
\pgfpathlineto{\pgfqpoint{1.314097in}{2.103429in}}%
\pgfpathlineto{\pgfqpoint{1.314301in}{2.086973in}}%
\pgfpathlineto{\pgfqpoint{1.314505in}{2.119886in}}%
\pgfpathlineto{\pgfqpoint{1.315116in}{2.103429in}}%
\pgfpathlineto{\pgfqpoint{1.315524in}{2.193939in}}%
\pgfpathlineto{\pgfqpoint{1.315932in}{2.251537in}}%
\pgfpathlineto{\pgfqpoint{1.316747in}{2.095201in}}%
\pgfpathlineto{\pgfqpoint{1.316951in}{2.152798in}}%
\pgfpathlineto{\pgfqpoint{1.317970in}{2.128114in}}%
\pgfpathlineto{\pgfqpoint{1.318785in}{2.045832in}}%
\pgfpathlineto{\pgfqpoint{1.318581in}{2.136342in}}%
\pgfpathlineto{\pgfqpoint{1.319192in}{2.078745in}}%
\pgfpathlineto{\pgfqpoint{1.319396in}{2.078745in}}%
\pgfpathlineto{\pgfqpoint{1.320823in}{2.235080in}}%
\pgfpathlineto{\pgfqpoint{1.321434in}{2.276221in}}%
\pgfpathlineto{\pgfqpoint{1.322249in}{2.095201in}}%
\pgfpathlineto{\pgfqpoint{1.322453in}{2.070516in}}%
\pgfpathlineto{\pgfqpoint{1.322657in}{2.136342in}}%
\pgfpathlineto{\pgfqpoint{1.323065in}{2.119886in}}%
\pgfpathlineto{\pgfqpoint{1.325510in}{2.325591in}}%
\pgfpathlineto{\pgfqpoint{1.325714in}{2.276221in}}%
\pgfpathlineto{\pgfqpoint{1.327344in}{2.152798in}}%
\pgfpathlineto{\pgfqpoint{1.327956in}{2.226852in}}%
\pgfpathlineto{\pgfqpoint{1.328363in}{2.144570in}}%
\pgfpathlineto{\pgfqpoint{1.328567in}{2.193939in}}%
\pgfpathlineto{\pgfqpoint{1.330198in}{1.971778in}}%
\pgfpathlineto{\pgfqpoint{1.330809in}{2.226852in}}%
\pgfpathlineto{\pgfqpoint{1.331421in}{2.144570in}}%
\pgfpathlineto{\pgfqpoint{1.332032in}{2.070516in}}%
\pgfpathlineto{\pgfqpoint{1.332236in}{2.161027in}}%
\pgfpathlineto{\pgfqpoint{1.332440in}{2.161027in}}%
\pgfpathlineto{\pgfqpoint{1.333662in}{2.259765in}}%
\pgfpathlineto{\pgfqpoint{1.334885in}{2.161027in}}%
\pgfpathlineto{\pgfqpoint{1.335089in}{2.169255in}}%
\pgfpathlineto{\pgfqpoint{1.335293in}{2.193939in}}%
\pgfpathlineto{\pgfqpoint{1.335497in}{1.922409in}}%
\pgfpathlineto{\pgfqpoint{1.336312in}{2.144570in}}%
\pgfpathlineto{\pgfqpoint{1.336923in}{2.086973in}}%
\pgfpathlineto{\pgfqpoint{1.337535in}{2.095201in}}%
\pgfpathlineto{\pgfqpoint{1.338554in}{2.152798in}}%
\pgfpathlineto{\pgfqpoint{1.338757in}{2.136342in}}%
\pgfpathlineto{\pgfqpoint{1.340795in}{2.267993in}}%
\pgfpathlineto{\pgfqpoint{1.341814in}{2.152798in}}%
\pgfpathlineto{\pgfqpoint{1.342018in}{2.226852in}}%
\pgfpathlineto{\pgfqpoint{1.342630in}{2.333819in}}%
\pgfpathlineto{\pgfqpoint{1.342834in}{2.284450in}}%
\pgfpathlineto{\pgfqpoint{1.344260in}{2.095201in}}%
\pgfpathlineto{\pgfqpoint{1.344464in}{2.086973in}}%
\pgfpathlineto{\pgfqpoint{1.344668in}{2.103429in}}%
\pgfpathlineto{\pgfqpoint{1.344872in}{2.095201in}}%
\pgfpathlineto{\pgfqpoint{1.346094in}{2.193939in}}%
\pgfpathlineto{\pgfqpoint{1.345687in}{2.070516in}}%
\pgfpathlineto{\pgfqpoint{1.346502in}{2.185711in}}%
\pgfpathlineto{\pgfqpoint{1.346910in}{2.128114in}}%
\pgfpathlineto{\pgfqpoint{1.347521in}{2.012919in}}%
\pgfpathlineto{\pgfqpoint{1.347725in}{2.078745in}}%
\pgfpathlineto{\pgfqpoint{1.348948in}{2.177483in}}%
\pgfpathlineto{\pgfqpoint{1.350170in}{2.111657in}}%
\pgfpathlineto{\pgfqpoint{1.350782in}{2.152798in}}%
\pgfpathlineto{\pgfqpoint{1.351189in}{2.144570in}}%
\pgfpathlineto{\pgfqpoint{1.351393in}{2.111657in}}%
\pgfpathlineto{\pgfqpoint{1.351597in}{2.193939in}}%
\pgfpathlineto{\pgfqpoint{1.351801in}{2.177483in}}%
\pgfpathlineto{\pgfqpoint{1.352005in}{2.193939in}}%
\pgfpathlineto{\pgfqpoint{1.352208in}{2.177483in}}%
\pgfpathlineto{\pgfqpoint{1.353024in}{2.095201in}}%
\pgfpathlineto{\pgfqpoint{1.353431in}{2.119886in}}%
\pgfpathlineto{\pgfqpoint{1.353635in}{2.128114in}}%
\pgfpathlineto{\pgfqpoint{1.353839in}{2.111657in}}%
\pgfpathlineto{\pgfqpoint{1.354043in}{2.086973in}}%
\pgfpathlineto{\pgfqpoint{1.354246in}{2.128114in}}%
\pgfpathlineto{\pgfqpoint{1.354858in}{2.103429in}}%
\pgfpathlineto{\pgfqpoint{1.355062in}{2.144570in}}%
\pgfpathlineto{\pgfqpoint{1.355673in}{2.086973in}}%
\pgfpathlineto{\pgfqpoint{1.355877in}{2.119886in}}%
\pgfpathlineto{\pgfqpoint{1.356285in}{2.037604in}}%
\pgfpathlineto{\pgfqpoint{1.356488in}{2.086973in}}%
\pgfpathlineto{\pgfqpoint{1.357507in}{2.169255in}}%
\pgfpathlineto{\pgfqpoint{1.358119in}{2.103429in}}%
\pgfpathlineto{\pgfqpoint{1.358526in}{2.144570in}}%
\pgfpathlineto{\pgfqpoint{1.359342in}{2.177483in}}%
\pgfpathlineto{\pgfqpoint{1.360564in}{2.078745in}}%
\pgfpathlineto{\pgfqpoint{1.361176in}{2.136342in}}%
\pgfpathlineto{\pgfqpoint{1.360972in}{2.062288in}}%
\pgfpathlineto{\pgfqpoint{1.361380in}{2.062288in}}%
\pgfpathlineto{\pgfqpoint{1.361583in}{1.955322in}}%
\pgfpathlineto{\pgfqpoint{1.361787in}{2.136342in}}%
\pgfpathlineto{\pgfqpoint{1.362195in}{2.095201in}}%
\pgfpathlineto{\pgfqpoint{1.363214in}{2.259765in}}%
\pgfpathlineto{\pgfqpoint{1.364029in}{2.185711in}}%
\pgfpathlineto{\pgfqpoint{1.364437in}{2.218624in}}%
\pgfpathlineto{\pgfqpoint{1.365456in}{1.947093in}}%
\pgfpathlineto{\pgfqpoint{1.366475in}{2.235080in}}%
\pgfpathlineto{\pgfqpoint{1.366678in}{2.185711in}}%
\pgfpathlineto{\pgfqpoint{1.366882in}{2.152798in}}%
\pgfpathlineto{\pgfqpoint{1.367290in}{2.202168in}}%
\pgfpathlineto{\pgfqpoint{1.367494in}{2.185711in}}%
\pgfpathlineto{\pgfqpoint{1.367697in}{2.243309in}}%
\pgfpathlineto{\pgfqpoint{1.368513in}{2.218624in}}%
\pgfpathlineto{\pgfqpoint{1.369124in}{2.136342in}}%
\pgfpathlineto{\pgfqpoint{1.368920in}{2.235080in}}%
\pgfpathlineto{\pgfqpoint{1.369532in}{2.161027in}}%
\pgfpathlineto{\pgfqpoint{1.370143in}{2.276221in}}%
\pgfpathlineto{\pgfqpoint{1.370551in}{2.152798in}}%
\pgfpathlineto{\pgfqpoint{1.371162in}{2.054060in}}%
\pgfpathlineto{\pgfqpoint{1.371977in}{2.062288in}}%
\pgfpathlineto{\pgfqpoint{1.372385in}{2.029375in}}%
\pgfpathlineto{\pgfqpoint{1.373404in}{2.128114in}}%
\pgfpathlineto{\pgfqpoint{1.374831in}{2.054060in}}%
\pgfpathlineto{\pgfqpoint{1.375034in}{1.864811in}}%
\pgfpathlineto{\pgfqpoint{1.375646in}{2.111657in}}%
\pgfpathlineto{\pgfqpoint{1.375850in}{2.062288in}}%
\pgfpathlineto{\pgfqpoint{1.376461in}{2.169255in}}%
\pgfpathlineto{\pgfqpoint{1.377072in}{2.103429in}}%
\pgfpathlineto{\pgfqpoint{1.378499in}{2.267993in}}%
\pgfpathlineto{\pgfqpoint{1.377480in}{2.086973in}}%
\pgfpathlineto{\pgfqpoint{1.378907in}{2.169255in}}%
\pgfpathlineto{\pgfqpoint{1.379722in}{2.086973in}}%
\pgfpathlineto{\pgfqpoint{1.379926in}{2.152798in}}%
\pgfpathlineto{\pgfqpoint{1.380129in}{2.177483in}}%
\pgfpathlineto{\pgfqpoint{1.380333in}{2.078745in}}%
\pgfpathlineto{\pgfqpoint{1.380537in}{2.070516in}}%
\pgfpathlineto{\pgfqpoint{1.381352in}{2.004691in}}%
\pgfpathlineto{\pgfqpoint{1.381556in}{2.070516in}}%
\pgfpathlineto{\pgfqpoint{1.381760in}{2.095201in}}%
\pgfpathlineto{\pgfqpoint{1.382167in}{2.086973in}}%
\pgfpathlineto{\pgfqpoint{1.382371in}{2.004691in}}%
\pgfpathlineto{\pgfqpoint{1.382575in}{2.095201in}}%
\pgfpathlineto{\pgfqpoint{1.383187in}{2.095201in}}%
\pgfpathlineto{\pgfqpoint{1.384206in}{2.021147in}}%
\pgfpathlineto{\pgfqpoint{1.384613in}{2.029375in}}%
\pgfpathlineto{\pgfqpoint{1.385428in}{1.963550in}}%
\pgfpathlineto{\pgfqpoint{1.385632in}{1.996463in}}%
\pgfpathlineto{\pgfqpoint{1.385836in}{2.045832in}}%
\pgfpathlineto{\pgfqpoint{1.386244in}{1.889496in}}%
\pgfpathlineto{\pgfqpoint{1.388689in}{2.136342in}}%
\pgfpathlineto{\pgfqpoint{1.389301in}{2.054060in}}%
\pgfpathlineto{\pgfqpoint{1.389708in}{2.062288in}}%
\pgfpathlineto{\pgfqpoint{1.390320in}{2.136342in}}%
\pgfpathlineto{\pgfqpoint{1.390931in}{2.103429in}}%
\pgfpathlineto{\pgfqpoint{1.391339in}{2.144570in}}%
\pgfpathlineto{\pgfqpoint{1.391542in}{2.111657in}}%
\pgfpathlineto{\pgfqpoint{1.392765in}{2.037604in}}%
\pgfpathlineto{\pgfqpoint{1.393784in}{2.235080in}}%
\pgfpathlineto{\pgfqpoint{1.394396in}{2.185711in}}%
\pgfpathlineto{\pgfqpoint{1.394599in}{2.144570in}}%
\pgfpathlineto{\pgfqpoint{1.395211in}{2.152798in}}%
\pgfpathlineto{\pgfqpoint{1.396026in}{2.259765in}}%
\pgfpathlineto{\pgfqpoint{1.396434in}{2.251537in}}%
\pgfpathlineto{\pgfqpoint{1.397657in}{2.070516in}}%
\pgfpathlineto{\pgfqpoint{1.396841in}{2.259765in}}%
\pgfpathlineto{\pgfqpoint{1.398064in}{2.119886in}}%
\pgfpathlineto{\pgfqpoint{1.399083in}{2.251537in}}%
\pgfpathlineto{\pgfqpoint{1.399287in}{2.226852in}}%
\pgfpathlineto{\pgfqpoint{1.399491in}{2.243309in}}%
\pgfpathlineto{\pgfqpoint{1.399695in}{2.185711in}}%
\pgfpathlineto{\pgfqpoint{1.399898in}{2.185711in}}%
\pgfpathlineto{\pgfqpoint{1.401121in}{2.119886in}}%
\pgfpathlineto{\pgfqpoint{1.401936in}{2.235080in}}%
\pgfpathlineto{\pgfqpoint{1.402548in}{2.210396in}}%
\pgfpathlineto{\pgfqpoint{1.402955in}{2.152798in}}%
\pgfpathlineto{\pgfqpoint{1.403567in}{2.169255in}}%
\pgfpathlineto{\pgfqpoint{1.404586in}{2.235080in}}%
\pgfpathlineto{\pgfqpoint{1.404993in}{2.226852in}}%
\pgfpathlineto{\pgfqpoint{1.405809in}{2.152798in}}%
\pgfpathlineto{\pgfqpoint{1.406012in}{2.169255in}}%
\pgfpathlineto{\pgfqpoint{1.406216in}{2.235080in}}%
\pgfpathlineto{\pgfqpoint{1.406624in}{2.161027in}}%
\pgfpathlineto{\pgfqpoint{1.406828in}{2.161027in}}%
\pgfpathlineto{\pgfqpoint{1.407031in}{2.136342in}}%
\pgfpathlineto{\pgfqpoint{1.407235in}{2.177483in}}%
\pgfpathlineto{\pgfqpoint{1.407439in}{2.161027in}}%
\pgfpathlineto{\pgfqpoint{1.408254in}{2.218624in}}%
\pgfpathlineto{\pgfqpoint{1.408662in}{2.128114in}}%
\pgfpathlineto{\pgfqpoint{1.409069in}{2.152798in}}%
\pgfpathlineto{\pgfqpoint{1.409477in}{2.267993in}}%
\pgfpathlineto{\pgfqpoint{1.410089in}{2.251537in}}%
\pgfpathlineto{\pgfqpoint{1.410292in}{2.202168in}}%
\pgfpathlineto{\pgfqpoint{1.410700in}{2.267993in}}%
\pgfpathlineto{\pgfqpoint{1.411108in}{2.243309in}}%
\pgfpathlineto{\pgfqpoint{1.412534in}{2.144570in}}%
\pgfpathlineto{\pgfqpoint{1.412738in}{2.218624in}}%
\pgfpathlineto{\pgfqpoint{1.413553in}{2.193939in}}%
\pgfpathlineto{\pgfqpoint{1.414368in}{2.144570in}}%
\pgfpathlineto{\pgfqpoint{1.414980in}{2.218624in}}%
\pgfpathlineto{\pgfqpoint{1.415184in}{2.136342in}}%
\pgfpathlineto{\pgfqpoint{1.415387in}{2.177483in}}%
\pgfpathlineto{\pgfqpoint{1.415591in}{2.169255in}}%
\pgfpathlineto{\pgfqpoint{1.417222in}{2.292678in}}%
\pgfpathlineto{\pgfqpoint{1.417629in}{2.185711in}}%
\pgfpathlineto{\pgfqpoint{1.418241in}{2.218624in}}%
\pgfpathlineto{\pgfqpoint{1.419056in}{2.243309in}}%
\pgfpathlineto{\pgfqpoint{1.420075in}{2.169255in}}%
\pgfpathlineto{\pgfqpoint{1.420279in}{2.226852in}}%
\pgfpathlineto{\pgfqpoint{1.420890in}{2.152798in}}%
\pgfpathlineto{\pgfqpoint{1.421094in}{2.161027in}}%
\pgfpathlineto{\pgfqpoint{1.421501in}{2.193939in}}%
\pgfpathlineto{\pgfqpoint{1.422317in}{2.177483in}}%
\pgfpathlineto{\pgfqpoint{1.422724in}{2.119886in}}%
\pgfpathlineto{\pgfqpoint{1.422928in}{2.202168in}}%
\pgfpathlineto{\pgfqpoint{1.423336in}{2.193939in}}%
\pgfpathlineto{\pgfqpoint{1.424559in}{2.251537in}}%
\pgfpathlineto{\pgfqpoint{1.424151in}{2.177483in}}%
\pgfpathlineto{\pgfqpoint{1.424762in}{2.226852in}}%
\pgfpathlineto{\pgfqpoint{1.426393in}{2.136342in}}%
\pgfpathlineto{\pgfqpoint{1.427004in}{2.095201in}}%
\pgfpathlineto{\pgfqpoint{1.427616in}{2.210396in}}%
\pgfpathlineto{\pgfqpoint{1.428838in}{2.128114in}}%
\pgfpathlineto{\pgfqpoint{1.429654in}{2.226852in}}%
\pgfpathlineto{\pgfqpoint{1.430265in}{2.218624in}}%
\pgfpathlineto{\pgfqpoint{1.430469in}{2.185711in}}%
\pgfpathlineto{\pgfqpoint{1.430876in}{2.243309in}}%
\pgfpathlineto{\pgfqpoint{1.431284in}{2.276221in}}%
\pgfpathlineto{\pgfqpoint{1.431692in}{2.218624in}}%
\pgfpathlineto{\pgfqpoint{1.431895in}{2.235080in}}%
\pgfpathlineto{\pgfqpoint{1.432099in}{2.243309in}}%
\pgfpathlineto{\pgfqpoint{1.432303in}{2.226852in}}%
\pgfpathlineto{\pgfqpoint{1.432507in}{2.193939in}}%
\pgfpathlineto{\pgfqpoint{1.433118in}{2.251537in}}%
\pgfpathlineto{\pgfqpoint{1.433322in}{2.235080in}}%
\pgfpathlineto{\pgfqpoint{1.433730in}{2.284450in}}%
\pgfpathlineto{\pgfqpoint{1.433933in}{2.267993in}}%
\pgfpathlineto{\pgfqpoint{1.434137in}{2.210396in}}%
\pgfpathlineto{\pgfqpoint{1.434952in}{2.292678in}}%
\pgfpathlineto{\pgfqpoint{1.435156in}{2.300906in}}%
\pgfpathlineto{\pgfqpoint{1.435360in}{2.358504in}}%
\pgfpathlineto{\pgfqpoint{1.435971in}{2.251537in}}%
\pgfpathlineto{\pgfqpoint{1.436175in}{2.309134in}}%
\pgfpathlineto{\pgfqpoint{1.436379in}{2.300906in}}%
\pgfpathlineto{\pgfqpoint{1.436583in}{2.309134in}}%
\pgfpathlineto{\pgfqpoint{1.436991in}{2.342047in}}%
\pgfpathlineto{\pgfqpoint{1.437602in}{2.309134in}}%
\pgfpathlineto{\pgfqpoint{1.437806in}{2.259765in}}%
\pgfpathlineto{\pgfqpoint{1.438213in}{2.374960in}}%
\pgfpathlineto{\pgfqpoint{1.438417in}{2.383188in}}%
\pgfpathlineto{\pgfqpoint{1.438621in}{2.276221in}}%
\pgfpathlineto{\pgfqpoint{1.439640in}{2.300906in}}%
\pgfpathlineto{\pgfqpoint{1.440863in}{2.374960in}}%
\pgfpathlineto{\pgfqpoint{1.442289in}{2.276221in}}%
\pgfpathlineto{\pgfqpoint{1.442901in}{2.243309in}}%
\pgfpathlineto{\pgfqpoint{1.443308in}{2.292678in}}%
\pgfpathlineto{\pgfqpoint{1.443716in}{2.235080in}}%
\pgfpathlineto{\pgfqpoint{1.444531in}{2.251537in}}%
\pgfpathlineto{\pgfqpoint{1.444735in}{2.243309in}}%
\pgfpathlineto{\pgfqpoint{1.445550in}{2.292678in}}%
\pgfpathlineto{\pgfqpoint{1.445754in}{2.259765in}}%
\pgfpathlineto{\pgfqpoint{1.445958in}{2.243309in}}%
\pgfpathlineto{\pgfqpoint{1.446162in}{2.276221in}}%
\pgfpathlineto{\pgfqpoint{1.446365in}{2.267993in}}%
\pgfpathlineto{\pgfqpoint{1.446569in}{2.300906in}}%
\pgfpathlineto{\pgfqpoint{1.446977in}{2.243309in}}%
\pgfpathlineto{\pgfqpoint{1.447384in}{2.267993in}}%
\pgfpathlineto{\pgfqpoint{1.447996in}{2.210396in}}%
\pgfpathlineto{\pgfqpoint{1.448607in}{2.243309in}}%
\pgfpathlineto{\pgfqpoint{1.449219in}{2.095201in}}%
\pgfpathlineto{\pgfqpoint{1.450034in}{2.185711in}}%
\pgfpathlineto{\pgfqpoint{1.450238in}{2.235080in}}%
\pgfpathlineto{\pgfqpoint{1.450442in}{2.177483in}}%
\pgfpathlineto{\pgfqpoint{1.451053in}{2.185711in}}%
\pgfpathlineto{\pgfqpoint{1.452276in}{2.292678in}}%
\pgfpathlineto{\pgfqpoint{1.452683in}{2.259765in}}%
\pgfpathlineto{\pgfqpoint{1.453091in}{2.267993in}}%
\pgfpathlineto{\pgfqpoint{1.453702in}{2.136342in}}%
\pgfpathlineto{\pgfqpoint{1.454314in}{2.333819in}}%
\pgfpathlineto{\pgfqpoint{1.454925in}{2.259765in}}%
\pgfpathlineto{\pgfqpoint{1.455129in}{2.259765in}}%
\pgfpathlineto{\pgfqpoint{1.455333in}{2.177483in}}%
\pgfpathlineto{\pgfqpoint{1.455537in}{2.300906in}}%
\pgfpathlineto{\pgfqpoint{1.456148in}{2.259765in}}%
\pgfpathlineto{\pgfqpoint{1.457167in}{2.350275in}}%
\pgfpathlineto{\pgfqpoint{1.457778in}{2.226852in}}%
\pgfpathlineto{\pgfqpoint{1.458390in}{2.284450in}}%
\pgfpathlineto{\pgfqpoint{1.458797in}{2.218624in}}%
\pgfpathlineto{\pgfqpoint{1.459205in}{2.292678in}}%
\pgfpathlineto{\pgfqpoint{1.459613in}{2.235080in}}%
\pgfpathlineto{\pgfqpoint{1.460224in}{2.177483in}}%
\pgfpathlineto{\pgfqpoint{1.460020in}{2.267993in}}%
\pgfpathlineto{\pgfqpoint{1.460632in}{2.185711in}}%
\pgfpathlineto{\pgfqpoint{1.462058in}{2.342047in}}%
\pgfpathlineto{\pgfqpoint{1.462262in}{2.342047in}}%
\pgfpathlineto{\pgfqpoint{1.463485in}{2.218624in}}%
\pgfpathlineto{\pgfqpoint{1.463689in}{2.235080in}}%
\pgfpathlineto{\pgfqpoint{1.463893in}{2.259765in}}%
\pgfpathlineto{\pgfqpoint{1.464504in}{2.202168in}}%
\pgfpathlineto{\pgfqpoint{1.464708in}{2.243309in}}%
\pgfpathlineto{\pgfqpoint{1.464912in}{1.980006in}}%
\pgfpathlineto{\pgfqpoint{1.465523in}{2.292678in}}%
\pgfpathlineto{\pgfqpoint{1.465727in}{2.202168in}}%
\pgfpathlineto{\pgfqpoint{1.465931in}{2.259765in}}%
\pgfpathlineto{\pgfqpoint{1.466746in}{2.185711in}}%
\pgfpathlineto{\pgfqpoint{1.467357in}{2.284450in}}%
\pgfpathlineto{\pgfqpoint{1.467969in}{2.235080in}}%
\pgfpathlineto{\pgfqpoint{1.468172in}{2.202168in}}%
\pgfpathlineto{\pgfqpoint{1.468376in}{2.243309in}}%
\pgfpathlineto{\pgfqpoint{1.468784in}{2.243309in}}%
\pgfpathlineto{\pgfqpoint{1.468988in}{2.317363in}}%
\pgfpathlineto{\pgfqpoint{1.469803in}{2.243309in}}%
\pgfpathlineto{\pgfqpoint{1.470414in}{2.193939in}}%
\pgfpathlineto{\pgfqpoint{1.470618in}{2.251537in}}%
\pgfpathlineto{\pgfqpoint{1.471026in}{2.218624in}}%
\pgfpathlineto{\pgfqpoint{1.471229in}{2.210396in}}%
\pgfpathlineto{\pgfqpoint{1.471433in}{2.235080in}}%
\pgfpathlineto{\pgfqpoint{1.471637in}{2.226852in}}%
\pgfpathlineto{\pgfqpoint{1.472045in}{2.309134in}}%
\pgfpathlineto{\pgfqpoint{1.472452in}{2.292678in}}%
\pgfpathlineto{\pgfqpoint{1.473675in}{2.136342in}}%
\pgfpathlineto{\pgfqpoint{1.474898in}{2.350275in}}%
\pgfpathlineto{\pgfqpoint{1.475102in}{2.309134in}}%
\pgfpathlineto{\pgfqpoint{1.475305in}{2.325591in}}%
\pgfpathlineto{\pgfqpoint{1.475713in}{2.111657in}}%
\pgfpathlineto{\pgfqpoint{1.476324in}{2.276221in}}%
\pgfpathlineto{\pgfqpoint{1.477547in}{2.218624in}}%
\pgfpathlineto{\pgfqpoint{1.477955in}{2.259765in}}%
\pgfpathlineto{\pgfqpoint{1.478566in}{2.210396in}}%
\pgfpathlineto{\pgfqpoint{1.478770in}{2.210396in}}%
\pgfpathlineto{\pgfqpoint{1.479178in}{2.317363in}}%
\pgfpathlineto{\pgfqpoint{1.479993in}{2.251537in}}%
\pgfpathlineto{\pgfqpoint{1.480808in}{2.292678in}}%
\pgfpathlineto{\pgfqpoint{1.481012in}{2.243309in}}%
\pgfpathlineto{\pgfqpoint{1.481420in}{2.300906in}}%
\pgfpathlineto{\pgfqpoint{1.481827in}{2.251537in}}%
\pgfpathlineto{\pgfqpoint{1.482031in}{2.267993in}}%
\pgfpathlineto{\pgfqpoint{1.482439in}{2.235080in}}%
\pgfpathlineto{\pgfqpoint{1.482642in}{2.226852in}}%
\pgfpathlineto{\pgfqpoint{1.483458in}{2.292678in}}%
\pgfpathlineto{\pgfqpoint{1.483661in}{2.103429in}}%
\pgfpathlineto{\pgfqpoint{1.484069in}{2.350275in}}%
\pgfpathlineto{\pgfqpoint{1.484477in}{2.333819in}}%
\pgfpathlineto{\pgfqpoint{1.484680in}{2.391416in}}%
\pgfpathlineto{\pgfqpoint{1.485699in}{2.374960in}}%
\pgfpathlineto{\pgfqpoint{1.487737in}{2.226852in}}%
\pgfpathlineto{\pgfqpoint{1.488349in}{2.235080in}}%
\pgfpathlineto{\pgfqpoint{1.488960in}{2.267993in}}%
\pgfpathlineto{\pgfqpoint{1.489368in}{1.996463in}}%
\pgfpathlineto{\pgfqpoint{1.489979in}{2.243309in}}%
\pgfpathlineto{\pgfqpoint{1.491406in}{2.333819in}}%
\pgfpathlineto{\pgfqpoint{1.491610in}{2.350275in}}%
\pgfpathlineto{\pgfqpoint{1.491814in}{2.309134in}}%
\pgfpathlineto{\pgfqpoint{1.492221in}{2.333819in}}%
\pgfpathlineto{\pgfqpoint{1.494667in}{1.807214in}}%
\pgfpathlineto{\pgfqpoint{1.495074in}{1.873040in}}%
\pgfpathlineto{\pgfqpoint{1.495482in}{1.914181in}}%
\pgfpathlineto{\pgfqpoint{1.495686in}{1.823670in}}%
\pgfpathlineto{\pgfqpoint{1.496093in}{1.683791in}}%
\pgfpathlineto{\pgfqpoint{1.496909in}{1.988234in}}%
\pgfpathlineto{\pgfqpoint{1.497316in}{1.823670in}}%
\pgfpathlineto{\pgfqpoint{1.497520in}{1.897724in}}%
\pgfpathlineto{\pgfqpoint{1.498131in}{1.807214in}}%
\pgfpathlineto{\pgfqpoint{1.498335in}{1.815442in}}%
\pgfpathlineto{\pgfqpoint{1.498539in}{1.831899in}}%
\pgfpathlineto{\pgfqpoint{1.498947in}{1.798986in}}%
\pgfpathlineto{\pgfqpoint{1.499150in}{1.798986in}}%
\pgfpathlineto{\pgfqpoint{1.499762in}{1.675563in}}%
\pgfpathlineto{\pgfqpoint{1.500169in}{1.782529in}}%
\pgfpathlineto{\pgfqpoint{1.500373in}{1.897724in}}%
\pgfpathlineto{\pgfqpoint{1.501392in}{1.864811in}}%
\pgfpathlineto{\pgfqpoint{1.502411in}{1.873040in}}%
\pgfpathlineto{\pgfqpoint{1.502819in}{1.749617in}}%
\pgfpathlineto{\pgfqpoint{1.504449in}{1.914181in}}%
\pgfpathlineto{\pgfqpoint{1.505061in}{1.864811in}}%
\pgfpathlineto{\pgfqpoint{1.504857in}{1.930637in}}%
\pgfpathlineto{\pgfqpoint{1.505265in}{1.922409in}}%
\pgfpathlineto{\pgfqpoint{1.505876in}{1.905952in}}%
\pgfpathlineto{\pgfqpoint{1.506487in}{1.971778in}}%
\pgfpathlineto{\pgfqpoint{1.507914in}{1.856583in}}%
\pgfpathlineto{\pgfqpoint{1.508118in}{1.922409in}}%
\pgfpathlineto{\pgfqpoint{1.508729in}{1.831899in}}%
\pgfpathlineto{\pgfqpoint{1.508933in}{1.840127in}}%
\pgfpathlineto{\pgfqpoint{1.509544in}{1.947093in}}%
\pgfpathlineto{\pgfqpoint{1.509952in}{1.930637in}}%
\pgfpathlineto{\pgfqpoint{1.511175in}{1.733160in}}%
\pgfpathlineto{\pgfqpoint{1.511990in}{1.864811in}}%
\pgfpathlineto{\pgfqpoint{1.512398in}{1.823670in}}%
\pgfpathlineto{\pgfqpoint{1.513213in}{1.938865in}}%
\pgfpathlineto{\pgfqpoint{1.513620in}{1.897724in}}%
\pgfpathlineto{\pgfqpoint{1.514232in}{1.749617in}}%
\pgfpathlineto{\pgfqpoint{1.514436in}{1.881268in}}%
\pgfpathlineto{\pgfqpoint{1.514639in}{1.922409in}}%
\pgfpathlineto{\pgfqpoint{1.514843in}{1.675563in}}%
\pgfpathlineto{\pgfqpoint{1.515658in}{1.938865in}}%
\pgfpathlineto{\pgfqpoint{1.516270in}{1.823670in}}%
\pgfpathlineto{\pgfqpoint{1.517289in}{1.856583in}}%
\pgfpathlineto{\pgfqpoint{1.518104in}{1.955322in}}%
\pgfpathlineto{\pgfqpoint{1.518512in}{1.889496in}}%
\pgfpathlineto{\pgfqpoint{1.519327in}{1.790758in}}%
\pgfpathlineto{\pgfqpoint{1.520142in}{1.840127in}}%
\pgfpathlineto{\pgfqpoint{1.520550in}{1.963550in}}%
\pgfpathlineto{\pgfqpoint{1.521569in}{1.930637in}}%
\pgfpathlineto{\pgfqpoint{1.521976in}{1.848355in}}%
\pgfpathlineto{\pgfqpoint{1.522384in}{1.914181in}}%
\pgfpathlineto{\pgfqpoint{1.523403in}{2.045832in}}%
\pgfpathlineto{\pgfqpoint{1.523811in}{1.963550in}}%
\pgfpathlineto{\pgfqpoint{1.524218in}{1.980006in}}%
\pgfpathlineto{\pgfqpoint{1.525033in}{1.889496in}}%
\pgfpathlineto{\pgfqpoint{1.525645in}{1.881268in}}%
\pgfpathlineto{\pgfqpoint{1.526256in}{1.980006in}}%
\pgfpathlineto{\pgfqpoint{1.527479in}{1.782529in}}%
\pgfpathlineto{\pgfqpoint{1.528090in}{2.070516in}}%
\pgfpathlineto{\pgfqpoint{1.528906in}{2.029375in}}%
\pgfpathlineto{\pgfqpoint{1.529721in}{2.037604in}}%
\pgfpathlineto{\pgfqpoint{1.530128in}{1.947093in}}%
\pgfpathlineto{\pgfqpoint{1.530944in}{2.062288in}}%
\pgfpathlineto{\pgfqpoint{1.531351in}{2.045832in}}%
\pgfpathlineto{\pgfqpoint{1.531555in}{2.045832in}}%
\pgfpathlineto{\pgfqpoint{1.531963in}{2.029375in}}%
\pgfpathlineto{\pgfqpoint{1.532370in}{1.947093in}}%
\pgfpathlineto{\pgfqpoint{1.532982in}{2.012919in}}%
\pgfpathlineto{\pgfqpoint{1.533186in}{2.021147in}}%
\pgfpathlineto{\pgfqpoint{1.533797in}{1.930637in}}%
\pgfpathlineto{\pgfqpoint{1.534001in}{2.029375in}}%
\pgfpathlineto{\pgfqpoint{1.534205in}{2.012919in}}%
\pgfpathlineto{\pgfqpoint{1.536243in}{1.831899in}}%
\pgfpathlineto{\pgfqpoint{1.536446in}{1.897724in}}%
\pgfpathlineto{\pgfqpoint{1.537058in}{1.790758in}}%
\pgfpathlineto{\pgfqpoint{1.537262in}{1.815442in}}%
\pgfpathlineto{\pgfqpoint{1.538688in}{1.667335in}}%
\pgfpathlineto{\pgfqpoint{1.538892in}{1.667335in}}%
\pgfpathlineto{\pgfqpoint{1.539911in}{1.733160in}}%
\pgfpathlineto{\pgfqpoint{1.540115in}{1.700247in}}%
\pgfpathlineto{\pgfqpoint{1.540930in}{1.650878in}}%
\pgfpathlineto{\pgfqpoint{1.541134in}{1.700247in}}%
\pgfpathlineto{\pgfqpoint{1.541338in}{1.733160in}}%
\pgfpathlineto{\pgfqpoint{1.541745in}{1.634422in}}%
\pgfpathlineto{\pgfqpoint{1.542357in}{1.593281in}}%
\pgfpathlineto{\pgfqpoint{1.542560in}{1.733160in}}%
\pgfpathlineto{\pgfqpoint{1.543579in}{1.700247in}}%
\pgfpathlineto{\pgfqpoint{1.543783in}{1.700247in}}%
\pgfpathlineto{\pgfqpoint{1.543987in}{1.683791in}}%
\pgfpathlineto{\pgfqpoint{1.544395in}{1.733160in}}%
\pgfpathlineto{\pgfqpoint{1.544599in}{1.840127in}}%
\pgfpathlineto{\pgfqpoint{1.545414in}{1.716704in}}%
\pgfpathlineto{\pgfqpoint{1.545821in}{1.790758in}}%
\pgfpathlineto{\pgfqpoint{1.546637in}{1.749617in}}%
\pgfpathlineto{\pgfqpoint{1.547248in}{1.700247in}}%
\pgfpathlineto{\pgfqpoint{1.548063in}{1.683791in}}%
\pgfpathlineto{\pgfqpoint{1.548267in}{1.807214in}}%
\pgfpathlineto{\pgfqpoint{1.548878in}{1.667335in}}%
\pgfpathlineto{\pgfqpoint{1.549490in}{1.724932in}}%
\pgfpathlineto{\pgfqpoint{1.550509in}{1.774301in}}%
\pgfpathlineto{\pgfqpoint{1.550916in}{1.766073in}}%
\pgfpathlineto{\pgfqpoint{1.552139in}{1.724932in}}%
\pgfpathlineto{\pgfqpoint{1.552343in}{1.782529in}}%
\pgfpathlineto{\pgfqpoint{1.553362in}{1.774301in}}%
\pgfpathlineto{\pgfqpoint{1.553770in}{1.774301in}}%
\pgfpathlineto{\pgfqpoint{1.554177in}{1.700247in}}%
\pgfpathlineto{\pgfqpoint{1.554381in}{1.774301in}}%
\pgfpathlineto{\pgfqpoint{1.554585in}{1.831899in}}%
\pgfpathlineto{\pgfqpoint{1.554992in}{1.708476in}}%
\pgfpathlineto{\pgfqpoint{1.555400in}{1.774301in}}%
\pgfpathlineto{\pgfqpoint{1.556011in}{1.708476in}}%
\pgfpathlineto{\pgfqpoint{1.556419in}{1.716704in}}%
\pgfpathlineto{\pgfqpoint{1.556623in}{1.749617in}}%
\pgfpathlineto{\pgfqpoint{1.556827in}{1.576824in}}%
\pgfpathlineto{\pgfqpoint{1.557438in}{1.807214in}}%
\pgfpathlineto{\pgfqpoint{1.557642in}{1.798986in}}%
\pgfpathlineto{\pgfqpoint{1.557846in}{1.823670in}}%
\pgfpathlineto{\pgfqpoint{1.558253in}{1.807214in}}%
\pgfpathlineto{\pgfqpoint{1.558457in}{1.716704in}}%
\pgfpathlineto{\pgfqpoint{1.559069in}{1.856583in}}%
\pgfpathlineto{\pgfqpoint{1.559272in}{1.823670in}}%
\pgfpathlineto{\pgfqpoint{1.559476in}{1.848355in}}%
\pgfpathlineto{\pgfqpoint{1.559680in}{1.774301in}}%
\pgfpathlineto{\pgfqpoint{1.560291in}{1.823670in}}%
\pgfpathlineto{\pgfqpoint{1.560495in}{1.659106in}}%
\pgfpathlineto{\pgfqpoint{1.561310in}{1.716704in}}%
\pgfpathlineto{\pgfqpoint{1.561718in}{1.749617in}}%
\pgfpathlineto{\pgfqpoint{1.562126in}{1.692019in}}%
\pgfpathlineto{\pgfqpoint{1.562329in}{1.683791in}}%
\pgfpathlineto{\pgfqpoint{1.562533in}{1.708476in}}%
\pgfpathlineto{\pgfqpoint{1.562737in}{1.766073in}}%
\pgfpathlineto{\pgfqpoint{1.563552in}{1.692019in}}%
\pgfpathlineto{\pgfqpoint{1.563960in}{1.700247in}}%
\pgfpathlineto{\pgfqpoint{1.564775in}{1.634422in}}%
\pgfpathlineto{\pgfqpoint{1.565183in}{1.659106in}}%
\pgfpathlineto{\pgfqpoint{1.565386in}{1.733160in}}%
\pgfpathlineto{\pgfqpoint{1.565998in}{1.650878in}}%
\pgfpathlineto{\pgfqpoint{1.566202in}{1.692019in}}%
\pgfpathlineto{\pgfqpoint{1.567424in}{1.617965in}}%
\pgfpathlineto{\pgfqpoint{1.567628in}{1.617965in}}%
\pgfpathlineto{\pgfqpoint{1.567832in}{1.626194in}}%
\pgfpathlineto{\pgfqpoint{1.568443in}{1.535683in}}%
\pgfpathlineto{\pgfqpoint{1.569259in}{1.543912in}}%
\pgfpathlineto{\pgfqpoint{1.570889in}{1.650878in}}%
\pgfpathlineto{\pgfqpoint{1.571297in}{1.642650in}}%
\pgfpathlineto{\pgfqpoint{1.571501in}{1.617965in}}%
\pgfpathlineto{\pgfqpoint{1.571704in}{1.724932in}}%
\pgfpathlineto{\pgfqpoint{1.571908in}{1.675563in}}%
\pgfpathlineto{\pgfqpoint{1.572723in}{1.634422in}}%
\pgfpathlineto{\pgfqpoint{1.573335in}{1.782529in}}%
\pgfpathlineto{\pgfqpoint{1.574965in}{1.576824in}}%
\pgfpathlineto{\pgfqpoint{1.575577in}{1.650878in}}%
\pgfpathlineto{\pgfqpoint{1.575984in}{1.609737in}}%
\pgfpathlineto{\pgfqpoint{1.576188in}{2.136342in}}%
\pgfpathlineto{\pgfqpoint{1.577003in}{1.609737in}}%
\pgfpathlineto{\pgfqpoint{1.577207in}{1.527455in}}%
\pgfpathlineto{\pgfqpoint{1.577615in}{1.675563in}}%
\pgfpathlineto{\pgfqpoint{1.577818in}{1.650878in}}%
\pgfpathlineto{\pgfqpoint{1.578022in}{1.659106in}}%
\pgfpathlineto{\pgfqpoint{1.578226in}{1.626194in}}%
\pgfpathlineto{\pgfqpoint{1.578430in}{1.601509in}}%
\pgfpathlineto{\pgfqpoint{1.578634in}{1.692019in}}%
\pgfpathlineto{\pgfqpoint{1.578837in}{1.667335in}}%
\pgfpathlineto{\pgfqpoint{1.579041in}{1.683791in}}%
\pgfpathlineto{\pgfqpoint{1.579449in}{1.445173in}}%
\pgfpathlineto{\pgfqpoint{1.580060in}{1.650878in}}%
\pgfpathlineto{\pgfqpoint{1.580264in}{1.642650in}}%
\pgfpathlineto{\pgfqpoint{1.580468in}{1.659106in}}%
\pgfpathlineto{\pgfqpoint{1.580875in}{1.749617in}}%
\pgfpathlineto{\pgfqpoint{1.581283in}{1.552140in}}%
\pgfpathlineto{\pgfqpoint{1.581487in}{1.667335in}}%
\pgfpathlineto{\pgfqpoint{1.581691in}{1.675563in}}%
\pgfpathlineto{\pgfqpoint{1.581894in}{1.667335in}}%
\pgfpathlineto{\pgfqpoint{1.582098in}{1.766073in}}%
\pgfpathlineto{\pgfqpoint{1.582913in}{1.667335in}}%
\pgfpathlineto{\pgfqpoint{1.583321in}{1.617965in}}%
\pgfpathlineto{\pgfqpoint{1.583932in}{1.667335in}}%
\pgfpathlineto{\pgfqpoint{1.584544in}{1.774301in}}%
\pgfpathlineto{\pgfqpoint{1.584748in}{1.741388in}}%
\pgfpathlineto{\pgfqpoint{1.585359in}{1.428717in}}%
\pgfpathlineto{\pgfqpoint{1.585971in}{1.659106in}}%
\pgfpathlineto{\pgfqpoint{1.586174in}{1.642650in}}%
\pgfpathlineto{\pgfqpoint{1.586378in}{1.708476in}}%
\pgfpathlineto{\pgfqpoint{1.586990in}{1.659106in}}%
\pgfpathlineto{\pgfqpoint{1.587193in}{1.683791in}}%
\pgfpathlineto{\pgfqpoint{1.588009in}{1.749617in}}%
\pgfpathlineto{\pgfqpoint{1.588212in}{1.683791in}}%
\pgfpathlineto{\pgfqpoint{1.588416in}{1.626194in}}%
\pgfpathlineto{\pgfqpoint{1.589028in}{1.708476in}}%
\pgfpathlineto{\pgfqpoint{1.589231in}{1.733160in}}%
\pgfpathlineto{\pgfqpoint{1.589639in}{1.535683in}}%
\pgfpathlineto{\pgfqpoint{1.590047in}{1.774301in}}%
\pgfpathlineto{\pgfqpoint{1.590250in}{1.790758in}}%
\pgfpathlineto{\pgfqpoint{1.590862in}{1.757845in}}%
\pgfpathlineto{\pgfqpoint{1.591881in}{1.642650in}}%
\pgfpathlineto{\pgfqpoint{1.592288in}{1.650878in}}%
\pgfpathlineto{\pgfqpoint{1.592492in}{1.683791in}}%
\pgfpathlineto{\pgfqpoint{1.592900in}{1.724932in}}%
\pgfpathlineto{\pgfqpoint{1.593511in}{1.445173in}}%
\pgfpathlineto{\pgfqpoint{1.594530in}{1.708476in}}%
\pgfpathlineto{\pgfqpoint{1.594734in}{1.692019in}}%
\pgfpathlineto{\pgfqpoint{1.594938in}{1.634422in}}%
\pgfpathlineto{\pgfqpoint{1.595957in}{1.642650in}}%
\pgfpathlineto{\pgfqpoint{1.596161in}{1.642650in}}%
\pgfpathlineto{\pgfqpoint{1.596568in}{1.617965in}}%
\pgfpathlineto{\pgfqpoint{1.596772in}{1.659106in}}%
\pgfpathlineto{\pgfqpoint{1.596976in}{1.634422in}}%
\pgfpathlineto{\pgfqpoint{1.597995in}{1.675563in}}%
\pgfpathlineto{\pgfqpoint{1.598403in}{1.708476in}}%
\pgfpathlineto{\pgfqpoint{1.599218in}{1.560368in}}%
\pgfpathlineto{\pgfqpoint{1.599625in}{1.626194in}}%
\pgfpathlineto{\pgfqpoint{1.599829in}{1.362891in}}%
\pgfpathlineto{\pgfqpoint{1.600441in}{1.659106in}}%
\pgfpathlineto{\pgfqpoint{1.600644in}{1.634422in}}%
\pgfpathlineto{\pgfqpoint{1.601052in}{1.609737in}}%
\pgfpathlineto{\pgfqpoint{1.601867in}{1.716704in}}%
\pgfpathlineto{\pgfqpoint{1.603090in}{1.338207in}}%
\pgfpathlineto{\pgfqpoint{1.603294in}{1.346435in}}%
\pgfpathlineto{\pgfqpoint{1.603498in}{1.642650in}}%
\pgfpathlineto{\pgfqpoint{1.604517in}{1.601509in}}%
\pgfpathlineto{\pgfqpoint{1.606147in}{1.790758in}}%
\pgfpathlineto{\pgfqpoint{1.606351in}{1.782529in}}%
\pgfpathlineto{\pgfqpoint{1.606758in}{1.782529in}}%
\pgfpathlineto{\pgfqpoint{1.607166in}{1.881268in}}%
\pgfpathlineto{\pgfqpoint{1.607574in}{1.807214in}}%
\pgfpathlineto{\pgfqpoint{1.607777in}{1.708476in}}%
\pgfpathlineto{\pgfqpoint{1.608796in}{1.716704in}}%
\pgfpathlineto{\pgfqpoint{1.609612in}{1.790758in}}%
\pgfpathlineto{\pgfqpoint{1.610019in}{1.766073in}}%
\pgfpathlineto{\pgfqpoint{1.610427in}{1.774301in}}%
\pgfpathlineto{\pgfqpoint{1.611038in}{1.593281in}}%
\pgfpathlineto{\pgfqpoint{1.611242in}{1.823670in}}%
\pgfpathlineto{\pgfqpoint{1.612057in}{1.815442in}}%
\pgfpathlineto{\pgfqpoint{1.613280in}{1.395804in}}%
\pgfpathlineto{\pgfqpoint{1.613484in}{1.428717in}}%
\pgfpathlineto{\pgfqpoint{1.614095in}{1.634422in}}%
\pgfpathlineto{\pgfqpoint{1.614707in}{1.576824in}}%
\pgfpathlineto{\pgfqpoint{1.614911in}{1.585053in}}%
\pgfpathlineto{\pgfqpoint{1.615726in}{1.593281in}}%
\pgfpathlineto{\pgfqpoint{1.615930in}{1.519227in}}%
\pgfpathlineto{\pgfqpoint{1.616541in}{1.609737in}}%
\pgfpathlineto{\pgfqpoint{1.616949in}{1.519227in}}%
\pgfpathlineto{\pgfqpoint{1.618783in}{1.626194in}}%
\pgfpathlineto{\pgfqpoint{1.618987in}{1.626194in}}%
\pgfpathlineto{\pgfqpoint{1.620006in}{1.543912in}}%
\pgfpathlineto{\pgfqpoint{1.620413in}{1.560368in}}%
\pgfpathlineto{\pgfqpoint{1.621228in}{1.617965in}}%
\pgfpathlineto{\pgfqpoint{1.621432in}{1.609737in}}%
\pgfpathlineto{\pgfqpoint{1.622044in}{1.552140in}}%
\pgfpathlineto{\pgfqpoint{1.621840in}{1.617965in}}%
\pgfpathlineto{\pgfqpoint{1.622247in}{1.585053in}}%
\pgfpathlineto{\pgfqpoint{1.622451in}{1.650878in}}%
\pgfpathlineto{\pgfqpoint{1.623063in}{1.494542in}}%
\pgfpathlineto{\pgfqpoint{1.623266in}{1.494542in}}%
\pgfpathlineto{\pgfqpoint{1.623674in}{1.486314in}}%
\pgfpathlineto{\pgfqpoint{1.624489in}{1.535683in}}%
\pgfpathlineto{\pgfqpoint{1.625508in}{1.445173in}}%
\pgfpathlineto{\pgfqpoint{1.625712in}{1.478086in}}%
\pgfpathlineto{\pgfqpoint{1.625916in}{1.527455in}}%
\pgfpathlineto{\pgfqpoint{1.626120in}{1.453401in}}%
\pgfpathlineto{\pgfqpoint{1.626527in}{1.469858in}}%
\pgfpathlineto{\pgfqpoint{1.627546in}{1.428717in}}%
\pgfpathlineto{\pgfqpoint{1.627750in}{1.453401in}}%
\pgfpathlineto{\pgfqpoint{1.628158in}{1.543912in}}%
\pgfpathlineto{\pgfqpoint{1.628769in}{1.494542in}}%
\pgfpathlineto{\pgfqpoint{1.629788in}{1.272381in}}%
\pgfpathlineto{\pgfqpoint{1.630196in}{1.313522in}}%
\pgfpathlineto{\pgfqpoint{1.630807in}{1.519227in}}%
\pgfpathlineto{\pgfqpoint{1.631419in}{1.469858in}}%
\pgfpathlineto{\pgfqpoint{1.632438in}{1.436945in}}%
\pgfpathlineto{\pgfqpoint{1.633660in}{1.527455in}}%
\pgfpathlineto{\pgfqpoint{1.633049in}{1.412260in}}%
\pgfpathlineto{\pgfqpoint{1.633864in}{1.502771in}}%
\pgfpathlineto{\pgfqpoint{1.634476in}{1.469858in}}%
\pgfpathlineto{\pgfqpoint{1.634679in}{1.486314in}}%
\pgfpathlineto{\pgfqpoint{1.635087in}{1.478086in}}%
\pgfpathlineto{\pgfqpoint{1.636106in}{1.585053in}}%
\pgfpathlineto{\pgfqpoint{1.636717in}{1.469858in}}%
\pgfpathlineto{\pgfqpoint{1.637329in}{1.502771in}}%
\pgfpathlineto{\pgfqpoint{1.637533in}{1.453401in}}%
\pgfpathlineto{\pgfqpoint{1.637736in}{1.552140in}}%
\pgfpathlineto{\pgfqpoint{1.638348in}{1.494542in}}%
\pgfpathlineto{\pgfqpoint{1.638552in}{1.519227in}}%
\pgfpathlineto{\pgfqpoint{1.638756in}{1.461630in}}%
\pgfpathlineto{\pgfqpoint{1.638959in}{1.387576in}}%
\pgfpathlineto{\pgfqpoint{1.639775in}{1.428717in}}%
\pgfpathlineto{\pgfqpoint{1.639978in}{1.445173in}}%
\pgfpathlineto{\pgfqpoint{1.640182in}{1.436945in}}%
\pgfpathlineto{\pgfqpoint{1.640794in}{1.198327in}}%
\pgfpathlineto{\pgfqpoint{1.640997in}{1.305294in}}%
\pgfpathlineto{\pgfqpoint{1.641609in}{1.510999in}}%
\pgfpathlineto{\pgfqpoint{1.642220in}{1.436945in}}%
\pgfpathlineto{\pgfqpoint{1.642424in}{1.478086in}}%
\pgfpathlineto{\pgfqpoint{1.642628in}{1.395804in}}%
\pgfpathlineto{\pgfqpoint{1.643035in}{1.412260in}}%
\pgfpathlineto{\pgfqpoint{1.643239in}{1.395804in}}%
\pgfpathlineto{\pgfqpoint{1.643647in}{1.445173in}}%
\pgfpathlineto{\pgfqpoint{1.643851in}{1.461630in}}%
\pgfpathlineto{\pgfqpoint{1.644258in}{1.428717in}}%
\pgfpathlineto{\pgfqpoint{1.645073in}{1.321750in}}%
\pgfpathlineto{\pgfqpoint{1.645481in}{1.362891in}}%
\pgfpathlineto{\pgfqpoint{1.645889in}{1.395804in}}%
\pgfpathlineto{\pgfqpoint{1.646092in}{1.239468in}}%
\pgfpathlineto{\pgfqpoint{1.646908in}{1.461630in}}%
\pgfpathlineto{\pgfqpoint{1.647315in}{1.527455in}}%
\pgfpathlineto{\pgfqpoint{1.647723in}{1.453401in}}%
\pgfpathlineto{\pgfqpoint{1.647927in}{1.453401in}}%
\pgfpathlineto{\pgfqpoint{1.648130in}{1.346435in}}%
\pgfpathlineto{\pgfqpoint{1.648946in}{1.404032in}}%
\pgfpathlineto{\pgfqpoint{1.649353in}{1.510999in}}%
\pgfpathlineto{\pgfqpoint{1.650168in}{1.478086in}}%
\pgfpathlineto{\pgfqpoint{1.651391in}{1.412260in}}%
\pgfpathlineto{\pgfqpoint{1.651595in}{1.469858in}}%
\pgfpathlineto{\pgfqpoint{1.652410in}{1.453401in}}%
\pgfpathlineto{\pgfqpoint{1.653226in}{1.420489in}}%
\pgfpathlineto{\pgfqpoint{1.652818in}{1.494542in}}%
\pgfpathlineto{\pgfqpoint{1.653429in}{1.453401in}}%
\pgfpathlineto{\pgfqpoint{1.653633in}{1.453401in}}%
\pgfpathlineto{\pgfqpoint{1.653837in}{1.543912in}}%
\pgfpathlineto{\pgfqpoint{1.654245in}{1.428717in}}%
\pgfpathlineto{\pgfqpoint{1.654448in}{1.445173in}}%
\pgfpathlineto{\pgfqpoint{1.654856in}{1.181871in}}%
\pgfpathlineto{\pgfqpoint{1.655671in}{1.387576in}}%
\pgfpathlineto{\pgfqpoint{1.655875in}{1.478086in}}%
\pgfpathlineto{\pgfqpoint{1.656486in}{1.379348in}}%
\pgfpathlineto{\pgfqpoint{1.656690in}{1.404032in}}%
\pgfpathlineto{\pgfqpoint{1.657505in}{1.297066in}}%
\pgfpathlineto{\pgfqpoint{1.657709in}{1.371119in}}%
\pgfpathlineto{\pgfqpoint{1.658321in}{1.469858in}}%
\pgfpathlineto{\pgfqpoint{1.658728in}{1.371119in}}%
\pgfpathlineto{\pgfqpoint{1.659340in}{1.206555in}}%
\pgfpathlineto{\pgfqpoint{1.659543in}{1.058448in}}%
\pgfpathlineto{\pgfqpoint{1.659951in}{1.494542in}}%
\pgfpathlineto{\pgfqpoint{1.660155in}{1.387576in}}%
\pgfpathlineto{\pgfqpoint{1.661785in}{1.198327in}}%
\pgfpathlineto{\pgfqpoint{1.663212in}{1.395804in}}%
\pgfpathlineto{\pgfqpoint{1.664231in}{1.272381in}}%
\pgfpathlineto{\pgfqpoint{1.664027in}{1.412260in}}%
\pgfpathlineto{\pgfqpoint{1.664435in}{1.280609in}}%
\pgfpathlineto{\pgfqpoint{1.664638in}{1.338207in}}%
\pgfpathlineto{\pgfqpoint{1.665250in}{1.247696in}}%
\pgfpathlineto{\pgfqpoint{1.665454in}{1.074904in}}%
\pgfpathlineto{\pgfqpoint{1.665861in}{1.346435in}}%
\pgfpathlineto{\pgfqpoint{1.666269in}{1.264153in}}%
\pgfpathlineto{\pgfqpoint{1.666677in}{1.247696in}}%
\pgfpathlineto{\pgfqpoint{1.666880in}{1.272381in}}%
\pgfpathlineto{\pgfqpoint{1.667288in}{1.321750in}}%
\pgfpathlineto{\pgfqpoint{1.667696in}{1.239468in}}%
\pgfpathlineto{\pgfqpoint{1.668918in}{1.379348in}}%
\pgfpathlineto{\pgfqpoint{1.669122in}{1.346435in}}%
\pgfpathlineto{\pgfqpoint{1.669937in}{1.231240in}}%
\pgfpathlineto{\pgfqpoint{1.670345in}{1.239468in}}%
\pgfpathlineto{\pgfqpoint{1.671364in}{1.313522in}}%
\pgfpathlineto{\pgfqpoint{1.670956in}{1.223012in}}%
\pgfpathlineto{\pgfqpoint{1.671568in}{1.280609in}}%
\pgfpathlineto{\pgfqpoint{1.672179in}{1.387576in}}%
\pgfpathlineto{\pgfqpoint{1.671975in}{1.272381in}}%
\pgfpathlineto{\pgfqpoint{1.672383in}{1.297066in}}%
\pgfpathlineto{\pgfqpoint{1.672791in}{1.321750in}}%
\pgfpathlineto{\pgfqpoint{1.673198in}{1.132502in}}%
\pgfpathlineto{\pgfqpoint{1.674421in}{1.395804in}}%
\pgfpathlineto{\pgfqpoint{1.675032in}{1.338207in}}%
\pgfpathlineto{\pgfqpoint{1.675236in}{1.387576in}}%
\pgfpathlineto{\pgfqpoint{1.675440in}{1.404032in}}%
\pgfpathlineto{\pgfqpoint{1.675644in}{1.362891in}}%
\pgfpathlineto{\pgfqpoint{1.676051in}{1.387576in}}%
\pgfpathlineto{\pgfqpoint{1.676255in}{1.313522in}}%
\pgfpathlineto{\pgfqpoint{1.676459in}{1.404032in}}%
\pgfpathlineto{\pgfqpoint{1.677070in}{1.395804in}}%
\pgfpathlineto{\pgfqpoint{1.677274in}{1.404032in}}%
\pgfpathlineto{\pgfqpoint{1.677478in}{1.395804in}}%
\pgfpathlineto{\pgfqpoint{1.678701in}{1.116045in}}%
\pgfpathlineto{\pgfqpoint{1.678905in}{1.288837in}}%
\pgfpathlineto{\pgfqpoint{1.679109in}{1.297066in}}%
\pgfpathlineto{\pgfqpoint{1.679312in}{1.552140in}}%
\pgfpathlineto{\pgfqpoint{1.680128in}{1.395804in}}%
\pgfpathlineto{\pgfqpoint{1.680331in}{1.395804in}}%
\pgfpathlineto{\pgfqpoint{1.680535in}{1.412260in}}%
\pgfpathlineto{\pgfqpoint{1.681147in}{1.165414in}}%
\pgfpathlineto{\pgfqpoint{1.681554in}{1.445173in}}%
\pgfpathlineto{\pgfqpoint{1.682166in}{1.535683in}}%
\pgfpathlineto{\pgfqpoint{1.682573in}{1.478086in}}%
\pgfpathlineto{\pgfqpoint{1.684000in}{1.379348in}}%
\pgfpathlineto{\pgfqpoint{1.683388in}{1.502771in}}%
\pgfpathlineto{\pgfqpoint{1.684204in}{1.395804in}}%
\pgfpathlineto{\pgfqpoint{1.684407in}{1.395804in}}%
\pgfpathlineto{\pgfqpoint{1.684815in}{1.173643in}}%
\pgfpathlineto{\pgfqpoint{1.685426in}{1.404032in}}%
\pgfpathlineto{\pgfqpoint{1.685630in}{1.329978in}}%
\pgfpathlineto{\pgfqpoint{1.685834in}{1.436945in}}%
\pgfpathlineto{\pgfqpoint{1.686649in}{1.379348in}}%
\pgfpathlineto{\pgfqpoint{1.687057in}{1.280609in}}%
\pgfpathlineto{\pgfqpoint{1.687464in}{1.453401in}}%
\pgfpathlineto{\pgfqpoint{1.688280in}{1.313522in}}%
\pgfpathlineto{\pgfqpoint{1.691133in}{0.967938in}}%
\pgfpathlineto{\pgfqpoint{1.688891in}{1.338207in}}%
\pgfpathlineto{\pgfqpoint{1.691337in}{0.976166in}}%
\pgfpathlineto{\pgfqpoint{1.692356in}{0.967938in}}%
\pgfpathlineto{\pgfqpoint{1.692967in}{1.362891in}}%
\pgfpathlineto{\pgfqpoint{1.693986in}{1.058448in}}%
\pgfpathlineto{\pgfqpoint{1.694190in}{1.239468in}}%
\pgfpathlineto{\pgfqpoint{1.694598in}{1.190099in}}%
\pgfpathlineto{\pgfqpoint{1.694801in}{1.247696in}}%
\pgfpathlineto{\pgfqpoint{1.695005in}{1.206555in}}%
\pgfpathlineto{\pgfqpoint{1.695209in}{1.280609in}}%
\pgfpathlineto{\pgfqpoint{1.695617in}{1.190099in}}%
\pgfpathlineto{\pgfqpoint{1.695820in}{1.190099in}}%
\pgfpathlineto{\pgfqpoint{1.696432in}{1.214784in}}%
\pgfpathlineto{\pgfqpoint{1.696839in}{1.107817in}}%
\pgfpathlineto{\pgfqpoint{1.697043in}{1.264153in}}%
\pgfpathlineto{\pgfqpoint{1.697858in}{1.206555in}}%
\pgfpathlineto{\pgfqpoint{1.698266in}{1.165414in}}%
\pgfpathlineto{\pgfqpoint{1.698470in}{1.181871in}}%
\pgfpathlineto{\pgfqpoint{1.699081in}{1.313522in}}%
\pgfpathlineto{\pgfqpoint{1.699285in}{1.272381in}}%
\pgfpathlineto{\pgfqpoint{1.700304in}{0.967938in}}%
\pgfpathlineto{\pgfqpoint{1.700508in}{1.107817in}}%
\pgfpathlineto{\pgfqpoint{1.701323in}{1.132502in}}%
\pgfpathlineto{\pgfqpoint{1.701934in}{1.033763in}}%
\pgfpathlineto{\pgfqpoint{1.703157in}{1.214784in}}%
\pgfpathlineto{\pgfqpoint{1.703361in}{1.181871in}}%
\pgfpathlineto{\pgfqpoint{1.704991in}{0.976166in}}%
\pgfpathlineto{\pgfqpoint{1.705195in}{1.009079in}}%
\pgfpathlineto{\pgfqpoint{1.705399in}{0.836286in}}%
\pgfpathlineto{\pgfqpoint{1.705603in}{1.116045in}}%
\pgfpathlineto{\pgfqpoint{1.706214in}{1.009079in}}%
\pgfpathlineto{\pgfqpoint{1.706826in}{1.091361in}}%
\pgfpathlineto{\pgfqpoint{1.707437in}{1.066676in}}%
\pgfpathlineto{\pgfqpoint{1.707641in}{1.000850in}}%
\pgfpathlineto{\pgfqpoint{1.708252in}{1.132502in}}%
\pgfpathlineto{\pgfqpoint{1.708456in}{1.033763in}}%
\pgfpathlineto{\pgfqpoint{1.709068in}{1.231240in}}%
\pgfpathlineto{\pgfqpoint{1.709679in}{1.099589in}}%
\pgfpathlineto{\pgfqpoint{1.709883in}{1.066676in}}%
\pgfpathlineto{\pgfqpoint{1.710290in}{1.099589in}}%
\pgfpathlineto{\pgfqpoint{1.710494in}{1.157186in}}%
\pgfpathlineto{\pgfqpoint{1.710698in}{1.050220in}}%
\pgfpathlineto{\pgfqpoint{1.711106in}{1.116045in}}%
\pgfpathlineto{\pgfqpoint{1.711513in}{0.836286in}}%
\pgfpathlineto{\pgfqpoint{1.712125in}{0.959709in}}%
\pgfpathlineto{\pgfqpoint{1.712940in}{1.066676in}}%
\pgfpathlineto{\pgfqpoint{1.712736in}{0.918568in}}%
\pgfpathlineto{\pgfqpoint{1.713144in}{1.009079in}}%
\pgfpathlineto{\pgfqpoint{1.714570in}{0.902112in}}%
\pgfpathlineto{\pgfqpoint{1.715589in}{1.116045in}}%
\pgfpathlineto{\pgfqpoint{1.715793in}{1.083132in}}%
\pgfpathlineto{\pgfqpoint{1.716812in}{1.017307in}}%
\pgfpathlineto{\pgfqpoint{1.717016in}{1.025535in}}%
\pgfpathlineto{\pgfqpoint{1.718646in}{1.255925in}}%
\pgfpathlineto{\pgfqpoint{1.720073in}{1.420489in}}%
\pgfpathlineto{\pgfqpoint{1.720888in}{1.404032in}}%
\pgfpathlineto{\pgfqpoint{1.721092in}{1.387576in}}%
\pgfpathlineto{\pgfqpoint{1.721296in}{1.428717in}}%
\pgfpathlineto{\pgfqpoint{1.721703in}{1.420489in}}%
\pgfpathlineto{\pgfqpoint{1.721907in}{1.428717in}}%
\pgfpathlineto{\pgfqpoint{1.722111in}{1.387576in}}%
\pgfpathlineto{\pgfqpoint{1.722722in}{1.486314in}}%
\pgfpathlineto{\pgfqpoint{1.722926in}{1.445173in}}%
\pgfpathlineto{\pgfqpoint{1.723130in}{1.445173in}}%
\pgfpathlineto{\pgfqpoint{1.723538in}{1.436945in}}%
\pgfpathlineto{\pgfqpoint{1.723945in}{1.519227in}}%
\pgfpathlineto{\pgfqpoint{1.724760in}{1.642650in}}%
\pgfpathlineto{\pgfqpoint{1.725168in}{1.609737in}}%
\pgfpathlineto{\pgfqpoint{1.725372in}{1.486314in}}%
\pgfpathlineto{\pgfqpoint{1.726187in}{1.650878in}}%
\pgfpathlineto{\pgfqpoint{1.727002in}{1.568596in}}%
\pgfpathlineto{\pgfqpoint{1.727410in}{1.626194in}}%
\pgfpathlineto{\pgfqpoint{1.728836in}{1.741388in}}%
\pgfpathlineto{\pgfqpoint{1.728225in}{1.617965in}}%
\pgfpathlineto{\pgfqpoint{1.729244in}{1.716704in}}%
\pgfpathlineto{\pgfqpoint{1.730263in}{1.338207in}}%
\pgfpathlineto{\pgfqpoint{1.730671in}{1.527455in}}%
\pgfpathlineto{\pgfqpoint{1.730874in}{1.535683in}}%
\pgfpathlineto{\pgfqpoint{1.731486in}{1.453401in}}%
\pgfpathlineto{\pgfqpoint{1.731893in}{1.478086in}}%
\pgfpathlineto{\pgfqpoint{1.733320in}{1.692019in}}%
\pgfpathlineto{\pgfqpoint{1.733728in}{1.609737in}}%
\pgfpathlineto{\pgfqpoint{1.734135in}{1.626194in}}%
\pgfpathlineto{\pgfqpoint{1.735358in}{1.856583in}}%
\pgfpathlineto{\pgfqpoint{1.735562in}{1.831899in}}%
\pgfpathlineto{\pgfqpoint{1.736173in}{1.766073in}}%
\pgfpathlineto{\pgfqpoint{1.736785in}{1.782529in}}%
\pgfpathlineto{\pgfqpoint{1.736989in}{1.790758in}}%
\pgfpathlineto{\pgfqpoint{1.738008in}{1.947093in}}%
\pgfpathlineto{\pgfqpoint{1.738211in}{1.930637in}}%
\pgfpathlineto{\pgfqpoint{1.738415in}{1.733160in}}%
\pgfpathlineto{\pgfqpoint{1.738619in}{2.012919in}}%
\pgfpathlineto{\pgfqpoint{1.739027in}{2.004691in}}%
\pgfpathlineto{\pgfqpoint{1.739230in}{2.045832in}}%
\pgfpathlineto{\pgfqpoint{1.740249in}{2.037604in}}%
\pgfpathlineto{\pgfqpoint{1.740453in}{2.029375in}}%
\pgfpathlineto{\pgfqpoint{1.741065in}{1.856583in}}%
\pgfpathlineto{\pgfqpoint{1.741472in}{1.938865in}}%
\pgfpathlineto{\pgfqpoint{1.742899in}{2.045832in}}%
\pgfpathlineto{\pgfqpoint{1.743103in}{2.045832in}}%
\pgfpathlineto{\pgfqpoint{1.743714in}{2.078745in}}%
\pgfpathlineto{\pgfqpoint{1.743918in}{1.938865in}}%
\pgfpathlineto{\pgfqpoint{1.744733in}{2.111657in}}%
\pgfpathlineto{\pgfqpoint{1.744937in}{2.111657in}}%
\pgfpathlineto{\pgfqpoint{1.745141in}{2.078745in}}%
\pgfpathlineto{\pgfqpoint{1.745548in}{2.144570in}}%
\pgfpathlineto{\pgfqpoint{1.745752in}{2.218624in}}%
\pgfpathlineto{\pgfqpoint{1.746567in}{2.136342in}}%
\pgfpathlineto{\pgfqpoint{1.747383in}{2.078745in}}%
\pgfpathlineto{\pgfqpoint{1.747790in}{2.103429in}}%
\pgfpathlineto{\pgfqpoint{1.748402in}{2.169255in}}%
\pgfpathlineto{\pgfqpoint{1.748809in}{2.111657in}}%
\pgfpathlineto{\pgfqpoint{1.749828in}{2.037604in}}%
\pgfpathlineto{\pgfqpoint{1.749421in}{2.128114in}}%
\pgfpathlineto{\pgfqpoint{1.750440in}{2.086973in}}%
\pgfpathlineto{\pgfqpoint{1.751866in}{2.202168in}}%
\pgfpathlineto{\pgfqpoint{1.752070in}{2.021147in}}%
\pgfpathlineto{\pgfqpoint{1.752885in}{2.276221in}}%
\pgfpathlineto{\pgfqpoint{1.753293in}{2.177483in}}%
\pgfpathlineto{\pgfqpoint{1.753700in}{2.218624in}}%
\pgfpathlineto{\pgfqpoint{1.753904in}{2.045832in}}%
\pgfpathlineto{\pgfqpoint{1.754719in}{2.193939in}}%
\pgfpathlineto{\pgfqpoint{1.754923in}{2.185711in}}%
\pgfpathlineto{\pgfqpoint{1.755535in}{2.292678in}}%
\pgfpathlineto{\pgfqpoint{1.756146in}{2.259765in}}%
\pgfpathlineto{\pgfqpoint{1.756350in}{2.226852in}}%
\pgfpathlineto{\pgfqpoint{1.756961in}{2.284450in}}%
\pgfpathlineto{\pgfqpoint{1.757573in}{2.325591in}}%
\pgfpathlineto{\pgfqpoint{1.757776in}{2.276221in}}%
\pgfpathlineto{\pgfqpoint{1.757980in}{2.292678in}}%
\pgfpathlineto{\pgfqpoint{1.758999in}{2.128114in}}%
\pgfpathlineto{\pgfqpoint{1.759611in}{2.161027in}}%
\pgfpathlineto{\pgfqpoint{1.759815in}{2.218624in}}%
\pgfpathlineto{\pgfqpoint{1.760630in}{2.193939in}}%
\pgfpathlineto{\pgfqpoint{1.760834in}{2.136342in}}%
\pgfpathlineto{\pgfqpoint{1.761241in}{2.226852in}}%
\pgfpathlineto{\pgfqpoint{1.761445in}{2.177483in}}%
\pgfpathlineto{\pgfqpoint{1.762872in}{2.292678in}}%
\pgfpathlineto{\pgfqpoint{1.763687in}{2.226852in}}%
\pgfpathlineto{\pgfqpoint{1.764094in}{2.276221in}}%
\pgfpathlineto{\pgfqpoint{1.764706in}{2.325591in}}%
\pgfpathlineto{\pgfqpoint{1.765113in}{2.284450in}}%
\pgfpathlineto{\pgfqpoint{1.765317in}{2.251537in}}%
\pgfpathlineto{\pgfqpoint{1.765725in}{2.350275in}}%
\pgfpathlineto{\pgfqpoint{1.766132in}{2.292678in}}%
\pgfpathlineto{\pgfqpoint{1.766540in}{2.342047in}}%
\pgfpathlineto{\pgfqpoint{1.766744in}{2.226852in}}%
\pgfpathlineto{\pgfqpoint{1.767559in}{2.325591in}}%
\pgfpathlineto{\pgfqpoint{1.767763in}{2.317363in}}%
\pgfpathlineto{\pgfqpoint{1.767967in}{2.358504in}}%
\pgfpathlineto{\pgfqpoint{1.768374in}{2.309134in}}%
\pgfpathlineto{\pgfqpoint{1.768782in}{2.333819in}}%
\pgfpathlineto{\pgfqpoint{1.769393in}{2.276221in}}%
\pgfpathlineto{\pgfqpoint{1.769801in}{2.342047in}}%
\pgfpathlineto{\pgfqpoint{1.770005in}{2.350275in}}%
\pgfpathlineto{\pgfqpoint{1.771227in}{2.259765in}}%
\pgfpathlineto{\pgfqpoint{1.772246in}{2.366732in}}%
\pgfpathlineto{\pgfqpoint{1.772654in}{2.358504in}}%
\pgfpathlineto{\pgfqpoint{1.773469in}{2.276221in}}%
\pgfpathlineto{\pgfqpoint{1.773673in}{2.309134in}}%
\pgfpathlineto{\pgfqpoint{1.773877in}{2.366732in}}%
\pgfpathlineto{\pgfqpoint{1.774285in}{2.259765in}}%
\pgfpathlineto{\pgfqpoint{1.774692in}{2.309134in}}%
\pgfpathlineto{\pgfqpoint{1.774896in}{2.284450in}}%
\pgfpathlineto{\pgfqpoint{1.775507in}{2.333819in}}%
\pgfpathlineto{\pgfqpoint{1.775711in}{2.342047in}}%
\pgfpathlineto{\pgfqpoint{1.776526in}{2.226852in}}%
\pgfpathlineto{\pgfqpoint{1.776934in}{2.300906in}}%
\pgfpathlineto{\pgfqpoint{1.777138in}{2.333819in}}%
\pgfpathlineto{\pgfqpoint{1.777545in}{2.243309in}}%
\pgfpathlineto{\pgfqpoint{1.777749in}{2.292678in}}%
\pgfpathlineto{\pgfqpoint{1.777953in}{2.276221in}}%
\pgfpathlineto{\pgfqpoint{1.778361in}{2.309134in}}%
\pgfpathlineto{\pgfqpoint{1.778564in}{2.333819in}}%
\pgfpathlineto{\pgfqpoint{1.778972in}{2.267993in}}%
\pgfpathlineto{\pgfqpoint{1.779176in}{2.276221in}}%
\pgfpathlineto{\pgfqpoint{1.780195in}{2.218624in}}%
\pgfpathlineto{\pgfqpoint{1.780399in}{2.251537in}}%
\pgfpathlineto{\pgfqpoint{1.781825in}{2.424329in}}%
\pgfpathlineto{\pgfqpoint{1.782029in}{2.424329in}}%
\pgfpathlineto{\pgfqpoint{1.782233in}{2.490155in}}%
\pgfpathlineto{\pgfqpoint{1.783252in}{2.465470in}}%
\pgfpathlineto{\pgfqpoint{1.783659in}{2.514839in}}%
\pgfpathlineto{\pgfqpoint{1.784882in}{2.383188in}}%
\pgfpathlineto{\pgfqpoint{1.785697in}{2.531296in}}%
\pgfpathlineto{\pgfqpoint{1.785901in}{2.523068in}}%
\pgfpathlineto{\pgfqpoint{1.786105in}{2.284450in}}%
\pgfpathlineto{\pgfqpoint{1.786920in}{2.597121in}}%
\pgfpathlineto{\pgfqpoint{1.787124in}{2.621806in}}%
\pgfpathlineto{\pgfqpoint{1.787328in}{2.514839in}}%
\pgfpathlineto{\pgfqpoint{1.787532in}{2.531296in}}%
\pgfpathlineto{\pgfqpoint{1.788143in}{2.416101in}}%
\pgfpathlineto{\pgfqpoint{1.788755in}{2.457242in}}%
\pgfpathlineto{\pgfqpoint{1.789162in}{2.457242in}}%
\pgfpathlineto{\pgfqpoint{1.789366in}{2.465470in}}%
\pgfpathlineto{\pgfqpoint{1.789570in}{2.407873in}}%
\pgfpathlineto{\pgfqpoint{1.789774in}{2.531296in}}%
\pgfpathlineto{\pgfqpoint{1.790589in}{2.424329in}}%
\pgfpathlineto{\pgfqpoint{1.790793in}{2.424329in}}%
\pgfpathlineto{\pgfqpoint{1.791404in}{2.416101in}}%
\pgfpathlineto{\pgfqpoint{1.791812in}{2.473698in}}%
\pgfpathlineto{\pgfqpoint{1.792423in}{2.432557in}}%
\pgfpathlineto{\pgfqpoint{1.792627in}{2.424329in}}%
\pgfpathlineto{\pgfqpoint{1.792831in}{2.473698in}}%
\pgfpathlineto{\pgfqpoint{1.793646in}{2.465470in}}%
\pgfpathlineto{\pgfqpoint{1.794053in}{2.383188in}}%
\pgfpathlineto{\pgfqpoint{1.794665in}{2.465470in}}%
\pgfpathlineto{\pgfqpoint{1.796499in}{2.555980in}}%
\pgfpathlineto{\pgfqpoint{1.797110in}{2.506611in}}%
\pgfpathlineto{\pgfqpoint{1.797314in}{2.539524in}}%
\pgfpathlineto{\pgfqpoint{1.798129in}{2.621806in}}%
\pgfpathlineto{\pgfqpoint{1.798333in}{2.547752in}}%
\pgfpathlineto{\pgfqpoint{1.798537in}{2.539524in}}%
\pgfpathlineto{\pgfqpoint{1.799556in}{2.399645in}}%
\pgfpathlineto{\pgfqpoint{1.800575in}{2.514839in}}%
\pgfpathlineto{\pgfqpoint{1.800779in}{2.506611in}}%
\pgfpathlineto{\pgfqpoint{1.801187in}{2.465470in}}%
\pgfpathlineto{\pgfqpoint{1.801594in}{2.547752in}}%
\pgfpathlineto{\pgfqpoint{1.801798in}{2.498383in}}%
\pgfpathlineto{\pgfqpoint{1.803021in}{2.564209in}}%
\pgfpathlineto{\pgfqpoint{1.804040in}{2.506611in}}%
\pgfpathlineto{\pgfqpoint{1.804244in}{2.564209in}}%
\pgfpathlineto{\pgfqpoint{1.804855in}{2.465470in}}%
\pgfpathlineto{\pgfqpoint{1.805059in}{2.498383in}}%
\pgfpathlineto{\pgfqpoint{1.806078in}{2.243309in}}%
\pgfpathlineto{\pgfqpoint{1.806689in}{2.333819in}}%
\pgfpathlineto{\pgfqpoint{1.808727in}{2.597121in}}%
\pgfpathlineto{\pgfqpoint{1.809135in}{2.621806in}}%
\pgfpathlineto{\pgfqpoint{1.809339in}{2.588893in}}%
\pgfpathlineto{\pgfqpoint{1.809950in}{2.292678in}}%
\pgfpathlineto{\pgfqpoint{1.810358in}{2.407873in}}%
\pgfpathlineto{\pgfqpoint{1.810561in}{2.646491in}}%
\pgfpathlineto{\pgfqpoint{1.811580in}{2.572437in}}%
\pgfpathlineto{\pgfqpoint{1.812599in}{2.358504in}}%
\pgfpathlineto{\pgfqpoint{1.812803in}{2.588893in}}%
\pgfpathlineto{\pgfqpoint{1.813415in}{2.514839in}}%
\pgfpathlineto{\pgfqpoint{1.813619in}{2.251537in}}%
\pgfpathlineto{\pgfqpoint{1.814434in}{2.613578in}}%
\pgfpathlineto{\pgfqpoint{1.815249in}{2.366732in}}%
\pgfpathlineto{\pgfqpoint{1.816064in}{2.457242in}}%
\pgfpathlineto{\pgfqpoint{1.816472in}{2.473698in}}%
\pgfpathlineto{\pgfqpoint{1.816676in}{2.440786in}}%
\pgfpathlineto{\pgfqpoint{1.817287in}{2.473698in}}%
\pgfpathlineto{\pgfqpoint{1.817898in}{2.333819in}}%
\pgfpathlineto{\pgfqpoint{1.818510in}{2.490155in}}%
\pgfpathlineto{\pgfqpoint{1.819121in}{2.473698in}}%
\pgfpathlineto{\pgfqpoint{1.819325in}{2.531296in}}%
\pgfpathlineto{\pgfqpoint{1.819936in}{2.407873in}}%
\pgfpathlineto{\pgfqpoint{1.821567in}{2.531296in}}%
\pgfpathlineto{\pgfqpoint{1.821974in}{2.210396in}}%
\pgfpathlineto{\pgfqpoint{1.822586in}{2.407873in}}%
\pgfpathlineto{\pgfqpoint{1.822993in}{2.531296in}}%
\pgfpathlineto{\pgfqpoint{1.823401in}{2.399645in}}%
\pgfpathlineto{\pgfqpoint{1.823605in}{2.407873in}}%
\pgfpathlineto{\pgfqpoint{1.824420in}{2.309134in}}%
\pgfpathlineto{\pgfqpoint{1.824012in}{2.465470in}}%
\pgfpathlineto{\pgfqpoint{1.824828in}{2.366732in}}%
\pgfpathlineto{\pgfqpoint{1.825031in}{2.366732in}}%
\pgfpathlineto{\pgfqpoint{1.825439in}{2.391416in}}%
\pgfpathlineto{\pgfqpoint{1.826662in}{2.251537in}}%
\pgfpathlineto{\pgfqpoint{1.826866in}{2.259765in}}%
\pgfpathlineto{\pgfqpoint{1.827070in}{2.300906in}}%
\pgfpathlineto{\pgfqpoint{1.827273in}{2.226852in}}%
\pgfpathlineto{\pgfqpoint{1.827885in}{2.251537in}}%
\pgfpathlineto{\pgfqpoint{1.828904in}{2.202168in}}%
\pgfpathlineto{\pgfqpoint{1.830127in}{2.267993in}}%
\pgfpathlineto{\pgfqpoint{1.830534in}{2.152798in}}%
\pgfpathlineto{\pgfqpoint{1.831146in}{2.226852in}}%
\pgfpathlineto{\pgfqpoint{1.831349in}{2.251537in}}%
\pgfpathlineto{\pgfqpoint{1.831553in}{2.210396in}}%
\pgfpathlineto{\pgfqpoint{1.832165in}{2.218624in}}%
\pgfpathlineto{\pgfqpoint{1.832368in}{2.218624in}}%
\pgfpathlineto{\pgfqpoint{1.832572in}{2.267993in}}%
\pgfpathlineto{\pgfqpoint{1.832980in}{2.202168in}}%
\pgfpathlineto{\pgfqpoint{1.833387in}{2.202168in}}%
\pgfpathlineto{\pgfqpoint{1.833795in}{2.128114in}}%
\pgfpathlineto{\pgfqpoint{1.833999in}{2.202168in}}%
\pgfpathlineto{\pgfqpoint{1.834203in}{2.284450in}}%
\pgfpathlineto{\pgfqpoint{1.834814in}{2.177483in}}%
\pgfpathlineto{\pgfqpoint{1.835018in}{2.177483in}}%
\pgfpathlineto{\pgfqpoint{1.835425in}{2.193939in}}%
\pgfpathlineto{\pgfqpoint{1.836037in}{2.152798in}}%
\pgfpathlineto{\pgfqpoint{1.836241in}{1.980006in}}%
\pgfpathlineto{\pgfqpoint{1.836852in}{2.235080in}}%
\pgfpathlineto{\pgfqpoint{1.837056in}{2.243309in}}%
\pgfpathlineto{\pgfqpoint{1.837260in}{2.235080in}}%
\pgfpathlineto{\pgfqpoint{1.837871in}{1.938865in}}%
\pgfpathlineto{\pgfqpoint{1.838279in}{2.202168in}}%
\pgfpathlineto{\pgfqpoint{1.839501in}{2.259765in}}%
\pgfpathlineto{\pgfqpoint{1.839909in}{2.226852in}}%
\pgfpathlineto{\pgfqpoint{1.840113in}{2.218624in}}%
\pgfpathlineto{\pgfqpoint{1.840317in}{2.251537in}}%
\pgfpathlineto{\pgfqpoint{1.840521in}{2.251537in}}%
\pgfpathlineto{\pgfqpoint{1.840724in}{2.276221in}}%
\pgfpathlineto{\pgfqpoint{1.841132in}{2.235080in}}%
\pgfpathlineto{\pgfqpoint{1.841336in}{2.251537in}}%
\pgfpathlineto{\pgfqpoint{1.841540in}{2.193939in}}%
\pgfpathlineto{\pgfqpoint{1.841947in}{2.267993in}}%
\pgfpathlineto{\pgfqpoint{1.842762in}{2.383188in}}%
\pgfpathlineto{\pgfqpoint{1.843170in}{2.309134in}}%
\pgfpathlineto{\pgfqpoint{1.843985in}{2.358504in}}%
\pgfpathlineto{\pgfqpoint{1.843781in}{2.292678in}}%
\pgfpathlineto{\pgfqpoint{1.844597in}{2.325591in}}%
\pgfpathlineto{\pgfqpoint{1.844800in}{2.292678in}}%
\pgfpathlineto{\pgfqpoint{1.845412in}{2.333819in}}%
\pgfpathlineto{\pgfqpoint{1.845616in}{2.317363in}}%
\pgfpathlineto{\pgfqpoint{1.846431in}{2.358504in}}%
\pgfpathlineto{\pgfqpoint{1.846838in}{2.333819in}}%
\pgfpathlineto{\pgfqpoint{1.847042in}{2.235080in}}%
\pgfpathlineto{\pgfqpoint{1.847857in}{2.358504in}}%
\pgfpathlineto{\pgfqpoint{1.848061in}{2.383188in}}%
\pgfpathlineto{\pgfqpoint{1.848265in}{2.342047in}}%
\pgfpathlineto{\pgfqpoint{1.848469in}{2.259765in}}%
\pgfpathlineto{\pgfqpoint{1.849284in}{2.309134in}}%
\pgfpathlineto{\pgfqpoint{1.850507in}{2.218624in}}%
\pgfpathlineto{\pgfqpoint{1.850711in}{2.202168in}}%
\pgfpathlineto{\pgfqpoint{1.850914in}{2.226852in}}%
\pgfpathlineto{\pgfqpoint{1.851118in}{2.259765in}}%
\pgfpathlineto{\pgfqpoint{1.851322in}{2.202168in}}%
\pgfpathlineto{\pgfqpoint{1.851933in}{2.218624in}}%
\pgfpathlineto{\pgfqpoint{1.852137in}{2.226852in}}%
\pgfpathlineto{\pgfqpoint{1.852341in}{2.152798in}}%
\pgfpathlineto{\pgfqpoint{1.853156in}{2.177483in}}%
\pgfpathlineto{\pgfqpoint{1.854175in}{2.243309in}}%
\pgfpathlineto{\pgfqpoint{1.854379in}{2.177483in}}%
\pgfpathlineto{\pgfqpoint{1.854991in}{2.317363in}}%
\pgfpathlineto{\pgfqpoint{1.855194in}{2.317363in}}%
\pgfpathlineto{\pgfqpoint{1.855602in}{2.284450in}}%
\pgfpathlineto{\pgfqpoint{1.856213in}{2.300906in}}%
\pgfpathlineto{\pgfqpoint{1.856417in}{2.309134in}}%
\pgfpathlineto{\pgfqpoint{1.857029in}{2.366732in}}%
\pgfpathlineto{\pgfqpoint{1.857640in}{2.226852in}}%
\pgfpathlineto{\pgfqpoint{1.858863in}{2.342047in}}%
\pgfpathlineto{\pgfqpoint{1.859067in}{2.317363in}}%
\pgfpathlineto{\pgfqpoint{1.859882in}{2.054060in}}%
\pgfpathlineto{\pgfqpoint{1.860289in}{2.267993in}}%
\pgfpathlineto{\pgfqpoint{1.861105in}{2.251537in}}%
\pgfpathlineto{\pgfqpoint{1.861308in}{2.309134in}}%
\pgfpathlineto{\pgfqpoint{1.861920in}{2.078745in}}%
\pgfpathlineto{\pgfqpoint{1.862735in}{2.226852in}}%
\pgfpathlineto{\pgfqpoint{1.862939in}{2.267993in}}%
\pgfpathlineto{\pgfqpoint{1.863550in}{2.169255in}}%
\pgfpathlineto{\pgfqpoint{1.865384in}{2.342047in}}%
\pgfpathlineto{\pgfqpoint{1.866403in}{2.300906in}}%
\pgfpathlineto{\pgfqpoint{1.866607in}{2.358504in}}%
\pgfpathlineto{\pgfqpoint{1.867423in}{2.300906in}}%
\pgfpathlineto{\pgfqpoint{1.867626in}{2.267993in}}%
\pgfpathlineto{\pgfqpoint{1.868238in}{2.342047in}}%
\pgfpathlineto{\pgfqpoint{1.868442in}{2.300906in}}%
\pgfpathlineto{\pgfqpoint{1.868849in}{2.251537in}}%
\pgfpathlineto{\pgfqpoint{1.869461in}{2.383188in}}%
\pgfpathlineto{\pgfqpoint{1.870683in}{2.284450in}}%
\pgfpathlineto{\pgfqpoint{1.871091in}{2.276221in}}%
\pgfpathlineto{\pgfqpoint{1.871906in}{2.366732in}}%
\pgfpathlineto{\pgfqpoint{1.872314in}{2.309134in}}%
\pgfpathlineto{\pgfqpoint{1.872721in}{2.317363in}}%
\pgfpathlineto{\pgfqpoint{1.873129in}{2.366732in}}%
\pgfpathlineto{\pgfqpoint{1.873944in}{2.251537in}}%
\pgfpathlineto{\pgfqpoint{1.874556in}{2.309134in}}%
\pgfpathlineto{\pgfqpoint{1.875167in}{2.300906in}}%
\pgfpathlineto{\pgfqpoint{1.875778in}{2.292678in}}%
\pgfpathlineto{\pgfqpoint{1.875982in}{2.333819in}}%
\pgfpathlineto{\pgfqpoint{1.876390in}{2.267993in}}%
\pgfpathlineto{\pgfqpoint{1.876797in}{2.350275in}}%
\pgfpathlineto{\pgfqpoint{1.877001in}{2.366732in}}%
\pgfpathlineto{\pgfqpoint{1.877205in}{2.358504in}}%
\pgfpathlineto{\pgfqpoint{1.878224in}{2.054060in}}%
\pgfpathlineto{\pgfqpoint{1.878428in}{2.095201in}}%
\pgfpathlineto{\pgfqpoint{1.878835in}{2.424329in}}%
\pgfpathlineto{\pgfqpoint{1.879651in}{2.383188in}}%
\pgfpathlineto{\pgfqpoint{1.880874in}{2.078745in}}%
\pgfpathlineto{\pgfqpoint{1.881893in}{2.342047in}}%
\pgfpathlineto{\pgfqpoint{1.882096in}{2.325591in}}%
\pgfpathlineto{\pgfqpoint{1.882912in}{2.407873in}}%
\pgfpathlineto{\pgfqpoint{1.883115in}{2.366732in}}%
\pgfpathlineto{\pgfqpoint{1.884134in}{2.111657in}}%
\pgfpathlineto{\pgfqpoint{1.883727in}{2.416101in}}%
\pgfpathlineto{\pgfqpoint{1.884338in}{2.218624in}}%
\pgfpathlineto{\pgfqpoint{1.885561in}{2.407873in}}%
\pgfpathlineto{\pgfqpoint{1.885765in}{2.399645in}}%
\pgfpathlineto{\pgfqpoint{1.886580in}{2.449014in}}%
\pgfpathlineto{\pgfqpoint{1.887803in}{2.284450in}}%
\pgfpathlineto{\pgfqpoint{1.888822in}{2.457242in}}%
\pgfpathlineto{\pgfqpoint{1.888414in}{2.177483in}}%
\pgfpathlineto{\pgfqpoint{1.889026in}{2.407873in}}%
\pgfpathlineto{\pgfqpoint{1.889229in}{2.144570in}}%
\pgfpathlineto{\pgfqpoint{1.890045in}{2.292678in}}%
\pgfpathlineto{\pgfqpoint{1.890248in}{2.366732in}}%
\pgfpathlineto{\pgfqpoint{1.890656in}{2.128114in}}%
\pgfpathlineto{\pgfqpoint{1.890860in}{2.062288in}}%
\pgfpathlineto{\pgfqpoint{1.891064in}{2.317363in}}%
\pgfpathlineto{\pgfqpoint{1.891267in}{2.317363in}}%
\pgfpathlineto{\pgfqpoint{1.891675in}{2.333819in}}%
\pgfpathlineto{\pgfqpoint{1.891879in}{2.407873in}}%
\pgfpathlineto{\pgfqpoint{1.892490in}{2.226852in}}%
\pgfpathlineto{\pgfqpoint{1.893713in}{2.358504in}}%
\pgfpathlineto{\pgfqpoint{1.893917in}{2.350275in}}%
\pgfpathlineto{\pgfqpoint{1.894325in}{2.457242in}}%
\pgfpathlineto{\pgfqpoint{1.895547in}{2.416101in}}%
\pgfpathlineto{\pgfqpoint{1.896974in}{2.333819in}}%
\pgfpathlineto{\pgfqpoint{1.897585in}{2.424329in}}%
\pgfpathlineto{\pgfqpoint{1.897993in}{2.333819in}}%
\pgfpathlineto{\pgfqpoint{1.898401in}{2.292678in}}%
\pgfpathlineto{\pgfqpoint{1.898604in}{2.350275in}}%
\pgfpathlineto{\pgfqpoint{1.899216in}{2.358504in}}%
\pgfpathlineto{\pgfqpoint{1.900439in}{2.078745in}}%
\pgfpathlineto{\pgfqpoint{1.899827in}{2.374960in}}%
\pgfpathlineto{\pgfqpoint{1.900642in}{2.292678in}}%
\pgfpathlineto{\pgfqpoint{1.901865in}{2.374960in}}%
\pgfpathlineto{\pgfqpoint{1.902273in}{2.358504in}}%
\pgfpathlineto{\pgfqpoint{1.903496in}{2.267993in}}%
\pgfpathlineto{\pgfqpoint{1.902680in}{2.383188in}}%
\pgfpathlineto{\pgfqpoint{1.903903in}{2.292678in}}%
\pgfpathlineto{\pgfqpoint{1.904922in}{2.407873in}}%
\pgfpathlineto{\pgfqpoint{1.905126in}{2.342047in}}%
\pgfpathlineto{\pgfqpoint{1.905330in}{2.350275in}}%
\pgfpathlineto{\pgfqpoint{1.905534in}{2.325591in}}%
\pgfpathlineto{\pgfqpoint{1.905737in}{2.325591in}}%
\pgfpathlineto{\pgfqpoint{1.905941in}{2.317363in}}%
\pgfpathlineto{\pgfqpoint{1.906349in}{2.300906in}}%
\pgfpathlineto{\pgfqpoint{1.907572in}{2.432557in}}%
\pgfpathlineto{\pgfqpoint{1.909610in}{2.333819in}}%
\pgfpathlineto{\pgfqpoint{1.909814in}{2.399645in}}%
\pgfpathlineto{\pgfqpoint{1.910629in}{2.317363in}}%
\pgfpathlineto{\pgfqpoint{1.910833in}{2.374960in}}%
\pgfpathlineto{\pgfqpoint{1.911036in}{2.276221in}}%
\pgfpathlineto{\pgfqpoint{1.911852in}{2.399645in}}%
\pgfpathlineto{\pgfqpoint{1.912667in}{2.300906in}}%
\pgfpathlineto{\pgfqpoint{1.912259in}{2.416101in}}%
\pgfpathlineto{\pgfqpoint{1.912871in}{2.350275in}}%
\pgfpathlineto{\pgfqpoint{1.913074in}{2.407873in}}%
\pgfpathlineto{\pgfqpoint{1.913890in}{2.333819in}}%
\pgfpathlineto{\pgfqpoint{1.914093in}{2.325591in}}%
\pgfpathlineto{\pgfqpoint{1.914297in}{2.358504in}}%
\pgfpathlineto{\pgfqpoint{1.914501in}{2.366732in}}%
\pgfpathlineto{\pgfqpoint{1.915520in}{2.235080in}}%
\pgfpathlineto{\pgfqpoint{1.915724in}{2.333819in}}%
\pgfpathlineto{\pgfqpoint{1.916131in}{2.325591in}}%
\pgfpathlineto{\pgfqpoint{1.916335in}{2.342047in}}%
\pgfpathlineto{\pgfqpoint{1.916743in}{2.309134in}}%
\pgfpathlineto{\pgfqpoint{1.917762in}{2.416101in}}%
\pgfpathlineto{\pgfqpoint{1.919392in}{2.350275in}}%
\pgfpathlineto{\pgfqpoint{1.919596in}{2.432557in}}%
\pgfpathlineto{\pgfqpoint{1.920207in}{2.317363in}}%
\pgfpathlineto{\pgfqpoint{1.921023in}{2.136342in}}%
\pgfpathlineto{\pgfqpoint{1.920615in}{2.374960in}}%
\pgfpathlineto{\pgfqpoint{1.921227in}{2.161027in}}%
\pgfpathlineto{\pgfqpoint{1.922246in}{2.383188in}}%
\pgfpathlineto{\pgfqpoint{1.922449in}{2.358504in}}%
\pgfpathlineto{\pgfqpoint{1.923061in}{2.078745in}}%
\pgfpathlineto{\pgfqpoint{1.923468in}{2.374960in}}%
\pgfpathlineto{\pgfqpoint{1.923672in}{2.358504in}}%
\pgfpathlineto{\pgfqpoint{1.923876in}{2.399645in}}%
\pgfpathlineto{\pgfqpoint{1.924284in}{2.457242in}}%
\pgfpathlineto{\pgfqpoint{1.924487in}{2.407873in}}%
\pgfpathlineto{\pgfqpoint{1.925303in}{2.333819in}}%
\pgfpathlineto{\pgfqpoint{1.925506in}{2.383188in}}%
\pgfpathlineto{\pgfqpoint{1.926118in}{2.407873in}}%
\pgfpathlineto{\pgfqpoint{1.927544in}{2.292678in}}%
\pgfpathlineto{\pgfqpoint{1.927748in}{2.267993in}}%
\pgfpathlineto{\pgfqpoint{1.927952in}{2.350275in}}%
\pgfpathlineto{\pgfqpoint{1.928563in}{2.342047in}}%
\pgfpathlineto{\pgfqpoint{1.929175in}{2.366732in}}%
\pgfpathlineto{\pgfqpoint{1.929379in}{2.333819in}}%
\pgfpathlineto{\pgfqpoint{1.929582in}{2.358504in}}%
\pgfpathlineto{\pgfqpoint{1.929786in}{2.086973in}}%
\pgfpathlineto{\pgfqpoint{1.930601in}{2.226852in}}%
\pgfpathlineto{\pgfqpoint{1.932232in}{2.292678in}}%
\pgfpathlineto{\pgfqpoint{1.932436in}{2.284450in}}%
\pgfpathlineto{\pgfqpoint{1.933251in}{2.251537in}}%
\pgfpathlineto{\pgfqpoint{1.933658in}{2.333819in}}%
\pgfpathlineto{\pgfqpoint{1.934881in}{2.267993in}}%
\pgfpathlineto{\pgfqpoint{1.934066in}{2.366732in}}%
\pgfpathlineto{\pgfqpoint{1.935085in}{2.292678in}}%
\pgfpathlineto{\pgfqpoint{1.935900in}{2.333819in}}%
\pgfpathlineto{\pgfqpoint{1.937327in}{2.210396in}}%
\pgfpathlineto{\pgfqpoint{1.938550in}{2.342047in}}%
\pgfpathlineto{\pgfqpoint{1.938957in}{2.350275in}}%
\pgfpathlineto{\pgfqpoint{1.939161in}{2.317363in}}%
\pgfpathlineto{\pgfqpoint{1.939365in}{2.374960in}}%
\pgfpathlineto{\pgfqpoint{1.939773in}{2.366732in}}%
\pgfpathlineto{\pgfqpoint{1.939976in}{2.374960in}}%
\pgfpathlineto{\pgfqpoint{1.940180in}{2.366732in}}%
\pgfpathlineto{\pgfqpoint{1.940792in}{2.276221in}}%
\pgfpathlineto{\pgfqpoint{1.941403in}{2.333819in}}%
\pgfpathlineto{\pgfqpoint{1.941607in}{2.325591in}}%
\pgfpathlineto{\pgfqpoint{1.941811in}{2.333819in}}%
\pgfpathlineto{\pgfqpoint{1.942422in}{2.498383in}}%
\pgfpathlineto{\pgfqpoint{1.942830in}{2.391416in}}%
\pgfpathlineto{\pgfqpoint{1.943849in}{2.309134in}}%
\pgfpathlineto{\pgfqpoint{1.944052in}{2.333819in}}%
\pgfpathlineto{\pgfqpoint{1.944256in}{2.309134in}}%
\pgfpathlineto{\pgfqpoint{1.944460in}{2.358504in}}%
\pgfpathlineto{\pgfqpoint{1.945683in}{2.424329in}}%
\pgfpathlineto{\pgfqpoint{1.946294in}{2.374960in}}%
\pgfpathlineto{\pgfqpoint{1.946498in}{2.325591in}}%
\pgfpathlineto{\pgfqpoint{1.947109in}{2.416101in}}%
\pgfpathlineto{\pgfqpoint{1.947517in}{2.391416in}}%
\pgfpathlineto{\pgfqpoint{1.947721in}{2.399645in}}%
\pgfpathlineto{\pgfqpoint{1.947925in}{2.449014in}}%
\pgfpathlineto{\pgfqpoint{1.948536in}{2.407873in}}%
\pgfpathlineto{\pgfqpoint{1.949148in}{2.440786in}}%
\pgfpathlineto{\pgfqpoint{1.950167in}{2.169255in}}%
\pgfpathlineto{\pgfqpoint{1.951389in}{2.424329in}}%
\pgfpathlineto{\pgfqpoint{1.951593in}{2.407873in}}%
\pgfpathlineto{\pgfqpoint{1.951797in}{2.424329in}}%
\pgfpathlineto{\pgfqpoint{1.952001in}{2.358504in}}%
\pgfpathlineto{\pgfqpoint{1.952408in}{2.366732in}}%
\pgfpathlineto{\pgfqpoint{1.952612in}{2.465470in}}%
\pgfpathlineto{\pgfqpoint{1.953427in}{2.383188in}}%
\pgfpathlineto{\pgfqpoint{1.954243in}{2.465470in}}%
\pgfpathlineto{\pgfqpoint{1.954446in}{2.407873in}}%
\pgfpathlineto{\pgfqpoint{1.954854in}{2.309134in}}%
\pgfpathlineto{\pgfqpoint{1.955873in}{2.366732in}}%
\pgfpathlineto{\pgfqpoint{1.956077in}{2.473698in}}%
\pgfpathlineto{\pgfqpoint{1.956892in}{2.358504in}}%
\pgfpathlineto{\pgfqpoint{1.957096in}{2.218624in}}%
\pgfpathlineto{\pgfqpoint{1.957503in}{2.465470in}}%
\pgfpathlineto{\pgfqpoint{1.957911in}{2.350275in}}%
\pgfpathlineto{\pgfqpoint{1.959134in}{2.539524in}}%
\pgfpathlineto{\pgfqpoint{1.959541in}{2.325591in}}%
\pgfpathlineto{\pgfqpoint{1.959745in}{2.580665in}}%
\pgfpathlineto{\pgfqpoint{1.960153in}{2.539524in}}%
\pgfpathlineto{\pgfqpoint{1.960764in}{2.490155in}}%
\pgfpathlineto{\pgfqpoint{1.960968in}{2.358504in}}%
\pgfpathlineto{\pgfqpoint{1.961783in}{2.498383in}}%
\pgfpathlineto{\pgfqpoint{1.962395in}{2.251537in}}%
\pgfpathlineto{\pgfqpoint{1.962802in}{2.506611in}}%
\pgfpathlineto{\pgfqpoint{1.963210in}{2.457242in}}%
\pgfpathlineto{\pgfqpoint{1.963414in}{2.210396in}}%
\pgfpathlineto{\pgfqpoint{1.963821in}{2.490155in}}%
\pgfpathlineto{\pgfqpoint{1.964229in}{2.449014in}}%
\pgfpathlineto{\pgfqpoint{1.965044in}{2.490155in}}%
\pgfpathlineto{\pgfqpoint{1.964840in}{2.432557in}}%
\pgfpathlineto{\pgfqpoint{1.965452in}{2.481927in}}%
\pgfpathlineto{\pgfqpoint{1.965656in}{2.498383in}}%
\pgfpathlineto{\pgfqpoint{1.966063in}{2.490155in}}%
\pgfpathlineto{\pgfqpoint{1.966878in}{2.350275in}}%
\pgfpathlineto{\pgfqpoint{1.966471in}{2.498383in}}%
\pgfpathlineto{\pgfqpoint{1.967082in}{2.424329in}}%
\pgfpathlineto{\pgfqpoint{1.968101in}{2.523068in}}%
\pgfpathlineto{\pgfqpoint{1.967694in}{2.383188in}}%
\pgfpathlineto{\pgfqpoint{1.968305in}{2.514839in}}%
\pgfpathlineto{\pgfqpoint{1.968713in}{2.424329in}}%
\pgfpathlineto{\pgfqpoint{1.969324in}{2.531296in}}%
\pgfpathlineto{\pgfqpoint{1.969732in}{2.481927in}}%
\pgfpathlineto{\pgfqpoint{1.969935in}{2.523068in}}%
\pgfpathlineto{\pgfqpoint{1.970139in}{2.383188in}}%
\pgfpathlineto{\pgfqpoint{1.970954in}{2.498383in}}%
\pgfpathlineto{\pgfqpoint{1.971566in}{2.465470in}}%
\pgfpathlineto{\pgfqpoint{1.972381in}{2.473698in}}%
\pgfpathlineto{\pgfqpoint{1.972585in}{2.490155in}}%
\pgfpathlineto{\pgfqpoint{1.972992in}{2.473698in}}%
\pgfpathlineto{\pgfqpoint{1.973196in}{2.432557in}}%
\pgfpathlineto{\pgfqpoint{1.973400in}{2.481927in}}%
\pgfpathlineto{\pgfqpoint{1.974011in}{2.465470in}}%
\pgfpathlineto{\pgfqpoint{1.975031in}{2.399645in}}%
\pgfpathlineto{\pgfqpoint{1.975642in}{2.325591in}}%
\pgfpathlineto{\pgfqpoint{1.976050in}{2.383188in}}%
\pgfpathlineto{\pgfqpoint{1.976253in}{2.407873in}}%
\pgfpathlineto{\pgfqpoint{1.976661in}{2.342047in}}%
\pgfpathlineto{\pgfqpoint{1.976865in}{2.358504in}}%
\pgfpathlineto{\pgfqpoint{1.977884in}{2.416101in}}%
\pgfpathlineto{\pgfqpoint{1.978088in}{2.399645in}}%
\pgfpathlineto{\pgfqpoint{1.979514in}{2.111657in}}%
\pgfpathlineto{\pgfqpoint{1.980737in}{2.226852in}}%
\pgfpathlineto{\pgfqpoint{1.982164in}{1.947093in}}%
\pgfpathlineto{\pgfqpoint{1.982571in}{2.054060in}}%
\pgfpathlineto{\pgfqpoint{1.983183in}{1.955322in}}%
\pgfpathlineto{\pgfqpoint{1.984202in}{2.012919in}}%
\pgfpathlineto{\pgfqpoint{1.985424in}{1.864811in}}%
\pgfpathlineto{\pgfqpoint{1.985628in}{1.881268in}}%
\pgfpathlineto{\pgfqpoint{1.985832in}{1.881268in}}%
\pgfpathlineto{\pgfqpoint{1.986240in}{1.675563in}}%
\pgfpathlineto{\pgfqpoint{1.986647in}{1.980006in}}%
\pgfpathlineto{\pgfqpoint{1.987055in}{2.029375in}}%
\pgfpathlineto{\pgfqpoint{1.987666in}{1.963550in}}%
\pgfpathlineto{\pgfqpoint{1.988889in}{2.144570in}}%
\pgfpathlineto{\pgfqpoint{1.989297in}{2.070516in}}%
\pgfpathlineto{\pgfqpoint{1.989501in}{1.996463in}}%
\pgfpathlineto{\pgfqpoint{1.990112in}{2.136342in}}%
\pgfpathlineto{\pgfqpoint{1.991335in}{2.070516in}}%
\pgfpathlineto{\pgfqpoint{1.991742in}{2.095201in}}%
\pgfpathlineto{\pgfqpoint{1.992761in}{2.136342in}}%
\pgfpathlineto{\pgfqpoint{1.994188in}{1.996463in}}%
\pgfpathlineto{\pgfqpoint{1.994799in}{2.045832in}}%
\pgfpathlineto{\pgfqpoint{1.995003in}{1.963550in}}%
\pgfpathlineto{\pgfqpoint{1.995207in}{1.881268in}}%
\pgfpathlineto{\pgfqpoint{1.995818in}{2.070516in}}%
\pgfpathlineto{\pgfqpoint{1.996226in}{2.004691in}}%
\pgfpathlineto{\pgfqpoint{1.996634in}{2.144570in}}%
\pgfpathlineto{\pgfqpoint{1.997653in}{1.980006in}}%
\pgfpathlineto{\pgfqpoint{1.998060in}{1.988234in}}%
\pgfpathlineto{\pgfqpoint{1.999487in}{1.840127in}}%
\pgfpathlineto{\pgfqpoint{2.000506in}{1.922409in}}%
\pgfpathlineto{\pgfqpoint{2.000710in}{1.864811in}}%
\pgfpathlineto{\pgfqpoint{2.000913in}{1.634422in}}%
\pgfpathlineto{\pgfqpoint{2.001729in}{1.955322in}}%
\pgfpathlineto{\pgfqpoint{2.001933in}{1.914181in}}%
\pgfpathlineto{\pgfqpoint{2.002544in}{1.988234in}}%
\pgfpathlineto{\pgfqpoint{2.002748in}{1.955322in}}%
\pgfpathlineto{\pgfqpoint{2.002952in}{1.947093in}}%
\pgfpathlineto{\pgfqpoint{2.003359in}{1.971778in}}%
\pgfpathlineto{\pgfqpoint{2.003971in}{2.045832in}}%
\pgfpathlineto{\pgfqpoint{2.004174in}{1.996463in}}%
\pgfpathlineto{\pgfqpoint{2.005397in}{1.757845in}}%
\pgfpathlineto{\pgfqpoint{2.006416in}{1.988234in}}%
\pgfpathlineto{\pgfqpoint{2.006620in}{1.938865in}}%
\pgfpathlineto{\pgfqpoint{2.007028in}{1.996463in}}%
\pgfpathlineto{\pgfqpoint{2.007435in}{1.905952in}}%
\pgfpathlineto{\pgfqpoint{2.007639in}{1.938865in}}%
\pgfpathlineto{\pgfqpoint{2.008047in}{1.881268in}}%
\pgfpathlineto{\pgfqpoint{2.008658in}{1.922409in}}%
\pgfpathlineto{\pgfqpoint{2.009269in}{1.905952in}}%
\pgfpathlineto{\pgfqpoint{2.010288in}{2.012919in}}%
\pgfpathlineto{\pgfqpoint{2.010492in}{1.905952in}}%
\pgfpathlineto{\pgfqpoint{2.011307in}{1.980006in}}%
\pgfpathlineto{\pgfqpoint{2.012530in}{1.914181in}}%
\pgfpathlineto{\pgfqpoint{2.011919in}{1.996463in}}%
\pgfpathlineto{\pgfqpoint{2.012734in}{1.938865in}}%
\pgfpathlineto{\pgfqpoint{2.012938in}{1.947093in}}%
\pgfpathlineto{\pgfqpoint{2.013142in}{1.905952in}}%
\pgfpathlineto{\pgfqpoint{2.013549in}{1.980006in}}%
\pgfpathlineto{\pgfqpoint{2.013957in}{1.947093in}}%
\pgfpathlineto{\pgfqpoint{2.014976in}{1.897724in}}%
\pgfpathlineto{\pgfqpoint{2.015995in}{1.971778in}}%
\pgfpathlineto{\pgfqpoint{2.016199in}{1.947093in}}%
\pgfpathlineto{\pgfqpoint{2.017014in}{1.889496in}}%
\pgfpathlineto{\pgfqpoint{2.016606in}{1.996463in}}%
\pgfpathlineto{\pgfqpoint{2.017218in}{1.930637in}}%
\pgfpathlineto{\pgfqpoint{2.018033in}{2.045832in}}%
\pgfpathlineto{\pgfqpoint{2.018237in}{1.963550in}}%
\pgfpathlineto{\pgfqpoint{2.019052in}{1.864811in}}%
\pgfpathlineto{\pgfqpoint{2.019460in}{1.914181in}}%
\pgfpathlineto{\pgfqpoint{2.020275in}{2.037604in}}%
\pgfpathlineto{\pgfqpoint{2.020886in}{1.996463in}}%
\pgfpathlineto{\pgfqpoint{2.022313in}{1.798986in}}%
\pgfpathlineto{\pgfqpoint{2.023332in}{1.881268in}}%
\pgfpathlineto{\pgfqpoint{2.023536in}{1.831899in}}%
\pgfpathlineto{\pgfqpoint{2.024147in}{1.741388in}}%
\pgfpathlineto{\pgfqpoint{2.024962in}{1.757845in}}%
\pgfpathlineto{\pgfqpoint{2.027204in}{1.988234in}}%
\pgfpathlineto{\pgfqpoint{2.027408in}{1.980006in}}%
\pgfpathlineto{\pgfqpoint{2.028223in}{1.996463in}}%
\pgfpathlineto{\pgfqpoint{2.027815in}{1.971778in}}%
\pgfpathlineto{\pgfqpoint{2.028427in}{1.980006in}}%
\pgfpathlineto{\pgfqpoint{2.028631in}{1.980006in}}%
\pgfpathlineto{\pgfqpoint{2.028835in}{1.971778in}}%
\pgfpathlineto{\pgfqpoint{2.029038in}{2.004691in}}%
\pgfpathlineto{\pgfqpoint{2.029242in}{2.004691in}}%
\pgfpathlineto{\pgfqpoint{2.030261in}{1.905952in}}%
\pgfpathlineto{\pgfqpoint{2.030465in}{1.914181in}}%
\pgfpathlineto{\pgfqpoint{2.031280in}{1.988234in}}%
\pgfpathlineto{\pgfqpoint{2.031688in}{1.971778in}}%
\pgfpathlineto{\pgfqpoint{2.032911in}{1.889496in}}%
\pgfpathlineto{\pgfqpoint{2.033114in}{1.914181in}}%
\pgfpathlineto{\pgfqpoint{2.033318in}{1.988234in}}%
\pgfpathlineto{\pgfqpoint{2.033930in}{1.831899in}}%
\pgfpathlineto{\pgfqpoint{2.034133in}{1.955322in}}%
\pgfpathlineto{\pgfqpoint{2.034541in}{1.634422in}}%
\pgfpathlineto{\pgfqpoint{2.035356in}{1.840127in}}%
\pgfpathlineto{\pgfqpoint{2.036375in}{1.683791in}}%
\pgfpathlineto{\pgfqpoint{2.036783in}{1.692019in}}%
\pgfpathlineto{\pgfqpoint{2.037190in}{1.757845in}}%
\pgfpathlineto{\pgfqpoint{2.038006in}{1.749617in}}%
\pgfpathlineto{\pgfqpoint{2.038209in}{1.708476in}}%
\pgfpathlineto{\pgfqpoint{2.038617in}{1.815442in}}%
\pgfpathlineto{\pgfqpoint{2.039025in}{1.757845in}}%
\pgfpathlineto{\pgfqpoint{2.039432in}{1.733160in}}%
\pgfpathlineto{\pgfqpoint{2.039840in}{1.815442in}}%
\pgfpathlineto{\pgfqpoint{2.040655in}{1.692019in}}%
\pgfpathlineto{\pgfqpoint{2.041063in}{1.749617in}}%
\pgfpathlineto{\pgfqpoint{2.041878in}{1.724932in}}%
\pgfpathlineto{\pgfqpoint{2.042082in}{1.733160in}}%
\pgfpathlineto{\pgfqpoint{2.043101in}{1.790758in}}%
\pgfpathlineto{\pgfqpoint{2.043305in}{1.609737in}}%
\pgfpathlineto{\pgfqpoint{2.044120in}{1.856583in}}%
\pgfpathlineto{\pgfqpoint{2.044324in}{1.848355in}}%
\pgfpathlineto{\pgfqpoint{2.044527in}{1.864811in}}%
\pgfpathlineto{\pgfqpoint{2.044935in}{1.938865in}}%
\pgfpathlineto{\pgfqpoint{2.045343in}{1.823670in}}%
\pgfpathlineto{\pgfqpoint{2.045546in}{1.905952in}}%
\pgfpathlineto{\pgfqpoint{2.046973in}{1.815442in}}%
\pgfpathlineto{\pgfqpoint{2.047381in}{1.914181in}}%
\pgfpathlineto{\pgfqpoint{2.047584in}{1.790758in}}%
\pgfpathlineto{\pgfqpoint{2.047788in}{1.856583in}}%
\pgfpathlineto{\pgfqpoint{2.049419in}{1.576824in}}%
\pgfpathlineto{\pgfqpoint{2.049826in}{1.782529in}}%
\pgfpathlineto{\pgfqpoint{2.050641in}{1.741388in}}%
\pgfpathlineto{\pgfqpoint{2.051457in}{1.848355in}}%
\pgfpathlineto{\pgfqpoint{2.052068in}{1.840127in}}%
\pgfpathlineto{\pgfqpoint{2.053087in}{1.733160in}}%
\pgfpathlineto{\pgfqpoint{2.053291in}{1.749617in}}%
\pgfpathlineto{\pgfqpoint{2.053495in}{1.741388in}}%
\pgfpathlineto{\pgfqpoint{2.053902in}{1.823670in}}%
\pgfpathlineto{\pgfqpoint{2.054514in}{1.724932in}}%
\pgfpathlineto{\pgfqpoint{2.054921in}{1.831899in}}%
\pgfpathlineto{\pgfqpoint{2.055329in}{1.823670in}}%
\pgfpathlineto{\pgfqpoint{2.055533in}{1.642650in}}%
\pgfpathlineto{\pgfqpoint{2.055737in}{1.864811in}}%
\pgfpathlineto{\pgfqpoint{2.056348in}{1.766073in}}%
\pgfpathlineto{\pgfqpoint{2.056756in}{1.889496in}}%
\pgfpathlineto{\pgfqpoint{2.057367in}{1.782529in}}%
\pgfpathlineto{\pgfqpoint{2.057571in}{1.766073in}}%
\pgfpathlineto{\pgfqpoint{2.057775in}{1.774301in}}%
\pgfpathlineto{\pgfqpoint{2.057978in}{1.831899in}}%
\pgfpathlineto{\pgfqpoint{2.058182in}{1.766073in}}%
\pgfpathlineto{\pgfqpoint{2.058794in}{1.790758in}}%
\pgfpathlineto{\pgfqpoint{2.058997in}{1.766073in}}%
\pgfpathlineto{\pgfqpoint{2.059201in}{1.856583in}}%
\pgfpathlineto{\pgfqpoint{2.059405in}{1.897724in}}%
\pgfpathlineto{\pgfqpoint{2.059813in}{1.840127in}}%
\pgfpathlineto{\pgfqpoint{2.060016in}{1.848355in}}%
\pgfpathlineto{\pgfqpoint{2.060220in}{1.617965in}}%
\pgfpathlineto{\pgfqpoint{2.060832in}{1.881268in}}%
\pgfpathlineto{\pgfqpoint{2.061035in}{1.873040in}}%
\pgfpathlineto{\pgfqpoint{2.061239in}{1.897724in}}%
\pgfpathlineto{\pgfqpoint{2.061443in}{1.823670in}}%
\pgfpathlineto{\pgfqpoint{2.061851in}{1.856583in}}%
\pgfpathlineto{\pgfqpoint{2.063073in}{1.692019in}}%
\pgfpathlineto{\pgfqpoint{2.063481in}{1.782529in}}%
\pgfpathlineto{\pgfqpoint{2.063685in}{1.881268in}}%
\pgfpathlineto{\pgfqpoint{2.063889in}{1.667335in}}%
\pgfpathlineto{\pgfqpoint{2.064500in}{1.741388in}}%
\pgfpathlineto{\pgfqpoint{2.065111in}{1.708476in}}%
\pgfpathlineto{\pgfqpoint{2.064908in}{1.757845in}}%
\pgfpathlineto{\pgfqpoint{2.065519in}{1.741388in}}%
\pgfpathlineto{\pgfqpoint{2.065927in}{1.774301in}}%
\pgfpathlineto{\pgfqpoint{2.066538in}{1.757845in}}%
\pgfpathlineto{\pgfqpoint{2.066742in}{1.700247in}}%
\pgfpathlineto{\pgfqpoint{2.066946in}{1.782529in}}%
\pgfpathlineto{\pgfqpoint{2.067761in}{1.716704in}}%
\pgfpathlineto{\pgfqpoint{2.068576in}{1.831899in}}%
\pgfpathlineto{\pgfqpoint{2.069799in}{1.601509in}}%
\pgfpathlineto{\pgfqpoint{2.071226in}{1.757845in}}%
\pgfpathlineto{\pgfqpoint{2.072245in}{1.659106in}}%
\pgfpathlineto{\pgfqpoint{2.072448in}{1.692019in}}%
\pgfpathlineto{\pgfqpoint{2.072856in}{1.724932in}}%
\pgfpathlineto{\pgfqpoint{2.073264in}{1.667335in}}%
\pgfpathlineto{\pgfqpoint{2.073467in}{1.642650in}}%
\pgfpathlineto{\pgfqpoint{2.074283in}{1.667335in}}%
\pgfpathlineto{\pgfqpoint{2.075709in}{1.790758in}}%
\pgfpathlineto{\pgfqpoint{2.075913in}{1.774301in}}%
\pgfpathlineto{\pgfqpoint{2.076117in}{1.741388in}}%
\pgfpathlineto{\pgfqpoint{2.076321in}{1.798986in}}%
\pgfpathlineto{\pgfqpoint{2.076728in}{1.757845in}}%
\pgfpathlineto{\pgfqpoint{2.077543in}{1.905952in}}%
\pgfpathlineto{\pgfqpoint{2.077747in}{1.864811in}}%
\pgfpathlineto{\pgfqpoint{2.078970in}{1.724932in}}%
\pgfpathlineto{\pgfqpoint{2.079989in}{1.807214in}}%
\pgfpathlineto{\pgfqpoint{2.080193in}{1.766073in}}%
\pgfpathlineto{\pgfqpoint{2.080804in}{1.823670in}}%
\pgfpathlineto{\pgfqpoint{2.081212in}{1.766073in}}%
\pgfpathlineto{\pgfqpoint{2.081619in}{1.667335in}}%
\pgfpathlineto{\pgfqpoint{2.082027in}{1.774301in}}%
\pgfpathlineto{\pgfqpoint{2.082231in}{1.790758in}}%
\pgfpathlineto{\pgfqpoint{2.082435in}{1.766073in}}%
\pgfpathlineto{\pgfqpoint{2.083046in}{1.469858in}}%
\pgfpathlineto{\pgfqpoint{2.083454in}{1.741388in}}%
\pgfpathlineto{\pgfqpoint{2.083658in}{1.757845in}}%
\pgfpathlineto{\pgfqpoint{2.084880in}{1.593281in}}%
\pgfpathlineto{\pgfqpoint{2.085899in}{1.667335in}}%
\pgfpathlineto{\pgfqpoint{2.086511in}{1.716704in}}%
\pgfpathlineto{\pgfqpoint{2.086715in}{1.659106in}}%
\pgfpathlineto{\pgfqpoint{2.087530in}{1.527455in}}%
\pgfpathlineto{\pgfqpoint{2.087734in}{1.659106in}}%
\pgfpathlineto{\pgfqpoint{2.088141in}{1.634422in}}%
\pgfpathlineto{\pgfqpoint{2.088345in}{1.733160in}}%
\pgfpathlineto{\pgfqpoint{2.089160in}{1.626194in}}%
\pgfpathlineto{\pgfqpoint{2.089975in}{1.700247in}}%
\pgfpathlineto{\pgfqpoint{2.090179in}{1.650878in}}%
\pgfpathlineto{\pgfqpoint{2.090791in}{1.535683in}}%
\pgfpathlineto{\pgfqpoint{2.091606in}{1.568596in}}%
\pgfpathlineto{\pgfqpoint{2.092421in}{1.650878in}}%
\pgfpathlineto{\pgfqpoint{2.092829in}{1.601509in}}%
\pgfpathlineto{\pgfqpoint{2.093236in}{1.609737in}}%
\pgfpathlineto{\pgfqpoint{2.093848in}{1.560368in}}%
\pgfpathlineto{\pgfqpoint{2.094051in}{1.585053in}}%
\pgfpathlineto{\pgfqpoint{2.094255in}{1.510999in}}%
\pgfpathlineto{\pgfqpoint{2.094867in}{1.568596in}}%
\pgfpathlineto{\pgfqpoint{2.095070in}{1.568596in}}%
\pgfpathlineto{\pgfqpoint{2.095274in}{1.560368in}}%
\pgfpathlineto{\pgfqpoint{2.095478in}{1.593281in}}%
\pgfpathlineto{\pgfqpoint{2.095682in}{1.576824in}}%
\pgfpathlineto{\pgfqpoint{2.097312in}{1.700247in}}%
\pgfpathlineto{\pgfqpoint{2.097516in}{1.617965in}}%
\pgfpathlineto{\pgfqpoint{2.098331in}{1.733160in}}%
\pgfpathlineto{\pgfqpoint{2.099758in}{1.445173in}}%
\pgfpathlineto{\pgfqpoint{2.099962in}{1.560368in}}%
\pgfpathlineto{\pgfqpoint{2.101185in}{1.675563in}}%
\pgfpathlineto{\pgfqpoint{2.102407in}{1.593281in}}%
\pgfpathlineto{\pgfqpoint{2.102611in}{1.642650in}}%
\pgfpathlineto{\pgfqpoint{2.102815in}{1.494542in}}%
\pgfpathlineto{\pgfqpoint{2.103630in}{1.626194in}}%
\pgfpathlineto{\pgfqpoint{2.104242in}{1.371119in}}%
\pgfpathlineto{\pgfqpoint{2.104038in}{1.650878in}}%
\pgfpathlineto{\pgfqpoint{2.104649in}{1.585053in}}%
\pgfpathlineto{\pgfqpoint{2.104853in}{1.609737in}}%
\pgfpathlineto{\pgfqpoint{2.105261in}{1.552140in}}%
\pgfpathlineto{\pgfqpoint{2.105668in}{1.593281in}}%
\pgfpathlineto{\pgfqpoint{2.106280in}{1.552140in}}%
\pgfpathlineto{\pgfqpoint{2.106483in}{1.634422in}}%
\pgfpathlineto{\pgfqpoint{2.106687in}{1.601509in}}%
\pgfpathlineto{\pgfqpoint{2.107502in}{1.552140in}}%
\pgfpathlineto{\pgfqpoint{2.107095in}{1.617965in}}%
\pgfpathlineto{\pgfqpoint{2.107706in}{1.576824in}}%
\pgfpathlineto{\pgfqpoint{2.107910in}{1.634422in}}%
\pgfpathlineto{\pgfqpoint{2.108318in}{1.552140in}}%
\pgfpathlineto{\pgfqpoint{2.108725in}{1.601509in}}%
\pgfpathlineto{\pgfqpoint{2.108929in}{1.593281in}}%
\pgfpathlineto{\pgfqpoint{2.109337in}{1.609737in}}%
\pgfpathlineto{\pgfqpoint{2.109948in}{1.634422in}}%
\pgfpathlineto{\pgfqpoint{2.110152in}{1.585053in}}%
\pgfpathlineto{\pgfqpoint{2.110560in}{1.626194in}}%
\pgfpathlineto{\pgfqpoint{2.110763in}{1.659106in}}%
\pgfpathlineto{\pgfqpoint{2.111171in}{1.576824in}}%
\pgfpathlineto{\pgfqpoint{2.111375in}{1.585053in}}%
\pgfpathlineto{\pgfqpoint{2.112394in}{1.510999in}}%
\pgfpathlineto{\pgfqpoint{2.111986in}{1.609737in}}%
\pgfpathlineto{\pgfqpoint{2.112598in}{1.543912in}}%
\pgfpathlineto{\pgfqpoint{2.113005in}{1.510999in}}%
\pgfpathlineto{\pgfqpoint{2.113209in}{1.519227in}}%
\pgfpathlineto{\pgfqpoint{2.114228in}{1.510999in}}%
\pgfpathlineto{\pgfqpoint{2.114432in}{1.585053in}}%
\pgfpathlineto{\pgfqpoint{2.115043in}{1.486314in}}%
\pgfpathlineto{\pgfqpoint{2.115451in}{1.576824in}}%
\pgfpathlineto{\pgfqpoint{2.115858in}{1.568596in}}%
\pgfpathlineto{\pgfqpoint{2.116062in}{1.617965in}}%
\pgfpathlineto{\pgfqpoint{2.116470in}{1.510999in}}%
\pgfpathlineto{\pgfqpoint{2.117081in}{1.428717in}}%
\pgfpathlineto{\pgfqpoint{2.117489in}{1.486314in}}%
\pgfpathlineto{\pgfqpoint{2.118304in}{1.543912in}}%
\pgfpathlineto{\pgfqpoint{2.117896in}{1.453401in}}%
\pgfpathlineto{\pgfqpoint{2.118712in}{1.502771in}}%
\pgfpathlineto{\pgfqpoint{2.118915in}{1.494542in}}%
\pgfpathlineto{\pgfqpoint{2.119323in}{1.552140in}}%
\pgfpathlineto{\pgfqpoint{2.119731in}{1.445173in}}%
\pgfpathlineto{\pgfqpoint{2.119934in}{1.486314in}}%
\pgfpathlineto{\pgfqpoint{2.120138in}{1.469858in}}%
\pgfpathlineto{\pgfqpoint{2.120342in}{1.502771in}}%
\pgfpathlineto{\pgfqpoint{2.121769in}{1.650878in}}%
\pgfpathlineto{\pgfqpoint{2.123195in}{1.494542in}}%
\pgfpathlineto{\pgfqpoint{2.124622in}{1.659106in}}%
\pgfpathlineto{\pgfqpoint{2.125233in}{1.650878in}}%
\pgfpathlineto{\pgfqpoint{2.126049in}{1.667335in}}%
\pgfpathlineto{\pgfqpoint{2.126252in}{1.626194in}}%
\pgfpathlineto{\pgfqpoint{2.126864in}{1.733160in}}%
\pgfpathlineto{\pgfqpoint{2.127271in}{1.650878in}}%
\pgfpathlineto{\pgfqpoint{2.127475in}{1.642650in}}%
\pgfpathlineto{\pgfqpoint{2.127883in}{1.716704in}}%
\pgfpathlineto{\pgfqpoint{2.128087in}{1.634422in}}%
\pgfpathlineto{\pgfqpoint{2.128494in}{1.692019in}}%
\pgfpathlineto{\pgfqpoint{2.128902in}{1.527455in}}%
\pgfpathlineto{\pgfqpoint{2.129921in}{1.552140in}}%
\pgfpathlineto{\pgfqpoint{2.130125in}{1.626194in}}%
\pgfpathlineto{\pgfqpoint{2.130940in}{1.552140in}}%
\pgfpathlineto{\pgfqpoint{2.131551in}{1.576824in}}%
\pgfpathlineto{\pgfqpoint{2.131347in}{1.543912in}}%
\pgfpathlineto{\pgfqpoint{2.131755in}{1.552140in}}%
\pgfpathlineto{\pgfqpoint{2.131959in}{1.510999in}}%
\pgfpathlineto{\pgfqpoint{2.132366in}{1.585053in}}%
\pgfpathlineto{\pgfqpoint{2.132570in}{1.617965in}}%
\pgfpathlineto{\pgfqpoint{2.132774in}{1.502771in}}%
\pgfpathlineto{\pgfqpoint{2.132978in}{1.519227in}}%
\pgfpathlineto{\pgfqpoint{2.133182in}{1.519227in}}%
\pgfpathlineto{\pgfqpoint{2.133385in}{1.510999in}}%
\pgfpathlineto{\pgfqpoint{2.133589in}{1.527455in}}%
\pgfpathlineto{\pgfqpoint{2.133793in}{1.527455in}}%
\pgfpathlineto{\pgfqpoint{2.134201in}{1.560368in}}%
\pgfpathlineto{\pgfqpoint{2.134404in}{1.527455in}}%
\pgfpathlineto{\pgfqpoint{2.135016in}{1.469858in}}%
\pgfpathlineto{\pgfqpoint{2.135220in}{1.543912in}}%
\pgfpathlineto{\pgfqpoint{2.135423in}{1.617965in}}%
\pgfpathlineto{\pgfqpoint{2.136239in}{1.527455in}}%
\pgfpathlineto{\pgfqpoint{2.137054in}{1.543912in}}%
\pgfpathlineto{\pgfqpoint{2.137665in}{1.420489in}}%
\pgfpathlineto{\pgfqpoint{2.138888in}{1.502771in}}%
\pgfpathlineto{\pgfqpoint{2.139092in}{1.436945in}}%
\pgfpathlineto{\pgfqpoint{2.139500in}{1.543912in}}%
\pgfpathlineto{\pgfqpoint{2.139907in}{1.494542in}}%
\pgfpathlineto{\pgfqpoint{2.140111in}{1.543912in}}%
\pgfpathlineto{\pgfqpoint{2.140315in}{1.321750in}}%
\pgfpathlineto{\pgfqpoint{2.140519in}{1.568596in}}%
\pgfpathlineto{\pgfqpoint{2.141130in}{1.486314in}}%
\pgfpathlineto{\pgfqpoint{2.141334in}{1.478086in}}%
\pgfpathlineto{\pgfqpoint{2.141538in}{1.527455in}}%
\pgfpathlineto{\pgfqpoint{2.141945in}{1.461630in}}%
\pgfpathlineto{\pgfqpoint{2.142149in}{1.486314in}}%
\pgfpathlineto{\pgfqpoint{2.142557in}{1.354663in}}%
\pgfpathlineto{\pgfqpoint{2.143168in}{1.387576in}}%
\pgfpathlineto{\pgfqpoint{2.144187in}{1.519227in}}%
\pgfpathlineto{\pgfqpoint{2.143576in}{1.362891in}}%
\pgfpathlineto{\pgfqpoint{2.144391in}{1.510999in}}%
\pgfpathlineto{\pgfqpoint{2.144798in}{1.165414in}}%
\pgfpathlineto{\pgfqpoint{2.145410in}{1.214784in}}%
\pgfpathlineto{\pgfqpoint{2.146429in}{1.478086in}}%
\pgfpathlineto{\pgfqpoint{2.146633in}{1.412260in}}%
\pgfpathlineto{\pgfqpoint{2.147652in}{1.510999in}}%
\pgfpathlineto{\pgfqpoint{2.148059in}{1.486314in}}%
\pgfpathlineto{\pgfqpoint{2.148874in}{1.453401in}}%
\pgfpathlineto{\pgfqpoint{2.149078in}{1.510999in}}%
\pgfpathlineto{\pgfqpoint{2.149893in}{1.494542in}}%
\pgfpathlineto{\pgfqpoint{2.150709in}{1.404032in}}%
\pgfpathlineto{\pgfqpoint{2.150913in}{1.469858in}}%
\pgfpathlineto{\pgfqpoint{2.151320in}{1.568596in}}%
\pgfpathlineto{\pgfqpoint{2.151932in}{1.469858in}}%
\pgfpathlineto{\pgfqpoint{2.152135in}{1.239468in}}%
\pgfpathlineto{\pgfqpoint{2.152747in}{1.568596in}}%
\pgfpathlineto{\pgfqpoint{2.152951in}{1.527455in}}%
\pgfpathlineto{\pgfqpoint{2.153154in}{1.527455in}}%
\pgfpathlineto{\pgfqpoint{2.153358in}{1.510999in}}%
\pgfpathlineto{\pgfqpoint{2.153562in}{1.560368in}}%
\pgfpathlineto{\pgfqpoint{2.153766in}{1.535683in}}%
\pgfpathlineto{\pgfqpoint{2.154785in}{1.650878in}}%
\pgfpathlineto{\pgfqpoint{2.155192in}{1.609737in}}%
\pgfpathlineto{\pgfqpoint{2.156211in}{1.527455in}}%
\pgfpathlineto{\pgfqpoint{2.156415in}{1.552140in}}%
\pgfpathlineto{\pgfqpoint{2.156619in}{1.560368in}}%
\pgfpathlineto{\pgfqpoint{2.157638in}{1.774301in}}%
\pgfpathlineto{\pgfqpoint{2.157842in}{1.700247in}}%
\pgfpathlineto{\pgfqpoint{2.158046in}{1.445173in}}%
\pgfpathlineto{\pgfqpoint{2.159065in}{1.560368in}}%
\pgfpathlineto{\pgfqpoint{2.159880in}{1.576824in}}%
\pgfpathlineto{\pgfqpoint{2.160287in}{1.478086in}}%
\pgfpathlineto{\pgfqpoint{2.160491in}{1.486314in}}%
\pgfpathlineto{\pgfqpoint{2.160695in}{1.552140in}}%
\pgfpathlineto{\pgfqpoint{2.160899in}{1.469858in}}%
\pgfpathlineto{\pgfqpoint{2.161510in}{1.527455in}}%
\pgfpathlineto{\pgfqpoint{2.162325in}{1.428717in}}%
\pgfpathlineto{\pgfqpoint{2.162733in}{1.469858in}}%
\pgfpathlineto{\pgfqpoint{2.163141in}{1.510999in}}%
\pgfpathlineto{\pgfqpoint{2.163344in}{1.461630in}}%
\pgfpathlineto{\pgfqpoint{2.163752in}{1.568596in}}%
\pgfpathlineto{\pgfqpoint{2.164364in}{1.478086in}}%
\pgfpathlineto{\pgfqpoint{2.165179in}{1.379348in}}%
\pgfpathlineto{\pgfqpoint{2.165586in}{1.412260in}}%
\pgfpathlineto{\pgfqpoint{2.166198in}{1.461630in}}%
\pgfpathlineto{\pgfqpoint{2.167421in}{1.321750in}}%
\pgfpathlineto{\pgfqpoint{2.168440in}{1.395804in}}%
\pgfpathlineto{\pgfqpoint{2.167828in}{1.288837in}}%
\pgfpathlineto{\pgfqpoint{2.168847in}{1.387576in}}%
\pgfpathlineto{\pgfqpoint{2.169459in}{1.173643in}}%
\pgfpathlineto{\pgfqpoint{2.169662in}{1.395804in}}%
\pgfpathlineto{\pgfqpoint{2.170681in}{1.543912in}}%
\pgfpathlineto{\pgfqpoint{2.171089in}{1.502771in}}%
\pgfpathlineto{\pgfqpoint{2.171293in}{1.288837in}}%
\pgfpathlineto{\pgfqpoint{2.171904in}{1.568596in}}%
\pgfpathlineto{\pgfqpoint{2.172108in}{1.543912in}}%
\pgfpathlineto{\pgfqpoint{2.173127in}{1.478086in}}%
\pgfpathlineto{\pgfqpoint{2.173331in}{1.486314in}}%
\pgfpathlineto{\pgfqpoint{2.173535in}{1.519227in}}%
\pgfpathlineto{\pgfqpoint{2.174146in}{1.593281in}}%
\pgfpathlineto{\pgfqpoint{2.174554in}{1.329978in}}%
\pgfpathlineto{\pgfqpoint{2.174961in}{1.560368in}}%
\pgfpathlineto{\pgfqpoint{2.175776in}{1.478086in}}%
\pgfpathlineto{\pgfqpoint{2.176592in}{1.420489in}}%
\pgfpathlineto{\pgfqpoint{2.176795in}{1.478086in}}%
\pgfpathlineto{\pgfqpoint{2.177611in}{1.404032in}}%
\pgfpathlineto{\pgfqpoint{2.178222in}{1.132502in}}%
\pgfpathlineto{\pgfqpoint{2.178426in}{1.428717in}}%
\pgfpathlineto{\pgfqpoint{2.179241in}{1.494542in}}%
\pgfpathlineto{\pgfqpoint{2.179445in}{1.478086in}}%
\pgfpathlineto{\pgfqpoint{2.181687in}{1.140730in}}%
\pgfpathlineto{\pgfqpoint{2.181891in}{0.885656in}}%
\pgfpathlineto{\pgfqpoint{2.182298in}{1.157186in}}%
\pgfpathlineto{\pgfqpoint{2.182706in}{1.099589in}}%
\pgfpathlineto{\pgfqpoint{2.183521in}{1.173643in}}%
\pgfpathlineto{\pgfqpoint{2.183113in}{1.091361in}}%
\pgfpathlineto{\pgfqpoint{2.183725in}{1.157186in}}%
\pgfpathlineto{\pgfqpoint{2.184540in}{0.869199in}}%
\pgfpathlineto{\pgfqpoint{2.184744in}{1.041991in}}%
\pgfpathlineto{\pgfqpoint{2.184948in}{1.165414in}}%
\pgfpathlineto{\pgfqpoint{2.185763in}{1.041991in}}%
\pgfpathlineto{\pgfqpoint{2.186170in}{1.099589in}}%
\pgfpathlineto{\pgfqpoint{2.186578in}{1.017307in}}%
\pgfpathlineto{\pgfqpoint{2.186782in}{1.058448in}}%
\pgfpathlineto{\pgfqpoint{2.186986in}{0.754004in}}%
\pgfpathlineto{\pgfqpoint{2.187393in}{1.066676in}}%
\pgfpathlineto{\pgfqpoint{2.187801in}{0.836286in}}%
\pgfpathlineto{\pgfqpoint{2.188412in}{1.025535in}}%
\pgfpathlineto{\pgfqpoint{2.189024in}{0.943253in}}%
\pgfpathlineto{\pgfqpoint{2.189227in}{0.688179in}}%
\pgfpathlineto{\pgfqpoint{2.189635in}{1.009079in}}%
\pgfpathlineto{\pgfqpoint{2.190043in}{0.918568in}}%
\pgfpathlineto{\pgfqpoint{2.190246in}{0.852743in}}%
\pgfpathlineto{\pgfqpoint{2.190858in}{0.967938in}}%
\pgfpathlineto{\pgfqpoint{2.191062in}{1.058448in}}%
\pgfpathlineto{\pgfqpoint{2.191266in}{0.935025in}}%
\pgfpathlineto{\pgfqpoint{2.191469in}{0.721091in}}%
\pgfpathlineto{\pgfqpoint{2.192285in}{0.885656in}}%
\pgfpathlineto{\pgfqpoint{2.193711in}{0.992622in}}%
\pgfpathlineto{\pgfqpoint{2.192692in}{0.852743in}}%
\pgfpathlineto{\pgfqpoint{2.193915in}{0.951481in}}%
\pgfpathlineto{\pgfqpoint{2.194323in}{0.688179in}}%
\pgfpathlineto{\pgfqpoint{2.195138in}{0.860971in}}%
\pgfpathlineto{\pgfqpoint{2.196157in}{1.083132in}}%
\pgfpathlineto{\pgfqpoint{2.196564in}{1.009079in}}%
\pgfpathlineto{\pgfqpoint{2.196768in}{0.910340in}}%
\pgfpathlineto{\pgfqpoint{2.197583in}{1.000850in}}%
\pgfpathlineto{\pgfqpoint{2.199010in}{1.148958in}}%
\pgfpathlineto{\pgfqpoint{2.199418in}{1.181871in}}%
\pgfpathlineto{\pgfqpoint{2.199825in}{1.107817in}}%
\pgfpathlineto{\pgfqpoint{2.200029in}{1.165414in}}%
\pgfpathlineto{\pgfqpoint{2.200640in}{1.066676in}}%
\pgfpathlineto{\pgfqpoint{2.201048in}{1.000850in}}%
\pgfpathlineto{\pgfqpoint{2.201456in}{1.066676in}}%
\pgfpathlineto{\pgfqpoint{2.202067in}{1.148958in}}%
\pgfpathlineto{\pgfqpoint{2.202271in}{1.041991in}}%
\pgfpathlineto{\pgfqpoint{2.202678in}{1.058448in}}%
\pgfpathlineto{\pgfqpoint{2.203086in}{0.910340in}}%
\pgfpathlineto{\pgfqpoint{2.203290in}{0.696407in}}%
\pgfpathlineto{\pgfqpoint{2.203697in}{1.025535in}}%
\pgfpathlineto{\pgfqpoint{2.204105in}{0.926797in}}%
\pgfpathlineto{\pgfqpoint{2.204717in}{0.967938in}}%
\pgfpathlineto{\pgfqpoint{2.204920in}{0.951481in}}%
\pgfpathlineto{\pgfqpoint{2.205328in}{0.803374in}}%
\pgfpathlineto{\pgfqpoint{2.205939in}{0.893884in}}%
\pgfpathlineto{\pgfqpoint{2.207366in}{1.083132in}}%
\pgfpathlineto{\pgfqpoint{2.207570in}{1.050220in}}%
\pgfpathlineto{\pgfqpoint{2.207977in}{1.116045in}}%
\pgfpathlineto{\pgfqpoint{2.208181in}{1.074904in}}%
\pgfpathlineto{\pgfqpoint{2.208589in}{1.132502in}}%
\pgfpathlineto{\pgfqpoint{2.208996in}{1.009079in}}%
\pgfpathlineto{\pgfqpoint{2.209404in}{1.041991in}}%
\pgfpathlineto{\pgfqpoint{2.209812in}{1.017307in}}%
\pgfpathlineto{\pgfqpoint{2.210015in}{0.984394in}}%
\pgfpathlineto{\pgfqpoint{2.210627in}{1.058448in}}%
\pgfpathlineto{\pgfqpoint{2.210831in}{1.025535in}}%
\pgfpathlineto{\pgfqpoint{2.212257in}{1.198327in}}%
\pgfpathlineto{\pgfqpoint{2.212461in}{1.280609in}}%
\pgfpathlineto{\pgfqpoint{2.213072in}{1.157186in}}%
\pgfpathlineto{\pgfqpoint{2.213276in}{1.231240in}}%
\pgfpathlineto{\pgfqpoint{2.213480in}{1.181871in}}%
\pgfpathlineto{\pgfqpoint{2.214295in}{1.223012in}}%
\pgfpathlineto{\pgfqpoint{2.214499in}{1.239468in}}%
\pgfpathlineto{\pgfqpoint{2.214907in}{1.190099in}}%
\pgfpathlineto{\pgfqpoint{2.215110in}{1.190099in}}%
\pgfpathlineto{\pgfqpoint{2.215314in}{1.181871in}}%
\pgfpathlineto{\pgfqpoint{2.215722in}{1.305294in}}%
\pgfpathlineto{\pgfqpoint{2.216129in}{1.247696in}}%
\pgfpathlineto{\pgfqpoint{2.216333in}{1.148958in}}%
\pgfpathlineto{\pgfqpoint{2.216537in}{1.338207in}}%
\pgfpathlineto{\pgfqpoint{2.217148in}{1.288837in}}%
\pgfpathlineto{\pgfqpoint{2.217760in}{1.371119in}}%
\pgfpathlineto{\pgfqpoint{2.218168in}{1.206555in}}%
\pgfpathlineto{\pgfqpoint{2.218371in}{1.527455in}}%
\pgfpathlineto{\pgfqpoint{2.218779in}{1.223012in}}%
\pgfpathlineto{\pgfqpoint{2.218983in}{1.313522in}}%
\pgfpathlineto{\pgfqpoint{2.219594in}{1.173643in}}%
\pgfpathlineto{\pgfqpoint{2.219798in}{1.190099in}}%
\pgfpathlineto{\pgfqpoint{2.220409in}{1.074904in}}%
\pgfpathlineto{\pgfqpoint{2.221021in}{1.157186in}}%
\pgfpathlineto{\pgfqpoint{2.221836in}{1.066676in}}%
\pgfpathlineto{\pgfqpoint{2.222040in}{1.116045in}}%
\pgfpathlineto{\pgfqpoint{2.223466in}{1.223012in}}%
\pgfpathlineto{\pgfqpoint{2.223670in}{1.206555in}}%
\pgfpathlineto{\pgfqpoint{2.224485in}{1.264153in}}%
\pgfpathlineto{\pgfqpoint{2.224282in}{1.198327in}}%
\pgfpathlineto{\pgfqpoint{2.225097in}{1.247696in}}%
\pgfpathlineto{\pgfqpoint{2.225301in}{1.223012in}}%
\pgfpathlineto{\pgfqpoint{2.225708in}{1.280609in}}%
\pgfpathlineto{\pgfqpoint{2.225912in}{1.255925in}}%
\pgfpathlineto{\pgfqpoint{2.226116in}{1.346435in}}%
\pgfpathlineto{\pgfqpoint{2.226523in}{1.223012in}}%
\pgfpathlineto{\pgfqpoint{2.226931in}{1.272381in}}%
\pgfpathlineto{\pgfqpoint{2.228561in}{1.379348in}}%
\pgfpathlineto{\pgfqpoint{2.228969in}{1.329978in}}%
\pgfpathlineto{\pgfqpoint{2.229173in}{1.288837in}}%
\pgfpathlineto{\pgfqpoint{2.229580in}{1.412260in}}%
\pgfpathlineto{\pgfqpoint{2.230396in}{1.478086in}}%
\pgfpathlineto{\pgfqpoint{2.231211in}{1.552140in}}%
\pgfpathlineto{\pgfqpoint{2.231007in}{1.461630in}}%
\pgfpathlineto{\pgfqpoint{2.231415in}{1.543912in}}%
\pgfpathlineto{\pgfqpoint{2.232026in}{1.469858in}}%
\pgfpathlineto{\pgfqpoint{2.232434in}{1.519227in}}%
\pgfpathlineto{\pgfqpoint{2.233657in}{1.601509in}}%
\pgfpathlineto{\pgfqpoint{2.233860in}{1.560368in}}%
\pgfpathlineto{\pgfqpoint{2.234268in}{1.650878in}}%
\pgfpathlineto{\pgfqpoint{2.234472in}{1.626194in}}%
\pgfpathlineto{\pgfqpoint{2.234879in}{1.659106in}}%
\pgfpathlineto{\pgfqpoint{2.235083in}{1.478086in}}%
\pgfpathlineto{\pgfqpoint{2.235898in}{1.733160in}}%
\pgfpathlineto{\pgfqpoint{2.236102in}{1.708476in}}%
\pgfpathlineto{\pgfqpoint{2.236306in}{1.766073in}}%
\pgfpathlineto{\pgfqpoint{2.237325in}{1.864811in}}%
\pgfpathlineto{\pgfqpoint{2.237121in}{1.700247in}}%
\pgfpathlineto{\pgfqpoint{2.237529in}{1.848355in}}%
\pgfpathlineto{\pgfqpoint{2.238140in}{1.856583in}}%
\pgfpathlineto{\pgfqpoint{2.238548in}{1.798986in}}%
\pgfpathlineto{\pgfqpoint{2.239771in}{1.856583in}}%
\pgfpathlineto{\pgfqpoint{2.239974in}{1.848355in}}%
\pgfpathlineto{\pgfqpoint{2.240382in}{1.840127in}}%
\pgfpathlineto{\pgfqpoint{2.240993in}{1.889496in}}%
\pgfpathlineto{\pgfqpoint{2.241197in}{1.848355in}}%
\pgfpathlineto{\pgfqpoint{2.241605in}{1.914181in}}%
\pgfpathlineto{\pgfqpoint{2.242012in}{1.889496in}}%
\pgfpathlineto{\pgfqpoint{2.243847in}{2.037604in}}%
\pgfpathlineto{\pgfqpoint{2.244050in}{2.054060in}}%
\pgfpathlineto{\pgfqpoint{2.244866in}{2.062288in}}%
\pgfpathlineto{\pgfqpoint{2.245273in}{1.873040in}}%
\pgfpathlineto{\pgfqpoint{2.246089in}{2.062288in}}%
\pgfpathlineto{\pgfqpoint{2.246292in}{1.807214in}}%
\pgfpathlineto{\pgfqpoint{2.246700in}{2.111657in}}%
\pgfpathlineto{\pgfqpoint{2.247108in}{2.095201in}}%
\pgfpathlineto{\pgfqpoint{2.247515in}{2.045832in}}%
\pgfpathlineto{\pgfqpoint{2.247719in}{2.012919in}}%
\pgfpathlineto{\pgfqpoint{2.248127in}{2.054060in}}%
\pgfpathlineto{\pgfqpoint{2.248942in}{2.152798in}}%
\pgfpathlineto{\pgfqpoint{2.249349in}{2.086973in}}%
\pgfpathlineto{\pgfqpoint{2.250368in}{2.169255in}}%
\pgfpathlineto{\pgfqpoint{2.250980in}{2.152798in}}%
\pgfpathlineto{\pgfqpoint{2.251184in}{2.210396in}}%
\pgfpathlineto{\pgfqpoint{2.251999in}{2.128114in}}%
\pgfpathlineto{\pgfqpoint{2.252406in}{2.086973in}}%
\pgfpathlineto{\pgfqpoint{2.252814in}{2.161027in}}%
\pgfpathlineto{\pgfqpoint{2.253018in}{2.136342in}}%
\pgfpathlineto{\pgfqpoint{2.253833in}{2.202168in}}%
\pgfpathlineto{\pgfqpoint{2.253629in}{2.119886in}}%
\pgfpathlineto{\pgfqpoint{2.254444in}{2.177483in}}%
\pgfpathlineto{\pgfqpoint{2.255260in}{2.218624in}}%
\pgfpathlineto{\pgfqpoint{2.255667in}{2.103429in}}%
\pgfpathlineto{\pgfqpoint{2.255871in}{2.103429in}}%
\pgfpathlineto{\pgfqpoint{2.256075in}{1.922409in}}%
\pgfpathlineto{\pgfqpoint{2.256482in}{2.136342in}}%
\pgfpathlineto{\pgfqpoint{2.256890in}{2.086973in}}%
\pgfpathlineto{\pgfqpoint{2.257501in}{2.144570in}}%
\pgfpathlineto{\pgfqpoint{2.257705in}{2.054060in}}%
\pgfpathlineto{\pgfqpoint{2.257909in}{2.095201in}}%
\pgfpathlineto{\pgfqpoint{2.258724in}{2.152798in}}%
\pgfpathlineto{\pgfqpoint{2.258317in}{2.086973in}}%
\pgfpathlineto{\pgfqpoint{2.258928in}{2.144570in}}%
\pgfpathlineto{\pgfqpoint{2.259540in}{1.889496in}}%
\pgfpathlineto{\pgfqpoint{2.259947in}{2.095201in}}%
\pgfpathlineto{\pgfqpoint{2.260151in}{2.078745in}}%
\pgfpathlineto{\pgfqpoint{2.260355in}{2.136342in}}%
\pgfpathlineto{\pgfqpoint{2.260762in}{2.103429in}}%
\pgfpathlineto{\pgfqpoint{2.261170in}{2.128114in}}%
\pgfpathlineto{\pgfqpoint{2.261374in}{2.086973in}}%
\pgfpathlineto{\pgfqpoint{2.262393in}{2.004691in}}%
\pgfpathlineto{\pgfqpoint{2.262597in}{2.012919in}}%
\pgfpathlineto{\pgfqpoint{2.264431in}{2.103429in}}%
\pgfpathlineto{\pgfqpoint{2.265246in}{2.037604in}}%
\pgfpathlineto{\pgfqpoint{2.265654in}{2.045832in}}%
\pgfpathlineto{\pgfqpoint{2.265857in}{2.054060in}}%
\pgfpathlineto{\pgfqpoint{2.266061in}{2.029375in}}%
\pgfpathlineto{\pgfqpoint{2.266469in}{1.963550in}}%
\pgfpathlineto{\pgfqpoint{2.267080in}{2.029375in}}%
\pgfpathlineto{\pgfqpoint{2.267488in}{2.029375in}}%
\pgfpathlineto{\pgfqpoint{2.267895in}{2.111657in}}%
\pgfpathlineto{\pgfqpoint{2.268507in}{2.054060in}}%
\pgfpathlineto{\pgfqpoint{2.269118in}{2.021147in}}%
\pgfpathlineto{\pgfqpoint{2.269526in}{2.029375in}}%
\pgfpathlineto{\pgfqpoint{2.270749in}{2.095201in}}%
\pgfpathlineto{\pgfqpoint{2.270952in}{2.095201in}}%
\pgfpathlineto{\pgfqpoint{2.271156in}{2.111657in}}%
\pgfpathlineto{\pgfqpoint{2.271564in}{2.062288in}}%
\pgfpathlineto{\pgfqpoint{2.271768in}{2.054060in}}%
\pgfpathlineto{\pgfqpoint{2.271972in}{2.119886in}}%
\pgfpathlineto{\pgfqpoint{2.272583in}{2.045832in}}%
\pgfpathlineto{\pgfqpoint{2.272787in}{2.054060in}}%
\pgfpathlineto{\pgfqpoint{2.272991in}{2.062288in}}%
\pgfpathlineto{\pgfqpoint{2.273194in}{2.037604in}}%
\pgfpathlineto{\pgfqpoint{2.273398in}{2.029375in}}%
\pgfpathlineto{\pgfqpoint{2.273602in}{2.037604in}}%
\pgfpathlineto{\pgfqpoint{2.273806in}{2.086973in}}%
\pgfpathlineto{\pgfqpoint{2.274213in}{2.021147in}}%
\pgfpathlineto{\pgfqpoint{2.274417in}{2.021147in}}%
\pgfpathlineto{\pgfqpoint{2.274621in}{1.963550in}}%
\pgfpathlineto{\pgfqpoint{2.275640in}{1.988234in}}%
\pgfpathlineto{\pgfqpoint{2.275844in}{1.996463in}}%
\pgfpathlineto{\pgfqpoint{2.276251in}{1.889496in}}%
\pgfpathlineto{\pgfqpoint{2.276659in}{2.021147in}}%
\pgfpathlineto{\pgfqpoint{2.276863in}{1.980006in}}%
\pgfpathlineto{\pgfqpoint{2.277678in}{2.111657in}}%
\pgfpathlineto{\pgfqpoint{2.278086in}{2.086973in}}%
\pgfpathlineto{\pgfqpoint{2.278493in}{2.062288in}}%
\pgfpathlineto{\pgfqpoint{2.278901in}{2.095201in}}%
\pgfpathlineto{\pgfqpoint{2.279105in}{2.086973in}}%
\pgfpathlineto{\pgfqpoint{2.279308in}{2.095201in}}%
\pgfpathlineto{\pgfqpoint{2.279512in}{2.054060in}}%
\pgfpathlineto{\pgfqpoint{2.280124in}{2.144570in}}%
\pgfpathlineto{\pgfqpoint{2.280531in}{2.070516in}}%
\pgfpathlineto{\pgfqpoint{2.281143in}{2.037604in}}%
\pgfpathlineto{\pgfqpoint{2.281958in}{2.161027in}}%
\pgfpathlineto{\pgfqpoint{2.282365in}{2.119886in}}%
\pgfpathlineto{\pgfqpoint{2.282977in}{2.169255in}}%
\pgfpathlineto{\pgfqpoint{2.283384in}{2.062288in}}%
\pgfpathlineto{\pgfqpoint{2.284200in}{2.161027in}}%
\pgfpathlineto{\pgfqpoint{2.283996in}{2.012919in}}%
\pgfpathlineto{\pgfqpoint{2.284607in}{2.128114in}}%
\pgfpathlineto{\pgfqpoint{2.284811in}{2.136342in}}%
\pgfpathlineto{\pgfqpoint{2.286238in}{2.054060in}}%
\pgfpathlineto{\pgfqpoint{2.286442in}{2.078745in}}%
\pgfpathlineto{\pgfqpoint{2.286849in}{2.004691in}}%
\pgfpathlineto{\pgfqpoint{2.287257in}{2.070516in}}%
\pgfpathlineto{\pgfqpoint{2.287664in}{2.029375in}}%
\pgfpathlineto{\pgfqpoint{2.288072in}{2.078745in}}%
\pgfpathlineto{\pgfqpoint{2.288276in}{2.078745in}}%
\pgfpathlineto{\pgfqpoint{2.289906in}{2.235080in}}%
\pgfpathlineto{\pgfqpoint{2.290110in}{2.235080in}}%
\pgfpathlineto{\pgfqpoint{2.290721in}{2.004691in}}%
\pgfpathlineto{\pgfqpoint{2.291333in}{2.111657in}}%
\pgfpathlineto{\pgfqpoint{2.291537in}{1.996463in}}%
\pgfpathlineto{\pgfqpoint{2.291740in}{2.161027in}}%
\pgfpathlineto{\pgfqpoint{2.292352in}{2.136342in}}%
\pgfpathlineto{\pgfqpoint{2.293167in}{2.119886in}}%
\pgfpathlineto{\pgfqpoint{2.293575in}{2.177483in}}%
\pgfpathlineto{\pgfqpoint{2.293982in}{2.169255in}}%
\pgfpathlineto{\pgfqpoint{2.294390in}{1.873040in}}%
\pgfpathlineto{\pgfqpoint{2.294797in}{2.202168in}}%
\pgfpathlineto{\pgfqpoint{2.295001in}{2.177483in}}%
\pgfpathlineto{\pgfqpoint{2.295613in}{2.177483in}}%
\pgfpathlineto{\pgfqpoint{2.296020in}{2.136342in}}%
\pgfpathlineto{\pgfqpoint{2.296835in}{2.161027in}}%
\pgfpathlineto{\pgfqpoint{2.297243in}{2.136342in}}%
\pgfpathlineto{\pgfqpoint{2.298262in}{2.243309in}}%
\pgfpathlineto{\pgfqpoint{2.298874in}{2.037604in}}%
\pgfpathlineto{\pgfqpoint{2.298670in}{2.251537in}}%
\pgfpathlineto{\pgfqpoint{2.299485in}{2.144570in}}%
\pgfpathlineto{\pgfqpoint{2.300708in}{2.276221in}}%
\pgfpathlineto{\pgfqpoint{2.301319in}{2.202168in}}%
\pgfpathlineto{\pgfqpoint{2.301727in}{2.259765in}}%
\pgfpathlineto{\pgfqpoint{2.301931in}{2.259765in}}%
\pgfpathlineto{\pgfqpoint{2.302134in}{2.210396in}}%
\pgfpathlineto{\pgfqpoint{2.302746in}{2.292678in}}%
\pgfpathlineto{\pgfqpoint{2.302950in}{2.292678in}}%
\pgfpathlineto{\pgfqpoint{2.303357in}{2.350275in}}%
\pgfpathlineto{\pgfqpoint{2.303969in}{2.300906in}}%
\pgfpathlineto{\pgfqpoint{2.305599in}{2.012919in}}%
\pgfpathlineto{\pgfqpoint{2.305803in}{2.062288in}}%
\pgfpathlineto{\pgfqpoint{2.306210in}{2.267993in}}%
\pgfpathlineto{\pgfqpoint{2.307026in}{2.243309in}}%
\pgfpathlineto{\pgfqpoint{2.307229in}{2.193939in}}%
\pgfpathlineto{\pgfqpoint{2.307841in}{2.284450in}}%
\pgfpathlineto{\pgfqpoint{2.308656in}{2.251537in}}%
\pgfpathlineto{\pgfqpoint{2.308860in}{2.325591in}}%
\pgfpathlineto{\pgfqpoint{2.309471in}{2.193939in}}%
\pgfpathlineto{\pgfqpoint{2.309675in}{2.235080in}}%
\pgfpathlineto{\pgfqpoint{2.310490in}{2.169255in}}%
\pgfpathlineto{\pgfqpoint{2.310694in}{2.210396in}}%
\pgfpathlineto{\pgfqpoint{2.311305in}{2.193939in}}%
\pgfpathlineto{\pgfqpoint{2.312121in}{2.300906in}}%
\pgfpathlineto{\pgfqpoint{2.312732in}{1.988234in}}%
\pgfpathlineto{\pgfqpoint{2.312936in}{2.325591in}}%
\pgfpathlineto{\pgfqpoint{2.313140in}{2.284450in}}%
\pgfpathlineto{\pgfqpoint{2.313547in}{2.317363in}}%
\pgfpathlineto{\pgfqpoint{2.313751in}{2.276221in}}%
\pgfpathlineto{\pgfqpoint{2.313955in}{2.210396in}}%
\pgfpathlineto{\pgfqpoint{2.314566in}{2.284450in}}%
\pgfpathlineto{\pgfqpoint{2.314770in}{2.267993in}}%
\pgfpathlineto{\pgfqpoint{2.315382in}{2.309134in}}%
\pgfpathlineto{\pgfqpoint{2.316197in}{2.119886in}}%
\pgfpathlineto{\pgfqpoint{2.316604in}{2.185711in}}%
\pgfpathlineto{\pgfqpoint{2.316808in}{2.185711in}}%
\pgfpathlineto{\pgfqpoint{2.317012in}{2.218624in}}%
\pgfpathlineto{\pgfqpoint{2.317420in}{2.144570in}}%
\pgfpathlineto{\pgfqpoint{2.317623in}{2.177483in}}%
\pgfpathlineto{\pgfqpoint{2.317827in}{2.161027in}}%
\pgfpathlineto{\pgfqpoint{2.318031in}{2.210396in}}%
\pgfpathlineto{\pgfqpoint{2.318439in}{2.185711in}}%
\pgfpathlineto{\pgfqpoint{2.318642in}{2.210396in}}%
\pgfpathlineto{\pgfqpoint{2.319050in}{2.202168in}}%
\pgfpathlineto{\pgfqpoint{2.319254in}{1.996463in}}%
\pgfpathlineto{\pgfqpoint{2.320069in}{2.243309in}}%
\pgfpathlineto{\pgfqpoint{2.320477in}{2.012919in}}%
\pgfpathlineto{\pgfqpoint{2.321088in}{2.226852in}}%
\pgfpathlineto{\pgfqpoint{2.322107in}{2.152798in}}%
\pgfpathlineto{\pgfqpoint{2.322311in}{2.161027in}}%
\pgfpathlineto{\pgfqpoint{2.322515in}{2.119886in}}%
\pgfpathlineto{\pgfqpoint{2.322922in}{2.210396in}}%
\pgfpathlineto{\pgfqpoint{2.323126in}{2.210396in}}%
\pgfpathlineto{\pgfqpoint{2.324349in}{2.309134in}}%
\pgfpathlineto{\pgfqpoint{2.324756in}{2.251537in}}%
\pgfpathlineto{\pgfqpoint{2.324960in}{2.342047in}}%
\pgfpathlineto{\pgfqpoint{2.325368in}{2.276221in}}%
\pgfpathlineto{\pgfqpoint{2.325979in}{2.416101in}}%
\pgfpathlineto{\pgfqpoint{2.326795in}{2.374960in}}%
\pgfpathlineto{\pgfqpoint{2.329036in}{2.235080in}}%
\pgfpathlineto{\pgfqpoint{2.329240in}{2.218624in}}%
\pgfpathlineto{\pgfqpoint{2.329852in}{2.243309in}}%
\pgfpathlineto{\pgfqpoint{2.330055in}{2.235080in}}%
\pgfpathlineto{\pgfqpoint{2.331482in}{2.383188in}}%
\pgfpathlineto{\pgfqpoint{2.331890in}{2.292678in}}%
\pgfpathlineto{\pgfqpoint{2.332705in}{2.350275in}}%
\pgfpathlineto{\pgfqpoint{2.333520in}{2.366732in}}%
\pgfpathlineto{\pgfqpoint{2.333724in}{2.350275in}}%
\pgfpathlineto{\pgfqpoint{2.334335in}{2.292678in}}%
\pgfpathlineto{\pgfqpoint{2.335558in}{2.300906in}}%
\pgfpathlineto{\pgfqpoint{2.335762in}{2.333819in}}%
\pgfpathlineto{\pgfqpoint{2.336169in}{2.243309in}}%
\pgfpathlineto{\pgfqpoint{2.336373in}{2.317363in}}%
\pgfpathlineto{\pgfqpoint{2.336577in}{2.259765in}}%
\pgfpathlineto{\pgfqpoint{2.337392in}{2.300906in}}%
\pgfpathlineto{\pgfqpoint{2.338615in}{2.218624in}}%
\pgfpathlineto{\pgfqpoint{2.338819in}{2.267993in}}%
\pgfpathlineto{\pgfqpoint{2.340042in}{2.350275in}}%
\pgfpathlineto{\pgfqpoint{2.339227in}{2.210396in}}%
\pgfpathlineto{\pgfqpoint{2.340246in}{2.325591in}}%
\pgfpathlineto{\pgfqpoint{2.340449in}{2.325591in}}%
\pgfpathlineto{\pgfqpoint{2.340857in}{2.251537in}}%
\pgfpathlineto{\pgfqpoint{2.341265in}{2.333819in}}%
\pgfpathlineto{\pgfqpoint{2.341468in}{2.317363in}}%
\pgfpathlineto{\pgfqpoint{2.341876in}{2.391416in}}%
\pgfpathlineto{\pgfqpoint{2.342487in}{2.350275in}}%
\pgfpathlineto{\pgfqpoint{2.343099in}{2.383188in}}%
\pgfpathlineto{\pgfqpoint{2.343303in}{2.333819in}}%
\pgfpathlineto{\pgfqpoint{2.344525in}{2.473698in}}%
\pgfpathlineto{\pgfqpoint{2.346563in}{2.309134in}}%
\pgfpathlineto{\pgfqpoint{2.347990in}{2.539524in}}%
\pgfpathlineto{\pgfqpoint{2.348194in}{2.523068in}}%
\pgfpathlineto{\pgfqpoint{2.348398in}{2.523068in}}%
\pgfpathlineto{\pgfqpoint{2.348601in}{2.564209in}}%
\pgfpathlineto{\pgfqpoint{2.349213in}{2.506611in}}%
\pgfpathlineto{\pgfqpoint{2.349417in}{2.547752in}}%
\pgfpathlineto{\pgfqpoint{2.350639in}{2.465470in}}%
\pgfpathlineto{\pgfqpoint{2.351658in}{2.531296in}}%
\pgfpathlineto{\pgfqpoint{2.352474in}{2.440786in}}%
\pgfpathlineto{\pgfqpoint{2.352881in}{2.449014in}}%
\pgfpathlineto{\pgfqpoint{2.353085in}{2.432557in}}%
\pgfpathlineto{\pgfqpoint{2.353289in}{2.449014in}}%
\pgfpathlineto{\pgfqpoint{2.353493in}{2.506611in}}%
\pgfpathlineto{\pgfqpoint{2.354104in}{2.498383in}}%
\pgfpathlineto{\pgfqpoint{2.354308in}{2.399645in}}%
\pgfpathlineto{\pgfqpoint{2.355123in}{2.498383in}}%
\pgfpathlineto{\pgfqpoint{2.355327in}{2.449014in}}%
\pgfpathlineto{\pgfqpoint{2.355938in}{2.547752in}}%
\pgfpathlineto{\pgfqpoint{2.356346in}{2.457242in}}%
\pgfpathlineto{\pgfqpoint{2.356550in}{2.449014in}}%
\pgfpathlineto{\pgfqpoint{2.356754in}{2.457242in}}%
\pgfpathlineto{\pgfqpoint{2.357773in}{2.564209in}}%
\pgfpathlineto{\pgfqpoint{2.358180in}{2.531296in}}%
\pgfpathlineto{\pgfqpoint{2.358384in}{2.539524in}}%
\pgfpathlineto{\pgfqpoint{2.358588in}{2.514839in}}%
\pgfpathlineto{\pgfqpoint{2.359199in}{2.449014in}}%
\pgfpathlineto{\pgfqpoint{2.359607in}{2.523068in}}%
\pgfpathlineto{\pgfqpoint{2.359811in}{2.523068in}}%
\pgfpathlineto{\pgfqpoint{2.361033in}{2.440786in}}%
\pgfpathlineto{\pgfqpoint{2.361441in}{2.449014in}}%
\pgfpathlineto{\pgfqpoint{2.361849in}{2.531296in}}%
\pgfpathlineto{\pgfqpoint{2.362460in}{2.473698in}}%
\pgfpathlineto{\pgfqpoint{2.362664in}{2.473698in}}%
\pgfpathlineto{\pgfqpoint{2.363683in}{2.391416in}}%
\pgfpathlineto{\pgfqpoint{2.363887in}{2.465470in}}%
\pgfpathlineto{\pgfqpoint{2.364906in}{2.449014in}}%
\pgfpathlineto{\pgfqpoint{2.365925in}{2.407873in}}%
\pgfpathlineto{\pgfqpoint{2.366129in}{2.416101in}}%
\pgfpathlineto{\pgfqpoint{2.366944in}{2.391416in}}%
\pgfpathlineto{\pgfqpoint{2.367148in}{2.399645in}}%
\pgfpathlineto{\pgfqpoint{2.367963in}{2.391416in}}%
\pgfpathlineto{\pgfqpoint{2.368778in}{2.572437in}}%
\pgfpathlineto{\pgfqpoint{2.369593in}{2.465470in}}%
\pgfpathlineto{\pgfqpoint{2.370001in}{2.473698in}}%
\pgfpathlineto{\pgfqpoint{2.370205in}{2.539524in}}%
\pgfpathlineto{\pgfqpoint{2.371020in}{2.498383in}}%
\pgfpathlineto{\pgfqpoint{2.371224in}{2.481927in}}%
\pgfpathlineto{\pgfqpoint{2.371427in}{2.498383in}}%
\pgfpathlineto{\pgfqpoint{2.371631in}{2.325591in}}%
\pgfpathlineto{\pgfqpoint{2.371835in}{2.539524in}}%
\pgfpathlineto{\pgfqpoint{2.372446in}{2.514839in}}%
\pgfpathlineto{\pgfqpoint{2.372650in}{2.531296in}}%
\pgfpathlineto{\pgfqpoint{2.373262in}{2.514839in}}%
\pgfpathlineto{\pgfqpoint{2.374484in}{2.416101in}}%
\pgfpathlineto{\pgfqpoint{2.374688in}{2.424329in}}%
\pgfpathlineto{\pgfqpoint{2.375707in}{2.465470in}}%
\pgfpathlineto{\pgfqpoint{2.375911in}{2.399645in}}%
\pgfpathlineto{\pgfqpoint{2.376115in}{2.473698in}}%
\pgfpathlineto{\pgfqpoint{2.376726in}{2.457242in}}%
\pgfpathlineto{\pgfqpoint{2.376930in}{2.457242in}}%
\pgfpathlineto{\pgfqpoint{2.377541in}{2.424329in}}%
\pgfpathlineto{\pgfqpoint{2.378764in}{2.218624in}}%
\pgfpathlineto{\pgfqpoint{2.378153in}{2.457242in}}%
\pgfpathlineto{\pgfqpoint{2.378968in}{2.292678in}}%
\pgfpathlineto{\pgfqpoint{2.380599in}{2.457242in}}%
\pgfpathlineto{\pgfqpoint{2.381414in}{2.391416in}}%
\pgfpathlineto{\pgfqpoint{2.381618in}{2.424329in}}%
\pgfpathlineto{\pgfqpoint{2.381821in}{2.457242in}}%
\pgfpathlineto{\pgfqpoint{2.382025in}{2.416101in}}%
\pgfpathlineto{\pgfqpoint{2.382637in}{2.185711in}}%
\pgfpathlineto{\pgfqpoint{2.382840in}{2.481927in}}%
\pgfpathlineto{\pgfqpoint{2.383044in}{2.457242in}}%
\pgfpathlineto{\pgfqpoint{2.383248in}{2.572437in}}%
\pgfpathlineto{\pgfqpoint{2.383452in}{2.416101in}}%
\pgfpathlineto{\pgfqpoint{2.384063in}{2.506611in}}%
\pgfpathlineto{\pgfqpoint{2.385286in}{2.424329in}}%
\pgfpathlineto{\pgfqpoint{2.385490in}{2.432557in}}%
\pgfpathlineto{\pgfqpoint{2.385694in}{2.506611in}}%
\pgfpathlineto{\pgfqpoint{2.386101in}{2.383188in}}%
\pgfpathlineto{\pgfqpoint{2.386509in}{2.440786in}}%
\pgfpathlineto{\pgfqpoint{2.387324in}{2.333819in}}%
\pgfpathlineto{\pgfqpoint{2.387935in}{2.342047in}}%
\pgfpathlineto{\pgfqpoint{2.388751in}{2.383188in}}%
\pgfpathlineto{\pgfqpoint{2.388954in}{2.374960in}}%
\pgfpathlineto{\pgfqpoint{2.389770in}{2.177483in}}%
\pgfpathlineto{\pgfqpoint{2.389973in}{2.325591in}}%
\pgfpathlineto{\pgfqpoint{2.390789in}{2.416101in}}%
\pgfpathlineto{\pgfqpoint{2.390585in}{2.309134in}}%
\pgfpathlineto{\pgfqpoint{2.390992in}{2.358504in}}%
\pgfpathlineto{\pgfqpoint{2.391196in}{2.333819in}}%
\pgfpathlineto{\pgfqpoint{2.392623in}{2.547752in}}%
\pgfpathlineto{\pgfqpoint{2.392827in}{2.547752in}}%
\pgfpathlineto{\pgfqpoint{2.393031in}{2.555980in}}%
\pgfpathlineto{\pgfqpoint{2.393234in}{2.613578in}}%
\pgfpathlineto{\pgfqpoint{2.393642in}{2.457242in}}%
\pgfpathlineto{\pgfqpoint{2.394050in}{2.424329in}}%
\pgfpathlineto{\pgfqpoint{2.394253in}{2.383188in}}%
\pgfpathlineto{\pgfqpoint{2.394865in}{2.465470in}}%
\pgfpathlineto{\pgfqpoint{2.395069in}{2.473698in}}%
\pgfpathlineto{\pgfqpoint{2.395272in}{2.449014in}}%
\pgfpathlineto{\pgfqpoint{2.395680in}{2.391416in}}%
\pgfpathlineto{\pgfqpoint{2.395884in}{2.523068in}}%
\pgfpathlineto{\pgfqpoint{2.396903in}{2.481927in}}%
\pgfpathlineto{\pgfqpoint{2.397310in}{2.481927in}}%
\pgfpathlineto{\pgfqpoint{2.397718in}{2.514839in}}%
\pgfpathlineto{\pgfqpoint{2.398126in}{2.490155in}}%
\pgfpathlineto{\pgfqpoint{2.398533in}{2.391416in}}%
\pgfpathlineto{\pgfqpoint{2.399552in}{2.432557in}}%
\pgfpathlineto{\pgfqpoint{2.401183in}{2.539524in}}%
\pgfpathlineto{\pgfqpoint{2.401386in}{2.473698in}}%
\pgfpathlineto{\pgfqpoint{2.402202in}{2.564209in}}%
\pgfpathlineto{\pgfqpoint{2.403832in}{2.473698in}}%
\pgfpathlineto{\pgfqpoint{2.404240in}{2.449014in}}%
\pgfpathlineto{\pgfqpoint{2.405055in}{2.514839in}}%
\pgfpathlineto{\pgfqpoint{2.405259in}{2.514839in}}%
\pgfpathlineto{\pgfqpoint{2.406074in}{2.432557in}}%
\pgfpathlineto{\pgfqpoint{2.406278in}{2.457242in}}%
\pgfpathlineto{\pgfqpoint{2.406482in}{2.498383in}}%
\pgfpathlineto{\pgfqpoint{2.406889in}{2.432557in}}%
\pgfpathlineto{\pgfqpoint{2.407501in}{2.490155in}}%
\pgfpathlineto{\pgfqpoint{2.408723in}{2.432557in}}%
\pgfpathlineto{\pgfqpoint{2.409742in}{2.514839in}}%
\pgfpathlineto{\pgfqpoint{2.409335in}{2.416101in}}%
\pgfpathlineto{\pgfqpoint{2.409946in}{2.490155in}}%
\pgfpathlineto{\pgfqpoint{2.410354in}{2.514839in}}%
\pgfpathlineto{\pgfqpoint{2.411373in}{2.416101in}}%
\pgfpathlineto{\pgfqpoint{2.411984in}{2.481927in}}%
\pgfpathlineto{\pgfqpoint{2.412392in}{2.440786in}}%
\pgfpathlineto{\pgfqpoint{2.412596in}{2.292678in}}%
\pgfpathlineto{\pgfqpoint{2.412799in}{2.481927in}}%
\pgfpathlineto{\pgfqpoint{2.413411in}{2.473698in}}%
\pgfpathlineto{\pgfqpoint{2.413818in}{2.481927in}}%
\pgfpathlineto{\pgfqpoint{2.414226in}{2.597121in}}%
\pgfpathlineto{\pgfqpoint{2.414837in}{2.580665in}}%
\pgfpathlineto{\pgfqpoint{2.415245in}{2.457242in}}%
\pgfpathlineto{\pgfqpoint{2.415856in}{2.514839in}}%
\pgfpathlineto{\pgfqpoint{2.416468in}{2.539524in}}%
\pgfpathlineto{\pgfqpoint{2.417079in}{2.416101in}}%
\pgfpathlineto{\pgfqpoint{2.417487in}{2.449014in}}%
\pgfpathlineto{\pgfqpoint{2.417691in}{2.514839in}}%
\pgfpathlineto{\pgfqpoint{2.417894in}{2.424329in}}%
\pgfpathlineto{\pgfqpoint{2.418506in}{2.440786in}}%
\pgfpathlineto{\pgfqpoint{2.419321in}{2.498383in}}%
\pgfpathlineto{\pgfqpoint{2.419933in}{2.473698in}}%
\pgfpathlineto{\pgfqpoint{2.421155in}{2.416101in}}%
\pgfpathlineto{\pgfqpoint{2.422174in}{2.481927in}}%
\pgfpathlineto{\pgfqpoint{2.422378in}{2.300906in}}%
\pgfpathlineto{\pgfqpoint{2.423193in}{2.523068in}}%
\pgfpathlineto{\pgfqpoint{2.423397in}{2.547752in}}%
\pgfpathlineto{\pgfqpoint{2.423601in}{2.490155in}}%
\pgfpathlineto{\pgfqpoint{2.423805in}{2.506611in}}%
\pgfpathlineto{\pgfqpoint{2.425435in}{2.416101in}}%
\pgfpathlineto{\pgfqpoint{2.426454in}{2.547752in}}%
\pgfpathlineto{\pgfqpoint{2.426047in}{2.383188in}}%
\pgfpathlineto{\pgfqpoint{2.426862in}{2.481927in}}%
\pgfpathlineto{\pgfqpoint{2.427269in}{2.152798in}}%
\pgfpathlineto{\pgfqpoint{2.428085in}{2.317363in}}%
\pgfpathlineto{\pgfqpoint{2.429307in}{2.449014in}}%
\pgfpathlineto{\pgfqpoint{2.429511in}{2.317363in}}%
\pgfpathlineto{\pgfqpoint{2.430326in}{2.465470in}}%
\pgfpathlineto{\pgfqpoint{2.431957in}{2.169255in}}%
\pgfpathlineto{\pgfqpoint{2.433180in}{2.572437in}}%
\pgfpathlineto{\pgfqpoint{2.435218in}{2.416101in}}%
\pgfpathlineto{\pgfqpoint{2.433587in}{2.588893in}}%
\pgfpathlineto{\pgfqpoint{2.435422in}{2.449014in}}%
\pgfpathlineto{\pgfqpoint{2.436848in}{2.555980in}}%
\pgfpathlineto{\pgfqpoint{2.437256in}{2.555980in}}%
\pgfpathlineto{\pgfqpoint{2.437460in}{2.481927in}}%
\pgfpathlineto{\pgfqpoint{2.438071in}{2.572437in}}%
\pgfpathlineto{\pgfqpoint{2.438275in}{2.506611in}}%
\pgfpathlineto{\pgfqpoint{2.438886in}{2.490155in}}%
\pgfpathlineto{\pgfqpoint{2.439294in}{2.555980in}}%
\pgfpathlineto{\pgfqpoint{2.440109in}{2.473698in}}%
\pgfpathlineto{\pgfqpoint{2.440313in}{2.555980in}}%
\pgfpathlineto{\pgfqpoint{2.441128in}{2.531296in}}%
\pgfpathlineto{\pgfqpoint{2.441536in}{2.605350in}}%
\pgfpathlineto{\pgfqpoint{2.441943in}{2.498383in}}%
\pgfpathlineto{\pgfqpoint{2.442758in}{2.547752in}}%
\pgfpathlineto{\pgfqpoint{2.442962in}{2.572437in}}%
\pgfpathlineto{\pgfqpoint{2.443370in}{2.547752in}}%
\pgfpathlineto{\pgfqpoint{2.443777in}{2.251537in}}%
\pgfpathlineto{\pgfqpoint{2.444389in}{2.457242in}}%
\pgfpathlineto{\pgfqpoint{2.445408in}{2.572437in}}%
\pgfpathlineto{\pgfqpoint{2.445612in}{2.531296in}}%
\pgfpathlineto{\pgfqpoint{2.446019in}{2.506611in}}%
\pgfpathlineto{\pgfqpoint{2.446631in}{2.555980in}}%
\pgfpathlineto{\pgfqpoint{2.446835in}{2.292678in}}%
\pgfpathlineto{\pgfqpoint{2.447854in}{2.399645in}}%
\pgfpathlineto{\pgfqpoint{2.448261in}{2.383188in}}%
\pgfpathlineto{\pgfqpoint{2.449076in}{2.440786in}}%
\pgfpathlineto{\pgfqpoint{2.449280in}{2.391416in}}%
\pgfpathlineto{\pgfqpoint{2.449484in}{2.457242in}}%
\pgfpathlineto{\pgfqpoint{2.449892in}{2.432557in}}%
\pgfpathlineto{\pgfqpoint{2.450095in}{2.473698in}}%
\pgfpathlineto{\pgfqpoint{2.450707in}{2.383188in}}%
\pgfpathlineto{\pgfqpoint{2.450911in}{2.374960in}}%
\pgfpathlineto{\pgfqpoint{2.451114in}{2.152798in}}%
\pgfpathlineto{\pgfqpoint{2.451930in}{2.432557in}}%
\pgfpathlineto{\pgfqpoint{2.452133in}{2.399645in}}%
\pgfpathlineto{\pgfqpoint{2.452337in}{2.457242in}}%
\pgfpathlineto{\pgfqpoint{2.453152in}{2.597121in}}%
\pgfpathlineto{\pgfqpoint{2.453560in}{2.572437in}}%
\pgfpathlineto{\pgfqpoint{2.454579in}{2.588893in}}%
\pgfpathlineto{\pgfqpoint{2.454783in}{2.490155in}}%
\pgfpathlineto{\pgfqpoint{2.455598in}{2.613578in}}%
\pgfpathlineto{\pgfqpoint{2.455802in}{2.539524in}}%
\pgfpathlineto{\pgfqpoint{2.456006in}{2.333819in}}%
\pgfpathlineto{\pgfqpoint{2.457025in}{2.374960in}}%
\pgfpathlineto{\pgfqpoint{2.458859in}{2.136342in}}%
\pgfpathlineto{\pgfqpoint{2.459878in}{2.284450in}}%
\pgfpathlineto{\pgfqpoint{2.460082in}{2.235080in}}%
\pgfpathlineto{\pgfqpoint{2.460286in}{2.284450in}}%
\pgfpathlineto{\pgfqpoint{2.460489in}{2.152798in}}%
\pgfpathlineto{\pgfqpoint{2.460693in}{2.078745in}}%
\pgfpathlineto{\pgfqpoint{2.461101in}{2.267993in}}%
\pgfpathlineto{\pgfqpoint{2.461305in}{2.235080in}}%
\pgfpathlineto{\pgfqpoint{2.462324in}{2.012919in}}%
\pgfpathlineto{\pgfqpoint{2.462120in}{2.243309in}}%
\pgfpathlineto{\pgfqpoint{2.462527in}{2.128114in}}%
\pgfpathlineto{\pgfqpoint{2.463546in}{2.350275in}}%
\pgfpathlineto{\pgfqpoint{2.463750in}{2.251537in}}%
\pgfpathlineto{\pgfqpoint{2.463954in}{2.226852in}}%
\pgfpathlineto{\pgfqpoint{2.464362in}{2.309134in}}%
\pgfpathlineto{\pgfqpoint{2.464565in}{2.383188in}}%
\pgfpathlineto{\pgfqpoint{2.464973in}{2.259765in}}%
\pgfpathlineto{\pgfqpoint{2.465177in}{2.317363in}}%
\pgfpathlineto{\pgfqpoint{2.465381in}{2.251537in}}%
\pgfpathlineto{\pgfqpoint{2.465584in}{2.325591in}}%
\pgfpathlineto{\pgfqpoint{2.466196in}{2.276221in}}%
\pgfpathlineto{\pgfqpoint{2.467419in}{2.391416in}}%
\pgfpathlineto{\pgfqpoint{2.468030in}{2.383188in}}%
\pgfpathlineto{\pgfqpoint{2.469253in}{2.136342in}}%
\pgfpathlineto{\pgfqpoint{2.469457in}{2.350275in}}%
\pgfpathlineto{\pgfqpoint{2.470476in}{2.292678in}}%
\pgfpathlineto{\pgfqpoint{2.470679in}{2.325591in}}%
\pgfpathlineto{\pgfqpoint{2.470883in}{2.251537in}}%
\pgfpathlineto{\pgfqpoint{2.471087in}{2.193939in}}%
\pgfpathlineto{\pgfqpoint{2.471698in}{2.333819in}}%
\pgfpathlineto{\pgfqpoint{2.471902in}{2.342047in}}%
\pgfpathlineto{\pgfqpoint{2.472310in}{2.218624in}}%
\pgfpathlineto{\pgfqpoint{2.472921in}{2.292678in}}%
\pgfpathlineto{\pgfqpoint{2.473329in}{2.325591in}}%
\pgfpathlineto{\pgfqpoint{2.473533in}{2.317363in}}%
\pgfpathlineto{\pgfqpoint{2.473737in}{2.226852in}}%
\pgfpathlineto{\pgfqpoint{2.473940in}{2.358504in}}%
\pgfpathlineto{\pgfqpoint{2.474756in}{2.243309in}}%
\pgfpathlineto{\pgfqpoint{2.474959in}{2.292678in}}%
\pgfpathlineto{\pgfqpoint{2.475571in}{2.226852in}}%
\pgfpathlineto{\pgfqpoint{2.475775in}{2.259765in}}%
\pgfpathlineto{\pgfqpoint{2.476182in}{1.988234in}}%
\pgfpathlineto{\pgfqpoint{2.476997in}{2.193939in}}%
\pgfpathlineto{\pgfqpoint{2.478220in}{2.342047in}}%
\pgfpathlineto{\pgfqpoint{2.478424in}{2.259765in}}%
\pgfpathlineto{\pgfqpoint{2.479239in}{2.342047in}}%
\pgfpathlineto{\pgfqpoint{2.479647in}{2.432557in}}%
\pgfpathlineto{\pgfqpoint{2.479851in}{2.333819in}}%
\pgfpathlineto{\pgfqpoint{2.480258in}{2.416101in}}%
\pgfpathlineto{\pgfqpoint{2.481481in}{2.193939in}}%
\pgfpathlineto{\pgfqpoint{2.481889in}{2.243309in}}%
\pgfpathlineto{\pgfqpoint{2.482092in}{2.300906in}}%
\pgfpathlineto{\pgfqpoint{2.482704in}{2.235080in}}%
\pgfpathlineto{\pgfqpoint{2.482908in}{2.259765in}}%
\pgfpathlineto{\pgfqpoint{2.483927in}{2.210396in}}%
\pgfpathlineto{\pgfqpoint{2.484130in}{2.119886in}}%
\pgfpathlineto{\pgfqpoint{2.484538in}{2.243309in}}%
\pgfpathlineto{\pgfqpoint{2.484946in}{2.226852in}}%
\pgfpathlineto{\pgfqpoint{2.485965in}{2.333819in}}%
\pgfpathlineto{\pgfqpoint{2.485557in}{2.210396in}}%
\pgfpathlineto{\pgfqpoint{2.486372in}{2.259765in}}%
\pgfpathlineto{\pgfqpoint{2.487799in}{1.947093in}}%
\pgfpathlineto{\pgfqpoint{2.488003in}{1.996463in}}%
\pgfpathlineto{\pgfqpoint{2.488410in}{2.185711in}}%
\pgfpathlineto{\pgfqpoint{2.489226in}{2.177483in}}%
\pgfpathlineto{\pgfqpoint{2.490448in}{2.086973in}}%
\pgfpathlineto{\pgfqpoint{2.491264in}{2.144570in}}%
\pgfpathlineto{\pgfqpoint{2.492486in}{1.971778in}}%
\pgfpathlineto{\pgfqpoint{2.493913in}{2.185711in}}%
\pgfpathlineto{\pgfqpoint{2.494932in}{2.128114in}}%
\pgfpathlineto{\pgfqpoint{2.494728in}{2.218624in}}%
\pgfpathlineto{\pgfqpoint{2.495136in}{2.161027in}}%
\pgfpathlineto{\pgfqpoint{2.495340in}{2.226852in}}%
\pgfpathlineto{\pgfqpoint{2.495543in}{2.128114in}}%
\pgfpathlineto{\pgfqpoint{2.495747in}{2.185711in}}%
\pgfpathlineto{\pgfqpoint{2.496970in}{2.012919in}}%
\pgfpathlineto{\pgfqpoint{2.497989in}{2.152798in}}%
\pgfpathlineto{\pgfqpoint{2.498397in}{2.128114in}}%
\pgfpathlineto{\pgfqpoint{2.498804in}{2.193939in}}%
\pgfpathlineto{\pgfqpoint{2.499416in}{2.185711in}}%
\pgfpathlineto{\pgfqpoint{2.499619in}{2.128114in}}%
\pgfpathlineto{\pgfqpoint{2.500435in}{2.202168in}}%
\pgfpathlineto{\pgfqpoint{2.500842in}{2.177483in}}%
\pgfpathlineto{\pgfqpoint{2.501250in}{2.235080in}}%
\pgfpathlineto{\pgfqpoint{2.501658in}{2.144570in}}%
\pgfpathlineto{\pgfqpoint{2.502269in}{2.226852in}}%
\pgfpathlineto{\pgfqpoint{2.502473in}{2.243309in}}%
\pgfpathlineto{\pgfqpoint{2.502677in}{2.210396in}}%
\pgfpathlineto{\pgfqpoint{2.503492in}{2.161027in}}%
\pgfpathlineto{\pgfqpoint{2.503696in}{2.169255in}}%
\pgfpathlineto{\pgfqpoint{2.503899in}{2.202168in}}%
\pgfpathlineto{\pgfqpoint{2.504511in}{2.152798in}}%
\pgfpathlineto{\pgfqpoint{2.505734in}{2.021147in}}%
\pgfpathlineto{\pgfqpoint{2.505937in}{2.029375in}}%
\pgfpathlineto{\pgfqpoint{2.506345in}{2.144570in}}%
\pgfpathlineto{\pgfqpoint{2.506753in}{2.021147in}}%
\pgfpathlineto{\pgfqpoint{2.506956in}{2.045832in}}%
\pgfpathlineto{\pgfqpoint{2.507160in}{1.996463in}}%
\pgfpathlineto{\pgfqpoint{2.507975in}{2.062288in}}%
\pgfpathlineto{\pgfqpoint{2.508179in}{2.062288in}}%
\pgfpathlineto{\pgfqpoint{2.508383in}{2.021147in}}%
\pgfpathlineto{\pgfqpoint{2.508791in}{2.078745in}}%
\pgfpathlineto{\pgfqpoint{2.508994in}{2.169255in}}%
\pgfpathlineto{\pgfqpoint{2.509606in}{1.971778in}}%
\pgfpathlineto{\pgfqpoint{2.510217in}{2.136342in}}%
\pgfpathlineto{\pgfqpoint{2.510625in}{1.971778in}}%
\pgfpathlineto{\pgfqpoint{2.510829in}{1.996463in}}%
\pgfpathlineto{\pgfqpoint{2.511032in}{1.938865in}}%
\pgfpathlineto{\pgfqpoint{2.511644in}{1.971778in}}%
\pgfpathlineto{\pgfqpoint{2.513070in}{1.823670in}}%
\pgfpathlineto{\pgfqpoint{2.512051in}{1.980006in}}%
\pgfpathlineto{\pgfqpoint{2.513274in}{1.848355in}}%
\pgfpathlineto{\pgfqpoint{2.513478in}{1.897724in}}%
\pgfpathlineto{\pgfqpoint{2.513886in}{1.840127in}}%
\pgfpathlineto{\pgfqpoint{2.514090in}{1.634422in}}%
\pgfpathlineto{\pgfqpoint{2.514905in}{1.897724in}}%
\pgfpathlineto{\pgfqpoint{2.515516in}{1.938865in}}%
\pgfpathlineto{\pgfqpoint{2.515312in}{1.873040in}}%
\pgfpathlineto{\pgfqpoint{2.515720in}{1.889496in}}%
\pgfpathlineto{\pgfqpoint{2.516943in}{1.741388in}}%
\pgfpathlineto{\pgfqpoint{2.517962in}{1.930637in}}%
\pgfpathlineto{\pgfqpoint{2.518166in}{1.889496in}}%
\pgfpathlineto{\pgfqpoint{2.518981in}{1.741388in}}%
\pgfpathlineto{\pgfqpoint{2.519185in}{1.766073in}}%
\pgfpathlineto{\pgfqpoint{2.519796in}{1.922409in}}%
\pgfpathlineto{\pgfqpoint{2.520407in}{1.881268in}}%
\pgfpathlineto{\pgfqpoint{2.521834in}{1.988234in}}%
\pgfpathlineto{\pgfqpoint{2.522038in}{2.012919in}}%
\pgfpathlineto{\pgfqpoint{2.522445in}{1.971778in}}%
\pgfpathlineto{\pgfqpoint{2.522853in}{2.004691in}}%
\pgfpathlineto{\pgfqpoint{2.523464in}{1.914181in}}%
\pgfpathlineto{\pgfqpoint{2.524687in}{2.045832in}}%
\pgfpathlineto{\pgfqpoint{2.524891in}{2.037604in}}%
\pgfpathlineto{\pgfqpoint{2.525299in}{2.086973in}}%
\pgfpathlineto{\pgfqpoint{2.525706in}{2.029375in}}%
\pgfpathlineto{\pgfqpoint{2.525910in}{2.078745in}}%
\pgfpathlineto{\pgfqpoint{2.526318in}{2.029375in}}%
\pgfpathlineto{\pgfqpoint{2.526725in}{2.128114in}}%
\pgfpathlineto{\pgfqpoint{2.527744in}{2.070516in}}%
\pgfpathlineto{\pgfqpoint{2.528152in}{2.095201in}}%
\pgfpathlineto{\pgfqpoint{2.528356in}{2.119886in}}%
\pgfpathlineto{\pgfqpoint{2.528763in}{2.054060in}}%
\pgfpathlineto{\pgfqpoint{2.528967in}{2.062288in}}%
\pgfpathlineto{\pgfqpoint{2.529171in}{2.062288in}}%
\pgfpathlineto{\pgfqpoint{2.529579in}{2.111657in}}%
\pgfpathlineto{\pgfqpoint{2.530190in}{2.078745in}}%
\pgfpathlineto{\pgfqpoint{2.530801in}{1.996463in}}%
\pgfpathlineto{\pgfqpoint{2.531413in}{2.004691in}}%
\pgfpathlineto{\pgfqpoint{2.531617in}{2.062288in}}%
\pgfpathlineto{\pgfqpoint{2.532432in}{1.988234in}}%
\pgfpathlineto{\pgfqpoint{2.532636in}{2.037604in}}%
\pgfpathlineto{\pgfqpoint{2.532839in}{2.012919in}}%
\pgfpathlineto{\pgfqpoint{2.533247in}{2.078745in}}%
\pgfpathlineto{\pgfqpoint{2.533451in}{2.062288in}}%
\pgfpathlineto{\pgfqpoint{2.533655in}{2.070516in}}%
\pgfpathlineto{\pgfqpoint{2.534266in}{2.078745in}}%
\pgfpathlineto{\pgfqpoint{2.535081in}{1.971778in}}%
\pgfpathlineto{\pgfqpoint{2.535285in}{1.971778in}}%
\pgfpathlineto{\pgfqpoint{2.535896in}{1.930637in}}%
\pgfpathlineto{\pgfqpoint{2.536100in}{1.963550in}}%
\pgfpathlineto{\pgfqpoint{2.536304in}{2.004691in}}%
\pgfpathlineto{\pgfqpoint{2.536712in}{1.914181in}}%
\pgfpathlineto{\pgfqpoint{2.537119in}{1.955322in}}%
\pgfpathlineto{\pgfqpoint{2.538138in}{1.889496in}}%
\pgfpathlineto{\pgfqpoint{2.537934in}{1.963550in}}%
\pgfpathlineto{\pgfqpoint{2.538546in}{1.914181in}}%
\pgfpathlineto{\pgfqpoint{2.538750in}{1.914181in}}%
\pgfpathlineto{\pgfqpoint{2.540176in}{1.782529in}}%
\pgfpathlineto{\pgfqpoint{2.539157in}{1.930637in}}%
\pgfpathlineto{\pgfqpoint{2.540380in}{1.831899in}}%
\pgfpathlineto{\pgfqpoint{2.540788in}{1.864811in}}%
\pgfpathlineto{\pgfqpoint{2.540992in}{1.840127in}}%
\pgfpathlineto{\pgfqpoint{2.542011in}{1.724932in}}%
\pgfpathlineto{\pgfqpoint{2.542214in}{1.782529in}}%
\pgfpathlineto{\pgfqpoint{2.542418in}{1.782529in}}%
\pgfpathlineto{\pgfqpoint{2.543845in}{1.897724in}}%
\pgfpathlineto{\pgfqpoint{2.544049in}{1.807214in}}%
\pgfpathlineto{\pgfqpoint{2.544660in}{1.922409in}}%
\pgfpathlineto{\pgfqpoint{2.544864in}{1.864811in}}%
\pgfpathlineto{\pgfqpoint{2.545271in}{1.823670in}}%
\pgfpathlineto{\pgfqpoint{2.545679in}{1.848355in}}%
\pgfpathlineto{\pgfqpoint{2.546494in}{1.716704in}}%
\pgfpathlineto{\pgfqpoint{2.546698in}{1.823670in}}%
\pgfpathlineto{\pgfqpoint{2.546902in}{1.897724in}}%
\pgfpathlineto{\pgfqpoint{2.547513in}{1.741388in}}%
\pgfpathlineto{\pgfqpoint{2.547921in}{1.535683in}}%
\pgfpathlineto{\pgfqpoint{2.548328in}{1.601509in}}%
\pgfpathlineto{\pgfqpoint{2.548736in}{1.840127in}}%
\pgfpathlineto{\pgfqpoint{2.549551in}{1.798986in}}%
\pgfpathlineto{\pgfqpoint{2.550570in}{1.741388in}}%
\pgfpathlineto{\pgfqpoint{2.551182in}{1.848355in}}%
\pgfpathlineto{\pgfqpoint{2.551793in}{1.840127in}}%
\pgfpathlineto{\pgfqpoint{2.553627in}{1.683791in}}%
\pgfpathlineto{\pgfqpoint{2.552404in}{1.856583in}}%
\pgfpathlineto{\pgfqpoint{2.553831in}{1.724932in}}%
\pgfpathlineto{\pgfqpoint{2.554239in}{1.692019in}}%
\pgfpathlineto{\pgfqpoint{2.554646in}{1.741388in}}%
\pgfpathlineto{\pgfqpoint{2.555054in}{1.700247in}}%
\pgfpathlineto{\pgfqpoint{2.555258in}{1.724932in}}%
\pgfpathlineto{\pgfqpoint{2.555869in}{1.708476in}}%
\pgfpathlineto{\pgfqpoint{2.556073in}{1.650878in}}%
\pgfpathlineto{\pgfqpoint{2.556684in}{1.774301in}}%
\pgfpathlineto{\pgfqpoint{2.556888in}{1.733160in}}%
\pgfpathlineto{\pgfqpoint{2.557092in}{1.733160in}}%
\pgfpathlineto{\pgfqpoint{2.557296in}{1.766073in}}%
\pgfpathlineto{\pgfqpoint{2.557907in}{1.708476in}}%
\pgfpathlineto{\pgfqpoint{2.558111in}{1.733160in}}%
\pgfpathlineto{\pgfqpoint{2.559130in}{1.864811in}}%
\pgfpathlineto{\pgfqpoint{2.559334in}{1.790758in}}%
\pgfpathlineto{\pgfqpoint{2.559538in}{1.749617in}}%
\pgfpathlineto{\pgfqpoint{2.560149in}{1.807214in}}%
\pgfpathlineto{\pgfqpoint{2.560353in}{1.798986in}}%
\pgfpathlineto{\pgfqpoint{2.562187in}{1.988234in}}%
\pgfpathlineto{\pgfqpoint{2.562595in}{1.741388in}}%
\pgfpathlineto{\pgfqpoint{2.563206in}{1.996463in}}%
\pgfpathlineto{\pgfqpoint{2.563410in}{1.848355in}}%
\pgfpathlineto{\pgfqpoint{2.563817in}{1.980006in}}%
\pgfpathlineto{\pgfqpoint{2.564225in}{1.963550in}}%
\pgfpathlineto{\pgfqpoint{2.564429in}{1.741388in}}%
\pgfpathlineto{\pgfqpoint{2.565040in}{1.971778in}}%
\pgfpathlineto{\pgfqpoint{2.565448in}{1.790758in}}%
\pgfpathlineto{\pgfqpoint{2.566059in}{2.012919in}}%
\pgfpathlineto{\pgfqpoint{2.566671in}{1.930637in}}%
\pgfpathlineto{\pgfqpoint{2.566874in}{1.922409in}}%
\pgfpathlineto{\pgfqpoint{2.567078in}{1.938865in}}%
\pgfpathlineto{\pgfqpoint{2.567282in}{1.930637in}}%
\pgfpathlineto{\pgfqpoint{2.567894in}{2.012919in}}%
\pgfpathlineto{\pgfqpoint{2.568301in}{1.897724in}}%
\pgfpathlineto{\pgfqpoint{2.569524in}{1.848355in}}%
\pgfpathlineto{\pgfqpoint{2.569728in}{1.848355in}}%
\pgfpathlineto{\pgfqpoint{2.569932in}{1.881268in}}%
\pgfpathlineto{\pgfqpoint{2.570339in}{1.798986in}}%
\pgfpathlineto{\pgfqpoint{2.570951in}{1.873040in}}%
\pgfpathlineto{\pgfqpoint{2.571766in}{1.938865in}}%
\pgfpathlineto{\pgfqpoint{2.572377in}{1.922409in}}%
\pgfpathlineto{\pgfqpoint{2.572581in}{1.922409in}}%
\pgfpathlineto{\pgfqpoint{2.572785in}{1.938865in}}%
\pgfpathlineto{\pgfqpoint{2.572989in}{1.914181in}}%
\pgfpathlineto{\pgfqpoint{2.574415in}{1.626194in}}%
\pgfpathlineto{\pgfqpoint{2.575027in}{1.996463in}}%
\pgfpathlineto{\pgfqpoint{2.575638in}{1.856583in}}%
\pgfpathlineto{\pgfqpoint{2.576046in}{2.062288in}}%
\pgfpathlineto{\pgfqpoint{2.576861in}{1.971778in}}%
\pgfpathlineto{\pgfqpoint{2.577676in}{1.782529in}}%
\pgfpathlineto{\pgfqpoint{2.577472in}{2.029375in}}%
\pgfpathlineto{\pgfqpoint{2.577880in}{1.930637in}}%
\pgfpathlineto{\pgfqpoint{2.578084in}{1.955322in}}%
\pgfpathlineto{\pgfqpoint{2.578491in}{1.938865in}}%
\pgfpathlineto{\pgfqpoint{2.579918in}{1.766073in}}%
\pgfpathlineto{\pgfqpoint{2.581141in}{1.831899in}}%
\pgfpathlineto{\pgfqpoint{2.581345in}{1.848355in}}%
\pgfpathlineto{\pgfqpoint{2.581548in}{1.807214in}}%
\pgfpathlineto{\pgfqpoint{2.582771in}{1.601509in}}%
\pgfpathlineto{\pgfqpoint{2.582975in}{1.609737in}}%
\pgfpathlineto{\pgfqpoint{2.583383in}{1.873040in}}%
\pgfpathlineto{\pgfqpoint{2.584198in}{1.807214in}}%
\pgfpathlineto{\pgfqpoint{2.584605in}{1.831899in}}%
\pgfpathlineto{\pgfqpoint{2.584809in}{1.790758in}}%
\pgfpathlineto{\pgfqpoint{2.585217in}{1.807214in}}%
\pgfpathlineto{\pgfqpoint{2.585624in}{1.766073in}}%
\pgfpathlineto{\pgfqpoint{2.585828in}{1.815442in}}%
\pgfpathlineto{\pgfqpoint{2.586032in}{1.774301in}}%
\pgfpathlineto{\pgfqpoint{2.587051in}{1.881268in}}%
\pgfpathlineto{\pgfqpoint{2.587255in}{1.798986in}}%
\pgfpathlineto{\pgfqpoint{2.588274in}{1.815442in}}%
\pgfpathlineto{\pgfqpoint{2.588478in}{1.856583in}}%
\pgfpathlineto{\pgfqpoint{2.588681in}{1.798986in}}%
\pgfpathlineto{\pgfqpoint{2.588885in}{1.798986in}}%
\pgfpathlineto{\pgfqpoint{2.589293in}{1.692019in}}%
\pgfpathlineto{\pgfqpoint{2.590108in}{1.700247in}}%
\pgfpathlineto{\pgfqpoint{2.590923in}{1.905952in}}%
\pgfpathlineto{\pgfqpoint{2.591331in}{1.815442in}}%
\pgfpathlineto{\pgfqpoint{2.591535in}{1.798986in}}%
\pgfpathlineto{\pgfqpoint{2.591942in}{1.848355in}}%
\pgfpathlineto{\pgfqpoint{2.592146in}{1.840127in}}%
\pgfpathlineto{\pgfqpoint{2.592350in}{1.864811in}}%
\pgfpathlineto{\pgfqpoint{2.592554in}{1.881268in}}%
\pgfpathlineto{\pgfqpoint{2.592757in}{1.831899in}}%
\pgfpathlineto{\pgfqpoint{2.593369in}{1.609737in}}%
\pgfpathlineto{\pgfqpoint{2.593573in}{1.848355in}}%
\pgfpathlineto{\pgfqpoint{2.593776in}{1.831899in}}%
\pgfpathlineto{\pgfqpoint{2.593980in}{1.815442in}}%
\pgfpathlineto{\pgfqpoint{2.594184in}{1.856583in}}%
\pgfpathlineto{\pgfqpoint{2.595611in}{1.980006in}}%
\pgfpathlineto{\pgfqpoint{2.596426in}{1.914181in}}%
\pgfpathlineto{\pgfqpoint{2.596834in}{1.922409in}}%
\pgfpathlineto{\pgfqpoint{2.598872in}{1.749617in}}%
\pgfpathlineto{\pgfqpoint{2.599075in}{1.823670in}}%
\pgfpathlineto{\pgfqpoint{2.599279in}{1.634422in}}%
\pgfpathlineto{\pgfqpoint{2.599891in}{1.930637in}}%
\pgfpathlineto{\pgfqpoint{2.600094in}{1.905952in}}%
\pgfpathlineto{\pgfqpoint{2.600502in}{1.922409in}}%
\pgfpathlineto{\pgfqpoint{2.600910in}{1.856583in}}%
\pgfpathlineto{\pgfqpoint{2.601113in}{1.914181in}}%
\pgfpathlineto{\pgfqpoint{2.601521in}{1.815442in}}%
\pgfpathlineto{\pgfqpoint{2.601929in}{1.864811in}}%
\pgfpathlineto{\pgfqpoint{2.603151in}{1.774301in}}%
\pgfpathlineto{\pgfqpoint{2.603355in}{1.881268in}}%
\pgfpathlineto{\pgfqpoint{2.604374in}{1.848355in}}%
\pgfpathlineto{\pgfqpoint{2.604578in}{1.692019in}}%
\pgfpathlineto{\pgfqpoint{2.605189in}{1.815442in}}%
\pgfpathlineto{\pgfqpoint{2.605393in}{1.980006in}}%
\pgfpathlineto{\pgfqpoint{2.606208in}{1.864811in}}%
\pgfpathlineto{\pgfqpoint{2.607431in}{2.012919in}}%
\pgfpathlineto{\pgfqpoint{2.607839in}{1.955322in}}%
\pgfpathlineto{\pgfqpoint{2.609062in}{1.864811in}}%
\pgfpathlineto{\pgfqpoint{2.609469in}{1.856583in}}%
\pgfpathlineto{\pgfqpoint{2.609877in}{1.905952in}}%
\pgfpathlineto{\pgfqpoint{2.610081in}{1.790758in}}%
\pgfpathlineto{\pgfqpoint{2.611100in}{1.831899in}}%
\pgfpathlineto{\pgfqpoint{2.611304in}{1.930637in}}%
\pgfpathlineto{\pgfqpoint{2.611915in}{1.741388in}}%
\pgfpathlineto{\pgfqpoint{2.612119in}{1.692019in}}%
\pgfpathlineto{\pgfqpoint{2.612934in}{1.716704in}}%
\pgfpathlineto{\pgfqpoint{2.613749in}{1.757845in}}%
\pgfpathlineto{\pgfqpoint{2.614361in}{1.404032in}}%
\pgfpathlineto{\pgfqpoint{2.614972in}{1.601509in}}%
\pgfpathlineto{\pgfqpoint{2.616399in}{1.807214in}}%
\pgfpathlineto{\pgfqpoint{2.616806in}{1.864811in}}%
\pgfpathlineto{\pgfqpoint{2.617214in}{1.766073in}}%
\pgfpathlineto{\pgfqpoint{2.618233in}{1.708476in}}%
\pgfpathlineto{\pgfqpoint{2.618437in}{1.716704in}}%
\pgfpathlineto{\pgfqpoint{2.618640in}{1.560368in}}%
\pgfpathlineto{\pgfqpoint{2.618844in}{1.757845in}}%
\pgfpathlineto{\pgfqpoint{2.619456in}{1.692019in}}%
\pgfpathlineto{\pgfqpoint{2.619863in}{1.576824in}}%
\pgfpathlineto{\pgfqpoint{2.620271in}{1.626194in}}%
\pgfpathlineto{\pgfqpoint{2.620475in}{1.346435in}}%
\pgfpathlineto{\pgfqpoint{2.621086in}{1.708476in}}%
\pgfpathlineto{\pgfqpoint{2.621290in}{1.757845in}}%
\pgfpathlineto{\pgfqpoint{2.621494in}{1.659106in}}%
\pgfpathlineto{\pgfqpoint{2.621901in}{1.667335in}}%
\pgfpathlineto{\pgfqpoint{2.622105in}{1.675563in}}%
\pgfpathlineto{\pgfqpoint{2.622513in}{1.659106in}}%
\pgfpathlineto{\pgfqpoint{2.622717in}{1.650878in}}%
\pgfpathlineto{\pgfqpoint{2.622920in}{1.667335in}}%
\pgfpathlineto{\pgfqpoint{2.623124in}{1.659106in}}%
\pgfpathlineto{\pgfqpoint{2.623939in}{1.387576in}}%
\pgfpathlineto{\pgfqpoint{2.624143in}{1.453401in}}%
\pgfpathlineto{\pgfqpoint{2.625366in}{1.733160in}}%
\pgfpathlineto{\pgfqpoint{2.625570in}{1.724932in}}%
\pgfpathlineto{\pgfqpoint{2.625774in}{1.757845in}}%
\pgfpathlineto{\pgfqpoint{2.625977in}{1.757845in}}%
\pgfpathlineto{\pgfqpoint{2.626589in}{1.733160in}}%
\pgfpathlineto{\pgfqpoint{2.626996in}{1.757845in}}%
\pgfpathlineto{\pgfqpoint{2.627812in}{1.840127in}}%
\pgfpathlineto{\pgfqpoint{2.628015in}{1.716704in}}%
\pgfpathlineto{\pgfqpoint{2.628831in}{1.864811in}}%
\pgfpathlineto{\pgfqpoint{2.629850in}{1.873040in}}%
\pgfpathlineto{\pgfqpoint{2.630053in}{1.741388in}}%
\pgfpathlineto{\pgfqpoint{2.630461in}{1.930637in}}%
\pgfpathlineto{\pgfqpoint{2.630665in}{1.716704in}}%
\pgfpathlineto{\pgfqpoint{2.631072in}{1.823670in}}%
\pgfpathlineto{\pgfqpoint{2.632499in}{1.642650in}}%
\pgfpathlineto{\pgfqpoint{2.632703in}{1.692019in}}%
\pgfpathlineto{\pgfqpoint{2.633110in}{1.757845in}}%
\pgfpathlineto{\pgfqpoint{2.633518in}{1.675563in}}%
\pgfpathlineto{\pgfqpoint{2.633722in}{1.708476in}}%
\pgfpathlineto{\pgfqpoint{2.634741in}{1.593281in}}%
\pgfpathlineto{\pgfqpoint{2.635352in}{1.617965in}}%
\pgfpathlineto{\pgfqpoint{2.636371in}{1.585053in}}%
\pgfpathlineto{\pgfqpoint{2.636575in}{1.593281in}}%
\pgfpathlineto{\pgfqpoint{2.637390in}{1.552140in}}%
\pgfpathlineto{\pgfqpoint{2.637798in}{1.675563in}}%
\pgfpathlineto{\pgfqpoint{2.638002in}{1.593281in}}%
\pgfpathlineto{\pgfqpoint{2.639021in}{1.617965in}}%
\pgfpathlineto{\pgfqpoint{2.639632in}{1.601509in}}%
\pgfpathlineto{\pgfqpoint{2.640244in}{1.692019in}}%
\pgfpathlineto{\pgfqpoint{2.640447in}{1.659106in}}%
\pgfpathlineto{\pgfqpoint{2.640651in}{1.692019in}}%
\pgfpathlineto{\pgfqpoint{2.640855in}{1.782529in}}%
\pgfpathlineto{\pgfqpoint{2.641466in}{1.733160in}}%
\pgfpathlineto{\pgfqpoint{2.641874in}{1.634422in}}%
\pgfpathlineto{\pgfqpoint{2.642689in}{1.642650in}}%
\pgfpathlineto{\pgfqpoint{2.644116in}{1.535683in}}%
\pgfpathlineto{\pgfqpoint{2.645339in}{1.601509in}}%
\pgfpathlineto{\pgfqpoint{2.645542in}{1.601509in}}%
\pgfpathlineto{\pgfqpoint{2.646154in}{1.675563in}}%
\pgfpathlineto{\pgfqpoint{2.646765in}{1.634422in}}%
\pgfpathlineto{\pgfqpoint{2.647580in}{1.568596in}}%
\pgfpathlineto{\pgfqpoint{2.647784in}{1.642650in}}%
\pgfpathlineto{\pgfqpoint{2.648192in}{1.650878in}}%
\pgfpathlineto{\pgfqpoint{2.648600in}{1.568596in}}%
\pgfpathlineto{\pgfqpoint{2.649822in}{1.757845in}}%
\pgfpathlineto{\pgfqpoint{2.651453in}{1.206555in}}%
\pgfpathlineto{\pgfqpoint{2.651860in}{1.231240in}}%
\pgfpathlineto{\pgfqpoint{2.652472in}{1.453401in}}%
\pgfpathlineto{\pgfqpoint{2.652879in}{1.338207in}}%
\pgfpathlineto{\pgfqpoint{2.653287in}{1.280609in}}%
\pgfpathlineto{\pgfqpoint{2.653695in}{1.371119in}}%
\pgfpathlineto{\pgfqpoint{2.654510in}{1.502771in}}%
\pgfpathlineto{\pgfqpoint{2.654917in}{1.469858in}}%
\pgfpathlineto{\pgfqpoint{2.655325in}{1.305294in}}%
\pgfpathlineto{\pgfqpoint{2.655936in}{1.354663in}}%
\pgfpathlineto{\pgfqpoint{2.656140in}{1.461630in}}%
\pgfpathlineto{\pgfqpoint{2.656752in}{1.223012in}}%
\pgfpathlineto{\pgfqpoint{2.657567in}{1.255925in}}%
\pgfpathlineto{\pgfqpoint{2.657974in}{1.091361in}}%
\pgfpathlineto{\pgfqpoint{2.658382in}{1.247696in}}%
\pgfpathlineto{\pgfqpoint{2.658993in}{1.074904in}}%
\pgfpathlineto{\pgfqpoint{2.659197in}{1.173643in}}%
\pgfpathlineto{\pgfqpoint{2.660216in}{1.091361in}}%
\pgfpathlineto{\pgfqpoint{2.660420in}{1.132502in}}%
\pgfpathlineto{\pgfqpoint{2.660624in}{1.223012in}}%
\pgfpathlineto{\pgfqpoint{2.661439in}{1.165414in}}%
\pgfpathlineto{\pgfqpoint{2.661847in}{1.157186in}}%
\pgfpathlineto{\pgfqpoint{2.662051in}{1.190099in}}%
\pgfpathlineto{\pgfqpoint{2.662662in}{1.297066in}}%
\pgfpathlineto{\pgfqpoint{2.662866in}{1.239468in}}%
\pgfpathlineto{\pgfqpoint{2.664089in}{1.083132in}}%
\pgfpathlineto{\pgfqpoint{2.664292in}{1.091361in}}%
\pgfpathlineto{\pgfqpoint{2.664700in}{0.992622in}}%
\pgfpathlineto{\pgfqpoint{2.664904in}{1.066676in}}%
\pgfpathlineto{\pgfqpoint{2.665311in}{1.190099in}}%
\pgfpathlineto{\pgfqpoint{2.665719in}{1.033763in}}%
\pgfpathlineto{\pgfqpoint{2.665923in}{0.984394in}}%
\pgfpathlineto{\pgfqpoint{2.666330in}{1.074904in}}%
\pgfpathlineto{\pgfqpoint{2.666534in}{1.157186in}}%
\pgfpathlineto{\pgfqpoint{2.666942in}{1.058448in}}%
\pgfpathlineto{\pgfqpoint{2.667349in}{1.091361in}}%
\pgfpathlineto{\pgfqpoint{2.667757in}{1.041991in}}%
\pgfpathlineto{\pgfqpoint{2.667961in}{1.083132in}}%
\pgfpathlineto{\pgfqpoint{2.668165in}{1.009079in}}%
\pgfpathlineto{\pgfqpoint{2.668572in}{1.132502in}}%
\pgfpathlineto{\pgfqpoint{2.668776in}{1.107817in}}%
\pgfpathlineto{\pgfqpoint{2.669387in}{1.214784in}}%
\pgfpathlineto{\pgfqpoint{2.669999in}{1.157186in}}%
\pgfpathlineto{\pgfqpoint{2.670406in}{1.173643in}}%
\pgfpathlineto{\pgfqpoint{2.670814in}{1.132502in}}%
\pgfpathlineto{\pgfqpoint{2.671425in}{1.148958in}}%
\pgfpathlineto{\pgfqpoint{2.671629in}{1.157186in}}%
\pgfpathlineto{\pgfqpoint{2.671833in}{1.099589in}}%
\pgfpathlineto{\pgfqpoint{2.672444in}{1.165414in}}%
\pgfpathlineto{\pgfqpoint{2.672648in}{1.148958in}}%
\pgfpathlineto{\pgfqpoint{2.675502in}{1.436945in}}%
\pgfpathlineto{\pgfqpoint{2.675909in}{1.412260in}}%
\pgfpathlineto{\pgfqpoint{2.676317in}{1.280609in}}%
\pgfpathlineto{\pgfqpoint{2.677132in}{1.305294in}}%
\pgfpathlineto{\pgfqpoint{2.677336in}{1.371119in}}%
\pgfpathlineto{\pgfqpoint{2.677743in}{1.239468in}}%
\pgfpathlineto{\pgfqpoint{2.677947in}{1.239468in}}%
\pgfpathlineto{\pgfqpoint{2.679781in}{1.519227in}}%
\pgfpathlineto{\pgfqpoint{2.678355in}{1.223012in}}%
\pgfpathlineto{\pgfqpoint{2.680189in}{1.379348in}}%
\pgfpathlineto{\pgfqpoint{2.681412in}{1.247696in}}%
\pgfpathlineto{\pgfqpoint{2.683042in}{1.420489in}}%
\pgfpathlineto{\pgfqpoint{2.684469in}{1.132502in}}%
\pgfpathlineto{\pgfqpoint{2.684876in}{1.140730in}}%
\pgfpathlineto{\pgfqpoint{2.685284in}{1.107817in}}%
\pgfpathlineto{\pgfqpoint{2.686099in}{1.239468in}}%
\pgfpathlineto{\pgfqpoint{2.686303in}{1.181871in}}%
\pgfpathlineto{\pgfqpoint{2.686507in}{1.132502in}}%
\pgfpathlineto{\pgfqpoint{2.686711in}{1.297066in}}%
\pgfpathlineto{\pgfqpoint{2.687322in}{1.469858in}}%
\pgfpathlineto{\pgfqpoint{2.687730in}{1.280609in}}%
\pgfpathlineto{\pgfqpoint{2.687933in}{1.280609in}}%
\pgfpathlineto{\pgfqpoint{2.688545in}{1.354663in}}%
\pgfpathlineto{\pgfqpoint{2.688953in}{1.288837in}}%
\pgfpathlineto{\pgfqpoint{2.689768in}{1.321750in}}%
\pgfpathlineto{\pgfqpoint{2.690175in}{1.198327in}}%
\pgfpathlineto{\pgfqpoint{2.690379in}{1.198327in}}%
\pgfpathlineto{\pgfqpoint{2.690583in}{1.000850in}}%
\pgfpathlineto{\pgfqpoint{2.690787in}{1.206555in}}%
\pgfpathlineto{\pgfqpoint{2.691602in}{1.074904in}}%
\pgfpathlineto{\pgfqpoint{2.692213in}{1.313522in}}%
\pgfpathlineto{\pgfqpoint{2.692825in}{1.181871in}}%
\pgfpathlineto{\pgfqpoint{2.693844in}{1.107817in}}%
\pgfpathlineto{\pgfqpoint{2.695067in}{1.264153in}}%
\pgfpathlineto{\pgfqpoint{2.695270in}{1.247696in}}%
\pgfpathlineto{\pgfqpoint{2.695882in}{1.255925in}}%
\pgfpathlineto{\pgfqpoint{2.697920in}{1.585053in}}%
\pgfpathlineto{\pgfqpoint{2.698124in}{1.593281in}}%
\pgfpathlineto{\pgfqpoint{2.698327in}{1.560368in}}%
\pgfpathlineto{\pgfqpoint{2.699754in}{1.420489in}}%
\pgfpathlineto{\pgfqpoint{2.699958in}{1.428717in}}%
\pgfpathlineto{\pgfqpoint{2.701996in}{1.585053in}}%
\pgfpathlineto{\pgfqpoint{2.700365in}{1.420489in}}%
\pgfpathlineto{\pgfqpoint{2.702200in}{1.560368in}}%
\pgfpathlineto{\pgfqpoint{2.702404in}{1.535683in}}%
\pgfpathlineto{\pgfqpoint{2.702811in}{1.593281in}}%
\pgfpathlineto{\pgfqpoint{2.703015in}{1.576824in}}%
\pgfpathlineto{\pgfqpoint{2.703219in}{1.626194in}}%
\pgfpathlineto{\pgfqpoint{2.703626in}{1.535683in}}%
\pgfpathlineto{\pgfqpoint{2.703830in}{1.552140in}}%
\pgfpathlineto{\pgfqpoint{2.704034in}{1.527455in}}%
\pgfpathlineto{\pgfqpoint{2.704238in}{1.552140in}}%
\pgfpathlineto{\pgfqpoint{2.704849in}{1.659106in}}%
\pgfpathlineto{\pgfqpoint{2.705461in}{1.642650in}}%
\pgfpathlineto{\pgfqpoint{2.705664in}{1.650878in}}%
\pgfpathlineto{\pgfqpoint{2.705868in}{1.642650in}}%
\pgfpathlineto{\pgfqpoint{2.706276in}{1.601509in}}%
\pgfpathlineto{\pgfqpoint{2.706480in}{1.692019in}}%
\pgfpathlineto{\pgfqpoint{2.706683in}{1.741388in}}%
\pgfpathlineto{\pgfqpoint{2.706887in}{1.642650in}}%
\pgfpathlineto{\pgfqpoint{2.707091in}{1.642650in}}%
\pgfpathlineto{\pgfqpoint{2.707499in}{1.601509in}}%
\pgfpathlineto{\pgfqpoint{2.707906in}{1.642650in}}%
\pgfpathlineto{\pgfqpoint{2.709129in}{1.716704in}}%
\pgfpathlineto{\pgfqpoint{2.709333in}{1.675563in}}%
\pgfpathlineto{\pgfqpoint{2.709537in}{1.650878in}}%
\pgfpathlineto{\pgfqpoint{2.710148in}{1.708476in}}%
\pgfpathlineto{\pgfqpoint{2.711167in}{1.757845in}}%
\pgfpathlineto{\pgfqpoint{2.711371in}{1.749617in}}%
\pgfpathlineto{\pgfqpoint{2.711778in}{1.807214in}}%
\pgfpathlineto{\pgfqpoint{2.712186in}{1.782529in}}%
\pgfpathlineto{\pgfqpoint{2.712390in}{1.708476in}}%
\pgfpathlineto{\pgfqpoint{2.713001in}{1.881268in}}%
\pgfpathlineto{\pgfqpoint{2.713409in}{1.889496in}}%
\pgfpathlineto{\pgfqpoint{2.713613in}{1.856583in}}%
\pgfpathlineto{\pgfqpoint{2.714428in}{1.938865in}}%
\pgfpathlineto{\pgfqpoint{2.714632in}{1.889496in}}%
\pgfpathlineto{\pgfqpoint{2.715243in}{1.930637in}}%
\pgfpathlineto{\pgfqpoint{2.715651in}{1.864811in}}%
\pgfpathlineto{\pgfqpoint{2.716874in}{2.029375in}}%
\pgfpathlineto{\pgfqpoint{2.717689in}{1.980006in}}%
\pgfpathlineto{\pgfqpoint{2.717281in}{2.037604in}}%
\pgfpathlineto{\pgfqpoint{2.717893in}{2.004691in}}%
\pgfpathlineto{\pgfqpoint{2.719727in}{2.136342in}}%
\pgfpathlineto{\pgfqpoint{2.719931in}{2.128114in}}%
\pgfpathlineto{\pgfqpoint{2.720542in}{2.103429in}}%
\pgfpathlineto{\pgfqpoint{2.720338in}{2.185711in}}%
\pgfpathlineto{\pgfqpoint{2.720746in}{2.161027in}}%
\pgfpathlineto{\pgfqpoint{2.721561in}{2.111657in}}%
\pgfpathlineto{\pgfqpoint{2.722784in}{2.243309in}}%
\pgfpathlineto{\pgfqpoint{2.723191in}{2.210396in}}%
\pgfpathlineto{\pgfqpoint{2.724210in}{2.086973in}}%
\pgfpathlineto{\pgfqpoint{2.725229in}{2.185711in}}%
\pgfpathlineto{\pgfqpoint{2.725433in}{2.169255in}}%
\pgfpathlineto{\pgfqpoint{2.725841in}{2.136342in}}%
\pgfpathlineto{\pgfqpoint{2.726045in}{2.185711in}}%
\pgfpathlineto{\pgfqpoint{2.726452in}{2.169255in}}%
\pgfpathlineto{\pgfqpoint{2.726656in}{2.243309in}}%
\pgfpathlineto{\pgfqpoint{2.727267in}{2.202168in}}%
\pgfpathlineto{\pgfqpoint{2.727471in}{2.119886in}}%
\pgfpathlineto{\pgfqpoint{2.727675in}{2.267993in}}%
\pgfpathlineto{\pgfqpoint{2.728490in}{2.136342in}}%
\pgfpathlineto{\pgfqpoint{2.728694in}{2.144570in}}%
\pgfpathlineto{\pgfqpoint{2.728898in}{2.136342in}}%
\pgfpathlineto{\pgfqpoint{2.729102in}{2.111657in}}%
\pgfpathlineto{\pgfqpoint{2.729509in}{2.152798in}}%
\pgfpathlineto{\pgfqpoint{2.729917in}{2.144570in}}%
\pgfpathlineto{\pgfqpoint{2.730121in}{2.210396in}}%
\pgfpathlineto{\pgfqpoint{2.730936in}{2.136342in}}%
\pgfpathlineto{\pgfqpoint{2.731140in}{2.128114in}}%
\pgfpathlineto{\pgfqpoint{2.732159in}{2.210396in}}%
\pgfpathlineto{\pgfqpoint{2.732363in}{1.988234in}}%
\pgfpathlineto{\pgfqpoint{2.733178in}{2.152798in}}%
\pgfpathlineto{\pgfqpoint{2.733382in}{2.152798in}}%
\pgfpathlineto{\pgfqpoint{2.733585in}{2.119886in}}%
\pgfpathlineto{\pgfqpoint{2.734197in}{2.152798in}}%
\pgfpathlineto{\pgfqpoint{2.735623in}{2.317363in}}%
\pgfpathlineto{\pgfqpoint{2.735827in}{2.284450in}}%
\pgfpathlineto{\pgfqpoint{2.736439in}{2.251537in}}%
\pgfpathlineto{\pgfqpoint{2.736642in}{2.284450in}}%
\pgfpathlineto{\pgfqpoint{2.737050in}{2.317363in}}%
\pgfpathlineto{\pgfqpoint{2.737254in}{2.226852in}}%
\pgfpathlineto{\pgfqpoint{2.738069in}{2.325591in}}%
\pgfpathlineto{\pgfqpoint{2.738273in}{2.235080in}}%
\pgfpathlineto{\pgfqpoint{2.739903in}{2.358504in}}%
\pgfpathlineto{\pgfqpoint{2.740311in}{2.374960in}}%
\pgfpathlineto{\pgfqpoint{2.740515in}{2.366732in}}%
\pgfpathlineto{\pgfqpoint{2.740922in}{2.144570in}}%
\pgfpathlineto{\pgfqpoint{2.741737in}{2.276221in}}%
\pgfpathlineto{\pgfqpoint{2.741941in}{2.309134in}}%
\pgfpathlineto{\pgfqpoint{2.742349in}{2.267993in}}%
\pgfpathlineto{\pgfqpoint{2.742757in}{1.980006in}}%
\pgfpathlineto{\pgfqpoint{2.743368in}{2.210396in}}%
\pgfpathlineto{\pgfqpoint{2.744183in}{2.342047in}}%
\pgfpathlineto{\pgfqpoint{2.744387in}{2.267993in}}%
\pgfpathlineto{\pgfqpoint{2.744795in}{2.086973in}}%
\pgfpathlineto{\pgfqpoint{2.745610in}{2.218624in}}%
\pgfpathlineto{\pgfqpoint{2.746833in}{2.358504in}}%
\pgfpathlineto{\pgfqpoint{2.748055in}{2.226852in}}%
\pgfpathlineto{\pgfqpoint{2.748667in}{2.325591in}}%
\pgfpathlineto{\pgfqpoint{2.749278in}{2.309134in}}%
\pgfpathlineto{\pgfqpoint{2.749890in}{2.226852in}}%
\pgfpathlineto{\pgfqpoint{2.750501in}{2.243309in}}%
\pgfpathlineto{\pgfqpoint{2.751112in}{2.342047in}}%
\pgfpathlineto{\pgfqpoint{2.751520in}{2.226852in}}%
\pgfpathlineto{\pgfqpoint{2.752131in}{2.300906in}}%
\pgfpathlineto{\pgfqpoint{2.752335in}{2.169255in}}%
\pgfpathlineto{\pgfqpoint{2.752947in}{2.317363in}}%
\pgfpathlineto{\pgfqpoint{2.753150in}{2.292678in}}%
\pgfpathlineto{\pgfqpoint{2.753558in}{2.292678in}}%
\pgfpathlineto{\pgfqpoint{2.754373in}{2.358504in}}%
\pgfpathlineto{\pgfqpoint{2.754985in}{2.333819in}}%
\pgfpathlineto{\pgfqpoint{2.755596in}{2.276221in}}%
\pgfpathlineto{\pgfqpoint{2.755800in}{2.317363in}}%
\pgfpathlineto{\pgfqpoint{2.756411in}{2.300906in}}%
\pgfpathlineto{\pgfqpoint{2.757023in}{2.399645in}}%
\pgfpathlineto{\pgfqpoint{2.758653in}{2.259765in}}%
\pgfpathlineto{\pgfqpoint{2.759468in}{2.358504in}}%
\pgfpathlineto{\pgfqpoint{2.759672in}{2.259765in}}%
\pgfpathlineto{\pgfqpoint{2.759876in}{2.267993in}}%
\pgfpathlineto{\pgfqpoint{2.760080in}{2.259765in}}%
\pgfpathlineto{\pgfqpoint{2.760284in}{2.235080in}}%
\pgfpathlineto{\pgfqpoint{2.760691in}{2.309134in}}%
\pgfpathlineto{\pgfqpoint{2.761303in}{2.350275in}}%
\pgfpathlineto{\pgfqpoint{2.761506in}{2.292678in}}%
\pgfpathlineto{\pgfqpoint{2.761710in}{2.317363in}}%
\pgfpathlineto{\pgfqpoint{2.761914in}{2.276221in}}%
\pgfpathlineto{\pgfqpoint{2.762322in}{2.366732in}}%
\pgfpathlineto{\pgfqpoint{2.762729in}{2.333819in}}%
\pgfpathlineto{\pgfqpoint{2.762933in}{2.383188in}}%
\pgfpathlineto{\pgfqpoint{2.763544in}{2.325591in}}%
\pgfpathlineto{\pgfqpoint{2.764156in}{2.276221in}}%
\pgfpathlineto{\pgfqpoint{2.764563in}{2.317363in}}%
\pgfpathlineto{\pgfqpoint{2.764767in}{2.342047in}}%
\pgfpathlineto{\pgfqpoint{2.764971in}{2.292678in}}%
\pgfpathlineto{\pgfqpoint{2.765175in}{2.300906in}}%
\pgfpathlineto{\pgfqpoint{2.766194in}{2.202168in}}%
\pgfpathlineto{\pgfqpoint{2.766805in}{2.218624in}}%
\pgfpathlineto{\pgfqpoint{2.767417in}{2.366732in}}%
\pgfpathlineto{\pgfqpoint{2.768232in}{2.325591in}}%
\pgfpathlineto{\pgfqpoint{2.769251in}{2.292678in}}%
\pgfpathlineto{\pgfqpoint{2.769455in}{2.300906in}}%
\pgfpathlineto{\pgfqpoint{2.770066in}{2.374960in}}%
\pgfpathlineto{\pgfqpoint{2.769862in}{2.292678in}}%
\pgfpathlineto{\pgfqpoint{2.770881in}{2.358504in}}%
\pgfpathlineto{\pgfqpoint{2.771085in}{2.358504in}}%
\pgfpathlineto{\pgfqpoint{2.771493in}{2.366732in}}%
\pgfpathlineto{\pgfqpoint{2.772512in}{2.276221in}}%
\pgfpathlineto{\pgfqpoint{2.772919in}{2.366732in}}%
\pgfpathlineto{\pgfqpoint{2.773531in}{2.350275in}}%
\pgfpathlineto{\pgfqpoint{2.774957in}{2.193939in}}%
\pgfpathlineto{\pgfqpoint{2.775161in}{2.226852in}}%
\pgfpathlineto{\pgfqpoint{2.775773in}{2.276221in}}%
\pgfpathlineto{\pgfqpoint{2.775569in}{2.210396in}}%
\pgfpathlineto{\pgfqpoint{2.775976in}{2.251537in}}%
\pgfpathlineto{\pgfqpoint{2.776792in}{2.202168in}}%
\pgfpathlineto{\pgfqpoint{2.776995in}{2.259765in}}%
\pgfpathlineto{\pgfqpoint{2.777607in}{2.292678in}}%
\pgfpathlineto{\pgfqpoint{2.777403in}{2.251537in}}%
\pgfpathlineto{\pgfqpoint{2.778014in}{2.259765in}}%
\pgfpathlineto{\pgfqpoint{2.778218in}{2.259765in}}%
\pgfpathlineto{\pgfqpoint{2.778422in}{2.210396in}}%
\pgfpathlineto{\pgfqpoint{2.779237in}{2.276221in}}%
\pgfpathlineto{\pgfqpoint{2.779441in}{2.267993in}}%
\pgfpathlineto{\pgfqpoint{2.779645in}{2.193939in}}%
\pgfpathlineto{\pgfqpoint{2.780256in}{2.243309in}}%
\pgfpathlineto{\pgfqpoint{2.780460in}{2.317363in}}%
\pgfpathlineto{\pgfqpoint{2.781275in}{2.226852in}}%
\pgfpathlineto{\pgfqpoint{2.781479in}{2.284450in}}%
\pgfpathlineto{\pgfqpoint{2.782498in}{2.226852in}}%
\pgfpathlineto{\pgfqpoint{2.782702in}{2.251537in}}%
\pgfpathlineto{\pgfqpoint{2.783925in}{2.399645in}}%
\pgfpathlineto{\pgfqpoint{2.784536in}{2.383188in}}%
\pgfpathlineto{\pgfqpoint{2.785351in}{2.259765in}}%
\pgfpathlineto{\pgfqpoint{2.785555in}{2.300906in}}%
\pgfpathlineto{\pgfqpoint{2.786167in}{2.366732in}}%
\pgfpathlineto{\pgfqpoint{2.786370in}{2.292678in}}%
\pgfpathlineto{\pgfqpoint{2.786778in}{2.276221in}}%
\pgfpathlineto{\pgfqpoint{2.786982in}{2.317363in}}%
\pgfpathlineto{\pgfqpoint{2.787186in}{2.309134in}}%
\pgfpathlineto{\pgfqpoint{2.788001in}{2.399645in}}%
\pgfpathlineto{\pgfqpoint{2.788205in}{2.350275in}}%
\pgfpathlineto{\pgfqpoint{2.788816in}{2.276221in}}%
\pgfpathlineto{\pgfqpoint{2.789224in}{2.325591in}}%
\pgfpathlineto{\pgfqpoint{2.790243in}{2.366732in}}%
\pgfpathlineto{\pgfqpoint{2.791873in}{2.103429in}}%
\pgfpathlineto{\pgfqpoint{2.793096in}{2.407873in}}%
\pgfpathlineto{\pgfqpoint{2.794319in}{2.292678in}}%
\pgfpathlineto{\pgfqpoint{2.794522in}{2.383188in}}%
\pgfpathlineto{\pgfqpoint{2.795338in}{2.374960in}}%
\pgfpathlineto{\pgfqpoint{2.795541in}{2.325591in}}%
\pgfpathlineto{\pgfqpoint{2.796357in}{2.350275in}}%
\pgfpathlineto{\pgfqpoint{2.796968in}{2.399645in}}%
\pgfpathlineto{\pgfqpoint{2.797376in}{2.391416in}}%
\pgfpathlineto{\pgfqpoint{2.798191in}{2.276221in}}%
\pgfpathlineto{\pgfqpoint{2.797987in}{2.399645in}}%
\pgfpathlineto{\pgfqpoint{2.798395in}{2.391416in}}%
\pgfpathlineto{\pgfqpoint{2.798599in}{2.383188in}}%
\pgfpathlineto{\pgfqpoint{2.798802in}{2.449014in}}%
\pgfpathlineto{\pgfqpoint{2.799210in}{2.374960in}}%
\pgfpathlineto{\pgfqpoint{2.799414in}{2.374960in}}%
\pgfpathlineto{\pgfqpoint{2.800025in}{2.424329in}}%
\pgfpathlineto{\pgfqpoint{2.800637in}{2.325591in}}%
\pgfpathlineto{\pgfqpoint{2.801656in}{2.383188in}}%
\pgfpathlineto{\pgfqpoint{2.802878in}{2.119886in}}%
\pgfpathlineto{\pgfqpoint{2.802267in}{2.440786in}}%
\pgfpathlineto{\pgfqpoint{2.803082in}{2.317363in}}%
\pgfpathlineto{\pgfqpoint{2.803490in}{2.325591in}}%
\pgfpathlineto{\pgfqpoint{2.803694in}{2.300906in}}%
\pgfpathlineto{\pgfqpoint{2.803897in}{2.136342in}}%
\pgfpathlineto{\pgfqpoint{2.804713in}{2.325591in}}%
\pgfpathlineto{\pgfqpoint{2.805120in}{2.391416in}}%
\pgfpathlineto{\pgfqpoint{2.805528in}{2.309134in}}%
\pgfpathlineto{\pgfqpoint{2.805732in}{2.350275in}}%
\pgfpathlineto{\pgfqpoint{2.807158in}{2.251537in}}%
\pgfpathlineto{\pgfqpoint{2.807362in}{2.276221in}}%
\pgfpathlineto{\pgfqpoint{2.808585in}{2.440786in}}%
\pgfpathlineto{\pgfqpoint{2.809400in}{2.325591in}}%
\pgfpathlineto{\pgfqpoint{2.809808in}{2.350275in}}%
\pgfpathlineto{\pgfqpoint{2.810012in}{2.366732in}}%
\pgfpathlineto{\pgfqpoint{2.810215in}{2.325591in}}%
\pgfpathlineto{\pgfqpoint{2.810419in}{2.325591in}}%
\pgfpathlineto{\pgfqpoint{2.811642in}{2.095201in}}%
\pgfpathlineto{\pgfqpoint{2.810827in}{2.366732in}}%
\pgfpathlineto{\pgfqpoint{2.811846in}{2.226852in}}%
\pgfpathlineto{\pgfqpoint{2.812253in}{2.399645in}}%
\pgfpathlineto{\pgfqpoint{2.812865in}{2.251537in}}%
\pgfpathlineto{\pgfqpoint{2.813476in}{2.416101in}}%
\pgfpathlineto{\pgfqpoint{2.814088in}{2.325591in}}%
\pgfpathlineto{\pgfqpoint{2.814291in}{2.309134in}}%
\pgfpathlineto{\pgfqpoint{2.814495in}{2.358504in}}%
\pgfpathlineto{\pgfqpoint{2.814699in}{2.342047in}}%
\pgfpathlineto{\pgfqpoint{2.815310in}{2.317363in}}%
\pgfpathlineto{\pgfqpoint{2.815514in}{2.358504in}}%
\pgfpathlineto{\pgfqpoint{2.816941in}{2.202168in}}%
\pgfpathlineto{\pgfqpoint{2.817552in}{2.490155in}}%
\pgfpathlineto{\pgfqpoint{2.818367in}{2.350275in}}%
\pgfpathlineto{\pgfqpoint{2.818571in}{2.358504in}}%
\pgfpathlineto{\pgfqpoint{2.818775in}{2.128114in}}%
\pgfpathlineto{\pgfqpoint{2.819183in}{2.432557in}}%
\pgfpathlineto{\pgfqpoint{2.819590in}{2.399645in}}%
\pgfpathlineto{\pgfqpoint{2.820202in}{2.481927in}}%
\pgfpathlineto{\pgfqpoint{2.820609in}{2.449014in}}%
\pgfpathlineto{\pgfqpoint{2.821221in}{2.136342in}}%
\pgfpathlineto{\pgfqpoint{2.821628in}{2.333819in}}%
\pgfpathlineto{\pgfqpoint{2.822036in}{2.267993in}}%
\pgfpathlineto{\pgfqpoint{2.822443in}{2.309134in}}%
\pgfpathlineto{\pgfqpoint{2.823259in}{2.383188in}}%
\pgfpathlineto{\pgfqpoint{2.823666in}{2.374960in}}%
\pgfpathlineto{\pgfqpoint{2.824278in}{2.407873in}}%
\pgfpathlineto{\pgfqpoint{2.824482in}{2.259765in}}%
\pgfpathlineto{\pgfqpoint{2.825501in}{2.539524in}}%
\pgfpathlineto{\pgfqpoint{2.825704in}{2.449014in}}%
\pgfpathlineto{\pgfqpoint{2.826520in}{2.613578in}}%
\pgfpathlineto{\pgfqpoint{2.827335in}{2.572437in}}%
\pgfpathlineto{\pgfqpoint{2.827539in}{2.523068in}}%
\pgfpathlineto{\pgfqpoint{2.828558in}{2.531296in}}%
\pgfpathlineto{\pgfqpoint{2.828761in}{2.547752in}}%
\pgfpathlineto{\pgfqpoint{2.828965in}{2.514839in}}%
\pgfpathlineto{\pgfqpoint{2.829169in}{2.514839in}}%
\pgfpathlineto{\pgfqpoint{2.829373in}{2.440786in}}%
\pgfpathlineto{\pgfqpoint{2.829984in}{2.547752in}}%
\pgfpathlineto{\pgfqpoint{2.830392in}{2.449014in}}%
\pgfpathlineto{\pgfqpoint{2.830596in}{2.465470in}}%
\pgfpathlineto{\pgfqpoint{2.830799in}{2.391416in}}%
\pgfpathlineto{\pgfqpoint{2.831003in}{2.506611in}}%
\pgfpathlineto{\pgfqpoint{2.831615in}{2.449014in}}%
\pgfpathlineto{\pgfqpoint{2.831818in}{2.506611in}}%
\pgfpathlineto{\pgfqpoint{2.832022in}{2.276221in}}%
\pgfpathlineto{\pgfqpoint{2.832430in}{2.465470in}}%
\pgfpathlineto{\pgfqpoint{2.832634in}{2.399645in}}%
\pgfpathlineto{\pgfqpoint{2.833041in}{2.473698in}}%
\pgfpathlineto{\pgfqpoint{2.833653in}{2.407873in}}%
\pgfpathlineto{\pgfqpoint{2.835079in}{2.531296in}}%
\pgfpathlineto{\pgfqpoint{2.836302in}{2.407873in}}%
\pgfpathlineto{\pgfqpoint{2.836710in}{2.391416in}}%
\pgfpathlineto{\pgfqpoint{2.837525in}{2.457242in}}%
\pgfpathlineto{\pgfqpoint{2.837729in}{2.416101in}}%
\pgfpathlineto{\pgfqpoint{2.838136in}{2.490155in}}%
\pgfpathlineto{\pgfqpoint{2.838340in}{2.481927in}}%
\pgfpathlineto{\pgfqpoint{2.838544in}{2.490155in}}%
\pgfpathlineto{\pgfqpoint{2.838748in}{2.555980in}}%
\pgfpathlineto{\pgfqpoint{2.839359in}{2.449014in}}%
\pgfpathlineto{\pgfqpoint{2.839563in}{2.490155in}}%
\pgfpathlineto{\pgfqpoint{2.839767in}{2.424329in}}%
\pgfpathlineto{\pgfqpoint{2.840582in}{2.514839in}}%
\pgfpathlineto{\pgfqpoint{2.840786in}{2.572437in}}%
\pgfpathlineto{\pgfqpoint{2.841601in}{2.531296in}}%
\pgfpathlineto{\pgfqpoint{2.842009in}{2.580665in}}%
\pgfpathlineto{\pgfqpoint{2.842212in}{2.506611in}}%
\pgfpathlineto{\pgfqpoint{2.843028in}{2.555980in}}%
\pgfpathlineto{\pgfqpoint{2.843231in}{2.572437in}}%
\pgfpathlineto{\pgfqpoint{2.843435in}{2.523068in}}%
\pgfpathlineto{\pgfqpoint{2.843843in}{2.564209in}}%
\pgfpathlineto{\pgfqpoint{2.844047in}{2.523068in}}%
\pgfpathlineto{\pgfqpoint{2.844454in}{2.580665in}}%
\pgfpathlineto{\pgfqpoint{2.844862in}{2.555980in}}%
\pgfpathlineto{\pgfqpoint{2.845066in}{2.572437in}}%
\pgfpathlineto{\pgfqpoint{2.845269in}{2.539524in}}%
\pgfpathlineto{\pgfqpoint{2.845881in}{2.555980in}}%
\pgfpathlineto{\pgfqpoint{2.846492in}{2.481927in}}%
\pgfpathlineto{\pgfqpoint{2.847104in}{2.523068in}}%
\pgfpathlineto{\pgfqpoint{2.848123in}{2.473698in}}%
\pgfpathlineto{\pgfqpoint{2.848938in}{2.399645in}}%
\pgfpathlineto{\pgfqpoint{2.849142in}{2.449014in}}%
\pgfpathlineto{\pgfqpoint{2.849753in}{2.555980in}}%
\pgfpathlineto{\pgfqpoint{2.849957in}{2.440786in}}%
\pgfpathlineto{\pgfqpoint{2.850772in}{2.514839in}}%
\pgfpathlineto{\pgfqpoint{2.851587in}{2.440786in}}%
\pgfpathlineto{\pgfqpoint{2.851180in}{2.539524in}}%
\pgfpathlineto{\pgfqpoint{2.851995in}{2.457242in}}%
\pgfpathlineto{\pgfqpoint{2.852403in}{2.531296in}}%
\pgfpathlineto{\pgfqpoint{2.852606in}{2.424329in}}%
\pgfpathlineto{\pgfqpoint{2.852810in}{2.440786in}}%
\pgfpathlineto{\pgfqpoint{2.853422in}{2.473698in}}%
\pgfpathlineto{\pgfqpoint{2.854033in}{2.374960in}}%
\pgfpathlineto{\pgfqpoint{2.854848in}{2.407873in}}%
\pgfpathlineto{\pgfqpoint{2.855052in}{2.333819in}}%
\pgfpathlineto{\pgfqpoint{2.855663in}{2.498383in}}%
\pgfpathlineto{\pgfqpoint{2.855867in}{2.465470in}}%
\pgfpathlineto{\pgfqpoint{2.856275in}{2.531296in}}%
\pgfpathlineto{\pgfqpoint{2.856479in}{2.547752in}}%
\pgfpathlineto{\pgfqpoint{2.856886in}{2.276221in}}%
\pgfpathlineto{\pgfqpoint{2.857498in}{2.292678in}}%
\pgfpathlineto{\pgfqpoint{2.857701in}{2.539524in}}%
\pgfpathlineto{\pgfqpoint{2.858313in}{2.284450in}}%
\pgfpathlineto{\pgfqpoint{2.858517in}{2.284450in}}%
\pgfpathlineto{\pgfqpoint{2.860147in}{2.539524in}}%
\pgfpathlineto{\pgfqpoint{2.858924in}{2.251537in}}%
\pgfpathlineto{\pgfqpoint{2.860351in}{2.523068in}}%
\pgfpathlineto{\pgfqpoint{2.861370in}{2.416101in}}%
\pgfpathlineto{\pgfqpoint{2.861574in}{2.457242in}}%
\pgfpathlineto{\pgfqpoint{2.862185in}{2.473698in}}%
\pgfpathlineto{\pgfqpoint{2.862796in}{2.342047in}}%
\pgfpathlineto{\pgfqpoint{2.864019in}{2.514839in}}%
\pgfpathlineto{\pgfqpoint{2.865242in}{2.416101in}}%
\pgfpathlineto{\pgfqpoint{2.866057in}{2.506611in}}%
\pgfpathlineto{\pgfqpoint{2.866465in}{2.490155in}}%
\pgfpathlineto{\pgfqpoint{2.867076in}{2.473698in}}%
\pgfpathlineto{\pgfqpoint{2.867280in}{2.506611in}}%
\pgfpathlineto{\pgfqpoint{2.867484in}{2.407873in}}%
\pgfpathlineto{\pgfqpoint{2.868299in}{2.490155in}}%
\pgfpathlineto{\pgfqpoint{2.869114in}{2.523068in}}%
\pgfpathlineto{\pgfqpoint{2.869318in}{2.473698in}}%
\pgfpathlineto{\pgfqpoint{2.869522in}{2.564209in}}%
\pgfpathlineto{\pgfqpoint{2.870133in}{2.539524in}}%
\pgfpathlineto{\pgfqpoint{2.870337in}{2.564209in}}%
\pgfpathlineto{\pgfqpoint{2.870541in}{2.523068in}}%
\pgfpathlineto{\pgfqpoint{2.871152in}{2.276221in}}%
\pgfpathlineto{\pgfqpoint{2.871560in}{2.514839in}}%
\pgfpathlineto{\pgfqpoint{2.871968in}{2.473698in}}%
\pgfpathlineto{\pgfqpoint{2.872171in}{2.539524in}}%
\pgfpathlineto{\pgfqpoint{2.872375in}{2.514839in}}%
\pgfpathlineto{\pgfqpoint{2.873190in}{2.654719in}}%
\pgfpathlineto{\pgfqpoint{2.873394in}{2.621806in}}%
\pgfpathlineto{\pgfqpoint{2.874209in}{2.457242in}}%
\pgfpathlineto{\pgfqpoint{2.874413in}{2.498383in}}%
\pgfpathlineto{\pgfqpoint{2.874821in}{2.630034in}}%
\pgfpathlineto{\pgfqpoint{2.875636in}{2.588893in}}%
\pgfpathlineto{\pgfqpoint{2.876655in}{2.481927in}}%
\pgfpathlineto{\pgfqpoint{2.876859in}{2.514839in}}%
\pgfpathlineto{\pgfqpoint{2.877063in}{2.523068in}}%
\pgfpathlineto{\pgfqpoint{2.877267in}{2.514839in}}%
\pgfpathlineto{\pgfqpoint{2.877470in}{2.490155in}}%
\pgfpathlineto{\pgfqpoint{2.877878in}{2.547752in}}%
\pgfpathlineto{\pgfqpoint{2.878082in}{2.555980in}}%
\pgfpathlineto{\pgfqpoint{2.879101in}{2.449014in}}%
\pgfpathlineto{\pgfqpoint{2.879916in}{2.580665in}}%
\pgfpathlineto{\pgfqpoint{2.880324in}{2.523068in}}%
\pgfpathlineto{\pgfqpoint{2.881139in}{2.391416in}}%
\pgfpathlineto{\pgfqpoint{2.881954in}{2.432557in}}%
\pgfpathlineto{\pgfqpoint{2.882158in}{2.424329in}}%
\pgfpathlineto{\pgfqpoint{2.882362in}{2.449014in}}%
\pgfpathlineto{\pgfqpoint{2.882973in}{2.490155in}}%
\pgfpathlineto{\pgfqpoint{2.882769in}{2.432557in}}%
\pgfpathlineto{\pgfqpoint{2.883584in}{2.465470in}}%
\pgfpathlineto{\pgfqpoint{2.883788in}{2.424329in}}%
\pgfpathlineto{\pgfqpoint{2.883992in}{2.481927in}}%
\pgfpathlineto{\pgfqpoint{2.884400in}{2.465470in}}%
\pgfpathlineto{\pgfqpoint{2.885011in}{2.490155in}}%
\pgfpathlineto{\pgfqpoint{2.885215in}{2.465470in}}%
\pgfpathlineto{\pgfqpoint{2.885622in}{2.391416in}}%
\pgfpathlineto{\pgfqpoint{2.886234in}{2.440786in}}%
\pgfpathlineto{\pgfqpoint{2.886641in}{2.465470in}}%
\pgfpathlineto{\pgfqpoint{2.886845in}{2.449014in}}%
\pgfpathlineto{\pgfqpoint{2.887457in}{2.399645in}}%
\pgfpathlineto{\pgfqpoint{2.887864in}{2.424329in}}%
\pgfpathlineto{\pgfqpoint{2.888068in}{2.440786in}}%
\pgfpathlineto{\pgfqpoint{2.888272in}{2.391416in}}%
\pgfpathlineto{\pgfqpoint{2.888476in}{2.383188in}}%
\pgfpathlineto{\pgfqpoint{2.888679in}{2.416101in}}%
\pgfpathlineto{\pgfqpoint{2.888883in}{2.391416in}}%
\pgfpathlineto{\pgfqpoint{2.889902in}{2.490155in}}%
\pgfpathlineto{\pgfqpoint{2.890106in}{2.432557in}}%
\pgfpathlineto{\pgfqpoint{2.891737in}{2.539524in}}%
\pgfpathlineto{\pgfqpoint{2.892959in}{2.457242in}}%
\pgfpathlineto{\pgfqpoint{2.892756in}{2.555980in}}%
\pgfpathlineto{\pgfqpoint{2.893163in}{2.465470in}}%
\pgfpathlineto{\pgfqpoint{2.893367in}{2.465470in}}%
\pgfpathlineto{\pgfqpoint{2.894590in}{2.555980in}}%
\pgfpathlineto{\pgfqpoint{2.895813in}{2.358504in}}%
\pgfpathlineto{\pgfqpoint{2.897035in}{2.621806in}}%
\pgfpathlineto{\pgfqpoint{2.897239in}{2.539524in}}%
\pgfpathlineto{\pgfqpoint{2.897647in}{2.564209in}}%
\pgfpathlineto{\pgfqpoint{2.897851in}{2.531296in}}%
\pgfpathlineto{\pgfqpoint{2.899073in}{2.317363in}}%
\pgfpathlineto{\pgfqpoint{2.899277in}{2.325591in}}%
\pgfpathlineto{\pgfqpoint{2.899685in}{2.374960in}}%
\pgfpathlineto{\pgfqpoint{2.900296in}{2.342047in}}%
\pgfpathlineto{\pgfqpoint{2.901723in}{2.276221in}}%
\pgfpathlineto{\pgfqpoint{2.902334in}{2.399645in}}%
\pgfpathlineto{\pgfqpoint{2.903149in}{2.366732in}}%
\pgfpathlineto{\pgfqpoint{2.903761in}{2.325591in}}%
\pgfpathlineto{\pgfqpoint{2.904576in}{2.267993in}}%
\pgfpathlineto{\pgfqpoint{2.904780in}{2.292678in}}%
\pgfpathlineto{\pgfqpoint{2.906003in}{2.383188in}}%
\pgfpathlineto{\pgfqpoint{2.905188in}{2.276221in}}%
\pgfpathlineto{\pgfqpoint{2.906410in}{2.358504in}}%
\pgfpathlineto{\pgfqpoint{2.906614in}{2.284450in}}%
\pgfpathlineto{\pgfqpoint{2.907226in}{2.366732in}}%
\pgfpathlineto{\pgfqpoint{2.907429in}{2.366732in}}%
\pgfpathlineto{\pgfqpoint{2.907837in}{2.399645in}}%
\pgfpathlineto{\pgfqpoint{2.908041in}{2.333819in}}%
\pgfpathlineto{\pgfqpoint{2.908245in}{2.350275in}}%
\pgfpathlineto{\pgfqpoint{2.908856in}{2.366732in}}%
\pgfpathlineto{\pgfqpoint{2.909875in}{2.136342in}}%
\pgfpathlineto{\pgfqpoint{2.910283in}{2.383188in}}%
\pgfpathlineto{\pgfqpoint{2.910690in}{2.070516in}}%
\pgfpathlineto{\pgfqpoint{2.911098in}{2.284450in}}%
\pgfpathlineto{\pgfqpoint{2.911709in}{2.407873in}}%
\pgfpathlineto{\pgfqpoint{2.912321in}{2.325591in}}%
\pgfpathlineto{\pgfqpoint{2.912524in}{2.333819in}}%
\pgfpathlineto{\pgfqpoint{2.912728in}{2.317363in}}%
\pgfpathlineto{\pgfqpoint{2.912932in}{2.284450in}}%
\pgfpathlineto{\pgfqpoint{2.913340in}{2.342047in}}%
\pgfpathlineto{\pgfqpoint{2.913543in}{2.333819in}}%
\pgfpathlineto{\pgfqpoint{2.913747in}{2.391416in}}%
\pgfpathlineto{\pgfqpoint{2.914562in}{2.342047in}}%
\pgfpathlineto{\pgfqpoint{2.914766in}{2.342047in}}%
\pgfpathlineto{\pgfqpoint{2.915174in}{2.259765in}}%
\pgfpathlineto{\pgfqpoint{2.915989in}{2.284450in}}%
\pgfpathlineto{\pgfqpoint{2.916804in}{2.358504in}}%
\pgfpathlineto{\pgfqpoint{2.916397in}{2.276221in}}%
\pgfpathlineto{\pgfqpoint{2.917008in}{2.292678in}}%
\pgfpathlineto{\pgfqpoint{2.917416in}{2.317363in}}%
\pgfpathlineto{\pgfqpoint{2.918842in}{2.235080in}}%
\pgfpathlineto{\pgfqpoint{2.919658in}{2.317363in}}%
\pgfpathlineto{\pgfqpoint{2.920065in}{2.300906in}}%
\pgfpathlineto{\pgfqpoint{2.920473in}{2.366732in}}%
\pgfpathlineto{\pgfqpoint{2.920880in}{2.325591in}}%
\pgfpathlineto{\pgfqpoint{2.921492in}{2.144570in}}%
\pgfpathlineto{\pgfqpoint{2.921899in}{2.185711in}}%
\pgfpathlineto{\pgfqpoint{2.922511in}{2.407873in}}%
\pgfpathlineto{\pgfqpoint{2.923122in}{2.333819in}}%
\pgfpathlineto{\pgfqpoint{2.924141in}{2.407873in}}%
\pgfpathlineto{\pgfqpoint{2.923734in}{2.309134in}}%
\pgfpathlineto{\pgfqpoint{2.924345in}{2.374960in}}%
\pgfpathlineto{\pgfqpoint{2.924753in}{2.235080in}}%
\pgfpathlineto{\pgfqpoint{2.925364in}{2.350275in}}%
\pgfpathlineto{\pgfqpoint{2.925568in}{2.407873in}}%
\pgfpathlineto{\pgfqpoint{2.925975in}{2.342047in}}%
\pgfpathlineto{\pgfqpoint{2.926791in}{2.086973in}}%
\pgfpathlineto{\pgfqpoint{2.927402in}{2.144570in}}%
\pgfpathlineto{\pgfqpoint{2.928829in}{2.391416in}}%
\pgfpathlineto{\pgfqpoint{2.929440in}{2.383188in}}%
\pgfpathlineto{\pgfqpoint{2.930663in}{2.317363in}}%
\pgfpathlineto{\pgfqpoint{2.930255in}{2.391416in}}%
\pgfpathlineto{\pgfqpoint{2.931071in}{2.333819in}}%
\pgfpathlineto{\pgfqpoint{2.932293in}{2.407873in}}%
\pgfpathlineto{\pgfqpoint{2.932497in}{2.391416in}}%
\pgfpathlineto{\pgfqpoint{2.933109in}{2.440786in}}%
\pgfpathlineto{\pgfqpoint{2.933924in}{2.325591in}}%
\pgfpathlineto{\pgfqpoint{2.934535in}{2.350275in}}%
\pgfpathlineto{\pgfqpoint{2.934739in}{2.342047in}}%
\pgfpathlineto{\pgfqpoint{2.935554in}{2.251537in}}%
\pgfpathlineto{\pgfqpoint{2.935758in}{2.276221in}}%
\pgfpathlineto{\pgfqpoint{2.936369in}{2.350275in}}%
\pgfpathlineto{\pgfqpoint{2.936573in}{2.144570in}}%
\pgfpathlineto{\pgfqpoint{2.937185in}{2.383188in}}%
\pgfpathlineto{\pgfqpoint{2.937388in}{2.366732in}}%
\pgfpathlineto{\pgfqpoint{2.938611in}{2.095201in}}%
\pgfpathlineto{\pgfqpoint{2.938815in}{2.193939in}}%
\pgfpathlineto{\pgfqpoint{2.939019in}{2.342047in}}%
\pgfpathlineto{\pgfqpoint{2.940038in}{2.292678in}}%
\pgfpathlineto{\pgfqpoint{2.941464in}{2.391416in}}%
\pgfpathlineto{\pgfqpoint{2.943910in}{2.136342in}}%
\pgfpathlineto{\pgfqpoint{2.944318in}{2.193939in}}%
\pgfpathlineto{\pgfqpoint{2.945541in}{1.930637in}}%
\pgfpathlineto{\pgfqpoint{2.946152in}{2.169255in}}%
\pgfpathlineto{\pgfqpoint{2.946560in}{2.004691in}}%
\pgfpathlineto{\pgfqpoint{2.948394in}{1.848355in}}%
\pgfpathlineto{\pgfqpoint{2.949413in}{2.004691in}}%
\pgfpathlineto{\pgfqpoint{2.949617in}{1.996463in}}%
\pgfpathlineto{\pgfqpoint{2.949820in}{1.922409in}}%
\pgfpathlineto{\pgfqpoint{2.950024in}{2.029375in}}%
\pgfpathlineto{\pgfqpoint{2.950636in}{2.021147in}}%
\pgfpathlineto{\pgfqpoint{2.950839in}{2.029375in}}%
\pgfpathlineto{\pgfqpoint{2.951247in}{1.930637in}}%
\pgfpathlineto{\pgfqpoint{2.952062in}{1.938865in}}%
\pgfpathlineto{\pgfqpoint{2.952266in}{1.914181in}}%
\pgfpathlineto{\pgfqpoint{2.952674in}{1.980006in}}%
\pgfpathlineto{\pgfqpoint{2.952877in}{1.963550in}}%
\pgfpathlineto{\pgfqpoint{2.953081in}{2.037604in}}%
\pgfpathlineto{\pgfqpoint{2.953693in}{1.938865in}}%
\pgfpathlineto{\pgfqpoint{2.954100in}{1.856583in}}%
\pgfpathlineto{\pgfqpoint{2.954508in}{2.004691in}}%
\pgfpathlineto{\pgfqpoint{2.954915in}{1.963550in}}%
\pgfpathlineto{\pgfqpoint{2.955527in}{1.980006in}}%
\pgfpathlineto{\pgfqpoint{2.955731in}{2.037604in}}%
\pgfpathlineto{\pgfqpoint{2.956342in}{1.947093in}}%
\pgfpathlineto{\pgfqpoint{2.956546in}{1.996463in}}%
\pgfpathlineto{\pgfqpoint{2.957361in}{2.012919in}}%
\pgfpathlineto{\pgfqpoint{2.957973in}{1.922409in}}%
\pgfpathlineto{\pgfqpoint{2.959603in}{2.037604in}}%
\pgfpathlineto{\pgfqpoint{2.961030in}{1.905952in}}%
\pgfpathlineto{\pgfqpoint{2.961845in}{1.971778in}}%
\pgfpathlineto{\pgfqpoint{2.962252in}{1.947093in}}%
\pgfpathlineto{\pgfqpoint{2.962456in}{1.980006in}}%
\pgfpathlineto{\pgfqpoint{2.962660in}{1.930637in}}%
\pgfpathlineto{\pgfqpoint{2.963271in}{1.963550in}}%
\pgfpathlineto{\pgfqpoint{2.963475in}{1.889496in}}%
\pgfpathlineto{\pgfqpoint{2.963679in}{1.971778in}}%
\pgfpathlineto{\pgfqpoint{2.964290in}{1.930637in}}%
\pgfpathlineto{\pgfqpoint{2.965513in}{2.070516in}}%
\pgfpathlineto{\pgfqpoint{2.965921in}{2.029375in}}%
\pgfpathlineto{\pgfqpoint{2.966125in}{2.029375in}}%
\pgfpathlineto{\pgfqpoint{2.967755in}{2.095201in}}%
\pgfpathlineto{\pgfqpoint{2.967959in}{2.029375in}}%
\pgfpathlineto{\pgfqpoint{2.968570in}{2.169255in}}%
\pgfpathlineto{\pgfqpoint{2.968774in}{2.119886in}}%
\pgfpathlineto{\pgfqpoint{2.969589in}{2.169255in}}%
\pgfpathlineto{\pgfqpoint{2.970608in}{2.021147in}}%
\pgfpathlineto{\pgfqpoint{2.970812in}{2.045832in}}%
\pgfpathlineto{\pgfqpoint{2.971016in}{2.136342in}}%
\pgfpathlineto{\pgfqpoint{2.971424in}{2.021147in}}%
\pgfpathlineto{\pgfqpoint{2.971627in}{2.037604in}}%
\pgfpathlineto{\pgfqpoint{2.972443in}{1.930637in}}%
\pgfpathlineto{\pgfqpoint{2.972850in}{1.947093in}}%
\pgfpathlineto{\pgfqpoint{2.973869in}{2.054060in}}%
\pgfpathlineto{\pgfqpoint{2.974073in}{2.037604in}}%
\pgfpathlineto{\pgfqpoint{2.974481in}{1.807214in}}%
\pgfpathlineto{\pgfqpoint{2.975296in}{1.955322in}}%
\pgfpathlineto{\pgfqpoint{2.975500in}{1.980006in}}%
\pgfpathlineto{\pgfqpoint{2.975907in}{1.905952in}}%
\pgfpathlineto{\pgfqpoint{2.976111in}{1.922409in}}%
\pgfpathlineto{\pgfqpoint{2.976315in}{1.848355in}}%
\pgfpathlineto{\pgfqpoint{2.977130in}{1.889496in}}%
\pgfpathlineto{\pgfqpoint{2.977334in}{1.889496in}}%
\pgfpathlineto{\pgfqpoint{2.977741in}{1.881268in}}%
\pgfpathlineto{\pgfqpoint{2.977945in}{1.963550in}}%
\pgfpathlineto{\pgfqpoint{2.978964in}{1.930637in}}%
\pgfpathlineto{\pgfqpoint{2.980391in}{1.831899in}}%
\pgfpathlineto{\pgfqpoint{2.980595in}{1.930637in}}%
\pgfpathlineto{\pgfqpoint{2.981614in}{1.914181in}}%
\pgfpathlineto{\pgfqpoint{2.982021in}{1.897724in}}%
\pgfpathlineto{\pgfqpoint{2.982429in}{1.963550in}}%
\pgfpathlineto{\pgfqpoint{2.983244in}{1.831899in}}%
\pgfpathlineto{\pgfqpoint{2.983652in}{1.840127in}}%
\pgfpathlineto{\pgfqpoint{2.984875in}{1.922409in}}%
\pgfpathlineto{\pgfqpoint{2.985078in}{1.914181in}}%
\pgfpathlineto{\pgfqpoint{2.985282in}{1.889496in}}%
\pgfpathlineto{\pgfqpoint{2.985894in}{1.905952in}}%
\pgfpathlineto{\pgfqpoint{2.986097in}{1.955322in}}%
\pgfpathlineto{\pgfqpoint{2.986709in}{1.848355in}}%
\pgfpathlineto{\pgfqpoint{2.986913in}{1.716704in}}%
\pgfpathlineto{\pgfqpoint{2.987320in}{1.938865in}}%
\pgfpathlineto{\pgfqpoint{2.987728in}{1.873040in}}%
\pgfpathlineto{\pgfqpoint{2.988747in}{2.037604in}}%
\pgfpathlineto{\pgfqpoint{2.989154in}{1.955322in}}%
\pgfpathlineto{\pgfqpoint{2.990377in}{2.062288in}}%
\pgfpathlineto{\pgfqpoint{2.990581in}{2.021147in}}%
\pgfpathlineto{\pgfqpoint{2.992008in}{1.864811in}}%
\pgfpathlineto{\pgfqpoint{2.990989in}{2.037604in}}%
\pgfpathlineto{\pgfqpoint{2.992415in}{1.881268in}}%
\pgfpathlineto{\pgfqpoint{2.993842in}{2.078745in}}%
\pgfpathlineto{\pgfqpoint{2.993434in}{1.864811in}}%
\pgfpathlineto{\pgfqpoint{2.994046in}{2.029375in}}%
\pgfpathlineto{\pgfqpoint{2.995065in}{1.856583in}}%
\pgfpathlineto{\pgfqpoint{2.995268in}{1.873040in}}%
\pgfpathlineto{\pgfqpoint{2.996084in}{1.914181in}}%
\pgfpathlineto{\pgfqpoint{2.996491in}{1.889496in}}%
\pgfpathlineto{\pgfqpoint{2.997103in}{1.774301in}}%
\pgfpathlineto{\pgfqpoint{2.997306in}{1.617965in}}%
\pgfpathlineto{\pgfqpoint{2.997918in}{1.980006in}}%
\pgfpathlineto{\pgfqpoint{2.999752in}{1.585053in}}%
\pgfpathlineto{\pgfqpoint{3.000160in}{1.749617in}}%
\pgfpathlineto{\pgfqpoint{3.001586in}{1.938865in}}%
\pgfpathlineto{\pgfqpoint{3.001994in}{1.692019in}}%
\pgfpathlineto{\pgfqpoint{3.002402in}{1.955322in}}%
\pgfpathlineto{\pgfqpoint{3.002809in}{1.988234in}}%
\pgfpathlineto{\pgfqpoint{3.003013in}{1.955322in}}%
\pgfpathlineto{\pgfqpoint{3.003421in}{1.790758in}}%
\pgfpathlineto{\pgfqpoint{3.004032in}{2.004691in}}%
\pgfpathlineto{\pgfqpoint{3.004236in}{2.004691in}}%
\pgfpathlineto{\pgfqpoint{3.005662in}{1.848355in}}%
\pgfpathlineto{\pgfqpoint{3.004643in}{2.012919in}}%
\pgfpathlineto{\pgfqpoint{3.005866in}{1.914181in}}%
\pgfpathlineto{\pgfqpoint{3.006070in}{1.914181in}}%
\pgfpathlineto{\pgfqpoint{3.006274in}{1.831899in}}%
\pgfpathlineto{\pgfqpoint{3.007089in}{1.930637in}}%
\pgfpathlineto{\pgfqpoint{3.007293in}{1.930637in}}%
\pgfpathlineto{\pgfqpoint{3.008108in}{1.840127in}}%
\pgfpathlineto{\pgfqpoint{3.008516in}{1.881268in}}%
\pgfpathlineto{\pgfqpoint{3.009127in}{1.864811in}}%
\pgfpathlineto{\pgfqpoint{3.009535in}{1.947093in}}%
\pgfpathlineto{\pgfqpoint{3.009738in}{1.905952in}}%
\pgfpathlineto{\pgfqpoint{3.010554in}{1.955322in}}%
\pgfpathlineto{\pgfqpoint{3.011573in}{1.889496in}}%
\pgfpathlineto{\pgfqpoint{3.011777in}{1.905952in}}%
\pgfpathlineto{\pgfqpoint{3.012184in}{1.971778in}}%
\pgfpathlineto{\pgfqpoint{3.012388in}{1.881268in}}%
\pgfpathlineto{\pgfqpoint{3.012592in}{1.626194in}}%
\pgfpathlineto{\pgfqpoint{3.013203in}{2.029375in}}%
\pgfpathlineto{\pgfqpoint{3.013407in}{1.963550in}}%
\pgfpathlineto{\pgfqpoint{3.014018in}{1.766073in}}%
\pgfpathlineto{\pgfqpoint{3.015241in}{1.790758in}}%
\pgfpathlineto{\pgfqpoint{3.015649in}{1.782529in}}%
\pgfpathlineto{\pgfqpoint{3.016668in}{1.848355in}}%
\pgfpathlineto{\pgfqpoint{3.017687in}{1.807214in}}%
\pgfpathlineto{\pgfqpoint{3.017075in}{1.881268in}}%
\pgfpathlineto{\pgfqpoint{3.017891in}{1.815442in}}%
\pgfpathlineto{\pgfqpoint{3.018706in}{1.881268in}}%
\pgfpathlineto{\pgfqpoint{3.018910in}{1.823670in}}%
\pgfpathlineto{\pgfqpoint{3.019521in}{1.766073in}}%
\pgfpathlineto{\pgfqpoint{3.019725in}{1.798986in}}%
\pgfpathlineto{\pgfqpoint{3.021151in}{2.021147in}}%
\pgfpathlineto{\pgfqpoint{3.021559in}{1.856583in}}%
\pgfpathlineto{\pgfqpoint{3.022374in}{1.864811in}}%
\pgfpathlineto{\pgfqpoint{3.023801in}{1.519227in}}%
\pgfpathlineto{\pgfqpoint{3.024820in}{1.864811in}}%
\pgfpathlineto{\pgfqpoint{3.025024in}{1.798986in}}%
\pgfpathlineto{\pgfqpoint{3.025228in}{1.798986in}}%
\pgfpathlineto{\pgfqpoint{3.025839in}{1.749617in}}%
\pgfpathlineto{\pgfqpoint{3.026450in}{1.774301in}}%
\pgfpathlineto{\pgfqpoint{3.027062in}{1.716704in}}%
\pgfpathlineto{\pgfqpoint{3.027266in}{1.535683in}}%
\pgfpathlineto{\pgfqpoint{3.027673in}{1.807214in}}%
\pgfpathlineto{\pgfqpoint{3.027877in}{1.766073in}}%
\pgfpathlineto{\pgfqpoint{3.028081in}{1.897724in}}%
\pgfpathlineto{\pgfqpoint{3.028285in}{1.609737in}}%
\pgfpathlineto{\pgfqpoint{3.029100in}{1.873040in}}%
\pgfpathlineto{\pgfqpoint{3.029304in}{1.848355in}}%
\pgfpathlineto{\pgfqpoint{3.029711in}{1.930637in}}%
\pgfpathlineto{\pgfqpoint{3.031953in}{1.782529in}}%
\pgfpathlineto{\pgfqpoint{3.032768in}{1.848355in}}%
\pgfpathlineto{\pgfqpoint{3.032972in}{1.774301in}}%
\pgfpathlineto{\pgfqpoint{3.034602in}{1.914181in}}%
\pgfpathlineto{\pgfqpoint{3.035010in}{1.873040in}}%
\pgfpathlineto{\pgfqpoint{3.036029in}{1.700247in}}%
\pgfpathlineto{\pgfqpoint{3.036233in}{1.708476in}}%
\pgfpathlineto{\pgfqpoint{3.036437in}{1.733160in}}%
\pgfpathlineto{\pgfqpoint{3.036844in}{1.675563in}}%
\pgfpathlineto{\pgfqpoint{3.037048in}{1.642650in}}%
\pgfpathlineto{\pgfqpoint{3.037456in}{1.716704in}}%
\pgfpathlineto{\pgfqpoint{3.038475in}{1.790758in}}%
\pgfpathlineto{\pgfqpoint{3.038067in}{1.683791in}}%
\pgfpathlineto{\pgfqpoint{3.038679in}{1.757845in}}%
\pgfpathlineto{\pgfqpoint{3.039494in}{1.659106in}}%
\pgfpathlineto{\pgfqpoint{3.039901in}{1.708476in}}%
\pgfpathlineto{\pgfqpoint{3.040309in}{1.823670in}}%
\pgfpathlineto{\pgfqpoint{3.040513in}{1.700247in}}%
\pgfpathlineto{\pgfqpoint{3.040920in}{1.782529in}}%
\pgfpathlineto{\pgfqpoint{3.041939in}{1.724932in}}%
\pgfpathlineto{\pgfqpoint{3.041736in}{1.815442in}}%
\pgfpathlineto{\pgfqpoint{3.042143in}{1.733160in}}%
\pgfpathlineto{\pgfqpoint{3.042958in}{1.667335in}}%
\pgfpathlineto{\pgfqpoint{3.044385in}{1.831899in}}%
\pgfpathlineto{\pgfqpoint{3.044793in}{1.749617in}}%
\pgfpathlineto{\pgfqpoint{3.045200in}{1.798986in}}%
\pgfpathlineto{\pgfqpoint{3.045608in}{1.856583in}}%
\pgfpathlineto{\pgfqpoint{3.045812in}{1.782529in}}%
\pgfpathlineto{\pgfqpoint{3.046015in}{1.823670in}}%
\pgfpathlineto{\pgfqpoint{3.046423in}{1.708476in}}%
\pgfpathlineto{\pgfqpoint{3.047034in}{1.749617in}}%
\pgfpathlineto{\pgfqpoint{3.047442in}{1.856583in}}%
\pgfpathlineto{\pgfqpoint{3.047646in}{1.741388in}}%
\pgfpathlineto{\pgfqpoint{3.048053in}{1.831899in}}%
\pgfpathlineto{\pgfqpoint{3.049072in}{1.617965in}}%
\pgfpathlineto{\pgfqpoint{3.049480in}{1.659106in}}%
\pgfpathlineto{\pgfqpoint{3.050295in}{1.766073in}}%
\pgfpathlineto{\pgfqpoint{3.050703in}{1.749617in}}%
\pgfpathlineto{\pgfqpoint{3.051518in}{1.667335in}}%
\pgfpathlineto{\pgfqpoint{3.051722in}{1.675563in}}%
\pgfpathlineto{\pgfqpoint{3.052333in}{1.782529in}}%
\pgfpathlineto{\pgfqpoint{3.052741in}{1.716704in}}%
\pgfpathlineto{\pgfqpoint{3.053149in}{1.634422in}}%
\pgfpathlineto{\pgfqpoint{3.053760in}{1.692019in}}%
\pgfpathlineto{\pgfqpoint{3.053964in}{1.716704in}}%
\pgfpathlineto{\pgfqpoint{3.054168in}{1.642650in}}%
\pgfpathlineto{\pgfqpoint{3.054575in}{1.609737in}}%
\pgfpathlineto{\pgfqpoint{3.054983in}{1.683791in}}%
\pgfpathlineto{\pgfqpoint{3.055187in}{1.741388in}}%
\pgfpathlineto{\pgfqpoint{3.055594in}{1.675563in}}%
\pgfpathlineto{\pgfqpoint{3.056002in}{1.708476in}}%
\pgfpathlineto{\pgfqpoint{3.056206in}{1.708476in}}%
\pgfpathlineto{\pgfqpoint{3.057428in}{1.626194in}}%
\pgfpathlineto{\pgfqpoint{3.057632in}{1.634422in}}%
\pgfpathlineto{\pgfqpoint{3.057836in}{1.609737in}}%
\pgfpathlineto{\pgfqpoint{3.058244in}{1.560368in}}%
\pgfpathlineto{\pgfqpoint{3.058651in}{1.642650in}}%
\pgfpathlineto{\pgfqpoint{3.058855in}{1.576824in}}%
\pgfpathlineto{\pgfqpoint{3.059059in}{1.642650in}}%
\pgfpathlineto{\pgfqpoint{3.059670in}{1.527455in}}%
\pgfpathlineto{\pgfqpoint{3.059874in}{1.560368in}}%
\pgfpathlineto{\pgfqpoint{3.060282in}{1.593281in}}%
\pgfpathlineto{\pgfqpoint{3.060485in}{1.585053in}}%
\pgfpathlineto{\pgfqpoint{3.061301in}{1.379348in}}%
\pgfpathlineto{\pgfqpoint{3.061504in}{1.609737in}}%
\pgfpathlineto{\pgfqpoint{3.062320in}{1.716704in}}%
\pgfpathlineto{\pgfqpoint{3.062523in}{1.469858in}}%
\pgfpathlineto{\pgfqpoint{3.063339in}{1.733160in}}%
\pgfpathlineto{\pgfqpoint{3.063542in}{1.716704in}}%
\pgfpathlineto{\pgfqpoint{3.063746in}{1.749617in}}%
\pgfpathlineto{\pgfqpoint{3.063950in}{1.749617in}}%
\pgfpathlineto{\pgfqpoint{3.064358in}{1.626194in}}%
\pgfpathlineto{\pgfqpoint{3.064969in}{1.807214in}}%
\pgfpathlineto{\pgfqpoint{3.065988in}{1.766073in}}%
\pgfpathlineto{\pgfqpoint{3.066396in}{1.905952in}}%
\pgfpathlineto{\pgfqpoint{3.066803in}{1.798986in}}%
\pgfpathlineto{\pgfqpoint{3.068026in}{1.716704in}}%
\pgfpathlineto{\pgfqpoint{3.068638in}{1.749617in}}%
\pgfpathlineto{\pgfqpoint{3.069860in}{1.609737in}}%
\pgfpathlineto{\pgfqpoint{3.070064in}{1.617965in}}%
\pgfpathlineto{\pgfqpoint{3.071898in}{1.807214in}}%
\pgfpathlineto{\pgfqpoint{3.072102in}{1.741388in}}%
\pgfpathlineto{\pgfqpoint{3.072306in}{1.445173in}}%
\pgfpathlineto{\pgfqpoint{3.073121in}{1.823670in}}%
\pgfpathlineto{\pgfqpoint{3.074140in}{1.576824in}}%
\pgfpathlineto{\pgfqpoint{3.074548in}{1.626194in}}%
\pgfpathlineto{\pgfqpoint{3.074955in}{1.700247in}}%
\pgfpathlineto{\pgfqpoint{3.075567in}{1.667335in}}%
\pgfpathlineto{\pgfqpoint{3.075771in}{1.617965in}}%
\pgfpathlineto{\pgfqpoint{3.076382in}{1.675563in}}%
\pgfpathlineto{\pgfqpoint{3.077401in}{1.774301in}}%
\pgfpathlineto{\pgfqpoint{3.077809in}{1.724932in}}%
\pgfpathlineto{\pgfqpoint{3.078828in}{1.642650in}}%
\pgfpathlineto{\pgfqpoint{3.079032in}{1.683791in}}%
\pgfpathlineto{\pgfqpoint{3.079847in}{1.708476in}}%
\pgfpathlineto{\pgfqpoint{3.080051in}{1.683791in}}%
\pgfpathlineto{\pgfqpoint{3.080662in}{1.642650in}}%
\pgfpathlineto{\pgfqpoint{3.081070in}{1.692019in}}%
\pgfpathlineto{\pgfqpoint{3.081681in}{1.659106in}}%
\pgfpathlineto{\pgfqpoint{3.082292in}{1.741388in}}%
\pgfpathlineto{\pgfqpoint{3.083923in}{1.617965in}}%
\pgfpathlineto{\pgfqpoint{3.084330in}{1.626194in}}%
\pgfpathlineto{\pgfqpoint{3.085961in}{1.757845in}}%
\pgfpathlineto{\pgfqpoint{3.086368in}{1.766073in}}%
\pgfpathlineto{\pgfqpoint{3.087387in}{1.560368in}}%
\pgfpathlineto{\pgfqpoint{3.087999in}{1.782529in}}%
\pgfpathlineto{\pgfqpoint{3.088406in}{1.683791in}}%
\pgfpathlineto{\pgfqpoint{3.088610in}{1.552140in}}%
\pgfpathlineto{\pgfqpoint{3.088814in}{1.774301in}}%
\pgfpathlineto{\pgfqpoint{3.089425in}{1.683791in}}%
\pgfpathlineto{\pgfqpoint{3.090037in}{1.642650in}}%
\pgfpathlineto{\pgfqpoint{3.090648in}{1.749617in}}%
\pgfpathlineto{\pgfqpoint{3.092075in}{1.659106in}}%
\pgfpathlineto{\pgfqpoint{3.092279in}{1.692019in}}%
\pgfpathlineto{\pgfqpoint{3.092686in}{1.601509in}}%
\pgfpathlineto{\pgfqpoint{3.092890in}{1.642650in}}%
\pgfpathlineto{\pgfqpoint{3.093705in}{1.601509in}}%
\pgfpathlineto{\pgfqpoint{3.094521in}{1.724932in}}%
\pgfpathlineto{\pgfqpoint{3.094928in}{1.675563in}}%
\pgfpathlineto{\pgfqpoint{3.095540in}{1.535683in}}%
\pgfpathlineto{\pgfqpoint{3.096355in}{1.609737in}}%
\pgfpathlineto{\pgfqpoint{3.096762in}{1.576824in}}%
\pgfpathlineto{\pgfqpoint{3.097170in}{1.305294in}}%
\pgfpathlineto{\pgfqpoint{3.098189in}{1.354663in}}%
\pgfpathlineto{\pgfqpoint{3.098393in}{1.346435in}}%
\pgfpathlineto{\pgfqpoint{3.099412in}{1.716704in}}%
\pgfpathlineto{\pgfqpoint{3.099819in}{1.700247in}}%
\pgfpathlineto{\pgfqpoint{3.100227in}{1.716704in}}%
\pgfpathlineto{\pgfqpoint{3.101042in}{1.659106in}}%
\pgfpathlineto{\pgfqpoint{3.101450in}{1.700247in}}%
\pgfpathlineto{\pgfqpoint{3.101654in}{1.601509in}}%
\pgfpathlineto{\pgfqpoint{3.102469in}{1.708476in}}%
\pgfpathlineto{\pgfqpoint{3.102876in}{1.626194in}}%
\pgfpathlineto{\pgfqpoint{3.103284in}{1.568596in}}%
\pgfpathlineto{\pgfqpoint{3.103692in}{1.601509in}}%
\pgfpathlineto{\pgfqpoint{3.104507in}{1.733160in}}%
\pgfpathlineto{\pgfqpoint{3.104303in}{1.576824in}}%
\pgfpathlineto{\pgfqpoint{3.104711in}{1.593281in}}%
\pgfpathlineto{\pgfqpoint{3.105730in}{1.461630in}}%
\pgfpathlineto{\pgfqpoint{3.105934in}{1.478086in}}%
\pgfpathlineto{\pgfqpoint{3.106749in}{1.519227in}}%
\pgfpathlineto{\pgfqpoint{3.106545in}{1.428717in}}%
\pgfpathlineto{\pgfqpoint{3.106953in}{1.478086in}}%
\pgfpathlineto{\pgfqpoint{3.107156in}{1.461630in}}%
\pgfpathlineto{\pgfqpoint{3.107360in}{1.560368in}}%
\pgfpathlineto{\pgfqpoint{3.108175in}{1.494542in}}%
\pgfpathlineto{\pgfqpoint{3.108787in}{1.576824in}}%
\pgfpathlineto{\pgfqpoint{3.108991in}{1.552140in}}%
\pgfpathlineto{\pgfqpoint{3.109602in}{1.461630in}}%
\pgfpathlineto{\pgfqpoint{3.110010in}{1.585053in}}%
\pgfpathlineto{\pgfqpoint{3.111029in}{1.478086in}}%
\pgfpathlineto{\pgfqpoint{3.111232in}{1.535683in}}%
\pgfpathlineto{\pgfqpoint{3.113067in}{1.642650in}}%
\pgfpathlineto{\pgfqpoint{3.113270in}{1.469858in}}%
\pgfpathlineto{\pgfqpoint{3.114086in}{1.683791in}}%
\pgfpathlineto{\pgfqpoint{3.115105in}{1.700247in}}%
\pgfpathlineto{\pgfqpoint{3.115512in}{1.560368in}}%
\pgfpathlineto{\pgfqpoint{3.115716in}{1.659106in}}%
\pgfpathlineto{\pgfqpoint{3.116327in}{1.461630in}}%
\pgfpathlineto{\pgfqpoint{3.116531in}{1.387576in}}%
\pgfpathlineto{\pgfqpoint{3.117346in}{1.461630in}}%
\pgfpathlineto{\pgfqpoint{3.117754in}{1.445173in}}%
\pgfpathlineto{\pgfqpoint{3.118569in}{1.502771in}}%
\pgfpathlineto{\pgfqpoint{3.119385in}{1.428717in}}%
\pgfpathlineto{\pgfqpoint{3.119588in}{1.453401in}}%
\pgfpathlineto{\pgfqpoint{3.120200in}{1.445173in}}%
\pgfpathlineto{\pgfqpoint{3.120811in}{1.510999in}}%
\pgfpathlineto{\pgfqpoint{3.122238in}{1.420489in}}%
\pgfpathlineto{\pgfqpoint{3.123461in}{1.601509in}}%
\pgfpathlineto{\pgfqpoint{3.123868in}{1.329978in}}%
\pgfpathlineto{\pgfqpoint{3.124480in}{1.519227in}}%
\pgfpathlineto{\pgfqpoint{3.126925in}{1.708476in}}%
\pgfpathlineto{\pgfqpoint{3.127129in}{1.683791in}}%
\pgfpathlineto{\pgfqpoint{3.127740in}{1.634422in}}%
\pgfpathlineto{\pgfqpoint{3.128352in}{1.659106in}}%
\pgfpathlineto{\pgfqpoint{3.129167in}{1.757845in}}%
\pgfpathlineto{\pgfqpoint{3.129371in}{1.642650in}}%
\pgfpathlineto{\pgfqpoint{3.130390in}{1.469858in}}%
\pgfpathlineto{\pgfqpoint{3.130594in}{1.494542in}}%
\pgfpathlineto{\pgfqpoint{3.131409in}{1.617965in}}%
\pgfpathlineto{\pgfqpoint{3.131816in}{1.543912in}}%
\pgfpathlineto{\pgfqpoint{3.132224in}{1.502771in}}%
\pgfpathlineto{\pgfqpoint{3.132428in}{1.519227in}}%
\pgfpathlineto{\pgfqpoint{3.132632in}{1.576824in}}%
\pgfpathlineto{\pgfqpoint{3.133039in}{1.494542in}}%
\pgfpathlineto{\pgfqpoint{3.133447in}{1.568596in}}%
\pgfpathlineto{\pgfqpoint{3.135077in}{1.453401in}}%
\pgfpathlineto{\pgfqpoint{3.135281in}{1.428717in}}%
\pgfpathlineto{\pgfqpoint{3.135485in}{1.264153in}}%
\pgfpathlineto{\pgfqpoint{3.135893in}{1.469858in}}%
\pgfpathlineto{\pgfqpoint{3.136300in}{1.445173in}}%
\pgfpathlineto{\pgfqpoint{3.136504in}{1.527455in}}%
\pgfpathlineto{\pgfqpoint{3.137523in}{1.502771in}}%
\pgfpathlineto{\pgfqpoint{3.138134in}{1.206555in}}%
\pgfpathlineto{\pgfqpoint{3.138542in}{1.445173in}}%
\pgfpathlineto{\pgfqpoint{3.139765in}{1.560368in}}%
\pgfpathlineto{\pgfqpoint{3.139969in}{1.527455in}}%
\pgfpathlineto{\pgfqpoint{3.140988in}{1.420489in}}%
\pgfpathlineto{\pgfqpoint{3.141191in}{1.428717in}}%
\pgfpathlineto{\pgfqpoint{3.142618in}{1.568596in}}%
\pgfpathlineto{\pgfqpoint{3.142822in}{1.502771in}}%
\pgfpathlineto{\pgfqpoint{3.143637in}{1.535683in}}%
\pgfpathlineto{\pgfqpoint{3.144248in}{1.576824in}}%
\pgfpathlineto{\pgfqpoint{3.144656in}{1.535683in}}%
\pgfpathlineto{\pgfqpoint{3.144860in}{1.535683in}}%
\pgfpathlineto{\pgfqpoint{3.145471in}{1.510999in}}%
\pgfpathlineto{\pgfqpoint{3.146083in}{1.585053in}}%
\pgfpathlineto{\pgfqpoint{3.147713in}{1.412260in}}%
\pgfpathlineto{\pgfqpoint{3.148528in}{1.469858in}}%
\pgfpathlineto{\pgfqpoint{3.148121in}{1.404032in}}%
\pgfpathlineto{\pgfqpoint{3.148732in}{1.436945in}}%
\pgfpathlineto{\pgfqpoint{3.148936in}{1.428717in}}%
\pgfpathlineto{\pgfqpoint{3.149140in}{1.461630in}}%
\pgfpathlineto{\pgfqpoint{3.149344in}{1.469858in}}%
\pgfpathlineto{\pgfqpoint{3.149547in}{1.445173in}}%
\pgfpathlineto{\pgfqpoint{3.150159in}{1.395804in}}%
\pgfpathlineto{\pgfqpoint{3.149955in}{1.453401in}}%
\pgfpathlineto{\pgfqpoint{3.150566in}{1.428717in}}%
\pgfpathlineto{\pgfqpoint{3.151789in}{1.527455in}}%
\pgfpathlineto{\pgfqpoint{3.151382in}{1.387576in}}%
\pgfpathlineto{\pgfqpoint{3.151993in}{1.519227in}}%
\pgfpathlineto{\pgfqpoint{3.152197in}{1.519227in}}%
\pgfpathlineto{\pgfqpoint{3.152401in}{1.461630in}}%
\pgfpathlineto{\pgfqpoint{3.153216in}{1.543912in}}%
\pgfpathlineto{\pgfqpoint{3.153827in}{1.510999in}}%
\pgfpathlineto{\pgfqpoint{3.154031in}{1.576824in}}%
\pgfpathlineto{\pgfqpoint{3.154235in}{1.543912in}}%
\pgfpathlineto{\pgfqpoint{3.154439in}{2.309134in}}%
\pgfpathlineto{\pgfqpoint{3.154846in}{1.535683in}}%
\pgfpathlineto{\pgfqpoint{3.155254in}{1.634422in}}%
\pgfpathlineto{\pgfqpoint{3.156069in}{1.461630in}}%
\pgfpathlineto{\pgfqpoint{3.156680in}{1.478086in}}%
\pgfpathlineto{\pgfqpoint{3.157292in}{1.543912in}}%
\pgfpathlineto{\pgfqpoint{3.157903in}{1.527455in}}%
\pgfpathlineto{\pgfqpoint{3.158311in}{1.478086in}}%
\pgfpathlineto{\pgfqpoint{3.158922in}{1.527455in}}%
\pgfpathlineto{\pgfqpoint{3.159534in}{1.568596in}}%
\pgfpathlineto{\pgfqpoint{3.160145in}{1.543912in}}%
\pgfpathlineto{\pgfqpoint{3.160349in}{1.543912in}}%
\pgfpathlineto{\pgfqpoint{3.160553in}{1.576824in}}%
\pgfpathlineto{\pgfqpoint{3.160960in}{1.469858in}}%
\pgfpathlineto{\pgfqpoint{3.161164in}{1.502771in}}%
\pgfpathlineto{\pgfqpoint{3.161572in}{1.379348in}}%
\pgfpathlineto{\pgfqpoint{3.161776in}{1.132502in}}%
\pgfpathlineto{\pgfqpoint{3.162387in}{1.412260in}}%
\pgfpathlineto{\pgfqpoint{3.162591in}{1.379348in}}%
\pgfpathlineto{\pgfqpoint{3.162998in}{1.305294in}}%
\pgfpathlineto{\pgfqpoint{3.163610in}{1.362891in}}%
\pgfpathlineto{\pgfqpoint{3.163814in}{1.379348in}}%
\pgfpathlineto{\pgfqpoint{3.164017in}{1.338207in}}%
\pgfpathlineto{\pgfqpoint{3.164221in}{1.305294in}}%
\pgfpathlineto{\pgfqpoint{3.164425in}{1.379348in}}%
\pgfpathlineto{\pgfqpoint{3.165036in}{1.354663in}}%
\pgfpathlineto{\pgfqpoint{3.165444in}{1.486314in}}%
\pgfpathlineto{\pgfqpoint{3.167074in}{1.083132in}}%
\pgfpathlineto{\pgfqpoint{3.168705in}{1.354663in}}%
\pgfpathlineto{\pgfqpoint{3.169316in}{0.877427in}}%
\pgfpathlineto{\pgfqpoint{3.169928in}{1.058448in}}%
\pgfpathlineto{\pgfqpoint{3.170131in}{1.017307in}}%
\pgfpathlineto{\pgfqpoint{3.170539in}{1.107817in}}%
\pgfpathlineto{\pgfqpoint{3.170743in}{1.074904in}}%
\pgfpathlineto{\pgfqpoint{3.171966in}{1.223012in}}%
\pgfpathlineto{\pgfqpoint{3.172169in}{1.140730in}}%
\pgfpathlineto{\pgfqpoint{3.172985in}{1.214784in}}%
\pgfpathlineto{\pgfqpoint{3.172577in}{1.050220in}}%
\pgfpathlineto{\pgfqpoint{3.173392in}{1.190099in}}%
\pgfpathlineto{\pgfqpoint{3.173596in}{1.181871in}}%
\pgfpathlineto{\pgfqpoint{3.173800in}{1.338207in}}%
\pgfpathlineto{\pgfqpoint{3.174615in}{1.206555in}}%
\pgfpathlineto{\pgfqpoint{3.175430in}{1.321750in}}%
\pgfpathlineto{\pgfqpoint{3.176246in}{1.313522in}}%
\pgfpathlineto{\pgfqpoint{3.177468in}{1.083132in}}%
\pgfpathlineto{\pgfqpoint{3.177672in}{1.247696in}}%
\pgfpathlineto{\pgfqpoint{3.178080in}{1.305294in}}%
\pgfpathlineto{\pgfqpoint{3.178895in}{1.198327in}}%
\pgfpathlineto{\pgfqpoint{3.179099in}{1.272381in}}%
\pgfpathlineto{\pgfqpoint{3.179506in}{1.190099in}}%
\pgfpathlineto{\pgfqpoint{3.179710in}{1.206555in}}%
\pgfpathlineto{\pgfqpoint{3.179914in}{1.148958in}}%
\pgfpathlineto{\pgfqpoint{3.180322in}{1.329978in}}%
\pgfpathlineto{\pgfqpoint{3.180729in}{1.214784in}}%
\pgfpathlineto{\pgfqpoint{3.181341in}{1.264153in}}%
\pgfpathlineto{\pgfqpoint{3.181544in}{1.239468in}}%
\pgfpathlineto{\pgfqpoint{3.181952in}{1.025535in}}%
\pgfpathlineto{\pgfqpoint{3.182563in}{1.206555in}}%
\pgfpathlineto{\pgfqpoint{3.182767in}{1.247696in}}%
\pgfpathlineto{\pgfqpoint{3.182971in}{1.239468in}}%
\pgfpathlineto{\pgfqpoint{3.183175in}{1.000850in}}%
\pgfpathlineto{\pgfqpoint{3.183582in}{1.297066in}}%
\pgfpathlineto{\pgfqpoint{3.183990in}{1.297066in}}%
\pgfpathlineto{\pgfqpoint{3.184601in}{1.321750in}}%
\pgfpathlineto{\pgfqpoint{3.185213in}{1.247696in}}%
\pgfpathlineto{\pgfqpoint{3.185620in}{1.313522in}}%
\pgfpathlineto{\pgfqpoint{3.186028in}{1.223012in}}%
\pgfpathlineto{\pgfqpoint{3.186232in}{1.198327in}}%
\pgfpathlineto{\pgfqpoint{3.186436in}{1.223012in}}%
\pgfpathlineto{\pgfqpoint{3.187659in}{1.395804in}}%
\pgfpathlineto{\pgfqpoint{3.187862in}{1.387576in}}%
\pgfpathlineto{\pgfqpoint{3.188066in}{1.387576in}}%
\pgfpathlineto{\pgfqpoint{3.188270in}{1.354663in}}%
\pgfpathlineto{\pgfqpoint{3.188678in}{1.436945in}}%
\pgfpathlineto{\pgfqpoint{3.188881in}{1.445173in}}%
\pgfpathlineto{\pgfqpoint{3.189085in}{1.412260in}}%
\pgfpathlineto{\pgfqpoint{3.189289in}{1.412260in}}%
\pgfpathlineto{\pgfqpoint{3.189697in}{1.387576in}}%
\pgfpathlineto{\pgfqpoint{3.190716in}{1.280609in}}%
\pgfpathlineto{\pgfqpoint{3.190919in}{1.305294in}}%
\pgfpathlineto{\pgfqpoint{3.191735in}{1.231240in}}%
\pgfpathlineto{\pgfqpoint{3.191327in}{1.313522in}}%
\pgfpathlineto{\pgfqpoint{3.191938in}{1.297066in}}%
\pgfpathlineto{\pgfqpoint{3.192142in}{1.362891in}}%
\pgfpathlineto{\pgfqpoint{3.192550in}{1.313522in}}%
\pgfpathlineto{\pgfqpoint{3.192754in}{1.181871in}}%
\pgfpathlineto{\pgfqpoint{3.193773in}{1.214784in}}%
\pgfpathlineto{\pgfqpoint{3.193976in}{1.206555in}}%
\pgfpathlineto{\pgfqpoint{3.194180in}{1.239468in}}%
\pgfpathlineto{\pgfqpoint{3.194384in}{1.223012in}}%
\pgfpathlineto{\pgfqpoint{3.195199in}{1.346435in}}%
\pgfpathlineto{\pgfqpoint{3.194995in}{1.214784in}}%
\pgfpathlineto{\pgfqpoint{3.195607in}{1.329978in}}%
\pgfpathlineto{\pgfqpoint{3.196422in}{1.255925in}}%
\pgfpathlineto{\pgfqpoint{3.197237in}{1.428717in}}%
\pgfpathlineto{\pgfqpoint{3.197645in}{1.412260in}}%
\pgfpathlineto{\pgfqpoint{3.197849in}{1.371119in}}%
\pgfpathlineto{\pgfqpoint{3.198460in}{1.461630in}}%
\pgfpathlineto{\pgfqpoint{3.198664in}{1.412260in}}%
\pgfpathlineto{\pgfqpoint{3.198868in}{1.420489in}}%
\pgfpathlineto{\pgfqpoint{3.199071in}{1.395804in}}%
\pgfpathlineto{\pgfqpoint{3.199683in}{1.223012in}}%
\pgfpathlineto{\pgfqpoint{3.200294in}{1.288837in}}%
\pgfpathlineto{\pgfqpoint{3.200702in}{1.346435in}}%
\pgfpathlineto{\pgfqpoint{3.200906in}{1.321750in}}%
\pgfpathlineto{\pgfqpoint{3.201925in}{1.469858in}}%
\pgfpathlineto{\pgfqpoint{3.202129in}{1.461630in}}%
\pgfpathlineto{\pgfqpoint{3.203555in}{1.206555in}}%
\pgfpathlineto{\pgfqpoint{3.205186in}{1.404032in}}%
\pgfpathlineto{\pgfqpoint{3.205797in}{1.297066in}}%
\pgfpathlineto{\pgfqpoint{3.206205in}{1.395804in}}%
\pgfpathlineto{\pgfqpoint{3.206408in}{1.428717in}}%
\pgfpathlineto{\pgfqpoint{3.206816in}{1.371119in}}%
\pgfpathlineto{\pgfqpoint{3.207224in}{1.387576in}}%
\pgfpathlineto{\pgfqpoint{3.208243in}{1.453401in}}%
\pgfpathlineto{\pgfqpoint{3.208650in}{1.404032in}}%
\pgfpathlineto{\pgfqpoint{3.208854in}{1.404032in}}%
\pgfpathlineto{\pgfqpoint{3.209669in}{1.297066in}}%
\pgfpathlineto{\pgfqpoint{3.209873in}{1.362891in}}%
\pgfpathlineto{\pgfqpoint{3.210077in}{1.379348in}}%
\pgfpathlineto{\pgfqpoint{3.210484in}{1.371119in}}%
\pgfpathlineto{\pgfqpoint{3.210688in}{1.288837in}}%
\pgfpathlineto{\pgfqpoint{3.211300in}{1.436945in}}%
\pgfpathlineto{\pgfqpoint{3.211707in}{1.461630in}}%
\pgfpathlineto{\pgfqpoint{3.211911in}{1.420489in}}%
\pgfpathlineto{\pgfqpoint{3.213134in}{1.601509in}}%
\pgfpathlineto{\pgfqpoint{3.213338in}{1.527455in}}%
\pgfpathlineto{\pgfqpoint{3.214153in}{1.585053in}}%
\pgfpathlineto{\pgfqpoint{3.214561in}{1.617965in}}%
\pgfpathlineto{\pgfqpoint{3.214764in}{1.692019in}}%
\pgfpathlineto{\pgfqpoint{3.214968in}{1.469858in}}%
\pgfpathlineto{\pgfqpoint{3.215783in}{1.675563in}}%
\pgfpathlineto{\pgfqpoint{3.217006in}{1.585053in}}%
\pgfpathlineto{\pgfqpoint{3.217210in}{1.593281in}}%
\pgfpathlineto{\pgfqpoint{3.218229in}{1.716704in}}%
\pgfpathlineto{\pgfqpoint{3.218433in}{1.568596in}}%
\pgfpathlineto{\pgfqpoint{3.219248in}{1.692019in}}%
\pgfpathlineto{\pgfqpoint{3.219452in}{1.667335in}}%
\pgfpathlineto{\pgfqpoint{3.219656in}{1.716704in}}%
\pgfpathlineto{\pgfqpoint{3.219859in}{1.708476in}}%
\pgfpathlineto{\pgfqpoint{3.220675in}{1.798986in}}%
\pgfpathlineto{\pgfqpoint{3.220878in}{1.519227in}}%
\pgfpathlineto{\pgfqpoint{3.221694in}{1.873040in}}%
\pgfpathlineto{\pgfqpoint{3.222916in}{2.029375in}}%
\pgfpathlineto{\pgfqpoint{3.223120in}{1.823670in}}%
\pgfpathlineto{\pgfqpoint{3.223935in}{2.086973in}}%
\pgfpathlineto{\pgfqpoint{3.224343in}{2.111657in}}%
\pgfpathlineto{\pgfqpoint{3.225362in}{1.782529in}}%
\pgfpathlineto{\pgfqpoint{3.225566in}{1.864811in}}%
\pgfpathlineto{\pgfqpoint{3.226381in}{1.708476in}}%
\pgfpathlineto{\pgfqpoint{3.226789in}{2.012919in}}%
\pgfpathlineto{\pgfqpoint{3.227604in}{1.840127in}}%
\pgfpathlineto{\pgfqpoint{3.227400in}{2.037604in}}%
\pgfpathlineto{\pgfqpoint{3.227808in}{2.021147in}}%
\pgfpathlineto{\pgfqpoint{3.228012in}{1.988234in}}%
\pgfpathlineto{\pgfqpoint{3.228215in}{2.070516in}}%
\pgfpathlineto{\pgfqpoint{3.228827in}{2.012919in}}%
\pgfpathlineto{\pgfqpoint{3.229234in}{2.004691in}}%
\pgfpathlineto{\pgfqpoint{3.230050in}{1.831899in}}%
\pgfpathlineto{\pgfqpoint{3.230253in}{2.021147in}}%
\pgfpathlineto{\pgfqpoint{3.230457in}{1.963550in}}%
\pgfpathlineto{\pgfqpoint{3.230661in}{2.021147in}}%
\pgfpathlineto{\pgfqpoint{3.231272in}{1.938865in}}%
\pgfpathlineto{\pgfqpoint{3.231476in}{1.922409in}}%
\pgfpathlineto{\pgfqpoint{3.231680in}{1.955322in}}%
\pgfpathlineto{\pgfqpoint{3.233107in}{2.128114in}}%
\pgfpathlineto{\pgfqpoint{3.233310in}{2.144570in}}%
\pgfpathlineto{\pgfqpoint{3.233514in}{2.111657in}}%
\pgfpathlineto{\pgfqpoint{3.233718in}{2.086973in}}%
\pgfpathlineto{\pgfqpoint{3.234126in}{2.136342in}}%
\pgfpathlineto{\pgfqpoint{3.234329in}{2.136342in}}%
\pgfpathlineto{\pgfqpoint{3.235348in}{2.086973in}}%
\pgfpathlineto{\pgfqpoint{3.235552in}{2.103429in}}%
\pgfpathlineto{\pgfqpoint{3.236164in}{2.202168in}}%
\pgfpathlineto{\pgfqpoint{3.236979in}{2.161027in}}%
\pgfpathlineto{\pgfqpoint{3.237183in}{2.161027in}}%
\pgfpathlineto{\pgfqpoint{3.237590in}{2.210396in}}%
\pgfpathlineto{\pgfqpoint{3.237794in}{2.136342in}}%
\pgfpathlineto{\pgfqpoint{3.237998in}{2.136342in}}%
\pgfpathlineto{\pgfqpoint{3.239628in}{2.193939in}}%
\pgfpathlineto{\pgfqpoint{3.239832in}{2.136342in}}%
\pgfpathlineto{\pgfqpoint{3.240444in}{2.251537in}}%
\pgfpathlineto{\pgfqpoint{3.240851in}{2.235080in}}%
\pgfpathlineto{\pgfqpoint{3.241055in}{2.243309in}}%
\pgfpathlineto{\pgfqpoint{3.241259in}{2.161027in}}%
\pgfpathlineto{\pgfqpoint{3.241870in}{2.292678in}}%
\pgfpathlineto{\pgfqpoint{3.242074in}{2.251537in}}%
\pgfpathlineto{\pgfqpoint{3.242278in}{2.243309in}}%
\pgfpathlineto{\pgfqpoint{3.242482in}{2.284450in}}%
\pgfpathlineto{\pgfqpoint{3.242685in}{2.136342in}}%
\pgfpathlineto{\pgfqpoint{3.243501in}{2.021147in}}%
\pgfpathlineto{\pgfqpoint{3.243704in}{2.128114in}}%
\pgfpathlineto{\pgfqpoint{3.244723in}{2.267993in}}%
\pgfpathlineto{\pgfqpoint{3.244927in}{2.259765in}}%
\pgfpathlineto{\pgfqpoint{3.245131in}{2.292678in}}%
\pgfpathlineto{\pgfqpoint{3.245335in}{2.243309in}}%
\pgfpathlineto{\pgfqpoint{3.245539in}{2.251537in}}%
\pgfpathlineto{\pgfqpoint{3.246761in}{2.045832in}}%
\pgfpathlineto{\pgfqpoint{3.247984in}{2.177483in}}%
\pgfpathlineto{\pgfqpoint{3.248188in}{1.938865in}}%
\pgfpathlineto{\pgfqpoint{3.249003in}{2.119886in}}%
\pgfpathlineto{\pgfqpoint{3.249207in}{2.128114in}}%
\pgfpathlineto{\pgfqpoint{3.250226in}{1.971778in}}%
\pgfpathlineto{\pgfqpoint{3.250634in}{2.045832in}}%
\pgfpathlineto{\pgfqpoint{3.250837in}{2.004691in}}%
\pgfpathlineto{\pgfqpoint{3.251041in}{2.054060in}}%
\pgfpathlineto{\pgfqpoint{3.251449in}{2.045832in}}%
\pgfpathlineto{\pgfqpoint{3.252672in}{2.210396in}}%
\pgfpathlineto{\pgfqpoint{3.252875in}{1.980006in}}%
\pgfpathlineto{\pgfqpoint{3.253487in}{2.292678in}}%
\pgfpathlineto{\pgfqpoint{3.253691in}{2.292678in}}%
\pgfpathlineto{\pgfqpoint{3.254914in}{2.078745in}}%
\pgfpathlineto{\pgfqpoint{3.255321in}{2.128114in}}%
\pgfpathlineto{\pgfqpoint{3.255525in}{2.128114in}}%
\pgfpathlineto{\pgfqpoint{3.255729in}{2.119886in}}%
\pgfpathlineto{\pgfqpoint{3.257155in}{2.243309in}}%
\pgfpathlineto{\pgfqpoint{3.258174in}{2.300906in}}%
\pgfpathlineto{\pgfqpoint{3.259601in}{2.078745in}}%
\pgfpathlineto{\pgfqpoint{3.260212in}{2.317363in}}%
\pgfpathlineto{\pgfqpoint{3.260824in}{2.300906in}}%
\pgfpathlineto{\pgfqpoint{3.261028in}{2.243309in}}%
\pgfpathlineto{\pgfqpoint{3.261639in}{2.317363in}}%
\pgfpathlineto{\pgfqpoint{3.261843in}{2.284450in}}%
\pgfpathlineto{\pgfqpoint{3.262047in}{2.292678in}}%
\pgfpathlineto{\pgfqpoint{3.262250in}{2.276221in}}%
\pgfpathlineto{\pgfqpoint{3.262658in}{2.276221in}}%
\pgfpathlineto{\pgfqpoint{3.262862in}{2.235080in}}%
\pgfpathlineto{\pgfqpoint{3.263269in}{2.309134in}}%
\pgfpathlineto{\pgfqpoint{3.263677in}{2.292678in}}%
\pgfpathlineto{\pgfqpoint{3.264696in}{2.391416in}}%
\pgfpathlineto{\pgfqpoint{3.265307in}{2.243309in}}%
\pgfpathlineto{\pgfqpoint{3.266123in}{2.276221in}}%
\pgfpathlineto{\pgfqpoint{3.266938in}{2.399645in}}%
\pgfpathlineto{\pgfqpoint{3.267346in}{2.309134in}}%
\pgfpathlineto{\pgfqpoint{3.267549in}{2.325591in}}%
\pgfpathlineto{\pgfqpoint{3.267753in}{2.267993in}}%
\pgfpathlineto{\pgfqpoint{3.268161in}{2.300906in}}%
\pgfpathlineto{\pgfqpoint{3.268568in}{2.259765in}}%
\pgfpathlineto{\pgfqpoint{3.269180in}{2.185711in}}%
\pgfpathlineto{\pgfqpoint{3.269587in}{2.259765in}}%
\pgfpathlineto{\pgfqpoint{3.270606in}{2.317363in}}%
\pgfpathlineto{\pgfqpoint{3.270403in}{2.243309in}}%
\pgfpathlineto{\pgfqpoint{3.270810in}{2.300906in}}%
\pgfpathlineto{\pgfqpoint{3.271014in}{2.292678in}}%
\pgfpathlineto{\pgfqpoint{3.271218in}{2.325591in}}%
\pgfpathlineto{\pgfqpoint{3.271422in}{2.012919in}}%
\pgfpathlineto{\pgfqpoint{3.272237in}{2.366732in}}%
\pgfpathlineto{\pgfqpoint{3.273256in}{2.391416in}}%
\pgfpathlineto{\pgfqpoint{3.273460in}{2.383188in}}%
\pgfpathlineto{\pgfqpoint{3.274275in}{2.267993in}}%
\pgfpathlineto{\pgfqpoint{3.274682in}{2.325591in}}%
\pgfpathlineto{\pgfqpoint{3.274886in}{2.350275in}}%
\pgfpathlineto{\pgfqpoint{3.275294in}{2.136342in}}%
\pgfpathlineto{\pgfqpoint{3.275498in}{2.399645in}}%
\pgfpathlineto{\pgfqpoint{3.275905in}{2.333819in}}%
\pgfpathlineto{\pgfqpoint{3.276109in}{2.333819in}}%
\pgfpathlineto{\pgfqpoint{3.277332in}{2.481927in}}%
\pgfpathlineto{\pgfqpoint{3.276720in}{2.276221in}}%
\pgfpathlineto{\pgfqpoint{3.277739in}{2.432557in}}%
\pgfpathlineto{\pgfqpoint{3.278147in}{2.465470in}}%
\pgfpathlineto{\pgfqpoint{3.278351in}{2.391416in}}%
\pgfpathlineto{\pgfqpoint{3.278555in}{2.259765in}}%
\pgfpathlineto{\pgfqpoint{3.279166in}{2.432557in}}%
\pgfpathlineto{\pgfqpoint{3.279574in}{2.473698in}}%
\pgfpathlineto{\pgfqpoint{3.279777in}{2.424329in}}%
\pgfpathlineto{\pgfqpoint{3.280185in}{2.432557in}}%
\pgfpathlineto{\pgfqpoint{3.280389in}{2.432557in}}%
\pgfpathlineto{\pgfqpoint{3.281612in}{2.333819in}}%
\pgfpathlineto{\pgfqpoint{3.281816in}{2.333819in}}%
\pgfpathlineto{\pgfqpoint{3.282427in}{2.325591in}}%
\pgfpathlineto{\pgfqpoint{3.283038in}{2.416101in}}%
\pgfpathlineto{\pgfqpoint{3.283242in}{2.465470in}}%
\pgfpathlineto{\pgfqpoint{3.283854in}{2.407873in}}%
\pgfpathlineto{\pgfqpoint{3.284873in}{2.276221in}}%
\pgfpathlineto{\pgfqpoint{3.285280in}{2.317363in}}%
\pgfpathlineto{\pgfqpoint{3.285484in}{2.317363in}}%
\pgfpathlineto{\pgfqpoint{3.285892in}{2.284450in}}%
\pgfpathlineto{\pgfqpoint{3.286299in}{2.342047in}}%
\pgfpathlineto{\pgfqpoint{3.286503in}{2.366732in}}%
\pgfpathlineto{\pgfqpoint{3.286911in}{2.300906in}}%
\pgfpathlineto{\pgfqpoint{3.287114in}{2.309134in}}%
\pgfpathlineto{\pgfqpoint{3.288541in}{2.432557in}}%
\pgfpathlineto{\pgfqpoint{3.289356in}{2.407873in}}%
\pgfpathlineto{\pgfqpoint{3.289968in}{2.317363in}}%
\pgfpathlineto{\pgfqpoint{3.290579in}{2.342047in}}%
\pgfpathlineto{\pgfqpoint{3.290783in}{2.342047in}}%
\pgfpathlineto{\pgfqpoint{3.291190in}{2.276221in}}%
\pgfpathlineto{\pgfqpoint{3.291802in}{2.358504in}}%
\pgfpathlineto{\pgfqpoint{3.292209in}{2.407873in}}%
\pgfpathlineto{\pgfqpoint{3.292821in}{2.358504in}}%
\pgfpathlineto{\pgfqpoint{3.293025in}{2.358504in}}%
\pgfpathlineto{\pgfqpoint{3.294044in}{2.284450in}}%
\pgfpathlineto{\pgfqpoint{3.294451in}{2.292678in}}%
\pgfpathlineto{\pgfqpoint{3.295063in}{2.366732in}}%
\pgfpathlineto{\pgfqpoint{3.295267in}{2.309134in}}%
\pgfpathlineto{\pgfqpoint{3.296082in}{2.161027in}}%
\pgfpathlineto{\pgfqpoint{3.296286in}{2.218624in}}%
\pgfpathlineto{\pgfqpoint{3.297101in}{2.284450in}}%
\pgfpathlineto{\pgfqpoint{3.297305in}{2.226852in}}%
\pgfpathlineto{\pgfqpoint{3.297508in}{2.226852in}}%
\pgfpathlineto{\pgfqpoint{3.297916in}{2.276221in}}%
\pgfpathlineto{\pgfqpoint{3.298527in}{2.342047in}}%
\pgfpathlineto{\pgfqpoint{3.298935in}{2.292678in}}%
\pgfpathlineto{\pgfqpoint{3.299954in}{2.366732in}}%
\pgfpathlineto{\pgfqpoint{3.300362in}{2.333819in}}%
\pgfpathlineto{\pgfqpoint{3.301381in}{2.259765in}}%
\pgfpathlineto{\pgfqpoint{3.301584in}{2.292678in}}%
\pgfpathlineto{\pgfqpoint{3.301788in}{2.350275in}}%
\pgfpathlineto{\pgfqpoint{3.302196in}{2.243309in}}%
\pgfpathlineto{\pgfqpoint{3.302807in}{2.325591in}}%
\pgfpathlineto{\pgfqpoint{3.303419in}{2.416101in}}%
\pgfpathlineto{\pgfqpoint{3.303826in}{2.325591in}}%
\pgfpathlineto{\pgfqpoint{3.304641in}{2.317363in}}%
\pgfpathlineto{\pgfqpoint{3.305049in}{2.342047in}}%
\pgfpathlineto{\pgfqpoint{3.305457in}{2.276221in}}%
\pgfpathlineto{\pgfqpoint{3.306272in}{2.309134in}}%
\pgfpathlineto{\pgfqpoint{3.306679in}{2.333819in}}%
\pgfpathlineto{\pgfqpoint{3.307087in}{2.218624in}}%
\pgfpathlineto{\pgfqpoint{3.307291in}{2.004691in}}%
\pgfpathlineto{\pgfqpoint{3.308106in}{2.218624in}}%
\pgfpathlineto{\pgfqpoint{3.308310in}{2.243309in}}%
\pgfpathlineto{\pgfqpoint{3.308718in}{2.226852in}}%
\pgfpathlineto{\pgfqpoint{3.308921in}{2.177483in}}%
\pgfpathlineto{\pgfqpoint{3.309125in}{2.243309in}}%
\pgfpathlineto{\pgfqpoint{3.309329in}{2.226852in}}%
\pgfpathlineto{\pgfqpoint{3.309940in}{2.350275in}}%
\pgfpathlineto{\pgfqpoint{3.310552in}{2.300906in}}%
\pgfpathlineto{\pgfqpoint{3.310756in}{2.309134in}}%
\pgfpathlineto{\pgfqpoint{3.310959in}{2.300906in}}%
\pgfpathlineto{\pgfqpoint{3.311367in}{2.062288in}}%
\pgfpathlineto{\pgfqpoint{3.311978in}{2.366732in}}%
\pgfpathlineto{\pgfqpoint{3.312182in}{2.193939in}}%
\pgfpathlineto{\pgfqpoint{3.313201in}{2.383188in}}%
\pgfpathlineto{\pgfqpoint{3.313609in}{2.342047in}}%
\pgfpathlineto{\pgfqpoint{3.313813in}{2.276221in}}%
\pgfpathlineto{\pgfqpoint{3.314628in}{2.350275in}}%
\pgfpathlineto{\pgfqpoint{3.315239in}{2.267993in}}%
\pgfpathlineto{\pgfqpoint{3.315443in}{2.284450in}}%
\pgfpathlineto{\pgfqpoint{3.315647in}{2.062288in}}%
\pgfpathlineto{\pgfqpoint{3.316054in}{2.300906in}}%
\pgfpathlineto{\pgfqpoint{3.316462in}{2.251537in}}%
\pgfpathlineto{\pgfqpoint{3.316666in}{2.292678in}}%
\pgfpathlineto{\pgfqpoint{3.316870in}{2.086973in}}%
\pgfpathlineto{\pgfqpoint{3.317481in}{2.416101in}}%
\pgfpathlineto{\pgfqpoint{3.317685in}{2.374960in}}%
\pgfpathlineto{\pgfqpoint{3.318908in}{2.317363in}}%
\pgfpathlineto{\pgfqpoint{3.319723in}{2.399645in}}%
\pgfpathlineto{\pgfqpoint{3.319927in}{2.383188in}}%
\pgfpathlineto{\pgfqpoint{3.320334in}{2.416101in}}%
\pgfpathlineto{\pgfqpoint{3.320946in}{2.325591in}}%
\pgfpathlineto{\pgfqpoint{3.321557in}{2.309134in}}%
\pgfpathlineto{\pgfqpoint{3.322169in}{2.383188in}}%
\pgfpathlineto{\pgfqpoint{3.323188in}{2.333819in}}%
\pgfpathlineto{\pgfqpoint{3.323595in}{2.342047in}}%
\pgfpathlineto{\pgfqpoint{3.323799in}{2.259765in}}%
\pgfpathlineto{\pgfqpoint{3.324818in}{2.276221in}}%
\pgfpathlineto{\pgfqpoint{3.325633in}{2.342047in}}%
\pgfpathlineto{\pgfqpoint{3.326041in}{2.325591in}}%
\pgfpathlineto{\pgfqpoint{3.326245in}{2.309134in}}%
\pgfpathlineto{\pgfqpoint{3.326448in}{2.333819in}}%
\pgfpathlineto{\pgfqpoint{3.327467in}{2.407873in}}%
\pgfpathlineto{\pgfqpoint{3.327671in}{2.366732in}}%
\pgfpathlineto{\pgfqpoint{3.328486in}{2.374960in}}%
\pgfpathlineto{\pgfqpoint{3.329302in}{2.333819in}}%
\pgfpathlineto{\pgfqpoint{3.330117in}{2.399645in}}%
\pgfpathlineto{\pgfqpoint{3.330321in}{2.342047in}}%
\pgfpathlineto{\pgfqpoint{3.330728in}{2.333819in}}%
\pgfpathlineto{\pgfqpoint{3.330932in}{2.342047in}}%
\pgfpathlineto{\pgfqpoint{3.331747in}{2.424329in}}%
\pgfpathlineto{\pgfqpoint{3.331951in}{2.399645in}}%
\pgfpathlineto{\pgfqpoint{3.332359in}{2.202168in}}%
\pgfpathlineto{\pgfqpoint{3.333174in}{2.342047in}}%
\pgfpathlineto{\pgfqpoint{3.333785in}{2.300906in}}%
\pgfpathlineto{\pgfqpoint{3.333581in}{2.358504in}}%
\pgfpathlineto{\pgfqpoint{3.333989in}{2.350275in}}%
\pgfpathlineto{\pgfqpoint{3.334601in}{2.342047in}}%
\pgfpathlineto{\pgfqpoint{3.335416in}{2.416101in}}%
\pgfpathlineto{\pgfqpoint{3.335620in}{2.416101in}}%
\pgfpathlineto{\pgfqpoint{3.336027in}{2.350275in}}%
\pgfpathlineto{\pgfqpoint{3.336639in}{2.416101in}}%
\pgfpathlineto{\pgfqpoint{3.337658in}{2.342047in}}%
\pgfpathlineto{\pgfqpoint{3.337046in}{2.440786in}}%
\pgfpathlineto{\pgfqpoint{3.337861in}{2.383188in}}%
\pgfpathlineto{\pgfqpoint{3.338065in}{2.383188in}}%
\pgfpathlineto{\pgfqpoint{3.338880in}{2.424329in}}%
\pgfpathlineto{\pgfqpoint{3.338677in}{2.358504in}}%
\pgfpathlineto{\pgfqpoint{3.339084in}{2.407873in}}%
\pgfpathlineto{\pgfqpoint{3.339696in}{2.333819in}}%
\pgfpathlineto{\pgfqpoint{3.340307in}{2.342047in}}%
\pgfpathlineto{\pgfqpoint{3.341122in}{2.416101in}}%
\pgfpathlineto{\pgfqpoint{3.341326in}{2.399645in}}%
\pgfpathlineto{\pgfqpoint{3.342345in}{2.325591in}}%
\pgfpathlineto{\pgfqpoint{3.342549in}{2.358504in}}%
\pgfpathlineto{\pgfqpoint{3.342753in}{2.366732in}}%
\pgfpathlineto{\pgfqpoint{3.343568in}{2.259765in}}%
\pgfpathlineto{\pgfqpoint{3.343975in}{2.284450in}}%
\pgfpathlineto{\pgfqpoint{3.344179in}{2.309134in}}%
\pgfpathlineto{\pgfqpoint{3.344791in}{2.259765in}}%
\pgfpathlineto{\pgfqpoint{3.344994in}{2.243309in}}%
\pgfpathlineto{\pgfqpoint{3.345198in}{2.309134in}}%
\pgfpathlineto{\pgfqpoint{3.345402in}{2.292678in}}%
\pgfpathlineto{\pgfqpoint{3.346217in}{2.366732in}}%
\pgfpathlineto{\pgfqpoint{3.346625in}{2.342047in}}%
\pgfpathlineto{\pgfqpoint{3.347236in}{2.383188in}}%
\pgfpathlineto{\pgfqpoint{3.347644in}{2.350275in}}%
\pgfpathlineto{\pgfqpoint{3.347848in}{2.342047in}}%
\pgfpathlineto{\pgfqpoint{3.348052in}{2.374960in}}%
\pgfpathlineto{\pgfqpoint{3.348255in}{2.449014in}}%
\pgfpathlineto{\pgfqpoint{3.349071in}{2.342047in}}%
\pgfpathlineto{\pgfqpoint{3.349274in}{2.309134in}}%
\pgfpathlineto{\pgfqpoint{3.349478in}{2.391416in}}%
\pgfpathlineto{\pgfqpoint{3.349682in}{2.374960in}}%
\pgfpathlineto{\pgfqpoint{3.350090in}{2.416101in}}%
\pgfpathlineto{\pgfqpoint{3.350497in}{2.366732in}}%
\pgfpathlineto{\pgfqpoint{3.351924in}{2.284450in}}%
\pgfpathlineto{\pgfqpoint{3.352128in}{2.300906in}}%
\pgfpathlineto{\pgfqpoint{3.352331in}{2.300906in}}%
\pgfpathlineto{\pgfqpoint{3.352535in}{2.292678in}}%
\pgfpathlineto{\pgfqpoint{3.352739in}{2.325591in}}%
\pgfpathlineto{\pgfqpoint{3.352943in}{2.325591in}}%
\pgfpathlineto{\pgfqpoint{3.353147in}{2.284450in}}%
\pgfpathlineto{\pgfqpoint{3.353554in}{2.358504in}}%
\pgfpathlineto{\pgfqpoint{3.353758in}{2.424329in}}%
\pgfpathlineto{\pgfqpoint{3.354777in}{2.407873in}}%
\pgfpathlineto{\pgfqpoint{3.354981in}{2.440786in}}%
\pgfpathlineto{\pgfqpoint{3.355388in}{2.383188in}}%
\pgfpathlineto{\pgfqpoint{3.355796in}{2.407873in}}%
\pgfpathlineto{\pgfqpoint{3.356407in}{2.416101in}}%
\pgfpathlineto{\pgfqpoint{3.356611in}{2.374960in}}%
\pgfpathlineto{\pgfqpoint{3.357223in}{2.465470in}}%
\pgfpathlineto{\pgfqpoint{3.357426in}{2.432557in}}%
\pgfpathlineto{\pgfqpoint{3.358242in}{2.473698in}}%
\pgfpathlineto{\pgfqpoint{3.358445in}{2.407873in}}%
\pgfpathlineto{\pgfqpoint{3.359261in}{2.498383in}}%
\pgfpathlineto{\pgfqpoint{3.359464in}{2.457242in}}%
\pgfpathlineto{\pgfqpoint{3.359668in}{2.523068in}}%
\pgfpathlineto{\pgfqpoint{3.360280in}{2.481927in}}%
\pgfpathlineto{\pgfqpoint{3.360483in}{2.498383in}}%
\pgfpathlineto{\pgfqpoint{3.360891in}{2.465470in}}%
\pgfpathlineto{\pgfqpoint{3.361503in}{2.391416in}}%
\pgfpathlineto{\pgfqpoint{3.361910in}{2.432557in}}%
\pgfpathlineto{\pgfqpoint{3.363133in}{2.506611in}}%
\pgfpathlineto{\pgfqpoint{3.363948in}{2.399645in}}%
\pgfpathlineto{\pgfqpoint{3.364560in}{2.465470in}}%
\pgfpathlineto{\pgfqpoint{3.364763in}{2.506611in}}%
\pgfpathlineto{\pgfqpoint{3.365579in}{2.465470in}}%
\pgfpathlineto{\pgfqpoint{3.367005in}{2.399645in}}%
\pgfpathlineto{\pgfqpoint{3.367209in}{2.383188in}}%
\pgfpathlineto{\pgfqpoint{3.367413in}{2.407873in}}%
\pgfpathlineto{\pgfqpoint{3.367820in}{2.498383in}}%
\pgfpathlineto{\pgfqpoint{3.368432in}{2.391416in}}%
\pgfpathlineto{\pgfqpoint{3.368636in}{2.391416in}}%
\pgfpathlineto{\pgfqpoint{3.368839in}{2.383188in}}%
\pgfpathlineto{\pgfqpoint{3.369655in}{2.506611in}}%
\pgfpathlineto{\pgfqpoint{3.370266in}{2.473698in}}%
\pgfpathlineto{\pgfqpoint{3.370470in}{2.514839in}}%
\pgfpathlineto{\pgfqpoint{3.370877in}{2.399645in}}%
\pgfpathlineto{\pgfqpoint{3.371081in}{2.383188in}}%
\pgfpathlineto{\pgfqpoint{3.371285in}{2.424329in}}%
\pgfpathlineto{\pgfqpoint{3.371489in}{2.416101in}}%
\pgfpathlineto{\pgfqpoint{3.372712in}{2.490155in}}%
\pgfpathlineto{\pgfqpoint{3.374546in}{2.391416in}}%
\pgfpathlineto{\pgfqpoint{3.374750in}{2.399645in}}%
\pgfpathlineto{\pgfqpoint{3.375769in}{2.457242in}}%
\pgfpathlineto{\pgfqpoint{3.375973in}{2.432557in}}%
\pgfpathlineto{\pgfqpoint{3.376992in}{2.317363in}}%
\pgfpathlineto{\pgfqpoint{3.376380in}{2.465470in}}%
\pgfpathlineto{\pgfqpoint{3.377399in}{2.325591in}}%
\pgfpathlineto{\pgfqpoint{3.377603in}{2.350275in}}%
\pgfpathlineto{\pgfqpoint{3.378011in}{2.292678in}}%
\pgfpathlineto{\pgfqpoint{3.378214in}{2.317363in}}%
\pgfpathlineto{\pgfqpoint{3.379437in}{2.284450in}}%
\pgfpathlineto{\pgfqpoint{3.380456in}{2.473698in}}%
\pgfpathlineto{\pgfqpoint{3.381068in}{2.267993in}}%
\pgfpathlineto{\pgfqpoint{3.381679in}{2.317363in}}%
\pgfpathlineto{\pgfqpoint{3.383106in}{2.391416in}}%
\pgfpathlineto{\pgfqpoint{3.383309in}{2.374960in}}%
\pgfpathlineto{\pgfqpoint{3.384940in}{2.111657in}}%
\pgfpathlineto{\pgfqpoint{3.386163in}{2.449014in}}%
\pgfpathlineto{\pgfqpoint{3.386366in}{2.440786in}}%
\pgfpathlineto{\pgfqpoint{3.386570in}{2.449014in}}%
\pgfpathlineto{\pgfqpoint{3.387793in}{2.531296in}}%
\pgfpathlineto{\pgfqpoint{3.387997in}{2.309134in}}%
\pgfpathlineto{\pgfqpoint{3.388812in}{2.432557in}}%
\pgfpathlineto{\pgfqpoint{3.389424in}{2.374960in}}%
\pgfpathlineto{\pgfqpoint{3.389627in}{2.193939in}}%
\pgfpathlineto{\pgfqpoint{3.390239in}{2.399645in}}%
\pgfpathlineto{\pgfqpoint{3.390443in}{2.366732in}}%
\pgfpathlineto{\pgfqpoint{3.390646in}{2.358504in}}%
\pgfpathlineto{\pgfqpoint{3.390850in}{2.366732in}}%
\pgfpathlineto{\pgfqpoint{3.391869in}{2.481927in}}%
\pgfpathlineto{\pgfqpoint{3.391258in}{2.193939in}}%
\pgfpathlineto{\pgfqpoint{3.392073in}{2.449014in}}%
\pgfpathlineto{\pgfqpoint{3.392481in}{2.152798in}}%
\pgfpathlineto{\pgfqpoint{3.392888in}{2.383188in}}%
\pgfpathlineto{\pgfqpoint{3.393907in}{2.539524in}}%
\pgfpathlineto{\pgfqpoint{3.394315in}{2.259765in}}%
\pgfpathlineto{\pgfqpoint{3.395130in}{2.449014in}}%
\pgfpathlineto{\pgfqpoint{3.395334in}{2.523068in}}%
\pgfpathlineto{\pgfqpoint{3.396149in}{2.440786in}}%
\pgfpathlineto{\pgfqpoint{3.397168in}{2.144570in}}%
\pgfpathlineto{\pgfqpoint{3.397372in}{2.284450in}}%
\pgfpathlineto{\pgfqpoint{3.397576in}{2.465470in}}%
\pgfpathlineto{\pgfqpoint{3.398595in}{2.432557in}}%
\pgfpathlineto{\pgfqpoint{3.399206in}{2.473698in}}%
\pgfpathlineto{\pgfqpoint{3.399817in}{2.350275in}}%
\pgfpathlineto{\pgfqpoint{3.400021in}{2.440786in}}%
\pgfpathlineto{\pgfqpoint{3.400429in}{2.193939in}}%
\pgfpathlineto{\pgfqpoint{3.401040in}{2.416101in}}%
\pgfpathlineto{\pgfqpoint{3.402059in}{2.358504in}}%
\pgfpathlineto{\pgfqpoint{3.402263in}{2.399645in}}%
\pgfpathlineto{\pgfqpoint{3.402875in}{2.449014in}}%
\pgfpathlineto{\pgfqpoint{3.403078in}{2.358504in}}%
\pgfpathlineto{\pgfqpoint{3.403282in}{2.325591in}}%
\pgfpathlineto{\pgfqpoint{3.403894in}{2.399645in}}%
\pgfpathlineto{\pgfqpoint{3.405116in}{2.333819in}}%
\pgfpathlineto{\pgfqpoint{3.405320in}{2.350275in}}%
\pgfpathlineto{\pgfqpoint{3.405728in}{2.391416in}}%
\pgfpathlineto{\pgfqpoint{3.405932in}{2.342047in}}%
\pgfpathlineto{\pgfqpoint{3.406135in}{2.251537in}}%
\pgfpathlineto{\pgfqpoint{3.406747in}{2.424329in}}%
\pgfpathlineto{\pgfqpoint{3.406951in}{2.440786in}}%
\pgfpathlineto{\pgfqpoint{3.407154in}{2.399645in}}%
\pgfpathlineto{\pgfqpoint{3.407562in}{2.416101in}}%
\pgfpathlineto{\pgfqpoint{3.408581in}{2.317363in}}%
\pgfpathlineto{\pgfqpoint{3.408785in}{2.366732in}}%
\pgfpathlineto{\pgfqpoint{3.409396in}{2.317363in}}%
\pgfpathlineto{\pgfqpoint{3.409600in}{2.366732in}}%
\pgfpathlineto{\pgfqpoint{3.410415in}{2.416101in}}%
\pgfpathlineto{\pgfqpoint{3.411230in}{2.333819in}}%
\pgfpathlineto{\pgfqpoint{3.411434in}{2.358504in}}%
\pgfpathlineto{\pgfqpoint{3.411842in}{2.391416in}}%
\pgfpathlineto{\pgfqpoint{3.412046in}{2.342047in}}%
\pgfpathlineto{\pgfqpoint{3.412249in}{2.202168in}}%
\pgfpathlineto{\pgfqpoint{3.412861in}{2.358504in}}%
\pgfpathlineto{\pgfqpoint{3.413065in}{2.317363in}}%
\pgfpathlineto{\pgfqpoint{3.413268in}{2.342047in}}%
\pgfpathlineto{\pgfqpoint{3.413472in}{2.161027in}}%
\pgfpathlineto{\pgfqpoint{3.414287in}{2.276221in}}%
\pgfpathlineto{\pgfqpoint{3.415510in}{2.374960in}}%
\pgfpathlineto{\pgfqpoint{3.415714in}{2.366732in}}%
\pgfpathlineto{\pgfqpoint{3.416122in}{2.350275in}}%
\pgfpathlineto{\pgfqpoint{3.416529in}{2.366732in}}%
\pgfpathlineto{\pgfqpoint{3.417548in}{2.407873in}}%
\pgfpathlineto{\pgfqpoint{3.417752in}{2.374960in}}%
\pgfpathlineto{\pgfqpoint{3.417956in}{2.432557in}}%
\pgfpathlineto{\pgfqpoint{3.418160in}{2.399645in}}%
\pgfpathlineto{\pgfqpoint{3.418975in}{2.465470in}}%
\pgfpathlineto{\pgfqpoint{3.419179in}{2.399645in}}%
\pgfpathlineto{\pgfqpoint{3.419586in}{2.374960in}}%
\pgfpathlineto{\pgfqpoint{3.419790in}{2.473698in}}%
\pgfpathlineto{\pgfqpoint{3.420605in}{2.383188in}}%
\pgfpathlineto{\pgfqpoint{3.421421in}{2.391416in}}%
\pgfpathlineto{\pgfqpoint{3.422032in}{2.333819in}}%
\pgfpathlineto{\pgfqpoint{3.423051in}{2.424329in}}%
\pgfpathlineto{\pgfqpoint{3.422440in}{2.276221in}}%
\pgfpathlineto{\pgfqpoint{3.423255in}{2.391416in}}%
\pgfpathlineto{\pgfqpoint{3.423459in}{2.309134in}}%
\pgfpathlineto{\pgfqpoint{3.424274in}{2.399645in}}%
\pgfpathlineto{\pgfqpoint{3.424885in}{2.374960in}}%
\pgfpathlineto{\pgfqpoint{3.425293in}{2.440786in}}%
\pgfpathlineto{\pgfqpoint{3.425497in}{2.399645in}}%
\pgfpathlineto{\pgfqpoint{3.425904in}{2.416101in}}%
\pgfpathlineto{\pgfqpoint{3.426516in}{2.309134in}}%
\pgfpathlineto{\pgfqpoint{3.427127in}{2.539524in}}%
\pgfpathlineto{\pgfqpoint{3.428961in}{2.309134in}}%
\pgfpathlineto{\pgfqpoint{3.429165in}{2.342047in}}%
\pgfpathlineto{\pgfqpoint{3.429573in}{2.267993in}}%
\pgfpathlineto{\pgfqpoint{3.430388in}{2.177483in}}%
\pgfpathlineto{\pgfqpoint{3.430796in}{2.218624in}}%
\pgfpathlineto{\pgfqpoint{3.430999in}{2.284450in}}%
\pgfpathlineto{\pgfqpoint{3.431203in}{2.210396in}}%
\pgfpathlineto{\pgfqpoint{3.431611in}{2.243309in}}%
\pgfpathlineto{\pgfqpoint{3.431815in}{2.161027in}}%
\pgfpathlineto{\pgfqpoint{3.432426in}{2.259765in}}%
\pgfpathlineto{\pgfqpoint{3.432630in}{2.235080in}}%
\pgfpathlineto{\pgfqpoint{3.433241in}{1.963550in}}%
\pgfpathlineto{\pgfqpoint{3.433853in}{2.185711in}}%
\pgfpathlineto{\pgfqpoint{3.434260in}{2.202168in}}%
\pgfpathlineto{\pgfqpoint{3.434464in}{2.128114in}}%
\pgfpathlineto{\pgfqpoint{3.435075in}{2.210396in}}%
\pgfpathlineto{\pgfqpoint{3.435279in}{2.185711in}}%
\pgfpathlineto{\pgfqpoint{3.436502in}{2.309134in}}%
\pgfpathlineto{\pgfqpoint{3.436706in}{2.251537in}}%
\pgfpathlineto{\pgfqpoint{3.437929in}{2.333819in}}%
\pgfpathlineto{\pgfqpoint{3.438948in}{2.218624in}}%
\pgfpathlineto{\pgfqpoint{3.439151in}{2.276221in}}%
\pgfpathlineto{\pgfqpoint{3.440986in}{2.078745in}}%
\pgfpathlineto{\pgfqpoint{3.441597in}{2.029375in}}%
\pgfpathlineto{\pgfqpoint{3.441801in}{2.054060in}}%
\pgfpathlineto{\pgfqpoint{3.443024in}{2.152798in}}%
\pgfpathlineto{\pgfqpoint{3.444043in}{2.095201in}}%
\pgfpathlineto{\pgfqpoint{3.444247in}{2.136342in}}%
\pgfpathlineto{\pgfqpoint{3.444654in}{2.062288in}}%
\pgfpathlineto{\pgfqpoint{3.445062in}{2.086973in}}%
\pgfpathlineto{\pgfqpoint{3.445673in}{2.161027in}}%
\pgfpathlineto{\pgfqpoint{3.446081in}{2.103429in}}%
\pgfpathlineto{\pgfqpoint{3.446285in}{2.078745in}}%
\pgfpathlineto{\pgfqpoint{3.446488in}{2.111657in}}%
\pgfpathlineto{\pgfqpoint{3.446692in}{2.169255in}}%
\pgfpathlineto{\pgfqpoint{3.447304in}{2.086973in}}%
\pgfpathlineto{\pgfqpoint{3.447711in}{2.070516in}}%
\pgfpathlineto{\pgfqpoint{3.447915in}{2.086973in}}%
\pgfpathlineto{\pgfqpoint{3.448934in}{2.251537in}}%
\pgfpathlineto{\pgfqpoint{3.449342in}{2.210396in}}%
\pgfpathlineto{\pgfqpoint{3.449749in}{1.988234in}}%
\pgfpathlineto{\pgfqpoint{3.450157in}{2.309134in}}%
\pgfpathlineto{\pgfqpoint{3.450768in}{2.267993in}}%
\pgfpathlineto{\pgfqpoint{3.450972in}{2.333819in}}%
\pgfpathlineto{\pgfqpoint{3.451380in}{2.243309in}}%
\pgfpathlineto{\pgfqpoint{3.451787in}{2.284450in}}%
\pgfpathlineto{\pgfqpoint{3.451991in}{2.292678in}}%
\pgfpathlineto{\pgfqpoint{3.452195in}{2.259765in}}%
\pgfpathlineto{\pgfqpoint{3.452399in}{2.276221in}}%
\pgfpathlineto{\pgfqpoint{3.452602in}{2.235080in}}%
\pgfpathlineto{\pgfqpoint{3.453418in}{2.259765in}}%
\pgfpathlineto{\pgfqpoint{3.454029in}{2.325591in}}%
\pgfpathlineto{\pgfqpoint{3.454233in}{2.259765in}}%
\pgfpathlineto{\pgfqpoint{3.455659in}{2.193939in}}%
\pgfpathlineto{\pgfqpoint{3.456271in}{2.062288in}}%
\pgfpathlineto{\pgfqpoint{3.456475in}{1.963550in}}%
\pgfpathlineto{\pgfqpoint{3.456679in}{2.226852in}}%
\pgfpathlineto{\pgfqpoint{3.456882in}{2.177483in}}%
\pgfpathlineto{\pgfqpoint{3.457290in}{2.235080in}}%
\pgfpathlineto{\pgfqpoint{3.457698in}{2.152798in}}%
\pgfpathlineto{\pgfqpoint{3.458105in}{2.226852in}}%
\pgfpathlineto{\pgfqpoint{3.458309in}{2.226852in}}%
\pgfpathlineto{\pgfqpoint{3.458513in}{2.202168in}}%
\pgfpathlineto{\pgfqpoint{3.459124in}{2.243309in}}%
\pgfpathlineto{\pgfqpoint{3.459532in}{2.284450in}}%
\pgfpathlineto{\pgfqpoint{3.459736in}{2.210396in}}%
\pgfpathlineto{\pgfqpoint{3.460755in}{2.243309in}}%
\pgfpathlineto{\pgfqpoint{3.460958in}{2.243309in}}%
\pgfpathlineto{\pgfqpoint{3.462385in}{2.300906in}}%
\pgfpathlineto{\pgfqpoint{3.462996in}{2.202168in}}%
\pgfpathlineto{\pgfqpoint{3.463200in}{2.243309in}}%
\pgfpathlineto{\pgfqpoint{3.464015in}{2.358504in}}%
\pgfpathlineto{\pgfqpoint{3.463608in}{2.202168in}}%
\pgfpathlineto{\pgfqpoint{3.464219in}{2.317363in}}%
\pgfpathlineto{\pgfqpoint{3.465646in}{2.086973in}}%
\pgfpathlineto{\pgfqpoint{3.466257in}{2.276221in}}%
\pgfpathlineto{\pgfqpoint{3.466869in}{2.193939in}}%
\pgfpathlineto{\pgfqpoint{3.467072in}{2.226852in}}%
\pgfpathlineto{\pgfqpoint{3.467888in}{1.897724in}}%
\pgfpathlineto{\pgfqpoint{3.468091in}{2.012919in}}%
\pgfpathlineto{\pgfqpoint{3.469110in}{2.161027in}}%
\pgfpathlineto{\pgfqpoint{3.469722in}{2.012919in}}%
\pgfpathlineto{\pgfqpoint{3.470333in}{2.062288in}}%
\pgfpathlineto{\pgfqpoint{3.470945in}{2.012919in}}%
\pgfpathlineto{\pgfqpoint{3.470741in}{2.078745in}}%
\pgfpathlineto{\pgfqpoint{3.471149in}{2.037604in}}%
\pgfpathlineto{\pgfqpoint{3.471352in}{2.070516in}}%
\pgfpathlineto{\pgfqpoint{3.471760in}{2.004691in}}%
\pgfpathlineto{\pgfqpoint{3.472371in}{2.054060in}}%
\pgfpathlineto{\pgfqpoint{3.472983in}{1.840127in}}%
\pgfpathlineto{\pgfqpoint{3.473594in}{2.070516in}}%
\pgfpathlineto{\pgfqpoint{3.473798in}{1.897724in}}%
\pgfpathlineto{\pgfqpoint{3.474002in}{1.741388in}}%
\pgfpathlineto{\pgfqpoint{3.474613in}{2.070516in}}%
\pgfpathlineto{\pgfqpoint{3.474817in}{2.004691in}}%
\pgfpathlineto{\pgfqpoint{3.475428in}{2.086973in}}%
\pgfpathlineto{\pgfqpoint{3.475632in}{2.062288in}}%
\pgfpathlineto{\pgfqpoint{3.476244in}{2.004691in}}%
\pgfpathlineto{\pgfqpoint{3.476447in}{2.012919in}}%
\pgfpathlineto{\pgfqpoint{3.476651in}{1.724932in}}%
\pgfpathlineto{\pgfqpoint{3.477466in}{2.029375in}}%
\pgfpathlineto{\pgfqpoint{3.477874in}{2.012919in}}%
\pgfpathlineto{\pgfqpoint{3.478078in}{2.045832in}}%
\pgfpathlineto{\pgfqpoint{3.478282in}{1.988234in}}%
\pgfpathlineto{\pgfqpoint{3.478893in}{2.119886in}}%
\pgfpathlineto{\pgfqpoint{3.479097in}{2.185711in}}%
\pgfpathlineto{\pgfqpoint{3.479504in}{2.037604in}}%
\pgfpathlineto{\pgfqpoint{3.479912in}{2.095201in}}%
\pgfpathlineto{\pgfqpoint{3.480116in}{2.103429in}}%
\pgfpathlineto{\pgfqpoint{3.480320in}{2.062288in}}%
\pgfpathlineto{\pgfqpoint{3.480727in}{2.128114in}}%
\pgfpathlineto{\pgfqpoint{3.481135in}{2.095201in}}%
\pgfpathlineto{\pgfqpoint{3.481339in}{2.095201in}}%
\pgfpathlineto{\pgfqpoint{3.481542in}{2.144570in}}%
\pgfpathlineto{\pgfqpoint{3.481950in}{2.045832in}}%
\pgfpathlineto{\pgfqpoint{3.482358in}{2.103429in}}%
\pgfpathlineto{\pgfqpoint{3.482561in}{2.111657in}}%
\pgfpathlineto{\pgfqpoint{3.483581in}{1.914181in}}%
\pgfpathlineto{\pgfqpoint{3.484192in}{1.955322in}}%
\pgfpathlineto{\pgfqpoint{3.484396in}{1.938865in}}%
\pgfpathlineto{\pgfqpoint{3.484600in}{1.980006in}}%
\pgfpathlineto{\pgfqpoint{3.484803in}{2.037604in}}%
\pgfpathlineto{\pgfqpoint{3.485619in}{1.955322in}}%
\pgfpathlineto{\pgfqpoint{3.486230in}{1.914181in}}%
\pgfpathlineto{\pgfqpoint{3.487045in}{2.054060in}}%
\pgfpathlineto{\pgfqpoint{3.487453in}{2.029375in}}%
\pgfpathlineto{\pgfqpoint{3.488676in}{1.955322in}}%
\pgfpathlineto{\pgfqpoint{3.489083in}{1.980006in}}%
\pgfpathlineto{\pgfqpoint{3.489287in}{1.980006in}}%
\pgfpathlineto{\pgfqpoint{3.489491in}{1.996463in}}%
\pgfpathlineto{\pgfqpoint{3.489695in}{1.971778in}}%
\pgfpathlineto{\pgfqpoint{3.490306in}{1.905952in}}%
\pgfpathlineto{\pgfqpoint{3.490510in}{1.980006in}}%
\pgfpathlineto{\pgfqpoint{3.490714in}{1.963550in}}%
\pgfpathlineto{\pgfqpoint{3.491121in}{2.004691in}}%
\pgfpathlineto{\pgfqpoint{3.491325in}{1.897724in}}%
\pgfpathlineto{\pgfqpoint{3.491529in}{1.914181in}}%
\pgfpathlineto{\pgfqpoint{3.491733in}{1.881268in}}%
\pgfpathlineto{\pgfqpoint{3.492344in}{1.905952in}}%
\pgfpathlineto{\pgfqpoint{3.492955in}{1.798986in}}%
\pgfpathlineto{\pgfqpoint{3.493363in}{1.848355in}}%
\pgfpathlineto{\pgfqpoint{3.493974in}{1.798986in}}%
\pgfpathlineto{\pgfqpoint{3.494178in}{1.798986in}}%
\pgfpathlineto{\pgfqpoint{3.495197in}{1.864811in}}%
\pgfpathlineto{\pgfqpoint{3.494586in}{1.774301in}}%
\pgfpathlineto{\pgfqpoint{3.495605in}{1.815442in}}%
\pgfpathlineto{\pgfqpoint{3.496624in}{1.774301in}}%
\pgfpathlineto{\pgfqpoint{3.496216in}{1.856583in}}%
\pgfpathlineto{\pgfqpoint{3.496828in}{1.798986in}}%
\pgfpathlineto{\pgfqpoint{3.498458in}{1.971778in}}%
\pgfpathlineto{\pgfqpoint{3.499681in}{1.659106in}}%
\pgfpathlineto{\pgfqpoint{3.499885in}{1.815442in}}%
\pgfpathlineto{\pgfqpoint{3.501108in}{2.029375in}}%
\pgfpathlineto{\pgfqpoint{3.501311in}{2.004691in}}%
\pgfpathlineto{\pgfqpoint{3.502330in}{1.881268in}}%
\pgfpathlineto{\pgfqpoint{3.502738in}{1.889496in}}%
\pgfpathlineto{\pgfqpoint{3.503553in}{1.856583in}}%
\pgfpathlineto{\pgfqpoint{3.503146in}{1.905952in}}%
\pgfpathlineto{\pgfqpoint{3.503961in}{1.873040in}}%
\pgfpathlineto{\pgfqpoint{3.504165in}{1.881268in}}%
\pgfpathlineto{\pgfqpoint{3.504368in}{1.864811in}}%
\pgfpathlineto{\pgfqpoint{3.505184in}{1.798986in}}%
\pgfpathlineto{\pgfqpoint{3.505387in}{1.873040in}}%
\pgfpathlineto{\pgfqpoint{3.506203in}{1.774301in}}%
\pgfpathlineto{\pgfqpoint{3.506610in}{1.790758in}}%
\pgfpathlineto{\pgfqpoint{3.507018in}{1.749617in}}%
\pgfpathlineto{\pgfqpoint{3.507425in}{1.881268in}}%
\pgfpathlineto{\pgfqpoint{3.507833in}{1.692019in}}%
\pgfpathlineto{\pgfqpoint{3.508037in}{1.757845in}}%
\pgfpathlineto{\pgfqpoint{3.509463in}{1.552140in}}%
\pgfpathlineto{\pgfqpoint{3.509667in}{1.626194in}}%
\pgfpathlineto{\pgfqpoint{3.510279in}{1.601509in}}%
\pgfpathlineto{\pgfqpoint{3.510483in}{1.724932in}}%
\pgfpathlineto{\pgfqpoint{3.511298in}{1.609737in}}%
\pgfpathlineto{\pgfqpoint{3.511502in}{1.576824in}}%
\pgfpathlineto{\pgfqpoint{3.512317in}{1.593281in}}%
\pgfpathlineto{\pgfqpoint{3.513336in}{1.634422in}}%
\pgfpathlineto{\pgfqpoint{3.513947in}{1.642650in}}%
\pgfpathlineto{\pgfqpoint{3.514151in}{1.576824in}}%
\pgfpathlineto{\pgfqpoint{3.515374in}{1.700247in}}%
\pgfpathlineto{\pgfqpoint{3.515985in}{1.626194in}}%
\pgfpathlineto{\pgfqpoint{3.516393in}{1.700247in}}%
\pgfpathlineto{\pgfqpoint{3.516597in}{1.700247in}}%
\pgfpathlineto{\pgfqpoint{3.517412in}{1.585053in}}%
\pgfpathlineto{\pgfqpoint{3.518023in}{1.626194in}}%
\pgfpathlineto{\pgfqpoint{3.518227in}{1.675563in}}%
\pgfpathlineto{\pgfqpoint{3.518431in}{1.626194in}}%
\pgfpathlineto{\pgfqpoint{3.518635in}{1.379348in}}%
\pgfpathlineto{\pgfqpoint{3.519450in}{1.560368in}}%
\pgfpathlineto{\pgfqpoint{3.519654in}{1.601509in}}%
\pgfpathlineto{\pgfqpoint{3.519857in}{1.486314in}}%
\pgfpathlineto{\pgfqpoint{3.520265in}{1.502771in}}%
\pgfpathlineto{\pgfqpoint{3.520876in}{1.535683in}}%
\pgfpathlineto{\pgfqpoint{3.521080in}{1.486314in}}%
\pgfpathlineto{\pgfqpoint{3.521284in}{1.510999in}}%
\pgfpathlineto{\pgfqpoint{3.522507in}{1.346435in}}%
\pgfpathlineto{\pgfqpoint{3.523730in}{1.510999in}}%
\pgfpathlineto{\pgfqpoint{3.524341in}{1.494542in}}%
\pgfpathlineto{\pgfqpoint{3.524545in}{1.527455in}}%
\pgfpathlineto{\pgfqpoint{3.524749in}{1.519227in}}%
\pgfpathlineto{\pgfqpoint{3.526175in}{1.642650in}}%
\pgfpathlineto{\pgfqpoint{3.526991in}{1.552140in}}%
\pgfpathlineto{\pgfqpoint{3.528010in}{1.700247in}}%
\pgfpathlineto{\pgfqpoint{3.528213in}{1.659106in}}%
\pgfpathlineto{\pgfqpoint{3.528417in}{1.724932in}}%
\pgfpathlineto{\pgfqpoint{3.528621in}{1.626194in}}%
\pgfpathlineto{\pgfqpoint{3.529232in}{1.642650in}}%
\pgfpathlineto{\pgfqpoint{3.530659in}{1.733160in}}%
\pgfpathlineto{\pgfqpoint{3.530863in}{1.700247in}}%
\pgfpathlineto{\pgfqpoint{3.532086in}{1.560368in}}%
\pgfpathlineto{\pgfqpoint{3.531270in}{1.708476in}}%
\pgfpathlineto{\pgfqpoint{3.532697in}{1.593281in}}%
\pgfpathlineto{\pgfqpoint{3.533512in}{1.692019in}}%
\pgfpathlineto{\pgfqpoint{3.533920in}{1.650878in}}%
\pgfpathlineto{\pgfqpoint{3.534327in}{1.650878in}}%
\pgfpathlineto{\pgfqpoint{3.535143in}{1.700247in}}%
\pgfpathlineto{\pgfqpoint{3.534735in}{1.634422in}}%
\pgfpathlineto{\pgfqpoint{3.535550in}{1.659106in}}%
\pgfpathlineto{\pgfqpoint{3.535754in}{1.659106in}}%
\pgfpathlineto{\pgfqpoint{3.535958in}{1.667335in}}%
\pgfpathlineto{\pgfqpoint{3.536162in}{1.659106in}}%
\pgfpathlineto{\pgfqpoint{3.536365in}{1.757845in}}%
\pgfpathlineto{\pgfqpoint{3.537181in}{1.667335in}}%
\pgfpathlineto{\pgfqpoint{3.537588in}{1.642650in}}%
\pgfpathlineto{\pgfqpoint{3.537792in}{1.667335in}}%
\pgfpathlineto{\pgfqpoint{3.538607in}{1.708476in}}%
\pgfpathlineto{\pgfqpoint{3.538811in}{1.692019in}}%
\pgfpathlineto{\pgfqpoint{3.539626in}{1.642650in}}%
\pgfpathlineto{\pgfqpoint{3.539830in}{1.667335in}}%
\pgfpathlineto{\pgfqpoint{3.540645in}{1.733160in}}%
\pgfpathlineto{\pgfqpoint{3.540238in}{1.617965in}}%
\pgfpathlineto{\pgfqpoint{3.540849in}{1.700247in}}%
\pgfpathlineto{\pgfqpoint{3.541664in}{1.659106in}}%
\pgfpathlineto{\pgfqpoint{3.541461in}{1.708476in}}%
\pgfpathlineto{\pgfqpoint{3.541868in}{1.675563in}}%
\pgfpathlineto{\pgfqpoint{3.543499in}{1.848355in}}%
\pgfpathlineto{\pgfqpoint{3.544721in}{1.749617in}}%
\pgfpathlineto{\pgfqpoint{3.547167in}{1.930637in}}%
\pgfpathlineto{\pgfqpoint{3.547778in}{1.815442in}}%
\pgfpathlineto{\pgfqpoint{3.548594in}{1.881268in}}%
\pgfpathlineto{\pgfqpoint{3.549205in}{1.848355in}}%
\pgfpathlineto{\pgfqpoint{3.549001in}{1.889496in}}%
\pgfpathlineto{\pgfqpoint{3.549409in}{1.873040in}}%
\pgfpathlineto{\pgfqpoint{3.549613in}{1.889496in}}%
\pgfpathlineto{\pgfqpoint{3.549816in}{1.848355in}}%
\pgfpathlineto{\pgfqpoint{3.551039in}{1.716704in}}%
\pgfpathlineto{\pgfqpoint{3.550224in}{1.914181in}}%
\pgfpathlineto{\pgfqpoint{3.551243in}{1.774301in}}%
\pgfpathlineto{\pgfqpoint{3.551651in}{1.815442in}}%
\pgfpathlineto{\pgfqpoint{3.551855in}{1.774301in}}%
\pgfpathlineto{\pgfqpoint{3.553077in}{1.642650in}}%
\pgfpathlineto{\pgfqpoint{3.553281in}{1.700247in}}%
\pgfpathlineto{\pgfqpoint{3.554096in}{1.650878in}}%
\pgfpathlineto{\pgfqpoint{3.554708in}{1.593281in}}%
\pgfpathlineto{\pgfqpoint{3.554912in}{1.683791in}}%
\pgfpathlineto{\pgfqpoint{3.555115in}{1.617965in}}%
\pgfpathlineto{\pgfqpoint{3.555523in}{1.708476in}}%
\pgfpathlineto{\pgfqpoint{3.555931in}{1.568596in}}%
\pgfpathlineto{\pgfqpoint{3.556746in}{1.593281in}}%
\pgfpathlineto{\pgfqpoint{3.557765in}{1.527455in}}%
\pgfpathlineto{\pgfqpoint{3.557357in}{1.617965in}}%
\pgfpathlineto{\pgfqpoint{3.557969in}{1.568596in}}%
\pgfpathlineto{\pgfqpoint{3.558376in}{1.659106in}}%
\pgfpathlineto{\pgfqpoint{3.559191in}{1.626194in}}%
\pgfpathlineto{\pgfqpoint{3.559395in}{1.626194in}}%
\pgfpathlineto{\pgfqpoint{3.560618in}{1.724932in}}%
\pgfpathlineto{\pgfqpoint{3.560822in}{1.716704in}}%
\pgfpathlineto{\pgfqpoint{3.561026in}{1.733160in}}%
\pgfpathlineto{\pgfqpoint{3.561229in}{1.692019in}}%
\pgfpathlineto{\pgfqpoint{3.561637in}{1.708476in}}%
\pgfpathlineto{\pgfqpoint{3.562452in}{1.642650in}}%
\pgfpathlineto{\pgfqpoint{3.562656in}{1.617965in}}%
\pgfpathlineto{\pgfqpoint{3.562860in}{1.667335in}}%
\pgfpathlineto{\pgfqpoint{3.563064in}{1.650878in}}%
\pgfpathlineto{\pgfqpoint{3.564490in}{1.807214in}}%
\pgfpathlineto{\pgfqpoint{3.565917in}{1.650878in}}%
\pgfpathlineto{\pgfqpoint{3.566121in}{1.659106in}}%
\pgfpathlineto{\pgfqpoint{3.566325in}{1.626194in}}%
\pgfpathlineto{\pgfqpoint{3.566528in}{1.617965in}}%
\pgfpathlineto{\pgfqpoint{3.567751in}{1.724932in}}%
\pgfpathlineto{\pgfqpoint{3.567955in}{1.683791in}}%
\pgfpathlineto{\pgfqpoint{3.568566in}{1.733160in}}%
\pgfpathlineto{\pgfqpoint{3.569382in}{1.807214in}}%
\pgfpathlineto{\pgfqpoint{3.569585in}{1.733160in}}%
\pgfpathlineto{\pgfqpoint{3.571012in}{1.650878in}}%
\pgfpathlineto{\pgfqpoint{3.571420in}{1.766073in}}%
\pgfpathlineto{\pgfqpoint{3.572642in}{1.724932in}}%
\pgfpathlineto{\pgfqpoint{3.574069in}{1.585053in}}%
\pgfpathlineto{\pgfqpoint{3.573050in}{1.733160in}}%
\pgfpathlineto{\pgfqpoint{3.574477in}{1.626194in}}%
\pgfpathlineto{\pgfqpoint{3.576311in}{1.815442in}}%
\pgfpathlineto{\pgfqpoint{3.577738in}{1.642650in}}%
\pgfpathlineto{\pgfqpoint{3.577941in}{1.675563in}}%
\pgfpathlineto{\pgfqpoint{3.578145in}{1.675563in}}%
\pgfpathlineto{\pgfqpoint{3.578553in}{1.659106in}}%
\pgfpathlineto{\pgfqpoint{3.579368in}{1.749617in}}%
\pgfpathlineto{\pgfqpoint{3.580387in}{1.609737in}}%
\pgfpathlineto{\pgfqpoint{3.580795in}{1.642650in}}%
\pgfpathlineto{\pgfqpoint{3.582221in}{1.733160in}}%
\pgfpathlineto{\pgfqpoint{3.583240in}{1.617965in}}%
\pgfpathlineto{\pgfqpoint{3.583444in}{1.650878in}}%
\pgfpathlineto{\pgfqpoint{3.584463in}{1.798986in}}%
\pgfpathlineto{\pgfqpoint{3.584871in}{1.782529in}}%
\pgfpathlineto{\pgfqpoint{3.585686in}{1.692019in}}%
\pgfpathlineto{\pgfqpoint{3.585890in}{1.733160in}}%
\pgfpathlineto{\pgfqpoint{3.586093in}{1.782529in}}%
\pgfpathlineto{\pgfqpoint{3.586297in}{1.659106in}}%
\pgfpathlineto{\pgfqpoint{3.586501in}{1.683791in}}%
\pgfpathlineto{\pgfqpoint{3.587112in}{1.576824in}}%
\pgfpathlineto{\pgfqpoint{3.587724in}{1.634422in}}%
\pgfpathlineto{\pgfqpoint{3.588131in}{1.675563in}}%
\pgfpathlineto{\pgfqpoint{3.588539in}{1.626194in}}%
\pgfpathlineto{\pgfqpoint{3.589558in}{1.593281in}}%
\pgfpathlineto{\pgfqpoint{3.589762in}{1.634422in}}%
\pgfpathlineto{\pgfqpoint{3.590169in}{1.552140in}}%
\pgfpathlineto{\pgfqpoint{3.590373in}{1.601509in}}%
\pgfpathlineto{\pgfqpoint{3.590577in}{1.568596in}}%
\pgfpathlineto{\pgfqpoint{3.590781in}{1.626194in}}%
\pgfpathlineto{\pgfqpoint{3.591392in}{1.576824in}}%
\pgfpathlineto{\pgfqpoint{3.591596in}{1.667335in}}%
\pgfpathlineto{\pgfqpoint{3.592615in}{1.642650in}}%
\pgfpathlineto{\pgfqpoint{3.592819in}{1.617965in}}%
\pgfpathlineto{\pgfqpoint{3.593023in}{1.692019in}}%
\pgfpathlineto{\pgfqpoint{3.593227in}{1.692019in}}%
\pgfpathlineto{\pgfqpoint{3.593430in}{1.766073in}}%
\pgfpathlineto{\pgfqpoint{3.594042in}{1.650878in}}%
\pgfpathlineto{\pgfqpoint{3.594246in}{1.716704in}}%
\pgfpathlineto{\pgfqpoint{3.595265in}{1.585053in}}%
\pgfpathlineto{\pgfqpoint{3.595468in}{1.634422in}}%
\pgfpathlineto{\pgfqpoint{3.595672in}{1.642650in}}%
\pgfpathlineto{\pgfqpoint{3.595876in}{1.478086in}}%
\pgfpathlineto{\pgfqpoint{3.596487in}{1.716704in}}%
\pgfpathlineto{\pgfqpoint{3.596691in}{1.724932in}}%
\pgfpathlineto{\pgfqpoint{3.596895in}{1.708476in}}%
\pgfpathlineto{\pgfqpoint{3.597506in}{1.626194in}}%
\pgfpathlineto{\pgfqpoint{3.598322in}{1.634422in}}%
\pgfpathlineto{\pgfqpoint{3.598729in}{1.692019in}}%
\pgfpathlineto{\pgfqpoint{3.598933in}{1.362891in}}%
\pgfpathlineto{\pgfqpoint{3.599748in}{1.634422in}}%
\pgfpathlineto{\pgfqpoint{3.600156in}{1.757845in}}%
\pgfpathlineto{\pgfqpoint{3.600360in}{1.552140in}}%
\pgfpathlineto{\pgfqpoint{3.600971in}{1.708476in}}%
\pgfpathlineto{\pgfqpoint{3.601582in}{1.766073in}}%
\pgfpathlineto{\pgfqpoint{3.601786in}{1.700247in}}%
\pgfpathlineto{\pgfqpoint{3.601990in}{1.724932in}}%
\pgfpathlineto{\pgfqpoint{3.602398in}{1.700247in}}%
\pgfpathlineto{\pgfqpoint{3.602805in}{1.749617in}}%
\pgfpathlineto{\pgfqpoint{3.603009in}{1.766073in}}%
\pgfpathlineto{\pgfqpoint{3.603213in}{1.708476in}}%
\pgfpathlineto{\pgfqpoint{3.603620in}{1.733160in}}%
\pgfpathlineto{\pgfqpoint{3.603824in}{1.560368in}}%
\pgfpathlineto{\pgfqpoint{3.604232in}{1.749617in}}%
\pgfpathlineto{\pgfqpoint{3.604640in}{1.667335in}}%
\pgfpathlineto{\pgfqpoint{3.605659in}{1.774301in}}%
\pgfpathlineto{\pgfqpoint{3.605862in}{1.724932in}}%
\pgfpathlineto{\pgfqpoint{3.606474in}{1.436945in}}%
\pgfpathlineto{\pgfqpoint{3.607085in}{1.667335in}}%
\pgfpathlineto{\pgfqpoint{3.607289in}{1.749617in}}%
\pgfpathlineto{\pgfqpoint{3.607900in}{1.659106in}}%
\pgfpathlineto{\pgfqpoint{3.608104in}{1.659106in}}%
\pgfpathlineto{\pgfqpoint{3.608308in}{1.428717in}}%
\pgfpathlineto{\pgfqpoint{3.608919in}{1.749617in}}%
\pgfpathlineto{\pgfqpoint{3.609123in}{1.700247in}}%
\pgfpathlineto{\pgfqpoint{3.609327in}{1.724932in}}%
\pgfpathlineto{\pgfqpoint{3.609938in}{1.683791in}}%
\pgfpathlineto{\pgfqpoint{3.610142in}{1.675563in}}%
\pgfpathlineto{\pgfqpoint{3.610550in}{1.692019in}}%
\pgfpathlineto{\pgfqpoint{3.611569in}{1.815442in}}%
\pgfpathlineto{\pgfqpoint{3.611976in}{1.774301in}}%
\pgfpathlineto{\pgfqpoint{3.612180in}{1.560368in}}%
\pgfpathlineto{\pgfqpoint{3.612995in}{1.716704in}}%
\pgfpathlineto{\pgfqpoint{3.613199in}{1.757845in}}%
\pgfpathlineto{\pgfqpoint{3.613403in}{1.700247in}}%
\pgfpathlineto{\pgfqpoint{3.613811in}{1.708476in}}%
\pgfpathlineto{\pgfqpoint{3.614422in}{1.724932in}}%
\pgfpathlineto{\pgfqpoint{3.615033in}{1.659106in}}%
\pgfpathlineto{\pgfqpoint{3.616868in}{1.840127in}}%
\pgfpathlineto{\pgfqpoint{3.615441in}{1.617965in}}%
\pgfpathlineto{\pgfqpoint{3.617071in}{1.831899in}}%
\pgfpathlineto{\pgfqpoint{3.618498in}{1.510999in}}%
\pgfpathlineto{\pgfqpoint{3.618702in}{1.494542in}}%
\pgfpathlineto{\pgfqpoint{3.618906in}{1.552140in}}%
\pgfpathlineto{\pgfqpoint{3.619110in}{1.560368in}}%
\pgfpathlineto{\pgfqpoint{3.619517in}{1.486314in}}%
\pgfpathlineto{\pgfqpoint{3.620129in}{1.535683in}}%
\pgfpathlineto{\pgfqpoint{3.620536in}{1.568596in}}%
\pgfpathlineto{\pgfqpoint{3.621555in}{1.404032in}}%
\pgfpathlineto{\pgfqpoint{3.621759in}{1.453401in}}%
\pgfpathlineto{\pgfqpoint{3.623389in}{1.305294in}}%
\pgfpathlineto{\pgfqpoint{3.622574in}{1.494542in}}%
\pgfpathlineto{\pgfqpoint{3.623593in}{1.329978in}}%
\pgfpathlineto{\pgfqpoint{3.623797in}{1.305294in}}%
\pgfpathlineto{\pgfqpoint{3.624001in}{1.362891in}}%
\pgfpathlineto{\pgfqpoint{3.624408in}{1.346435in}}%
\pgfpathlineto{\pgfqpoint{3.624612in}{1.371119in}}%
\pgfpathlineto{\pgfqpoint{3.624816in}{1.305294in}}%
\pgfpathlineto{\pgfqpoint{3.625835in}{1.099589in}}%
\pgfpathlineto{\pgfqpoint{3.626446in}{1.148958in}}%
\pgfpathlineto{\pgfqpoint{3.626650in}{1.379348in}}%
\pgfpathlineto{\pgfqpoint{3.627669in}{1.280609in}}%
\pgfpathlineto{\pgfqpoint{3.627873in}{1.404032in}}%
\pgfpathlineto{\pgfqpoint{3.628688in}{1.321750in}}%
\pgfpathlineto{\pgfqpoint{3.629707in}{1.107817in}}%
\pgfpathlineto{\pgfqpoint{3.630319in}{1.223012in}}%
\pgfpathlineto{\pgfqpoint{3.630522in}{1.206555in}}%
\pgfpathlineto{\pgfqpoint{3.630726in}{1.272381in}}%
\pgfpathlineto{\pgfqpoint{3.630930in}{1.313522in}}%
\pgfpathlineto{\pgfqpoint{3.631134in}{1.239468in}}%
\pgfpathlineto{\pgfqpoint{3.631542in}{1.041991in}}%
\pgfpathlineto{\pgfqpoint{3.632357in}{1.132502in}}%
\pgfpathlineto{\pgfqpoint{3.632764in}{1.173643in}}%
\pgfpathlineto{\pgfqpoint{3.634191in}{0.893884in}}%
\pgfpathlineto{\pgfqpoint{3.634395in}{0.918568in}}%
\pgfpathlineto{\pgfqpoint{3.634802in}{0.910340in}}%
\pgfpathlineto{\pgfqpoint{3.635618in}{0.729320in}}%
\pgfpathlineto{\pgfqpoint{3.635821in}{0.745776in}}%
\pgfpathlineto{\pgfqpoint{3.637044in}{1.000850in}}%
\pgfpathlineto{\pgfqpoint{3.637452in}{1.074904in}}%
\pgfpathlineto{\pgfqpoint{3.638267in}{0.893884in}}%
\pgfpathlineto{\pgfqpoint{3.639082in}{1.148958in}}%
\pgfpathlineto{\pgfqpoint{3.639490in}{0.992622in}}%
\pgfpathlineto{\pgfqpoint{3.639694in}{1.009079in}}%
\pgfpathlineto{\pgfqpoint{3.640101in}{0.647038in}}%
\pgfpathlineto{\pgfqpoint{3.640509in}{1.066676in}}%
\pgfpathlineto{\pgfqpoint{3.640713in}{1.025535in}}%
\pgfpathlineto{\pgfqpoint{3.641120in}{0.918568in}}%
\pgfpathlineto{\pgfqpoint{3.641528in}{1.091361in}}%
\pgfpathlineto{\pgfqpoint{3.641732in}{1.033763in}}%
\pgfpathlineto{\pgfqpoint{3.642139in}{1.124273in}}%
\pgfpathlineto{\pgfqpoint{3.642751in}{1.041991in}}%
\pgfpathlineto{\pgfqpoint{3.643362in}{0.984394in}}%
\pgfpathlineto{\pgfqpoint{3.643770in}{0.992622in}}%
\pgfpathlineto{\pgfqpoint{3.643973in}{1.025535in}}%
\pgfpathlineto{\pgfqpoint{3.644177in}{1.009079in}}%
\pgfpathlineto{\pgfqpoint{3.645400in}{0.811602in}}%
\pgfpathlineto{\pgfqpoint{3.645604in}{0.852743in}}%
\pgfpathlineto{\pgfqpoint{3.645808in}{0.795145in}}%
\pgfpathlineto{\pgfqpoint{3.646419in}{0.877427in}}%
\pgfpathlineto{\pgfqpoint{3.646623in}{0.869199in}}%
\pgfpathlineto{\pgfqpoint{3.647642in}{1.132502in}}%
\pgfpathlineto{\pgfqpoint{3.647846in}{1.091361in}}%
\pgfpathlineto{\pgfqpoint{3.649069in}{0.844515in}}%
\pgfpathlineto{\pgfqpoint{3.650291in}{1.214784in}}%
\pgfpathlineto{\pgfqpoint{3.650699in}{1.066676in}}%
\pgfpathlineto{\pgfqpoint{3.651310in}{1.140730in}}%
\pgfpathlineto{\pgfqpoint{3.652126in}{1.255925in}}%
\pgfpathlineto{\pgfqpoint{3.652329in}{1.066676in}}%
\pgfpathlineto{\pgfqpoint{3.653145in}{1.214784in}}%
\pgfpathlineto{\pgfqpoint{3.653756in}{1.313522in}}%
\pgfpathlineto{\pgfqpoint{3.653960in}{1.181871in}}%
\pgfpathlineto{\pgfqpoint{3.654164in}{1.181871in}}%
\pgfpathlineto{\pgfqpoint{3.654367in}{1.223012in}}%
\pgfpathlineto{\pgfqpoint{3.654775in}{1.124273in}}%
\pgfpathlineto{\pgfqpoint{3.654979in}{1.041991in}}%
\pgfpathlineto{\pgfqpoint{3.655794in}{1.157186in}}%
\pgfpathlineto{\pgfqpoint{3.655998in}{1.132502in}}%
\pgfpathlineto{\pgfqpoint{3.656405in}{1.198327in}}%
\pgfpathlineto{\pgfqpoint{3.656609in}{1.239468in}}%
\pgfpathlineto{\pgfqpoint{3.657017in}{1.132502in}}%
\pgfpathlineto{\pgfqpoint{3.657221in}{1.198327in}}%
\pgfpathlineto{\pgfqpoint{3.657424in}{1.041991in}}%
\pgfpathlineto{\pgfqpoint{3.658240in}{1.148958in}}%
\pgfpathlineto{\pgfqpoint{3.658444in}{1.255925in}}%
\pgfpathlineto{\pgfqpoint{3.658851in}{1.091361in}}%
\pgfpathlineto{\pgfqpoint{3.659259in}{1.181871in}}%
\pgfpathlineto{\pgfqpoint{3.660074in}{1.140730in}}%
\pgfpathlineto{\pgfqpoint{3.660482in}{1.165414in}}%
\pgfpathlineto{\pgfqpoint{3.661093in}{1.132502in}}%
\pgfpathlineto{\pgfqpoint{3.661297in}{1.173643in}}%
\pgfpathlineto{\pgfqpoint{3.662112in}{1.255925in}}%
\pgfpathlineto{\pgfqpoint{3.662927in}{1.231240in}}%
\pgfpathlineto{\pgfqpoint{3.663131in}{1.157186in}}%
\pgfpathlineto{\pgfqpoint{3.663742in}{1.272381in}}%
\pgfpathlineto{\pgfqpoint{3.663946in}{1.255925in}}%
\pgfpathlineto{\pgfqpoint{3.665373in}{1.354663in}}%
\pgfpathlineto{\pgfqpoint{3.666799in}{1.519227in}}%
\pgfpathlineto{\pgfqpoint{3.667207in}{1.478086in}}%
\pgfpathlineto{\pgfqpoint{3.667615in}{1.453401in}}%
\pgfpathlineto{\pgfqpoint{3.667818in}{1.486314in}}%
\pgfpathlineto{\pgfqpoint{3.669041in}{1.527455in}}%
\pgfpathlineto{\pgfqpoint{3.669245in}{1.486314in}}%
\pgfpathlineto{\pgfqpoint{3.669449in}{1.560368in}}%
\pgfpathlineto{\pgfqpoint{3.669653in}{1.552140in}}%
\pgfpathlineto{\pgfqpoint{3.670264in}{1.741388in}}%
\pgfpathlineto{\pgfqpoint{3.670672in}{1.552140in}}%
\pgfpathlineto{\pgfqpoint{3.671079in}{1.568596in}}%
\pgfpathlineto{\pgfqpoint{3.671283in}{1.543912in}}%
\pgfpathlineto{\pgfqpoint{3.671487in}{1.634422in}}%
\pgfpathlineto{\pgfqpoint{3.672302in}{1.585053in}}%
\pgfpathlineto{\pgfqpoint{3.673525in}{1.486314in}}%
\pgfpathlineto{\pgfqpoint{3.673729in}{1.494542in}}%
\pgfpathlineto{\pgfqpoint{3.673933in}{1.527455in}}%
\pgfpathlineto{\pgfqpoint{3.674340in}{1.478086in}}%
\pgfpathlineto{\pgfqpoint{3.674748in}{1.502771in}}%
\pgfpathlineto{\pgfqpoint{3.675155in}{1.510999in}}%
\pgfpathlineto{\pgfqpoint{3.675359in}{1.626194in}}%
\pgfpathlineto{\pgfqpoint{3.675971in}{1.453401in}}%
\pgfpathlineto{\pgfqpoint{3.676174in}{1.478086in}}%
\pgfpathlineto{\pgfqpoint{3.676378in}{1.453401in}}%
\pgfpathlineto{\pgfqpoint{3.676786in}{1.486314in}}%
\pgfpathlineto{\pgfqpoint{3.678212in}{1.642650in}}%
\pgfpathlineto{\pgfqpoint{3.678620in}{1.617965in}}%
\pgfpathlineto{\pgfqpoint{3.678824in}{1.576824in}}%
\pgfpathlineto{\pgfqpoint{3.679231in}{1.675563in}}%
\pgfpathlineto{\pgfqpoint{3.679435in}{1.634422in}}%
\pgfpathlineto{\pgfqpoint{3.680047in}{1.798986in}}%
\pgfpathlineto{\pgfqpoint{3.680658in}{1.757845in}}%
\pgfpathlineto{\pgfqpoint{3.683919in}{2.021147in}}%
\pgfpathlineto{\pgfqpoint{3.684123in}{2.029375in}}%
\pgfpathlineto{\pgfqpoint{3.684326in}{1.922409in}}%
\pgfpathlineto{\pgfqpoint{3.685142in}{2.021147in}}%
\pgfpathlineto{\pgfqpoint{3.685346in}{2.029375in}}%
\pgfpathlineto{\pgfqpoint{3.686161in}{2.161027in}}%
\pgfpathlineto{\pgfqpoint{3.686365in}{2.103429in}}%
\pgfpathlineto{\pgfqpoint{3.687384in}{2.111657in}}%
\pgfpathlineto{\pgfqpoint{3.687791in}{1.790758in}}%
\pgfpathlineto{\pgfqpoint{3.688403in}{2.086973in}}%
\pgfpathlineto{\pgfqpoint{3.689014in}{2.070516in}}%
\pgfpathlineto{\pgfqpoint{3.689625in}{2.004691in}}%
\pgfpathlineto{\pgfqpoint{3.690033in}{2.078745in}}%
\pgfpathlineto{\pgfqpoint{3.690237in}{2.070516in}}%
\pgfpathlineto{\pgfqpoint{3.690441in}{2.136342in}}%
\pgfpathlineto{\pgfqpoint{3.690848in}{2.012919in}}%
\pgfpathlineto{\pgfqpoint{3.691052in}{2.054060in}}%
\pgfpathlineto{\pgfqpoint{3.691256in}{1.790758in}}%
\pgfpathlineto{\pgfqpoint{3.692071in}{1.963550in}}%
\pgfpathlineto{\pgfqpoint{3.692275in}{1.963550in}}%
\pgfpathlineto{\pgfqpoint{3.694109in}{2.177483in}}%
\pgfpathlineto{\pgfqpoint{3.694720in}{2.054060in}}%
\pgfpathlineto{\pgfqpoint{3.695536in}{2.086973in}}%
\pgfpathlineto{\pgfqpoint{3.695739in}{2.086973in}}%
\pgfpathlineto{\pgfqpoint{3.696147in}{2.128114in}}%
\pgfpathlineto{\pgfqpoint{3.696758in}{2.086973in}}%
\pgfpathlineto{\pgfqpoint{3.697370in}{2.095201in}}%
\pgfpathlineto{\pgfqpoint{3.697777in}{2.054060in}}%
\pgfpathlineto{\pgfqpoint{3.699000in}{2.169255in}}%
\pgfpathlineto{\pgfqpoint{3.699408in}{2.136342in}}%
\pgfpathlineto{\pgfqpoint{3.699816in}{2.152798in}}%
\pgfpathlineto{\pgfqpoint{3.700019in}{2.202168in}}%
\pgfpathlineto{\pgfqpoint{3.700835in}{2.161027in}}%
\pgfpathlineto{\pgfqpoint{3.701038in}{2.161027in}}%
\pgfpathlineto{\pgfqpoint{3.701446in}{2.226852in}}%
\pgfpathlineto{\pgfqpoint{3.701650in}{2.152798in}}%
\pgfpathlineto{\pgfqpoint{3.701854in}{2.119886in}}%
\pgfpathlineto{\pgfqpoint{3.702057in}{2.193939in}}%
\pgfpathlineto{\pgfqpoint{3.702465in}{2.185711in}}%
\pgfpathlineto{\pgfqpoint{3.702669in}{2.185711in}}%
\pgfpathlineto{\pgfqpoint{3.703076in}{2.259765in}}%
\pgfpathlineto{\pgfqpoint{3.703892in}{2.243309in}}%
\pgfpathlineto{\pgfqpoint{3.704299in}{2.185711in}}%
\pgfpathlineto{\pgfqpoint{3.704503in}{2.169255in}}%
\pgfpathlineto{\pgfqpoint{3.704707in}{2.210396in}}%
\pgfpathlineto{\pgfqpoint{3.704911in}{2.193939in}}%
\pgfpathlineto{\pgfqpoint{3.705114in}{2.218624in}}%
\pgfpathlineto{\pgfqpoint{3.705930in}{2.210396in}}%
\pgfpathlineto{\pgfqpoint{3.706337in}{2.177483in}}%
\pgfpathlineto{\pgfqpoint{3.706745in}{2.193939in}}%
\pgfpathlineto{\pgfqpoint{3.706949in}{2.251537in}}%
\pgfpathlineto{\pgfqpoint{3.707968in}{2.243309in}}%
\pgfpathlineto{\pgfqpoint{3.709802in}{2.193939in}}%
\pgfpathlineto{\pgfqpoint{3.710006in}{2.202168in}}%
\pgfpathlineto{\pgfqpoint{3.712044in}{2.366732in}}%
\pgfpathlineto{\pgfqpoint{3.712451in}{2.317363in}}%
\pgfpathlineto{\pgfqpoint{3.712859in}{2.342047in}}%
\pgfpathlineto{\pgfqpoint{3.713878in}{2.185711in}}%
\pgfpathlineto{\pgfqpoint{3.714286in}{2.218624in}}%
\pgfpathlineto{\pgfqpoint{3.714693in}{2.161027in}}%
\pgfpathlineto{\pgfqpoint{3.714897in}{2.185711in}}%
\pgfpathlineto{\pgfqpoint{3.715508in}{2.169255in}}%
\pgfpathlineto{\pgfqpoint{3.715712in}{2.193939in}}%
\pgfpathlineto{\pgfqpoint{3.716324in}{2.276221in}}%
\pgfpathlineto{\pgfqpoint{3.716935in}{2.267993in}}%
\pgfpathlineto{\pgfqpoint{3.718362in}{2.202168in}}%
\pgfpathlineto{\pgfqpoint{3.719381in}{2.152798in}}%
\pgfpathlineto{\pgfqpoint{3.718769in}{2.210396in}}%
\pgfpathlineto{\pgfqpoint{3.719584in}{2.185711in}}%
\pgfpathlineto{\pgfqpoint{3.719992in}{2.243309in}}%
\pgfpathlineto{\pgfqpoint{3.720603in}{2.226852in}}%
\pgfpathlineto{\pgfqpoint{3.720807in}{2.185711in}}%
\pgfpathlineto{\pgfqpoint{3.721215in}{2.267993in}}%
\pgfpathlineto{\pgfqpoint{3.721419in}{2.243309in}}%
\pgfpathlineto{\pgfqpoint{3.721622in}{2.267993in}}%
\pgfpathlineto{\pgfqpoint{3.722030in}{2.218624in}}%
\pgfpathlineto{\pgfqpoint{3.722234in}{2.243309in}}%
\pgfpathlineto{\pgfqpoint{3.722845in}{1.897724in}}%
\pgfpathlineto{\pgfqpoint{3.723457in}{2.128114in}}%
\pgfpathlineto{\pgfqpoint{3.723864in}{2.202168in}}%
\pgfpathlineto{\pgfqpoint{3.724068in}{2.119886in}}%
\pgfpathlineto{\pgfqpoint{3.724272in}{1.881268in}}%
\pgfpathlineto{\pgfqpoint{3.724883in}{2.235080in}}%
\pgfpathlineto{\pgfqpoint{3.725087in}{2.152798in}}%
\pgfpathlineto{\pgfqpoint{3.725495in}{2.309134in}}%
\pgfpathlineto{\pgfqpoint{3.726310in}{2.284450in}}%
\pgfpathlineto{\pgfqpoint{3.728959in}{2.054060in}}%
\pgfpathlineto{\pgfqpoint{3.729978in}{2.300906in}}%
\pgfpathlineto{\pgfqpoint{3.730386in}{2.276221in}}%
\pgfpathlineto{\pgfqpoint{3.730590in}{2.276221in}}%
\pgfpathlineto{\pgfqpoint{3.730997in}{2.267993in}}%
\pgfpathlineto{\pgfqpoint{3.731813in}{2.325591in}}%
\pgfpathlineto{\pgfqpoint{3.732016in}{2.317363in}}%
\pgfpathlineto{\pgfqpoint{3.732220in}{2.374960in}}%
\pgfpathlineto{\pgfqpoint{3.732628in}{2.267993in}}%
\pgfpathlineto{\pgfqpoint{3.732832in}{2.267993in}}%
\pgfpathlineto{\pgfqpoint{3.733035in}{2.029375in}}%
\pgfpathlineto{\pgfqpoint{3.733443in}{2.300906in}}%
\pgfpathlineto{\pgfqpoint{3.733851in}{2.284450in}}%
\pgfpathlineto{\pgfqpoint{3.735073in}{2.243309in}}%
\pgfpathlineto{\pgfqpoint{3.735685in}{2.317363in}}%
\pgfpathlineto{\pgfqpoint{3.736092in}{2.259765in}}%
\pgfpathlineto{\pgfqpoint{3.736500in}{2.317363in}}%
\pgfpathlineto{\pgfqpoint{3.737315in}{2.300906in}}%
\pgfpathlineto{\pgfqpoint{3.737519in}{2.267993in}}%
\pgfpathlineto{\pgfqpoint{3.737927in}{2.358504in}}%
\pgfpathlineto{\pgfqpoint{3.738130in}{2.358504in}}%
\pgfpathlineto{\pgfqpoint{3.738334in}{2.424329in}}%
\pgfpathlineto{\pgfqpoint{3.738946in}{2.309134in}}%
\pgfpathlineto{\pgfqpoint{3.739150in}{2.350275in}}%
\pgfpathlineto{\pgfqpoint{3.739965in}{2.292678in}}%
\pgfpathlineto{\pgfqpoint{3.740576in}{2.366732in}}%
\pgfpathlineto{\pgfqpoint{3.740984in}{2.325591in}}%
\pgfpathlineto{\pgfqpoint{3.741188in}{2.309134in}}%
\pgfpathlineto{\pgfqpoint{3.741595in}{2.424329in}}%
\pgfpathlineto{\pgfqpoint{3.742207in}{2.383188in}}%
\pgfpathlineto{\pgfqpoint{3.743429in}{2.292678in}}%
\pgfpathlineto{\pgfqpoint{3.742614in}{2.399645in}}%
\pgfpathlineto{\pgfqpoint{3.743633in}{2.317363in}}%
\pgfpathlineto{\pgfqpoint{3.743837in}{2.333819in}}%
\pgfpathlineto{\pgfqpoint{3.744041in}{2.309134in}}%
\pgfpathlineto{\pgfqpoint{3.744245in}{2.251537in}}%
\pgfpathlineto{\pgfqpoint{3.744448in}{2.325591in}}%
\pgfpathlineto{\pgfqpoint{3.744856in}{2.325591in}}%
\pgfpathlineto{\pgfqpoint{3.745467in}{2.374960in}}%
\pgfpathlineto{\pgfqpoint{3.745875in}{2.342047in}}%
\pgfpathlineto{\pgfqpoint{3.746283in}{2.300906in}}%
\pgfpathlineto{\pgfqpoint{3.746486in}{2.391416in}}%
\pgfpathlineto{\pgfqpoint{3.747302in}{2.292678in}}%
\pgfpathlineto{\pgfqpoint{3.748524in}{2.399645in}}%
\pgfpathlineto{\pgfqpoint{3.748932in}{2.358504in}}%
\pgfpathlineto{\pgfqpoint{3.749543in}{2.325591in}}%
\pgfpathlineto{\pgfqpoint{3.749951in}{2.374960in}}%
\pgfpathlineto{\pgfqpoint{3.750562in}{2.350275in}}%
\pgfpathlineto{\pgfqpoint{3.751785in}{2.267993in}}%
\pgfpathlineto{\pgfqpoint{3.752193in}{2.276221in}}%
\pgfpathlineto{\pgfqpoint{3.752397in}{2.325591in}}%
\pgfpathlineto{\pgfqpoint{3.753212in}{2.284450in}}%
\pgfpathlineto{\pgfqpoint{3.753620in}{2.300906in}}%
\pgfpathlineto{\pgfqpoint{3.753823in}{2.251537in}}%
\pgfpathlineto{\pgfqpoint{3.754027in}{2.267993in}}%
\pgfpathlineto{\pgfqpoint{3.754231in}{2.226852in}}%
\pgfpathlineto{\pgfqpoint{3.754639in}{2.317363in}}%
\pgfpathlineto{\pgfqpoint{3.755046in}{2.276221in}}%
\pgfpathlineto{\pgfqpoint{3.756269in}{2.366732in}}%
\pgfpathlineto{\pgfqpoint{3.757084in}{2.449014in}}%
\pgfpathlineto{\pgfqpoint{3.757492in}{2.399645in}}%
\pgfpathlineto{\pgfqpoint{3.758918in}{2.333819in}}%
\pgfpathlineto{\pgfqpoint{3.759122in}{2.391416in}}%
\pgfpathlineto{\pgfqpoint{3.759326in}{2.309134in}}%
\pgfpathlineto{\pgfqpoint{3.759530in}{2.185711in}}%
\pgfpathlineto{\pgfqpoint{3.760141in}{2.374960in}}%
\pgfpathlineto{\pgfqpoint{3.760345in}{2.440786in}}%
\pgfpathlineto{\pgfqpoint{3.761160in}{2.358504in}}%
\pgfpathlineto{\pgfqpoint{3.761364in}{2.416101in}}%
\pgfpathlineto{\pgfqpoint{3.761772in}{2.399645in}}%
\pgfpathlineto{\pgfqpoint{3.761975in}{2.424329in}}%
\pgfpathlineto{\pgfqpoint{3.762587in}{2.111657in}}%
\pgfpathlineto{\pgfqpoint{3.762994in}{2.276221in}}%
\pgfpathlineto{\pgfqpoint{3.763810in}{2.449014in}}%
\pgfpathlineto{\pgfqpoint{3.764013in}{2.202168in}}%
\pgfpathlineto{\pgfqpoint{3.764829in}{2.407873in}}%
\pgfpathlineto{\pgfqpoint{3.765032in}{2.407873in}}%
\pgfpathlineto{\pgfqpoint{3.765440in}{2.374960in}}%
\pgfpathlineto{\pgfqpoint{3.766052in}{2.399645in}}%
\pgfpathlineto{\pgfqpoint{3.766459in}{2.416101in}}%
\pgfpathlineto{\pgfqpoint{3.766663in}{2.391416in}}%
\pgfpathlineto{\pgfqpoint{3.766867in}{2.399645in}}%
\pgfpathlineto{\pgfqpoint{3.767071in}{2.374960in}}%
\pgfpathlineto{\pgfqpoint{3.767274in}{2.103429in}}%
\pgfpathlineto{\pgfqpoint{3.768090in}{2.317363in}}%
\pgfpathlineto{\pgfqpoint{3.769109in}{2.383188in}}%
\pgfpathlineto{\pgfqpoint{3.769312in}{2.358504in}}%
\pgfpathlineto{\pgfqpoint{3.770128in}{2.284450in}}%
\pgfpathlineto{\pgfqpoint{3.770331in}{2.309134in}}%
\pgfpathlineto{\pgfqpoint{3.770535in}{2.383188in}}%
\pgfpathlineto{\pgfqpoint{3.771350in}{2.325591in}}%
\pgfpathlineto{\pgfqpoint{3.772573in}{2.251537in}}%
\pgfpathlineto{\pgfqpoint{3.772777in}{2.292678in}}%
\pgfpathlineto{\pgfqpoint{3.773592in}{2.259765in}}%
\pgfpathlineto{\pgfqpoint{3.773796in}{2.259765in}}%
\pgfpathlineto{\pgfqpoint{3.774000in}{2.300906in}}%
\pgfpathlineto{\pgfqpoint{3.774407in}{2.226852in}}%
\pgfpathlineto{\pgfqpoint{3.774815in}{2.243309in}}%
\pgfpathlineto{\pgfqpoint{3.776649in}{2.161027in}}%
\pgfpathlineto{\pgfqpoint{3.777872in}{2.243309in}}%
\pgfpathlineto{\pgfqpoint{3.778076in}{2.210396in}}%
\pgfpathlineto{\pgfqpoint{3.778280in}{2.169255in}}%
\pgfpathlineto{\pgfqpoint{3.778891in}{2.267993in}}%
\pgfpathlineto{\pgfqpoint{3.779503in}{2.276221in}}%
\pgfpathlineto{\pgfqpoint{3.780114in}{2.202168in}}%
\pgfpathlineto{\pgfqpoint{3.781541in}{2.309134in}}%
\pgfpathlineto{\pgfqpoint{3.781948in}{2.226852in}}%
\pgfpathlineto{\pgfqpoint{3.782152in}{2.119886in}}%
\pgfpathlineto{\pgfqpoint{3.782763in}{2.284450in}}%
\pgfpathlineto{\pgfqpoint{3.782967in}{2.235080in}}%
\pgfpathlineto{\pgfqpoint{3.783579in}{2.152798in}}%
\pgfpathlineto{\pgfqpoint{3.783782in}{2.169255in}}%
\pgfpathlineto{\pgfqpoint{3.783986in}{1.971778in}}%
\pgfpathlineto{\pgfqpoint{3.784394in}{2.259765in}}%
\pgfpathlineto{\pgfqpoint{3.784801in}{1.988234in}}%
\pgfpathlineto{\pgfqpoint{3.785005in}{2.243309in}}%
\pgfpathlineto{\pgfqpoint{3.785413in}{1.864811in}}%
\pgfpathlineto{\pgfqpoint{3.786024in}{2.161027in}}%
\pgfpathlineto{\pgfqpoint{3.786432in}{2.193939in}}%
\pgfpathlineto{\pgfqpoint{3.786839in}{2.136342in}}%
\pgfpathlineto{\pgfqpoint{3.787043in}{2.070516in}}%
\pgfpathlineto{\pgfqpoint{3.787247in}{2.235080in}}%
\pgfpathlineto{\pgfqpoint{3.787655in}{2.161027in}}%
\pgfpathlineto{\pgfqpoint{3.789489in}{2.350275in}}%
\pgfpathlineto{\pgfqpoint{3.789693in}{2.309134in}}%
\pgfpathlineto{\pgfqpoint{3.790304in}{2.383188in}}%
\pgfpathlineto{\pgfqpoint{3.790508in}{2.399645in}}%
\pgfpathlineto{\pgfqpoint{3.791527in}{2.235080in}}%
\pgfpathlineto{\pgfqpoint{3.791934in}{2.251537in}}%
\pgfpathlineto{\pgfqpoint{3.792954in}{2.243309in}}%
\pgfpathlineto{\pgfqpoint{3.793769in}{2.342047in}}%
\pgfpathlineto{\pgfqpoint{3.793973in}{2.309134in}}%
\pgfpathlineto{\pgfqpoint{3.794176in}{2.350275in}}%
\pgfpathlineto{\pgfqpoint{3.794584in}{2.325591in}}%
\pgfpathlineto{\pgfqpoint{3.794788in}{2.399645in}}%
\pgfpathlineto{\pgfqpoint{3.795603in}{2.333819in}}%
\pgfpathlineto{\pgfqpoint{3.797030in}{2.399645in}}%
\pgfpathlineto{\pgfqpoint{3.797845in}{2.267993in}}%
\pgfpathlineto{\pgfqpoint{3.798252in}{2.284450in}}%
\pgfpathlineto{\pgfqpoint{3.798660in}{2.267993in}}%
\pgfpathlineto{\pgfqpoint{3.799068in}{2.111657in}}%
\pgfpathlineto{\pgfqpoint{3.799271in}{2.350275in}}%
\pgfpathlineto{\pgfqpoint{3.799475in}{2.325591in}}%
\pgfpathlineto{\pgfqpoint{3.800290in}{2.383188in}}%
\pgfpathlineto{\pgfqpoint{3.800698in}{2.342047in}}%
\pgfpathlineto{\pgfqpoint{3.802125in}{2.243309in}}%
\pgfpathlineto{\pgfqpoint{3.801106in}{2.350275in}}%
\pgfpathlineto{\pgfqpoint{3.802328in}{2.267993in}}%
\pgfpathlineto{\pgfqpoint{3.803144in}{2.342047in}}%
\pgfpathlineto{\pgfqpoint{3.803551in}{2.062288in}}%
\pgfpathlineto{\pgfqpoint{3.804163in}{2.366732in}}%
\pgfpathlineto{\pgfqpoint{3.804570in}{2.086973in}}%
\pgfpathlineto{\pgfqpoint{3.805385in}{2.325591in}}%
\pgfpathlineto{\pgfqpoint{3.806405in}{2.202168in}}%
\pgfpathlineto{\pgfqpoint{3.806812in}{2.243309in}}%
\pgfpathlineto{\pgfqpoint{3.807016in}{2.317363in}}%
\pgfpathlineto{\pgfqpoint{3.808035in}{2.292678in}}%
\pgfpathlineto{\pgfqpoint{3.808239in}{2.292678in}}%
\pgfpathlineto{\pgfqpoint{3.808443in}{2.267993in}}%
\pgfpathlineto{\pgfqpoint{3.808646in}{2.333819in}}%
\pgfpathlineto{\pgfqpoint{3.808850in}{2.317363in}}%
\pgfpathlineto{\pgfqpoint{3.809462in}{2.300906in}}%
\pgfpathlineto{\pgfqpoint{3.809869in}{2.342047in}}%
\pgfpathlineto{\pgfqpoint{3.810684in}{2.210396in}}%
\pgfpathlineto{\pgfqpoint{3.810888in}{2.276221in}}%
\pgfpathlineto{\pgfqpoint{3.812315in}{2.366732in}}%
\pgfpathlineto{\pgfqpoint{3.813945in}{2.284450in}}%
\pgfpathlineto{\pgfqpoint{3.814149in}{2.325591in}}%
\pgfpathlineto{\pgfqpoint{3.814557in}{2.243309in}}%
\pgfpathlineto{\pgfqpoint{3.814964in}{2.284450in}}%
\pgfpathlineto{\pgfqpoint{3.815168in}{2.259765in}}%
\pgfpathlineto{\pgfqpoint{3.815372in}{2.317363in}}%
\pgfpathlineto{\pgfqpoint{3.815576in}{2.317363in}}%
\pgfpathlineto{\pgfqpoint{3.815983in}{2.366732in}}%
\pgfpathlineto{\pgfqpoint{3.816391in}{2.333819in}}%
\pgfpathlineto{\pgfqpoint{3.817410in}{2.226852in}}%
\pgfpathlineto{\pgfqpoint{3.817614in}{2.259765in}}%
\pgfpathlineto{\pgfqpoint{3.819040in}{2.366732in}}%
\pgfpathlineto{\pgfqpoint{3.819244in}{2.416101in}}%
\pgfpathlineto{\pgfqpoint{3.819856in}{2.325591in}}%
\pgfpathlineto{\pgfqpoint{3.820059in}{2.333819in}}%
\pgfpathlineto{\pgfqpoint{3.820263in}{2.325591in}}%
\pgfpathlineto{\pgfqpoint{3.820467in}{2.251537in}}%
\pgfpathlineto{\pgfqpoint{3.821282in}{2.292678in}}%
\pgfpathlineto{\pgfqpoint{3.822301in}{2.342047in}}%
\pgfpathlineto{\pgfqpoint{3.822913in}{2.218624in}}%
\pgfpathlineto{\pgfqpoint{3.823524in}{2.317363in}}%
\pgfpathlineto{\pgfqpoint{3.825562in}{2.473698in}}%
\pgfpathlineto{\pgfqpoint{3.826173in}{2.342047in}}%
\pgfpathlineto{\pgfqpoint{3.826785in}{2.416101in}}%
\pgfpathlineto{\pgfqpoint{3.826989in}{2.424329in}}%
\pgfpathlineto{\pgfqpoint{3.827192in}{2.391416in}}%
\pgfpathlineto{\pgfqpoint{3.827804in}{2.366732in}}%
\pgfpathlineto{\pgfqpoint{3.828415in}{2.432557in}}%
\pgfpathlineto{\pgfqpoint{3.828823in}{2.391416in}}%
\pgfpathlineto{\pgfqpoint{3.829027in}{2.358504in}}%
\pgfpathlineto{\pgfqpoint{3.829434in}{2.432557in}}%
\pgfpathlineto{\pgfqpoint{3.829638in}{2.440786in}}%
\pgfpathlineto{\pgfqpoint{3.829842in}{2.391416in}}%
\pgfpathlineto{\pgfqpoint{3.830453in}{2.473698in}}%
\pgfpathlineto{\pgfqpoint{3.830657in}{2.432557in}}%
\pgfpathlineto{\pgfqpoint{3.830861in}{2.465470in}}%
\pgfpathlineto{\pgfqpoint{3.831065in}{2.416101in}}%
\pgfpathlineto{\pgfqpoint{3.831472in}{2.416101in}}%
\pgfpathlineto{\pgfqpoint{3.831676in}{2.416101in}}%
\pgfpathlineto{\pgfqpoint{3.832287in}{2.391416in}}%
\pgfpathlineto{\pgfqpoint{3.832491in}{2.457242in}}%
\pgfpathlineto{\pgfqpoint{3.832899in}{2.383188in}}%
\pgfpathlineto{\pgfqpoint{3.833510in}{2.432557in}}%
\pgfpathlineto{\pgfqpoint{3.833918in}{2.383188in}}%
\pgfpathlineto{\pgfqpoint{3.834122in}{2.424329in}}%
\pgfpathlineto{\pgfqpoint{3.835141in}{2.473698in}}%
\pgfpathlineto{\pgfqpoint{3.835548in}{2.465470in}}%
\pgfpathlineto{\pgfqpoint{3.836771in}{2.383188in}}%
\pgfpathlineto{\pgfqpoint{3.837179in}{2.407873in}}%
\pgfpathlineto{\pgfqpoint{3.837383in}{2.465470in}}%
\pgfpathlineto{\pgfqpoint{3.837790in}{2.374960in}}%
\pgfpathlineto{\pgfqpoint{3.838198in}{2.416101in}}%
\pgfpathlineto{\pgfqpoint{3.839828in}{2.333819in}}%
\pgfpathlineto{\pgfqpoint{3.840236in}{2.251537in}}%
\pgfpathlineto{\pgfqpoint{3.840847in}{2.317363in}}%
\pgfpathlineto{\pgfqpoint{3.842070in}{2.391416in}}%
\pgfpathlineto{\pgfqpoint{3.842274in}{2.383188in}}%
\pgfpathlineto{\pgfqpoint{3.842885in}{2.366732in}}%
\pgfpathlineto{\pgfqpoint{3.843089in}{2.399645in}}%
\pgfpathlineto{\pgfqpoint{3.843497in}{2.374960in}}%
\pgfpathlineto{\pgfqpoint{3.843700in}{2.399645in}}%
\pgfpathlineto{\pgfqpoint{3.843904in}{2.432557in}}%
\pgfpathlineto{\pgfqpoint{3.844312in}{2.358504in}}%
\pgfpathlineto{\pgfqpoint{3.845331in}{2.259765in}}%
\pgfpathlineto{\pgfqpoint{3.845738in}{2.292678in}}%
\pgfpathlineto{\pgfqpoint{3.846146in}{2.251537in}}%
\pgfpathlineto{\pgfqpoint{3.846350in}{2.350275in}}%
\pgfpathlineto{\pgfqpoint{3.847165in}{2.333819in}}%
\pgfpathlineto{\pgfqpoint{3.847573in}{2.218624in}}%
\pgfpathlineto{\pgfqpoint{3.848388in}{2.235080in}}%
\pgfpathlineto{\pgfqpoint{3.849815in}{2.317363in}}%
\pgfpathlineto{\pgfqpoint{3.850018in}{2.317363in}}%
\pgfpathlineto{\pgfqpoint{3.850834in}{2.342047in}}%
\pgfpathlineto{\pgfqpoint{3.851241in}{2.243309in}}%
\pgfpathlineto{\pgfqpoint{3.852056in}{2.284450in}}%
\pgfpathlineto{\pgfqpoint{3.852872in}{2.243309in}}%
\pgfpathlineto{\pgfqpoint{3.853279in}{2.259765in}}%
\pgfpathlineto{\pgfqpoint{3.854298in}{2.424329in}}%
\pgfpathlineto{\pgfqpoint{3.854502in}{2.111657in}}%
\pgfpathlineto{\pgfqpoint{3.855317in}{2.350275in}}%
\pgfpathlineto{\pgfqpoint{3.855521in}{2.374960in}}%
\pgfpathlineto{\pgfqpoint{3.855929in}{2.226852in}}%
\pgfpathlineto{\pgfqpoint{3.856540in}{2.300906in}}%
\pgfpathlineto{\pgfqpoint{3.856948in}{2.333819in}}%
\pgfpathlineto{\pgfqpoint{3.857151in}{2.152798in}}%
\pgfpathlineto{\pgfqpoint{3.857967in}{2.383188in}}%
\pgfpathlineto{\pgfqpoint{3.858374in}{2.292678in}}%
\pgfpathlineto{\pgfqpoint{3.858986in}{2.374960in}}%
\pgfpathlineto{\pgfqpoint{3.859189in}{2.358504in}}%
\pgfpathlineto{\pgfqpoint{3.859597in}{2.144570in}}%
\pgfpathlineto{\pgfqpoint{3.860209in}{2.350275in}}%
\pgfpathlineto{\pgfqpoint{3.860820in}{2.374960in}}%
\pgfpathlineto{\pgfqpoint{3.861635in}{2.284450in}}%
\pgfpathlineto{\pgfqpoint{3.861839in}{2.284450in}}%
\pgfpathlineto{\pgfqpoint{3.862043in}{2.078745in}}%
\pgfpathlineto{\pgfqpoint{3.862654in}{2.407873in}}%
\pgfpathlineto{\pgfqpoint{3.862858in}{2.333819in}}%
\pgfpathlineto{\pgfqpoint{3.863266in}{2.358504in}}%
\pgfpathlineto{\pgfqpoint{3.863469in}{2.317363in}}%
\pgfpathlineto{\pgfqpoint{3.864081in}{2.424329in}}%
\pgfpathlineto{\pgfqpoint{3.864285in}{2.300906in}}%
\pgfpathlineto{\pgfqpoint{3.864488in}{2.300906in}}%
\pgfpathlineto{\pgfqpoint{3.864692in}{2.325591in}}%
\pgfpathlineto{\pgfqpoint{3.865304in}{2.309134in}}%
\pgfpathlineto{\pgfqpoint{3.866323in}{2.169255in}}%
\pgfpathlineto{\pgfqpoint{3.867138in}{2.210396in}}%
\pgfpathlineto{\pgfqpoint{3.868564in}{2.333819in}}%
\pgfpathlineto{\pgfqpoint{3.867953in}{2.202168in}}%
\pgfpathlineto{\pgfqpoint{3.868768in}{2.325591in}}%
\pgfpathlineto{\pgfqpoint{3.868972in}{2.342047in}}%
\pgfpathlineto{\pgfqpoint{3.869176in}{2.300906in}}%
\pgfpathlineto{\pgfqpoint{3.869380in}{2.259765in}}%
\pgfpathlineto{\pgfqpoint{3.869991in}{2.333819in}}%
\pgfpathlineto{\pgfqpoint{3.870195in}{2.350275in}}%
\pgfpathlineto{\pgfqpoint{3.870602in}{2.309134in}}%
\pgfpathlineto{\pgfqpoint{3.871418in}{2.243309in}}%
\pgfpathlineto{\pgfqpoint{3.871214in}{2.350275in}}%
\pgfpathlineto{\pgfqpoint{3.871621in}{2.251537in}}%
\pgfpathlineto{\pgfqpoint{3.871825in}{2.300906in}}%
\pgfpathlineto{\pgfqpoint{3.872233in}{2.235080in}}%
\pgfpathlineto{\pgfqpoint{3.872437in}{2.243309in}}%
\pgfpathlineto{\pgfqpoint{3.873048in}{2.161027in}}%
\pgfpathlineto{\pgfqpoint{3.873456in}{2.210396in}}%
\pgfpathlineto{\pgfqpoint{3.874271in}{2.276221in}}%
\pgfpathlineto{\pgfqpoint{3.874475in}{2.226852in}}%
\pgfpathlineto{\pgfqpoint{3.875086in}{2.202168in}}%
\pgfpathlineto{\pgfqpoint{3.874882in}{2.235080in}}%
\pgfpathlineto{\pgfqpoint{3.875290in}{2.226852in}}%
\pgfpathlineto{\pgfqpoint{3.875901in}{2.267993in}}%
\pgfpathlineto{\pgfqpoint{3.876717in}{2.259765in}}%
\pgfpathlineto{\pgfqpoint{3.877939in}{2.119886in}}%
\pgfpathlineto{\pgfqpoint{3.878347in}{2.177483in}}%
\pgfpathlineto{\pgfqpoint{3.878755in}{2.062288in}}%
\pgfpathlineto{\pgfqpoint{3.879366in}{2.095201in}}%
\pgfpathlineto{\pgfqpoint{3.880181in}{1.790758in}}%
\pgfpathlineto{\pgfqpoint{3.881200in}{2.021147in}}%
\pgfpathlineto{\pgfqpoint{3.881404in}{1.988234in}}%
\pgfpathlineto{\pgfqpoint{3.881608in}{2.012919in}}%
\pgfpathlineto{\pgfqpoint{3.881812in}{1.947093in}}%
\pgfpathlineto{\pgfqpoint{3.882015in}{1.980006in}}%
\pgfpathlineto{\pgfqpoint{3.883034in}{1.848355in}}%
\pgfpathlineto{\pgfqpoint{3.883646in}{1.856583in}}%
\pgfpathlineto{\pgfqpoint{3.884461in}{1.930637in}}%
\pgfpathlineto{\pgfqpoint{3.884869in}{2.070516in}}%
\pgfpathlineto{\pgfqpoint{3.885480in}{1.922409in}}%
\pgfpathlineto{\pgfqpoint{3.885684in}{1.930637in}}%
\pgfpathlineto{\pgfqpoint{3.885888in}{2.054060in}}%
\pgfpathlineto{\pgfqpoint{3.886703in}{1.980006in}}%
\pgfpathlineto{\pgfqpoint{3.886907in}{1.988234in}}%
\pgfpathlineto{\pgfqpoint{3.887111in}{1.971778in}}%
\pgfpathlineto{\pgfqpoint{3.887518in}{1.947093in}}%
\pgfpathlineto{\pgfqpoint{3.887926in}{1.980006in}}%
\pgfpathlineto{\pgfqpoint{3.888945in}{2.152798in}}%
\pgfpathlineto{\pgfqpoint{3.889352in}{2.103429in}}%
\pgfpathlineto{\pgfqpoint{3.890371in}{2.045832in}}%
\pgfpathlineto{\pgfqpoint{3.891187in}{2.095201in}}%
\pgfpathlineto{\pgfqpoint{3.891390in}{2.070516in}}%
\pgfpathlineto{\pgfqpoint{3.892002in}{2.095201in}}%
\pgfpathlineto{\pgfqpoint{3.892409in}{2.045832in}}%
\pgfpathlineto{\pgfqpoint{3.893225in}{2.095201in}}%
\pgfpathlineto{\pgfqpoint{3.893428in}{2.037604in}}%
\pgfpathlineto{\pgfqpoint{3.893632in}{2.029375in}}%
\pgfpathlineto{\pgfqpoint{3.894040in}{2.103429in}}%
\pgfpathlineto{\pgfqpoint{3.894651in}{2.012919in}}%
\pgfpathlineto{\pgfqpoint{3.895263in}{2.070516in}}%
\pgfpathlineto{\pgfqpoint{3.895874in}{2.045832in}}%
\pgfpathlineto{\pgfqpoint{3.896282in}{1.938865in}}%
\pgfpathlineto{\pgfqpoint{3.896893in}{2.045832in}}%
\pgfpathlineto{\pgfqpoint{3.897097in}{2.037604in}}%
\pgfpathlineto{\pgfqpoint{3.897301in}{2.086973in}}%
\pgfpathlineto{\pgfqpoint{3.897504in}{1.955322in}}%
\pgfpathlineto{\pgfqpoint{3.897912in}{1.971778in}}%
\pgfpathlineto{\pgfqpoint{3.898320in}{1.955322in}}%
\pgfpathlineto{\pgfqpoint{3.898523in}{2.045832in}}%
\pgfpathlineto{\pgfqpoint{3.899339in}{1.971778in}}%
\pgfpathlineto{\pgfqpoint{3.899950in}{1.980006in}}%
\pgfpathlineto{\pgfqpoint{3.900562in}{1.914181in}}%
\pgfpathlineto{\pgfqpoint{3.900765in}{1.905952in}}%
\pgfpathlineto{\pgfqpoint{3.900969in}{1.971778in}}%
\pgfpathlineto{\pgfqpoint{3.901377in}{1.955322in}}%
\pgfpathlineto{\pgfqpoint{3.901988in}{2.021147in}}%
\pgfpathlineto{\pgfqpoint{3.902600in}{1.782529in}}%
\pgfpathlineto{\pgfqpoint{3.903415in}{2.086973in}}%
\pgfpathlineto{\pgfqpoint{3.903822in}{2.012919in}}%
\pgfpathlineto{\pgfqpoint{3.904026in}{1.980006in}}%
\pgfpathlineto{\pgfqpoint{3.904230in}{2.045832in}}%
\pgfpathlineto{\pgfqpoint{3.904434in}{2.021147in}}%
\pgfpathlineto{\pgfqpoint{3.905249in}{2.086973in}}%
\pgfpathlineto{\pgfqpoint{3.905657in}{1.782529in}}%
\pgfpathlineto{\pgfqpoint{3.906268in}{2.054060in}}%
\pgfpathlineto{\pgfqpoint{3.906676in}{2.029375in}}%
\pgfpathlineto{\pgfqpoint{3.907083in}{1.930637in}}%
\pgfpathlineto{\pgfqpoint{3.907695in}{2.004691in}}%
\pgfpathlineto{\pgfqpoint{3.908102in}{2.045832in}}%
\pgfpathlineto{\pgfqpoint{3.908306in}{2.021147in}}%
\pgfpathlineto{\pgfqpoint{3.908510in}{1.955322in}}%
\pgfpathlineto{\pgfqpoint{3.909121in}{2.054060in}}%
\pgfpathlineto{\pgfqpoint{3.909325in}{2.054060in}}%
\pgfpathlineto{\pgfqpoint{3.909529in}{2.086973in}}%
\pgfpathlineto{\pgfqpoint{3.909936in}{1.980006in}}%
\pgfpathlineto{\pgfqpoint{3.910344in}{2.045832in}}%
\pgfpathlineto{\pgfqpoint{3.910548in}{2.062288in}}%
\pgfpathlineto{\pgfqpoint{3.910955in}{1.766073in}}%
\pgfpathlineto{\pgfqpoint{3.911771in}{1.955322in}}%
\pgfpathlineto{\pgfqpoint{3.912790in}{2.004691in}}%
\pgfpathlineto{\pgfqpoint{3.912993in}{1.980006in}}%
\pgfpathlineto{\pgfqpoint{3.913401in}{2.029375in}}%
\pgfpathlineto{\pgfqpoint{3.913605in}{2.045832in}}%
\pgfpathlineto{\pgfqpoint{3.914420in}{2.029375in}}%
\pgfpathlineto{\pgfqpoint{3.914624in}{2.021147in}}%
\pgfpathlineto{\pgfqpoint{3.915032in}{1.757845in}}%
\pgfpathlineto{\pgfqpoint{3.915847in}{1.774301in}}%
\pgfpathlineto{\pgfqpoint{3.916866in}{1.963550in}}%
\pgfpathlineto{\pgfqpoint{3.917070in}{1.938865in}}%
\pgfpathlineto{\pgfqpoint{3.918089in}{1.889496in}}%
\pgfpathlineto{\pgfqpoint{3.918904in}{2.045832in}}%
\pgfpathlineto{\pgfqpoint{3.919515in}{1.988234in}}%
\pgfpathlineto{\pgfqpoint{3.920127in}{1.856583in}}%
\pgfpathlineto{\pgfqpoint{3.920534in}{1.930637in}}%
\pgfpathlineto{\pgfqpoint{3.921146in}{1.996463in}}%
\pgfpathlineto{\pgfqpoint{3.921757in}{1.955322in}}%
\pgfpathlineto{\pgfqpoint{3.923184in}{1.897724in}}%
\pgfpathlineto{\pgfqpoint{3.923591in}{1.922409in}}%
\pgfpathlineto{\pgfqpoint{3.923795in}{1.947093in}}%
\pgfpathlineto{\pgfqpoint{3.924203in}{1.905952in}}%
\pgfpathlineto{\pgfqpoint{3.924406in}{1.864811in}}%
\pgfpathlineto{\pgfqpoint{3.924814in}{1.938865in}}%
\pgfpathlineto{\pgfqpoint{3.925018in}{1.930637in}}%
\pgfpathlineto{\pgfqpoint{3.925222in}{1.938865in}}%
\pgfpathlineto{\pgfqpoint{3.925629in}{1.700247in}}%
\pgfpathlineto{\pgfqpoint{3.926444in}{1.873040in}}%
\pgfpathlineto{\pgfqpoint{3.927260in}{1.922409in}}%
\pgfpathlineto{\pgfqpoint{3.927667in}{1.790758in}}%
\pgfpathlineto{\pgfqpoint{3.928483in}{1.831899in}}%
\pgfpathlineto{\pgfqpoint{3.928686in}{1.831899in}}%
\pgfpathlineto{\pgfqpoint{3.929909in}{1.955322in}}%
\pgfpathlineto{\pgfqpoint{3.930113in}{1.914181in}}%
\pgfpathlineto{\pgfqpoint{3.930317in}{1.914181in}}%
\pgfpathlineto{\pgfqpoint{3.931132in}{1.873040in}}%
\pgfpathlineto{\pgfqpoint{3.930928in}{1.922409in}}%
\pgfpathlineto{\pgfqpoint{3.931336in}{1.889496in}}%
\pgfpathlineto{\pgfqpoint{3.931743in}{1.683791in}}%
\pgfpathlineto{\pgfqpoint{3.932151in}{1.905952in}}%
\pgfpathlineto{\pgfqpoint{3.932355in}{1.881268in}}%
\pgfpathlineto{\pgfqpoint{3.932762in}{1.955322in}}%
\pgfpathlineto{\pgfqpoint{3.933374in}{1.938865in}}%
\pgfpathlineto{\pgfqpoint{3.933578in}{1.897724in}}%
\pgfpathlineto{\pgfqpoint{3.933985in}{1.947093in}}%
\pgfpathlineto{\pgfqpoint{3.934393in}{2.021147in}}%
\pgfpathlineto{\pgfqpoint{3.934800in}{1.930637in}}%
\pgfpathlineto{\pgfqpoint{3.935004in}{1.971778in}}%
\pgfpathlineto{\pgfqpoint{3.936023in}{1.889496in}}%
\pgfpathlineto{\pgfqpoint{3.936227in}{1.922409in}}%
\pgfpathlineto{\pgfqpoint{3.936431in}{1.963550in}}%
\pgfpathlineto{\pgfqpoint{3.937246in}{1.914181in}}%
\pgfpathlineto{\pgfqpoint{3.937654in}{2.012919in}}%
\pgfpathlineto{\pgfqpoint{3.938061in}{1.930637in}}%
\pgfpathlineto{\pgfqpoint{3.938265in}{1.905952in}}%
\pgfpathlineto{\pgfqpoint{3.938469in}{1.980006in}}%
\pgfpathlineto{\pgfqpoint{3.938673in}{1.947093in}}%
\pgfpathlineto{\pgfqpoint{3.939284in}{2.012919in}}%
\pgfpathlineto{\pgfqpoint{3.939080in}{1.930637in}}%
\pgfpathlineto{\pgfqpoint{3.939692in}{2.004691in}}%
\pgfpathlineto{\pgfqpoint{3.940915in}{1.881268in}}%
\pgfpathlineto{\pgfqpoint{3.941118in}{1.881268in}}%
\pgfpathlineto{\pgfqpoint{3.941322in}{1.642650in}}%
\pgfpathlineto{\pgfqpoint{3.942137in}{1.955322in}}%
\pgfpathlineto{\pgfqpoint{3.942545in}{1.889496in}}%
\pgfpathlineto{\pgfqpoint{3.942953in}{1.798986in}}%
\pgfpathlineto{\pgfqpoint{3.943564in}{1.897724in}}%
\pgfpathlineto{\pgfqpoint{3.943768in}{1.897724in}}%
\pgfpathlineto{\pgfqpoint{3.944175in}{1.848355in}}%
\pgfpathlineto{\pgfqpoint{3.945398in}{1.757845in}}%
\pgfpathlineto{\pgfqpoint{3.945602in}{1.807214in}}%
\pgfpathlineto{\pgfqpoint{3.945806in}{1.724932in}}%
\pgfpathlineto{\pgfqpoint{3.946417in}{1.782529in}}%
\pgfpathlineto{\pgfqpoint{3.947436in}{1.708476in}}%
\pgfpathlineto{\pgfqpoint{3.947640in}{1.733160in}}%
\pgfpathlineto{\pgfqpoint{3.948455in}{1.790758in}}%
\pgfpathlineto{\pgfqpoint{3.948659in}{1.782529in}}%
\pgfpathlineto{\pgfqpoint{3.948863in}{1.840127in}}%
\pgfpathlineto{\pgfqpoint{3.949270in}{1.766073in}}%
\pgfpathlineto{\pgfqpoint{3.949474in}{1.593281in}}%
\pgfpathlineto{\pgfqpoint{3.949678in}{1.848355in}}%
\pgfpathlineto{\pgfqpoint{3.950289in}{1.815442in}}%
\pgfpathlineto{\pgfqpoint{3.951308in}{1.955322in}}%
\pgfpathlineto{\pgfqpoint{3.951716in}{1.905952in}}%
\pgfpathlineto{\pgfqpoint{3.953958in}{1.683791in}}%
\pgfpathlineto{\pgfqpoint{3.954366in}{1.757845in}}%
\pgfpathlineto{\pgfqpoint{3.954569in}{1.502771in}}%
\pgfpathlineto{\pgfqpoint{3.955181in}{1.831899in}}%
\pgfpathlineto{\pgfqpoint{3.955385in}{1.807214in}}%
\pgfpathlineto{\pgfqpoint{3.955588in}{1.881268in}}%
\pgfpathlineto{\pgfqpoint{3.955996in}{1.741388in}}%
\pgfpathlineto{\pgfqpoint{3.956200in}{1.741388in}}%
\pgfpathlineto{\pgfqpoint{3.956404in}{1.724932in}}%
\pgfpathlineto{\pgfqpoint{3.956607in}{1.823670in}}%
\pgfpathlineto{\pgfqpoint{3.957626in}{1.790758in}}%
\pgfpathlineto{\pgfqpoint{3.957830in}{1.733160in}}%
\pgfpathlineto{\pgfqpoint{3.958238in}{1.815442in}}%
\pgfpathlineto{\pgfqpoint{3.958645in}{1.757845in}}%
\pgfpathlineto{\pgfqpoint{3.958849in}{1.782529in}}%
\pgfpathlineto{\pgfqpoint{3.959053in}{1.683791in}}%
\pgfpathlineto{\pgfqpoint{3.959257in}{1.716704in}}%
\pgfpathlineto{\pgfqpoint{3.959461in}{1.675563in}}%
\pgfpathlineto{\pgfqpoint{3.959664in}{1.831899in}}%
\pgfpathlineto{\pgfqpoint{3.960072in}{1.774301in}}%
\pgfpathlineto{\pgfqpoint{3.961295in}{1.840127in}}%
\pgfpathlineto{\pgfqpoint{3.960480in}{1.749617in}}%
\pgfpathlineto{\pgfqpoint{3.961499in}{1.823670in}}%
\pgfpathlineto{\pgfqpoint{3.962314in}{1.766073in}}%
\pgfpathlineto{\pgfqpoint{3.961906in}{1.831899in}}%
\pgfpathlineto{\pgfqpoint{3.962518in}{1.798986in}}%
\pgfpathlineto{\pgfqpoint{3.962925in}{1.889496in}}%
\pgfpathlineto{\pgfqpoint{3.963129in}{1.807214in}}%
\pgfpathlineto{\pgfqpoint{3.963333in}{1.757845in}}%
\pgfpathlineto{\pgfqpoint{3.963944in}{1.864811in}}%
\pgfpathlineto{\pgfqpoint{3.964148in}{1.864811in}}%
\pgfpathlineto{\pgfqpoint{3.965167in}{1.831899in}}%
\pgfpathlineto{\pgfqpoint{3.965371in}{1.873040in}}%
\pgfpathlineto{\pgfqpoint{3.965778in}{1.807214in}}%
\pgfpathlineto{\pgfqpoint{3.967205in}{1.568596in}}%
\pgfpathlineto{\pgfqpoint{3.968632in}{1.749617in}}%
\pgfpathlineto{\pgfqpoint{3.968836in}{1.708476in}}%
\pgfpathlineto{\pgfqpoint{3.970466in}{1.535683in}}%
\pgfpathlineto{\pgfqpoint{3.969447in}{1.733160in}}%
\pgfpathlineto{\pgfqpoint{3.970874in}{1.576824in}}%
\pgfpathlineto{\pgfqpoint{3.971689in}{1.733160in}}%
\pgfpathlineto{\pgfqpoint{3.972504in}{1.675563in}}%
\pgfpathlineto{\pgfqpoint{3.972708in}{1.626194in}}%
\pgfpathlineto{\pgfqpoint{3.973115in}{1.724932in}}%
\pgfpathlineto{\pgfqpoint{3.973319in}{1.700247in}}%
\pgfpathlineto{\pgfqpoint{3.975357in}{1.856583in}}%
\pgfpathlineto{\pgfqpoint{3.976376in}{1.716704in}}%
\pgfpathlineto{\pgfqpoint{3.976580in}{1.733160in}}%
\pgfpathlineto{\pgfqpoint{3.976784in}{1.749617in}}%
\pgfpathlineto{\pgfqpoint{3.977191in}{1.708476in}}%
\pgfpathlineto{\pgfqpoint{3.977395in}{1.716704in}}%
\pgfpathlineto{\pgfqpoint{3.977803in}{1.683791in}}%
\pgfpathlineto{\pgfqpoint{3.978007in}{1.774301in}}%
\pgfpathlineto{\pgfqpoint{3.978618in}{1.700247in}}%
\pgfpathlineto{\pgfqpoint{3.980045in}{1.593281in}}%
\pgfpathlineto{\pgfqpoint{3.981064in}{1.700247in}}%
\pgfpathlineto{\pgfqpoint{3.981268in}{1.650878in}}%
\pgfpathlineto{\pgfqpoint{3.981471in}{1.626194in}}%
\pgfpathlineto{\pgfqpoint{3.981879in}{1.708476in}}%
\pgfpathlineto{\pgfqpoint{3.982898in}{1.642650in}}%
\pgfpathlineto{\pgfqpoint{3.983102in}{1.716704in}}%
\pgfpathlineto{\pgfqpoint{3.983509in}{1.601509in}}%
\pgfpathlineto{\pgfqpoint{3.983917in}{1.675563in}}%
\pgfpathlineto{\pgfqpoint{3.984121in}{1.675563in}}%
\pgfpathlineto{\pgfqpoint{3.984325in}{1.659106in}}%
\pgfpathlineto{\pgfqpoint{3.984528in}{1.683791in}}%
\pgfpathlineto{\pgfqpoint{3.984732in}{1.683791in}}%
\pgfpathlineto{\pgfqpoint{3.985547in}{1.741388in}}%
\pgfpathlineto{\pgfqpoint{3.985955in}{1.716704in}}%
\pgfpathlineto{\pgfqpoint{3.986363in}{1.659106in}}%
\pgfpathlineto{\pgfqpoint{3.986566in}{1.741388in}}%
\pgfpathlineto{\pgfqpoint{3.986974in}{1.683791in}}%
\pgfpathlineto{\pgfqpoint{3.987382in}{1.675563in}}%
\pgfpathlineto{\pgfqpoint{3.987789in}{1.733160in}}%
\pgfpathlineto{\pgfqpoint{3.989012in}{1.626194in}}%
\pgfpathlineto{\pgfqpoint{3.989420in}{1.741388in}}%
\pgfpathlineto{\pgfqpoint{3.990439in}{1.683791in}}%
\pgfpathlineto{\pgfqpoint{3.990846in}{1.733160in}}%
\pgfpathlineto{\pgfqpoint{3.991254in}{1.675563in}}%
\pgfpathlineto{\pgfqpoint{3.991458in}{1.716704in}}%
\pgfpathlineto{\pgfqpoint{3.992680in}{1.601509in}}%
\pgfpathlineto{\pgfqpoint{3.992884in}{1.659106in}}%
\pgfpathlineto{\pgfqpoint{3.993699in}{1.617965in}}%
\pgfpathlineto{\pgfqpoint{3.994107in}{1.675563in}}%
\pgfpathlineto{\pgfqpoint{3.995534in}{1.790758in}}%
\pgfpathlineto{\pgfqpoint{3.995738in}{1.741388in}}%
\pgfpathlineto{\pgfqpoint{3.996349in}{1.774301in}}%
\pgfpathlineto{\pgfqpoint{3.996757in}{1.692019in}}%
\pgfpathlineto{\pgfqpoint{3.996960in}{1.733160in}}%
\pgfpathlineto{\pgfqpoint{3.997572in}{1.667335in}}%
\pgfpathlineto{\pgfqpoint{3.997979in}{1.626194in}}%
\pgfpathlineto{\pgfqpoint{3.998591in}{1.675563in}}%
\pgfpathlineto{\pgfqpoint{3.998795in}{1.683791in}}%
\pgfpathlineto{\pgfqpoint{3.999406in}{1.708476in}}%
\pgfpathlineto{\pgfqpoint{3.999814in}{1.617965in}}%
\pgfpathlineto{\pgfqpoint{4.001036in}{1.692019in}}%
\pgfpathlineto{\pgfqpoint{4.002667in}{1.535683in}}%
\pgfpathlineto{\pgfqpoint{4.003278in}{1.708476in}}%
\pgfpathlineto{\pgfqpoint{4.003890in}{1.642650in}}%
\pgfpathlineto{\pgfqpoint{4.004093in}{1.634422in}}%
\pgfpathlineto{\pgfqpoint{4.004297in}{1.650878in}}%
\pgfpathlineto{\pgfqpoint{4.004501in}{1.692019in}}%
\pgfpathlineto{\pgfqpoint{4.004909in}{1.609737in}}%
\pgfpathlineto{\pgfqpoint{4.005112in}{1.371119in}}%
\pgfpathlineto{\pgfqpoint{4.005928in}{1.576824in}}%
\pgfpathlineto{\pgfqpoint{4.006131in}{1.568596in}}%
\pgfpathlineto{\pgfqpoint{4.006335in}{1.601509in}}%
\pgfpathlineto{\pgfqpoint{4.006539in}{1.601509in}}%
\pgfpathlineto{\pgfqpoint{4.007558in}{1.708476in}}%
\pgfpathlineto{\pgfqpoint{4.007762in}{1.634422in}}%
\pgfpathlineto{\pgfqpoint{4.007966in}{1.420489in}}%
\pgfpathlineto{\pgfqpoint{4.008170in}{1.659106in}}%
\pgfpathlineto{\pgfqpoint{4.008781in}{1.576824in}}%
\pgfpathlineto{\pgfqpoint{4.008985in}{1.552140in}}%
\pgfpathlineto{\pgfqpoint{4.009392in}{1.609737in}}%
\pgfpathlineto{\pgfqpoint{4.009596in}{1.609737in}}%
\pgfpathlineto{\pgfqpoint{4.010411in}{1.683791in}}%
\pgfpathlineto{\pgfqpoint{4.010615in}{1.642650in}}%
\pgfpathlineto{\pgfqpoint{4.011634in}{1.543912in}}%
\pgfpathlineto{\pgfqpoint{4.011023in}{1.683791in}}%
\pgfpathlineto{\pgfqpoint{4.011838in}{1.593281in}}%
\pgfpathlineto{\pgfqpoint{4.012449in}{1.675563in}}%
\pgfpathlineto{\pgfqpoint{4.012653in}{1.560368in}}%
\pgfpathlineto{\pgfqpoint{4.012857in}{1.519227in}}%
\pgfpathlineto{\pgfqpoint{4.013468in}{1.609737in}}%
\pgfpathlineto{\pgfqpoint{4.013672in}{1.609737in}}%
\pgfpathlineto{\pgfqpoint{4.014284in}{1.601509in}}%
\pgfpathlineto{\pgfqpoint{4.015303in}{1.692019in}}%
\pgfpathlineto{\pgfqpoint{4.016322in}{1.527455in}}%
\pgfpathlineto{\pgfqpoint{4.016729in}{1.626194in}}%
\pgfpathlineto{\pgfqpoint{4.018360in}{1.708476in}}%
\pgfpathlineto{\pgfqpoint{4.017544in}{1.617965in}}%
\pgfpathlineto{\pgfqpoint{4.018563in}{1.675563in}}%
\pgfpathlineto{\pgfqpoint{4.018971in}{1.609737in}}%
\pgfpathlineto{\pgfqpoint{4.019379in}{1.708476in}}%
\pgfpathlineto{\pgfqpoint{4.019582in}{1.683791in}}%
\pgfpathlineto{\pgfqpoint{4.020194in}{1.650878in}}%
\pgfpathlineto{\pgfqpoint{4.020805in}{1.774301in}}%
\pgfpathlineto{\pgfqpoint{4.021213in}{1.667335in}}%
\pgfpathlineto{\pgfqpoint{4.021417in}{1.659106in}}%
\pgfpathlineto{\pgfqpoint{4.021621in}{1.667335in}}%
\pgfpathlineto{\pgfqpoint{4.021824in}{1.692019in}}%
\pgfpathlineto{\pgfqpoint{4.022232in}{1.642650in}}%
\pgfpathlineto{\pgfqpoint{4.022640in}{1.675563in}}%
\pgfpathlineto{\pgfqpoint{4.023047in}{1.609737in}}%
\pgfpathlineto{\pgfqpoint{4.023455in}{1.626194in}}%
\pgfpathlineto{\pgfqpoint{4.023659in}{1.412260in}}%
\pgfpathlineto{\pgfqpoint{4.024474in}{1.667335in}}%
\pgfpathlineto{\pgfqpoint{4.025289in}{1.650878in}}%
\pgfpathlineto{\pgfqpoint{4.025493in}{1.724932in}}%
\pgfpathlineto{\pgfqpoint{4.025697in}{1.478086in}}%
\pgfpathlineto{\pgfqpoint{4.026512in}{1.667335in}}%
\pgfpathlineto{\pgfqpoint{4.026716in}{1.650878in}}%
\pgfpathlineto{\pgfqpoint{4.027123in}{1.700247in}}%
\pgfpathlineto{\pgfqpoint{4.027327in}{1.708476in}}%
\pgfpathlineto{\pgfqpoint{4.027531in}{1.700247in}}%
\pgfpathlineto{\pgfqpoint{4.027735in}{1.461630in}}%
\pgfpathlineto{\pgfqpoint{4.028142in}{1.716704in}}%
\pgfpathlineto{\pgfqpoint{4.028550in}{1.667335in}}%
\pgfpathlineto{\pgfqpoint{4.028754in}{1.683791in}}%
\pgfpathlineto{\pgfqpoint{4.028957in}{1.617965in}}%
\pgfpathlineto{\pgfqpoint{4.029569in}{1.667335in}}%
\pgfpathlineto{\pgfqpoint{4.029773in}{1.593281in}}%
\pgfpathlineto{\pgfqpoint{4.030792in}{1.617965in}}%
\pgfpathlineto{\pgfqpoint{4.030995in}{1.634422in}}%
\pgfpathlineto{\pgfqpoint{4.031199in}{1.617965in}}%
\pgfpathlineto{\pgfqpoint{4.031403in}{1.560368in}}%
\pgfpathlineto{\pgfqpoint{4.032218in}{1.568596in}}%
\pgfpathlineto{\pgfqpoint{4.033645in}{1.650878in}}%
\pgfpathlineto{\pgfqpoint{4.034460in}{1.601509in}}%
\pgfpathlineto{\pgfqpoint{4.034868in}{1.708476in}}%
\pgfpathlineto{\pgfqpoint{4.035479in}{1.601509in}}%
\pgfpathlineto{\pgfqpoint{4.035683in}{1.601509in}}%
\pgfpathlineto{\pgfqpoint{4.036906in}{1.478086in}}%
\pgfpathlineto{\pgfqpoint{4.037110in}{1.519227in}}%
\pgfpathlineto{\pgfqpoint{4.037313in}{1.510999in}}%
\pgfpathlineto{\pgfqpoint{4.037925in}{1.617965in}}%
\pgfpathlineto{\pgfqpoint{4.038536in}{1.576824in}}%
\pgfpathlineto{\pgfqpoint{4.039555in}{1.469858in}}%
\pgfpathlineto{\pgfqpoint{4.039759in}{1.535683in}}%
\pgfpathlineto{\pgfqpoint{4.041389in}{1.371119in}}%
\pgfpathlineto{\pgfqpoint{4.041797in}{1.420489in}}%
\pgfpathlineto{\pgfqpoint{4.043020in}{1.519227in}}%
\pgfpathlineto{\pgfqpoint{4.043224in}{1.510999in}}%
\pgfpathlineto{\pgfqpoint{4.043427in}{1.461630in}}%
\pgfpathlineto{\pgfqpoint{4.043631in}{1.527455in}}%
\pgfpathlineto{\pgfqpoint{4.044243in}{1.527455in}}%
\pgfpathlineto{\pgfqpoint{4.044854in}{1.568596in}}%
\pgfpathlineto{\pgfqpoint{4.046077in}{1.420489in}}%
\pgfpathlineto{\pgfqpoint{4.046281in}{1.420489in}}%
\pgfpathlineto{\pgfqpoint{4.047707in}{1.552140in}}%
\pgfpathlineto{\pgfqpoint{4.048523in}{1.469858in}}%
\pgfpathlineto{\pgfqpoint{4.048930in}{1.478086in}}%
\pgfpathlineto{\pgfqpoint{4.049542in}{1.543912in}}%
\pgfpathlineto{\pgfqpoint{4.049949in}{1.486314in}}%
\pgfpathlineto{\pgfqpoint{4.050153in}{1.461630in}}%
\pgfpathlineto{\pgfqpoint{4.050561in}{1.535683in}}%
\pgfpathlineto{\pgfqpoint{4.052191in}{1.585053in}}%
\pgfpathlineto{\pgfqpoint{4.052599in}{1.560368in}}%
\pgfpathlineto{\pgfqpoint{4.053210in}{1.609737in}}%
\pgfpathlineto{\pgfqpoint{4.054229in}{1.527455in}}%
\pgfpathlineto{\pgfqpoint{4.053618in}{1.642650in}}%
\pgfpathlineto{\pgfqpoint{4.054637in}{1.543912in}}%
\pgfpathlineto{\pgfqpoint{4.056063in}{1.667335in}}%
\pgfpathlineto{\pgfqpoint{4.056675in}{1.659106in}}%
\pgfpathlineto{\pgfqpoint{4.056878in}{1.568596in}}%
\pgfpathlineto{\pgfqpoint{4.057694in}{1.626194in}}%
\pgfpathlineto{\pgfqpoint{4.057897in}{1.609737in}}%
\pgfpathlineto{\pgfqpoint{4.058101in}{1.650878in}}%
\pgfpathlineto{\pgfqpoint{4.058305in}{1.634422in}}%
\pgfpathlineto{\pgfqpoint{4.059324in}{1.749617in}}%
\pgfpathlineto{\pgfqpoint{4.059528in}{1.683791in}}%
\pgfpathlineto{\pgfqpoint{4.060954in}{1.552140in}}%
\pgfpathlineto{\pgfqpoint{4.061158in}{1.568596in}}%
\pgfpathlineto{\pgfqpoint{4.061362in}{1.502771in}}%
\pgfpathlineto{\pgfqpoint{4.061566in}{1.502771in}}%
\pgfpathlineto{\pgfqpoint{4.063808in}{1.617965in}}%
\pgfpathlineto{\pgfqpoint{4.064012in}{1.601509in}}%
\pgfpathlineto{\pgfqpoint{4.064215in}{1.659106in}}%
\pgfpathlineto{\pgfqpoint{4.064419in}{1.856583in}}%
\pgfpathlineto{\pgfqpoint{4.064623in}{1.585053in}}%
\pgfpathlineto{\pgfqpoint{4.065234in}{1.650878in}}%
\pgfpathlineto{\pgfqpoint{4.065438in}{1.453401in}}%
\pgfpathlineto{\pgfqpoint{4.065642in}{1.749617in}}%
\pgfpathlineto{\pgfqpoint{4.066253in}{1.683791in}}%
\pgfpathlineto{\pgfqpoint{4.067680in}{1.593281in}}%
\pgfpathlineto{\pgfqpoint{4.068903in}{1.708476in}}%
\pgfpathlineto{\pgfqpoint{4.069310in}{1.634422in}}%
\pgfpathlineto{\pgfqpoint{4.069514in}{1.667335in}}%
\pgfpathlineto{\pgfqpoint{4.069922in}{1.354663in}}%
\pgfpathlineto{\pgfqpoint{4.070533in}{1.568596in}}%
\pgfpathlineto{\pgfqpoint{4.072164in}{1.724932in}}%
\pgfpathlineto{\pgfqpoint{4.073183in}{1.527455in}}%
\pgfpathlineto{\pgfqpoint{4.073386in}{1.568596in}}%
\pgfpathlineto{\pgfqpoint{4.074813in}{1.436945in}}%
\pgfpathlineto{\pgfqpoint{4.075628in}{1.585053in}}%
\pgfpathlineto{\pgfqpoint{4.075832in}{1.453401in}}%
\pgfpathlineto{\pgfqpoint{4.076444in}{1.214784in}}%
\pgfpathlineto{\pgfqpoint{4.077259in}{1.280609in}}%
\pgfpathlineto{\pgfqpoint{4.077463in}{1.288837in}}%
\pgfpathlineto{\pgfqpoint{4.078278in}{1.017307in}}%
\pgfpathlineto{\pgfqpoint{4.078482in}{1.346435in}}%
\pgfpathlineto{\pgfqpoint{4.079501in}{1.247696in}}%
\pgfpathlineto{\pgfqpoint{4.080723in}{0.951481in}}%
\pgfpathlineto{\pgfqpoint{4.080927in}{1.066676in}}%
\pgfpathlineto{\pgfqpoint{4.082354in}{1.214784in}}%
\pgfpathlineto{\pgfqpoint{4.083780in}{0.819830in}}%
\pgfpathlineto{\pgfqpoint{4.084188in}{0.951481in}}%
\pgfpathlineto{\pgfqpoint{4.084596in}{0.935025in}}%
\pgfpathlineto{\pgfqpoint{4.085818in}{1.066676in}}%
\pgfpathlineto{\pgfqpoint{4.086022in}{1.066676in}}%
\pgfpathlineto{\pgfqpoint{4.086226in}{1.058448in}}%
\pgfpathlineto{\pgfqpoint{4.086430in}{1.198327in}}%
\pgfpathlineto{\pgfqpoint{4.086837in}{1.000850in}}%
\pgfpathlineto{\pgfqpoint{4.087041in}{1.041991in}}%
\pgfpathlineto{\pgfqpoint{4.087653in}{0.762232in}}%
\pgfpathlineto{\pgfqpoint{4.088060in}{1.025535in}}%
\pgfpathlineto{\pgfqpoint{4.088264in}{1.058448in}}%
\pgfpathlineto{\pgfqpoint{4.088468in}{1.000850in}}%
\pgfpathlineto{\pgfqpoint{4.088672in}{1.025535in}}%
\pgfpathlineto{\pgfqpoint{4.089691in}{0.786917in}}%
\pgfpathlineto{\pgfqpoint{4.090302in}{0.828058in}}%
\pgfpathlineto{\pgfqpoint{4.091729in}{1.157186in}}%
\pgfpathlineto{\pgfqpoint{4.092340in}{1.000850in}}%
\pgfpathlineto{\pgfqpoint{4.092544in}{0.967938in}}%
\pgfpathlineto{\pgfqpoint{4.093155in}{1.041991in}}%
\pgfpathlineto{\pgfqpoint{4.093971in}{1.181871in}}%
\pgfpathlineto{\pgfqpoint{4.094378in}{1.140730in}}%
\pgfpathlineto{\pgfqpoint{4.095193in}{0.803374in}}%
\pgfpathlineto{\pgfqpoint{4.095601in}{0.885656in}}%
\pgfpathlineto{\pgfqpoint{4.097028in}{1.214784in}}%
\pgfpathlineto{\pgfqpoint{4.097843in}{1.091361in}}%
\pgfpathlineto{\pgfqpoint{4.098250in}{1.157186in}}%
\pgfpathlineto{\pgfqpoint{4.098454in}{1.231240in}}%
\pgfpathlineto{\pgfqpoint{4.099066in}{1.124273in}}%
\pgfpathlineto{\pgfqpoint{4.099269in}{1.157186in}}%
\pgfpathlineto{\pgfqpoint{4.099881in}{1.083132in}}%
\pgfpathlineto{\pgfqpoint{4.100492in}{1.132502in}}%
\pgfpathlineto{\pgfqpoint{4.100696in}{1.165414in}}%
\pgfpathlineto{\pgfqpoint{4.100900in}{1.124273in}}%
\pgfpathlineto{\pgfqpoint{4.101307in}{1.041991in}}%
\pgfpathlineto{\pgfqpoint{4.102123in}{1.066676in}}%
\pgfpathlineto{\pgfqpoint{4.102734in}{1.116045in}}%
\pgfpathlineto{\pgfqpoint{4.102938in}{1.058448in}}%
\pgfpathlineto{\pgfqpoint{4.103346in}{1.025535in}}%
\pgfpathlineto{\pgfqpoint{4.103549in}{1.083132in}}%
\pgfpathlineto{\pgfqpoint{4.103753in}{1.083132in}}%
\pgfpathlineto{\pgfqpoint{4.104161in}{1.124273in}}%
\pgfpathlineto{\pgfqpoint{4.104568in}{1.116045in}}%
\pgfpathlineto{\pgfqpoint{4.104772in}{1.074904in}}%
\pgfpathlineto{\pgfqpoint{4.104976in}{1.173643in}}%
\pgfpathlineto{\pgfqpoint{4.105384in}{1.297066in}}%
\pgfpathlineto{\pgfqpoint{4.105587in}{1.198327in}}%
\pgfpathlineto{\pgfqpoint{4.105791in}{1.099589in}}%
\pgfpathlineto{\pgfqpoint{4.106403in}{1.313522in}}%
\pgfpathlineto{\pgfqpoint{4.107625in}{1.206555in}}%
\pgfpathlineto{\pgfqpoint{4.108237in}{1.362891in}}%
\pgfpathlineto{\pgfqpoint{4.108848in}{1.264153in}}%
\pgfpathlineto{\pgfqpoint{4.109663in}{1.354663in}}%
\pgfpathlineto{\pgfqpoint{4.109867in}{1.297066in}}%
\pgfpathlineto{\pgfqpoint{4.110071in}{1.206555in}}%
\pgfpathlineto{\pgfqpoint{4.110479in}{1.404032in}}%
\pgfpathlineto{\pgfqpoint{4.110682in}{1.428717in}}%
\pgfpathlineto{\pgfqpoint{4.110886in}{1.362891in}}%
\pgfpathlineto{\pgfqpoint{4.111498in}{1.305294in}}%
\pgfpathlineto{\pgfqpoint{4.111294in}{1.379348in}}%
\pgfpathlineto{\pgfqpoint{4.111701in}{1.346435in}}%
\pgfpathlineto{\pgfqpoint{4.111905in}{1.387576in}}%
\pgfpathlineto{\pgfqpoint{4.112109in}{1.297066in}}%
\pgfpathlineto{\pgfqpoint{4.112924in}{1.379348in}}%
\pgfpathlineto{\pgfqpoint{4.113128in}{1.379348in}}%
\pgfpathlineto{\pgfqpoint{4.114351in}{1.510999in}}%
\pgfpathlineto{\pgfqpoint{4.113536in}{1.362891in}}%
\pgfpathlineto{\pgfqpoint{4.114555in}{1.486314in}}%
\pgfpathlineto{\pgfqpoint{4.115370in}{1.502771in}}%
\pgfpathlineto{\pgfqpoint{4.115778in}{1.445173in}}%
\pgfpathlineto{\pgfqpoint{4.115981in}{1.510999in}}%
\pgfpathlineto{\pgfqpoint{4.116389in}{1.387576in}}%
\pgfpathlineto{\pgfqpoint{4.117000in}{1.486314in}}%
\pgfpathlineto{\pgfqpoint{4.118427in}{1.198327in}}%
\pgfpathlineto{\pgfqpoint{4.119650in}{1.519227in}}%
\pgfpathlineto{\pgfqpoint{4.120669in}{1.338207in}}%
\pgfpathlineto{\pgfqpoint{4.120873in}{1.436945in}}%
\pgfpathlineto{\pgfqpoint{4.121484in}{1.494542in}}%
\pgfpathlineto{\pgfqpoint{4.122095in}{1.478086in}}%
\pgfpathlineto{\pgfqpoint{4.122503in}{1.502771in}}%
\pgfpathlineto{\pgfqpoint{4.123318in}{1.404032in}}%
\pgfpathlineto{\pgfqpoint{4.124745in}{1.560368in}}%
\pgfpathlineto{\pgfqpoint{4.124949in}{1.552140in}}%
\pgfpathlineto{\pgfqpoint{4.125560in}{1.609737in}}%
\pgfpathlineto{\pgfqpoint{4.125764in}{1.568596in}}%
\pgfpathlineto{\pgfqpoint{4.127598in}{1.395804in}}%
\pgfpathlineto{\pgfqpoint{4.128617in}{1.535683in}}%
\pgfpathlineto{\pgfqpoint{4.129025in}{1.527455in}}%
\pgfpathlineto{\pgfqpoint{4.129840in}{1.494542in}}%
\pgfpathlineto{\pgfqpoint{4.131267in}{1.634422in}}%
\pgfpathlineto{\pgfqpoint{4.132286in}{1.568596in}}%
\pgfpathlineto{\pgfqpoint{4.132489in}{1.601509in}}%
\pgfpathlineto{\pgfqpoint{4.132897in}{1.601509in}}%
\pgfpathlineto{\pgfqpoint{4.133508in}{1.683791in}}%
\pgfpathlineto{\pgfqpoint{4.134324in}{1.659106in}}%
\pgfpathlineto{\pgfqpoint{4.134527in}{1.642650in}}%
\pgfpathlineto{\pgfqpoint{4.134731in}{1.659106in}}%
\pgfpathlineto{\pgfqpoint{4.135139in}{1.716704in}}%
\pgfpathlineto{\pgfqpoint{4.135750in}{1.700247in}}%
\pgfpathlineto{\pgfqpoint{4.135954in}{1.650878in}}%
\pgfpathlineto{\pgfqpoint{4.136565in}{1.766073in}}%
\pgfpathlineto{\pgfqpoint{4.136769in}{1.724932in}}%
\pgfpathlineto{\pgfqpoint{4.136973in}{1.774301in}}%
\pgfpathlineto{\pgfqpoint{4.137788in}{1.741388in}}%
\pgfpathlineto{\pgfqpoint{4.138807in}{1.823670in}}%
\pgfpathlineto{\pgfqpoint{4.139011in}{1.807214in}}%
\pgfpathlineto{\pgfqpoint{4.139215in}{1.766073in}}%
\pgfpathlineto{\pgfqpoint{4.139622in}{1.881268in}}%
\pgfpathlineto{\pgfqpoint{4.139826in}{1.831899in}}%
\pgfpathlineto{\pgfqpoint{4.140030in}{1.831899in}}%
\pgfpathlineto{\pgfqpoint{4.141457in}{1.947093in}}%
\pgfpathlineto{\pgfqpoint{4.141864in}{1.914181in}}%
\pgfpathlineto{\pgfqpoint{4.142068in}{1.889496in}}%
\pgfpathlineto{\pgfqpoint{4.142476in}{1.971778in}}%
\pgfpathlineto{\pgfqpoint{4.142680in}{1.963550in}}%
\pgfpathlineto{\pgfqpoint{4.143902in}{2.021147in}}%
\pgfpathlineto{\pgfqpoint{4.144514in}{2.012919in}}%
\pgfpathlineto{\pgfqpoint{4.144718in}{2.037604in}}%
\pgfpathlineto{\pgfqpoint{4.144921in}{1.848355in}}%
\pgfpathlineto{\pgfqpoint{4.145737in}{2.095201in}}%
\pgfpathlineto{\pgfqpoint{4.145940in}{2.070516in}}%
\pgfpathlineto{\pgfqpoint{4.146144in}{2.128114in}}%
\pgfpathlineto{\pgfqpoint{4.146756in}{2.103429in}}%
\pgfpathlineto{\pgfqpoint{4.146959in}{2.086973in}}%
\pgfpathlineto{\pgfqpoint{4.147163in}{1.971778in}}%
\pgfpathlineto{\pgfqpoint{4.147978in}{2.062288in}}%
\pgfpathlineto{\pgfqpoint{4.148590in}{1.889496in}}%
\pgfpathlineto{\pgfqpoint{4.149201in}{1.905952in}}%
\pgfpathlineto{\pgfqpoint{4.150016in}{2.128114in}}%
\pgfpathlineto{\pgfqpoint{4.150424in}{2.111657in}}%
\pgfpathlineto{\pgfqpoint{4.150832in}{2.152798in}}%
\pgfpathlineto{\pgfqpoint{4.151035in}{1.840127in}}%
\pgfpathlineto{\pgfqpoint{4.151851in}{2.169255in}}%
\pgfpathlineto{\pgfqpoint{4.152054in}{2.202168in}}%
\pgfpathlineto{\pgfqpoint{4.152666in}{2.136342in}}%
\pgfpathlineto{\pgfqpoint{4.152870in}{2.119886in}}%
\pgfpathlineto{\pgfqpoint{4.153073in}{2.152798in}}%
\pgfpathlineto{\pgfqpoint{4.153277in}{2.136342in}}%
\pgfpathlineto{\pgfqpoint{4.153481in}{2.193939in}}%
\pgfpathlineto{\pgfqpoint{4.154296in}{2.169255in}}%
\pgfpathlineto{\pgfqpoint{4.155723in}{2.045832in}}%
\pgfpathlineto{\pgfqpoint{4.156742in}{2.037604in}}%
\pgfpathlineto{\pgfqpoint{4.156946in}{2.144570in}}%
\pgfpathlineto{\pgfqpoint{4.157353in}{2.103429in}}%
\pgfpathlineto{\pgfqpoint{4.157761in}{2.177483in}}%
\pgfpathlineto{\pgfqpoint{4.157965in}{2.185711in}}%
\pgfpathlineto{\pgfqpoint{4.158780in}{2.210396in}}%
\pgfpathlineto{\pgfqpoint{4.159188in}{2.004691in}}%
\pgfpathlineto{\pgfqpoint{4.159391in}{2.021147in}}%
\pgfpathlineto{\pgfqpoint{4.159595in}{2.004691in}}%
\pgfpathlineto{\pgfqpoint{4.159799in}{1.963550in}}%
\pgfpathlineto{\pgfqpoint{4.160207in}{2.226852in}}%
\pgfpathlineto{\pgfqpoint{4.161022in}{2.136342in}}%
\pgfpathlineto{\pgfqpoint{4.162041in}{2.226852in}}%
\pgfpathlineto{\pgfqpoint{4.162448in}{2.202168in}}%
\pgfpathlineto{\pgfqpoint{4.163264in}{2.062288in}}%
\pgfpathlineto{\pgfqpoint{4.163671in}{2.119886in}}%
\pgfpathlineto{\pgfqpoint{4.164283in}{2.193939in}}%
\pgfpathlineto{\pgfqpoint{4.164690in}{2.103429in}}%
\pgfpathlineto{\pgfqpoint{4.164894in}{2.078745in}}%
\pgfpathlineto{\pgfqpoint{4.165098in}{2.136342in}}%
\pgfpathlineto{\pgfqpoint{4.165302in}{2.136342in}}%
\pgfpathlineto{\pgfqpoint{4.165505in}{2.235080in}}%
\pgfpathlineto{\pgfqpoint{4.166117in}{2.095201in}}%
\pgfpathlineto{\pgfqpoint{4.166524in}{2.202168in}}%
\pgfpathlineto{\pgfqpoint{4.166932in}{2.161027in}}%
\pgfpathlineto{\pgfqpoint{4.167747in}{2.226852in}}%
\pgfpathlineto{\pgfqpoint{4.168155in}{2.169255in}}%
\pgfpathlineto{\pgfqpoint{4.168562in}{2.226852in}}%
\pgfpathlineto{\pgfqpoint{4.168766in}{2.292678in}}%
\pgfpathlineto{\pgfqpoint{4.169582in}{2.202168in}}%
\pgfpathlineto{\pgfqpoint{4.169785in}{2.284450in}}%
\pgfpathlineto{\pgfqpoint{4.171212in}{2.202168in}}%
\pgfpathlineto{\pgfqpoint{4.172231in}{2.309134in}}%
\pgfpathlineto{\pgfqpoint{4.173250in}{2.235080in}}%
\pgfpathlineto{\pgfqpoint{4.174065in}{2.317363in}}%
\pgfpathlineto{\pgfqpoint{4.174473in}{2.300906in}}%
\pgfpathlineto{\pgfqpoint{4.174677in}{2.309134in}}%
\pgfpathlineto{\pgfqpoint{4.175084in}{2.333819in}}%
\pgfpathlineto{\pgfqpoint{4.176307in}{2.144570in}}%
\pgfpathlineto{\pgfqpoint{4.178345in}{2.259765in}}%
\pgfpathlineto{\pgfqpoint{4.178753in}{2.029375in}}%
\pgfpathlineto{\pgfqpoint{4.179160in}{2.350275in}}%
\pgfpathlineto{\pgfqpoint{4.179364in}{2.259765in}}%
\pgfpathlineto{\pgfqpoint{4.179568in}{2.251537in}}%
\pgfpathlineto{\pgfqpoint{4.180383in}{2.317363in}}%
\pgfpathlineto{\pgfqpoint{4.180587in}{2.292678in}}%
\pgfpathlineto{\pgfqpoint{4.181198in}{2.251537in}}%
\pgfpathlineto{\pgfqpoint{4.181810in}{2.267993in}}%
\pgfpathlineto{\pgfqpoint{4.182013in}{2.103429in}}%
\pgfpathlineto{\pgfqpoint{4.182829in}{2.317363in}}%
\pgfpathlineto{\pgfqpoint{4.183644in}{2.350275in}}%
\pgfpathlineto{\pgfqpoint{4.183440in}{2.292678in}}%
\pgfpathlineto{\pgfqpoint{4.183848in}{2.325591in}}%
\pgfpathlineto{\pgfqpoint{4.184867in}{2.070516in}}%
\pgfpathlineto{\pgfqpoint{4.184459in}{2.350275in}}%
\pgfpathlineto{\pgfqpoint{4.185274in}{2.284450in}}%
\pgfpathlineto{\pgfqpoint{4.185478in}{2.292678in}}%
\pgfpathlineto{\pgfqpoint{4.186090in}{2.226852in}}%
\pgfpathlineto{\pgfqpoint{4.186293in}{2.251537in}}%
\pgfpathlineto{\pgfqpoint{4.186497in}{2.317363in}}%
\pgfpathlineto{\pgfqpoint{4.187312in}{2.243309in}}%
\pgfpathlineto{\pgfqpoint{4.188128in}{2.226852in}}%
\pgfpathlineto{\pgfqpoint{4.189554in}{2.350275in}}%
\pgfpathlineto{\pgfqpoint{4.189758in}{2.358504in}}%
\pgfpathlineto{\pgfqpoint{4.190573in}{2.226852in}}%
\pgfpathlineto{\pgfqpoint{4.190981in}{2.292678in}}%
\pgfpathlineto{\pgfqpoint{4.192000in}{2.342047in}}%
\pgfpathlineto{\pgfqpoint{4.192204in}{2.276221in}}%
\pgfpathlineto{\pgfqpoint{4.192815in}{2.366732in}}%
\pgfpathlineto{\pgfqpoint{4.193019in}{2.342047in}}%
\pgfpathlineto{\pgfqpoint{4.193223in}{2.374960in}}%
\pgfpathlineto{\pgfqpoint{4.193630in}{2.317363in}}%
\pgfpathlineto{\pgfqpoint{4.193834in}{2.350275in}}%
\pgfpathlineto{\pgfqpoint{4.194649in}{2.111657in}}%
\pgfpathlineto{\pgfqpoint{4.194853in}{2.300906in}}%
\pgfpathlineto{\pgfqpoint{4.195872in}{2.432557in}}%
\pgfpathlineto{\pgfqpoint{4.196076in}{2.383188in}}%
\pgfpathlineto{\pgfqpoint{4.196687in}{2.440786in}}%
\pgfpathlineto{\pgfqpoint{4.197095in}{2.416101in}}%
\pgfpathlineto{\pgfqpoint{4.198725in}{2.070516in}}%
\pgfpathlineto{\pgfqpoint{4.199948in}{2.333819in}}%
\pgfpathlineto{\pgfqpoint{4.200967in}{2.366732in}}%
\pgfpathlineto{\pgfqpoint{4.201171in}{2.358504in}}%
\pgfpathlineto{\pgfqpoint{4.201986in}{2.267993in}}%
\pgfpathlineto{\pgfqpoint{4.202190in}{2.366732in}}%
\pgfpathlineto{\pgfqpoint{4.202598in}{2.309134in}}%
\pgfpathlineto{\pgfqpoint{4.203005in}{2.366732in}}%
\pgfpathlineto{\pgfqpoint{4.203209in}{2.391416in}}%
\pgfpathlineto{\pgfqpoint{4.203413in}{2.300906in}}%
\pgfpathlineto{\pgfqpoint{4.203617in}{2.300906in}}%
\pgfpathlineto{\pgfqpoint{4.204024in}{2.251537in}}%
\pgfpathlineto{\pgfqpoint{4.204228in}{2.317363in}}%
\pgfpathlineto{\pgfqpoint{4.204636in}{2.399645in}}%
\pgfpathlineto{\pgfqpoint{4.205247in}{2.374960in}}%
\pgfpathlineto{\pgfqpoint{4.206877in}{2.152798in}}%
\pgfpathlineto{\pgfqpoint{4.207285in}{2.333819in}}%
\pgfpathlineto{\pgfqpoint{4.208100in}{2.276221in}}%
\pgfpathlineto{\pgfqpoint{4.208304in}{2.267993in}}%
\pgfpathlineto{\pgfqpoint{4.209323in}{2.383188in}}%
\pgfpathlineto{\pgfqpoint{4.208712in}{2.259765in}}%
\pgfpathlineto{\pgfqpoint{4.209527in}{2.358504in}}%
\pgfpathlineto{\pgfqpoint{4.210750in}{2.284450in}}%
\pgfpathlineto{\pgfqpoint{4.211973in}{2.416101in}}%
\pgfpathlineto{\pgfqpoint{4.212992in}{2.317363in}}%
\pgfpathlineto{\pgfqpoint{4.213195in}{2.358504in}}%
\pgfpathlineto{\pgfqpoint{4.214011in}{2.325591in}}%
\pgfpathlineto{\pgfqpoint{4.213603in}{2.383188in}}%
\pgfpathlineto{\pgfqpoint{4.214214in}{2.358504in}}%
\pgfpathlineto{\pgfqpoint{4.214622in}{2.374960in}}%
\pgfpathlineto{\pgfqpoint{4.214826in}{2.342047in}}%
\pgfpathlineto{\pgfqpoint{4.215030in}{2.358504in}}%
\pgfpathlineto{\pgfqpoint{4.216049in}{2.309134in}}%
\pgfpathlineto{\pgfqpoint{4.216252in}{2.399645in}}%
\pgfpathlineto{\pgfqpoint{4.217068in}{2.374960in}}%
\pgfpathlineto{\pgfqpoint{4.217271in}{2.333819in}}%
\pgfpathlineto{\pgfqpoint{4.217679in}{2.391416in}}%
\pgfpathlineto{\pgfqpoint{4.218087in}{2.391416in}}%
\pgfpathlineto{\pgfqpoint{4.218290in}{2.407873in}}%
\pgfpathlineto{\pgfqpoint{4.218494in}{2.383188in}}%
\pgfpathlineto{\pgfqpoint{4.218902in}{2.383188in}}%
\pgfpathlineto{\pgfqpoint{4.220532in}{2.267993in}}%
\pgfpathlineto{\pgfqpoint{4.222366in}{2.366732in}}%
\pgfpathlineto{\pgfqpoint{4.221144in}{2.251537in}}%
\pgfpathlineto{\pgfqpoint{4.222570in}{2.350275in}}%
\pgfpathlineto{\pgfqpoint{4.222774in}{2.210396in}}%
\pgfpathlineto{\pgfqpoint{4.223386in}{2.383188in}}%
\pgfpathlineto{\pgfqpoint{4.223589in}{2.350275in}}%
\pgfpathlineto{\pgfqpoint{4.224201in}{2.317363in}}%
\pgfpathlineto{\pgfqpoint{4.225016in}{2.416101in}}%
\pgfpathlineto{\pgfqpoint{4.225627in}{2.333819in}}%
\pgfpathlineto{\pgfqpoint{4.225831in}{2.424329in}}%
\pgfpathlineto{\pgfqpoint{4.226035in}{2.498383in}}%
\pgfpathlineto{\pgfqpoint{4.226646in}{2.350275in}}%
\pgfpathlineto{\pgfqpoint{4.226850in}{2.416101in}}%
\pgfpathlineto{\pgfqpoint{4.227462in}{2.424329in}}%
\pgfpathlineto{\pgfqpoint{4.228277in}{2.243309in}}%
\pgfpathlineto{\pgfqpoint{4.229092in}{2.333819in}}%
\pgfpathlineto{\pgfqpoint{4.229703in}{2.358504in}}%
\pgfpathlineto{\pgfqpoint{4.230315in}{2.292678in}}%
\pgfpathlineto{\pgfqpoint{4.230519in}{2.342047in}}%
\pgfpathlineto{\pgfqpoint{4.231334in}{2.333819in}}%
\pgfpathlineto{\pgfqpoint{4.232557in}{2.284450in}}%
\pgfpathlineto{\pgfqpoint{4.232964in}{2.350275in}}%
\pgfpathlineto{\pgfqpoint{4.233372in}{2.300906in}}%
\pgfpathlineto{\pgfqpoint{4.233576in}{2.259765in}}%
\pgfpathlineto{\pgfqpoint{4.234187in}{2.309134in}}%
\pgfpathlineto{\pgfqpoint{4.234391in}{2.309134in}}%
\pgfpathlineto{\pgfqpoint{4.236225in}{2.136342in}}%
\pgfpathlineto{\pgfqpoint{4.236429in}{2.185711in}}%
\pgfpathlineto{\pgfqpoint{4.237244in}{2.300906in}}%
\pgfpathlineto{\pgfqpoint{4.237856in}{2.267993in}}%
\pgfpathlineto{\pgfqpoint{4.239486in}{2.177483in}}%
\pgfpathlineto{\pgfqpoint{4.239894in}{2.128114in}}%
\pgfpathlineto{\pgfqpoint{4.240301in}{2.226852in}}%
\pgfpathlineto{\pgfqpoint{4.240913in}{2.284450in}}%
\pgfpathlineto{\pgfqpoint{4.241320in}{2.070516in}}%
\pgfpathlineto{\pgfqpoint{4.241728in}{2.325591in}}%
\pgfpathlineto{\pgfqpoint{4.241932in}{2.309134in}}%
\pgfpathlineto{\pgfqpoint{4.243154in}{2.424329in}}%
\pgfpathlineto{\pgfqpoint{4.243358in}{2.391416in}}%
\pgfpathlineto{\pgfqpoint{4.243970in}{2.317363in}}%
\pgfpathlineto{\pgfqpoint{4.243766in}{2.424329in}}%
\pgfpathlineto{\pgfqpoint{4.244581in}{2.342047in}}%
\pgfpathlineto{\pgfqpoint{4.245192in}{2.374960in}}%
\pgfpathlineto{\pgfqpoint{4.245804in}{2.070516in}}%
\pgfpathlineto{\pgfqpoint{4.246211in}{2.267993in}}%
\pgfpathlineto{\pgfqpoint{4.247434in}{2.399645in}}%
\pgfpathlineto{\pgfqpoint{4.246619in}{2.251537in}}%
\pgfpathlineto{\pgfqpoint{4.247638in}{2.342047in}}%
\pgfpathlineto{\pgfqpoint{4.249472in}{2.276221in}}%
\pgfpathlineto{\pgfqpoint{4.250084in}{2.325591in}}%
\pgfpathlineto{\pgfqpoint{4.250695in}{2.317363in}}%
\pgfpathlineto{\pgfqpoint{4.251510in}{2.300906in}}%
\pgfpathlineto{\pgfqpoint{4.251714in}{2.374960in}}%
\pgfpathlineto{\pgfqpoint{4.252937in}{2.095201in}}%
\pgfpathlineto{\pgfqpoint{4.253141in}{2.284450in}}%
\pgfpathlineto{\pgfqpoint{4.253345in}{2.350275in}}%
\pgfpathlineto{\pgfqpoint{4.254160in}{2.259765in}}%
\pgfpathlineto{\pgfqpoint{4.254364in}{2.284450in}}%
\pgfpathlineto{\pgfqpoint{4.254567in}{2.276221in}}%
\pgfpathlineto{\pgfqpoint{4.254771in}{2.078745in}}%
\pgfpathlineto{\pgfqpoint{4.255179in}{2.374960in}}%
\pgfpathlineto{\pgfqpoint{4.255586in}{2.144570in}}%
\pgfpathlineto{\pgfqpoint{4.256605in}{2.374960in}}%
\pgfpathlineto{\pgfqpoint{4.256809in}{2.342047in}}%
\pgfpathlineto{\pgfqpoint{4.257013in}{2.333819in}}%
\pgfpathlineto{\pgfqpoint{4.257217in}{2.342047in}}%
\pgfpathlineto{\pgfqpoint{4.257828in}{2.383188in}}%
\pgfpathlineto{\pgfqpoint{4.259051in}{2.284450in}}%
\pgfpathlineto{\pgfqpoint{4.259255in}{2.292678in}}%
\pgfpathlineto{\pgfqpoint{4.259459in}{2.276221in}}%
\pgfpathlineto{\pgfqpoint{4.259866in}{2.152798in}}%
\pgfpathlineto{\pgfqpoint{4.260681in}{2.251537in}}%
\pgfpathlineto{\pgfqpoint{4.261089in}{2.235080in}}%
\pgfpathlineto{\pgfqpoint{4.261293in}{2.251537in}}%
\pgfpathlineto{\pgfqpoint{4.261497in}{2.292678in}}%
\pgfpathlineto{\pgfqpoint{4.261700in}{2.095201in}}%
\pgfpathlineto{\pgfqpoint{4.261904in}{2.325591in}}%
\pgfpathlineto{\pgfqpoint{4.262516in}{2.267993in}}%
\pgfpathlineto{\pgfqpoint{4.263127in}{2.358504in}}%
\pgfpathlineto{\pgfqpoint{4.263535in}{2.309134in}}%
\pgfpathlineto{\pgfqpoint{4.263739in}{2.276221in}}%
\pgfpathlineto{\pgfqpoint{4.263942in}{2.383188in}}%
\pgfpathlineto{\pgfqpoint{4.264146in}{2.383188in}}%
\pgfpathlineto{\pgfqpoint{4.264554in}{2.358504in}}%
\pgfpathlineto{\pgfqpoint{4.264758in}{2.416101in}}%
\pgfpathlineto{\pgfqpoint{4.265980in}{2.111657in}}%
\pgfpathlineto{\pgfqpoint{4.266999in}{2.374960in}}%
\pgfpathlineto{\pgfqpoint{4.267203in}{2.317363in}}%
\pgfpathlineto{\pgfqpoint{4.267407in}{2.243309in}}%
\pgfpathlineto{\pgfqpoint{4.267611in}{2.424329in}}%
\pgfpathlineto{\pgfqpoint{4.268018in}{2.358504in}}%
\pgfpathlineto{\pgfqpoint{4.268426in}{2.407873in}}%
\pgfpathlineto{\pgfqpoint{4.269037in}{2.366732in}}%
\pgfpathlineto{\pgfqpoint{4.269649in}{2.383188in}}%
\pgfpathlineto{\pgfqpoint{4.270056in}{2.333819in}}%
\pgfpathlineto{\pgfqpoint{4.270464in}{2.325591in}}%
\pgfpathlineto{\pgfqpoint{4.271279in}{2.374960in}}%
\pgfpathlineto{\pgfqpoint{4.271483in}{2.366732in}}%
\pgfpathlineto{\pgfqpoint{4.271891in}{2.416101in}}%
\pgfpathlineto{\pgfqpoint{4.272298in}{2.350275in}}%
\pgfpathlineto{\pgfqpoint{4.272502in}{2.350275in}}%
\pgfpathlineto{\pgfqpoint{4.272706in}{2.309134in}}%
\pgfpathlineto{\pgfqpoint{4.273113in}{2.366732in}}%
\pgfpathlineto{\pgfqpoint{4.273725in}{2.473698in}}%
\pgfpathlineto{\pgfqpoint{4.274336in}{2.457242in}}%
\pgfpathlineto{\pgfqpoint{4.275355in}{2.399645in}}%
\pgfpathlineto{\pgfqpoint{4.275559in}{2.407873in}}%
\pgfpathlineto{\pgfqpoint{4.275763in}{2.440786in}}%
\pgfpathlineto{\pgfqpoint{4.276170in}{2.374960in}}%
\pgfpathlineto{\pgfqpoint{4.277190in}{2.276221in}}%
\pgfpathlineto{\pgfqpoint{4.277393in}{2.333819in}}%
\pgfpathlineto{\pgfqpoint{4.277801in}{2.103429in}}%
\pgfpathlineto{\pgfqpoint{4.278412in}{2.292678in}}%
\pgfpathlineto{\pgfqpoint{4.279228in}{2.416101in}}%
\pgfpathlineto{\pgfqpoint{4.279635in}{2.374960in}}%
\pgfpathlineto{\pgfqpoint{4.279839in}{2.366732in}}%
\pgfpathlineto{\pgfqpoint{4.280247in}{2.490155in}}%
\pgfpathlineto{\pgfqpoint{4.280450in}{2.169255in}}%
\pgfpathlineto{\pgfqpoint{4.281266in}{2.416101in}}%
\pgfpathlineto{\pgfqpoint{4.281469in}{2.399645in}}%
\pgfpathlineto{\pgfqpoint{4.281673in}{2.465470in}}%
\pgfpathlineto{\pgfqpoint{4.281877in}{2.440786in}}%
\pgfpathlineto{\pgfqpoint{4.282285in}{2.498383in}}%
\pgfpathlineto{\pgfqpoint{4.282692in}{2.350275in}}%
\pgfpathlineto{\pgfqpoint{4.283507in}{2.383188in}}%
\pgfpathlineto{\pgfqpoint{4.283711in}{2.391416in}}%
\pgfpathlineto{\pgfqpoint{4.283915in}{2.366732in}}%
\pgfpathlineto{\pgfqpoint{4.284323in}{2.383188in}}%
\pgfpathlineto{\pgfqpoint{4.284526in}{2.333819in}}%
\pgfpathlineto{\pgfqpoint{4.285342in}{2.358504in}}%
\pgfpathlineto{\pgfqpoint{4.285545in}{2.424329in}}%
\pgfpathlineto{\pgfqpoint{4.286157in}{2.284450in}}%
\pgfpathlineto{\pgfqpoint{4.286361in}{2.276221in}}%
\pgfpathlineto{\pgfqpoint{4.286564in}{2.284450in}}%
\pgfpathlineto{\pgfqpoint{4.288195in}{2.358504in}}%
\pgfpathlineto{\pgfqpoint{4.289010in}{2.276221in}}%
\pgfpathlineto{\pgfqpoint{4.289418in}{2.325591in}}%
\pgfpathlineto{\pgfqpoint{4.290437in}{2.251537in}}%
\pgfpathlineto{\pgfqpoint{4.291660in}{2.383188in}}%
\pgfpathlineto{\pgfqpoint{4.291863in}{2.374960in}}%
\pgfpathlineto{\pgfqpoint{4.293698in}{2.473698in}}%
\pgfpathlineto{\pgfqpoint{4.294920in}{2.391416in}}%
\pgfpathlineto{\pgfqpoint{4.294105in}{2.481927in}}%
\pgfpathlineto{\pgfqpoint{4.295124in}{2.407873in}}%
\pgfpathlineto{\pgfqpoint{4.295532in}{2.473698in}}%
\pgfpathlineto{\pgfqpoint{4.295939in}{2.383188in}}%
\pgfpathlineto{\pgfqpoint{4.296347in}{2.358504in}}%
\pgfpathlineto{\pgfqpoint{4.296551in}{2.374960in}}%
\pgfpathlineto{\pgfqpoint{4.296755in}{2.424329in}}%
\pgfpathlineto{\pgfqpoint{4.297570in}{2.374960in}}%
\pgfpathlineto{\pgfqpoint{4.297977in}{2.358504in}}%
\pgfpathlineto{\pgfqpoint{4.298181in}{2.407873in}}%
\pgfpathlineto{\pgfqpoint{4.298385in}{2.374960in}}%
\pgfpathlineto{\pgfqpoint{4.299200in}{2.366732in}}%
\pgfpathlineto{\pgfqpoint{4.299608in}{2.465470in}}%
\pgfpathlineto{\pgfqpoint{4.300219in}{2.342047in}}%
\pgfpathlineto{\pgfqpoint{4.300627in}{2.407873in}}%
\pgfpathlineto{\pgfqpoint{4.300831in}{2.399645in}}%
\pgfpathlineto{\pgfqpoint{4.302053in}{2.449014in}}%
\pgfpathlineto{\pgfqpoint{4.302461in}{2.465470in}}%
\pgfpathlineto{\pgfqpoint{4.302665in}{2.432557in}}%
\pgfpathlineto{\pgfqpoint{4.303888in}{2.514839in}}%
\pgfpathlineto{\pgfqpoint{4.304703in}{2.407873in}}%
\pgfpathlineto{\pgfqpoint{4.305111in}{2.424329in}}%
\pgfpathlineto{\pgfqpoint{4.305314in}{2.481927in}}%
\pgfpathlineto{\pgfqpoint{4.306130in}{2.399645in}}%
\pgfpathlineto{\pgfqpoint{4.306741in}{2.440786in}}%
\pgfpathlineto{\pgfqpoint{4.306945in}{2.374960in}}%
\pgfpathlineto{\pgfqpoint{4.307149in}{2.416101in}}%
\pgfpathlineto{\pgfqpoint{4.307352in}{2.358504in}}%
\pgfpathlineto{\pgfqpoint{4.308168in}{2.407873in}}%
\pgfpathlineto{\pgfqpoint{4.308371in}{2.424329in}}%
\pgfpathlineto{\pgfqpoint{4.308983in}{2.333819in}}%
\pgfpathlineto{\pgfqpoint{4.309594in}{2.350275in}}%
\pgfpathlineto{\pgfqpoint{4.310409in}{2.498383in}}%
\pgfpathlineto{\pgfqpoint{4.311021in}{2.449014in}}%
\pgfpathlineto{\pgfqpoint{4.311225in}{2.309134in}}%
\pgfpathlineto{\pgfqpoint{4.312040in}{2.391416in}}%
\pgfpathlineto{\pgfqpoint{4.312244in}{2.432557in}}%
\pgfpathlineto{\pgfqpoint{4.312855in}{2.350275in}}%
\pgfpathlineto{\pgfqpoint{4.313263in}{2.399645in}}%
\pgfpathlineto{\pgfqpoint{4.313670in}{2.366732in}}%
\pgfpathlineto{\pgfqpoint{4.314485in}{2.457242in}}%
\pgfpathlineto{\pgfqpoint{4.314689in}{2.276221in}}%
\pgfpathlineto{\pgfqpoint{4.315504in}{2.366732in}}%
\pgfpathlineto{\pgfqpoint{4.316523in}{2.309134in}}%
\pgfpathlineto{\pgfqpoint{4.316727in}{2.078745in}}%
\pgfpathlineto{\pgfqpoint{4.317543in}{2.383188in}}%
\pgfpathlineto{\pgfqpoint{4.317746in}{2.440786in}}%
\pgfpathlineto{\pgfqpoint{4.318358in}{2.342047in}}%
\pgfpathlineto{\pgfqpoint{4.318765in}{2.416101in}}%
\pgfpathlineto{\pgfqpoint{4.318969in}{2.366732in}}%
\pgfpathlineto{\pgfqpoint{4.319173in}{2.424329in}}%
\pgfpathlineto{\pgfqpoint{4.319581in}{2.407873in}}%
\pgfpathlineto{\pgfqpoint{4.320396in}{2.498383in}}%
\pgfpathlineto{\pgfqpoint{4.319988in}{2.383188in}}%
\pgfpathlineto{\pgfqpoint{4.320600in}{2.473698in}}%
\pgfpathlineto{\pgfqpoint{4.321211in}{2.424329in}}%
\pgfpathlineto{\pgfqpoint{4.321619in}{2.432557in}}%
\pgfpathlineto{\pgfqpoint{4.321822in}{2.498383in}}%
\pgfpathlineto{\pgfqpoint{4.322026in}{2.407873in}}%
\pgfpathlineto{\pgfqpoint{4.322638in}{2.432557in}}%
\pgfpathlineto{\pgfqpoint{4.322841in}{2.449014in}}%
\pgfpathlineto{\pgfqpoint{4.323249in}{2.177483in}}%
\pgfpathlineto{\pgfqpoint{4.323860in}{2.383188in}}%
\pgfpathlineto{\pgfqpoint{4.324268in}{2.424329in}}%
\pgfpathlineto{\pgfqpoint{4.324676in}{2.350275in}}%
\pgfpathlineto{\pgfqpoint{4.324879in}{2.407873in}}%
\pgfpathlineto{\pgfqpoint{4.325287in}{2.358504in}}%
\pgfpathlineto{\pgfqpoint{4.325491in}{2.416101in}}%
\pgfpathlineto{\pgfqpoint{4.325695in}{2.416101in}}%
\pgfpathlineto{\pgfqpoint{4.328548in}{2.572437in}}%
\pgfpathlineto{\pgfqpoint{4.328752in}{2.605350in}}%
\pgfpathlineto{\pgfqpoint{4.329159in}{2.514839in}}%
\pgfpathlineto{\pgfqpoint{4.329363in}{2.580665in}}%
\pgfpathlineto{\pgfqpoint{4.330586in}{2.531296in}}%
\pgfpathlineto{\pgfqpoint{4.330790in}{2.597121in}}%
\pgfpathlineto{\pgfqpoint{4.330994in}{2.383188in}}%
\pgfpathlineto{\pgfqpoint{4.331401in}{2.317363in}}%
\pgfpathlineto{\pgfqpoint{4.331401in}{2.317363in}}%
\pgfusepath{stroke}%
\end{pgfscope}%
\begin{pgfscope}%
\pgfsetrectcap%
\pgfsetmiterjoin%
\pgfsetlinewidth{0.803000pt}%
\definecolor{currentstroke}{rgb}{0.000000,0.000000,0.000000}%
\pgfsetstrokecolor{currentstroke}%
\pgfsetdash{}{0pt}%
\pgfpathmoveto{\pgfqpoint{0.633813in}{0.538014in}}%
\pgfpathlineto{\pgfqpoint{0.633813in}{2.936535in}}%
\pgfusepath{stroke}%
\end{pgfscope}%
\begin{pgfscope}%
\pgfsetrectcap%
\pgfsetmiterjoin%
\pgfsetlinewidth{0.803000pt}%
\definecolor{currentstroke}{rgb}{0.000000,0.000000,0.000000}%
\pgfsetstrokecolor{currentstroke}%
\pgfsetdash{}{0pt}%
\pgfpathmoveto{\pgfqpoint{4.507477in}{0.538014in}}%
\pgfpathlineto{\pgfqpoint{4.507477in}{2.936535in}}%
\pgfusepath{stroke}%
\end{pgfscope}%
\begin{pgfscope}%
\pgfsetrectcap%
\pgfsetmiterjoin%
\pgfsetlinewidth{0.803000pt}%
\definecolor{currentstroke}{rgb}{0.000000,0.000000,0.000000}%
\pgfsetstrokecolor{currentstroke}%
\pgfsetdash{}{0pt}%
\pgfpathmoveto{\pgfqpoint{0.633813in}{0.538014in}}%
\pgfpathlineto{\pgfqpoint{4.507477in}{0.538014in}}%
\pgfusepath{stroke}%
\end{pgfscope}%
\begin{pgfscope}%
\pgfsetrectcap%
\pgfsetmiterjoin%
\pgfsetlinewidth{0.803000pt}%
\definecolor{currentstroke}{rgb}{0.000000,0.000000,0.000000}%
\pgfsetstrokecolor{currentstroke}%
\pgfsetdash{}{0pt}%
\pgfpathmoveto{\pgfqpoint{0.633813in}{2.936535in}}%
\pgfpathlineto{\pgfqpoint{4.507477in}{2.936535in}}%
\pgfusepath{stroke}%
\end{pgfscope}%
\begin{pgfscope}%
\pgfsetbuttcap%
\pgfsetroundjoin%
\definecolor{currentfill}{rgb}{0.000000,0.000000,0.000000}%
\pgfsetfillcolor{currentfill}%
\pgfsetlinewidth{0.803000pt}%
\definecolor{currentstroke}{rgb}{0.000000,0.000000,0.000000}%
\pgfsetstrokecolor{currentstroke}%
\pgfsetdash{}{0pt}%
\pgfsys@defobject{currentmarker}{\pgfqpoint{0.000000in}{0.000000in}}{\pgfqpoint{0.048611in}{0.000000in}}{%
\pgfpathmoveto{\pgfqpoint{0.000000in}{0.000000in}}%
\pgfpathlineto{\pgfqpoint{0.048611in}{0.000000in}}%
\pgfusepath{stroke,fill}%
}%
\begin{pgfscope}%
\pgfsys@transformshift{4.507477in}{0.723354in}%
\pgfsys@useobject{currentmarker}{}%
\end{pgfscope}%
\end{pgfscope}%
\begin{pgfscope}%
\definecolor{textcolor}{rgb}{0.000000,0.000000,0.000000}%
\pgfsetstrokecolor{textcolor}%
\pgfsetfillcolor{textcolor}%
\pgftext[x=4.604699in, y=0.684799in, left, base]{\color{textcolor}\rmfamily\fontsize{8.000000}{9.600000}\selectfont \(\displaystyle {20.00}\)}%
\end{pgfscope}%
\begin{pgfscope}%
\pgfsetbuttcap%
\pgfsetroundjoin%
\definecolor{currentfill}{rgb}{0.000000,0.000000,0.000000}%
\pgfsetfillcolor{currentfill}%
\pgfsetlinewidth{0.803000pt}%
\definecolor{currentstroke}{rgb}{0.000000,0.000000,0.000000}%
\pgfsetstrokecolor{currentstroke}%
\pgfsetdash{}{0pt}%
\pgfsys@defobject{currentmarker}{\pgfqpoint{0.000000in}{0.000000in}}{\pgfqpoint{0.048611in}{0.000000in}}{%
\pgfpathmoveto{\pgfqpoint{0.000000in}{0.000000in}}%
\pgfpathlineto{\pgfqpoint{0.048611in}{0.000000in}}%
\pgfusepath{stroke,fill}%
}%
\begin{pgfscope}%
\pgfsys@transformshift{4.507477in}{0.995913in}%
\pgfsys@useobject{currentmarker}{}%
\end{pgfscope}%
\end{pgfscope}%
\begin{pgfscope}%
\definecolor{textcolor}{rgb}{0.000000,0.000000,0.000000}%
\pgfsetstrokecolor{textcolor}%
\pgfsetfillcolor{textcolor}%
\pgftext[x=4.604699in, y=0.957358in, left, base]{\color{textcolor}\rmfamily\fontsize{8.000000}{9.600000}\selectfont \(\displaystyle {20.25}\)}%
\end{pgfscope}%
\begin{pgfscope}%
\pgfsetbuttcap%
\pgfsetroundjoin%
\definecolor{currentfill}{rgb}{0.000000,0.000000,0.000000}%
\pgfsetfillcolor{currentfill}%
\pgfsetlinewidth{0.803000pt}%
\definecolor{currentstroke}{rgb}{0.000000,0.000000,0.000000}%
\pgfsetstrokecolor{currentstroke}%
\pgfsetdash{}{0pt}%
\pgfsys@defobject{currentmarker}{\pgfqpoint{0.000000in}{0.000000in}}{\pgfqpoint{0.048611in}{0.000000in}}{%
\pgfpathmoveto{\pgfqpoint{0.000000in}{0.000000in}}%
\pgfpathlineto{\pgfqpoint{0.048611in}{0.000000in}}%
\pgfusepath{stroke,fill}%
}%
\begin{pgfscope}%
\pgfsys@transformshift{4.507477in}{1.268473in}%
\pgfsys@useobject{currentmarker}{}%
\end{pgfscope}%
\end{pgfscope}%
\begin{pgfscope}%
\definecolor{textcolor}{rgb}{0.000000,0.000000,0.000000}%
\pgfsetstrokecolor{textcolor}%
\pgfsetfillcolor{textcolor}%
\pgftext[x=4.604699in, y=1.229917in, left, base]{\color{textcolor}\rmfamily\fontsize{8.000000}{9.600000}\selectfont \(\displaystyle {20.50}\)}%
\end{pgfscope}%
\begin{pgfscope}%
\pgfsetbuttcap%
\pgfsetroundjoin%
\definecolor{currentfill}{rgb}{0.000000,0.000000,0.000000}%
\pgfsetfillcolor{currentfill}%
\pgfsetlinewidth{0.803000pt}%
\definecolor{currentstroke}{rgb}{0.000000,0.000000,0.000000}%
\pgfsetstrokecolor{currentstroke}%
\pgfsetdash{}{0pt}%
\pgfsys@defobject{currentmarker}{\pgfqpoint{0.000000in}{0.000000in}}{\pgfqpoint{0.048611in}{0.000000in}}{%
\pgfpathmoveto{\pgfqpoint{0.000000in}{0.000000in}}%
\pgfpathlineto{\pgfqpoint{0.048611in}{0.000000in}}%
\pgfusepath{stroke,fill}%
}%
\begin{pgfscope}%
\pgfsys@transformshift{4.507477in}{1.541032in}%
\pgfsys@useobject{currentmarker}{}%
\end{pgfscope}%
\end{pgfscope}%
\begin{pgfscope}%
\definecolor{textcolor}{rgb}{0.000000,0.000000,0.000000}%
\pgfsetstrokecolor{textcolor}%
\pgfsetfillcolor{textcolor}%
\pgftext[x=4.604699in, y=1.502476in, left, base]{\color{textcolor}\rmfamily\fontsize{8.000000}{9.600000}\selectfont \(\displaystyle {20.75}\)}%
\end{pgfscope}%
\begin{pgfscope}%
\pgfsetbuttcap%
\pgfsetroundjoin%
\definecolor{currentfill}{rgb}{0.000000,0.000000,0.000000}%
\pgfsetfillcolor{currentfill}%
\pgfsetlinewidth{0.803000pt}%
\definecolor{currentstroke}{rgb}{0.000000,0.000000,0.000000}%
\pgfsetstrokecolor{currentstroke}%
\pgfsetdash{}{0pt}%
\pgfsys@defobject{currentmarker}{\pgfqpoint{0.000000in}{0.000000in}}{\pgfqpoint{0.048611in}{0.000000in}}{%
\pgfpathmoveto{\pgfqpoint{0.000000in}{0.000000in}}%
\pgfpathlineto{\pgfqpoint{0.048611in}{0.000000in}}%
\pgfusepath{stroke,fill}%
}%
\begin{pgfscope}%
\pgfsys@transformshift{4.507477in}{1.813591in}%
\pgfsys@useobject{currentmarker}{}%
\end{pgfscope}%
\end{pgfscope}%
\begin{pgfscope}%
\definecolor{textcolor}{rgb}{0.000000,0.000000,0.000000}%
\pgfsetstrokecolor{textcolor}%
\pgfsetfillcolor{textcolor}%
\pgftext[x=4.604699in, y=1.775035in, left, base]{\color{textcolor}\rmfamily\fontsize{8.000000}{9.600000}\selectfont \(\displaystyle {21.00}\)}%
\end{pgfscope}%
\begin{pgfscope}%
\pgfsetbuttcap%
\pgfsetroundjoin%
\definecolor{currentfill}{rgb}{0.000000,0.000000,0.000000}%
\pgfsetfillcolor{currentfill}%
\pgfsetlinewidth{0.803000pt}%
\definecolor{currentstroke}{rgb}{0.000000,0.000000,0.000000}%
\pgfsetstrokecolor{currentstroke}%
\pgfsetdash{}{0pt}%
\pgfsys@defobject{currentmarker}{\pgfqpoint{0.000000in}{0.000000in}}{\pgfqpoint{0.048611in}{0.000000in}}{%
\pgfpathmoveto{\pgfqpoint{0.000000in}{0.000000in}}%
\pgfpathlineto{\pgfqpoint{0.048611in}{0.000000in}}%
\pgfusepath{stroke,fill}%
}%
\begin{pgfscope}%
\pgfsys@transformshift{4.507477in}{2.086150in}%
\pgfsys@useobject{currentmarker}{}%
\end{pgfscope}%
\end{pgfscope}%
\begin{pgfscope}%
\definecolor{textcolor}{rgb}{0.000000,0.000000,0.000000}%
\pgfsetstrokecolor{textcolor}%
\pgfsetfillcolor{textcolor}%
\pgftext[x=4.604699in, y=2.047594in, left, base]{\color{textcolor}\rmfamily\fontsize{8.000000}{9.600000}\selectfont \(\displaystyle {21.25}\)}%
\end{pgfscope}%
\begin{pgfscope}%
\pgfsetbuttcap%
\pgfsetroundjoin%
\definecolor{currentfill}{rgb}{0.000000,0.000000,0.000000}%
\pgfsetfillcolor{currentfill}%
\pgfsetlinewidth{0.803000pt}%
\definecolor{currentstroke}{rgb}{0.000000,0.000000,0.000000}%
\pgfsetstrokecolor{currentstroke}%
\pgfsetdash{}{0pt}%
\pgfsys@defobject{currentmarker}{\pgfqpoint{0.000000in}{0.000000in}}{\pgfqpoint{0.048611in}{0.000000in}}{%
\pgfpathmoveto{\pgfqpoint{0.000000in}{0.000000in}}%
\pgfpathlineto{\pgfqpoint{0.048611in}{0.000000in}}%
\pgfusepath{stroke,fill}%
}%
\begin{pgfscope}%
\pgfsys@transformshift{4.507477in}{2.358709in}%
\pgfsys@useobject{currentmarker}{}%
\end{pgfscope}%
\end{pgfscope}%
\begin{pgfscope}%
\definecolor{textcolor}{rgb}{0.000000,0.000000,0.000000}%
\pgfsetstrokecolor{textcolor}%
\pgfsetfillcolor{textcolor}%
\pgftext[x=4.604699in, y=2.320154in, left, base]{\color{textcolor}\rmfamily\fontsize{8.000000}{9.600000}\selectfont \(\displaystyle {21.50}\)}%
\end{pgfscope}%
\begin{pgfscope}%
\pgfsetbuttcap%
\pgfsetroundjoin%
\definecolor{currentfill}{rgb}{0.000000,0.000000,0.000000}%
\pgfsetfillcolor{currentfill}%
\pgfsetlinewidth{0.803000pt}%
\definecolor{currentstroke}{rgb}{0.000000,0.000000,0.000000}%
\pgfsetstrokecolor{currentstroke}%
\pgfsetdash{}{0pt}%
\pgfsys@defobject{currentmarker}{\pgfqpoint{0.000000in}{0.000000in}}{\pgfqpoint{0.048611in}{0.000000in}}{%
\pgfpathmoveto{\pgfqpoint{0.000000in}{0.000000in}}%
\pgfpathlineto{\pgfqpoint{0.048611in}{0.000000in}}%
\pgfusepath{stroke,fill}%
}%
\begin{pgfscope}%
\pgfsys@transformshift{4.507477in}{2.631268in}%
\pgfsys@useobject{currentmarker}{}%
\end{pgfscope}%
\end{pgfscope}%
\begin{pgfscope}%
\definecolor{textcolor}{rgb}{0.000000,0.000000,0.000000}%
\pgfsetstrokecolor{textcolor}%
\pgfsetfillcolor{textcolor}%
\pgftext[x=4.604699in, y=2.592713in, left, base]{\color{textcolor}\rmfamily\fontsize{8.000000}{9.600000}\selectfont \(\displaystyle {21.75}\)}%
\end{pgfscope}%
\begin{pgfscope}%
\pgfsetbuttcap%
\pgfsetroundjoin%
\definecolor{currentfill}{rgb}{0.000000,0.000000,0.000000}%
\pgfsetfillcolor{currentfill}%
\pgfsetlinewidth{0.803000pt}%
\definecolor{currentstroke}{rgb}{0.000000,0.000000,0.000000}%
\pgfsetstrokecolor{currentstroke}%
\pgfsetdash{}{0pt}%
\pgfsys@defobject{currentmarker}{\pgfqpoint{0.000000in}{0.000000in}}{\pgfqpoint{0.048611in}{0.000000in}}{%
\pgfpathmoveto{\pgfqpoint{0.000000in}{0.000000in}}%
\pgfpathlineto{\pgfqpoint{0.048611in}{0.000000in}}%
\pgfusepath{stroke,fill}%
}%
\begin{pgfscope}%
\pgfsys@transformshift{4.507477in}{2.903828in}%
\pgfsys@useobject{currentmarker}{}%
\end{pgfscope}%
\end{pgfscope}%
\begin{pgfscope}%
\definecolor{textcolor}{rgb}{0.000000,0.000000,0.000000}%
\pgfsetstrokecolor{textcolor}%
\pgfsetfillcolor{textcolor}%
\pgftext[x=4.604699in, y=2.865272in, left, base]{\color{textcolor}\rmfamily\fontsize{8.000000}{9.600000}\selectfont \(\displaystyle {22.00}\)}%
\end{pgfscope}%
\begin{pgfscope}%
\definecolor{textcolor}{rgb}{0.000000,0.000000,0.000000}%
\pgfsetstrokecolor{textcolor}%
\pgfsetfillcolor{textcolor}%
\pgftext[x=4.929163in,y=1.737274in,,top,rotate=90.000000]{\color{textcolor}\rmfamily\fontsize{10.000000}{12.000000}\selectfont Temperature in °C}%
\end{pgfscope}%
\begin{pgfscope}%
\pgfpathrectangle{\pgfqpoint{0.633813in}{0.538014in}}{\pgfqpoint{3.873664in}{2.398521in}}%
\pgfusepath{clip}%
\pgfsetrectcap%
\pgfsetroundjoin%
\pgfsetlinewidth{0.501875pt}%
\definecolor{currentstroke}{rgb}{0.698039,0.133333,0.133333}%
\pgfsetstrokecolor{currentstroke}%
\pgfsetstrokeopacity{0.700000}%
\pgfsetdash{}{0pt}%
\pgfpathmoveto{\pgfqpoint{0.810420in}{1.944419in}}%
\pgfpathlineto{\pgfqpoint{0.816575in}{2.086150in}}%
\pgfpathlineto{\pgfqpoint{0.819142in}{2.009833in}}%
\pgfpathlineto{\pgfqpoint{0.824401in}{2.151564in}}%
\pgfpathlineto{\pgfqpoint{0.826846in}{2.086150in}}%
\pgfpathlineto{\pgfqpoint{0.832797in}{2.216978in}}%
\pgfpathlineto{\pgfqpoint{0.835243in}{2.151564in}}%
\pgfpathlineto{\pgfqpoint{0.837689in}{2.216978in}}%
\pgfpathlineto{\pgfqpoint{0.843354in}{2.282393in}}%
\pgfpathlineto{\pgfqpoint{0.845800in}{2.216978in}}%
\pgfpathlineto{\pgfqpoint{0.848246in}{2.282393in}}%
\pgfpathlineto{\pgfqpoint{0.856765in}{2.358709in}}%
\pgfpathlineto{\pgfqpoint{0.859210in}{2.282393in}}%
\pgfpathlineto{\pgfqpoint{0.861656in}{2.358709in}}%
\pgfpathlineto{\pgfqpoint{0.868096in}{2.424123in}}%
\pgfpathlineto{\pgfqpoint{0.870542in}{2.358709in}}%
\pgfpathlineto{\pgfqpoint{0.872987in}{2.424123in}}%
\pgfpathlineto{\pgfqpoint{0.875840in}{2.358709in}}%
\pgfpathlineto{\pgfqpoint{0.878286in}{2.424123in}}%
\pgfpathlineto{\pgfqpoint{0.884808in}{2.489538in}}%
\pgfpathlineto{\pgfqpoint{0.887253in}{2.424123in}}%
\pgfpathlineto{\pgfqpoint{0.889699in}{2.489538in}}%
\pgfpathlineto{\pgfqpoint{0.901642in}{2.554952in}}%
\pgfpathlineto{\pgfqpoint{0.904088in}{2.489538in}}%
\pgfpathlineto{\pgfqpoint{0.906574in}{2.554952in}}%
\pgfpathlineto{\pgfqpoint{0.909223in}{2.489538in}}%
\pgfpathlineto{\pgfqpoint{0.911669in}{2.554952in}}%
\pgfpathlineto{\pgfqpoint{0.922430in}{2.631268in}}%
\pgfpathlineto{\pgfqpoint{0.924875in}{2.554952in}}%
\pgfpathlineto{\pgfqpoint{0.927321in}{2.631268in}}%
\pgfpathlineto{\pgfqpoint{0.930011in}{2.554952in}}%
\pgfpathlineto{\pgfqpoint{0.932457in}{2.631268in}}%
\pgfpathlineto{\pgfqpoint{0.945256in}{2.696683in}}%
\pgfpathlineto{\pgfqpoint{0.947701in}{2.631268in}}%
\pgfpathlineto{\pgfqpoint{0.950147in}{2.696683in}}%
\pgfpathlineto{\pgfqpoint{0.952593in}{2.631268in}}%
\pgfpathlineto{\pgfqpoint{0.955038in}{2.696683in}}%
\pgfpathlineto{\pgfqpoint{0.969590in}{2.762097in}}%
\pgfpathlineto{\pgfqpoint{0.972036in}{2.696683in}}%
\pgfpathlineto{\pgfqpoint{0.974522in}{2.762097in}}%
\pgfpathlineto{\pgfqpoint{0.976968in}{2.696683in}}%
\pgfpathlineto{\pgfqpoint{0.979413in}{2.762097in}}%
\pgfpathlineto{\pgfqpoint{0.982144in}{2.696683in}}%
\pgfpathlineto{\pgfqpoint{0.984590in}{2.762097in}}%
\pgfpathlineto{\pgfqpoint{0.997144in}{2.827511in}}%
\pgfpathlineto{\pgfqpoint{0.999590in}{2.762097in}}%
\pgfpathlineto{\pgfqpoint{1.002035in}{2.827511in}}%
\pgfpathlineto{\pgfqpoint{1.004481in}{2.762097in}}%
\pgfpathlineto{\pgfqpoint{1.006927in}{2.827511in}}%
\pgfpathlineto{\pgfqpoint{1.009494in}{2.762097in}}%
\pgfpathlineto{\pgfqpoint{1.011940in}{2.827511in}}%
\pgfpathlineto{\pgfqpoint{1.017647in}{2.696683in}}%
\pgfpathlineto{\pgfqpoint{1.020092in}{2.631268in}}%
\pgfpathlineto{\pgfqpoint{1.024984in}{2.358709in}}%
\pgfpathlineto{\pgfqpoint{1.032320in}{2.151564in}}%
\pgfpathlineto{\pgfqpoint{1.034766in}{2.009833in}}%
\pgfpathlineto{\pgfqpoint{1.050867in}{1.606446in}}%
\pgfpathlineto{\pgfqpoint{1.057388in}{1.464715in}}%
\pgfpathlineto{\pgfqpoint{1.066763in}{1.333887in}}%
\pgfpathlineto{\pgfqpoint{1.069250in}{1.399301in}}%
\pgfpathlineto{\pgfqpoint{1.074304in}{1.268473in}}%
\pgfpathlineto{\pgfqpoint{1.079684in}{1.192156in}}%
\pgfpathlineto{\pgfqpoint{1.087510in}{1.126742in}}%
\pgfpathlineto{\pgfqpoint{1.090037in}{1.192156in}}%
\pgfpathlineto{\pgfqpoint{1.095336in}{1.061328in}}%
\pgfpathlineto{\pgfqpoint{1.097782in}{1.126742in}}%
\pgfpathlineto{\pgfqpoint{1.100228in}{1.061328in}}%
\pgfpathlineto{\pgfqpoint{1.102755in}{1.126742in}}%
\pgfpathlineto{\pgfqpoint{1.105200in}{1.061328in}}%
\pgfpathlineto{\pgfqpoint{1.111355in}{0.995913in}}%
\pgfpathlineto{\pgfqpoint{1.113801in}{1.061328in}}%
\pgfpathlineto{\pgfqpoint{1.116287in}{0.995913in}}%
\pgfpathlineto{\pgfqpoint{1.118855in}{1.061328in}}%
\pgfpathlineto{\pgfqpoint{1.121301in}{0.995913in}}%
\pgfpathlineto{\pgfqpoint{1.125988in}{0.919597in}}%
\pgfpathlineto{\pgfqpoint{1.128475in}{0.995913in}}%
\pgfpathlineto{\pgfqpoint{1.130920in}{0.919597in}}%
\pgfpathlineto{\pgfqpoint{1.138216in}{0.854183in}}%
\pgfpathlineto{\pgfqpoint{1.140662in}{0.919597in}}%
\pgfpathlineto{\pgfqpoint{1.143108in}{0.854183in}}%
\pgfpathlineto{\pgfqpoint{1.146287in}{0.919597in}}%
\pgfpathlineto{\pgfqpoint{1.151341in}{0.788768in}}%
\pgfpathlineto{\pgfqpoint{1.153787in}{0.854183in}}%
\pgfpathlineto{\pgfqpoint{1.156273in}{0.788768in}}%
\pgfpathlineto{\pgfqpoint{1.158719in}{0.854183in}}%
\pgfpathlineto{\pgfqpoint{1.161165in}{0.788768in}}%
\pgfpathlineto{\pgfqpoint{1.171722in}{0.854183in}}%
\pgfpathlineto{\pgfqpoint{1.174167in}{0.788768in}}%
\pgfpathlineto{\pgfqpoint{1.176654in}{0.854183in}}%
\pgfpathlineto{\pgfqpoint{1.179099in}{0.788768in}}%
\pgfpathlineto{\pgfqpoint{1.181830in}{0.854183in}}%
\pgfpathlineto{\pgfqpoint{1.184276in}{0.788768in}}%
\pgfpathlineto{\pgfqpoint{1.202577in}{0.854183in}}%
\pgfpathlineto{\pgfqpoint{1.207183in}{0.919597in}}%
\pgfpathlineto{\pgfqpoint{1.224303in}{1.268473in}}%
\pgfpathlineto{\pgfqpoint{1.250390in}{1.671860in}}%
\pgfpathlineto{\pgfqpoint{1.273909in}{1.944419in}}%
\pgfpathlineto{\pgfqpoint{1.276476in}{1.879005in}}%
\pgfpathlineto{\pgfqpoint{1.281449in}{2.009833in}}%
\pgfpathlineto{\pgfqpoint{1.284710in}{1.944419in}}%
\pgfpathlineto{\pgfqpoint{1.289968in}{2.086150in}}%
\pgfpathlineto{\pgfqpoint{1.292658in}{2.009833in}}%
\pgfpathlineto{\pgfqpoint{1.295104in}{2.086150in}}%
\pgfpathlineto{\pgfqpoint{1.298487in}{2.151564in}}%
\pgfpathlineto{\pgfqpoint{1.300933in}{2.086150in}}%
\pgfpathlineto{\pgfqpoint{1.303378in}{2.151564in}}%
\pgfpathlineto{\pgfqpoint{1.308351in}{2.216978in}}%
\pgfpathlineto{\pgfqpoint{1.310838in}{2.151564in}}%
\pgfpathlineto{\pgfqpoint{1.313283in}{2.216978in}}%
\pgfpathlineto{\pgfqpoint{1.318704in}{2.282393in}}%
\pgfpathlineto{\pgfqpoint{1.321150in}{2.216978in}}%
\pgfpathlineto{\pgfqpoint{1.323596in}{2.282393in}}%
\pgfpathlineto{\pgfqpoint{1.332237in}{2.358709in}}%
\pgfpathlineto{\pgfqpoint{1.334723in}{2.282393in}}%
\pgfpathlineto{\pgfqpoint{1.337169in}{2.358709in}}%
\pgfpathlineto{\pgfqpoint{1.344180in}{2.424123in}}%
\pgfpathlineto{\pgfqpoint{1.346625in}{2.358709in}}%
\pgfpathlineto{\pgfqpoint{1.349071in}{2.424123in}}%
\pgfpathlineto{\pgfqpoint{1.351924in}{2.358709in}}%
\pgfpathlineto{\pgfqpoint{1.354370in}{2.424123in}}%
\pgfpathlineto{\pgfqpoint{1.359139in}{2.489538in}}%
\pgfpathlineto{\pgfqpoint{1.361585in}{2.424123in}}%
\pgfpathlineto{\pgfqpoint{1.364030in}{2.489538in}}%
\pgfpathlineto{\pgfqpoint{1.368677in}{2.424123in}}%
\pgfpathlineto{\pgfqpoint{1.371123in}{2.489538in}}%
\pgfpathlineto{\pgfqpoint{1.378215in}{2.554952in}}%
\pgfpathlineto{\pgfqpoint{1.380661in}{2.489538in}}%
\pgfpathlineto{\pgfqpoint{1.383106in}{2.554952in}}%
\pgfpathlineto{\pgfqpoint{1.385878in}{2.489538in}}%
\pgfpathlineto{\pgfqpoint{1.388324in}{2.554952in}}%
\pgfpathlineto{\pgfqpoint{1.397128in}{2.631268in}}%
\pgfpathlineto{\pgfqpoint{1.399573in}{2.554952in}}%
\pgfpathlineto{\pgfqpoint{1.402019in}{2.631268in}}%
\pgfpathlineto{\pgfqpoint{1.404506in}{2.554952in}}%
\pgfpathlineto{\pgfqpoint{1.406951in}{2.631268in}}%
\pgfpathlineto{\pgfqpoint{1.418894in}{2.696683in}}%
\pgfpathlineto{\pgfqpoint{1.421340in}{2.631268in}}%
\pgfpathlineto{\pgfqpoint{1.423826in}{2.696683in}}%
\pgfpathlineto{\pgfqpoint{1.426272in}{2.631268in}}%
\pgfpathlineto{\pgfqpoint{1.428717in}{2.696683in}}%
\pgfpathlineto{\pgfqpoint{1.431693in}{2.631268in}}%
\pgfpathlineto{\pgfqpoint{1.434138in}{2.696683in}}%
\pgfpathlineto{\pgfqpoint{1.441801in}{2.762097in}}%
\pgfpathlineto{\pgfqpoint{1.444247in}{2.696683in}}%
\pgfpathlineto{\pgfqpoint{1.446734in}{2.762097in}}%
\pgfpathlineto{\pgfqpoint{1.449179in}{2.696683in}}%
\pgfpathlineto{\pgfqpoint{1.451625in}{2.762097in}}%
\pgfpathlineto{\pgfqpoint{1.454070in}{2.696683in}}%
\pgfpathlineto{\pgfqpoint{1.456516in}{2.762097in}}%
\pgfpathlineto{\pgfqpoint{1.473432in}{2.827511in}}%
\pgfpathlineto{\pgfqpoint{1.475877in}{2.762097in}}%
\pgfpathlineto{\pgfqpoint{1.478445in}{2.827511in}}%
\pgfpathlineto{\pgfqpoint{1.480891in}{2.762097in}}%
\pgfpathlineto{\pgfqpoint{1.483418in}{2.827511in}}%
\pgfpathlineto{\pgfqpoint{1.485864in}{2.762097in}}%
\pgfpathlineto{\pgfqpoint{1.488309in}{2.827511in}}%
\pgfpathlineto{\pgfqpoint{1.494016in}{2.696683in}}%
\pgfpathlineto{\pgfqpoint{1.496461in}{2.631268in}}%
\pgfpathlineto{\pgfqpoint{1.498907in}{2.489538in}}%
\pgfpathlineto{\pgfqpoint{1.501353in}{2.424123in}}%
\pgfpathlineto{\pgfqpoint{1.503798in}{2.282393in}}%
\pgfpathlineto{\pgfqpoint{1.508690in}{2.151564in}}%
\pgfpathlineto{\pgfqpoint{1.511135in}{2.009833in}}%
\pgfpathlineto{\pgfqpoint{1.527276in}{1.606446in}}%
\pgfpathlineto{\pgfqpoint{1.536285in}{1.464715in}}%
\pgfpathlineto{\pgfqpoint{1.541787in}{1.399301in}}%
\pgfpathlineto{\pgfqpoint{1.544274in}{1.464715in}}%
\pgfpathlineto{\pgfqpoint{1.549165in}{1.333887in}}%
\pgfpathlineto{\pgfqpoint{1.556013in}{1.268473in}}%
\pgfpathlineto{\pgfqpoint{1.558458in}{1.333887in}}%
\pgfpathlineto{\pgfqpoint{1.563716in}{1.192156in}}%
\pgfpathlineto{\pgfqpoint{1.566488in}{1.268473in}}%
\pgfpathlineto{\pgfqpoint{1.568934in}{1.192156in}}%
\pgfpathlineto{\pgfqpoint{1.575048in}{1.126742in}}%
\pgfpathlineto{\pgfqpoint{1.577534in}{1.192156in}}%
\pgfpathlineto{\pgfqpoint{1.579980in}{1.126742in}}%
\pgfpathlineto{\pgfqpoint{1.583608in}{1.061328in}}%
\pgfpathlineto{\pgfqpoint{1.589314in}{0.995913in}}%
\pgfpathlineto{\pgfqpoint{1.591760in}{1.061328in}}%
\pgfpathlineto{\pgfqpoint{1.594205in}{0.995913in}}%
\pgfpathlineto{\pgfqpoint{1.596692in}{1.061328in}}%
\pgfpathlineto{\pgfqpoint{1.599137in}{0.995913in}}%
\pgfpathlineto{\pgfqpoint{1.603132in}{0.919597in}}%
\pgfpathlineto{\pgfqpoint{1.605578in}{0.995913in}}%
\pgfpathlineto{\pgfqpoint{1.608023in}{0.919597in}}%
\pgfpathlineto{\pgfqpoint{1.617724in}{0.854183in}}%
\pgfpathlineto{\pgfqpoint{1.620170in}{0.919597in}}%
\pgfpathlineto{\pgfqpoint{1.622616in}{0.854183in}}%
\pgfpathlineto{\pgfqpoint{1.625102in}{0.919597in}}%
\pgfpathlineto{\pgfqpoint{1.627548in}{0.854183in}}%
\pgfpathlineto{\pgfqpoint{1.632887in}{0.788768in}}%
\pgfpathlineto{\pgfqpoint{1.635333in}{0.854183in}}%
\pgfpathlineto{\pgfqpoint{1.637778in}{0.788768in}}%
\pgfpathlineto{\pgfqpoint{1.640387in}{0.854183in}}%
\pgfpathlineto{\pgfqpoint{1.642833in}{0.788768in}}%
\pgfpathlineto{\pgfqpoint{1.645319in}{0.854183in}}%
\pgfpathlineto{\pgfqpoint{1.647765in}{0.788768in}}%
\pgfpathlineto{\pgfqpoint{1.650659in}{0.854183in}}%
\pgfpathlineto{\pgfqpoint{1.653104in}{0.788768in}}%
\pgfpathlineto{\pgfqpoint{1.671569in}{0.723354in}}%
\pgfpathlineto{\pgfqpoint{1.674015in}{0.788768in}}%
\pgfpathlineto{\pgfqpoint{1.676460in}{0.723354in}}%
\pgfpathlineto{\pgfqpoint{1.678906in}{0.788768in}}%
\pgfpathlineto{\pgfqpoint{1.681352in}{0.723354in}}%
\pgfpathlineto{\pgfqpoint{1.689463in}{0.919597in}}%
\pgfpathlineto{\pgfqpoint{1.711066in}{1.333887in}}%
\pgfpathlineto{\pgfqpoint{1.729979in}{1.606446in}}%
\pgfpathlineto{\pgfqpoint{1.745590in}{1.813591in}}%
\pgfpathlineto{\pgfqpoint{1.757085in}{1.944419in}}%
\pgfpathlineto{\pgfqpoint{1.759530in}{1.879005in}}%
\pgfpathlineto{\pgfqpoint{1.765237in}{2.009833in}}%
\pgfpathlineto{\pgfqpoint{1.768172in}{1.944419in}}%
\pgfpathlineto{\pgfqpoint{1.773307in}{2.086150in}}%
\pgfpathlineto{\pgfqpoint{1.776038in}{2.009833in}}%
\pgfpathlineto{\pgfqpoint{1.778484in}{2.086150in}}%
\pgfpathlineto{\pgfqpoint{1.782764in}{2.151564in}}%
\pgfpathlineto{\pgfqpoint{1.785250in}{2.086150in}}%
\pgfpathlineto{\pgfqpoint{1.790875in}{2.216978in}}%
\pgfpathlineto{\pgfqpoint{1.793362in}{2.151564in}}%
\pgfpathlineto{\pgfqpoint{1.795807in}{2.216978in}}%
\pgfpathlineto{\pgfqpoint{1.801229in}{2.282393in}}%
\pgfpathlineto{\pgfqpoint{1.803674in}{2.216978in}}%
\pgfpathlineto{\pgfqpoint{1.806120in}{2.282393in}}%
\pgfpathlineto{\pgfqpoint{1.815169in}{2.358709in}}%
\pgfpathlineto{\pgfqpoint{1.817614in}{2.282393in}}%
\pgfpathlineto{\pgfqpoint{1.820060in}{2.358709in}}%
\pgfpathlineto{\pgfqpoint{1.826663in}{2.424123in}}%
\pgfpathlineto{\pgfqpoint{1.829109in}{2.358709in}}%
\pgfpathlineto{\pgfqpoint{1.831595in}{2.424123in}}%
\pgfpathlineto{\pgfqpoint{1.834897in}{2.358709in}}%
\pgfpathlineto{\pgfqpoint{1.837342in}{2.424123in}}%
\pgfpathlineto{\pgfqpoint{1.844231in}{2.489538in}}%
\pgfpathlineto{\pgfqpoint{1.846677in}{2.424123in}}%
\pgfpathlineto{\pgfqpoint{1.849122in}{2.489538in}}%
\pgfpathlineto{\pgfqpoint{1.862492in}{2.554952in}}%
\pgfpathlineto{\pgfqpoint{1.864937in}{2.489538in}}%
\pgfpathlineto{\pgfqpoint{1.867383in}{2.554952in}}%
\pgfpathlineto{\pgfqpoint{1.869829in}{2.489538in}}%
\pgfpathlineto{\pgfqpoint{1.872274in}{2.554952in}}%
\pgfpathlineto{\pgfqpoint{1.881853in}{2.631268in}}%
\pgfpathlineto{\pgfqpoint{1.884299in}{2.554952in}}%
\pgfpathlineto{\pgfqpoint{1.886744in}{2.631268in}}%
\pgfpathlineto{\pgfqpoint{1.889271in}{2.554952in}}%
\pgfpathlineto{\pgfqpoint{1.891717in}{2.631268in}}%
\pgfpathlineto{\pgfqpoint{1.904964in}{2.696683in}}%
\pgfpathlineto{\pgfqpoint{1.907410in}{2.631268in}}%
\pgfpathlineto{\pgfqpoint{1.909856in}{2.696683in}}%
\pgfpathlineto{\pgfqpoint{1.912342in}{2.631268in}}%
\pgfpathlineto{\pgfqpoint{1.914788in}{2.696683in}}%
\pgfpathlineto{\pgfqpoint{1.929421in}{2.762097in}}%
\pgfpathlineto{\pgfqpoint{1.931866in}{2.696683in}}%
\pgfpathlineto{\pgfqpoint{1.934760in}{2.762097in}}%
\pgfpathlineto{\pgfqpoint{1.937206in}{2.696683in}}%
\pgfpathlineto{\pgfqpoint{1.939692in}{2.762097in}}%
\pgfpathlineto{\pgfqpoint{1.942179in}{2.696683in}}%
\pgfpathlineto{\pgfqpoint{1.944624in}{2.762097in}}%
\pgfpathlineto{\pgfqpoint{1.959543in}{2.827511in}}%
\pgfpathlineto{\pgfqpoint{1.961988in}{2.762097in}}%
\pgfpathlineto{\pgfqpoint{1.964842in}{2.827511in}}%
\pgfpathlineto{\pgfqpoint{1.967287in}{2.762097in}}%
\pgfpathlineto{\pgfqpoint{1.969733in}{2.827511in}}%
\pgfpathlineto{\pgfqpoint{1.972219in}{2.762097in}}%
\pgfpathlineto{\pgfqpoint{1.974665in}{2.827511in}}%
\pgfpathlineto{\pgfqpoint{1.979923in}{2.696683in}}%
\pgfpathlineto{\pgfqpoint{1.982369in}{2.554952in}}%
\pgfpathlineto{\pgfqpoint{1.984814in}{2.489538in}}%
\pgfpathlineto{\pgfqpoint{1.989706in}{2.282393in}}%
\pgfpathlineto{\pgfqpoint{1.992151in}{2.216978in}}%
\pgfpathlineto{\pgfqpoint{1.997042in}{2.009833in}}%
\pgfpathlineto{\pgfqpoint{2.012531in}{1.606446in}}%
\pgfpathlineto{\pgfqpoint{2.021132in}{1.464715in}}%
\pgfpathlineto{\pgfqpoint{2.023618in}{1.541032in}}%
\pgfpathlineto{\pgfqpoint{2.028510in}{1.399301in}}%
\pgfpathlineto{\pgfqpoint{2.032219in}{1.333887in}}%
\pgfpathlineto{\pgfqpoint{2.034705in}{1.399301in}}%
\pgfpathlineto{\pgfqpoint{2.039597in}{1.268473in}}%
\pgfpathlineto{\pgfqpoint{2.044529in}{1.192156in}}%
\pgfpathlineto{\pgfqpoint{2.047015in}{1.268473in}}%
\pgfpathlineto{\pgfqpoint{2.052273in}{1.126742in}}%
\pgfpathlineto{\pgfqpoint{2.064053in}{0.995913in}}%
\pgfpathlineto{\pgfqpoint{2.066499in}{1.061328in}}%
\pgfpathlineto{\pgfqpoint{2.068944in}{0.995913in}}%
\pgfpathlineto{\pgfqpoint{2.077667in}{0.919597in}}%
\pgfpathlineto{\pgfqpoint{2.080113in}{0.995913in}}%
\pgfpathlineto{\pgfqpoint{2.082558in}{0.919597in}}%
\pgfpathlineto{\pgfqpoint{2.085126in}{0.995913in}}%
\pgfpathlineto{\pgfqpoint{2.087572in}{0.919597in}}%
\pgfpathlineto{\pgfqpoint{2.095316in}{0.854183in}}%
\pgfpathlineto{\pgfqpoint{2.097762in}{0.919597in}}%
\pgfpathlineto{\pgfqpoint{2.100208in}{0.854183in}}%
\pgfpathlineto{\pgfqpoint{2.102653in}{0.919597in}}%
\pgfpathlineto{\pgfqpoint{2.105099in}{0.854183in}}%
\pgfpathlineto{\pgfqpoint{2.108074in}{0.919597in}}%
\pgfpathlineto{\pgfqpoint{2.110520in}{0.854183in}}%
\pgfpathlineto{\pgfqpoint{2.115493in}{0.788768in}}%
\pgfpathlineto{\pgfqpoint{2.117938in}{0.854183in}}%
\pgfpathlineto{\pgfqpoint{2.120384in}{0.788768in}}%
\pgfpathlineto{\pgfqpoint{2.122911in}{0.854183in}}%
\pgfpathlineto{\pgfqpoint{2.125357in}{0.788768in}}%
\pgfpathlineto{\pgfqpoint{2.139501in}{0.723354in}}%
\pgfpathlineto{\pgfqpoint{2.141946in}{0.788768in}}%
\pgfpathlineto{\pgfqpoint{2.144433in}{0.723354in}}%
\pgfpathlineto{\pgfqpoint{2.146919in}{0.788768in}}%
\pgfpathlineto{\pgfqpoint{2.149365in}{0.723354in}}%
\pgfpathlineto{\pgfqpoint{2.154664in}{0.647038in}}%
\pgfpathlineto{\pgfqpoint{2.157109in}{0.723354in}}%
\pgfpathlineto{\pgfqpoint{2.159555in}{0.647038in}}%
\pgfpathlineto{\pgfqpoint{2.162001in}{0.723354in}}%
\pgfpathlineto{\pgfqpoint{2.164446in}{0.647038in}}%
\pgfpathlineto{\pgfqpoint{2.174351in}{0.723354in}}%
\pgfpathlineto{\pgfqpoint{2.198196in}{1.192156in}}%
\pgfpathlineto{\pgfqpoint{2.210628in}{1.399301in}}%
\pgfpathlineto{\pgfqpoint{2.246212in}{1.879005in}}%
\pgfpathlineto{\pgfqpoint{2.260152in}{2.009833in}}%
\pgfpathlineto{\pgfqpoint{2.262679in}{1.944419in}}%
\pgfpathlineto{\pgfqpoint{2.267978in}{2.086150in}}%
\pgfpathlineto{\pgfqpoint{2.270505in}{2.009833in}}%
\pgfpathlineto{\pgfqpoint{2.272951in}{2.086150in}}%
\pgfpathlineto{\pgfqpoint{2.276864in}{2.151564in}}%
\pgfpathlineto{\pgfqpoint{2.279432in}{2.086150in}}%
\pgfpathlineto{\pgfqpoint{2.284772in}{2.216978in}}%
\pgfpathlineto{\pgfqpoint{2.287217in}{2.151564in}}%
\pgfpathlineto{\pgfqpoint{2.289663in}{2.216978in}}%
\pgfpathlineto{\pgfqpoint{2.295125in}{2.282393in}}%
\pgfpathlineto{\pgfqpoint{2.297570in}{2.216978in}}%
\pgfpathlineto{\pgfqpoint{2.300016in}{2.282393in}}%
\pgfpathlineto{\pgfqpoint{2.307027in}{2.358709in}}%
\pgfpathlineto{\pgfqpoint{2.309472in}{2.282393in}}%
\pgfpathlineto{\pgfqpoint{2.311918in}{2.358709in}}%
\pgfpathlineto{\pgfqpoint{2.319703in}{2.424123in}}%
\pgfpathlineto{\pgfqpoint{2.322149in}{2.358709in}}%
\pgfpathlineto{\pgfqpoint{2.324595in}{2.424123in}}%
\pgfpathlineto{\pgfqpoint{2.334785in}{2.489538in}}%
\pgfpathlineto{\pgfqpoint{2.337230in}{2.424123in}}%
\pgfpathlineto{\pgfqpoint{2.339676in}{2.489538in}}%
\pgfpathlineto{\pgfqpoint{2.342122in}{2.424123in}}%
\pgfpathlineto{\pgfqpoint{2.344567in}{2.489538in}}%
\pgfpathlineto{\pgfqpoint{2.353779in}{2.554952in}}%
\pgfpathlineto{\pgfqpoint{2.356225in}{2.489538in}}%
\pgfpathlineto{\pgfqpoint{2.358711in}{2.554952in}}%
\pgfpathlineto{\pgfqpoint{2.361442in}{2.489538in}}%
\pgfpathlineto{\pgfqpoint{2.363888in}{2.554952in}}%
\pgfpathlineto{\pgfqpoint{2.374404in}{2.631268in}}%
\pgfpathlineto{\pgfqpoint{2.376850in}{2.554952in}}%
\pgfpathlineto{\pgfqpoint{2.379295in}{2.631268in}}%
\pgfpathlineto{\pgfqpoint{2.381741in}{2.554952in}}%
\pgfpathlineto{\pgfqpoint{2.384187in}{2.631268in}}%
\pgfpathlineto{\pgfqpoint{2.387488in}{2.554952in}}%
\pgfpathlineto{\pgfqpoint{2.389934in}{2.631268in}}%
\pgfpathlineto{\pgfqpoint{2.400328in}{2.696683in}}%
\pgfpathlineto{\pgfqpoint{2.402774in}{2.631268in}}%
\pgfpathlineto{\pgfqpoint{2.405219in}{2.696683in}}%
\pgfpathlineto{\pgfqpoint{2.407746in}{2.631268in}}%
\pgfpathlineto{\pgfqpoint{2.410192in}{2.696683in}}%
\pgfpathlineto{\pgfqpoint{2.413167in}{2.631268in}}%
\pgfpathlineto{\pgfqpoint{2.415613in}{2.696683in}}%
\pgfpathlineto{\pgfqpoint{2.426252in}{2.762097in}}%
\pgfpathlineto{\pgfqpoint{2.428697in}{2.696683in}}%
\pgfpathlineto{\pgfqpoint{2.431184in}{2.762097in}}%
\pgfpathlineto{\pgfqpoint{2.433629in}{2.696683in}}%
\pgfpathlineto{\pgfqpoint{2.436116in}{2.762097in}}%
\pgfpathlineto{\pgfqpoint{2.438643in}{2.696683in}}%
\pgfpathlineto{\pgfqpoint{2.441088in}{2.762097in}}%
\pgfpathlineto{\pgfqpoint{2.456414in}{2.696683in}}%
\pgfpathlineto{\pgfqpoint{2.458860in}{2.631268in}}%
\pgfpathlineto{\pgfqpoint{2.461306in}{2.489538in}}%
\pgfpathlineto{\pgfqpoint{2.463751in}{2.424123in}}%
\pgfpathlineto{\pgfqpoint{2.466197in}{2.282393in}}%
\pgfpathlineto{\pgfqpoint{2.468643in}{2.216978in}}%
\pgfpathlineto{\pgfqpoint{2.473534in}{2.009833in}}%
\pgfpathlineto{\pgfqpoint{2.488575in}{1.606446in}}%
\pgfpathlineto{\pgfqpoint{2.510789in}{1.268473in}}%
\pgfpathlineto{\pgfqpoint{2.513520in}{1.333887in}}%
\pgfpathlineto{\pgfqpoint{2.518778in}{1.192156in}}%
\pgfpathlineto{\pgfqpoint{2.521631in}{1.268473in}}%
\pgfpathlineto{\pgfqpoint{2.524077in}{1.192156in}}%
\pgfpathlineto{\pgfqpoint{2.529335in}{1.126742in}}%
\pgfpathlineto{\pgfqpoint{2.531781in}{1.192156in}}%
\pgfpathlineto{\pgfqpoint{2.534226in}{1.126742in}}%
\pgfpathlineto{\pgfqpoint{2.538588in}{1.061328in}}%
\pgfpathlineto{\pgfqpoint{2.541074in}{1.126742in}}%
\pgfpathlineto{\pgfqpoint{2.543520in}{1.061328in}}%
\pgfpathlineto{\pgfqpoint{2.548656in}{0.995913in}}%
\pgfpathlineto{\pgfqpoint{2.551142in}{1.061328in}}%
\pgfpathlineto{\pgfqpoint{2.553588in}{0.995913in}}%
\pgfpathlineto{\pgfqpoint{2.560354in}{0.919597in}}%
\pgfpathlineto{\pgfqpoint{2.562800in}{0.995913in}}%
\pgfpathlineto{\pgfqpoint{2.565245in}{0.919597in}}%
\pgfpathlineto{\pgfqpoint{2.567732in}{0.995913in}}%
\pgfpathlineto{\pgfqpoint{2.570177in}{0.919597in}}%
\pgfpathlineto{\pgfqpoint{2.574009in}{0.854183in}}%
\pgfpathlineto{\pgfqpoint{2.576536in}{0.919597in}}%
\pgfpathlineto{\pgfqpoint{2.578982in}{0.854183in}}%
\pgfpathlineto{\pgfqpoint{2.586645in}{0.788768in}}%
\pgfpathlineto{\pgfqpoint{2.589090in}{0.854183in}}%
\pgfpathlineto{\pgfqpoint{2.592188in}{0.788768in}}%
\pgfpathlineto{\pgfqpoint{2.594634in}{0.854183in}}%
\pgfpathlineto{\pgfqpoint{2.597079in}{0.788768in}}%
\pgfpathlineto{\pgfqpoint{2.604824in}{0.723354in}}%
\pgfpathlineto{\pgfqpoint{2.607269in}{0.788768in}}%
\pgfpathlineto{\pgfqpoint{2.609797in}{0.723354in}}%
\pgfpathlineto{\pgfqpoint{2.612242in}{0.788768in}}%
\pgfpathlineto{\pgfqpoint{2.614729in}{0.723354in}}%
\pgfpathlineto{\pgfqpoint{2.617174in}{0.788768in}}%
\pgfpathlineto{\pgfqpoint{2.619620in}{0.723354in}}%
\pgfpathlineto{\pgfqpoint{2.622066in}{0.788768in}}%
\pgfpathlineto{\pgfqpoint{2.624511in}{0.723354in}}%
\pgfpathlineto{\pgfqpoint{2.635598in}{0.647038in}}%
\pgfpathlineto{\pgfqpoint{2.638044in}{0.723354in}}%
\pgfpathlineto{\pgfqpoint{2.640489in}{0.647038in}}%
\pgfpathlineto{\pgfqpoint{2.643017in}{0.723354in}}%
\pgfpathlineto{\pgfqpoint{2.645462in}{0.647038in}}%
\pgfpathlineto{\pgfqpoint{2.648560in}{0.723354in}}%
\pgfpathlineto{\pgfqpoint{2.656468in}{0.854183in}}%
\pgfpathlineto{\pgfqpoint{2.660218in}{0.919597in}}%
\pgfpathlineto{\pgfqpoint{2.666780in}{1.061328in}}%
\pgfpathlineto{\pgfqpoint{2.686671in}{1.399301in}}%
\pgfpathlineto{\pgfqpoint{2.705421in}{1.671860in}}%
\pgfpathlineto{\pgfqpoint{2.708193in}{1.606446in}}%
\pgfpathlineto{\pgfqpoint{2.713084in}{1.737274in}}%
\pgfpathlineto{\pgfqpoint{2.717527in}{1.813591in}}%
\pgfpathlineto{\pgfqpoint{2.723030in}{1.879005in}}%
\pgfpathlineto{\pgfqpoint{2.725557in}{1.813591in}}%
\pgfpathlineto{\pgfqpoint{2.730530in}{1.944419in}}%
\pgfpathlineto{\pgfqpoint{2.737703in}{2.009833in}}%
\pgfpathlineto{\pgfqpoint{2.740231in}{1.944419in}}%
\pgfpathlineto{\pgfqpoint{2.745733in}{2.086150in}}%
\pgfpathlineto{\pgfqpoint{2.748301in}{2.009833in}}%
\pgfpathlineto{\pgfqpoint{2.750747in}{2.086150in}}%
\pgfpathlineto{\pgfqpoint{2.754619in}{2.151564in}}%
\pgfpathlineto{\pgfqpoint{2.757146in}{2.086150in}}%
\pgfpathlineto{\pgfqpoint{2.759592in}{2.151564in}}%
\pgfpathlineto{\pgfqpoint{2.763342in}{2.216978in}}%
\pgfpathlineto{\pgfqpoint{2.765787in}{2.151564in}}%
\pgfpathlineto{\pgfqpoint{2.768233in}{2.216978in}}%
\pgfpathlineto{\pgfqpoint{2.773328in}{2.282393in}}%
\pgfpathlineto{\pgfqpoint{2.775774in}{2.216978in}}%
\pgfpathlineto{\pgfqpoint{2.778219in}{2.282393in}}%
\pgfpathlineto{\pgfqpoint{2.786005in}{2.358709in}}%
\pgfpathlineto{\pgfqpoint{2.788450in}{2.282393in}}%
\pgfpathlineto{\pgfqpoint{2.790896in}{2.358709in}}%
\pgfpathlineto{\pgfqpoint{2.797173in}{2.424123in}}%
\pgfpathlineto{\pgfqpoint{2.799619in}{2.358709in}}%
\pgfpathlineto{\pgfqpoint{2.802064in}{2.424123in}}%
\pgfpathlineto{\pgfqpoint{2.804592in}{2.358709in}}%
\pgfpathlineto{\pgfqpoint{2.807037in}{2.424123in}}%
\pgfpathlineto{\pgfqpoint{2.815678in}{2.489538in}}%
\pgfpathlineto{\pgfqpoint{2.818124in}{2.424123in}}%
\pgfpathlineto{\pgfqpoint{2.820570in}{2.489538in}}%
\pgfpathlineto{\pgfqpoint{2.832920in}{2.554952in}}%
\pgfpathlineto{\pgfqpoint{2.835366in}{2.489538in}}%
\pgfpathlineto{\pgfqpoint{2.837811in}{2.554952in}}%
\pgfpathlineto{\pgfqpoint{2.840379in}{2.489538in}}%
\pgfpathlineto{\pgfqpoint{2.842825in}{2.554952in}}%
\pgfpathlineto{\pgfqpoint{2.852322in}{2.631268in}}%
\pgfpathlineto{\pgfqpoint{2.854768in}{2.554952in}}%
\pgfpathlineto{\pgfqpoint{2.857254in}{2.631268in}}%
\pgfpathlineto{\pgfqpoint{2.859700in}{2.554952in}}%
\pgfpathlineto{\pgfqpoint{2.862146in}{2.631268in}}%
\pgfpathlineto{\pgfqpoint{2.878898in}{2.696683in}}%
\pgfpathlineto{\pgfqpoint{2.881344in}{2.631268in}}%
\pgfpathlineto{\pgfqpoint{2.883871in}{2.696683in}}%
\pgfpathlineto{\pgfqpoint{2.886317in}{2.631268in}}%
\pgfpathlineto{\pgfqpoint{2.888762in}{2.696683in}}%
\pgfpathlineto{\pgfqpoint{2.891371in}{2.631268in}}%
\pgfpathlineto{\pgfqpoint{2.893817in}{2.696683in}}%
\pgfpathlineto{\pgfqpoint{2.904659in}{2.762097in}}%
\pgfpathlineto{\pgfqpoint{2.907104in}{2.696683in}}%
\pgfpathlineto{\pgfqpoint{2.909632in}{2.762097in}}%
\pgfpathlineto{\pgfqpoint{2.912077in}{2.696683in}}%
\pgfpathlineto{\pgfqpoint{2.914523in}{2.762097in}}%
\pgfpathlineto{\pgfqpoint{2.917009in}{2.696683in}}%
\pgfpathlineto{\pgfqpoint{2.919455in}{2.762097in}}%
\pgfpathlineto{\pgfqpoint{2.923001in}{2.696683in}}%
\pgfpathlineto{\pgfqpoint{2.925447in}{2.762097in}}%
\pgfpathlineto{\pgfqpoint{2.936248in}{2.827511in}}%
\pgfpathlineto{\pgfqpoint{2.938694in}{2.762097in}}%
\pgfpathlineto{\pgfqpoint{2.941140in}{2.827511in}}%
\pgfpathlineto{\pgfqpoint{2.953368in}{2.216978in}}%
\pgfpathlineto{\pgfqpoint{2.955813in}{2.151564in}}%
\pgfpathlineto{\pgfqpoint{2.958259in}{2.009833in}}%
\pgfpathlineto{\pgfqpoint{2.975745in}{1.541032in}}%
\pgfpathlineto{\pgfqpoint{2.984550in}{1.399301in}}%
\pgfpathlineto{\pgfqpoint{2.989400in}{1.333887in}}%
\pgfpathlineto{\pgfqpoint{2.991968in}{1.399301in}}%
\pgfpathlineto{\pgfqpoint{2.996859in}{1.268473in}}%
\pgfpathlineto{\pgfqpoint{3.002077in}{1.192156in}}%
\pgfpathlineto{\pgfqpoint{3.004563in}{1.268473in}}%
\pgfpathlineto{\pgfqpoint{3.007009in}{1.192156in}}%
\pgfpathlineto{\pgfqpoint{3.010514in}{1.126742in}}%
\pgfpathlineto{\pgfqpoint{3.012960in}{1.192156in}}%
\pgfpathlineto{\pgfqpoint{3.015405in}{1.126742in}}%
\pgfpathlineto{\pgfqpoint{3.020419in}{1.061328in}}%
\pgfpathlineto{\pgfqpoint{3.022865in}{1.126742in}}%
\pgfpathlineto{\pgfqpoint{3.025310in}{1.061328in}}%
\pgfpathlineto{\pgfqpoint{3.027756in}{1.126742in}}%
\pgfpathlineto{\pgfqpoint{3.030202in}{1.061328in}}%
\pgfpathlineto{\pgfqpoint{3.040840in}{0.995913in}}%
\pgfpathlineto{\pgfqpoint{3.043367in}{1.061328in}}%
\pgfpathlineto{\pgfqpoint{3.045813in}{0.995913in}}%
\pgfpathlineto{\pgfqpoint{3.051927in}{0.919597in}}%
\pgfpathlineto{\pgfqpoint{3.054373in}{0.995913in}}%
\pgfpathlineto{\pgfqpoint{3.056818in}{0.919597in}}%
\pgfpathlineto{\pgfqpoint{3.065256in}{0.854183in}}%
\pgfpathlineto{\pgfqpoint{3.067701in}{0.919597in}}%
\pgfpathlineto{\pgfqpoint{3.070147in}{0.854183in}}%
\pgfpathlineto{\pgfqpoint{3.072633in}{0.919597in}}%
\pgfpathlineto{\pgfqpoint{3.075079in}{0.854183in}}%
\pgfpathlineto{\pgfqpoint{3.090323in}{0.788768in}}%
\pgfpathlineto{\pgfqpoint{3.092769in}{0.854183in}}%
\pgfpathlineto{\pgfqpoint{3.095215in}{0.788768in}}%
\pgfpathlineto{\pgfqpoint{3.097742in}{0.854183in}}%
\pgfpathlineto{\pgfqpoint{3.100187in}{0.788768in}}%
\pgfpathlineto{\pgfqpoint{3.105323in}{0.854183in}}%
\pgfpathlineto{\pgfqpoint{3.107769in}{0.788768in}}%
\pgfpathlineto{\pgfqpoint{3.114331in}{0.723354in}}%
\pgfpathlineto{\pgfqpoint{3.116777in}{0.788768in}}%
\pgfpathlineto{\pgfqpoint{3.119345in}{0.723354in}}%
\pgfpathlineto{\pgfqpoint{3.121831in}{0.788768in}}%
\pgfpathlineto{\pgfqpoint{3.124277in}{0.723354in}}%
\pgfpathlineto{\pgfqpoint{3.126763in}{0.788768in}}%
\pgfpathlineto{\pgfqpoint{3.129209in}{0.723354in}}%
\pgfpathlineto{\pgfqpoint{3.131655in}{0.788768in}}%
\pgfpathlineto{\pgfqpoint{3.134100in}{0.723354in}}%
\pgfpathlineto{\pgfqpoint{3.149467in}{0.647038in}}%
\pgfpathlineto{\pgfqpoint{3.151913in}{0.723354in}}%
\pgfpathlineto{\pgfqpoint{3.154399in}{0.647038in}}%
\pgfpathlineto{\pgfqpoint{3.156845in}{0.723354in}}%
\pgfpathlineto{\pgfqpoint{3.187537in}{1.333887in}}%
\pgfpathlineto{\pgfqpoint{3.196668in}{1.464715in}}%
\pgfpathlineto{\pgfqpoint{3.200988in}{1.541032in}}%
\pgfpathlineto{\pgfqpoint{3.206776in}{1.606446in}}%
\pgfpathlineto{\pgfqpoint{3.210608in}{1.671860in}}%
\pgfpathlineto{\pgfqpoint{3.213135in}{1.606446in}}%
\pgfpathlineto{\pgfqpoint{3.218026in}{1.737274in}}%
\pgfpathlineto{\pgfqpoint{3.222225in}{1.813591in}}%
\pgfpathlineto{\pgfqpoint{3.224670in}{1.737274in}}%
\pgfpathlineto{\pgfqpoint{3.229602in}{1.879005in}}%
\pgfpathlineto{\pgfqpoint{3.242972in}{2.009833in}}%
\pgfpathlineto{\pgfqpoint{3.245499in}{1.944419in}}%
\pgfpathlineto{\pgfqpoint{3.250553in}{2.086150in}}%
\pgfpathlineto{\pgfqpoint{3.252999in}{2.009833in}}%
\pgfpathlineto{\pgfqpoint{3.258542in}{2.151564in}}%
\pgfpathlineto{\pgfqpoint{3.261029in}{2.086150in}}%
\pgfpathlineto{\pgfqpoint{3.263474in}{2.151564in}}%
\pgfpathlineto{\pgfqpoint{3.267224in}{2.216978in}}%
\pgfpathlineto{\pgfqpoint{3.269711in}{2.151564in}}%
\pgfpathlineto{\pgfqpoint{3.272156in}{2.216978in}}%
\pgfpathlineto{\pgfqpoint{3.276885in}{2.282393in}}%
\pgfpathlineto{\pgfqpoint{3.279330in}{2.216978in}}%
\pgfpathlineto{\pgfqpoint{3.281776in}{2.282393in}}%
\pgfpathlineto{\pgfqpoint{3.290010in}{2.358709in}}%
\pgfpathlineto{\pgfqpoint{3.292455in}{2.282393in}}%
\pgfpathlineto{\pgfqpoint{3.294901in}{2.358709in}}%
\pgfpathlineto{\pgfqpoint{3.301993in}{2.424123in}}%
\pgfpathlineto{\pgfqpoint{3.304439in}{2.358709in}}%
\pgfpathlineto{\pgfqpoint{3.306885in}{2.424123in}}%
\pgfpathlineto{\pgfqpoint{3.317197in}{2.489538in}}%
\pgfpathlineto{\pgfqpoint{3.319643in}{2.424123in}}%
\pgfpathlineto{\pgfqpoint{3.322088in}{2.489538in}}%
\pgfpathlineto{\pgfqpoint{3.324656in}{2.424123in}}%
\pgfpathlineto{\pgfqpoint{3.327102in}{2.489538in}}%
\pgfpathlineto{\pgfqpoint{3.334765in}{2.554952in}}%
\pgfpathlineto{\pgfqpoint{3.337210in}{2.489538in}}%
\pgfpathlineto{\pgfqpoint{3.339656in}{2.554952in}}%
\pgfpathlineto{\pgfqpoint{3.342102in}{2.489538in}}%
\pgfpathlineto{\pgfqpoint{3.344547in}{2.554952in}}%
\pgfpathlineto{\pgfqpoint{3.356735in}{2.631268in}}%
\pgfpathlineto{\pgfqpoint{3.359180in}{2.554952in}}%
\pgfpathlineto{\pgfqpoint{3.361626in}{2.631268in}}%
\pgfpathlineto{\pgfqpoint{3.364235in}{2.554952in}}%
\pgfpathlineto{\pgfqpoint{3.366680in}{2.631268in}}%
\pgfpathlineto{\pgfqpoint{3.377889in}{2.696683in}}%
\pgfpathlineto{\pgfqpoint{3.380335in}{2.631268in}}%
\pgfpathlineto{\pgfqpoint{3.382781in}{2.696683in}}%
\pgfpathlineto{\pgfqpoint{3.385267in}{2.631268in}}%
\pgfpathlineto{\pgfqpoint{3.387713in}{2.696683in}}%
\pgfpathlineto{\pgfqpoint{3.390729in}{2.631268in}}%
\pgfpathlineto{\pgfqpoint{3.393175in}{2.696683in}}%
\pgfpathlineto{\pgfqpoint{3.403813in}{2.762097in}}%
\pgfpathlineto{\pgfqpoint{3.406259in}{2.696683in}}%
\pgfpathlineto{\pgfqpoint{3.408786in}{2.762097in}}%
\pgfpathlineto{\pgfqpoint{3.411232in}{2.696683in}}%
\pgfpathlineto{\pgfqpoint{3.413677in}{2.762097in}}%
\pgfpathlineto{\pgfqpoint{3.416204in}{2.696683in}}%
\pgfpathlineto{\pgfqpoint{3.418650in}{2.762097in}}%
\pgfpathlineto{\pgfqpoint{3.428799in}{2.696683in}}%
\pgfpathlineto{\pgfqpoint{3.433691in}{2.554952in}}%
\pgfpathlineto{\pgfqpoint{3.441028in}{2.216978in}}%
\pgfpathlineto{\pgfqpoint{3.443473in}{2.151564in}}%
\pgfpathlineto{\pgfqpoint{3.445919in}{2.009833in}}%
\pgfpathlineto{\pgfqpoint{3.461041in}{1.606446in}}%
\pgfpathlineto{\pgfqpoint{3.480362in}{1.333887in}}%
\pgfpathlineto{\pgfqpoint{3.482807in}{1.399301in}}%
\pgfpathlineto{\pgfqpoint{3.487699in}{1.268473in}}%
\pgfpathlineto{\pgfqpoint{3.492182in}{1.192156in}}%
\pgfpathlineto{\pgfqpoint{3.494669in}{1.268473in}}%
\pgfpathlineto{\pgfqpoint{3.497114in}{1.192156in}}%
\pgfpathlineto{\pgfqpoint{3.500905in}{1.126742in}}%
\pgfpathlineto{\pgfqpoint{3.503391in}{1.192156in}}%
\pgfpathlineto{\pgfqpoint{3.508650in}{1.061328in}}%
\pgfpathlineto{\pgfqpoint{3.511095in}{1.126742in}}%
\pgfpathlineto{\pgfqpoint{3.516516in}{0.995913in}}%
\pgfpathlineto{\pgfqpoint{3.518962in}{1.061328in}}%
\pgfpathlineto{\pgfqpoint{3.521408in}{0.995913in}}%
\pgfpathlineto{\pgfqpoint{3.531149in}{0.919597in}}%
\pgfpathlineto{\pgfqpoint{3.533677in}{0.995913in}}%
\pgfpathlineto{\pgfqpoint{3.536122in}{0.919597in}}%
\pgfpathlineto{\pgfqpoint{3.542236in}{0.854183in}}%
\pgfpathlineto{\pgfqpoint{3.544682in}{0.919597in}}%
\pgfpathlineto{\pgfqpoint{3.547128in}{0.854183in}}%
\pgfpathlineto{\pgfqpoint{3.559682in}{0.788768in}}%
\pgfpathlineto{\pgfqpoint{3.562127in}{0.854183in}}%
\pgfpathlineto{\pgfqpoint{3.564573in}{0.788768in}}%
\pgfpathlineto{\pgfqpoint{3.573540in}{0.723354in}}%
\pgfpathlineto{\pgfqpoint{3.575986in}{0.788768in}}%
\pgfpathlineto{\pgfqpoint{3.578432in}{0.723354in}}%
\pgfpathlineto{\pgfqpoint{3.580959in}{0.788768in}}%
\pgfpathlineto{\pgfqpoint{3.583404in}{0.723354in}}%
\pgfpathlineto{\pgfqpoint{3.590415in}{0.647038in}}%
\pgfpathlineto{\pgfqpoint{3.592942in}{0.723354in}}%
\pgfpathlineto{\pgfqpoint{3.595388in}{0.647038in}}%
\pgfpathlineto{\pgfqpoint{3.597834in}{0.723354in}}%
\pgfpathlineto{\pgfqpoint{3.600279in}{0.647038in}}%
\pgfpathlineto{\pgfqpoint{3.615931in}{0.723354in}}%
\pgfpathlineto{\pgfqpoint{3.630564in}{0.995913in}}%
\pgfpathlineto{\pgfqpoint{3.645279in}{1.268473in}}%
\pgfpathlineto{\pgfqpoint{3.653187in}{1.399301in}}%
\pgfpathlineto{\pgfqpoint{3.658608in}{1.464715in}}%
\pgfpathlineto{\pgfqpoint{3.673404in}{1.671860in}}%
\pgfpathlineto{\pgfqpoint{3.679273in}{1.737274in}}%
\pgfpathlineto{\pgfqpoint{3.684695in}{1.813591in}}%
\pgfpathlineto{\pgfqpoint{3.687426in}{1.737274in}}%
\pgfpathlineto{\pgfqpoint{3.692317in}{1.879005in}}%
\pgfpathlineto{\pgfqpoint{3.697127in}{1.944419in}}%
\pgfpathlineto{\pgfqpoint{3.699654in}{1.879005in}}%
\pgfpathlineto{\pgfqpoint{3.705279in}{2.009833in}}%
\pgfpathlineto{\pgfqpoint{3.707806in}{1.944419in}}%
\pgfpathlineto{\pgfqpoint{3.712697in}{2.086150in}}%
\pgfpathlineto{\pgfqpoint{3.715183in}{2.009833in}}%
\pgfpathlineto{\pgfqpoint{3.720768in}{2.151564in}}%
\pgfpathlineto{\pgfqpoint{3.723213in}{2.086150in}}%
\pgfpathlineto{\pgfqpoint{3.725659in}{2.151564in}}%
\pgfpathlineto{\pgfqpoint{3.730102in}{2.216978in}}%
\pgfpathlineto{\pgfqpoint{3.732548in}{2.151564in}}%
\pgfpathlineto{\pgfqpoint{3.734993in}{2.216978in}}%
\pgfpathlineto{\pgfqpoint{3.740496in}{2.282393in}}%
\pgfpathlineto{\pgfqpoint{3.742941in}{2.216978in}}%
\pgfpathlineto{\pgfqpoint{3.745387in}{2.282393in}}%
\pgfpathlineto{\pgfqpoint{3.752031in}{2.358709in}}%
\pgfpathlineto{\pgfqpoint{3.754477in}{2.282393in}}%
\pgfpathlineto{\pgfqpoint{3.756922in}{2.358709in}}%
\pgfpathlineto{\pgfqpoint{3.763852in}{2.424123in}}%
\pgfpathlineto{\pgfqpoint{3.766297in}{2.358709in}}%
\pgfpathlineto{\pgfqpoint{3.768743in}{2.424123in}}%
\pgfpathlineto{\pgfqpoint{3.779096in}{2.489538in}}%
\pgfpathlineto{\pgfqpoint{3.781542in}{2.424123in}}%
\pgfpathlineto{\pgfqpoint{3.783987in}{2.489538in}}%
\pgfpathlineto{\pgfqpoint{3.786433in}{2.424123in}}%
\pgfpathlineto{\pgfqpoint{3.788879in}{2.489538in}}%
\pgfpathlineto{\pgfqpoint{3.797398in}{2.554952in}}%
\pgfpathlineto{\pgfqpoint{3.799843in}{2.489538in}}%
\pgfpathlineto{\pgfqpoint{3.802330in}{2.554952in}}%
\pgfpathlineto{\pgfqpoint{3.804775in}{2.489538in}}%
\pgfpathlineto{\pgfqpoint{3.807221in}{2.554952in}}%
\pgfpathlineto{\pgfqpoint{3.818552in}{2.631268in}}%
\pgfpathlineto{\pgfqpoint{3.820998in}{2.554952in}}%
\pgfpathlineto{\pgfqpoint{3.823444in}{2.631268in}}%
\pgfpathlineto{\pgfqpoint{3.825930in}{2.554952in}}%
\pgfpathlineto{\pgfqpoint{3.828376in}{2.631268in}}%
\pgfpathlineto{\pgfqpoint{3.841052in}{2.696683in}}%
\pgfpathlineto{\pgfqpoint{3.843498in}{2.631268in}}%
\pgfpathlineto{\pgfqpoint{3.845984in}{2.696683in}}%
\pgfpathlineto{\pgfqpoint{3.848430in}{2.631268in}}%
\pgfpathlineto{\pgfqpoint{3.850916in}{2.696683in}}%
\pgfpathlineto{\pgfqpoint{3.855074in}{2.631268in}}%
\pgfpathlineto{\pgfqpoint{3.857520in}{2.696683in}}%
\pgfpathlineto{\pgfqpoint{3.866976in}{2.762097in}}%
\pgfpathlineto{\pgfqpoint{3.869422in}{2.696683in}}%
\pgfpathlineto{\pgfqpoint{3.871908in}{2.762097in}}%
\pgfpathlineto{\pgfqpoint{3.874354in}{2.696683in}}%
\pgfpathlineto{\pgfqpoint{3.876799in}{2.762097in}}%
\pgfpathlineto{\pgfqpoint{3.884136in}{2.424123in}}%
\pgfpathlineto{\pgfqpoint{3.889027in}{2.282393in}}%
\pgfpathlineto{\pgfqpoint{3.891473in}{2.151564in}}%
\pgfpathlineto{\pgfqpoint{3.901256in}{1.879005in}}%
\pgfpathlineto{\pgfqpoint{3.903701in}{1.737274in}}%
\pgfpathlineto{\pgfqpoint{3.917397in}{1.464715in}}%
\pgfpathlineto{\pgfqpoint{3.927709in}{1.333887in}}%
\pgfpathlineto{\pgfqpoint{3.934027in}{1.268473in}}%
\pgfpathlineto{\pgfqpoint{3.936717in}{1.333887in}}%
\pgfpathlineto{\pgfqpoint{3.941609in}{1.192156in}}%
\pgfpathlineto{\pgfqpoint{3.946908in}{1.126742in}}%
\pgfpathlineto{\pgfqpoint{3.949435in}{1.192156in}}%
\pgfpathlineto{\pgfqpoint{3.951880in}{1.126742in}}%
\pgfpathlineto{\pgfqpoint{3.955956in}{1.061328in}}%
\pgfpathlineto{\pgfqpoint{3.958402in}{1.126742in}}%
\pgfpathlineto{\pgfqpoint{3.963905in}{0.995913in}}%
\pgfpathlineto{\pgfqpoint{3.966350in}{1.061328in}}%
\pgfpathlineto{\pgfqpoint{3.968796in}{0.995913in}}%
\pgfpathlineto{\pgfqpoint{3.975725in}{0.919597in}}%
\pgfpathlineto{\pgfqpoint{3.978171in}{0.995913in}}%
\pgfpathlineto{\pgfqpoint{3.980617in}{0.919597in}}%
\pgfpathlineto{\pgfqpoint{3.983103in}{0.995913in}}%
\pgfpathlineto{\pgfqpoint{3.985549in}{0.919597in}}%
\pgfpathlineto{\pgfqpoint{3.993252in}{0.854183in}}%
\pgfpathlineto{\pgfqpoint{3.995698in}{0.919597in}}%
\pgfpathlineto{\pgfqpoint{3.999000in}{0.854183in}}%
\pgfpathlineto{\pgfqpoint{4.001445in}{0.919597in}}%
\pgfpathlineto{\pgfqpoint{4.003891in}{0.854183in}}%
\pgfpathlineto{\pgfqpoint{4.007111in}{0.919597in}}%
\pgfpathlineto{\pgfqpoint{4.009557in}{0.854183in}}%
\pgfpathlineto{\pgfqpoint{4.014448in}{0.788768in}}%
\pgfpathlineto{\pgfqpoint{4.016894in}{0.854183in}}%
\pgfpathlineto{\pgfqpoint{4.019339in}{0.788768in}}%
\pgfpathlineto{\pgfqpoint{4.021907in}{0.854183in}}%
\pgfpathlineto{\pgfqpoint{4.024353in}{0.788768in}}%
\pgfpathlineto{\pgfqpoint{4.034543in}{0.723354in}}%
\pgfpathlineto{\pgfqpoint{4.036988in}{0.788768in}}%
\pgfpathlineto{\pgfqpoint{4.039434in}{0.723354in}}%
\pgfpathlineto{\pgfqpoint{4.041880in}{0.788768in}}%
\pgfpathlineto{\pgfqpoint{4.044325in}{0.723354in}}%
\pgfpathlineto{\pgfqpoint{4.051255in}{0.647038in}}%
\pgfpathlineto{\pgfqpoint{4.053700in}{0.723354in}}%
\pgfpathlineto{\pgfqpoint{4.056146in}{0.647038in}}%
\pgfpathlineto{\pgfqpoint{4.058632in}{0.723354in}}%
\pgfpathlineto{\pgfqpoint{4.061078in}{0.647038in}}%
\pgfpathlineto{\pgfqpoint{4.063564in}{0.723354in}}%
\pgfpathlineto{\pgfqpoint{4.066010in}{0.647038in}}%
\pgfpathlineto{\pgfqpoint{4.070983in}{0.788768in}}%
\pgfpathlineto{\pgfqpoint{4.085697in}{1.061328in}}%
\pgfpathlineto{\pgfqpoint{4.101064in}{1.333887in}}%
\pgfpathlineto{\pgfqpoint{4.125724in}{1.671860in}}%
\pgfpathlineto{\pgfqpoint{4.128333in}{1.606446in}}%
\pgfpathlineto{\pgfqpoint{4.133224in}{1.737274in}}%
\pgfpathlineto{\pgfqpoint{4.138116in}{1.813591in}}%
\pgfpathlineto{\pgfqpoint{4.143781in}{1.879005in}}%
\pgfpathlineto{\pgfqpoint{4.158496in}{2.009833in}}%
\pgfpathlineto{\pgfqpoint{4.160982in}{1.944419in}}%
\pgfpathlineto{\pgfqpoint{4.167259in}{2.086150in}}%
\pgfpathlineto{\pgfqpoint{4.169705in}{2.009833in}}%
\pgfpathlineto{\pgfqpoint{4.172151in}{2.086150in}}%
\pgfpathlineto{\pgfqpoint{4.176431in}{2.151564in}}%
\pgfpathlineto{\pgfqpoint{4.178876in}{2.086150in}}%
\pgfpathlineto{\pgfqpoint{4.181322in}{2.151564in}}%
\pgfpathlineto{\pgfqpoint{4.186050in}{2.216978in}}%
\pgfpathlineto{\pgfqpoint{4.188496in}{2.151564in}}%
\pgfpathlineto{\pgfqpoint{4.190941in}{2.216978in}}%
\pgfpathlineto{\pgfqpoint{4.197545in}{2.282393in}}%
\pgfpathlineto{\pgfqpoint{4.199990in}{2.216978in}}%
\pgfpathlineto{\pgfqpoint{4.202436in}{2.282393in}}%
\pgfpathlineto{\pgfqpoint{4.210384in}{2.358709in}}%
\pgfpathlineto{\pgfqpoint{4.212870in}{2.282393in}}%
\pgfpathlineto{\pgfqpoint{4.215316in}{2.358709in}}%
\pgfpathlineto{\pgfqpoint{4.224039in}{2.424123in}}%
\pgfpathlineto{\pgfqpoint{4.226485in}{2.358709in}}%
\pgfpathlineto{\pgfqpoint{4.228930in}{2.424123in}}%
\pgfpathlineto{\pgfqpoint{4.241077in}{2.489538in}}%
\pgfpathlineto{\pgfqpoint{4.243522in}{2.424123in}}%
\pgfpathlineto{\pgfqpoint{4.246009in}{2.489538in}}%
\pgfpathlineto{\pgfqpoint{4.248577in}{2.424123in}}%
\pgfpathlineto{\pgfqpoint{4.251022in}{2.489538in}}%
\pgfpathlineto{\pgfqpoint{4.259134in}{2.554952in}}%
\pgfpathlineto{\pgfqpoint{4.261579in}{2.489538in}}%
\pgfpathlineto{\pgfqpoint{4.264188in}{2.554952in}}%
\pgfpathlineto{\pgfqpoint{4.266634in}{2.489538in}}%
\pgfpathlineto{\pgfqpoint{4.269079in}{2.554952in}}%
\pgfpathlineto{\pgfqpoint{4.282327in}{2.631268in}}%
\pgfpathlineto{\pgfqpoint{4.284772in}{2.554952in}}%
\pgfpathlineto{\pgfqpoint{4.287218in}{2.631268in}}%
\pgfpathlineto{\pgfqpoint{4.289704in}{2.554952in}}%
\pgfpathlineto{\pgfqpoint{4.292150in}{2.631268in}}%
\pgfpathlineto{\pgfqpoint{4.295288in}{2.554952in}}%
\pgfpathlineto{\pgfqpoint{4.297734in}{2.631268in}}%
\pgfpathlineto{\pgfqpoint{4.311063in}{2.696683in}}%
\pgfpathlineto{\pgfqpoint{4.313508in}{2.631268in}}%
\pgfpathlineto{\pgfqpoint{4.315995in}{2.696683in}}%
\pgfpathlineto{\pgfqpoint{4.318440in}{2.631268in}}%
\pgfpathlineto{\pgfqpoint{4.320886in}{2.696683in}}%
\pgfpathlineto{\pgfqpoint{4.323943in}{2.631268in}}%
\pgfpathlineto{\pgfqpoint{4.326389in}{2.696683in}}%
\pgfpathlineto{\pgfqpoint{4.326389in}{2.696683in}}%
\pgfusepath{stroke}%
\end{pgfscope}%
\begin{pgfscope}%
\pgfsetrectcap%
\pgfsetmiterjoin%
\pgfsetlinewidth{0.803000pt}%
\definecolor{currentstroke}{rgb}{0.000000,0.000000,0.000000}%
\pgfsetstrokecolor{currentstroke}%
\pgfsetdash{}{0pt}%
\pgfpathmoveto{\pgfqpoint{0.633813in}{0.538014in}}%
\pgfpathlineto{\pgfqpoint{0.633813in}{2.936535in}}%
\pgfusepath{stroke}%
\end{pgfscope}%
\begin{pgfscope}%
\pgfsetrectcap%
\pgfsetmiterjoin%
\pgfsetlinewidth{0.803000pt}%
\definecolor{currentstroke}{rgb}{0.000000,0.000000,0.000000}%
\pgfsetstrokecolor{currentstroke}%
\pgfsetdash{}{0pt}%
\pgfpathmoveto{\pgfqpoint{4.507477in}{0.538014in}}%
\pgfpathlineto{\pgfqpoint{4.507477in}{2.936535in}}%
\pgfusepath{stroke}%
\end{pgfscope}%
\begin{pgfscope}%
\pgfsetrectcap%
\pgfsetmiterjoin%
\pgfsetlinewidth{0.803000pt}%
\definecolor{currentstroke}{rgb}{0.000000,0.000000,0.000000}%
\pgfsetstrokecolor{currentstroke}%
\pgfsetdash{}{0pt}%
\pgfpathmoveto{\pgfqpoint{0.633813in}{0.538014in}}%
\pgfpathlineto{\pgfqpoint{4.507477in}{0.538014in}}%
\pgfusepath{stroke}%
\end{pgfscope}%
\begin{pgfscope}%
\pgfsetrectcap%
\pgfsetmiterjoin%
\pgfsetlinewidth{0.803000pt}%
\definecolor{currentstroke}{rgb}{0.000000,0.000000,0.000000}%
\pgfsetstrokecolor{currentstroke}%
\pgfsetdash{}{0pt}%
\pgfpathmoveto{\pgfqpoint{0.633813in}{2.936535in}}%
\pgfpathlineto{\pgfqpoint{4.507477in}{2.936535in}}%
\pgfusepath{stroke}%
\end{pgfscope}%
\begin{pgfscope}%
\pgfsetbuttcap%
\pgfsetmiterjoin%
\definecolor{currentfill}{rgb}{1.000000,1.000000,1.000000}%
\pgfsetfillcolor{currentfill}%
\pgfsetfillopacity{0.800000}%
\pgfsetlinewidth{1.003750pt}%
\definecolor{currentstroke}{rgb}{0.800000,0.800000,0.800000}%
\pgfsetstrokecolor{currentstroke}%
\pgfsetstrokeopacity{0.800000}%
\pgfsetdash{}{0pt}%
\pgfpathmoveto{\pgfqpoint{0.711591in}{2.536313in}}%
\pgfpathlineto{\pgfqpoint{2.211146in}{2.536313in}}%
\pgfpathquadraticcurveto{\pgfqpoint{2.233369in}{2.536313in}}{\pgfqpoint{2.233369in}{2.558535in}}%
\pgfpathlineto{\pgfqpoint{2.233369in}{2.858757in}}%
\pgfpathquadraticcurveto{\pgfqpoint{2.233369in}{2.880979in}}{\pgfqpoint{2.211146in}{2.880979in}}%
\pgfpathlineto{\pgfqpoint{0.711591in}{2.880979in}}%
\pgfpathquadraticcurveto{\pgfqpoint{0.689369in}{2.880979in}}{\pgfqpoint{0.689369in}{2.858757in}}%
\pgfpathlineto{\pgfqpoint{0.689369in}{2.558535in}}%
\pgfpathquadraticcurveto{\pgfqpoint{0.689369in}{2.536313in}}{\pgfqpoint{0.711591in}{2.536313in}}%
\pgfpathlineto{\pgfqpoint{0.711591in}{2.536313in}}%
\pgfpathclose%
\pgfusepath{stroke,fill}%
\end{pgfscope}%
\begin{pgfscope}%
\pgfsetrectcap%
\pgfsetroundjoin%
\pgfsetlinewidth{0.501875pt}%
\definecolor{currentstroke}{rgb}{0.121569,0.466667,0.705882}%
\pgfsetstrokecolor{currentstroke}%
\pgfsetstrokeopacity{0.700000}%
\pgfsetdash{}{0pt}%
\pgfpathmoveto{\pgfqpoint{0.733813in}{2.797646in}}%
\pgfpathlineto{\pgfqpoint{0.844924in}{2.797646in}}%
\pgfpathlineto{\pgfqpoint{0.956035in}{2.797646in}}%
\pgfusepath{stroke}%
\end{pgfscope}%
\begin{pgfscope}%
\definecolor{textcolor}{rgb}{0.000000,0.000000,0.000000}%
\pgfsetstrokecolor{textcolor}%
\pgfsetfillcolor{textcolor}%
\pgftext[x=1.044924in,y=2.758757in,left,base]{\color{textcolor}\rmfamily\fontsize{8.000000}{9.600000}\selectfont DUT vs KS34470A}%
\end{pgfscope}%
\begin{pgfscope}%
\pgfsetrectcap%
\pgfsetroundjoin%
\pgfsetlinewidth{0.501875pt}%
\definecolor{currentstroke}{rgb}{0.698039,0.133333,0.133333}%
\pgfsetstrokecolor{currentstroke}%
\pgfsetstrokeopacity{0.700000}%
\pgfsetdash{}{0pt}%
\pgfpathmoveto{\pgfqpoint{0.733813in}{2.641201in}}%
\pgfpathlineto{\pgfqpoint{0.844924in}{2.641201in}}%
\pgfpathlineto{\pgfqpoint{0.956035in}{2.641201in}}%
\pgfusepath{stroke}%
\end{pgfscope}%
\begin{pgfscope}%
\definecolor{textcolor}{rgb}{0.000000,0.000000,0.000000}%
\pgfsetstrokecolor{textcolor}%
\pgfsetfillcolor{textcolor}%
\pgftext[x=1.044924in,y=2.602313in,left,base]{\color{textcolor}\rmfamily\fontsize{8.000000}{9.600000}\selectfont Ambient Temperature}%
\end{pgfscope}%
\end{pgfpicture}%
\makeatother%
\endgroup%

    \caption{Voltage deviation from the mean voltage of a LM399 negative \qty{10}{\volt} reference measured with a Keysight 34470A at \qty{100}{\plc}.}
    \label{fig:lm399_vs_34470a}
\end{figure}

Figure \ref{fig:lm399_vs_34470a} shows an example of such measurement. This measurement highlights one the problems encountered during those measurements. From this measurement it is unclear whether the features seen in the graph are only a result of ambient temperature changes due to the cycling of the air conditioning or popcorn noise on top of that. These results hightlight the fact, that sub-\unit{ppm} measurements not only requires high-end gear, but also a very stable environment. From the data it follows, that the temperature coefficient of the DMM in linear approximation is:
\begin{equation}
    \alpha_\device{34470A} \approx \frac{\qty{6.08}{\micro\volt}-(-\qty{9.30}{\micro\volt})}{(\qty{21.85}{\celsius}-\qty{19.96}{\celsius})\qty{10}{\volt}} = \qty[per-mode=symbol]{0.86}{\micro\volt \per \volt \per \kelvin}
\end{equation}

While the temperature coeeficient is vastly better than the specified \qty[per-mode=symbol]{2}{\micro\volt \per \volt \per \kelvin} \cite{datasheet_keysight34470A}, it is not low enough for this type of measurement. The multimeter must therefore be kept in a temperature controlled environment. This issue was resolved by replacing the stock air conditioning controller with a custom PID controller as discussed in section \ref{}. Lastly, the noise floor of the measurement is \qty{1.5}{\micro\volt_{RMS}}, resulting in an estimated signal-to-noise ratio (SNR) of about \qty{10}{\decibel}, which is suffient to detect the popcorn noise.

While the temperature issue was being worked on, testing of the Zener diodes continued. To work around the temperature drift of the DMM, the amplification of the reference voltage was increased to \qty{15}{\volt}, the same voltage required by the digital current driver, and a differential measurement was realized. This measurement was done against a primary \qty{15}{\volt} reference board. To ensure, that any popcorn noise found originates only in the DUT and not the primary reference used, several reference boards where tested against a \device{Fluke 5440B}. The \device{5440B} does not exhibit popcorn noise as it uses a different voltage reference ic, namely two Motorola SZA263 in series \cite{service_manual_fluke_5440b}. Finally, a board that did not show popcorn noise in a period of three days was selected. The serial number of this primary or golden reference is \textit{\#1}.

Using this differential technique, the results greatly improved

In order to test a large amount of Zener diodes, and considering the duration of the burn-in process, which can take anything between \qtyrange{100}{1000}{\hour}, it is necessary to have an automated setup. This consists of a digital multimeter (DMM) a scanner and test board, that holds the Zener diodes and provides the necessary infrastructure for the diodes.



To conclude, we need a high performance DMM, a scanner, and a test fixture. The choices will detailed in the following sections.



\subsection{A Scanner System for Testing Zener Diodes}
As discussed before the diodes need to be tested for \qty{1000}{\hour} and it is not be feasible to test them individually. So a minimum of 10 diodes must be tested at the same time. To keep the system compact, the test setup a scanner to multiplex a single multimeter input. Several commercial options currently available were considered for this project and are shown in table \ref{tab:list_of_daqs}.

\begin{table}[h]
    \centering
    \small
    \begin{tabular}{ |l|l|l|l|l|l|l|l| }
        \hline
        \multirow{2}{*}{} & \multicolumn{2}{l|}{Keysight} & \multicolumn{3}{l|}{Keithley} & Fluke & Rigol \\
        \cline{2-8}
        & DAQ973A & 34980A & DAQ6510 & 2750 & 3706 & 2680 & M300 \\
        \hline
        DMM & \num{6.5} & \num{6.5} & \num{6.5} & \num{6.5} & \num{7.5} & \qty{18}{\bit} & \num{6.5} \\
        \hline
        Channels & 3x20 & 8x40 & 2x10 & 5x20 & 6x60 & 6x20 & 5x32 \\
        \hline
        FET & \textcolor{green!60!black}{\checkmark} & \textcolor{green!60!black}{\checkmark} & \textcolor{green!60!black}{\checkmark} & \textcolor{green!60!black}{\checkmark} & \textcolor{green!60!black}{\checkmark} & \textcolor{red!80!black}{\ding{55}} & \textcolor{red!80!black}{\ding{55}} \\
        \hline
        Voltage & \qty{120}{\volt} & \qty{80}{\volt} & \qty{60}{\volt} & \qty{60}{\volt} & \qty{200}{\volt} & \qty{75}{\volt} & \qty{300}{\volt} \\
        \hline
        Card & DAQM900A & 34925A & 7710 & 7710 & 3724 & 2680A-PAI & MC3132 \\
        \hline
        USB & \textcolor{green!60!black}{\checkmark} & \textcolor{green!60!black}{\checkmark} & \textcolor{green!60!black}{\checkmark} & \textcolor{red!80!black}{\ding{55}} & \textcolor{green!60!black}{\checkmark} & \textcolor{red!80!black}{\ding{55}} & \textcolor{green!60!black}{\checkmark} \\
        \hline
        Ethernet & \textcolor{green!60!black}{\checkmark} & \textcolor{green!60!black}{\checkmark} & \textcolor{green!60!black}{\checkmark} & \textcolor{red!80!black}{\ding{55}} & \textcolor{green!60!black}{\checkmark} & \textcolor{green!60!black}{\checkmark} & \textcolor{green!60!black}{\checkmark} \\
        \hline
        GPIB & \textcolor{green!60!black}{\checkmark} & \textcolor{green!60!black}{\checkmark} & \textcolor{green!60!black}{\checkmark} & \textcolor{green!60!black}{\checkmark} & \textcolor{green!60!black}{\checkmark} & \textcolor{red!80!black}{\ding{55}} & \textcolor{green!60!black}{\checkmark} \\
        %DMM & & & & & & & \\
        \hline
    \end{tabular}
    \caption{Overview of scanner mainframes}
    \label{tab:list_of_daqs}
\end{table}

A recent trend to more compact devices has led major manufacturers to include multimeters in the scanner mainframe creating so called data acquisition units. Legacy devices, that only have switching capabilities are no available. For example Keithley replaced the small desktop switch mainframe \device{Model 7001} with the \device{DAQ6510} and Keysight is offering the \device{DAQ973A}, a scanning \num{6.5} digit DMM, that accepts extension cards. Unfortunately, for this project, as discussed above, the integrated \num{6.5} digit multimeter does not add any value.

The simplest option is to go with an \num{8.5} digit multimeter that already included a scanner option or buy a used \device{Keithley 7001} from a second-hand dealer. The author has tested both options and the simplicity of only having a single device to connect and program makes the integrated scanner card of the \device{Model 2002} very attractive.

\begin{figure}[ht]
    \centering
    \scalebox{0.7}{%
        \import{figures/}{simplified_scanner.tex}
    } % scalebox
    \caption{Simplified schematic of the scanner front-end with parasitic elements}
\end{figure}

The scanner card used to multiplex the DMM does have to meet several specifications. The most important aspects are the number channels and the lifetime of the relays. Other factors, such as channel to channel isolation, the contact potential, resistance and maximum voltage is not the limiting factor.

The reason is, that in this case, the voltage is low, there is no ac component involved and the the typical input impedance of high-end multimeters is far more than \qty{100}{\giga\ohm} \cite{datasheet_fluke8588A,article_3458A_input_mpedance_2,datasheet_keithley2002,article_3458A_input_mpedance}.



In this work the Keithley (now Tektronix) \device{Model 2002} was chosen for three reasons. It is a very compact system requiring only a half-sized 2U rack in comparison to the other DMMs, that are typically full-sized 2U rack devices. The other two advantages are the integrated scanner card slot, that allows to to fit a 10 channel scanner card and finally the \qty{20}{\volt} range. The latter is interesting for testing the final voltage reference boards, as these have a \qty{15}{\volt} output, which is too much for the \qty{10}{\volt} range of most DMMs, so that testing the voltage reference Printed circuit boards
(PCBs) one would have to switch to the \qty{100}{\volt} range and forgo an extra digit of resolution and add more noise.



The test setup consists of a mounting PCB, that holds up to 20 Zener diode. It provides power regulation and a minimal circuit required to support each diode. This circuit is given here:

\begin{figure}[ht]
    \centering
    \scalebox{0.7}{%
        \import{figures/}{zener_burnin.tex}
    } % scalebox
    \caption{Circuit used for burning in the Zener diodes}
\end{figure}

The compensation network is required when using the ADR1399, because of its very low dynamic impedance as recommended in the data sheet \cite{datasheet_ADR1399}. It is not strictly required for the LM399, but fitted nonetheless, because there are no downsides to it. This makes the board compatible with both types of references. Each Zener output is protected using an output buffer, which provides isolation and short circuit protection. Finally there is a common mode filter at the output to suppress high frequency noise via ground loops.

The two key metrics of concern, that need to measured are popcorn noise and drift.

digital multimeter and a scanner card


\begin{figure}[ht]
    \centering
    \import{figures/}{DIN_41612.tex}
    \caption{The extension connector used in several Keithley multimeters}
\end{figure}

\begin{table}[ht]
    \centering
    \begin{tabular}{llllll}
        \toprule
        Pin    & Function    & Cable Colour    & Pin    & Function    &  Cable Colour\\
        \midrule
        a1, b1    & \SIrange[explicit-sign=+]{6}{20}{\volt}    & brown    & \num{6}    & GND    & green/white\\
        a2, b2    & PD cathode    & red    & \num{7}    & LD Cathode    & blue/white\\
        a3, b3    & LD case (GND)    & red/white    & \num{8}    & LD Anode    & blue\\
        a4    & PD anode (GND)    & red/white    & \num{9}    & LD current    & green\\
        a5    & \SIrange{-6}{-20}{\volt}    & brown/white\\
        \bottomrule
    \end{tabular}
\end{table}

As a sidenote, for the pure entertainment of the author, several batches of LM399 Zener diodes were purchased from non-authorized dealers. Some were marked as refurbished, the others were not marked as such, but clearly were. These so-called refurbished diodes are not to be used in production devices. To entertain and warn the reader a small selection of examples are shown here in figure \ref{fig:fake_lm399}. All but one diode, which is shown for comparison, are refurbished.

\begin{figure}[h]
    \centering
    %\includegraphics[width=0.75\textwidth]{images/foo.png}
    \caption{Refurbished LM399 Zener diodes. From left to right: }
    \label{fig:fake_lm399}
\end{figure}

As it can be clearly seen, the sellers have gone to some effort to hide the fact, that these diodes have been used before. When a through-hole is soldered to the PCB, its legs will be trimmed to match the PCB thickness. In order to conceal this, the legs need to be extended to their original length. The legs of the LM399 are Kovar, because the LM399 is hermetically sealed with a glass seal and Kovar has the same coefficient of expansion as borosilicate glass. The forgers typically weld steel legs to the Kovar legs and then either gold-plate or tin them, as can be seen in fig. \ref{fig:fake_lm399_legs}.

\begin{figure}[h]
    \centering
    %\includegraphics[width=0.75\textwidth]{images/foo.png}
    \caption{Fake steel legs of a refurbished LM399.}
    \label{fig:fake_lm399_legs}
\end{figure}

Much to the delight of the author the refurbished diodes prove valuable for educational purposes. As the origin and method of extraction from the original circuit is unknown, but can be imagined to be rather savage, the diodes are typically faulty. They can therefore be used to validate the test setup and demonstrate the popcorn noise found in the LM399. A very good example in shown in fig. \ref{fig:fake_lm399_popcorn_noise}.

\begin{figure}[h]
    \centering
    %% Creator: Matplotlib, PGF backend
%%
%% To include the figure in your LaTeX document, write
%%   \input{<filename>.pgf}
%%
%% Make sure the required packages are loaded in your preamble
%%   \usepackage{pgf}
%%
%% Also ensure that all the required font packages are loaded; for instance,
%% the lmodern package is sometimes necessary when using math font.
%%   \usepackage{lmodern}
%%
%% Figures using additional raster images can only be included by \input if
%% they are in the same directory as the main LaTeX file. For loading figures
%% from other directories you can use the `import` package
%%   \usepackage{import}
%%
%% and then include the figures with
%%   \import{<path to file>}{<filename>.pgf}
%%
%% Matplotlib used the following preamble
%%   \usepackage{siunitx}
%%   \usepackage{fontspec}
%%
\begingroup%
\makeatletter%
\begin{pgfpicture}%
\pgfpathrectangle{\pgfpointorigin}{\pgfqpoint{5.208662in}{3.219130in}}%
\pgfusepath{use as bounding box, clip}%
\begin{pgfscope}%
\pgfsetbuttcap%
\pgfsetmiterjoin%
\definecolor{currentfill}{rgb}{1.000000,1.000000,1.000000}%
\pgfsetfillcolor{currentfill}%
\pgfsetlinewidth{0.000000pt}%
\definecolor{currentstroke}{rgb}{1.000000,1.000000,1.000000}%
\pgfsetstrokecolor{currentstroke}%
\pgfsetdash{}{0pt}%
\pgfpathmoveto{\pgfqpoint{0.000000in}{0.000000in}}%
\pgfpathlineto{\pgfqpoint{5.208662in}{0.000000in}}%
\pgfpathlineto{\pgfqpoint{5.208662in}{3.219130in}}%
\pgfpathlineto{\pgfqpoint{0.000000in}{3.219130in}}%
\pgfpathlineto{\pgfqpoint{0.000000in}{0.000000in}}%
\pgfpathclose%
\pgfusepath{fill}%
\end{pgfscope}%
\begin{pgfscope}%
\pgfsetbuttcap%
\pgfsetmiterjoin%
\definecolor{currentfill}{rgb}{1.000000,1.000000,1.000000}%
\pgfsetfillcolor{currentfill}%
\pgfsetlinewidth{0.000000pt}%
\definecolor{currentstroke}{rgb}{0.000000,0.000000,0.000000}%
\pgfsetstrokecolor{currentstroke}%
\pgfsetstrokeopacity{0.000000}%
\pgfsetdash{}{0pt}%
\pgfpathmoveto{\pgfqpoint{0.667540in}{0.539544in}}%
\pgfpathlineto{\pgfqpoint{5.058662in}{0.539544in}}%
\pgfpathlineto{\pgfqpoint{5.058662in}{2.944887in}}%
\pgfpathlineto{\pgfqpoint{0.667540in}{2.944887in}}%
\pgfpathlineto{\pgfqpoint{0.667540in}{0.539544in}}%
\pgfpathclose%
\pgfusepath{fill}%
\end{pgfscope}%
\begin{pgfscope}%
\pgfsetbuttcap%
\pgfsetroundjoin%
\definecolor{currentfill}{rgb}{0.000000,0.000000,0.000000}%
\pgfsetfillcolor{currentfill}%
\pgfsetlinewidth{0.803000pt}%
\definecolor{currentstroke}{rgb}{0.000000,0.000000,0.000000}%
\pgfsetstrokecolor{currentstroke}%
\pgfsetdash{}{0pt}%
\pgfsys@defobject{currentmarker}{\pgfqpoint{0.000000in}{-0.048611in}}{\pgfqpoint{0.000000in}{0.000000in}}{%
\pgfpathmoveto{\pgfqpoint{0.000000in}{0.000000in}}%
\pgfpathlineto{\pgfqpoint{0.000000in}{-0.048611in}}%
\pgfusepath{stroke,fill}%
}%
\begin{pgfscope}%
\pgfsys@transformshift{0.866046in}{0.539544in}%
\pgfsys@useobject{currentmarker}{}%
\end{pgfscope}%
\end{pgfscope}%
\begin{pgfscope}%
\definecolor{textcolor}{rgb}{0.000000,0.000000,0.000000}%
\pgfsetstrokecolor{textcolor}%
\pgfsetfillcolor{textcolor}%
\pgftext[x=0.866046in,y=0.442322in,,top]{\color{textcolor}\rmfamily\fontsize{8.000000}{9.600000}\selectfont \(\displaystyle {06{:}45}\)}%
\end{pgfscope}%
\begin{pgfscope}%
\pgfsetbuttcap%
\pgfsetroundjoin%
\definecolor{currentfill}{rgb}{0.000000,0.000000,0.000000}%
\pgfsetfillcolor{currentfill}%
\pgfsetlinewidth{0.803000pt}%
\definecolor{currentstroke}{rgb}{0.000000,0.000000,0.000000}%
\pgfsetstrokecolor{currentstroke}%
\pgfsetdash{}{0pt}%
\pgfsys@defobject{currentmarker}{\pgfqpoint{0.000000in}{-0.048611in}}{\pgfqpoint{0.000000in}{0.000000in}}{%
\pgfpathmoveto{\pgfqpoint{0.000000in}{0.000000in}}%
\pgfpathlineto{\pgfqpoint{0.000000in}{-0.048611in}}%
\pgfusepath{stroke,fill}%
}%
\begin{pgfscope}%
\pgfsys@transformshift{2.197480in}{0.539544in}%
\pgfsys@useobject{currentmarker}{}%
\end{pgfscope}%
\end{pgfscope}%
\begin{pgfscope}%
\definecolor{textcolor}{rgb}{0.000000,0.000000,0.000000}%
\pgfsetstrokecolor{textcolor}%
\pgfsetfillcolor{textcolor}%
\pgftext[x=2.197480in,y=0.442322in,,top]{\color{textcolor}\rmfamily\fontsize{8.000000}{9.600000}\selectfont \(\displaystyle {06{:}50}\)}%
\end{pgfscope}%
\begin{pgfscope}%
\pgfsetbuttcap%
\pgfsetroundjoin%
\definecolor{currentfill}{rgb}{0.000000,0.000000,0.000000}%
\pgfsetfillcolor{currentfill}%
\pgfsetlinewidth{0.803000pt}%
\definecolor{currentstroke}{rgb}{0.000000,0.000000,0.000000}%
\pgfsetstrokecolor{currentstroke}%
\pgfsetdash{}{0pt}%
\pgfsys@defobject{currentmarker}{\pgfqpoint{0.000000in}{-0.048611in}}{\pgfqpoint{0.000000in}{0.000000in}}{%
\pgfpathmoveto{\pgfqpoint{0.000000in}{0.000000in}}%
\pgfpathlineto{\pgfqpoint{0.000000in}{-0.048611in}}%
\pgfusepath{stroke,fill}%
}%
\begin{pgfscope}%
\pgfsys@transformshift{3.528915in}{0.539544in}%
\pgfsys@useobject{currentmarker}{}%
\end{pgfscope}%
\end{pgfscope}%
\begin{pgfscope}%
\definecolor{textcolor}{rgb}{0.000000,0.000000,0.000000}%
\pgfsetstrokecolor{textcolor}%
\pgfsetfillcolor{textcolor}%
\pgftext[x=3.528915in,y=0.442322in,,top]{\color{textcolor}\rmfamily\fontsize{8.000000}{9.600000}\selectfont \(\displaystyle {06{:}55}\)}%
\end{pgfscope}%
\begin{pgfscope}%
\pgfsetbuttcap%
\pgfsetroundjoin%
\definecolor{currentfill}{rgb}{0.000000,0.000000,0.000000}%
\pgfsetfillcolor{currentfill}%
\pgfsetlinewidth{0.803000pt}%
\definecolor{currentstroke}{rgb}{0.000000,0.000000,0.000000}%
\pgfsetstrokecolor{currentstroke}%
\pgfsetdash{}{0pt}%
\pgfsys@defobject{currentmarker}{\pgfqpoint{0.000000in}{-0.048611in}}{\pgfqpoint{0.000000in}{0.000000in}}{%
\pgfpathmoveto{\pgfqpoint{0.000000in}{0.000000in}}%
\pgfpathlineto{\pgfqpoint{0.000000in}{-0.048611in}}%
\pgfusepath{stroke,fill}%
}%
\begin{pgfscope}%
\pgfsys@transformshift{4.860349in}{0.539544in}%
\pgfsys@useobject{currentmarker}{}%
\end{pgfscope}%
\end{pgfscope}%
\begin{pgfscope}%
\definecolor{textcolor}{rgb}{0.000000,0.000000,0.000000}%
\pgfsetstrokecolor{textcolor}%
\pgfsetfillcolor{textcolor}%
\pgftext[x=4.860349in,y=0.442322in,,top]{\color{textcolor}\rmfamily\fontsize{8.000000}{9.600000}\selectfont \(\displaystyle {07{:}00}\)}%
\end{pgfscope}%
\begin{pgfscope}%
\definecolor{textcolor}{rgb}{0.000000,0.000000,0.000000}%
\pgfsetstrokecolor{textcolor}%
\pgfsetfillcolor{textcolor}%
\pgftext[x=2.863101in,y=0.288100in,,top]{\color{textcolor}\rmfamily\fontsize{10.000000}{12.000000}\selectfont Time (UTC)}%
\end{pgfscope}%
\begin{pgfscope}%
\pgfsetbuttcap%
\pgfsetroundjoin%
\definecolor{currentfill}{rgb}{0.000000,0.000000,0.000000}%
\pgfsetfillcolor{currentfill}%
\pgfsetlinewidth{0.803000pt}%
\definecolor{currentstroke}{rgb}{0.000000,0.000000,0.000000}%
\pgfsetstrokecolor{currentstroke}%
\pgfsetdash{}{0pt}%
\pgfsys@defobject{currentmarker}{\pgfqpoint{-0.048611in}{0.000000in}}{\pgfqpoint{-0.000000in}{0.000000in}}{%
\pgfpathmoveto{\pgfqpoint{-0.000000in}{0.000000in}}%
\pgfpathlineto{\pgfqpoint{-0.048611in}{0.000000in}}%
\pgfusepath{stroke,fill}%
}%
\begin{pgfscope}%
\pgfsys@transformshift{0.667540in}{0.611455in}%
\pgfsys@useobject{currentmarker}{}%
\end{pgfscope}%
\end{pgfscope}%
\begin{pgfscope}%
\definecolor{textcolor}{rgb}{0.000000,0.000000,0.000000}%
\pgfsetstrokecolor{textcolor}%
\pgfsetfillcolor{textcolor}%
\pgftext[x=0.327644in, y=0.572899in, left, base]{\color{textcolor}\rmfamily\fontsize{8.000000}{9.600000}\selectfont \(\displaystyle {\ensuremath{-}7.5}\)}%
\end{pgfscope}%
\begin{pgfscope}%
\pgfsetbuttcap%
\pgfsetroundjoin%
\definecolor{currentfill}{rgb}{0.000000,0.000000,0.000000}%
\pgfsetfillcolor{currentfill}%
\pgfsetlinewidth{0.803000pt}%
\definecolor{currentstroke}{rgb}{0.000000,0.000000,0.000000}%
\pgfsetstrokecolor{currentstroke}%
\pgfsetdash{}{0pt}%
\pgfsys@defobject{currentmarker}{\pgfqpoint{-0.048611in}{0.000000in}}{\pgfqpoint{-0.000000in}{0.000000in}}{%
\pgfpathmoveto{\pgfqpoint{-0.000000in}{0.000000in}}%
\pgfpathlineto{\pgfqpoint{-0.048611in}{0.000000in}}%
\pgfusepath{stroke,fill}%
}%
\begin{pgfscope}%
\pgfsys@transformshift{0.667540in}{0.925106in}%
\pgfsys@useobject{currentmarker}{}%
\end{pgfscope}%
\end{pgfscope}%
\begin{pgfscope}%
\definecolor{textcolor}{rgb}{0.000000,0.000000,0.000000}%
\pgfsetstrokecolor{textcolor}%
\pgfsetfillcolor{textcolor}%
\pgftext[x=0.327644in, y=0.886551in, left, base]{\color{textcolor}\rmfamily\fontsize{8.000000}{9.600000}\selectfont \(\displaystyle {\ensuremath{-}5.0}\)}%
\end{pgfscope}%
\begin{pgfscope}%
\pgfsetbuttcap%
\pgfsetroundjoin%
\definecolor{currentfill}{rgb}{0.000000,0.000000,0.000000}%
\pgfsetfillcolor{currentfill}%
\pgfsetlinewidth{0.803000pt}%
\definecolor{currentstroke}{rgb}{0.000000,0.000000,0.000000}%
\pgfsetstrokecolor{currentstroke}%
\pgfsetdash{}{0pt}%
\pgfsys@defobject{currentmarker}{\pgfqpoint{-0.048611in}{0.000000in}}{\pgfqpoint{-0.000000in}{0.000000in}}{%
\pgfpathmoveto{\pgfqpoint{-0.000000in}{0.000000in}}%
\pgfpathlineto{\pgfqpoint{-0.048611in}{0.000000in}}%
\pgfusepath{stroke,fill}%
}%
\begin{pgfscope}%
\pgfsys@transformshift{0.667540in}{1.238757in}%
\pgfsys@useobject{currentmarker}{}%
\end{pgfscope}%
\end{pgfscope}%
\begin{pgfscope}%
\definecolor{textcolor}{rgb}{0.000000,0.000000,0.000000}%
\pgfsetstrokecolor{textcolor}%
\pgfsetfillcolor{textcolor}%
\pgftext[x=0.327644in, y=1.200202in, left, base]{\color{textcolor}\rmfamily\fontsize{8.000000}{9.600000}\selectfont \(\displaystyle {\ensuremath{-}2.5}\)}%
\end{pgfscope}%
\begin{pgfscope}%
\pgfsetbuttcap%
\pgfsetroundjoin%
\definecolor{currentfill}{rgb}{0.000000,0.000000,0.000000}%
\pgfsetfillcolor{currentfill}%
\pgfsetlinewidth{0.803000pt}%
\definecolor{currentstroke}{rgb}{0.000000,0.000000,0.000000}%
\pgfsetstrokecolor{currentstroke}%
\pgfsetdash{}{0pt}%
\pgfsys@defobject{currentmarker}{\pgfqpoint{-0.048611in}{0.000000in}}{\pgfqpoint{-0.000000in}{0.000000in}}{%
\pgfpathmoveto{\pgfqpoint{-0.000000in}{0.000000in}}%
\pgfpathlineto{\pgfqpoint{-0.048611in}{0.000000in}}%
\pgfusepath{stroke,fill}%
}%
\begin{pgfscope}%
\pgfsys@transformshift{0.667540in}{1.552408in}%
\pgfsys@useobject{currentmarker}{}%
\end{pgfscope}%
\end{pgfscope}%
\begin{pgfscope}%
\definecolor{textcolor}{rgb}{0.000000,0.000000,0.000000}%
\pgfsetstrokecolor{textcolor}%
\pgfsetfillcolor{textcolor}%
\pgftext[x=0.419467in, y=1.513853in, left, base]{\color{textcolor}\rmfamily\fontsize{8.000000}{9.600000}\selectfont \(\displaystyle {0.0}\)}%
\end{pgfscope}%
\begin{pgfscope}%
\pgfsetbuttcap%
\pgfsetroundjoin%
\definecolor{currentfill}{rgb}{0.000000,0.000000,0.000000}%
\pgfsetfillcolor{currentfill}%
\pgfsetlinewidth{0.803000pt}%
\definecolor{currentstroke}{rgb}{0.000000,0.000000,0.000000}%
\pgfsetstrokecolor{currentstroke}%
\pgfsetdash{}{0pt}%
\pgfsys@defobject{currentmarker}{\pgfqpoint{-0.048611in}{0.000000in}}{\pgfqpoint{-0.000000in}{0.000000in}}{%
\pgfpathmoveto{\pgfqpoint{-0.000000in}{0.000000in}}%
\pgfpathlineto{\pgfqpoint{-0.048611in}{0.000000in}}%
\pgfusepath{stroke,fill}%
}%
\begin{pgfscope}%
\pgfsys@transformshift{0.667540in}{1.866059in}%
\pgfsys@useobject{currentmarker}{}%
\end{pgfscope}%
\end{pgfscope}%
\begin{pgfscope}%
\definecolor{textcolor}{rgb}{0.000000,0.000000,0.000000}%
\pgfsetstrokecolor{textcolor}%
\pgfsetfillcolor{textcolor}%
\pgftext[x=0.419467in, y=1.827504in, left, base]{\color{textcolor}\rmfamily\fontsize{8.000000}{9.600000}\selectfont \(\displaystyle {2.5}\)}%
\end{pgfscope}%
\begin{pgfscope}%
\pgfsetbuttcap%
\pgfsetroundjoin%
\definecolor{currentfill}{rgb}{0.000000,0.000000,0.000000}%
\pgfsetfillcolor{currentfill}%
\pgfsetlinewidth{0.803000pt}%
\definecolor{currentstroke}{rgb}{0.000000,0.000000,0.000000}%
\pgfsetstrokecolor{currentstroke}%
\pgfsetdash{}{0pt}%
\pgfsys@defobject{currentmarker}{\pgfqpoint{-0.048611in}{0.000000in}}{\pgfqpoint{-0.000000in}{0.000000in}}{%
\pgfpathmoveto{\pgfqpoint{-0.000000in}{0.000000in}}%
\pgfpathlineto{\pgfqpoint{-0.048611in}{0.000000in}}%
\pgfusepath{stroke,fill}%
}%
\begin{pgfscope}%
\pgfsys@transformshift{0.667540in}{2.179710in}%
\pgfsys@useobject{currentmarker}{}%
\end{pgfscope}%
\end{pgfscope}%
\begin{pgfscope}%
\definecolor{textcolor}{rgb}{0.000000,0.000000,0.000000}%
\pgfsetstrokecolor{textcolor}%
\pgfsetfillcolor{textcolor}%
\pgftext[x=0.419467in, y=2.141155in, left, base]{\color{textcolor}\rmfamily\fontsize{8.000000}{9.600000}\selectfont \(\displaystyle {5.0}\)}%
\end{pgfscope}%
\begin{pgfscope}%
\pgfsetbuttcap%
\pgfsetroundjoin%
\definecolor{currentfill}{rgb}{0.000000,0.000000,0.000000}%
\pgfsetfillcolor{currentfill}%
\pgfsetlinewidth{0.803000pt}%
\definecolor{currentstroke}{rgb}{0.000000,0.000000,0.000000}%
\pgfsetstrokecolor{currentstroke}%
\pgfsetdash{}{0pt}%
\pgfsys@defobject{currentmarker}{\pgfqpoint{-0.048611in}{0.000000in}}{\pgfqpoint{-0.000000in}{0.000000in}}{%
\pgfpathmoveto{\pgfqpoint{-0.000000in}{0.000000in}}%
\pgfpathlineto{\pgfqpoint{-0.048611in}{0.000000in}}%
\pgfusepath{stroke,fill}%
}%
\begin{pgfscope}%
\pgfsys@transformshift{0.667540in}{2.493362in}%
\pgfsys@useobject{currentmarker}{}%
\end{pgfscope}%
\end{pgfscope}%
\begin{pgfscope}%
\definecolor{textcolor}{rgb}{0.000000,0.000000,0.000000}%
\pgfsetstrokecolor{textcolor}%
\pgfsetfillcolor{textcolor}%
\pgftext[x=0.419467in, y=2.454806in, left, base]{\color{textcolor}\rmfamily\fontsize{8.000000}{9.600000}\selectfont \(\displaystyle {7.5}\)}%
\end{pgfscope}%
\begin{pgfscope}%
\pgfsetbuttcap%
\pgfsetroundjoin%
\definecolor{currentfill}{rgb}{0.000000,0.000000,0.000000}%
\pgfsetfillcolor{currentfill}%
\pgfsetlinewidth{0.803000pt}%
\definecolor{currentstroke}{rgb}{0.000000,0.000000,0.000000}%
\pgfsetstrokecolor{currentstroke}%
\pgfsetdash{}{0pt}%
\pgfsys@defobject{currentmarker}{\pgfqpoint{-0.048611in}{0.000000in}}{\pgfqpoint{-0.000000in}{0.000000in}}{%
\pgfpathmoveto{\pgfqpoint{-0.000000in}{0.000000in}}%
\pgfpathlineto{\pgfqpoint{-0.048611in}{0.000000in}}%
\pgfusepath{stroke,fill}%
}%
\begin{pgfscope}%
\pgfsys@transformshift{0.667540in}{2.807013in}%
\pgfsys@useobject{currentmarker}{}%
\end{pgfscope}%
\end{pgfscope}%
\begin{pgfscope}%
\definecolor{textcolor}{rgb}{0.000000,0.000000,0.000000}%
\pgfsetstrokecolor{textcolor}%
\pgfsetfillcolor{textcolor}%
\pgftext[x=0.360438in, y=2.768457in, left, base]{\color{textcolor}\rmfamily\fontsize{8.000000}{9.600000}\selectfont \(\displaystyle {10.0}\)}%
\end{pgfscope}%
\begin{pgfscope}%
\definecolor{textcolor}{rgb}{0.000000,0.000000,0.000000}%
\pgfsetstrokecolor{textcolor}%
\pgfsetfillcolor{textcolor}%
\pgftext[x=0.272089in,y=1.742216in,,bottom,rotate=90.000000]{\color{textcolor}\rmfamily\fontsize{10.000000}{12.000000}\selectfont Voltage deviation in V}%
\end{pgfscope}%
\begin{pgfscope}%
\definecolor{textcolor}{rgb}{0.000000,0.000000,0.000000}%
\pgfsetstrokecolor{textcolor}%
\pgfsetfillcolor{textcolor}%
\pgftext[x=0.667540in,y=2.986554in,left,base]{\color{textcolor}\rmfamily\fontsize{8.000000}{9.600000}\selectfont \(\displaystyle \times{10^{\ensuremath{-}6}}{}\)}%
\end{pgfscope}%
\begin{pgfscope}%
\pgfpathrectangle{\pgfqpoint{0.667540in}{0.539544in}}{\pgfqpoint{4.391122in}{2.405343in}}%
\pgfusepath{clip}%
\pgfsetrectcap%
\pgfsetroundjoin%
\pgfsetlinewidth{0.501875pt}%
\definecolor{currentstroke}{rgb}{0.121569,0.466667,0.705882}%
\pgfsetstrokecolor{currentstroke}%
\pgfsetstrokeopacity{0.700000}%
\pgfsetdash{}{0pt}%
\pgfpathmoveto{\pgfqpoint{0.867136in}{1.087387in}}%
\pgfpathlineto{\pgfqpoint{0.870808in}{1.118966in}}%
\pgfpathlineto{\pgfqpoint{0.872643in}{1.181144in}}%
\pgfpathlineto{\pgfqpoint{0.874481in}{0.958565in}}%
\pgfpathlineto{\pgfqpoint{0.876316in}{1.101426in}}%
\pgfpathlineto{\pgfqpoint{0.878153in}{0.951576in}}%
\pgfpathlineto{\pgfqpoint{0.881823in}{1.075920in}}%
\pgfpathlineto{\pgfqpoint{0.883659in}{1.252581in}}%
\pgfpathlineto{\pgfqpoint{0.885494in}{1.091590in}}%
\pgfpathlineto{\pgfqpoint{0.887330in}{1.123470in}}%
\pgfpathlineto{\pgfqpoint{0.889165in}{1.181997in}}%
\pgfpathlineto{\pgfqpoint{0.892838in}{1.123909in}}%
\pgfpathlineto{\pgfqpoint{0.894674in}{1.172450in}}%
\pgfpathlineto{\pgfqpoint{0.896510in}{1.289040in}}%
\pgfpathlineto{\pgfqpoint{0.898346in}{1.152940in}}%
\pgfpathlineto{\pgfqpoint{0.902018in}{1.215583in}}%
\pgfpathlineto{\pgfqpoint{0.905690in}{1.384741in}}%
\pgfpathlineto{\pgfqpoint{0.909363in}{1.223286in}}%
\pgfpathlineto{\pgfqpoint{0.911199in}{1.421614in}}%
\pgfpathlineto{\pgfqpoint{0.913035in}{1.319953in}}%
\pgfpathlineto{\pgfqpoint{0.914871in}{1.291248in}}%
\pgfpathlineto{\pgfqpoint{0.918543in}{1.115240in}}%
\pgfpathlineto{\pgfqpoint{0.920378in}{1.088742in}}%
\pgfpathlineto{\pgfqpoint{0.922213in}{1.270585in}}%
\pgfpathlineto{\pgfqpoint{0.924048in}{1.272203in}}%
\pgfpathlineto{\pgfqpoint{0.925883in}{1.289065in}}%
\pgfpathlineto{\pgfqpoint{0.929556in}{1.198307in}}%
\pgfpathlineto{\pgfqpoint{0.931392in}{1.190227in}}%
\pgfpathlineto{\pgfqpoint{0.933228in}{1.146442in}}%
\pgfpathlineto{\pgfqpoint{0.935065in}{1.283369in}}%
\pgfpathlineto{\pgfqpoint{0.938735in}{1.029588in}}%
\pgfpathlineto{\pgfqpoint{0.946079in}{1.626290in}}%
\pgfpathlineto{\pgfqpoint{0.949753in}{1.444096in}}%
\pgfpathlineto{\pgfqpoint{0.951589in}{1.616956in}}%
\pgfpathlineto{\pgfqpoint{0.953425in}{1.532446in}}%
\pgfpathlineto{\pgfqpoint{0.955260in}{1.652950in}}%
\pgfpathlineto{\pgfqpoint{0.957096in}{1.510703in}}%
\pgfpathlineto{\pgfqpoint{0.958931in}{1.617922in}}%
\pgfpathlineto{\pgfqpoint{0.960766in}{1.610194in}}%
\pgfpathlineto{\pgfqpoint{0.962600in}{1.313542in}}%
\pgfpathlineto{\pgfqpoint{0.964437in}{1.329589in}}%
\pgfpathlineto{\pgfqpoint{0.966272in}{1.506526in}}%
\pgfpathlineto{\pgfqpoint{0.968108in}{1.501068in}}%
\pgfpathlineto{\pgfqpoint{0.969944in}{1.597183in}}%
\pgfpathlineto{\pgfqpoint{0.971780in}{1.593921in}}%
\pgfpathlineto{\pgfqpoint{0.973616in}{1.657530in}}%
\pgfpathlineto{\pgfqpoint{0.975451in}{1.470130in}}%
\pgfpathlineto{\pgfqpoint{0.979123in}{1.384151in}}%
\pgfpathlineto{\pgfqpoint{0.980959in}{1.395556in}}%
\pgfpathlineto{\pgfqpoint{0.982794in}{1.581990in}}%
\pgfpathlineto{\pgfqpoint{0.984630in}{1.471886in}}%
\pgfpathlineto{\pgfqpoint{0.986466in}{1.237476in}}%
\pgfpathlineto{\pgfqpoint{0.990138in}{1.537000in}}%
\pgfpathlineto{\pgfqpoint{0.991973in}{1.372346in}}%
\pgfpathlineto{\pgfqpoint{0.995643in}{1.405605in}}%
\pgfpathlineto{\pgfqpoint{0.997478in}{1.710361in}}%
\pgfpathlineto{\pgfqpoint{0.999315in}{1.353740in}}%
\pgfpathlineto{\pgfqpoint{1.001150in}{1.559683in}}%
\pgfpathlineto{\pgfqpoint{1.002986in}{1.584449in}}%
\pgfpathlineto{\pgfqpoint{1.004822in}{1.532120in}}%
\pgfpathlineto{\pgfqpoint{1.006658in}{1.099607in}}%
\pgfpathlineto{\pgfqpoint{1.008497in}{1.116206in}}%
\pgfpathlineto{\pgfqpoint{1.010333in}{1.363701in}}%
\pgfpathlineto{\pgfqpoint{1.012168in}{1.297822in}}%
\pgfpathlineto{\pgfqpoint{1.014005in}{1.069810in}}%
\pgfpathlineto{\pgfqpoint{1.017676in}{1.366574in}}%
\pgfpathlineto{\pgfqpoint{1.019512in}{1.219133in}}%
\pgfpathlineto{\pgfqpoint{1.021348in}{1.249871in}}%
\pgfpathlineto{\pgfqpoint{1.023183in}{1.354882in}}%
\pgfpathlineto{\pgfqpoint{1.025021in}{1.274637in}}%
\pgfpathlineto{\pgfqpoint{1.026856in}{1.591236in}}%
\pgfpathlineto{\pgfqpoint{1.028692in}{1.362409in}}%
\pgfpathlineto{\pgfqpoint{1.030527in}{1.339613in}}%
\pgfpathlineto{\pgfqpoint{1.034199in}{1.115679in}}%
\pgfpathlineto{\pgfqpoint{1.036034in}{1.211944in}}%
\pgfpathlineto{\pgfqpoint{1.037869in}{1.103158in}}%
\pgfpathlineto{\pgfqpoint{1.039706in}{1.258302in}}%
\pgfpathlineto{\pgfqpoint{1.041541in}{1.234803in}}%
\pgfpathlineto{\pgfqpoint{1.043377in}{1.057089in}}%
\pgfpathlineto{\pgfqpoint{1.048886in}{2.234585in}}%
\pgfpathlineto{\pgfqpoint{1.050723in}{2.532253in}}%
\pgfpathlineto{\pgfqpoint{1.052559in}{2.137178in}}%
\pgfpathlineto{\pgfqpoint{1.054395in}{1.468800in}}%
\pgfpathlineto{\pgfqpoint{1.056245in}{1.494895in}}%
\pgfpathlineto{\pgfqpoint{1.058082in}{1.229421in}}%
\pgfpathlineto{\pgfqpoint{1.061752in}{1.352648in}}%
\pgfpathlineto{\pgfqpoint{1.063588in}{1.260711in}}%
\pgfpathlineto{\pgfqpoint{1.065424in}{1.255956in}}%
\pgfpathlineto{\pgfqpoint{1.067259in}{1.191256in}}%
\pgfpathlineto{\pgfqpoint{1.069094in}{1.201155in}}%
\pgfpathlineto{\pgfqpoint{1.070929in}{1.157846in}}%
\pgfpathlineto{\pgfqpoint{1.072765in}{1.334406in}}%
\pgfpathlineto{\pgfqpoint{1.074601in}{1.051305in}}%
\pgfpathlineto{\pgfqpoint{1.076437in}{1.210828in}}%
\pgfpathlineto{\pgfqpoint{1.078273in}{1.206675in}}%
\pgfpathlineto{\pgfqpoint{1.080109in}{1.457370in}}%
\pgfpathlineto{\pgfqpoint{1.081946in}{2.094308in}}%
\pgfpathlineto{\pgfqpoint{1.083781in}{2.144203in}}%
\pgfpathlineto{\pgfqpoint{1.085616in}{2.054763in}}%
\pgfpathlineto{\pgfqpoint{1.087454in}{2.151279in}}%
\pgfpathlineto{\pgfqpoint{1.089290in}{2.347499in}}%
\pgfpathlineto{\pgfqpoint{1.091125in}{2.151543in}}%
\pgfpathlineto{\pgfqpoint{1.092961in}{2.094546in}}%
\pgfpathlineto{\pgfqpoint{1.094797in}{2.186985in}}%
\pgfpathlineto{\pgfqpoint{1.096632in}{2.073795in}}%
\pgfpathlineto{\pgfqpoint{1.098468in}{2.131156in}}%
\pgfpathlineto{\pgfqpoint{1.100303in}{2.013587in}}%
\pgfpathlineto{\pgfqpoint{1.102138in}{2.176083in}}%
\pgfpathlineto{\pgfqpoint{1.105808in}{2.148871in}}%
\pgfpathlineto{\pgfqpoint{1.107646in}{2.056369in}}%
\pgfpathlineto{\pgfqpoint{1.111317in}{2.206846in}}%
\pgfpathlineto{\pgfqpoint{1.113154in}{2.051614in}}%
\pgfpathlineto{\pgfqpoint{1.114989in}{2.200761in}}%
\pgfpathlineto{\pgfqpoint{1.116824in}{2.146725in}}%
\pgfpathlineto{\pgfqpoint{1.118661in}{2.019144in}}%
\pgfpathlineto{\pgfqpoint{1.120497in}{2.141468in}}%
\pgfpathlineto{\pgfqpoint{1.124169in}{1.269732in}}%
\pgfpathlineto{\pgfqpoint{1.126004in}{1.341721in}}%
\pgfpathlineto{\pgfqpoint{1.127845in}{1.286606in}}%
\pgfpathlineto{\pgfqpoint{1.129681in}{1.146241in}}%
\pgfpathlineto{\pgfqpoint{1.133355in}{1.080562in}}%
\pgfpathlineto{\pgfqpoint{1.135191in}{1.274085in}}%
\pgfpathlineto{\pgfqpoint{1.137027in}{1.059849in}}%
\pgfpathlineto{\pgfqpoint{1.138862in}{1.087538in}}%
\pgfpathlineto{\pgfqpoint{1.140696in}{1.174469in}}%
\pgfpathlineto{\pgfqpoint{1.142533in}{1.206085in}}%
\pgfpathlineto{\pgfqpoint{1.144368in}{1.114060in}}%
\pgfpathlineto{\pgfqpoint{1.146205in}{1.310669in}}%
\pgfpathlineto{\pgfqpoint{1.148041in}{1.983175in}}%
\pgfpathlineto{\pgfqpoint{1.149878in}{2.057134in}}%
\pgfpathlineto{\pgfqpoint{1.151713in}{1.299027in}}%
\pgfpathlineto{\pgfqpoint{1.153550in}{2.024163in}}%
\pgfpathlineto{\pgfqpoint{1.155386in}{1.916505in}}%
\pgfpathlineto{\pgfqpoint{1.157223in}{1.936228in}}%
\pgfpathlineto{\pgfqpoint{1.159058in}{2.108121in}}%
\pgfpathlineto{\pgfqpoint{1.160894in}{2.131682in}}%
\pgfpathlineto{\pgfqpoint{1.162731in}{2.048352in}}%
\pgfpathlineto{\pgfqpoint{1.166405in}{2.197474in}}%
\pgfpathlineto{\pgfqpoint{1.168241in}{2.119337in}}%
\pgfpathlineto{\pgfqpoint{1.170076in}{2.177965in}}%
\pgfpathlineto{\pgfqpoint{1.171914in}{2.155921in}}%
\pgfpathlineto{\pgfqpoint{1.173748in}{2.047235in}}%
\pgfpathlineto{\pgfqpoint{1.175582in}{2.072151in}}%
\pgfpathlineto{\pgfqpoint{1.177418in}{2.185630in}}%
\pgfpathlineto{\pgfqpoint{1.181087in}{1.964230in}}%
\pgfpathlineto{\pgfqpoint{1.186596in}{2.183071in}}%
\pgfpathlineto{\pgfqpoint{1.190268in}{1.984066in}}%
\pgfpathlineto{\pgfqpoint{1.192105in}{1.972047in}}%
\pgfpathlineto{\pgfqpoint{1.193940in}{1.421413in}}%
\pgfpathlineto{\pgfqpoint{1.197612in}{1.164069in}}%
\pgfpathlineto{\pgfqpoint{1.199448in}{1.245317in}}%
\pgfpathlineto{\pgfqpoint{1.201285in}{1.047541in}}%
\pgfpathlineto{\pgfqpoint{1.203121in}{1.305262in}}%
\pgfpathlineto{\pgfqpoint{1.204957in}{1.405982in}}%
\pgfpathlineto{\pgfqpoint{1.208632in}{1.280609in}}%
\pgfpathlineto{\pgfqpoint{1.210467in}{1.242118in}}%
\pgfpathlineto{\pgfqpoint{1.212302in}{1.359611in}}%
\pgfpathlineto{\pgfqpoint{1.214137in}{1.275666in}}%
\pgfpathlineto{\pgfqpoint{1.215972in}{1.375043in}}%
\pgfpathlineto{\pgfqpoint{1.217807in}{1.207415in}}%
\pgfpathlineto{\pgfqpoint{1.221479in}{1.316428in}}%
\pgfpathlineto{\pgfqpoint{1.223316in}{1.354831in}}%
\pgfpathlineto{\pgfqpoint{1.225150in}{1.049398in}}%
\pgfpathlineto{\pgfqpoint{1.226988in}{1.149678in}}%
\pgfpathlineto{\pgfqpoint{1.228824in}{1.126845in}}%
\pgfpathlineto{\pgfqpoint{1.230659in}{1.174971in}}%
\pgfpathlineto{\pgfqpoint{1.232495in}{1.334820in}}%
\pgfpathlineto{\pgfqpoint{1.234331in}{1.203852in}}%
\pgfpathlineto{\pgfqpoint{1.236168in}{1.285715in}}%
\pgfpathlineto{\pgfqpoint{1.238004in}{1.260862in}}%
\pgfpathlineto{\pgfqpoint{1.239840in}{1.427523in}}%
\pgfpathlineto{\pgfqpoint{1.241677in}{1.373400in}}%
\pgfpathlineto{\pgfqpoint{1.243513in}{1.231328in}}%
\pgfpathlineto{\pgfqpoint{1.245348in}{1.389910in}}%
\pgfpathlineto{\pgfqpoint{1.249019in}{1.220689in}}%
\pgfpathlineto{\pgfqpoint{1.250854in}{1.366261in}}%
\pgfpathlineto{\pgfqpoint{1.252690in}{1.261564in}}%
\pgfpathlineto{\pgfqpoint{1.256359in}{1.400549in}}%
\pgfpathlineto{\pgfqpoint{1.258196in}{1.241904in}}%
\pgfpathlineto{\pgfqpoint{1.260032in}{1.239847in}}%
\pgfpathlineto{\pgfqpoint{1.261867in}{1.542972in}}%
\pgfpathlineto{\pgfqpoint{1.263704in}{2.158004in}}%
\pgfpathlineto{\pgfqpoint{1.265540in}{2.350322in}}%
\pgfpathlineto{\pgfqpoint{1.269213in}{2.139147in}}%
\pgfpathlineto{\pgfqpoint{1.271049in}{2.259288in}}%
\pgfpathlineto{\pgfqpoint{1.272886in}{2.184689in}}%
\pgfpathlineto{\pgfqpoint{1.274721in}{2.231022in}}%
\pgfpathlineto{\pgfqpoint{1.276557in}{2.170939in}}%
\pgfpathlineto{\pgfqpoint{1.278395in}{2.312145in}}%
\pgfpathlineto{\pgfqpoint{1.280229in}{2.275749in}}%
\pgfpathlineto{\pgfqpoint{1.283924in}{2.348616in}}%
\pgfpathlineto{\pgfqpoint{1.285760in}{2.339119in}}%
\pgfpathlineto{\pgfqpoint{1.287594in}{2.442611in}}%
\pgfpathlineto{\pgfqpoint{1.289430in}{2.232402in}}%
\pgfpathlineto{\pgfqpoint{1.291265in}{2.217159in}}%
\pgfpathlineto{\pgfqpoint{1.294938in}{2.254496in}}%
\pgfpathlineto{\pgfqpoint{1.296774in}{2.273478in}}%
\pgfpathlineto{\pgfqpoint{1.298608in}{2.227710in}}%
\pgfpathlineto{\pgfqpoint{1.300446in}{2.217096in}}%
\pgfpathlineto{\pgfqpoint{1.302282in}{2.156800in}}%
\pgfpathlineto{\pgfqpoint{1.304117in}{1.349863in}}%
\pgfpathlineto{\pgfqpoint{1.305953in}{1.111012in}}%
\pgfpathlineto{\pgfqpoint{1.307790in}{1.240211in}}%
\pgfpathlineto{\pgfqpoint{1.309627in}{1.209059in}}%
\pgfpathlineto{\pgfqpoint{1.311463in}{1.149503in}}%
\pgfpathlineto{\pgfqpoint{1.315135in}{1.233423in}}%
\pgfpathlineto{\pgfqpoint{1.316970in}{1.146203in}}%
\pgfpathlineto{\pgfqpoint{1.318807in}{1.237614in}}%
\pgfpathlineto{\pgfqpoint{1.320643in}{1.170279in}}%
\pgfpathlineto{\pgfqpoint{1.322477in}{1.275264in}}%
\pgfpathlineto{\pgfqpoint{1.326148in}{1.116733in}}%
\pgfpathlineto{\pgfqpoint{1.327983in}{1.249369in}}%
\pgfpathlineto{\pgfqpoint{1.329819in}{1.117410in}}%
\pgfpathlineto{\pgfqpoint{1.331656in}{1.345899in}}%
\pgfpathlineto{\pgfqpoint{1.333492in}{1.277786in}}%
\pgfpathlineto{\pgfqpoint{1.335329in}{1.164044in}}%
\pgfpathlineto{\pgfqpoint{1.337164in}{1.195396in}}%
\pgfpathlineto{\pgfqpoint{1.339000in}{1.120935in}}%
\pgfpathlineto{\pgfqpoint{1.342670in}{1.215344in}}%
\pgfpathlineto{\pgfqpoint{1.344507in}{1.183641in}}%
\pgfpathlineto{\pgfqpoint{1.347225in}{1.035560in}}%
\pgfpathlineto{\pgfqpoint{1.349060in}{1.134811in}}%
\pgfpathlineto{\pgfqpoint{1.350897in}{1.093146in}}%
\pgfpathlineto{\pgfqpoint{1.352733in}{1.092556in}}%
\pgfpathlineto{\pgfqpoint{1.354568in}{1.137546in}}%
\pgfpathlineto{\pgfqpoint{1.358240in}{0.961801in}}%
\pgfpathlineto{\pgfqpoint{1.360072in}{1.107900in}}%
\pgfpathlineto{\pgfqpoint{1.361907in}{1.141072in}}%
\pgfpathlineto{\pgfqpoint{1.363739in}{1.103434in}}%
\pgfpathlineto{\pgfqpoint{1.365572in}{1.188345in}}%
\pgfpathlineto{\pgfqpoint{1.367406in}{1.130960in}}%
\pgfpathlineto{\pgfqpoint{1.369240in}{1.462226in}}%
\pgfpathlineto{\pgfqpoint{1.372906in}{1.214930in}}%
\pgfpathlineto{\pgfqpoint{1.374739in}{1.296116in}}%
\pgfpathlineto{\pgfqpoint{1.376572in}{1.258653in}}%
\pgfpathlineto{\pgfqpoint{1.378404in}{1.542846in}}%
\pgfpathlineto{\pgfqpoint{1.383907in}{1.135514in}}%
\pgfpathlineto{\pgfqpoint{1.385743in}{0.977346in}}%
\pgfpathlineto{\pgfqpoint{1.387576in}{1.130307in}}%
\pgfpathlineto{\pgfqpoint{1.389409in}{1.049160in}}%
\pgfpathlineto{\pgfqpoint{1.391244in}{1.216725in}}%
\pgfpathlineto{\pgfqpoint{1.393077in}{1.149152in}}%
\pgfpathlineto{\pgfqpoint{1.396744in}{1.240123in}}%
\pgfpathlineto{\pgfqpoint{1.398578in}{1.367377in}}%
\pgfpathlineto{\pgfqpoint{1.402245in}{1.101690in}}%
\pgfpathlineto{\pgfqpoint{1.404078in}{1.141423in}}%
\pgfpathlineto{\pgfqpoint{1.405913in}{1.019538in}}%
\pgfpathlineto{\pgfqpoint{1.407746in}{1.120622in}}%
\pgfpathlineto{\pgfqpoint{1.409581in}{1.145764in}}%
\pgfpathlineto{\pgfqpoint{1.413248in}{0.967886in}}%
\pgfpathlineto{\pgfqpoint{1.415082in}{1.012161in}}%
\pgfpathlineto{\pgfqpoint{1.416914in}{0.978927in}}%
\pgfpathlineto{\pgfqpoint{1.420582in}{1.154847in}}%
\pgfpathlineto{\pgfqpoint{1.424249in}{1.255780in}}%
\pgfpathlineto{\pgfqpoint{1.426084in}{1.173892in}}%
\pgfpathlineto{\pgfqpoint{1.427917in}{1.306793in}}%
\pgfpathlineto{\pgfqpoint{1.429749in}{1.031294in}}%
\pgfpathlineto{\pgfqpoint{1.431585in}{1.137459in}}%
\pgfpathlineto{\pgfqpoint{1.433419in}{1.176301in}}%
\pgfpathlineto{\pgfqpoint{1.437087in}{1.023239in}}%
\pgfpathlineto{\pgfqpoint{1.438920in}{1.168046in}}%
\pgfpathlineto{\pgfqpoint{1.440752in}{1.202008in}}%
\pgfpathlineto{\pgfqpoint{1.442587in}{1.384064in}}%
\pgfpathlineto{\pgfqpoint{1.448088in}{1.099670in}}%
\pgfpathlineto{\pgfqpoint{1.449922in}{1.246283in}}%
\pgfpathlineto{\pgfqpoint{1.453590in}{1.226661in}}%
\pgfpathlineto{\pgfqpoint{1.455424in}{1.263183in}}%
\pgfpathlineto{\pgfqpoint{1.457277in}{1.175297in}}%
\pgfpathlineto{\pgfqpoint{1.459481in}{1.229948in}}%
\pgfpathlineto{\pgfqpoint{1.461314in}{1.053388in}}%
\pgfpathlineto{\pgfqpoint{1.463148in}{1.119819in}}%
\pgfpathlineto{\pgfqpoint{1.464981in}{0.999415in}}%
\pgfpathlineto{\pgfqpoint{1.466813in}{1.487932in}}%
\pgfpathlineto{\pgfqpoint{1.472313in}{1.085242in}}%
\pgfpathlineto{\pgfqpoint{1.474146in}{1.262216in}}%
\pgfpathlineto{\pgfqpoint{1.475980in}{1.228681in}}%
\pgfpathlineto{\pgfqpoint{1.477812in}{1.015599in}}%
\pgfpathlineto{\pgfqpoint{1.479648in}{2.006573in}}%
\pgfpathlineto{\pgfqpoint{1.481481in}{1.754410in}}%
\pgfpathlineto{\pgfqpoint{1.483314in}{0.936647in}}%
\pgfpathlineto{\pgfqpoint{1.486984in}{1.164922in}}%
\pgfpathlineto{\pgfqpoint{1.488817in}{1.159377in}}%
\pgfpathlineto{\pgfqpoint{1.490650in}{1.092908in}}%
\pgfpathlineto{\pgfqpoint{1.492484in}{1.070902in}}%
\pgfpathlineto{\pgfqpoint{1.494318in}{0.980257in}}%
\pgfpathlineto{\pgfqpoint{1.496151in}{1.178233in}}%
\pgfpathlineto{\pgfqpoint{1.497985in}{1.212873in}}%
\pgfpathlineto{\pgfqpoint{1.499818in}{1.077953in}}%
\pgfpathlineto{\pgfqpoint{1.503486in}{1.267034in}}%
\pgfpathlineto{\pgfqpoint{1.505321in}{1.082269in}}%
\pgfpathlineto{\pgfqpoint{1.507154in}{1.092494in}}%
\pgfpathlineto{\pgfqpoint{1.508989in}{1.132515in}}%
\pgfpathlineto{\pgfqpoint{1.510822in}{0.911943in}}%
\pgfpathlineto{\pgfqpoint{1.512654in}{1.099319in}}%
\pgfpathlineto{\pgfqpoint{1.514489in}{1.161133in}}%
\pgfpathlineto{\pgfqpoint{1.516322in}{1.102781in}}%
\pgfpathlineto{\pgfqpoint{1.518156in}{0.998825in}}%
\pgfpathlineto{\pgfqpoint{1.523657in}{1.369636in}}%
\pgfpathlineto{\pgfqpoint{1.525491in}{1.289366in}}%
\pgfpathlineto{\pgfqpoint{1.527325in}{1.364379in}}%
\pgfpathlineto{\pgfqpoint{1.529158in}{1.253961in}}%
\pgfpathlineto{\pgfqpoint{1.531015in}{1.482964in}}%
\pgfpathlineto{\pgfqpoint{1.532848in}{1.217289in}}%
\pgfpathlineto{\pgfqpoint{1.534681in}{1.226310in}}%
\pgfpathlineto{\pgfqpoint{1.536516in}{1.525370in}}%
\pgfpathlineto{\pgfqpoint{1.540182in}{1.090762in}}%
\pgfpathlineto{\pgfqpoint{1.542016in}{1.211468in}}%
\pgfpathlineto{\pgfqpoint{1.543849in}{1.068230in}}%
\pgfpathlineto{\pgfqpoint{1.547517in}{1.425528in}}%
\pgfpathlineto{\pgfqpoint{1.549349in}{1.303756in}}%
\pgfpathlineto{\pgfqpoint{1.551181in}{1.381981in}}%
\pgfpathlineto{\pgfqpoint{1.553015in}{1.334469in}}%
\pgfpathlineto{\pgfqpoint{1.554849in}{1.321509in}}%
\pgfpathlineto{\pgfqpoint{1.556681in}{1.274988in}}%
\pgfpathlineto{\pgfqpoint{1.558518in}{1.349713in}}%
\pgfpathlineto{\pgfqpoint{1.560351in}{1.174796in}}%
\pgfpathlineto{\pgfqpoint{1.562183in}{1.204392in}}%
\pgfpathlineto{\pgfqpoint{1.564017in}{1.295238in}}%
\pgfpathlineto{\pgfqpoint{1.565850in}{1.920482in}}%
\pgfpathlineto{\pgfqpoint{1.567684in}{2.088135in}}%
\pgfpathlineto{\pgfqpoint{1.569520in}{2.135358in}}%
\pgfpathlineto{\pgfqpoint{1.571352in}{2.117430in}}%
\pgfpathlineto{\pgfqpoint{1.575017in}{1.991731in}}%
\pgfpathlineto{\pgfqpoint{1.578712in}{2.385790in}}%
\pgfpathlineto{\pgfqpoint{1.580545in}{2.154278in}}%
\pgfpathlineto{\pgfqpoint{1.582379in}{2.229529in}}%
\pgfpathlineto{\pgfqpoint{1.584213in}{2.219003in}}%
\pgfpathlineto{\pgfqpoint{1.586048in}{2.173461in}}%
\pgfpathlineto{\pgfqpoint{1.589716in}{1.343314in}}%
\pgfpathlineto{\pgfqpoint{1.593382in}{1.132867in}}%
\pgfpathlineto{\pgfqpoint{1.595217in}{1.185999in}}%
\pgfpathlineto{\pgfqpoint{1.597050in}{1.140834in}}%
\pgfpathlineto{\pgfqpoint{1.598882in}{1.236234in}}%
\pgfpathlineto{\pgfqpoint{1.600716in}{2.220007in}}%
\pgfpathlineto{\pgfqpoint{1.602565in}{2.084020in}}%
\pgfpathlineto{\pgfqpoint{1.604399in}{2.099125in}}%
\pgfpathlineto{\pgfqpoint{1.606234in}{2.076555in}}%
\pgfpathlineto{\pgfqpoint{1.608066in}{2.084283in}}%
\pgfpathlineto{\pgfqpoint{1.609901in}{2.215302in}}%
\pgfpathlineto{\pgfqpoint{1.611736in}{2.072214in}}%
\pgfpathlineto{\pgfqpoint{1.613569in}{2.225351in}}%
\pgfpathlineto{\pgfqpoint{1.615402in}{2.118221in}}%
\pgfpathlineto{\pgfqpoint{1.617235in}{2.101007in}}%
\pgfpathlineto{\pgfqpoint{1.619070in}{2.177752in}}%
\pgfpathlineto{\pgfqpoint{1.620903in}{2.181954in}}%
\pgfpathlineto{\pgfqpoint{1.626403in}{1.094463in}}%
\pgfpathlineto{\pgfqpoint{1.628237in}{1.137283in}}%
\pgfpathlineto{\pgfqpoint{1.630070in}{1.216135in}}%
\pgfpathlineto{\pgfqpoint{1.631903in}{1.150996in}}%
\pgfpathlineto{\pgfqpoint{1.633737in}{1.274022in}}%
\pgfpathlineto{\pgfqpoint{1.635570in}{1.224716in}}%
\pgfpathlineto{\pgfqpoint{1.637403in}{1.308286in}}%
\pgfpathlineto{\pgfqpoint{1.639237in}{1.318925in}}%
\pgfpathlineto{\pgfqpoint{1.642905in}{0.961099in}}%
\pgfpathlineto{\pgfqpoint{1.644738in}{1.003404in}}%
\pgfpathlineto{\pgfqpoint{1.648404in}{1.246760in}}%
\pgfpathlineto{\pgfqpoint{1.650238in}{1.113270in}}%
\pgfpathlineto{\pgfqpoint{1.652073in}{1.205032in}}%
\pgfpathlineto{\pgfqpoint{1.653906in}{1.088203in}}%
\pgfpathlineto{\pgfqpoint{1.655738in}{1.330680in}}%
\pgfpathlineto{\pgfqpoint{1.659405in}{1.214930in}}%
\pgfpathlineto{\pgfqpoint{1.661238in}{1.358319in}}%
\pgfpathlineto{\pgfqpoint{1.663073in}{1.318448in}}%
\pgfpathlineto{\pgfqpoint{1.664906in}{1.330204in}}%
\pgfpathlineto{\pgfqpoint{1.666740in}{1.293419in}}%
\pgfpathlineto{\pgfqpoint{1.668574in}{1.187555in}}%
\pgfpathlineto{\pgfqpoint{1.670408in}{1.339111in}}%
\pgfpathlineto{\pgfqpoint{1.672240in}{1.212559in}}%
\pgfpathlineto{\pgfqpoint{1.674074in}{1.192486in}}%
\pgfpathlineto{\pgfqpoint{1.675908in}{1.292239in}}%
\pgfpathlineto{\pgfqpoint{1.677742in}{1.175862in}}%
\pgfpathlineto{\pgfqpoint{1.679575in}{1.524278in}}%
\pgfpathlineto{\pgfqpoint{1.681409in}{2.329019in}}%
\pgfpathlineto{\pgfqpoint{1.685075in}{1.986261in}}%
\pgfpathlineto{\pgfqpoint{1.686909in}{1.267536in}}%
\pgfpathlineto{\pgfqpoint{1.688743in}{1.353326in}}%
\pgfpathlineto{\pgfqpoint{1.690577in}{1.222195in}}%
\pgfpathlineto{\pgfqpoint{1.692411in}{1.172625in}}%
\pgfpathlineto{\pgfqpoint{1.694244in}{1.195020in}}%
\pgfpathlineto{\pgfqpoint{1.696076in}{1.164345in}}%
\pgfpathlineto{\pgfqpoint{1.697909in}{1.208507in}}%
\pgfpathlineto{\pgfqpoint{1.699744in}{1.574902in}}%
\pgfpathlineto{\pgfqpoint{1.701578in}{1.477419in}}%
\pgfpathlineto{\pgfqpoint{1.703412in}{1.219083in}}%
\pgfpathlineto{\pgfqpoint{1.705247in}{1.361268in}}%
\pgfpathlineto{\pgfqpoint{1.707080in}{1.268615in}}%
\pgfpathlineto{\pgfqpoint{1.710745in}{1.311786in}}%
\pgfpathlineto{\pgfqpoint{1.712579in}{1.307282in}}%
\pgfpathlineto{\pgfqpoint{1.714412in}{1.367729in}}%
\pgfpathlineto{\pgfqpoint{1.716246in}{1.267298in}}%
\pgfpathlineto{\pgfqpoint{1.718078in}{1.320154in}}%
\pgfpathlineto{\pgfqpoint{1.719912in}{1.241026in}}%
\pgfpathlineto{\pgfqpoint{1.721745in}{1.353326in}}%
\pgfpathlineto{\pgfqpoint{1.723579in}{1.252255in}}%
\pgfpathlineto{\pgfqpoint{1.725413in}{1.259682in}}%
\pgfpathlineto{\pgfqpoint{1.727245in}{1.202246in}}%
\pgfpathlineto{\pgfqpoint{1.729080in}{1.195396in}}%
\pgfpathlineto{\pgfqpoint{1.730913in}{1.052033in}}%
\pgfpathlineto{\pgfqpoint{1.732745in}{1.267624in}}%
\pgfpathlineto{\pgfqpoint{1.734578in}{1.143656in}}%
\pgfpathlineto{\pgfqpoint{1.738246in}{1.307784in}}%
\pgfpathlineto{\pgfqpoint{1.740080in}{1.190779in}}%
\pgfpathlineto{\pgfqpoint{1.741914in}{1.490718in}}%
\pgfpathlineto{\pgfqpoint{1.743747in}{1.342047in}}%
\pgfpathlineto{\pgfqpoint{1.745581in}{1.308900in}}%
\pgfpathlineto{\pgfqpoint{1.747414in}{1.156491in}}%
\pgfpathlineto{\pgfqpoint{1.749249in}{1.336138in}}%
\pgfpathlineto{\pgfqpoint{1.751084in}{1.295087in}}%
\pgfpathlineto{\pgfqpoint{1.752918in}{1.298650in}}%
\pgfpathlineto{\pgfqpoint{1.754750in}{1.439066in}}%
\pgfpathlineto{\pgfqpoint{1.756586in}{1.238856in}}%
\pgfpathlineto{\pgfqpoint{1.758418in}{1.359172in}}%
\pgfpathlineto{\pgfqpoint{1.762086in}{1.262304in}}%
\pgfpathlineto{\pgfqpoint{1.763919in}{1.389032in}}%
\pgfpathlineto{\pgfqpoint{1.765752in}{1.300645in}}%
\pgfpathlineto{\pgfqpoint{1.767584in}{1.365759in}}%
\pgfpathlineto{\pgfqpoint{1.769418in}{1.488547in}}%
\pgfpathlineto{\pgfqpoint{1.771251in}{1.353125in}}%
\pgfpathlineto{\pgfqpoint{1.773085in}{1.071454in}}%
\pgfpathlineto{\pgfqpoint{1.776754in}{1.204003in}}%
\pgfpathlineto{\pgfqpoint{1.778589in}{1.116595in}}%
\pgfpathlineto{\pgfqpoint{1.780423in}{1.329526in}}%
\pgfpathlineto{\pgfqpoint{1.784088in}{1.396785in}}%
\pgfpathlineto{\pgfqpoint{1.785923in}{1.105604in}}%
\pgfpathlineto{\pgfqpoint{1.791425in}{1.381128in}}%
\pgfpathlineto{\pgfqpoint{1.793259in}{1.368318in}}%
\pgfpathlineto{\pgfqpoint{1.795091in}{1.368958in}}%
\pgfpathlineto{\pgfqpoint{1.796923in}{1.396961in}}%
\pgfpathlineto{\pgfqpoint{1.804261in}{1.188759in}}%
\pgfpathlineto{\pgfqpoint{1.806094in}{1.199950in}}%
\pgfpathlineto{\pgfqpoint{1.807926in}{1.299616in}}%
\pgfpathlineto{\pgfqpoint{1.809760in}{1.281048in}}%
\pgfpathlineto{\pgfqpoint{1.811593in}{1.423697in}}%
\pgfpathlineto{\pgfqpoint{1.813426in}{1.365082in}}%
\pgfpathlineto{\pgfqpoint{1.815261in}{1.409156in}}%
\pgfpathlineto{\pgfqpoint{1.818926in}{1.198332in}}%
\pgfpathlineto{\pgfqpoint{1.820760in}{1.233950in}}%
\pgfpathlineto{\pgfqpoint{1.822594in}{1.157520in}}%
\pgfpathlineto{\pgfqpoint{1.824427in}{1.136016in}}%
\pgfpathlineto{\pgfqpoint{1.826262in}{1.382157in}}%
\pgfpathlineto{\pgfqpoint{1.828096in}{1.172475in}}%
\pgfpathlineto{\pgfqpoint{1.829929in}{1.186727in}}%
\pgfpathlineto{\pgfqpoint{1.831763in}{1.150055in}}%
\pgfpathlineto{\pgfqpoint{1.833596in}{1.177857in}}%
\pgfpathlineto{\pgfqpoint{1.835429in}{1.140005in}}%
\pgfpathlineto{\pgfqpoint{1.839098in}{1.297094in}}%
\pgfpathlineto{\pgfqpoint{1.840930in}{1.399458in}}%
\pgfpathlineto{\pgfqpoint{1.842762in}{1.623216in}}%
\pgfpathlineto{\pgfqpoint{1.844596in}{2.350285in}}%
\pgfpathlineto{\pgfqpoint{1.846431in}{2.251058in}}%
\pgfpathlineto{\pgfqpoint{1.848264in}{2.248950in}}%
\pgfpathlineto{\pgfqpoint{1.850099in}{2.485719in}}%
\pgfpathlineto{\pgfqpoint{1.853767in}{2.166297in}}%
\pgfpathlineto{\pgfqpoint{1.855603in}{2.253354in}}%
\pgfpathlineto{\pgfqpoint{1.857436in}{2.148080in}}%
\pgfpathlineto{\pgfqpoint{1.859269in}{2.115347in}}%
\pgfpathlineto{\pgfqpoint{1.861102in}{2.146311in}}%
\pgfpathlineto{\pgfqpoint{1.862937in}{2.134330in}}%
\pgfpathlineto{\pgfqpoint{1.864769in}{2.200284in}}%
\pgfpathlineto{\pgfqpoint{1.868437in}{1.999159in}}%
\pgfpathlineto{\pgfqpoint{1.870270in}{2.312383in}}%
\pgfpathlineto{\pgfqpoint{1.872104in}{1.429631in}}%
\pgfpathlineto{\pgfqpoint{1.873938in}{1.215495in}}%
\pgfpathlineto{\pgfqpoint{1.875771in}{1.158047in}}%
\pgfpathlineto{\pgfqpoint{1.877984in}{1.237212in}}%
\pgfpathlineto{\pgfqpoint{1.881649in}{1.528895in}}%
\pgfpathlineto{\pgfqpoint{1.883483in}{2.110969in}}%
\pgfpathlineto{\pgfqpoint{1.885316in}{2.210547in}}%
\pgfpathlineto{\pgfqpoint{1.887148in}{2.115573in}}%
\pgfpathlineto{\pgfqpoint{1.888982in}{2.242778in}}%
\pgfpathlineto{\pgfqpoint{1.890815in}{2.198967in}}%
\pgfpathlineto{\pgfqpoint{1.894481in}{2.279011in}}%
\pgfpathlineto{\pgfqpoint{1.896315in}{2.248536in}}%
\pgfpathlineto{\pgfqpoint{1.898148in}{2.142999in}}%
\pgfpathlineto{\pgfqpoint{1.899983in}{2.221060in}}%
\pgfpathlineto{\pgfqpoint{1.901816in}{2.358101in}}%
\pgfpathlineto{\pgfqpoint{1.903651in}{2.255035in}}%
\pgfpathlineto{\pgfqpoint{1.907317in}{2.443075in}}%
\pgfpathlineto{\pgfqpoint{1.909150in}{2.615270in}}%
\pgfpathlineto{\pgfqpoint{1.910984in}{2.267606in}}%
\pgfpathlineto{\pgfqpoint{1.912818in}{2.255060in}}%
\pgfpathlineto{\pgfqpoint{1.916485in}{2.465708in}}%
\pgfpathlineto{\pgfqpoint{1.920151in}{2.112261in}}%
\pgfpathlineto{\pgfqpoint{1.921984in}{2.204136in}}%
\pgfpathlineto{\pgfqpoint{1.923819in}{2.207134in}}%
\pgfpathlineto{\pgfqpoint{1.925653in}{2.334652in}}%
\pgfpathlineto{\pgfqpoint{1.927486in}{1.253836in}}%
\pgfpathlineto{\pgfqpoint{1.929321in}{1.344079in}}%
\pgfpathlineto{\pgfqpoint{1.931154in}{1.131248in}}%
\pgfpathlineto{\pgfqpoint{1.932987in}{1.597384in}}%
\pgfpathlineto{\pgfqpoint{1.934824in}{1.776378in}}%
\pgfpathlineto{\pgfqpoint{1.936657in}{1.695105in}}%
\pgfpathlineto{\pgfqpoint{1.938490in}{1.516613in}}%
\pgfpathlineto{\pgfqpoint{1.940325in}{1.617219in}}%
\pgfpathlineto{\pgfqpoint{1.942159in}{1.541855in}}%
\pgfpathlineto{\pgfqpoint{1.945827in}{1.306052in}}%
\pgfpathlineto{\pgfqpoint{1.947660in}{1.349625in}}%
\pgfpathlineto{\pgfqpoint{1.949493in}{1.056035in}}%
\pgfpathlineto{\pgfqpoint{1.951341in}{1.290269in}}%
\pgfpathlineto{\pgfqpoint{1.953174in}{1.252845in}}%
\pgfpathlineto{\pgfqpoint{1.955006in}{1.070965in}}%
\pgfpathlineto{\pgfqpoint{1.956840in}{1.311874in}}%
\pgfpathlineto{\pgfqpoint{1.960506in}{1.188935in}}%
\pgfpathlineto{\pgfqpoint{1.962341in}{1.290771in}}%
\pgfpathlineto{\pgfqpoint{1.964174in}{1.229798in}}%
\pgfpathlineto{\pgfqpoint{1.966007in}{1.209937in}}%
\pgfpathlineto{\pgfqpoint{1.967841in}{1.210916in}}%
\pgfpathlineto{\pgfqpoint{1.971508in}{2.044324in}}%
\pgfpathlineto{\pgfqpoint{1.973343in}{2.051413in}}%
\pgfpathlineto{\pgfqpoint{1.977011in}{2.339031in}}%
\pgfpathlineto{\pgfqpoint{1.978845in}{2.248361in}}%
\pgfpathlineto{\pgfqpoint{1.980679in}{1.942099in}}%
\pgfpathlineto{\pgfqpoint{1.982512in}{2.126012in}}%
\pgfpathlineto{\pgfqpoint{1.984346in}{2.094283in}}%
\pgfpathlineto{\pgfqpoint{1.986181in}{2.142372in}}%
\pgfpathlineto{\pgfqpoint{1.988015in}{2.394158in}}%
\pgfpathlineto{\pgfqpoint{1.989847in}{2.089189in}}%
\pgfpathlineto{\pgfqpoint{1.991680in}{2.035868in}}%
\pgfpathlineto{\pgfqpoint{1.995348in}{2.190862in}}%
\pgfpathlineto{\pgfqpoint{1.997182in}{2.178981in}}%
\pgfpathlineto{\pgfqpoint{1.999016in}{2.002132in}}%
\pgfpathlineto{\pgfqpoint{2.000873in}{2.293903in}}%
\pgfpathlineto{\pgfqpoint{2.002706in}{1.409507in}}%
\pgfpathlineto{\pgfqpoint{2.004539in}{1.377778in}}%
\pgfpathlineto{\pgfqpoint{2.006374in}{1.125051in}}%
\pgfpathlineto{\pgfqpoint{2.008208in}{1.186727in}}%
\pgfpathlineto{\pgfqpoint{2.010042in}{1.146391in}}%
\pgfpathlineto{\pgfqpoint{2.011876in}{1.228330in}}%
\pgfpathlineto{\pgfqpoint{2.013709in}{1.123733in}}%
\pgfpathlineto{\pgfqpoint{2.015542in}{1.282014in}}%
\pgfpathlineto{\pgfqpoint{2.017375in}{0.981963in}}%
\pgfpathlineto{\pgfqpoint{2.019208in}{1.294886in}}%
\pgfpathlineto{\pgfqpoint{2.022876in}{1.117849in}}%
\pgfpathlineto{\pgfqpoint{2.024711in}{1.052033in}}%
\pgfpathlineto{\pgfqpoint{2.026543in}{1.145149in}}%
\pgfpathlineto{\pgfqpoint{2.028377in}{1.134397in}}%
\pgfpathlineto{\pgfqpoint{2.030211in}{1.103923in}}%
\pgfpathlineto{\pgfqpoint{2.032044in}{1.309227in}}%
\pgfpathlineto{\pgfqpoint{2.033877in}{1.243849in}}%
\pgfpathlineto{\pgfqpoint{2.035712in}{1.300093in}}%
\pgfpathlineto{\pgfqpoint{2.039381in}{1.157143in}}%
\pgfpathlineto{\pgfqpoint{2.041214in}{1.375984in}}%
\pgfpathlineto{\pgfqpoint{2.043048in}{2.166347in}}%
\pgfpathlineto{\pgfqpoint{2.044882in}{1.992083in}}%
\pgfpathlineto{\pgfqpoint{2.046716in}{2.385577in}}%
\pgfpathlineto{\pgfqpoint{2.050384in}{2.229491in}}%
\pgfpathlineto{\pgfqpoint{2.054049in}{2.055252in}}%
\pgfpathlineto{\pgfqpoint{2.055883in}{2.069216in}}%
\pgfpathlineto{\pgfqpoint{2.057717in}{2.196671in}}%
\pgfpathlineto{\pgfqpoint{2.059552in}{1.349775in}}%
\pgfpathlineto{\pgfqpoint{2.061385in}{1.258980in}}%
\pgfpathlineto{\pgfqpoint{2.063219in}{1.267586in}}%
\pgfpathlineto{\pgfqpoint{2.065051in}{1.313580in}}%
\pgfpathlineto{\pgfqpoint{2.066885in}{1.268477in}}%
\pgfpathlineto{\pgfqpoint{2.068720in}{1.133569in}}%
\pgfpathlineto{\pgfqpoint{2.070554in}{1.164809in}}%
\pgfpathlineto{\pgfqpoint{2.072387in}{1.166453in}}%
\pgfpathlineto{\pgfqpoint{2.074222in}{1.187818in}}%
\pgfpathlineto{\pgfqpoint{2.076056in}{1.271827in}}%
\pgfpathlineto{\pgfqpoint{2.077889in}{1.118062in}}%
\pgfpathlineto{\pgfqpoint{2.079722in}{1.189725in}}%
\pgfpathlineto{\pgfqpoint{2.081556in}{0.964386in}}%
\pgfpathlineto{\pgfqpoint{2.083390in}{1.215257in}}%
\pgfpathlineto{\pgfqpoint{2.085224in}{1.270472in}}%
\pgfpathlineto{\pgfqpoint{2.087060in}{1.137484in}}%
\pgfpathlineto{\pgfqpoint{2.088893in}{1.109217in}}%
\pgfpathlineto{\pgfqpoint{2.090725in}{1.192511in}}%
\pgfpathlineto{\pgfqpoint{2.092559in}{1.181470in}}%
\pgfpathlineto{\pgfqpoint{2.094392in}{1.283131in}}%
\pgfpathlineto{\pgfqpoint{2.096225in}{1.178534in}}%
\pgfpathlineto{\pgfqpoint{2.098059in}{1.221894in}}%
\pgfpathlineto{\pgfqpoint{2.099907in}{1.421238in}}%
\pgfpathlineto{\pgfqpoint{2.101741in}{1.193903in}}%
\pgfpathlineto{\pgfqpoint{2.103575in}{1.272002in}}%
\pgfpathlineto{\pgfqpoint{2.105407in}{1.269945in}}%
\pgfpathlineto{\pgfqpoint{2.109075in}{1.055157in}}%
\pgfpathlineto{\pgfqpoint{2.110908in}{1.064315in}}%
\pgfpathlineto{\pgfqpoint{2.112743in}{1.028672in}}%
\pgfpathlineto{\pgfqpoint{2.114577in}{1.128701in}}%
\pgfpathlineto{\pgfqpoint{2.116411in}{1.030943in}}%
\pgfpathlineto{\pgfqpoint{2.118245in}{1.024005in}}%
\pgfpathlineto{\pgfqpoint{2.121912in}{1.177029in}}%
\pgfpathlineto{\pgfqpoint{2.125580in}{1.027530in}}%
\pgfpathlineto{\pgfqpoint{2.127413in}{1.272906in}}%
\pgfpathlineto{\pgfqpoint{2.131080in}{1.159577in}}%
\pgfpathlineto{\pgfqpoint{2.132913in}{1.205709in}}%
\pgfpathlineto{\pgfqpoint{2.136582in}{1.199599in}}%
\pgfpathlineto{\pgfqpoint{2.138416in}{1.144447in}}%
\pgfpathlineto{\pgfqpoint{2.140250in}{1.135514in}}%
\pgfpathlineto{\pgfqpoint{2.142083in}{1.154910in}}%
\pgfpathlineto{\pgfqpoint{2.143917in}{1.290219in}}%
\pgfpathlineto{\pgfqpoint{2.145750in}{1.320656in}}%
\pgfpathlineto{\pgfqpoint{2.147586in}{1.234891in}}%
\pgfpathlineto{\pgfqpoint{2.149421in}{1.246634in}}%
\pgfpathlineto{\pgfqpoint{2.153089in}{1.157821in}}%
\pgfpathlineto{\pgfqpoint{2.154922in}{1.071454in}}%
\pgfpathlineto{\pgfqpoint{2.156755in}{1.217753in}}%
\pgfpathlineto{\pgfqpoint{2.158588in}{1.094639in}}%
\pgfpathlineto{\pgfqpoint{2.160423in}{1.142239in}}%
\pgfpathlineto{\pgfqpoint{2.162254in}{1.275703in}}%
\pgfpathlineto{\pgfqpoint{2.164087in}{1.283432in}}%
\pgfpathlineto{\pgfqpoint{2.167753in}{1.059937in}}%
\pgfpathlineto{\pgfqpoint{2.169586in}{1.231240in}}%
\pgfpathlineto{\pgfqpoint{2.171421in}{1.087852in}}%
\pgfpathlineto{\pgfqpoint{2.173255in}{1.199775in}}%
\pgfpathlineto{\pgfqpoint{2.175088in}{1.118000in}}%
\pgfpathlineto{\pgfqpoint{2.178756in}{1.935839in}}%
\pgfpathlineto{\pgfqpoint{2.180589in}{2.126338in}}%
\pgfpathlineto{\pgfqpoint{2.182425in}{2.032305in}}%
\pgfpathlineto{\pgfqpoint{2.184258in}{2.207373in}}%
\pgfpathlineto{\pgfqpoint{2.186092in}{2.025167in}}%
\pgfpathlineto{\pgfqpoint{2.187926in}{1.065030in}}%
\pgfpathlineto{\pgfqpoint{2.189760in}{1.289805in}}%
\pgfpathlineto{\pgfqpoint{2.191595in}{1.217251in}}%
\pgfpathlineto{\pgfqpoint{2.193430in}{1.218544in}}%
\pgfpathlineto{\pgfqpoint{2.195262in}{1.127196in}}%
\pgfpathlineto{\pgfqpoint{2.197096in}{1.322387in}}%
\pgfpathlineto{\pgfqpoint{2.198950in}{1.382684in}}%
\pgfpathlineto{\pgfqpoint{2.200782in}{1.168071in}}%
\pgfpathlineto{\pgfqpoint{2.202617in}{1.245054in}}%
\pgfpathlineto{\pgfqpoint{2.204450in}{1.213086in}}%
\pgfpathlineto{\pgfqpoint{2.206283in}{1.372283in}}%
\pgfpathlineto{\pgfqpoint{2.208117in}{1.284661in}}%
\pgfpathlineto{\pgfqpoint{2.209951in}{1.447936in}}%
\pgfpathlineto{\pgfqpoint{2.211786in}{1.227778in}}%
\pgfpathlineto{\pgfqpoint{2.213619in}{1.462640in}}%
\pgfpathlineto{\pgfqpoint{2.219122in}{1.289391in}}%
\pgfpathlineto{\pgfqpoint{2.220957in}{1.236146in}}%
\pgfpathlineto{\pgfqpoint{2.222791in}{1.074164in}}%
\pgfpathlineto{\pgfqpoint{2.224623in}{1.046537in}}%
\pgfpathlineto{\pgfqpoint{2.226457in}{1.119242in}}%
\pgfpathlineto{\pgfqpoint{2.228291in}{0.930863in}}%
\pgfpathlineto{\pgfqpoint{2.230125in}{1.210201in}}%
\pgfpathlineto{\pgfqpoint{2.231958in}{1.108038in}}%
\pgfpathlineto{\pgfqpoint{2.233792in}{1.210966in}}%
\pgfpathlineto{\pgfqpoint{2.237460in}{1.083335in}}%
\pgfpathlineto{\pgfqpoint{2.239295in}{1.061643in}}%
\pgfpathlineto{\pgfqpoint{2.241128in}{1.072520in}}%
\pgfpathlineto{\pgfqpoint{2.242963in}{1.029174in}}%
\pgfpathlineto{\pgfqpoint{2.246630in}{1.215495in}}%
\pgfpathlineto{\pgfqpoint{2.248478in}{0.970145in}}%
\pgfpathlineto{\pgfqpoint{2.250312in}{1.126255in}}%
\pgfpathlineto{\pgfqpoint{2.252145in}{1.032147in}}%
\pgfpathlineto{\pgfqpoint{2.253979in}{1.223161in}}%
\pgfpathlineto{\pgfqpoint{2.255812in}{1.093911in}}%
\pgfpathlineto{\pgfqpoint{2.257648in}{1.120710in}}%
\pgfpathlineto{\pgfqpoint{2.261314in}{0.861283in}}%
\pgfpathlineto{\pgfqpoint{2.263148in}{0.931867in}}%
\pgfpathlineto{\pgfqpoint{2.264982in}{1.100548in}}%
\pgfpathlineto{\pgfqpoint{2.266816in}{1.111212in}}%
\pgfpathlineto{\pgfqpoint{2.268648in}{0.909271in}}%
\pgfpathlineto{\pgfqpoint{2.270482in}{0.943171in}}%
\pgfpathlineto{\pgfqpoint{2.272317in}{1.219108in}}%
\pgfpathlineto{\pgfqpoint{2.274151in}{1.189462in}}%
\pgfpathlineto{\pgfqpoint{2.275986in}{1.033640in}}%
\pgfpathlineto{\pgfqpoint{2.277818in}{1.310807in}}%
\pgfpathlineto{\pgfqpoint{2.279652in}{0.981549in}}%
\pgfpathlineto{\pgfqpoint{2.281486in}{0.945316in}}%
\pgfpathlineto{\pgfqpoint{2.285153in}{0.801024in}}%
\pgfpathlineto{\pgfqpoint{2.286987in}{0.983983in}}%
\pgfpathlineto{\pgfqpoint{2.288821in}{0.738821in}}%
\pgfpathlineto{\pgfqpoint{2.290654in}{0.790410in}}%
\pgfpathlineto{\pgfqpoint{2.292489in}{0.986342in}}%
\pgfpathlineto{\pgfqpoint{2.294323in}{1.005499in}}%
\pgfpathlineto{\pgfqpoint{2.296157in}{0.935743in}}%
\pgfpathlineto{\pgfqpoint{2.298007in}{0.963395in}}%
\pgfpathlineto{\pgfqpoint{2.299841in}{0.803746in}}%
\pgfpathlineto{\pgfqpoint{2.301675in}{0.839716in}}%
\pgfpathlineto{\pgfqpoint{2.305342in}{0.944613in}}%
\pgfpathlineto{\pgfqpoint{2.307176in}{0.989014in}}%
\pgfpathlineto{\pgfqpoint{2.309010in}{0.999000in}}%
\pgfpathlineto{\pgfqpoint{2.310843in}{1.024005in}}%
\pgfpathlineto{\pgfqpoint{2.312677in}{1.002288in}}%
\pgfpathlineto{\pgfqpoint{2.316343in}{0.874569in}}%
\pgfpathlineto{\pgfqpoint{2.320010in}{1.113445in}}%
\pgfpathlineto{\pgfqpoint{2.323678in}{1.022035in}}%
\pgfpathlineto{\pgfqpoint{2.325512in}{1.086973in}}%
\pgfpathlineto{\pgfqpoint{2.327345in}{1.044480in}}%
\pgfpathlineto{\pgfqpoint{2.329179in}{0.861345in}}%
\pgfpathlineto{\pgfqpoint{2.331014in}{0.815126in}}%
\pgfpathlineto{\pgfqpoint{2.332847in}{0.970822in}}%
\pgfpathlineto{\pgfqpoint{2.334681in}{0.995563in}}%
\pgfpathlineto{\pgfqpoint{2.336516in}{0.912408in}}%
\pgfpathlineto{\pgfqpoint{2.338350in}{1.038232in}}%
\pgfpathlineto{\pgfqpoint{2.340183in}{1.060288in}}%
\pgfpathlineto{\pgfqpoint{2.342017in}{0.985752in}}%
\pgfpathlineto{\pgfqpoint{2.343851in}{1.101602in}}%
\pgfpathlineto{\pgfqpoint{2.345685in}{1.035434in}}%
\pgfpathlineto{\pgfqpoint{2.347518in}{1.112944in}}%
\pgfpathlineto{\pgfqpoint{2.349353in}{1.019890in}}%
\pgfpathlineto{\pgfqpoint{2.351186in}{1.018422in}}%
\pgfpathlineto{\pgfqpoint{2.353019in}{0.900564in}}%
\pgfpathlineto{\pgfqpoint{2.354854in}{1.045132in}}%
\pgfpathlineto{\pgfqpoint{2.356687in}{1.070162in}}%
\pgfpathlineto{\pgfqpoint{2.358520in}{1.025623in}}%
\pgfpathlineto{\pgfqpoint{2.360354in}{1.040202in}}%
\pgfpathlineto{\pgfqpoint{2.362187in}{0.987308in}}%
\pgfpathlineto{\pgfqpoint{2.364021in}{0.991385in}}%
\pgfpathlineto{\pgfqpoint{2.365856in}{0.912056in}}%
\pgfpathlineto{\pgfqpoint{2.369522in}{0.871771in}}%
\pgfpathlineto{\pgfqpoint{2.371356in}{0.873389in}}%
\pgfpathlineto{\pgfqpoint{2.373192in}{1.075280in}}%
\pgfpathlineto{\pgfqpoint{2.375026in}{0.964913in}}%
\pgfpathlineto{\pgfqpoint{2.376861in}{0.979291in}}%
\pgfpathlineto{\pgfqpoint{2.378694in}{1.046011in}}%
\pgfpathlineto{\pgfqpoint{2.380528in}{0.849941in}}%
\pgfpathlineto{\pgfqpoint{2.382365in}{0.875648in}}%
\pgfpathlineto{\pgfqpoint{2.386030in}{0.971964in}}%
\pgfpathlineto{\pgfqpoint{2.389698in}{0.835137in}}%
\pgfpathlineto{\pgfqpoint{2.391531in}{0.648878in}}%
\pgfpathlineto{\pgfqpoint{2.393366in}{0.690744in}}%
\pgfpathlineto{\pgfqpoint{2.397033in}{1.258428in}}%
\pgfpathlineto{\pgfqpoint{2.400702in}{0.871595in}}%
\pgfpathlineto{\pgfqpoint{2.402535in}{1.079358in}}%
\pgfpathlineto{\pgfqpoint{2.404369in}{0.981925in}}%
\pgfpathlineto{\pgfqpoint{2.406204in}{1.019363in}}%
\pgfpathlineto{\pgfqpoint{2.408037in}{0.916962in}}%
\pgfpathlineto{\pgfqpoint{2.409871in}{0.967974in}}%
\pgfpathlineto{\pgfqpoint{2.411705in}{1.051242in}}%
\pgfpathlineto{\pgfqpoint{2.415371in}{0.880240in}}%
\pgfpathlineto{\pgfqpoint{2.417205in}{0.915645in}}%
\pgfpathlineto{\pgfqpoint{2.419039in}{1.045596in}}%
\pgfpathlineto{\pgfqpoint{2.420872in}{0.994622in}}%
\pgfpathlineto{\pgfqpoint{2.424540in}{1.062320in}}%
\pgfpathlineto{\pgfqpoint{2.426374in}{0.912822in}}%
\pgfpathlineto{\pgfqpoint{2.428209in}{1.008586in}}%
\pgfpathlineto{\pgfqpoint{2.430043in}{1.014457in}}%
\pgfpathlineto{\pgfqpoint{2.431877in}{1.037793in}}%
\pgfpathlineto{\pgfqpoint{2.433711in}{0.772983in}}%
\pgfpathlineto{\pgfqpoint{2.435544in}{0.991385in}}%
\pgfpathlineto{\pgfqpoint{2.437378in}{0.962630in}}%
\pgfpathlineto{\pgfqpoint{2.439213in}{1.210765in}}%
\pgfpathlineto{\pgfqpoint{2.441046in}{1.815610in}}%
\pgfpathlineto{\pgfqpoint{2.442879in}{1.656739in}}%
\pgfpathlineto{\pgfqpoint{2.444714in}{1.981004in}}%
\pgfpathlineto{\pgfqpoint{2.446550in}{1.928411in}}%
\pgfpathlineto{\pgfqpoint{2.448385in}{1.950154in}}%
\pgfpathlineto{\pgfqpoint{2.450220in}{1.290621in}}%
\pgfpathlineto{\pgfqpoint{2.452054in}{1.294096in}}%
\pgfpathlineto{\pgfqpoint{2.453888in}{1.511406in}}%
\pgfpathlineto{\pgfqpoint{2.455723in}{1.110773in}}%
\pgfpathlineto{\pgfqpoint{2.459389in}{1.204304in}}%
\pgfpathlineto{\pgfqpoint{2.461223in}{1.071078in}}%
\pgfpathlineto{\pgfqpoint{2.463057in}{1.193953in}}%
\pgfpathlineto{\pgfqpoint{2.464891in}{1.058619in}}%
\pgfpathlineto{\pgfqpoint{2.466724in}{1.107097in}}%
\pgfpathlineto{\pgfqpoint{2.468558in}{1.103522in}}%
\pgfpathlineto{\pgfqpoint{2.470392in}{1.147320in}}%
\pgfpathlineto{\pgfqpoint{2.474059in}{2.006511in}}%
\pgfpathlineto{\pgfqpoint{2.475893in}{2.000049in}}%
\pgfpathlineto{\pgfqpoint{2.477727in}{1.851692in}}%
\pgfpathlineto{\pgfqpoint{2.479561in}{1.029036in}}%
\pgfpathlineto{\pgfqpoint{2.481394in}{1.026715in}}%
\pgfpathlineto{\pgfqpoint{2.483228in}{0.866866in}}%
\pgfpathlineto{\pgfqpoint{2.485062in}{0.929897in}}%
\pgfpathlineto{\pgfqpoint{2.486896in}{1.048156in}}%
\pgfpathlineto{\pgfqpoint{2.488729in}{0.928918in}}%
\pgfpathlineto{\pgfqpoint{2.490563in}{0.961889in}}%
\pgfpathlineto{\pgfqpoint{2.492397in}{0.857995in}}%
\pgfpathlineto{\pgfqpoint{2.494230in}{0.893375in}}%
\pgfpathlineto{\pgfqpoint{2.496065in}{0.872009in}}%
\pgfpathlineto{\pgfqpoint{2.497897in}{1.098641in}}%
\pgfpathlineto{\pgfqpoint{2.499731in}{0.992953in}}%
\pgfpathlineto{\pgfqpoint{2.503403in}{1.168096in}}%
\pgfpathlineto{\pgfqpoint{2.505236in}{1.478849in}}%
\pgfpathlineto{\pgfqpoint{2.507071in}{1.334406in}}%
\pgfpathlineto{\pgfqpoint{2.508905in}{1.393523in}}%
\pgfpathlineto{\pgfqpoint{2.510738in}{1.348333in}}%
\pgfpathlineto{\pgfqpoint{2.512572in}{1.338898in}}%
\pgfpathlineto{\pgfqpoint{2.514406in}{1.346752in}}%
\pgfpathlineto{\pgfqpoint{2.518075in}{1.100398in}}%
\pgfpathlineto{\pgfqpoint{2.519909in}{1.250461in}}%
\pgfpathlineto{\pgfqpoint{2.521742in}{1.113245in}}%
\pgfpathlineto{\pgfqpoint{2.523576in}{1.221718in}}%
\pgfpathlineto{\pgfqpoint{2.527245in}{1.021332in}}%
\pgfpathlineto{\pgfqpoint{2.529079in}{1.287773in}}%
\pgfpathlineto{\pgfqpoint{2.530912in}{1.257775in}}%
\pgfpathlineto{\pgfqpoint{2.532745in}{1.174118in}}%
\pgfpathlineto{\pgfqpoint{2.534578in}{1.298738in}}%
\pgfpathlineto{\pgfqpoint{2.536412in}{1.293795in}}%
\pgfpathlineto{\pgfqpoint{2.538246in}{1.268301in}}%
\pgfpathlineto{\pgfqpoint{2.540080in}{1.260008in}}%
\pgfpathlineto{\pgfqpoint{2.541915in}{1.242444in}}%
\pgfpathlineto{\pgfqpoint{2.547418in}{2.259941in}}%
\pgfpathlineto{\pgfqpoint{2.549252in}{2.315319in}}%
\pgfpathlineto{\pgfqpoint{2.551087in}{2.307653in}}%
\pgfpathlineto{\pgfqpoint{2.552922in}{2.231060in}}%
\pgfpathlineto{\pgfqpoint{2.554755in}{2.348641in}}%
\pgfpathlineto{\pgfqpoint{2.556590in}{2.236868in}}%
\pgfpathlineto{\pgfqpoint{2.558425in}{2.250795in}}%
\pgfpathlineto{\pgfqpoint{2.560258in}{2.104909in}}%
\pgfpathlineto{\pgfqpoint{2.562091in}{2.130089in}}%
\pgfpathlineto{\pgfqpoint{2.563926in}{2.072854in}}%
\pgfpathlineto{\pgfqpoint{2.565761in}{2.329609in}}%
\pgfpathlineto{\pgfqpoint{2.567594in}{2.319472in}}%
\pgfpathlineto{\pgfqpoint{2.569431in}{2.393305in}}%
\pgfpathlineto{\pgfqpoint{2.571263in}{2.301079in}}%
\pgfpathlineto{\pgfqpoint{2.573096in}{2.430354in}}%
\pgfpathlineto{\pgfqpoint{2.574931in}{2.383406in}}%
\pgfpathlineto{\pgfqpoint{2.576766in}{2.444104in}}%
\pgfpathlineto{\pgfqpoint{2.578600in}{2.332043in}}%
\pgfpathlineto{\pgfqpoint{2.580434in}{2.401736in}}%
\pgfpathlineto{\pgfqpoint{2.584101in}{2.219856in}}%
\pgfpathlineto{\pgfqpoint{2.585936in}{2.234610in}}%
\pgfpathlineto{\pgfqpoint{2.589603in}{2.443878in}}%
\pgfpathlineto{\pgfqpoint{2.591437in}{2.428597in}}%
\pgfpathlineto{\pgfqpoint{2.595105in}{2.225941in}}%
\pgfpathlineto{\pgfqpoint{2.596940in}{2.246805in}}%
\pgfpathlineto{\pgfqpoint{2.598773in}{2.450189in}}%
\pgfpathlineto{\pgfqpoint{2.600607in}{2.379843in}}%
\pgfpathlineto{\pgfqpoint{2.602442in}{2.401799in}}%
\pgfpathlineto{\pgfqpoint{2.604276in}{2.479835in}}%
\pgfpathlineto{\pgfqpoint{2.606110in}{2.304981in}}%
\pgfpathlineto{\pgfqpoint{2.607943in}{2.478756in}}%
\pgfpathlineto{\pgfqpoint{2.611611in}{1.396873in}}%
\pgfpathlineto{\pgfqpoint{2.613445in}{1.390788in}}%
\pgfpathlineto{\pgfqpoint{2.615280in}{1.459189in}}%
\pgfpathlineto{\pgfqpoint{2.617113in}{1.454823in}}%
\pgfpathlineto{\pgfqpoint{2.618945in}{1.555894in}}%
\pgfpathlineto{\pgfqpoint{2.620781in}{1.540525in}}%
\pgfpathlineto{\pgfqpoint{2.622614in}{1.455702in}}%
\pgfpathlineto{\pgfqpoint{2.624446in}{1.550838in}}%
\pgfpathlineto{\pgfqpoint{2.626282in}{1.843588in}}%
\pgfpathlineto{\pgfqpoint{2.628117in}{1.812938in}}%
\pgfpathlineto{\pgfqpoint{2.629951in}{1.620895in}}%
\pgfpathlineto{\pgfqpoint{2.633620in}{1.527365in}}%
\pgfpathlineto{\pgfqpoint{2.637287in}{1.633027in}}%
\pgfpathlineto{\pgfqpoint{2.639121in}{1.552544in}}%
\pgfpathlineto{\pgfqpoint{2.640953in}{1.653917in}}%
\pgfpathlineto{\pgfqpoint{2.642788in}{1.496715in}}%
\pgfpathlineto{\pgfqpoint{2.646455in}{1.703511in}}%
\pgfpathlineto{\pgfqpoint{2.648288in}{1.722380in}}%
\pgfpathlineto{\pgfqpoint{2.650122in}{1.792488in}}%
\pgfpathlineto{\pgfqpoint{2.657458in}{1.602553in}}%
\pgfpathlineto{\pgfqpoint{2.659291in}{1.605614in}}%
\pgfpathlineto{\pgfqpoint{2.661125in}{1.775714in}}%
\pgfpathlineto{\pgfqpoint{2.662959in}{1.770858in}}%
\pgfpathlineto{\pgfqpoint{2.664792in}{1.387326in}}%
\pgfpathlineto{\pgfqpoint{2.666626in}{1.444498in}}%
\pgfpathlineto{\pgfqpoint{2.668460in}{1.272994in}}%
\pgfpathlineto{\pgfqpoint{2.670294in}{1.391378in}}%
\pgfpathlineto{\pgfqpoint{2.672129in}{1.301147in}}%
\pgfpathlineto{\pgfqpoint{2.675797in}{1.282077in}}%
\pgfpathlineto{\pgfqpoint{2.677631in}{1.327180in}}%
\pgfpathlineto{\pgfqpoint{2.679465in}{1.298587in}}%
\pgfpathlineto{\pgfqpoint{2.681298in}{1.296530in}}%
\pgfpathlineto{\pgfqpoint{2.683133in}{1.372433in}}%
\pgfpathlineto{\pgfqpoint{2.684967in}{1.294008in}}%
\pgfpathlineto{\pgfqpoint{2.686800in}{1.320305in}}%
\pgfpathlineto{\pgfqpoint{2.688636in}{1.226636in}}%
\pgfpathlineto{\pgfqpoint{2.690470in}{1.308173in}}%
\pgfpathlineto{\pgfqpoint{2.692303in}{1.294385in}}%
\pgfpathlineto{\pgfqpoint{2.694139in}{1.293481in}}%
\pgfpathlineto{\pgfqpoint{2.695974in}{1.375959in}}%
\pgfpathlineto{\pgfqpoint{2.697807in}{1.311610in}}%
\pgfpathlineto{\pgfqpoint{2.699642in}{1.287396in}}%
\pgfpathlineto{\pgfqpoint{2.701476in}{1.193226in}}%
\pgfpathlineto{\pgfqpoint{2.703309in}{1.280396in}}%
\pgfpathlineto{\pgfqpoint{2.705144in}{1.439856in}}%
\pgfpathlineto{\pgfqpoint{2.706978in}{1.375921in}}%
\pgfpathlineto{\pgfqpoint{2.708810in}{1.504656in}}%
\pgfpathlineto{\pgfqpoint{2.710645in}{1.347893in}}%
\pgfpathlineto{\pgfqpoint{2.712479in}{1.353301in}}%
\pgfpathlineto{\pgfqpoint{2.714312in}{1.232119in}}%
\pgfpathlineto{\pgfqpoint{2.716146in}{1.324244in}}%
\pgfpathlineto{\pgfqpoint{2.719814in}{1.198395in}}%
\pgfpathlineto{\pgfqpoint{2.721649in}{1.311221in}}%
\pgfpathlineto{\pgfqpoint{2.723486in}{1.338434in}}%
\pgfpathlineto{\pgfqpoint{2.725319in}{1.394088in}}%
\pgfpathlineto{\pgfqpoint{2.727153in}{1.236999in}}%
\pgfpathlineto{\pgfqpoint{2.728986in}{1.307934in}}%
\pgfpathlineto{\pgfqpoint{2.730821in}{1.280433in}}%
\pgfpathlineto{\pgfqpoint{2.732655in}{1.329940in}}%
\pgfpathlineto{\pgfqpoint{2.734489in}{1.281424in}}%
\pgfpathlineto{\pgfqpoint{2.736323in}{1.154584in}}%
\pgfpathlineto{\pgfqpoint{2.738156in}{1.132666in}}%
\pgfpathlineto{\pgfqpoint{2.739990in}{1.187254in}}%
\pgfpathlineto{\pgfqpoint{2.741824in}{1.146028in}}%
\pgfpathlineto{\pgfqpoint{2.743657in}{1.356763in}}%
\pgfpathlineto{\pgfqpoint{2.747324in}{2.163537in}}%
\pgfpathlineto{\pgfqpoint{2.749158in}{2.217748in}}%
\pgfpathlineto{\pgfqpoint{2.750992in}{2.141029in}}%
\pgfpathlineto{\pgfqpoint{2.754676in}{2.262789in}}%
\pgfpathlineto{\pgfqpoint{2.756510in}{2.219982in}}%
\pgfpathlineto{\pgfqpoint{2.758345in}{2.138558in}}%
\pgfpathlineto{\pgfqpoint{2.762014in}{2.354638in}}%
\pgfpathlineto{\pgfqpoint{2.763847in}{2.189005in}}%
\pgfpathlineto{\pgfqpoint{2.765683in}{2.189306in}}%
\pgfpathlineto{\pgfqpoint{2.767516in}{2.292171in}}%
\pgfpathlineto{\pgfqpoint{2.769350in}{2.295082in}}%
\pgfpathlineto{\pgfqpoint{2.771184in}{2.274657in}}%
\pgfpathlineto{\pgfqpoint{2.773018in}{2.351263in}}%
\pgfpathlineto{\pgfqpoint{2.776687in}{2.170262in}}%
\pgfpathlineto{\pgfqpoint{2.778521in}{2.590466in}}%
\pgfpathlineto{\pgfqpoint{2.780356in}{2.303626in}}%
\pgfpathlineto{\pgfqpoint{2.782191in}{2.256528in}}%
\pgfpathlineto{\pgfqpoint{2.784025in}{2.169383in}}%
\pgfpathlineto{\pgfqpoint{2.785858in}{1.276017in}}%
\pgfpathlineto{\pgfqpoint{2.787692in}{1.340253in}}%
\pgfpathlineto{\pgfqpoint{2.789527in}{1.261125in}}%
\pgfpathlineto{\pgfqpoint{2.791360in}{1.346338in}}%
\pgfpathlineto{\pgfqpoint{2.793195in}{1.252042in}}%
\pgfpathlineto{\pgfqpoint{2.795029in}{1.292829in}}%
\pgfpathlineto{\pgfqpoint{2.796863in}{1.220049in}}%
\pgfpathlineto{\pgfqpoint{2.798697in}{1.217640in}}%
\pgfpathlineto{\pgfqpoint{2.800531in}{1.317218in}}%
\pgfpathlineto{\pgfqpoint{2.802364in}{1.265968in}}%
\pgfpathlineto{\pgfqpoint{2.804199in}{1.392620in}}%
\pgfpathlineto{\pgfqpoint{2.806034in}{1.422881in}}%
\pgfpathlineto{\pgfqpoint{2.811535in}{1.242758in}}%
\pgfpathlineto{\pgfqpoint{2.813369in}{1.462539in}}%
\pgfpathlineto{\pgfqpoint{2.815202in}{1.126782in}}%
\pgfpathlineto{\pgfqpoint{2.817037in}{1.348546in}}%
\pgfpathlineto{\pgfqpoint{2.818870in}{1.245555in}}%
\pgfpathlineto{\pgfqpoint{2.820704in}{1.281286in}}%
\pgfpathlineto{\pgfqpoint{2.822539in}{1.267737in}}%
\pgfpathlineto{\pgfqpoint{2.826205in}{1.419092in}}%
\pgfpathlineto{\pgfqpoint{2.828040in}{1.313580in}}%
\pgfpathlineto{\pgfqpoint{2.831708in}{1.424638in}}%
\pgfpathlineto{\pgfqpoint{2.833542in}{1.194280in}}%
\pgfpathlineto{\pgfqpoint{2.835376in}{1.393197in}}%
\pgfpathlineto{\pgfqpoint{2.837209in}{1.312049in}}%
\pgfpathlineto{\pgfqpoint{2.839043in}{1.344983in}}%
\pgfpathlineto{\pgfqpoint{2.840878in}{1.414061in}}%
\pgfpathlineto{\pgfqpoint{2.842711in}{1.411100in}}%
\pgfpathlineto{\pgfqpoint{2.844547in}{1.444360in}}%
\pgfpathlineto{\pgfqpoint{2.846380in}{1.272755in}}%
\pgfpathlineto{\pgfqpoint{2.848214in}{1.269318in}}%
\pgfpathlineto{\pgfqpoint{2.851883in}{1.443181in}}%
\pgfpathlineto{\pgfqpoint{2.853718in}{1.241026in}}%
\pgfpathlineto{\pgfqpoint{2.857387in}{1.493666in}}%
\pgfpathlineto{\pgfqpoint{2.859250in}{1.399583in}}%
\pgfpathlineto{\pgfqpoint{2.861437in}{1.383323in}}%
\pgfpathlineto{\pgfqpoint{2.863271in}{1.552720in}}%
\pgfpathlineto{\pgfqpoint{2.866939in}{1.375871in}}%
\pgfpathlineto{\pgfqpoint{2.870607in}{1.266796in}}%
\pgfpathlineto{\pgfqpoint{2.874275in}{2.185242in}}%
\pgfpathlineto{\pgfqpoint{2.876108in}{1.586720in}}%
\pgfpathlineto{\pgfqpoint{2.877943in}{1.824217in}}%
\pgfpathlineto{\pgfqpoint{2.879777in}{1.612753in}}%
\pgfpathlineto{\pgfqpoint{2.881615in}{1.246935in}}%
\pgfpathlineto{\pgfqpoint{2.883449in}{1.504531in}}%
\pgfpathlineto{\pgfqpoint{2.885282in}{1.295978in}}%
\pgfpathlineto{\pgfqpoint{2.888951in}{1.228593in}}%
\pgfpathlineto{\pgfqpoint{2.892618in}{1.422116in}}%
\pgfpathlineto{\pgfqpoint{2.894452in}{1.428728in}}%
\pgfpathlineto{\pgfqpoint{2.896285in}{1.293117in}}%
\pgfpathlineto{\pgfqpoint{2.898120in}{1.445351in}}%
\pgfpathlineto{\pgfqpoint{2.899953in}{1.768688in}}%
\pgfpathlineto{\pgfqpoint{2.901787in}{1.678306in}}%
\pgfpathlineto{\pgfqpoint{2.903622in}{1.415153in}}%
\pgfpathlineto{\pgfqpoint{2.905456in}{1.437836in}}%
\pgfpathlineto{\pgfqpoint{2.907289in}{1.567324in}}%
\pgfpathlineto{\pgfqpoint{2.909124in}{1.300319in}}%
\pgfpathlineto{\pgfqpoint{2.910958in}{1.322149in}}%
\pgfpathlineto{\pgfqpoint{2.912791in}{1.422705in}}%
\pgfpathlineto{\pgfqpoint{2.914626in}{1.453581in}}%
\pgfpathlineto{\pgfqpoint{2.916459in}{1.364818in}}%
\pgfpathlineto{\pgfqpoint{2.918292in}{1.373186in}}%
\pgfpathlineto{\pgfqpoint{2.920127in}{1.415918in}}%
\pgfpathlineto{\pgfqpoint{2.921961in}{1.604322in}}%
\pgfpathlineto{\pgfqpoint{2.923795in}{1.547401in}}%
\pgfpathlineto{\pgfqpoint{2.925630in}{1.266821in}}%
\pgfpathlineto{\pgfqpoint{2.927464in}{1.285514in}}%
\pgfpathlineto{\pgfqpoint{2.929298in}{1.337581in}}%
\pgfpathlineto{\pgfqpoint{2.931134in}{1.538995in}}%
\pgfpathlineto{\pgfqpoint{2.932967in}{1.204153in}}%
\pgfpathlineto{\pgfqpoint{2.934799in}{1.228656in}}%
\pgfpathlineto{\pgfqpoint{2.936633in}{1.316215in}}%
\pgfpathlineto{\pgfqpoint{2.938468in}{1.276519in}}%
\pgfpathlineto{\pgfqpoint{2.940301in}{1.208118in}}%
\pgfpathlineto{\pgfqpoint{2.943970in}{1.382859in}}%
\pgfpathlineto{\pgfqpoint{2.945803in}{1.625036in}}%
\pgfpathlineto{\pgfqpoint{2.947636in}{1.333353in}}%
\pgfpathlineto{\pgfqpoint{2.949470in}{1.384980in}}%
\pgfpathlineto{\pgfqpoint{2.951303in}{1.399545in}}%
\pgfpathlineto{\pgfqpoint{2.954971in}{1.128727in}}%
\pgfpathlineto{\pgfqpoint{2.956806in}{1.183352in}}%
\pgfpathlineto{\pgfqpoint{2.958640in}{1.129166in}}%
\pgfpathlineto{\pgfqpoint{2.960474in}{1.364354in}}%
\pgfpathlineto{\pgfqpoint{2.962307in}{1.189048in}}%
\pgfpathlineto{\pgfqpoint{2.964141in}{1.322500in}}%
\pgfpathlineto{\pgfqpoint{2.965976in}{1.118150in}}%
\pgfpathlineto{\pgfqpoint{2.967811in}{1.460457in}}%
\pgfpathlineto{\pgfqpoint{2.969644in}{1.254187in}}%
\pgfpathlineto{\pgfqpoint{2.971478in}{1.235443in}}%
\pgfpathlineto{\pgfqpoint{2.973311in}{1.271764in}}%
\pgfpathlineto{\pgfqpoint{2.975146in}{1.344807in}}%
\pgfpathlineto{\pgfqpoint{2.976981in}{1.275026in}}%
\pgfpathlineto{\pgfqpoint{2.978814in}{1.300319in}}%
\pgfpathlineto{\pgfqpoint{2.980649in}{1.169740in}}%
\pgfpathlineto{\pgfqpoint{2.982484in}{1.399345in}}%
\pgfpathlineto{\pgfqpoint{2.984318in}{1.235506in}}%
\pgfpathlineto{\pgfqpoint{2.986151in}{1.231102in}}%
\pgfpathlineto{\pgfqpoint{2.987986in}{1.174796in}}%
\pgfpathlineto{\pgfqpoint{2.989819in}{1.318624in}}%
\pgfpathlineto{\pgfqpoint{2.991653in}{1.280960in}}%
\pgfpathlineto{\pgfqpoint{2.993487in}{1.282491in}}%
\pgfpathlineto{\pgfqpoint{2.995321in}{1.362346in}}%
\pgfpathlineto{\pgfqpoint{2.997154in}{1.356450in}}%
\pgfpathlineto{\pgfqpoint{2.998989in}{1.258867in}}%
\pgfpathlineto{\pgfqpoint{3.000823in}{1.441976in}}%
\pgfpathlineto{\pgfqpoint{3.002655in}{1.174093in}}%
\pgfpathlineto{\pgfqpoint{3.004488in}{1.242845in}}%
\pgfpathlineto{\pgfqpoint{3.006324in}{1.427021in}}%
\pgfpathlineto{\pgfqpoint{3.008156in}{1.451235in}}%
\pgfpathlineto{\pgfqpoint{3.015492in}{1.220689in}}%
\pgfpathlineto{\pgfqpoint{3.017327in}{1.163228in}}%
\pgfpathlineto{\pgfqpoint{3.019162in}{1.466955in}}%
\pgfpathlineto{\pgfqpoint{3.024663in}{1.341257in}}%
\pgfpathlineto{\pgfqpoint{3.028332in}{1.136455in}}%
\pgfpathlineto{\pgfqpoint{3.033835in}{1.361355in}}%
\pgfpathlineto{\pgfqpoint{3.035669in}{1.186024in}}%
\pgfpathlineto{\pgfqpoint{3.039337in}{2.118459in}}%
\pgfpathlineto{\pgfqpoint{3.041170in}{1.711001in}}%
\pgfpathlineto{\pgfqpoint{3.043003in}{1.868178in}}%
\pgfpathlineto{\pgfqpoint{3.046673in}{2.457654in}}%
\pgfpathlineto{\pgfqpoint{3.050343in}{1.904524in}}%
\pgfpathlineto{\pgfqpoint{3.052175in}{2.014766in}}%
\pgfpathlineto{\pgfqpoint{3.054009in}{2.029520in}}%
\pgfpathlineto{\pgfqpoint{3.055844in}{2.119400in}}%
\pgfpathlineto{\pgfqpoint{3.057678in}{2.118283in}}%
\pgfpathlineto{\pgfqpoint{3.059511in}{2.213131in}}%
\pgfpathlineto{\pgfqpoint{3.061345in}{2.242426in}}%
\pgfpathlineto{\pgfqpoint{3.063179in}{1.405191in}}%
\pgfpathlineto{\pgfqpoint{3.065013in}{1.188433in}}%
\pgfpathlineto{\pgfqpoint{3.066847in}{1.467896in}}%
\pgfpathlineto{\pgfqpoint{3.068682in}{1.312137in}}%
\pgfpathlineto{\pgfqpoint{3.070517in}{1.404363in}}%
\pgfpathlineto{\pgfqpoint{3.072352in}{1.429606in}}%
\pgfpathlineto{\pgfqpoint{3.076019in}{1.215909in}}%
\pgfpathlineto{\pgfqpoint{3.077854in}{1.211794in}}%
\pgfpathlineto{\pgfqpoint{3.079689in}{1.255454in}}%
\pgfpathlineto{\pgfqpoint{3.081521in}{1.115177in}}%
\pgfpathlineto{\pgfqpoint{3.085189in}{1.337693in}}%
\pgfpathlineto{\pgfqpoint{3.087023in}{1.201393in}}%
\pgfpathlineto{\pgfqpoint{3.088855in}{1.352448in}}%
\pgfpathlineto{\pgfqpoint{3.090689in}{1.256069in}}%
\pgfpathlineto{\pgfqpoint{3.092523in}{1.229772in}}%
\pgfpathlineto{\pgfqpoint{3.094357in}{1.420585in}}%
\pgfpathlineto{\pgfqpoint{3.096190in}{1.273671in}}%
\pgfpathlineto{\pgfqpoint{3.098024in}{1.340642in}}%
\pgfpathlineto{\pgfqpoint{3.099858in}{1.320455in}}%
\pgfpathlineto{\pgfqpoint{3.101693in}{1.257888in}}%
\pgfpathlineto{\pgfqpoint{3.103527in}{1.393937in}}%
\pgfpathlineto{\pgfqpoint{3.107195in}{2.219806in}}%
\pgfpathlineto{\pgfqpoint{3.109029in}{2.172871in}}%
\pgfpathlineto{\pgfqpoint{3.110862in}{2.395363in}}%
\pgfpathlineto{\pgfqpoint{3.112696in}{2.438998in}}%
\pgfpathlineto{\pgfqpoint{3.114530in}{2.311856in}}%
\pgfpathlineto{\pgfqpoint{3.116365in}{2.079905in}}%
\pgfpathlineto{\pgfqpoint{3.118199in}{2.026697in}}%
\pgfpathlineto{\pgfqpoint{3.120033in}{2.254408in}}%
\pgfpathlineto{\pgfqpoint{3.121867in}{2.274306in}}%
\pgfpathlineto{\pgfqpoint{3.123701in}{2.375615in}}%
\pgfpathlineto{\pgfqpoint{3.125535in}{2.372704in}}%
\pgfpathlineto{\pgfqpoint{3.127369in}{2.071474in}}%
\pgfpathlineto{\pgfqpoint{3.132873in}{2.440792in}}%
\pgfpathlineto{\pgfqpoint{3.134707in}{1.208708in}}%
\pgfpathlineto{\pgfqpoint{3.136542in}{1.386498in}}%
\pgfpathlineto{\pgfqpoint{3.138375in}{1.210878in}}%
\pgfpathlineto{\pgfqpoint{3.140209in}{1.424136in}}%
\pgfpathlineto{\pgfqpoint{3.142043in}{1.361932in}}%
\pgfpathlineto{\pgfqpoint{3.143877in}{1.405304in}}%
\pgfpathlineto{\pgfqpoint{3.145710in}{1.349926in}}%
\pgfpathlineto{\pgfqpoint{3.147544in}{1.442152in}}%
\pgfpathlineto{\pgfqpoint{3.149377in}{1.307959in}}%
\pgfpathlineto{\pgfqpoint{3.151211in}{1.360678in}}%
\pgfpathlineto{\pgfqpoint{3.153045in}{1.355509in}}%
\pgfpathlineto{\pgfqpoint{3.154878in}{1.408654in}}%
\pgfpathlineto{\pgfqpoint{3.156713in}{1.553373in}}%
\pgfpathlineto{\pgfqpoint{3.158548in}{1.235381in}}%
\pgfpathlineto{\pgfqpoint{3.164050in}{1.501507in}}%
\pgfpathlineto{\pgfqpoint{3.165884in}{1.799451in}}%
\pgfpathlineto{\pgfqpoint{3.167717in}{1.806150in}}%
\pgfpathlineto{\pgfqpoint{3.169551in}{1.397927in}}%
\pgfpathlineto{\pgfqpoint{3.171384in}{1.345485in}}%
\pgfpathlineto{\pgfqpoint{3.173218in}{1.433156in}}%
\pgfpathlineto{\pgfqpoint{3.175055in}{1.248755in}}%
\pgfpathlineto{\pgfqpoint{3.176889in}{1.566885in}}%
\pgfpathlineto{\pgfqpoint{3.178722in}{1.473504in}}%
\pgfpathlineto{\pgfqpoint{3.180569in}{1.252368in}}%
\pgfpathlineto{\pgfqpoint{3.182403in}{1.340604in}}%
\pgfpathlineto{\pgfqpoint{3.184237in}{1.269995in}}%
\pgfpathlineto{\pgfqpoint{3.186071in}{1.435603in}}%
\pgfpathlineto{\pgfqpoint{3.187905in}{2.051701in}}%
\pgfpathlineto{\pgfqpoint{3.189739in}{2.101421in}}%
\pgfpathlineto{\pgfqpoint{3.191573in}{2.276301in}}%
\pgfpathlineto{\pgfqpoint{3.193406in}{2.291431in}}%
\pgfpathlineto{\pgfqpoint{3.195241in}{2.229843in}}%
\pgfpathlineto{\pgfqpoint{3.197075in}{2.045504in}}%
\pgfpathlineto{\pgfqpoint{3.198909in}{2.223883in}}%
\pgfpathlineto{\pgfqpoint{3.202579in}{2.192067in}}%
\pgfpathlineto{\pgfqpoint{3.204428in}{2.160037in}}%
\pgfpathlineto{\pgfqpoint{3.206260in}{2.211990in}}%
\pgfpathlineto{\pgfqpoint{3.208094in}{2.057636in}}%
\pgfpathlineto{\pgfqpoint{3.209927in}{2.248210in}}%
\pgfpathlineto{\pgfqpoint{3.211761in}{1.515847in}}%
\pgfpathlineto{\pgfqpoint{3.213595in}{1.387526in}}%
\pgfpathlineto{\pgfqpoint{3.215429in}{1.343603in}}%
\pgfpathlineto{\pgfqpoint{3.217263in}{1.178710in}}%
\pgfpathlineto{\pgfqpoint{3.222765in}{1.463869in}}%
\pgfpathlineto{\pgfqpoint{3.224599in}{1.310230in}}%
\pgfpathlineto{\pgfqpoint{3.226432in}{1.311610in}}%
\pgfpathlineto{\pgfqpoint{3.228266in}{1.445765in}}%
\pgfpathlineto{\pgfqpoint{3.230099in}{1.192021in}}%
\pgfpathlineto{\pgfqpoint{3.231933in}{1.092619in}}%
\pgfpathlineto{\pgfqpoint{3.235603in}{1.318097in}}%
\pgfpathlineto{\pgfqpoint{3.237436in}{1.130169in}}%
\pgfpathlineto{\pgfqpoint{3.239271in}{1.210702in}}%
\pgfpathlineto{\pgfqpoint{3.241104in}{1.145237in}}%
\pgfpathlineto{\pgfqpoint{3.242938in}{1.394088in}}%
\pgfpathlineto{\pgfqpoint{3.244772in}{1.172562in}}%
\pgfpathlineto{\pgfqpoint{3.246606in}{1.132252in}}%
\pgfpathlineto{\pgfqpoint{3.248439in}{1.143267in}}%
\pgfpathlineto{\pgfqpoint{3.250273in}{1.583220in}}%
\pgfpathlineto{\pgfqpoint{3.252109in}{1.613481in}}%
\pgfpathlineto{\pgfqpoint{3.253942in}{1.321685in}}%
\pgfpathlineto{\pgfqpoint{3.255778in}{1.368870in}}%
\pgfpathlineto{\pgfqpoint{3.257612in}{1.342511in}}%
\pgfpathlineto{\pgfqpoint{3.259444in}{1.284047in}}%
\pgfpathlineto{\pgfqpoint{3.261278in}{1.122993in}}%
\pgfpathlineto{\pgfqpoint{3.263112in}{1.341219in}}%
\pgfpathlineto{\pgfqpoint{3.264945in}{2.062253in}}%
\pgfpathlineto{\pgfqpoint{3.266779in}{1.943743in}}%
\pgfpathlineto{\pgfqpoint{3.268613in}{2.065251in}}%
\pgfpathlineto{\pgfqpoint{3.270446in}{1.999108in}}%
\pgfpathlineto{\pgfqpoint{3.272279in}{2.039971in}}%
\pgfpathlineto{\pgfqpoint{3.274116in}{2.165946in}}%
\pgfpathlineto{\pgfqpoint{3.275950in}{2.197474in}}%
\pgfpathlineto{\pgfqpoint{3.277784in}{1.971633in}}%
\pgfpathlineto{\pgfqpoint{3.279618in}{1.219485in}}%
\pgfpathlineto{\pgfqpoint{3.283286in}{1.114951in}}%
\pgfpathlineto{\pgfqpoint{3.285121in}{1.140156in}}%
\pgfpathlineto{\pgfqpoint{3.286954in}{1.291449in}}%
\pgfpathlineto{\pgfqpoint{3.290622in}{1.146115in}}%
\pgfpathlineto{\pgfqpoint{3.292455in}{1.202359in}}%
\pgfpathlineto{\pgfqpoint{3.294288in}{1.189550in}}%
\pgfpathlineto{\pgfqpoint{3.296121in}{1.245204in}}%
\pgfpathlineto{\pgfqpoint{3.297955in}{1.365169in}}%
\pgfpathlineto{\pgfqpoint{3.299789in}{1.406948in}}%
\pgfpathlineto{\pgfqpoint{3.301622in}{1.333942in}}%
\pgfpathlineto{\pgfqpoint{3.303456in}{1.128074in}}%
\pgfpathlineto{\pgfqpoint{3.307123in}{1.334231in}}%
\pgfpathlineto{\pgfqpoint{3.308957in}{1.215344in}}%
\pgfpathlineto{\pgfqpoint{3.312625in}{1.255040in}}%
\pgfpathlineto{\pgfqpoint{3.314460in}{1.193427in}}%
\pgfpathlineto{\pgfqpoint{3.316293in}{1.323215in}}%
\pgfpathlineto{\pgfqpoint{3.318126in}{1.115302in}}%
\pgfpathlineto{\pgfqpoint{3.323628in}{1.260159in}}%
\pgfpathlineto{\pgfqpoint{3.325463in}{2.005369in}}%
\pgfpathlineto{\pgfqpoint{3.327297in}{2.055603in}}%
\pgfpathlineto{\pgfqpoint{3.329130in}{2.170763in}}%
\pgfpathlineto{\pgfqpoint{3.330965in}{2.090895in}}%
\pgfpathlineto{\pgfqpoint{3.332797in}{2.090657in}}%
\pgfpathlineto{\pgfqpoint{3.334631in}{2.024966in}}%
\pgfpathlineto{\pgfqpoint{3.336466in}{2.023372in}}%
\pgfpathlineto{\pgfqpoint{3.338299in}{2.009471in}}%
\pgfpathlineto{\pgfqpoint{3.340132in}{2.154215in}}%
\pgfpathlineto{\pgfqpoint{3.343803in}{2.003600in}}%
\pgfpathlineto{\pgfqpoint{3.345637in}{2.074146in}}%
\pgfpathlineto{\pgfqpoint{3.349319in}{2.099677in}}%
\pgfpathlineto{\pgfqpoint{3.351153in}{2.242803in}}%
\pgfpathlineto{\pgfqpoint{3.352986in}{2.098297in}}%
\pgfpathlineto{\pgfqpoint{3.354819in}{2.155984in}}%
\pgfpathlineto{\pgfqpoint{3.356653in}{1.976952in}}%
\pgfpathlineto{\pgfqpoint{3.358488in}{1.399784in}}%
\pgfpathlineto{\pgfqpoint{3.360321in}{1.198508in}}%
\pgfpathlineto{\pgfqpoint{3.362156in}{1.255655in}}%
\pgfpathlineto{\pgfqpoint{3.363989in}{1.180943in}}%
\pgfpathlineto{\pgfqpoint{3.365824in}{1.160581in}}%
\pgfpathlineto{\pgfqpoint{3.367658in}{1.330818in}}%
\pgfpathlineto{\pgfqpoint{3.369491in}{1.255780in}}%
\pgfpathlineto{\pgfqpoint{3.371324in}{1.307081in}}%
\pgfpathlineto{\pgfqpoint{3.373159in}{1.420610in}}%
\pgfpathlineto{\pgfqpoint{3.374993in}{1.362698in}}%
\pgfpathlineto{\pgfqpoint{3.376827in}{1.266859in}}%
\pgfpathlineto{\pgfqpoint{3.378660in}{1.417411in}}%
\pgfpathlineto{\pgfqpoint{3.380495in}{1.195748in}}%
\pgfpathlineto{\pgfqpoint{3.382330in}{1.266683in}}%
\pgfpathlineto{\pgfqpoint{3.384164in}{1.226071in}}%
\pgfpathlineto{\pgfqpoint{3.386001in}{1.245932in}}%
\pgfpathlineto{\pgfqpoint{3.387834in}{1.332788in}}%
\pgfpathlineto{\pgfqpoint{3.389668in}{1.263421in}}%
\pgfpathlineto{\pgfqpoint{3.391503in}{1.124825in}}%
\pgfpathlineto{\pgfqpoint{3.393338in}{1.261037in}}%
\pgfpathlineto{\pgfqpoint{3.395171in}{1.265391in}}%
\pgfpathlineto{\pgfqpoint{3.397006in}{1.099670in}}%
\pgfpathlineto{\pgfqpoint{3.398840in}{1.234327in}}%
\pgfpathlineto{\pgfqpoint{3.400674in}{1.287007in}}%
\pgfpathlineto{\pgfqpoint{3.402510in}{1.138952in}}%
\pgfpathlineto{\pgfqpoint{3.404343in}{1.092494in}}%
\pgfpathlineto{\pgfqpoint{3.406177in}{1.233599in}}%
\pgfpathlineto{\pgfqpoint{3.408012in}{1.111513in}}%
\pgfpathlineto{\pgfqpoint{3.409845in}{1.350741in}}%
\pgfpathlineto{\pgfqpoint{3.411678in}{1.161785in}}%
\pgfpathlineto{\pgfqpoint{3.413513in}{1.243197in}}%
\pgfpathlineto{\pgfqpoint{3.415347in}{1.195045in}}%
\pgfpathlineto{\pgfqpoint{3.417181in}{1.086647in}}%
\pgfpathlineto{\pgfqpoint{3.419015in}{1.252493in}}%
\pgfpathlineto{\pgfqpoint{3.420849in}{1.223073in}}%
\pgfpathlineto{\pgfqpoint{3.422686in}{1.018271in}}%
\pgfpathlineto{\pgfqpoint{3.424521in}{1.031031in}}%
\pgfpathlineto{\pgfqpoint{3.426354in}{1.276168in}}%
\pgfpathlineto{\pgfqpoint{3.428187in}{1.104488in}}%
\pgfpathlineto{\pgfqpoint{3.430022in}{1.161020in}}%
\pgfpathlineto{\pgfqpoint{3.431855in}{1.147972in}}%
\pgfpathlineto{\pgfqpoint{3.435523in}{1.417499in}}%
\pgfpathlineto{\pgfqpoint{3.437356in}{1.311723in}}%
\pgfpathlineto{\pgfqpoint{3.439189in}{1.349951in}}%
\pgfpathlineto{\pgfqpoint{3.441023in}{1.173089in}}%
\pgfpathlineto{\pgfqpoint{3.442856in}{1.293331in}}%
\pgfpathlineto{\pgfqpoint{3.444691in}{1.658797in}}%
\pgfpathlineto{\pgfqpoint{3.446524in}{2.249101in}}%
\pgfpathlineto{\pgfqpoint{3.448382in}{2.235049in}}%
\pgfpathlineto{\pgfqpoint{3.450217in}{2.271746in}}%
\pgfpathlineto{\pgfqpoint{3.452050in}{2.094069in}}%
\pgfpathlineto{\pgfqpoint{3.455718in}{2.261961in}}%
\pgfpathlineto{\pgfqpoint{3.457551in}{2.291318in}}%
\pgfpathlineto{\pgfqpoint{3.461218in}{2.217422in}}%
\pgfpathlineto{\pgfqpoint{3.463056in}{2.192267in}}%
\pgfpathlineto{\pgfqpoint{3.464890in}{2.139825in}}%
\pgfpathlineto{\pgfqpoint{3.466727in}{2.141054in}}%
\pgfpathlineto{\pgfqpoint{3.468561in}{2.050522in}}%
\pgfpathlineto{\pgfqpoint{3.470395in}{2.179872in}}%
\pgfpathlineto{\pgfqpoint{3.474066in}{2.010588in}}%
\pgfpathlineto{\pgfqpoint{3.475899in}{2.344062in}}%
\pgfpathlineto{\pgfqpoint{3.477733in}{2.143965in}}%
\pgfpathlineto{\pgfqpoint{3.479567in}{2.309623in}}%
\pgfpathlineto{\pgfqpoint{3.481400in}{2.241398in}}%
\pgfpathlineto{\pgfqpoint{3.483235in}{2.483097in}}%
\pgfpathlineto{\pgfqpoint{3.486903in}{2.232051in}}%
\pgfpathlineto{\pgfqpoint{3.488739in}{2.200937in}}%
\pgfpathlineto{\pgfqpoint{3.490574in}{2.377409in}}%
\pgfpathlineto{\pgfqpoint{3.492407in}{2.192418in}}%
\pgfpathlineto{\pgfqpoint{3.494242in}{2.152484in}}%
\pgfpathlineto{\pgfqpoint{3.496076in}{2.186421in}}%
\pgfpathlineto{\pgfqpoint{3.497910in}{2.158807in}}%
\pgfpathlineto{\pgfqpoint{3.499744in}{2.431533in}}%
\pgfpathlineto{\pgfqpoint{3.501578in}{2.351815in}}%
\pgfpathlineto{\pgfqpoint{3.503411in}{2.183046in}}%
\pgfpathlineto{\pgfqpoint{3.505250in}{2.229730in}}%
\pgfpathlineto{\pgfqpoint{3.507084in}{2.382290in}}%
\pgfpathlineto{\pgfqpoint{3.508918in}{2.074974in}}%
\pgfpathlineto{\pgfqpoint{3.510752in}{2.204751in}}%
\pgfpathlineto{\pgfqpoint{3.512586in}{2.241423in}}%
\pgfpathlineto{\pgfqpoint{3.516251in}{1.332085in}}%
\pgfpathlineto{\pgfqpoint{3.518086in}{1.356174in}}%
\pgfpathlineto{\pgfqpoint{3.519919in}{1.276469in}}%
\pgfpathlineto{\pgfqpoint{3.521752in}{1.257487in}}%
\pgfpathlineto{\pgfqpoint{3.523588in}{1.351394in}}%
\pgfpathlineto{\pgfqpoint{3.525421in}{1.387326in}}%
\pgfpathlineto{\pgfqpoint{3.527256in}{1.190102in}}%
\pgfpathlineto{\pgfqpoint{3.529091in}{1.186614in}}%
\pgfpathlineto{\pgfqpoint{3.530924in}{1.361707in}}%
\pgfpathlineto{\pgfqpoint{3.532759in}{1.217929in}}%
\pgfpathlineto{\pgfqpoint{3.534593in}{1.300557in}}%
\pgfpathlineto{\pgfqpoint{3.536425in}{1.128701in}}%
\pgfpathlineto{\pgfqpoint{3.538258in}{1.259067in}}%
\pgfpathlineto{\pgfqpoint{3.540092in}{1.187379in}}%
\pgfpathlineto{\pgfqpoint{3.541927in}{1.243761in}}%
\pgfpathlineto{\pgfqpoint{3.543760in}{1.079860in}}%
\pgfpathlineto{\pgfqpoint{3.545594in}{1.216173in}}%
\pgfpathlineto{\pgfqpoint{3.547428in}{1.206299in}}%
\pgfpathlineto{\pgfqpoint{3.549263in}{1.213262in}}%
\pgfpathlineto{\pgfqpoint{3.551099in}{1.177380in}}%
\pgfpathlineto{\pgfqpoint{3.554765in}{1.013893in}}%
\pgfpathlineto{\pgfqpoint{3.556600in}{1.177531in}}%
\pgfpathlineto{\pgfqpoint{3.558433in}{0.895195in}}%
\pgfpathlineto{\pgfqpoint{3.562100in}{1.285339in}}%
\pgfpathlineto{\pgfqpoint{3.563934in}{1.035459in}}%
\pgfpathlineto{\pgfqpoint{3.565768in}{1.001083in}}%
\pgfpathlineto{\pgfqpoint{3.567603in}{1.187204in}}%
\pgfpathlineto{\pgfqpoint{3.569436in}{0.962567in}}%
\pgfpathlineto{\pgfqpoint{3.573103in}{1.134335in}}%
\pgfpathlineto{\pgfqpoint{3.576774in}{1.013140in}}%
\pgfpathlineto{\pgfqpoint{3.578608in}{1.080186in}}%
\pgfpathlineto{\pgfqpoint{3.580443in}{1.216173in}}%
\pgfpathlineto{\pgfqpoint{3.584114in}{1.005324in}}%
\pgfpathlineto{\pgfqpoint{3.585947in}{1.240963in}}%
\pgfpathlineto{\pgfqpoint{3.587780in}{1.265880in}}%
\pgfpathlineto{\pgfqpoint{3.589616in}{1.274901in}}%
\pgfpathlineto{\pgfqpoint{3.591449in}{1.469063in}}%
\pgfpathlineto{\pgfqpoint{3.595117in}{1.342486in}}%
\pgfpathlineto{\pgfqpoint{3.602455in}{0.984284in}}%
\pgfpathlineto{\pgfqpoint{3.604289in}{1.122403in}}%
\pgfpathlineto{\pgfqpoint{3.606124in}{1.041456in}}%
\pgfpathlineto{\pgfqpoint{3.607958in}{1.108302in}}%
\pgfpathlineto{\pgfqpoint{3.609790in}{1.069396in}}%
\pgfpathlineto{\pgfqpoint{3.611625in}{1.107750in}}%
\pgfpathlineto{\pgfqpoint{3.613459in}{1.110685in}}%
\pgfpathlineto{\pgfqpoint{3.615293in}{1.064265in}}%
\pgfpathlineto{\pgfqpoint{3.617126in}{1.049486in}}%
\pgfpathlineto{\pgfqpoint{3.618961in}{1.066548in}}%
\pgfpathlineto{\pgfqpoint{3.620794in}{1.072633in}}%
\pgfpathlineto{\pgfqpoint{3.622628in}{1.136191in}}%
\pgfpathlineto{\pgfqpoint{3.624462in}{1.414588in}}%
\pgfpathlineto{\pgfqpoint{3.626295in}{1.142364in}}%
\pgfpathlineto{\pgfqpoint{3.628128in}{1.125314in}}%
\pgfpathlineto{\pgfqpoint{3.631796in}{1.388154in}}%
\pgfpathlineto{\pgfqpoint{3.633629in}{1.422053in}}%
\pgfpathlineto{\pgfqpoint{3.635464in}{1.361757in}}%
\pgfpathlineto{\pgfqpoint{3.637298in}{1.382420in}}%
\pgfpathlineto{\pgfqpoint{3.639130in}{1.091879in}}%
\pgfpathlineto{\pgfqpoint{3.640965in}{1.317507in}}%
\pgfpathlineto{\pgfqpoint{3.644632in}{1.169740in}}%
\pgfpathlineto{\pgfqpoint{3.646465in}{1.215232in}}%
\pgfpathlineto{\pgfqpoint{3.648299in}{1.142151in}}%
\pgfpathlineto{\pgfqpoint{3.650133in}{1.251552in}}%
\pgfpathlineto{\pgfqpoint{3.651968in}{1.235092in}}%
\pgfpathlineto{\pgfqpoint{3.653803in}{1.106784in}}%
\pgfpathlineto{\pgfqpoint{3.655636in}{1.222019in}}%
\pgfpathlineto{\pgfqpoint{3.657471in}{1.034293in}}%
\pgfpathlineto{\pgfqpoint{3.662969in}{1.721025in}}%
\pgfpathlineto{\pgfqpoint{3.664803in}{1.622213in}}%
\pgfpathlineto{\pgfqpoint{3.670303in}{2.439111in}}%
\pgfpathlineto{\pgfqpoint{3.672137in}{2.351728in}}%
\pgfpathlineto{\pgfqpoint{3.673971in}{2.581684in}}%
\pgfpathlineto{\pgfqpoint{3.675806in}{2.414985in}}%
\pgfpathlineto{\pgfqpoint{3.677640in}{2.567018in}}%
\pgfpathlineto{\pgfqpoint{3.679474in}{2.565612in}}%
\pgfpathlineto{\pgfqpoint{3.681309in}{2.517687in}}%
\pgfpathlineto{\pgfqpoint{3.683143in}{2.542565in}}%
\pgfpathlineto{\pgfqpoint{3.684977in}{2.494790in}}%
\pgfpathlineto{\pgfqpoint{3.686813in}{2.498090in}}%
\pgfpathlineto{\pgfqpoint{3.688646in}{2.614793in}}%
\pgfpathlineto{\pgfqpoint{3.690479in}{2.633926in}}%
\pgfpathlineto{\pgfqpoint{3.692315in}{2.509481in}}%
\pgfpathlineto{\pgfqpoint{3.694149in}{2.495969in}}%
\pgfpathlineto{\pgfqpoint{3.695982in}{2.643448in}}%
\pgfpathlineto{\pgfqpoint{3.697816in}{2.486359in}}%
\pgfpathlineto{\pgfqpoint{3.699648in}{2.614994in}}%
\pgfpathlineto{\pgfqpoint{3.701483in}{2.419689in}}%
\pgfpathlineto{\pgfqpoint{3.703318in}{2.683457in}}%
\pgfpathlineto{\pgfqpoint{3.705152in}{2.482545in}}%
\pgfpathlineto{\pgfqpoint{3.706986in}{2.576778in}}%
\pgfpathlineto{\pgfqpoint{3.708820in}{2.414320in}}%
\pgfpathlineto{\pgfqpoint{3.710655in}{2.377786in}}%
\pgfpathlineto{\pgfqpoint{3.712488in}{2.541361in}}%
\pgfpathlineto{\pgfqpoint{3.714321in}{2.452134in}}%
\pgfpathlineto{\pgfqpoint{3.716156in}{2.436740in}}%
\pgfpathlineto{\pgfqpoint{3.717990in}{2.389453in}}%
\pgfpathlineto{\pgfqpoint{3.719822in}{2.598069in}}%
\pgfpathlineto{\pgfqpoint{3.721657in}{2.423541in}}%
\pgfpathlineto{\pgfqpoint{3.723491in}{2.546593in}}%
\pgfpathlineto{\pgfqpoint{3.725324in}{2.422976in}}%
\pgfpathlineto{\pgfqpoint{3.727159in}{2.712878in}}%
\pgfpathlineto{\pgfqpoint{3.728993in}{2.478957in}}%
\pgfpathlineto{\pgfqpoint{3.732661in}{2.370534in}}%
\pgfpathlineto{\pgfqpoint{3.734494in}{2.388011in}}%
\pgfpathlineto{\pgfqpoint{3.736327in}{2.574219in}}%
\pgfpathlineto{\pgfqpoint{3.738163in}{2.532077in}}%
\pgfpathlineto{\pgfqpoint{3.739999in}{2.572337in}}%
\pgfpathlineto{\pgfqpoint{3.741832in}{2.541750in}}%
\pgfpathlineto{\pgfqpoint{3.745501in}{2.604192in}}%
\pgfpathlineto{\pgfqpoint{3.747335in}{2.498315in}}%
\pgfpathlineto{\pgfqpoint{3.749169in}{2.535866in}}%
\pgfpathlineto{\pgfqpoint{3.751003in}{2.519970in}}%
\pgfpathlineto{\pgfqpoint{3.752838in}{2.389717in}}%
\pgfpathlineto{\pgfqpoint{3.754673in}{2.579237in}}%
\pgfpathlineto{\pgfqpoint{3.758339in}{2.384899in}}%
\pgfpathlineto{\pgfqpoint{3.760174in}{2.417218in}}%
\pgfpathlineto{\pgfqpoint{3.762009in}{2.410430in}}%
\pgfpathlineto{\pgfqpoint{3.763842in}{2.305006in}}%
\pgfpathlineto{\pgfqpoint{3.765677in}{2.378902in}}%
\pgfpathlineto{\pgfqpoint{3.767511in}{2.337688in}}%
\pgfpathlineto{\pgfqpoint{3.769345in}{2.527585in}}%
\pgfpathlineto{\pgfqpoint{3.771179in}{2.469961in}}%
\pgfpathlineto{\pgfqpoint{3.773011in}{2.506433in}}%
\pgfpathlineto{\pgfqpoint{3.774845in}{2.394484in}}%
\pgfpathlineto{\pgfqpoint{3.776679in}{2.403267in}}%
\pgfpathlineto{\pgfqpoint{3.778514in}{2.543720in}}%
\pgfpathlineto{\pgfqpoint{3.780349in}{2.486974in}}%
\pgfpathlineto{\pgfqpoint{3.782186in}{2.637828in}}%
\pgfpathlineto{\pgfqpoint{3.784019in}{1.627557in}}%
\pgfpathlineto{\pgfqpoint{3.787687in}{1.566232in}}%
\pgfpathlineto{\pgfqpoint{3.789522in}{1.507027in}}%
\pgfpathlineto{\pgfqpoint{3.791355in}{1.594674in}}%
\pgfpathlineto{\pgfqpoint{3.793189in}{1.462138in}}%
\pgfpathlineto{\pgfqpoint{3.795023in}{1.574588in}}%
\pgfpathlineto{\pgfqpoint{3.796856in}{1.554803in}}%
\pgfpathlineto{\pgfqpoint{3.798692in}{1.590835in}}%
\pgfpathlineto{\pgfqpoint{3.802362in}{1.383035in}}%
\pgfpathlineto{\pgfqpoint{3.804196in}{1.489312in}}%
\pgfpathlineto{\pgfqpoint{3.806032in}{1.387827in}}%
\pgfpathlineto{\pgfqpoint{3.807865in}{1.401816in}}%
\pgfpathlineto{\pgfqpoint{3.809697in}{1.539760in}}%
\pgfpathlineto{\pgfqpoint{3.811531in}{1.429694in}}%
\pgfpathlineto{\pgfqpoint{3.813365in}{1.393611in}}%
\pgfpathlineto{\pgfqpoint{3.815198in}{1.600583in}}%
\pgfpathlineto{\pgfqpoint{3.817032in}{1.399081in}}%
\pgfpathlineto{\pgfqpoint{3.818866in}{1.481057in}}%
\pgfpathlineto{\pgfqpoint{3.820699in}{1.312049in}}%
\pgfpathlineto{\pgfqpoint{3.822534in}{1.432717in}}%
\pgfpathlineto{\pgfqpoint{3.824367in}{1.416621in}}%
\pgfpathlineto{\pgfqpoint{3.826201in}{1.426432in}}%
\pgfpathlineto{\pgfqpoint{3.828036in}{1.360640in}}%
\pgfpathlineto{\pgfqpoint{3.829870in}{1.483842in}}%
\pgfpathlineto{\pgfqpoint{3.831703in}{1.433922in}}%
\pgfpathlineto{\pgfqpoint{3.833538in}{1.441299in}}%
\pgfpathlineto{\pgfqpoint{3.835371in}{1.568792in}}%
\pgfpathlineto{\pgfqpoint{3.837205in}{1.524742in}}%
\pgfpathlineto{\pgfqpoint{3.839039in}{1.433608in}}%
\pgfpathlineto{\pgfqpoint{3.840875in}{1.514643in}}%
\pgfpathlineto{\pgfqpoint{3.842708in}{1.358470in}}%
\pgfpathlineto{\pgfqpoint{3.844542in}{1.567211in}}%
\pgfpathlineto{\pgfqpoint{3.848208in}{1.461636in}}%
\pgfpathlineto{\pgfqpoint{3.850044in}{1.552921in}}%
\pgfpathlineto{\pgfqpoint{3.851891in}{1.584838in}}%
\pgfpathlineto{\pgfqpoint{3.853727in}{1.500328in}}%
\pgfpathlineto{\pgfqpoint{3.855561in}{1.478560in}}%
\pgfpathlineto{\pgfqpoint{3.857394in}{1.340730in}}%
\pgfpathlineto{\pgfqpoint{3.861063in}{1.523450in}}%
\pgfpathlineto{\pgfqpoint{3.862896in}{1.620067in}}%
\pgfpathlineto{\pgfqpoint{3.866563in}{1.322036in}}%
\pgfpathlineto{\pgfqpoint{3.868397in}{1.460983in}}%
\pgfpathlineto{\pgfqpoint{3.870230in}{1.477996in}}%
\pgfpathlineto{\pgfqpoint{3.872064in}{1.483491in}}%
\pgfpathlineto{\pgfqpoint{3.873898in}{1.588125in}}%
\pgfpathlineto{\pgfqpoint{3.877566in}{1.393022in}}%
\pgfpathlineto{\pgfqpoint{3.879399in}{1.462903in}}%
\pgfpathlineto{\pgfqpoint{3.881233in}{1.443921in}}%
\pgfpathlineto{\pgfqpoint{3.883067in}{1.365194in}}%
\pgfpathlineto{\pgfqpoint{3.884901in}{1.440508in}}%
\pgfpathlineto{\pgfqpoint{3.886734in}{1.327330in}}%
\pgfpathlineto{\pgfqpoint{3.888568in}{1.367553in}}%
\pgfpathlineto{\pgfqpoint{3.890402in}{1.552921in}}%
\pgfpathlineto{\pgfqpoint{3.894071in}{1.421639in}}%
\pgfpathlineto{\pgfqpoint{3.895904in}{1.447007in}}%
\pgfpathlineto{\pgfqpoint{3.897738in}{1.542056in}}%
\pgfpathlineto{\pgfqpoint{3.899574in}{1.356676in}}%
\pgfpathlineto{\pgfqpoint{3.901409in}{1.340228in}}%
\pgfpathlineto{\pgfqpoint{3.905078in}{1.476440in}}%
\pgfpathlineto{\pgfqpoint{3.906913in}{1.320919in}}%
\pgfpathlineto{\pgfqpoint{3.908748in}{1.358382in}}%
\pgfpathlineto{\pgfqpoint{3.910582in}{1.442829in}}%
\pgfpathlineto{\pgfqpoint{3.912415in}{1.597146in}}%
\pgfpathlineto{\pgfqpoint{3.914249in}{1.323128in}}%
\pgfpathlineto{\pgfqpoint{3.916085in}{1.429016in}}%
\pgfpathlineto{\pgfqpoint{3.917917in}{1.452264in}}%
\pgfpathlineto{\pgfqpoint{3.919750in}{1.408955in}}%
\pgfpathlineto{\pgfqpoint{3.923418in}{1.530802in}}%
\pgfpathlineto{\pgfqpoint{3.925252in}{1.457696in}}%
\pgfpathlineto{\pgfqpoint{3.927086in}{1.433483in}}%
\pgfpathlineto{\pgfqpoint{3.928919in}{1.599090in}}%
\pgfpathlineto{\pgfqpoint{3.932587in}{1.333528in}}%
\pgfpathlineto{\pgfqpoint{3.934422in}{1.484733in}}%
\pgfpathlineto{\pgfqpoint{3.936255in}{1.262392in}}%
\pgfpathlineto{\pgfqpoint{3.938088in}{1.510051in}}%
\pgfpathlineto{\pgfqpoint{3.939922in}{1.434160in}}%
\pgfpathlineto{\pgfqpoint{3.941755in}{1.457232in}}%
\pgfpathlineto{\pgfqpoint{3.943588in}{1.444448in}}%
\pgfpathlineto{\pgfqpoint{3.945425in}{1.465538in}}%
\pgfpathlineto{\pgfqpoint{3.947259in}{1.359210in}}%
\pgfpathlineto{\pgfqpoint{3.950929in}{1.514467in}}%
\pgfpathlineto{\pgfqpoint{3.952762in}{1.422843in}}%
\pgfpathlineto{\pgfqpoint{3.954594in}{1.427172in}}%
\pgfpathlineto{\pgfqpoint{3.956429in}{1.598262in}}%
\pgfpathlineto{\pgfqpoint{3.958263in}{1.519486in}}%
\pgfpathlineto{\pgfqpoint{3.960096in}{1.546196in}}%
\pgfpathlineto{\pgfqpoint{3.961930in}{1.552281in}}%
\pgfpathlineto{\pgfqpoint{3.963764in}{1.442478in}}%
\pgfpathlineto{\pgfqpoint{3.965599in}{1.600671in}}%
\pgfpathlineto{\pgfqpoint{3.967434in}{1.558504in}}%
\pgfpathlineto{\pgfqpoint{3.969267in}{1.594235in}}%
\pgfpathlineto{\pgfqpoint{3.971101in}{1.607371in}}%
\pgfpathlineto{\pgfqpoint{3.972938in}{1.422994in}}%
\pgfpathlineto{\pgfqpoint{3.974772in}{1.639225in}}%
\pgfpathlineto{\pgfqpoint{3.976604in}{1.614572in}}%
\pgfpathlineto{\pgfqpoint{3.980273in}{1.810905in}}%
\pgfpathlineto{\pgfqpoint{3.982106in}{1.556358in}}%
\pgfpathlineto{\pgfqpoint{3.983941in}{1.637080in}}%
\pgfpathlineto{\pgfqpoint{3.987610in}{1.467721in}}%
\pgfpathlineto{\pgfqpoint{3.989444in}{1.530150in}}%
\pgfpathlineto{\pgfqpoint{3.991276in}{1.634282in}}%
\pgfpathlineto{\pgfqpoint{3.993109in}{1.554125in}}%
\pgfpathlineto{\pgfqpoint{3.994943in}{1.790756in}}%
\pgfpathlineto{\pgfqpoint{3.998610in}{1.493603in}}%
\pgfpathlineto{\pgfqpoint{4.002279in}{1.611988in}}%
\pgfpathlineto{\pgfqpoint{4.004112in}{1.550286in}}%
\pgfpathlineto{\pgfqpoint{4.005945in}{1.533562in}}%
\pgfpathlineto{\pgfqpoint{4.007780in}{1.588564in}}%
\pgfpathlineto{\pgfqpoint{4.009614in}{1.503414in}}%
\pgfpathlineto{\pgfqpoint{4.011448in}{1.614723in}}%
\pgfpathlineto{\pgfqpoint{4.015116in}{1.443206in}}%
\pgfpathlineto{\pgfqpoint{4.016949in}{1.511531in}}%
\pgfpathlineto{\pgfqpoint{4.020618in}{1.328384in}}%
\pgfpathlineto{\pgfqpoint{4.022452in}{1.415065in}}%
\pgfpathlineto{\pgfqpoint{4.024289in}{1.661319in}}%
\pgfpathlineto{\pgfqpoint{4.026122in}{1.615225in}}%
\pgfpathlineto{\pgfqpoint{4.029789in}{1.419820in}}%
\pgfpathlineto{\pgfqpoint{4.031623in}{1.363789in}}%
\pgfpathlineto{\pgfqpoint{4.033457in}{1.433520in}}%
\pgfpathlineto{\pgfqpoint{4.035291in}{1.647016in}}%
\pgfpathlineto{\pgfqpoint{4.037125in}{1.504832in}}%
\pgfpathlineto{\pgfqpoint{4.040794in}{1.569933in}}%
\pgfpathlineto{\pgfqpoint{4.042629in}{1.631033in}}%
\pgfpathlineto{\pgfqpoint{4.044463in}{1.576909in}}%
\pgfpathlineto{\pgfqpoint{4.046298in}{1.622501in}}%
\pgfpathlineto{\pgfqpoint{4.048133in}{1.695845in}}%
\pgfpathlineto{\pgfqpoint{4.049965in}{1.907347in}}%
\pgfpathlineto{\pgfqpoint{4.051799in}{1.966815in}}%
\pgfpathlineto{\pgfqpoint{4.055467in}{1.556835in}}%
\pgfpathlineto{\pgfqpoint{4.057301in}{1.620657in}}%
\pgfpathlineto{\pgfqpoint{4.059136in}{1.768035in}}%
\pgfpathlineto{\pgfqpoint{4.060970in}{1.552369in}}%
\pgfpathlineto{\pgfqpoint{4.062804in}{1.476352in}}%
\pgfpathlineto{\pgfqpoint{4.064637in}{1.482676in}}%
\pgfpathlineto{\pgfqpoint{4.066470in}{1.557713in}}%
\pgfpathlineto{\pgfqpoint{4.068303in}{1.501620in}}%
\pgfpathlineto{\pgfqpoint{4.070138in}{1.586858in}}%
\pgfpathlineto{\pgfqpoint{4.071971in}{1.576545in}}%
\pgfpathlineto{\pgfqpoint{4.073806in}{1.447321in}}%
\pgfpathlineto{\pgfqpoint{4.075640in}{1.622413in}}%
\pgfpathlineto{\pgfqpoint{4.077474in}{1.547488in}}%
\pgfpathlineto{\pgfqpoint{4.081142in}{1.553774in}}%
\pgfpathlineto{\pgfqpoint{4.082976in}{1.225544in}}%
\pgfpathlineto{\pgfqpoint{4.084809in}{1.531003in}}%
\pgfpathlineto{\pgfqpoint{4.086642in}{1.455840in}}%
\pgfpathlineto{\pgfqpoint{4.088476in}{1.546259in}}%
\pgfpathlineto{\pgfqpoint{4.090309in}{1.365470in}}%
\pgfpathlineto{\pgfqpoint{4.092143in}{1.611599in}}%
\pgfpathlineto{\pgfqpoint{4.095810in}{1.299641in}}%
\pgfpathlineto{\pgfqpoint{4.097644in}{0.944463in}}%
\pgfpathlineto{\pgfqpoint{4.099480in}{1.302351in}}%
\pgfpathlineto{\pgfqpoint{4.101313in}{1.390199in}}%
\pgfpathlineto{\pgfqpoint{4.103146in}{1.319213in}}%
\pgfpathlineto{\pgfqpoint{4.104981in}{1.164307in}}%
\pgfpathlineto{\pgfqpoint{4.106815in}{1.356036in}}%
\pgfpathlineto{\pgfqpoint{4.108648in}{1.170279in}}%
\pgfpathlineto{\pgfqpoint{4.110482in}{1.220225in}}%
\pgfpathlineto{\pgfqpoint{4.112316in}{1.342248in}}%
\pgfpathlineto{\pgfqpoint{4.114150in}{1.191984in}}%
\pgfpathlineto{\pgfqpoint{4.115985in}{1.172475in}}%
\pgfpathlineto{\pgfqpoint{4.117818in}{1.168485in}}%
\pgfpathlineto{\pgfqpoint{4.123319in}{1.273408in}}%
\pgfpathlineto{\pgfqpoint{4.125154in}{1.267825in}}%
\pgfpathlineto{\pgfqpoint{4.126989in}{1.296116in}}%
\pgfpathlineto{\pgfqpoint{4.128822in}{1.529861in}}%
\pgfpathlineto{\pgfqpoint{4.130656in}{1.581513in}}%
\pgfpathlineto{\pgfqpoint{4.136157in}{1.188960in}}%
\pgfpathlineto{\pgfqpoint{4.139824in}{1.967367in}}%
\pgfpathlineto{\pgfqpoint{4.141657in}{2.147955in}}%
\pgfpathlineto{\pgfqpoint{4.143492in}{2.087696in}}%
\pgfpathlineto{\pgfqpoint{4.147158in}{2.331779in}}%
\pgfpathlineto{\pgfqpoint{4.148992in}{2.327137in}}%
\pgfpathlineto{\pgfqpoint{4.150827in}{2.360723in}}%
\pgfpathlineto{\pgfqpoint{4.154496in}{2.265875in}}%
\pgfpathlineto{\pgfqpoint{4.158163in}{2.399478in}}%
\pgfpathlineto{\pgfqpoint{4.159995in}{2.169321in}}%
\pgfpathlineto{\pgfqpoint{4.161830in}{2.220534in}}%
\pgfpathlineto{\pgfqpoint{4.163663in}{2.185191in}}%
\pgfpathlineto{\pgfqpoint{4.165497in}{2.347826in}}%
\pgfpathlineto{\pgfqpoint{4.167331in}{2.116314in}}%
\pgfpathlineto{\pgfqpoint{4.169164in}{2.238286in}}%
\pgfpathlineto{\pgfqpoint{4.170998in}{2.243894in}}%
\pgfpathlineto{\pgfqpoint{4.172834in}{2.218100in}}%
\pgfpathlineto{\pgfqpoint{4.174667in}{2.249628in}}%
\pgfpathlineto{\pgfqpoint{4.178336in}{1.245468in}}%
\pgfpathlineto{\pgfqpoint{4.180169in}{1.347517in}}%
\pgfpathlineto{\pgfqpoint{4.182002in}{1.072282in}}%
\pgfpathlineto{\pgfqpoint{4.183837in}{1.124674in}}%
\pgfpathlineto{\pgfqpoint{4.185671in}{1.104751in}}%
\pgfpathlineto{\pgfqpoint{4.187504in}{1.125113in}}%
\pgfpathlineto{\pgfqpoint{4.189339in}{1.255655in}}%
\pgfpathlineto{\pgfqpoint{4.191172in}{0.986981in}}%
\pgfpathlineto{\pgfqpoint{4.193005in}{1.023503in}}%
\pgfpathlineto{\pgfqpoint{4.194838in}{1.104136in}}%
\pgfpathlineto{\pgfqpoint{4.196673in}{1.130583in}}%
\pgfpathlineto{\pgfqpoint{4.198506in}{0.996830in}}%
\pgfpathlineto{\pgfqpoint{4.202200in}{1.083297in}}%
\pgfpathlineto{\pgfqpoint{4.204035in}{0.929069in}}%
\pgfpathlineto{\pgfqpoint{4.205869in}{1.217490in}}%
\pgfpathlineto{\pgfqpoint{4.207702in}{1.169627in}}%
\pgfpathlineto{\pgfqpoint{4.209536in}{1.069585in}}%
\pgfpathlineto{\pgfqpoint{4.211368in}{1.271438in}}%
\pgfpathlineto{\pgfqpoint{4.213201in}{1.271501in}}%
\pgfpathlineto{\pgfqpoint{4.215035in}{1.348596in}}%
\pgfpathlineto{\pgfqpoint{4.216869in}{1.361644in}}%
\pgfpathlineto{\pgfqpoint{4.218702in}{1.327180in}}%
\pgfpathlineto{\pgfqpoint{4.220537in}{1.238241in}}%
\pgfpathlineto{\pgfqpoint{4.222371in}{1.343189in}}%
\pgfpathlineto{\pgfqpoint{4.226039in}{1.094589in}}%
\pgfpathlineto{\pgfqpoint{4.227874in}{1.070024in}}%
\pgfpathlineto{\pgfqpoint{4.229707in}{0.991360in}}%
\pgfpathlineto{\pgfqpoint{4.231541in}{1.040465in}}%
\pgfpathlineto{\pgfqpoint{4.233374in}{1.252782in}}%
\pgfpathlineto{\pgfqpoint{4.235208in}{1.261125in}}%
\pgfpathlineto{\pgfqpoint{4.237041in}{1.161459in}}%
\pgfpathlineto{\pgfqpoint{4.238876in}{1.176565in}}%
\pgfpathlineto{\pgfqpoint{4.240710in}{1.310431in}}%
\pgfpathlineto{\pgfqpoint{4.242543in}{1.298148in}}%
\pgfpathlineto{\pgfqpoint{4.244379in}{1.008962in}}%
\pgfpathlineto{\pgfqpoint{4.248044in}{1.158762in}}%
\pgfpathlineto{\pgfqpoint{4.249878in}{1.154496in}}%
\pgfpathlineto{\pgfqpoint{4.251713in}{1.218017in}}%
\pgfpathlineto{\pgfqpoint{4.253547in}{1.084038in}}%
\pgfpathlineto{\pgfqpoint{4.255380in}{1.117448in}}%
\pgfpathlineto{\pgfqpoint{4.257214in}{1.178534in}}%
\pgfpathlineto{\pgfqpoint{4.259047in}{1.055357in}}%
\pgfpathlineto{\pgfqpoint{4.260882in}{1.310042in}}%
\pgfpathlineto{\pgfqpoint{4.262717in}{1.153053in}}%
\pgfpathlineto{\pgfqpoint{4.266386in}{1.243347in}}%
\pgfpathlineto{\pgfqpoint{4.268220in}{1.212735in}}%
\pgfpathlineto{\pgfqpoint{4.273723in}{1.433043in}}%
\pgfpathlineto{\pgfqpoint{4.275557in}{2.301543in}}%
\pgfpathlineto{\pgfqpoint{4.277392in}{2.229968in}}%
\pgfpathlineto{\pgfqpoint{4.279227in}{2.207573in}}%
\pgfpathlineto{\pgfqpoint{4.281060in}{2.328317in}}%
\pgfpathlineto{\pgfqpoint{4.282894in}{2.276037in}}%
\pgfpathlineto{\pgfqpoint{4.286560in}{2.077647in}}%
\pgfpathlineto{\pgfqpoint{4.288394in}{2.307829in}}%
\pgfpathlineto{\pgfqpoint{4.292060in}{2.333536in}}%
\pgfpathlineto{\pgfqpoint{4.293894in}{2.300452in}}%
\pgfpathlineto{\pgfqpoint{4.295728in}{2.325632in}}%
\pgfpathlineto{\pgfqpoint{4.297561in}{2.213156in}}%
\pgfpathlineto{\pgfqpoint{4.299395in}{2.285208in}}%
\pgfpathlineto{\pgfqpoint{4.303066in}{1.192611in}}%
\pgfpathlineto{\pgfqpoint{4.304899in}{1.243285in}}%
\pgfpathlineto{\pgfqpoint{4.306733in}{1.435603in}}%
\pgfpathlineto{\pgfqpoint{4.308566in}{1.345246in}}%
\pgfpathlineto{\pgfqpoint{4.312234in}{2.107293in}}%
\pgfpathlineto{\pgfqpoint{4.314067in}{2.162910in}}%
\pgfpathlineto{\pgfqpoint{4.315900in}{2.180512in}}%
\pgfpathlineto{\pgfqpoint{4.317734in}{2.163499in}}%
\pgfpathlineto{\pgfqpoint{4.319568in}{2.412526in}}%
\pgfpathlineto{\pgfqpoint{4.321401in}{2.469083in}}%
\pgfpathlineto{\pgfqpoint{4.325069in}{2.297253in}}%
\pgfpathlineto{\pgfqpoint{4.326902in}{2.462911in}}%
\pgfpathlineto{\pgfqpoint{4.328736in}{2.214248in}}%
\pgfpathlineto{\pgfqpoint{4.330571in}{2.145282in}}%
\pgfpathlineto{\pgfqpoint{4.332404in}{2.262312in}}%
\pgfpathlineto{\pgfqpoint{4.337905in}{2.039532in}}%
\pgfpathlineto{\pgfqpoint{4.339740in}{2.080231in}}%
\pgfpathlineto{\pgfqpoint{4.341576in}{2.219179in}}%
\pgfpathlineto{\pgfqpoint{4.343408in}{2.173762in}}%
\pgfpathlineto{\pgfqpoint{4.345242in}{2.096102in}}%
\pgfpathlineto{\pgfqpoint{4.347076in}{2.188541in}}%
\pgfpathlineto{\pgfqpoint{4.348910in}{2.155834in}}%
\pgfpathlineto{\pgfqpoint{4.350743in}{2.199757in}}%
\pgfpathlineto{\pgfqpoint{4.352577in}{2.290641in}}%
\pgfpathlineto{\pgfqpoint{4.354411in}{1.228305in}}%
\pgfpathlineto{\pgfqpoint{4.356245in}{1.269995in}}%
\pgfpathlineto{\pgfqpoint{4.358079in}{1.269004in}}%
\pgfpathlineto{\pgfqpoint{4.359912in}{1.320091in}}%
\pgfpathlineto{\pgfqpoint{4.361745in}{1.236359in}}%
\pgfpathlineto{\pgfqpoint{4.363581in}{1.376448in}}%
\pgfpathlineto{\pgfqpoint{4.365414in}{1.386058in}}%
\pgfpathlineto{\pgfqpoint{4.367247in}{1.491282in}}%
\pgfpathlineto{\pgfqpoint{4.369083in}{1.437899in}}%
\pgfpathlineto{\pgfqpoint{4.370916in}{1.427674in}}%
\pgfpathlineto{\pgfqpoint{4.372750in}{1.120446in}}%
\pgfpathlineto{\pgfqpoint{4.374585in}{1.203212in}}%
\pgfpathlineto{\pgfqpoint{4.376419in}{1.429631in}}%
\pgfpathlineto{\pgfqpoint{4.378253in}{1.358031in}}%
\pgfpathlineto{\pgfqpoint{4.380088in}{1.207880in}}%
\pgfpathlineto{\pgfqpoint{4.381923in}{1.388354in}}%
\pgfpathlineto{\pgfqpoint{4.383756in}{1.398253in}}%
\pgfpathlineto{\pgfqpoint{4.385591in}{1.305965in}}%
\pgfpathlineto{\pgfqpoint{4.387426in}{1.318598in}}%
\pgfpathlineto{\pgfqpoint{4.389259in}{1.313166in}}%
\pgfpathlineto{\pgfqpoint{4.391093in}{1.476239in}}%
\pgfpathlineto{\pgfqpoint{4.392929in}{1.337468in}}%
\pgfpathlineto{\pgfqpoint{4.394762in}{1.340755in}}%
\pgfpathlineto{\pgfqpoint{4.396597in}{1.354442in}}%
\pgfpathlineto{\pgfqpoint{4.398431in}{1.386209in}}%
\pgfpathlineto{\pgfqpoint{4.400264in}{1.308022in}}%
\pgfpathlineto{\pgfqpoint{4.403933in}{1.481647in}}%
\pgfpathlineto{\pgfqpoint{4.405798in}{1.339375in}}%
\pgfpathlineto{\pgfqpoint{4.407631in}{1.347868in}}%
\pgfpathlineto{\pgfqpoint{4.409466in}{1.256420in}}%
\pgfpathlineto{\pgfqpoint{4.411300in}{1.295325in}}%
\pgfpathlineto{\pgfqpoint{4.413134in}{1.297446in}}%
\pgfpathlineto{\pgfqpoint{4.414969in}{1.222809in}}%
\pgfpathlineto{\pgfqpoint{4.416802in}{1.453167in}}%
\pgfpathlineto{\pgfqpoint{4.418637in}{1.297910in}}%
\pgfpathlineto{\pgfqpoint{4.420471in}{1.393937in}}%
\pgfpathlineto{\pgfqpoint{4.422305in}{1.321798in}}%
\pgfpathlineto{\pgfqpoint{4.424138in}{1.379974in}}%
\pgfpathlineto{\pgfqpoint{4.425971in}{1.298889in}}%
\pgfpathlineto{\pgfqpoint{4.427805in}{1.402569in}}%
\pgfpathlineto{\pgfqpoint{4.429639in}{1.381830in}}%
\pgfpathlineto{\pgfqpoint{4.431472in}{1.341570in}}%
\pgfpathlineto{\pgfqpoint{4.435139in}{1.513790in}}%
\pgfpathlineto{\pgfqpoint{4.436973in}{1.574048in}}%
\pgfpathlineto{\pgfqpoint{4.438807in}{1.573521in}}%
\pgfpathlineto{\pgfqpoint{4.442474in}{1.284398in}}%
\pgfpathlineto{\pgfqpoint{4.444309in}{1.479740in}}%
\pgfpathlineto{\pgfqpoint{4.446143in}{1.241503in}}%
\pgfpathlineto{\pgfqpoint{4.447977in}{1.303669in}}%
\pgfpathlineto{\pgfqpoint{4.449812in}{1.283607in}}%
\pgfpathlineto{\pgfqpoint{4.451645in}{1.410097in}}%
\pgfpathlineto{\pgfqpoint{4.453478in}{1.260598in}}%
\pgfpathlineto{\pgfqpoint{4.458998in}{1.494456in}}%
\pgfpathlineto{\pgfqpoint{4.462665in}{1.328999in}}%
\pgfpathlineto{\pgfqpoint{4.464499in}{1.633642in}}%
\pgfpathlineto{\pgfqpoint{4.466333in}{2.176221in}}%
\pgfpathlineto{\pgfqpoint{4.469999in}{2.438973in}}%
\pgfpathlineto{\pgfqpoint{4.473668in}{2.248248in}}%
\pgfpathlineto{\pgfqpoint{4.477335in}{2.515566in}}%
\pgfpathlineto{\pgfqpoint{4.479184in}{2.207573in}}%
\pgfpathlineto{\pgfqpoint{4.481019in}{2.481930in}}%
\pgfpathlineto{\pgfqpoint{4.482853in}{2.321441in}}%
\pgfpathlineto{\pgfqpoint{4.484685in}{2.323813in}}%
\pgfpathlineto{\pgfqpoint{4.486520in}{2.311505in}}%
\pgfpathlineto{\pgfqpoint{4.488354in}{2.491152in}}%
\pgfpathlineto{\pgfqpoint{4.490186in}{2.382616in}}%
\pgfpathlineto{\pgfqpoint{4.492020in}{2.628719in}}%
\pgfpathlineto{\pgfqpoint{4.493854in}{2.442021in}}%
\pgfpathlineto{\pgfqpoint{4.495689in}{2.515504in}}%
\pgfpathlineto{\pgfqpoint{4.499358in}{2.405914in}}%
\pgfpathlineto{\pgfqpoint{4.501192in}{2.585849in}}%
\pgfpathlineto{\pgfqpoint{4.503027in}{2.337914in}}%
\pgfpathlineto{\pgfqpoint{4.506693in}{2.643937in}}%
\pgfpathlineto{\pgfqpoint{4.508528in}{2.462208in}}%
\pgfpathlineto{\pgfqpoint{4.510362in}{2.442498in}}%
\pgfpathlineto{\pgfqpoint{4.512194in}{2.406679in}}%
\pgfpathlineto{\pgfqpoint{4.514031in}{2.275749in}}%
\pgfpathlineto{\pgfqpoint{4.517699in}{2.494025in}}%
\pgfpathlineto{\pgfqpoint{4.521370in}{2.389127in}}%
\pgfpathlineto{\pgfqpoint{4.523203in}{2.303802in}}%
\pgfpathlineto{\pgfqpoint{4.525039in}{2.448784in}}%
\pgfpathlineto{\pgfqpoint{4.528706in}{2.320137in}}%
\pgfpathlineto{\pgfqpoint{4.530541in}{2.235702in}}%
\pgfpathlineto{\pgfqpoint{4.532377in}{2.276426in}}%
\pgfpathlineto{\pgfqpoint{4.534210in}{2.194563in}}%
\pgfpathlineto{\pgfqpoint{4.537878in}{2.315996in}}%
\pgfpathlineto{\pgfqpoint{4.541544in}{2.282360in}}%
\pgfpathlineto{\pgfqpoint{4.543377in}{2.193672in}}%
\pgfpathlineto{\pgfqpoint{4.545211in}{2.222591in}}%
\pgfpathlineto{\pgfqpoint{4.547045in}{2.590554in}}%
\pgfpathlineto{\pgfqpoint{4.548878in}{2.601331in}}%
\pgfpathlineto{\pgfqpoint{4.550711in}{2.835553in}}%
\pgfpathlineto{\pgfqpoint{4.552574in}{2.192380in}}%
\pgfpathlineto{\pgfqpoint{4.554408in}{2.299925in}}%
\pgfpathlineto{\pgfqpoint{4.556243in}{2.255888in}}%
\pgfpathlineto{\pgfqpoint{4.558077in}{2.241460in}}%
\pgfpathlineto{\pgfqpoint{4.559910in}{2.245864in}}%
\pgfpathlineto{\pgfqpoint{4.561745in}{2.306298in}}%
\pgfpathlineto{\pgfqpoint{4.565411in}{2.137027in}}%
\pgfpathlineto{\pgfqpoint{4.567246in}{2.415110in}}%
\pgfpathlineto{\pgfqpoint{4.569079in}{2.262136in}}%
\pgfpathlineto{\pgfqpoint{4.572747in}{2.401498in}}%
\pgfpathlineto{\pgfqpoint{4.574580in}{2.298846in}}%
\pgfpathlineto{\pgfqpoint{4.576415in}{2.314830in}}%
\pgfpathlineto{\pgfqpoint{4.578250in}{2.207260in}}%
\pgfpathlineto{\pgfqpoint{4.580083in}{2.321705in}}%
\pgfpathlineto{\pgfqpoint{4.581917in}{2.319610in}}%
\pgfpathlineto{\pgfqpoint{4.583750in}{2.331629in}}%
\pgfpathlineto{\pgfqpoint{4.585584in}{1.448914in}}%
\pgfpathlineto{\pgfqpoint{4.587417in}{1.422492in}}%
\pgfpathlineto{\pgfqpoint{4.589252in}{1.508997in}}%
\pgfpathlineto{\pgfqpoint{4.591086in}{1.481521in}}%
\pgfpathlineto{\pgfqpoint{4.592920in}{1.490454in}}%
\pgfpathlineto{\pgfqpoint{4.596587in}{1.432805in}}%
\pgfpathlineto{\pgfqpoint{4.598421in}{1.491659in}}%
\pgfpathlineto{\pgfqpoint{4.600254in}{1.447496in}}%
\pgfpathlineto{\pgfqpoint{4.602089in}{1.472099in}}%
\pgfpathlineto{\pgfqpoint{4.605756in}{1.294623in}}%
\pgfpathlineto{\pgfqpoint{4.607590in}{1.417323in}}%
\pgfpathlineto{\pgfqpoint{4.609425in}{2.295760in}}%
\pgfpathlineto{\pgfqpoint{4.611259in}{2.346295in}}%
\pgfpathlineto{\pgfqpoint{4.614929in}{2.163825in}}%
\pgfpathlineto{\pgfqpoint{4.616761in}{1.949301in}}%
\pgfpathlineto{\pgfqpoint{4.618596in}{2.319321in}}%
\pgfpathlineto{\pgfqpoint{4.620430in}{2.292962in}}%
\pgfpathlineto{\pgfqpoint{4.622262in}{2.119011in}}%
\pgfpathlineto{\pgfqpoint{4.624096in}{2.287705in}}%
\pgfpathlineto{\pgfqpoint{4.625930in}{2.334163in}}%
\pgfpathlineto{\pgfqpoint{4.629613in}{2.104821in}}%
\pgfpathlineto{\pgfqpoint{4.631447in}{2.167790in}}%
\pgfpathlineto{\pgfqpoint{4.633281in}{2.148017in}}%
\pgfpathlineto{\pgfqpoint{4.635115in}{1.205069in}}%
\pgfpathlineto{\pgfqpoint{4.636949in}{1.250072in}}%
\pgfpathlineto{\pgfqpoint{4.638784in}{1.253510in}}%
\pgfpathlineto{\pgfqpoint{4.642452in}{1.396672in}}%
\pgfpathlineto{\pgfqpoint{4.644286in}{1.339312in}}%
\pgfpathlineto{\pgfqpoint{4.646122in}{1.344456in}}%
\pgfpathlineto{\pgfqpoint{4.647955in}{1.353765in}}%
\pgfpathlineto{\pgfqpoint{4.651625in}{2.255700in}}%
\pgfpathlineto{\pgfqpoint{4.653458in}{2.160714in}}%
\pgfpathlineto{\pgfqpoint{4.655293in}{1.984091in}}%
\pgfpathlineto{\pgfqpoint{4.657854in}{2.046407in}}%
\pgfpathlineto{\pgfqpoint{4.659687in}{2.042330in}}%
\pgfpathlineto{\pgfqpoint{4.661523in}{2.114407in}}%
\pgfpathlineto{\pgfqpoint{4.663356in}{1.999660in}}%
\pgfpathlineto{\pgfqpoint{4.665190in}{1.964958in}}%
\pgfpathlineto{\pgfqpoint{4.667025in}{1.806941in}}%
\pgfpathlineto{\pgfqpoint{4.668858in}{1.970842in}}%
\pgfpathlineto{\pgfqpoint{4.672527in}{1.864678in}}%
\pgfpathlineto{\pgfqpoint{4.674360in}{2.003901in}}%
\pgfpathlineto{\pgfqpoint{4.676194in}{2.272048in}}%
\pgfpathlineto{\pgfqpoint{4.679861in}{2.104232in}}%
\pgfpathlineto{\pgfqpoint{4.681693in}{2.218150in}}%
\pgfpathlineto{\pgfqpoint{4.683527in}{2.132410in}}%
\pgfpathlineto{\pgfqpoint{4.685359in}{2.311354in}}%
\pgfpathlineto{\pgfqpoint{4.687192in}{2.137002in}}%
\pgfpathlineto{\pgfqpoint{4.689027in}{2.097043in}}%
\pgfpathlineto{\pgfqpoint{4.690860in}{2.243744in}}%
\pgfpathlineto{\pgfqpoint{4.694526in}{2.149071in}}%
\pgfpathlineto{\pgfqpoint{4.696359in}{1.854478in}}%
\pgfpathlineto{\pgfqpoint{4.698192in}{1.233474in}}%
\pgfpathlineto{\pgfqpoint{4.700028in}{1.407625in}}%
\pgfpathlineto{\pgfqpoint{4.701864in}{1.122905in}}%
\pgfpathlineto{\pgfqpoint{4.703698in}{1.129931in}}%
\pgfpathlineto{\pgfqpoint{4.705532in}{1.118351in}}%
\pgfpathlineto{\pgfqpoint{4.707367in}{1.212120in}}%
\pgfpathlineto{\pgfqpoint{4.709200in}{0.858083in}}%
\pgfpathlineto{\pgfqpoint{4.712867in}{1.312601in}}%
\pgfpathlineto{\pgfqpoint{4.714701in}{1.254488in}}%
\pgfpathlineto{\pgfqpoint{4.716533in}{1.062471in}}%
\pgfpathlineto{\pgfqpoint{4.718366in}{1.239671in}}%
\pgfpathlineto{\pgfqpoint{4.720199in}{1.284247in}}%
\pgfpathlineto{\pgfqpoint{4.722032in}{1.356625in}}%
\pgfpathlineto{\pgfqpoint{4.723865in}{1.988996in}}%
\pgfpathlineto{\pgfqpoint{4.725698in}{2.017413in}}%
\pgfpathlineto{\pgfqpoint{4.727532in}{1.936378in}}%
\pgfpathlineto{\pgfqpoint{4.729366in}{1.976362in}}%
\pgfpathlineto{\pgfqpoint{4.731200in}{2.057109in}}%
\pgfpathlineto{\pgfqpoint{4.733032in}{2.232440in}}%
\pgfpathlineto{\pgfqpoint{4.734866in}{2.219505in}}%
\pgfpathlineto{\pgfqpoint{4.736700in}{2.248160in}}%
\pgfpathlineto{\pgfqpoint{4.738533in}{2.256854in}}%
\pgfpathlineto{\pgfqpoint{4.740366in}{2.181954in}}%
\pgfpathlineto{\pgfqpoint{4.742199in}{2.159735in}}%
\pgfpathlineto{\pgfqpoint{4.744033in}{2.093016in}}%
\pgfpathlineto{\pgfqpoint{4.745866in}{2.253203in}}%
\pgfpathlineto{\pgfqpoint{4.747700in}{2.146110in}}%
\pgfpathlineto{\pgfqpoint{4.749534in}{2.152747in}}%
\pgfpathlineto{\pgfqpoint{4.751367in}{2.187161in}}%
\pgfpathlineto{\pgfqpoint{4.753202in}{2.157778in}}%
\pgfpathlineto{\pgfqpoint{4.755035in}{2.196972in}}%
\pgfpathlineto{\pgfqpoint{4.756868in}{1.926241in}}%
\pgfpathlineto{\pgfqpoint{4.758701in}{2.092049in}}%
\pgfpathlineto{\pgfqpoint{4.760535in}{2.138056in}}%
\pgfpathlineto{\pgfqpoint{4.764201in}{1.274587in}}%
\pgfpathlineto{\pgfqpoint{4.766036in}{1.197479in}}%
\pgfpathlineto{\pgfqpoint{4.767868in}{1.317570in}}%
\pgfpathlineto{\pgfqpoint{4.771535in}{1.194543in}}%
\pgfpathlineto{\pgfqpoint{4.773368in}{2.037838in}}%
\pgfpathlineto{\pgfqpoint{4.777037in}{2.132824in}}%
\pgfpathlineto{\pgfqpoint{4.778871in}{2.149699in}}%
\pgfpathlineto{\pgfqpoint{4.780704in}{2.062604in}}%
\pgfpathlineto{\pgfqpoint{4.782543in}{2.207072in}}%
\pgfpathlineto{\pgfqpoint{4.784383in}{2.222528in}}%
\pgfpathlineto{\pgfqpoint{4.788065in}{1.263270in}}%
\pgfpathlineto{\pgfqpoint{4.789906in}{1.241352in}}%
\pgfpathlineto{\pgfqpoint{4.791746in}{1.111802in}}%
\pgfpathlineto{\pgfqpoint{4.795426in}{1.164395in}}%
\pgfpathlineto{\pgfqpoint{4.797636in}{1.171948in}}%
\pgfpathlineto{\pgfqpoint{4.799475in}{2.813949in}}%
\pgfpathlineto{\pgfqpoint{4.801317in}{1.208469in}}%
\pgfpathlineto{\pgfqpoint{4.803157in}{1.281286in}}%
\pgfpathlineto{\pgfqpoint{4.804997in}{2.003186in}}%
\pgfpathlineto{\pgfqpoint{4.806836in}{2.013674in}}%
\pgfpathlineto{\pgfqpoint{4.810517in}{1.872117in}}%
\pgfpathlineto{\pgfqpoint{4.812358in}{1.914247in}}%
\pgfpathlineto{\pgfqpoint{4.814200in}{2.164616in}}%
\pgfpathlineto{\pgfqpoint{4.816040in}{2.025907in}}%
\pgfpathlineto{\pgfqpoint{4.817881in}{2.000689in}}%
\pgfpathlineto{\pgfqpoint{4.819721in}{2.196295in}}%
\pgfpathlineto{\pgfqpoint{4.821560in}{1.897711in}}%
\pgfpathlineto{\pgfqpoint{4.823400in}{1.955122in}}%
\pgfpathlineto{\pgfqpoint{4.825241in}{2.069454in}}%
\pgfpathlineto{\pgfqpoint{4.827080in}{2.062892in}}%
\pgfpathlineto{\pgfqpoint{4.828920in}{2.021930in}}%
\pgfpathlineto{\pgfqpoint{4.830759in}{2.116163in}}%
\pgfpathlineto{\pgfqpoint{4.836650in}{2.209581in}}%
\pgfpathlineto{\pgfqpoint{4.838489in}{2.140352in}}%
\pgfpathlineto{\pgfqpoint{4.840330in}{2.135760in}}%
\pgfpathlineto{\pgfqpoint{4.842169in}{2.034488in}}%
\pgfpathlineto{\pgfqpoint{4.844009in}{2.063683in}}%
\pgfpathlineto{\pgfqpoint{4.846214in}{2.033045in}}%
\pgfpathlineto{\pgfqpoint{4.848056in}{2.254031in}}%
\pgfpathlineto{\pgfqpoint{4.849893in}{2.000338in}}%
\pgfpathlineto{\pgfqpoint{4.851730in}{1.916831in}}%
\pgfpathlineto{\pgfqpoint{4.853564in}{1.924008in}}%
\pgfpathlineto{\pgfqpoint{4.857232in}{2.171077in}}%
\pgfpathlineto{\pgfqpoint{4.859065in}{2.124456in}}%
\pgfpathlineto{\pgfqpoint{4.859065in}{2.124456in}}%
\pgfusepath{stroke}%
\end{pgfscope}%
\begin{pgfscope}%
\pgfsetrectcap%
\pgfsetmiterjoin%
\pgfsetlinewidth{0.803000pt}%
\definecolor{currentstroke}{rgb}{0.000000,0.000000,0.000000}%
\pgfsetstrokecolor{currentstroke}%
\pgfsetdash{}{0pt}%
\pgfpathmoveto{\pgfqpoint{0.667540in}{0.539544in}}%
\pgfpathlineto{\pgfqpoint{0.667540in}{2.944887in}}%
\pgfusepath{stroke}%
\end{pgfscope}%
\begin{pgfscope}%
\pgfsetrectcap%
\pgfsetmiterjoin%
\pgfsetlinewidth{0.803000pt}%
\definecolor{currentstroke}{rgb}{0.000000,0.000000,0.000000}%
\pgfsetstrokecolor{currentstroke}%
\pgfsetdash{}{0pt}%
\pgfpathmoveto{\pgfqpoint{5.058662in}{0.539544in}}%
\pgfpathlineto{\pgfqpoint{5.058662in}{2.944887in}}%
\pgfusepath{stroke}%
\end{pgfscope}%
\begin{pgfscope}%
\pgfsetrectcap%
\pgfsetmiterjoin%
\pgfsetlinewidth{0.803000pt}%
\definecolor{currentstroke}{rgb}{0.000000,0.000000,0.000000}%
\pgfsetstrokecolor{currentstroke}%
\pgfsetdash{}{0pt}%
\pgfpathmoveto{\pgfqpoint{0.667540in}{0.539544in}}%
\pgfpathlineto{\pgfqpoint{5.058662in}{0.539544in}}%
\pgfusepath{stroke}%
\end{pgfscope}%
\begin{pgfscope}%
\pgfsetrectcap%
\pgfsetmiterjoin%
\pgfsetlinewidth{0.803000pt}%
\definecolor{currentstroke}{rgb}{0.000000,0.000000,0.000000}%
\pgfsetstrokecolor{currentstroke}%
\pgfsetdash{}{0pt}%
\pgfpathmoveto{\pgfqpoint{0.667540in}{2.944887in}}%
\pgfpathlineto{\pgfqpoint{5.058662in}{2.944887in}}%
\pgfusepath{stroke}%
\end{pgfscope}%
\begin{pgfscope}%
\pgfsetbuttcap%
\pgfsetmiterjoin%
\definecolor{currentfill}{rgb}{1.000000,1.000000,1.000000}%
\pgfsetfillcolor{currentfill}%
\pgfsetfillopacity{0.800000}%
\pgfsetlinewidth{1.003750pt}%
\definecolor{currentstroke}{rgb}{0.800000,0.800000,0.800000}%
\pgfsetstrokecolor{currentstroke}%
\pgfsetstrokeopacity{0.800000}%
\pgfsetdash{}{0pt}%
\pgfpathmoveto{\pgfqpoint{0.745318in}{2.701109in}}%
\pgfpathlineto{\pgfqpoint{2.092873in}{2.701109in}}%
\pgfpathquadraticcurveto{\pgfqpoint{2.115096in}{2.701109in}}{\pgfqpoint{2.115096in}{2.723331in}}%
\pgfpathlineto{\pgfqpoint{2.115096in}{2.867109in}}%
\pgfpathquadraticcurveto{\pgfqpoint{2.115096in}{2.889331in}}{\pgfqpoint{2.092873in}{2.889331in}}%
\pgfpathlineto{\pgfqpoint{0.745318in}{2.889331in}}%
\pgfpathquadraticcurveto{\pgfqpoint{0.723095in}{2.889331in}}{\pgfqpoint{0.723095in}{2.867109in}}%
\pgfpathlineto{\pgfqpoint{0.723095in}{2.723331in}}%
\pgfpathquadraticcurveto{\pgfqpoint{0.723095in}{2.701109in}}{\pgfqpoint{0.745318in}{2.701109in}}%
\pgfpathlineto{\pgfqpoint{0.745318in}{2.701109in}}%
\pgfpathclose%
\pgfusepath{stroke,fill}%
\end{pgfscope}%
\begin{pgfscope}%
\pgfsetrectcap%
\pgfsetroundjoin%
\pgfsetlinewidth{0.501875pt}%
\definecolor{currentstroke}{rgb}{0.121569,0.466667,0.705882}%
\pgfsetstrokecolor{currentstroke}%
\pgfsetstrokeopacity{0.700000}%
\pgfsetdash{}{0pt}%
\pgfpathmoveto{\pgfqpoint{0.767540in}{2.805998in}}%
\pgfpathlineto{\pgfqpoint{0.878651in}{2.805998in}}%
\pgfpathlineto{\pgfqpoint{0.989762in}{2.805998in}}%
\pgfusepath{stroke}%
\end{pgfscope}%
\begin{pgfscope}%
\definecolor{textcolor}{rgb}{0.000000,0.000000,0.000000}%
\pgfsetstrokecolor{textcolor}%
\pgfsetfillcolor{textcolor}%
\pgftext[x=1.078651in,y=2.767109in,left,base]{\color{textcolor}\rmfamily\fontsize{8.000000}{9.600000}\selectfont DUT vs KS34470A}%
\end{pgfscope}%
\end{pgfpicture}%
\makeatother%
\endgroup%

    \caption{Popcorn noise of a refurbished LM399 (\#15) over a period of \qty{15}{\minute}.}
    \label{fig:fake_lm399_popcorn_noise}
\end{figure}

TODO: Chinese/Ebay Zeners. Welded legs. Photots. Decap one of those.

%\begin{figure}[h]
%    \centering
    %\import{figures/}{dgDrive_protocol.tex}
%\end{figure}

%\subsection{Current Sources}
%Discuss Op amp choice (AD797)

%\subsection{Temperature Coeeficient}
%Discuss each section (Reference, DAC, Buffer/Divider, Filter, CC)
%\subsubsection{Voltage Reference}
%\subsubsection{DAC}
%\subsubsection{Divider}
%\subsubsection{Filter}
%Choice of components. Leakage current, size of resistor (input bias current of AD797), size of capacitor

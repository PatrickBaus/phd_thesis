\section*{Zusammenfassung}
Die Entwicklung von Laserdioden ist in den letzten Jahren immer weiter voran geschritten. Mittlerweile wird nahezu der gesamte Wellenlängenbereich von Ultraviolett bis Nahinfrarot abgedeckt, wodurch eine Vielzahl von Lasersystemen zu Verfügung steht. Dabei hat sich eine Kluft aufgetan zwischen der Entwicklung von Laserdioden und der zugehörigen Steuerelektronik. Dies hat unweigerlich zu einer eingeschränkten Leistungsfähigkeit von Lasersystemen mit diesen modernen Dioden geführt.

Das Interesse an hochauflösender Laserspektroskopie von hochgeladenen Ionen, welche optisch zugängliche Übergänge im ultravioletten und blauen Wellenlängenbereich aufweisen führte zu einem gesteigerten Bedarf an kompakten und hoch performanten Lasersystemen. Zeitgleich ergeben sich aus der sprunghaften Entwicklung von Quantensystemen, z.B. bestehend aus Registern aus einzeln kohärent adressierbaren neutralen Atomen, welche Quantencomputer mit tausenden von Qubits in greifbaren Nähe rücken lassen, neue Anforderungen an die Stabilität, Steuerbarkeit und Integration der Lasersysteme. All diese Anwendungsfälle benötigen Systeme auf dem neusten Stand der Technik, welche die Leistung aktuell verfügbarer Lösungen hinsichtlich Stabilität und Rauschverhalten bei weitem übertrifft.

Geleitet durch diese Anforderungen beschäftigt sich diese Arbeit sich mit dem Aufbau entsprechender Lasersysteme. Es werden hierzu zunächst kommerziell verfügbare und aus wissenschaftlichen Arbeiten abgeleitete Geräte, wie Laserstromtreiber und Temperaturregler, für diese Anwendungen getestet. Die Stromtreiber werden bezüglich ihrer Konformitätsspannung, dem Ausgangsrauschen, der Stabilität, in Bezug auf Temperatur und Zeit, und der Ausgangsimpedanz untersucht. Die Grenzen der getesteten Geräte werden aufgezeigt und die Ursachen analytisch und durch Simulationen illustriert. Die Temperaturregler werden hinsichtlich ihres Messrauschens und der Stabilität gegen einen Referenzwiderstand vermessen.

Diese Erkenntnisse führten zu der Notwendigkeit einer Eigenentwicklung eines neuartigen digital Laserstromtreibers, welcher die Leistung bestehender Lösungen um mindestens eine Größenordnung übertrifft. Die dabei entstandene Open-Source Lösung ist modular und kann einfach an die Bedürfnisse der hochauflösenden Spektroskopie und der kohärenten Manipulation von Quantensystemen angepasst werden. Der Laserstromtreiber baut hierzu auf einer einzigartigen Architektur auf, welche die Stromquelle von der getriebenen Last trennt, um so die Anforderungen einer hohen Konformitätsspannung durch blaue Laserdioden mit dem Bedürfnis nach niedrigstem Stromrauschen, welches das Schottky-Rauschen bereits oberhalb von \qty{20}{\mA} unterbietet, und höchster Stabilität nahe an der Grenze des physikalisch Machbaren ermöglicht. Dies wird mit einer außergewöhnlichen Eingangsrauschunterdrückung gepaart, welche den Betrieb mit modernen und kompakten Schaltnetzteilen ermöglicht, ohne Einbußen in der Leistung befürchten zu müssen.

Der digitale Temperaturregler, welcher ebenso als Open-Source Entwicklung vorangetrieben wird, ermöglicht eine Regelstabilität von weniger als \qty{1}{\milli\kelvin} mit einer \unit{\micro\kelvin}-Auflösung. Zusammen mit dem Stromtreiber verschiebt er hierdurch die Grenzen der Laserstabilität klar in Richtung der mechanischen Stabilität des externen Laserresonators.

Zuletzt wird noch ein Datenerfassungssystem vorgestellt, welches im Zusammenspiel mit den vorgestellten Geräten sowohl die Laborumgebung, wie auch Experiment- und Geräteparameter erfassen kann, um dem Experimentator eine Echtzeitüberwachung des Experiments über mehrere Endgeräte hinweg zu ermöglichen.

Diese Neuentwicklungen werden aktuell bereits intensiv von mehreren wegweisenden Experimenten genutzt und sind mittlerweile unabdingbar für deren Erfolg geworden.

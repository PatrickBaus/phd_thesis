\chapter{Introduction}
Highly charged ions offer a unique insight into the very fine details of the world described by quantum electrodynamics (QED). The predictions of the electron’s magnetic moment (g-factor) can easily be regarded as the most accurate prediction in all science, matching the experimental value to \num{10} significant figures \cite{gfactor_theory}.

For free electrons, experimental measurements of the g-factor have most recently pushed the boundaries as far as an uncertainty of \num{1.3e-13}. The comparison of experimental values of the g-factor with theory therefore represents the most stringent test of the QED theory \cite{bi82+_qed}. Extending the application of QED to bound states requires new tests of these calculations. Computing of the g-factor in complex systems such as neutral atoms with many electrons is extremely difficult and currently impossible with decent uncertainty \cite{gfactor_theory_codegen}. Using heavy highly charged ions helps both theory and experiment because it reduces the complexity and at the same times scales the QED contribution with the nuclear charge number as $Z\alpha$, reducing the required accuracy \cite{gfactor_ions_scaling}. While research has primarily relied on non-optical measurement in Penning traps in the past, laser spectroscopy opens new opportunities \cite{penning_trap_laser_spectroscopy} as the strong scaling of the fine- and hyperfine-structures with $Z^4$ and $Z^3$ brings those transitions into the region of visible and ultraviolet (UV) laser spectroscopy.

Highly charged ions are therefore an interesting field of research with the GSI Helmholtz Center for Heavy Ion Research at the forefront and capable of providing up to bare uranium. The AsymmetRic Trap for the measurement of Electron Magnetic moments in IonS (ARTEMIS) experiment at GSI as part of the Highly charged Ions Trap (HITRAP) platform and Facility for Antiproton and Ion Research (FAIR) aims to perform high-precision measurements of the bound electron's g-factor using a combination of laser- and microwave spectroscopy referred to as laser-microwave double-resonance spectroscopy \cite{laser_microwave_double_resonance_spectroscopy}. The ARTEMIS experiment provides a unique optically accessible Penning trap design \cite{penning_trap_transparent,penning_trap_half_open} with the aim of measurements on hydrogen-like bismuth, \ce{^{209}Bi^{82+}}. Currently the experiment is in its commissioning phase using boron-like Argon, \ce{^{40}Ar^{13+}}, targeting a g-factor measurement at the \num{e-9} level \cite{BoundElectronGFactorMeasurement}.

Both the commissioning and the measurement on hydrogen-like bismuth require a precision laser system consisting of multiple lasers for the targeted closed transition driven by lasers and microwaves. The accessible wavelength of \ce{^{40}Ar^{13+}} is \qty{441}{\nm} \cite{ar13+_wavelength}, while \ce{^{209}Bi^{82+}} requires laser radiation at \qty{243.8}{\nm} \cite{bi82+_wavelength}. Recent years have seen the development of new laser diodes giving diode lasers access to an increasing part of the spectrum and for both targeted ion species the fundamental wavelengths are now covered by diode lasers. \qty{244}{\nm} can be reached via a quadrupled \qty{976}{\nm} diode laser source and \qty{441}{\nm} is directly accessible due the invention of blue laser diodes. Blue laser diodes were first presented in 1994 by \citeauthor{Nakamura_1996} \cite{Nakamura_1996}. This was followed by the first pulsed UV laser diode developed by \citeauthor{uv_pulsed_laser_diode_first} \cite{uv_pulsed_laser_diode_first} in 1995 and 1996 the first continuous wave UV laser diode \cite{uv_cw_laser_diode_first}. These developments then warranted the 2014 Nobel Prize in physics. This progress created the opportunity to build a compact and economic laser systems for the spectroscopy of highly charged ions based entirely on diode lasers.

For both applications, laser systems were proposed and preliminary tests conducted \cite{thesis_baus,thesis_alex,thesis_tilman,thesis_seppo}. These tests have shown severe limitations in the current state of the art in diode laser technology. Most laser diode drivers commercially available are based on the work of \citeauthor{libbrecht_hall} presented in 1993 \cite{libbrecht_hall} which was designed for near-infrared laser diodes and the characteristics of blue laser diodes could not be foreseen at that time. The rapid development of blue light emitting diodes and laser diodes outpaced the development of the electronics to drive them which led to subpar performance of blue and UV diode lasers in comparison to their near-infrared brethren. Several groups have reported issues with the existing designs \cite{laser_driver_mosfet_noise} without offering solutions while others attempted to improve the front end situation by including modern digital controls \cite{laser_driver_digital}.

Such digital control is critical to stay ahead of the ever increasing complexity introduced by modern experiments. Such experiments can, for example, be found in the field of quantum computing. A promising approach is the use of large arrays of individual neutral atoms, captured using optical traps to solve the scaling problem \cite{quatum_computer_scaling_ions}. Having hundreds to thousands of quantum systems that need addressing and manipulation requires dozens of compact laser sources that need to be orchestrated. Such orchestration is no longer feasible by hand. This level of automation requires a high degree of stability over time and temperature from the laser electronics to ensure the repeatability and reliability of the system. These qualities must be paired with an outstanding noise performance to produce the high fidelity of quantum-state manipulation necessary for quantum computers. A combination of these digital features with the performance level is currently not available on the market.

This work has has now closed the gap and provides state of the art open-source laser electronics incorporating novel approaches to the design of laser current drivers and temperature controllers for the application in high precision spectroscopy and quantum computing experiments. These solutions include modern remote-controllable digital interfaces for controlling large scale setups.

This work is split into three parts. \par\vspace{12pt}

The \hyperref[sec:preparation]{\textbf{Preparation}} develops the theoretical background by giving a quick introduction into control theory, noise types and current sources. It also presents the requirements for the laser system designed for ARTEMIS.\par\vspace{12pt}

The \hyperref[sec:results]{\textbf{Results}} give a detailed comparison of several laser drivers, both commercial and academic to outline the problems discovered during the testing of a blue laser systems. A laser driver design outperforming all solutions currently available is presented along with a high stability temperature controller specifically designed for the stringent needs of high precision laser spectroscopy. Additionally a compact PID controller system for lab application is presented in the context of lab temperature control together with a data monitoring system capable of logging the manifold data accumulated in a modern experiment and environmental monitoring systems. This data can then be accessed in real-time using a graphical web front end. \par\vspace{12pt}

The \hyperref[sec:outlook]{\textbf{Outlook}} summarizes the results developed and, with the sources of electronic noise in diode lasers suppressed, exposes the final barrier imposed by the mechanical design of laser resonators.

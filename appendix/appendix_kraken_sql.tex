\documentclass[12pt]{book}
\usepackage{docmute}
\usepackage{caption}  % used in the appendix
\usepackage{listings}
\usepackage{siunitx}
\usepackage{xcolor}

\DeclareCaptionFont{white}{\color{white}}
\DeclareCaptionFormat{listing}{\colorbox{gray}{\parbox{0.95\textwidth}{#1#2#3}}}
\captionsetup[lstlisting]{format=listing,labelfont=white,textfont=white}

\begin{document}
\section{Querying the TimescaleDB via SQL}%
\label{sec:appendix_query_kraken}
SQL query to extract binned data from the Timescale DB of two sensors in lab 011: \textsf{011\_humidity} and \textsf{011\_temperature}. The data returned is of the form \textsf{date,humidity,temperature}. The database groups the asynchronous data into bins of \qty{6}{\hour} and averages the data inside those bins. The timeframe is from 2022-01-01 until 2023-01-01. In addition, the Tinkerforge sensors will only send a new value, if it has changed, the last observation must be carried forward. This is done using the \textit{locf()} function call in SQL query.

\begin{lstlisting}[language=sql,title=SQL query]
  SELECT
    time
    ,data_values [1] humidity  --1st value in the array
    ,data_values [2] temperature --2nd value
  FROM (
    SELECT
      bucket as "time"
      ,array_agg("data") as data_values
    FROM (
      SELECT
        time_bucket('6h',"time") AS "bucket"
        ,sensor_id
        ,locf(avg(value)) AS "data"
      FROM sensor_data
        WHERE
        time BETWEEN
          '2022-01-01T00:00:00.00Z' AND '2023-01-01T00:00:00Z'
        AND sensor_id IN (
          SELECT id
          FROM sensors
          WHERE
            label = '011_humidity' OR
            label = '011_temperature' AND
            enabled
        )
      GROUP BY bucket, sensor_id
      ORDER BY bucket
    ) t1
    GROUP BY "bucket"
  ) t2
\end{lstlisting}

\clearpage
When attempting to derive PID parameter for the lab temperature controller, controller output and the sensor output is needed. This query will compile the data in buckets of \qty{2}{\s}. Missing data from the Tinkerforge sensors, which will only update their output on a change, is interpolated by filling the gap with the previous value. The order of the array depends on the values of the sensor ids and needs to be adjusted accordingling for each query.

\begin{lstlisting}[language=sql,title=SQL query]
  SELECT
    timestamp as time,
    arr[2] as output,
    arr[1] as "temperature room",
    arr[3] as "temperature labnode"
  FROM (
    SELECT
      timestamp,
      array_agg(value) arr
    FROM (
      SELECT
        time_bucket_gapfill('2.000s',"time") AS "timestamp",
        sensor_id,
        locf(avg(value)) as value
      FROM sensor_data
      WHERE
        "time" BETWEEN
          '2022-09-22T04:10:00Z' AND '2022-09-22T10:30:00Z'
        AND
        sensor_id IN (
          SELECT id FROM sensors
          WHERE
            label = '011_temperature_back' OR
            label = '011_temperature_labnode_back' OR
            label ='011_output_aircon_back'
        )
      GROUP BY timestamp,sensor_id
      ORDER BY timestamp,sensor_id
    ) t1
    GROUP BY timestamp
  ) t2
\end{lstlisting}
\end{document}


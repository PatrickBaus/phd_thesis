\documentclass[12pt]{book}
\usepackage{docmute}
\usepackage{caption}  % used in the appendix
\usepackage{listings}
\usepackage{siunitx}
\usepackage{xcolor}

\DeclareCaptionFont{white}{\color{white}}
\DeclareCaptionFormat{listing}{\colorbox{gray}{\parbox{0.95\textwidth}{#1#2#3}}}
\captionsetup[lstlisting]{format=listing,labelfont=white,textfont=white}

\begin{document}
\section{Querying the TimescaleDB via SQL}
SQL query to extract binned data from the Timescale DB of two sensors in lab 011: \textsf{011\_humidity} and \textsf{011\_temperature}. The data returned is of the form \textsf{date,humidity,temperature}. The database groups the asynchronous data into bins of \qty{6}{\hour} and averages the data inside those bins. The timeframe is from 2022-01-01 until 2023-01-01.

\begin{lstlisting}[language=sql,title=SQL query]
  SELECT
    time
    ,data_values [1] humidity  --1st value in the array
    ,data_values [2] temperature --2nd value
  FROM (
    SELECT
      bucket as "time"
      ,array_agg("data") as data_values
    FROM (
      SELECT
        time_bucket('6h',"time") AS "bucket"
        ,sensor_id
        ,avg(value) AS "data"
      FROM sensor_data
        WHERE
        time BETWEEN
          '2022-01-01T00:00:00.00Z' AND '2023-01-01T00:00:00Z'
        AND sensor_id IN (
          SELECT id
          FROM sensors
          WHERE label = '011_humidity' OR label = '011_temperature'
            AND enabled
        )
      GROUP BY bucket, sensor_id
      ORDER BY 1
    ) t1
    GROUP BY "bucket"
  ) t2
\end{lstlisting}
\end{document}


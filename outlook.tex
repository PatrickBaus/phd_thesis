\chapter{Discussion and Future Perspective}%
\label{sec:outlook}
This work has pushed the boundaries of diode laser systems to a new level. Diode lasers built using the technology presented in this work will see their limits in the mechanical design rather than the control electronics. The design presented allows precision laser spectroscopy even in the presence of adverse environmental conditions that would otherwise pose serious challenges to the experimenter.

The development of such high precision devices as they are typically required for university labs has become more challenging over the recent years due to the decline in availability of many electronic parts. These shortages mostly affect microcontrollers and power electronics. It is the authors desire to make the instruments presented in this work available to the public via an open-source contribution of the design files and also to keep them available. This necessitates a few updates to the designs to make them more resilient to supply chain issues. The author proposes the following changes to the designs presented in this work.

The \device{DgDrive} laser current driver should be modified with the careful addition of an FPGA on the analog side to simplify the digital isolation between the front panel and the analog electronics. Recent developments have demonstrated that reducing the number of critical components like the digital isolators makes the construction more resilient to the varying supply and requires less requalification when replacing unavailable parts. The use of an FPGA can also simplify many of the logic elements on the board by bringing them into the FPGA core. Being able to change this logic software also improves the compatibility with future part changes. The FGPA also allows for smaller and simpler microcontroller, giving more options when adapting to supply changes. Another interesting new development that is currently pursued is the integration of the new ADI \device{ADR1399} reference Zener diode which has shown promising results regarding the burst noise issue as well as deliver a lower floor. Given those early results future batches of the current driver may not even need the additional screening process of the Zener diodes discussed in section \ref{sec:zener_diode_selection}. This would greatly simplify the manufacturing process in terms of hardware and time required.

Much of the same as said above can be said about the temperature controller. The supply situation is even more dire in this case, because many of the specialised parts come with tight specifications. For example the \device{LTC2508-32} ADC used in the analog front end is explicitly tailored to the microcontroller. Adding an FPGA front end for the ADCs can greatly simplify and even improve the performance of the system as it is right now. With the advent of new high performance ADCs in recent years an FPGA also gives the needed flexibility when choosing suitable substations if the need arises. The work on these changes was already started, but is not yet finalised.

Regarding the driver board of the temperature controller, the availability of the new ADI \device{LT8722} TEC driver is much anticipated because it integrates many of the components used on the driver board into a single chip allowing a simpler design, reducing the production cost and complexity. This presents a key feature to further the widespread adoption of the design.

These developments are independent of the benefits than can already be reaped from the designs currently available. The new laser system built for the ARTEMIS experiment is expected to deliver the much desired stability for the spectroscopy of \ce{Ar^{13+}}. Clearing the commissioning of the Penning trap is major step forward and and also opens up the road towards the spectroscopy of Bismuth (\ce{^{209}Bi^{82+}}) as the new electronics can provide the needed stability for the seed laser at \qty{976}{\nm}. Parts of this system integrating the new electronic are currently being performance tested. Some of this work was already presented in \cite{thesis_tilman}.

Additionally, the new control electronics allows the use of a wide variety of modern laser diodes and opens up other experimental avenues like the spectroscopy of highly charged Krypton \ce{Kr^{17+}} at \qty{636.9}{\nm} \cite{krypton17+,krypton17+_2}.

A final aspect that has come up with the new electronics are the limits of the mechanical design of the laser resonators. While not surprising, the unsealed laser makes for a good barometer as shown in figure \ref{fig:atomics_master_barometer}.
\begin{figure}[ht]
    \centering
    %% Creator: Matplotlib, PGF backend
%%
%% To include the figure in your LaTeX document, write
%%   \input{<filename>.pgf}
%%
%% Make sure the required packages are loaded in your preamble
%%   \usepackage{pgf}
%%
%% Also ensure that all the required font packages are loaded; for instance,
%% the lmodern package is sometimes necessary when using math font.
%%   \usepackage{lmodern}
%%
%% Figures using additional raster images can only be included by \input if
%% they are in the same directory as the main LaTeX file. For loading figures
%% from other directories you can use the `import` package
%%   \usepackage{import}
%%
%% and then include the figures with
%%   \import{<path to file>}{<filename>.pgf}
%%
%% Matplotlib used the following preamble
%%   \usepackage{siunitx}
%%   \usepackage{fontspec}
%%   \makeatletter\@ifpackageloaded{underscore}{}{\usepackage[strings]{underscore}}\makeatother
%%
\begingroup%
\makeatletter%
\begin{pgfpicture}%
\pgfpathrectangle{\pgfpointorigin}{\pgfqpoint{5.431103in}{3.356606in}}%
\pgfusepath{use as bounding box, clip}%
\begin{pgfscope}%
\pgfsetbuttcap%
\pgfsetmiterjoin%
\definecolor{currentfill}{rgb}{1.000000,1.000000,1.000000}%
\pgfsetfillcolor{currentfill}%
\pgfsetlinewidth{0.000000pt}%
\definecolor{currentstroke}{rgb}{1.000000,1.000000,1.000000}%
\pgfsetstrokecolor{currentstroke}%
\pgfsetdash{}{0pt}%
\pgfpathmoveto{\pgfqpoint{0.000000in}{0.000000in}}%
\pgfpathlineto{\pgfqpoint{5.431103in}{0.000000in}}%
\pgfpathlineto{\pgfqpoint{5.431103in}{3.356606in}}%
\pgfpathlineto{\pgfqpoint{0.000000in}{3.356606in}}%
\pgfpathlineto{\pgfqpoint{0.000000in}{0.000000in}}%
\pgfpathclose%
\pgfusepath{fill}%
\end{pgfscope}%
\begin{pgfscope}%
\pgfsetbuttcap%
\pgfsetmiterjoin%
\definecolor{currentfill}{rgb}{1.000000,1.000000,1.000000}%
\pgfsetfillcolor{currentfill}%
\pgfsetlinewidth{0.000000pt}%
\definecolor{currentstroke}{rgb}{0.000000,0.000000,0.000000}%
\pgfsetstrokecolor{currentstroke}%
\pgfsetstrokeopacity{0.000000}%
\pgfsetdash{}{0pt}%
\pgfpathmoveto{\pgfqpoint{0.661006in}{2.135160in}}%
\pgfpathlineto{\pgfqpoint{4.855042in}{2.135160in}}%
\pgfpathlineto{\pgfqpoint{4.855042in}{3.206606in}}%
\pgfpathlineto{\pgfqpoint{0.661006in}{3.206606in}}%
\pgfpathlineto{\pgfqpoint{0.661006in}{2.135160in}}%
\pgfpathclose%
\pgfusepath{fill}%
\end{pgfscope}%
\begin{pgfscope}%
\pgfpathrectangle{\pgfqpoint{0.661006in}{2.135160in}}{\pgfqpoint{4.194036in}{1.071446in}}%
\pgfusepath{clip}%
\pgfsetrectcap%
\pgfsetroundjoin%
\pgfsetlinewidth{0.803000pt}%
\definecolor{currentstroke}{rgb}{0.450000,0.450000,0.450000}%
\pgfsetstrokecolor{currentstroke}%
\pgfsetdash{}{0pt}%
\pgfpathmoveto{\pgfqpoint{0.851644in}{2.135160in}}%
\pgfpathlineto{\pgfqpoint{0.851644in}{3.206606in}}%
\pgfusepath{stroke}%
\end{pgfscope}%
\begin{pgfscope}%
\pgfsetbuttcap%
\pgfsetroundjoin%
\definecolor{currentfill}{rgb}{0.000000,0.000000,0.000000}%
\pgfsetfillcolor{currentfill}%
\pgfsetlinewidth{0.803000pt}%
\definecolor{currentstroke}{rgb}{0.000000,0.000000,0.000000}%
\pgfsetstrokecolor{currentstroke}%
\pgfsetdash{}{0pt}%
\pgfsys@defobject{currentmarker}{\pgfqpoint{0.000000in}{-0.048611in}}{\pgfqpoint{0.000000in}{0.000000in}}{%
\pgfpathmoveto{\pgfqpoint{0.000000in}{0.000000in}}%
\pgfpathlineto{\pgfqpoint{0.000000in}{-0.048611in}}%
\pgfusepath{stroke,fill}%
}%
\begin{pgfscope}%
\pgfsys@transformshift{0.851644in}{2.135160in}%
\pgfsys@useobject{currentmarker}{}%
\end{pgfscope}%
\end{pgfscope}%
\begin{pgfscope}%
\definecolor{textcolor}{rgb}{0.000000,0.000000,0.000000}%
\pgfsetstrokecolor{textcolor}%
\pgfsetfillcolor{textcolor}%
\pgftext[x=0.851644in,y=2.037938in,,top]{\color{textcolor}\rmfamily\fontsize{8.000000}{9.600000}\selectfont \(\displaystyle {04{:}00}\)}%
\end{pgfscope}%
\begin{pgfscope}%
\pgfpathrectangle{\pgfqpoint{0.661006in}{2.135160in}}{\pgfqpoint{4.194036in}{1.071446in}}%
\pgfusepath{clip}%
\pgfsetrectcap%
\pgfsetroundjoin%
\pgfsetlinewidth{0.803000pt}%
\definecolor{currentstroke}{rgb}{0.450000,0.450000,0.450000}%
\pgfsetstrokecolor{currentstroke}%
\pgfsetdash{}{0pt}%
\pgfpathmoveto{\pgfqpoint{1.232920in}{2.135160in}}%
\pgfpathlineto{\pgfqpoint{1.232920in}{3.206606in}}%
\pgfusepath{stroke}%
\end{pgfscope}%
\begin{pgfscope}%
\pgfsetbuttcap%
\pgfsetroundjoin%
\definecolor{currentfill}{rgb}{0.000000,0.000000,0.000000}%
\pgfsetfillcolor{currentfill}%
\pgfsetlinewidth{0.803000pt}%
\definecolor{currentstroke}{rgb}{0.000000,0.000000,0.000000}%
\pgfsetstrokecolor{currentstroke}%
\pgfsetdash{}{0pt}%
\pgfsys@defobject{currentmarker}{\pgfqpoint{0.000000in}{-0.048611in}}{\pgfqpoint{0.000000in}{0.000000in}}{%
\pgfpathmoveto{\pgfqpoint{0.000000in}{0.000000in}}%
\pgfpathlineto{\pgfqpoint{0.000000in}{-0.048611in}}%
\pgfusepath{stroke,fill}%
}%
\begin{pgfscope}%
\pgfsys@transformshift{1.232920in}{2.135160in}%
\pgfsys@useobject{currentmarker}{}%
\end{pgfscope}%
\end{pgfscope}%
\begin{pgfscope}%
\definecolor{textcolor}{rgb}{0.000000,0.000000,0.000000}%
\pgfsetstrokecolor{textcolor}%
\pgfsetfillcolor{textcolor}%
\pgftext[x=1.232920in,y=2.037938in,,top]{\color{textcolor}\rmfamily\fontsize{8.000000}{9.600000}\selectfont \(\displaystyle {05{:}00}\)}%
\end{pgfscope}%
\begin{pgfscope}%
\pgfpathrectangle{\pgfqpoint{0.661006in}{2.135160in}}{\pgfqpoint{4.194036in}{1.071446in}}%
\pgfusepath{clip}%
\pgfsetrectcap%
\pgfsetroundjoin%
\pgfsetlinewidth{0.803000pt}%
\definecolor{currentstroke}{rgb}{0.450000,0.450000,0.450000}%
\pgfsetstrokecolor{currentstroke}%
\pgfsetdash{}{0pt}%
\pgfpathmoveto{\pgfqpoint{1.614196in}{2.135160in}}%
\pgfpathlineto{\pgfqpoint{1.614196in}{3.206606in}}%
\pgfusepath{stroke}%
\end{pgfscope}%
\begin{pgfscope}%
\pgfsetbuttcap%
\pgfsetroundjoin%
\definecolor{currentfill}{rgb}{0.000000,0.000000,0.000000}%
\pgfsetfillcolor{currentfill}%
\pgfsetlinewidth{0.803000pt}%
\definecolor{currentstroke}{rgb}{0.000000,0.000000,0.000000}%
\pgfsetstrokecolor{currentstroke}%
\pgfsetdash{}{0pt}%
\pgfsys@defobject{currentmarker}{\pgfqpoint{0.000000in}{-0.048611in}}{\pgfqpoint{0.000000in}{0.000000in}}{%
\pgfpathmoveto{\pgfqpoint{0.000000in}{0.000000in}}%
\pgfpathlineto{\pgfqpoint{0.000000in}{-0.048611in}}%
\pgfusepath{stroke,fill}%
}%
\begin{pgfscope}%
\pgfsys@transformshift{1.614196in}{2.135160in}%
\pgfsys@useobject{currentmarker}{}%
\end{pgfscope}%
\end{pgfscope}%
\begin{pgfscope}%
\definecolor{textcolor}{rgb}{0.000000,0.000000,0.000000}%
\pgfsetstrokecolor{textcolor}%
\pgfsetfillcolor{textcolor}%
\pgftext[x=1.614196in,y=2.037938in,,top]{\color{textcolor}\rmfamily\fontsize{8.000000}{9.600000}\selectfont \(\displaystyle {06{:}00}\)}%
\end{pgfscope}%
\begin{pgfscope}%
\pgfpathrectangle{\pgfqpoint{0.661006in}{2.135160in}}{\pgfqpoint{4.194036in}{1.071446in}}%
\pgfusepath{clip}%
\pgfsetrectcap%
\pgfsetroundjoin%
\pgfsetlinewidth{0.803000pt}%
\definecolor{currentstroke}{rgb}{0.450000,0.450000,0.450000}%
\pgfsetstrokecolor{currentstroke}%
\pgfsetdash{}{0pt}%
\pgfpathmoveto{\pgfqpoint{1.995472in}{2.135160in}}%
\pgfpathlineto{\pgfqpoint{1.995472in}{3.206606in}}%
\pgfusepath{stroke}%
\end{pgfscope}%
\begin{pgfscope}%
\pgfsetbuttcap%
\pgfsetroundjoin%
\definecolor{currentfill}{rgb}{0.000000,0.000000,0.000000}%
\pgfsetfillcolor{currentfill}%
\pgfsetlinewidth{0.803000pt}%
\definecolor{currentstroke}{rgb}{0.000000,0.000000,0.000000}%
\pgfsetstrokecolor{currentstroke}%
\pgfsetdash{}{0pt}%
\pgfsys@defobject{currentmarker}{\pgfqpoint{0.000000in}{-0.048611in}}{\pgfqpoint{0.000000in}{0.000000in}}{%
\pgfpathmoveto{\pgfqpoint{0.000000in}{0.000000in}}%
\pgfpathlineto{\pgfqpoint{0.000000in}{-0.048611in}}%
\pgfusepath{stroke,fill}%
}%
\begin{pgfscope}%
\pgfsys@transformshift{1.995472in}{2.135160in}%
\pgfsys@useobject{currentmarker}{}%
\end{pgfscope}%
\end{pgfscope}%
\begin{pgfscope}%
\definecolor{textcolor}{rgb}{0.000000,0.000000,0.000000}%
\pgfsetstrokecolor{textcolor}%
\pgfsetfillcolor{textcolor}%
\pgftext[x=1.995472in,y=2.037938in,,top]{\color{textcolor}\rmfamily\fontsize{8.000000}{9.600000}\selectfont \(\displaystyle {07{:}00}\)}%
\end{pgfscope}%
\begin{pgfscope}%
\pgfpathrectangle{\pgfqpoint{0.661006in}{2.135160in}}{\pgfqpoint{4.194036in}{1.071446in}}%
\pgfusepath{clip}%
\pgfsetrectcap%
\pgfsetroundjoin%
\pgfsetlinewidth{0.803000pt}%
\definecolor{currentstroke}{rgb}{0.450000,0.450000,0.450000}%
\pgfsetstrokecolor{currentstroke}%
\pgfsetdash{}{0pt}%
\pgfpathmoveto{\pgfqpoint{2.376748in}{2.135160in}}%
\pgfpathlineto{\pgfqpoint{2.376748in}{3.206606in}}%
\pgfusepath{stroke}%
\end{pgfscope}%
\begin{pgfscope}%
\pgfsetbuttcap%
\pgfsetroundjoin%
\definecolor{currentfill}{rgb}{0.000000,0.000000,0.000000}%
\pgfsetfillcolor{currentfill}%
\pgfsetlinewidth{0.803000pt}%
\definecolor{currentstroke}{rgb}{0.000000,0.000000,0.000000}%
\pgfsetstrokecolor{currentstroke}%
\pgfsetdash{}{0pt}%
\pgfsys@defobject{currentmarker}{\pgfqpoint{0.000000in}{-0.048611in}}{\pgfqpoint{0.000000in}{0.000000in}}{%
\pgfpathmoveto{\pgfqpoint{0.000000in}{0.000000in}}%
\pgfpathlineto{\pgfqpoint{0.000000in}{-0.048611in}}%
\pgfusepath{stroke,fill}%
}%
\begin{pgfscope}%
\pgfsys@transformshift{2.376748in}{2.135160in}%
\pgfsys@useobject{currentmarker}{}%
\end{pgfscope}%
\end{pgfscope}%
\begin{pgfscope}%
\definecolor{textcolor}{rgb}{0.000000,0.000000,0.000000}%
\pgfsetstrokecolor{textcolor}%
\pgfsetfillcolor{textcolor}%
\pgftext[x=2.376748in,y=2.037938in,,top]{\color{textcolor}\rmfamily\fontsize{8.000000}{9.600000}\selectfont \(\displaystyle {08{:}00}\)}%
\end{pgfscope}%
\begin{pgfscope}%
\pgfpathrectangle{\pgfqpoint{0.661006in}{2.135160in}}{\pgfqpoint{4.194036in}{1.071446in}}%
\pgfusepath{clip}%
\pgfsetrectcap%
\pgfsetroundjoin%
\pgfsetlinewidth{0.803000pt}%
\definecolor{currentstroke}{rgb}{0.450000,0.450000,0.450000}%
\pgfsetstrokecolor{currentstroke}%
\pgfsetdash{}{0pt}%
\pgfpathmoveto{\pgfqpoint{2.758024in}{2.135160in}}%
\pgfpathlineto{\pgfqpoint{2.758024in}{3.206606in}}%
\pgfusepath{stroke}%
\end{pgfscope}%
\begin{pgfscope}%
\pgfsetbuttcap%
\pgfsetroundjoin%
\definecolor{currentfill}{rgb}{0.000000,0.000000,0.000000}%
\pgfsetfillcolor{currentfill}%
\pgfsetlinewidth{0.803000pt}%
\definecolor{currentstroke}{rgb}{0.000000,0.000000,0.000000}%
\pgfsetstrokecolor{currentstroke}%
\pgfsetdash{}{0pt}%
\pgfsys@defobject{currentmarker}{\pgfqpoint{0.000000in}{-0.048611in}}{\pgfqpoint{0.000000in}{0.000000in}}{%
\pgfpathmoveto{\pgfqpoint{0.000000in}{0.000000in}}%
\pgfpathlineto{\pgfqpoint{0.000000in}{-0.048611in}}%
\pgfusepath{stroke,fill}%
}%
\begin{pgfscope}%
\pgfsys@transformshift{2.758024in}{2.135160in}%
\pgfsys@useobject{currentmarker}{}%
\end{pgfscope}%
\end{pgfscope}%
\begin{pgfscope}%
\definecolor{textcolor}{rgb}{0.000000,0.000000,0.000000}%
\pgfsetstrokecolor{textcolor}%
\pgfsetfillcolor{textcolor}%
\pgftext[x=2.758024in,y=2.037938in,,top]{\color{textcolor}\rmfamily\fontsize{8.000000}{9.600000}\selectfont \(\displaystyle {09{:}00}\)}%
\end{pgfscope}%
\begin{pgfscope}%
\pgfpathrectangle{\pgfqpoint{0.661006in}{2.135160in}}{\pgfqpoint{4.194036in}{1.071446in}}%
\pgfusepath{clip}%
\pgfsetrectcap%
\pgfsetroundjoin%
\pgfsetlinewidth{0.803000pt}%
\definecolor{currentstroke}{rgb}{0.450000,0.450000,0.450000}%
\pgfsetstrokecolor{currentstroke}%
\pgfsetdash{}{0pt}%
\pgfpathmoveto{\pgfqpoint{3.139300in}{2.135160in}}%
\pgfpathlineto{\pgfqpoint{3.139300in}{3.206606in}}%
\pgfusepath{stroke}%
\end{pgfscope}%
\begin{pgfscope}%
\pgfsetbuttcap%
\pgfsetroundjoin%
\definecolor{currentfill}{rgb}{0.000000,0.000000,0.000000}%
\pgfsetfillcolor{currentfill}%
\pgfsetlinewidth{0.803000pt}%
\definecolor{currentstroke}{rgb}{0.000000,0.000000,0.000000}%
\pgfsetstrokecolor{currentstroke}%
\pgfsetdash{}{0pt}%
\pgfsys@defobject{currentmarker}{\pgfqpoint{0.000000in}{-0.048611in}}{\pgfqpoint{0.000000in}{0.000000in}}{%
\pgfpathmoveto{\pgfqpoint{0.000000in}{0.000000in}}%
\pgfpathlineto{\pgfqpoint{0.000000in}{-0.048611in}}%
\pgfusepath{stroke,fill}%
}%
\begin{pgfscope}%
\pgfsys@transformshift{3.139300in}{2.135160in}%
\pgfsys@useobject{currentmarker}{}%
\end{pgfscope}%
\end{pgfscope}%
\begin{pgfscope}%
\definecolor{textcolor}{rgb}{0.000000,0.000000,0.000000}%
\pgfsetstrokecolor{textcolor}%
\pgfsetfillcolor{textcolor}%
\pgftext[x=3.139300in,y=2.037938in,,top]{\color{textcolor}\rmfamily\fontsize{8.000000}{9.600000}\selectfont \(\displaystyle {10{:}00}\)}%
\end{pgfscope}%
\begin{pgfscope}%
\pgfpathrectangle{\pgfqpoint{0.661006in}{2.135160in}}{\pgfqpoint{4.194036in}{1.071446in}}%
\pgfusepath{clip}%
\pgfsetrectcap%
\pgfsetroundjoin%
\pgfsetlinewidth{0.803000pt}%
\definecolor{currentstroke}{rgb}{0.450000,0.450000,0.450000}%
\pgfsetstrokecolor{currentstroke}%
\pgfsetdash{}{0pt}%
\pgfpathmoveto{\pgfqpoint{3.520576in}{2.135160in}}%
\pgfpathlineto{\pgfqpoint{3.520576in}{3.206606in}}%
\pgfusepath{stroke}%
\end{pgfscope}%
\begin{pgfscope}%
\pgfsetbuttcap%
\pgfsetroundjoin%
\definecolor{currentfill}{rgb}{0.000000,0.000000,0.000000}%
\pgfsetfillcolor{currentfill}%
\pgfsetlinewidth{0.803000pt}%
\definecolor{currentstroke}{rgb}{0.000000,0.000000,0.000000}%
\pgfsetstrokecolor{currentstroke}%
\pgfsetdash{}{0pt}%
\pgfsys@defobject{currentmarker}{\pgfqpoint{0.000000in}{-0.048611in}}{\pgfqpoint{0.000000in}{0.000000in}}{%
\pgfpathmoveto{\pgfqpoint{0.000000in}{0.000000in}}%
\pgfpathlineto{\pgfqpoint{0.000000in}{-0.048611in}}%
\pgfusepath{stroke,fill}%
}%
\begin{pgfscope}%
\pgfsys@transformshift{3.520576in}{2.135160in}%
\pgfsys@useobject{currentmarker}{}%
\end{pgfscope}%
\end{pgfscope}%
\begin{pgfscope}%
\definecolor{textcolor}{rgb}{0.000000,0.000000,0.000000}%
\pgfsetstrokecolor{textcolor}%
\pgfsetfillcolor{textcolor}%
\pgftext[x=3.520576in,y=2.037938in,,top]{\color{textcolor}\rmfamily\fontsize{8.000000}{9.600000}\selectfont \(\displaystyle {11{:}00}\)}%
\end{pgfscope}%
\begin{pgfscope}%
\pgfpathrectangle{\pgfqpoint{0.661006in}{2.135160in}}{\pgfqpoint{4.194036in}{1.071446in}}%
\pgfusepath{clip}%
\pgfsetrectcap%
\pgfsetroundjoin%
\pgfsetlinewidth{0.803000pt}%
\definecolor{currentstroke}{rgb}{0.450000,0.450000,0.450000}%
\pgfsetstrokecolor{currentstroke}%
\pgfsetdash{}{0pt}%
\pgfpathmoveto{\pgfqpoint{3.901852in}{2.135160in}}%
\pgfpathlineto{\pgfqpoint{3.901852in}{3.206606in}}%
\pgfusepath{stroke}%
\end{pgfscope}%
\begin{pgfscope}%
\pgfsetbuttcap%
\pgfsetroundjoin%
\definecolor{currentfill}{rgb}{0.000000,0.000000,0.000000}%
\pgfsetfillcolor{currentfill}%
\pgfsetlinewidth{0.803000pt}%
\definecolor{currentstroke}{rgb}{0.000000,0.000000,0.000000}%
\pgfsetstrokecolor{currentstroke}%
\pgfsetdash{}{0pt}%
\pgfsys@defobject{currentmarker}{\pgfqpoint{0.000000in}{-0.048611in}}{\pgfqpoint{0.000000in}{0.000000in}}{%
\pgfpathmoveto{\pgfqpoint{0.000000in}{0.000000in}}%
\pgfpathlineto{\pgfqpoint{0.000000in}{-0.048611in}}%
\pgfusepath{stroke,fill}%
}%
\begin{pgfscope}%
\pgfsys@transformshift{3.901852in}{2.135160in}%
\pgfsys@useobject{currentmarker}{}%
\end{pgfscope}%
\end{pgfscope}%
\begin{pgfscope}%
\definecolor{textcolor}{rgb}{0.000000,0.000000,0.000000}%
\pgfsetstrokecolor{textcolor}%
\pgfsetfillcolor{textcolor}%
\pgftext[x=3.901852in,y=2.037938in,,top]{\color{textcolor}\rmfamily\fontsize{8.000000}{9.600000}\selectfont \(\displaystyle {12{:}00}\)}%
\end{pgfscope}%
\begin{pgfscope}%
\pgfpathrectangle{\pgfqpoint{0.661006in}{2.135160in}}{\pgfqpoint{4.194036in}{1.071446in}}%
\pgfusepath{clip}%
\pgfsetrectcap%
\pgfsetroundjoin%
\pgfsetlinewidth{0.803000pt}%
\definecolor{currentstroke}{rgb}{0.450000,0.450000,0.450000}%
\pgfsetstrokecolor{currentstroke}%
\pgfsetdash{}{0pt}%
\pgfpathmoveto{\pgfqpoint{4.283128in}{2.135160in}}%
\pgfpathlineto{\pgfqpoint{4.283128in}{3.206606in}}%
\pgfusepath{stroke}%
\end{pgfscope}%
\begin{pgfscope}%
\pgfsetbuttcap%
\pgfsetroundjoin%
\definecolor{currentfill}{rgb}{0.000000,0.000000,0.000000}%
\pgfsetfillcolor{currentfill}%
\pgfsetlinewidth{0.803000pt}%
\definecolor{currentstroke}{rgb}{0.000000,0.000000,0.000000}%
\pgfsetstrokecolor{currentstroke}%
\pgfsetdash{}{0pt}%
\pgfsys@defobject{currentmarker}{\pgfqpoint{0.000000in}{-0.048611in}}{\pgfqpoint{0.000000in}{0.000000in}}{%
\pgfpathmoveto{\pgfqpoint{0.000000in}{0.000000in}}%
\pgfpathlineto{\pgfqpoint{0.000000in}{-0.048611in}}%
\pgfusepath{stroke,fill}%
}%
\begin{pgfscope}%
\pgfsys@transformshift{4.283128in}{2.135160in}%
\pgfsys@useobject{currentmarker}{}%
\end{pgfscope}%
\end{pgfscope}%
\begin{pgfscope}%
\definecolor{textcolor}{rgb}{0.000000,0.000000,0.000000}%
\pgfsetstrokecolor{textcolor}%
\pgfsetfillcolor{textcolor}%
\pgftext[x=4.283128in,y=2.037938in,,top]{\color{textcolor}\rmfamily\fontsize{8.000000}{9.600000}\selectfont \(\displaystyle {13{:}00}\)}%
\end{pgfscope}%
\begin{pgfscope}%
\pgfpathrectangle{\pgfqpoint{0.661006in}{2.135160in}}{\pgfqpoint{4.194036in}{1.071446in}}%
\pgfusepath{clip}%
\pgfsetrectcap%
\pgfsetroundjoin%
\pgfsetlinewidth{0.803000pt}%
\definecolor{currentstroke}{rgb}{0.450000,0.450000,0.450000}%
\pgfsetstrokecolor{currentstroke}%
\pgfsetdash{}{0pt}%
\pgfpathmoveto{\pgfqpoint{4.664404in}{2.135160in}}%
\pgfpathlineto{\pgfqpoint{4.664404in}{3.206606in}}%
\pgfusepath{stroke}%
\end{pgfscope}%
\begin{pgfscope}%
\pgfsetbuttcap%
\pgfsetroundjoin%
\definecolor{currentfill}{rgb}{0.000000,0.000000,0.000000}%
\pgfsetfillcolor{currentfill}%
\pgfsetlinewidth{0.803000pt}%
\definecolor{currentstroke}{rgb}{0.000000,0.000000,0.000000}%
\pgfsetstrokecolor{currentstroke}%
\pgfsetdash{}{0pt}%
\pgfsys@defobject{currentmarker}{\pgfqpoint{0.000000in}{-0.048611in}}{\pgfqpoint{0.000000in}{0.000000in}}{%
\pgfpathmoveto{\pgfqpoint{0.000000in}{0.000000in}}%
\pgfpathlineto{\pgfqpoint{0.000000in}{-0.048611in}}%
\pgfusepath{stroke,fill}%
}%
\begin{pgfscope}%
\pgfsys@transformshift{4.664404in}{2.135160in}%
\pgfsys@useobject{currentmarker}{}%
\end{pgfscope}%
\end{pgfscope}%
\begin{pgfscope}%
\definecolor{textcolor}{rgb}{0.000000,0.000000,0.000000}%
\pgfsetstrokecolor{textcolor}%
\pgfsetfillcolor{textcolor}%
\pgftext[x=4.664404in,y=2.037938in,,top]{\color{textcolor}\rmfamily\fontsize{8.000000}{9.600000}\selectfont \(\displaystyle {14{:}00}\)}%
\end{pgfscope}%
\begin{pgfscope}%
\definecolor{textcolor}{rgb}{0.000000,0.000000,0.000000}%
\pgfsetstrokecolor{textcolor}%
\pgfsetfillcolor{textcolor}%
\pgftext[x=2.758024in,y=1.883716in,,top]{\color{textcolor}\rmfamily\fontsize{10.000000}{12.000000}\selectfont Time (UTC)}%
\end{pgfscope}%
\begin{pgfscope}%
\definecolor{textcolor}{rgb}{0.000000,0.000000,0.000000}%
\pgfsetstrokecolor{textcolor}%
\pgfsetfillcolor{textcolor}%
\pgftext[x=4.855042in,y=1.897605in,right,top]{\color{textcolor}\rmfamily\fontsize{8.000000}{9.600000}\selectfont \(\displaystyle {2022{-}}\)Jun\(\displaystyle {{-}20}\)}%
\end{pgfscope}%
\begin{pgfscope}%
\pgfpathrectangle{\pgfqpoint{0.661006in}{2.135160in}}{\pgfqpoint{4.194036in}{1.071446in}}%
\pgfusepath{clip}%
\pgfsetrectcap%
\pgfsetroundjoin%
\pgfsetlinewidth{0.803000pt}%
\definecolor{currentstroke}{rgb}{0.450000,0.450000,0.450000}%
\pgfsetstrokecolor{currentstroke}%
\pgfsetdash{}{0pt}%
\pgfpathmoveto{\pgfqpoint{0.661006in}{2.322984in}}%
\pgfpathlineto{\pgfqpoint{4.855042in}{2.322984in}}%
\pgfusepath{stroke}%
\end{pgfscope}%
\begin{pgfscope}%
\pgfsetbuttcap%
\pgfsetroundjoin%
\definecolor{currentfill}{rgb}{0.000000,0.000000,0.000000}%
\pgfsetfillcolor{currentfill}%
\pgfsetlinewidth{0.803000pt}%
\definecolor{currentstroke}{rgb}{0.000000,0.000000,0.000000}%
\pgfsetstrokecolor{currentstroke}%
\pgfsetdash{}{0pt}%
\pgfsys@defobject{currentmarker}{\pgfqpoint{-0.048611in}{0.000000in}}{\pgfqpoint{-0.000000in}{0.000000in}}{%
\pgfpathmoveto{\pgfqpoint{-0.000000in}{0.000000in}}%
\pgfpathlineto{\pgfqpoint{-0.048611in}{0.000000in}}%
\pgfusepath{stroke,fill}%
}%
\begin{pgfscope}%
\pgfsys@transformshift{0.661006in}{2.322984in}%
\pgfsys@useobject{currentmarker}{}%
\end{pgfscope}%
\end{pgfscope}%
\begin{pgfscope}%
\definecolor{textcolor}{rgb}{0.000000,0.000000,0.000000}%
\pgfsetstrokecolor{textcolor}%
\pgfsetfillcolor{textcolor}%
\pgftext[x=0.386698in, y=2.284428in, left, base]{\color{textcolor}\rmfamily\fontsize{8.000000}{9.600000}\selectfont \(\displaystyle {996}\)}%
\end{pgfscope}%
\begin{pgfscope}%
\pgfpathrectangle{\pgfqpoint{0.661006in}{2.135160in}}{\pgfqpoint{4.194036in}{1.071446in}}%
\pgfusepath{clip}%
\pgfsetrectcap%
\pgfsetroundjoin%
\pgfsetlinewidth{0.803000pt}%
\definecolor{currentstroke}{rgb}{0.450000,0.450000,0.450000}%
\pgfsetstrokecolor{currentstroke}%
\pgfsetdash{}{0pt}%
\pgfpathmoveto{\pgfqpoint{0.661006in}{2.631695in}}%
\pgfpathlineto{\pgfqpoint{4.855042in}{2.631695in}}%
\pgfusepath{stroke}%
\end{pgfscope}%
\begin{pgfscope}%
\pgfsetbuttcap%
\pgfsetroundjoin%
\definecolor{currentfill}{rgb}{0.000000,0.000000,0.000000}%
\pgfsetfillcolor{currentfill}%
\pgfsetlinewidth{0.803000pt}%
\definecolor{currentstroke}{rgb}{0.000000,0.000000,0.000000}%
\pgfsetstrokecolor{currentstroke}%
\pgfsetdash{}{0pt}%
\pgfsys@defobject{currentmarker}{\pgfqpoint{-0.048611in}{0.000000in}}{\pgfqpoint{-0.000000in}{0.000000in}}{%
\pgfpathmoveto{\pgfqpoint{-0.000000in}{0.000000in}}%
\pgfpathlineto{\pgfqpoint{-0.048611in}{0.000000in}}%
\pgfusepath{stroke,fill}%
}%
\begin{pgfscope}%
\pgfsys@transformshift{0.661006in}{2.631695in}%
\pgfsys@useobject{currentmarker}{}%
\end{pgfscope}%
\end{pgfscope}%
\begin{pgfscope}%
\definecolor{textcolor}{rgb}{0.000000,0.000000,0.000000}%
\pgfsetstrokecolor{textcolor}%
\pgfsetfillcolor{textcolor}%
\pgftext[x=0.386698in, y=2.593139in, left, base]{\color{textcolor}\rmfamily\fontsize{8.000000}{9.600000}\selectfont \(\displaystyle {998}\)}%
\end{pgfscope}%
\begin{pgfscope}%
\pgfpathrectangle{\pgfqpoint{0.661006in}{2.135160in}}{\pgfqpoint{4.194036in}{1.071446in}}%
\pgfusepath{clip}%
\pgfsetrectcap%
\pgfsetroundjoin%
\pgfsetlinewidth{0.803000pt}%
\definecolor{currentstroke}{rgb}{0.450000,0.450000,0.450000}%
\pgfsetstrokecolor{currentstroke}%
\pgfsetdash{}{0pt}%
\pgfpathmoveto{\pgfqpoint{0.661006in}{2.940405in}}%
\pgfpathlineto{\pgfqpoint{4.855042in}{2.940405in}}%
\pgfusepath{stroke}%
\end{pgfscope}%
\begin{pgfscope}%
\pgfsetbuttcap%
\pgfsetroundjoin%
\definecolor{currentfill}{rgb}{0.000000,0.000000,0.000000}%
\pgfsetfillcolor{currentfill}%
\pgfsetlinewidth{0.803000pt}%
\definecolor{currentstroke}{rgb}{0.000000,0.000000,0.000000}%
\pgfsetstrokecolor{currentstroke}%
\pgfsetdash{}{0pt}%
\pgfsys@defobject{currentmarker}{\pgfqpoint{-0.048611in}{0.000000in}}{\pgfqpoint{-0.000000in}{0.000000in}}{%
\pgfpathmoveto{\pgfqpoint{-0.000000in}{0.000000in}}%
\pgfpathlineto{\pgfqpoint{-0.048611in}{0.000000in}}%
\pgfusepath{stroke,fill}%
}%
\begin{pgfscope}%
\pgfsys@transformshift{0.661006in}{2.940405in}%
\pgfsys@useobject{currentmarker}{}%
\end{pgfscope}%
\end{pgfscope}%
\begin{pgfscope}%
\definecolor{textcolor}{rgb}{0.000000,0.000000,0.000000}%
\pgfsetstrokecolor{textcolor}%
\pgfsetfillcolor{textcolor}%
\pgftext[x=0.327669in, y=2.901850in, left, base]{\color{textcolor}\rmfamily\fontsize{8.000000}{9.600000}\selectfont \(\displaystyle {1000}\)}%
\end{pgfscope}%
\begin{pgfscope}%
\definecolor{textcolor}{rgb}{0.000000,0.000000,0.000000}%
\pgfsetstrokecolor{textcolor}%
\pgfsetfillcolor{textcolor}%
\pgftext[x=0.272113in,y=2.670883in,,bottom,rotate=90.000000]{\color{textcolor}\rmfamily\fontsize{10.000000}{12.000000}\selectfont Pressue in \unit{\hecto\pascal}}%
\end{pgfscope}%
\begin{pgfscope}%
\pgfpathrectangle{\pgfqpoint{0.661006in}{2.135160in}}{\pgfqpoint{4.194036in}{1.071446in}}%
\pgfusepath{clip}%
\pgfsetrectcap%
\pgfsetroundjoin%
\pgfsetlinewidth{1.505625pt}%
\definecolor{currentstroke}{rgb}{0.003922,0.450980,0.698039}%
\pgfsetstrokecolor{currentstroke}%
\pgfsetstrokeopacity{0.700000}%
\pgfsetdash{}{0pt}%
\pgfpathmoveto{\pgfqpoint{0.851644in}{2.194156in}}%
\pgfpathlineto{\pgfqpoint{0.857998in}{2.190571in}}%
\pgfpathlineto{\pgfqpoint{0.864353in}{2.190559in}}%
\pgfpathlineto{\pgfqpoint{0.870707in}{2.189146in}}%
\pgfpathlineto{\pgfqpoint{0.883417in}{2.193123in}}%
\pgfpathlineto{\pgfqpoint{0.889771in}{2.197232in}}%
\pgfpathlineto{\pgfqpoint{0.896126in}{2.196270in}}%
\pgfpathlineto{\pgfqpoint{0.915190in}{2.183862in}}%
\pgfpathlineto{\pgfqpoint{0.946963in}{2.188089in}}%
\pgfpathlineto{\pgfqpoint{0.953317in}{2.190856in}}%
\pgfpathlineto{\pgfqpoint{0.959672in}{2.195122in}}%
\pgfpathlineto{\pgfqpoint{0.966026in}{2.196650in}}%
\pgfpathlineto{\pgfqpoint{0.972381in}{2.199559in}}%
\pgfpathlineto{\pgfqpoint{0.991445in}{2.214377in}}%
\pgfpathlineto{\pgfqpoint{0.997799in}{2.216194in}}%
\pgfpathlineto{\pgfqpoint{1.004154in}{2.216645in}}%
\pgfpathlineto{\pgfqpoint{1.016863in}{2.213700in}}%
\pgfpathlineto{\pgfqpoint{1.023218in}{2.214116in}}%
\pgfpathlineto{\pgfqpoint{1.035927in}{2.216964in}}%
\pgfpathlineto{\pgfqpoint{1.042282in}{2.214341in}}%
\pgfpathlineto{\pgfqpoint{1.048636in}{2.213807in}}%
\pgfpathlineto{\pgfqpoint{1.067700in}{2.219221in}}%
\pgfpathlineto{\pgfqpoint{1.074055in}{2.221311in}}%
\pgfpathlineto{\pgfqpoint{1.080409in}{2.225206in}}%
\pgfpathlineto{\pgfqpoint{1.086764in}{2.230798in}}%
\pgfpathlineto{\pgfqpoint{1.105828in}{2.242648in}}%
\pgfpathlineto{\pgfqpoint{1.112182in}{2.247302in}}%
\pgfpathlineto{\pgfqpoint{1.124891in}{2.267261in}}%
\pgfpathlineto{\pgfqpoint{1.131246in}{2.270978in}}%
\pgfpathlineto{\pgfqpoint{1.137601in}{2.266846in}}%
\pgfpathlineto{\pgfqpoint{1.143955in}{2.260019in}}%
\pgfpathlineto{\pgfqpoint{1.150310in}{2.248786in}}%
\pgfpathlineto{\pgfqpoint{1.163019in}{2.233707in}}%
\pgfpathlineto{\pgfqpoint{1.182083in}{2.227473in}}%
\pgfpathlineto{\pgfqpoint{1.194792in}{2.238575in}}%
\pgfpathlineto{\pgfqpoint{1.201147in}{2.240790in}}%
\pgfpathlineto{\pgfqpoint{1.207501in}{2.240392in}}%
\pgfpathlineto{\pgfqpoint{1.213856in}{2.234871in}}%
\pgfpathlineto{\pgfqpoint{1.220210in}{2.237578in}}%
\pgfpathlineto{\pgfqpoint{1.226565in}{2.246946in}}%
\pgfpathlineto{\pgfqpoint{1.239274in}{2.271168in}}%
\pgfpathlineto{\pgfqpoint{1.245629in}{2.286259in}}%
\pgfpathlineto{\pgfqpoint{1.258338in}{2.321203in}}%
\pgfpathlineto{\pgfqpoint{1.271047in}{2.351136in}}%
\pgfpathlineto{\pgfqpoint{1.277402in}{2.361145in}}%
\pgfpathlineto{\pgfqpoint{1.283757in}{2.367306in}}%
\pgfpathlineto{\pgfqpoint{1.290111in}{2.371238in}}%
\pgfpathlineto{\pgfqpoint{1.302820in}{2.365776in}}%
\pgfpathlineto{\pgfqpoint{1.309175in}{2.363140in}}%
\pgfpathlineto{\pgfqpoint{1.315530in}{2.362475in}}%
\pgfpathlineto{\pgfqpoint{1.347303in}{2.377911in}}%
\pgfpathlineto{\pgfqpoint{1.353657in}{2.382339in}}%
\pgfpathlineto{\pgfqpoint{1.360012in}{2.383028in}}%
\pgfpathlineto{\pgfqpoint{1.366366in}{2.388217in}}%
\pgfpathlineto{\pgfqpoint{1.372721in}{2.390069in}}%
\pgfpathlineto{\pgfqpoint{1.379076in}{2.386424in}}%
\pgfpathlineto{\pgfqpoint{1.391785in}{2.384667in}}%
\pgfpathlineto{\pgfqpoint{1.398139in}{2.384690in}}%
\pgfpathlineto{\pgfqpoint{1.404494in}{2.386091in}}%
\pgfpathlineto{\pgfqpoint{1.423558in}{2.402714in}}%
\pgfpathlineto{\pgfqpoint{1.429912in}{2.404317in}}%
\pgfpathlineto{\pgfqpoint{1.436267in}{2.401503in}}%
\pgfpathlineto{\pgfqpoint{1.448976in}{2.390651in}}%
\pgfpathlineto{\pgfqpoint{1.455331in}{2.387801in}}%
\pgfpathlineto{\pgfqpoint{1.461685in}{2.390817in}}%
\pgfpathlineto{\pgfqpoint{1.480749in}{2.413994in}}%
\pgfpathlineto{\pgfqpoint{1.487104in}{2.424787in}}%
\pgfpathlineto{\pgfqpoint{1.506168in}{2.462699in}}%
\pgfpathlineto{\pgfqpoint{1.512522in}{2.468956in}}%
\pgfpathlineto{\pgfqpoint{1.518877in}{2.472507in}}%
\pgfpathlineto{\pgfqpoint{1.525231in}{2.471220in}}%
\pgfpathlineto{\pgfqpoint{1.531586in}{2.473267in}}%
\pgfpathlineto{\pgfqpoint{1.537941in}{2.473183in}}%
\pgfpathlineto{\pgfqpoint{1.544295in}{2.471046in}}%
\pgfpathlineto{\pgfqpoint{1.550650in}{2.466475in}}%
\pgfpathlineto{\pgfqpoint{1.557004in}{2.472067in}}%
\pgfpathlineto{\pgfqpoint{1.563359in}{2.481946in}}%
\pgfpathlineto{\pgfqpoint{1.569714in}{2.487515in}}%
\pgfpathlineto{\pgfqpoint{1.576068in}{2.477173in}}%
\pgfpathlineto{\pgfqpoint{1.582423in}{2.463459in}}%
\pgfpathlineto{\pgfqpoint{1.588777in}{2.457332in}}%
\pgfpathlineto{\pgfqpoint{1.595132in}{2.454221in}}%
\pgfpathlineto{\pgfqpoint{1.601487in}{2.437492in}}%
\pgfpathlineto{\pgfqpoint{1.607841in}{2.411308in}}%
\pgfpathlineto{\pgfqpoint{1.614196in}{2.396457in}}%
\pgfpathlineto{\pgfqpoint{1.620550in}{2.403510in}}%
\pgfpathlineto{\pgfqpoint{1.626905in}{2.416856in}}%
\pgfpathlineto{\pgfqpoint{1.645969in}{2.469799in}}%
\pgfpathlineto{\pgfqpoint{1.652323in}{2.481756in}}%
\pgfpathlineto{\pgfqpoint{1.671387in}{2.504945in}}%
\pgfpathlineto{\pgfqpoint{1.677742in}{2.502760in}}%
\pgfpathlineto{\pgfqpoint{1.690451in}{2.475079in}}%
\pgfpathlineto{\pgfqpoint{1.696806in}{2.465157in}}%
\pgfpathlineto{\pgfqpoint{1.709515in}{2.458377in}}%
\pgfpathlineto{\pgfqpoint{1.715869in}{2.458425in}}%
\pgfpathlineto{\pgfqpoint{1.722224in}{2.466938in}}%
\pgfpathlineto{\pgfqpoint{1.734933in}{2.489830in}}%
\pgfpathlineto{\pgfqpoint{1.741288in}{2.496432in}}%
\pgfpathlineto{\pgfqpoint{1.747642in}{2.504779in}}%
\pgfpathlineto{\pgfqpoint{1.753997in}{2.515489in}}%
\pgfpathlineto{\pgfqpoint{1.760352in}{2.534391in}}%
\pgfpathlineto{\pgfqpoint{1.773061in}{2.583271in}}%
\pgfpathlineto{\pgfqpoint{1.785770in}{2.628952in}}%
\pgfpathlineto{\pgfqpoint{1.792125in}{2.642143in}}%
\pgfpathlineto{\pgfqpoint{1.798479in}{2.634651in}}%
\pgfpathlineto{\pgfqpoint{1.804834in}{2.602509in}}%
\pgfpathlineto{\pgfqpoint{1.817543in}{2.500077in}}%
\pgfpathlineto{\pgfqpoint{1.823898in}{2.488037in}}%
\pgfpathlineto{\pgfqpoint{1.830252in}{2.496812in}}%
\pgfpathlineto{\pgfqpoint{1.836607in}{2.482385in}}%
\pgfpathlineto{\pgfqpoint{1.849316in}{2.431721in}}%
\pgfpathlineto{\pgfqpoint{1.855671in}{2.422258in}}%
\pgfpathlineto{\pgfqpoint{1.862025in}{2.419478in}}%
\pgfpathlineto{\pgfqpoint{1.868380in}{2.394403in}}%
\pgfpathlineto{\pgfqpoint{1.874734in}{2.327353in}}%
\pgfpathlineto{\pgfqpoint{1.881089in}{2.236485in}}%
\pgfpathlineto{\pgfqpoint{1.887444in}{2.193765in}}%
\pgfpathlineto{\pgfqpoint{1.893798in}{2.234669in}}%
\pgfpathlineto{\pgfqpoint{1.900153in}{2.309009in}}%
\pgfpathlineto{\pgfqpoint{1.906507in}{2.370264in}}%
\pgfpathlineto{\pgfqpoint{1.912862in}{2.414671in}}%
\pgfpathlineto{\pgfqpoint{1.919217in}{2.446041in}}%
\pgfpathlineto{\pgfqpoint{1.925571in}{2.468042in}}%
\pgfpathlineto{\pgfqpoint{1.938280in}{2.477612in}}%
\pgfpathlineto{\pgfqpoint{1.950990in}{2.479346in}}%
\pgfpathlineto{\pgfqpoint{1.963699in}{2.489783in}}%
\pgfpathlineto{\pgfqpoint{1.970053in}{2.492644in}}%
\pgfpathlineto{\pgfqpoint{1.976408in}{2.494259in}}%
\pgfpathlineto{\pgfqpoint{1.982763in}{2.493725in}}%
\pgfpathlineto{\pgfqpoint{2.001826in}{2.503057in}}%
\pgfpathlineto{\pgfqpoint{2.008181in}{2.506322in}}%
\pgfpathlineto{\pgfqpoint{2.014536in}{2.505657in}}%
\pgfpathlineto{\pgfqpoint{2.020890in}{2.495458in}}%
\pgfpathlineto{\pgfqpoint{2.027245in}{2.487550in}}%
\pgfpathlineto{\pgfqpoint{2.033599in}{2.492168in}}%
\pgfpathlineto{\pgfqpoint{2.039954in}{2.499839in}}%
\pgfpathlineto{\pgfqpoint{2.046309in}{2.501229in}}%
\pgfpathlineto{\pgfqpoint{2.052663in}{2.497037in}}%
\pgfpathlineto{\pgfqpoint{2.065372in}{2.486161in}}%
\pgfpathlineto{\pgfqpoint{2.071727in}{2.487764in}}%
\pgfpathlineto{\pgfqpoint{2.078082in}{2.494116in}}%
\pgfpathlineto{\pgfqpoint{2.084436in}{2.502511in}}%
\pgfpathlineto{\pgfqpoint{2.097145in}{2.510906in}}%
\pgfpathlineto{\pgfqpoint{2.103500in}{2.509801in}}%
\pgfpathlineto{\pgfqpoint{2.116209in}{2.511419in}}%
\pgfpathlineto{\pgfqpoint{2.122564in}{2.517768in}}%
\pgfpathlineto{\pgfqpoint{2.128918in}{2.519383in}}%
\pgfpathlineto{\pgfqpoint{2.135273in}{2.519466in}}%
\pgfpathlineto{\pgfqpoint{2.147982in}{2.514954in}}%
\pgfpathlineto{\pgfqpoint{2.154337in}{2.506952in}}%
\pgfpathlineto{\pgfqpoint{2.167046in}{2.483228in}}%
\pgfpathlineto{\pgfqpoint{2.192464in}{2.465181in}}%
\pgfpathlineto{\pgfqpoint{2.198819in}{2.462984in}}%
\pgfpathlineto{\pgfqpoint{2.205174in}{2.473943in}}%
\pgfpathlineto{\pgfqpoint{2.211528in}{2.479833in}}%
\pgfpathlineto{\pgfqpoint{2.217883in}{2.480592in}}%
\pgfpathlineto{\pgfqpoint{2.224237in}{2.483691in}}%
\pgfpathlineto{\pgfqpoint{2.230592in}{2.488678in}}%
\pgfpathlineto{\pgfqpoint{2.249656in}{2.517187in}}%
\pgfpathlineto{\pgfqpoint{2.256010in}{2.519573in}}%
\pgfpathlineto{\pgfqpoint{2.262365in}{2.517840in}}%
\pgfpathlineto{\pgfqpoint{2.268720in}{2.512152in}}%
\pgfpathlineto{\pgfqpoint{2.281429in}{2.496148in}}%
\pgfpathlineto{\pgfqpoint{2.287783in}{2.486292in}}%
\pgfpathlineto{\pgfqpoint{2.294138in}{2.480082in}}%
\pgfpathlineto{\pgfqpoint{2.300493in}{2.477375in}}%
\pgfpathlineto{\pgfqpoint{2.306847in}{2.478883in}}%
\pgfpathlineto{\pgfqpoint{2.319556in}{2.473231in}}%
\pgfpathlineto{\pgfqpoint{2.325911in}{2.465846in}}%
\pgfpathlineto{\pgfqpoint{2.332266in}{2.461464in}}%
\pgfpathlineto{\pgfqpoint{2.344975in}{2.451051in}}%
\pgfpathlineto{\pgfqpoint{2.351329in}{2.449163in}}%
\pgfpathlineto{\pgfqpoint{2.357684in}{2.450339in}}%
\pgfpathlineto{\pgfqpoint{2.376748in}{2.456596in}}%
\pgfpathlineto{\pgfqpoint{2.383102in}{2.461536in}}%
\pgfpathlineto{\pgfqpoint{2.389457in}{2.469978in}}%
\pgfpathlineto{\pgfqpoint{2.395812in}{2.472578in}}%
\pgfpathlineto{\pgfqpoint{2.402166in}{2.481174in}}%
\pgfpathlineto{\pgfqpoint{2.408521in}{2.483525in}}%
\pgfpathlineto{\pgfqpoint{2.414875in}{2.483205in}}%
\pgfpathlineto{\pgfqpoint{2.421230in}{2.480604in}}%
\pgfpathlineto{\pgfqpoint{2.427585in}{2.479405in}}%
\pgfpathlineto{\pgfqpoint{2.433939in}{2.483335in}}%
\pgfpathlineto{\pgfqpoint{2.440294in}{2.490020in}}%
\pgfpathlineto{\pgfqpoint{2.446648in}{2.492532in}}%
\pgfpathlineto{\pgfqpoint{2.453003in}{2.496349in}}%
\pgfpathlineto{\pgfqpoint{2.459358in}{2.496764in}}%
\pgfpathlineto{\pgfqpoint{2.465712in}{2.494318in}}%
\pgfpathlineto{\pgfqpoint{2.472067in}{2.490471in}}%
\pgfpathlineto{\pgfqpoint{2.478421in}{2.490744in}}%
\pgfpathlineto{\pgfqpoint{2.484776in}{2.495411in}}%
\pgfpathlineto{\pgfqpoint{2.491131in}{2.498569in}}%
\pgfpathlineto{\pgfqpoint{2.516549in}{2.518220in}}%
\pgfpathlineto{\pgfqpoint{2.529258in}{2.521992in}}%
\pgfpathlineto{\pgfqpoint{2.535613in}{2.518481in}}%
\pgfpathlineto{\pgfqpoint{2.541967in}{2.519977in}}%
\pgfpathlineto{\pgfqpoint{2.567386in}{2.537229in}}%
\pgfpathlineto{\pgfqpoint{2.592804in}{2.529737in}}%
\pgfpathlineto{\pgfqpoint{2.611868in}{2.536038in}}%
\pgfpathlineto{\pgfqpoint{2.624577in}{2.548307in}}%
\pgfpathlineto{\pgfqpoint{2.630932in}{2.552047in}}%
\pgfpathlineto{\pgfqpoint{2.637286in}{2.547274in}}%
\pgfpathlineto{\pgfqpoint{2.643641in}{2.546621in}}%
\pgfpathlineto{\pgfqpoint{2.656350in}{2.555277in}}%
\pgfpathlineto{\pgfqpoint{2.662705in}{2.560549in}}%
\pgfpathlineto{\pgfqpoint{2.669059in}{2.560703in}}%
\pgfpathlineto{\pgfqpoint{2.675414in}{2.563754in}}%
\pgfpathlineto{\pgfqpoint{2.681769in}{2.560905in}}%
\pgfpathlineto{\pgfqpoint{2.688123in}{2.567162in}}%
\pgfpathlineto{\pgfqpoint{2.694478in}{2.566996in}}%
\pgfpathlineto{\pgfqpoint{2.700832in}{2.569930in}}%
\pgfpathlineto{\pgfqpoint{2.713542in}{2.568183in}}%
\pgfpathlineto{\pgfqpoint{2.719896in}{2.570190in}}%
\pgfpathlineto{\pgfqpoint{2.726251in}{2.576863in}}%
\pgfpathlineto{\pgfqpoint{2.738960in}{2.580817in}}%
\pgfpathlineto{\pgfqpoint{2.745315in}{2.580995in}}%
\pgfpathlineto{\pgfqpoint{2.751669in}{2.579819in}}%
\pgfpathlineto{\pgfqpoint{2.758024in}{2.583358in}}%
\pgfpathlineto{\pgfqpoint{2.764378in}{2.584557in}}%
\pgfpathlineto{\pgfqpoint{2.770733in}{2.584272in}}%
\pgfpathlineto{\pgfqpoint{2.783442in}{2.592863in}}%
\pgfpathlineto{\pgfqpoint{2.789797in}{2.600372in}}%
\pgfpathlineto{\pgfqpoint{2.796151in}{2.612056in}}%
\pgfpathlineto{\pgfqpoint{2.802506in}{2.619489in}}%
\pgfpathlineto{\pgfqpoint{2.808861in}{2.628845in}}%
\pgfpathlineto{\pgfqpoint{2.821570in}{2.656759in}}%
\pgfpathlineto{\pgfqpoint{2.827924in}{2.667517in}}%
\pgfpathlineto{\pgfqpoint{2.853343in}{2.723821in}}%
\pgfpathlineto{\pgfqpoint{2.866052in}{2.740052in}}%
\pgfpathlineto{\pgfqpoint{2.878761in}{2.755143in}}%
\pgfpathlineto{\pgfqpoint{2.891470in}{2.767420in}}%
\pgfpathlineto{\pgfqpoint{2.897825in}{2.761591in}}%
\pgfpathlineto{\pgfqpoint{2.904180in}{2.758682in}}%
\pgfpathlineto{\pgfqpoint{2.910534in}{2.764381in}}%
\pgfpathlineto{\pgfqpoint{2.929598in}{2.786964in}}%
\pgfpathlineto{\pgfqpoint{2.935953in}{2.787926in}}%
\pgfpathlineto{\pgfqpoint{2.942307in}{2.783029in}}%
\pgfpathlineto{\pgfqpoint{2.955016in}{2.780327in}}%
\pgfpathlineto{\pgfqpoint{2.961371in}{2.786513in}}%
\pgfpathlineto{\pgfqpoint{2.974080in}{2.786430in}}%
\pgfpathlineto{\pgfqpoint{2.999499in}{2.776551in}}%
\pgfpathlineto{\pgfqpoint{3.005853in}{2.772633in}}%
\pgfpathlineto{\pgfqpoint{3.012208in}{2.767029in}}%
\pgfpathlineto{\pgfqpoint{3.018562in}{2.764369in}}%
\pgfpathlineto{\pgfqpoint{3.037626in}{2.773238in}}%
\pgfpathlineto{\pgfqpoint{3.050335in}{2.785243in}}%
\pgfpathlineto{\pgfqpoint{3.063045in}{2.782381in}}%
\pgfpathlineto{\pgfqpoint{3.082108in}{2.801687in}}%
\pgfpathlineto{\pgfqpoint{3.088463in}{2.801925in}}%
\pgfpathlineto{\pgfqpoint{3.094818in}{2.804216in}}%
\pgfpathlineto{\pgfqpoint{3.101172in}{2.803624in}}%
\pgfpathlineto{\pgfqpoint{3.113881in}{2.811032in}}%
\pgfpathlineto{\pgfqpoint{3.126591in}{2.826503in}}%
\pgfpathlineto{\pgfqpoint{3.132945in}{2.831858in}}%
\pgfpathlineto{\pgfqpoint{3.139300in}{2.840822in}}%
\pgfpathlineto{\pgfqpoint{3.152009in}{2.852601in}}%
\pgfpathlineto{\pgfqpoint{3.164718in}{2.856032in}}%
\pgfpathlineto{\pgfqpoint{3.171073in}{2.852304in}}%
\pgfpathlineto{\pgfqpoint{3.177427in}{2.854275in}}%
\pgfpathlineto{\pgfqpoint{3.202846in}{2.868701in}}%
\pgfpathlineto{\pgfqpoint{3.209200in}{2.872240in}}%
\pgfpathlineto{\pgfqpoint{3.215555in}{2.874306in}}%
\pgfpathlineto{\pgfqpoint{3.234619in}{2.884303in}}%
\pgfpathlineto{\pgfqpoint{3.253683in}{2.886108in}}%
\pgfpathlineto{\pgfqpoint{3.266392in}{2.892279in}}%
\pgfpathlineto{\pgfqpoint{3.285456in}{2.890465in}}%
\pgfpathlineto{\pgfqpoint{3.291810in}{2.892935in}}%
\pgfpathlineto{\pgfqpoint{3.298165in}{2.893268in}}%
\pgfpathlineto{\pgfqpoint{3.304519in}{2.891938in}}%
\pgfpathlineto{\pgfqpoint{3.310874in}{2.893552in}}%
\pgfpathlineto{\pgfqpoint{3.317229in}{2.893101in}}%
\pgfpathlineto{\pgfqpoint{3.329938in}{2.900748in}}%
\pgfpathlineto{\pgfqpoint{3.336292in}{2.902564in}}%
\pgfpathlineto{\pgfqpoint{3.342647in}{2.902006in}}%
\pgfpathlineto{\pgfqpoint{3.349002in}{2.905786in}}%
\pgfpathlineto{\pgfqpoint{3.361711in}{2.909712in}}%
\pgfpathlineto{\pgfqpoint{3.368065in}{2.907516in}}%
\pgfpathlineto{\pgfqpoint{3.380775in}{2.901068in}}%
\pgfpathlineto{\pgfqpoint{3.387129in}{2.902849in}}%
\pgfpathlineto{\pgfqpoint{3.406193in}{2.898955in}}%
\pgfpathlineto{\pgfqpoint{3.412548in}{2.901567in}}%
\pgfpathlineto{\pgfqpoint{3.418902in}{2.896117in}}%
\pgfpathlineto{\pgfqpoint{3.425257in}{2.897910in}}%
\pgfpathlineto{\pgfqpoint{3.431611in}{2.889931in}}%
\pgfpathlineto{\pgfqpoint{3.437966in}{2.887186in}}%
\pgfpathlineto{\pgfqpoint{3.444321in}{2.882926in}}%
\pgfpathlineto{\pgfqpoint{3.450675in}{2.876989in}}%
\pgfpathlineto{\pgfqpoint{3.457030in}{2.880824in}}%
\pgfpathlineto{\pgfqpoint{3.463384in}{2.887283in}}%
\pgfpathlineto{\pgfqpoint{3.476094in}{2.905509in}}%
\pgfpathlineto{\pgfqpoint{3.482448in}{2.906364in}}%
\pgfpathlineto{\pgfqpoint{3.488803in}{2.909712in}}%
\pgfpathlineto{\pgfqpoint{3.495157in}{2.915542in}}%
\pgfpathlineto{\pgfqpoint{3.501512in}{2.918843in}}%
\pgfpathlineto{\pgfqpoint{3.520576in}{2.911927in}}%
\pgfpathlineto{\pgfqpoint{3.526930in}{2.907575in}}%
\pgfpathlineto{\pgfqpoint{3.539640in}{2.903372in}}%
\pgfpathlineto{\pgfqpoint{3.552349in}{2.897138in}}%
\pgfpathlineto{\pgfqpoint{3.558703in}{2.899727in}}%
\pgfpathlineto{\pgfqpoint{3.571413in}{2.916053in}}%
\pgfpathlineto{\pgfqpoint{3.577767in}{2.919484in}}%
\pgfpathlineto{\pgfqpoint{3.584122in}{2.915376in}}%
\pgfpathlineto{\pgfqpoint{3.590476in}{2.907634in}}%
\pgfpathlineto{\pgfqpoint{3.596831in}{2.903277in}}%
\pgfpathlineto{\pgfqpoint{3.603186in}{2.904077in}}%
\pgfpathlineto{\pgfqpoint{3.609540in}{2.909867in}}%
\pgfpathlineto{\pgfqpoint{3.622249in}{2.917656in}}%
\pgfpathlineto{\pgfqpoint{3.628604in}{2.918772in}}%
\pgfpathlineto{\pgfqpoint{3.634959in}{2.923509in}}%
\pgfpathlineto{\pgfqpoint{3.641313in}{2.922643in}}%
\pgfpathlineto{\pgfqpoint{3.660377in}{2.936071in}}%
\pgfpathlineto{\pgfqpoint{3.666732in}{2.937627in}}%
\pgfpathlineto{\pgfqpoint{3.673086in}{2.940346in}}%
\pgfpathlineto{\pgfqpoint{3.679441in}{2.948717in}}%
\pgfpathlineto{\pgfqpoint{3.685795in}{2.952070in}}%
\pgfpathlineto{\pgfqpoint{3.698505in}{2.961018in}}%
\pgfpathlineto{\pgfqpoint{3.704859in}{2.965494in}}%
\pgfpathlineto{\pgfqpoint{3.711214in}{2.965197in}}%
\pgfpathlineto{\pgfqpoint{3.717568in}{2.968082in}}%
\pgfpathlineto{\pgfqpoint{3.723923in}{2.969709in}}%
\pgfpathlineto{\pgfqpoint{3.730278in}{2.973794in}}%
\pgfpathlineto{\pgfqpoint{3.742987in}{2.974423in}}%
\pgfpathlineto{\pgfqpoint{3.749341in}{2.974328in}}%
\pgfpathlineto{\pgfqpoint{3.755696in}{2.972677in}}%
\pgfpathlineto{\pgfqpoint{3.762051in}{2.973307in}}%
\pgfpathlineto{\pgfqpoint{3.768405in}{2.975477in}}%
\pgfpathlineto{\pgfqpoint{3.774760in}{2.982675in}}%
\pgfpathlineto{\pgfqpoint{3.781114in}{2.995154in}}%
\pgfpathlineto{\pgfqpoint{3.793824in}{3.007716in}}%
\pgfpathlineto{\pgfqpoint{3.800178in}{3.008939in}}%
\pgfpathlineto{\pgfqpoint{3.806533in}{3.012228in}}%
\pgfpathlineto{\pgfqpoint{3.812887in}{3.013665in}}%
\pgfpathlineto{\pgfqpoint{3.838306in}{3.025562in}}%
\pgfpathlineto{\pgfqpoint{3.863724in}{3.028732in}}%
\pgfpathlineto{\pgfqpoint{3.876433in}{3.021964in}}%
\pgfpathlineto{\pgfqpoint{3.882788in}{3.024434in}}%
\pgfpathlineto{\pgfqpoint{3.889143in}{3.023294in}}%
\pgfpathlineto{\pgfqpoint{3.895497in}{3.027070in}}%
\pgfpathlineto{\pgfqpoint{3.901852in}{3.028020in}}%
\pgfpathlineto{\pgfqpoint{3.908206in}{3.031712in}}%
\pgfpathlineto{\pgfqpoint{3.914561in}{3.031712in}}%
\pgfpathlineto{\pgfqpoint{3.920916in}{3.028613in}}%
\pgfpathlineto{\pgfqpoint{3.927270in}{3.028091in}}%
\pgfpathlineto{\pgfqpoint{3.933625in}{3.024242in}}%
\pgfpathlineto{\pgfqpoint{3.946334in}{3.010079in}}%
\pgfpathlineto{\pgfqpoint{3.952689in}{3.010494in}}%
\pgfpathlineto{\pgfqpoint{3.978107in}{3.022784in}}%
\pgfpathlineto{\pgfqpoint{3.984462in}{3.027106in}}%
\pgfpathlineto{\pgfqpoint{4.009880in}{3.034135in}}%
\pgfpathlineto{\pgfqpoint{4.028944in}{3.022534in}}%
\pgfpathlineto{\pgfqpoint{4.041653in}{3.029884in}}%
\pgfpathlineto{\pgfqpoint{4.060717in}{3.059390in}}%
\pgfpathlineto{\pgfqpoint{4.067071in}{3.059698in}}%
\pgfpathlineto{\pgfqpoint{4.073426in}{3.062999in}}%
\pgfpathlineto{\pgfqpoint{4.079781in}{3.069245in}}%
\pgfpathlineto{\pgfqpoint{4.086135in}{3.073911in}}%
\pgfpathlineto{\pgfqpoint{4.092490in}{3.073044in}}%
\pgfpathlineto{\pgfqpoint{4.105199in}{3.067429in}}%
\pgfpathlineto{\pgfqpoint{4.111554in}{3.069874in}}%
\pgfpathlineto{\pgfqpoint{4.124263in}{3.070171in}}%
\pgfpathlineto{\pgfqpoint{4.130617in}{3.066454in}}%
\pgfpathlineto{\pgfqpoint{4.136972in}{3.059532in}}%
\pgfpathlineto{\pgfqpoint{4.143327in}{3.047397in}}%
\pgfpathlineto{\pgfqpoint{4.149681in}{3.038813in}}%
\pgfpathlineto{\pgfqpoint{4.156036in}{3.036889in}}%
\pgfpathlineto{\pgfqpoint{4.162390in}{3.029658in}}%
\pgfpathlineto{\pgfqpoint{4.168745in}{3.028233in}}%
\pgfpathlineto{\pgfqpoint{4.181454in}{3.032650in}}%
\pgfpathlineto{\pgfqpoint{4.194163in}{3.041769in}}%
\pgfpathlineto{\pgfqpoint{4.206873in}{3.053583in}}%
\pgfpathlineto{\pgfqpoint{4.213227in}{3.055839in}}%
\pgfpathlineto{\pgfqpoint{4.219582in}{3.064673in}}%
\pgfpathlineto{\pgfqpoint{4.225936in}{3.062120in}}%
\pgfpathlineto{\pgfqpoint{4.238646in}{3.063605in}}%
\pgfpathlineto{\pgfqpoint{4.251355in}{3.069850in}}%
\pgfpathlineto{\pgfqpoint{4.264064in}{3.074398in}}%
\pgfpathlineto{\pgfqpoint{4.270419in}{3.079623in}}%
\pgfpathlineto{\pgfqpoint{4.276773in}{3.082555in}}%
\pgfpathlineto{\pgfqpoint{4.289482in}{3.078625in}}%
\pgfpathlineto{\pgfqpoint{4.295837in}{3.080299in}}%
\pgfpathlineto{\pgfqpoint{4.302192in}{3.085191in}}%
\pgfpathlineto{\pgfqpoint{4.308546in}{3.092077in}}%
\pgfpathlineto{\pgfqpoint{4.314901in}{3.090486in}}%
\pgfpathlineto{\pgfqpoint{4.321255in}{3.086010in}}%
\pgfpathlineto{\pgfqpoint{4.340319in}{3.093906in}}%
\pgfpathlineto{\pgfqpoint{4.353028in}{3.093901in}}%
\pgfpathlineto{\pgfqpoint{4.359383in}{3.100092in}}%
\pgfpathlineto{\pgfqpoint{4.365738in}{3.099795in}}%
\pgfpathlineto{\pgfqpoint{4.378447in}{3.105435in}}%
\pgfpathlineto{\pgfqpoint{4.384801in}{3.107347in}}%
\pgfpathlineto{\pgfqpoint{4.391156in}{3.111431in}}%
\pgfpathlineto{\pgfqpoint{4.397511in}{3.112072in}}%
\pgfpathlineto{\pgfqpoint{4.403865in}{3.118543in}}%
\pgfpathlineto{\pgfqpoint{4.410220in}{3.126653in}}%
\pgfpathlineto{\pgfqpoint{4.422929in}{3.138942in}}%
\pgfpathlineto{\pgfqpoint{4.429284in}{3.140664in}}%
\pgfpathlineto{\pgfqpoint{4.435638in}{3.139270in}}%
\pgfpathlineto{\pgfqpoint{4.461057in}{3.146980in}}%
\pgfpathlineto{\pgfqpoint{4.467411in}{3.145104in}}%
\pgfpathlineto{\pgfqpoint{4.473766in}{3.144902in}}%
\pgfpathlineto{\pgfqpoint{4.480120in}{3.140925in}}%
\pgfpathlineto{\pgfqpoint{4.486475in}{3.135356in}}%
\pgfpathlineto{\pgfqpoint{4.492830in}{3.132922in}}%
\pgfpathlineto{\pgfqpoint{4.511893in}{3.132071in}}%
\pgfpathlineto{\pgfqpoint{4.518248in}{3.128814in}}%
\pgfpathlineto{\pgfqpoint{4.524603in}{3.131687in}}%
\pgfpathlineto{\pgfqpoint{4.537312in}{3.132067in}}%
\pgfpathlineto{\pgfqpoint{4.550021in}{3.139073in}}%
\pgfpathlineto{\pgfqpoint{4.556376in}{3.144166in}}%
\pgfpathlineto{\pgfqpoint{4.562730in}{3.151635in}}%
\pgfpathlineto{\pgfqpoint{4.569085in}{3.154104in}}%
\pgfpathlineto{\pgfqpoint{4.575439in}{3.154924in}}%
\pgfpathlineto{\pgfqpoint{4.581794in}{3.152893in}}%
\pgfpathlineto{\pgfqpoint{4.594503in}{3.140533in}}%
\pgfpathlineto{\pgfqpoint{4.600858in}{3.142291in}}%
\pgfpathlineto{\pgfqpoint{4.613567in}{3.154342in}}%
\pgfpathlineto{\pgfqpoint{4.619922in}{3.156467in}}%
\pgfpathlineto{\pgfqpoint{4.626276in}{3.156230in}}%
\pgfpathlineto{\pgfqpoint{4.632631in}{3.157904in}}%
\pgfpathlineto{\pgfqpoint{4.651695in}{3.149070in}}%
\pgfpathlineto{\pgfqpoint{4.664404in}{3.140806in}}%
\pgfpathlineto{\pgfqpoint{4.664404in}{3.140806in}}%
\pgfusepath{stroke}%
\end{pgfscope}%
\begin{pgfscope}%
\pgfsetrectcap%
\pgfsetmiterjoin%
\pgfsetlinewidth{0.803000pt}%
\definecolor{currentstroke}{rgb}{0.000000,0.000000,0.000000}%
\pgfsetstrokecolor{currentstroke}%
\pgfsetdash{}{0pt}%
\pgfpathmoveto{\pgfqpoint{0.661006in}{2.135160in}}%
\pgfpathlineto{\pgfqpoint{0.661006in}{3.206606in}}%
\pgfusepath{stroke}%
\end{pgfscope}%
\begin{pgfscope}%
\pgfsetrectcap%
\pgfsetmiterjoin%
\pgfsetlinewidth{0.803000pt}%
\definecolor{currentstroke}{rgb}{0.000000,0.000000,0.000000}%
\pgfsetstrokecolor{currentstroke}%
\pgfsetdash{}{0pt}%
\pgfpathmoveto{\pgfqpoint{4.855042in}{2.135160in}}%
\pgfpathlineto{\pgfqpoint{4.855042in}{3.206606in}}%
\pgfusepath{stroke}%
\end{pgfscope}%
\begin{pgfscope}%
\pgfsetrectcap%
\pgfsetmiterjoin%
\pgfsetlinewidth{0.803000pt}%
\definecolor{currentstroke}{rgb}{0.000000,0.000000,0.000000}%
\pgfsetstrokecolor{currentstroke}%
\pgfsetdash{}{0pt}%
\pgfpathmoveto{\pgfqpoint{0.661006in}{2.135160in}}%
\pgfpathlineto{\pgfqpoint{4.855042in}{2.135160in}}%
\pgfusepath{stroke}%
\end{pgfscope}%
\begin{pgfscope}%
\pgfsetrectcap%
\pgfsetmiterjoin%
\pgfsetlinewidth{0.803000pt}%
\definecolor{currentstroke}{rgb}{0.000000,0.000000,0.000000}%
\pgfsetstrokecolor{currentstroke}%
\pgfsetdash{}{0pt}%
\pgfpathmoveto{\pgfqpoint{0.661006in}{3.206606in}}%
\pgfpathlineto{\pgfqpoint{4.855042in}{3.206606in}}%
\pgfusepath{stroke}%
\end{pgfscope}%
\begin{pgfscope}%
\pgfsetbuttcap%
\pgfsetroundjoin%
\definecolor{currentfill}{rgb}{0.000000,0.000000,0.000000}%
\pgfsetfillcolor{currentfill}%
\pgfsetlinewidth{0.803000pt}%
\definecolor{currentstroke}{rgb}{0.000000,0.000000,0.000000}%
\pgfsetstrokecolor{currentstroke}%
\pgfsetdash{}{0pt}%
\pgfsys@defobject{currentmarker}{\pgfqpoint{0.000000in}{0.000000in}}{\pgfqpoint{0.048611in}{0.000000in}}{%
\pgfpathmoveto{\pgfqpoint{0.000000in}{0.000000in}}%
\pgfpathlineto{\pgfqpoint{0.048611in}{0.000000in}}%
\pgfusepath{stroke,fill}%
}%
\begin{pgfscope}%
\pgfsys@transformshift{4.855042in}{3.122751in}%
\pgfsys@useobject{currentmarker}{}%
\end{pgfscope}%
\end{pgfscope}%
\begin{pgfscope}%
\definecolor{textcolor}{rgb}{0.000000,0.000000,0.000000}%
\pgfsetstrokecolor{textcolor}%
\pgfsetfillcolor{textcolor}%
\pgftext[x=4.952264in, y=3.084196in, left, base]{\color{textcolor}\rmfamily\fontsize{8.000000}{9.600000}\selectfont \(\displaystyle {1.0}\)}%
\end{pgfscope}%
\begin{pgfscope}%
\pgfsetbuttcap%
\pgfsetroundjoin%
\definecolor{currentfill}{rgb}{0.000000,0.000000,0.000000}%
\pgfsetfillcolor{currentfill}%
\pgfsetlinewidth{0.803000pt}%
\definecolor{currentstroke}{rgb}{0.000000,0.000000,0.000000}%
\pgfsetstrokecolor{currentstroke}%
\pgfsetdash{}{0pt}%
\pgfsys@defobject{currentmarker}{\pgfqpoint{0.000000in}{0.000000in}}{\pgfqpoint{0.048611in}{0.000000in}}{%
\pgfpathmoveto{\pgfqpoint{0.000000in}{0.000000in}}%
\pgfpathlineto{\pgfqpoint{0.048611in}{0.000000in}}%
\pgfusepath{stroke,fill}%
}%
\begin{pgfscope}%
\pgfsys@transformshift{4.855042in}{2.817526in}%
\pgfsys@useobject{currentmarker}{}%
\end{pgfscope}%
\end{pgfscope}%
\begin{pgfscope}%
\definecolor{textcolor}{rgb}{0.000000,0.000000,0.000000}%
\pgfsetstrokecolor{textcolor}%
\pgfsetfillcolor{textcolor}%
\pgftext[x=4.952264in, y=2.778970in, left, base]{\color{textcolor}\rmfamily\fontsize{8.000000}{9.600000}\selectfont \(\displaystyle {1.5}\)}%
\end{pgfscope}%
\begin{pgfscope}%
\pgfsetbuttcap%
\pgfsetroundjoin%
\definecolor{currentfill}{rgb}{0.000000,0.000000,0.000000}%
\pgfsetfillcolor{currentfill}%
\pgfsetlinewidth{0.803000pt}%
\definecolor{currentstroke}{rgb}{0.000000,0.000000,0.000000}%
\pgfsetstrokecolor{currentstroke}%
\pgfsetdash{}{0pt}%
\pgfsys@defobject{currentmarker}{\pgfqpoint{0.000000in}{0.000000in}}{\pgfqpoint{0.048611in}{0.000000in}}{%
\pgfpathmoveto{\pgfqpoint{0.000000in}{0.000000in}}%
\pgfpathlineto{\pgfqpoint{0.048611in}{0.000000in}}%
\pgfusepath{stroke,fill}%
}%
\begin{pgfscope}%
\pgfsys@transformshift{4.855042in}{2.512300in}%
\pgfsys@useobject{currentmarker}{}%
\end{pgfscope}%
\end{pgfscope}%
\begin{pgfscope}%
\definecolor{textcolor}{rgb}{0.000000,0.000000,0.000000}%
\pgfsetstrokecolor{textcolor}%
\pgfsetfillcolor{textcolor}%
\pgftext[x=4.952264in, y=2.473744in, left, base]{\color{textcolor}\rmfamily\fontsize{8.000000}{9.600000}\selectfont \(\displaystyle {2.0}\)}%
\end{pgfscope}%
\begin{pgfscope}%
\pgfsetbuttcap%
\pgfsetroundjoin%
\definecolor{currentfill}{rgb}{0.000000,0.000000,0.000000}%
\pgfsetfillcolor{currentfill}%
\pgfsetlinewidth{0.803000pt}%
\definecolor{currentstroke}{rgb}{0.000000,0.000000,0.000000}%
\pgfsetstrokecolor{currentstroke}%
\pgfsetdash{}{0pt}%
\pgfsys@defobject{currentmarker}{\pgfqpoint{0.000000in}{0.000000in}}{\pgfqpoint{0.048611in}{0.000000in}}{%
\pgfpathmoveto{\pgfqpoint{0.000000in}{0.000000in}}%
\pgfpathlineto{\pgfqpoint{0.048611in}{0.000000in}}%
\pgfusepath{stroke,fill}%
}%
\begin{pgfscope}%
\pgfsys@transformshift{4.855042in}{2.207074in}%
\pgfsys@useobject{currentmarker}{}%
\end{pgfscope}%
\end{pgfscope}%
\begin{pgfscope}%
\definecolor{textcolor}{rgb}{0.000000,0.000000,0.000000}%
\pgfsetstrokecolor{textcolor}%
\pgfsetfillcolor{textcolor}%
\pgftext[x=4.952264in, y=2.168519in, left, base]{\color{textcolor}\rmfamily\fontsize{8.000000}{9.600000}\selectfont \(\displaystyle {2.5}\)}%
\end{pgfscope}%
\begin{pgfscope}%
\definecolor{textcolor}{rgb}{0.000000,0.000000,0.000000}%
\pgfsetstrokecolor{textcolor}%
\pgfsetfillcolor{textcolor}%
\pgftext[x=5.158671in,y=2.670883in,,top,rotate=90.000000]{\color{textcolor}\rmfamily\fontsize{10.000000}{12.000000}\selectfont Voltage in \unit{\V}}%
\end{pgfscope}%
\begin{pgfscope}%
\pgfpathrectangle{\pgfqpoint{0.661006in}{2.135160in}}{\pgfqpoint{4.194036in}{1.071446in}}%
\pgfusepath{clip}%
\pgfsetrectcap%
\pgfsetroundjoin%
\pgfsetlinewidth{1.505625pt}%
\definecolor{currentstroke}{rgb}{0.835294,0.368627,0.000000}%
\pgfsetstrokecolor{currentstroke}%
\pgfsetstrokeopacity{0.700000}%
\pgfsetdash{}{0pt}%
\pgfpathmoveto{\pgfqpoint{0.851644in}{2.188940in}}%
\pgfpathlineto{\pgfqpoint{0.857998in}{2.186064in}}%
\pgfpathlineto{\pgfqpoint{0.877062in}{2.188105in}}%
\pgfpathlineto{\pgfqpoint{0.883417in}{2.190942in}}%
\pgfpathlineto{\pgfqpoint{0.889771in}{2.195325in}}%
\pgfpathlineto{\pgfqpoint{0.896126in}{2.194632in}}%
\pgfpathlineto{\pgfqpoint{0.915190in}{2.183862in}}%
\pgfpathlineto{\pgfqpoint{0.921544in}{2.186039in}}%
\pgfpathlineto{\pgfqpoint{0.927899in}{2.186144in}}%
\pgfpathlineto{\pgfqpoint{0.940608in}{2.189302in}}%
\pgfpathlineto{\pgfqpoint{0.959672in}{2.196818in}}%
\pgfpathlineto{\pgfqpoint{0.972381in}{2.202547in}}%
\pgfpathlineto{\pgfqpoint{0.985090in}{2.212374in}}%
\pgfpathlineto{\pgfqpoint{0.997799in}{2.217992in}}%
\pgfpathlineto{\pgfqpoint{1.004154in}{2.218605in}}%
\pgfpathlineto{\pgfqpoint{1.016863in}{2.216170in}}%
\pgfpathlineto{\pgfqpoint{1.035927in}{2.220072in}}%
\pgfpathlineto{\pgfqpoint{1.048636in}{2.218083in}}%
\pgfpathlineto{\pgfqpoint{1.067700in}{2.224692in}}%
\pgfpathlineto{\pgfqpoint{1.074055in}{2.227124in}}%
\pgfpathlineto{\pgfqpoint{1.080409in}{2.231204in}}%
\pgfpathlineto{\pgfqpoint{1.086764in}{2.236661in}}%
\pgfpathlineto{\pgfqpoint{1.093118in}{2.240549in}}%
\pgfpathlineto{\pgfqpoint{1.099473in}{2.243017in}}%
\pgfpathlineto{\pgfqpoint{1.105828in}{2.247396in}}%
\pgfpathlineto{\pgfqpoint{1.112182in}{2.253411in}}%
\pgfpathlineto{\pgfqpoint{1.124891in}{2.270660in}}%
\pgfpathlineto{\pgfqpoint{1.131246in}{2.273564in}}%
\pgfpathlineto{\pgfqpoint{1.137601in}{2.270307in}}%
\pgfpathlineto{\pgfqpoint{1.143955in}{2.263794in}}%
\pgfpathlineto{\pgfqpoint{1.156664in}{2.247835in}}%
\pgfpathlineto{\pgfqpoint{1.163019in}{2.243043in}}%
\pgfpathlineto{\pgfqpoint{1.175728in}{2.238605in}}%
\pgfpathlineto{\pgfqpoint{1.182083in}{2.238330in}}%
\pgfpathlineto{\pgfqpoint{1.194792in}{2.249236in}}%
\pgfpathlineto{\pgfqpoint{1.201147in}{2.251295in}}%
\pgfpathlineto{\pgfqpoint{1.213856in}{2.246592in}}%
\pgfpathlineto{\pgfqpoint{1.220210in}{2.249663in}}%
\pgfpathlineto{\pgfqpoint{1.232920in}{2.270629in}}%
\pgfpathlineto{\pgfqpoint{1.245629in}{2.297192in}}%
\pgfpathlineto{\pgfqpoint{1.264693in}{2.342840in}}%
\pgfpathlineto{\pgfqpoint{1.271047in}{2.355506in}}%
\pgfpathlineto{\pgfqpoint{1.277402in}{2.364832in}}%
\pgfpathlineto{\pgfqpoint{1.283757in}{2.370358in}}%
\pgfpathlineto{\pgfqpoint{1.290111in}{2.371233in}}%
\pgfpathlineto{\pgfqpoint{1.315530in}{2.366365in}}%
\pgfpathlineto{\pgfqpoint{1.334593in}{2.374171in}}%
\pgfpathlineto{\pgfqpoint{1.360012in}{2.386572in}}%
\pgfpathlineto{\pgfqpoint{1.366366in}{2.390239in}}%
\pgfpathlineto{\pgfqpoint{1.372721in}{2.391662in}}%
\pgfpathlineto{\pgfqpoint{1.385430in}{2.388046in}}%
\pgfpathlineto{\pgfqpoint{1.398139in}{2.387955in}}%
\pgfpathlineto{\pgfqpoint{1.404494in}{2.389213in}}%
\pgfpathlineto{\pgfqpoint{1.423558in}{2.406027in}}%
\pgfpathlineto{\pgfqpoint{1.429912in}{2.406748in}}%
\pgfpathlineto{\pgfqpoint{1.436267in}{2.405019in}}%
\pgfpathlineto{\pgfqpoint{1.455331in}{2.393742in}}%
\pgfpathlineto{\pgfqpoint{1.461685in}{2.398237in}}%
\pgfpathlineto{\pgfqpoint{1.480749in}{2.419149in}}%
\pgfpathlineto{\pgfqpoint{1.487104in}{2.428795in}}%
\pgfpathlineto{\pgfqpoint{1.499813in}{2.453132in}}%
\pgfpathlineto{\pgfqpoint{1.506168in}{2.462992in}}%
\pgfpathlineto{\pgfqpoint{1.512522in}{2.469723in}}%
\pgfpathlineto{\pgfqpoint{1.518877in}{2.471707in}}%
\pgfpathlineto{\pgfqpoint{1.525231in}{2.470756in}}%
\pgfpathlineto{\pgfqpoint{1.531586in}{2.472395in}}%
\pgfpathlineto{\pgfqpoint{1.537941in}{2.472271in}}%
\pgfpathlineto{\pgfqpoint{1.550650in}{2.467589in}}%
\pgfpathlineto{\pgfqpoint{1.557004in}{2.471958in}}%
\pgfpathlineto{\pgfqpoint{1.563359in}{2.481521in}}%
\pgfpathlineto{\pgfqpoint{1.569714in}{2.485628in}}%
\pgfpathlineto{\pgfqpoint{1.576068in}{2.477019in}}%
\pgfpathlineto{\pgfqpoint{1.582423in}{2.464137in}}%
\pgfpathlineto{\pgfqpoint{1.595132in}{2.456827in}}%
\pgfpathlineto{\pgfqpoint{1.601487in}{2.440342in}}%
\pgfpathlineto{\pgfqpoint{1.607841in}{2.417527in}}%
\pgfpathlineto{\pgfqpoint{1.614196in}{2.406338in}}%
\pgfpathlineto{\pgfqpoint{1.620550in}{2.413253in}}%
\pgfpathlineto{\pgfqpoint{1.633260in}{2.441881in}}%
\pgfpathlineto{\pgfqpoint{1.645969in}{2.473633in}}%
\pgfpathlineto{\pgfqpoint{1.652323in}{2.484562in}}%
\pgfpathlineto{\pgfqpoint{1.671387in}{2.504732in}}%
\pgfpathlineto{\pgfqpoint{1.677742in}{2.501884in}}%
\pgfpathlineto{\pgfqpoint{1.690451in}{2.476992in}}%
\pgfpathlineto{\pgfqpoint{1.696806in}{2.469292in}}%
\pgfpathlineto{\pgfqpoint{1.703160in}{2.465516in}}%
\pgfpathlineto{\pgfqpoint{1.709515in}{2.463223in}}%
\pgfpathlineto{\pgfqpoint{1.715869in}{2.464193in}}%
\pgfpathlineto{\pgfqpoint{1.722224in}{2.472665in}}%
\pgfpathlineto{\pgfqpoint{1.734933in}{2.494214in}}%
\pgfpathlineto{\pgfqpoint{1.747642in}{2.506080in}}%
\pgfpathlineto{\pgfqpoint{1.753997in}{2.516970in}}%
\pgfpathlineto{\pgfqpoint{1.766706in}{2.555904in}}%
\pgfpathlineto{\pgfqpoint{1.785770in}{2.618813in}}%
\pgfpathlineto{\pgfqpoint{1.792125in}{2.629057in}}%
\pgfpathlineto{\pgfqpoint{1.798479in}{2.621210in}}%
\pgfpathlineto{\pgfqpoint{1.804834in}{2.589377in}}%
\pgfpathlineto{\pgfqpoint{1.817543in}{2.499822in}}%
\pgfpathlineto{\pgfqpoint{1.823898in}{2.493466in}}%
\pgfpathlineto{\pgfqpoint{1.830252in}{2.500550in}}%
\pgfpathlineto{\pgfqpoint{1.836607in}{2.486447in}}%
\pgfpathlineto{\pgfqpoint{1.842961in}{2.460514in}}%
\pgfpathlineto{\pgfqpoint{1.849316in}{2.442156in}}%
\pgfpathlineto{\pgfqpoint{1.855671in}{2.437042in}}%
\pgfpathlineto{\pgfqpoint{1.862025in}{2.434040in}}%
\pgfpathlineto{\pgfqpoint{1.868380in}{2.411589in}}%
\pgfpathlineto{\pgfqpoint{1.874734in}{2.352398in}}%
\pgfpathlineto{\pgfqpoint{1.881089in}{2.270560in}}%
\pgfpathlineto{\pgfqpoint{1.887444in}{2.236081in}}%
\pgfpathlineto{\pgfqpoint{1.893798in}{2.279831in}}%
\pgfpathlineto{\pgfqpoint{1.900153in}{2.347533in}}%
\pgfpathlineto{\pgfqpoint{1.906507in}{2.402206in}}%
\pgfpathlineto{\pgfqpoint{1.912862in}{2.442526in}}%
\pgfpathlineto{\pgfqpoint{1.919217in}{2.470365in}}%
\pgfpathlineto{\pgfqpoint{1.925571in}{2.486916in}}%
\pgfpathlineto{\pgfqpoint{1.931926in}{2.492561in}}%
\pgfpathlineto{\pgfqpoint{1.944635in}{2.496550in}}%
\pgfpathlineto{\pgfqpoint{1.950990in}{2.496866in}}%
\pgfpathlineto{\pgfqpoint{1.970053in}{2.509563in}}%
\pgfpathlineto{\pgfqpoint{1.989117in}{2.511225in}}%
\pgfpathlineto{\pgfqpoint{1.995472in}{2.513335in}}%
\pgfpathlineto{\pgfqpoint{2.008181in}{2.518909in}}%
\pgfpathlineto{\pgfqpoint{2.014536in}{2.515740in}}%
\pgfpathlineto{\pgfqpoint{2.027245in}{2.500406in}}%
\pgfpathlineto{\pgfqpoint{2.033599in}{2.503520in}}%
\pgfpathlineto{\pgfqpoint{2.039954in}{2.511403in}}%
\pgfpathlineto{\pgfqpoint{2.046309in}{2.512318in}}%
\pgfpathlineto{\pgfqpoint{2.052663in}{2.507844in}}%
\pgfpathlineto{\pgfqpoint{2.059018in}{2.501489in}}%
\pgfpathlineto{\pgfqpoint{2.065372in}{2.497623in}}%
\pgfpathlineto{\pgfqpoint{2.071727in}{2.498848in}}%
\pgfpathlineto{\pgfqpoint{2.078082in}{2.504763in}}%
\pgfpathlineto{\pgfqpoint{2.084436in}{2.512442in}}%
\pgfpathlineto{\pgfqpoint{2.090791in}{2.517600in}}%
\pgfpathlineto{\pgfqpoint{2.097145in}{2.518572in}}%
\pgfpathlineto{\pgfqpoint{2.103500in}{2.517018in}}%
\pgfpathlineto{\pgfqpoint{2.109855in}{2.517106in}}%
\pgfpathlineto{\pgfqpoint{2.122564in}{2.524905in}}%
\pgfpathlineto{\pgfqpoint{2.128918in}{2.526139in}}%
\pgfpathlineto{\pgfqpoint{2.141628in}{2.524383in}}%
\pgfpathlineto{\pgfqpoint{2.147982in}{2.521452in}}%
\pgfpathlineto{\pgfqpoint{2.154337in}{2.514148in}}%
\pgfpathlineto{\pgfqpoint{2.167046in}{2.493254in}}%
\pgfpathlineto{\pgfqpoint{2.173401in}{2.487457in}}%
\pgfpathlineto{\pgfqpoint{2.186110in}{2.480211in}}%
\pgfpathlineto{\pgfqpoint{2.192464in}{2.475268in}}%
\pgfpathlineto{\pgfqpoint{2.198819in}{2.476139in}}%
\pgfpathlineto{\pgfqpoint{2.211528in}{2.491543in}}%
\pgfpathlineto{\pgfqpoint{2.224237in}{2.494403in}}%
\pgfpathlineto{\pgfqpoint{2.230592in}{2.498594in}}%
\pgfpathlineto{\pgfqpoint{2.249656in}{2.525121in}}%
\pgfpathlineto{\pgfqpoint{2.256010in}{2.527567in}}%
\pgfpathlineto{\pgfqpoint{2.262365in}{2.525978in}}%
\pgfpathlineto{\pgfqpoint{2.268720in}{2.521521in}}%
\pgfpathlineto{\pgfqpoint{2.281429in}{2.506116in}}%
\pgfpathlineto{\pgfqpoint{2.287783in}{2.498052in}}%
\pgfpathlineto{\pgfqpoint{2.294138in}{2.493600in}}%
\pgfpathlineto{\pgfqpoint{2.300493in}{2.492377in}}%
\pgfpathlineto{\pgfqpoint{2.306847in}{2.493087in}}%
\pgfpathlineto{\pgfqpoint{2.313202in}{2.491750in}}%
\pgfpathlineto{\pgfqpoint{2.332266in}{2.478727in}}%
\pgfpathlineto{\pgfqpoint{2.344975in}{2.470210in}}%
\pgfpathlineto{\pgfqpoint{2.351329in}{2.469108in}}%
\pgfpathlineto{\pgfqpoint{2.370393in}{2.475765in}}%
\pgfpathlineto{\pgfqpoint{2.376748in}{2.477975in}}%
\pgfpathlineto{\pgfqpoint{2.395812in}{2.492933in}}%
\pgfpathlineto{\pgfqpoint{2.402166in}{2.499256in}}%
\pgfpathlineto{\pgfqpoint{2.408521in}{2.502981in}}%
\pgfpathlineto{\pgfqpoint{2.414875in}{2.501806in}}%
\pgfpathlineto{\pgfqpoint{2.421230in}{2.497278in}}%
\pgfpathlineto{\pgfqpoint{2.427585in}{2.496706in}}%
\pgfpathlineto{\pgfqpoint{2.446648in}{2.509628in}}%
\pgfpathlineto{\pgfqpoint{2.453003in}{2.510997in}}%
\pgfpathlineto{\pgfqpoint{2.459358in}{2.511145in}}%
\pgfpathlineto{\pgfqpoint{2.472067in}{2.506406in}}%
\pgfpathlineto{\pgfqpoint{2.478421in}{2.507615in}}%
\pgfpathlineto{\pgfqpoint{2.484776in}{2.511539in}}%
\pgfpathlineto{\pgfqpoint{2.503840in}{2.527540in}}%
\pgfpathlineto{\pgfqpoint{2.529258in}{2.540005in}}%
\pgfpathlineto{\pgfqpoint{2.535613in}{2.542502in}}%
\pgfpathlineto{\pgfqpoint{2.541967in}{2.540395in}}%
\pgfpathlineto{\pgfqpoint{2.548322in}{2.542385in}}%
\pgfpathlineto{\pgfqpoint{2.567386in}{2.555816in}}%
\pgfpathlineto{\pgfqpoint{2.573740in}{2.558122in}}%
\pgfpathlineto{\pgfqpoint{2.586450in}{2.550660in}}%
\pgfpathlineto{\pgfqpoint{2.592804in}{2.551474in}}%
\pgfpathlineto{\pgfqpoint{2.599159in}{2.550373in}}%
\pgfpathlineto{\pgfqpoint{2.611868in}{2.552696in}}%
\pgfpathlineto{\pgfqpoint{2.624577in}{2.565783in}}%
\pgfpathlineto{\pgfqpoint{2.630932in}{2.567885in}}%
\pgfpathlineto{\pgfqpoint{2.643641in}{2.565068in}}%
\pgfpathlineto{\pgfqpoint{2.649996in}{2.568273in}}%
\pgfpathlineto{\pgfqpoint{2.656350in}{2.574423in}}%
\pgfpathlineto{\pgfqpoint{2.662705in}{2.577005in}}%
\pgfpathlineto{\pgfqpoint{2.675414in}{2.576514in}}%
\pgfpathlineto{\pgfqpoint{2.700832in}{2.583185in}}%
\pgfpathlineto{\pgfqpoint{2.713542in}{2.582996in}}%
\pgfpathlineto{\pgfqpoint{2.719896in}{2.585079in}}%
\pgfpathlineto{\pgfqpoint{2.732605in}{2.592036in}}%
\pgfpathlineto{\pgfqpoint{2.738960in}{2.593960in}}%
\pgfpathlineto{\pgfqpoint{2.751669in}{2.593953in}}%
\pgfpathlineto{\pgfqpoint{2.764378in}{2.598134in}}%
\pgfpathlineto{\pgfqpoint{2.770733in}{2.598358in}}%
\pgfpathlineto{\pgfqpoint{2.783442in}{2.606359in}}%
\pgfpathlineto{\pgfqpoint{2.789797in}{2.613627in}}%
\pgfpathlineto{\pgfqpoint{2.808861in}{2.640999in}}%
\pgfpathlineto{\pgfqpoint{2.815215in}{2.654168in}}%
\pgfpathlineto{\pgfqpoint{2.840634in}{2.696152in}}%
\pgfpathlineto{\pgfqpoint{2.846988in}{2.711028in}}%
\pgfpathlineto{\pgfqpoint{2.853343in}{2.722180in}}%
\pgfpathlineto{\pgfqpoint{2.872407in}{2.742384in}}%
\pgfpathlineto{\pgfqpoint{2.878761in}{2.747726in}}%
\pgfpathlineto{\pgfqpoint{2.885116in}{2.755933in}}%
\pgfpathlineto{\pgfqpoint{2.891470in}{2.758224in}}%
\pgfpathlineto{\pgfqpoint{2.904180in}{2.749967in}}%
\pgfpathlineto{\pgfqpoint{2.929598in}{2.774365in}}%
\pgfpathlineto{\pgfqpoint{2.935953in}{2.775592in}}%
\pgfpathlineto{\pgfqpoint{2.942307in}{2.771082in}}%
\pgfpathlineto{\pgfqpoint{2.955016in}{2.770786in}}%
\pgfpathlineto{\pgfqpoint{2.967726in}{2.774971in}}%
\pgfpathlineto{\pgfqpoint{2.974080in}{2.774649in}}%
\pgfpathlineto{\pgfqpoint{2.986789in}{2.769876in}}%
\pgfpathlineto{\pgfqpoint{2.993144in}{2.768710in}}%
\pgfpathlineto{\pgfqpoint{2.999499in}{2.766243in}}%
\pgfpathlineto{\pgfqpoint{3.012208in}{2.758548in}}%
\pgfpathlineto{\pgfqpoint{3.018562in}{2.757974in}}%
\pgfpathlineto{\pgfqpoint{3.024917in}{2.759378in}}%
\pgfpathlineto{\pgfqpoint{3.037626in}{2.766800in}}%
\pgfpathlineto{\pgfqpoint{3.043981in}{2.772240in}}%
\pgfpathlineto{\pgfqpoint{3.050335in}{2.775849in}}%
\pgfpathlineto{\pgfqpoint{3.063045in}{2.775301in}}%
\pgfpathlineto{\pgfqpoint{3.082108in}{2.791226in}}%
\pgfpathlineto{\pgfqpoint{3.088463in}{2.793732in}}%
\pgfpathlineto{\pgfqpoint{3.101172in}{2.795451in}}%
\pgfpathlineto{\pgfqpoint{3.107527in}{2.798306in}}%
\pgfpathlineto{\pgfqpoint{3.120236in}{2.808638in}}%
\pgfpathlineto{\pgfqpoint{3.139300in}{2.829637in}}%
\pgfpathlineto{\pgfqpoint{3.152009in}{2.838684in}}%
\pgfpathlineto{\pgfqpoint{3.158364in}{2.841349in}}%
\pgfpathlineto{\pgfqpoint{3.164718in}{2.841013in}}%
\pgfpathlineto{\pgfqpoint{3.171073in}{2.838769in}}%
\pgfpathlineto{\pgfqpoint{3.183782in}{2.844424in}}%
\pgfpathlineto{\pgfqpoint{3.202846in}{2.855652in}}%
\pgfpathlineto{\pgfqpoint{3.215555in}{2.860826in}}%
\pgfpathlineto{\pgfqpoint{3.221910in}{2.864503in}}%
\pgfpathlineto{\pgfqpoint{3.234619in}{2.868761in}}%
\pgfpathlineto{\pgfqpoint{3.240973in}{2.868179in}}%
\pgfpathlineto{\pgfqpoint{3.253683in}{2.871496in}}%
\pgfpathlineto{\pgfqpoint{3.260037in}{2.874829in}}%
\pgfpathlineto{\pgfqpoint{3.266392in}{2.876906in}}%
\pgfpathlineto{\pgfqpoint{3.272746in}{2.875997in}}%
\pgfpathlineto{\pgfqpoint{3.279101in}{2.877855in}}%
\pgfpathlineto{\pgfqpoint{3.285456in}{2.877364in}}%
\pgfpathlineto{\pgfqpoint{3.298165in}{2.878491in}}%
\pgfpathlineto{\pgfqpoint{3.304519in}{2.879131in}}%
\pgfpathlineto{\pgfqpoint{3.310874in}{2.881394in}}%
\pgfpathlineto{\pgfqpoint{3.317229in}{2.882001in}}%
\pgfpathlineto{\pgfqpoint{3.323583in}{2.885761in}}%
\pgfpathlineto{\pgfqpoint{3.361711in}{2.901267in}}%
\pgfpathlineto{\pgfqpoint{3.368065in}{2.899858in}}%
\pgfpathlineto{\pgfqpoint{3.374420in}{2.897112in}}%
\pgfpathlineto{\pgfqpoint{3.380775in}{2.895646in}}%
\pgfpathlineto{\pgfqpoint{3.418902in}{2.898180in}}%
\pgfpathlineto{\pgfqpoint{3.425257in}{2.895589in}}%
\pgfpathlineto{\pgfqpoint{3.437966in}{2.888171in}}%
\pgfpathlineto{\pgfqpoint{3.450675in}{2.883376in}}%
\pgfpathlineto{\pgfqpoint{3.457030in}{2.887811in}}%
\pgfpathlineto{\pgfqpoint{3.469739in}{2.902403in}}%
\pgfpathlineto{\pgfqpoint{3.476094in}{2.911965in}}%
\pgfpathlineto{\pgfqpoint{3.488803in}{2.919041in}}%
\pgfpathlineto{\pgfqpoint{3.495157in}{2.924359in}}%
\pgfpathlineto{\pgfqpoint{3.501512in}{2.928109in}}%
\pgfpathlineto{\pgfqpoint{3.507867in}{2.924690in}}%
\pgfpathlineto{\pgfqpoint{3.520576in}{2.921860in}}%
\pgfpathlineto{\pgfqpoint{3.533285in}{2.916712in}}%
\pgfpathlineto{\pgfqpoint{3.539640in}{2.915827in}}%
\pgfpathlineto{\pgfqpoint{3.552349in}{2.910166in}}%
\pgfpathlineto{\pgfqpoint{3.558703in}{2.913491in}}%
\pgfpathlineto{\pgfqpoint{3.571413in}{2.928088in}}%
\pgfpathlineto{\pgfqpoint{3.577767in}{2.930812in}}%
\pgfpathlineto{\pgfqpoint{3.584122in}{2.926356in}}%
\pgfpathlineto{\pgfqpoint{3.590476in}{2.920235in}}%
\pgfpathlineto{\pgfqpoint{3.596831in}{2.916894in}}%
\pgfpathlineto{\pgfqpoint{3.603186in}{2.916591in}}%
\pgfpathlineto{\pgfqpoint{3.609540in}{2.920779in}}%
\pgfpathlineto{\pgfqpoint{3.615895in}{2.926553in}}%
\pgfpathlineto{\pgfqpoint{3.622249in}{2.929454in}}%
\pgfpathlineto{\pgfqpoint{3.647668in}{2.936696in}}%
\pgfpathlineto{\pgfqpoint{3.660377in}{2.944585in}}%
\pgfpathlineto{\pgfqpoint{3.666732in}{2.946701in}}%
\pgfpathlineto{\pgfqpoint{3.698505in}{2.966731in}}%
\pgfpathlineto{\pgfqpoint{3.704859in}{2.969142in}}%
\pgfpathlineto{\pgfqpoint{3.717568in}{2.971154in}}%
\pgfpathlineto{\pgfqpoint{3.736632in}{2.977553in}}%
\pgfpathlineto{\pgfqpoint{3.742987in}{2.978862in}}%
\pgfpathlineto{\pgfqpoint{3.749341in}{2.975613in}}%
\pgfpathlineto{\pgfqpoint{3.762051in}{2.975638in}}%
\pgfpathlineto{\pgfqpoint{3.768405in}{2.979420in}}%
\pgfpathlineto{\pgfqpoint{3.774760in}{2.984979in}}%
\pgfpathlineto{\pgfqpoint{3.781114in}{2.994459in}}%
\pgfpathlineto{\pgfqpoint{3.793824in}{3.005821in}}%
\pgfpathlineto{\pgfqpoint{3.800178in}{3.009189in}}%
\pgfpathlineto{\pgfqpoint{3.806533in}{3.007981in}}%
\pgfpathlineto{\pgfqpoint{3.831951in}{3.018159in}}%
\pgfpathlineto{\pgfqpoint{3.844660in}{3.021317in}}%
\pgfpathlineto{\pgfqpoint{3.857370in}{3.023569in}}%
\pgfpathlineto{\pgfqpoint{3.876433in}{3.017686in}}%
\pgfpathlineto{\pgfqpoint{3.882788in}{3.020180in}}%
\pgfpathlineto{\pgfqpoint{3.895497in}{3.022314in}}%
\pgfpathlineto{\pgfqpoint{3.908206in}{3.026690in}}%
\pgfpathlineto{\pgfqpoint{3.920916in}{3.022503in}}%
\pgfpathlineto{\pgfqpoint{3.927270in}{3.023252in}}%
\pgfpathlineto{\pgfqpoint{3.933625in}{3.019088in}}%
\pgfpathlineto{\pgfqpoint{3.939979in}{3.011509in}}%
\pgfpathlineto{\pgfqpoint{3.946334in}{3.006919in}}%
\pgfpathlineto{\pgfqpoint{3.952689in}{3.006288in}}%
\pgfpathlineto{\pgfqpoint{3.959043in}{3.007443in}}%
\pgfpathlineto{\pgfqpoint{3.965398in}{3.010334in}}%
\pgfpathlineto{\pgfqpoint{3.978107in}{3.018626in}}%
\pgfpathlineto{\pgfqpoint{3.990816in}{3.021782in}}%
\pgfpathlineto{\pgfqpoint{4.009880in}{3.026659in}}%
\pgfpathlineto{\pgfqpoint{4.022589in}{3.019658in}}%
\pgfpathlineto{\pgfqpoint{4.035298in}{3.018339in}}%
\pgfpathlineto{\pgfqpoint{4.041653in}{3.024097in}}%
\pgfpathlineto{\pgfqpoint{4.054362in}{3.042400in}}%
\pgfpathlineto{\pgfqpoint{4.060717in}{3.048210in}}%
\pgfpathlineto{\pgfqpoint{4.073426in}{3.052885in}}%
\pgfpathlineto{\pgfqpoint{4.079781in}{3.058235in}}%
\pgfpathlineto{\pgfqpoint{4.086135in}{3.061916in}}%
\pgfpathlineto{\pgfqpoint{4.092490in}{3.061735in}}%
\pgfpathlineto{\pgfqpoint{4.098844in}{3.057736in}}%
\pgfpathlineto{\pgfqpoint{4.105199in}{3.056493in}}%
\pgfpathlineto{\pgfqpoint{4.124263in}{3.059316in}}%
\pgfpathlineto{\pgfqpoint{4.130617in}{3.056119in}}%
\pgfpathlineto{\pgfqpoint{4.149681in}{3.033651in}}%
\pgfpathlineto{\pgfqpoint{4.168745in}{3.025545in}}%
\pgfpathlineto{\pgfqpoint{4.175100in}{3.025921in}}%
\pgfpathlineto{\pgfqpoint{4.187809in}{3.033564in}}%
\pgfpathlineto{\pgfqpoint{4.200518in}{3.041803in}}%
\pgfpathlineto{\pgfqpoint{4.206873in}{3.047389in}}%
\pgfpathlineto{\pgfqpoint{4.213227in}{3.050951in}}%
\pgfpathlineto{\pgfqpoint{4.219582in}{3.056265in}}%
\pgfpathlineto{\pgfqpoint{4.225936in}{3.056294in}}%
\pgfpathlineto{\pgfqpoint{4.232291in}{3.055158in}}%
\pgfpathlineto{\pgfqpoint{4.238646in}{3.056399in}}%
\pgfpathlineto{\pgfqpoint{4.245000in}{3.061076in}}%
\pgfpathlineto{\pgfqpoint{4.251355in}{3.064230in}}%
\pgfpathlineto{\pgfqpoint{4.257709in}{3.066138in}}%
\pgfpathlineto{\pgfqpoint{4.270419in}{3.074821in}}%
\pgfpathlineto{\pgfqpoint{4.276773in}{3.076086in}}%
\pgfpathlineto{\pgfqpoint{4.289482in}{3.075236in}}%
\pgfpathlineto{\pgfqpoint{4.302192in}{3.081522in}}%
\pgfpathlineto{\pgfqpoint{4.308546in}{3.086645in}}%
\pgfpathlineto{\pgfqpoint{4.314901in}{3.087094in}}%
\pgfpathlineto{\pgfqpoint{4.321255in}{3.084854in}}%
\pgfpathlineto{\pgfqpoint{4.333965in}{3.089427in}}%
\pgfpathlineto{\pgfqpoint{4.340319in}{3.091424in}}%
\pgfpathlineto{\pgfqpoint{4.346674in}{3.091802in}}%
\pgfpathlineto{\pgfqpoint{4.353028in}{3.093913in}}%
\pgfpathlineto{\pgfqpoint{4.359383in}{3.098775in}}%
\pgfpathlineto{\pgfqpoint{4.365738in}{3.099903in}}%
\pgfpathlineto{\pgfqpoint{4.378447in}{3.104385in}}%
\pgfpathlineto{\pgfqpoint{4.384801in}{3.108385in}}%
\pgfpathlineto{\pgfqpoint{4.391156in}{3.110140in}}%
\pgfpathlineto{\pgfqpoint{4.397511in}{3.113181in}}%
\pgfpathlineto{\pgfqpoint{4.416574in}{3.130124in}}%
\pgfpathlineto{\pgfqpoint{4.422929in}{3.136482in}}%
\pgfpathlineto{\pgfqpoint{4.429284in}{3.139231in}}%
\pgfpathlineto{\pgfqpoint{4.441993in}{3.137712in}}%
\pgfpathlineto{\pgfqpoint{4.448347in}{3.140083in}}%
\pgfpathlineto{\pgfqpoint{4.454702in}{3.143676in}}%
\pgfpathlineto{\pgfqpoint{4.461057in}{3.143918in}}%
\pgfpathlineto{\pgfqpoint{4.467411in}{3.142041in}}%
\pgfpathlineto{\pgfqpoint{4.473766in}{3.142135in}}%
\pgfpathlineto{\pgfqpoint{4.492830in}{3.131283in}}%
\pgfpathlineto{\pgfqpoint{4.505539in}{3.133285in}}%
\pgfpathlineto{\pgfqpoint{4.518248in}{3.128074in}}%
\pgfpathlineto{\pgfqpoint{4.524603in}{3.131527in}}%
\pgfpathlineto{\pgfqpoint{4.543666in}{3.135228in}}%
\pgfpathlineto{\pgfqpoint{4.550021in}{3.138205in}}%
\pgfpathlineto{\pgfqpoint{4.569085in}{3.153860in}}%
\pgfpathlineto{\pgfqpoint{4.581794in}{3.151898in}}%
\pgfpathlineto{\pgfqpoint{4.594503in}{3.142597in}}%
\pgfpathlineto{\pgfqpoint{4.600858in}{3.144100in}}%
\pgfpathlineto{\pgfqpoint{4.613567in}{3.154512in}}%
\pgfpathlineto{\pgfqpoint{4.626276in}{3.157293in}}%
\pgfpathlineto{\pgfqpoint{4.632631in}{3.157904in}}%
\pgfpathlineto{\pgfqpoint{4.638985in}{3.156773in}}%
\pgfpathlineto{\pgfqpoint{4.651695in}{3.152077in}}%
\pgfpathlineto{\pgfqpoint{4.658049in}{3.147620in}}%
\pgfpathlineto{\pgfqpoint{4.664404in}{3.144668in}}%
\pgfpathlineto{\pgfqpoint{4.664404in}{3.144668in}}%
\pgfusepath{stroke}%
\end{pgfscope}%
\begin{pgfscope}%
\pgfsetrectcap%
\pgfsetmiterjoin%
\pgfsetlinewidth{0.803000pt}%
\definecolor{currentstroke}{rgb}{0.000000,0.000000,0.000000}%
\pgfsetstrokecolor{currentstroke}%
\pgfsetdash{}{0pt}%
\pgfpathmoveto{\pgfqpoint{0.661006in}{2.135160in}}%
\pgfpathlineto{\pgfqpoint{0.661006in}{3.206606in}}%
\pgfusepath{stroke}%
\end{pgfscope}%
\begin{pgfscope}%
\pgfsetrectcap%
\pgfsetmiterjoin%
\pgfsetlinewidth{0.803000pt}%
\definecolor{currentstroke}{rgb}{0.000000,0.000000,0.000000}%
\pgfsetstrokecolor{currentstroke}%
\pgfsetdash{}{0pt}%
\pgfpathmoveto{\pgfqpoint{4.855042in}{2.135160in}}%
\pgfpathlineto{\pgfqpoint{4.855042in}{3.206606in}}%
\pgfusepath{stroke}%
\end{pgfscope}%
\begin{pgfscope}%
\pgfsetrectcap%
\pgfsetmiterjoin%
\pgfsetlinewidth{0.803000pt}%
\definecolor{currentstroke}{rgb}{0.000000,0.000000,0.000000}%
\pgfsetstrokecolor{currentstroke}%
\pgfsetdash{}{0pt}%
\pgfpathmoveto{\pgfqpoint{0.661006in}{2.135160in}}%
\pgfpathlineto{\pgfqpoint{4.855042in}{2.135160in}}%
\pgfusepath{stroke}%
\end{pgfscope}%
\begin{pgfscope}%
\pgfsetrectcap%
\pgfsetmiterjoin%
\pgfsetlinewidth{0.803000pt}%
\definecolor{currentstroke}{rgb}{0.000000,0.000000,0.000000}%
\pgfsetstrokecolor{currentstroke}%
\pgfsetdash{}{0pt}%
\pgfpathmoveto{\pgfqpoint{0.661006in}{3.206606in}}%
\pgfpathlineto{\pgfqpoint{4.855042in}{3.206606in}}%
\pgfusepath{stroke}%
\end{pgfscope}%
\begin{pgfscope}%
\pgfsetbuttcap%
\pgfsetmiterjoin%
\definecolor{currentfill}{rgb}{1.000000,1.000000,1.000000}%
\pgfsetfillcolor{currentfill}%
\pgfsetfillopacity{0.800000}%
\pgfsetlinewidth{1.003750pt}%
\definecolor{currentstroke}{rgb}{0.800000,0.800000,0.800000}%
\pgfsetstrokecolor{currentstroke}%
\pgfsetstrokeopacity{0.800000}%
\pgfsetdash{}{0pt}%
\pgfpathmoveto{\pgfqpoint{0.738783in}{2.805162in}}%
\pgfpathlineto{\pgfqpoint{1.777784in}{2.805162in}}%
\pgfpathquadraticcurveto{\pgfqpoint{1.800006in}{2.805162in}}{\pgfqpoint{1.800006in}{2.827384in}}%
\pgfpathlineto{\pgfqpoint{1.800006in}{3.128828in}}%
\pgfpathquadraticcurveto{\pgfqpoint{1.800006in}{3.151050in}}{\pgfqpoint{1.777784in}{3.151050in}}%
\pgfpathlineto{\pgfqpoint{0.738783in}{3.151050in}}%
\pgfpathquadraticcurveto{\pgfqpoint{0.716561in}{3.151050in}}{\pgfqpoint{0.716561in}{3.128828in}}%
\pgfpathlineto{\pgfqpoint{0.716561in}{2.827384in}}%
\pgfpathquadraticcurveto{\pgfqpoint{0.716561in}{2.805162in}}{\pgfqpoint{0.738783in}{2.805162in}}%
\pgfpathlineto{\pgfqpoint{0.738783in}{2.805162in}}%
\pgfpathclose%
\pgfusepath{stroke,fill}%
\end{pgfscope}%
\begin{pgfscope}%
\pgfsetrectcap%
\pgfsetroundjoin%
\pgfsetlinewidth{1.505625pt}%
\definecolor{currentstroke}{rgb}{0.003922,0.450980,0.698039}%
\pgfsetstrokecolor{currentstroke}%
\pgfsetstrokeopacity{0.700000}%
\pgfsetdash{}{0pt}%
\pgfpathmoveto{\pgfqpoint{0.761006in}{3.066162in}}%
\pgfpathlineto{\pgfqpoint{0.872117in}{3.066162in}}%
\pgfpathlineto{\pgfqpoint{0.983228in}{3.066162in}}%
\pgfusepath{stroke}%
\end{pgfscope}%
\begin{pgfscope}%
\definecolor{textcolor}{rgb}{0.000000,0.000000,0.000000}%
\pgfsetstrokecolor{textcolor}%
\pgfsetfillcolor{textcolor}%
\pgftext[x=1.072117in,y=3.027273in,left,base]{\color{textcolor}\rmfamily\fontsize{8.000000}{9.600000}\selectfont Air pressue}%
\end{pgfscope}%
\begin{pgfscope}%
\pgfsetrectcap%
\pgfsetroundjoin%
\pgfsetlinewidth{1.505625pt}%
\definecolor{currentstroke}{rgb}{0.835294,0.368627,0.000000}%
\pgfsetstrokecolor{currentstroke}%
\pgfsetstrokeopacity{0.700000}%
\pgfsetdash{}{0pt}%
\pgfpathmoveto{\pgfqpoint{0.761006in}{2.911273in}}%
\pgfpathlineto{\pgfqpoint{0.872117in}{2.911273in}}%
\pgfpathlineto{\pgfqpoint{0.983228in}{2.911273in}}%
\pgfusepath{stroke}%
\end{pgfscope}%
\begin{pgfscope}%
\definecolor{textcolor}{rgb}{0.000000,0.000000,0.000000}%
\pgfsetstrokecolor{textcolor}%
\pgfsetfillcolor{textcolor}%
\pgftext[x=1.072117in,y=2.872384in,left,base]{\color{textcolor}\rmfamily\fontsize{8.000000}{9.600000}\selectfont Piezo voltage}%
\end{pgfscope}%
\begin{pgfscope}%
\pgfsetbuttcap%
\pgfsetmiterjoin%
\definecolor{currentfill}{rgb}{1.000000,1.000000,1.000000}%
\pgfsetfillcolor{currentfill}%
\pgfsetlinewidth{0.000000pt}%
\definecolor{currentstroke}{rgb}{0.000000,0.000000,0.000000}%
\pgfsetstrokecolor{currentstroke}%
\pgfsetstrokeopacity{0.000000}%
\pgfsetdash{}{0pt}%
\pgfpathmoveto{\pgfqpoint{0.661006in}{0.524170in}}%
\pgfpathlineto{\pgfqpoint{4.855042in}{0.524170in}}%
\pgfpathlineto{\pgfqpoint{4.855042in}{1.595616in}}%
\pgfpathlineto{\pgfqpoint{0.661006in}{1.595616in}}%
\pgfpathlineto{\pgfqpoint{0.661006in}{0.524170in}}%
\pgfpathclose%
\pgfusepath{fill}%
\end{pgfscope}%
\begin{pgfscope}%
\pgfpathrectangle{\pgfqpoint{0.661006in}{0.524170in}}{\pgfqpoint{4.194036in}{1.071446in}}%
\pgfusepath{clip}%
\pgfsetbuttcap%
\pgfsetroundjoin%
\definecolor{currentfill}{rgb}{0.780108,0.874195,0.919757}%
\pgfsetfillcolor{currentfill}%
\pgfsetfillopacity{0.700000}%
\pgfsetlinewidth{1.003750pt}%
\definecolor{currentstroke}{rgb}{0.780108,0.874195,0.919757}%
\pgfsetstrokecolor{currentstroke}%
\pgfsetstrokeopacity{0.700000}%
\pgfsetdash{}{0pt}%
\pgfpathmoveto{\pgfqpoint{0.891939in}{1.534892in}}%
\pgfpathcurveto{\pgfqpoint{0.893781in}{1.534892in}}{\pgfqpoint{0.895548in}{1.535624in}}{\pgfqpoint{0.896850in}{1.536926in}}%
\pgfpathcurveto{\pgfqpoint{0.898152in}{1.538228in}}{\pgfqpoint{0.898884in}{1.539995in}}{\pgfqpoint{0.898884in}{1.541836in}}%
\pgfpathcurveto{\pgfqpoint{0.898884in}{1.543678in}}{\pgfqpoint{0.898152in}{1.545444in}}{\pgfqpoint{0.896850in}{1.546747in}}%
\pgfpathcurveto{\pgfqpoint{0.895548in}{1.548049in}}{\pgfqpoint{0.893781in}{1.548781in}}{\pgfqpoint{0.891939in}{1.548781in}}%
\pgfpathcurveto{\pgfqpoint{0.890098in}{1.548781in}}{\pgfqpoint{0.888331in}{1.548049in}}{\pgfqpoint{0.887029in}{1.546747in}}%
\pgfpathcurveto{\pgfqpoint{0.885727in}{1.545444in}}{\pgfqpoint{0.884995in}{1.543678in}}{\pgfqpoint{0.884995in}{1.541836in}}%
\pgfpathcurveto{\pgfqpoint{0.884995in}{1.539995in}}{\pgfqpoint{0.885727in}{1.538228in}}{\pgfqpoint{0.887029in}{1.536926in}}%
\pgfpathcurveto{\pgfqpoint{0.888331in}{1.535624in}}{\pgfqpoint{0.890098in}{1.534892in}}{\pgfqpoint{0.891939in}{1.534892in}}%
\pgfpathlineto{\pgfqpoint{0.891939in}{1.534892in}}%
\pgfpathclose%
\pgfusepath{stroke,fill}%
\end{pgfscope}%
\begin{pgfscope}%
\pgfpathrectangle{\pgfqpoint{0.661006in}{0.524170in}}{\pgfqpoint{4.194036in}{1.071446in}}%
\pgfusepath{clip}%
\pgfsetbuttcap%
\pgfsetroundjoin%
\definecolor{currentfill}{rgb}{0.780108,0.874195,0.919757}%
\pgfsetfillcolor{currentfill}%
\pgfsetfillopacity{0.700000}%
\pgfsetlinewidth{1.003750pt}%
\definecolor{currentstroke}{rgb}{0.780108,0.874195,0.919757}%
\pgfsetstrokecolor{currentstroke}%
\pgfsetstrokeopacity{0.700000}%
\pgfsetdash{}{0pt}%
\pgfpathmoveto{\pgfqpoint{0.877903in}{1.537767in}}%
\pgfpathcurveto{\pgfqpoint{0.879745in}{1.537767in}}{\pgfqpoint{0.881512in}{1.538499in}}{\pgfqpoint{0.882814in}{1.539801in}}%
\pgfpathcurveto{\pgfqpoint{0.884116in}{1.541104in}}{\pgfqpoint{0.884848in}{1.542870in}}{\pgfqpoint{0.884848in}{1.544712in}}%
\pgfpathcurveto{\pgfqpoint{0.884848in}{1.546554in}}{\pgfqpoint{0.884116in}{1.548320in}}{\pgfqpoint{0.882814in}{1.549622in}}%
\pgfpathcurveto{\pgfqpoint{0.881512in}{1.550925in}}{\pgfqpoint{0.879745in}{1.551656in}}{\pgfqpoint{0.877903in}{1.551656in}}%
\pgfpathcurveto{\pgfqpoint{0.876062in}{1.551656in}}{\pgfqpoint{0.874295in}{1.550925in}}{\pgfqpoint{0.872993in}{1.549622in}}%
\pgfpathcurveto{\pgfqpoint{0.871691in}{1.548320in}}{\pgfqpoint{0.870959in}{1.546554in}}{\pgfqpoint{0.870959in}{1.544712in}}%
\pgfpathcurveto{\pgfqpoint{0.870959in}{1.542870in}}{\pgfqpoint{0.871691in}{1.541104in}}{\pgfqpoint{0.872993in}{1.539801in}}%
\pgfpathcurveto{\pgfqpoint{0.874295in}{1.538499in}}{\pgfqpoint{0.876062in}{1.537767in}}{\pgfqpoint{0.877903in}{1.537767in}}%
\pgfpathlineto{\pgfqpoint{0.877903in}{1.537767in}}%
\pgfpathclose%
\pgfusepath{stroke,fill}%
\end{pgfscope}%
\begin{pgfscope}%
\pgfpathrectangle{\pgfqpoint{0.661006in}{0.524170in}}{\pgfqpoint{4.194036in}{1.071446in}}%
\pgfusepath{clip}%
\pgfsetbuttcap%
\pgfsetroundjoin%
\definecolor{currentfill}{rgb}{0.780108,0.874195,0.919757}%
\pgfsetfillcolor{currentfill}%
\pgfsetfillopacity{0.700000}%
\pgfsetlinewidth{1.003750pt}%
\definecolor{currentstroke}{rgb}{0.780108,0.874195,0.919757}%
\pgfsetstrokecolor{currentstroke}%
\pgfsetstrokeopacity{0.700000}%
\pgfsetdash{}{0pt}%
\pgfpathmoveto{\pgfqpoint{0.877857in}{1.536904in}}%
\pgfpathcurveto{\pgfqpoint{0.879699in}{1.536904in}}{\pgfqpoint{0.881465in}{1.537635in}}{\pgfqpoint{0.882767in}{1.538938in}}%
\pgfpathcurveto{\pgfqpoint{0.884070in}{1.540240in}}{\pgfqpoint{0.884801in}{1.542006in}}{\pgfqpoint{0.884801in}{1.543848in}}%
\pgfpathcurveto{\pgfqpoint{0.884801in}{1.545690in}}{\pgfqpoint{0.884070in}{1.547456in}}{\pgfqpoint{0.882767in}{1.548759in}}%
\pgfpathcurveto{\pgfqpoint{0.881465in}{1.550061in}}{\pgfqpoint{0.879699in}{1.550793in}}{\pgfqpoint{0.877857in}{1.550793in}}%
\pgfpathcurveto{\pgfqpoint{0.876015in}{1.550793in}}{\pgfqpoint{0.874249in}{1.550061in}}{\pgfqpoint{0.872946in}{1.548759in}}%
\pgfpathcurveto{\pgfqpoint{0.871644in}{1.547456in}}{\pgfqpoint{0.870912in}{1.545690in}}{\pgfqpoint{0.870912in}{1.543848in}}%
\pgfpathcurveto{\pgfqpoint{0.870912in}{1.542006in}}{\pgfqpoint{0.871644in}{1.540240in}}{\pgfqpoint{0.872946in}{1.538938in}}%
\pgfpathcurveto{\pgfqpoint{0.874249in}{1.537635in}}{\pgfqpoint{0.876015in}{1.536904in}}{\pgfqpoint{0.877857in}{1.536904in}}%
\pgfpathlineto{\pgfqpoint{0.877857in}{1.536904in}}%
\pgfpathclose%
\pgfusepath{stroke,fill}%
\end{pgfscope}%
\begin{pgfscope}%
\pgfpathrectangle{\pgfqpoint{0.661006in}{0.524170in}}{\pgfqpoint{4.194036in}{1.071446in}}%
\pgfusepath{clip}%
\pgfsetbuttcap%
\pgfsetroundjoin%
\definecolor{currentfill}{rgb}{0.775014,0.870507,0.917777}%
\pgfsetfillcolor{currentfill}%
\pgfsetfillopacity{0.700000}%
\pgfsetlinewidth{1.003750pt}%
\definecolor{currentstroke}{rgb}{0.775014,0.870507,0.917777}%
\pgfsetstrokecolor{currentstroke}%
\pgfsetstrokeopacity{0.700000}%
\pgfsetdash{}{0pt}%
\pgfpathmoveto{\pgfqpoint{0.872326in}{1.536727in}}%
\pgfpathcurveto{\pgfqpoint{0.874168in}{1.536727in}}{\pgfqpoint{0.875934in}{1.537459in}}{\pgfqpoint{0.877237in}{1.538761in}}%
\pgfpathcurveto{\pgfqpoint{0.878539in}{1.540064in}}{\pgfqpoint{0.879270in}{1.541830in}}{\pgfqpoint{0.879270in}{1.543672in}}%
\pgfpathcurveto{\pgfqpoint{0.879270in}{1.545514in}}{\pgfqpoint{0.878539in}{1.547280in}}{\pgfqpoint{0.877237in}{1.548582in}}%
\pgfpathcurveto{\pgfqpoint{0.875934in}{1.549885in}}{\pgfqpoint{0.874168in}{1.550616in}}{\pgfqpoint{0.872326in}{1.550616in}}%
\pgfpathcurveto{\pgfqpoint{0.870484in}{1.550616in}}{\pgfqpoint{0.868718in}{1.549885in}}{\pgfqpoint{0.867416in}{1.548582in}}%
\pgfpathcurveto{\pgfqpoint{0.866113in}{1.547280in}}{\pgfqpoint{0.865382in}{1.545514in}}{\pgfqpoint{0.865382in}{1.543672in}}%
\pgfpathcurveto{\pgfqpoint{0.865382in}{1.541830in}}{\pgfqpoint{0.866113in}{1.540064in}}{\pgfqpoint{0.867416in}{1.538761in}}%
\pgfpathcurveto{\pgfqpoint{0.868718in}{1.537459in}}{\pgfqpoint{0.870484in}{1.536727in}}{\pgfqpoint{0.872326in}{1.536727in}}%
\pgfpathlineto{\pgfqpoint{0.872326in}{1.536727in}}%
\pgfpathclose%
\pgfusepath{stroke,fill}%
\end{pgfscope}%
\begin{pgfscope}%
\pgfpathrectangle{\pgfqpoint{0.661006in}{0.524170in}}{\pgfqpoint{4.194036in}{1.071446in}}%
\pgfusepath{clip}%
\pgfsetbuttcap%
\pgfsetroundjoin%
\definecolor{currentfill}{rgb}{0.775014,0.870507,0.917777}%
\pgfsetfillcolor{currentfill}%
\pgfsetfillopacity{0.700000}%
\pgfsetlinewidth{1.003750pt}%
\definecolor{currentstroke}{rgb}{0.775014,0.870507,0.917777}%
\pgfsetstrokecolor{currentstroke}%
\pgfsetstrokeopacity{0.700000}%
\pgfsetdash{}{0pt}%
\pgfpathmoveto{\pgfqpoint{0.878415in}{1.535727in}}%
\pgfpathcurveto{\pgfqpoint{0.880256in}{1.535727in}}{\pgfqpoint{0.882023in}{1.536458in}}{\pgfqpoint{0.883325in}{1.537761in}}%
\pgfpathcurveto{\pgfqpoint{0.884627in}{1.539063in}}{\pgfqpoint{0.885359in}{1.540829in}}{\pgfqpoint{0.885359in}{1.542671in}}%
\pgfpathcurveto{\pgfqpoint{0.885359in}{1.544513in}}{\pgfqpoint{0.884627in}{1.546279in}}{\pgfqpoint{0.883325in}{1.547581in}}%
\pgfpathcurveto{\pgfqpoint{0.882023in}{1.548884in}}{\pgfqpoint{0.880256in}{1.549615in}}{\pgfqpoint{0.878415in}{1.549615in}}%
\pgfpathcurveto{\pgfqpoint{0.876573in}{1.549615in}}{\pgfqpoint{0.874806in}{1.548884in}}{\pgfqpoint{0.873504in}{1.547581in}}%
\pgfpathcurveto{\pgfqpoint{0.872202in}{1.546279in}}{\pgfqpoint{0.871470in}{1.544513in}}{\pgfqpoint{0.871470in}{1.542671in}}%
\pgfpathcurveto{\pgfqpoint{0.871470in}{1.540829in}}{\pgfqpoint{0.872202in}{1.539063in}}{\pgfqpoint{0.873504in}{1.537761in}}%
\pgfpathcurveto{\pgfqpoint{0.874806in}{1.536458in}}{\pgfqpoint{0.876573in}{1.535727in}}{\pgfqpoint{0.878415in}{1.535727in}}%
\pgfpathlineto{\pgfqpoint{0.878415in}{1.535727in}}%
\pgfpathclose%
\pgfusepath{stroke,fill}%
\end{pgfscope}%
\begin{pgfscope}%
\pgfpathrectangle{\pgfqpoint{0.661006in}{0.524170in}}{\pgfqpoint{4.194036in}{1.071446in}}%
\pgfusepath{clip}%
\pgfsetbuttcap%
\pgfsetroundjoin%
\definecolor{currentfill}{rgb}{0.775014,0.870507,0.917777}%
\pgfsetfillcolor{currentfill}%
\pgfsetfillopacity{0.700000}%
\pgfsetlinewidth{1.003750pt}%
\definecolor{currentstroke}{rgb}{0.775014,0.870507,0.917777}%
\pgfsetstrokecolor{currentstroke}%
\pgfsetstrokeopacity{0.700000}%
\pgfsetdash{}{0pt}%
\pgfpathmoveto{\pgfqpoint{0.887896in}{1.532889in}}%
\pgfpathcurveto{\pgfqpoint{0.889738in}{1.532889in}}{\pgfqpoint{0.891504in}{1.533621in}}{\pgfqpoint{0.892806in}{1.534923in}}%
\pgfpathcurveto{\pgfqpoint{0.894109in}{1.536225in}}{\pgfqpoint{0.894840in}{1.537992in}}{\pgfqpoint{0.894840in}{1.539833in}}%
\pgfpathcurveto{\pgfqpoint{0.894840in}{1.541675in}}{\pgfqpoint{0.894109in}{1.543441in}}{\pgfqpoint{0.892806in}{1.544744in}}%
\pgfpathcurveto{\pgfqpoint{0.891504in}{1.546046in}}{\pgfqpoint{0.889738in}{1.546778in}}{\pgfqpoint{0.887896in}{1.546778in}}%
\pgfpathcurveto{\pgfqpoint{0.886054in}{1.546778in}}{\pgfqpoint{0.884288in}{1.546046in}}{\pgfqpoint{0.882985in}{1.544744in}}%
\pgfpathcurveto{\pgfqpoint{0.881683in}{1.543441in}}{\pgfqpoint{0.880951in}{1.541675in}}{\pgfqpoint{0.880951in}{1.539833in}}%
\pgfpathcurveto{\pgfqpoint{0.880951in}{1.537992in}}{\pgfqpoint{0.881683in}{1.536225in}}{\pgfqpoint{0.882985in}{1.534923in}}%
\pgfpathcurveto{\pgfqpoint{0.884288in}{1.533621in}}{\pgfqpoint{0.886054in}{1.532889in}}{\pgfqpoint{0.887896in}{1.532889in}}%
\pgfpathlineto{\pgfqpoint{0.887896in}{1.532889in}}%
\pgfpathclose%
\pgfusepath{stroke,fill}%
\end{pgfscope}%
\begin{pgfscope}%
\pgfpathrectangle{\pgfqpoint{0.661006in}{0.524170in}}{\pgfqpoint{4.194036in}{1.071446in}}%
\pgfusepath{clip}%
\pgfsetbuttcap%
\pgfsetroundjoin%
\definecolor{currentfill}{rgb}{0.775014,0.870507,0.917777}%
\pgfsetfillcolor{currentfill}%
\pgfsetfillopacity{0.700000}%
\pgfsetlinewidth{1.003750pt}%
\definecolor{currentstroke}{rgb}{0.775014,0.870507,0.917777}%
\pgfsetstrokecolor{currentstroke}%
\pgfsetstrokeopacity{0.700000}%
\pgfsetdash{}{0pt}%
\pgfpathmoveto{\pgfqpoint{0.903977in}{1.528506in}}%
\pgfpathcurveto{\pgfqpoint{0.905819in}{1.528506in}}{\pgfqpoint{0.907585in}{1.529238in}}{\pgfqpoint{0.908888in}{1.530540in}}%
\pgfpathcurveto{\pgfqpoint{0.910190in}{1.531842in}}{\pgfqpoint{0.910921in}{1.533609in}}{\pgfqpoint{0.910921in}{1.535450in}}%
\pgfpathcurveto{\pgfqpoint{0.910921in}{1.537292in}}{\pgfqpoint{0.910190in}{1.539058in}}{\pgfqpoint{0.908888in}{1.540361in}}%
\pgfpathcurveto{\pgfqpoint{0.907585in}{1.541663in}}{\pgfqpoint{0.905819in}{1.542395in}}{\pgfqpoint{0.903977in}{1.542395in}}%
\pgfpathcurveto{\pgfqpoint{0.902135in}{1.542395in}}{\pgfqpoint{0.900369in}{1.541663in}}{\pgfqpoint{0.899067in}{1.540361in}}%
\pgfpathcurveto{\pgfqpoint{0.897764in}{1.539058in}}{\pgfqpoint{0.897033in}{1.537292in}}{\pgfqpoint{0.897033in}{1.535450in}}%
\pgfpathcurveto{\pgfqpoint{0.897033in}{1.533609in}}{\pgfqpoint{0.897764in}{1.531842in}}{\pgfqpoint{0.899067in}{1.530540in}}%
\pgfpathcurveto{\pgfqpoint{0.900369in}{1.529238in}}{\pgfqpoint{0.902135in}{1.528506in}}{\pgfqpoint{0.903977in}{1.528506in}}%
\pgfpathlineto{\pgfqpoint{0.903977in}{1.528506in}}%
\pgfpathclose%
\pgfusepath{stroke,fill}%
\end{pgfscope}%
\begin{pgfscope}%
\pgfpathrectangle{\pgfqpoint{0.661006in}{0.524170in}}{\pgfqpoint{4.194036in}{1.071446in}}%
\pgfusepath{clip}%
\pgfsetbuttcap%
\pgfsetroundjoin%
\definecolor{currentfill}{rgb}{0.775014,0.870507,0.917777}%
\pgfsetfillcolor{currentfill}%
\pgfsetfillopacity{0.700000}%
\pgfsetlinewidth{1.003750pt}%
\definecolor{currentstroke}{rgb}{0.775014,0.870507,0.917777}%
\pgfsetstrokecolor{currentstroke}%
\pgfsetstrokeopacity{0.700000}%
\pgfsetdash{}{0pt}%
\pgfpathmoveto{\pgfqpoint{0.900212in}{1.529199in}}%
\pgfpathcurveto{\pgfqpoint{0.902054in}{1.529199in}}{\pgfqpoint{0.903821in}{1.529931in}}{\pgfqpoint{0.905123in}{1.531233in}}%
\pgfpathcurveto{\pgfqpoint{0.906425in}{1.532535in}}{\pgfqpoint{0.907157in}{1.534302in}}{\pgfqpoint{0.907157in}{1.536143in}}%
\pgfpathcurveto{\pgfqpoint{0.907157in}{1.537985in}}{\pgfqpoint{0.906425in}{1.539752in}}{\pgfqpoint{0.905123in}{1.541054in}}%
\pgfpathcurveto{\pgfqpoint{0.903821in}{1.542356in}}{\pgfqpoint{0.902054in}{1.543088in}}{\pgfqpoint{0.900212in}{1.543088in}}%
\pgfpathcurveto{\pgfqpoint{0.898371in}{1.543088in}}{\pgfqpoint{0.896604in}{1.542356in}}{\pgfqpoint{0.895302in}{1.541054in}}%
\pgfpathcurveto{\pgfqpoint{0.894000in}{1.539752in}}{\pgfqpoint{0.893268in}{1.537985in}}{\pgfqpoint{0.893268in}{1.536143in}}%
\pgfpathcurveto{\pgfqpoint{0.893268in}{1.534302in}}{\pgfqpoint{0.894000in}{1.532535in}}{\pgfqpoint{0.895302in}{1.531233in}}%
\pgfpathcurveto{\pgfqpoint{0.896604in}{1.529931in}}{\pgfqpoint{0.898371in}{1.529199in}}{\pgfqpoint{0.900212in}{1.529199in}}%
\pgfpathlineto{\pgfqpoint{0.900212in}{1.529199in}}%
\pgfpathclose%
\pgfusepath{stroke,fill}%
\end{pgfscope}%
\begin{pgfscope}%
\pgfpathrectangle{\pgfqpoint{0.661006in}{0.524170in}}{\pgfqpoint{4.194036in}{1.071446in}}%
\pgfusepath{clip}%
\pgfsetbuttcap%
\pgfsetroundjoin%
\definecolor{currentfill}{rgb}{0.769956,0.866802,0.915795}%
\pgfsetfillcolor{currentfill}%
\pgfsetfillopacity{0.700000}%
\pgfsetlinewidth{1.003750pt}%
\definecolor{currentstroke}{rgb}{0.769956,0.866802,0.915795}%
\pgfsetstrokecolor{currentstroke}%
\pgfsetstrokeopacity{0.700000}%
\pgfsetdash{}{0pt}%
\pgfpathmoveto{\pgfqpoint{0.884178in}{1.533023in}}%
\pgfpathcurveto{\pgfqpoint{0.886019in}{1.533023in}}{\pgfqpoint{0.887786in}{1.533755in}}{\pgfqpoint{0.889088in}{1.535057in}}%
\pgfpathcurveto{\pgfqpoint{0.890390in}{1.536359in}}{\pgfqpoint{0.891122in}{1.538126in}}{\pgfqpoint{0.891122in}{1.539968in}}%
\pgfpathcurveto{\pgfqpoint{0.891122in}{1.541809in}}{\pgfqpoint{0.890390in}{1.543576in}}{\pgfqpoint{0.889088in}{1.544878in}}%
\pgfpathcurveto{\pgfqpoint{0.887786in}{1.546180in}}{\pgfqpoint{0.886019in}{1.546912in}}{\pgfqpoint{0.884178in}{1.546912in}}%
\pgfpathcurveto{\pgfqpoint{0.882336in}{1.546912in}}{\pgfqpoint{0.880570in}{1.546180in}}{\pgfqpoint{0.879267in}{1.544878in}}%
\pgfpathcurveto{\pgfqpoint{0.877965in}{1.543576in}}{\pgfqpoint{0.877233in}{1.541809in}}{\pgfqpoint{0.877233in}{1.539968in}}%
\pgfpathcurveto{\pgfqpoint{0.877233in}{1.538126in}}{\pgfqpoint{0.877965in}{1.536359in}}{\pgfqpoint{0.879267in}{1.535057in}}%
\pgfpathcurveto{\pgfqpoint{0.880570in}{1.533755in}}{\pgfqpoint{0.882336in}{1.533023in}}{\pgfqpoint{0.884178in}{1.533023in}}%
\pgfpathlineto{\pgfqpoint{0.884178in}{1.533023in}}%
\pgfpathclose%
\pgfusepath{stroke,fill}%
\end{pgfscope}%
\begin{pgfscope}%
\pgfpathrectangle{\pgfqpoint{0.661006in}{0.524170in}}{\pgfqpoint{4.194036in}{1.071446in}}%
\pgfusepath{clip}%
\pgfsetbuttcap%
\pgfsetroundjoin%
\definecolor{currentfill}{rgb}{0.769956,0.866802,0.915795}%
\pgfsetfillcolor{currentfill}%
\pgfsetfillopacity{0.700000}%
\pgfsetlinewidth{1.003750pt}%
\definecolor{currentstroke}{rgb}{0.769956,0.866802,0.915795}%
\pgfsetstrokecolor{currentstroke}%
\pgfsetstrokeopacity{0.700000}%
\pgfsetdash{}{0pt}%
\pgfpathmoveto{\pgfqpoint{0.867353in}{1.537290in}}%
\pgfpathcurveto{\pgfqpoint{0.869195in}{1.537290in}}{\pgfqpoint{0.870961in}{1.538022in}}{\pgfqpoint{0.872263in}{1.539324in}}%
\pgfpathcurveto{\pgfqpoint{0.873566in}{1.540627in}}{\pgfqpoint{0.874297in}{1.542393in}}{\pgfqpoint{0.874297in}{1.544235in}}%
\pgfpathcurveto{\pgfqpoint{0.874297in}{1.546077in}}{\pgfqpoint{0.873566in}{1.547843in}}{\pgfqpoint{0.872263in}{1.549145in}}%
\pgfpathcurveto{\pgfqpoint{0.870961in}{1.550448in}}{\pgfqpoint{0.869195in}{1.551179in}}{\pgfqpoint{0.867353in}{1.551179in}}%
\pgfpathcurveto{\pgfqpoint{0.865511in}{1.551179in}}{\pgfqpoint{0.863745in}{1.550448in}}{\pgfqpoint{0.862443in}{1.549145in}}%
\pgfpathcurveto{\pgfqpoint{0.861140in}{1.547843in}}{\pgfqpoint{0.860409in}{1.546077in}}{\pgfqpoint{0.860409in}{1.544235in}}%
\pgfpathcurveto{\pgfqpoint{0.860409in}{1.542393in}}{\pgfqpoint{0.861140in}{1.540627in}}{\pgfqpoint{0.862443in}{1.539324in}}%
\pgfpathcurveto{\pgfqpoint{0.863745in}{1.538022in}}{\pgfqpoint{0.865511in}{1.537290in}}{\pgfqpoint{0.867353in}{1.537290in}}%
\pgfpathlineto{\pgfqpoint{0.867353in}{1.537290in}}%
\pgfpathclose%
\pgfusepath{stroke,fill}%
\end{pgfscope}%
\begin{pgfscope}%
\pgfpathrectangle{\pgfqpoint{0.661006in}{0.524170in}}{\pgfqpoint{4.194036in}{1.071446in}}%
\pgfusepath{clip}%
\pgfsetbuttcap%
\pgfsetroundjoin%
\definecolor{currentfill}{rgb}{0.764932,0.863079,0.913813}%
\pgfsetfillcolor{currentfill}%
\pgfsetfillopacity{0.700000}%
\pgfsetlinewidth{1.003750pt}%
\definecolor{currentstroke}{rgb}{0.764932,0.863079,0.913813}%
\pgfsetstrokecolor{currentstroke}%
\pgfsetstrokeopacity{0.700000}%
\pgfsetdash{}{0pt}%
\pgfpathmoveto{\pgfqpoint{0.851644in}{1.539969in}}%
\pgfpathcurveto{\pgfqpoint{0.853485in}{1.539969in}}{\pgfqpoint{0.855252in}{1.540701in}}{\pgfqpoint{0.856554in}{1.542003in}}%
\pgfpathcurveto{\pgfqpoint{0.857856in}{1.543305in}}{\pgfqpoint{0.858588in}{1.545072in}}{\pgfqpoint{0.858588in}{1.546914in}}%
\pgfpathcurveto{\pgfqpoint{0.858588in}{1.548755in}}{\pgfqpoint{0.857856in}{1.550522in}}{\pgfqpoint{0.856554in}{1.551824in}}%
\pgfpathcurveto{\pgfqpoint{0.855252in}{1.553126in}}{\pgfqpoint{0.853485in}{1.553858in}}{\pgfqpoint{0.851644in}{1.553858in}}%
\pgfpathcurveto{\pgfqpoint{0.849802in}{1.553858in}}{\pgfqpoint{0.848035in}{1.553126in}}{\pgfqpoint{0.846733in}{1.551824in}}%
\pgfpathcurveto{\pgfqpoint{0.845431in}{1.550522in}}{\pgfqpoint{0.844699in}{1.548755in}}{\pgfqpoint{0.844699in}{1.546914in}}%
\pgfpathcurveto{\pgfqpoint{0.844699in}{1.545072in}}{\pgfqpoint{0.845431in}{1.543305in}}{\pgfqpoint{0.846733in}{1.542003in}}%
\pgfpathcurveto{\pgfqpoint{0.848035in}{1.540701in}}{\pgfqpoint{0.849802in}{1.539969in}}{\pgfqpoint{0.851644in}{1.539969in}}%
\pgfpathlineto{\pgfqpoint{0.851644in}{1.539969in}}%
\pgfpathclose%
\pgfusepath{stroke,fill}%
\end{pgfscope}%
\begin{pgfscope}%
\pgfpathrectangle{\pgfqpoint{0.661006in}{0.524170in}}{\pgfqpoint{4.194036in}{1.071446in}}%
\pgfusepath{clip}%
\pgfsetbuttcap%
\pgfsetroundjoin%
\definecolor{currentfill}{rgb}{0.764932,0.863079,0.913813}%
\pgfsetfillcolor{currentfill}%
\pgfsetfillopacity{0.700000}%
\pgfsetlinewidth{1.003750pt}%
\definecolor{currentstroke}{rgb}{0.764932,0.863079,0.913813}%
\pgfsetstrokecolor{currentstroke}%
\pgfsetstrokeopacity{0.700000}%
\pgfsetdash{}{0pt}%
\pgfpathmoveto{\pgfqpoint{0.855037in}{1.537793in}}%
\pgfpathcurveto{\pgfqpoint{0.856878in}{1.537793in}}{\pgfqpoint{0.858645in}{1.538524in}}{\pgfqpoint{0.859947in}{1.539826in}}%
\pgfpathcurveto{\pgfqpoint{0.861249in}{1.541129in}}{\pgfqpoint{0.861981in}{1.542895in}}{\pgfqpoint{0.861981in}{1.544737in}}%
\pgfpathcurveto{\pgfqpoint{0.861981in}{1.546579in}}{\pgfqpoint{0.861249in}{1.548345in}}{\pgfqpoint{0.859947in}{1.549647in}}%
\pgfpathcurveto{\pgfqpoint{0.858645in}{1.550950in}}{\pgfqpoint{0.856878in}{1.551681in}}{\pgfqpoint{0.855037in}{1.551681in}}%
\pgfpathcurveto{\pgfqpoint{0.853195in}{1.551681in}}{\pgfqpoint{0.851428in}{1.550950in}}{\pgfqpoint{0.850126in}{1.549647in}}%
\pgfpathcurveto{\pgfqpoint{0.848824in}{1.548345in}}{\pgfqpoint{0.848092in}{1.546579in}}{\pgfqpoint{0.848092in}{1.544737in}}%
\pgfpathcurveto{\pgfqpoint{0.848092in}{1.542895in}}{\pgfqpoint{0.848824in}{1.541129in}}{\pgfqpoint{0.850126in}{1.539826in}}%
\pgfpathcurveto{\pgfqpoint{0.851428in}{1.538524in}}{\pgfqpoint{0.853195in}{1.537793in}}{\pgfqpoint{0.855037in}{1.537793in}}%
\pgfpathlineto{\pgfqpoint{0.855037in}{1.537793in}}%
\pgfpathclose%
\pgfusepath{stroke,fill}%
\end{pgfscope}%
\begin{pgfscope}%
\pgfpathrectangle{\pgfqpoint{0.661006in}{0.524170in}}{\pgfqpoint{4.194036in}{1.071446in}}%
\pgfusepath{clip}%
\pgfsetbuttcap%
\pgfsetroundjoin%
\definecolor{currentfill}{rgb}{0.764932,0.863079,0.913813}%
\pgfsetfillcolor{currentfill}%
\pgfsetfillopacity{0.700000}%
\pgfsetlinewidth{1.003750pt}%
\definecolor{currentstroke}{rgb}{0.764932,0.863079,0.913813}%
\pgfsetstrokecolor{currentstroke}%
\pgfsetstrokeopacity{0.700000}%
\pgfsetdash{}{0pt}%
\pgfpathmoveto{\pgfqpoint{0.858708in}{1.537687in}}%
\pgfpathcurveto{\pgfqpoint{0.860550in}{1.537687in}}{\pgfqpoint{0.862316in}{1.538419in}}{\pgfqpoint{0.863619in}{1.539721in}}%
\pgfpathcurveto{\pgfqpoint{0.864921in}{1.541023in}}{\pgfqpoint{0.865653in}{1.542790in}}{\pgfqpoint{0.865653in}{1.544632in}}%
\pgfpathcurveto{\pgfqpoint{0.865653in}{1.546473in}}{\pgfqpoint{0.864921in}{1.548240in}}{\pgfqpoint{0.863619in}{1.549542in}}%
\pgfpathcurveto{\pgfqpoint{0.862316in}{1.550844in}}{\pgfqpoint{0.860550in}{1.551576in}}{\pgfqpoint{0.858708in}{1.551576in}}%
\pgfpathcurveto{\pgfqpoint{0.856867in}{1.551576in}}{\pgfqpoint{0.855100in}{1.550844in}}{\pgfqpoint{0.853798in}{1.549542in}}%
\pgfpathcurveto{\pgfqpoint{0.852495in}{1.548240in}}{\pgfqpoint{0.851764in}{1.546473in}}{\pgfqpoint{0.851764in}{1.544632in}}%
\pgfpathcurveto{\pgfqpoint{0.851764in}{1.542790in}}{\pgfqpoint{0.852495in}{1.541023in}}{\pgfqpoint{0.853798in}{1.539721in}}%
\pgfpathcurveto{\pgfqpoint{0.855100in}{1.538419in}}{\pgfqpoint{0.856867in}{1.537687in}}{\pgfqpoint{0.858708in}{1.537687in}}%
\pgfpathlineto{\pgfqpoint{0.858708in}{1.537687in}}%
\pgfpathclose%
\pgfusepath{stroke,fill}%
\end{pgfscope}%
\begin{pgfscope}%
\pgfpathrectangle{\pgfqpoint{0.661006in}{0.524170in}}{\pgfqpoint{4.194036in}{1.071446in}}%
\pgfusepath{clip}%
\pgfsetbuttcap%
\pgfsetroundjoin%
\definecolor{currentfill}{rgb}{0.764932,0.863079,0.913813}%
\pgfsetfillcolor{currentfill}%
\pgfsetfillopacity{0.700000}%
\pgfsetlinewidth{1.003750pt}%
\definecolor{currentstroke}{rgb}{0.764932,0.863079,0.913813}%
\pgfsetstrokecolor{currentstroke}%
\pgfsetstrokeopacity{0.700000}%
\pgfsetdash{}{0pt}%
\pgfpathmoveto{\pgfqpoint{0.863728in}{1.536279in}}%
\pgfpathcurveto{\pgfqpoint{0.865569in}{1.536279in}}{\pgfqpoint{0.867336in}{1.537010in}}{\pgfqpoint{0.868638in}{1.538313in}}%
\pgfpathcurveto{\pgfqpoint{0.869940in}{1.539615in}}{\pgfqpoint{0.870672in}{1.541381in}}{\pgfqpoint{0.870672in}{1.543223in}}%
\pgfpathcurveto{\pgfqpoint{0.870672in}{1.545065in}}{\pgfqpoint{0.869940in}{1.546831in}}{\pgfqpoint{0.868638in}{1.548134in}}%
\pgfpathcurveto{\pgfqpoint{0.867336in}{1.549436in}}{\pgfqpoint{0.865569in}{1.550168in}}{\pgfqpoint{0.863728in}{1.550168in}}%
\pgfpathcurveto{\pgfqpoint{0.861886in}{1.550168in}}{\pgfqpoint{0.860120in}{1.549436in}}{\pgfqpoint{0.858817in}{1.548134in}}%
\pgfpathcurveto{\pgfqpoint{0.857515in}{1.546831in}}{\pgfqpoint{0.856783in}{1.545065in}}{\pgfqpoint{0.856783in}{1.543223in}}%
\pgfpathcurveto{\pgfqpoint{0.856783in}{1.541381in}}{\pgfqpoint{0.857515in}{1.539615in}}{\pgfqpoint{0.858817in}{1.538313in}}%
\pgfpathcurveto{\pgfqpoint{0.860120in}{1.537010in}}{\pgfqpoint{0.861886in}{1.536279in}}{\pgfqpoint{0.863728in}{1.536279in}}%
\pgfpathlineto{\pgfqpoint{0.863728in}{1.536279in}}%
\pgfpathclose%
\pgfusepath{stroke,fill}%
\end{pgfscope}%
\begin{pgfscope}%
\pgfpathrectangle{\pgfqpoint{0.661006in}{0.524170in}}{\pgfqpoint{4.194036in}{1.071446in}}%
\pgfusepath{clip}%
\pgfsetbuttcap%
\pgfsetroundjoin%
\definecolor{currentfill}{rgb}{0.764932,0.863079,0.913813}%
\pgfsetfillcolor{currentfill}%
\pgfsetfillopacity{0.700000}%
\pgfsetlinewidth{1.003750pt}%
\definecolor{currentstroke}{rgb}{0.764932,0.863079,0.913813}%
\pgfsetstrokecolor{currentstroke}%
\pgfsetstrokeopacity{0.700000}%
\pgfsetdash{}{0pt}%
\pgfpathmoveto{\pgfqpoint{0.860986in}{1.534530in}}%
\pgfpathcurveto{\pgfqpoint{0.862827in}{1.534530in}}{\pgfqpoint{0.864594in}{1.535261in}}{\pgfqpoint{0.865896in}{1.536564in}}%
\pgfpathcurveto{\pgfqpoint{0.867198in}{1.537866in}}{\pgfqpoint{0.867930in}{1.539633in}}{\pgfqpoint{0.867930in}{1.541474in}}%
\pgfpathcurveto{\pgfqpoint{0.867930in}{1.543316in}}{\pgfqpoint{0.867198in}{1.545082in}}{\pgfqpoint{0.865896in}{1.546385in}}%
\pgfpathcurveto{\pgfqpoint{0.864594in}{1.547687in}}{\pgfqpoint{0.862827in}{1.548419in}}{\pgfqpoint{0.860986in}{1.548419in}}%
\pgfpathcurveto{\pgfqpoint{0.859144in}{1.548419in}}{\pgfqpoint{0.857377in}{1.547687in}}{\pgfqpoint{0.856075in}{1.546385in}}%
\pgfpathcurveto{\pgfqpoint{0.854773in}{1.545082in}}{\pgfqpoint{0.854041in}{1.543316in}}{\pgfqpoint{0.854041in}{1.541474in}}%
\pgfpathcurveto{\pgfqpoint{0.854041in}{1.539633in}}{\pgfqpoint{0.854773in}{1.537866in}}{\pgfqpoint{0.856075in}{1.536564in}}%
\pgfpathcurveto{\pgfqpoint{0.857377in}{1.535261in}}{\pgfqpoint{0.859144in}{1.534530in}}{\pgfqpoint{0.860986in}{1.534530in}}%
\pgfpathlineto{\pgfqpoint{0.860986in}{1.534530in}}%
\pgfpathclose%
\pgfusepath{stroke,fill}%
\end{pgfscope}%
\begin{pgfscope}%
\pgfpathrectangle{\pgfqpoint{0.661006in}{0.524170in}}{\pgfqpoint{4.194036in}{1.071446in}}%
\pgfusepath{clip}%
\pgfsetbuttcap%
\pgfsetroundjoin%
\definecolor{currentfill}{rgb}{0.759943,0.859339,0.911830}%
\pgfsetfillcolor{currentfill}%
\pgfsetfillopacity{0.700000}%
\pgfsetlinewidth{1.003750pt}%
\definecolor{currentstroke}{rgb}{0.759943,0.859339,0.911830}%
\pgfsetstrokecolor{currentstroke}%
\pgfsetstrokeopacity{0.700000}%
\pgfsetdash{}{0pt}%
\pgfpathmoveto{\pgfqpoint{0.868190in}{1.532294in}}%
\pgfpathcurveto{\pgfqpoint{0.870031in}{1.532294in}}{\pgfqpoint{0.871798in}{1.533025in}}{\pgfqpoint{0.873100in}{1.534328in}}%
\pgfpathcurveto{\pgfqpoint{0.874402in}{1.535630in}}{\pgfqpoint{0.875134in}{1.537397in}}{\pgfqpoint{0.875134in}{1.539238in}}%
\pgfpathcurveto{\pgfqpoint{0.875134in}{1.541080in}}{\pgfqpoint{0.874402in}{1.542846in}}{\pgfqpoint{0.873100in}{1.544149in}}%
\pgfpathcurveto{\pgfqpoint{0.871798in}{1.545451in}}{\pgfqpoint{0.870031in}{1.546183in}}{\pgfqpoint{0.868190in}{1.546183in}}%
\pgfpathcurveto{\pgfqpoint{0.866348in}{1.546183in}}{\pgfqpoint{0.864581in}{1.545451in}}{\pgfqpoint{0.863279in}{1.544149in}}%
\pgfpathcurveto{\pgfqpoint{0.861977in}{1.542846in}}{\pgfqpoint{0.861245in}{1.541080in}}{\pgfqpoint{0.861245in}{1.539238in}}%
\pgfpathcurveto{\pgfqpoint{0.861245in}{1.537397in}}{\pgfqpoint{0.861977in}{1.535630in}}{\pgfqpoint{0.863279in}{1.534328in}}%
\pgfpathcurveto{\pgfqpoint{0.864581in}{1.533025in}}{\pgfqpoint{0.866348in}{1.532294in}}{\pgfqpoint{0.868190in}{1.532294in}}%
\pgfpathlineto{\pgfqpoint{0.868190in}{1.532294in}}%
\pgfpathclose%
\pgfusepath{stroke,fill}%
\end{pgfscope}%
\begin{pgfscope}%
\pgfpathrectangle{\pgfqpoint{0.661006in}{0.524170in}}{\pgfqpoint{4.194036in}{1.071446in}}%
\pgfusepath{clip}%
\pgfsetbuttcap%
\pgfsetroundjoin%
\definecolor{currentfill}{rgb}{0.759943,0.859339,0.911830}%
\pgfsetfillcolor{currentfill}%
\pgfsetfillopacity{0.700000}%
\pgfsetlinewidth{1.003750pt}%
\definecolor{currentstroke}{rgb}{0.759943,0.859339,0.911830}%
\pgfsetstrokecolor{currentstroke}%
\pgfsetstrokeopacity{0.700000}%
\pgfsetdash{}{0pt}%
\pgfpathmoveto{\pgfqpoint{0.879019in}{1.529989in}}%
\pgfpathcurveto{\pgfqpoint{0.880860in}{1.529989in}}{\pgfqpoint{0.882627in}{1.530720in}}{\pgfqpoint{0.883929in}{1.532022in}}%
\pgfpathcurveto{\pgfqpoint{0.885232in}{1.533325in}}{\pgfqpoint{0.885963in}{1.535091in}}{\pgfqpoint{0.885963in}{1.536933in}}%
\pgfpathcurveto{\pgfqpoint{0.885963in}{1.538775in}}{\pgfqpoint{0.885232in}{1.540541in}}{\pgfqpoint{0.883929in}{1.541843in}}%
\pgfpathcurveto{\pgfqpoint{0.882627in}{1.543146in}}{\pgfqpoint{0.880860in}{1.543877in}}{\pgfqpoint{0.879019in}{1.543877in}}%
\pgfpathcurveto{\pgfqpoint{0.877177in}{1.543877in}}{\pgfqpoint{0.875411in}{1.543146in}}{\pgfqpoint{0.874108in}{1.541843in}}%
\pgfpathcurveto{\pgfqpoint{0.872806in}{1.540541in}}{\pgfqpoint{0.872074in}{1.538775in}}{\pgfqpoint{0.872074in}{1.536933in}}%
\pgfpathcurveto{\pgfqpoint{0.872074in}{1.535091in}}{\pgfqpoint{0.872806in}{1.533325in}}{\pgfqpoint{0.874108in}{1.532022in}}%
\pgfpathcurveto{\pgfqpoint{0.875411in}{1.530720in}}{\pgfqpoint{0.877177in}{1.529989in}}{\pgfqpoint{0.879019in}{1.529989in}}%
\pgfpathlineto{\pgfqpoint{0.879019in}{1.529989in}}%
\pgfpathclose%
\pgfusepath{stroke,fill}%
\end{pgfscope}%
\begin{pgfscope}%
\pgfpathrectangle{\pgfqpoint{0.661006in}{0.524170in}}{\pgfqpoint{4.194036in}{1.071446in}}%
\pgfusepath{clip}%
\pgfsetbuttcap%
\pgfsetroundjoin%
\definecolor{currentfill}{rgb}{0.754989,0.855581,0.909845}%
\pgfsetfillcolor{currentfill}%
\pgfsetfillopacity{0.700000}%
\pgfsetlinewidth{1.003750pt}%
\definecolor{currentstroke}{rgb}{0.754989,0.855581,0.909845}%
\pgfsetstrokecolor{currentstroke}%
\pgfsetstrokeopacity{0.700000}%
\pgfsetdash{}{0pt}%
\pgfpathmoveto{\pgfqpoint{0.895721in}{1.527013in}}%
\pgfpathcurveto{\pgfqpoint{0.897562in}{1.527013in}}{\pgfqpoint{0.899329in}{1.527745in}}{\pgfqpoint{0.900631in}{1.529047in}}%
\pgfpathcurveto{\pgfqpoint{0.901933in}{1.530349in}}{\pgfqpoint{0.902665in}{1.532116in}}{\pgfqpoint{0.902665in}{1.533958in}}%
\pgfpathcurveto{\pgfqpoint{0.902665in}{1.535799in}}{\pgfqpoint{0.901933in}{1.537566in}}{\pgfqpoint{0.900631in}{1.538868in}}%
\pgfpathcurveto{\pgfqpoint{0.899329in}{1.540170in}}{\pgfqpoint{0.897562in}{1.540902in}}{\pgfqpoint{0.895721in}{1.540902in}}%
\pgfpathcurveto{\pgfqpoint{0.893879in}{1.540902in}}{\pgfqpoint{0.892113in}{1.540170in}}{\pgfqpoint{0.890810in}{1.538868in}}%
\pgfpathcurveto{\pgfqpoint{0.889508in}{1.537566in}}{\pgfqpoint{0.888776in}{1.535799in}}{\pgfqpoint{0.888776in}{1.533958in}}%
\pgfpathcurveto{\pgfqpoint{0.888776in}{1.532116in}}{\pgfqpoint{0.889508in}{1.530349in}}{\pgfqpoint{0.890810in}{1.529047in}}%
\pgfpathcurveto{\pgfqpoint{0.892113in}{1.527745in}}{\pgfqpoint{0.893879in}{1.527013in}}{\pgfqpoint{0.895721in}{1.527013in}}%
\pgfpathlineto{\pgfqpoint{0.895721in}{1.527013in}}%
\pgfpathclose%
\pgfusepath{stroke,fill}%
\end{pgfscope}%
\begin{pgfscope}%
\pgfpathrectangle{\pgfqpoint{0.661006in}{0.524170in}}{\pgfqpoint{4.194036in}{1.071446in}}%
\pgfusepath{clip}%
\pgfsetbuttcap%
\pgfsetroundjoin%
\definecolor{currentfill}{rgb}{0.754989,0.855581,0.909845}%
\pgfsetfillcolor{currentfill}%
\pgfsetfillopacity{0.700000}%
\pgfsetlinewidth{1.003750pt}%
\definecolor{currentstroke}{rgb}{0.754989,0.855581,0.909845}%
\pgfsetstrokecolor{currentstroke}%
\pgfsetstrokeopacity{0.700000}%
\pgfsetdash{}{0pt}%
\pgfpathmoveto{\pgfqpoint{0.901700in}{1.524086in}}%
\pgfpathcurveto{\pgfqpoint{0.903541in}{1.524086in}}{\pgfqpoint{0.905308in}{1.524818in}}{\pgfqpoint{0.906610in}{1.526120in}}%
\pgfpathcurveto{\pgfqpoint{0.907912in}{1.527422in}}{\pgfqpoint{0.908644in}{1.529189in}}{\pgfqpoint{0.908644in}{1.531030in}}%
\pgfpathcurveto{\pgfqpoint{0.908644in}{1.532872in}}{\pgfqpoint{0.907912in}{1.534639in}}{\pgfqpoint{0.906610in}{1.535941in}}%
\pgfpathcurveto{\pgfqpoint{0.905308in}{1.537243in}}{\pgfqpoint{0.903541in}{1.537975in}}{\pgfqpoint{0.901700in}{1.537975in}}%
\pgfpathcurveto{\pgfqpoint{0.899858in}{1.537975in}}{\pgfqpoint{0.898091in}{1.537243in}}{\pgfqpoint{0.896789in}{1.535941in}}%
\pgfpathcurveto{\pgfqpoint{0.895487in}{1.534639in}}{\pgfqpoint{0.894755in}{1.532872in}}{\pgfqpoint{0.894755in}{1.531030in}}%
\pgfpathcurveto{\pgfqpoint{0.894755in}{1.529189in}}{\pgfqpoint{0.895487in}{1.527422in}}{\pgfqpoint{0.896789in}{1.526120in}}%
\pgfpathcurveto{\pgfqpoint{0.898091in}{1.524818in}}{\pgfqpoint{0.899858in}{1.524086in}}{\pgfqpoint{0.901700in}{1.524086in}}%
\pgfpathlineto{\pgfqpoint{0.901700in}{1.524086in}}%
\pgfpathclose%
\pgfusepath{stroke,fill}%
\end{pgfscope}%
\begin{pgfscope}%
\pgfpathrectangle{\pgfqpoint{0.661006in}{0.524170in}}{\pgfqpoint{4.194036in}{1.071446in}}%
\pgfusepath{clip}%
\pgfsetbuttcap%
\pgfsetroundjoin%
\definecolor{currentfill}{rgb}{0.750069,0.851807,0.907857}%
\pgfsetfillcolor{currentfill}%
\pgfsetfillopacity{0.700000}%
\pgfsetlinewidth{1.003750pt}%
\definecolor{currentstroke}{rgb}{0.750069,0.851807,0.907857}%
\pgfsetstrokecolor{currentstroke}%
\pgfsetstrokeopacity{0.700000}%
\pgfsetdash{}{0pt}%
\pgfpathmoveto{\pgfqpoint{0.913087in}{1.521284in}}%
\pgfpathcurveto{\pgfqpoint{0.914928in}{1.521284in}}{\pgfqpoint{0.916695in}{1.522016in}}{\pgfqpoint{0.917997in}{1.523318in}}%
\pgfpathcurveto{\pgfqpoint{0.919299in}{1.524620in}}{\pgfqpoint{0.920031in}{1.526387in}}{\pgfqpoint{0.920031in}{1.528229in}}%
\pgfpathcurveto{\pgfqpoint{0.920031in}{1.530070in}}{\pgfqpoint{0.919299in}{1.531837in}}{\pgfqpoint{0.917997in}{1.533139in}}%
\pgfpathcurveto{\pgfqpoint{0.916695in}{1.534441in}}{\pgfqpoint{0.914928in}{1.535173in}}{\pgfqpoint{0.913087in}{1.535173in}}%
\pgfpathcurveto{\pgfqpoint{0.911245in}{1.535173in}}{\pgfqpoint{0.909478in}{1.534441in}}{\pgfqpoint{0.908176in}{1.533139in}}%
\pgfpathcurveto{\pgfqpoint{0.906874in}{1.531837in}}{\pgfqpoint{0.906142in}{1.530070in}}{\pgfqpoint{0.906142in}{1.528229in}}%
\pgfpathcurveto{\pgfqpoint{0.906142in}{1.526387in}}{\pgfqpoint{0.906874in}{1.524620in}}{\pgfqpoint{0.908176in}{1.523318in}}%
\pgfpathcurveto{\pgfqpoint{0.909478in}{1.522016in}}{\pgfqpoint{0.911245in}{1.521284in}}{\pgfqpoint{0.913087in}{1.521284in}}%
\pgfpathlineto{\pgfqpoint{0.913087in}{1.521284in}}%
\pgfpathclose%
\pgfusepath{stroke,fill}%
\end{pgfscope}%
\begin{pgfscope}%
\pgfpathrectangle{\pgfqpoint{0.661006in}{0.524170in}}{\pgfqpoint{4.194036in}{1.071446in}}%
\pgfusepath{clip}%
\pgfsetbuttcap%
\pgfsetroundjoin%
\definecolor{currentfill}{rgb}{0.750069,0.851807,0.907857}%
\pgfsetfillcolor{currentfill}%
\pgfsetfillopacity{0.700000}%
\pgfsetlinewidth{1.003750pt}%
\definecolor{currentstroke}{rgb}{0.750069,0.851807,0.907857}%
\pgfsetstrokecolor{currentstroke}%
\pgfsetstrokeopacity{0.700000}%
\pgfsetdash{}{0pt}%
\pgfpathmoveto{\pgfqpoint{0.932282in}{1.516508in}}%
\pgfpathcurveto{\pgfqpoint{0.934123in}{1.516508in}}{\pgfqpoint{0.935890in}{1.517240in}}{\pgfqpoint{0.937192in}{1.518542in}}%
\pgfpathcurveto{\pgfqpoint{0.938494in}{1.519844in}}{\pgfqpoint{0.939226in}{1.521611in}}{\pgfqpoint{0.939226in}{1.523453in}}%
\pgfpathcurveto{\pgfqpoint{0.939226in}{1.525294in}}{\pgfqpoint{0.938494in}{1.527061in}}{\pgfqpoint{0.937192in}{1.528363in}}%
\pgfpathcurveto{\pgfqpoint{0.935890in}{1.529665in}}{\pgfqpoint{0.934123in}{1.530397in}}{\pgfqpoint{0.932282in}{1.530397in}}%
\pgfpathcurveto{\pgfqpoint{0.930440in}{1.530397in}}{\pgfqpoint{0.928673in}{1.529665in}}{\pgfqpoint{0.927371in}{1.528363in}}%
\pgfpathcurveto{\pgfqpoint{0.926069in}{1.527061in}}{\pgfqpoint{0.925337in}{1.525294in}}{\pgfqpoint{0.925337in}{1.523453in}}%
\pgfpathcurveto{\pgfqpoint{0.925337in}{1.521611in}}{\pgfqpoint{0.926069in}{1.519844in}}{\pgfqpoint{0.927371in}{1.518542in}}%
\pgfpathcurveto{\pgfqpoint{0.928673in}{1.517240in}}{\pgfqpoint{0.930440in}{1.516508in}}{\pgfqpoint{0.932282in}{1.516508in}}%
\pgfpathlineto{\pgfqpoint{0.932282in}{1.516508in}}%
\pgfpathclose%
\pgfusepath{stroke,fill}%
\end{pgfscope}%
\begin{pgfscope}%
\pgfpathrectangle{\pgfqpoint{0.661006in}{0.524170in}}{\pgfqpoint{4.194036in}{1.071446in}}%
\pgfusepath{clip}%
\pgfsetbuttcap%
\pgfsetroundjoin%
\definecolor{currentfill}{rgb}{0.750069,0.851807,0.907857}%
\pgfsetfillcolor{currentfill}%
\pgfsetfillopacity{0.700000}%
\pgfsetlinewidth{1.003750pt}%
\definecolor{currentstroke}{rgb}{0.750069,0.851807,0.907857}%
\pgfsetstrokecolor{currentstroke}%
\pgfsetstrokeopacity{0.700000}%
\pgfsetdash{}{0pt}%
\pgfpathmoveto{\pgfqpoint{0.954265in}{1.511458in}}%
\pgfpathcurveto{\pgfqpoint{0.956107in}{1.511458in}}{\pgfqpoint{0.957874in}{1.512189in}}{\pgfqpoint{0.959176in}{1.513492in}}%
\pgfpathcurveto{\pgfqpoint{0.960478in}{1.514794in}}{\pgfqpoint{0.961210in}{1.516560in}}{\pgfqpoint{0.961210in}{1.518402in}}%
\pgfpathcurveto{\pgfqpoint{0.961210in}{1.520244in}}{\pgfqpoint{0.960478in}{1.522010in}}{\pgfqpoint{0.959176in}{1.523313in}}%
\pgfpathcurveto{\pgfqpoint{0.957874in}{1.524615in}}{\pgfqpoint{0.956107in}{1.525347in}}{\pgfqpoint{0.954265in}{1.525347in}}%
\pgfpathcurveto{\pgfqpoint{0.952424in}{1.525347in}}{\pgfqpoint{0.950657in}{1.524615in}}{\pgfqpoint{0.949355in}{1.523313in}}%
\pgfpathcurveto{\pgfqpoint{0.948053in}{1.522010in}}{\pgfqpoint{0.947321in}{1.520244in}}{\pgfqpoint{0.947321in}{1.518402in}}%
\pgfpathcurveto{\pgfqpoint{0.947321in}{1.516560in}}{\pgfqpoint{0.948053in}{1.514794in}}{\pgfqpoint{0.949355in}{1.513492in}}%
\pgfpathcurveto{\pgfqpoint{0.950657in}{1.512189in}}{\pgfqpoint{0.952424in}{1.511458in}}{\pgfqpoint{0.954265in}{1.511458in}}%
\pgfpathlineto{\pgfqpoint{0.954265in}{1.511458in}}%
\pgfpathclose%
\pgfusepath{stroke,fill}%
\end{pgfscope}%
\begin{pgfscope}%
\pgfpathrectangle{\pgfqpoint{0.661006in}{0.524170in}}{\pgfqpoint{4.194036in}{1.071446in}}%
\pgfusepath{clip}%
\pgfsetbuttcap%
\pgfsetroundjoin%
\definecolor{currentfill}{rgb}{0.750069,0.851807,0.907857}%
\pgfsetfillcolor{currentfill}%
\pgfsetfillopacity{0.700000}%
\pgfsetlinewidth{1.003750pt}%
\definecolor{currentstroke}{rgb}{0.750069,0.851807,0.907857}%
\pgfsetstrokecolor{currentstroke}%
\pgfsetstrokeopacity{0.700000}%
\pgfsetdash{}{0pt}%
\pgfpathmoveto{\pgfqpoint{0.971090in}{1.508412in}}%
\pgfpathcurveto{\pgfqpoint{0.972932in}{1.508412in}}{\pgfqpoint{0.974698in}{1.509144in}}{\pgfqpoint{0.976001in}{1.510446in}}%
\pgfpathcurveto{\pgfqpoint{0.977303in}{1.511749in}}{\pgfqpoint{0.978035in}{1.513515in}}{\pgfqpoint{0.978035in}{1.515357in}}%
\pgfpathcurveto{\pgfqpoint{0.978035in}{1.517199in}}{\pgfqpoint{0.977303in}{1.518965in}}{\pgfqpoint{0.976001in}{1.520267in}}%
\pgfpathcurveto{\pgfqpoint{0.974698in}{1.521570in}}{\pgfqpoint{0.972932in}{1.522301in}}{\pgfqpoint{0.971090in}{1.522301in}}%
\pgfpathcurveto{\pgfqpoint{0.969248in}{1.522301in}}{\pgfqpoint{0.967482in}{1.521570in}}{\pgfqpoint{0.966180in}{1.520267in}}%
\pgfpathcurveto{\pgfqpoint{0.964877in}{1.518965in}}{\pgfqpoint{0.964146in}{1.517199in}}{\pgfqpoint{0.964146in}{1.515357in}}%
\pgfpathcurveto{\pgfqpoint{0.964146in}{1.513515in}}{\pgfqpoint{0.964877in}{1.511749in}}{\pgfqpoint{0.966180in}{1.510446in}}%
\pgfpathcurveto{\pgfqpoint{0.967482in}{1.509144in}}{\pgfqpoint{0.969248in}{1.508412in}}{\pgfqpoint{0.971090in}{1.508412in}}%
\pgfpathlineto{\pgfqpoint{0.971090in}{1.508412in}}%
\pgfpathclose%
\pgfusepath{stroke,fill}%
\end{pgfscope}%
\begin{pgfscope}%
\pgfpathrectangle{\pgfqpoint{0.661006in}{0.524170in}}{\pgfqpoint{4.194036in}{1.071446in}}%
\pgfusepath{clip}%
\pgfsetbuttcap%
\pgfsetroundjoin%
\definecolor{currentfill}{rgb}{0.750069,0.851807,0.907857}%
\pgfsetfillcolor{currentfill}%
\pgfsetfillopacity{0.700000}%
\pgfsetlinewidth{1.003750pt}%
\definecolor{currentstroke}{rgb}{0.750069,0.851807,0.907857}%
\pgfsetstrokecolor{currentstroke}%
\pgfsetstrokeopacity{0.700000}%
\pgfsetdash{}{0pt}%
\pgfpathmoveto{\pgfqpoint{0.978201in}{1.505839in}}%
\pgfpathcurveto{\pgfqpoint{0.980043in}{1.505839in}}{\pgfqpoint{0.981809in}{1.506571in}}{\pgfqpoint{0.983112in}{1.507873in}}%
\pgfpathcurveto{\pgfqpoint{0.984414in}{1.509175in}}{\pgfqpoint{0.985146in}{1.510942in}}{\pgfqpoint{0.985146in}{1.512784in}}%
\pgfpathcurveto{\pgfqpoint{0.985146in}{1.514625in}}{\pgfqpoint{0.984414in}{1.516392in}}{\pgfqpoint{0.983112in}{1.517694in}}%
\pgfpathcurveto{\pgfqpoint{0.981809in}{1.518996in}}{\pgfqpoint{0.980043in}{1.519728in}}{\pgfqpoint{0.978201in}{1.519728in}}%
\pgfpathcurveto{\pgfqpoint{0.976360in}{1.519728in}}{\pgfqpoint{0.974593in}{1.518996in}}{\pgfqpoint{0.973291in}{1.517694in}}%
\pgfpathcurveto{\pgfqpoint{0.971988in}{1.516392in}}{\pgfqpoint{0.971257in}{1.514625in}}{\pgfqpoint{0.971257in}{1.512784in}}%
\pgfpathcurveto{\pgfqpoint{0.971257in}{1.510942in}}{\pgfqpoint{0.971988in}{1.509175in}}{\pgfqpoint{0.973291in}{1.507873in}}%
\pgfpathcurveto{\pgfqpoint{0.974593in}{1.506571in}}{\pgfqpoint{0.976360in}{1.505839in}}{\pgfqpoint{0.978201in}{1.505839in}}%
\pgfpathlineto{\pgfqpoint{0.978201in}{1.505839in}}%
\pgfpathclose%
\pgfusepath{stroke,fill}%
\end{pgfscope}%
\begin{pgfscope}%
\pgfpathrectangle{\pgfqpoint{0.661006in}{0.524170in}}{\pgfqpoint{4.194036in}{1.071446in}}%
\pgfusepath{clip}%
\pgfsetbuttcap%
\pgfsetroundjoin%
\definecolor{currentfill}{rgb}{0.745184,0.848015,0.905867}%
\pgfsetfillcolor{currentfill}%
\pgfsetfillopacity{0.700000}%
\pgfsetlinewidth{1.003750pt}%
\definecolor{currentstroke}{rgb}{0.745184,0.848015,0.905867}%
\pgfsetstrokecolor{currentstroke}%
\pgfsetstrokeopacity{0.700000}%
\pgfsetdash{}{0pt}%
\pgfpathmoveto{\pgfqpoint{0.979967in}{1.505227in}}%
\pgfpathcurveto{\pgfqpoint{0.981809in}{1.505227in}}{\pgfqpoint{0.983576in}{1.505958in}}{\pgfqpoint{0.984878in}{1.507261in}}%
\pgfpathcurveto{\pgfqpoint{0.986180in}{1.508563in}}{\pgfqpoint{0.986912in}{1.510329in}}{\pgfqpoint{0.986912in}{1.512171in}}%
\pgfpathcurveto{\pgfqpoint{0.986912in}{1.514013in}}{\pgfqpoint{0.986180in}{1.515779in}}{\pgfqpoint{0.984878in}{1.517081in}}%
\pgfpathcurveto{\pgfqpoint{0.983576in}{1.518384in}}{\pgfqpoint{0.981809in}{1.519115in}}{\pgfqpoint{0.979967in}{1.519115in}}%
\pgfpathcurveto{\pgfqpoint{0.978126in}{1.519115in}}{\pgfqpoint{0.976359in}{1.518384in}}{\pgfqpoint{0.975057in}{1.517081in}}%
\pgfpathcurveto{\pgfqpoint{0.973755in}{1.515779in}}{\pgfqpoint{0.973023in}{1.514013in}}{\pgfqpoint{0.973023in}{1.512171in}}%
\pgfpathcurveto{\pgfqpoint{0.973023in}{1.510329in}}{\pgfqpoint{0.973755in}{1.508563in}}{\pgfqpoint{0.975057in}{1.507261in}}%
\pgfpathcurveto{\pgfqpoint{0.976359in}{1.505958in}}{\pgfqpoint{0.978126in}{1.505227in}}{\pgfqpoint{0.979967in}{1.505227in}}%
\pgfpathlineto{\pgfqpoint{0.979967in}{1.505227in}}%
\pgfpathclose%
\pgfusepath{stroke,fill}%
\end{pgfscope}%
\begin{pgfscope}%
\pgfpathrectangle{\pgfqpoint{0.661006in}{0.524170in}}{\pgfqpoint{4.194036in}{1.071446in}}%
\pgfusepath{clip}%
\pgfsetbuttcap%
\pgfsetroundjoin%
\definecolor{currentfill}{rgb}{0.745184,0.848015,0.905867}%
\pgfsetfillcolor{currentfill}%
\pgfsetfillopacity{0.700000}%
\pgfsetlinewidth{1.003750pt}%
\definecolor{currentstroke}{rgb}{0.745184,0.848015,0.905867}%
\pgfsetstrokecolor{currentstroke}%
\pgfsetstrokeopacity{0.700000}%
\pgfsetdash{}{0pt}%
\pgfpathmoveto{\pgfqpoint{0.975180in}{1.506050in}}%
\pgfpathcurveto{\pgfqpoint{0.977022in}{1.506050in}}{\pgfqpoint{0.978788in}{1.506782in}}{\pgfqpoint{0.980091in}{1.508084in}}%
\pgfpathcurveto{\pgfqpoint{0.981393in}{1.509386in}}{\pgfqpoint{0.982125in}{1.511153in}}{\pgfqpoint{0.982125in}{1.512995in}}%
\pgfpathcurveto{\pgfqpoint{0.982125in}{1.514836in}}{\pgfqpoint{0.981393in}{1.516603in}}{\pgfqpoint{0.980091in}{1.517905in}}%
\pgfpathcurveto{\pgfqpoint{0.978788in}{1.519207in}}{\pgfqpoint{0.977022in}{1.519939in}}{\pgfqpoint{0.975180in}{1.519939in}}%
\pgfpathcurveto{\pgfqpoint{0.973338in}{1.519939in}}{\pgfqpoint{0.971572in}{1.519207in}}{\pgfqpoint{0.970270in}{1.517905in}}%
\pgfpathcurveto{\pgfqpoint{0.968967in}{1.516603in}}{\pgfqpoint{0.968236in}{1.514836in}}{\pgfqpoint{0.968236in}{1.512995in}}%
\pgfpathcurveto{\pgfqpoint{0.968236in}{1.511153in}}{\pgfqpoint{0.968967in}{1.509386in}}{\pgfqpoint{0.970270in}{1.508084in}}%
\pgfpathcurveto{\pgfqpoint{0.971572in}{1.506782in}}{\pgfqpoint{0.973338in}{1.506050in}}{\pgfqpoint{0.975180in}{1.506050in}}%
\pgfpathlineto{\pgfqpoint{0.975180in}{1.506050in}}%
\pgfpathclose%
\pgfusepath{stroke,fill}%
\end{pgfscope}%
\begin{pgfscope}%
\pgfpathrectangle{\pgfqpoint{0.661006in}{0.524170in}}{\pgfqpoint{4.194036in}{1.071446in}}%
\pgfusepath{clip}%
\pgfsetbuttcap%
\pgfsetroundjoin%
\definecolor{currentfill}{rgb}{0.740332,0.844207,0.903873}%
\pgfsetfillcolor{currentfill}%
\pgfsetfillopacity{0.700000}%
\pgfsetlinewidth{1.003750pt}%
\definecolor{currentstroke}{rgb}{0.740332,0.844207,0.903873}%
\pgfsetstrokecolor{currentstroke}%
\pgfsetstrokeopacity{0.700000}%
\pgfsetdash{}{0pt}%
\pgfpathmoveto{\pgfqpoint{0.968441in}{1.507662in}}%
\pgfpathcurveto{\pgfqpoint{0.970283in}{1.507662in}}{\pgfqpoint{0.972049in}{1.508394in}}{\pgfqpoint{0.973351in}{1.509696in}}%
\pgfpathcurveto{\pgfqpoint{0.974654in}{1.510998in}}{\pgfqpoint{0.975385in}{1.512765in}}{\pgfqpoint{0.975385in}{1.514606in}}%
\pgfpathcurveto{\pgfqpoint{0.975385in}{1.516448in}}{\pgfqpoint{0.974654in}{1.518214in}}{\pgfqpoint{0.973351in}{1.519517in}}%
\pgfpathcurveto{\pgfqpoint{0.972049in}{1.520819in}}{\pgfqpoint{0.970283in}{1.521551in}}{\pgfqpoint{0.968441in}{1.521551in}}%
\pgfpathcurveto{\pgfqpoint{0.966599in}{1.521551in}}{\pgfqpoint{0.964833in}{1.520819in}}{\pgfqpoint{0.963531in}{1.519517in}}%
\pgfpathcurveto{\pgfqpoint{0.962228in}{1.518214in}}{\pgfqpoint{0.961497in}{1.516448in}}{\pgfqpoint{0.961497in}{1.514606in}}%
\pgfpathcurveto{\pgfqpoint{0.961497in}{1.512765in}}{\pgfqpoint{0.962228in}{1.510998in}}{\pgfqpoint{0.963531in}{1.509696in}}%
\pgfpathcurveto{\pgfqpoint{0.964833in}{1.508394in}}{\pgfqpoint{0.966599in}{1.507662in}}{\pgfqpoint{0.968441in}{1.507662in}}%
\pgfpathlineto{\pgfqpoint{0.968441in}{1.507662in}}%
\pgfpathclose%
\pgfusepath{stroke,fill}%
\end{pgfscope}%
\begin{pgfscope}%
\pgfpathrectangle{\pgfqpoint{0.661006in}{0.524170in}}{\pgfqpoint{4.194036in}{1.071446in}}%
\pgfusepath{clip}%
\pgfsetbuttcap%
\pgfsetroundjoin%
\definecolor{currentfill}{rgb}{0.740332,0.844207,0.903873}%
\pgfsetfillcolor{currentfill}%
\pgfsetfillopacity{0.700000}%
\pgfsetlinewidth{1.003750pt}%
\definecolor{currentstroke}{rgb}{0.740332,0.844207,0.903873}%
\pgfsetstrokecolor{currentstroke}%
\pgfsetstrokeopacity{0.700000}%
\pgfsetdash{}{0pt}%
\pgfpathmoveto{\pgfqpoint{0.970068in}{1.506551in}}%
\pgfpathcurveto{\pgfqpoint{0.971909in}{1.506551in}}{\pgfqpoint{0.973676in}{1.507283in}}{\pgfqpoint{0.974978in}{1.508585in}}%
\pgfpathcurveto{\pgfqpoint{0.976280in}{1.509887in}}{\pgfqpoint{0.977012in}{1.511654in}}{\pgfqpoint{0.977012in}{1.513495in}}%
\pgfpathcurveto{\pgfqpoint{0.977012in}{1.515337in}}{\pgfqpoint{0.976280in}{1.517103in}}{\pgfqpoint{0.974978in}{1.518406in}}%
\pgfpathcurveto{\pgfqpoint{0.973676in}{1.519708in}}{\pgfqpoint{0.971909in}{1.520440in}}{\pgfqpoint{0.970068in}{1.520440in}}%
\pgfpathcurveto{\pgfqpoint{0.968226in}{1.520440in}}{\pgfqpoint{0.966459in}{1.519708in}}{\pgfqpoint{0.965157in}{1.518406in}}%
\pgfpathcurveto{\pgfqpoint{0.963855in}{1.517103in}}{\pgfqpoint{0.963123in}{1.515337in}}{\pgfqpoint{0.963123in}{1.513495in}}%
\pgfpathcurveto{\pgfqpoint{0.963123in}{1.511654in}}{\pgfqpoint{0.963855in}{1.509887in}}{\pgfqpoint{0.965157in}{1.508585in}}%
\pgfpathcurveto{\pgfqpoint{0.966459in}{1.507283in}}{\pgfqpoint{0.968226in}{1.506551in}}{\pgfqpoint{0.970068in}{1.506551in}}%
\pgfpathlineto{\pgfqpoint{0.970068in}{1.506551in}}%
\pgfpathclose%
\pgfusepath{stroke,fill}%
\end{pgfscope}%
\begin{pgfscope}%
\pgfpathrectangle{\pgfqpoint{0.661006in}{0.524170in}}{\pgfqpoint{4.194036in}{1.071446in}}%
\pgfusepath{clip}%
\pgfsetbuttcap%
\pgfsetroundjoin%
\definecolor{currentfill}{rgb}{0.740332,0.844207,0.903873}%
\pgfsetfillcolor{currentfill}%
\pgfsetfillopacity{0.700000}%
\pgfsetlinewidth{1.003750pt}%
\definecolor{currentstroke}{rgb}{0.740332,0.844207,0.903873}%
\pgfsetstrokecolor{currentstroke}%
\pgfsetstrokeopacity{0.700000}%
\pgfsetdash{}{0pt}%
\pgfpathmoveto{\pgfqpoint{0.981216in}{1.503759in}}%
\pgfpathcurveto{\pgfqpoint{0.983057in}{1.503759in}}{\pgfqpoint{0.984824in}{1.504491in}}{\pgfqpoint{0.986126in}{1.505793in}}%
\pgfpathcurveto{\pgfqpoint{0.987428in}{1.507096in}}{\pgfqpoint{0.988160in}{1.508862in}}{\pgfqpoint{0.988160in}{1.510704in}}%
\pgfpathcurveto{\pgfqpoint{0.988160in}{1.512546in}}{\pgfqpoint{0.987428in}{1.514312in}}{\pgfqpoint{0.986126in}{1.515614in}}%
\pgfpathcurveto{\pgfqpoint{0.984824in}{1.516917in}}{\pgfqpoint{0.983057in}{1.517648in}}{\pgfqpoint{0.981216in}{1.517648in}}%
\pgfpathcurveto{\pgfqpoint{0.979374in}{1.517648in}}{\pgfqpoint{0.977607in}{1.516917in}}{\pgfqpoint{0.976305in}{1.515614in}}%
\pgfpathcurveto{\pgfqpoint{0.975003in}{1.514312in}}{\pgfqpoint{0.974271in}{1.512546in}}{\pgfqpoint{0.974271in}{1.510704in}}%
\pgfpathcurveto{\pgfqpoint{0.974271in}{1.508862in}}{\pgfqpoint{0.975003in}{1.507096in}}{\pgfqpoint{0.976305in}{1.505793in}}%
\pgfpathcurveto{\pgfqpoint{0.977607in}{1.504491in}}{\pgfqpoint{0.979374in}{1.503759in}}{\pgfqpoint{0.981216in}{1.503759in}}%
\pgfpathlineto{\pgfqpoint{0.981216in}{1.503759in}}%
\pgfpathclose%
\pgfusepath{stroke,fill}%
\end{pgfscope}%
\begin{pgfscope}%
\pgfpathrectangle{\pgfqpoint{0.661006in}{0.524170in}}{\pgfqpoint{4.194036in}{1.071446in}}%
\pgfusepath{clip}%
\pgfsetbuttcap%
\pgfsetroundjoin%
\definecolor{currentfill}{rgb}{0.740332,0.844207,0.903873}%
\pgfsetfillcolor{currentfill}%
\pgfsetfillopacity{0.700000}%
\pgfsetlinewidth{1.003750pt}%
\definecolor{currentstroke}{rgb}{0.740332,0.844207,0.903873}%
\pgfsetstrokecolor{currentstroke}%
\pgfsetstrokeopacity{0.700000}%
\pgfsetdash{}{0pt}%
\pgfpathmoveto{\pgfqpoint{0.970951in}{1.505055in}}%
\pgfpathcurveto{\pgfqpoint{0.972792in}{1.505055in}}{\pgfqpoint{0.974559in}{1.505787in}}{\pgfqpoint{0.975861in}{1.507089in}}%
\pgfpathcurveto{\pgfqpoint{0.977163in}{1.508391in}}{\pgfqpoint{0.977895in}{1.510158in}}{\pgfqpoint{0.977895in}{1.512000in}}%
\pgfpathcurveto{\pgfqpoint{0.977895in}{1.513841in}}{\pgfqpoint{0.977163in}{1.515608in}}{\pgfqpoint{0.975861in}{1.516910in}}%
\pgfpathcurveto{\pgfqpoint{0.974559in}{1.518212in}}{\pgfqpoint{0.972792in}{1.518944in}}{\pgfqpoint{0.970951in}{1.518944in}}%
\pgfpathcurveto{\pgfqpoint{0.969109in}{1.518944in}}{\pgfqpoint{0.967343in}{1.518212in}}{\pgfqpoint{0.966040in}{1.516910in}}%
\pgfpathcurveto{\pgfqpoint{0.964738in}{1.515608in}}{\pgfqpoint{0.964006in}{1.513841in}}{\pgfqpoint{0.964006in}{1.512000in}}%
\pgfpathcurveto{\pgfqpoint{0.964006in}{1.510158in}}{\pgfqpoint{0.964738in}{1.508391in}}{\pgfqpoint{0.966040in}{1.507089in}}%
\pgfpathcurveto{\pgfqpoint{0.967343in}{1.505787in}}{\pgfqpoint{0.969109in}{1.505055in}}{\pgfqpoint{0.970951in}{1.505055in}}%
\pgfpathlineto{\pgfqpoint{0.970951in}{1.505055in}}%
\pgfpathclose%
\pgfusepath{stroke,fill}%
\end{pgfscope}%
\begin{pgfscope}%
\pgfpathrectangle{\pgfqpoint{0.661006in}{0.524170in}}{\pgfqpoint{4.194036in}{1.071446in}}%
\pgfusepath{clip}%
\pgfsetbuttcap%
\pgfsetroundjoin%
\definecolor{currentfill}{rgb}{0.735515,0.840382,0.901875}%
\pgfsetfillcolor{currentfill}%
\pgfsetfillopacity{0.700000}%
\pgfsetlinewidth{1.003750pt}%
\definecolor{currentstroke}{rgb}{0.735515,0.840382,0.901875}%
\pgfsetstrokecolor{currentstroke}%
\pgfsetstrokeopacity{0.700000}%
\pgfsetdash{}{0pt}%
\pgfpathmoveto{\pgfqpoint{0.968859in}{1.505748in}}%
\pgfpathcurveto{\pgfqpoint{0.970701in}{1.505748in}}{\pgfqpoint{0.972467in}{1.506480in}}{\pgfqpoint{0.973770in}{1.507782in}}%
\pgfpathcurveto{\pgfqpoint{0.975072in}{1.509085in}}{\pgfqpoint{0.975804in}{1.510851in}}{\pgfqpoint{0.975804in}{1.512693in}}%
\pgfpathcurveto{\pgfqpoint{0.975804in}{1.514535in}}{\pgfqpoint{0.975072in}{1.516301in}}{\pgfqpoint{0.973770in}{1.517603in}}%
\pgfpathcurveto{\pgfqpoint{0.972467in}{1.518906in}}{\pgfqpoint{0.970701in}{1.519637in}}{\pgfqpoint{0.968859in}{1.519637in}}%
\pgfpathcurveto{\pgfqpoint{0.967018in}{1.519637in}}{\pgfqpoint{0.965251in}{1.518906in}}{\pgfqpoint{0.963949in}{1.517603in}}%
\pgfpathcurveto{\pgfqpoint{0.962647in}{1.516301in}}{\pgfqpoint{0.961915in}{1.514535in}}{\pgfqpoint{0.961915in}{1.512693in}}%
\pgfpathcurveto{\pgfqpoint{0.961915in}{1.510851in}}{\pgfqpoint{0.962647in}{1.509085in}}{\pgfqpoint{0.963949in}{1.507782in}}%
\pgfpathcurveto{\pgfqpoint{0.965251in}{1.506480in}}{\pgfqpoint{0.967018in}{1.505748in}}{\pgfqpoint{0.968859in}{1.505748in}}%
\pgfpathlineto{\pgfqpoint{0.968859in}{1.505748in}}%
\pgfpathclose%
\pgfusepath{stroke,fill}%
\end{pgfscope}%
\begin{pgfscope}%
\pgfpathrectangle{\pgfqpoint{0.661006in}{0.524170in}}{\pgfqpoint{4.194036in}{1.071446in}}%
\pgfusepath{clip}%
\pgfsetbuttcap%
\pgfsetroundjoin%
\definecolor{currentfill}{rgb}{0.735515,0.840382,0.901875}%
\pgfsetfillcolor{currentfill}%
\pgfsetfillopacity{0.700000}%
\pgfsetlinewidth{1.003750pt}%
\definecolor{currentstroke}{rgb}{0.735515,0.840382,0.901875}%
\pgfsetstrokecolor{currentstroke}%
\pgfsetstrokeopacity{0.700000}%
\pgfsetdash{}{0pt}%
\pgfpathmoveto{\pgfqpoint{0.977365in}{1.503811in}}%
\pgfpathcurveto{\pgfqpoint{0.979206in}{1.503811in}}{\pgfqpoint{0.980973in}{1.504542in}}{\pgfqpoint{0.982275in}{1.505845in}}%
\pgfpathcurveto{\pgfqpoint{0.983577in}{1.507147in}}{\pgfqpoint{0.984309in}{1.508913in}}{\pgfqpoint{0.984309in}{1.510755in}}%
\pgfpathcurveto{\pgfqpoint{0.984309in}{1.512597in}}{\pgfqpoint{0.983577in}{1.514363in}}{\pgfqpoint{0.982275in}{1.515666in}}%
\pgfpathcurveto{\pgfqpoint{0.980973in}{1.516968in}}{\pgfqpoint{0.979206in}{1.517700in}}{\pgfqpoint{0.977365in}{1.517700in}}%
\pgfpathcurveto{\pgfqpoint{0.975523in}{1.517700in}}{\pgfqpoint{0.973756in}{1.516968in}}{\pgfqpoint{0.972454in}{1.515666in}}%
\pgfpathcurveto{\pgfqpoint{0.971152in}{1.514363in}}{\pgfqpoint{0.970420in}{1.512597in}}{\pgfqpoint{0.970420in}{1.510755in}}%
\pgfpathcurveto{\pgfqpoint{0.970420in}{1.508913in}}{\pgfqpoint{0.971152in}{1.507147in}}{\pgfqpoint{0.972454in}{1.505845in}}%
\pgfpathcurveto{\pgfqpoint{0.973756in}{1.504542in}}{\pgfqpoint{0.975523in}{1.503811in}}{\pgfqpoint{0.977365in}{1.503811in}}%
\pgfpathlineto{\pgfqpoint{0.977365in}{1.503811in}}%
\pgfpathclose%
\pgfusepath{stroke,fill}%
\end{pgfscope}%
\begin{pgfscope}%
\pgfpathrectangle{\pgfqpoint{0.661006in}{0.524170in}}{\pgfqpoint{4.194036in}{1.071446in}}%
\pgfusepath{clip}%
\pgfsetbuttcap%
\pgfsetroundjoin%
\definecolor{currentfill}{rgb}{0.730732,0.836541,0.899873}%
\pgfsetfillcolor{currentfill}%
\pgfsetfillopacity{0.700000}%
\pgfsetlinewidth{1.003750pt}%
\definecolor{currentstroke}{rgb}{0.730732,0.836541,0.899873}%
\pgfsetstrokecolor{currentstroke}%
\pgfsetstrokeopacity{0.700000}%
\pgfsetdash{}{0pt}%
\pgfpathmoveto{\pgfqpoint{0.982059in}{1.501261in}}%
\pgfpathcurveto{\pgfqpoint{0.983900in}{1.501261in}}{\pgfqpoint{0.985667in}{1.501992in}}{\pgfqpoint{0.986969in}{1.503295in}}%
\pgfpathcurveto{\pgfqpoint{0.988272in}{1.504597in}}{\pgfqpoint{0.989003in}{1.506363in}}{\pgfqpoint{0.989003in}{1.508205in}}%
\pgfpathcurveto{\pgfqpoint{0.989003in}{1.510047in}}{\pgfqpoint{0.988272in}{1.511813in}}{\pgfqpoint{0.986969in}{1.513116in}}%
\pgfpathcurveto{\pgfqpoint{0.985667in}{1.514418in}}{\pgfqpoint{0.983900in}{1.515150in}}{\pgfqpoint{0.982059in}{1.515150in}}%
\pgfpathcurveto{\pgfqpoint{0.980217in}{1.515150in}}{\pgfqpoint{0.978451in}{1.514418in}}{\pgfqpoint{0.977148in}{1.513116in}}%
\pgfpathcurveto{\pgfqpoint{0.975846in}{1.511813in}}{\pgfqpoint{0.975114in}{1.510047in}}{\pgfqpoint{0.975114in}{1.508205in}}%
\pgfpathcurveto{\pgfqpoint{0.975114in}{1.506363in}}{\pgfqpoint{0.975846in}{1.504597in}}{\pgfqpoint{0.977148in}{1.503295in}}%
\pgfpathcurveto{\pgfqpoint{0.978451in}{1.501992in}}{\pgfqpoint{0.980217in}{1.501261in}}{\pgfqpoint{0.982059in}{1.501261in}}%
\pgfpathlineto{\pgfqpoint{0.982059in}{1.501261in}}%
\pgfpathclose%
\pgfusepath{stroke,fill}%
\end{pgfscope}%
\begin{pgfscope}%
\pgfpathrectangle{\pgfqpoint{0.661006in}{0.524170in}}{\pgfqpoint{4.194036in}{1.071446in}}%
\pgfusepath{clip}%
\pgfsetbuttcap%
\pgfsetroundjoin%
\definecolor{currentfill}{rgb}{0.730732,0.836541,0.899873}%
\pgfsetfillcolor{currentfill}%
\pgfsetfillopacity{0.700000}%
\pgfsetlinewidth{1.003750pt}%
\definecolor{currentstroke}{rgb}{0.730732,0.836541,0.899873}%
\pgfsetstrokecolor{currentstroke}%
\pgfsetstrokeopacity{0.700000}%
\pgfsetdash{}{0pt}%
\pgfpathmoveto{\pgfqpoint{0.990053in}{1.499140in}}%
\pgfpathcurveto{\pgfqpoint{0.991895in}{1.499140in}}{\pgfqpoint{0.993661in}{1.499871in}}{\pgfqpoint{0.994963in}{1.501174in}}%
\pgfpathcurveto{\pgfqpoint{0.996266in}{1.502476in}}{\pgfqpoint{0.996997in}{1.504242in}}{\pgfqpoint{0.996997in}{1.506084in}}%
\pgfpathcurveto{\pgfqpoint{0.996997in}{1.507926in}}{\pgfqpoint{0.996266in}{1.509692in}}{\pgfqpoint{0.994963in}{1.510995in}}%
\pgfpathcurveto{\pgfqpoint{0.993661in}{1.512297in}}{\pgfqpoint{0.991895in}{1.513029in}}{\pgfqpoint{0.990053in}{1.513029in}}%
\pgfpathcurveto{\pgfqpoint{0.988211in}{1.513029in}}{\pgfqpoint{0.986445in}{1.512297in}}{\pgfqpoint{0.985142in}{1.510995in}}%
\pgfpathcurveto{\pgfqpoint{0.983840in}{1.509692in}}{\pgfqpoint{0.983108in}{1.507926in}}{\pgfqpoint{0.983108in}{1.506084in}}%
\pgfpathcurveto{\pgfqpoint{0.983108in}{1.504242in}}{\pgfqpoint{0.983840in}{1.502476in}}{\pgfqpoint{0.985142in}{1.501174in}}%
\pgfpathcurveto{\pgfqpoint{0.986445in}{1.499871in}}{\pgfqpoint{0.988211in}{1.499140in}}{\pgfqpoint{0.990053in}{1.499140in}}%
\pgfpathlineto{\pgfqpoint{0.990053in}{1.499140in}}%
\pgfpathclose%
\pgfusepath{stroke,fill}%
\end{pgfscope}%
\begin{pgfscope}%
\pgfpathrectangle{\pgfqpoint{0.661006in}{0.524170in}}{\pgfqpoint{4.194036in}{1.071446in}}%
\pgfusepath{clip}%
\pgfsetbuttcap%
\pgfsetroundjoin%
\definecolor{currentfill}{rgb}{0.730732,0.836541,0.899873}%
\pgfsetfillcolor{currentfill}%
\pgfsetfillopacity{0.700000}%
\pgfsetlinewidth{1.003750pt}%
\definecolor{currentstroke}{rgb}{0.730732,0.836541,0.899873}%
\pgfsetstrokecolor{currentstroke}%
\pgfsetstrokeopacity{0.700000}%
\pgfsetdash{}{0pt}%
\pgfpathmoveto{\pgfqpoint{0.998233in}{1.496708in}}%
\pgfpathcurveto{\pgfqpoint{1.000075in}{1.496708in}}{\pgfqpoint{1.001841in}{1.497439in}}{\pgfqpoint{1.003143in}{1.498741in}}%
\pgfpathcurveto{\pgfqpoint{1.004446in}{1.500044in}}{\pgfqpoint{1.005177in}{1.501810in}}{\pgfqpoint{1.005177in}{1.503652in}}%
\pgfpathcurveto{\pgfqpoint{1.005177in}{1.505494in}}{\pgfqpoint{1.004446in}{1.507260in}}{\pgfqpoint{1.003143in}{1.508562in}}%
\pgfpathcurveto{\pgfqpoint{1.001841in}{1.509865in}}{\pgfqpoint{1.000075in}{1.510596in}}{\pgfqpoint{0.998233in}{1.510596in}}%
\pgfpathcurveto{\pgfqpoint{0.996391in}{1.510596in}}{\pgfqpoint{0.994625in}{1.509865in}}{\pgfqpoint{0.993322in}{1.508562in}}%
\pgfpathcurveto{\pgfqpoint{0.992020in}{1.507260in}}{\pgfqpoint{0.991288in}{1.505494in}}{\pgfqpoint{0.991288in}{1.503652in}}%
\pgfpathcurveto{\pgfqpoint{0.991288in}{1.501810in}}{\pgfqpoint{0.992020in}{1.500044in}}{\pgfqpoint{0.993322in}{1.498741in}}%
\pgfpathcurveto{\pgfqpoint{0.994625in}{1.497439in}}{\pgfqpoint{0.996391in}{1.496708in}}{\pgfqpoint{0.998233in}{1.496708in}}%
\pgfpathlineto{\pgfqpoint{0.998233in}{1.496708in}}%
\pgfpathclose%
\pgfusepath{stroke,fill}%
\end{pgfscope}%
\begin{pgfscope}%
\pgfpathrectangle{\pgfqpoint{0.661006in}{0.524170in}}{\pgfqpoint{4.194036in}{1.071446in}}%
\pgfusepath{clip}%
\pgfsetbuttcap%
\pgfsetroundjoin%
\definecolor{currentfill}{rgb}{0.730732,0.836541,0.899873}%
\pgfsetfillcolor{currentfill}%
\pgfsetfillopacity{0.700000}%
\pgfsetlinewidth{1.003750pt}%
\definecolor{currentstroke}{rgb}{0.730732,0.836541,0.899873}%
\pgfsetstrokecolor{currentstroke}%
\pgfsetstrokeopacity{0.700000}%
\pgfsetdash{}{0pt}%
\pgfpathmoveto{\pgfqpoint{1.013477in}{1.492627in}}%
\pgfpathcurveto{\pgfqpoint{1.015319in}{1.492627in}}{\pgfqpoint{1.017086in}{1.493359in}}{\pgfqpoint{1.018388in}{1.494661in}}%
\pgfpathcurveto{\pgfqpoint{1.019690in}{1.495963in}}{\pgfqpoint{1.020422in}{1.497730in}}{\pgfqpoint{1.020422in}{1.499572in}}%
\pgfpathcurveto{\pgfqpoint{1.020422in}{1.501413in}}{\pgfqpoint{1.019690in}{1.503180in}}{\pgfqpoint{1.018388in}{1.504482in}}%
\pgfpathcurveto{\pgfqpoint{1.017086in}{1.505784in}}{\pgfqpoint{1.015319in}{1.506516in}}{\pgfqpoint{1.013477in}{1.506516in}}%
\pgfpathcurveto{\pgfqpoint{1.011636in}{1.506516in}}{\pgfqpoint{1.009869in}{1.505784in}}{\pgfqpoint{1.008567in}{1.504482in}}%
\pgfpathcurveto{\pgfqpoint{1.007265in}{1.503180in}}{\pgfqpoint{1.006533in}{1.501413in}}{\pgfqpoint{1.006533in}{1.499572in}}%
\pgfpathcurveto{\pgfqpoint{1.006533in}{1.497730in}}{\pgfqpoint{1.007265in}{1.495963in}}{\pgfqpoint{1.008567in}{1.494661in}}%
\pgfpathcurveto{\pgfqpoint{1.009869in}{1.493359in}}{\pgfqpoint{1.011636in}{1.492627in}}{\pgfqpoint{1.013477in}{1.492627in}}%
\pgfpathlineto{\pgfqpoint{1.013477in}{1.492627in}}%
\pgfpathclose%
\pgfusepath{stroke,fill}%
\end{pgfscope}%
\begin{pgfscope}%
\pgfpathrectangle{\pgfqpoint{0.661006in}{0.524170in}}{\pgfqpoint{4.194036in}{1.071446in}}%
\pgfusepath{clip}%
\pgfsetbuttcap%
\pgfsetroundjoin%
\definecolor{currentfill}{rgb}{0.730732,0.836541,0.899873}%
\pgfsetfillcolor{currentfill}%
\pgfsetfillopacity{0.700000}%
\pgfsetlinewidth{1.003750pt}%
\definecolor{currentstroke}{rgb}{0.730732,0.836541,0.899873}%
\pgfsetstrokecolor{currentstroke}%
\pgfsetstrokeopacity{0.700000}%
\pgfsetdash{}{0pt}%
\pgfpathmoveto{\pgfqpoint{1.035368in}{1.487170in}}%
\pgfpathcurveto{\pgfqpoint{1.037210in}{1.487170in}}{\pgfqpoint{1.038976in}{1.487902in}}{\pgfqpoint{1.040279in}{1.489204in}}%
\pgfpathcurveto{\pgfqpoint{1.041581in}{1.490507in}}{\pgfqpoint{1.042313in}{1.492273in}}{\pgfqpoint{1.042313in}{1.494115in}}%
\pgfpathcurveto{\pgfqpoint{1.042313in}{1.495956in}}{\pgfqpoint{1.041581in}{1.497723in}}{\pgfqpoint{1.040279in}{1.499025in}}%
\pgfpathcurveto{\pgfqpoint{1.038976in}{1.500327in}}{\pgfqpoint{1.037210in}{1.501059in}}{\pgfqpoint{1.035368in}{1.501059in}}%
\pgfpathcurveto{\pgfqpoint{1.033527in}{1.501059in}}{\pgfqpoint{1.031760in}{1.500327in}}{\pgfqpoint{1.030458in}{1.499025in}}%
\pgfpathcurveto{\pgfqpoint{1.029155in}{1.497723in}}{\pgfqpoint{1.028424in}{1.495956in}}{\pgfqpoint{1.028424in}{1.494115in}}%
\pgfpathcurveto{\pgfqpoint{1.028424in}{1.492273in}}{\pgfqpoint{1.029155in}{1.490507in}}{\pgfqpoint{1.030458in}{1.489204in}}%
\pgfpathcurveto{\pgfqpoint{1.031760in}{1.487902in}}{\pgfqpoint{1.033527in}{1.487170in}}{\pgfqpoint{1.035368in}{1.487170in}}%
\pgfpathlineto{\pgfqpoint{1.035368in}{1.487170in}}%
\pgfpathclose%
\pgfusepath{stroke,fill}%
\end{pgfscope}%
\begin{pgfscope}%
\pgfpathrectangle{\pgfqpoint{0.661006in}{0.524170in}}{\pgfqpoint{4.194036in}{1.071446in}}%
\pgfusepath{clip}%
\pgfsetbuttcap%
\pgfsetroundjoin%
\definecolor{currentfill}{rgb}{0.725983,0.832683,0.897866}%
\pgfsetfillcolor{currentfill}%
\pgfsetfillopacity{0.700000}%
\pgfsetlinewidth{1.003750pt}%
\definecolor{currentstroke}{rgb}{0.725983,0.832683,0.897866}%
\pgfsetstrokecolor{currentstroke}%
\pgfsetstrokeopacity{0.700000}%
\pgfsetdash{}{0pt}%
\pgfpathmoveto{\pgfqpoint{1.049823in}{1.483282in}}%
\pgfpathcurveto{\pgfqpoint{1.051664in}{1.483282in}}{\pgfqpoint{1.053431in}{1.484014in}}{\pgfqpoint{1.054733in}{1.485316in}}%
\pgfpathcurveto{\pgfqpoint{1.056035in}{1.486619in}}{\pgfqpoint{1.056767in}{1.488385in}}{\pgfqpoint{1.056767in}{1.490227in}}%
\pgfpathcurveto{\pgfqpoint{1.056767in}{1.492069in}}{\pgfqpoint{1.056035in}{1.493835in}}{\pgfqpoint{1.054733in}{1.495137in}}%
\pgfpathcurveto{\pgfqpoint{1.053431in}{1.496440in}}{\pgfqpoint{1.051664in}{1.497171in}}{\pgfqpoint{1.049823in}{1.497171in}}%
\pgfpathcurveto{\pgfqpoint{1.047981in}{1.497171in}}{\pgfqpoint{1.046214in}{1.496440in}}{\pgfqpoint{1.044912in}{1.495137in}}%
\pgfpathcurveto{\pgfqpoint{1.043610in}{1.493835in}}{\pgfqpoint{1.042878in}{1.492069in}}{\pgfqpoint{1.042878in}{1.490227in}}%
\pgfpathcurveto{\pgfqpoint{1.042878in}{1.488385in}}{\pgfqpoint{1.043610in}{1.486619in}}{\pgfqpoint{1.044912in}{1.485316in}}%
\pgfpathcurveto{\pgfqpoint{1.046214in}{1.484014in}}{\pgfqpoint{1.047981in}{1.483282in}}{\pgfqpoint{1.049823in}{1.483282in}}%
\pgfpathlineto{\pgfqpoint{1.049823in}{1.483282in}}%
\pgfpathclose%
\pgfusepath{stroke,fill}%
\end{pgfscope}%
\begin{pgfscope}%
\pgfpathrectangle{\pgfqpoint{0.661006in}{0.524170in}}{\pgfqpoint{4.194036in}{1.071446in}}%
\pgfusepath{clip}%
\pgfsetbuttcap%
\pgfsetroundjoin%
\definecolor{currentfill}{rgb}{0.725983,0.832683,0.897866}%
\pgfsetfillcolor{currentfill}%
\pgfsetfillopacity{0.700000}%
\pgfsetlinewidth{1.003750pt}%
\definecolor{currentstroke}{rgb}{0.725983,0.832683,0.897866}%
\pgfsetstrokecolor{currentstroke}%
\pgfsetstrokeopacity{0.700000}%
\pgfsetdash{}{0pt}%
\pgfpathmoveto{\pgfqpoint{1.061210in}{1.480814in}}%
\pgfpathcurveto{\pgfqpoint{1.063051in}{1.480814in}}{\pgfqpoint{1.064818in}{1.481546in}}{\pgfqpoint{1.066120in}{1.482848in}}%
\pgfpathcurveto{\pgfqpoint{1.067422in}{1.484150in}}{\pgfqpoint{1.068154in}{1.485917in}}{\pgfqpoint{1.068154in}{1.487758in}}%
\pgfpathcurveto{\pgfqpoint{1.068154in}{1.489600in}}{\pgfqpoint{1.067422in}{1.491367in}}{\pgfqpoint{1.066120in}{1.492669in}}%
\pgfpathcurveto{\pgfqpoint{1.064818in}{1.493971in}}{\pgfqpoint{1.063051in}{1.494703in}}{\pgfqpoint{1.061210in}{1.494703in}}%
\pgfpathcurveto{\pgfqpoint{1.059368in}{1.494703in}}{\pgfqpoint{1.057601in}{1.493971in}}{\pgfqpoint{1.056299in}{1.492669in}}%
\pgfpathcurveto{\pgfqpoint{1.054997in}{1.491367in}}{\pgfqpoint{1.054265in}{1.489600in}}{\pgfqpoint{1.054265in}{1.487758in}}%
\pgfpathcurveto{\pgfqpoint{1.054265in}{1.485917in}}{\pgfqpoint{1.054997in}{1.484150in}}{\pgfqpoint{1.056299in}{1.482848in}}%
\pgfpathcurveto{\pgfqpoint{1.057601in}{1.481546in}}{\pgfqpoint{1.059368in}{1.480814in}}{\pgfqpoint{1.061210in}{1.480814in}}%
\pgfpathlineto{\pgfqpoint{1.061210in}{1.480814in}}%
\pgfpathclose%
\pgfusepath{stroke,fill}%
\end{pgfscope}%
\begin{pgfscope}%
\pgfpathrectangle{\pgfqpoint{0.661006in}{0.524170in}}{\pgfqpoint{4.194036in}{1.071446in}}%
\pgfusepath{clip}%
\pgfsetbuttcap%
\pgfsetroundjoin%
\definecolor{currentfill}{rgb}{0.721268,0.828809,0.895854}%
\pgfsetfillcolor{currentfill}%
\pgfsetfillopacity{0.700000}%
\pgfsetlinewidth{1.003750pt}%
\definecolor{currentstroke}{rgb}{0.721268,0.828809,0.895854}%
\pgfsetstrokecolor{currentstroke}%
\pgfsetstrokeopacity{0.700000}%
\pgfsetdash{}{0pt}%
\pgfpathmoveto{\pgfqpoint{1.081752in}{1.476435in}}%
\pgfpathcurveto{\pgfqpoint{1.083594in}{1.476435in}}{\pgfqpoint{1.085361in}{1.477167in}}{\pgfqpoint{1.086663in}{1.478469in}}%
\pgfpathcurveto{\pgfqpoint{1.087965in}{1.479772in}}{\pgfqpoint{1.088697in}{1.481538in}}{\pgfqpoint{1.088697in}{1.483380in}}%
\pgfpathcurveto{\pgfqpoint{1.088697in}{1.485221in}}{\pgfqpoint{1.087965in}{1.486988in}}{\pgfqpoint{1.086663in}{1.488290in}}%
\pgfpathcurveto{\pgfqpoint{1.085361in}{1.489592in}}{\pgfqpoint{1.083594in}{1.490324in}}{\pgfqpoint{1.081752in}{1.490324in}}%
\pgfpathcurveto{\pgfqpoint{1.079911in}{1.490324in}}{\pgfqpoint{1.078144in}{1.489592in}}{\pgfqpoint{1.076842in}{1.488290in}}%
\pgfpathcurveto{\pgfqpoint{1.075540in}{1.486988in}}{\pgfqpoint{1.074808in}{1.485221in}}{\pgfqpoint{1.074808in}{1.483380in}}%
\pgfpathcurveto{\pgfqpoint{1.074808in}{1.481538in}}{\pgfqpoint{1.075540in}{1.479772in}}{\pgfqpoint{1.076842in}{1.478469in}}%
\pgfpathcurveto{\pgfqpoint{1.078144in}{1.477167in}}{\pgfqpoint{1.079911in}{1.476435in}}{\pgfqpoint{1.081752in}{1.476435in}}%
\pgfpathlineto{\pgfqpoint{1.081752in}{1.476435in}}%
\pgfpathclose%
\pgfusepath{stroke,fill}%
\end{pgfscope}%
\begin{pgfscope}%
\pgfpathrectangle{\pgfqpoint{0.661006in}{0.524170in}}{\pgfqpoint{4.194036in}{1.071446in}}%
\pgfusepath{clip}%
\pgfsetbuttcap%
\pgfsetroundjoin%
\definecolor{currentfill}{rgb}{0.721268,0.828809,0.895854}%
\pgfsetfillcolor{currentfill}%
\pgfsetfillopacity{0.700000}%
\pgfsetlinewidth{1.003750pt}%
\definecolor{currentstroke}{rgb}{0.721268,0.828809,0.895854}%
\pgfsetstrokecolor{currentstroke}%
\pgfsetstrokeopacity{0.700000}%
\pgfsetdash{}{0pt}%
\pgfpathmoveto{\pgfqpoint{1.099972in}{1.470420in}}%
\pgfpathcurveto{\pgfqpoint{1.101813in}{1.470420in}}{\pgfqpoint{1.103580in}{1.471152in}}{\pgfqpoint{1.104882in}{1.472454in}}%
\pgfpathcurveto{\pgfqpoint{1.106184in}{1.473756in}}{\pgfqpoint{1.106916in}{1.475523in}}{\pgfqpoint{1.106916in}{1.477365in}}%
\pgfpathcurveto{\pgfqpoint{1.106916in}{1.479206in}}{\pgfqpoint{1.106184in}{1.480973in}}{\pgfqpoint{1.104882in}{1.482275in}}%
\pgfpathcurveto{\pgfqpoint{1.103580in}{1.483577in}}{\pgfqpoint{1.101813in}{1.484309in}}{\pgfqpoint{1.099972in}{1.484309in}}%
\pgfpathcurveto{\pgfqpoint{1.098130in}{1.484309in}}{\pgfqpoint{1.096363in}{1.483577in}}{\pgfqpoint{1.095061in}{1.482275in}}%
\pgfpathcurveto{\pgfqpoint{1.093759in}{1.480973in}}{\pgfqpoint{1.093027in}{1.479206in}}{\pgfqpoint{1.093027in}{1.477365in}}%
\pgfpathcurveto{\pgfqpoint{1.093027in}{1.475523in}}{\pgfqpoint{1.093759in}{1.473756in}}{\pgfqpoint{1.095061in}{1.472454in}}%
\pgfpathcurveto{\pgfqpoint{1.096363in}{1.471152in}}{\pgfqpoint{1.098130in}{1.470420in}}{\pgfqpoint{1.099972in}{1.470420in}}%
\pgfpathlineto{\pgfqpoint{1.099972in}{1.470420in}}%
\pgfpathclose%
\pgfusepath{stroke,fill}%
\end{pgfscope}%
\begin{pgfscope}%
\pgfpathrectangle{\pgfqpoint{0.661006in}{0.524170in}}{\pgfqpoint{4.194036in}{1.071446in}}%
\pgfusepath{clip}%
\pgfsetbuttcap%
\pgfsetroundjoin%
\definecolor{currentfill}{rgb}{0.721268,0.828809,0.895854}%
\pgfsetfillcolor{currentfill}%
\pgfsetfillopacity{0.700000}%
\pgfsetlinewidth{1.003750pt}%
\definecolor{currentstroke}{rgb}{0.721268,0.828809,0.895854}%
\pgfsetstrokecolor{currentstroke}%
\pgfsetstrokeopacity{0.700000}%
\pgfsetdash{}{0pt}%
\pgfpathmoveto{\pgfqpoint{1.135806in}{1.461245in}}%
\pgfpathcurveto{\pgfqpoint{1.137647in}{1.461245in}}{\pgfqpoint{1.139414in}{1.461977in}}{\pgfqpoint{1.140716in}{1.463279in}}%
\pgfpathcurveto{\pgfqpoint{1.142018in}{1.464581in}}{\pgfqpoint{1.142750in}{1.466348in}}{\pgfqpoint{1.142750in}{1.468189in}}%
\pgfpathcurveto{\pgfqpoint{1.142750in}{1.470031in}}{\pgfqpoint{1.142018in}{1.471797in}}{\pgfqpoint{1.140716in}{1.473100in}}%
\pgfpathcurveto{\pgfqpoint{1.139414in}{1.474402in}}{\pgfqpoint{1.137647in}{1.475134in}}{\pgfqpoint{1.135806in}{1.475134in}}%
\pgfpathcurveto{\pgfqpoint{1.133964in}{1.475134in}}{\pgfqpoint{1.132197in}{1.474402in}}{\pgfqpoint{1.130895in}{1.473100in}}%
\pgfpathcurveto{\pgfqpoint{1.129593in}{1.471797in}}{\pgfqpoint{1.128861in}{1.470031in}}{\pgfqpoint{1.128861in}{1.468189in}}%
\pgfpathcurveto{\pgfqpoint{1.128861in}{1.466348in}}{\pgfqpoint{1.129593in}{1.464581in}}{\pgfqpoint{1.130895in}{1.463279in}}%
\pgfpathcurveto{\pgfqpoint{1.132197in}{1.461977in}}{\pgfqpoint{1.133964in}{1.461245in}}{\pgfqpoint{1.135806in}{1.461245in}}%
\pgfpathlineto{\pgfqpoint{1.135806in}{1.461245in}}%
\pgfpathclose%
\pgfusepath{stroke,fill}%
\end{pgfscope}%
\begin{pgfscope}%
\pgfpathrectangle{\pgfqpoint{0.661006in}{0.524170in}}{\pgfqpoint{4.194036in}{1.071446in}}%
\pgfusepath{clip}%
\pgfsetbuttcap%
\pgfsetroundjoin%
\definecolor{currentfill}{rgb}{0.716586,0.824919,0.893836}%
\pgfsetfillcolor{currentfill}%
\pgfsetfillopacity{0.700000}%
\pgfsetlinewidth{1.003750pt}%
\definecolor{currentstroke}{rgb}{0.716586,0.824919,0.893836}%
\pgfsetstrokecolor{currentstroke}%
\pgfsetstrokeopacity{0.700000}%
\pgfsetdash{}{0pt}%
\pgfpathmoveto{\pgfqpoint{1.178100in}{1.453171in}}%
\pgfpathcurveto{\pgfqpoint{1.179941in}{1.453171in}}{\pgfqpoint{1.181708in}{1.453903in}}{\pgfqpoint{1.183010in}{1.455205in}}%
\pgfpathcurveto{\pgfqpoint{1.184313in}{1.456507in}}{\pgfqpoint{1.185044in}{1.458274in}}{\pgfqpoint{1.185044in}{1.460116in}}%
\pgfpathcurveto{\pgfqpoint{1.185044in}{1.461957in}}{\pgfqpoint{1.184313in}{1.463724in}}{\pgfqpoint{1.183010in}{1.465026in}}%
\pgfpathcurveto{\pgfqpoint{1.181708in}{1.466328in}}{\pgfqpoint{1.179941in}{1.467060in}}{\pgfqpoint{1.178100in}{1.467060in}}%
\pgfpathcurveto{\pgfqpoint{1.176258in}{1.467060in}}{\pgfqpoint{1.174492in}{1.466328in}}{\pgfqpoint{1.173189in}{1.465026in}}%
\pgfpathcurveto{\pgfqpoint{1.171887in}{1.463724in}}{\pgfqpoint{1.171155in}{1.461957in}}{\pgfqpoint{1.171155in}{1.460116in}}%
\pgfpathcurveto{\pgfqpoint{1.171155in}{1.458274in}}{\pgfqpoint{1.171887in}{1.456507in}}{\pgfqpoint{1.173189in}{1.455205in}}%
\pgfpathcurveto{\pgfqpoint{1.174492in}{1.453903in}}{\pgfqpoint{1.176258in}{1.453171in}}{\pgfqpoint{1.178100in}{1.453171in}}%
\pgfpathlineto{\pgfqpoint{1.178100in}{1.453171in}}%
\pgfpathclose%
\pgfusepath{stroke,fill}%
\end{pgfscope}%
\begin{pgfscope}%
\pgfpathrectangle{\pgfqpoint{0.661006in}{0.524170in}}{\pgfqpoint{4.194036in}{1.071446in}}%
\pgfusepath{clip}%
\pgfsetbuttcap%
\pgfsetroundjoin%
\definecolor{currentfill}{rgb}{0.716586,0.824919,0.893836}%
\pgfsetfillcolor{currentfill}%
\pgfsetfillopacity{0.700000}%
\pgfsetlinewidth{1.003750pt}%
\definecolor{currentstroke}{rgb}{0.716586,0.824919,0.893836}%
\pgfsetstrokecolor{currentstroke}%
\pgfsetstrokeopacity{0.700000}%
\pgfsetdash{}{0pt}%
\pgfpathmoveto{\pgfqpoint{1.192647in}{1.450268in}}%
\pgfpathcurveto{\pgfqpoint{1.194489in}{1.450268in}}{\pgfqpoint{1.196255in}{1.450999in}}{\pgfqpoint{1.197558in}{1.452302in}}%
\pgfpathcurveto{\pgfqpoint{1.198860in}{1.453604in}}{\pgfqpoint{1.199592in}{1.455370in}}{\pgfqpoint{1.199592in}{1.457212in}}%
\pgfpathcurveto{\pgfqpoint{1.199592in}{1.459054in}}{\pgfqpoint{1.198860in}{1.460820in}}{\pgfqpoint{1.197558in}{1.462123in}}%
\pgfpathcurveto{\pgfqpoint{1.196255in}{1.463425in}}{\pgfqpoint{1.194489in}{1.464157in}}{\pgfqpoint{1.192647in}{1.464157in}}%
\pgfpathcurveto{\pgfqpoint{1.190805in}{1.464157in}}{\pgfqpoint{1.189039in}{1.463425in}}{\pgfqpoint{1.187737in}{1.462123in}}%
\pgfpathcurveto{\pgfqpoint{1.186434in}{1.460820in}}{\pgfqpoint{1.185703in}{1.459054in}}{\pgfqpoint{1.185703in}{1.457212in}}%
\pgfpathcurveto{\pgfqpoint{1.185703in}{1.455370in}}{\pgfqpoint{1.186434in}{1.453604in}}{\pgfqpoint{1.187737in}{1.452302in}}%
\pgfpathcurveto{\pgfqpoint{1.189039in}{1.450999in}}{\pgfqpoint{1.190805in}{1.450268in}}{\pgfqpoint{1.192647in}{1.450268in}}%
\pgfpathlineto{\pgfqpoint{1.192647in}{1.450268in}}%
\pgfpathclose%
\pgfusepath{stroke,fill}%
\end{pgfscope}%
\begin{pgfscope}%
\pgfpathrectangle{\pgfqpoint{0.661006in}{0.524170in}}{\pgfqpoint{4.194036in}{1.071446in}}%
\pgfusepath{clip}%
\pgfsetbuttcap%
\pgfsetroundjoin%
\definecolor{currentfill}{rgb}{0.716586,0.824919,0.893836}%
\pgfsetfillcolor{currentfill}%
\pgfsetfillopacity{0.700000}%
\pgfsetlinewidth{1.003750pt}%
\definecolor{currentstroke}{rgb}{0.716586,0.824919,0.893836}%
\pgfsetstrokecolor{currentstroke}%
\pgfsetstrokeopacity{0.700000}%
\pgfsetdash{}{0pt}%
\pgfpathmoveto{\pgfqpoint{1.176473in}{1.453524in}}%
\pgfpathcurveto{\pgfqpoint{1.178315in}{1.453524in}}{\pgfqpoint{1.180081in}{1.454256in}}{\pgfqpoint{1.181384in}{1.455558in}}%
\pgfpathcurveto{\pgfqpoint{1.182686in}{1.456860in}}{\pgfqpoint{1.183418in}{1.458627in}}{\pgfqpoint{1.183418in}{1.460468in}}%
\pgfpathcurveto{\pgfqpoint{1.183418in}{1.462310in}}{\pgfqpoint{1.182686in}{1.464077in}}{\pgfqpoint{1.181384in}{1.465379in}}%
\pgfpathcurveto{\pgfqpoint{1.180081in}{1.466681in}}{\pgfqpoint{1.178315in}{1.467413in}}{\pgfqpoint{1.176473in}{1.467413in}}%
\pgfpathcurveto{\pgfqpoint{1.174631in}{1.467413in}}{\pgfqpoint{1.172865in}{1.466681in}}{\pgfqpoint{1.171563in}{1.465379in}}%
\pgfpathcurveto{\pgfqpoint{1.170260in}{1.464077in}}{\pgfqpoint{1.169529in}{1.462310in}}{\pgfqpoint{1.169529in}{1.460468in}}%
\pgfpathcurveto{\pgfqpoint{1.169529in}{1.458627in}}{\pgfqpoint{1.170260in}{1.456860in}}{\pgfqpoint{1.171563in}{1.455558in}}%
\pgfpathcurveto{\pgfqpoint{1.172865in}{1.454256in}}{\pgfqpoint{1.174631in}{1.453524in}}{\pgfqpoint{1.176473in}{1.453524in}}%
\pgfpathlineto{\pgfqpoint{1.176473in}{1.453524in}}%
\pgfpathclose%
\pgfusepath{stroke,fill}%
\end{pgfscope}%
\begin{pgfscope}%
\pgfpathrectangle{\pgfqpoint{0.661006in}{0.524170in}}{\pgfqpoint{4.194036in}{1.071446in}}%
\pgfusepath{clip}%
\pgfsetbuttcap%
\pgfsetroundjoin%
\definecolor{currentfill}{rgb}{0.716586,0.824919,0.893836}%
\pgfsetfillcolor{currentfill}%
\pgfsetfillopacity{0.700000}%
\pgfsetlinewidth{1.003750pt}%
\definecolor{currentstroke}{rgb}{0.716586,0.824919,0.893836}%
\pgfsetstrokecolor{currentstroke}%
\pgfsetstrokeopacity{0.700000}%
\pgfsetdash{}{0pt}%
\pgfpathmoveto{\pgfqpoint{1.149749in}{1.460037in}}%
\pgfpathcurveto{\pgfqpoint{1.151590in}{1.460037in}}{\pgfqpoint{1.153357in}{1.460769in}}{\pgfqpoint{1.154659in}{1.462071in}}%
\pgfpathcurveto{\pgfqpoint{1.155961in}{1.463373in}}{\pgfqpoint{1.156693in}{1.465140in}}{\pgfqpoint{1.156693in}{1.466982in}}%
\pgfpathcurveto{\pgfqpoint{1.156693in}{1.468823in}}{\pgfqpoint{1.155961in}{1.470590in}}{\pgfqpoint{1.154659in}{1.471892in}}%
\pgfpathcurveto{\pgfqpoint{1.153357in}{1.473194in}}{\pgfqpoint{1.151590in}{1.473926in}}{\pgfqpoint{1.149749in}{1.473926in}}%
\pgfpathcurveto{\pgfqpoint{1.147907in}{1.473926in}}{\pgfqpoint{1.146140in}{1.473194in}}{\pgfqpoint{1.144838in}{1.471892in}}%
\pgfpathcurveto{\pgfqpoint{1.143536in}{1.470590in}}{\pgfqpoint{1.142804in}{1.468823in}}{\pgfqpoint{1.142804in}{1.466982in}}%
\pgfpathcurveto{\pgfqpoint{1.142804in}{1.465140in}}{\pgfqpoint{1.143536in}{1.463373in}}{\pgfqpoint{1.144838in}{1.462071in}}%
\pgfpathcurveto{\pgfqpoint{1.146140in}{1.460769in}}{\pgfqpoint{1.147907in}{1.460037in}}{\pgfqpoint{1.149749in}{1.460037in}}%
\pgfpathlineto{\pgfqpoint{1.149749in}{1.460037in}}%
\pgfpathclose%
\pgfusepath{stroke,fill}%
\end{pgfscope}%
\begin{pgfscope}%
\pgfpathrectangle{\pgfqpoint{0.661006in}{0.524170in}}{\pgfqpoint{4.194036in}{1.071446in}}%
\pgfusepath{clip}%
\pgfsetbuttcap%
\pgfsetroundjoin%
\definecolor{currentfill}{rgb}{0.711937,0.821013,0.891811}%
\pgfsetfillcolor{currentfill}%
\pgfsetfillopacity{0.700000}%
\pgfsetlinewidth{1.003750pt}%
\definecolor{currentstroke}{rgb}{0.711937,0.821013,0.891811}%
\pgfsetstrokecolor{currentstroke}%
\pgfsetstrokeopacity{0.700000}%
\pgfsetdash{}{0pt}%
\pgfpathmoveto{\pgfqpoint{1.105781in}{1.468829in}}%
\pgfpathcurveto{\pgfqpoint{1.107623in}{1.468829in}}{\pgfqpoint{1.109389in}{1.469561in}}{\pgfqpoint{1.110692in}{1.470863in}}%
\pgfpathcurveto{\pgfqpoint{1.111994in}{1.472165in}}{\pgfqpoint{1.112726in}{1.473932in}}{\pgfqpoint{1.112726in}{1.475774in}}%
\pgfpathcurveto{\pgfqpoint{1.112726in}{1.477615in}}{\pgfqpoint{1.111994in}{1.479382in}}{\pgfqpoint{1.110692in}{1.480684in}}%
\pgfpathcurveto{\pgfqpoint{1.109389in}{1.481986in}}{\pgfqpoint{1.107623in}{1.482718in}}{\pgfqpoint{1.105781in}{1.482718in}}%
\pgfpathcurveto{\pgfqpoint{1.103940in}{1.482718in}}{\pgfqpoint{1.102173in}{1.481986in}}{\pgfqpoint{1.100871in}{1.480684in}}%
\pgfpathcurveto{\pgfqpoint{1.099568in}{1.479382in}}{\pgfqpoint{1.098837in}{1.477615in}}{\pgfqpoint{1.098837in}{1.475774in}}%
\pgfpathcurveto{\pgfqpoint{1.098837in}{1.473932in}}{\pgfqpoint{1.099568in}{1.472165in}}{\pgfqpoint{1.100871in}{1.470863in}}%
\pgfpathcurveto{\pgfqpoint{1.102173in}{1.469561in}}{\pgfqpoint{1.103940in}{1.468829in}}{\pgfqpoint{1.105781in}{1.468829in}}%
\pgfpathlineto{\pgfqpoint{1.105781in}{1.468829in}}%
\pgfpathclose%
\pgfusepath{stroke,fill}%
\end{pgfscope}%
\begin{pgfscope}%
\pgfpathrectangle{\pgfqpoint{0.661006in}{0.524170in}}{\pgfqpoint{4.194036in}{1.071446in}}%
\pgfusepath{clip}%
\pgfsetbuttcap%
\pgfsetroundjoin%
\definecolor{currentfill}{rgb}{0.711937,0.821013,0.891811}%
\pgfsetfillcolor{currentfill}%
\pgfsetfillopacity{0.700000}%
\pgfsetlinewidth{1.003750pt}%
\definecolor{currentstroke}{rgb}{0.711937,0.821013,0.891811}%
\pgfsetstrokecolor{currentstroke}%
\pgfsetstrokeopacity{0.700000}%
\pgfsetdash{}{0pt}%
\pgfpathmoveto{\pgfqpoint{1.077895in}{1.475996in}}%
\pgfpathcurveto{\pgfqpoint{1.079737in}{1.475996in}}{\pgfqpoint{1.081503in}{1.476728in}}{\pgfqpoint{1.082805in}{1.478030in}}%
\pgfpathcurveto{\pgfqpoint{1.084108in}{1.479332in}}{\pgfqpoint{1.084839in}{1.481099in}}{\pgfqpoint{1.084839in}{1.482940in}}%
\pgfpathcurveto{\pgfqpoint{1.084839in}{1.484782in}}{\pgfqpoint{1.084108in}{1.486549in}}{\pgfqpoint{1.082805in}{1.487851in}}%
\pgfpathcurveto{\pgfqpoint{1.081503in}{1.489153in}}{\pgfqpoint{1.079737in}{1.489885in}}{\pgfqpoint{1.077895in}{1.489885in}}%
\pgfpathcurveto{\pgfqpoint{1.076053in}{1.489885in}}{\pgfqpoint{1.074287in}{1.489153in}}{\pgfqpoint{1.072984in}{1.487851in}}%
\pgfpathcurveto{\pgfqpoint{1.071682in}{1.486549in}}{\pgfqpoint{1.070950in}{1.484782in}}{\pgfqpoint{1.070950in}{1.482940in}}%
\pgfpathcurveto{\pgfqpoint{1.070950in}{1.481099in}}{\pgfqpoint{1.071682in}{1.479332in}}{\pgfqpoint{1.072984in}{1.478030in}}%
\pgfpathcurveto{\pgfqpoint{1.074287in}{1.476728in}}{\pgfqpoint{1.076053in}{1.475996in}}{\pgfqpoint{1.077895in}{1.475996in}}%
\pgfpathlineto{\pgfqpoint{1.077895in}{1.475996in}}%
\pgfpathclose%
\pgfusepath{stroke,fill}%
\end{pgfscope}%
\begin{pgfscope}%
\pgfpathrectangle{\pgfqpoint{0.661006in}{0.524170in}}{\pgfqpoint{4.194036in}{1.071446in}}%
\pgfusepath{clip}%
\pgfsetbuttcap%
\pgfsetroundjoin%
\definecolor{currentfill}{rgb}{0.711937,0.821013,0.891811}%
\pgfsetfillcolor{currentfill}%
\pgfsetfillopacity{0.700000}%
\pgfsetlinewidth{1.003750pt}%
\definecolor{currentstroke}{rgb}{0.711937,0.821013,0.891811}%
\pgfsetstrokecolor{currentstroke}%
\pgfsetstrokeopacity{0.700000}%
\pgfsetdash{}{0pt}%
\pgfpathmoveto{\pgfqpoint{1.046755in}{1.480788in}}%
\pgfpathcurveto{\pgfqpoint{1.048597in}{1.480788in}}{\pgfqpoint{1.050363in}{1.481520in}}{\pgfqpoint{1.051666in}{1.482822in}}%
\pgfpathcurveto{\pgfqpoint{1.052968in}{1.484124in}}{\pgfqpoint{1.053700in}{1.485891in}}{\pgfqpoint{1.053700in}{1.487733in}}%
\pgfpathcurveto{\pgfqpoint{1.053700in}{1.489574in}}{\pgfqpoint{1.052968in}{1.491341in}}{\pgfqpoint{1.051666in}{1.492643in}}%
\pgfpathcurveto{\pgfqpoint{1.050363in}{1.493945in}}{\pgfqpoint{1.048597in}{1.494677in}}{\pgfqpoint{1.046755in}{1.494677in}}%
\pgfpathcurveto{\pgfqpoint{1.044913in}{1.494677in}}{\pgfqpoint{1.043147in}{1.493945in}}{\pgfqpoint{1.041845in}{1.492643in}}%
\pgfpathcurveto{\pgfqpoint{1.040542in}{1.491341in}}{\pgfqpoint{1.039811in}{1.489574in}}{\pgfqpoint{1.039811in}{1.487733in}}%
\pgfpathcurveto{\pgfqpoint{1.039811in}{1.485891in}}{\pgfqpoint{1.040542in}{1.484124in}}{\pgfqpoint{1.041845in}{1.482822in}}%
\pgfpathcurveto{\pgfqpoint{1.043147in}{1.481520in}}{\pgfqpoint{1.044913in}{1.480788in}}{\pgfqpoint{1.046755in}{1.480788in}}%
\pgfpathlineto{\pgfqpoint{1.046755in}{1.480788in}}%
\pgfpathclose%
\pgfusepath{stroke,fill}%
\end{pgfscope}%
\begin{pgfscope}%
\pgfpathrectangle{\pgfqpoint{0.661006in}{0.524170in}}{\pgfqpoint{4.194036in}{1.071446in}}%
\pgfusepath{clip}%
\pgfsetbuttcap%
\pgfsetroundjoin%
\definecolor{currentfill}{rgb}{0.707322,0.817092,0.889780}%
\pgfsetfillcolor{currentfill}%
\pgfsetfillopacity{0.700000}%
\pgfsetlinewidth{1.003750pt}%
\definecolor{currentstroke}{rgb}{0.707322,0.817092,0.889780}%
\pgfsetstrokecolor{currentstroke}%
\pgfsetstrokeopacity{0.700000}%
\pgfsetdash{}{0pt}%
\pgfpathmoveto{\pgfqpoint{1.039691in}{1.482845in}}%
\pgfpathcurveto{\pgfqpoint{1.041532in}{1.482845in}}{\pgfqpoint{1.043299in}{1.483577in}}{\pgfqpoint{1.044601in}{1.484879in}}%
\pgfpathcurveto{\pgfqpoint{1.045903in}{1.486181in}}{\pgfqpoint{1.046635in}{1.487948in}}{\pgfqpoint{1.046635in}{1.489789in}}%
\pgfpathcurveto{\pgfqpoint{1.046635in}{1.491631in}}{\pgfqpoint{1.045903in}{1.493398in}}{\pgfqpoint{1.044601in}{1.494700in}}%
\pgfpathcurveto{\pgfqpoint{1.043299in}{1.496002in}}{\pgfqpoint{1.041532in}{1.496734in}}{\pgfqpoint{1.039691in}{1.496734in}}%
\pgfpathcurveto{\pgfqpoint{1.037849in}{1.496734in}}{\pgfqpoint{1.036082in}{1.496002in}}{\pgfqpoint{1.034780in}{1.494700in}}%
\pgfpathcurveto{\pgfqpoint{1.033478in}{1.493398in}}{\pgfqpoint{1.032746in}{1.491631in}}{\pgfqpoint{1.032746in}{1.489789in}}%
\pgfpathcurveto{\pgfqpoint{1.032746in}{1.487948in}}{\pgfqpoint{1.033478in}{1.486181in}}{\pgfqpoint{1.034780in}{1.484879in}}%
\pgfpathcurveto{\pgfqpoint{1.036082in}{1.483577in}}{\pgfqpoint{1.037849in}{1.482845in}}{\pgfqpoint{1.039691in}{1.482845in}}%
\pgfpathlineto{\pgfqpoint{1.039691in}{1.482845in}}%
\pgfpathclose%
\pgfusepath{stroke,fill}%
\end{pgfscope}%
\begin{pgfscope}%
\pgfpathrectangle{\pgfqpoint{0.661006in}{0.524170in}}{\pgfqpoint{4.194036in}{1.071446in}}%
\pgfusepath{clip}%
\pgfsetbuttcap%
\pgfsetroundjoin%
\definecolor{currentfill}{rgb}{0.707322,0.817092,0.889780}%
\pgfsetfillcolor{currentfill}%
\pgfsetfillopacity{0.700000}%
\pgfsetlinewidth{1.003750pt}%
\definecolor{currentstroke}{rgb}{0.707322,0.817092,0.889780}%
\pgfsetstrokecolor{currentstroke}%
\pgfsetstrokeopacity{0.700000}%
\pgfsetdash{}{0pt}%
\pgfpathmoveto{\pgfqpoint{1.029837in}{1.485226in}}%
\pgfpathcurveto{\pgfqpoint{1.031679in}{1.485226in}}{\pgfqpoint{1.033446in}{1.485958in}}{\pgfqpoint{1.034748in}{1.487260in}}%
\pgfpathcurveto{\pgfqpoint{1.036050in}{1.488562in}}{\pgfqpoint{1.036782in}{1.490329in}}{\pgfqpoint{1.036782in}{1.492171in}}%
\pgfpathcurveto{\pgfqpoint{1.036782in}{1.494012in}}{\pgfqpoint{1.036050in}{1.495779in}}{\pgfqpoint{1.034748in}{1.497081in}}%
\pgfpathcurveto{\pgfqpoint{1.033446in}{1.498383in}}{\pgfqpoint{1.031679in}{1.499115in}}{\pgfqpoint{1.029837in}{1.499115in}}%
\pgfpathcurveto{\pgfqpoint{1.027996in}{1.499115in}}{\pgfqpoint{1.026229in}{1.498383in}}{\pgfqpoint{1.024927in}{1.497081in}}%
\pgfpathcurveto{\pgfqpoint{1.023625in}{1.495779in}}{\pgfqpoint{1.022893in}{1.494012in}}{\pgfqpoint{1.022893in}{1.492171in}}%
\pgfpathcurveto{\pgfqpoint{1.022893in}{1.490329in}}{\pgfqpoint{1.023625in}{1.488562in}}{\pgfqpoint{1.024927in}{1.487260in}}%
\pgfpathcurveto{\pgfqpoint{1.026229in}{1.485958in}}{\pgfqpoint{1.027996in}{1.485226in}}{\pgfqpoint{1.029837in}{1.485226in}}%
\pgfpathlineto{\pgfqpoint{1.029837in}{1.485226in}}%
\pgfpathclose%
\pgfusepath{stroke,fill}%
\end{pgfscope}%
\begin{pgfscope}%
\pgfpathrectangle{\pgfqpoint{0.661006in}{0.524170in}}{\pgfqpoint{4.194036in}{1.071446in}}%
\pgfusepath{clip}%
\pgfsetbuttcap%
\pgfsetroundjoin%
\definecolor{currentfill}{rgb}{0.707322,0.817092,0.889780}%
\pgfsetfillcolor{currentfill}%
\pgfsetfillopacity{0.700000}%
\pgfsetlinewidth{1.003750pt}%
\definecolor{currentstroke}{rgb}{0.707322,0.817092,0.889780}%
\pgfsetstrokecolor{currentstroke}%
\pgfsetstrokeopacity{0.700000}%
\pgfsetdash{}{0pt}%
\pgfpathmoveto{\pgfqpoint{1.022355in}{1.485501in}}%
\pgfpathcurveto{\pgfqpoint{1.024196in}{1.485501in}}{\pgfqpoint{1.025963in}{1.486233in}}{\pgfqpoint{1.027265in}{1.487535in}}%
\pgfpathcurveto{\pgfqpoint{1.028567in}{1.488837in}}{\pgfqpoint{1.029299in}{1.490604in}}{\pgfqpoint{1.029299in}{1.492446in}}%
\pgfpathcurveto{\pgfqpoint{1.029299in}{1.494287in}}{\pgfqpoint{1.028567in}{1.496054in}}{\pgfqpoint{1.027265in}{1.497356in}}%
\pgfpathcurveto{\pgfqpoint{1.025963in}{1.498658in}}{\pgfqpoint{1.024196in}{1.499390in}}{\pgfqpoint{1.022355in}{1.499390in}}%
\pgfpathcurveto{\pgfqpoint{1.020513in}{1.499390in}}{\pgfqpoint{1.018746in}{1.498658in}}{\pgfqpoint{1.017444in}{1.497356in}}%
\pgfpathcurveto{\pgfqpoint{1.016142in}{1.496054in}}{\pgfqpoint{1.015410in}{1.494287in}}{\pgfqpoint{1.015410in}{1.492446in}}%
\pgfpathcurveto{\pgfqpoint{1.015410in}{1.490604in}}{\pgfqpoint{1.016142in}{1.488837in}}{\pgfqpoint{1.017444in}{1.487535in}}%
\pgfpathcurveto{\pgfqpoint{1.018746in}{1.486233in}}{\pgfqpoint{1.020513in}{1.485501in}}{\pgfqpoint{1.022355in}{1.485501in}}%
\pgfpathlineto{\pgfqpoint{1.022355in}{1.485501in}}%
\pgfpathclose%
\pgfusepath{stroke,fill}%
\end{pgfscope}%
\begin{pgfscope}%
\pgfpathrectangle{\pgfqpoint{0.661006in}{0.524170in}}{\pgfqpoint{4.194036in}{1.071446in}}%
\pgfusepath{clip}%
\pgfsetbuttcap%
\pgfsetroundjoin%
\definecolor{currentfill}{rgb}{0.707322,0.817092,0.889780}%
\pgfsetfillcolor{currentfill}%
\pgfsetfillopacity{0.700000}%
\pgfsetlinewidth{1.003750pt}%
\definecolor{currentstroke}{rgb}{0.707322,0.817092,0.889780}%
\pgfsetstrokecolor{currentstroke}%
\pgfsetstrokeopacity{0.700000}%
\pgfsetdash{}{0pt}%
\pgfpathmoveto{\pgfqpoint{1.045314in}{1.480593in}}%
\pgfpathcurveto{\pgfqpoint{1.047156in}{1.480593in}}{\pgfqpoint{1.048923in}{1.481325in}}{\pgfqpoint{1.050225in}{1.482627in}}%
\pgfpathcurveto{\pgfqpoint{1.051527in}{1.483929in}}{\pgfqpoint{1.052259in}{1.485696in}}{\pgfqpoint{1.052259in}{1.487537in}}%
\pgfpathcurveto{\pgfqpoint{1.052259in}{1.489379in}}{\pgfqpoint{1.051527in}{1.491146in}}{\pgfqpoint{1.050225in}{1.492448in}}%
\pgfpathcurveto{\pgfqpoint{1.048923in}{1.493750in}}{\pgfqpoint{1.047156in}{1.494482in}}{\pgfqpoint{1.045314in}{1.494482in}}%
\pgfpathcurveto{\pgfqpoint{1.043473in}{1.494482in}}{\pgfqpoint{1.041706in}{1.493750in}}{\pgfqpoint{1.040404in}{1.492448in}}%
\pgfpathcurveto{\pgfqpoint{1.039102in}{1.491146in}}{\pgfqpoint{1.038370in}{1.489379in}}{\pgfqpoint{1.038370in}{1.487537in}}%
\pgfpathcurveto{\pgfqpoint{1.038370in}{1.485696in}}{\pgfqpoint{1.039102in}{1.483929in}}{\pgfqpoint{1.040404in}{1.482627in}}%
\pgfpathcurveto{\pgfqpoint{1.041706in}{1.481325in}}{\pgfqpoint{1.043473in}{1.480593in}}{\pgfqpoint{1.045314in}{1.480593in}}%
\pgfpathlineto{\pgfqpoint{1.045314in}{1.480593in}}%
\pgfpathclose%
\pgfusepath{stroke,fill}%
\end{pgfscope}%
\begin{pgfscope}%
\pgfpathrectangle{\pgfqpoint{0.661006in}{0.524170in}}{\pgfqpoint{4.194036in}{1.071446in}}%
\pgfusepath{clip}%
\pgfsetbuttcap%
\pgfsetroundjoin%
\definecolor{currentfill}{rgb}{0.702740,0.813155,0.887742}%
\pgfsetfillcolor{currentfill}%
\pgfsetfillopacity{0.700000}%
\pgfsetlinewidth{1.003750pt}%
\definecolor{currentstroke}{rgb}{0.702740,0.813155,0.887742}%
\pgfsetstrokecolor{currentstroke}%
\pgfsetstrokeopacity{0.700000}%
\pgfsetdash{}{0pt}%
\pgfpathmoveto{\pgfqpoint{1.065811in}{1.474595in}}%
\pgfpathcurveto{\pgfqpoint{1.067652in}{1.474595in}}{\pgfqpoint{1.069419in}{1.475327in}}{\pgfqpoint{1.070721in}{1.476629in}}%
\pgfpathcurveto{\pgfqpoint{1.072024in}{1.477932in}}{\pgfqpoint{1.072755in}{1.479698in}}{\pgfqpoint{1.072755in}{1.481540in}}%
\pgfpathcurveto{\pgfqpoint{1.072755in}{1.483382in}}{\pgfqpoint{1.072024in}{1.485148in}}{\pgfqpoint{1.070721in}{1.486450in}}%
\pgfpathcurveto{\pgfqpoint{1.069419in}{1.487753in}}{\pgfqpoint{1.067652in}{1.488484in}}{\pgfqpoint{1.065811in}{1.488484in}}%
\pgfpathcurveto{\pgfqpoint{1.063969in}{1.488484in}}{\pgfqpoint{1.062203in}{1.487753in}}{\pgfqpoint{1.060900in}{1.486450in}}%
\pgfpathcurveto{\pgfqpoint{1.059598in}{1.485148in}}{\pgfqpoint{1.058866in}{1.483382in}}{\pgfqpoint{1.058866in}{1.481540in}}%
\pgfpathcurveto{\pgfqpoint{1.058866in}{1.479698in}}{\pgfqpoint{1.059598in}{1.477932in}}{\pgfqpoint{1.060900in}{1.476629in}}%
\pgfpathcurveto{\pgfqpoint{1.062203in}{1.475327in}}{\pgfqpoint{1.063969in}{1.474595in}}{\pgfqpoint{1.065811in}{1.474595in}}%
\pgfpathlineto{\pgfqpoint{1.065811in}{1.474595in}}%
\pgfpathclose%
\pgfusepath{stroke,fill}%
\end{pgfscope}%
\begin{pgfscope}%
\pgfpathrectangle{\pgfqpoint{0.661006in}{0.524170in}}{\pgfqpoint{4.194036in}{1.071446in}}%
\pgfusepath{clip}%
\pgfsetbuttcap%
\pgfsetroundjoin%
\definecolor{currentfill}{rgb}{0.702740,0.813155,0.887742}%
\pgfsetfillcolor{currentfill}%
\pgfsetfillopacity{0.700000}%
\pgfsetlinewidth{1.003750pt}%
\definecolor{currentstroke}{rgb}{0.702740,0.813155,0.887742}%
\pgfsetstrokecolor{currentstroke}%
\pgfsetstrokeopacity{0.700000}%
\pgfsetdash{}{0pt}%
\pgfpathmoveto{\pgfqpoint{1.074479in}{1.472536in}}%
\pgfpathcurveto{\pgfqpoint{1.076320in}{1.472536in}}{\pgfqpoint{1.078087in}{1.473268in}}{\pgfqpoint{1.079389in}{1.474570in}}%
\pgfpathcurveto{\pgfqpoint{1.080692in}{1.475872in}}{\pgfqpoint{1.081423in}{1.477639in}}{\pgfqpoint{1.081423in}{1.479480in}}%
\pgfpathcurveto{\pgfqpoint{1.081423in}{1.481322in}}{\pgfqpoint{1.080692in}{1.483088in}}{\pgfqpoint{1.079389in}{1.484391in}}%
\pgfpathcurveto{\pgfqpoint{1.078087in}{1.485693in}}{\pgfqpoint{1.076320in}{1.486425in}}{\pgfqpoint{1.074479in}{1.486425in}}%
\pgfpathcurveto{\pgfqpoint{1.072637in}{1.486425in}}{\pgfqpoint{1.070871in}{1.485693in}}{\pgfqpoint{1.069568in}{1.484391in}}%
\pgfpathcurveto{\pgfqpoint{1.068266in}{1.483088in}}{\pgfqpoint{1.067534in}{1.481322in}}{\pgfqpoint{1.067534in}{1.479480in}}%
\pgfpathcurveto{\pgfqpoint{1.067534in}{1.477639in}}{\pgfqpoint{1.068266in}{1.475872in}}{\pgfqpoint{1.069568in}{1.474570in}}%
\pgfpathcurveto{\pgfqpoint{1.070871in}{1.473268in}}{\pgfqpoint{1.072637in}{1.472536in}}{\pgfqpoint{1.074479in}{1.472536in}}%
\pgfpathlineto{\pgfqpoint{1.074479in}{1.472536in}}%
\pgfpathclose%
\pgfusepath{stroke,fill}%
\end{pgfscope}%
\begin{pgfscope}%
\pgfpathrectangle{\pgfqpoint{0.661006in}{0.524170in}}{\pgfqpoint{4.194036in}{1.071446in}}%
\pgfusepath{clip}%
\pgfsetbuttcap%
\pgfsetroundjoin%
\definecolor{currentfill}{rgb}{0.702740,0.813155,0.887742}%
\pgfsetfillcolor{currentfill}%
\pgfsetfillopacity{0.700000}%
\pgfsetlinewidth{1.003750pt}%
\definecolor{currentstroke}{rgb}{0.702740,0.813155,0.887742}%
\pgfsetstrokecolor{currentstroke}%
\pgfsetstrokeopacity{0.700000}%
\pgfsetdash{}{0pt}%
\pgfpathmoveto{\pgfqpoint{1.072922in}{1.474585in}}%
\pgfpathcurveto{\pgfqpoint{1.074763in}{1.474585in}}{\pgfqpoint{1.076530in}{1.475317in}}{\pgfqpoint{1.077832in}{1.476619in}}%
\pgfpathcurveto{\pgfqpoint{1.079135in}{1.477922in}}{\pgfqpoint{1.079866in}{1.479688in}}{\pgfqpoint{1.079866in}{1.481530in}}%
\pgfpathcurveto{\pgfqpoint{1.079866in}{1.483372in}}{\pgfqpoint{1.079135in}{1.485138in}}{\pgfqpoint{1.077832in}{1.486440in}}%
\pgfpathcurveto{\pgfqpoint{1.076530in}{1.487743in}}{\pgfqpoint{1.074763in}{1.488474in}}{\pgfqpoint{1.072922in}{1.488474in}}%
\pgfpathcurveto{\pgfqpoint{1.071080in}{1.488474in}}{\pgfqpoint{1.069314in}{1.487743in}}{\pgfqpoint{1.068011in}{1.486440in}}%
\pgfpathcurveto{\pgfqpoint{1.066709in}{1.485138in}}{\pgfqpoint{1.065977in}{1.483372in}}{\pgfqpoint{1.065977in}{1.481530in}}%
\pgfpathcurveto{\pgfqpoint{1.065977in}{1.479688in}}{\pgfqpoint{1.066709in}{1.477922in}}{\pgfqpoint{1.068011in}{1.476619in}}%
\pgfpathcurveto{\pgfqpoint{1.069314in}{1.475317in}}{\pgfqpoint{1.071080in}{1.474585in}}{\pgfqpoint{1.072922in}{1.474585in}}%
\pgfpathlineto{\pgfqpoint{1.072922in}{1.474585in}}%
\pgfpathclose%
\pgfusepath{stroke,fill}%
\end{pgfscope}%
\begin{pgfscope}%
\pgfpathrectangle{\pgfqpoint{0.661006in}{0.524170in}}{\pgfqpoint{4.194036in}{1.071446in}}%
\pgfusepath{clip}%
\pgfsetbuttcap%
\pgfsetroundjoin%
\definecolor{currentfill}{rgb}{0.698191,0.809203,0.885695}%
\pgfsetfillcolor{currentfill}%
\pgfsetfillopacity{0.700000}%
\pgfsetlinewidth{1.003750pt}%
\definecolor{currentstroke}{rgb}{0.698191,0.809203,0.885695}%
\pgfsetstrokecolor{currentstroke}%
\pgfsetstrokeopacity{0.700000}%
\pgfsetdash{}{0pt}%
\pgfpathmoveto{\pgfqpoint{1.051310in}{1.477239in}}%
\pgfpathcurveto{\pgfqpoint{1.053152in}{1.477239in}}{\pgfqpoint{1.054918in}{1.477971in}}{\pgfqpoint{1.056220in}{1.479273in}}%
\pgfpathcurveto{\pgfqpoint{1.057523in}{1.480576in}}{\pgfqpoint{1.058254in}{1.482342in}}{\pgfqpoint{1.058254in}{1.484184in}}%
\pgfpathcurveto{\pgfqpoint{1.058254in}{1.486026in}}{\pgfqpoint{1.057523in}{1.487792in}}{\pgfqpoint{1.056220in}{1.489094in}}%
\pgfpathcurveto{\pgfqpoint{1.054918in}{1.490397in}}{\pgfqpoint{1.053152in}{1.491128in}}{\pgfqpoint{1.051310in}{1.491128in}}%
\pgfpathcurveto{\pgfqpoint{1.049468in}{1.491128in}}{\pgfqpoint{1.047702in}{1.490397in}}{\pgfqpoint{1.046399in}{1.489094in}}%
\pgfpathcurveto{\pgfqpoint{1.045097in}{1.487792in}}{\pgfqpoint{1.044365in}{1.486026in}}{\pgfqpoint{1.044365in}{1.484184in}}%
\pgfpathcurveto{\pgfqpoint{1.044365in}{1.482342in}}{\pgfqpoint{1.045097in}{1.480576in}}{\pgfqpoint{1.046399in}{1.479273in}}%
\pgfpathcurveto{\pgfqpoint{1.047702in}{1.477971in}}{\pgfqpoint{1.049468in}{1.477239in}}{\pgfqpoint{1.051310in}{1.477239in}}%
\pgfpathlineto{\pgfqpoint{1.051310in}{1.477239in}}%
\pgfpathclose%
\pgfusepath{stroke,fill}%
\end{pgfscope}%
\begin{pgfscope}%
\pgfpathrectangle{\pgfqpoint{0.661006in}{0.524170in}}{\pgfqpoint{4.194036in}{1.071446in}}%
\pgfusepath{clip}%
\pgfsetbuttcap%
\pgfsetroundjoin%
\definecolor{currentfill}{rgb}{0.698191,0.809203,0.885695}%
\pgfsetfillcolor{currentfill}%
\pgfsetfillopacity{0.700000}%
\pgfsetlinewidth{1.003750pt}%
\definecolor{currentstroke}{rgb}{0.698191,0.809203,0.885695}%
\pgfsetstrokecolor{currentstroke}%
\pgfsetstrokeopacity{0.700000}%
\pgfsetdash{}{0pt}%
\pgfpathmoveto{\pgfqpoint{1.061907in}{1.474168in}}%
\pgfpathcurveto{\pgfqpoint{1.063748in}{1.474168in}}{\pgfqpoint{1.065515in}{1.474900in}}{\pgfqpoint{1.066817in}{1.476202in}}%
\pgfpathcurveto{\pgfqpoint{1.068119in}{1.477504in}}{\pgfqpoint{1.068851in}{1.479271in}}{\pgfqpoint{1.068851in}{1.481112in}}%
\pgfpathcurveto{\pgfqpoint{1.068851in}{1.482954in}}{\pgfqpoint{1.068119in}{1.484721in}}{\pgfqpoint{1.066817in}{1.486023in}}%
\pgfpathcurveto{\pgfqpoint{1.065515in}{1.487325in}}{\pgfqpoint{1.063748in}{1.488057in}}{\pgfqpoint{1.061907in}{1.488057in}}%
\pgfpathcurveto{\pgfqpoint{1.060065in}{1.488057in}}{\pgfqpoint{1.058299in}{1.487325in}}{\pgfqpoint{1.056996in}{1.486023in}}%
\pgfpathcurveto{\pgfqpoint{1.055694in}{1.484721in}}{\pgfqpoint{1.054962in}{1.482954in}}{\pgfqpoint{1.054962in}{1.481112in}}%
\pgfpathcurveto{\pgfqpoint{1.054962in}{1.479271in}}{\pgfqpoint{1.055694in}{1.477504in}}{\pgfqpoint{1.056996in}{1.476202in}}%
\pgfpathcurveto{\pgfqpoint{1.058299in}{1.474900in}}{\pgfqpoint{1.060065in}{1.474168in}}{\pgfqpoint{1.061907in}{1.474168in}}%
\pgfpathlineto{\pgfqpoint{1.061907in}{1.474168in}}%
\pgfpathclose%
\pgfusepath{stroke,fill}%
\end{pgfscope}%
\begin{pgfscope}%
\pgfpathrectangle{\pgfqpoint{0.661006in}{0.524170in}}{\pgfqpoint{4.194036in}{1.071446in}}%
\pgfusepath{clip}%
\pgfsetbuttcap%
\pgfsetroundjoin%
\definecolor{currentfill}{rgb}{0.698191,0.809203,0.885695}%
\pgfsetfillcolor{currentfill}%
\pgfsetfillopacity{0.700000}%
\pgfsetlinewidth{1.003750pt}%
\definecolor{currentstroke}{rgb}{0.698191,0.809203,0.885695}%
\pgfsetstrokecolor{currentstroke}%
\pgfsetstrokeopacity{0.700000}%
\pgfsetdash{}{0pt}%
\pgfpathmoveto{\pgfqpoint{1.098577in}{1.464277in}}%
\pgfpathcurveto{\pgfqpoint{1.100419in}{1.464277in}}{\pgfqpoint{1.102185in}{1.465009in}}{\pgfqpoint{1.103488in}{1.466311in}}%
\pgfpathcurveto{\pgfqpoint{1.104790in}{1.467613in}}{\pgfqpoint{1.105522in}{1.469380in}}{\pgfqpoint{1.105522in}{1.471221in}}%
\pgfpathcurveto{\pgfqpoint{1.105522in}{1.473063in}}{\pgfqpoint{1.104790in}{1.474830in}}{\pgfqpoint{1.103488in}{1.476132in}}%
\pgfpathcurveto{\pgfqpoint{1.102185in}{1.477434in}}{\pgfqpoint{1.100419in}{1.478166in}}{\pgfqpoint{1.098577in}{1.478166in}}%
\pgfpathcurveto{\pgfqpoint{1.096736in}{1.478166in}}{\pgfqpoint{1.094969in}{1.477434in}}{\pgfqpoint{1.093667in}{1.476132in}}%
\pgfpathcurveto{\pgfqpoint{1.092365in}{1.474830in}}{\pgfqpoint{1.091633in}{1.473063in}}{\pgfqpoint{1.091633in}{1.471221in}}%
\pgfpathcurveto{\pgfqpoint{1.091633in}{1.469380in}}{\pgfqpoint{1.092365in}{1.467613in}}{\pgfqpoint{1.093667in}{1.466311in}}%
\pgfpathcurveto{\pgfqpoint{1.094969in}{1.465009in}}{\pgfqpoint{1.096736in}{1.464277in}}{\pgfqpoint{1.098577in}{1.464277in}}%
\pgfpathlineto{\pgfqpoint{1.098577in}{1.464277in}}%
\pgfpathclose%
\pgfusepath{stroke,fill}%
\end{pgfscope}%
\begin{pgfscope}%
\pgfpathrectangle{\pgfqpoint{0.661006in}{0.524170in}}{\pgfqpoint{4.194036in}{1.071446in}}%
\pgfusepath{clip}%
\pgfsetbuttcap%
\pgfsetroundjoin%
\definecolor{currentfill}{rgb}{0.698191,0.809203,0.885695}%
\pgfsetfillcolor{currentfill}%
\pgfsetfillopacity{0.700000}%
\pgfsetlinewidth{1.003750pt}%
\definecolor{currentstroke}{rgb}{0.698191,0.809203,0.885695}%
\pgfsetstrokecolor{currentstroke}%
\pgfsetstrokeopacity{0.700000}%
\pgfsetdash{}{0pt}%
\pgfpathmoveto{\pgfqpoint{1.145333in}{1.453202in}}%
\pgfpathcurveto{\pgfqpoint{1.147175in}{1.453202in}}{\pgfqpoint{1.148942in}{1.453934in}}{\pgfqpoint{1.150244in}{1.455236in}}%
\pgfpathcurveto{\pgfqpoint{1.151546in}{1.456539in}}{\pgfqpoint{1.152278in}{1.458305in}}{\pgfqpoint{1.152278in}{1.460147in}}%
\pgfpathcurveto{\pgfqpoint{1.152278in}{1.461989in}}{\pgfqpoint{1.151546in}{1.463755in}}{\pgfqpoint{1.150244in}{1.465057in}}%
\pgfpathcurveto{\pgfqpoint{1.148942in}{1.466360in}}{\pgfqpoint{1.147175in}{1.467091in}}{\pgfqpoint{1.145333in}{1.467091in}}%
\pgfpathcurveto{\pgfqpoint{1.143492in}{1.467091in}}{\pgfqpoint{1.141725in}{1.466360in}}{\pgfqpoint{1.140423in}{1.465057in}}%
\pgfpathcurveto{\pgfqpoint{1.139121in}{1.463755in}}{\pgfqpoint{1.138389in}{1.461989in}}{\pgfqpoint{1.138389in}{1.460147in}}%
\pgfpathcurveto{\pgfqpoint{1.138389in}{1.458305in}}{\pgfqpoint{1.139121in}{1.456539in}}{\pgfqpoint{1.140423in}{1.455236in}}%
\pgfpathcurveto{\pgfqpoint{1.141725in}{1.453934in}}{\pgfqpoint{1.143492in}{1.453202in}}{\pgfqpoint{1.145333in}{1.453202in}}%
\pgfpathlineto{\pgfqpoint{1.145333in}{1.453202in}}%
\pgfpathclose%
\pgfusepath{stroke,fill}%
\end{pgfscope}%
\begin{pgfscope}%
\pgfpathrectangle{\pgfqpoint{0.661006in}{0.524170in}}{\pgfqpoint{4.194036in}{1.071446in}}%
\pgfusepath{clip}%
\pgfsetbuttcap%
\pgfsetroundjoin%
\definecolor{currentfill}{rgb}{0.693674,0.805236,0.883641}%
\pgfsetfillcolor{currentfill}%
\pgfsetfillopacity{0.700000}%
\pgfsetlinewidth{1.003750pt}%
\definecolor{currentstroke}{rgb}{0.693674,0.805236,0.883641}%
\pgfsetstrokecolor{currentstroke}%
\pgfsetstrokeopacity{0.700000}%
\pgfsetdash{}{0pt}%
\pgfpathmoveto{\pgfqpoint{1.193391in}{1.441086in}}%
\pgfpathcurveto{\pgfqpoint{1.195232in}{1.441086in}}{\pgfqpoint{1.196999in}{1.441818in}}{\pgfqpoint{1.198301in}{1.443120in}}%
\pgfpathcurveto{\pgfqpoint{1.199604in}{1.444423in}}{\pgfqpoint{1.200335in}{1.446189in}}{\pgfqpoint{1.200335in}{1.448031in}}%
\pgfpathcurveto{\pgfqpoint{1.200335in}{1.449873in}}{\pgfqpoint{1.199604in}{1.451639in}}{\pgfqpoint{1.198301in}{1.452941in}}%
\pgfpathcurveto{\pgfqpoint{1.196999in}{1.454244in}}{\pgfqpoint{1.195232in}{1.454975in}}{\pgfqpoint{1.193391in}{1.454975in}}%
\pgfpathcurveto{\pgfqpoint{1.191549in}{1.454975in}}{\pgfqpoint{1.189783in}{1.454244in}}{\pgfqpoint{1.188480in}{1.452941in}}%
\pgfpathcurveto{\pgfqpoint{1.187178in}{1.451639in}}{\pgfqpoint{1.186446in}{1.449873in}}{\pgfqpoint{1.186446in}{1.448031in}}%
\pgfpathcurveto{\pgfqpoint{1.186446in}{1.446189in}}{\pgfqpoint{1.187178in}{1.444423in}}{\pgfqpoint{1.188480in}{1.443120in}}%
\pgfpathcurveto{\pgfqpoint{1.189783in}{1.441818in}}{\pgfqpoint{1.191549in}{1.441086in}}{\pgfqpoint{1.193391in}{1.441086in}}%
\pgfpathlineto{\pgfqpoint{1.193391in}{1.441086in}}%
\pgfpathclose%
\pgfusepath{stroke,fill}%
\end{pgfscope}%
\begin{pgfscope}%
\pgfpathrectangle{\pgfqpoint{0.661006in}{0.524170in}}{\pgfqpoint{4.194036in}{1.071446in}}%
\pgfusepath{clip}%
\pgfsetbuttcap%
\pgfsetroundjoin%
\definecolor{currentfill}{rgb}{0.693674,0.805236,0.883641}%
\pgfsetfillcolor{currentfill}%
\pgfsetfillopacity{0.700000}%
\pgfsetlinewidth{1.003750pt}%
\definecolor{currentstroke}{rgb}{0.693674,0.805236,0.883641}%
\pgfsetstrokecolor{currentstroke}%
\pgfsetstrokeopacity{0.700000}%
\pgfsetdash{}{0pt}%
\pgfpathmoveto{\pgfqpoint{1.252463in}{1.426639in}}%
\pgfpathcurveto{\pgfqpoint{1.254305in}{1.426639in}}{\pgfqpoint{1.256072in}{1.427371in}}{\pgfqpoint{1.257374in}{1.428673in}}%
\pgfpathcurveto{\pgfqpoint{1.258676in}{1.429975in}}{\pgfqpoint{1.259408in}{1.431742in}}{\pgfqpoint{1.259408in}{1.433583in}}%
\pgfpathcurveto{\pgfqpoint{1.259408in}{1.435425in}}{\pgfqpoint{1.258676in}{1.437192in}}{\pgfqpoint{1.257374in}{1.438494in}}%
\pgfpathcurveto{\pgfqpoint{1.256072in}{1.439796in}}{\pgfqpoint{1.254305in}{1.440528in}}{\pgfqpoint{1.252463in}{1.440528in}}%
\pgfpathcurveto{\pgfqpoint{1.250622in}{1.440528in}}{\pgfqpoint{1.248855in}{1.439796in}}{\pgfqpoint{1.247553in}{1.438494in}}%
\pgfpathcurveto{\pgfqpoint{1.246251in}{1.437192in}}{\pgfqpoint{1.245519in}{1.435425in}}{\pgfqpoint{1.245519in}{1.433583in}}%
\pgfpathcurveto{\pgfqpoint{1.245519in}{1.431742in}}{\pgfqpoint{1.246251in}{1.429975in}}{\pgfqpoint{1.247553in}{1.428673in}}%
\pgfpathcurveto{\pgfqpoint{1.248855in}{1.427371in}}{\pgfqpoint{1.250622in}{1.426639in}}{\pgfqpoint{1.252463in}{1.426639in}}%
\pgfpathlineto{\pgfqpoint{1.252463in}{1.426639in}}%
\pgfpathclose%
\pgfusepath{stroke,fill}%
\end{pgfscope}%
\begin{pgfscope}%
\pgfpathrectangle{\pgfqpoint{0.661006in}{0.524170in}}{\pgfqpoint{4.194036in}{1.071446in}}%
\pgfusepath{clip}%
\pgfsetbuttcap%
\pgfsetroundjoin%
\definecolor{currentfill}{rgb}{0.693674,0.805236,0.883641}%
\pgfsetfillcolor{currentfill}%
\pgfsetfillopacity{0.700000}%
\pgfsetlinewidth{1.003750pt}%
\definecolor{currentstroke}{rgb}{0.693674,0.805236,0.883641}%
\pgfsetstrokecolor{currentstroke}%
\pgfsetstrokeopacity{0.700000}%
\pgfsetdash{}{0pt}%
\pgfpathmoveto{\pgfqpoint{1.323341in}{1.410516in}}%
\pgfpathcurveto{\pgfqpoint{1.325183in}{1.410516in}}{\pgfqpoint{1.326949in}{1.411248in}}{\pgfqpoint{1.328252in}{1.412550in}}%
\pgfpathcurveto{\pgfqpoint{1.329554in}{1.413852in}}{\pgfqpoint{1.330286in}{1.415619in}}{\pgfqpoint{1.330286in}{1.417460in}}%
\pgfpathcurveto{\pgfqpoint{1.330286in}{1.419302in}}{\pgfqpoint{1.329554in}{1.421069in}}{\pgfqpoint{1.328252in}{1.422371in}}%
\pgfpathcurveto{\pgfqpoint{1.326949in}{1.423673in}}{\pgfqpoint{1.325183in}{1.424405in}}{\pgfqpoint{1.323341in}{1.424405in}}%
\pgfpathcurveto{\pgfqpoint{1.321499in}{1.424405in}}{\pgfqpoint{1.319733in}{1.423673in}}{\pgfqpoint{1.318431in}{1.422371in}}%
\pgfpathcurveto{\pgfqpoint{1.317128in}{1.421069in}}{\pgfqpoint{1.316397in}{1.419302in}}{\pgfqpoint{1.316397in}{1.417460in}}%
\pgfpathcurveto{\pgfqpoint{1.316397in}{1.415619in}}{\pgfqpoint{1.317128in}{1.413852in}}{\pgfqpoint{1.318431in}{1.412550in}}%
\pgfpathcurveto{\pgfqpoint{1.319733in}{1.411248in}}{\pgfqpoint{1.321499in}{1.410516in}}{\pgfqpoint{1.323341in}{1.410516in}}%
\pgfpathlineto{\pgfqpoint{1.323341in}{1.410516in}}%
\pgfpathclose%
\pgfusepath{stroke,fill}%
\end{pgfscope}%
\begin{pgfscope}%
\pgfpathrectangle{\pgfqpoint{0.661006in}{0.524170in}}{\pgfqpoint{4.194036in}{1.071446in}}%
\pgfusepath{clip}%
\pgfsetbuttcap%
\pgfsetroundjoin%
\definecolor{currentfill}{rgb}{0.689191,0.801254,0.881578}%
\pgfsetfillcolor{currentfill}%
\pgfsetfillopacity{0.700000}%
\pgfsetlinewidth{1.003750pt}%
\definecolor{currentstroke}{rgb}{0.689191,0.801254,0.881578}%
\pgfsetstrokecolor{currentstroke}%
\pgfsetstrokeopacity{0.700000}%
\pgfsetdash{}{0pt}%
\pgfpathmoveto{\pgfqpoint{1.389246in}{1.395124in}}%
\pgfpathcurveto{\pgfqpoint{1.391088in}{1.395124in}}{\pgfqpoint{1.392854in}{1.395856in}}{\pgfqpoint{1.394156in}{1.397158in}}%
\pgfpathcurveto{\pgfqpoint{1.395459in}{1.398460in}}{\pgfqpoint{1.396190in}{1.400227in}}{\pgfqpoint{1.396190in}{1.402069in}}%
\pgfpathcurveto{\pgfqpoint{1.396190in}{1.403910in}}{\pgfqpoint{1.395459in}{1.405677in}}{\pgfqpoint{1.394156in}{1.406979in}}%
\pgfpathcurveto{\pgfqpoint{1.392854in}{1.408281in}}{\pgfqpoint{1.391088in}{1.409013in}}{\pgfqpoint{1.389246in}{1.409013in}}%
\pgfpathcurveto{\pgfqpoint{1.387404in}{1.409013in}}{\pgfqpoint{1.385638in}{1.408281in}}{\pgfqpoint{1.384335in}{1.406979in}}%
\pgfpathcurveto{\pgfqpoint{1.383033in}{1.405677in}}{\pgfqpoint{1.382301in}{1.403910in}}{\pgfqpoint{1.382301in}{1.402069in}}%
\pgfpathcurveto{\pgfqpoint{1.382301in}{1.400227in}}{\pgfqpoint{1.383033in}{1.398460in}}{\pgfqpoint{1.384335in}{1.397158in}}%
\pgfpathcurveto{\pgfqpoint{1.385638in}{1.395856in}}{\pgfqpoint{1.387404in}{1.395124in}}{\pgfqpoint{1.389246in}{1.395124in}}%
\pgfpathlineto{\pgfqpoint{1.389246in}{1.395124in}}%
\pgfpathclose%
\pgfusepath{stroke,fill}%
\end{pgfscope}%
\begin{pgfscope}%
\pgfpathrectangle{\pgfqpoint{0.661006in}{0.524170in}}{\pgfqpoint{4.194036in}{1.071446in}}%
\pgfusepath{clip}%
\pgfsetbuttcap%
\pgfsetroundjoin%
\definecolor{currentfill}{rgb}{0.689191,0.801254,0.881578}%
\pgfsetfillcolor{currentfill}%
\pgfsetfillopacity{0.700000}%
\pgfsetlinewidth{1.003750pt}%
\definecolor{currentstroke}{rgb}{0.689191,0.801254,0.881578}%
\pgfsetstrokecolor{currentstroke}%
\pgfsetstrokeopacity{0.700000}%
\pgfsetdash{}{0pt}%
\pgfpathmoveto{\pgfqpoint{1.449852in}{1.380991in}}%
\pgfpathcurveto{\pgfqpoint{1.451694in}{1.380991in}}{\pgfqpoint{1.453460in}{1.381723in}}{\pgfqpoint{1.454763in}{1.383025in}}%
\pgfpathcurveto{\pgfqpoint{1.456065in}{1.384327in}}{\pgfqpoint{1.456797in}{1.386094in}}{\pgfqpoint{1.456797in}{1.387935in}}%
\pgfpathcurveto{\pgfqpoint{1.456797in}{1.389777in}}{\pgfqpoint{1.456065in}{1.391544in}}{\pgfqpoint{1.454763in}{1.392846in}}%
\pgfpathcurveto{\pgfqpoint{1.453460in}{1.394148in}}{\pgfqpoint{1.451694in}{1.394880in}}{\pgfqpoint{1.449852in}{1.394880in}}%
\pgfpathcurveto{\pgfqpoint{1.448011in}{1.394880in}}{\pgfqpoint{1.446244in}{1.394148in}}{\pgfqpoint{1.444942in}{1.392846in}}%
\pgfpathcurveto{\pgfqpoint{1.443639in}{1.391544in}}{\pgfqpoint{1.442908in}{1.389777in}}{\pgfqpoint{1.442908in}{1.387935in}}%
\pgfpathcurveto{\pgfqpoint{1.442908in}{1.386094in}}{\pgfqpoint{1.443639in}{1.384327in}}{\pgfqpoint{1.444942in}{1.383025in}}%
\pgfpathcurveto{\pgfqpoint{1.446244in}{1.381723in}}{\pgfqpoint{1.448011in}{1.380991in}}{\pgfqpoint{1.449852in}{1.380991in}}%
\pgfpathlineto{\pgfqpoint{1.449852in}{1.380991in}}%
\pgfpathclose%
\pgfusepath{stroke,fill}%
\end{pgfscope}%
\begin{pgfscope}%
\pgfpathrectangle{\pgfqpoint{0.661006in}{0.524170in}}{\pgfqpoint{4.194036in}{1.071446in}}%
\pgfusepath{clip}%
\pgfsetbuttcap%
\pgfsetroundjoin%
\definecolor{currentfill}{rgb}{0.684739,0.797257,0.879506}%
\pgfsetfillcolor{currentfill}%
\pgfsetfillopacity{0.700000}%
\pgfsetlinewidth{1.003750pt}%
\definecolor{currentstroke}{rgb}{0.684739,0.797257,0.879506}%
\pgfsetstrokecolor{currentstroke}%
\pgfsetstrokeopacity{0.700000}%
\pgfsetdash{}{0pt}%
\pgfpathmoveto{\pgfqpoint{1.506415in}{1.368325in}}%
\pgfpathcurveto{\pgfqpoint{1.508257in}{1.368325in}}{\pgfqpoint{1.510023in}{1.369057in}}{\pgfqpoint{1.511325in}{1.370359in}}%
\pgfpathcurveto{\pgfqpoint{1.512628in}{1.371662in}}{\pgfqpoint{1.513359in}{1.373428in}}{\pgfqpoint{1.513359in}{1.375270in}}%
\pgfpathcurveto{\pgfqpoint{1.513359in}{1.377112in}}{\pgfqpoint{1.512628in}{1.378878in}}{\pgfqpoint{1.511325in}{1.380180in}}%
\pgfpathcurveto{\pgfqpoint{1.510023in}{1.381483in}}{\pgfqpoint{1.508257in}{1.382214in}}{\pgfqpoint{1.506415in}{1.382214in}}%
\pgfpathcurveto{\pgfqpoint{1.504573in}{1.382214in}}{\pgfqpoint{1.502807in}{1.381483in}}{\pgfqpoint{1.501505in}{1.380180in}}%
\pgfpathcurveto{\pgfqpoint{1.500202in}{1.378878in}}{\pgfqpoint{1.499471in}{1.377112in}}{\pgfqpoint{1.499471in}{1.375270in}}%
\pgfpathcurveto{\pgfqpoint{1.499471in}{1.373428in}}{\pgfqpoint{1.500202in}{1.371662in}}{\pgfqpoint{1.501505in}{1.370359in}}%
\pgfpathcurveto{\pgfqpoint{1.502807in}{1.369057in}}{\pgfqpoint{1.504573in}{1.368325in}}{\pgfqpoint{1.506415in}{1.368325in}}%
\pgfpathlineto{\pgfqpoint{1.506415in}{1.368325in}}%
\pgfpathclose%
\pgfusepath{stroke,fill}%
\end{pgfscope}%
\begin{pgfscope}%
\pgfpathrectangle{\pgfqpoint{0.661006in}{0.524170in}}{\pgfqpoint{4.194036in}{1.071446in}}%
\pgfusepath{clip}%
\pgfsetbuttcap%
\pgfsetroundjoin%
\definecolor{currentfill}{rgb}{0.684739,0.797257,0.879506}%
\pgfsetfillcolor{currentfill}%
\pgfsetfillopacity{0.700000}%
\pgfsetlinewidth{1.003750pt}%
\definecolor{currentstroke}{rgb}{0.684739,0.797257,0.879506}%
\pgfsetstrokecolor{currentstroke}%
\pgfsetstrokeopacity{0.700000}%
\pgfsetdash{}{0pt}%
\pgfpathmoveto{\pgfqpoint{1.545595in}{1.359000in}}%
\pgfpathcurveto{\pgfqpoint{1.547437in}{1.359000in}}{\pgfqpoint{1.549204in}{1.359731in}}{\pgfqpoint{1.550506in}{1.361034in}}%
\pgfpathcurveto{\pgfqpoint{1.551808in}{1.362336in}}{\pgfqpoint{1.552540in}{1.364103in}}{\pgfqpoint{1.552540in}{1.365944in}}%
\pgfpathcurveto{\pgfqpoint{1.552540in}{1.367786in}}{\pgfqpoint{1.551808in}{1.369552in}}{\pgfqpoint{1.550506in}{1.370855in}}%
\pgfpathcurveto{\pgfqpoint{1.549204in}{1.372157in}}{\pgfqpoint{1.547437in}{1.372889in}}{\pgfqpoint{1.545595in}{1.372889in}}%
\pgfpathcurveto{\pgfqpoint{1.543754in}{1.372889in}}{\pgfqpoint{1.541987in}{1.372157in}}{\pgfqpoint{1.540685in}{1.370855in}}%
\pgfpathcurveto{\pgfqpoint{1.539383in}{1.369552in}}{\pgfqpoint{1.538651in}{1.367786in}}{\pgfqpoint{1.538651in}{1.365944in}}%
\pgfpathcurveto{\pgfqpoint{1.538651in}{1.364103in}}{\pgfqpoint{1.539383in}{1.362336in}}{\pgfqpoint{1.540685in}{1.361034in}}%
\pgfpathcurveto{\pgfqpoint{1.541987in}{1.359731in}}{\pgfqpoint{1.543754in}{1.359000in}}{\pgfqpoint{1.545595in}{1.359000in}}%
\pgfpathlineto{\pgfqpoint{1.545595in}{1.359000in}}%
\pgfpathclose%
\pgfusepath{stroke,fill}%
\end{pgfscope}%
\begin{pgfscope}%
\pgfpathrectangle{\pgfqpoint{0.661006in}{0.524170in}}{\pgfqpoint{4.194036in}{1.071446in}}%
\pgfusepath{clip}%
\pgfsetbuttcap%
\pgfsetroundjoin%
\definecolor{currentfill}{rgb}{0.684739,0.797257,0.879506}%
\pgfsetfillcolor{currentfill}%
\pgfsetfillopacity{0.700000}%
\pgfsetlinewidth{1.003750pt}%
\definecolor{currentstroke}{rgb}{0.684739,0.797257,0.879506}%
\pgfsetstrokecolor{currentstroke}%
\pgfsetstrokeopacity{0.700000}%
\pgfsetdash{}{0pt}%
\pgfpathmoveto{\pgfqpoint{1.569710in}{1.353473in}}%
\pgfpathcurveto{\pgfqpoint{1.571552in}{1.353473in}}{\pgfqpoint{1.573319in}{1.354205in}}{\pgfqpoint{1.574621in}{1.355507in}}%
\pgfpathcurveto{\pgfqpoint{1.575923in}{1.356809in}}{\pgfqpoint{1.576655in}{1.358576in}}{\pgfqpoint{1.576655in}{1.360417in}}%
\pgfpathcurveto{\pgfqpoint{1.576655in}{1.362259in}}{\pgfqpoint{1.575923in}{1.364026in}}{\pgfqpoint{1.574621in}{1.365328in}}%
\pgfpathcurveto{\pgfqpoint{1.573319in}{1.366630in}}{\pgfqpoint{1.571552in}{1.367362in}}{\pgfqpoint{1.569710in}{1.367362in}}%
\pgfpathcurveto{\pgfqpoint{1.567869in}{1.367362in}}{\pgfqpoint{1.566102in}{1.366630in}}{\pgfqpoint{1.564800in}{1.365328in}}%
\pgfpathcurveto{\pgfqpoint{1.563498in}{1.364026in}}{\pgfqpoint{1.562766in}{1.362259in}}{\pgfqpoint{1.562766in}{1.360417in}}%
\pgfpathcurveto{\pgfqpoint{1.562766in}{1.358576in}}{\pgfqpoint{1.563498in}{1.356809in}}{\pgfqpoint{1.564800in}{1.355507in}}%
\pgfpathcurveto{\pgfqpoint{1.566102in}{1.354205in}}{\pgfqpoint{1.567869in}{1.353473in}}{\pgfqpoint{1.569710in}{1.353473in}}%
\pgfpathlineto{\pgfqpoint{1.569710in}{1.353473in}}%
\pgfpathclose%
\pgfusepath{stroke,fill}%
\end{pgfscope}%
\begin{pgfscope}%
\pgfpathrectangle{\pgfqpoint{0.661006in}{0.524170in}}{\pgfqpoint{4.194036in}{1.071446in}}%
\pgfusepath{clip}%
\pgfsetbuttcap%
\pgfsetroundjoin%
\definecolor{currentfill}{rgb}{0.684739,0.797257,0.879506}%
\pgfsetfillcolor{currentfill}%
\pgfsetfillopacity{0.700000}%
\pgfsetlinewidth{1.003750pt}%
\definecolor{currentstroke}{rgb}{0.684739,0.797257,0.879506}%
\pgfsetstrokecolor{currentstroke}%
\pgfsetstrokeopacity{0.700000}%
\pgfsetdash{}{0pt}%
\pgfpathmoveto{\pgfqpoint{1.585101in}{1.352599in}}%
\pgfpathcurveto{\pgfqpoint{1.586943in}{1.352599in}}{\pgfqpoint{1.588709in}{1.353330in}}{\pgfqpoint{1.590011in}{1.354633in}}%
\pgfpathcurveto{\pgfqpoint{1.591314in}{1.355935in}}{\pgfqpoint{1.592045in}{1.357701in}}{\pgfqpoint{1.592045in}{1.359543in}}%
\pgfpathcurveto{\pgfqpoint{1.592045in}{1.361385in}}{\pgfqpoint{1.591314in}{1.363151in}}{\pgfqpoint{1.590011in}{1.364453in}}%
\pgfpathcurveto{\pgfqpoint{1.588709in}{1.365756in}}{\pgfqpoint{1.586943in}{1.366487in}}{\pgfqpoint{1.585101in}{1.366487in}}%
\pgfpathcurveto{\pgfqpoint{1.583259in}{1.366487in}}{\pgfqpoint{1.581493in}{1.365756in}}{\pgfqpoint{1.580191in}{1.364453in}}%
\pgfpathcurveto{\pgfqpoint{1.578888in}{1.363151in}}{\pgfqpoint{1.578157in}{1.361385in}}{\pgfqpoint{1.578157in}{1.359543in}}%
\pgfpathcurveto{\pgfqpoint{1.578157in}{1.357701in}}{\pgfqpoint{1.578888in}{1.355935in}}{\pgfqpoint{1.580191in}{1.354633in}}%
\pgfpathcurveto{\pgfqpoint{1.581493in}{1.353330in}}{\pgfqpoint{1.583259in}{1.352599in}}{\pgfqpoint{1.585101in}{1.352599in}}%
\pgfpathlineto{\pgfqpoint{1.585101in}{1.352599in}}%
\pgfpathclose%
\pgfusepath{stroke,fill}%
\end{pgfscope}%
\begin{pgfscope}%
\pgfpathrectangle{\pgfqpoint{0.661006in}{0.524170in}}{\pgfqpoint{4.194036in}{1.071446in}}%
\pgfusepath{clip}%
\pgfsetbuttcap%
\pgfsetroundjoin%
\definecolor{currentfill}{rgb}{0.684739,0.797257,0.879506}%
\pgfsetfillcolor{currentfill}%
\pgfsetfillopacity{0.700000}%
\pgfsetlinewidth{1.003750pt}%
\definecolor{currentstroke}{rgb}{0.684739,0.797257,0.879506}%
\pgfsetstrokecolor{currentstroke}%
\pgfsetstrokeopacity{0.700000}%
\pgfsetdash{}{0pt}%
\pgfpathmoveto{\pgfqpoint{1.572506in}{1.353888in}}%
\pgfpathcurveto{\pgfqpoint{1.574347in}{1.353888in}}{\pgfqpoint{1.576114in}{1.354620in}}{\pgfqpoint{1.577416in}{1.355922in}}%
\pgfpathcurveto{\pgfqpoint{1.578718in}{1.357224in}}{\pgfqpoint{1.579450in}{1.358991in}}{\pgfqpoint{1.579450in}{1.360832in}}%
\pgfpathcurveto{\pgfqpoint{1.579450in}{1.362674in}}{\pgfqpoint{1.578718in}{1.364441in}}{\pgfqpoint{1.577416in}{1.365743in}}%
\pgfpathcurveto{\pgfqpoint{1.576114in}{1.367045in}}{\pgfqpoint{1.574347in}{1.367777in}}{\pgfqpoint{1.572506in}{1.367777in}}%
\pgfpathcurveto{\pgfqpoint{1.570664in}{1.367777in}}{\pgfqpoint{1.568897in}{1.367045in}}{\pgfqpoint{1.567595in}{1.365743in}}%
\pgfpathcurveto{\pgfqpoint{1.566293in}{1.364441in}}{\pgfqpoint{1.565561in}{1.362674in}}{\pgfqpoint{1.565561in}{1.360832in}}%
\pgfpathcurveto{\pgfqpoint{1.565561in}{1.358991in}}{\pgfqpoint{1.566293in}{1.357224in}}{\pgfqpoint{1.567595in}{1.355922in}}%
\pgfpathcurveto{\pgfqpoint{1.568897in}{1.354620in}}{\pgfqpoint{1.570664in}{1.353888in}}{\pgfqpoint{1.572506in}{1.353888in}}%
\pgfpathlineto{\pgfqpoint{1.572506in}{1.353888in}}%
\pgfpathclose%
\pgfusepath{stroke,fill}%
\end{pgfscope}%
\begin{pgfscope}%
\pgfpathrectangle{\pgfqpoint{0.661006in}{0.524170in}}{\pgfqpoint{4.194036in}{1.071446in}}%
\pgfusepath{clip}%
\pgfsetbuttcap%
\pgfsetroundjoin%
\definecolor{currentfill}{rgb}{0.680320,0.793245,0.877425}%
\pgfsetfillcolor{currentfill}%
\pgfsetfillopacity{0.700000}%
\pgfsetlinewidth{1.003750pt}%
\definecolor{currentstroke}{rgb}{0.680320,0.793245,0.877425}%
\pgfsetstrokecolor{currentstroke}%
\pgfsetstrokeopacity{0.700000}%
\pgfsetdash{}{0pt}%
\pgfpathmoveto{\pgfqpoint{1.563721in}{1.355456in}}%
\pgfpathcurveto{\pgfqpoint{1.565563in}{1.355456in}}{\pgfqpoint{1.567330in}{1.356188in}}{\pgfqpoint{1.568632in}{1.357490in}}%
\pgfpathcurveto{\pgfqpoint{1.569934in}{1.358793in}}{\pgfqpoint{1.570666in}{1.360559in}}{\pgfqpoint{1.570666in}{1.362401in}}%
\pgfpathcurveto{\pgfqpoint{1.570666in}{1.364243in}}{\pgfqpoint{1.569934in}{1.366009in}}{\pgfqpoint{1.568632in}{1.367311in}}%
\pgfpathcurveto{\pgfqpoint{1.567330in}{1.368614in}}{\pgfqpoint{1.565563in}{1.369345in}}{\pgfqpoint{1.563721in}{1.369345in}}%
\pgfpathcurveto{\pgfqpoint{1.561880in}{1.369345in}}{\pgfqpoint{1.560113in}{1.368614in}}{\pgfqpoint{1.558811in}{1.367311in}}%
\pgfpathcurveto{\pgfqpoint{1.557509in}{1.366009in}}{\pgfqpoint{1.556777in}{1.364243in}}{\pgfqpoint{1.556777in}{1.362401in}}%
\pgfpathcurveto{\pgfqpoint{1.556777in}{1.360559in}}{\pgfqpoint{1.557509in}{1.358793in}}{\pgfqpoint{1.558811in}{1.357490in}}%
\pgfpathcurveto{\pgfqpoint{1.560113in}{1.356188in}}{\pgfqpoint{1.561880in}{1.355456in}}{\pgfqpoint{1.563721in}{1.355456in}}%
\pgfpathlineto{\pgfqpoint{1.563721in}{1.355456in}}%
\pgfpathclose%
\pgfusepath{stroke,fill}%
\end{pgfscope}%
\begin{pgfscope}%
\pgfpathrectangle{\pgfqpoint{0.661006in}{0.524170in}}{\pgfqpoint{4.194036in}{1.071446in}}%
\pgfusepath{clip}%
\pgfsetbuttcap%
\pgfsetroundjoin%
\definecolor{currentfill}{rgb}{0.680320,0.793245,0.877425}%
\pgfsetfillcolor{currentfill}%
\pgfsetfillopacity{0.700000}%
\pgfsetlinewidth{1.003750pt}%
\definecolor{currentstroke}{rgb}{0.680320,0.793245,0.877425}%
\pgfsetstrokecolor{currentstroke}%
\pgfsetstrokeopacity{0.700000}%
\pgfsetdash{}{0pt}%
\pgfpathmoveto{\pgfqpoint{1.553403in}{1.357402in}}%
\pgfpathcurveto{\pgfqpoint{1.555245in}{1.357402in}}{\pgfqpoint{1.557012in}{1.358134in}}{\pgfqpoint{1.558314in}{1.359436in}}%
\pgfpathcurveto{\pgfqpoint{1.559616in}{1.360739in}}{\pgfqpoint{1.560348in}{1.362505in}}{\pgfqpoint{1.560348in}{1.364347in}}%
\pgfpathcurveto{\pgfqpoint{1.560348in}{1.366189in}}{\pgfqpoint{1.559616in}{1.367955in}}{\pgfqpoint{1.558314in}{1.369257in}}%
\pgfpathcurveto{\pgfqpoint{1.557012in}{1.370560in}}{\pgfqpoint{1.555245in}{1.371291in}}{\pgfqpoint{1.553403in}{1.371291in}}%
\pgfpathcurveto{\pgfqpoint{1.551562in}{1.371291in}}{\pgfqpoint{1.549795in}{1.370560in}}{\pgfqpoint{1.548493in}{1.369257in}}%
\pgfpathcurveto{\pgfqpoint{1.547191in}{1.367955in}}{\pgfqpoint{1.546459in}{1.366189in}}{\pgfqpoint{1.546459in}{1.364347in}}%
\pgfpathcurveto{\pgfqpoint{1.546459in}{1.362505in}}{\pgfqpoint{1.547191in}{1.360739in}}{\pgfqpoint{1.548493in}{1.359436in}}%
\pgfpathcurveto{\pgfqpoint{1.549795in}{1.358134in}}{\pgfqpoint{1.551562in}{1.357402in}}{\pgfqpoint{1.553403in}{1.357402in}}%
\pgfpathlineto{\pgfqpoint{1.553403in}{1.357402in}}%
\pgfpathclose%
\pgfusepath{stroke,fill}%
\end{pgfscope}%
\begin{pgfscope}%
\pgfpathrectangle{\pgfqpoint{0.661006in}{0.524170in}}{\pgfqpoint{4.194036in}{1.071446in}}%
\pgfusepath{clip}%
\pgfsetbuttcap%
\pgfsetroundjoin%
\definecolor{currentfill}{rgb}{0.675932,0.789220,0.875333}%
\pgfsetfillcolor{currentfill}%
\pgfsetfillopacity{0.700000}%
\pgfsetlinewidth{1.003750pt}%
\definecolor{currentstroke}{rgb}{0.675932,0.789220,0.875333}%
\pgfsetstrokecolor{currentstroke}%
\pgfsetstrokeopacity{0.700000}%
\pgfsetdash{}{0pt}%
\pgfpathmoveto{\pgfqpoint{1.550801in}{1.357467in}}%
\pgfpathcurveto{\pgfqpoint{1.552642in}{1.357467in}}{\pgfqpoint{1.554409in}{1.358198in}}{\pgfqpoint{1.555711in}{1.359501in}}%
\pgfpathcurveto{\pgfqpoint{1.557013in}{1.360803in}}{\pgfqpoint{1.557745in}{1.362569in}}{\pgfqpoint{1.557745in}{1.364411in}}%
\pgfpathcurveto{\pgfqpoint{1.557745in}{1.366253in}}{\pgfqpoint{1.557013in}{1.368019in}}{\pgfqpoint{1.555711in}{1.369322in}}%
\pgfpathcurveto{\pgfqpoint{1.554409in}{1.370624in}}{\pgfqpoint{1.552642in}{1.371356in}}{\pgfqpoint{1.550801in}{1.371356in}}%
\pgfpathcurveto{\pgfqpoint{1.548959in}{1.371356in}}{\pgfqpoint{1.547193in}{1.370624in}}{\pgfqpoint{1.545890in}{1.369322in}}%
\pgfpathcurveto{\pgfqpoint{1.544588in}{1.368019in}}{\pgfqpoint{1.543856in}{1.366253in}}{\pgfqpoint{1.543856in}{1.364411in}}%
\pgfpathcurveto{\pgfqpoint{1.543856in}{1.362569in}}{\pgfqpoint{1.544588in}{1.360803in}}{\pgfqpoint{1.545890in}{1.359501in}}%
\pgfpathcurveto{\pgfqpoint{1.547193in}{1.358198in}}{\pgfqpoint{1.548959in}{1.357467in}}{\pgfqpoint{1.550801in}{1.357467in}}%
\pgfpathlineto{\pgfqpoint{1.550801in}{1.357467in}}%
\pgfpathclose%
\pgfusepath{stroke,fill}%
\end{pgfscope}%
\begin{pgfscope}%
\pgfpathrectangle{\pgfqpoint{0.661006in}{0.524170in}}{\pgfqpoint{4.194036in}{1.071446in}}%
\pgfusepath{clip}%
\pgfsetbuttcap%
\pgfsetroundjoin%
\definecolor{currentfill}{rgb}{0.675932,0.789220,0.875333}%
\pgfsetfillcolor{currentfill}%
\pgfsetfillopacity{0.700000}%
\pgfsetlinewidth{1.003750pt}%
\definecolor{currentstroke}{rgb}{0.675932,0.789220,0.875333}%
\pgfsetstrokecolor{currentstroke}%
\pgfsetstrokeopacity{0.700000}%
\pgfsetdash{}{0pt}%
\pgfpathmoveto{\pgfqpoint{1.562374in}{1.355135in}}%
\pgfpathcurveto{\pgfqpoint{1.564215in}{1.355135in}}{\pgfqpoint{1.565982in}{1.355867in}}{\pgfqpoint{1.567284in}{1.357169in}}%
\pgfpathcurveto{\pgfqpoint{1.568586in}{1.358472in}}{\pgfqpoint{1.569318in}{1.360238in}}{\pgfqpoint{1.569318in}{1.362080in}}%
\pgfpathcurveto{\pgfqpoint{1.569318in}{1.363922in}}{\pgfqpoint{1.568586in}{1.365688in}}{\pgfqpoint{1.567284in}{1.366990in}}%
\pgfpathcurveto{\pgfqpoint{1.565982in}{1.368293in}}{\pgfqpoint{1.564215in}{1.369024in}}{\pgfqpoint{1.562374in}{1.369024in}}%
\pgfpathcurveto{\pgfqpoint{1.560532in}{1.369024in}}{\pgfqpoint{1.558765in}{1.368293in}}{\pgfqpoint{1.557463in}{1.366990in}}%
\pgfpathcurveto{\pgfqpoint{1.556161in}{1.365688in}}{\pgfqpoint{1.555429in}{1.363922in}}{\pgfqpoint{1.555429in}{1.362080in}}%
\pgfpathcurveto{\pgfqpoint{1.555429in}{1.360238in}}{\pgfqpoint{1.556161in}{1.358472in}}{\pgfqpoint{1.557463in}{1.357169in}}%
\pgfpathcurveto{\pgfqpoint{1.558765in}{1.355867in}}{\pgfqpoint{1.560532in}{1.355135in}}{\pgfqpoint{1.562374in}{1.355135in}}%
\pgfpathlineto{\pgfqpoint{1.562374in}{1.355135in}}%
\pgfpathclose%
\pgfusepath{stroke,fill}%
\end{pgfscope}%
\begin{pgfscope}%
\pgfpathrectangle{\pgfqpoint{0.661006in}{0.524170in}}{\pgfqpoint{4.194036in}{1.071446in}}%
\pgfusepath{clip}%
\pgfsetbuttcap%
\pgfsetroundjoin%
\definecolor{currentfill}{rgb}{0.675932,0.789220,0.875333}%
\pgfsetfillcolor{currentfill}%
\pgfsetfillopacity{0.700000}%
\pgfsetlinewidth{1.003750pt}%
\definecolor{currentstroke}{rgb}{0.675932,0.789220,0.875333}%
\pgfsetstrokecolor{currentstroke}%
\pgfsetstrokeopacity{0.700000}%
\pgfsetdash{}{0pt}%
\pgfpathmoveto{\pgfqpoint{1.577943in}{1.352454in}}%
\pgfpathcurveto{\pgfqpoint{1.579785in}{1.352454in}}{\pgfqpoint{1.581552in}{1.353185in}}{\pgfqpoint{1.582854in}{1.354488in}}%
\pgfpathcurveto{\pgfqpoint{1.584156in}{1.355790in}}{\pgfqpoint{1.584888in}{1.357556in}}{\pgfqpoint{1.584888in}{1.359398in}}%
\pgfpathcurveto{\pgfqpoint{1.584888in}{1.361240in}}{\pgfqpoint{1.584156in}{1.363006in}}{\pgfqpoint{1.582854in}{1.364309in}}%
\pgfpathcurveto{\pgfqpoint{1.581552in}{1.365611in}}{\pgfqpoint{1.579785in}{1.366343in}}{\pgfqpoint{1.577943in}{1.366343in}}%
\pgfpathcurveto{\pgfqpoint{1.576102in}{1.366343in}}{\pgfqpoint{1.574335in}{1.365611in}}{\pgfqpoint{1.573033in}{1.364309in}}%
\pgfpathcurveto{\pgfqpoint{1.571731in}{1.363006in}}{\pgfqpoint{1.570999in}{1.361240in}}{\pgfqpoint{1.570999in}{1.359398in}}%
\pgfpathcurveto{\pgfqpoint{1.570999in}{1.357556in}}{\pgfqpoint{1.571731in}{1.355790in}}{\pgfqpoint{1.573033in}{1.354488in}}%
\pgfpathcurveto{\pgfqpoint{1.574335in}{1.353185in}}{\pgfqpoint{1.576102in}{1.352454in}}{\pgfqpoint{1.577943in}{1.352454in}}%
\pgfpathlineto{\pgfqpoint{1.577943in}{1.352454in}}%
\pgfpathclose%
\pgfusepath{stroke,fill}%
\end{pgfscope}%
\begin{pgfscope}%
\pgfpathrectangle{\pgfqpoint{0.661006in}{0.524170in}}{\pgfqpoint{4.194036in}{1.071446in}}%
\pgfusepath{clip}%
\pgfsetbuttcap%
\pgfsetroundjoin%
\definecolor{currentfill}{rgb}{0.675932,0.789220,0.875333}%
\pgfsetfillcolor{currentfill}%
\pgfsetfillopacity{0.700000}%
\pgfsetlinewidth{1.003750pt}%
\definecolor{currentstroke}{rgb}{0.675932,0.789220,0.875333}%
\pgfsetstrokecolor{currentstroke}%
\pgfsetstrokeopacity{0.700000}%
\pgfsetdash{}{0pt}%
\pgfpathmoveto{\pgfqpoint{1.586681in}{1.349661in}}%
\pgfpathcurveto{\pgfqpoint{1.588523in}{1.349661in}}{\pgfqpoint{1.590289in}{1.350392in}}{\pgfqpoint{1.591592in}{1.351695in}}%
\pgfpathcurveto{\pgfqpoint{1.592894in}{1.352997in}}{\pgfqpoint{1.593626in}{1.354763in}}{\pgfqpoint{1.593626in}{1.356605in}}%
\pgfpathcurveto{\pgfqpoint{1.593626in}{1.358447in}}{\pgfqpoint{1.592894in}{1.360213in}}{\pgfqpoint{1.591592in}{1.361516in}}%
\pgfpathcurveto{\pgfqpoint{1.590289in}{1.362818in}}{\pgfqpoint{1.588523in}{1.363550in}}{\pgfqpoint{1.586681in}{1.363550in}}%
\pgfpathcurveto{\pgfqpoint{1.584840in}{1.363550in}}{\pgfqpoint{1.583073in}{1.362818in}}{\pgfqpoint{1.581771in}{1.361516in}}%
\pgfpathcurveto{\pgfqpoint{1.580468in}{1.360213in}}{\pgfqpoint{1.579737in}{1.358447in}}{\pgfqpoint{1.579737in}{1.356605in}}%
\pgfpathcurveto{\pgfqpoint{1.579737in}{1.354763in}}{\pgfqpoint{1.580468in}{1.352997in}}{\pgfqpoint{1.581771in}{1.351695in}}%
\pgfpathcurveto{\pgfqpoint{1.583073in}{1.350392in}}{\pgfqpoint{1.584840in}{1.349661in}}{\pgfqpoint{1.586681in}{1.349661in}}%
\pgfpathlineto{\pgfqpoint{1.586681in}{1.349661in}}%
\pgfpathclose%
\pgfusepath{stroke,fill}%
\end{pgfscope}%
\begin{pgfscope}%
\pgfpathrectangle{\pgfqpoint{0.661006in}{0.524170in}}{\pgfqpoint{4.194036in}{1.071446in}}%
\pgfusepath{clip}%
\pgfsetbuttcap%
\pgfsetroundjoin%
\definecolor{currentfill}{rgb}{0.675932,0.789220,0.875333}%
\pgfsetfillcolor{currentfill}%
\pgfsetfillopacity{0.700000}%
\pgfsetlinewidth{1.003750pt}%
\definecolor{currentstroke}{rgb}{0.675932,0.789220,0.875333}%
\pgfsetstrokecolor{currentstroke}%
\pgfsetstrokeopacity{0.700000}%
\pgfsetdash{}{0pt}%
\pgfpathmoveto{\pgfqpoint{1.592444in}{1.346702in}}%
\pgfpathcurveto{\pgfqpoint{1.594286in}{1.346702in}}{\pgfqpoint{1.596053in}{1.347433in}}{\pgfqpoint{1.597355in}{1.348736in}}%
\pgfpathcurveto{\pgfqpoint{1.598657in}{1.350038in}}{\pgfqpoint{1.599389in}{1.351805in}}{\pgfqpoint{1.599389in}{1.353646in}}%
\pgfpathcurveto{\pgfqpoint{1.599389in}{1.355488in}}{\pgfqpoint{1.598657in}{1.357254in}}{\pgfqpoint{1.597355in}{1.358557in}}%
\pgfpathcurveto{\pgfqpoint{1.596053in}{1.359859in}}{\pgfqpoint{1.594286in}{1.360591in}}{\pgfqpoint{1.592444in}{1.360591in}}%
\pgfpathcurveto{\pgfqpoint{1.590603in}{1.360591in}}{\pgfqpoint{1.588836in}{1.359859in}}{\pgfqpoint{1.587534in}{1.358557in}}%
\pgfpathcurveto{\pgfqpoint{1.586232in}{1.357254in}}{\pgfqpoint{1.585500in}{1.355488in}}{\pgfqpoint{1.585500in}{1.353646in}}%
\pgfpathcurveto{\pgfqpoint{1.585500in}{1.351805in}}{\pgfqpoint{1.586232in}{1.350038in}}{\pgfqpoint{1.587534in}{1.348736in}}%
\pgfpathcurveto{\pgfqpoint{1.588836in}{1.347433in}}{\pgfqpoint{1.590603in}{1.346702in}}{\pgfqpoint{1.592444in}{1.346702in}}%
\pgfpathlineto{\pgfqpoint{1.592444in}{1.346702in}}%
\pgfpathclose%
\pgfusepath{stroke,fill}%
\end{pgfscope}%
\begin{pgfscope}%
\pgfpathrectangle{\pgfqpoint{0.661006in}{0.524170in}}{\pgfqpoint{4.194036in}{1.071446in}}%
\pgfusepath{clip}%
\pgfsetbuttcap%
\pgfsetroundjoin%
\definecolor{currentfill}{rgb}{0.671577,0.785179,0.873231}%
\pgfsetfillcolor{currentfill}%
\pgfsetfillopacity{0.700000}%
\pgfsetlinewidth{1.003750pt}%
\definecolor{currentstroke}{rgb}{0.671577,0.785179,0.873231}%
\pgfsetstrokecolor{currentstroke}%
\pgfsetstrokeopacity{0.700000}%
\pgfsetdash{}{0pt}%
\pgfpathmoveto{\pgfqpoint{1.611221in}{1.342942in}}%
\pgfpathcurveto{\pgfqpoint{1.613063in}{1.342942in}}{\pgfqpoint{1.614829in}{1.343674in}}{\pgfqpoint{1.616132in}{1.344976in}}%
\pgfpathcurveto{\pgfqpoint{1.617434in}{1.346279in}}{\pgfqpoint{1.618166in}{1.348045in}}{\pgfqpoint{1.618166in}{1.349887in}}%
\pgfpathcurveto{\pgfqpoint{1.618166in}{1.351729in}}{\pgfqpoint{1.617434in}{1.353495in}}{\pgfqpoint{1.616132in}{1.354797in}}%
\pgfpathcurveto{\pgfqpoint{1.614829in}{1.356100in}}{\pgfqpoint{1.613063in}{1.356831in}}{\pgfqpoint{1.611221in}{1.356831in}}%
\pgfpathcurveto{\pgfqpoint{1.609379in}{1.356831in}}{\pgfqpoint{1.607613in}{1.356100in}}{\pgfqpoint{1.606311in}{1.354797in}}%
\pgfpathcurveto{\pgfqpoint{1.605008in}{1.353495in}}{\pgfqpoint{1.604277in}{1.351729in}}{\pgfqpoint{1.604277in}{1.349887in}}%
\pgfpathcurveto{\pgfqpoint{1.604277in}{1.348045in}}{\pgfqpoint{1.605008in}{1.346279in}}{\pgfqpoint{1.606311in}{1.344976in}}%
\pgfpathcurveto{\pgfqpoint{1.607613in}{1.343674in}}{\pgfqpoint{1.609379in}{1.342942in}}{\pgfqpoint{1.611221in}{1.342942in}}%
\pgfpathlineto{\pgfqpoint{1.611221in}{1.342942in}}%
\pgfpathclose%
\pgfusepath{stroke,fill}%
\end{pgfscope}%
\begin{pgfscope}%
\pgfpathrectangle{\pgfqpoint{0.661006in}{0.524170in}}{\pgfqpoint{4.194036in}{1.071446in}}%
\pgfusepath{clip}%
\pgfsetbuttcap%
\pgfsetroundjoin%
\definecolor{currentfill}{rgb}{0.671577,0.785179,0.873231}%
\pgfsetfillcolor{currentfill}%
\pgfsetfillopacity{0.700000}%
\pgfsetlinewidth{1.003750pt}%
\definecolor{currentstroke}{rgb}{0.671577,0.785179,0.873231}%
\pgfsetstrokecolor{currentstroke}%
\pgfsetstrokeopacity{0.700000}%
\pgfsetdash{}{0pt}%
\pgfpathmoveto{\pgfqpoint{1.628557in}{1.340040in}}%
\pgfpathcurveto{\pgfqpoint{1.630399in}{1.340040in}}{\pgfqpoint{1.632165in}{1.340771in}}{\pgfqpoint{1.633468in}{1.342074in}}%
\pgfpathcurveto{\pgfqpoint{1.634770in}{1.343376in}}{\pgfqpoint{1.635502in}{1.345142in}}{\pgfqpoint{1.635502in}{1.346984in}}%
\pgfpathcurveto{\pgfqpoint{1.635502in}{1.348826in}}{\pgfqpoint{1.634770in}{1.350592in}}{\pgfqpoint{1.633468in}{1.351895in}}%
\pgfpathcurveto{\pgfqpoint{1.632165in}{1.353197in}}{\pgfqpoint{1.630399in}{1.353929in}}{\pgfqpoint{1.628557in}{1.353929in}}%
\pgfpathcurveto{\pgfqpoint{1.626715in}{1.353929in}}{\pgfqpoint{1.624949in}{1.353197in}}{\pgfqpoint{1.623647in}{1.351895in}}%
\pgfpathcurveto{\pgfqpoint{1.622344in}{1.350592in}}{\pgfqpoint{1.621613in}{1.348826in}}{\pgfqpoint{1.621613in}{1.346984in}}%
\pgfpathcurveto{\pgfqpoint{1.621613in}{1.345142in}}{\pgfqpoint{1.622344in}{1.343376in}}{\pgfqpoint{1.623647in}{1.342074in}}%
\pgfpathcurveto{\pgfqpoint{1.624949in}{1.340771in}}{\pgfqpoint{1.626715in}{1.340040in}}{\pgfqpoint{1.628557in}{1.340040in}}%
\pgfpathlineto{\pgfqpoint{1.628557in}{1.340040in}}%
\pgfpathclose%
\pgfusepath{stroke,fill}%
\end{pgfscope}%
\begin{pgfscope}%
\pgfpathrectangle{\pgfqpoint{0.661006in}{0.524170in}}{\pgfqpoint{4.194036in}{1.071446in}}%
\pgfusepath{clip}%
\pgfsetbuttcap%
\pgfsetroundjoin%
\definecolor{currentfill}{rgb}{0.667253,0.781125,0.871118}%
\pgfsetfillcolor{currentfill}%
\pgfsetfillopacity{0.700000}%
\pgfsetlinewidth{1.003750pt}%
\definecolor{currentstroke}{rgb}{0.667253,0.781125,0.871118}%
\pgfsetstrokecolor{currentstroke}%
\pgfsetstrokeopacity{0.700000}%
\pgfsetdash{}{0pt}%
\pgfpathmoveto{\pgfqpoint{1.631253in}{1.337259in}}%
\pgfpathcurveto{\pgfqpoint{1.633095in}{1.337259in}}{\pgfqpoint{1.634861in}{1.337991in}}{\pgfqpoint{1.636163in}{1.339293in}}%
\pgfpathcurveto{\pgfqpoint{1.637466in}{1.340595in}}{\pgfqpoint{1.638197in}{1.342362in}}{\pgfqpoint{1.638197in}{1.344203in}}%
\pgfpathcurveto{\pgfqpoint{1.638197in}{1.346045in}}{\pgfqpoint{1.637466in}{1.347811in}}{\pgfqpoint{1.636163in}{1.349114in}}%
\pgfpathcurveto{\pgfqpoint{1.634861in}{1.350416in}}{\pgfqpoint{1.633095in}{1.351148in}}{\pgfqpoint{1.631253in}{1.351148in}}%
\pgfpathcurveto{\pgfqpoint{1.629411in}{1.351148in}}{\pgfqpoint{1.627645in}{1.350416in}}{\pgfqpoint{1.626342in}{1.349114in}}%
\pgfpathcurveto{\pgfqpoint{1.625040in}{1.347811in}}{\pgfqpoint{1.624308in}{1.346045in}}{\pgfqpoint{1.624308in}{1.344203in}}%
\pgfpathcurveto{\pgfqpoint{1.624308in}{1.342362in}}{\pgfqpoint{1.625040in}{1.340595in}}{\pgfqpoint{1.626342in}{1.339293in}}%
\pgfpathcurveto{\pgfqpoint{1.627645in}{1.337991in}}{\pgfqpoint{1.629411in}{1.337259in}}{\pgfqpoint{1.631253in}{1.337259in}}%
\pgfpathlineto{\pgfqpoint{1.631253in}{1.337259in}}%
\pgfpathclose%
\pgfusepath{stroke,fill}%
\end{pgfscope}%
\begin{pgfscope}%
\pgfpathrectangle{\pgfqpoint{0.661006in}{0.524170in}}{\pgfqpoint{4.194036in}{1.071446in}}%
\pgfusepath{clip}%
\pgfsetbuttcap%
\pgfsetroundjoin%
\definecolor{currentfill}{rgb}{0.667253,0.781125,0.871118}%
\pgfsetfillcolor{currentfill}%
\pgfsetfillopacity{0.700000}%
\pgfsetlinewidth{1.003750pt}%
\definecolor{currentstroke}{rgb}{0.667253,0.781125,0.871118}%
\pgfsetstrokecolor{currentstroke}%
\pgfsetstrokeopacity{0.700000}%
\pgfsetdash{}{0pt}%
\pgfpathmoveto{\pgfqpoint{1.651563in}{1.333592in}}%
\pgfpathcurveto{\pgfqpoint{1.653405in}{1.333592in}}{\pgfqpoint{1.655172in}{1.334324in}}{\pgfqpoint{1.656474in}{1.335626in}}%
\pgfpathcurveto{\pgfqpoint{1.657776in}{1.336929in}}{\pgfqpoint{1.658508in}{1.338695in}}{\pgfqpoint{1.658508in}{1.340537in}}%
\pgfpathcurveto{\pgfqpoint{1.658508in}{1.342379in}}{\pgfqpoint{1.657776in}{1.344145in}}{\pgfqpoint{1.656474in}{1.345447in}}%
\pgfpathcurveto{\pgfqpoint{1.655172in}{1.346750in}}{\pgfqpoint{1.653405in}{1.347481in}}{\pgfqpoint{1.651563in}{1.347481in}}%
\pgfpathcurveto{\pgfqpoint{1.649722in}{1.347481in}}{\pgfqpoint{1.647955in}{1.346750in}}{\pgfqpoint{1.646653in}{1.345447in}}%
\pgfpathcurveto{\pgfqpoint{1.645351in}{1.344145in}}{\pgfqpoint{1.644619in}{1.342379in}}{\pgfqpoint{1.644619in}{1.340537in}}%
\pgfpathcurveto{\pgfqpoint{1.644619in}{1.338695in}}{\pgfqpoint{1.645351in}{1.336929in}}{\pgfqpoint{1.646653in}{1.335626in}}%
\pgfpathcurveto{\pgfqpoint{1.647955in}{1.334324in}}{\pgfqpoint{1.649722in}{1.333592in}}{\pgfqpoint{1.651563in}{1.333592in}}%
\pgfpathlineto{\pgfqpoint{1.651563in}{1.333592in}}%
\pgfpathclose%
\pgfusepath{stroke,fill}%
\end{pgfscope}%
\begin{pgfscope}%
\pgfpathrectangle{\pgfqpoint{0.661006in}{0.524170in}}{\pgfqpoint{4.194036in}{1.071446in}}%
\pgfusepath{clip}%
\pgfsetbuttcap%
\pgfsetroundjoin%
\definecolor{currentfill}{rgb}{0.667253,0.781125,0.871118}%
\pgfsetfillcolor{currentfill}%
\pgfsetfillopacity{0.700000}%
\pgfsetlinewidth{1.003750pt}%
\definecolor{currentstroke}{rgb}{0.667253,0.781125,0.871118}%
\pgfsetstrokecolor{currentstroke}%
\pgfsetstrokeopacity{0.700000}%
\pgfsetdash{}{0pt}%
\pgfpathmoveto{\pgfqpoint{1.658814in}{1.332170in}}%
\pgfpathcurveto{\pgfqpoint{1.660656in}{1.332170in}}{\pgfqpoint{1.662422in}{1.332901in}}{\pgfqpoint{1.663724in}{1.334204in}}%
\pgfpathcurveto{\pgfqpoint{1.665027in}{1.335506in}}{\pgfqpoint{1.665758in}{1.337273in}}{\pgfqpoint{1.665758in}{1.339114in}}%
\pgfpathcurveto{\pgfqpoint{1.665758in}{1.340956in}}{\pgfqpoint{1.665027in}{1.342722in}}{\pgfqpoint{1.663724in}{1.344025in}}%
\pgfpathcurveto{\pgfqpoint{1.662422in}{1.345327in}}{\pgfqpoint{1.660656in}{1.346059in}}{\pgfqpoint{1.658814in}{1.346059in}}%
\pgfpathcurveto{\pgfqpoint{1.656972in}{1.346059in}}{\pgfqpoint{1.655206in}{1.345327in}}{\pgfqpoint{1.653903in}{1.344025in}}%
\pgfpathcurveto{\pgfqpoint{1.652601in}{1.342722in}}{\pgfqpoint{1.651869in}{1.340956in}}{\pgfqpoint{1.651869in}{1.339114in}}%
\pgfpathcurveto{\pgfqpoint{1.651869in}{1.337273in}}{\pgfqpoint{1.652601in}{1.335506in}}{\pgfqpoint{1.653903in}{1.334204in}}%
\pgfpathcurveto{\pgfqpoint{1.655206in}{1.332901in}}{\pgfqpoint{1.656972in}{1.332170in}}{\pgfqpoint{1.658814in}{1.332170in}}%
\pgfpathlineto{\pgfqpoint{1.658814in}{1.332170in}}%
\pgfpathclose%
\pgfusepath{stroke,fill}%
\end{pgfscope}%
\begin{pgfscope}%
\pgfpathrectangle{\pgfqpoint{0.661006in}{0.524170in}}{\pgfqpoint{4.194036in}{1.071446in}}%
\pgfusepath{clip}%
\pgfsetbuttcap%
\pgfsetroundjoin%
\definecolor{currentfill}{rgb}{0.667253,0.781125,0.871118}%
\pgfsetfillcolor{currentfill}%
\pgfsetfillopacity{0.700000}%
\pgfsetlinewidth{1.003750pt}%
\definecolor{currentstroke}{rgb}{0.667253,0.781125,0.871118}%
\pgfsetstrokecolor{currentstroke}%
\pgfsetstrokeopacity{0.700000}%
\pgfsetdash{}{0pt}%
\pgfpathmoveto{\pgfqpoint{1.644545in}{1.334125in}}%
\pgfpathcurveto{\pgfqpoint{1.646387in}{1.334125in}}{\pgfqpoint{1.648154in}{1.334857in}}{\pgfqpoint{1.649456in}{1.336159in}}%
\pgfpathcurveto{\pgfqpoint{1.650758in}{1.337462in}}{\pgfqpoint{1.651490in}{1.339228in}}{\pgfqpoint{1.651490in}{1.341070in}}%
\pgfpathcurveto{\pgfqpoint{1.651490in}{1.342912in}}{\pgfqpoint{1.650758in}{1.344678in}}{\pgfqpoint{1.649456in}{1.345980in}}%
\pgfpathcurveto{\pgfqpoint{1.648154in}{1.347283in}}{\pgfqpoint{1.646387in}{1.348014in}}{\pgfqpoint{1.644545in}{1.348014in}}%
\pgfpathcurveto{\pgfqpoint{1.642704in}{1.348014in}}{\pgfqpoint{1.640937in}{1.347283in}}{\pgfqpoint{1.639635in}{1.345980in}}%
\pgfpathcurveto{\pgfqpoint{1.638333in}{1.344678in}}{\pgfqpoint{1.637601in}{1.342912in}}{\pgfqpoint{1.637601in}{1.341070in}}%
\pgfpathcurveto{\pgfqpoint{1.637601in}{1.339228in}}{\pgfqpoint{1.638333in}{1.337462in}}{\pgfqpoint{1.639635in}{1.336159in}}%
\pgfpathcurveto{\pgfqpoint{1.640937in}{1.334857in}}{\pgfqpoint{1.642704in}{1.334125in}}{\pgfqpoint{1.644545in}{1.334125in}}%
\pgfpathlineto{\pgfqpoint{1.644545in}{1.334125in}}%
\pgfpathclose%
\pgfusepath{stroke,fill}%
\end{pgfscope}%
\begin{pgfscope}%
\pgfpathrectangle{\pgfqpoint{0.661006in}{0.524170in}}{\pgfqpoint{4.194036in}{1.071446in}}%
\pgfusepath{clip}%
\pgfsetbuttcap%
\pgfsetroundjoin%
\definecolor{currentfill}{rgb}{0.667253,0.781125,0.871118}%
\pgfsetfillcolor{currentfill}%
\pgfsetfillopacity{0.700000}%
\pgfsetlinewidth{1.003750pt}%
\definecolor{currentstroke}{rgb}{0.667253,0.781125,0.871118}%
\pgfsetstrokecolor{currentstroke}%
\pgfsetstrokeopacity{0.700000}%
\pgfsetdash{}{0pt}%
\pgfpathmoveto{\pgfqpoint{1.639758in}{1.335785in}}%
\pgfpathcurveto{\pgfqpoint{1.641600in}{1.335785in}}{\pgfqpoint{1.643366in}{1.336517in}}{\pgfqpoint{1.644669in}{1.337819in}}%
\pgfpathcurveto{\pgfqpoint{1.645971in}{1.339121in}}{\pgfqpoint{1.646703in}{1.340888in}}{\pgfqpoint{1.646703in}{1.342730in}}%
\pgfpathcurveto{\pgfqpoint{1.646703in}{1.344571in}}{\pgfqpoint{1.645971in}{1.346338in}}{\pgfqpoint{1.644669in}{1.347640in}}%
\pgfpathcurveto{\pgfqpoint{1.643366in}{1.348942in}}{\pgfqpoint{1.641600in}{1.349674in}}{\pgfqpoint{1.639758in}{1.349674in}}%
\pgfpathcurveto{\pgfqpoint{1.637917in}{1.349674in}}{\pgfqpoint{1.636150in}{1.348942in}}{\pgfqpoint{1.634848in}{1.347640in}}%
\pgfpathcurveto{\pgfqpoint{1.633545in}{1.346338in}}{\pgfqpoint{1.632814in}{1.344571in}}{\pgfqpoint{1.632814in}{1.342730in}}%
\pgfpathcurveto{\pgfqpoint{1.632814in}{1.340888in}}{\pgfqpoint{1.633545in}{1.339121in}}{\pgfqpoint{1.634848in}{1.337819in}}%
\pgfpathcurveto{\pgfqpoint{1.636150in}{1.336517in}}{\pgfqpoint{1.637917in}{1.335785in}}{\pgfqpoint{1.639758in}{1.335785in}}%
\pgfpathlineto{\pgfqpoint{1.639758in}{1.335785in}}%
\pgfpathclose%
\pgfusepath{stroke,fill}%
\end{pgfscope}%
\begin{pgfscope}%
\pgfpathrectangle{\pgfqpoint{0.661006in}{0.524170in}}{\pgfqpoint{4.194036in}{1.071446in}}%
\pgfusepath{clip}%
\pgfsetbuttcap%
\pgfsetroundjoin%
\definecolor{currentfill}{rgb}{0.662960,0.777057,0.868994}%
\pgfsetfillcolor{currentfill}%
\pgfsetfillopacity{0.700000}%
\pgfsetlinewidth{1.003750pt}%
\definecolor{currentstroke}{rgb}{0.662960,0.777057,0.868994}%
\pgfsetstrokecolor{currentstroke}%
\pgfsetstrokeopacity{0.700000}%
\pgfsetdash{}{0pt}%
\pgfpathmoveto{\pgfqpoint{1.637667in}{1.336241in}}%
\pgfpathcurveto{\pgfqpoint{1.639508in}{1.336241in}}{\pgfqpoint{1.641275in}{1.336973in}}{\pgfqpoint{1.642577in}{1.338275in}}%
\pgfpathcurveto{\pgfqpoint{1.643879in}{1.339577in}}{\pgfqpoint{1.644611in}{1.341344in}}{\pgfqpoint{1.644611in}{1.343185in}}%
\pgfpathcurveto{\pgfqpoint{1.644611in}{1.345027in}}{\pgfqpoint{1.643879in}{1.346794in}}{\pgfqpoint{1.642577in}{1.348096in}}%
\pgfpathcurveto{\pgfqpoint{1.641275in}{1.349398in}}{\pgfqpoint{1.639508in}{1.350130in}}{\pgfqpoint{1.637667in}{1.350130in}}%
\pgfpathcurveto{\pgfqpoint{1.635825in}{1.350130in}}{\pgfqpoint{1.634059in}{1.349398in}}{\pgfqpoint{1.632756in}{1.348096in}}%
\pgfpathcurveto{\pgfqpoint{1.631454in}{1.346794in}}{\pgfqpoint{1.630722in}{1.345027in}}{\pgfqpoint{1.630722in}{1.343185in}}%
\pgfpathcurveto{\pgfqpoint{1.630722in}{1.341344in}}{\pgfqpoint{1.631454in}{1.339577in}}{\pgfqpoint{1.632756in}{1.338275in}}%
\pgfpathcurveto{\pgfqpoint{1.634059in}{1.336973in}}{\pgfqpoint{1.635825in}{1.336241in}}{\pgfqpoint{1.637667in}{1.336241in}}%
\pgfpathlineto{\pgfqpoint{1.637667in}{1.336241in}}%
\pgfpathclose%
\pgfusepath{stroke,fill}%
\end{pgfscope}%
\begin{pgfscope}%
\pgfpathrectangle{\pgfqpoint{0.661006in}{0.524170in}}{\pgfqpoint{4.194036in}{1.071446in}}%
\pgfusepath{clip}%
\pgfsetbuttcap%
\pgfsetroundjoin%
\definecolor{currentfill}{rgb}{0.662960,0.777057,0.868994}%
\pgfsetfillcolor{currentfill}%
\pgfsetfillopacity{0.700000}%
\pgfsetlinewidth{1.003750pt}%
\definecolor{currentstroke}{rgb}{0.662960,0.777057,0.868994}%
\pgfsetstrokecolor{currentstroke}%
\pgfsetstrokeopacity{0.700000}%
\pgfsetdash{}{0pt}%
\pgfpathmoveto{\pgfqpoint{1.637760in}{1.335877in}}%
\pgfpathcurveto{\pgfqpoint{1.639601in}{1.335877in}}{\pgfqpoint{1.641368in}{1.336608in}}{\pgfqpoint{1.642670in}{1.337910in}}%
\pgfpathcurveto{\pgfqpoint{1.643972in}{1.339213in}}{\pgfqpoint{1.644704in}{1.340979in}}{\pgfqpoint{1.644704in}{1.342821in}}%
\pgfpathcurveto{\pgfqpoint{1.644704in}{1.344663in}}{\pgfqpoint{1.643972in}{1.346429in}}{\pgfqpoint{1.642670in}{1.347731in}}%
\pgfpathcurveto{\pgfqpoint{1.641368in}{1.349034in}}{\pgfqpoint{1.639601in}{1.349765in}}{\pgfqpoint{1.637760in}{1.349765in}}%
\pgfpathcurveto{\pgfqpoint{1.635918in}{1.349765in}}{\pgfqpoint{1.634151in}{1.349034in}}{\pgfqpoint{1.632849in}{1.347731in}}%
\pgfpathcurveto{\pgfqpoint{1.631547in}{1.346429in}}{\pgfqpoint{1.630815in}{1.344663in}}{\pgfqpoint{1.630815in}{1.342821in}}%
\pgfpathcurveto{\pgfqpoint{1.630815in}{1.340979in}}{\pgfqpoint{1.631547in}{1.339213in}}{\pgfqpoint{1.632849in}{1.337910in}}%
\pgfpathcurveto{\pgfqpoint{1.634151in}{1.336608in}}{\pgfqpoint{1.635918in}{1.335877in}}{\pgfqpoint{1.637760in}{1.335877in}}%
\pgfpathlineto{\pgfqpoint{1.637760in}{1.335877in}}%
\pgfpathclose%
\pgfusepath{stroke,fill}%
\end{pgfscope}%
\begin{pgfscope}%
\pgfpathrectangle{\pgfqpoint{0.661006in}{0.524170in}}{\pgfqpoint{4.194036in}{1.071446in}}%
\pgfusepath{clip}%
\pgfsetbuttcap%
\pgfsetroundjoin%
\definecolor{currentfill}{rgb}{0.658699,0.772976,0.866858}%
\pgfsetfillcolor{currentfill}%
\pgfsetfillopacity{0.700000}%
\pgfsetlinewidth{1.003750pt}%
\definecolor{currentstroke}{rgb}{0.658699,0.772976,0.866858}%
\pgfsetstrokecolor{currentstroke}%
\pgfsetstrokeopacity{0.700000}%
\pgfsetdash{}{0pt}%
\pgfpathmoveto{\pgfqpoint{1.643244in}{1.334618in}}%
\pgfpathcurveto{\pgfqpoint{1.645086in}{1.334618in}}{\pgfqpoint{1.646852in}{1.335350in}}{\pgfqpoint{1.648154in}{1.336652in}}%
\pgfpathcurveto{\pgfqpoint{1.649457in}{1.337954in}}{\pgfqpoint{1.650188in}{1.339721in}}{\pgfqpoint{1.650188in}{1.341562in}}%
\pgfpathcurveto{\pgfqpoint{1.650188in}{1.343404in}}{\pgfqpoint{1.649457in}{1.345171in}}{\pgfqpoint{1.648154in}{1.346473in}}%
\pgfpathcurveto{\pgfqpoint{1.646852in}{1.347775in}}{\pgfqpoint{1.645086in}{1.348507in}}{\pgfqpoint{1.643244in}{1.348507in}}%
\pgfpathcurveto{\pgfqpoint{1.641402in}{1.348507in}}{\pgfqpoint{1.639636in}{1.347775in}}{\pgfqpoint{1.638334in}{1.346473in}}%
\pgfpathcurveto{\pgfqpoint{1.637031in}{1.345171in}}{\pgfqpoint{1.636300in}{1.343404in}}{\pgfqpoint{1.636300in}{1.341562in}}%
\pgfpathcurveto{\pgfqpoint{1.636300in}{1.339721in}}{\pgfqpoint{1.637031in}{1.337954in}}{\pgfqpoint{1.638334in}{1.336652in}}%
\pgfpathcurveto{\pgfqpoint{1.639636in}{1.335350in}}{\pgfqpoint{1.641402in}{1.334618in}}{\pgfqpoint{1.643244in}{1.334618in}}%
\pgfpathlineto{\pgfqpoint{1.643244in}{1.334618in}}%
\pgfpathclose%
\pgfusepath{stroke,fill}%
\end{pgfscope}%
\begin{pgfscope}%
\pgfpathrectangle{\pgfqpoint{0.661006in}{0.524170in}}{\pgfqpoint{4.194036in}{1.071446in}}%
\pgfusepath{clip}%
\pgfsetbuttcap%
\pgfsetroundjoin%
\definecolor{currentfill}{rgb}{0.658699,0.772976,0.866858}%
\pgfsetfillcolor{currentfill}%
\pgfsetfillopacity{0.700000}%
\pgfsetlinewidth{1.003750pt}%
\definecolor{currentstroke}{rgb}{0.658699,0.772976,0.866858}%
\pgfsetstrokecolor{currentstroke}%
\pgfsetstrokeopacity{0.700000}%
\pgfsetdash{}{0pt}%
\pgfpathmoveto{\pgfqpoint{1.664856in}{1.328794in}}%
\pgfpathcurveto{\pgfqpoint{1.666698in}{1.328794in}}{\pgfqpoint{1.668464in}{1.329526in}}{\pgfqpoint{1.669766in}{1.330828in}}%
\pgfpathcurveto{\pgfqpoint{1.671069in}{1.332130in}}{\pgfqpoint{1.671800in}{1.333897in}}{\pgfqpoint{1.671800in}{1.335739in}}%
\pgfpathcurveto{\pgfqpoint{1.671800in}{1.337580in}}{\pgfqpoint{1.671069in}{1.339347in}}{\pgfqpoint{1.669766in}{1.340649in}}%
\pgfpathcurveto{\pgfqpoint{1.668464in}{1.341951in}}{\pgfqpoint{1.666698in}{1.342683in}}{\pgfqpoint{1.664856in}{1.342683in}}%
\pgfpathcurveto{\pgfqpoint{1.663014in}{1.342683in}}{\pgfqpoint{1.661248in}{1.341951in}}{\pgfqpoint{1.659945in}{1.340649in}}%
\pgfpathcurveto{\pgfqpoint{1.658643in}{1.339347in}}{\pgfqpoint{1.657911in}{1.337580in}}{\pgfqpoint{1.657911in}{1.335739in}}%
\pgfpathcurveto{\pgfqpoint{1.657911in}{1.333897in}}{\pgfqpoint{1.658643in}{1.332130in}}{\pgfqpoint{1.659945in}{1.330828in}}%
\pgfpathcurveto{\pgfqpoint{1.661248in}{1.329526in}}{\pgfqpoint{1.663014in}{1.328794in}}{\pgfqpoint{1.664856in}{1.328794in}}%
\pgfpathlineto{\pgfqpoint{1.664856in}{1.328794in}}%
\pgfpathclose%
\pgfusepath{stroke,fill}%
\end{pgfscope}%
\begin{pgfscope}%
\pgfpathrectangle{\pgfqpoint{0.661006in}{0.524170in}}{\pgfqpoint{4.194036in}{1.071446in}}%
\pgfusepath{clip}%
\pgfsetbuttcap%
\pgfsetroundjoin%
\definecolor{currentfill}{rgb}{0.658699,0.772976,0.866858}%
\pgfsetfillcolor{currentfill}%
\pgfsetfillopacity{0.700000}%
\pgfsetlinewidth{1.003750pt}%
\definecolor{currentstroke}{rgb}{0.658699,0.772976,0.866858}%
\pgfsetstrokecolor{currentstroke}%
\pgfsetstrokeopacity{0.700000}%
\pgfsetdash{}{0pt}%
\pgfpathmoveto{\pgfqpoint{1.688141in}{1.321573in}}%
\pgfpathcurveto{\pgfqpoint{1.689983in}{1.321573in}}{\pgfqpoint{1.691749in}{1.322304in}}{\pgfqpoint{1.693051in}{1.323607in}}%
\pgfpathcurveto{\pgfqpoint{1.694354in}{1.324909in}}{\pgfqpoint{1.695085in}{1.326675in}}{\pgfqpoint{1.695085in}{1.328517in}}%
\pgfpathcurveto{\pgfqpoint{1.695085in}{1.330359in}}{\pgfqpoint{1.694354in}{1.332125in}}{\pgfqpoint{1.693051in}{1.333428in}}%
\pgfpathcurveto{\pgfqpoint{1.691749in}{1.334730in}}{\pgfqpoint{1.689983in}{1.335462in}}{\pgfqpoint{1.688141in}{1.335462in}}%
\pgfpathcurveto{\pgfqpoint{1.686299in}{1.335462in}}{\pgfqpoint{1.684533in}{1.334730in}}{\pgfqpoint{1.683231in}{1.333428in}}%
\pgfpathcurveto{\pgfqpoint{1.681928in}{1.332125in}}{\pgfqpoint{1.681197in}{1.330359in}}{\pgfqpoint{1.681197in}{1.328517in}}%
\pgfpathcurveto{\pgfqpoint{1.681197in}{1.326675in}}{\pgfqpoint{1.681928in}{1.324909in}}{\pgfqpoint{1.683231in}{1.323607in}}%
\pgfpathcurveto{\pgfqpoint{1.684533in}{1.322304in}}{\pgfqpoint{1.686299in}{1.321573in}}{\pgfqpoint{1.688141in}{1.321573in}}%
\pgfpathlineto{\pgfqpoint{1.688141in}{1.321573in}}%
\pgfpathclose%
\pgfusepath{stroke,fill}%
\end{pgfscope}%
\begin{pgfscope}%
\pgfpathrectangle{\pgfqpoint{0.661006in}{0.524170in}}{\pgfqpoint{4.194036in}{1.071446in}}%
\pgfusepath{clip}%
\pgfsetbuttcap%
\pgfsetroundjoin%
\definecolor{currentfill}{rgb}{0.654469,0.768880,0.864709}%
\pgfsetfillcolor{currentfill}%
\pgfsetfillopacity{0.700000}%
\pgfsetlinewidth{1.003750pt}%
\definecolor{currentstroke}{rgb}{0.654469,0.768880,0.864709}%
\pgfsetstrokecolor{currentstroke}%
\pgfsetstrokeopacity{0.700000}%
\pgfsetdash{}{0pt}%
\pgfpathmoveto{\pgfqpoint{1.708312in}{1.317804in}}%
\pgfpathcurveto{\pgfqpoint{1.710154in}{1.317804in}}{\pgfqpoint{1.711920in}{1.318536in}}{\pgfqpoint{1.713223in}{1.319838in}}%
\pgfpathcurveto{\pgfqpoint{1.714525in}{1.321140in}}{\pgfqpoint{1.715257in}{1.322907in}}{\pgfqpoint{1.715257in}{1.324748in}}%
\pgfpathcurveto{\pgfqpoint{1.715257in}{1.326590in}}{\pgfqpoint{1.714525in}{1.328357in}}{\pgfqpoint{1.713223in}{1.329659in}}%
\pgfpathcurveto{\pgfqpoint{1.711920in}{1.330961in}}{\pgfqpoint{1.710154in}{1.331693in}}{\pgfqpoint{1.708312in}{1.331693in}}%
\pgfpathcurveto{\pgfqpoint{1.706470in}{1.331693in}}{\pgfqpoint{1.704704in}{1.330961in}}{\pgfqpoint{1.703402in}{1.329659in}}%
\pgfpathcurveto{\pgfqpoint{1.702099in}{1.328357in}}{\pgfqpoint{1.701368in}{1.326590in}}{\pgfqpoint{1.701368in}{1.324748in}}%
\pgfpathcurveto{\pgfqpoint{1.701368in}{1.322907in}}{\pgfqpoint{1.702099in}{1.321140in}}{\pgfqpoint{1.703402in}{1.319838in}}%
\pgfpathcurveto{\pgfqpoint{1.704704in}{1.318536in}}{\pgfqpoint{1.706470in}{1.317804in}}{\pgfqpoint{1.708312in}{1.317804in}}%
\pgfpathlineto{\pgfqpoint{1.708312in}{1.317804in}}%
\pgfpathclose%
\pgfusepath{stroke,fill}%
\end{pgfscope}%
\begin{pgfscope}%
\pgfpathrectangle{\pgfqpoint{0.661006in}{0.524170in}}{\pgfqpoint{4.194036in}{1.071446in}}%
\pgfusepath{clip}%
\pgfsetbuttcap%
\pgfsetroundjoin%
\definecolor{currentfill}{rgb}{0.654469,0.768880,0.864709}%
\pgfsetfillcolor{currentfill}%
\pgfsetfillopacity{0.700000}%
\pgfsetlinewidth{1.003750pt}%
\definecolor{currentstroke}{rgb}{0.654469,0.768880,0.864709}%
\pgfsetstrokecolor{currentstroke}%
\pgfsetstrokeopacity{0.700000}%
\pgfsetdash{}{0pt}%
\pgfpathmoveto{\pgfqpoint{1.714587in}{1.317083in}}%
\pgfpathcurveto{\pgfqpoint{1.716428in}{1.317083in}}{\pgfqpoint{1.718195in}{1.317815in}}{\pgfqpoint{1.719497in}{1.319117in}}%
\pgfpathcurveto{\pgfqpoint{1.720799in}{1.320420in}}{\pgfqpoint{1.721531in}{1.322186in}}{\pgfqpoint{1.721531in}{1.324028in}}%
\pgfpathcurveto{\pgfqpoint{1.721531in}{1.325870in}}{\pgfqpoint{1.720799in}{1.327636in}}{\pgfqpoint{1.719497in}{1.328938in}}%
\pgfpathcurveto{\pgfqpoint{1.718195in}{1.330241in}}{\pgfqpoint{1.716428in}{1.330972in}}{\pgfqpoint{1.714587in}{1.330972in}}%
\pgfpathcurveto{\pgfqpoint{1.712745in}{1.330972in}}{\pgfqpoint{1.710978in}{1.330241in}}{\pgfqpoint{1.709676in}{1.328938in}}%
\pgfpathcurveto{\pgfqpoint{1.708374in}{1.327636in}}{\pgfqpoint{1.707642in}{1.325870in}}{\pgfqpoint{1.707642in}{1.324028in}}%
\pgfpathcurveto{\pgfqpoint{1.707642in}{1.322186in}}{\pgfqpoint{1.708374in}{1.320420in}}{\pgfqpoint{1.709676in}{1.319117in}}%
\pgfpathcurveto{\pgfqpoint{1.710978in}{1.317815in}}{\pgfqpoint{1.712745in}{1.317083in}}{\pgfqpoint{1.714587in}{1.317083in}}%
\pgfpathlineto{\pgfqpoint{1.714587in}{1.317083in}}%
\pgfpathclose%
\pgfusepath{stroke,fill}%
\end{pgfscope}%
\begin{pgfscope}%
\pgfpathrectangle{\pgfqpoint{0.661006in}{0.524170in}}{\pgfqpoint{4.194036in}{1.071446in}}%
\pgfusepath{clip}%
\pgfsetbuttcap%
\pgfsetroundjoin%
\definecolor{currentfill}{rgb}{0.654469,0.768880,0.864709}%
\pgfsetfillcolor{currentfill}%
\pgfsetfillopacity{0.700000}%
\pgfsetlinewidth{1.003750pt}%
\definecolor{currentstroke}{rgb}{0.654469,0.768880,0.864709}%
\pgfsetstrokecolor{currentstroke}%
\pgfsetstrokeopacity{0.700000}%
\pgfsetdash{}{0pt}%
\pgfpathmoveto{\pgfqpoint{1.703571in}{1.318813in}}%
\pgfpathcurveto{\pgfqpoint{1.705413in}{1.318813in}}{\pgfqpoint{1.707180in}{1.319544in}}{\pgfqpoint{1.708482in}{1.320846in}}%
\pgfpathcurveto{\pgfqpoint{1.709784in}{1.322149in}}{\pgfqpoint{1.710516in}{1.323915in}}{\pgfqpoint{1.710516in}{1.325757in}}%
\pgfpathcurveto{\pgfqpoint{1.710516in}{1.327599in}}{\pgfqpoint{1.709784in}{1.329365in}}{\pgfqpoint{1.708482in}{1.330667in}}%
\pgfpathcurveto{\pgfqpoint{1.707180in}{1.331970in}}{\pgfqpoint{1.705413in}{1.332701in}}{\pgfqpoint{1.703571in}{1.332701in}}%
\pgfpathcurveto{\pgfqpoint{1.701730in}{1.332701in}}{\pgfqpoint{1.699963in}{1.331970in}}{\pgfqpoint{1.698661in}{1.330667in}}%
\pgfpathcurveto{\pgfqpoint{1.697359in}{1.329365in}}{\pgfqpoint{1.696627in}{1.327599in}}{\pgfqpoint{1.696627in}{1.325757in}}%
\pgfpathcurveto{\pgfqpoint{1.696627in}{1.323915in}}{\pgfqpoint{1.697359in}{1.322149in}}{\pgfqpoint{1.698661in}{1.320846in}}%
\pgfpathcurveto{\pgfqpoint{1.699963in}{1.319544in}}{\pgfqpoint{1.701730in}{1.318813in}}{\pgfqpoint{1.703571in}{1.318813in}}%
\pgfpathlineto{\pgfqpoint{1.703571in}{1.318813in}}%
\pgfpathclose%
\pgfusepath{stroke,fill}%
\end{pgfscope}%
\begin{pgfscope}%
\pgfpathrectangle{\pgfqpoint{0.661006in}{0.524170in}}{\pgfqpoint{4.194036in}{1.071446in}}%
\pgfusepath{clip}%
\pgfsetbuttcap%
\pgfsetroundjoin%
\definecolor{currentfill}{rgb}{0.654469,0.768880,0.864709}%
\pgfsetfillcolor{currentfill}%
\pgfsetfillopacity{0.700000}%
\pgfsetlinewidth{1.003750pt}%
\definecolor{currentstroke}{rgb}{0.654469,0.768880,0.864709}%
\pgfsetstrokecolor{currentstroke}%
\pgfsetstrokeopacity{0.700000}%
\pgfsetdash{}{0pt}%
\pgfpathmoveto{\pgfqpoint{1.681790in}{1.322970in}}%
\pgfpathcurveto{\pgfqpoint{1.683632in}{1.322970in}}{\pgfqpoint{1.685398in}{1.323702in}}{\pgfqpoint{1.686701in}{1.325004in}}%
\pgfpathcurveto{\pgfqpoint{1.688003in}{1.326307in}}{\pgfqpoint{1.688735in}{1.328073in}}{\pgfqpoint{1.688735in}{1.329915in}}%
\pgfpathcurveto{\pgfqpoint{1.688735in}{1.331757in}}{\pgfqpoint{1.688003in}{1.333523in}}{\pgfqpoint{1.686701in}{1.334825in}}%
\pgfpathcurveto{\pgfqpoint{1.685398in}{1.336128in}}{\pgfqpoint{1.683632in}{1.336859in}}{\pgfqpoint{1.681790in}{1.336859in}}%
\pgfpathcurveto{\pgfqpoint{1.679949in}{1.336859in}}{\pgfqpoint{1.678182in}{1.336128in}}{\pgfqpoint{1.676880in}{1.334825in}}%
\pgfpathcurveto{\pgfqpoint{1.675577in}{1.333523in}}{\pgfqpoint{1.674846in}{1.331757in}}{\pgfqpoint{1.674846in}{1.329915in}}%
\pgfpathcurveto{\pgfqpoint{1.674846in}{1.328073in}}{\pgfqpoint{1.675577in}{1.326307in}}{\pgfqpoint{1.676880in}{1.325004in}}%
\pgfpathcurveto{\pgfqpoint{1.678182in}{1.323702in}}{\pgfqpoint{1.679949in}{1.322970in}}{\pgfqpoint{1.681790in}{1.322970in}}%
\pgfpathlineto{\pgfqpoint{1.681790in}{1.322970in}}%
\pgfpathclose%
\pgfusepath{stroke,fill}%
\end{pgfscope}%
\begin{pgfscope}%
\pgfpathrectangle{\pgfqpoint{0.661006in}{0.524170in}}{\pgfqpoint{4.194036in}{1.071446in}}%
\pgfusepath{clip}%
\pgfsetbuttcap%
\pgfsetroundjoin%
\definecolor{currentfill}{rgb}{0.650269,0.764772,0.862549}%
\pgfsetfillcolor{currentfill}%
\pgfsetfillopacity{0.700000}%
\pgfsetlinewidth{1.003750pt}%
\definecolor{currentstroke}{rgb}{0.650269,0.764772,0.862549}%
\pgfsetstrokecolor{currentstroke}%
\pgfsetstrokeopacity{0.700000}%
\pgfsetdash{}{0pt}%
\pgfpathmoveto{\pgfqpoint{1.661091in}{1.328301in}}%
\pgfpathcurveto{\pgfqpoint{1.662933in}{1.328301in}}{\pgfqpoint{1.664699in}{1.329033in}}{\pgfqpoint{1.666002in}{1.330335in}}%
\pgfpathcurveto{\pgfqpoint{1.667304in}{1.331638in}}{\pgfqpoint{1.668036in}{1.333404in}}{\pgfqpoint{1.668036in}{1.335246in}}%
\pgfpathcurveto{\pgfqpoint{1.668036in}{1.337088in}}{\pgfqpoint{1.667304in}{1.338854in}}{\pgfqpoint{1.666002in}{1.340156in}}%
\pgfpathcurveto{\pgfqpoint{1.664699in}{1.341459in}}{\pgfqpoint{1.662933in}{1.342190in}}{\pgfqpoint{1.661091in}{1.342190in}}%
\pgfpathcurveto{\pgfqpoint{1.659250in}{1.342190in}}{\pgfqpoint{1.657483in}{1.341459in}}{\pgfqpoint{1.656181in}{1.340156in}}%
\pgfpathcurveto{\pgfqpoint{1.654879in}{1.338854in}}{\pgfqpoint{1.654147in}{1.337088in}}{\pgfqpoint{1.654147in}{1.335246in}}%
\pgfpathcurveto{\pgfqpoint{1.654147in}{1.333404in}}{\pgfqpoint{1.654879in}{1.331638in}}{\pgfqpoint{1.656181in}{1.330335in}}%
\pgfpathcurveto{\pgfqpoint{1.657483in}{1.329033in}}{\pgfqpoint{1.659250in}{1.328301in}}{\pgfqpoint{1.661091in}{1.328301in}}%
\pgfpathlineto{\pgfqpoint{1.661091in}{1.328301in}}%
\pgfpathclose%
\pgfusepath{stroke,fill}%
\end{pgfscope}%
\begin{pgfscope}%
\pgfpathrectangle{\pgfqpoint{0.661006in}{0.524170in}}{\pgfqpoint{4.194036in}{1.071446in}}%
\pgfusepath{clip}%
\pgfsetbuttcap%
\pgfsetroundjoin%
\definecolor{currentfill}{rgb}{0.650269,0.764772,0.862549}%
\pgfsetfillcolor{currentfill}%
\pgfsetfillopacity{0.700000}%
\pgfsetlinewidth{1.003750pt}%
\definecolor{currentstroke}{rgb}{0.650269,0.764772,0.862549}%
\pgfsetstrokecolor{currentstroke}%
\pgfsetstrokeopacity{0.700000}%
\pgfsetdash{}{0pt}%
\pgfpathmoveto{\pgfqpoint{1.649937in}{1.330089in}}%
\pgfpathcurveto{\pgfqpoint{1.651778in}{1.330089in}}{\pgfqpoint{1.653545in}{1.330821in}}{\pgfqpoint{1.654847in}{1.332123in}}%
\pgfpathcurveto{\pgfqpoint{1.656149in}{1.333426in}}{\pgfqpoint{1.656881in}{1.335192in}}{\pgfqpoint{1.656881in}{1.337034in}}%
\pgfpathcurveto{\pgfqpoint{1.656881in}{1.338876in}}{\pgfqpoint{1.656149in}{1.340642in}}{\pgfqpoint{1.654847in}{1.341944in}}%
\pgfpathcurveto{\pgfqpoint{1.653545in}{1.343247in}}{\pgfqpoint{1.651778in}{1.343978in}}{\pgfqpoint{1.649937in}{1.343978in}}%
\pgfpathcurveto{\pgfqpoint{1.648095in}{1.343978in}}{\pgfqpoint{1.646329in}{1.343247in}}{\pgfqpoint{1.645026in}{1.341944in}}%
\pgfpathcurveto{\pgfqpoint{1.643724in}{1.340642in}}{\pgfqpoint{1.642992in}{1.338876in}}{\pgfqpoint{1.642992in}{1.337034in}}%
\pgfpathcurveto{\pgfqpoint{1.642992in}{1.335192in}}{\pgfqpoint{1.643724in}{1.333426in}}{\pgfqpoint{1.645026in}{1.332123in}}%
\pgfpathcurveto{\pgfqpoint{1.646329in}{1.330821in}}{\pgfqpoint{1.648095in}{1.330089in}}{\pgfqpoint{1.649937in}{1.330089in}}%
\pgfpathlineto{\pgfqpoint{1.649937in}{1.330089in}}%
\pgfpathclose%
\pgfusepath{stroke,fill}%
\end{pgfscope}%
\begin{pgfscope}%
\pgfpathrectangle{\pgfqpoint{0.661006in}{0.524170in}}{\pgfqpoint{4.194036in}{1.071446in}}%
\pgfusepath{clip}%
\pgfsetbuttcap%
\pgfsetroundjoin%
\definecolor{currentfill}{rgb}{0.650269,0.764772,0.862549}%
\pgfsetfillcolor{currentfill}%
\pgfsetfillopacity{0.700000}%
\pgfsetlinewidth{1.003750pt}%
\definecolor{currentstroke}{rgb}{0.650269,0.764772,0.862549}%
\pgfsetstrokecolor{currentstroke}%
\pgfsetstrokeopacity{0.700000}%
\pgfsetdash{}{0pt}%
\pgfpathmoveto{\pgfqpoint{1.661742in}{1.325594in}}%
\pgfpathcurveto{\pgfqpoint{1.663584in}{1.325594in}}{\pgfqpoint{1.665350in}{1.326326in}}{\pgfqpoint{1.666652in}{1.327628in}}%
\pgfpathcurveto{\pgfqpoint{1.667955in}{1.328931in}}{\pgfqpoint{1.668686in}{1.330697in}}{\pgfqpoint{1.668686in}{1.332539in}}%
\pgfpathcurveto{\pgfqpoint{1.668686in}{1.334380in}}{\pgfqpoint{1.667955in}{1.336147in}}{\pgfqpoint{1.666652in}{1.337449in}}%
\pgfpathcurveto{\pgfqpoint{1.665350in}{1.338751in}}{\pgfqpoint{1.663584in}{1.339483in}}{\pgfqpoint{1.661742in}{1.339483in}}%
\pgfpathcurveto{\pgfqpoint{1.659900in}{1.339483in}}{\pgfqpoint{1.658134in}{1.338751in}}{\pgfqpoint{1.656831in}{1.337449in}}%
\pgfpathcurveto{\pgfqpoint{1.655529in}{1.336147in}}{\pgfqpoint{1.654797in}{1.334380in}}{\pgfqpoint{1.654797in}{1.332539in}}%
\pgfpathcurveto{\pgfqpoint{1.654797in}{1.330697in}}{\pgfqpoint{1.655529in}{1.328931in}}{\pgfqpoint{1.656831in}{1.327628in}}%
\pgfpathcurveto{\pgfqpoint{1.658134in}{1.326326in}}{\pgfqpoint{1.659900in}{1.325594in}}{\pgfqpoint{1.661742in}{1.325594in}}%
\pgfpathlineto{\pgfqpoint{1.661742in}{1.325594in}}%
\pgfpathclose%
\pgfusepath{stroke,fill}%
\end{pgfscope}%
\begin{pgfscope}%
\pgfpathrectangle{\pgfqpoint{0.661006in}{0.524170in}}{\pgfqpoint{4.194036in}{1.071446in}}%
\pgfusepath{clip}%
\pgfsetbuttcap%
\pgfsetroundjoin%
\definecolor{currentfill}{rgb}{0.646100,0.760650,0.860375}%
\pgfsetfillcolor{currentfill}%
\pgfsetfillopacity{0.700000}%
\pgfsetlinewidth{1.003750pt}%
\definecolor{currentstroke}{rgb}{0.646100,0.760650,0.860375}%
\pgfsetstrokecolor{currentstroke}%
\pgfsetstrokeopacity{0.700000}%
\pgfsetdash{}{0pt}%
\pgfpathmoveto{\pgfqpoint{1.691580in}{1.318571in}}%
\pgfpathcurveto{\pgfqpoint{1.693422in}{1.318571in}}{\pgfqpoint{1.695189in}{1.319303in}}{\pgfqpoint{1.696491in}{1.320605in}}%
\pgfpathcurveto{\pgfqpoint{1.697793in}{1.321907in}}{\pgfqpoint{1.698525in}{1.323674in}}{\pgfqpoint{1.698525in}{1.325515in}}%
\pgfpathcurveto{\pgfqpoint{1.698525in}{1.327357in}}{\pgfqpoint{1.697793in}{1.329123in}}{\pgfqpoint{1.696491in}{1.330426in}}%
\pgfpathcurveto{\pgfqpoint{1.695189in}{1.331728in}}{\pgfqpoint{1.693422in}{1.332460in}}{\pgfqpoint{1.691580in}{1.332460in}}%
\pgfpathcurveto{\pgfqpoint{1.689739in}{1.332460in}}{\pgfqpoint{1.687972in}{1.331728in}}{\pgfqpoint{1.686670in}{1.330426in}}%
\pgfpathcurveto{\pgfqpoint{1.685368in}{1.329123in}}{\pgfqpoint{1.684636in}{1.327357in}}{\pgfqpoint{1.684636in}{1.325515in}}%
\pgfpathcurveto{\pgfqpoint{1.684636in}{1.323674in}}{\pgfqpoint{1.685368in}{1.321907in}}{\pgfqpoint{1.686670in}{1.320605in}}%
\pgfpathcurveto{\pgfqpoint{1.687972in}{1.319303in}}{\pgfqpoint{1.689739in}{1.318571in}}{\pgfqpoint{1.691580in}{1.318571in}}%
\pgfpathlineto{\pgfqpoint{1.691580in}{1.318571in}}%
\pgfpathclose%
\pgfusepath{stroke,fill}%
\end{pgfscope}%
\begin{pgfscope}%
\pgfpathrectangle{\pgfqpoint{0.661006in}{0.524170in}}{\pgfqpoint{4.194036in}{1.071446in}}%
\pgfusepath{clip}%
\pgfsetbuttcap%
\pgfsetroundjoin%
\definecolor{currentfill}{rgb}{0.646100,0.760650,0.860375}%
\pgfsetfillcolor{currentfill}%
\pgfsetfillopacity{0.700000}%
\pgfsetlinewidth{1.003750pt}%
\definecolor{currentstroke}{rgb}{0.646100,0.760650,0.860375}%
\pgfsetstrokecolor{currentstroke}%
\pgfsetstrokeopacity{0.700000}%
\pgfsetdash{}{0pt}%
\pgfpathmoveto{\pgfqpoint{1.721744in}{1.312049in}}%
\pgfpathcurveto{\pgfqpoint{1.723586in}{1.312049in}}{\pgfqpoint{1.725352in}{1.312781in}}{\pgfqpoint{1.726655in}{1.314083in}}%
\pgfpathcurveto{\pgfqpoint{1.727957in}{1.315386in}}{\pgfqpoint{1.728688in}{1.317152in}}{\pgfqpoint{1.728688in}{1.318994in}}%
\pgfpathcurveto{\pgfqpoint{1.728688in}{1.320835in}}{\pgfqpoint{1.727957in}{1.322602in}}{\pgfqpoint{1.726655in}{1.323904in}}%
\pgfpathcurveto{\pgfqpoint{1.725352in}{1.325207in}}{\pgfqpoint{1.723586in}{1.325938in}}{\pgfqpoint{1.721744in}{1.325938in}}%
\pgfpathcurveto{\pgfqpoint{1.719902in}{1.325938in}}{\pgfqpoint{1.718136in}{1.325207in}}{\pgfqpoint{1.716834in}{1.323904in}}%
\pgfpathcurveto{\pgfqpoint{1.715531in}{1.322602in}}{\pgfqpoint{1.714800in}{1.320835in}}{\pgfqpoint{1.714800in}{1.318994in}}%
\pgfpathcurveto{\pgfqpoint{1.714800in}{1.317152in}}{\pgfqpoint{1.715531in}{1.315386in}}{\pgfqpoint{1.716834in}{1.314083in}}%
\pgfpathcurveto{\pgfqpoint{1.718136in}{1.312781in}}{\pgfqpoint{1.719902in}{1.312049in}}{\pgfqpoint{1.721744in}{1.312049in}}%
\pgfpathlineto{\pgfqpoint{1.721744in}{1.312049in}}%
\pgfpathclose%
\pgfusepath{stroke,fill}%
\end{pgfscope}%
\begin{pgfscope}%
\pgfpathrectangle{\pgfqpoint{0.661006in}{0.524170in}}{\pgfqpoint{4.194036in}{1.071446in}}%
\pgfusepath{clip}%
\pgfsetbuttcap%
\pgfsetroundjoin%
\definecolor{currentfill}{rgb}{0.646100,0.760650,0.860375}%
\pgfsetfillcolor{currentfill}%
\pgfsetfillopacity{0.700000}%
\pgfsetlinewidth{1.003750pt}%
\definecolor{currentstroke}{rgb}{0.646100,0.760650,0.860375}%
\pgfsetstrokecolor{currentstroke}%
\pgfsetstrokeopacity{0.700000}%
\pgfsetdash{}{0pt}%
\pgfpathmoveto{\pgfqpoint{1.752465in}{1.304683in}}%
\pgfpathcurveto{\pgfqpoint{1.754307in}{1.304683in}}{\pgfqpoint{1.756074in}{1.305414in}}{\pgfqpoint{1.757376in}{1.306717in}}%
\pgfpathcurveto{\pgfqpoint{1.758678in}{1.308019in}}{\pgfqpoint{1.759410in}{1.309785in}}{\pgfqpoint{1.759410in}{1.311627in}}%
\pgfpathcurveto{\pgfqpoint{1.759410in}{1.313469in}}{\pgfqpoint{1.758678in}{1.315235in}}{\pgfqpoint{1.757376in}{1.316538in}}%
\pgfpathcurveto{\pgfqpoint{1.756074in}{1.317840in}}{\pgfqpoint{1.754307in}{1.318572in}}{\pgfqpoint{1.752465in}{1.318572in}}%
\pgfpathcurveto{\pgfqpoint{1.750624in}{1.318572in}}{\pgfqpoint{1.748857in}{1.317840in}}{\pgfqpoint{1.747555in}{1.316538in}}%
\pgfpathcurveto{\pgfqpoint{1.746253in}{1.315235in}}{\pgfqpoint{1.745521in}{1.313469in}}{\pgfqpoint{1.745521in}{1.311627in}}%
\pgfpathcurveto{\pgfqpoint{1.745521in}{1.309785in}}{\pgfqpoint{1.746253in}{1.308019in}}{\pgfqpoint{1.747555in}{1.306717in}}%
\pgfpathcurveto{\pgfqpoint{1.748857in}{1.305414in}}{\pgfqpoint{1.750624in}{1.304683in}}{\pgfqpoint{1.752465in}{1.304683in}}%
\pgfpathlineto{\pgfqpoint{1.752465in}{1.304683in}}%
\pgfpathclose%
\pgfusepath{stroke,fill}%
\end{pgfscope}%
\begin{pgfscope}%
\pgfpathrectangle{\pgfqpoint{0.661006in}{0.524170in}}{\pgfqpoint{4.194036in}{1.071446in}}%
\pgfusepath{clip}%
\pgfsetbuttcap%
\pgfsetroundjoin%
\definecolor{currentfill}{rgb}{0.646100,0.760650,0.860375}%
\pgfsetfillcolor{currentfill}%
\pgfsetfillopacity{0.700000}%
\pgfsetlinewidth{1.003750pt}%
\definecolor{currentstroke}{rgb}{0.646100,0.760650,0.860375}%
\pgfsetstrokecolor{currentstroke}%
\pgfsetstrokeopacity{0.700000}%
\pgfsetdash{}{0pt}%
\pgfpathmoveto{\pgfqpoint{1.794713in}{1.295036in}}%
\pgfpathcurveto{\pgfqpoint{1.796555in}{1.295036in}}{\pgfqpoint{1.798322in}{1.295768in}}{\pgfqpoint{1.799624in}{1.297070in}}%
\pgfpathcurveto{\pgfqpoint{1.800926in}{1.298372in}}{\pgfqpoint{1.801658in}{1.300139in}}{\pgfqpoint{1.801658in}{1.301980in}}%
\pgfpathcurveto{\pgfqpoint{1.801658in}{1.303822in}}{\pgfqpoint{1.800926in}{1.305589in}}{\pgfqpoint{1.799624in}{1.306891in}}%
\pgfpathcurveto{\pgfqpoint{1.798322in}{1.308193in}}{\pgfqpoint{1.796555in}{1.308925in}}{\pgfqpoint{1.794713in}{1.308925in}}%
\pgfpathcurveto{\pgfqpoint{1.792872in}{1.308925in}}{\pgfqpoint{1.791105in}{1.308193in}}{\pgfqpoint{1.789803in}{1.306891in}}%
\pgfpathcurveto{\pgfqpoint{1.788501in}{1.305589in}}{\pgfqpoint{1.787769in}{1.303822in}}{\pgfqpoint{1.787769in}{1.301980in}}%
\pgfpathcurveto{\pgfqpoint{1.787769in}{1.300139in}}{\pgfqpoint{1.788501in}{1.298372in}}{\pgfqpoint{1.789803in}{1.297070in}}%
\pgfpathcurveto{\pgfqpoint{1.791105in}{1.295768in}}{\pgfqpoint{1.792872in}{1.295036in}}{\pgfqpoint{1.794713in}{1.295036in}}%
\pgfpathlineto{\pgfqpoint{1.794713in}{1.295036in}}%
\pgfpathclose%
\pgfusepath{stroke,fill}%
\end{pgfscope}%
\begin{pgfscope}%
\pgfpathrectangle{\pgfqpoint{0.661006in}{0.524170in}}{\pgfqpoint{4.194036in}{1.071446in}}%
\pgfusepath{clip}%
\pgfsetbuttcap%
\pgfsetroundjoin%
\definecolor{currentfill}{rgb}{0.641961,0.756516,0.858187}%
\pgfsetfillcolor{currentfill}%
\pgfsetfillopacity{0.700000}%
\pgfsetlinewidth{1.003750pt}%
\definecolor{currentstroke}{rgb}{0.641961,0.756516,0.858187}%
\pgfsetstrokecolor{currentstroke}%
\pgfsetstrokeopacity{0.700000}%
\pgfsetdash{}{0pt}%
\pgfpathmoveto{\pgfqpoint{1.845606in}{1.282566in}}%
\pgfpathcurveto{\pgfqpoint{1.847448in}{1.282566in}}{\pgfqpoint{1.849214in}{1.283298in}}{\pgfqpoint{1.850516in}{1.284600in}}%
\pgfpathcurveto{\pgfqpoint{1.851819in}{1.285902in}}{\pgfqpoint{1.852550in}{1.287669in}}{\pgfqpoint{1.852550in}{1.289510in}}%
\pgfpathcurveto{\pgfqpoint{1.852550in}{1.291352in}}{\pgfqpoint{1.851819in}{1.293119in}}{\pgfqpoint{1.850516in}{1.294421in}}%
\pgfpathcurveto{\pgfqpoint{1.849214in}{1.295723in}}{\pgfqpoint{1.847448in}{1.296455in}}{\pgfqpoint{1.845606in}{1.296455in}}%
\pgfpathcurveto{\pgfqpoint{1.843764in}{1.296455in}}{\pgfqpoint{1.841998in}{1.295723in}}{\pgfqpoint{1.840695in}{1.294421in}}%
\pgfpathcurveto{\pgfqpoint{1.839393in}{1.293119in}}{\pgfqpoint{1.838661in}{1.291352in}}{\pgfqpoint{1.838661in}{1.289510in}}%
\pgfpathcurveto{\pgfqpoint{1.838661in}{1.287669in}}{\pgfqpoint{1.839393in}{1.285902in}}{\pgfqpoint{1.840695in}{1.284600in}}%
\pgfpathcurveto{\pgfqpoint{1.841998in}{1.283298in}}{\pgfqpoint{1.843764in}{1.282566in}}{\pgfqpoint{1.845606in}{1.282566in}}%
\pgfpathlineto{\pgfqpoint{1.845606in}{1.282566in}}%
\pgfpathclose%
\pgfusepath{stroke,fill}%
\end{pgfscope}%
\begin{pgfscope}%
\pgfpathrectangle{\pgfqpoint{0.661006in}{0.524170in}}{\pgfqpoint{4.194036in}{1.071446in}}%
\pgfusepath{clip}%
\pgfsetbuttcap%
\pgfsetroundjoin%
\definecolor{currentfill}{rgb}{0.641961,0.756516,0.858187}%
\pgfsetfillcolor{currentfill}%
\pgfsetfillopacity{0.700000}%
\pgfsetlinewidth{1.003750pt}%
\definecolor{currentstroke}{rgb}{0.641961,0.756516,0.858187}%
\pgfsetstrokecolor{currentstroke}%
\pgfsetstrokeopacity{0.700000}%
\pgfsetdash{}{0pt}%
\pgfpathmoveto{\pgfqpoint{1.897335in}{1.270699in}}%
\pgfpathcurveto{\pgfqpoint{1.899177in}{1.270699in}}{\pgfqpoint{1.900943in}{1.271431in}}{\pgfqpoint{1.902246in}{1.272733in}}%
\pgfpathcurveto{\pgfqpoint{1.903548in}{1.274035in}}{\pgfqpoint{1.904279in}{1.275802in}}{\pgfqpoint{1.904279in}{1.277644in}}%
\pgfpathcurveto{\pgfqpoint{1.904279in}{1.279485in}}{\pgfqpoint{1.903548in}{1.281252in}}{\pgfqpoint{1.902246in}{1.282554in}}%
\pgfpathcurveto{\pgfqpoint{1.900943in}{1.283856in}}{\pgfqpoint{1.899177in}{1.284588in}}{\pgfqpoint{1.897335in}{1.284588in}}%
\pgfpathcurveto{\pgfqpoint{1.895493in}{1.284588in}}{\pgfqpoint{1.893727in}{1.283856in}}{\pgfqpoint{1.892425in}{1.282554in}}%
\pgfpathcurveto{\pgfqpoint{1.891122in}{1.281252in}}{\pgfqpoint{1.890391in}{1.279485in}}{\pgfqpoint{1.890391in}{1.277644in}}%
\pgfpathcurveto{\pgfqpoint{1.890391in}{1.275802in}}{\pgfqpoint{1.891122in}{1.274035in}}{\pgfqpoint{1.892425in}{1.272733in}}%
\pgfpathcurveto{\pgfqpoint{1.893727in}{1.271431in}}{\pgfqpoint{1.895493in}{1.270699in}}{\pgfqpoint{1.897335in}{1.270699in}}%
\pgfpathlineto{\pgfqpoint{1.897335in}{1.270699in}}%
\pgfpathclose%
\pgfusepath{stroke,fill}%
\end{pgfscope}%
\begin{pgfscope}%
\pgfpathrectangle{\pgfqpoint{0.661006in}{0.524170in}}{\pgfqpoint{4.194036in}{1.071446in}}%
\pgfusepath{clip}%
\pgfsetbuttcap%
\pgfsetroundjoin%
\definecolor{currentfill}{rgb}{0.641961,0.756516,0.858187}%
\pgfsetfillcolor{currentfill}%
\pgfsetfillopacity{0.700000}%
\pgfsetlinewidth{1.003750pt}%
\definecolor{currentstroke}{rgb}{0.641961,0.756516,0.858187}%
\pgfsetstrokecolor{currentstroke}%
\pgfsetstrokeopacity{0.700000}%
\pgfsetdash{}{0pt}%
\pgfpathmoveto{\pgfqpoint{1.943115in}{1.260840in}}%
\pgfpathcurveto{\pgfqpoint{1.944957in}{1.260840in}}{\pgfqpoint{1.946723in}{1.261571in}}{\pgfqpoint{1.948026in}{1.262874in}}%
\pgfpathcurveto{\pgfqpoint{1.949328in}{1.264176in}}{\pgfqpoint{1.950060in}{1.265942in}}{\pgfqpoint{1.950060in}{1.267784in}}%
\pgfpathcurveto{\pgfqpoint{1.950060in}{1.269626in}}{\pgfqpoint{1.949328in}{1.271392in}}{\pgfqpoint{1.948026in}{1.272695in}}%
\pgfpathcurveto{\pgfqpoint{1.946723in}{1.273997in}}{\pgfqpoint{1.944957in}{1.274729in}}{\pgfqpoint{1.943115in}{1.274729in}}%
\pgfpathcurveto{\pgfqpoint{1.941273in}{1.274729in}}{\pgfqpoint{1.939507in}{1.273997in}}{\pgfqpoint{1.938205in}{1.272695in}}%
\pgfpathcurveto{\pgfqpoint{1.936902in}{1.271392in}}{\pgfqpoint{1.936171in}{1.269626in}}{\pgfqpoint{1.936171in}{1.267784in}}%
\pgfpathcurveto{\pgfqpoint{1.936171in}{1.265942in}}{\pgfqpoint{1.936902in}{1.264176in}}{\pgfqpoint{1.938205in}{1.262874in}}%
\pgfpathcurveto{\pgfqpoint{1.939507in}{1.261571in}}{\pgfqpoint{1.941273in}{1.260840in}}{\pgfqpoint{1.943115in}{1.260840in}}%
\pgfpathlineto{\pgfqpoint{1.943115in}{1.260840in}}%
\pgfpathclose%
\pgfusepath{stroke,fill}%
\end{pgfscope}%
\begin{pgfscope}%
\pgfpathrectangle{\pgfqpoint{0.661006in}{0.524170in}}{\pgfqpoint{4.194036in}{1.071446in}}%
\pgfusepath{clip}%
\pgfsetbuttcap%
\pgfsetroundjoin%
\definecolor{currentfill}{rgb}{0.637852,0.752369,0.855986}%
\pgfsetfillcolor{currentfill}%
\pgfsetfillopacity{0.700000}%
\pgfsetlinewidth{1.003750pt}%
\definecolor{currentstroke}{rgb}{0.637852,0.752369,0.855986}%
\pgfsetstrokecolor{currentstroke}%
\pgfsetstrokeopacity{0.700000}%
\pgfsetdash{}{0pt}%
\pgfpathmoveto{\pgfqpoint{1.967609in}{1.254108in}}%
\pgfpathcurveto{\pgfqpoint{1.969450in}{1.254108in}}{\pgfqpoint{1.971217in}{1.254840in}}{\pgfqpoint{1.972519in}{1.256142in}}%
\pgfpathcurveto{\pgfqpoint{1.973821in}{1.257444in}}{\pgfqpoint{1.974553in}{1.259211in}}{\pgfqpoint{1.974553in}{1.261052in}}%
\pgfpathcurveto{\pgfqpoint{1.974553in}{1.262894in}}{\pgfqpoint{1.973821in}{1.264660in}}{\pgfqpoint{1.972519in}{1.265963in}}%
\pgfpathcurveto{\pgfqpoint{1.971217in}{1.267265in}}{\pgfqpoint{1.969450in}{1.267997in}}{\pgfqpoint{1.967609in}{1.267997in}}%
\pgfpathcurveto{\pgfqpoint{1.965767in}{1.267997in}}{\pgfqpoint{1.964000in}{1.267265in}}{\pgfqpoint{1.962698in}{1.265963in}}%
\pgfpathcurveto{\pgfqpoint{1.961396in}{1.264660in}}{\pgfqpoint{1.960664in}{1.262894in}}{\pgfqpoint{1.960664in}{1.261052in}}%
\pgfpathcurveto{\pgfqpoint{1.960664in}{1.259211in}}{\pgfqpoint{1.961396in}{1.257444in}}{\pgfqpoint{1.962698in}{1.256142in}}%
\pgfpathcurveto{\pgfqpoint{1.964000in}{1.254840in}}{\pgfqpoint{1.965767in}{1.254108in}}{\pgfqpoint{1.967609in}{1.254108in}}%
\pgfpathlineto{\pgfqpoint{1.967609in}{1.254108in}}%
\pgfpathclose%
\pgfusepath{stroke,fill}%
\end{pgfscope}%
\begin{pgfscope}%
\pgfpathrectangle{\pgfqpoint{0.661006in}{0.524170in}}{\pgfqpoint{4.194036in}{1.071446in}}%
\pgfusepath{clip}%
\pgfsetbuttcap%
\pgfsetroundjoin%
\definecolor{currentfill}{rgb}{0.637852,0.752369,0.855986}%
\pgfsetfillcolor{currentfill}%
\pgfsetfillopacity{0.700000}%
\pgfsetlinewidth{1.003750pt}%
\definecolor{currentstroke}{rgb}{0.637852,0.752369,0.855986}%
\pgfsetstrokecolor{currentstroke}%
\pgfsetstrokeopacity{0.700000}%
\pgfsetdash{}{0pt}%
\pgfpathmoveto{\pgfqpoint{1.981505in}{1.252125in}}%
\pgfpathcurveto{\pgfqpoint{1.983347in}{1.252125in}}{\pgfqpoint{1.985114in}{1.252856in}}{\pgfqpoint{1.986416in}{1.254159in}}%
\pgfpathcurveto{\pgfqpoint{1.987718in}{1.255461in}}{\pgfqpoint{1.988450in}{1.257227in}}{\pgfqpoint{1.988450in}{1.259069in}}%
\pgfpathcurveto{\pgfqpoint{1.988450in}{1.260911in}}{\pgfqpoint{1.987718in}{1.262677in}}{\pgfqpoint{1.986416in}{1.263980in}}%
\pgfpathcurveto{\pgfqpoint{1.985114in}{1.265282in}}{\pgfqpoint{1.983347in}{1.266014in}}{\pgfqpoint{1.981505in}{1.266014in}}%
\pgfpathcurveto{\pgfqpoint{1.979664in}{1.266014in}}{\pgfqpoint{1.977897in}{1.265282in}}{\pgfqpoint{1.976595in}{1.263980in}}%
\pgfpathcurveto{\pgfqpoint{1.975293in}{1.262677in}}{\pgfqpoint{1.974561in}{1.260911in}}{\pgfqpoint{1.974561in}{1.259069in}}%
\pgfpathcurveto{\pgfqpoint{1.974561in}{1.257227in}}{\pgfqpoint{1.975293in}{1.255461in}}{\pgfqpoint{1.976595in}{1.254159in}}%
\pgfpathcurveto{\pgfqpoint{1.977897in}{1.252856in}}{\pgfqpoint{1.979664in}{1.252125in}}{\pgfqpoint{1.981505in}{1.252125in}}%
\pgfpathlineto{\pgfqpoint{1.981505in}{1.252125in}}%
\pgfpathclose%
\pgfusepath{stroke,fill}%
\end{pgfscope}%
\begin{pgfscope}%
\pgfpathrectangle{\pgfqpoint{0.661006in}{0.524170in}}{\pgfqpoint{4.194036in}{1.071446in}}%
\pgfusepath{clip}%
\pgfsetbuttcap%
\pgfsetroundjoin%
\definecolor{currentfill}{rgb}{0.633773,0.748209,0.853770}%
\pgfsetfillcolor{currentfill}%
\pgfsetfillopacity{0.700000}%
\pgfsetlinewidth{1.003750pt}%
\definecolor{currentstroke}{rgb}{0.633773,0.748209,0.853770}%
\pgfsetstrokecolor{currentstroke}%
\pgfsetstrokeopacity{0.700000}%
\pgfsetdash{}{0pt}%
\pgfpathmoveto{\pgfqpoint{1.976469in}{1.253075in}}%
\pgfpathcurveto{\pgfqpoint{1.978311in}{1.253075in}}{\pgfqpoint{1.980077in}{1.253807in}}{\pgfqpoint{1.981380in}{1.255109in}}%
\pgfpathcurveto{\pgfqpoint{1.982682in}{1.256411in}}{\pgfqpoint{1.983414in}{1.258178in}}{\pgfqpoint{1.983414in}{1.260020in}}%
\pgfpathcurveto{\pgfqpoint{1.983414in}{1.261861in}}{\pgfqpoint{1.982682in}{1.263628in}}{\pgfqpoint{1.981380in}{1.264930in}}%
\pgfpathcurveto{\pgfqpoint{1.980077in}{1.266232in}}{\pgfqpoint{1.978311in}{1.266964in}}{\pgfqpoint{1.976469in}{1.266964in}}%
\pgfpathcurveto{\pgfqpoint{1.974627in}{1.266964in}}{\pgfqpoint{1.972861in}{1.266232in}}{\pgfqpoint{1.971559in}{1.264930in}}%
\pgfpathcurveto{\pgfqpoint{1.970256in}{1.263628in}}{\pgfqpoint{1.969525in}{1.261861in}}{\pgfqpoint{1.969525in}{1.260020in}}%
\pgfpathcurveto{\pgfqpoint{1.969525in}{1.258178in}}{\pgfqpoint{1.970256in}{1.256411in}}{\pgfqpoint{1.971559in}{1.255109in}}%
\pgfpathcurveto{\pgfqpoint{1.972861in}{1.253807in}}{\pgfqpoint{1.974627in}{1.253075in}}{\pgfqpoint{1.976469in}{1.253075in}}%
\pgfpathlineto{\pgfqpoint{1.976469in}{1.253075in}}%
\pgfpathclose%
\pgfusepath{stroke,fill}%
\end{pgfscope}%
\begin{pgfscope}%
\pgfpathrectangle{\pgfqpoint{0.661006in}{0.524170in}}{\pgfqpoint{4.194036in}{1.071446in}}%
\pgfusepath{clip}%
\pgfsetbuttcap%
\pgfsetroundjoin%
\definecolor{currentfill}{rgb}{0.633773,0.748209,0.853770}%
\pgfsetfillcolor{currentfill}%
\pgfsetfillopacity{0.700000}%
\pgfsetlinewidth{1.003750pt}%
\definecolor{currentstroke}{rgb}{0.633773,0.748209,0.853770}%
\pgfsetstrokecolor{currentstroke}%
\pgfsetstrokeopacity{0.700000}%
\pgfsetdash{}{0pt}%
\pgfpathmoveto{\pgfqpoint{1.984480in}{1.251436in}}%
\pgfpathcurveto{\pgfqpoint{1.986322in}{1.251436in}}{\pgfqpoint{1.988088in}{1.252168in}}{\pgfqpoint{1.989390in}{1.253470in}}%
\pgfpathcurveto{\pgfqpoint{1.990693in}{1.254772in}}{\pgfqpoint{1.991424in}{1.256539in}}{\pgfqpoint{1.991424in}{1.258380in}}%
\pgfpathcurveto{\pgfqpoint{1.991424in}{1.260222in}}{\pgfqpoint{1.990693in}{1.261988in}}{\pgfqpoint{1.989390in}{1.263291in}}%
\pgfpathcurveto{\pgfqpoint{1.988088in}{1.264593in}}{\pgfqpoint{1.986322in}{1.265325in}}{\pgfqpoint{1.984480in}{1.265325in}}%
\pgfpathcurveto{\pgfqpoint{1.982638in}{1.265325in}}{\pgfqpoint{1.980872in}{1.264593in}}{\pgfqpoint{1.979569in}{1.263291in}}%
\pgfpathcurveto{\pgfqpoint{1.978267in}{1.261988in}}{\pgfqpoint{1.977535in}{1.260222in}}{\pgfqpoint{1.977535in}{1.258380in}}%
\pgfpathcurveto{\pgfqpoint{1.977535in}{1.256539in}}{\pgfqpoint{1.978267in}{1.254772in}}{\pgfqpoint{1.979569in}{1.253470in}}%
\pgfpathcurveto{\pgfqpoint{1.980872in}{1.252168in}}{\pgfqpoint{1.982638in}{1.251436in}}{\pgfqpoint{1.984480in}{1.251436in}}%
\pgfpathlineto{\pgfqpoint{1.984480in}{1.251436in}}%
\pgfpathclose%
\pgfusepath{stroke,fill}%
\end{pgfscope}%
\begin{pgfscope}%
\pgfpathrectangle{\pgfqpoint{0.661006in}{0.524170in}}{\pgfqpoint{4.194036in}{1.071446in}}%
\pgfusepath{clip}%
\pgfsetbuttcap%
\pgfsetroundjoin%
\definecolor{currentfill}{rgb}{0.633773,0.748209,0.853770}%
\pgfsetfillcolor{currentfill}%
\pgfsetfillopacity{0.700000}%
\pgfsetlinewidth{1.003750pt}%
\definecolor{currentstroke}{rgb}{0.633773,0.748209,0.853770}%
\pgfsetstrokecolor{currentstroke}%
\pgfsetstrokeopacity{0.700000}%
\pgfsetdash{}{0pt}%
\pgfpathmoveto{\pgfqpoint{1.984155in}{1.251560in}}%
\pgfpathcurveto{\pgfqpoint{1.985996in}{1.251560in}}{\pgfqpoint{1.987763in}{1.252292in}}{\pgfqpoint{1.989065in}{1.253594in}}%
\pgfpathcurveto{\pgfqpoint{1.990367in}{1.254897in}}{\pgfqpoint{1.991099in}{1.256663in}}{\pgfqpoint{1.991099in}{1.258505in}}%
\pgfpathcurveto{\pgfqpoint{1.991099in}{1.260346in}}{\pgfqpoint{1.990367in}{1.262113in}}{\pgfqpoint{1.989065in}{1.263415in}}%
\pgfpathcurveto{\pgfqpoint{1.987763in}{1.264717in}}{\pgfqpoint{1.985996in}{1.265449in}}{\pgfqpoint{1.984155in}{1.265449in}}%
\pgfpathcurveto{\pgfqpoint{1.982313in}{1.265449in}}{\pgfqpoint{1.980546in}{1.264717in}}{\pgfqpoint{1.979244in}{1.263415in}}%
\pgfpathcurveto{\pgfqpoint{1.977942in}{1.262113in}}{\pgfqpoint{1.977210in}{1.260346in}}{\pgfqpoint{1.977210in}{1.258505in}}%
\pgfpathcurveto{\pgfqpoint{1.977210in}{1.256663in}}{\pgfqpoint{1.977942in}{1.254897in}}{\pgfqpoint{1.979244in}{1.253594in}}%
\pgfpathcurveto{\pgfqpoint{1.980546in}{1.252292in}}{\pgfqpoint{1.982313in}{1.251560in}}{\pgfqpoint{1.984155in}{1.251560in}}%
\pgfpathlineto{\pgfqpoint{1.984155in}{1.251560in}}%
\pgfpathclose%
\pgfusepath{stroke,fill}%
\end{pgfscope}%
\begin{pgfscope}%
\pgfpathrectangle{\pgfqpoint{0.661006in}{0.524170in}}{\pgfqpoint{4.194036in}{1.071446in}}%
\pgfusepath{clip}%
\pgfsetbuttcap%
\pgfsetroundjoin%
\definecolor{currentfill}{rgb}{0.633773,0.748209,0.853770}%
\pgfsetfillcolor{currentfill}%
\pgfsetfillopacity{0.700000}%
\pgfsetlinewidth{1.003750pt}%
\definecolor{currentstroke}{rgb}{0.633773,0.748209,0.853770}%
\pgfsetstrokecolor{currentstroke}%
\pgfsetstrokeopacity{0.700000}%
\pgfsetdash{}{0pt}%
\pgfpathmoveto{\pgfqpoint{1.975789in}{1.253908in}}%
\pgfpathcurveto{\pgfqpoint{1.977630in}{1.253908in}}{\pgfqpoint{1.979397in}{1.254639in}}{\pgfqpoint{1.980699in}{1.255942in}}%
\pgfpathcurveto{\pgfqpoint{1.982001in}{1.257244in}}{\pgfqpoint{1.982733in}{1.259010in}}{\pgfqpoint{1.982733in}{1.260852in}}%
\pgfpathcurveto{\pgfqpoint{1.982733in}{1.262694in}}{\pgfqpoint{1.982001in}{1.264460in}}{\pgfqpoint{1.980699in}{1.265763in}}%
\pgfpathcurveto{\pgfqpoint{1.979397in}{1.267065in}}{\pgfqpoint{1.977630in}{1.267797in}}{\pgfqpoint{1.975789in}{1.267797in}}%
\pgfpathcurveto{\pgfqpoint{1.973947in}{1.267797in}}{\pgfqpoint{1.972180in}{1.267065in}}{\pgfqpoint{1.970878in}{1.265763in}}%
\pgfpathcurveto{\pgfqpoint{1.969576in}{1.264460in}}{\pgfqpoint{1.968844in}{1.262694in}}{\pgfqpoint{1.968844in}{1.260852in}}%
\pgfpathcurveto{\pgfqpoint{1.968844in}{1.259010in}}{\pgfqpoint{1.969576in}{1.257244in}}{\pgfqpoint{1.970878in}{1.255942in}}%
\pgfpathcurveto{\pgfqpoint{1.972180in}{1.254639in}}{\pgfqpoint{1.973947in}{1.253908in}}{\pgfqpoint{1.975789in}{1.253908in}}%
\pgfpathlineto{\pgfqpoint{1.975789in}{1.253908in}}%
\pgfpathclose%
\pgfusepath{stroke,fill}%
\end{pgfscope}%
\begin{pgfscope}%
\pgfpathrectangle{\pgfqpoint{0.661006in}{0.524170in}}{\pgfqpoint{4.194036in}{1.071446in}}%
\pgfusepath{clip}%
\pgfsetbuttcap%
\pgfsetroundjoin%
\definecolor{currentfill}{rgb}{0.633773,0.748209,0.853770}%
\pgfsetfillcolor{currentfill}%
\pgfsetfillopacity{0.700000}%
\pgfsetlinewidth{1.003750pt}%
\definecolor{currentstroke}{rgb}{0.633773,0.748209,0.853770}%
\pgfsetstrokecolor{currentstroke}%
\pgfsetstrokeopacity{0.700000}%
\pgfsetdash{}{0pt}%
\pgfpathmoveto{\pgfqpoint{1.957895in}{1.256242in}}%
\pgfpathcurveto{\pgfqpoint{1.959737in}{1.256242in}}{\pgfqpoint{1.961503in}{1.256974in}}{\pgfqpoint{1.962805in}{1.258276in}}%
\pgfpathcurveto{\pgfqpoint{1.964108in}{1.259578in}}{\pgfqpoint{1.964839in}{1.261345in}}{\pgfqpoint{1.964839in}{1.263186in}}%
\pgfpathcurveto{\pgfqpoint{1.964839in}{1.265028in}}{\pgfqpoint{1.964108in}{1.266795in}}{\pgfqpoint{1.962805in}{1.268097in}}%
\pgfpathcurveto{\pgfqpoint{1.961503in}{1.269399in}}{\pgfqpoint{1.959737in}{1.270131in}}{\pgfqpoint{1.957895in}{1.270131in}}%
\pgfpathcurveto{\pgfqpoint{1.956053in}{1.270131in}}{\pgfqpoint{1.954287in}{1.269399in}}{\pgfqpoint{1.952984in}{1.268097in}}%
\pgfpathcurveto{\pgfqpoint{1.951682in}{1.266795in}}{\pgfqpoint{1.950950in}{1.265028in}}{\pgfqpoint{1.950950in}{1.263186in}}%
\pgfpathcurveto{\pgfqpoint{1.950950in}{1.261345in}}{\pgfqpoint{1.951682in}{1.259578in}}{\pgfqpoint{1.952984in}{1.258276in}}%
\pgfpathcurveto{\pgfqpoint{1.954287in}{1.256974in}}{\pgfqpoint{1.956053in}{1.256242in}}{\pgfqpoint{1.957895in}{1.256242in}}%
\pgfpathlineto{\pgfqpoint{1.957895in}{1.256242in}}%
\pgfpathclose%
\pgfusepath{stroke,fill}%
\end{pgfscope}%
\begin{pgfscope}%
\pgfpathrectangle{\pgfqpoint{0.661006in}{0.524170in}}{\pgfqpoint{4.194036in}{1.071446in}}%
\pgfusepath{clip}%
\pgfsetbuttcap%
\pgfsetroundjoin%
\definecolor{currentfill}{rgb}{0.629724,0.744038,0.851540}%
\pgfsetfillcolor{currentfill}%
\pgfsetfillopacity{0.700000}%
\pgfsetlinewidth{1.003750pt}%
\definecolor{currentstroke}{rgb}{0.629724,0.744038,0.851540}%
\pgfsetstrokecolor{currentstroke}%
\pgfsetstrokeopacity{0.700000}%
\pgfsetdash{}{0pt}%
\pgfpathmoveto{\pgfqpoint{1.979786in}{1.251873in}}%
\pgfpathcurveto{\pgfqpoint{1.981627in}{1.251873in}}{\pgfqpoint{1.983394in}{1.252605in}}{\pgfqpoint{1.984696in}{1.253907in}}%
\pgfpathcurveto{\pgfqpoint{1.985998in}{1.255209in}}{\pgfqpoint{1.986730in}{1.256976in}}{\pgfqpoint{1.986730in}{1.258817in}}%
\pgfpathcurveto{\pgfqpoint{1.986730in}{1.260659in}}{\pgfqpoint{1.985998in}{1.262426in}}{\pgfqpoint{1.984696in}{1.263728in}}%
\pgfpathcurveto{\pgfqpoint{1.983394in}{1.265030in}}{\pgfqpoint{1.981627in}{1.265762in}}{\pgfqpoint{1.979786in}{1.265762in}}%
\pgfpathcurveto{\pgfqpoint{1.977944in}{1.265762in}}{\pgfqpoint{1.976177in}{1.265030in}}{\pgfqpoint{1.974875in}{1.263728in}}%
\pgfpathcurveto{\pgfqpoint{1.973573in}{1.262426in}}{\pgfqpoint{1.972841in}{1.260659in}}{\pgfqpoint{1.972841in}{1.258817in}}%
\pgfpathcurveto{\pgfqpoint{1.972841in}{1.256976in}}{\pgfqpoint{1.973573in}{1.255209in}}{\pgfqpoint{1.974875in}{1.253907in}}%
\pgfpathcurveto{\pgfqpoint{1.976177in}{1.252605in}}{\pgfqpoint{1.977944in}{1.251873in}}{\pgfqpoint{1.979786in}{1.251873in}}%
\pgfpathlineto{\pgfqpoint{1.979786in}{1.251873in}}%
\pgfpathclose%
\pgfusepath{stroke,fill}%
\end{pgfscope}%
\begin{pgfscope}%
\pgfpathrectangle{\pgfqpoint{0.661006in}{0.524170in}}{\pgfqpoint{4.194036in}{1.071446in}}%
\pgfusepath{clip}%
\pgfsetbuttcap%
\pgfsetroundjoin%
\definecolor{currentfill}{rgb}{0.629724,0.744038,0.851540}%
\pgfsetfillcolor{currentfill}%
\pgfsetfillopacity{0.700000}%
\pgfsetlinewidth{1.003750pt}%
\definecolor{currentstroke}{rgb}{0.629724,0.744038,0.851540}%
\pgfsetstrokecolor{currentstroke}%
\pgfsetstrokeopacity{0.700000}%
\pgfsetdash{}{0pt}%
\pgfpathmoveto{\pgfqpoint{2.018455in}{1.242310in}}%
\pgfpathcurveto{\pgfqpoint{2.020296in}{1.242310in}}{\pgfqpoint{2.022063in}{1.243042in}}{\pgfqpoint{2.023365in}{1.244344in}}%
\pgfpathcurveto{\pgfqpoint{2.024667in}{1.245646in}}{\pgfqpoint{2.025399in}{1.247413in}}{\pgfqpoint{2.025399in}{1.249254in}}%
\pgfpathcurveto{\pgfqpoint{2.025399in}{1.251096in}}{\pgfqpoint{2.024667in}{1.252863in}}{\pgfqpoint{2.023365in}{1.254165in}}%
\pgfpathcurveto{\pgfqpoint{2.022063in}{1.255467in}}{\pgfqpoint{2.020296in}{1.256199in}}{\pgfqpoint{2.018455in}{1.256199in}}%
\pgfpathcurveto{\pgfqpoint{2.016613in}{1.256199in}}{\pgfqpoint{2.014847in}{1.255467in}}{\pgfqpoint{2.013544in}{1.254165in}}%
\pgfpathcurveto{\pgfqpoint{2.012242in}{1.252863in}}{\pgfqpoint{2.011510in}{1.251096in}}{\pgfqpoint{2.011510in}{1.249254in}}%
\pgfpathcurveto{\pgfqpoint{2.011510in}{1.247413in}}{\pgfqpoint{2.012242in}{1.245646in}}{\pgfqpoint{2.013544in}{1.244344in}}%
\pgfpathcurveto{\pgfqpoint{2.014847in}{1.243042in}}{\pgfqpoint{2.016613in}{1.242310in}}{\pgfqpoint{2.018455in}{1.242310in}}%
\pgfpathlineto{\pgfqpoint{2.018455in}{1.242310in}}%
\pgfpathclose%
\pgfusepath{stroke,fill}%
\end{pgfscope}%
\begin{pgfscope}%
\pgfpathrectangle{\pgfqpoint{0.661006in}{0.524170in}}{\pgfqpoint{4.194036in}{1.071446in}}%
\pgfusepath{clip}%
\pgfsetbuttcap%
\pgfsetroundjoin%
\definecolor{currentfill}{rgb}{0.625704,0.739854,0.849294}%
\pgfsetfillcolor{currentfill}%
\pgfsetfillopacity{0.700000}%
\pgfsetlinewidth{1.003750pt}%
\definecolor{currentstroke}{rgb}{0.625704,0.739854,0.849294}%
\pgfsetstrokecolor{currentstroke}%
\pgfsetstrokeopacity{0.700000}%
\pgfsetdash{}{0pt}%
\pgfpathmoveto{\pgfqpoint{2.040253in}{1.238203in}}%
\pgfpathcurveto{\pgfqpoint{2.042094in}{1.238203in}}{\pgfqpoint{2.043861in}{1.238935in}}{\pgfqpoint{2.045163in}{1.240237in}}%
\pgfpathcurveto{\pgfqpoint{2.046465in}{1.241539in}}{\pgfqpoint{2.047197in}{1.243306in}}{\pgfqpoint{2.047197in}{1.245148in}}%
\pgfpathcurveto{\pgfqpoint{2.047197in}{1.246989in}}{\pgfqpoint{2.046465in}{1.248756in}}{\pgfqpoint{2.045163in}{1.250058in}}%
\pgfpathcurveto{\pgfqpoint{2.043861in}{1.251360in}}{\pgfqpoint{2.042094in}{1.252092in}}{\pgfqpoint{2.040253in}{1.252092in}}%
\pgfpathcurveto{\pgfqpoint{2.038411in}{1.252092in}}{\pgfqpoint{2.036644in}{1.251360in}}{\pgfqpoint{2.035342in}{1.250058in}}%
\pgfpathcurveto{\pgfqpoint{2.034040in}{1.248756in}}{\pgfqpoint{2.033308in}{1.246989in}}{\pgfqpoint{2.033308in}{1.245148in}}%
\pgfpathcurveto{\pgfqpoint{2.033308in}{1.243306in}}{\pgfqpoint{2.034040in}{1.241539in}}{\pgfqpoint{2.035342in}{1.240237in}}%
\pgfpathcurveto{\pgfqpoint{2.036644in}{1.238935in}}{\pgfqpoint{2.038411in}{1.238203in}}{\pgfqpoint{2.040253in}{1.238203in}}%
\pgfpathlineto{\pgfqpoint{2.040253in}{1.238203in}}%
\pgfpathclose%
\pgfusepath{stroke,fill}%
\end{pgfscope}%
\begin{pgfscope}%
\pgfpathrectangle{\pgfqpoint{0.661006in}{0.524170in}}{\pgfqpoint{4.194036in}{1.071446in}}%
\pgfusepath{clip}%
\pgfsetbuttcap%
\pgfsetroundjoin%
\definecolor{currentfill}{rgb}{0.625704,0.739854,0.849294}%
\pgfsetfillcolor{currentfill}%
\pgfsetfillopacity{0.700000}%
\pgfsetlinewidth{1.003750pt}%
\definecolor{currentstroke}{rgb}{0.625704,0.739854,0.849294}%
\pgfsetstrokecolor{currentstroke}%
\pgfsetstrokeopacity{0.700000}%
\pgfsetdash{}{0pt}%
\pgfpathmoveto{\pgfqpoint{1.999771in}{1.246813in}}%
\pgfpathcurveto{\pgfqpoint{2.001613in}{1.246813in}}{\pgfqpoint{2.003379in}{1.247544in}}{\pgfqpoint{2.004681in}{1.248847in}}%
\pgfpathcurveto{\pgfqpoint{2.005984in}{1.250149in}}{\pgfqpoint{2.006715in}{1.251915in}}{\pgfqpoint{2.006715in}{1.253757in}}%
\pgfpathcurveto{\pgfqpoint{2.006715in}{1.255599in}}{\pgfqpoint{2.005984in}{1.257365in}}{\pgfqpoint{2.004681in}{1.258667in}}%
\pgfpathcurveto{\pgfqpoint{2.003379in}{1.259970in}}{\pgfqpoint{2.001613in}{1.260701in}}{\pgfqpoint{1.999771in}{1.260701in}}%
\pgfpathcurveto{\pgfqpoint{1.997929in}{1.260701in}}{\pgfqpoint{1.996163in}{1.259970in}}{\pgfqpoint{1.994860in}{1.258667in}}%
\pgfpathcurveto{\pgfqpoint{1.993558in}{1.257365in}}{\pgfqpoint{1.992826in}{1.255599in}}{\pgfqpoint{1.992826in}{1.253757in}}%
\pgfpathcurveto{\pgfqpoint{1.992826in}{1.251915in}}{\pgfqpoint{1.993558in}{1.250149in}}{\pgfqpoint{1.994860in}{1.248847in}}%
\pgfpathcurveto{\pgfqpoint{1.996163in}{1.247544in}}{\pgfqpoint{1.997929in}{1.246813in}}{\pgfqpoint{1.999771in}{1.246813in}}%
\pgfpathlineto{\pgfqpoint{1.999771in}{1.246813in}}%
\pgfpathclose%
\pgfusepath{stroke,fill}%
\end{pgfscope}%
\begin{pgfscope}%
\pgfpathrectangle{\pgfqpoint{0.661006in}{0.524170in}}{\pgfqpoint{4.194036in}{1.071446in}}%
\pgfusepath{clip}%
\pgfsetbuttcap%
\pgfsetroundjoin%
\definecolor{currentfill}{rgb}{0.625704,0.739854,0.849294}%
\pgfsetfillcolor{currentfill}%
\pgfsetfillopacity{0.700000}%
\pgfsetlinewidth{1.003750pt}%
\definecolor{currentstroke}{rgb}{0.625704,0.739854,0.849294}%
\pgfsetstrokecolor{currentstroke}%
\pgfsetstrokeopacity{0.700000}%
\pgfsetdash{}{0pt}%
\pgfpathmoveto{\pgfqpoint{1.946090in}{1.259695in}}%
\pgfpathcurveto{\pgfqpoint{1.947931in}{1.259695in}}{\pgfqpoint{1.949698in}{1.260426in}}{\pgfqpoint{1.951000in}{1.261729in}}%
\pgfpathcurveto{\pgfqpoint{1.952302in}{1.263031in}}{\pgfqpoint{1.953034in}{1.264797in}}{\pgfqpoint{1.953034in}{1.266639in}}%
\pgfpathcurveto{\pgfqpoint{1.953034in}{1.268481in}}{\pgfqpoint{1.952302in}{1.270247in}}{\pgfqpoint{1.951000in}{1.271549in}}%
\pgfpathcurveto{\pgfqpoint{1.949698in}{1.272852in}}{\pgfqpoint{1.947931in}{1.273583in}}{\pgfqpoint{1.946090in}{1.273583in}}%
\pgfpathcurveto{\pgfqpoint{1.944248in}{1.273583in}}{\pgfqpoint{1.942481in}{1.272852in}}{\pgfqpoint{1.941179in}{1.271549in}}%
\pgfpathcurveto{\pgfqpoint{1.939877in}{1.270247in}}{\pgfqpoint{1.939145in}{1.268481in}}{\pgfqpoint{1.939145in}{1.266639in}}%
\pgfpathcurveto{\pgfqpoint{1.939145in}{1.264797in}}{\pgfqpoint{1.939877in}{1.263031in}}{\pgfqpoint{1.941179in}{1.261729in}}%
\pgfpathcurveto{\pgfqpoint{1.942481in}{1.260426in}}{\pgfqpoint{1.944248in}{1.259695in}}{\pgfqpoint{1.946090in}{1.259695in}}%
\pgfpathlineto{\pgfqpoint{1.946090in}{1.259695in}}%
\pgfpathclose%
\pgfusepath{stroke,fill}%
\end{pgfscope}%
\begin{pgfscope}%
\pgfpathrectangle{\pgfqpoint{0.661006in}{0.524170in}}{\pgfqpoint{4.194036in}{1.071446in}}%
\pgfusepath{clip}%
\pgfsetbuttcap%
\pgfsetroundjoin%
\definecolor{currentfill}{rgb}{0.625704,0.739854,0.849294}%
\pgfsetfillcolor{currentfill}%
\pgfsetfillopacity{0.700000}%
\pgfsetlinewidth{1.003750pt}%
\definecolor{currentstroke}{rgb}{0.625704,0.739854,0.849294}%
\pgfsetstrokecolor{currentstroke}%
\pgfsetstrokeopacity{0.700000}%
\pgfsetdash{}{0pt}%
\pgfpathmoveto{\pgfqpoint{1.922107in}{1.263928in}}%
\pgfpathcurveto{\pgfqpoint{1.923949in}{1.263928in}}{\pgfqpoint{1.925716in}{1.264659in}}{\pgfqpoint{1.927018in}{1.265962in}}%
\pgfpathcurveto{\pgfqpoint{1.928320in}{1.267264in}}{\pgfqpoint{1.929052in}{1.269030in}}{\pgfqpoint{1.929052in}{1.270872in}}%
\pgfpathcurveto{\pgfqpoint{1.929052in}{1.272714in}}{\pgfqpoint{1.928320in}{1.274480in}}{\pgfqpoint{1.927018in}{1.275783in}}%
\pgfpathcurveto{\pgfqpoint{1.925716in}{1.277085in}}{\pgfqpoint{1.923949in}{1.277817in}}{\pgfqpoint{1.922107in}{1.277817in}}%
\pgfpathcurveto{\pgfqpoint{1.920266in}{1.277817in}}{\pgfqpoint{1.918499in}{1.277085in}}{\pgfqpoint{1.917197in}{1.275783in}}%
\pgfpathcurveto{\pgfqpoint{1.915895in}{1.274480in}}{\pgfqpoint{1.915163in}{1.272714in}}{\pgfqpoint{1.915163in}{1.270872in}}%
\pgfpathcurveto{\pgfqpoint{1.915163in}{1.269030in}}{\pgfqpoint{1.915895in}{1.267264in}}{\pgfqpoint{1.917197in}{1.265962in}}%
\pgfpathcurveto{\pgfqpoint{1.918499in}{1.264659in}}{\pgfqpoint{1.920266in}{1.263928in}}{\pgfqpoint{1.922107in}{1.263928in}}%
\pgfpathlineto{\pgfqpoint{1.922107in}{1.263928in}}%
\pgfpathclose%
\pgfusepath{stroke,fill}%
\end{pgfscope}%
\begin{pgfscope}%
\pgfpathrectangle{\pgfqpoint{0.661006in}{0.524170in}}{\pgfqpoint{4.194036in}{1.071446in}}%
\pgfusepath{clip}%
\pgfsetbuttcap%
\pgfsetroundjoin%
\definecolor{currentfill}{rgb}{0.625704,0.739854,0.849294}%
\pgfsetfillcolor{currentfill}%
\pgfsetfillopacity{0.700000}%
\pgfsetlinewidth{1.003750pt}%
\definecolor{currentstroke}{rgb}{0.625704,0.739854,0.849294}%
\pgfsetstrokecolor{currentstroke}%
\pgfsetstrokeopacity{0.700000}%
\pgfsetdash{}{0pt}%
\pgfpathmoveto{\pgfqpoint{1.909930in}{1.267005in}}%
\pgfpathcurveto{\pgfqpoint{1.911772in}{1.267005in}}{\pgfqpoint{1.913539in}{1.267736in}}{\pgfqpoint{1.914841in}{1.269038in}}%
\pgfpathcurveto{\pgfqpoint{1.916143in}{1.270341in}}{\pgfqpoint{1.916875in}{1.272107in}}{\pgfqpoint{1.916875in}{1.273949in}}%
\pgfpathcurveto{\pgfqpoint{1.916875in}{1.275791in}}{\pgfqpoint{1.916143in}{1.277557in}}{\pgfqpoint{1.914841in}{1.278859in}}%
\pgfpathcurveto{\pgfqpoint{1.913539in}{1.280162in}}{\pgfqpoint{1.911772in}{1.280893in}}{\pgfqpoint{1.909930in}{1.280893in}}%
\pgfpathcurveto{\pgfqpoint{1.908089in}{1.280893in}}{\pgfqpoint{1.906322in}{1.280162in}}{\pgfqpoint{1.905020in}{1.278859in}}%
\pgfpathcurveto{\pgfqpoint{1.903718in}{1.277557in}}{\pgfqpoint{1.902986in}{1.275791in}}{\pgfqpoint{1.902986in}{1.273949in}}%
\pgfpathcurveto{\pgfqpoint{1.902986in}{1.272107in}}{\pgfqpoint{1.903718in}{1.270341in}}{\pgfqpoint{1.905020in}{1.269038in}}%
\pgfpathcurveto{\pgfqpoint{1.906322in}{1.267736in}}{\pgfqpoint{1.908089in}{1.267005in}}{\pgfqpoint{1.909930in}{1.267005in}}%
\pgfpathlineto{\pgfqpoint{1.909930in}{1.267005in}}%
\pgfpathclose%
\pgfusepath{stroke,fill}%
\end{pgfscope}%
\begin{pgfscope}%
\pgfpathrectangle{\pgfqpoint{0.661006in}{0.524170in}}{\pgfqpoint{4.194036in}{1.071446in}}%
\pgfusepath{clip}%
\pgfsetbuttcap%
\pgfsetroundjoin%
\definecolor{currentfill}{rgb}{0.621713,0.735658,0.847033}%
\pgfsetfillcolor{currentfill}%
\pgfsetfillopacity{0.700000}%
\pgfsetlinewidth{1.003750pt}%
\definecolor{currentstroke}{rgb}{0.621713,0.735658,0.847033}%
\pgfsetstrokecolor{currentstroke}%
\pgfsetstrokeopacity{0.700000}%
\pgfsetdash{}{0pt}%
\pgfpathmoveto{\pgfqpoint{1.844444in}{1.283489in}}%
\pgfpathcurveto{\pgfqpoint{1.846286in}{1.283489in}}{\pgfqpoint{1.848052in}{1.284221in}}{\pgfqpoint{1.849354in}{1.285523in}}%
\pgfpathcurveto{\pgfqpoint{1.850657in}{1.286826in}}{\pgfqpoint{1.851388in}{1.288592in}}{\pgfqpoint{1.851388in}{1.290434in}}%
\pgfpathcurveto{\pgfqpoint{1.851388in}{1.292275in}}{\pgfqpoint{1.850657in}{1.294042in}}{\pgfqpoint{1.849354in}{1.295344in}}%
\pgfpathcurveto{\pgfqpoint{1.848052in}{1.296646in}}{\pgfqpoint{1.846286in}{1.297378in}}{\pgfqpoint{1.844444in}{1.297378in}}%
\pgfpathcurveto{\pgfqpoint{1.842602in}{1.297378in}}{\pgfqpoint{1.840836in}{1.296646in}}{\pgfqpoint{1.839533in}{1.295344in}}%
\pgfpathcurveto{\pgfqpoint{1.838231in}{1.294042in}}{\pgfqpoint{1.837500in}{1.292275in}}{\pgfqpoint{1.837500in}{1.290434in}}%
\pgfpathcurveto{\pgfqpoint{1.837500in}{1.288592in}}{\pgfqpoint{1.838231in}{1.286826in}}{\pgfqpoint{1.839533in}{1.285523in}}%
\pgfpathcurveto{\pgfqpoint{1.840836in}{1.284221in}}{\pgfqpoint{1.842602in}{1.283489in}}{\pgfqpoint{1.844444in}{1.283489in}}%
\pgfpathlineto{\pgfqpoint{1.844444in}{1.283489in}}%
\pgfpathclose%
\pgfusepath{stroke,fill}%
\end{pgfscope}%
\begin{pgfscope}%
\pgfpathrectangle{\pgfqpoint{0.661006in}{0.524170in}}{\pgfqpoint{4.194036in}{1.071446in}}%
\pgfusepath{clip}%
\pgfsetbuttcap%
\pgfsetroundjoin%
\definecolor{currentfill}{rgb}{0.621713,0.735658,0.847033}%
\pgfsetfillcolor{currentfill}%
\pgfsetfillopacity{0.700000}%
\pgfsetlinewidth{1.003750pt}%
\definecolor{currentstroke}{rgb}{0.621713,0.735658,0.847033}%
\pgfsetstrokecolor{currentstroke}%
\pgfsetstrokeopacity{0.700000}%
\pgfsetdash{}{0pt}%
\pgfpathmoveto{\pgfqpoint{1.741952in}{1.306304in}}%
\pgfpathcurveto{\pgfqpoint{1.743793in}{1.306304in}}{\pgfqpoint{1.745560in}{1.307036in}}{\pgfqpoint{1.746862in}{1.308338in}}%
\pgfpathcurveto{\pgfqpoint{1.748164in}{1.309640in}}{\pgfqpoint{1.748896in}{1.311407in}}{\pgfqpoint{1.748896in}{1.313249in}}%
\pgfpathcurveto{\pgfqpoint{1.748896in}{1.315090in}}{\pgfqpoint{1.748164in}{1.316857in}}{\pgfqpoint{1.746862in}{1.318159in}}%
\pgfpathcurveto{\pgfqpoint{1.745560in}{1.319461in}}{\pgfqpoint{1.743793in}{1.320193in}}{\pgfqpoint{1.741952in}{1.320193in}}%
\pgfpathcurveto{\pgfqpoint{1.740110in}{1.320193in}}{\pgfqpoint{1.738343in}{1.319461in}}{\pgfqpoint{1.737041in}{1.318159in}}%
\pgfpathcurveto{\pgfqpoint{1.735739in}{1.316857in}}{\pgfqpoint{1.735007in}{1.315090in}}{\pgfqpoint{1.735007in}{1.313249in}}%
\pgfpathcurveto{\pgfqpoint{1.735007in}{1.311407in}}{\pgfqpoint{1.735739in}{1.309640in}}{\pgfqpoint{1.737041in}{1.308338in}}%
\pgfpathcurveto{\pgfqpoint{1.738343in}{1.307036in}}{\pgfqpoint{1.740110in}{1.306304in}}{\pgfqpoint{1.741952in}{1.306304in}}%
\pgfpathlineto{\pgfqpoint{1.741952in}{1.306304in}}%
\pgfpathclose%
\pgfusepath{stroke,fill}%
\end{pgfscope}%
\begin{pgfscope}%
\pgfpathrectangle{\pgfqpoint{0.661006in}{0.524170in}}{\pgfqpoint{4.194036in}{1.071446in}}%
\pgfusepath{clip}%
\pgfsetbuttcap%
\pgfsetroundjoin%
\definecolor{currentfill}{rgb}{0.617750,0.731450,0.844756}%
\pgfsetfillcolor{currentfill}%
\pgfsetfillopacity{0.700000}%
\pgfsetlinewidth{1.003750pt}%
\definecolor{currentstroke}{rgb}{0.617750,0.731450,0.844756}%
\pgfsetstrokecolor{currentstroke}%
\pgfsetstrokeopacity{0.700000}%
\pgfsetdash{}{0pt}%
\pgfpathmoveto{\pgfqpoint{1.683819in}{1.317494in}}%
\pgfpathcurveto{\pgfqpoint{1.685660in}{1.317494in}}{\pgfqpoint{1.687427in}{1.318225in}}{\pgfqpoint{1.688729in}{1.319528in}}%
\pgfpathcurveto{\pgfqpoint{1.690031in}{1.320830in}}{\pgfqpoint{1.690763in}{1.322596in}}{\pgfqpoint{1.690763in}{1.324438in}}%
\pgfpathcurveto{\pgfqpoint{1.690763in}{1.326280in}}{\pgfqpoint{1.690031in}{1.328046in}}{\pgfqpoint{1.688729in}{1.329349in}}%
\pgfpathcurveto{\pgfqpoint{1.687427in}{1.330651in}}{\pgfqpoint{1.685660in}{1.331383in}}{\pgfqpoint{1.683819in}{1.331383in}}%
\pgfpathcurveto{\pgfqpoint{1.681977in}{1.331383in}}{\pgfqpoint{1.680210in}{1.330651in}}{\pgfqpoint{1.678908in}{1.329349in}}%
\pgfpathcurveto{\pgfqpoint{1.677606in}{1.328046in}}{\pgfqpoint{1.676874in}{1.326280in}}{\pgfqpoint{1.676874in}{1.324438in}}%
\pgfpathcurveto{\pgfqpoint{1.676874in}{1.322596in}}{\pgfqpoint{1.677606in}{1.320830in}}{\pgfqpoint{1.678908in}{1.319528in}}%
\pgfpathcurveto{\pgfqpoint{1.680210in}{1.318225in}}{\pgfqpoint{1.681977in}{1.317494in}}{\pgfqpoint{1.683819in}{1.317494in}}%
\pgfpathlineto{\pgfqpoint{1.683819in}{1.317494in}}%
\pgfpathclose%
\pgfusepath{stroke,fill}%
\end{pgfscope}%
\begin{pgfscope}%
\pgfpathrectangle{\pgfqpoint{0.661006in}{0.524170in}}{\pgfqpoint{4.194036in}{1.071446in}}%
\pgfusepath{clip}%
\pgfsetbuttcap%
\pgfsetroundjoin%
\definecolor{currentfill}{rgb}{0.617750,0.731450,0.844756}%
\pgfsetfillcolor{currentfill}%
\pgfsetfillopacity{0.700000}%
\pgfsetlinewidth{1.003750pt}%
\definecolor{currentstroke}{rgb}{0.617750,0.731450,0.844756}%
\pgfsetstrokecolor{currentstroke}%
\pgfsetstrokeopacity{0.700000}%
\pgfsetdash{}{0pt}%
\pgfpathmoveto{\pgfqpoint{1.711426in}{1.310578in}}%
\pgfpathcurveto{\pgfqpoint{1.713268in}{1.310578in}}{\pgfqpoint{1.715034in}{1.311310in}}{\pgfqpoint{1.716337in}{1.312612in}}%
\pgfpathcurveto{\pgfqpoint{1.717639in}{1.313914in}}{\pgfqpoint{1.718371in}{1.315681in}}{\pgfqpoint{1.718371in}{1.317523in}}%
\pgfpathcurveto{\pgfqpoint{1.718371in}{1.319364in}}{\pgfqpoint{1.717639in}{1.321131in}}{\pgfqpoint{1.716337in}{1.322433in}}%
\pgfpathcurveto{\pgfqpoint{1.715034in}{1.323735in}}{\pgfqpoint{1.713268in}{1.324467in}}{\pgfqpoint{1.711426in}{1.324467in}}%
\pgfpathcurveto{\pgfqpoint{1.709584in}{1.324467in}}{\pgfqpoint{1.707818in}{1.323735in}}{\pgfqpoint{1.706516in}{1.322433in}}%
\pgfpathcurveto{\pgfqpoint{1.705213in}{1.321131in}}{\pgfqpoint{1.704482in}{1.319364in}}{\pgfqpoint{1.704482in}{1.317523in}}%
\pgfpathcurveto{\pgfqpoint{1.704482in}{1.315681in}}{\pgfqpoint{1.705213in}{1.313914in}}{\pgfqpoint{1.706516in}{1.312612in}}%
\pgfpathcurveto{\pgfqpoint{1.707818in}{1.311310in}}{\pgfqpoint{1.709584in}{1.310578in}}{\pgfqpoint{1.711426in}{1.310578in}}%
\pgfpathlineto{\pgfqpoint{1.711426in}{1.310578in}}%
\pgfpathclose%
\pgfusepath{stroke,fill}%
\end{pgfscope}%
\begin{pgfscope}%
\pgfpathrectangle{\pgfqpoint{0.661006in}{0.524170in}}{\pgfqpoint{4.194036in}{1.071446in}}%
\pgfusepath{clip}%
\pgfsetbuttcap%
\pgfsetroundjoin%
\definecolor{currentfill}{rgb}{0.617750,0.731450,0.844756}%
\pgfsetfillcolor{currentfill}%
\pgfsetfillopacity{0.700000}%
\pgfsetlinewidth{1.003750pt}%
\definecolor{currentstroke}{rgb}{0.617750,0.731450,0.844756}%
\pgfsetstrokecolor{currentstroke}%
\pgfsetstrokeopacity{0.700000}%
\pgfsetdash{}{0pt}%
\pgfpathmoveto{\pgfqpoint{1.763667in}{1.297153in}}%
\pgfpathcurveto{\pgfqpoint{1.765508in}{1.297153in}}{\pgfqpoint{1.767275in}{1.297884in}}{\pgfqpoint{1.768577in}{1.299187in}}%
\pgfpathcurveto{\pgfqpoint{1.769879in}{1.300489in}}{\pgfqpoint{1.770611in}{1.302255in}}{\pgfqpoint{1.770611in}{1.304097in}}%
\pgfpathcurveto{\pgfqpoint{1.770611in}{1.305939in}}{\pgfqpoint{1.769879in}{1.307705in}}{\pgfqpoint{1.768577in}{1.309007in}}%
\pgfpathcurveto{\pgfqpoint{1.767275in}{1.310310in}}{\pgfqpoint{1.765508in}{1.311041in}}{\pgfqpoint{1.763667in}{1.311041in}}%
\pgfpathcurveto{\pgfqpoint{1.761825in}{1.311041in}}{\pgfqpoint{1.760058in}{1.310310in}}{\pgfqpoint{1.758756in}{1.309007in}}%
\pgfpathcurveto{\pgfqpoint{1.757454in}{1.307705in}}{\pgfqpoint{1.756722in}{1.305939in}}{\pgfqpoint{1.756722in}{1.304097in}}%
\pgfpathcurveto{\pgfqpoint{1.756722in}{1.302255in}}{\pgfqpoint{1.757454in}{1.300489in}}{\pgfqpoint{1.758756in}{1.299187in}}%
\pgfpathcurveto{\pgfqpoint{1.760058in}{1.297884in}}{\pgfqpoint{1.761825in}{1.297153in}}{\pgfqpoint{1.763667in}{1.297153in}}%
\pgfpathlineto{\pgfqpoint{1.763667in}{1.297153in}}%
\pgfpathclose%
\pgfusepath{stroke,fill}%
\end{pgfscope}%
\begin{pgfscope}%
\pgfpathrectangle{\pgfqpoint{0.661006in}{0.524170in}}{\pgfqpoint{4.194036in}{1.071446in}}%
\pgfusepath{clip}%
\pgfsetbuttcap%
\pgfsetroundjoin%
\definecolor{currentfill}{rgb}{0.617750,0.731450,0.844756}%
\pgfsetfillcolor{currentfill}%
\pgfsetfillopacity{0.700000}%
\pgfsetlinewidth{1.003750pt}%
\definecolor{currentstroke}{rgb}{0.617750,0.731450,0.844756}%
\pgfsetstrokecolor{currentstroke}%
\pgfsetstrokeopacity{0.700000}%
\pgfsetdash{}{0pt}%
\pgfpathmoveto{\pgfqpoint{1.834126in}{1.281950in}}%
\pgfpathcurveto{\pgfqpoint{1.835968in}{1.281950in}}{\pgfqpoint{1.837734in}{1.282682in}}{\pgfqpoint{1.839036in}{1.283984in}}%
\pgfpathcurveto{\pgfqpoint{1.840339in}{1.285286in}}{\pgfqpoint{1.841070in}{1.287053in}}{\pgfqpoint{1.841070in}{1.288894in}}%
\pgfpathcurveto{\pgfqpoint{1.841070in}{1.290736in}}{\pgfqpoint{1.840339in}{1.292503in}}{\pgfqpoint{1.839036in}{1.293805in}}%
\pgfpathcurveto{\pgfqpoint{1.837734in}{1.295107in}}{\pgfqpoint{1.835968in}{1.295839in}}{\pgfqpoint{1.834126in}{1.295839in}}%
\pgfpathcurveto{\pgfqpoint{1.832284in}{1.295839in}}{\pgfqpoint{1.830518in}{1.295107in}}{\pgfqpoint{1.829216in}{1.293805in}}%
\pgfpathcurveto{\pgfqpoint{1.827913in}{1.292503in}}{\pgfqpoint{1.827182in}{1.290736in}}{\pgfqpoint{1.827182in}{1.288894in}}%
\pgfpathcurveto{\pgfqpoint{1.827182in}{1.287053in}}{\pgfqpoint{1.827913in}{1.285286in}}{\pgfqpoint{1.829216in}{1.283984in}}%
\pgfpathcurveto{\pgfqpoint{1.830518in}{1.282682in}}{\pgfqpoint{1.832284in}{1.281950in}}{\pgfqpoint{1.834126in}{1.281950in}}%
\pgfpathlineto{\pgfqpoint{1.834126in}{1.281950in}}%
\pgfpathclose%
\pgfusepath{stroke,fill}%
\end{pgfscope}%
\begin{pgfscope}%
\pgfpathrectangle{\pgfqpoint{0.661006in}{0.524170in}}{\pgfqpoint{4.194036in}{1.071446in}}%
\pgfusepath{clip}%
\pgfsetbuttcap%
\pgfsetroundjoin%
\definecolor{currentfill}{rgb}{0.617750,0.731450,0.844756}%
\pgfsetfillcolor{currentfill}%
\pgfsetfillopacity{0.700000}%
\pgfsetlinewidth{1.003750pt}%
\definecolor{currentstroke}{rgb}{0.617750,0.731450,0.844756}%
\pgfsetstrokecolor{currentstroke}%
\pgfsetstrokeopacity{0.700000}%
\pgfsetdash{}{0pt}%
\pgfpathmoveto{\pgfqpoint{1.897289in}{1.266031in}}%
\pgfpathcurveto{\pgfqpoint{1.899130in}{1.266031in}}{\pgfqpoint{1.900897in}{1.266763in}}{\pgfqpoint{1.902199in}{1.268065in}}%
\pgfpathcurveto{\pgfqpoint{1.903501in}{1.269367in}}{\pgfqpoint{1.904233in}{1.271134in}}{\pgfqpoint{1.904233in}{1.272975in}}%
\pgfpathcurveto{\pgfqpoint{1.904233in}{1.274817in}}{\pgfqpoint{1.903501in}{1.276584in}}{\pgfqpoint{1.902199in}{1.277886in}}%
\pgfpathcurveto{\pgfqpoint{1.900897in}{1.279188in}}{\pgfqpoint{1.899130in}{1.279920in}}{\pgfqpoint{1.897289in}{1.279920in}}%
\pgfpathcurveto{\pgfqpoint{1.895447in}{1.279920in}}{\pgfqpoint{1.893680in}{1.279188in}}{\pgfqpoint{1.892378in}{1.277886in}}%
\pgfpathcurveto{\pgfqpoint{1.891076in}{1.276584in}}{\pgfqpoint{1.890344in}{1.274817in}}{\pgfqpoint{1.890344in}{1.272975in}}%
\pgfpathcurveto{\pgfqpoint{1.890344in}{1.271134in}}{\pgfqpoint{1.891076in}{1.269367in}}{\pgfqpoint{1.892378in}{1.268065in}}%
\pgfpathcurveto{\pgfqpoint{1.893680in}{1.266763in}}{\pgfqpoint{1.895447in}{1.266031in}}{\pgfqpoint{1.897289in}{1.266031in}}%
\pgfpathlineto{\pgfqpoint{1.897289in}{1.266031in}}%
\pgfpathclose%
\pgfusepath{stroke,fill}%
\end{pgfscope}%
\begin{pgfscope}%
\pgfpathrectangle{\pgfqpoint{0.661006in}{0.524170in}}{\pgfqpoint{4.194036in}{1.071446in}}%
\pgfusepath{clip}%
\pgfsetbuttcap%
\pgfsetroundjoin%
\definecolor{currentfill}{rgb}{0.613817,0.727231,0.842462}%
\pgfsetfillcolor{currentfill}%
\pgfsetfillopacity{0.700000}%
\pgfsetlinewidth{1.003750pt}%
\definecolor{currentstroke}{rgb}{0.613817,0.727231,0.842462}%
\pgfsetstrokecolor{currentstroke}%
\pgfsetstrokeopacity{0.700000}%
\pgfsetdash{}{0pt}%
\pgfpathmoveto{\pgfqpoint{1.970909in}{1.250198in}}%
\pgfpathcurveto{\pgfqpoint{1.972750in}{1.250198in}}{\pgfqpoint{1.974517in}{1.250930in}}{\pgfqpoint{1.975819in}{1.252232in}}%
\pgfpathcurveto{\pgfqpoint{1.977121in}{1.253535in}}{\pgfqpoint{1.977853in}{1.255301in}}{\pgfqpoint{1.977853in}{1.257143in}}%
\pgfpathcurveto{\pgfqpoint{1.977853in}{1.258985in}}{\pgfqpoint{1.977121in}{1.260751in}}{\pgfqpoint{1.975819in}{1.262053in}}%
\pgfpathcurveto{\pgfqpoint{1.974517in}{1.263356in}}{\pgfqpoint{1.972750in}{1.264087in}}{\pgfqpoint{1.970909in}{1.264087in}}%
\pgfpathcurveto{\pgfqpoint{1.969067in}{1.264087in}}{\pgfqpoint{1.967300in}{1.263356in}}{\pgfqpoint{1.965998in}{1.262053in}}%
\pgfpathcurveto{\pgfqpoint{1.964696in}{1.260751in}}{\pgfqpoint{1.963964in}{1.258985in}}{\pgfqpoint{1.963964in}{1.257143in}}%
\pgfpathcurveto{\pgfqpoint{1.963964in}{1.255301in}}{\pgfqpoint{1.964696in}{1.253535in}}{\pgfqpoint{1.965998in}{1.252232in}}%
\pgfpathcurveto{\pgfqpoint{1.967300in}{1.250930in}}{\pgfqpoint{1.969067in}{1.250198in}}{\pgfqpoint{1.970909in}{1.250198in}}%
\pgfpathlineto{\pgfqpoint{1.970909in}{1.250198in}}%
\pgfpathclose%
\pgfusepath{stroke,fill}%
\end{pgfscope}%
\begin{pgfscope}%
\pgfpathrectangle{\pgfqpoint{0.661006in}{0.524170in}}{\pgfqpoint{4.194036in}{1.071446in}}%
\pgfusepath{clip}%
\pgfsetbuttcap%
\pgfsetroundjoin%
\definecolor{currentfill}{rgb}{0.613817,0.727231,0.842462}%
\pgfsetfillcolor{currentfill}%
\pgfsetfillopacity{0.700000}%
\pgfsetlinewidth{1.003750pt}%
\definecolor{currentstroke}{rgb}{0.613817,0.727231,0.842462}%
\pgfsetstrokecolor{currentstroke}%
\pgfsetstrokeopacity{0.700000}%
\pgfsetdash{}{0pt}%
\pgfpathmoveto{\pgfqpoint{2.017711in}{1.239269in}}%
\pgfpathcurveto{\pgfqpoint{2.019553in}{1.239269in}}{\pgfqpoint{2.021319in}{1.240001in}}{\pgfqpoint{2.022622in}{1.241303in}}%
\pgfpathcurveto{\pgfqpoint{2.023924in}{1.242605in}}{\pgfqpoint{2.024656in}{1.244372in}}{\pgfqpoint{2.024656in}{1.246214in}}%
\pgfpathcurveto{\pgfqpoint{2.024656in}{1.248055in}}{\pgfqpoint{2.023924in}{1.249822in}}{\pgfqpoint{2.022622in}{1.251124in}}%
\pgfpathcurveto{\pgfqpoint{2.021319in}{1.252426in}}{\pgfqpoint{2.019553in}{1.253158in}}{\pgfqpoint{2.017711in}{1.253158in}}%
\pgfpathcurveto{\pgfqpoint{2.015869in}{1.253158in}}{\pgfqpoint{2.014103in}{1.252426in}}{\pgfqpoint{2.012801in}{1.251124in}}%
\pgfpathcurveto{\pgfqpoint{2.011498in}{1.249822in}}{\pgfqpoint{2.010767in}{1.248055in}}{\pgfqpoint{2.010767in}{1.246214in}}%
\pgfpathcurveto{\pgfqpoint{2.010767in}{1.244372in}}{\pgfqpoint{2.011498in}{1.242605in}}{\pgfqpoint{2.012801in}{1.241303in}}%
\pgfpathcurveto{\pgfqpoint{2.014103in}{1.240001in}}{\pgfqpoint{2.015869in}{1.239269in}}{\pgfqpoint{2.017711in}{1.239269in}}%
\pgfpathlineto{\pgfqpoint{2.017711in}{1.239269in}}%
\pgfpathclose%
\pgfusepath{stroke,fill}%
\end{pgfscope}%
\begin{pgfscope}%
\pgfpathrectangle{\pgfqpoint{0.661006in}{0.524170in}}{\pgfqpoint{4.194036in}{1.071446in}}%
\pgfusepath{clip}%
\pgfsetbuttcap%
\pgfsetroundjoin%
\definecolor{currentfill}{rgb}{0.609912,0.723001,0.840152}%
\pgfsetfillcolor{currentfill}%
\pgfsetfillopacity{0.700000}%
\pgfsetlinewidth{1.003750pt}%
\definecolor{currentstroke}{rgb}{0.609912,0.723001,0.840152}%
\pgfsetstrokecolor{currentstroke}%
\pgfsetstrokeopacity{0.700000}%
\pgfsetdash{}{0pt}%
\pgfpathmoveto{\pgfqpoint{2.048618in}{1.232436in}}%
\pgfpathcurveto{\pgfqpoint{2.050460in}{1.232436in}}{\pgfqpoint{2.052227in}{1.233168in}}{\pgfqpoint{2.053529in}{1.234470in}}%
\pgfpathcurveto{\pgfqpoint{2.054831in}{1.235772in}}{\pgfqpoint{2.055563in}{1.237539in}}{\pgfqpoint{2.055563in}{1.239381in}}%
\pgfpathcurveto{\pgfqpoint{2.055563in}{1.241222in}}{\pgfqpoint{2.054831in}{1.242989in}}{\pgfqpoint{2.053529in}{1.244291in}}%
\pgfpathcurveto{\pgfqpoint{2.052227in}{1.245593in}}{\pgfqpoint{2.050460in}{1.246325in}}{\pgfqpoint{2.048618in}{1.246325in}}%
\pgfpathcurveto{\pgfqpoint{2.046777in}{1.246325in}}{\pgfqpoint{2.045010in}{1.245593in}}{\pgfqpoint{2.043708in}{1.244291in}}%
\pgfpathcurveto{\pgfqpoint{2.042406in}{1.242989in}}{\pgfqpoint{2.041674in}{1.241222in}}{\pgfqpoint{2.041674in}{1.239381in}}%
\pgfpathcurveto{\pgfqpoint{2.041674in}{1.237539in}}{\pgfqpoint{2.042406in}{1.235772in}}{\pgfqpoint{2.043708in}{1.234470in}}%
\pgfpathcurveto{\pgfqpoint{2.045010in}{1.233168in}}{\pgfqpoint{2.046777in}{1.232436in}}{\pgfqpoint{2.048618in}{1.232436in}}%
\pgfpathlineto{\pgfqpoint{2.048618in}{1.232436in}}%
\pgfpathclose%
\pgfusepath{stroke,fill}%
\end{pgfscope}%
\begin{pgfscope}%
\pgfpathrectangle{\pgfqpoint{0.661006in}{0.524170in}}{\pgfqpoint{4.194036in}{1.071446in}}%
\pgfusepath{clip}%
\pgfsetbuttcap%
\pgfsetroundjoin%
\definecolor{currentfill}{rgb}{0.609912,0.723001,0.840152}%
\pgfsetfillcolor{currentfill}%
\pgfsetfillopacity{0.700000}%
\pgfsetlinewidth{1.003750pt}%
\definecolor{currentstroke}{rgb}{0.609912,0.723001,0.840152}%
\pgfsetstrokecolor{currentstroke}%
\pgfsetstrokeopacity{0.700000}%
\pgfsetdash{}{0pt}%
\pgfpathmoveto{\pgfqpoint{2.085196in}{1.224436in}}%
\pgfpathcurveto{\pgfqpoint{2.087038in}{1.224436in}}{\pgfqpoint{2.088804in}{1.225167in}}{\pgfqpoint{2.090107in}{1.226470in}}%
\pgfpathcurveto{\pgfqpoint{2.091409in}{1.227772in}}{\pgfqpoint{2.092140in}{1.229538in}}{\pgfqpoint{2.092140in}{1.231380in}}%
\pgfpathcurveto{\pgfqpoint{2.092140in}{1.233222in}}{\pgfqpoint{2.091409in}{1.234988in}}{\pgfqpoint{2.090107in}{1.236290in}}%
\pgfpathcurveto{\pgfqpoint{2.088804in}{1.237593in}}{\pgfqpoint{2.087038in}{1.238324in}}{\pgfqpoint{2.085196in}{1.238324in}}%
\pgfpathcurveto{\pgfqpoint{2.083354in}{1.238324in}}{\pgfqpoint{2.081588in}{1.237593in}}{\pgfqpoint{2.080286in}{1.236290in}}%
\pgfpathcurveto{\pgfqpoint{2.078983in}{1.234988in}}{\pgfqpoint{2.078252in}{1.233222in}}{\pgfqpoint{2.078252in}{1.231380in}}%
\pgfpathcurveto{\pgfqpoint{2.078252in}{1.229538in}}{\pgfqpoint{2.078983in}{1.227772in}}{\pgfqpoint{2.080286in}{1.226470in}}%
\pgfpathcurveto{\pgfqpoint{2.081588in}{1.225167in}}{\pgfqpoint{2.083354in}{1.224436in}}{\pgfqpoint{2.085196in}{1.224436in}}%
\pgfpathlineto{\pgfqpoint{2.085196in}{1.224436in}}%
\pgfpathclose%
\pgfusepath{stroke,fill}%
\end{pgfscope}%
\begin{pgfscope}%
\pgfpathrectangle{\pgfqpoint{0.661006in}{0.524170in}}{\pgfqpoint{4.194036in}{1.071446in}}%
\pgfusepath{clip}%
\pgfsetbuttcap%
\pgfsetroundjoin%
\definecolor{currentfill}{rgb}{0.606034,0.718760,0.837824}%
\pgfsetfillcolor{currentfill}%
\pgfsetfillopacity{0.700000}%
\pgfsetlinewidth{1.003750pt}%
\definecolor{currentstroke}{rgb}{0.606034,0.718760,0.837824}%
\pgfsetstrokecolor{currentstroke}%
\pgfsetstrokeopacity{0.700000}%
\pgfsetdash{}{0pt}%
\pgfpathmoveto{\pgfqpoint{2.108481in}{1.219100in}}%
\pgfpathcurveto{\pgfqpoint{2.110323in}{1.219100in}}{\pgfqpoint{2.112089in}{1.219831in}}{\pgfqpoint{2.113392in}{1.221134in}}%
\pgfpathcurveto{\pgfqpoint{2.114694in}{1.222436in}}{\pgfqpoint{2.115426in}{1.224202in}}{\pgfqpoint{2.115426in}{1.226044in}}%
\pgfpathcurveto{\pgfqpoint{2.115426in}{1.227886in}}{\pgfqpoint{2.114694in}{1.229652in}}{\pgfqpoint{2.113392in}{1.230955in}}%
\pgfpathcurveto{\pgfqpoint{2.112089in}{1.232257in}}{\pgfqpoint{2.110323in}{1.232989in}}{\pgfqpoint{2.108481in}{1.232989in}}%
\pgfpathcurveto{\pgfqpoint{2.106639in}{1.232989in}}{\pgfqpoint{2.104873in}{1.232257in}}{\pgfqpoint{2.103571in}{1.230955in}}%
\pgfpathcurveto{\pgfqpoint{2.102268in}{1.229652in}}{\pgfqpoint{2.101537in}{1.227886in}}{\pgfqpoint{2.101537in}{1.226044in}}%
\pgfpathcurveto{\pgfqpoint{2.101537in}{1.224202in}}{\pgfqpoint{2.102268in}{1.222436in}}{\pgfqpoint{2.103571in}{1.221134in}}%
\pgfpathcurveto{\pgfqpoint{2.104873in}{1.219831in}}{\pgfqpoint{2.106639in}{1.219100in}}{\pgfqpoint{2.108481in}{1.219100in}}%
\pgfpathlineto{\pgfqpoint{2.108481in}{1.219100in}}%
\pgfpathclose%
\pgfusepath{stroke,fill}%
\end{pgfscope}%
\begin{pgfscope}%
\pgfpathrectangle{\pgfqpoint{0.661006in}{0.524170in}}{\pgfqpoint{4.194036in}{1.071446in}}%
\pgfusepath{clip}%
\pgfsetbuttcap%
\pgfsetroundjoin%
\definecolor{currentfill}{rgb}{0.606034,0.718760,0.837824}%
\pgfsetfillcolor{currentfill}%
\pgfsetfillopacity{0.700000}%
\pgfsetlinewidth{1.003750pt}%
\definecolor{currentstroke}{rgb}{0.606034,0.718760,0.837824}%
\pgfsetstrokecolor{currentstroke}%
\pgfsetstrokeopacity{0.700000}%
\pgfsetdash{}{0pt}%
\pgfpathmoveto{\pgfqpoint{2.099929in}{1.221947in}}%
\pgfpathcurveto{\pgfqpoint{2.101771in}{1.221947in}}{\pgfqpoint{2.103538in}{1.222679in}}{\pgfqpoint{2.104840in}{1.223981in}}%
\pgfpathcurveto{\pgfqpoint{2.106142in}{1.225284in}}{\pgfqpoint{2.106874in}{1.227050in}}{\pgfqpoint{2.106874in}{1.228892in}}%
\pgfpathcurveto{\pgfqpoint{2.106874in}{1.230733in}}{\pgfqpoint{2.106142in}{1.232500in}}{\pgfqpoint{2.104840in}{1.233802in}}%
\pgfpathcurveto{\pgfqpoint{2.103538in}{1.235104in}}{\pgfqpoint{2.101771in}{1.235836in}}{\pgfqpoint{2.099929in}{1.235836in}}%
\pgfpathcurveto{\pgfqpoint{2.098088in}{1.235836in}}{\pgfqpoint{2.096321in}{1.235104in}}{\pgfqpoint{2.095019in}{1.233802in}}%
\pgfpathcurveto{\pgfqpoint{2.093717in}{1.232500in}}{\pgfqpoint{2.092985in}{1.230733in}}{\pgfqpoint{2.092985in}{1.228892in}}%
\pgfpathcurveto{\pgfqpoint{2.092985in}{1.227050in}}{\pgfqpoint{2.093717in}{1.225284in}}{\pgfqpoint{2.095019in}{1.223981in}}%
\pgfpathcurveto{\pgfqpoint{2.096321in}{1.222679in}}{\pgfqpoint{2.098088in}{1.221947in}}{\pgfqpoint{2.099929in}{1.221947in}}%
\pgfpathlineto{\pgfqpoint{2.099929in}{1.221947in}}%
\pgfpathclose%
\pgfusepath{stroke,fill}%
\end{pgfscope}%
\begin{pgfscope}%
\pgfpathrectangle{\pgfqpoint{0.661006in}{0.524170in}}{\pgfqpoint{4.194036in}{1.071446in}}%
\pgfusepath{clip}%
\pgfsetbuttcap%
\pgfsetroundjoin%
\definecolor{currentfill}{rgb}{0.606034,0.718760,0.837824}%
\pgfsetfillcolor{currentfill}%
\pgfsetfillopacity{0.700000}%
\pgfsetlinewidth{1.003750pt}%
\definecolor{currentstroke}{rgb}{0.606034,0.718760,0.837824}%
\pgfsetstrokecolor{currentstroke}%
\pgfsetstrokeopacity{0.700000}%
\pgfsetdash{}{0pt}%
\pgfpathmoveto{\pgfqpoint{2.047224in}{1.234004in}}%
\pgfpathcurveto{\pgfqpoint{2.049066in}{1.234004in}}{\pgfqpoint{2.050832in}{1.234736in}}{\pgfqpoint{2.052135in}{1.236038in}}%
\pgfpathcurveto{\pgfqpoint{2.053437in}{1.237340in}}{\pgfqpoint{2.054169in}{1.239107in}}{\pgfqpoint{2.054169in}{1.240948in}}%
\pgfpathcurveto{\pgfqpoint{2.054169in}{1.242790in}}{\pgfqpoint{2.053437in}{1.244557in}}{\pgfqpoint{2.052135in}{1.245859in}}%
\pgfpathcurveto{\pgfqpoint{2.050832in}{1.247161in}}{\pgfqpoint{2.049066in}{1.247893in}}{\pgfqpoint{2.047224in}{1.247893in}}%
\pgfpathcurveto{\pgfqpoint{2.045382in}{1.247893in}}{\pgfqpoint{2.043616in}{1.247161in}}{\pgfqpoint{2.042314in}{1.245859in}}%
\pgfpathcurveto{\pgfqpoint{2.041011in}{1.244557in}}{\pgfqpoint{2.040280in}{1.242790in}}{\pgfqpoint{2.040280in}{1.240948in}}%
\pgfpathcurveto{\pgfqpoint{2.040280in}{1.239107in}}{\pgfqpoint{2.041011in}{1.237340in}}{\pgfqpoint{2.042314in}{1.236038in}}%
\pgfpathcurveto{\pgfqpoint{2.043616in}{1.234736in}}{\pgfqpoint{2.045382in}{1.234004in}}{\pgfqpoint{2.047224in}{1.234004in}}%
\pgfpathlineto{\pgfqpoint{2.047224in}{1.234004in}}%
\pgfpathclose%
\pgfusepath{stroke,fill}%
\end{pgfscope}%
\begin{pgfscope}%
\pgfpathrectangle{\pgfqpoint{0.661006in}{0.524170in}}{\pgfqpoint{4.194036in}{1.071446in}}%
\pgfusepath{clip}%
\pgfsetbuttcap%
\pgfsetroundjoin%
\definecolor{currentfill}{rgb}{0.606034,0.718760,0.837824}%
\pgfsetfillcolor{currentfill}%
\pgfsetfillopacity{0.700000}%
\pgfsetlinewidth{1.003750pt}%
\definecolor{currentstroke}{rgb}{0.606034,0.718760,0.837824}%
\pgfsetstrokecolor{currentstroke}%
\pgfsetstrokeopacity{0.700000}%
\pgfsetdash{}{0pt}%
\pgfpathmoveto{\pgfqpoint{1.991574in}{1.246840in}}%
\pgfpathcurveto{\pgfqpoint{1.993416in}{1.246840in}}{\pgfqpoint{1.995182in}{1.247571in}}{\pgfqpoint{1.996485in}{1.248874in}}%
\pgfpathcurveto{\pgfqpoint{1.997787in}{1.250176in}}{\pgfqpoint{1.998519in}{1.251942in}}{\pgfqpoint{1.998519in}{1.253784in}}%
\pgfpathcurveto{\pgfqpoint{1.998519in}{1.255626in}}{\pgfqpoint{1.997787in}{1.257392in}}{\pgfqpoint{1.996485in}{1.258695in}}%
\pgfpathcurveto{\pgfqpoint{1.995182in}{1.259997in}}{\pgfqpoint{1.993416in}{1.260729in}}{\pgfqpoint{1.991574in}{1.260729in}}%
\pgfpathcurveto{\pgfqpoint{1.989733in}{1.260729in}}{\pgfqpoint{1.987966in}{1.259997in}}{\pgfqpoint{1.986664in}{1.258695in}}%
\pgfpathcurveto{\pgfqpoint{1.985362in}{1.257392in}}{\pgfqpoint{1.984630in}{1.255626in}}{\pgfqpoint{1.984630in}{1.253784in}}%
\pgfpathcurveto{\pgfqpoint{1.984630in}{1.251942in}}{\pgfqpoint{1.985362in}{1.250176in}}{\pgfqpoint{1.986664in}{1.248874in}}%
\pgfpathcurveto{\pgfqpoint{1.987966in}{1.247571in}}{\pgfqpoint{1.989733in}{1.246840in}}{\pgfqpoint{1.991574in}{1.246840in}}%
\pgfpathlineto{\pgfqpoint{1.991574in}{1.246840in}}%
\pgfpathclose%
\pgfusepath{stroke,fill}%
\end{pgfscope}%
\begin{pgfscope}%
\pgfpathrectangle{\pgfqpoint{0.661006in}{0.524170in}}{\pgfqpoint{4.194036in}{1.071446in}}%
\pgfusepath{clip}%
\pgfsetbuttcap%
\pgfsetroundjoin%
\definecolor{currentfill}{rgb}{0.606034,0.718760,0.837824}%
\pgfsetfillcolor{currentfill}%
\pgfsetfillopacity{0.700000}%
\pgfsetlinewidth{1.003750pt}%
\definecolor{currentstroke}{rgb}{0.606034,0.718760,0.837824}%
\pgfsetstrokecolor{currentstroke}%
\pgfsetstrokeopacity{0.700000}%
\pgfsetdash{}{0pt}%
\pgfpathmoveto{\pgfqpoint{1.952736in}{1.254539in}}%
\pgfpathcurveto{\pgfqpoint{1.954578in}{1.254539in}}{\pgfqpoint{1.956344in}{1.255271in}}{\pgfqpoint{1.957646in}{1.256573in}}%
\pgfpathcurveto{\pgfqpoint{1.958949in}{1.257875in}}{\pgfqpoint{1.959680in}{1.259642in}}{\pgfqpoint{1.959680in}{1.261484in}}%
\pgfpathcurveto{\pgfqpoint{1.959680in}{1.263325in}}{\pgfqpoint{1.958949in}{1.265092in}}{\pgfqpoint{1.957646in}{1.266394in}}%
\pgfpathcurveto{\pgfqpoint{1.956344in}{1.267696in}}{\pgfqpoint{1.954578in}{1.268428in}}{\pgfqpoint{1.952736in}{1.268428in}}%
\pgfpathcurveto{\pgfqpoint{1.950894in}{1.268428in}}{\pgfqpoint{1.949128in}{1.267696in}}{\pgfqpoint{1.947825in}{1.266394in}}%
\pgfpathcurveto{\pgfqpoint{1.946523in}{1.265092in}}{\pgfqpoint{1.945791in}{1.263325in}}{\pgfqpoint{1.945791in}{1.261484in}}%
\pgfpathcurveto{\pgfqpoint{1.945791in}{1.259642in}}{\pgfqpoint{1.946523in}{1.257875in}}{\pgfqpoint{1.947825in}{1.256573in}}%
\pgfpathcurveto{\pgfqpoint{1.949128in}{1.255271in}}{\pgfqpoint{1.950894in}{1.254539in}}{\pgfqpoint{1.952736in}{1.254539in}}%
\pgfpathlineto{\pgfqpoint{1.952736in}{1.254539in}}%
\pgfpathclose%
\pgfusepath{stroke,fill}%
\end{pgfscope}%
\begin{pgfscope}%
\pgfpathrectangle{\pgfqpoint{0.661006in}{0.524170in}}{\pgfqpoint{4.194036in}{1.071446in}}%
\pgfusepath{clip}%
\pgfsetbuttcap%
\pgfsetroundjoin%
\definecolor{currentfill}{rgb}{0.602185,0.714507,0.835480}%
\pgfsetfillcolor{currentfill}%
\pgfsetfillopacity{0.700000}%
\pgfsetlinewidth{1.003750pt}%
\definecolor{currentstroke}{rgb}{0.602185,0.714507,0.835480}%
\pgfsetstrokecolor{currentstroke}%
\pgfsetstrokeopacity{0.700000}%
\pgfsetdash{}{0pt}%
\pgfpathmoveto{\pgfqpoint{1.938560in}{1.258315in}}%
\pgfpathcurveto{\pgfqpoint{1.940402in}{1.258315in}}{\pgfqpoint{1.942169in}{1.259047in}}{\pgfqpoint{1.943471in}{1.260349in}}%
\pgfpathcurveto{\pgfqpoint{1.944773in}{1.261651in}}{\pgfqpoint{1.945505in}{1.263418in}}{\pgfqpoint{1.945505in}{1.265259in}}%
\pgfpathcurveto{\pgfqpoint{1.945505in}{1.267101in}}{\pgfqpoint{1.944773in}{1.268868in}}{\pgfqpoint{1.943471in}{1.270170in}}%
\pgfpathcurveto{\pgfqpoint{1.942169in}{1.271472in}}{\pgfqpoint{1.940402in}{1.272204in}}{\pgfqpoint{1.938560in}{1.272204in}}%
\pgfpathcurveto{\pgfqpoint{1.936719in}{1.272204in}}{\pgfqpoint{1.934952in}{1.271472in}}{\pgfqpoint{1.933650in}{1.270170in}}%
\pgfpathcurveto{\pgfqpoint{1.932348in}{1.268868in}}{\pgfqpoint{1.931616in}{1.267101in}}{\pgfqpoint{1.931616in}{1.265259in}}%
\pgfpathcurveto{\pgfqpoint{1.931616in}{1.263418in}}{\pgfqpoint{1.932348in}{1.261651in}}{\pgfqpoint{1.933650in}{1.260349in}}%
\pgfpathcurveto{\pgfqpoint{1.934952in}{1.259047in}}{\pgfqpoint{1.936719in}{1.258315in}}{\pgfqpoint{1.938560in}{1.258315in}}%
\pgfpathlineto{\pgfqpoint{1.938560in}{1.258315in}}%
\pgfpathclose%
\pgfusepath{stroke,fill}%
\end{pgfscope}%
\begin{pgfscope}%
\pgfpathrectangle{\pgfqpoint{0.661006in}{0.524170in}}{\pgfqpoint{4.194036in}{1.071446in}}%
\pgfusepath{clip}%
\pgfsetbuttcap%
\pgfsetroundjoin%
\definecolor{currentfill}{rgb}{0.602185,0.714507,0.835480}%
\pgfsetfillcolor{currentfill}%
\pgfsetfillopacity{0.700000}%
\pgfsetlinewidth{1.003750pt}%
\definecolor{currentstroke}{rgb}{0.602185,0.714507,0.835480}%
\pgfsetstrokecolor{currentstroke}%
\pgfsetstrokeopacity{0.700000}%
\pgfsetdash{}{0pt}%
\pgfpathmoveto{\pgfqpoint{1.926197in}{1.260608in}}%
\pgfpathcurveto{\pgfqpoint{1.928039in}{1.260608in}}{\pgfqpoint{1.929806in}{1.261340in}}{\pgfqpoint{1.931108in}{1.262642in}}%
\pgfpathcurveto{\pgfqpoint{1.932410in}{1.263944in}}{\pgfqpoint{1.933142in}{1.265711in}}{\pgfqpoint{1.933142in}{1.267552in}}%
\pgfpathcurveto{\pgfqpoint{1.933142in}{1.269394in}}{\pgfqpoint{1.932410in}{1.271161in}}{\pgfqpoint{1.931108in}{1.272463in}}%
\pgfpathcurveto{\pgfqpoint{1.929806in}{1.273765in}}{\pgfqpoint{1.928039in}{1.274497in}}{\pgfqpoint{1.926197in}{1.274497in}}%
\pgfpathcurveto{\pgfqpoint{1.924356in}{1.274497in}}{\pgfqpoint{1.922589in}{1.273765in}}{\pgfqpoint{1.921287in}{1.272463in}}%
\pgfpathcurveto{\pgfqpoint{1.919985in}{1.271161in}}{\pgfqpoint{1.919253in}{1.269394in}}{\pgfqpoint{1.919253in}{1.267552in}}%
\pgfpathcurveto{\pgfqpoint{1.919253in}{1.265711in}}{\pgfqpoint{1.919985in}{1.263944in}}{\pgfqpoint{1.921287in}{1.262642in}}%
\pgfpathcurveto{\pgfqpoint{1.922589in}{1.261340in}}{\pgfqpoint{1.924356in}{1.260608in}}{\pgfqpoint{1.926197in}{1.260608in}}%
\pgfpathlineto{\pgfqpoint{1.926197in}{1.260608in}}%
\pgfpathclose%
\pgfusepath{stroke,fill}%
\end{pgfscope}%
\begin{pgfscope}%
\pgfpathrectangle{\pgfqpoint{0.661006in}{0.524170in}}{\pgfqpoint{4.194036in}{1.071446in}}%
\pgfusepath{clip}%
\pgfsetbuttcap%
\pgfsetroundjoin%
\definecolor{currentfill}{rgb}{0.598363,0.710245,0.833117}%
\pgfsetfillcolor{currentfill}%
\pgfsetfillopacity{0.700000}%
\pgfsetlinewidth{1.003750pt}%
\definecolor{currentstroke}{rgb}{0.598363,0.710245,0.833117}%
\pgfsetstrokecolor{currentstroke}%
\pgfsetstrokeopacity{0.700000}%
\pgfsetdash{}{0pt}%
\pgfpathmoveto{\pgfqpoint{1.926383in}{1.259639in}}%
\pgfpathcurveto{\pgfqpoint{1.928225in}{1.259639in}}{\pgfqpoint{1.929992in}{1.260370in}}{\pgfqpoint{1.931294in}{1.261672in}}%
\pgfpathcurveto{\pgfqpoint{1.932596in}{1.262975in}}{\pgfqpoint{1.933328in}{1.264741in}}{\pgfqpoint{1.933328in}{1.266583in}}%
\pgfpathcurveto{\pgfqpoint{1.933328in}{1.268425in}}{\pgfqpoint{1.932596in}{1.270191in}}{\pgfqpoint{1.931294in}{1.271493in}}%
\pgfpathcurveto{\pgfqpoint{1.929992in}{1.272796in}}{\pgfqpoint{1.928225in}{1.273527in}}{\pgfqpoint{1.926383in}{1.273527in}}%
\pgfpathcurveto{\pgfqpoint{1.924542in}{1.273527in}}{\pgfqpoint{1.922775in}{1.272796in}}{\pgfqpoint{1.921473in}{1.271493in}}%
\pgfpathcurveto{\pgfqpoint{1.920171in}{1.270191in}}{\pgfqpoint{1.919439in}{1.268425in}}{\pgfqpoint{1.919439in}{1.266583in}}%
\pgfpathcurveto{\pgfqpoint{1.919439in}{1.264741in}}{\pgfqpoint{1.920171in}{1.262975in}}{\pgfqpoint{1.921473in}{1.261672in}}%
\pgfpathcurveto{\pgfqpoint{1.922775in}{1.260370in}}{\pgfqpoint{1.924542in}{1.259639in}}{\pgfqpoint{1.926383in}{1.259639in}}%
\pgfpathlineto{\pgfqpoint{1.926383in}{1.259639in}}%
\pgfpathclose%
\pgfusepath{stroke,fill}%
\end{pgfscope}%
\begin{pgfscope}%
\pgfpathrectangle{\pgfqpoint{0.661006in}{0.524170in}}{\pgfqpoint{4.194036in}{1.071446in}}%
\pgfusepath{clip}%
\pgfsetbuttcap%
\pgfsetroundjoin%
\definecolor{currentfill}{rgb}{0.598363,0.710245,0.833117}%
\pgfsetfillcolor{currentfill}%
\pgfsetfillopacity{0.700000}%
\pgfsetlinewidth{1.003750pt}%
\definecolor{currentstroke}{rgb}{0.598363,0.710245,0.833117}%
\pgfsetstrokecolor{currentstroke}%
\pgfsetstrokeopacity{0.700000}%
\pgfsetdash{}{0pt}%
\pgfpathmoveto{\pgfqpoint{1.959708in}{1.251166in}}%
\pgfpathcurveto{\pgfqpoint{1.961549in}{1.251166in}}{\pgfqpoint{1.963316in}{1.251898in}}{\pgfqpoint{1.964618in}{1.253200in}}%
\pgfpathcurveto{\pgfqpoint{1.965920in}{1.254503in}}{\pgfqpoint{1.966652in}{1.256269in}}{\pgfqpoint{1.966652in}{1.258111in}}%
\pgfpathcurveto{\pgfqpoint{1.966652in}{1.259953in}}{\pgfqpoint{1.965920in}{1.261719in}}{\pgfqpoint{1.964618in}{1.263021in}}%
\pgfpathcurveto{\pgfqpoint{1.963316in}{1.264324in}}{\pgfqpoint{1.961549in}{1.265055in}}{\pgfqpoint{1.959708in}{1.265055in}}%
\pgfpathcurveto{\pgfqpoint{1.957866in}{1.265055in}}{\pgfqpoint{1.956099in}{1.264324in}}{\pgfqpoint{1.954797in}{1.263021in}}%
\pgfpathcurveto{\pgfqpoint{1.953495in}{1.261719in}}{\pgfqpoint{1.952763in}{1.259953in}}{\pgfqpoint{1.952763in}{1.258111in}}%
\pgfpathcurveto{\pgfqpoint{1.952763in}{1.256269in}}{\pgfqpoint{1.953495in}{1.254503in}}{\pgfqpoint{1.954797in}{1.253200in}}%
\pgfpathcurveto{\pgfqpoint{1.956099in}{1.251898in}}{\pgfqpoint{1.957866in}{1.251166in}}{\pgfqpoint{1.959708in}{1.251166in}}%
\pgfpathlineto{\pgfqpoint{1.959708in}{1.251166in}}%
\pgfpathclose%
\pgfusepath{stroke,fill}%
\end{pgfscope}%
\begin{pgfscope}%
\pgfpathrectangle{\pgfqpoint{0.661006in}{0.524170in}}{\pgfqpoint{4.194036in}{1.071446in}}%
\pgfusepath{clip}%
\pgfsetbuttcap%
\pgfsetroundjoin%
\definecolor{currentfill}{rgb}{0.598363,0.710245,0.833117}%
\pgfsetfillcolor{currentfill}%
\pgfsetfillopacity{0.700000}%
\pgfsetlinewidth{1.003750pt}%
\definecolor{currentstroke}{rgb}{0.598363,0.710245,0.833117}%
\pgfsetstrokecolor{currentstroke}%
\pgfsetstrokeopacity{0.700000}%
\pgfsetdash{}{0pt}%
\pgfpathmoveto{\pgfqpoint{2.007440in}{1.239486in}}%
\pgfpathcurveto{\pgfqpoint{2.009281in}{1.239486in}}{\pgfqpoint{2.011048in}{1.240218in}}{\pgfqpoint{2.012350in}{1.241520in}}%
\pgfpathcurveto{\pgfqpoint{2.013652in}{1.242823in}}{\pgfqpoint{2.014384in}{1.244589in}}{\pgfqpoint{2.014384in}{1.246431in}}%
\pgfpathcurveto{\pgfqpoint{2.014384in}{1.248272in}}{\pgfqpoint{2.013652in}{1.250039in}}{\pgfqpoint{2.012350in}{1.251341in}}%
\pgfpathcurveto{\pgfqpoint{2.011048in}{1.252644in}}{\pgfqpoint{2.009281in}{1.253375in}}{\pgfqpoint{2.007440in}{1.253375in}}%
\pgfpathcurveto{\pgfqpoint{2.005598in}{1.253375in}}{\pgfqpoint{2.003831in}{1.252644in}}{\pgfqpoint{2.002529in}{1.251341in}}%
\pgfpathcurveto{\pgfqpoint{2.001227in}{1.250039in}}{\pgfqpoint{2.000495in}{1.248272in}}{\pgfqpoint{2.000495in}{1.246431in}}%
\pgfpathcurveto{\pgfqpoint{2.000495in}{1.244589in}}{\pgfqpoint{2.001227in}{1.242823in}}{\pgfqpoint{2.002529in}{1.241520in}}%
\pgfpathcurveto{\pgfqpoint{2.003831in}{1.240218in}}{\pgfqpoint{2.005598in}{1.239486in}}{\pgfqpoint{2.007440in}{1.239486in}}%
\pgfpathlineto{\pgfqpoint{2.007440in}{1.239486in}}%
\pgfpathclose%
\pgfusepath{stroke,fill}%
\end{pgfscope}%
\begin{pgfscope}%
\pgfpathrectangle{\pgfqpoint{0.661006in}{0.524170in}}{\pgfqpoint{4.194036in}{1.071446in}}%
\pgfusepath{clip}%
\pgfsetbuttcap%
\pgfsetroundjoin%
\definecolor{currentfill}{rgb}{0.598363,0.710245,0.833117}%
\pgfsetfillcolor{currentfill}%
\pgfsetfillopacity{0.700000}%
\pgfsetlinewidth{1.003750pt}%
\definecolor{currentstroke}{rgb}{0.598363,0.710245,0.833117}%
\pgfsetstrokecolor{currentstroke}%
\pgfsetstrokeopacity{0.700000}%
\pgfsetdash{}{0pt}%
\pgfpathmoveto{\pgfqpoint{2.049316in}{1.229617in}}%
\pgfpathcurveto{\pgfqpoint{2.051157in}{1.229617in}}{\pgfqpoint{2.052924in}{1.230349in}}{\pgfqpoint{2.054226in}{1.231651in}}%
\pgfpathcurveto{\pgfqpoint{2.055528in}{1.232954in}}{\pgfqpoint{2.056260in}{1.234720in}}{\pgfqpoint{2.056260in}{1.236562in}}%
\pgfpathcurveto{\pgfqpoint{2.056260in}{1.238403in}}{\pgfqpoint{2.055528in}{1.240170in}}{\pgfqpoint{2.054226in}{1.241472in}}%
\pgfpathcurveto{\pgfqpoint{2.052924in}{1.242774in}}{\pgfqpoint{2.051157in}{1.243506in}}{\pgfqpoint{2.049316in}{1.243506in}}%
\pgfpathcurveto{\pgfqpoint{2.047474in}{1.243506in}}{\pgfqpoint{2.045707in}{1.242774in}}{\pgfqpoint{2.044405in}{1.241472in}}%
\pgfpathcurveto{\pgfqpoint{2.043103in}{1.240170in}}{\pgfqpoint{2.042371in}{1.238403in}}{\pgfqpoint{2.042371in}{1.236562in}}%
\pgfpathcurveto{\pgfqpoint{2.042371in}{1.234720in}}{\pgfqpoint{2.043103in}{1.232954in}}{\pgfqpoint{2.044405in}{1.231651in}}%
\pgfpathcurveto{\pgfqpoint{2.045707in}{1.230349in}}{\pgfqpoint{2.047474in}{1.229617in}}{\pgfqpoint{2.049316in}{1.229617in}}%
\pgfpathlineto{\pgfqpoint{2.049316in}{1.229617in}}%
\pgfpathclose%
\pgfusepath{stroke,fill}%
\end{pgfscope}%
\begin{pgfscope}%
\pgfpathrectangle{\pgfqpoint{0.661006in}{0.524170in}}{\pgfqpoint{4.194036in}{1.071446in}}%
\pgfusepath{clip}%
\pgfsetbuttcap%
\pgfsetroundjoin%
\definecolor{currentfill}{rgb}{0.598363,0.710245,0.833117}%
\pgfsetfillcolor{currentfill}%
\pgfsetfillopacity{0.700000}%
\pgfsetlinewidth{1.003750pt}%
\definecolor{currentstroke}{rgb}{0.598363,0.710245,0.833117}%
\pgfsetstrokecolor{currentstroke}%
\pgfsetstrokeopacity{0.700000}%
\pgfsetdash{}{0pt}%
\pgfpathmoveto{\pgfqpoint{2.075157in}{1.223312in}}%
\pgfpathcurveto{\pgfqpoint{2.076999in}{1.223312in}}{\pgfqpoint{2.078765in}{1.224044in}}{\pgfqpoint{2.080067in}{1.225346in}}%
\pgfpathcurveto{\pgfqpoint{2.081370in}{1.226649in}}{\pgfqpoint{2.082101in}{1.228415in}}{\pgfqpoint{2.082101in}{1.230257in}}%
\pgfpathcurveto{\pgfqpoint{2.082101in}{1.232099in}}{\pgfqpoint{2.081370in}{1.233865in}}{\pgfqpoint{2.080067in}{1.235167in}}%
\pgfpathcurveto{\pgfqpoint{2.078765in}{1.236470in}}{\pgfqpoint{2.076999in}{1.237201in}}{\pgfqpoint{2.075157in}{1.237201in}}%
\pgfpathcurveto{\pgfqpoint{2.073315in}{1.237201in}}{\pgfqpoint{2.071549in}{1.236470in}}{\pgfqpoint{2.070246in}{1.235167in}}%
\pgfpathcurveto{\pgfqpoint{2.068944in}{1.233865in}}{\pgfqpoint{2.068213in}{1.232099in}}{\pgfqpoint{2.068213in}{1.230257in}}%
\pgfpathcurveto{\pgfqpoint{2.068213in}{1.228415in}}{\pgfqpoint{2.068944in}{1.226649in}}{\pgfqpoint{2.070246in}{1.225346in}}%
\pgfpathcurveto{\pgfqpoint{2.071549in}{1.224044in}}{\pgfqpoint{2.073315in}{1.223312in}}{\pgfqpoint{2.075157in}{1.223312in}}%
\pgfpathlineto{\pgfqpoint{2.075157in}{1.223312in}}%
\pgfpathclose%
\pgfusepath{stroke,fill}%
\end{pgfscope}%
\begin{pgfscope}%
\pgfpathrectangle{\pgfqpoint{0.661006in}{0.524170in}}{\pgfqpoint{4.194036in}{1.071446in}}%
\pgfusepath{clip}%
\pgfsetbuttcap%
\pgfsetroundjoin%
\definecolor{currentfill}{rgb}{0.594568,0.705971,0.830736}%
\pgfsetfillcolor{currentfill}%
\pgfsetfillopacity{0.700000}%
\pgfsetlinewidth{1.003750pt}%
\definecolor{currentstroke}{rgb}{0.594568,0.705971,0.830736}%
\pgfsetstrokecolor{currentstroke}%
\pgfsetstrokeopacity{0.700000}%
\pgfsetdash{}{0pt}%
\pgfpathmoveto{\pgfqpoint{2.107830in}{1.217751in}}%
\pgfpathcurveto{\pgfqpoint{2.109672in}{1.217751in}}{\pgfqpoint{2.111439in}{1.218483in}}{\pgfqpoint{2.112741in}{1.219785in}}%
\pgfpathcurveto{\pgfqpoint{2.114043in}{1.221088in}}{\pgfqpoint{2.114775in}{1.222854in}}{\pgfqpoint{2.114775in}{1.224696in}}%
\pgfpathcurveto{\pgfqpoint{2.114775in}{1.226537in}}{\pgfqpoint{2.114043in}{1.228304in}}{\pgfqpoint{2.112741in}{1.229606in}}%
\pgfpathcurveto{\pgfqpoint{2.111439in}{1.230908in}}{\pgfqpoint{2.109672in}{1.231640in}}{\pgfqpoint{2.107830in}{1.231640in}}%
\pgfpathcurveto{\pgfqpoint{2.105989in}{1.231640in}}{\pgfqpoint{2.104222in}{1.230908in}}{\pgfqpoint{2.102920in}{1.229606in}}%
\pgfpathcurveto{\pgfqpoint{2.101618in}{1.228304in}}{\pgfqpoint{2.100886in}{1.226537in}}{\pgfqpoint{2.100886in}{1.224696in}}%
\pgfpathcurveto{\pgfqpoint{2.100886in}{1.222854in}}{\pgfqpoint{2.101618in}{1.221088in}}{\pgfqpoint{2.102920in}{1.219785in}}%
\pgfpathcurveto{\pgfqpoint{2.104222in}{1.218483in}}{\pgfqpoint{2.105989in}{1.217751in}}{\pgfqpoint{2.107830in}{1.217751in}}%
\pgfpathlineto{\pgfqpoint{2.107830in}{1.217751in}}%
\pgfpathclose%
\pgfusepath{stroke,fill}%
\end{pgfscope}%
\begin{pgfscope}%
\pgfpathrectangle{\pgfqpoint{0.661006in}{0.524170in}}{\pgfqpoint{4.194036in}{1.071446in}}%
\pgfusepath{clip}%
\pgfsetbuttcap%
\pgfsetroundjoin%
\definecolor{currentfill}{rgb}{0.594568,0.705971,0.830736}%
\pgfsetfillcolor{currentfill}%
\pgfsetfillopacity{0.700000}%
\pgfsetlinewidth{1.003750pt}%
\definecolor{currentstroke}{rgb}{0.594568,0.705971,0.830736}%
\pgfsetstrokecolor{currentstroke}%
\pgfsetstrokeopacity{0.700000}%
\pgfsetdash{}{0pt}%
\pgfpathmoveto{\pgfqpoint{2.149753in}{1.206861in}}%
\pgfpathcurveto{\pgfqpoint{2.151595in}{1.206861in}}{\pgfqpoint{2.153361in}{1.207593in}}{\pgfqpoint{2.154663in}{1.208895in}}%
\pgfpathcurveto{\pgfqpoint{2.155966in}{1.210198in}}{\pgfqpoint{2.156697in}{1.211964in}}{\pgfqpoint{2.156697in}{1.213806in}}%
\pgfpathcurveto{\pgfqpoint{2.156697in}{1.215647in}}{\pgfqpoint{2.155966in}{1.217414in}}{\pgfqpoint{2.154663in}{1.218716in}}%
\pgfpathcurveto{\pgfqpoint{2.153361in}{1.220019in}}{\pgfqpoint{2.151595in}{1.220750in}}{\pgfqpoint{2.149753in}{1.220750in}}%
\pgfpathcurveto{\pgfqpoint{2.147911in}{1.220750in}}{\pgfqpoint{2.146145in}{1.220019in}}{\pgfqpoint{2.144842in}{1.218716in}}%
\pgfpathcurveto{\pgfqpoint{2.143540in}{1.217414in}}{\pgfqpoint{2.142808in}{1.215647in}}{\pgfqpoint{2.142808in}{1.213806in}}%
\pgfpathcurveto{\pgfqpoint{2.142808in}{1.211964in}}{\pgfqpoint{2.143540in}{1.210198in}}{\pgfqpoint{2.144842in}{1.208895in}}%
\pgfpathcurveto{\pgfqpoint{2.146145in}{1.207593in}}{\pgfqpoint{2.147911in}{1.206861in}}{\pgfqpoint{2.149753in}{1.206861in}}%
\pgfpathlineto{\pgfqpoint{2.149753in}{1.206861in}}%
\pgfpathclose%
\pgfusepath{stroke,fill}%
\end{pgfscope}%
\begin{pgfscope}%
\pgfpathrectangle{\pgfqpoint{0.661006in}{0.524170in}}{\pgfqpoint{4.194036in}{1.071446in}}%
\pgfusepath{clip}%
\pgfsetbuttcap%
\pgfsetroundjoin%
\definecolor{currentfill}{rgb}{0.590800,0.701688,0.828336}%
\pgfsetfillcolor{currentfill}%
\pgfsetfillopacity{0.700000}%
\pgfsetlinewidth{1.003750pt}%
\definecolor{currentstroke}{rgb}{0.590800,0.701688,0.828336}%
\pgfsetstrokecolor{currentstroke}%
\pgfsetstrokeopacity{0.700000}%
\pgfsetdash{}{0pt}%
\pgfpathmoveto{\pgfqpoint{2.223745in}{1.188834in}}%
\pgfpathcurveto{\pgfqpoint{2.225586in}{1.188834in}}{\pgfqpoint{2.227353in}{1.189566in}}{\pgfqpoint{2.228655in}{1.190868in}}%
\pgfpathcurveto{\pgfqpoint{2.229957in}{1.192171in}}{\pgfqpoint{2.230689in}{1.193937in}}{\pgfqpoint{2.230689in}{1.195779in}}%
\pgfpathcurveto{\pgfqpoint{2.230689in}{1.197621in}}{\pgfqpoint{2.229957in}{1.199387in}}{\pgfqpoint{2.228655in}{1.200689in}}%
\pgfpathcurveto{\pgfqpoint{2.227353in}{1.201992in}}{\pgfqpoint{2.225586in}{1.202723in}}{\pgfqpoint{2.223745in}{1.202723in}}%
\pgfpathcurveto{\pgfqpoint{2.221903in}{1.202723in}}{\pgfqpoint{2.220136in}{1.201992in}}{\pgfqpoint{2.218834in}{1.200689in}}%
\pgfpathcurveto{\pgfqpoint{2.217532in}{1.199387in}}{\pgfqpoint{2.216800in}{1.197621in}}{\pgfqpoint{2.216800in}{1.195779in}}%
\pgfpathcurveto{\pgfqpoint{2.216800in}{1.193937in}}{\pgfqpoint{2.217532in}{1.192171in}}{\pgfqpoint{2.218834in}{1.190868in}}%
\pgfpathcurveto{\pgfqpoint{2.220136in}{1.189566in}}{\pgfqpoint{2.221903in}{1.188834in}}{\pgfqpoint{2.223745in}{1.188834in}}%
\pgfpathlineto{\pgfqpoint{2.223745in}{1.188834in}}%
\pgfpathclose%
\pgfusepath{stroke,fill}%
\end{pgfscope}%
\begin{pgfscope}%
\pgfpathrectangle{\pgfqpoint{0.661006in}{0.524170in}}{\pgfqpoint{4.194036in}{1.071446in}}%
\pgfusepath{clip}%
\pgfsetbuttcap%
\pgfsetroundjoin%
\definecolor{currentfill}{rgb}{0.590800,0.701688,0.828336}%
\pgfsetfillcolor{currentfill}%
\pgfsetfillopacity{0.700000}%
\pgfsetlinewidth{1.003750pt}%
\definecolor{currentstroke}{rgb}{0.590800,0.701688,0.828336}%
\pgfsetstrokecolor{currentstroke}%
\pgfsetstrokeopacity{0.700000}%
\pgfsetdash{}{0pt}%
\pgfpathmoveto{\pgfqpoint{2.312981in}{1.167928in}}%
\pgfpathcurveto{\pgfqpoint{2.314823in}{1.167928in}}{\pgfqpoint{2.316589in}{1.168659in}}{\pgfqpoint{2.317891in}{1.169962in}}%
\pgfpathcurveto{\pgfqpoint{2.319194in}{1.171264in}}{\pgfqpoint{2.319925in}{1.173030in}}{\pgfqpoint{2.319925in}{1.174872in}}%
\pgfpathcurveto{\pgfqpoint{2.319925in}{1.176714in}}{\pgfqpoint{2.319194in}{1.178480in}}{\pgfqpoint{2.317891in}{1.179783in}}%
\pgfpathcurveto{\pgfqpoint{2.316589in}{1.181085in}}{\pgfqpoint{2.314823in}{1.181817in}}{\pgfqpoint{2.312981in}{1.181817in}}%
\pgfpathcurveto{\pgfqpoint{2.311139in}{1.181817in}}{\pgfqpoint{2.309373in}{1.181085in}}{\pgfqpoint{2.308071in}{1.179783in}}%
\pgfpathcurveto{\pgfqpoint{2.306768in}{1.178480in}}{\pgfqpoint{2.306037in}{1.176714in}}{\pgfqpoint{2.306037in}{1.174872in}}%
\pgfpathcurveto{\pgfqpoint{2.306037in}{1.173030in}}{\pgfqpoint{2.306768in}{1.171264in}}{\pgfqpoint{2.308071in}{1.169962in}}%
\pgfpathcurveto{\pgfqpoint{2.309373in}{1.168659in}}{\pgfqpoint{2.311139in}{1.167928in}}{\pgfqpoint{2.312981in}{1.167928in}}%
\pgfpathlineto{\pgfqpoint{2.312981in}{1.167928in}}%
\pgfpathclose%
\pgfusepath{stroke,fill}%
\end{pgfscope}%
\begin{pgfscope}%
\pgfpathrectangle{\pgfqpoint{0.661006in}{0.524170in}}{\pgfqpoint{4.194036in}{1.071446in}}%
\pgfusepath{clip}%
\pgfsetbuttcap%
\pgfsetroundjoin%
\definecolor{currentfill}{rgb}{0.590800,0.701688,0.828336}%
\pgfsetfillcolor{currentfill}%
\pgfsetfillopacity{0.700000}%
\pgfsetlinewidth{1.003750pt}%
\definecolor{currentstroke}{rgb}{0.590800,0.701688,0.828336}%
\pgfsetstrokecolor{currentstroke}%
\pgfsetstrokeopacity{0.700000}%
\pgfsetdash{}{0pt}%
\pgfpathmoveto{\pgfqpoint{2.415078in}{1.145976in}}%
\pgfpathcurveto{\pgfqpoint{2.416920in}{1.145976in}}{\pgfqpoint{2.418686in}{1.146708in}}{\pgfqpoint{2.419989in}{1.148010in}}%
\pgfpathcurveto{\pgfqpoint{2.421291in}{1.149313in}}{\pgfqpoint{2.422023in}{1.151079in}}{\pgfqpoint{2.422023in}{1.152921in}}%
\pgfpathcurveto{\pgfqpoint{2.422023in}{1.154762in}}{\pgfqpoint{2.421291in}{1.156529in}}{\pgfqpoint{2.419989in}{1.157831in}}%
\pgfpathcurveto{\pgfqpoint{2.418686in}{1.159133in}}{\pgfqpoint{2.416920in}{1.159865in}}{\pgfqpoint{2.415078in}{1.159865in}}%
\pgfpathcurveto{\pgfqpoint{2.413237in}{1.159865in}}{\pgfqpoint{2.411470in}{1.159133in}}{\pgfqpoint{2.410168in}{1.157831in}}%
\pgfpathcurveto{\pgfqpoint{2.408865in}{1.156529in}}{\pgfqpoint{2.408134in}{1.154762in}}{\pgfqpoint{2.408134in}{1.152921in}}%
\pgfpathcurveto{\pgfqpoint{2.408134in}{1.151079in}}{\pgfqpoint{2.408865in}{1.149313in}}{\pgfqpoint{2.410168in}{1.148010in}}%
\pgfpathcurveto{\pgfqpoint{2.411470in}{1.146708in}}{\pgfqpoint{2.413237in}{1.145976in}}{\pgfqpoint{2.415078in}{1.145976in}}%
\pgfpathlineto{\pgfqpoint{2.415078in}{1.145976in}}%
\pgfpathclose%
\pgfusepath{stroke,fill}%
\end{pgfscope}%
\begin{pgfscope}%
\pgfpathrectangle{\pgfqpoint{0.661006in}{0.524170in}}{\pgfqpoint{4.194036in}{1.071446in}}%
\pgfusepath{clip}%
\pgfsetbuttcap%
\pgfsetroundjoin%
\definecolor{currentfill}{rgb}{0.590800,0.701688,0.828336}%
\pgfsetfillcolor{currentfill}%
\pgfsetfillopacity{0.700000}%
\pgfsetlinewidth{1.003750pt}%
\definecolor{currentstroke}{rgb}{0.590800,0.701688,0.828336}%
\pgfsetstrokecolor{currentstroke}%
\pgfsetstrokeopacity{0.700000}%
\pgfsetdash{}{0pt}%
\pgfpathmoveto{\pgfqpoint{2.510649in}{1.124839in}}%
\pgfpathcurveto{\pgfqpoint{2.512490in}{1.124839in}}{\pgfqpoint{2.514257in}{1.125570in}}{\pgfqpoint{2.515559in}{1.126873in}}%
\pgfpathcurveto{\pgfqpoint{2.516861in}{1.128175in}}{\pgfqpoint{2.517593in}{1.129941in}}{\pgfqpoint{2.517593in}{1.131783in}}%
\pgfpathcurveto{\pgfqpoint{2.517593in}{1.133625in}}{\pgfqpoint{2.516861in}{1.135391in}}{\pgfqpoint{2.515559in}{1.136693in}}%
\pgfpathcurveto{\pgfqpoint{2.514257in}{1.137996in}}{\pgfqpoint{2.512490in}{1.138727in}}{\pgfqpoint{2.510649in}{1.138727in}}%
\pgfpathcurveto{\pgfqpoint{2.508807in}{1.138727in}}{\pgfqpoint{2.507040in}{1.137996in}}{\pgfqpoint{2.505738in}{1.136693in}}%
\pgfpathcurveto{\pgfqpoint{2.504436in}{1.135391in}}{\pgfqpoint{2.503704in}{1.133625in}}{\pgfqpoint{2.503704in}{1.131783in}}%
\pgfpathcurveto{\pgfqpoint{2.503704in}{1.129941in}}{\pgfqpoint{2.504436in}{1.128175in}}{\pgfqpoint{2.505738in}{1.126873in}}%
\pgfpathcurveto{\pgfqpoint{2.507040in}{1.125570in}}{\pgfqpoint{2.508807in}{1.124839in}}{\pgfqpoint{2.510649in}{1.124839in}}%
\pgfpathlineto{\pgfqpoint{2.510649in}{1.124839in}}%
\pgfpathclose%
\pgfusepath{stroke,fill}%
\end{pgfscope}%
\begin{pgfscope}%
\pgfpathrectangle{\pgfqpoint{0.661006in}{0.524170in}}{\pgfqpoint{4.194036in}{1.071446in}}%
\pgfusepath{clip}%
\pgfsetbuttcap%
\pgfsetroundjoin%
\definecolor{currentfill}{rgb}{0.590800,0.701688,0.828336}%
\pgfsetfillcolor{currentfill}%
\pgfsetfillopacity{0.700000}%
\pgfsetlinewidth{1.003750pt}%
\definecolor{currentstroke}{rgb}{0.590800,0.701688,0.828336}%
\pgfsetstrokecolor{currentstroke}%
\pgfsetstrokeopacity{0.700000}%
\pgfsetdash{}{0pt}%
\pgfpathmoveto{\pgfqpoint{2.593889in}{1.105018in}}%
\pgfpathcurveto{\pgfqpoint{2.595731in}{1.105018in}}{\pgfqpoint{2.597498in}{1.105750in}}{\pgfqpoint{2.598800in}{1.107052in}}%
\pgfpathcurveto{\pgfqpoint{2.600102in}{1.108354in}}{\pgfqpoint{2.600834in}{1.110121in}}{\pgfqpoint{2.600834in}{1.111963in}}%
\pgfpathcurveto{\pgfqpoint{2.600834in}{1.113804in}}{\pgfqpoint{2.600102in}{1.115571in}}{\pgfqpoint{2.598800in}{1.116873in}}%
\pgfpathcurveto{\pgfqpoint{2.597498in}{1.118175in}}{\pgfqpoint{2.595731in}{1.118907in}}{\pgfqpoint{2.593889in}{1.118907in}}%
\pgfpathcurveto{\pgfqpoint{2.592048in}{1.118907in}}{\pgfqpoint{2.590281in}{1.118175in}}{\pgfqpoint{2.588979in}{1.116873in}}%
\pgfpathcurveto{\pgfqpoint{2.587677in}{1.115571in}}{\pgfqpoint{2.586945in}{1.113804in}}{\pgfqpoint{2.586945in}{1.111963in}}%
\pgfpathcurveto{\pgfqpoint{2.586945in}{1.110121in}}{\pgfqpoint{2.587677in}{1.108354in}}{\pgfqpoint{2.588979in}{1.107052in}}%
\pgfpathcurveto{\pgfqpoint{2.590281in}{1.105750in}}{\pgfqpoint{2.592048in}{1.105018in}}{\pgfqpoint{2.593889in}{1.105018in}}%
\pgfpathlineto{\pgfqpoint{2.593889in}{1.105018in}}%
\pgfpathclose%
\pgfusepath{stroke,fill}%
\end{pgfscope}%
\begin{pgfscope}%
\pgfpathrectangle{\pgfqpoint{0.661006in}{0.524170in}}{\pgfqpoint{4.194036in}{1.071446in}}%
\pgfusepath{clip}%
\pgfsetbuttcap%
\pgfsetroundjoin%
\definecolor{currentfill}{rgb}{0.587059,0.697394,0.825917}%
\pgfsetfillcolor{currentfill}%
\pgfsetfillopacity{0.700000}%
\pgfsetlinewidth{1.003750pt}%
\definecolor{currentstroke}{rgb}{0.587059,0.697394,0.825917}%
\pgfsetstrokecolor{currentstroke}%
\pgfsetstrokeopacity{0.700000}%
\pgfsetdash{}{0pt}%
\pgfpathmoveto{\pgfqpoint{2.645526in}{1.094775in}}%
\pgfpathcurveto{\pgfqpoint{2.647367in}{1.094775in}}{\pgfqpoint{2.649134in}{1.095506in}}{\pgfqpoint{2.650436in}{1.096809in}}%
\pgfpathcurveto{\pgfqpoint{2.651738in}{1.098111in}}{\pgfqpoint{2.652470in}{1.099877in}}{\pgfqpoint{2.652470in}{1.101719in}}%
\pgfpathcurveto{\pgfqpoint{2.652470in}{1.103561in}}{\pgfqpoint{2.651738in}{1.105327in}}{\pgfqpoint{2.650436in}{1.106630in}}%
\pgfpathcurveto{\pgfqpoint{2.649134in}{1.107932in}}{\pgfqpoint{2.647367in}{1.108664in}}{\pgfqpoint{2.645526in}{1.108664in}}%
\pgfpathcurveto{\pgfqpoint{2.643684in}{1.108664in}}{\pgfqpoint{2.641917in}{1.107932in}}{\pgfqpoint{2.640615in}{1.106630in}}%
\pgfpathcurveto{\pgfqpoint{2.639313in}{1.105327in}}{\pgfqpoint{2.638581in}{1.103561in}}{\pgfqpoint{2.638581in}{1.101719in}}%
\pgfpathcurveto{\pgfqpoint{2.638581in}{1.099877in}}{\pgfqpoint{2.639313in}{1.098111in}}{\pgfqpoint{2.640615in}{1.096809in}}%
\pgfpathcurveto{\pgfqpoint{2.641917in}{1.095506in}}{\pgfqpoint{2.643684in}{1.094775in}}{\pgfqpoint{2.645526in}{1.094775in}}%
\pgfpathlineto{\pgfqpoint{2.645526in}{1.094775in}}%
\pgfpathclose%
\pgfusepath{stroke,fill}%
\end{pgfscope}%
\begin{pgfscope}%
\pgfpathrectangle{\pgfqpoint{0.661006in}{0.524170in}}{\pgfqpoint{4.194036in}{1.071446in}}%
\pgfusepath{clip}%
\pgfsetbuttcap%
\pgfsetroundjoin%
\definecolor{currentfill}{rgb}{0.587059,0.697394,0.825917}%
\pgfsetfillcolor{currentfill}%
\pgfsetfillopacity{0.700000}%
\pgfsetlinewidth{1.003750pt}%
\definecolor{currentstroke}{rgb}{0.587059,0.697394,0.825917}%
\pgfsetstrokecolor{currentstroke}%
\pgfsetstrokeopacity{0.700000}%
\pgfsetdash{}{0pt}%
\pgfpathmoveto{\pgfqpoint{2.616198in}{1.102621in}}%
\pgfpathcurveto{\pgfqpoint{2.618040in}{1.102621in}}{\pgfqpoint{2.619807in}{1.103353in}}{\pgfqpoint{2.621109in}{1.104655in}}%
\pgfpathcurveto{\pgfqpoint{2.622411in}{1.105958in}}{\pgfqpoint{2.623143in}{1.107724in}}{\pgfqpoint{2.623143in}{1.109566in}}%
\pgfpathcurveto{\pgfqpoint{2.623143in}{1.111407in}}{\pgfqpoint{2.622411in}{1.113174in}}{\pgfqpoint{2.621109in}{1.114476in}}%
\pgfpathcurveto{\pgfqpoint{2.619807in}{1.115779in}}{\pgfqpoint{2.618040in}{1.116510in}}{\pgfqpoint{2.616198in}{1.116510in}}%
\pgfpathcurveto{\pgfqpoint{2.614357in}{1.116510in}}{\pgfqpoint{2.612590in}{1.115779in}}{\pgfqpoint{2.611288in}{1.114476in}}%
\pgfpathcurveto{\pgfqpoint{2.609986in}{1.113174in}}{\pgfqpoint{2.609254in}{1.111407in}}{\pgfqpoint{2.609254in}{1.109566in}}%
\pgfpathcurveto{\pgfqpoint{2.609254in}{1.107724in}}{\pgfqpoint{2.609986in}{1.105958in}}{\pgfqpoint{2.611288in}{1.104655in}}%
\pgfpathcurveto{\pgfqpoint{2.612590in}{1.103353in}}{\pgfqpoint{2.614357in}{1.102621in}}{\pgfqpoint{2.616198in}{1.102621in}}%
\pgfpathlineto{\pgfqpoint{2.616198in}{1.102621in}}%
\pgfpathclose%
\pgfusepath{stroke,fill}%
\end{pgfscope}%
\begin{pgfscope}%
\pgfpathrectangle{\pgfqpoint{0.661006in}{0.524170in}}{\pgfqpoint{4.194036in}{1.071446in}}%
\pgfusepath{clip}%
\pgfsetbuttcap%
\pgfsetroundjoin%
\definecolor{currentfill}{rgb}{0.583344,0.693091,0.823479}%
\pgfsetfillcolor{currentfill}%
\pgfsetfillopacity{0.700000}%
\pgfsetlinewidth{1.003750pt}%
\definecolor{currentstroke}{rgb}{0.583344,0.693091,0.823479}%
\pgfsetstrokecolor{currentstroke}%
\pgfsetstrokeopacity{0.700000}%
\pgfsetdash{}{0pt}%
\pgfpathmoveto{\pgfqpoint{2.490385in}{1.134454in}}%
\pgfpathcurveto{\pgfqpoint{2.492226in}{1.134454in}}{\pgfqpoint{2.493993in}{1.135186in}}{\pgfqpoint{2.495295in}{1.136488in}}%
\pgfpathcurveto{\pgfqpoint{2.496597in}{1.137790in}}{\pgfqpoint{2.497329in}{1.139557in}}{\pgfqpoint{2.497329in}{1.141398in}}%
\pgfpathcurveto{\pgfqpoint{2.497329in}{1.143240in}}{\pgfqpoint{2.496597in}{1.145007in}}{\pgfqpoint{2.495295in}{1.146309in}}%
\pgfpathcurveto{\pgfqpoint{2.493993in}{1.147611in}}{\pgfqpoint{2.492226in}{1.148343in}}{\pgfqpoint{2.490385in}{1.148343in}}%
\pgfpathcurveto{\pgfqpoint{2.488543in}{1.148343in}}{\pgfqpoint{2.486776in}{1.147611in}}{\pgfqpoint{2.485474in}{1.146309in}}%
\pgfpathcurveto{\pgfqpoint{2.484172in}{1.145007in}}{\pgfqpoint{2.483440in}{1.143240in}}{\pgfqpoint{2.483440in}{1.141398in}}%
\pgfpathcurveto{\pgfqpoint{2.483440in}{1.139557in}}{\pgfqpoint{2.484172in}{1.137790in}}{\pgfqpoint{2.485474in}{1.136488in}}%
\pgfpathcurveto{\pgfqpoint{2.486776in}{1.135186in}}{\pgfqpoint{2.488543in}{1.134454in}}{\pgfqpoint{2.490385in}{1.134454in}}%
\pgfpathlineto{\pgfqpoint{2.490385in}{1.134454in}}%
\pgfpathclose%
\pgfusepath{stroke,fill}%
\end{pgfscope}%
\begin{pgfscope}%
\pgfpathrectangle{\pgfqpoint{0.661006in}{0.524170in}}{\pgfqpoint{4.194036in}{1.071446in}}%
\pgfusepath{clip}%
\pgfsetbuttcap%
\pgfsetroundjoin%
\definecolor{currentfill}{rgb}{0.583344,0.693091,0.823479}%
\pgfsetfillcolor{currentfill}%
\pgfsetfillopacity{0.700000}%
\pgfsetlinewidth{1.003750pt}%
\definecolor{currentstroke}{rgb}{0.583344,0.693091,0.823479}%
\pgfsetstrokecolor{currentstroke}%
\pgfsetstrokeopacity{0.700000}%
\pgfsetdash{}{0pt}%
\pgfpathmoveto{\pgfqpoint{2.275195in}{1.182483in}}%
\pgfpathcurveto{\pgfqpoint{2.277037in}{1.182483in}}{\pgfqpoint{2.278803in}{1.183214in}}{\pgfqpoint{2.280105in}{1.184517in}}%
\pgfpathcurveto{\pgfqpoint{2.281408in}{1.185819in}}{\pgfqpoint{2.282139in}{1.187585in}}{\pgfqpoint{2.282139in}{1.189427in}}%
\pgfpathcurveto{\pgfqpoint{2.282139in}{1.191269in}}{\pgfqpoint{2.281408in}{1.193035in}}{\pgfqpoint{2.280105in}{1.194338in}}%
\pgfpathcurveto{\pgfqpoint{2.278803in}{1.195640in}}{\pgfqpoint{2.277037in}{1.196372in}}{\pgfqpoint{2.275195in}{1.196372in}}%
\pgfpathcurveto{\pgfqpoint{2.273353in}{1.196372in}}{\pgfqpoint{2.271587in}{1.195640in}}{\pgfqpoint{2.270285in}{1.194338in}}%
\pgfpathcurveto{\pgfqpoint{2.268982in}{1.193035in}}{\pgfqpoint{2.268251in}{1.191269in}}{\pgfqpoint{2.268251in}{1.189427in}}%
\pgfpathcurveto{\pgfqpoint{2.268251in}{1.187585in}}{\pgfqpoint{2.268982in}{1.185819in}}{\pgfqpoint{2.270285in}{1.184517in}}%
\pgfpathcurveto{\pgfqpoint{2.271587in}{1.183214in}}{\pgfqpoint{2.273353in}{1.182483in}}{\pgfqpoint{2.275195in}{1.182483in}}%
\pgfpathlineto{\pgfqpoint{2.275195in}{1.182483in}}%
\pgfpathclose%
\pgfusepath{stroke,fill}%
\end{pgfscope}%
\begin{pgfscope}%
\pgfpathrectangle{\pgfqpoint{0.661006in}{0.524170in}}{\pgfqpoint{4.194036in}{1.071446in}}%
\pgfusepath{clip}%
\pgfsetbuttcap%
\pgfsetroundjoin%
\definecolor{currentfill}{rgb}{0.583344,0.693091,0.823479}%
\pgfsetfillcolor{currentfill}%
\pgfsetfillopacity{0.700000}%
\pgfsetlinewidth{1.003750pt}%
\definecolor{currentstroke}{rgb}{0.583344,0.693091,0.823479}%
\pgfsetstrokecolor{currentstroke}%
\pgfsetstrokeopacity{0.700000}%
\pgfsetdash{}{0pt}%
\pgfpathmoveto{\pgfqpoint{2.089425in}{1.224010in}}%
\pgfpathcurveto{\pgfqpoint{2.091267in}{1.224010in}}{\pgfqpoint{2.093034in}{1.224741in}}{\pgfqpoint{2.094336in}{1.226044in}}%
\pgfpathcurveto{\pgfqpoint{2.095638in}{1.227346in}}{\pgfqpoint{2.096370in}{1.229113in}}{\pgfqpoint{2.096370in}{1.230954in}}%
\pgfpathcurveto{\pgfqpoint{2.096370in}{1.232796in}}{\pgfqpoint{2.095638in}{1.234562in}}{\pgfqpoint{2.094336in}{1.235865in}}%
\pgfpathcurveto{\pgfqpoint{2.093034in}{1.237167in}}{\pgfqpoint{2.091267in}{1.237899in}}{\pgfqpoint{2.089425in}{1.237899in}}%
\pgfpathcurveto{\pgfqpoint{2.087584in}{1.237899in}}{\pgfqpoint{2.085817in}{1.237167in}}{\pgfqpoint{2.084515in}{1.235865in}}%
\pgfpathcurveto{\pgfqpoint{2.083213in}{1.234562in}}{\pgfqpoint{2.082481in}{1.232796in}}{\pgfqpoint{2.082481in}{1.230954in}}%
\pgfpathcurveto{\pgfqpoint{2.082481in}{1.229113in}}{\pgfqpoint{2.083213in}{1.227346in}}{\pgfqpoint{2.084515in}{1.226044in}}%
\pgfpathcurveto{\pgfqpoint{2.085817in}{1.224741in}}{\pgfqpoint{2.087584in}{1.224010in}}{\pgfqpoint{2.089425in}{1.224010in}}%
\pgfpathlineto{\pgfqpoint{2.089425in}{1.224010in}}%
\pgfpathclose%
\pgfusepath{stroke,fill}%
\end{pgfscope}%
\begin{pgfscope}%
\pgfpathrectangle{\pgfqpoint{0.661006in}{0.524170in}}{\pgfqpoint{4.194036in}{1.071446in}}%
\pgfusepath{clip}%
\pgfsetbuttcap%
\pgfsetroundjoin%
\definecolor{currentfill}{rgb}{0.579655,0.688778,0.821022}%
\pgfsetfillcolor{currentfill}%
\pgfsetfillopacity{0.700000}%
\pgfsetlinewidth{1.003750pt}%
\definecolor{currentstroke}{rgb}{0.579655,0.688778,0.821022}%
\pgfsetstrokecolor{currentstroke}%
\pgfsetstrokeopacity{0.700000}%
\pgfsetdash{}{0pt}%
\pgfpathmoveto{\pgfqpoint{2.042298in}{1.230365in}}%
\pgfpathcurveto{\pgfqpoint{2.044139in}{1.230365in}}{\pgfqpoint{2.045906in}{1.231097in}}{\pgfqpoint{2.047208in}{1.232399in}}%
\pgfpathcurveto{\pgfqpoint{2.048510in}{1.233701in}}{\pgfqpoint{2.049242in}{1.235468in}}{\pgfqpoint{2.049242in}{1.237310in}}%
\pgfpathcurveto{\pgfqpoint{2.049242in}{1.239151in}}{\pgfqpoint{2.048510in}{1.240918in}}{\pgfqpoint{2.047208in}{1.242220in}}%
\pgfpathcurveto{\pgfqpoint{2.045906in}{1.243522in}}{\pgfqpoint{2.044139in}{1.244254in}}{\pgfqpoint{2.042298in}{1.244254in}}%
\pgfpathcurveto{\pgfqpoint{2.040456in}{1.244254in}}{\pgfqpoint{2.038689in}{1.243522in}}{\pgfqpoint{2.037387in}{1.242220in}}%
\pgfpathcurveto{\pgfqpoint{2.036085in}{1.240918in}}{\pgfqpoint{2.035353in}{1.239151in}}{\pgfqpoint{2.035353in}{1.237310in}}%
\pgfpathcurveto{\pgfqpoint{2.035353in}{1.235468in}}{\pgfqpoint{2.036085in}{1.233701in}}{\pgfqpoint{2.037387in}{1.232399in}}%
\pgfpathcurveto{\pgfqpoint{2.038689in}{1.231097in}}{\pgfqpoint{2.040456in}{1.230365in}}{\pgfqpoint{2.042298in}{1.230365in}}%
\pgfpathlineto{\pgfqpoint{2.042298in}{1.230365in}}%
\pgfpathclose%
\pgfusepath{stroke,fill}%
\end{pgfscope}%
\begin{pgfscope}%
\pgfpathrectangle{\pgfqpoint{0.661006in}{0.524170in}}{\pgfqpoint{4.194036in}{1.071446in}}%
\pgfusepath{clip}%
\pgfsetbuttcap%
\pgfsetroundjoin%
\definecolor{currentfill}{rgb}{0.579655,0.688778,0.821022}%
\pgfsetfillcolor{currentfill}%
\pgfsetfillopacity{0.700000}%
\pgfsetlinewidth{1.003750pt}%
\definecolor{currentstroke}{rgb}{0.579655,0.688778,0.821022}%
\pgfsetstrokecolor{currentstroke}%
\pgfsetstrokeopacity{0.700000}%
\pgfsetdash{}{0pt}%
\pgfpathmoveto{\pgfqpoint{2.076644in}{1.223281in}}%
\pgfpathcurveto{\pgfqpoint{2.078486in}{1.223281in}}{\pgfqpoint{2.080252in}{1.224013in}}{\pgfqpoint{2.081555in}{1.225315in}}%
\pgfpathcurveto{\pgfqpoint{2.082857in}{1.226618in}}{\pgfqpoint{2.083589in}{1.228384in}}{\pgfqpoint{2.083589in}{1.230226in}}%
\pgfpathcurveto{\pgfqpoint{2.083589in}{1.232067in}}{\pgfqpoint{2.082857in}{1.233834in}}{\pgfqpoint{2.081555in}{1.235136in}}%
\pgfpathcurveto{\pgfqpoint{2.080252in}{1.236438in}}{\pgfqpoint{2.078486in}{1.237170in}}{\pgfqpoint{2.076644in}{1.237170in}}%
\pgfpathcurveto{\pgfqpoint{2.074803in}{1.237170in}}{\pgfqpoint{2.073036in}{1.236438in}}{\pgfqpoint{2.071734in}{1.235136in}}%
\pgfpathcurveto{\pgfqpoint{2.070431in}{1.233834in}}{\pgfqpoint{2.069700in}{1.232067in}}{\pgfqpoint{2.069700in}{1.230226in}}%
\pgfpathcurveto{\pgfqpoint{2.069700in}{1.228384in}}{\pgfqpoint{2.070431in}{1.226618in}}{\pgfqpoint{2.071734in}{1.225315in}}%
\pgfpathcurveto{\pgfqpoint{2.073036in}{1.224013in}}{\pgfqpoint{2.074803in}{1.223281in}}{\pgfqpoint{2.076644in}{1.223281in}}%
\pgfpathlineto{\pgfqpoint{2.076644in}{1.223281in}}%
\pgfpathclose%
\pgfusepath{stroke,fill}%
\end{pgfscope}%
\begin{pgfscope}%
\pgfpathrectangle{\pgfqpoint{0.661006in}{0.524170in}}{\pgfqpoint{4.194036in}{1.071446in}}%
\pgfusepath{clip}%
\pgfsetbuttcap%
\pgfsetroundjoin%
\definecolor{currentfill}{rgb}{0.579655,0.688778,0.821022}%
\pgfsetfillcolor{currentfill}%
\pgfsetfillopacity{0.700000}%
\pgfsetlinewidth{1.003750pt}%
\definecolor{currentstroke}{rgb}{0.579655,0.688778,0.821022}%
\pgfsetstrokecolor{currentstroke}%
\pgfsetstrokeopacity{0.700000}%
\pgfsetdash{}{0pt}%
\pgfpathmoveto{\pgfqpoint{2.020174in}{1.237384in}}%
\pgfpathcurveto{\pgfqpoint{2.022016in}{1.237384in}}{\pgfqpoint{2.023783in}{1.238116in}}{\pgfqpoint{2.025085in}{1.239418in}}%
\pgfpathcurveto{\pgfqpoint{2.026387in}{1.240720in}}{\pgfqpoint{2.027119in}{1.242487in}}{\pgfqpoint{2.027119in}{1.244329in}}%
\pgfpathcurveto{\pgfqpoint{2.027119in}{1.246170in}}{\pgfqpoint{2.026387in}{1.247937in}}{\pgfqpoint{2.025085in}{1.249239in}}%
\pgfpathcurveto{\pgfqpoint{2.023783in}{1.250541in}}{\pgfqpoint{2.022016in}{1.251273in}}{\pgfqpoint{2.020174in}{1.251273in}}%
\pgfpathcurveto{\pgfqpoint{2.018333in}{1.251273in}}{\pgfqpoint{2.016566in}{1.250541in}}{\pgfqpoint{2.015264in}{1.249239in}}%
\pgfpathcurveto{\pgfqpoint{2.013962in}{1.247937in}}{\pgfqpoint{2.013230in}{1.246170in}}{\pgfqpoint{2.013230in}{1.244329in}}%
\pgfpathcurveto{\pgfqpoint{2.013230in}{1.242487in}}{\pgfqpoint{2.013962in}{1.240720in}}{\pgfqpoint{2.015264in}{1.239418in}}%
\pgfpathcurveto{\pgfqpoint{2.016566in}{1.238116in}}{\pgfqpoint{2.018333in}{1.237384in}}{\pgfqpoint{2.020174in}{1.237384in}}%
\pgfpathlineto{\pgfqpoint{2.020174in}{1.237384in}}%
\pgfpathclose%
\pgfusepath{stroke,fill}%
\end{pgfscope}%
\begin{pgfscope}%
\pgfpathrectangle{\pgfqpoint{0.661006in}{0.524170in}}{\pgfqpoint{4.194036in}{1.071446in}}%
\pgfusepath{clip}%
\pgfsetbuttcap%
\pgfsetroundjoin%
\definecolor{currentfill}{rgb}{0.579655,0.688778,0.821022}%
\pgfsetfillcolor{currentfill}%
\pgfsetfillopacity{0.700000}%
\pgfsetlinewidth{1.003750pt}%
\definecolor{currentstroke}{rgb}{0.579655,0.688778,0.821022}%
\pgfsetstrokecolor{currentstroke}%
\pgfsetstrokeopacity{0.700000}%
\pgfsetdash{}{0pt}%
\pgfpathmoveto{\pgfqpoint{1.913230in}{1.263317in}}%
\pgfpathcurveto{\pgfqpoint{1.915072in}{1.263317in}}{\pgfqpoint{1.916838in}{1.264049in}}{\pgfqpoint{1.918141in}{1.265351in}}%
\pgfpathcurveto{\pgfqpoint{1.919443in}{1.266653in}}{\pgfqpoint{1.920175in}{1.268420in}}{\pgfqpoint{1.920175in}{1.270261in}}%
\pgfpathcurveto{\pgfqpoint{1.920175in}{1.272103in}}{\pgfqpoint{1.919443in}{1.273869in}}{\pgfqpoint{1.918141in}{1.275172in}}%
\pgfpathcurveto{\pgfqpoint{1.916838in}{1.276474in}}{\pgfqpoint{1.915072in}{1.277206in}}{\pgfqpoint{1.913230in}{1.277206in}}%
\pgfpathcurveto{\pgfqpoint{1.911389in}{1.277206in}}{\pgfqpoint{1.909622in}{1.276474in}}{\pgfqpoint{1.908320in}{1.275172in}}%
\pgfpathcurveto{\pgfqpoint{1.907018in}{1.273869in}}{\pgfqpoint{1.906286in}{1.272103in}}{\pgfqpoint{1.906286in}{1.270261in}}%
\pgfpathcurveto{\pgfqpoint{1.906286in}{1.268420in}}{\pgfqpoint{1.907018in}{1.266653in}}{\pgfqpoint{1.908320in}{1.265351in}}%
\pgfpathcurveto{\pgfqpoint{1.909622in}{1.264049in}}{\pgfqpoint{1.911389in}{1.263317in}}{\pgfqpoint{1.913230in}{1.263317in}}%
\pgfpathlineto{\pgfqpoint{1.913230in}{1.263317in}}%
\pgfpathclose%
\pgfusepath{stroke,fill}%
\end{pgfscope}%
\begin{pgfscope}%
\pgfpathrectangle{\pgfqpoint{0.661006in}{0.524170in}}{\pgfqpoint{4.194036in}{1.071446in}}%
\pgfusepath{clip}%
\pgfsetbuttcap%
\pgfsetroundjoin%
\definecolor{currentfill}{rgb}{0.579655,0.688778,0.821022}%
\pgfsetfillcolor{currentfill}%
\pgfsetfillopacity{0.700000}%
\pgfsetlinewidth{1.003750pt}%
\definecolor{currentstroke}{rgb}{0.579655,0.688778,0.821022}%
\pgfsetstrokecolor{currentstroke}%
\pgfsetstrokeopacity{0.700000}%
\pgfsetdash{}{0pt}%
\pgfpathmoveto{\pgfqpoint{1.821856in}{1.281675in}}%
\pgfpathcurveto{\pgfqpoint{1.823698in}{1.281675in}}{\pgfqpoint{1.825464in}{1.282407in}}{\pgfqpoint{1.826766in}{1.283709in}}%
\pgfpathcurveto{\pgfqpoint{1.828069in}{1.285011in}}{\pgfqpoint{1.828800in}{1.286778in}}{\pgfqpoint{1.828800in}{1.288620in}}%
\pgfpathcurveto{\pgfqpoint{1.828800in}{1.290461in}}{\pgfqpoint{1.828069in}{1.292228in}}{\pgfqpoint{1.826766in}{1.293530in}}%
\pgfpathcurveto{\pgfqpoint{1.825464in}{1.294832in}}{\pgfqpoint{1.823698in}{1.295564in}}{\pgfqpoint{1.821856in}{1.295564in}}%
\pgfpathcurveto{\pgfqpoint{1.820014in}{1.295564in}}{\pgfqpoint{1.818248in}{1.294832in}}{\pgfqpoint{1.816946in}{1.293530in}}%
\pgfpathcurveto{\pgfqpoint{1.815643in}{1.292228in}}{\pgfqpoint{1.814912in}{1.290461in}}{\pgfqpoint{1.814912in}{1.288620in}}%
\pgfpathcurveto{\pgfqpoint{1.814912in}{1.286778in}}{\pgfqpoint{1.815643in}{1.285011in}}{\pgfqpoint{1.816946in}{1.283709in}}%
\pgfpathcurveto{\pgfqpoint{1.818248in}{1.282407in}}{\pgfqpoint{1.820014in}{1.281675in}}{\pgfqpoint{1.821856in}{1.281675in}}%
\pgfpathlineto{\pgfqpoint{1.821856in}{1.281675in}}%
\pgfpathclose%
\pgfusepath{stroke,fill}%
\end{pgfscope}%
\begin{pgfscope}%
\pgfpathrectangle{\pgfqpoint{0.661006in}{0.524170in}}{\pgfqpoint{4.194036in}{1.071446in}}%
\pgfusepath{clip}%
\pgfsetbuttcap%
\pgfsetroundjoin%
\definecolor{currentfill}{rgb}{0.575992,0.684456,0.818544}%
\pgfsetfillcolor{currentfill}%
\pgfsetfillopacity{0.700000}%
\pgfsetlinewidth{1.003750pt}%
\definecolor{currentstroke}{rgb}{0.575992,0.684456,0.818544}%
\pgfsetstrokecolor{currentstroke}%
\pgfsetstrokeopacity{0.700000}%
\pgfsetdash{}{0pt}%
\pgfpathmoveto{\pgfqpoint{1.784814in}{1.286790in}}%
\pgfpathcurveto{\pgfqpoint{1.786655in}{1.286790in}}{\pgfqpoint{1.788422in}{1.287521in}}{\pgfqpoint{1.789724in}{1.288824in}}%
\pgfpathcurveto{\pgfqpoint{1.791026in}{1.290126in}}{\pgfqpoint{1.791758in}{1.291892in}}{\pgfqpoint{1.791758in}{1.293734in}}%
\pgfpathcurveto{\pgfqpoint{1.791758in}{1.295576in}}{\pgfqpoint{1.791026in}{1.297342in}}{\pgfqpoint{1.789724in}{1.298644in}}%
\pgfpathcurveto{\pgfqpoint{1.788422in}{1.299947in}}{\pgfqpoint{1.786655in}{1.300678in}}{\pgfqpoint{1.784814in}{1.300678in}}%
\pgfpathcurveto{\pgfqpoint{1.782972in}{1.300678in}}{\pgfqpoint{1.781205in}{1.299947in}}{\pgfqpoint{1.779903in}{1.298644in}}%
\pgfpathcurveto{\pgfqpoint{1.778601in}{1.297342in}}{\pgfqpoint{1.777869in}{1.295576in}}{\pgfqpoint{1.777869in}{1.293734in}}%
\pgfpathcurveto{\pgfqpoint{1.777869in}{1.291892in}}{\pgfqpoint{1.778601in}{1.290126in}}{\pgfqpoint{1.779903in}{1.288824in}}%
\pgfpathcurveto{\pgfqpoint{1.781205in}{1.287521in}}{\pgfqpoint{1.782972in}{1.286790in}}{\pgfqpoint{1.784814in}{1.286790in}}%
\pgfpathlineto{\pgfqpoint{1.784814in}{1.286790in}}%
\pgfpathclose%
\pgfusepath{stroke,fill}%
\end{pgfscope}%
\begin{pgfscope}%
\pgfpathrectangle{\pgfqpoint{0.661006in}{0.524170in}}{\pgfqpoint{4.194036in}{1.071446in}}%
\pgfusepath{clip}%
\pgfsetbuttcap%
\pgfsetroundjoin%
\definecolor{currentfill}{rgb}{0.575992,0.684456,0.818544}%
\pgfsetfillcolor{currentfill}%
\pgfsetfillopacity{0.700000}%
\pgfsetlinewidth{1.003750pt}%
\definecolor{currentstroke}{rgb}{0.575992,0.684456,0.818544}%
\pgfsetstrokecolor{currentstroke}%
\pgfsetstrokeopacity{0.700000}%
\pgfsetdash{}{0pt}%
\pgfpathmoveto{\pgfqpoint{1.773931in}{1.289792in}}%
\pgfpathcurveto{\pgfqpoint{1.775773in}{1.289792in}}{\pgfqpoint{1.777540in}{1.290523in}}{\pgfqpoint{1.778842in}{1.291826in}}%
\pgfpathcurveto{\pgfqpoint{1.780144in}{1.293128in}}{\pgfqpoint{1.780876in}{1.294894in}}{\pgfqpoint{1.780876in}{1.296736in}}%
\pgfpathcurveto{\pgfqpoint{1.780876in}{1.298578in}}{\pgfqpoint{1.780144in}{1.300344in}}{\pgfqpoint{1.778842in}{1.301647in}}%
\pgfpathcurveto{\pgfqpoint{1.777540in}{1.302949in}}{\pgfqpoint{1.775773in}{1.303681in}}{\pgfqpoint{1.773931in}{1.303681in}}%
\pgfpathcurveto{\pgfqpoint{1.772090in}{1.303681in}}{\pgfqpoint{1.770323in}{1.302949in}}{\pgfqpoint{1.769021in}{1.301647in}}%
\pgfpathcurveto{\pgfqpoint{1.767719in}{1.300344in}}{\pgfqpoint{1.766987in}{1.298578in}}{\pgfqpoint{1.766987in}{1.296736in}}%
\pgfpathcurveto{\pgfqpoint{1.766987in}{1.294894in}}{\pgfqpoint{1.767719in}{1.293128in}}{\pgfqpoint{1.769021in}{1.291826in}}%
\pgfpathcurveto{\pgfqpoint{1.770323in}{1.290523in}}{\pgfqpoint{1.772090in}{1.289792in}}{\pgfqpoint{1.773931in}{1.289792in}}%
\pgfpathlineto{\pgfqpoint{1.773931in}{1.289792in}}%
\pgfpathclose%
\pgfusepath{stroke,fill}%
\end{pgfscope}%
\begin{pgfscope}%
\pgfpathrectangle{\pgfqpoint{0.661006in}{0.524170in}}{\pgfqpoint{4.194036in}{1.071446in}}%
\pgfusepath{clip}%
\pgfsetbuttcap%
\pgfsetroundjoin%
\definecolor{currentfill}{rgb}{0.572355,0.680125,0.816045}%
\pgfsetfillcolor{currentfill}%
\pgfsetfillopacity{0.700000}%
\pgfsetlinewidth{1.003750pt}%
\definecolor{currentstroke}{rgb}{0.572355,0.680125,0.816045}%
\pgfsetstrokecolor{currentstroke}%
\pgfsetstrokeopacity{0.700000}%
\pgfsetdash{}{0pt}%
\pgfpathmoveto{\pgfqpoint{1.675778in}{1.312242in}}%
\pgfpathcurveto{\pgfqpoint{1.677620in}{1.312242in}}{\pgfqpoint{1.679386in}{1.312974in}}{\pgfqpoint{1.680689in}{1.314276in}}%
\pgfpathcurveto{\pgfqpoint{1.681991in}{1.315578in}}{\pgfqpoint{1.682723in}{1.317345in}}{\pgfqpoint{1.682723in}{1.319186in}}%
\pgfpathcurveto{\pgfqpoint{1.682723in}{1.321028in}}{\pgfqpoint{1.681991in}{1.322795in}}{\pgfqpoint{1.680689in}{1.324097in}}%
\pgfpathcurveto{\pgfqpoint{1.679386in}{1.325399in}}{\pgfqpoint{1.677620in}{1.326131in}}{\pgfqpoint{1.675778in}{1.326131in}}%
\pgfpathcurveto{\pgfqpoint{1.673936in}{1.326131in}}{\pgfqpoint{1.672170in}{1.325399in}}{\pgfqpoint{1.670868in}{1.324097in}}%
\pgfpathcurveto{\pgfqpoint{1.669565in}{1.322795in}}{\pgfqpoint{1.668834in}{1.321028in}}{\pgfqpoint{1.668834in}{1.319186in}}%
\pgfpathcurveto{\pgfqpoint{1.668834in}{1.317345in}}{\pgfqpoint{1.669565in}{1.315578in}}{\pgfqpoint{1.670868in}{1.314276in}}%
\pgfpathcurveto{\pgfqpoint{1.672170in}{1.312974in}}{\pgfqpoint{1.673936in}{1.312242in}}{\pgfqpoint{1.675778in}{1.312242in}}%
\pgfpathlineto{\pgfqpoint{1.675778in}{1.312242in}}%
\pgfpathclose%
\pgfusepath{stroke,fill}%
\end{pgfscope}%
\begin{pgfscope}%
\pgfpathrectangle{\pgfqpoint{0.661006in}{0.524170in}}{\pgfqpoint{4.194036in}{1.071446in}}%
\pgfusepath{clip}%
\pgfsetbuttcap%
\pgfsetroundjoin%
\definecolor{currentfill}{rgb}{0.572355,0.680125,0.816045}%
\pgfsetfillcolor{currentfill}%
\pgfsetfillopacity{0.700000}%
\pgfsetlinewidth{1.003750pt}%
\definecolor{currentstroke}{rgb}{0.572355,0.680125,0.816045}%
\pgfsetstrokecolor{currentstroke}%
\pgfsetstrokeopacity{0.700000}%
\pgfsetdash{}{0pt}%
\pgfpathmoveto{\pgfqpoint{1.413321in}{1.371433in}}%
\pgfpathcurveto{\pgfqpoint{1.415163in}{1.371433in}}{\pgfqpoint{1.416929in}{1.372165in}}{\pgfqpoint{1.418232in}{1.373467in}}%
\pgfpathcurveto{\pgfqpoint{1.419534in}{1.374770in}}{\pgfqpoint{1.420266in}{1.376536in}}{\pgfqpoint{1.420266in}{1.378378in}}%
\pgfpathcurveto{\pgfqpoint{1.420266in}{1.380219in}}{\pgfqpoint{1.419534in}{1.381986in}}{\pgfqpoint{1.418232in}{1.383288in}}%
\pgfpathcurveto{\pgfqpoint{1.416929in}{1.384591in}}{\pgfqpoint{1.415163in}{1.385322in}}{\pgfqpoint{1.413321in}{1.385322in}}%
\pgfpathcurveto{\pgfqpoint{1.411479in}{1.385322in}}{\pgfqpoint{1.409713in}{1.384591in}}{\pgfqpoint{1.408411in}{1.383288in}}%
\pgfpathcurveto{\pgfqpoint{1.407108in}{1.381986in}}{\pgfqpoint{1.406377in}{1.380219in}}{\pgfqpoint{1.406377in}{1.378378in}}%
\pgfpathcurveto{\pgfqpoint{1.406377in}{1.376536in}}{\pgfqpoint{1.407108in}{1.374770in}}{\pgfqpoint{1.408411in}{1.373467in}}%
\pgfpathcurveto{\pgfqpoint{1.409713in}{1.372165in}}{\pgfqpoint{1.411479in}{1.371433in}}{\pgfqpoint{1.413321in}{1.371433in}}%
\pgfpathlineto{\pgfqpoint{1.413321in}{1.371433in}}%
\pgfpathclose%
\pgfusepath{stroke,fill}%
\end{pgfscope}%
\begin{pgfscope}%
\pgfpathrectangle{\pgfqpoint{0.661006in}{0.524170in}}{\pgfqpoint{4.194036in}{1.071446in}}%
\pgfusepath{clip}%
\pgfsetbuttcap%
\pgfsetroundjoin%
\definecolor{currentfill}{rgb}{0.572355,0.680125,0.816045}%
\pgfsetfillcolor{currentfill}%
\pgfsetfillopacity{0.700000}%
\pgfsetlinewidth{1.003750pt}%
\definecolor{currentstroke}{rgb}{0.572355,0.680125,0.816045}%
\pgfsetstrokecolor{currentstroke}%
\pgfsetstrokeopacity{0.700000}%
\pgfsetdash{}{0pt}%
\pgfpathmoveto{\pgfqpoint{1.057631in}{1.453271in}}%
\pgfpathcurveto{\pgfqpoint{1.059472in}{1.453271in}}{\pgfqpoint{1.061239in}{1.454003in}}{\pgfqpoint{1.062541in}{1.455305in}}%
\pgfpathcurveto{\pgfqpoint{1.063844in}{1.456607in}}{\pgfqpoint{1.064575in}{1.458374in}}{\pgfqpoint{1.064575in}{1.460216in}}%
\pgfpathcurveto{\pgfqpoint{1.064575in}{1.462057in}}{\pgfqpoint{1.063844in}{1.463824in}}{\pgfqpoint{1.062541in}{1.465126in}}%
\pgfpathcurveto{\pgfqpoint{1.061239in}{1.466428in}}{\pgfqpoint{1.059472in}{1.467160in}}{\pgfqpoint{1.057631in}{1.467160in}}%
\pgfpathcurveto{\pgfqpoint{1.055789in}{1.467160in}}{\pgfqpoint{1.054023in}{1.466428in}}{\pgfqpoint{1.052720in}{1.465126in}}%
\pgfpathcurveto{\pgfqpoint{1.051418in}{1.463824in}}{\pgfqpoint{1.050686in}{1.462057in}}{\pgfqpoint{1.050686in}{1.460216in}}%
\pgfpathcurveto{\pgfqpoint{1.050686in}{1.458374in}}{\pgfqpoint{1.051418in}{1.456607in}}{\pgfqpoint{1.052720in}{1.455305in}}%
\pgfpathcurveto{\pgfqpoint{1.054023in}{1.454003in}}{\pgfqpoint{1.055789in}{1.453271in}}{\pgfqpoint{1.057631in}{1.453271in}}%
\pgfpathlineto{\pgfqpoint{1.057631in}{1.453271in}}%
\pgfpathclose%
\pgfusepath{stroke,fill}%
\end{pgfscope}%
\begin{pgfscope}%
\pgfpathrectangle{\pgfqpoint{0.661006in}{0.524170in}}{\pgfqpoint{4.194036in}{1.071446in}}%
\pgfusepath{clip}%
\pgfsetbuttcap%
\pgfsetroundjoin%
\definecolor{currentfill}{rgb}{0.572355,0.680125,0.816045}%
\pgfsetfillcolor{currentfill}%
\pgfsetfillopacity{0.700000}%
\pgfsetlinewidth{1.003750pt}%
\definecolor{currentstroke}{rgb}{0.572355,0.680125,0.816045}%
\pgfsetstrokecolor{currentstroke}%
\pgfsetstrokeopacity{0.700000}%
\pgfsetdash{}{0pt}%
\pgfpathmoveto{\pgfqpoint{0.890406in}{1.487750in}}%
\pgfpathcurveto{\pgfqpoint{0.892247in}{1.487750in}}{\pgfqpoint{0.894014in}{1.488482in}}{\pgfqpoint{0.895316in}{1.489784in}}%
\pgfpathcurveto{\pgfqpoint{0.896618in}{1.491086in}}{\pgfqpoint{0.897350in}{1.492853in}}{\pgfqpoint{0.897350in}{1.494694in}}%
\pgfpathcurveto{\pgfqpoint{0.897350in}{1.496536in}}{\pgfqpoint{0.896618in}{1.498302in}}{\pgfqpoint{0.895316in}{1.499605in}}%
\pgfpathcurveto{\pgfqpoint{0.894014in}{1.500907in}}{\pgfqpoint{0.892247in}{1.501639in}}{\pgfqpoint{0.890406in}{1.501639in}}%
\pgfpathcurveto{\pgfqpoint{0.888564in}{1.501639in}}{\pgfqpoint{0.886798in}{1.500907in}}{\pgfqpoint{0.885495in}{1.499605in}}%
\pgfpathcurveto{\pgfqpoint{0.884193in}{1.498302in}}{\pgfqpoint{0.883461in}{1.496536in}}{\pgfqpoint{0.883461in}{1.494694in}}%
\pgfpathcurveto{\pgfqpoint{0.883461in}{1.492853in}}{\pgfqpoint{0.884193in}{1.491086in}}{\pgfqpoint{0.885495in}{1.489784in}}%
\pgfpathcurveto{\pgfqpoint{0.886798in}{1.488482in}}{\pgfqpoint{0.888564in}{1.487750in}}{\pgfqpoint{0.890406in}{1.487750in}}%
\pgfpathlineto{\pgfqpoint{0.890406in}{1.487750in}}%
\pgfpathclose%
\pgfusepath{stroke,fill}%
\end{pgfscope}%
\begin{pgfscope}%
\pgfpathrectangle{\pgfqpoint{0.661006in}{0.524170in}}{\pgfqpoint{4.194036in}{1.071446in}}%
\pgfusepath{clip}%
\pgfsetbuttcap%
\pgfsetroundjoin%
\definecolor{currentfill}{rgb}{0.572355,0.680125,0.816045}%
\pgfsetfillcolor{currentfill}%
\pgfsetfillopacity{0.700000}%
\pgfsetlinewidth{1.003750pt}%
\definecolor{currentstroke}{rgb}{0.572355,0.680125,0.816045}%
\pgfsetstrokecolor{currentstroke}%
\pgfsetstrokeopacity{0.700000}%
\pgfsetdash{}{0pt}%
\pgfpathmoveto{\pgfqpoint{1.050520in}{1.444000in}}%
\pgfpathcurveto{\pgfqpoint{1.052361in}{1.444000in}}{\pgfqpoint{1.054128in}{1.444732in}}{\pgfqpoint{1.055430in}{1.446034in}}%
\pgfpathcurveto{\pgfqpoint{1.056733in}{1.447336in}}{\pgfqpoint{1.057464in}{1.449103in}}{\pgfqpoint{1.057464in}{1.450945in}}%
\pgfpathcurveto{\pgfqpoint{1.057464in}{1.452786in}}{\pgfqpoint{1.056733in}{1.454553in}}{\pgfqpoint{1.055430in}{1.455855in}}%
\pgfpathcurveto{\pgfqpoint{1.054128in}{1.457157in}}{\pgfqpoint{1.052361in}{1.457889in}}{\pgfqpoint{1.050520in}{1.457889in}}%
\pgfpathcurveto{\pgfqpoint{1.048678in}{1.457889in}}{\pgfqpoint{1.046912in}{1.457157in}}{\pgfqpoint{1.045609in}{1.455855in}}%
\pgfpathcurveto{\pgfqpoint{1.044307in}{1.454553in}}{\pgfqpoint{1.043575in}{1.452786in}}{\pgfqpoint{1.043575in}{1.450945in}}%
\pgfpathcurveto{\pgfqpoint{1.043575in}{1.449103in}}{\pgfqpoint{1.044307in}{1.447336in}}{\pgfqpoint{1.045609in}{1.446034in}}%
\pgfpathcurveto{\pgfqpoint{1.046912in}{1.444732in}}{\pgfqpoint{1.048678in}{1.444000in}}{\pgfqpoint{1.050520in}{1.444000in}}%
\pgfpathlineto{\pgfqpoint{1.050520in}{1.444000in}}%
\pgfpathclose%
\pgfusepath{stroke,fill}%
\end{pgfscope}%
\begin{pgfscope}%
\pgfpathrectangle{\pgfqpoint{0.661006in}{0.524170in}}{\pgfqpoint{4.194036in}{1.071446in}}%
\pgfusepath{clip}%
\pgfsetbuttcap%
\pgfsetroundjoin%
\definecolor{currentfill}{rgb}{0.568742,0.675785,0.813526}%
\pgfsetfillcolor{currentfill}%
\pgfsetfillopacity{0.700000}%
\pgfsetlinewidth{1.003750pt}%
\definecolor{currentstroke}{rgb}{0.568742,0.675785,0.813526}%
\pgfsetstrokecolor{currentstroke}%
\pgfsetstrokeopacity{0.700000}%
\pgfsetdash{}{0pt}%
\pgfpathmoveto{\pgfqpoint{1.341514in}{1.376298in}}%
\pgfpathcurveto{\pgfqpoint{1.343355in}{1.376298in}}{\pgfqpoint{1.345122in}{1.377030in}}{\pgfqpoint{1.346424in}{1.378332in}}%
\pgfpathcurveto{\pgfqpoint{1.347726in}{1.379634in}}{\pgfqpoint{1.348458in}{1.381401in}}{\pgfqpoint{1.348458in}{1.383242in}}%
\pgfpathcurveto{\pgfqpoint{1.348458in}{1.385084in}}{\pgfqpoint{1.347726in}{1.386851in}}{\pgfqpoint{1.346424in}{1.388153in}}%
\pgfpathcurveto{\pgfqpoint{1.345122in}{1.389455in}}{\pgfqpoint{1.343355in}{1.390187in}}{\pgfqpoint{1.341514in}{1.390187in}}%
\pgfpathcurveto{\pgfqpoint{1.339672in}{1.390187in}}{\pgfqpoint{1.337906in}{1.389455in}}{\pgfqpoint{1.336603in}{1.388153in}}%
\pgfpathcurveto{\pgfqpoint{1.335301in}{1.386851in}}{\pgfqpoint{1.334569in}{1.385084in}}{\pgfqpoint{1.334569in}{1.383242in}}%
\pgfpathcurveto{\pgfqpoint{1.334569in}{1.381401in}}{\pgfqpoint{1.335301in}{1.379634in}}{\pgfqpoint{1.336603in}{1.378332in}}%
\pgfpathcurveto{\pgfqpoint{1.337906in}{1.377030in}}{\pgfqpoint{1.339672in}{1.376298in}}{\pgfqpoint{1.341514in}{1.376298in}}%
\pgfpathlineto{\pgfqpoint{1.341514in}{1.376298in}}%
\pgfpathclose%
\pgfusepath{stroke,fill}%
\end{pgfscope}%
\begin{pgfscope}%
\pgfpathrectangle{\pgfqpoint{0.661006in}{0.524170in}}{\pgfqpoint{4.194036in}{1.071446in}}%
\pgfusepath{clip}%
\pgfsetbuttcap%
\pgfsetroundjoin%
\definecolor{currentfill}{rgb}{0.568742,0.675785,0.813526}%
\pgfsetfillcolor{currentfill}%
\pgfsetfillopacity{0.700000}%
\pgfsetlinewidth{1.003750pt}%
\definecolor{currentstroke}{rgb}{0.568742,0.675785,0.813526}%
\pgfsetstrokecolor{currentstroke}%
\pgfsetstrokeopacity{0.700000}%
\pgfsetdash{}{0pt}%
\pgfpathmoveto{\pgfqpoint{1.581290in}{1.321625in}}%
\pgfpathcurveto{\pgfqpoint{1.583132in}{1.321625in}}{\pgfqpoint{1.584898in}{1.322357in}}{\pgfqpoint{1.586200in}{1.323659in}}%
\pgfpathcurveto{\pgfqpoint{1.587503in}{1.324962in}}{\pgfqpoint{1.588234in}{1.326728in}}{\pgfqpoint{1.588234in}{1.328570in}}%
\pgfpathcurveto{\pgfqpoint{1.588234in}{1.330412in}}{\pgfqpoint{1.587503in}{1.332178in}}{\pgfqpoint{1.586200in}{1.333480in}}%
\pgfpathcurveto{\pgfqpoint{1.584898in}{1.334783in}}{\pgfqpoint{1.583132in}{1.335514in}}{\pgfqpoint{1.581290in}{1.335514in}}%
\pgfpathcurveto{\pgfqpoint{1.579448in}{1.335514in}}{\pgfqpoint{1.577682in}{1.334783in}}{\pgfqpoint{1.576379in}{1.333480in}}%
\pgfpathcurveto{\pgfqpoint{1.575077in}{1.332178in}}{\pgfqpoint{1.574345in}{1.330412in}}{\pgfqpoint{1.574345in}{1.328570in}}%
\pgfpathcurveto{\pgfqpoint{1.574345in}{1.326728in}}{\pgfqpoint{1.575077in}{1.324962in}}{\pgfqpoint{1.576379in}{1.323659in}}%
\pgfpathcurveto{\pgfqpoint{1.577682in}{1.322357in}}{\pgfqpoint{1.579448in}{1.321625in}}{\pgfqpoint{1.581290in}{1.321625in}}%
\pgfpathlineto{\pgfqpoint{1.581290in}{1.321625in}}%
\pgfpathclose%
\pgfusepath{stroke,fill}%
\end{pgfscope}%
\begin{pgfscope}%
\pgfpathrectangle{\pgfqpoint{0.661006in}{0.524170in}}{\pgfqpoint{4.194036in}{1.071446in}}%
\pgfusepath{clip}%
\pgfsetbuttcap%
\pgfsetroundjoin%
\definecolor{currentfill}{rgb}{0.565155,0.671436,0.810986}%
\pgfsetfillcolor{currentfill}%
\pgfsetfillopacity{0.700000}%
\pgfsetlinewidth{1.003750pt}%
\definecolor{currentstroke}{rgb}{0.565155,0.671436,0.810986}%
\pgfsetstrokecolor{currentstroke}%
\pgfsetstrokeopacity{0.700000}%
\pgfsetdash{}{0pt}%
\pgfpathmoveto{\pgfqpoint{1.755115in}{1.281305in}}%
\pgfpathcurveto{\pgfqpoint{1.756956in}{1.281305in}}{\pgfqpoint{1.758723in}{1.282037in}}{\pgfqpoint{1.760025in}{1.283339in}}%
\pgfpathcurveto{\pgfqpoint{1.761327in}{1.284641in}}{\pgfqpoint{1.762059in}{1.286408in}}{\pgfqpoint{1.762059in}{1.288249in}}%
\pgfpathcurveto{\pgfqpoint{1.762059in}{1.290091in}}{\pgfqpoint{1.761327in}{1.291858in}}{\pgfqpoint{1.760025in}{1.293160in}}%
\pgfpathcurveto{\pgfqpoint{1.758723in}{1.294462in}}{\pgfqpoint{1.756956in}{1.295194in}}{\pgfqpoint{1.755115in}{1.295194in}}%
\pgfpathcurveto{\pgfqpoint{1.753273in}{1.295194in}}{\pgfqpoint{1.751507in}{1.294462in}}{\pgfqpoint{1.750204in}{1.293160in}}%
\pgfpathcurveto{\pgfqpoint{1.748902in}{1.291858in}}{\pgfqpoint{1.748170in}{1.290091in}}{\pgfqpoint{1.748170in}{1.288249in}}%
\pgfpathcurveto{\pgfqpoint{1.748170in}{1.286408in}}{\pgfqpoint{1.748902in}{1.284641in}}{\pgfqpoint{1.750204in}{1.283339in}}%
\pgfpathcurveto{\pgfqpoint{1.751507in}{1.282037in}}{\pgfqpoint{1.753273in}{1.281305in}}{\pgfqpoint{1.755115in}{1.281305in}}%
\pgfpathlineto{\pgfqpoint{1.755115in}{1.281305in}}%
\pgfpathclose%
\pgfusepath{stroke,fill}%
\end{pgfscope}%
\begin{pgfscope}%
\pgfpathrectangle{\pgfqpoint{0.661006in}{0.524170in}}{\pgfqpoint{4.194036in}{1.071446in}}%
\pgfusepath{clip}%
\pgfsetbuttcap%
\pgfsetroundjoin%
\definecolor{currentfill}{rgb}{0.565155,0.671436,0.810986}%
\pgfsetfillcolor{currentfill}%
\pgfsetfillopacity{0.700000}%
\pgfsetlinewidth{1.003750pt}%
\definecolor{currentstroke}{rgb}{0.565155,0.671436,0.810986}%
\pgfsetstrokecolor{currentstroke}%
\pgfsetstrokeopacity{0.700000}%
\pgfsetdash{}{0pt}%
\pgfpathmoveto{\pgfqpoint{1.877908in}{1.253466in}}%
\pgfpathcurveto{\pgfqpoint{1.879749in}{1.253466in}}{\pgfqpoint{1.881516in}{1.254198in}}{\pgfqpoint{1.882818in}{1.255500in}}%
\pgfpathcurveto{\pgfqpoint{1.884120in}{1.256802in}}{\pgfqpoint{1.884852in}{1.258569in}}{\pgfqpoint{1.884852in}{1.260410in}}%
\pgfpathcurveto{\pgfqpoint{1.884852in}{1.262252in}}{\pgfqpoint{1.884120in}{1.264018in}}{\pgfqpoint{1.882818in}{1.265321in}}%
\pgfpathcurveto{\pgfqpoint{1.881516in}{1.266623in}}{\pgfqpoint{1.879749in}{1.267355in}}{\pgfqpoint{1.877908in}{1.267355in}}%
\pgfpathcurveto{\pgfqpoint{1.876066in}{1.267355in}}{\pgfqpoint{1.874299in}{1.266623in}}{\pgfqpoint{1.872997in}{1.265321in}}%
\pgfpathcurveto{\pgfqpoint{1.871695in}{1.264018in}}{\pgfqpoint{1.870963in}{1.262252in}}{\pgfqpoint{1.870963in}{1.260410in}}%
\pgfpathcurveto{\pgfqpoint{1.870963in}{1.258569in}}{\pgfqpoint{1.871695in}{1.256802in}}{\pgfqpoint{1.872997in}{1.255500in}}%
\pgfpathcurveto{\pgfqpoint{1.874299in}{1.254198in}}{\pgfqpoint{1.876066in}{1.253466in}}{\pgfqpoint{1.877908in}{1.253466in}}%
\pgfpathlineto{\pgfqpoint{1.877908in}{1.253466in}}%
\pgfpathclose%
\pgfusepath{stroke,fill}%
\end{pgfscope}%
\begin{pgfscope}%
\pgfpathrectangle{\pgfqpoint{0.661006in}{0.524170in}}{\pgfqpoint{4.194036in}{1.071446in}}%
\pgfusepath{clip}%
\pgfsetbuttcap%
\pgfsetroundjoin%
\definecolor{currentfill}{rgb}{0.565155,0.671436,0.810986}%
\pgfsetfillcolor{currentfill}%
\pgfsetfillopacity{0.700000}%
\pgfsetlinewidth{1.003750pt}%
\definecolor{currentstroke}{rgb}{0.565155,0.671436,0.810986}%
\pgfsetstrokecolor{currentstroke}%
\pgfsetstrokeopacity{0.700000}%
\pgfsetdash{}{0pt}%
\pgfpathmoveto{\pgfqpoint{1.964030in}{1.236915in}}%
\pgfpathcurveto{\pgfqpoint{1.965872in}{1.236915in}}{\pgfqpoint{1.967638in}{1.237647in}}{\pgfqpoint{1.968940in}{1.238949in}}%
\pgfpathcurveto{\pgfqpoint{1.970243in}{1.240252in}}{\pgfqpoint{1.970974in}{1.242018in}}{\pgfqpoint{1.970974in}{1.243860in}}%
\pgfpathcurveto{\pgfqpoint{1.970974in}{1.245702in}}{\pgfqpoint{1.970243in}{1.247468in}}{\pgfqpoint{1.968940in}{1.248770in}}%
\pgfpathcurveto{\pgfqpoint{1.967638in}{1.250073in}}{\pgfqpoint{1.965872in}{1.250804in}}{\pgfqpoint{1.964030in}{1.250804in}}%
\pgfpathcurveto{\pgfqpoint{1.962188in}{1.250804in}}{\pgfqpoint{1.960422in}{1.250073in}}{\pgfqpoint{1.959119in}{1.248770in}}%
\pgfpathcurveto{\pgfqpoint{1.957817in}{1.247468in}}{\pgfqpoint{1.957085in}{1.245702in}}{\pgfqpoint{1.957085in}{1.243860in}}%
\pgfpathcurveto{\pgfqpoint{1.957085in}{1.242018in}}{\pgfqpoint{1.957817in}{1.240252in}}{\pgfqpoint{1.959119in}{1.238949in}}%
\pgfpathcurveto{\pgfqpoint{1.960422in}{1.237647in}}{\pgfqpoint{1.962188in}{1.236915in}}{\pgfqpoint{1.964030in}{1.236915in}}%
\pgfpathlineto{\pgfqpoint{1.964030in}{1.236915in}}%
\pgfpathclose%
\pgfusepath{stroke,fill}%
\end{pgfscope}%
\begin{pgfscope}%
\pgfpathrectangle{\pgfqpoint{0.661006in}{0.524170in}}{\pgfqpoint{4.194036in}{1.071446in}}%
\pgfusepath{clip}%
\pgfsetbuttcap%
\pgfsetroundjoin%
\definecolor{currentfill}{rgb}{0.565155,0.671436,0.810986}%
\pgfsetfillcolor{currentfill}%
\pgfsetfillopacity{0.700000}%
\pgfsetlinewidth{1.003750pt}%
\definecolor{currentstroke}{rgb}{0.565155,0.671436,0.810986}%
\pgfsetstrokecolor{currentstroke}%
\pgfsetstrokeopacity{0.700000}%
\pgfsetdash{}{0pt}%
\pgfpathmoveto{\pgfqpoint{1.982409in}{1.231270in}}%
\pgfpathcurveto{\pgfqpoint{1.984250in}{1.231270in}}{\pgfqpoint{1.986017in}{1.232002in}}{\pgfqpoint{1.987319in}{1.233304in}}%
\pgfpathcurveto{\pgfqpoint{1.988621in}{1.234606in}}{\pgfqpoint{1.989353in}{1.236373in}}{\pgfqpoint{1.989353in}{1.238214in}}%
\pgfpathcurveto{\pgfqpoint{1.989353in}{1.240056in}}{\pgfqpoint{1.988621in}{1.241823in}}{\pgfqpoint{1.987319in}{1.243125in}}%
\pgfpathcurveto{\pgfqpoint{1.986017in}{1.244427in}}{\pgfqpoint{1.984250in}{1.245159in}}{\pgfqpoint{1.982409in}{1.245159in}}%
\pgfpathcurveto{\pgfqpoint{1.980567in}{1.245159in}}{\pgfqpoint{1.978801in}{1.244427in}}{\pgfqpoint{1.977498in}{1.243125in}}%
\pgfpathcurveto{\pgfqpoint{1.976196in}{1.241823in}}{\pgfqpoint{1.975464in}{1.240056in}}{\pgfqpoint{1.975464in}{1.238214in}}%
\pgfpathcurveto{\pgfqpoint{1.975464in}{1.236373in}}{\pgfqpoint{1.976196in}{1.234606in}}{\pgfqpoint{1.977498in}{1.233304in}}%
\pgfpathcurveto{\pgfqpoint{1.978801in}{1.232002in}}{\pgfqpoint{1.980567in}{1.231270in}}{\pgfqpoint{1.982409in}{1.231270in}}%
\pgfpathlineto{\pgfqpoint{1.982409in}{1.231270in}}%
\pgfpathclose%
\pgfusepath{stroke,fill}%
\end{pgfscope}%
\begin{pgfscope}%
\pgfpathrectangle{\pgfqpoint{0.661006in}{0.524170in}}{\pgfqpoint{4.194036in}{1.071446in}}%
\pgfusepath{clip}%
\pgfsetbuttcap%
\pgfsetroundjoin%
\definecolor{currentfill}{rgb}{0.565155,0.671436,0.810986}%
\pgfsetfillcolor{currentfill}%
\pgfsetfillopacity{0.700000}%
\pgfsetlinewidth{1.003750pt}%
\definecolor{currentstroke}{rgb}{0.565155,0.671436,0.810986}%
\pgfsetstrokecolor{currentstroke}%
\pgfsetstrokeopacity{0.700000}%
\pgfsetdash{}{0pt}%
\pgfpathmoveto{\pgfqpoint{2.001491in}{1.228723in}}%
\pgfpathcurveto{\pgfqpoint{2.003332in}{1.228723in}}{\pgfqpoint{2.005099in}{1.229454in}}{\pgfqpoint{2.006401in}{1.230757in}}%
\pgfpathcurveto{\pgfqpoint{2.007703in}{1.232059in}}{\pgfqpoint{2.008435in}{1.233825in}}{\pgfqpoint{2.008435in}{1.235667in}}%
\pgfpathcurveto{\pgfqpoint{2.008435in}{1.237509in}}{\pgfqpoint{2.007703in}{1.239275in}}{\pgfqpoint{2.006401in}{1.240578in}}%
\pgfpathcurveto{\pgfqpoint{2.005099in}{1.241880in}}{\pgfqpoint{2.003332in}{1.242612in}}{\pgfqpoint{2.001491in}{1.242612in}}%
\pgfpathcurveto{\pgfqpoint{1.999649in}{1.242612in}}{\pgfqpoint{1.997882in}{1.241880in}}{\pgfqpoint{1.996580in}{1.240578in}}%
\pgfpathcurveto{\pgfqpoint{1.995278in}{1.239275in}}{\pgfqpoint{1.994546in}{1.237509in}}{\pgfqpoint{1.994546in}{1.235667in}}%
\pgfpathcurveto{\pgfqpoint{1.994546in}{1.233825in}}{\pgfqpoint{1.995278in}{1.232059in}}{\pgfqpoint{1.996580in}{1.230757in}}%
\pgfpathcurveto{\pgfqpoint{1.997882in}{1.229454in}}{\pgfqpoint{1.999649in}{1.228723in}}{\pgfqpoint{2.001491in}{1.228723in}}%
\pgfpathlineto{\pgfqpoint{2.001491in}{1.228723in}}%
\pgfpathclose%
\pgfusepath{stroke,fill}%
\end{pgfscope}%
\begin{pgfscope}%
\pgfpathrectangle{\pgfqpoint{0.661006in}{0.524170in}}{\pgfqpoint{4.194036in}{1.071446in}}%
\pgfusepath{clip}%
\pgfsetbuttcap%
\pgfsetroundjoin%
\definecolor{currentfill}{rgb}{0.561592,0.667078,0.808424}%
\pgfsetfillcolor{currentfill}%
\pgfsetfillopacity{0.700000}%
\pgfsetlinewidth{1.003750pt}%
\definecolor{currentstroke}{rgb}{0.561592,0.667078,0.808424}%
\pgfsetstrokecolor{currentstroke}%
\pgfsetstrokeopacity{0.700000}%
\pgfsetdash{}{0pt}%
\pgfpathmoveto{\pgfqpoint{2.006138in}{1.227281in}}%
\pgfpathcurveto{\pgfqpoint{2.007980in}{1.227281in}}{\pgfqpoint{2.009746in}{1.228013in}}{\pgfqpoint{2.011049in}{1.229315in}}%
\pgfpathcurveto{\pgfqpoint{2.012351in}{1.230618in}}{\pgfqpoint{2.013083in}{1.232384in}}{\pgfqpoint{2.013083in}{1.234226in}}%
\pgfpathcurveto{\pgfqpoint{2.013083in}{1.236068in}}{\pgfqpoint{2.012351in}{1.237834in}}{\pgfqpoint{2.011049in}{1.239136in}}%
\pgfpathcurveto{\pgfqpoint{2.009746in}{1.240439in}}{\pgfqpoint{2.007980in}{1.241170in}}{\pgfqpoint{2.006138in}{1.241170in}}%
\pgfpathcurveto{\pgfqpoint{2.004297in}{1.241170in}}{\pgfqpoint{2.002530in}{1.240439in}}{\pgfqpoint{2.001228in}{1.239136in}}%
\pgfpathcurveto{\pgfqpoint{1.999926in}{1.237834in}}{\pgfqpoint{1.999194in}{1.236068in}}{\pgfqpoint{1.999194in}{1.234226in}}%
\pgfpathcurveto{\pgfqpoint{1.999194in}{1.232384in}}{\pgfqpoint{1.999926in}{1.230618in}}{\pgfqpoint{2.001228in}{1.229315in}}%
\pgfpathcurveto{\pgfqpoint{2.002530in}{1.228013in}}{\pgfqpoint{2.004297in}{1.227281in}}{\pgfqpoint{2.006138in}{1.227281in}}%
\pgfpathlineto{\pgfqpoint{2.006138in}{1.227281in}}%
\pgfpathclose%
\pgfusepath{stroke,fill}%
\end{pgfscope}%
\begin{pgfscope}%
\pgfpathrectangle{\pgfqpoint{0.661006in}{0.524170in}}{\pgfqpoint{4.194036in}{1.071446in}}%
\pgfusepath{clip}%
\pgfsetbuttcap%
\pgfsetroundjoin%
\definecolor{currentfill}{rgb}{0.561592,0.667078,0.808424}%
\pgfsetfillcolor{currentfill}%
\pgfsetfillopacity{0.700000}%
\pgfsetlinewidth{1.003750pt}%
\definecolor{currentstroke}{rgb}{0.561592,0.667078,0.808424}%
\pgfsetstrokecolor{currentstroke}%
\pgfsetstrokeopacity{0.700000}%
\pgfsetdash{}{0pt}%
\pgfpathmoveto{\pgfqpoint{2.008276in}{1.226965in}}%
\pgfpathcurveto{\pgfqpoint{2.010118in}{1.226965in}}{\pgfqpoint{2.011884in}{1.227697in}}{\pgfqpoint{2.013187in}{1.228999in}}%
\pgfpathcurveto{\pgfqpoint{2.014489in}{1.230301in}}{\pgfqpoint{2.015221in}{1.232068in}}{\pgfqpoint{2.015221in}{1.233909in}}%
\pgfpathcurveto{\pgfqpoint{2.015221in}{1.235751in}}{\pgfqpoint{2.014489in}{1.237518in}}{\pgfqpoint{2.013187in}{1.238820in}}%
\pgfpathcurveto{\pgfqpoint{2.011884in}{1.240122in}}{\pgfqpoint{2.010118in}{1.240854in}}{\pgfqpoint{2.008276in}{1.240854in}}%
\pgfpathcurveto{\pgfqpoint{2.006435in}{1.240854in}}{\pgfqpoint{2.004668in}{1.240122in}}{\pgfqpoint{2.003366in}{1.238820in}}%
\pgfpathcurveto{\pgfqpoint{2.002063in}{1.237518in}}{\pgfqpoint{2.001332in}{1.235751in}}{\pgfqpoint{2.001332in}{1.233909in}}%
\pgfpathcurveto{\pgfqpoint{2.001332in}{1.232068in}}{\pgfqpoint{2.002063in}{1.230301in}}{\pgfqpoint{2.003366in}{1.228999in}}%
\pgfpathcurveto{\pgfqpoint{2.004668in}{1.227697in}}{\pgfqpoint{2.006435in}{1.226965in}}{\pgfqpoint{2.008276in}{1.226965in}}%
\pgfpathlineto{\pgfqpoint{2.008276in}{1.226965in}}%
\pgfpathclose%
\pgfusepath{stroke,fill}%
\end{pgfscope}%
\begin{pgfscope}%
\pgfpathrectangle{\pgfqpoint{0.661006in}{0.524170in}}{\pgfqpoint{4.194036in}{1.071446in}}%
\pgfusepath{clip}%
\pgfsetbuttcap%
\pgfsetroundjoin%
\definecolor{currentfill}{rgb}{0.558053,0.662713,0.805841}%
\pgfsetfillcolor{currentfill}%
\pgfsetfillopacity{0.700000}%
\pgfsetlinewidth{1.003750pt}%
\definecolor{currentstroke}{rgb}{0.558053,0.662713,0.805841}%
\pgfsetstrokecolor{currentstroke}%
\pgfsetstrokeopacity{0.700000}%
\pgfsetdash{}{0pt}%
\pgfpathmoveto{\pgfqpoint{2.028076in}{1.222611in}}%
\pgfpathcurveto{\pgfqpoint{2.029917in}{1.222611in}}{\pgfqpoint{2.031684in}{1.223343in}}{\pgfqpoint{2.032986in}{1.224645in}}%
\pgfpathcurveto{\pgfqpoint{2.034288in}{1.225947in}}{\pgfqpoint{2.035020in}{1.227714in}}{\pgfqpoint{2.035020in}{1.229555in}}%
\pgfpathcurveto{\pgfqpoint{2.035020in}{1.231397in}}{\pgfqpoint{2.034288in}{1.233164in}}{\pgfqpoint{2.032986in}{1.234466in}}%
\pgfpathcurveto{\pgfqpoint{2.031684in}{1.235768in}}{\pgfqpoint{2.029917in}{1.236500in}}{\pgfqpoint{2.028076in}{1.236500in}}%
\pgfpathcurveto{\pgfqpoint{2.026234in}{1.236500in}}{\pgfqpoint{2.024467in}{1.235768in}}{\pgfqpoint{2.023165in}{1.234466in}}%
\pgfpathcurveto{\pgfqpoint{2.021863in}{1.233164in}}{\pgfqpoint{2.021131in}{1.231397in}}{\pgfqpoint{2.021131in}{1.229555in}}%
\pgfpathcurveto{\pgfqpoint{2.021131in}{1.227714in}}{\pgfqpoint{2.021863in}{1.225947in}}{\pgfqpoint{2.023165in}{1.224645in}}%
\pgfpathcurveto{\pgfqpoint{2.024467in}{1.223343in}}{\pgfqpoint{2.026234in}{1.222611in}}{\pgfqpoint{2.028076in}{1.222611in}}%
\pgfpathlineto{\pgfqpoint{2.028076in}{1.222611in}}%
\pgfpathclose%
\pgfusepath{stroke,fill}%
\end{pgfscope}%
\begin{pgfscope}%
\pgfpathrectangle{\pgfqpoint{0.661006in}{0.524170in}}{\pgfqpoint{4.194036in}{1.071446in}}%
\pgfusepath{clip}%
\pgfsetbuttcap%
\pgfsetroundjoin%
\definecolor{currentfill}{rgb}{0.558053,0.662713,0.805841}%
\pgfsetfillcolor{currentfill}%
\pgfsetfillopacity{0.700000}%
\pgfsetlinewidth{1.003750pt}%
\definecolor{currentstroke}{rgb}{0.558053,0.662713,0.805841}%
\pgfsetstrokecolor{currentstroke}%
\pgfsetstrokeopacity{0.700000}%
\pgfsetdash{}{0pt}%
\pgfpathmoveto{\pgfqpoint{2.049130in}{1.217494in}}%
\pgfpathcurveto{\pgfqpoint{2.050971in}{1.217494in}}{\pgfqpoint{2.052738in}{1.218226in}}{\pgfqpoint{2.054040in}{1.219528in}}%
\pgfpathcurveto{\pgfqpoint{2.055342in}{1.220830in}}{\pgfqpoint{2.056074in}{1.222597in}}{\pgfqpoint{2.056074in}{1.224439in}}%
\pgfpathcurveto{\pgfqpoint{2.056074in}{1.226280in}}{\pgfqpoint{2.055342in}{1.228047in}}{\pgfqpoint{2.054040in}{1.229349in}}%
\pgfpathcurveto{\pgfqpoint{2.052738in}{1.230651in}}{\pgfqpoint{2.050971in}{1.231383in}}{\pgfqpoint{2.049130in}{1.231383in}}%
\pgfpathcurveto{\pgfqpoint{2.047288in}{1.231383in}}{\pgfqpoint{2.045522in}{1.230651in}}{\pgfqpoint{2.044219in}{1.229349in}}%
\pgfpathcurveto{\pgfqpoint{2.042917in}{1.228047in}}{\pgfqpoint{2.042185in}{1.226280in}}{\pgfqpoint{2.042185in}{1.224439in}}%
\pgfpathcurveto{\pgfqpoint{2.042185in}{1.222597in}}{\pgfqpoint{2.042917in}{1.220830in}}{\pgfqpoint{2.044219in}{1.219528in}}%
\pgfpathcurveto{\pgfqpoint{2.045522in}{1.218226in}}{\pgfqpoint{2.047288in}{1.217494in}}{\pgfqpoint{2.049130in}{1.217494in}}%
\pgfpathlineto{\pgfqpoint{2.049130in}{1.217494in}}%
\pgfpathclose%
\pgfusepath{stroke,fill}%
\end{pgfscope}%
\begin{pgfscope}%
\pgfpathrectangle{\pgfqpoint{0.661006in}{0.524170in}}{\pgfqpoint{4.194036in}{1.071446in}}%
\pgfusepath{clip}%
\pgfsetbuttcap%
\pgfsetroundjoin%
\definecolor{currentfill}{rgb}{0.554538,0.658339,0.803236}%
\pgfsetfillcolor{currentfill}%
\pgfsetfillopacity{0.700000}%
\pgfsetlinewidth{1.003750pt}%
\definecolor{currentstroke}{rgb}{0.554538,0.658339,0.803236}%
\pgfsetstrokecolor{currentstroke}%
\pgfsetstrokeopacity{0.700000}%
\pgfsetdash{}{0pt}%
\pgfpathmoveto{\pgfqpoint{2.060331in}{1.214269in}}%
\pgfpathcurveto{\pgfqpoint{2.062172in}{1.214269in}}{\pgfqpoint{2.063939in}{1.215000in}}{\pgfqpoint{2.065241in}{1.216303in}}%
\pgfpathcurveto{\pgfqpoint{2.066543in}{1.217605in}}{\pgfqpoint{2.067275in}{1.219371in}}{\pgfqpoint{2.067275in}{1.221213in}}%
\pgfpathcurveto{\pgfqpoint{2.067275in}{1.223055in}}{\pgfqpoint{2.066543in}{1.224821in}}{\pgfqpoint{2.065241in}{1.226123in}}%
\pgfpathcurveto{\pgfqpoint{2.063939in}{1.227426in}}{\pgfqpoint{2.062172in}{1.228157in}}{\pgfqpoint{2.060331in}{1.228157in}}%
\pgfpathcurveto{\pgfqpoint{2.058489in}{1.228157in}}{\pgfqpoint{2.056723in}{1.227426in}}{\pgfqpoint{2.055420in}{1.226123in}}%
\pgfpathcurveto{\pgfqpoint{2.054118in}{1.224821in}}{\pgfqpoint{2.053386in}{1.223055in}}{\pgfqpoint{2.053386in}{1.221213in}}%
\pgfpathcurveto{\pgfqpoint{2.053386in}{1.219371in}}{\pgfqpoint{2.054118in}{1.217605in}}{\pgfqpoint{2.055420in}{1.216303in}}%
\pgfpathcurveto{\pgfqpoint{2.056723in}{1.215000in}}{\pgfqpoint{2.058489in}{1.214269in}}{\pgfqpoint{2.060331in}{1.214269in}}%
\pgfpathlineto{\pgfqpoint{2.060331in}{1.214269in}}%
\pgfpathclose%
\pgfusepath{stroke,fill}%
\end{pgfscope}%
\begin{pgfscope}%
\pgfpathrectangle{\pgfqpoint{0.661006in}{0.524170in}}{\pgfqpoint{4.194036in}{1.071446in}}%
\pgfusepath{clip}%
\pgfsetbuttcap%
\pgfsetroundjoin%
\definecolor{currentfill}{rgb}{0.554538,0.658339,0.803236}%
\pgfsetfillcolor{currentfill}%
\pgfsetfillopacity{0.700000}%
\pgfsetlinewidth{1.003750pt}%
\definecolor{currentstroke}{rgb}{0.554538,0.658339,0.803236}%
\pgfsetstrokecolor{currentstroke}%
\pgfsetstrokeopacity{0.700000}%
\pgfsetdash{}{0pt}%
\pgfpathmoveto{\pgfqpoint{2.066652in}{1.214030in}}%
\pgfpathcurveto{\pgfqpoint{2.068493in}{1.214030in}}{\pgfqpoint{2.070260in}{1.214762in}}{\pgfqpoint{2.071562in}{1.216064in}}%
\pgfpathcurveto{\pgfqpoint{2.072864in}{1.217366in}}{\pgfqpoint{2.073596in}{1.219133in}}{\pgfqpoint{2.073596in}{1.220975in}}%
\pgfpathcurveto{\pgfqpoint{2.073596in}{1.222816in}}{\pgfqpoint{2.072864in}{1.224583in}}{\pgfqpoint{2.071562in}{1.225885in}}%
\pgfpathcurveto{\pgfqpoint{2.070260in}{1.227187in}}{\pgfqpoint{2.068493in}{1.227919in}}{\pgfqpoint{2.066652in}{1.227919in}}%
\pgfpathcurveto{\pgfqpoint{2.064810in}{1.227919in}}{\pgfqpoint{2.063043in}{1.227187in}}{\pgfqpoint{2.061741in}{1.225885in}}%
\pgfpathcurveto{\pgfqpoint{2.060439in}{1.224583in}}{\pgfqpoint{2.059707in}{1.222816in}}{\pgfqpoint{2.059707in}{1.220975in}}%
\pgfpathcurveto{\pgfqpoint{2.059707in}{1.219133in}}{\pgfqpoint{2.060439in}{1.217366in}}{\pgfqpoint{2.061741in}{1.216064in}}%
\pgfpathcurveto{\pgfqpoint{2.063043in}{1.214762in}}{\pgfqpoint{2.064810in}{1.214030in}}{\pgfqpoint{2.066652in}{1.214030in}}%
\pgfpathlineto{\pgfqpoint{2.066652in}{1.214030in}}%
\pgfpathclose%
\pgfusepath{stroke,fill}%
\end{pgfscope}%
\begin{pgfscope}%
\pgfpathrectangle{\pgfqpoint{0.661006in}{0.524170in}}{\pgfqpoint{4.194036in}{1.071446in}}%
\pgfusepath{clip}%
\pgfsetbuttcap%
\pgfsetroundjoin%
\definecolor{currentfill}{rgb}{0.554538,0.658339,0.803236}%
\pgfsetfillcolor{currentfill}%
\pgfsetfillopacity{0.700000}%
\pgfsetlinewidth{1.003750pt}%
\definecolor{currentstroke}{rgb}{0.554538,0.658339,0.803236}%
\pgfsetstrokecolor{currentstroke}%
\pgfsetstrokeopacity{0.700000}%
\pgfsetdash{}{0pt}%
\pgfpathmoveto{\pgfqpoint{2.064560in}{1.214267in}}%
\pgfpathcurveto{\pgfqpoint{2.066402in}{1.214267in}}{\pgfqpoint{2.068168in}{1.214998in}}{\pgfqpoint{2.069471in}{1.216301in}}%
\pgfpathcurveto{\pgfqpoint{2.070773in}{1.217603in}}{\pgfqpoint{2.071505in}{1.219370in}}{\pgfqpoint{2.071505in}{1.221211in}}%
\pgfpathcurveto{\pgfqpoint{2.071505in}{1.223053in}}{\pgfqpoint{2.070773in}{1.224819in}}{\pgfqpoint{2.069471in}{1.226122in}}%
\pgfpathcurveto{\pgfqpoint{2.068168in}{1.227424in}}{\pgfqpoint{2.066402in}{1.228156in}}{\pgfqpoint{2.064560in}{1.228156in}}%
\pgfpathcurveto{\pgfqpoint{2.062718in}{1.228156in}}{\pgfqpoint{2.060952in}{1.227424in}}{\pgfqpoint{2.059650in}{1.226122in}}%
\pgfpathcurveto{\pgfqpoint{2.058347in}{1.224819in}}{\pgfqpoint{2.057616in}{1.223053in}}{\pgfqpoint{2.057616in}{1.221211in}}%
\pgfpathcurveto{\pgfqpoint{2.057616in}{1.219370in}}{\pgfqpoint{2.058347in}{1.217603in}}{\pgfqpoint{2.059650in}{1.216301in}}%
\pgfpathcurveto{\pgfqpoint{2.060952in}{1.214998in}}{\pgfqpoint{2.062718in}{1.214267in}}{\pgfqpoint{2.064560in}{1.214267in}}%
\pgfpathlineto{\pgfqpoint{2.064560in}{1.214267in}}%
\pgfpathclose%
\pgfusepath{stroke,fill}%
\end{pgfscope}%
\begin{pgfscope}%
\pgfpathrectangle{\pgfqpoint{0.661006in}{0.524170in}}{\pgfqpoint{4.194036in}{1.071446in}}%
\pgfusepath{clip}%
\pgfsetbuttcap%
\pgfsetroundjoin%
\definecolor{currentfill}{rgb}{0.554538,0.658339,0.803236}%
\pgfsetfillcolor{currentfill}%
\pgfsetfillopacity{0.700000}%
\pgfsetlinewidth{1.003750pt}%
\definecolor{currentstroke}{rgb}{0.554538,0.658339,0.803236}%
\pgfsetstrokecolor{currentstroke}%
\pgfsetstrokeopacity{0.700000}%
\pgfsetdash{}{0pt}%
\pgfpathmoveto{\pgfqpoint{2.075622in}{1.212606in}}%
\pgfpathcurveto{\pgfqpoint{2.077463in}{1.212606in}}{\pgfqpoint{2.079230in}{1.213338in}}{\pgfqpoint{2.080532in}{1.214640in}}%
\pgfpathcurveto{\pgfqpoint{2.081834in}{1.215942in}}{\pgfqpoint{2.082566in}{1.217709in}}{\pgfqpoint{2.082566in}{1.219550in}}%
\pgfpathcurveto{\pgfqpoint{2.082566in}{1.221392in}}{\pgfqpoint{2.081834in}{1.223158in}}{\pgfqpoint{2.080532in}{1.224461in}}%
\pgfpathcurveto{\pgfqpoint{2.079230in}{1.225763in}}{\pgfqpoint{2.077463in}{1.226495in}}{\pgfqpoint{2.075622in}{1.226495in}}%
\pgfpathcurveto{\pgfqpoint{2.073780in}{1.226495in}}{\pgfqpoint{2.072014in}{1.225763in}}{\pgfqpoint{2.070711in}{1.224461in}}%
\pgfpathcurveto{\pgfqpoint{2.069409in}{1.223158in}}{\pgfqpoint{2.068677in}{1.221392in}}{\pgfqpoint{2.068677in}{1.219550in}}%
\pgfpathcurveto{\pgfqpoint{2.068677in}{1.217709in}}{\pgfqpoint{2.069409in}{1.215942in}}{\pgfqpoint{2.070711in}{1.214640in}}%
\pgfpathcurveto{\pgfqpoint{2.072014in}{1.213338in}}{\pgfqpoint{2.073780in}{1.212606in}}{\pgfqpoint{2.075622in}{1.212606in}}%
\pgfpathlineto{\pgfqpoint{2.075622in}{1.212606in}}%
\pgfpathclose%
\pgfusepath{stroke,fill}%
\end{pgfscope}%
\begin{pgfscope}%
\pgfpathrectangle{\pgfqpoint{0.661006in}{0.524170in}}{\pgfqpoint{4.194036in}{1.071446in}}%
\pgfusepath{clip}%
\pgfsetbuttcap%
\pgfsetroundjoin%
\definecolor{currentfill}{rgb}{0.554538,0.658339,0.803236}%
\pgfsetfillcolor{currentfill}%
\pgfsetfillopacity{0.700000}%
\pgfsetlinewidth{1.003750pt}%
\definecolor{currentstroke}{rgb}{0.554538,0.658339,0.803236}%
\pgfsetstrokecolor{currentstroke}%
\pgfsetstrokeopacity{0.700000}%
\pgfsetdash{}{0pt}%
\pgfpathmoveto{\pgfqpoint{2.083616in}{1.210496in}}%
\pgfpathcurveto{\pgfqpoint{2.085458in}{1.210496in}}{\pgfqpoint{2.087224in}{1.211228in}}{\pgfqpoint{2.088526in}{1.212530in}}%
\pgfpathcurveto{\pgfqpoint{2.089829in}{1.213832in}}{\pgfqpoint{2.090560in}{1.215599in}}{\pgfqpoint{2.090560in}{1.217440in}}%
\pgfpathcurveto{\pgfqpoint{2.090560in}{1.219282in}}{\pgfqpoint{2.089829in}{1.221049in}}{\pgfqpoint{2.088526in}{1.222351in}}%
\pgfpathcurveto{\pgfqpoint{2.087224in}{1.223653in}}{\pgfqpoint{2.085458in}{1.224385in}}{\pgfqpoint{2.083616in}{1.224385in}}%
\pgfpathcurveto{\pgfqpoint{2.081774in}{1.224385in}}{\pgfqpoint{2.080008in}{1.223653in}}{\pgfqpoint{2.078705in}{1.222351in}}%
\pgfpathcurveto{\pgfqpoint{2.077403in}{1.221049in}}{\pgfqpoint{2.076671in}{1.219282in}}{\pgfqpoint{2.076671in}{1.217440in}}%
\pgfpathcurveto{\pgfqpoint{2.076671in}{1.215599in}}{\pgfqpoint{2.077403in}{1.213832in}}{\pgfqpoint{2.078705in}{1.212530in}}%
\pgfpathcurveto{\pgfqpoint{2.080008in}{1.211228in}}{\pgfqpoint{2.081774in}{1.210496in}}{\pgfqpoint{2.083616in}{1.210496in}}%
\pgfpathlineto{\pgfqpoint{2.083616in}{1.210496in}}%
\pgfpathclose%
\pgfusepath{stroke,fill}%
\end{pgfscope}%
\begin{pgfscope}%
\pgfpathrectangle{\pgfqpoint{0.661006in}{0.524170in}}{\pgfqpoint{4.194036in}{1.071446in}}%
\pgfusepath{clip}%
\pgfsetbuttcap%
\pgfsetroundjoin%
\definecolor{currentfill}{rgb}{0.551047,0.653957,0.800608}%
\pgfsetfillcolor{currentfill}%
\pgfsetfillopacity{0.700000}%
\pgfsetlinewidth{1.003750pt}%
\definecolor{currentstroke}{rgb}{0.551047,0.653957,0.800608}%
\pgfsetstrokecolor{currentstroke}%
\pgfsetstrokeopacity{0.700000}%
\pgfsetdash{}{0pt}%
\pgfpathmoveto{\pgfqpoint{2.101091in}{1.207084in}}%
\pgfpathcurveto{\pgfqpoint{2.102933in}{1.207084in}}{\pgfqpoint{2.104699in}{1.207815in}}{\pgfqpoint{2.106002in}{1.209118in}}%
\pgfpathcurveto{\pgfqpoint{2.107304in}{1.210420in}}{\pgfqpoint{2.108036in}{1.212187in}}{\pgfqpoint{2.108036in}{1.214028in}}%
\pgfpathcurveto{\pgfqpoint{2.108036in}{1.215870in}}{\pgfqpoint{2.107304in}{1.217636in}}{\pgfqpoint{2.106002in}{1.218939in}}%
\pgfpathcurveto{\pgfqpoint{2.104699in}{1.220241in}}{\pgfqpoint{2.102933in}{1.220973in}}{\pgfqpoint{2.101091in}{1.220973in}}%
\pgfpathcurveto{\pgfqpoint{2.099250in}{1.220973in}}{\pgfqpoint{2.097483in}{1.220241in}}{\pgfqpoint{2.096181in}{1.218939in}}%
\pgfpathcurveto{\pgfqpoint{2.094879in}{1.217636in}}{\pgfqpoint{2.094147in}{1.215870in}}{\pgfqpoint{2.094147in}{1.214028in}}%
\pgfpathcurveto{\pgfqpoint{2.094147in}{1.212187in}}{\pgfqpoint{2.094879in}{1.210420in}}{\pgfqpoint{2.096181in}{1.209118in}}%
\pgfpathcurveto{\pgfqpoint{2.097483in}{1.207815in}}{\pgfqpoint{2.099250in}{1.207084in}}{\pgfqpoint{2.101091in}{1.207084in}}%
\pgfpathlineto{\pgfqpoint{2.101091in}{1.207084in}}%
\pgfpathclose%
\pgfusepath{stroke,fill}%
\end{pgfscope}%
\begin{pgfscope}%
\pgfpathrectangle{\pgfqpoint{0.661006in}{0.524170in}}{\pgfqpoint{4.194036in}{1.071446in}}%
\pgfusepath{clip}%
\pgfsetbuttcap%
\pgfsetroundjoin%
\definecolor{currentfill}{rgb}{0.551047,0.653957,0.800608}%
\pgfsetfillcolor{currentfill}%
\pgfsetfillopacity{0.700000}%
\pgfsetlinewidth{1.003750pt}%
\definecolor{currentstroke}{rgb}{0.551047,0.653957,0.800608}%
\pgfsetstrokecolor{currentstroke}%
\pgfsetstrokeopacity{0.700000}%
\pgfsetdash{}{0pt}%
\pgfpathmoveto{\pgfqpoint{2.113873in}{1.204923in}}%
\pgfpathcurveto{\pgfqpoint{2.115714in}{1.204923in}}{\pgfqpoint{2.117481in}{1.205654in}}{\pgfqpoint{2.118783in}{1.206957in}}%
\pgfpathcurveto{\pgfqpoint{2.120085in}{1.208259in}}{\pgfqpoint{2.120817in}{1.210025in}}{\pgfqpoint{2.120817in}{1.211867in}}%
\pgfpathcurveto{\pgfqpoint{2.120817in}{1.213709in}}{\pgfqpoint{2.120085in}{1.215475in}}{\pgfqpoint{2.118783in}{1.216778in}}%
\pgfpathcurveto{\pgfqpoint{2.117481in}{1.218080in}}{\pgfqpoint{2.115714in}{1.218812in}}{\pgfqpoint{2.113873in}{1.218812in}}%
\pgfpathcurveto{\pgfqpoint{2.112031in}{1.218812in}}{\pgfqpoint{2.110264in}{1.218080in}}{\pgfqpoint{2.108962in}{1.216778in}}%
\pgfpathcurveto{\pgfqpoint{2.107660in}{1.215475in}}{\pgfqpoint{2.106928in}{1.213709in}}{\pgfqpoint{2.106928in}{1.211867in}}%
\pgfpathcurveto{\pgfqpoint{2.106928in}{1.210025in}}{\pgfqpoint{2.107660in}{1.208259in}}{\pgfqpoint{2.108962in}{1.206957in}}%
\pgfpathcurveto{\pgfqpoint{2.110264in}{1.205654in}}{\pgfqpoint{2.112031in}{1.204923in}}{\pgfqpoint{2.113873in}{1.204923in}}%
\pgfpathlineto{\pgfqpoint{2.113873in}{1.204923in}}%
\pgfpathclose%
\pgfusepath{stroke,fill}%
\end{pgfscope}%
\begin{pgfscope}%
\pgfpathrectangle{\pgfqpoint{0.661006in}{0.524170in}}{\pgfqpoint{4.194036in}{1.071446in}}%
\pgfusepath{clip}%
\pgfsetbuttcap%
\pgfsetroundjoin%
\definecolor{currentfill}{rgb}{0.547579,0.649568,0.797957}%
\pgfsetfillcolor{currentfill}%
\pgfsetfillopacity{0.700000}%
\pgfsetlinewidth{1.003750pt}%
\definecolor{currentstroke}{rgb}{0.547579,0.649568,0.797957}%
\pgfsetstrokecolor{currentstroke}%
\pgfsetstrokeopacity{0.700000}%
\pgfsetdash{}{0pt}%
\pgfpathmoveto{\pgfqpoint{2.111270in}{1.208091in}}%
\pgfpathcurveto{\pgfqpoint{2.113111in}{1.208091in}}{\pgfqpoint{2.114878in}{1.208823in}}{\pgfqpoint{2.116180in}{1.210125in}}%
\pgfpathcurveto{\pgfqpoint{2.117483in}{1.211427in}}{\pgfqpoint{2.118214in}{1.213194in}}{\pgfqpoint{2.118214in}{1.215035in}}%
\pgfpathcurveto{\pgfqpoint{2.118214in}{1.216877in}}{\pgfqpoint{2.117483in}{1.218644in}}{\pgfqpoint{2.116180in}{1.219946in}}%
\pgfpathcurveto{\pgfqpoint{2.114878in}{1.221248in}}{\pgfqpoint{2.113111in}{1.221980in}}{\pgfqpoint{2.111270in}{1.221980in}}%
\pgfpathcurveto{\pgfqpoint{2.109428in}{1.221980in}}{\pgfqpoint{2.107662in}{1.221248in}}{\pgfqpoint{2.106359in}{1.219946in}}%
\pgfpathcurveto{\pgfqpoint{2.105057in}{1.218644in}}{\pgfqpoint{2.104325in}{1.216877in}}{\pgfqpoint{2.104325in}{1.215035in}}%
\pgfpathcurveto{\pgfqpoint{2.104325in}{1.213194in}}{\pgfqpoint{2.105057in}{1.211427in}}{\pgfqpoint{2.106359in}{1.210125in}}%
\pgfpathcurveto{\pgfqpoint{2.107662in}{1.208823in}}{\pgfqpoint{2.109428in}{1.208091in}}{\pgfqpoint{2.111270in}{1.208091in}}%
\pgfpathlineto{\pgfqpoint{2.111270in}{1.208091in}}%
\pgfpathclose%
\pgfusepath{stroke,fill}%
\end{pgfscope}%
\begin{pgfscope}%
\pgfpathrectangle{\pgfqpoint{0.661006in}{0.524170in}}{\pgfqpoint{4.194036in}{1.071446in}}%
\pgfusepath{clip}%
\pgfsetbuttcap%
\pgfsetroundjoin%
\definecolor{currentfill}{rgb}{0.547579,0.649568,0.797957}%
\pgfsetfillcolor{currentfill}%
\pgfsetfillopacity{0.700000}%
\pgfsetlinewidth{1.003750pt}%
\definecolor{currentstroke}{rgb}{0.547579,0.649568,0.797957}%
\pgfsetstrokecolor{currentstroke}%
\pgfsetstrokeopacity{0.700000}%
\pgfsetdash{}{0pt}%
\pgfpathmoveto{\pgfqpoint{2.071346in}{1.216560in}}%
\pgfpathcurveto{\pgfqpoint{2.073188in}{1.216560in}}{\pgfqpoint{2.074954in}{1.217291in}}{\pgfqpoint{2.076256in}{1.218594in}}%
\pgfpathcurveto{\pgfqpoint{2.077559in}{1.219896in}}{\pgfqpoint{2.078290in}{1.221663in}}{\pgfqpoint{2.078290in}{1.223504in}}%
\pgfpathcurveto{\pgfqpoint{2.078290in}{1.225346in}}{\pgfqpoint{2.077559in}{1.227112in}}{\pgfqpoint{2.076256in}{1.228415in}}%
\pgfpathcurveto{\pgfqpoint{2.074954in}{1.229717in}}{\pgfqpoint{2.073188in}{1.230449in}}{\pgfqpoint{2.071346in}{1.230449in}}%
\pgfpathcurveto{\pgfqpoint{2.069504in}{1.230449in}}{\pgfqpoint{2.067738in}{1.229717in}}{\pgfqpoint{2.066435in}{1.228415in}}%
\pgfpathcurveto{\pgfqpoint{2.065133in}{1.227112in}}{\pgfqpoint{2.064401in}{1.225346in}}{\pgfqpoint{2.064401in}{1.223504in}}%
\pgfpathcurveto{\pgfqpoint{2.064401in}{1.221663in}}{\pgfqpoint{2.065133in}{1.219896in}}{\pgfqpoint{2.066435in}{1.218594in}}%
\pgfpathcurveto{\pgfqpoint{2.067738in}{1.217291in}}{\pgfqpoint{2.069504in}{1.216560in}}{\pgfqpoint{2.071346in}{1.216560in}}%
\pgfpathlineto{\pgfqpoint{2.071346in}{1.216560in}}%
\pgfpathclose%
\pgfusepath{stroke,fill}%
\end{pgfscope}%
\begin{pgfscope}%
\pgfpathrectangle{\pgfqpoint{0.661006in}{0.524170in}}{\pgfqpoint{4.194036in}{1.071446in}}%
\pgfusepath{clip}%
\pgfsetbuttcap%
\pgfsetroundjoin%
\definecolor{currentfill}{rgb}{0.547579,0.649568,0.797957}%
\pgfsetfillcolor{currentfill}%
\pgfsetfillopacity{0.700000}%
\pgfsetlinewidth{1.003750pt}%
\definecolor{currentstroke}{rgb}{0.547579,0.649568,0.797957}%
\pgfsetstrokecolor{currentstroke}%
\pgfsetstrokeopacity{0.700000}%
\pgfsetdash{}{0pt}%
\pgfpathmoveto{\pgfqpoint{2.040392in}{1.223426in}}%
\pgfpathcurveto{\pgfqpoint{2.042234in}{1.223426in}}{\pgfqpoint{2.044000in}{1.224157in}}{\pgfqpoint{2.045302in}{1.225460in}}%
\pgfpathcurveto{\pgfqpoint{2.046605in}{1.226762in}}{\pgfqpoint{2.047336in}{1.228528in}}{\pgfqpoint{2.047336in}{1.230370in}}%
\pgfpathcurveto{\pgfqpoint{2.047336in}{1.232212in}}{\pgfqpoint{2.046605in}{1.233978in}}{\pgfqpoint{2.045302in}{1.235281in}}%
\pgfpathcurveto{\pgfqpoint{2.044000in}{1.236583in}}{\pgfqpoint{2.042234in}{1.237314in}}{\pgfqpoint{2.040392in}{1.237314in}}%
\pgfpathcurveto{\pgfqpoint{2.038550in}{1.237314in}}{\pgfqpoint{2.036784in}{1.236583in}}{\pgfqpoint{2.035482in}{1.235281in}}%
\pgfpathcurveto{\pgfqpoint{2.034179in}{1.233978in}}{\pgfqpoint{2.033448in}{1.232212in}}{\pgfqpoint{2.033448in}{1.230370in}}%
\pgfpathcurveto{\pgfqpoint{2.033448in}{1.228528in}}{\pgfqpoint{2.034179in}{1.226762in}}{\pgfqpoint{2.035482in}{1.225460in}}%
\pgfpathcurveto{\pgfqpoint{2.036784in}{1.224157in}}{\pgfqpoint{2.038550in}{1.223426in}}{\pgfqpoint{2.040392in}{1.223426in}}%
\pgfpathlineto{\pgfqpoint{2.040392in}{1.223426in}}%
\pgfpathclose%
\pgfusepath{stroke,fill}%
\end{pgfscope}%
\begin{pgfscope}%
\pgfpathrectangle{\pgfqpoint{0.661006in}{0.524170in}}{\pgfqpoint{4.194036in}{1.071446in}}%
\pgfusepath{clip}%
\pgfsetbuttcap%
\pgfsetroundjoin%
\definecolor{currentfill}{rgb}{0.547579,0.649568,0.797957}%
\pgfsetfillcolor{currentfill}%
\pgfsetfillopacity{0.700000}%
\pgfsetlinewidth{1.003750pt}%
\definecolor{currentstroke}{rgb}{0.547579,0.649568,0.797957}%
\pgfsetstrokecolor{currentstroke}%
\pgfsetstrokeopacity{0.700000}%
\pgfsetdash{}{0pt}%
\pgfpathmoveto{\pgfqpoint{2.058468in}{1.220312in}}%
\pgfpathcurveto{\pgfqpoint{2.060310in}{1.220312in}}{\pgfqpoint{2.062077in}{1.221043in}}{\pgfqpoint{2.063379in}{1.222346in}}%
\pgfpathcurveto{\pgfqpoint{2.064681in}{1.223648in}}{\pgfqpoint{2.065413in}{1.225414in}}{\pgfqpoint{2.065413in}{1.227256in}}%
\pgfpathcurveto{\pgfqpoint{2.065413in}{1.229098in}}{\pgfqpoint{2.064681in}{1.230864in}}{\pgfqpoint{2.063379in}{1.232167in}}%
\pgfpathcurveto{\pgfqpoint{2.062077in}{1.233469in}}{\pgfqpoint{2.060310in}{1.234201in}}{\pgfqpoint{2.058468in}{1.234201in}}%
\pgfpathcurveto{\pgfqpoint{2.056627in}{1.234201in}}{\pgfqpoint{2.054860in}{1.233469in}}{\pgfqpoint{2.053558in}{1.232167in}}%
\pgfpathcurveto{\pgfqpoint{2.052256in}{1.230864in}}{\pgfqpoint{2.051524in}{1.229098in}}{\pgfqpoint{2.051524in}{1.227256in}}%
\pgfpathcurveto{\pgfqpoint{2.051524in}{1.225414in}}{\pgfqpoint{2.052256in}{1.223648in}}{\pgfqpoint{2.053558in}{1.222346in}}%
\pgfpathcurveto{\pgfqpoint{2.054860in}{1.221043in}}{\pgfqpoint{2.056627in}{1.220312in}}{\pgfqpoint{2.058468in}{1.220312in}}%
\pgfpathlineto{\pgfqpoint{2.058468in}{1.220312in}}%
\pgfpathclose%
\pgfusepath{stroke,fill}%
\end{pgfscope}%
\begin{pgfscope}%
\pgfpathrectangle{\pgfqpoint{0.661006in}{0.524170in}}{\pgfqpoint{4.194036in}{1.071446in}}%
\pgfusepath{clip}%
\pgfsetbuttcap%
\pgfsetroundjoin%
\definecolor{currentfill}{rgb}{0.547579,0.649568,0.797957}%
\pgfsetfillcolor{currentfill}%
\pgfsetfillopacity{0.700000}%
\pgfsetlinewidth{1.003750pt}%
\definecolor{currentstroke}{rgb}{0.547579,0.649568,0.797957}%
\pgfsetstrokecolor{currentstroke}%
\pgfsetstrokeopacity{0.700000}%
\pgfsetdash{}{0pt}%
\pgfpathmoveto{\pgfqpoint{2.088496in}{1.212429in}}%
\pgfpathcurveto{\pgfqpoint{2.090338in}{1.212429in}}{\pgfqpoint{2.092104in}{1.213160in}}{\pgfqpoint{2.093406in}{1.214463in}}%
\pgfpathcurveto{\pgfqpoint{2.094709in}{1.215765in}}{\pgfqpoint{2.095440in}{1.217531in}}{\pgfqpoint{2.095440in}{1.219373in}}%
\pgfpathcurveto{\pgfqpoint{2.095440in}{1.221215in}}{\pgfqpoint{2.094709in}{1.222981in}}{\pgfqpoint{2.093406in}{1.224284in}}%
\pgfpathcurveto{\pgfqpoint{2.092104in}{1.225586in}}{\pgfqpoint{2.090338in}{1.226318in}}{\pgfqpoint{2.088496in}{1.226318in}}%
\pgfpathcurveto{\pgfqpoint{2.086654in}{1.226318in}}{\pgfqpoint{2.084888in}{1.225586in}}{\pgfqpoint{2.083585in}{1.224284in}}%
\pgfpathcurveto{\pgfqpoint{2.082283in}{1.222981in}}{\pgfqpoint{2.081551in}{1.221215in}}{\pgfqpoint{2.081551in}{1.219373in}}%
\pgfpathcurveto{\pgfqpoint{2.081551in}{1.217531in}}{\pgfqpoint{2.082283in}{1.215765in}}{\pgfqpoint{2.083585in}{1.214463in}}%
\pgfpathcurveto{\pgfqpoint{2.084888in}{1.213160in}}{\pgfqpoint{2.086654in}{1.212429in}}{\pgfqpoint{2.088496in}{1.212429in}}%
\pgfpathlineto{\pgfqpoint{2.088496in}{1.212429in}}%
\pgfpathclose%
\pgfusepath{stroke,fill}%
\end{pgfscope}%
\begin{pgfscope}%
\pgfpathrectangle{\pgfqpoint{0.661006in}{0.524170in}}{\pgfqpoint{4.194036in}{1.071446in}}%
\pgfusepath{clip}%
\pgfsetbuttcap%
\pgfsetroundjoin%
\definecolor{currentfill}{rgb}{0.544133,0.645172,0.795283}%
\pgfsetfillcolor{currentfill}%
\pgfsetfillopacity{0.700000}%
\pgfsetlinewidth{1.003750pt}%
\definecolor{currentstroke}{rgb}{0.544133,0.645172,0.795283}%
\pgfsetstrokecolor{currentstroke}%
\pgfsetstrokeopacity{0.700000}%
\pgfsetdash{}{0pt}%
\pgfpathmoveto{\pgfqpoint{2.093934in}{1.211514in}}%
\pgfpathcurveto{\pgfqpoint{2.095775in}{1.211514in}}{\pgfqpoint{2.097542in}{1.212245in}}{\pgfqpoint{2.098844in}{1.213548in}}%
\pgfpathcurveto{\pgfqpoint{2.100146in}{1.214850in}}{\pgfqpoint{2.100878in}{1.216616in}}{\pgfqpoint{2.100878in}{1.218458in}}%
\pgfpathcurveto{\pgfqpoint{2.100878in}{1.220300in}}{\pgfqpoint{2.100146in}{1.222066in}}{\pgfqpoint{2.098844in}{1.223369in}}%
\pgfpathcurveto{\pgfqpoint{2.097542in}{1.224671in}}{\pgfqpoint{2.095775in}{1.225403in}}{\pgfqpoint{2.093934in}{1.225403in}}%
\pgfpathcurveto{\pgfqpoint{2.092092in}{1.225403in}}{\pgfqpoint{2.090326in}{1.224671in}}{\pgfqpoint{2.089023in}{1.223369in}}%
\pgfpathcurveto{\pgfqpoint{2.087721in}{1.222066in}}{\pgfqpoint{2.086989in}{1.220300in}}{\pgfqpoint{2.086989in}{1.218458in}}%
\pgfpathcurveto{\pgfqpoint{2.086989in}{1.216616in}}{\pgfqpoint{2.087721in}{1.214850in}}{\pgfqpoint{2.089023in}{1.213548in}}%
\pgfpathcurveto{\pgfqpoint{2.090326in}{1.212245in}}{\pgfqpoint{2.092092in}{1.211514in}}{\pgfqpoint{2.093934in}{1.211514in}}%
\pgfpathlineto{\pgfqpoint{2.093934in}{1.211514in}}%
\pgfpathclose%
\pgfusepath{stroke,fill}%
\end{pgfscope}%
\begin{pgfscope}%
\pgfpathrectangle{\pgfqpoint{0.661006in}{0.524170in}}{\pgfqpoint{4.194036in}{1.071446in}}%
\pgfusepath{clip}%
\pgfsetbuttcap%
\pgfsetroundjoin%
\definecolor{currentfill}{rgb}{0.544133,0.645172,0.795283}%
\pgfsetfillcolor{currentfill}%
\pgfsetfillopacity{0.700000}%
\pgfsetlinewidth{1.003750pt}%
\definecolor{currentstroke}{rgb}{0.544133,0.645172,0.795283}%
\pgfsetstrokecolor{currentstroke}%
\pgfsetstrokeopacity{0.700000}%
\pgfsetdash{}{0pt}%
\pgfpathmoveto{\pgfqpoint{2.077527in}{1.215987in}}%
\pgfpathcurveto{\pgfqpoint{2.079369in}{1.215987in}}{\pgfqpoint{2.081135in}{1.216719in}}{\pgfqpoint{2.082438in}{1.218021in}}%
\pgfpathcurveto{\pgfqpoint{2.083740in}{1.219324in}}{\pgfqpoint{2.084472in}{1.221090in}}{\pgfqpoint{2.084472in}{1.222932in}}%
\pgfpathcurveto{\pgfqpoint{2.084472in}{1.224773in}}{\pgfqpoint{2.083740in}{1.226540in}}{\pgfqpoint{2.082438in}{1.227842in}}%
\pgfpathcurveto{\pgfqpoint{2.081135in}{1.229145in}}{\pgfqpoint{2.079369in}{1.229876in}}{\pgfqpoint{2.077527in}{1.229876in}}%
\pgfpathcurveto{\pgfqpoint{2.075686in}{1.229876in}}{\pgfqpoint{2.073919in}{1.229145in}}{\pgfqpoint{2.072617in}{1.227842in}}%
\pgfpathcurveto{\pgfqpoint{2.071315in}{1.226540in}}{\pgfqpoint{2.070583in}{1.224773in}}{\pgfqpoint{2.070583in}{1.222932in}}%
\pgfpathcurveto{\pgfqpoint{2.070583in}{1.221090in}}{\pgfqpoint{2.071315in}{1.219324in}}{\pgfqpoint{2.072617in}{1.218021in}}%
\pgfpathcurveto{\pgfqpoint{2.073919in}{1.216719in}}{\pgfqpoint{2.075686in}{1.215987in}}{\pgfqpoint{2.077527in}{1.215987in}}%
\pgfpathlineto{\pgfqpoint{2.077527in}{1.215987in}}%
\pgfpathclose%
\pgfusepath{stroke,fill}%
\end{pgfscope}%
\begin{pgfscope}%
\pgfpathrectangle{\pgfqpoint{0.661006in}{0.524170in}}{\pgfqpoint{4.194036in}{1.071446in}}%
\pgfusepath{clip}%
\pgfsetbuttcap%
\pgfsetroundjoin%
\definecolor{currentfill}{rgb}{0.540711,0.640768,0.792586}%
\pgfsetfillcolor{currentfill}%
\pgfsetfillopacity{0.700000}%
\pgfsetlinewidth{1.003750pt}%
\definecolor{currentstroke}{rgb}{0.540711,0.640768,0.792586}%
\pgfsetstrokecolor{currentstroke}%
\pgfsetstrokeopacity{0.700000}%
\pgfsetdash{}{0pt}%
\pgfpathmoveto{\pgfqpoint{2.054707in}{1.222342in}}%
\pgfpathcurveto{\pgfqpoint{2.056549in}{1.222342in}}{\pgfqpoint{2.058315in}{1.223074in}}{\pgfqpoint{2.059617in}{1.224376in}}%
\pgfpathcurveto{\pgfqpoint{2.060920in}{1.225678in}}{\pgfqpoint{2.061651in}{1.227445in}}{\pgfqpoint{2.061651in}{1.229287in}}%
\pgfpathcurveto{\pgfqpoint{2.061651in}{1.231128in}}{\pgfqpoint{2.060920in}{1.232895in}}{\pgfqpoint{2.059617in}{1.234197in}}%
\pgfpathcurveto{\pgfqpoint{2.058315in}{1.235499in}}{\pgfqpoint{2.056549in}{1.236231in}}{\pgfqpoint{2.054707in}{1.236231in}}%
\pgfpathcurveto{\pgfqpoint{2.052865in}{1.236231in}}{\pgfqpoint{2.051099in}{1.235499in}}{\pgfqpoint{2.049797in}{1.234197in}}%
\pgfpathcurveto{\pgfqpoint{2.048494in}{1.232895in}}{\pgfqpoint{2.047763in}{1.231128in}}{\pgfqpoint{2.047763in}{1.229287in}}%
\pgfpathcurveto{\pgfqpoint{2.047763in}{1.227445in}}{\pgfqpoint{2.048494in}{1.225678in}}{\pgfqpoint{2.049797in}{1.224376in}}%
\pgfpathcurveto{\pgfqpoint{2.051099in}{1.223074in}}{\pgfqpoint{2.052865in}{1.222342in}}{\pgfqpoint{2.054707in}{1.222342in}}%
\pgfpathlineto{\pgfqpoint{2.054707in}{1.222342in}}%
\pgfpathclose%
\pgfusepath{stroke,fill}%
\end{pgfscope}%
\begin{pgfscope}%
\pgfpathrectangle{\pgfqpoint{0.661006in}{0.524170in}}{\pgfqpoint{4.194036in}{1.071446in}}%
\pgfusepath{clip}%
\pgfsetbuttcap%
\pgfsetroundjoin%
\definecolor{currentfill}{rgb}{0.540711,0.640768,0.792586}%
\pgfsetfillcolor{currentfill}%
\pgfsetfillopacity{0.700000}%
\pgfsetlinewidth{1.003750pt}%
\definecolor{currentstroke}{rgb}{0.540711,0.640768,0.792586}%
\pgfsetstrokecolor{currentstroke}%
\pgfsetstrokeopacity{0.700000}%
\pgfsetdash{}{0pt}%
\pgfpathmoveto{\pgfqpoint{2.034954in}{1.226208in}}%
\pgfpathcurveto{\pgfqpoint{2.036796in}{1.226208in}}{\pgfqpoint{2.038562in}{1.226940in}}{\pgfqpoint{2.039865in}{1.228242in}}%
\pgfpathcurveto{\pgfqpoint{2.041167in}{1.229544in}}{\pgfqpoint{2.041899in}{1.231311in}}{\pgfqpoint{2.041899in}{1.233152in}}%
\pgfpathcurveto{\pgfqpoint{2.041899in}{1.234994in}}{\pgfqpoint{2.041167in}{1.236760in}}{\pgfqpoint{2.039865in}{1.238063in}}%
\pgfpathcurveto{\pgfqpoint{2.038562in}{1.239365in}}{\pgfqpoint{2.036796in}{1.240097in}}{\pgfqpoint{2.034954in}{1.240097in}}%
\pgfpathcurveto{\pgfqpoint{2.033112in}{1.240097in}}{\pgfqpoint{2.031346in}{1.239365in}}{\pgfqpoint{2.030044in}{1.238063in}}%
\pgfpathcurveto{\pgfqpoint{2.028741in}{1.236760in}}{\pgfqpoint{2.028010in}{1.234994in}}{\pgfqpoint{2.028010in}{1.233152in}}%
\pgfpathcurveto{\pgfqpoint{2.028010in}{1.231311in}}{\pgfqpoint{2.028741in}{1.229544in}}{\pgfqpoint{2.030044in}{1.228242in}}%
\pgfpathcurveto{\pgfqpoint{2.031346in}{1.226940in}}{\pgfqpoint{2.033112in}{1.226208in}}{\pgfqpoint{2.034954in}{1.226208in}}%
\pgfpathlineto{\pgfqpoint{2.034954in}{1.226208in}}%
\pgfpathclose%
\pgfusepath{stroke,fill}%
\end{pgfscope}%
\begin{pgfscope}%
\pgfpathrectangle{\pgfqpoint{0.661006in}{0.524170in}}{\pgfqpoint{4.194036in}{1.071446in}}%
\pgfusepath{clip}%
\pgfsetbuttcap%
\pgfsetroundjoin%
\definecolor{currentfill}{rgb}{0.540711,0.640768,0.792586}%
\pgfsetfillcolor{currentfill}%
\pgfsetfillopacity{0.700000}%
\pgfsetlinewidth{1.003750pt}%
\definecolor{currentstroke}{rgb}{0.540711,0.640768,0.792586}%
\pgfsetstrokecolor{currentstroke}%
\pgfsetstrokeopacity{0.700000}%
\pgfsetdash{}{0pt}%
\pgfpathmoveto{\pgfqpoint{2.041229in}{1.224983in}}%
\pgfpathcurveto{\pgfqpoint{2.043070in}{1.224983in}}{\pgfqpoint{2.044837in}{1.225715in}}{\pgfqpoint{2.046139in}{1.227017in}}%
\pgfpathcurveto{\pgfqpoint{2.047441in}{1.228320in}}{\pgfqpoint{2.048173in}{1.230086in}}{\pgfqpoint{2.048173in}{1.231928in}}%
\pgfpathcurveto{\pgfqpoint{2.048173in}{1.233770in}}{\pgfqpoint{2.047441in}{1.235536in}}{\pgfqpoint{2.046139in}{1.236838in}}%
\pgfpathcurveto{\pgfqpoint{2.044837in}{1.238141in}}{\pgfqpoint{2.043070in}{1.238872in}}{\pgfqpoint{2.041229in}{1.238872in}}%
\pgfpathcurveto{\pgfqpoint{2.039387in}{1.238872in}}{\pgfqpoint{2.037620in}{1.238141in}}{\pgfqpoint{2.036318in}{1.236838in}}%
\pgfpathcurveto{\pgfqpoint{2.035016in}{1.235536in}}{\pgfqpoint{2.034284in}{1.233770in}}{\pgfqpoint{2.034284in}{1.231928in}}%
\pgfpathcurveto{\pgfqpoint{2.034284in}{1.230086in}}{\pgfqpoint{2.035016in}{1.228320in}}{\pgfqpoint{2.036318in}{1.227017in}}%
\pgfpathcurveto{\pgfqpoint{2.037620in}{1.225715in}}{\pgfqpoint{2.039387in}{1.224983in}}{\pgfqpoint{2.041229in}{1.224983in}}%
\pgfpathlineto{\pgfqpoint{2.041229in}{1.224983in}}%
\pgfpathclose%
\pgfusepath{stroke,fill}%
\end{pgfscope}%
\begin{pgfscope}%
\pgfpathrectangle{\pgfqpoint{0.661006in}{0.524170in}}{\pgfqpoint{4.194036in}{1.071446in}}%
\pgfusepath{clip}%
\pgfsetbuttcap%
\pgfsetroundjoin%
\definecolor{currentfill}{rgb}{0.537310,0.636358,0.789865}%
\pgfsetfillcolor{currentfill}%
\pgfsetfillopacity{0.700000}%
\pgfsetlinewidth{1.003750pt}%
\definecolor{currentstroke}{rgb}{0.537310,0.636358,0.789865}%
\pgfsetstrokecolor{currentstroke}%
\pgfsetstrokeopacity{0.700000}%
\pgfsetdash{}{0pt}%
\pgfpathmoveto{\pgfqpoint{2.066094in}{1.219068in}}%
\pgfpathcurveto{\pgfqpoint{2.067936in}{1.219068in}}{\pgfqpoint{2.069702in}{1.219800in}}{\pgfqpoint{2.071004in}{1.221102in}}%
\pgfpathcurveto{\pgfqpoint{2.072307in}{1.222404in}}{\pgfqpoint{2.073038in}{1.224171in}}{\pgfqpoint{2.073038in}{1.226013in}}%
\pgfpathcurveto{\pgfqpoint{2.073038in}{1.227854in}}{\pgfqpoint{2.072307in}{1.229621in}}{\pgfqpoint{2.071004in}{1.230923in}}%
\pgfpathcurveto{\pgfqpoint{2.069702in}{1.232225in}}{\pgfqpoint{2.067936in}{1.232957in}}{\pgfqpoint{2.066094in}{1.232957in}}%
\pgfpathcurveto{\pgfqpoint{2.064252in}{1.232957in}}{\pgfqpoint{2.062486in}{1.232225in}}{\pgfqpoint{2.061183in}{1.230923in}}%
\pgfpathcurveto{\pgfqpoint{2.059881in}{1.229621in}}{\pgfqpoint{2.059149in}{1.227854in}}{\pgfqpoint{2.059149in}{1.226013in}}%
\pgfpathcurveto{\pgfqpoint{2.059149in}{1.224171in}}{\pgfqpoint{2.059881in}{1.222404in}}{\pgfqpoint{2.061183in}{1.221102in}}%
\pgfpathcurveto{\pgfqpoint{2.062486in}{1.219800in}}{\pgfqpoint{2.064252in}{1.219068in}}{\pgfqpoint{2.066094in}{1.219068in}}%
\pgfpathlineto{\pgfqpoint{2.066094in}{1.219068in}}%
\pgfpathclose%
\pgfusepath{stroke,fill}%
\end{pgfscope}%
\begin{pgfscope}%
\pgfpathrectangle{\pgfqpoint{0.661006in}{0.524170in}}{\pgfqpoint{4.194036in}{1.071446in}}%
\pgfusepath{clip}%
\pgfsetbuttcap%
\pgfsetroundjoin%
\definecolor{currentfill}{rgb}{0.537310,0.636358,0.789865}%
\pgfsetfillcolor{currentfill}%
\pgfsetfillopacity{0.700000}%
\pgfsetlinewidth{1.003750pt}%
\definecolor{currentstroke}{rgb}{0.537310,0.636358,0.789865}%
\pgfsetstrokecolor{currentstroke}%
\pgfsetstrokeopacity{0.700000}%
\pgfsetdash{}{0pt}%
\pgfpathmoveto{\pgfqpoint{2.098953in}{1.211390in}}%
\pgfpathcurveto{\pgfqpoint{2.100795in}{1.211390in}}{\pgfqpoint{2.102562in}{1.212121in}}{\pgfqpoint{2.103864in}{1.213424in}}%
\pgfpathcurveto{\pgfqpoint{2.105166in}{1.214726in}}{\pgfqpoint{2.105898in}{1.216492in}}{\pgfqpoint{2.105898in}{1.218334in}}%
\pgfpathcurveto{\pgfqpoint{2.105898in}{1.220176in}}{\pgfqpoint{2.105166in}{1.221942in}}{\pgfqpoint{2.103864in}{1.223244in}}%
\pgfpathcurveto{\pgfqpoint{2.102562in}{1.224547in}}{\pgfqpoint{2.100795in}{1.225278in}}{\pgfqpoint{2.098953in}{1.225278in}}%
\pgfpathcurveto{\pgfqpoint{2.097112in}{1.225278in}}{\pgfqpoint{2.095345in}{1.224547in}}{\pgfqpoint{2.094043in}{1.223244in}}%
\pgfpathcurveto{\pgfqpoint{2.092741in}{1.221942in}}{\pgfqpoint{2.092009in}{1.220176in}}{\pgfqpoint{2.092009in}{1.218334in}}%
\pgfpathcurveto{\pgfqpoint{2.092009in}{1.216492in}}{\pgfqpoint{2.092741in}{1.214726in}}{\pgfqpoint{2.094043in}{1.213424in}}%
\pgfpathcurveto{\pgfqpoint{2.095345in}{1.212121in}}{\pgfqpoint{2.097112in}{1.211390in}}{\pgfqpoint{2.098953in}{1.211390in}}%
\pgfpathlineto{\pgfqpoint{2.098953in}{1.211390in}}%
\pgfpathclose%
\pgfusepath{stroke,fill}%
\end{pgfscope}%
\begin{pgfscope}%
\pgfpathrectangle{\pgfqpoint{0.661006in}{0.524170in}}{\pgfqpoint{4.194036in}{1.071446in}}%
\pgfusepath{clip}%
\pgfsetbuttcap%
\pgfsetroundjoin%
\definecolor{currentfill}{rgb}{0.537310,0.636358,0.789865}%
\pgfsetfillcolor{currentfill}%
\pgfsetfillopacity{0.700000}%
\pgfsetlinewidth{1.003750pt}%
\definecolor{currentstroke}{rgb}{0.537310,0.636358,0.789865}%
\pgfsetstrokecolor{currentstroke}%
\pgfsetstrokeopacity{0.700000}%
\pgfsetdash{}{0pt}%
\pgfpathmoveto{\pgfqpoint{2.117730in}{1.206231in}}%
\pgfpathcurveto{\pgfqpoint{2.119572in}{1.206231in}}{\pgfqpoint{2.121338in}{1.206963in}}{\pgfqpoint{2.122641in}{1.208265in}}%
\pgfpathcurveto{\pgfqpoint{2.123943in}{1.209567in}}{\pgfqpoint{2.124675in}{1.211334in}}{\pgfqpoint{2.124675in}{1.213175in}}%
\pgfpathcurveto{\pgfqpoint{2.124675in}{1.215017in}}{\pgfqpoint{2.123943in}{1.216784in}}{\pgfqpoint{2.122641in}{1.218086in}}%
\pgfpathcurveto{\pgfqpoint{2.121338in}{1.219388in}}{\pgfqpoint{2.119572in}{1.220120in}}{\pgfqpoint{2.117730in}{1.220120in}}%
\pgfpathcurveto{\pgfqpoint{2.115888in}{1.220120in}}{\pgfqpoint{2.114122in}{1.219388in}}{\pgfqpoint{2.112820in}{1.218086in}}%
\pgfpathcurveto{\pgfqpoint{2.111517in}{1.216784in}}{\pgfqpoint{2.110786in}{1.215017in}}{\pgfqpoint{2.110786in}{1.213175in}}%
\pgfpathcurveto{\pgfqpoint{2.110786in}{1.211334in}}{\pgfqpoint{2.111517in}{1.209567in}}{\pgfqpoint{2.112820in}{1.208265in}}%
\pgfpathcurveto{\pgfqpoint{2.114122in}{1.206963in}}{\pgfqpoint{2.115888in}{1.206231in}}{\pgfqpoint{2.117730in}{1.206231in}}%
\pgfpathlineto{\pgfqpoint{2.117730in}{1.206231in}}%
\pgfpathclose%
\pgfusepath{stroke,fill}%
\end{pgfscope}%
\begin{pgfscope}%
\pgfpathrectangle{\pgfqpoint{0.661006in}{0.524170in}}{\pgfqpoint{4.194036in}{1.071446in}}%
\pgfusepath{clip}%
\pgfsetbuttcap%
\pgfsetroundjoin%
\definecolor{currentfill}{rgb}{0.537310,0.636358,0.789865}%
\pgfsetfillcolor{currentfill}%
\pgfsetfillopacity{0.700000}%
\pgfsetlinewidth{1.003750pt}%
\definecolor{currentstroke}{rgb}{0.537310,0.636358,0.789865}%
\pgfsetstrokecolor{currentstroke}%
\pgfsetstrokeopacity{0.700000}%
\pgfsetdash{}{0pt}%
\pgfpathmoveto{\pgfqpoint{2.131813in}{1.205259in}}%
\pgfpathcurveto{\pgfqpoint{2.133654in}{1.205259in}}{\pgfqpoint{2.135421in}{1.205991in}}{\pgfqpoint{2.136723in}{1.207293in}}%
\pgfpathcurveto{\pgfqpoint{2.138025in}{1.208596in}}{\pgfqpoint{2.138757in}{1.210362in}}{\pgfqpoint{2.138757in}{1.212204in}}%
\pgfpathcurveto{\pgfqpoint{2.138757in}{1.214046in}}{\pgfqpoint{2.138025in}{1.215812in}}{\pgfqpoint{2.136723in}{1.217114in}}%
\pgfpathcurveto{\pgfqpoint{2.135421in}{1.218417in}}{\pgfqpoint{2.133654in}{1.219148in}}{\pgfqpoint{2.131813in}{1.219148in}}%
\pgfpathcurveto{\pgfqpoint{2.129971in}{1.219148in}}{\pgfqpoint{2.128205in}{1.218417in}}{\pgfqpoint{2.126902in}{1.217114in}}%
\pgfpathcurveto{\pgfqpoint{2.125600in}{1.215812in}}{\pgfqpoint{2.124868in}{1.214046in}}{\pgfqpoint{2.124868in}{1.212204in}}%
\pgfpathcurveto{\pgfqpoint{2.124868in}{1.210362in}}{\pgfqpoint{2.125600in}{1.208596in}}{\pgfqpoint{2.126902in}{1.207293in}}%
\pgfpathcurveto{\pgfqpoint{2.128205in}{1.205991in}}{\pgfqpoint{2.129971in}{1.205259in}}{\pgfqpoint{2.131813in}{1.205259in}}%
\pgfpathlineto{\pgfqpoint{2.131813in}{1.205259in}}%
\pgfpathclose%
\pgfusepath{stroke,fill}%
\end{pgfscope}%
\begin{pgfscope}%
\pgfpathrectangle{\pgfqpoint{0.661006in}{0.524170in}}{\pgfqpoint{4.194036in}{1.071446in}}%
\pgfusepath{clip}%
\pgfsetbuttcap%
\pgfsetroundjoin%
\definecolor{currentfill}{rgb}{0.533932,0.631941,0.787121}%
\pgfsetfillcolor{currentfill}%
\pgfsetfillopacity{0.700000}%
\pgfsetlinewidth{1.003750pt}%
\definecolor{currentstroke}{rgb}{0.533932,0.631941,0.787121}%
\pgfsetstrokecolor{currentstroke}%
\pgfsetstrokeopacity{0.700000}%
\pgfsetdash{}{0pt}%
\pgfpathmoveto{\pgfqpoint{2.127490in}{1.206813in}}%
\pgfpathcurveto{\pgfqpoint{2.129332in}{1.206813in}}{\pgfqpoint{2.131099in}{1.207545in}}{\pgfqpoint{2.132401in}{1.208847in}}%
\pgfpathcurveto{\pgfqpoint{2.133703in}{1.210149in}}{\pgfqpoint{2.134435in}{1.211916in}}{\pgfqpoint{2.134435in}{1.213757in}}%
\pgfpathcurveto{\pgfqpoint{2.134435in}{1.215599in}}{\pgfqpoint{2.133703in}{1.217366in}}{\pgfqpoint{2.132401in}{1.218668in}}%
\pgfpathcurveto{\pgfqpoint{2.131099in}{1.219970in}}{\pgfqpoint{2.129332in}{1.220702in}}{\pgfqpoint{2.127490in}{1.220702in}}%
\pgfpathcurveto{\pgfqpoint{2.125649in}{1.220702in}}{\pgfqpoint{2.123882in}{1.219970in}}{\pgfqpoint{2.122580in}{1.218668in}}%
\pgfpathcurveto{\pgfqpoint{2.121278in}{1.217366in}}{\pgfqpoint{2.120546in}{1.215599in}}{\pgfqpoint{2.120546in}{1.213757in}}%
\pgfpathcurveto{\pgfqpoint{2.120546in}{1.211916in}}{\pgfqpoint{2.121278in}{1.210149in}}{\pgfqpoint{2.122580in}{1.208847in}}%
\pgfpathcurveto{\pgfqpoint{2.123882in}{1.207545in}}{\pgfqpoint{2.125649in}{1.206813in}}{\pgfqpoint{2.127490in}{1.206813in}}%
\pgfpathlineto{\pgfqpoint{2.127490in}{1.206813in}}%
\pgfpathclose%
\pgfusepath{stroke,fill}%
\end{pgfscope}%
\begin{pgfscope}%
\pgfpathrectangle{\pgfqpoint{0.661006in}{0.524170in}}{\pgfqpoint{4.194036in}{1.071446in}}%
\pgfusepath{clip}%
\pgfsetbuttcap%
\pgfsetroundjoin%
\definecolor{currentfill}{rgb}{0.533932,0.631941,0.787121}%
\pgfsetfillcolor{currentfill}%
\pgfsetfillopacity{0.700000}%
\pgfsetlinewidth{1.003750pt}%
\definecolor{currentstroke}{rgb}{0.533932,0.631941,0.787121}%
\pgfsetstrokecolor{currentstroke}%
\pgfsetstrokeopacity{0.700000}%
\pgfsetdash{}{0pt}%
\pgfpathmoveto{\pgfqpoint{2.131069in}{1.206726in}}%
\pgfpathcurveto{\pgfqpoint{2.132911in}{1.206726in}}{\pgfqpoint{2.134677in}{1.207457in}}{\pgfqpoint{2.135980in}{1.208760in}}%
\pgfpathcurveto{\pgfqpoint{2.137282in}{1.210062in}}{\pgfqpoint{2.138014in}{1.211828in}}{\pgfqpoint{2.138014in}{1.213670in}}%
\pgfpathcurveto{\pgfqpoint{2.138014in}{1.215512in}}{\pgfqpoint{2.137282in}{1.217278in}}{\pgfqpoint{2.135980in}{1.218581in}}%
\pgfpathcurveto{\pgfqpoint{2.134677in}{1.219883in}}{\pgfqpoint{2.132911in}{1.220615in}}{\pgfqpoint{2.131069in}{1.220615in}}%
\pgfpathcurveto{\pgfqpoint{2.129227in}{1.220615in}}{\pgfqpoint{2.127461in}{1.219883in}}{\pgfqpoint{2.126159in}{1.218581in}}%
\pgfpathcurveto{\pgfqpoint{2.124856in}{1.217278in}}{\pgfqpoint{2.124125in}{1.215512in}}{\pgfqpoint{2.124125in}{1.213670in}}%
\pgfpathcurveto{\pgfqpoint{2.124125in}{1.211828in}}{\pgfqpoint{2.124856in}{1.210062in}}{\pgfqpoint{2.126159in}{1.208760in}}%
\pgfpathcurveto{\pgfqpoint{2.127461in}{1.207457in}}{\pgfqpoint{2.129227in}{1.206726in}}{\pgfqpoint{2.131069in}{1.206726in}}%
\pgfpathlineto{\pgfqpoint{2.131069in}{1.206726in}}%
\pgfpathclose%
\pgfusepath{stroke,fill}%
\end{pgfscope}%
\begin{pgfscope}%
\pgfpathrectangle{\pgfqpoint{0.661006in}{0.524170in}}{\pgfqpoint{4.194036in}{1.071446in}}%
\pgfusepath{clip}%
\pgfsetbuttcap%
\pgfsetroundjoin%
\definecolor{currentfill}{rgb}{0.533932,0.631941,0.787121}%
\pgfsetfillcolor{currentfill}%
\pgfsetfillopacity{0.700000}%
\pgfsetlinewidth{1.003750pt}%
\definecolor{currentstroke}{rgb}{0.533932,0.631941,0.787121}%
\pgfsetstrokecolor{currentstroke}%
\pgfsetstrokeopacity{0.700000}%
\pgfsetdash{}{0pt}%
\pgfpathmoveto{\pgfqpoint{2.133821in}{1.203087in}}%
\pgfpathcurveto{\pgfqpoint{2.135663in}{1.203087in}}{\pgfqpoint{2.137429in}{1.203819in}}{\pgfqpoint{2.138732in}{1.205121in}}%
\pgfpathcurveto{\pgfqpoint{2.140034in}{1.206423in}}{\pgfqpoint{2.140766in}{1.208190in}}{\pgfqpoint{2.140766in}{1.210031in}}%
\pgfpathcurveto{\pgfqpoint{2.140766in}{1.211873in}}{\pgfqpoint{2.140034in}{1.213639in}}{\pgfqpoint{2.138732in}{1.214942in}}%
\pgfpathcurveto{\pgfqpoint{2.137429in}{1.216244in}}{\pgfqpoint{2.135663in}{1.216976in}}{\pgfqpoint{2.133821in}{1.216976in}}%
\pgfpathcurveto{\pgfqpoint{2.131980in}{1.216976in}}{\pgfqpoint{2.130213in}{1.216244in}}{\pgfqpoint{2.128911in}{1.214942in}}%
\pgfpathcurveto{\pgfqpoint{2.127608in}{1.213639in}}{\pgfqpoint{2.126877in}{1.211873in}}{\pgfqpoint{2.126877in}{1.210031in}}%
\pgfpathcurveto{\pgfqpoint{2.126877in}{1.208190in}}{\pgfqpoint{2.127608in}{1.206423in}}{\pgfqpoint{2.128911in}{1.205121in}}%
\pgfpathcurveto{\pgfqpoint{2.130213in}{1.203819in}}{\pgfqpoint{2.131980in}{1.203087in}}{\pgfqpoint{2.133821in}{1.203087in}}%
\pgfpathlineto{\pgfqpoint{2.133821in}{1.203087in}}%
\pgfpathclose%
\pgfusepath{stroke,fill}%
\end{pgfscope}%
\begin{pgfscope}%
\pgfpathrectangle{\pgfqpoint{0.661006in}{0.524170in}}{\pgfqpoint{4.194036in}{1.071446in}}%
\pgfusepath{clip}%
\pgfsetbuttcap%
\pgfsetroundjoin%
\definecolor{currentfill}{rgb}{0.530575,0.627517,0.784352}%
\pgfsetfillcolor{currentfill}%
\pgfsetfillopacity{0.700000}%
\pgfsetlinewidth{1.003750pt}%
\definecolor{currentstroke}{rgb}{0.530575,0.627517,0.784352}%
\pgfsetstrokecolor{currentstroke}%
\pgfsetstrokeopacity{0.700000}%
\pgfsetdash{}{0pt}%
\pgfpathmoveto{\pgfqpoint{2.158677in}{1.198927in}}%
\pgfpathcurveto{\pgfqpoint{2.160518in}{1.198927in}}{\pgfqpoint{2.162285in}{1.199658in}}{\pgfqpoint{2.163587in}{1.200961in}}%
\pgfpathcurveto{\pgfqpoint{2.164889in}{1.202263in}}{\pgfqpoint{2.165621in}{1.204030in}}{\pgfqpoint{2.165621in}{1.205871in}}%
\pgfpathcurveto{\pgfqpoint{2.165621in}{1.207713in}}{\pgfqpoint{2.164889in}{1.209479in}}{\pgfqpoint{2.163587in}{1.210782in}}%
\pgfpathcurveto{\pgfqpoint{2.162285in}{1.212084in}}{\pgfqpoint{2.160518in}{1.212816in}}{\pgfqpoint{2.158677in}{1.212816in}}%
\pgfpathcurveto{\pgfqpoint{2.156835in}{1.212816in}}{\pgfqpoint{2.155068in}{1.212084in}}{\pgfqpoint{2.153766in}{1.210782in}}%
\pgfpathcurveto{\pgfqpoint{2.152464in}{1.209479in}}{\pgfqpoint{2.151732in}{1.207713in}}{\pgfqpoint{2.151732in}{1.205871in}}%
\pgfpathcurveto{\pgfqpoint{2.151732in}{1.204030in}}{\pgfqpoint{2.152464in}{1.202263in}}{\pgfqpoint{2.153766in}{1.200961in}}%
\pgfpathcurveto{\pgfqpoint{2.155068in}{1.199658in}}{\pgfqpoint{2.156835in}{1.198927in}}{\pgfqpoint{2.158677in}{1.198927in}}%
\pgfpathlineto{\pgfqpoint{2.158677in}{1.198927in}}%
\pgfpathclose%
\pgfusepath{stroke,fill}%
\end{pgfscope}%
\begin{pgfscope}%
\pgfpathrectangle{\pgfqpoint{0.661006in}{0.524170in}}{\pgfqpoint{4.194036in}{1.071446in}}%
\pgfusepath{clip}%
\pgfsetbuttcap%
\pgfsetroundjoin%
\definecolor{currentfill}{rgb}{0.530575,0.627517,0.784352}%
\pgfsetfillcolor{currentfill}%
\pgfsetfillopacity{0.700000}%
\pgfsetlinewidth{1.003750pt}%
\definecolor{currentstroke}{rgb}{0.530575,0.627517,0.784352}%
\pgfsetstrokecolor{currentstroke}%
\pgfsetstrokeopacity{0.700000}%
\pgfsetdash{}{0pt}%
\pgfpathmoveto{\pgfqpoint{2.164997in}{1.197692in}}%
\pgfpathcurveto{\pgfqpoint{2.166839in}{1.197692in}}{\pgfqpoint{2.168606in}{1.198424in}}{\pgfqpoint{2.169908in}{1.199726in}}%
\pgfpathcurveto{\pgfqpoint{2.171210in}{1.201029in}}{\pgfqpoint{2.171942in}{1.202795in}}{\pgfqpoint{2.171942in}{1.204637in}}%
\pgfpathcurveto{\pgfqpoint{2.171942in}{1.206479in}}{\pgfqpoint{2.171210in}{1.208245in}}{\pgfqpoint{2.169908in}{1.209547in}}%
\pgfpathcurveto{\pgfqpoint{2.168606in}{1.210850in}}{\pgfqpoint{2.166839in}{1.211581in}}{\pgfqpoint{2.164997in}{1.211581in}}%
\pgfpathcurveto{\pgfqpoint{2.163156in}{1.211581in}}{\pgfqpoint{2.161389in}{1.210850in}}{\pgfqpoint{2.160087in}{1.209547in}}%
\pgfpathcurveto{\pgfqpoint{2.158785in}{1.208245in}}{\pgfqpoint{2.158053in}{1.206479in}}{\pgfqpoint{2.158053in}{1.204637in}}%
\pgfpathcurveto{\pgfqpoint{2.158053in}{1.202795in}}{\pgfqpoint{2.158785in}{1.201029in}}{\pgfqpoint{2.160087in}{1.199726in}}%
\pgfpathcurveto{\pgfqpoint{2.161389in}{1.198424in}}{\pgfqpoint{2.163156in}{1.197692in}}{\pgfqpoint{2.164997in}{1.197692in}}%
\pgfpathlineto{\pgfqpoint{2.164997in}{1.197692in}}%
\pgfpathclose%
\pgfusepath{stroke,fill}%
\end{pgfscope}%
\begin{pgfscope}%
\pgfpathrectangle{\pgfqpoint{0.661006in}{0.524170in}}{\pgfqpoint{4.194036in}{1.071446in}}%
\pgfusepath{clip}%
\pgfsetbuttcap%
\pgfsetroundjoin%
\definecolor{currentfill}{rgb}{0.530575,0.627517,0.784352}%
\pgfsetfillcolor{currentfill}%
\pgfsetfillopacity{0.700000}%
\pgfsetlinewidth{1.003750pt}%
\definecolor{currentstroke}{rgb}{0.530575,0.627517,0.784352}%
\pgfsetstrokecolor{currentstroke}%
\pgfsetstrokeopacity{0.700000}%
\pgfsetdash{}{0pt}%
\pgfpathmoveto{\pgfqpoint{2.165323in}{1.198410in}}%
\pgfpathcurveto{\pgfqpoint{2.167164in}{1.198410in}}{\pgfqpoint{2.168931in}{1.199141in}}{\pgfqpoint{2.170233in}{1.200444in}}%
\pgfpathcurveto{\pgfqpoint{2.171536in}{1.201746in}}{\pgfqpoint{2.172267in}{1.203512in}}{\pgfqpoint{2.172267in}{1.205354in}}%
\pgfpathcurveto{\pgfqpoint{2.172267in}{1.207196in}}{\pgfqpoint{2.171536in}{1.208962in}}{\pgfqpoint{2.170233in}{1.210264in}}%
\pgfpathcurveto{\pgfqpoint{2.168931in}{1.211567in}}{\pgfqpoint{2.167164in}{1.212298in}}{\pgfqpoint{2.165323in}{1.212298in}}%
\pgfpathcurveto{\pgfqpoint{2.163481in}{1.212298in}}{\pgfqpoint{2.161715in}{1.211567in}}{\pgfqpoint{2.160412in}{1.210264in}}%
\pgfpathcurveto{\pgfqpoint{2.159110in}{1.208962in}}{\pgfqpoint{2.158378in}{1.207196in}}{\pgfqpoint{2.158378in}{1.205354in}}%
\pgfpathcurveto{\pgfqpoint{2.158378in}{1.203512in}}{\pgfqpoint{2.159110in}{1.201746in}}{\pgfqpoint{2.160412in}{1.200444in}}%
\pgfpathcurveto{\pgfqpoint{2.161715in}{1.199141in}}{\pgfqpoint{2.163481in}{1.198410in}}{\pgfqpoint{2.165323in}{1.198410in}}%
\pgfpathlineto{\pgfqpoint{2.165323in}{1.198410in}}%
\pgfpathclose%
\pgfusepath{stroke,fill}%
\end{pgfscope}%
\begin{pgfscope}%
\pgfpathrectangle{\pgfqpoint{0.661006in}{0.524170in}}{\pgfqpoint{4.194036in}{1.071446in}}%
\pgfusepath{clip}%
\pgfsetbuttcap%
\pgfsetroundjoin%
\definecolor{currentfill}{rgb}{0.530575,0.627517,0.784352}%
\pgfsetfillcolor{currentfill}%
\pgfsetfillopacity{0.700000}%
\pgfsetlinewidth{1.003750pt}%
\definecolor{currentstroke}{rgb}{0.530575,0.627517,0.784352}%
\pgfsetstrokecolor{currentstroke}%
\pgfsetstrokeopacity{0.700000}%
\pgfsetdash{}{0pt}%
\pgfpathmoveto{\pgfqpoint{2.155330in}{1.199448in}}%
\pgfpathcurveto{\pgfqpoint{2.157172in}{1.199448in}}{\pgfqpoint{2.158938in}{1.200180in}}{\pgfqpoint{2.160241in}{1.201482in}}%
\pgfpathcurveto{\pgfqpoint{2.161543in}{1.202784in}}{\pgfqpoint{2.162275in}{1.204551in}}{\pgfqpoint{2.162275in}{1.206392in}}%
\pgfpathcurveto{\pgfqpoint{2.162275in}{1.208234in}}{\pgfqpoint{2.161543in}{1.210000in}}{\pgfqpoint{2.160241in}{1.211303in}}%
\pgfpathcurveto{\pgfqpoint{2.158938in}{1.212605in}}{\pgfqpoint{2.157172in}{1.213337in}}{\pgfqpoint{2.155330in}{1.213337in}}%
\pgfpathcurveto{\pgfqpoint{2.153489in}{1.213337in}}{\pgfqpoint{2.151722in}{1.212605in}}{\pgfqpoint{2.150420in}{1.211303in}}%
\pgfpathcurveto{\pgfqpoint{2.149117in}{1.210000in}}{\pgfqpoint{2.148386in}{1.208234in}}{\pgfqpoint{2.148386in}{1.206392in}}%
\pgfpathcurveto{\pgfqpoint{2.148386in}{1.204551in}}{\pgfqpoint{2.149117in}{1.202784in}}{\pgfqpoint{2.150420in}{1.201482in}}%
\pgfpathcurveto{\pgfqpoint{2.151722in}{1.200180in}}{\pgfqpoint{2.153489in}{1.199448in}}{\pgfqpoint{2.155330in}{1.199448in}}%
\pgfpathlineto{\pgfqpoint{2.155330in}{1.199448in}}%
\pgfpathclose%
\pgfusepath{stroke,fill}%
\end{pgfscope}%
\begin{pgfscope}%
\pgfpathrectangle{\pgfqpoint{0.661006in}{0.524170in}}{\pgfqpoint{4.194036in}{1.071446in}}%
\pgfusepath{clip}%
\pgfsetbuttcap%
\pgfsetroundjoin%
\definecolor{currentfill}{rgb}{0.527239,0.623087,0.781558}%
\pgfsetfillcolor{currentfill}%
\pgfsetfillopacity{0.700000}%
\pgfsetlinewidth{1.003750pt}%
\definecolor{currentstroke}{rgb}{0.527239,0.623087,0.781558}%
\pgfsetstrokecolor{currentstroke}%
\pgfsetstrokeopacity{0.700000}%
\pgfsetdash{}{0pt}%
\pgfpathmoveto{\pgfqpoint{2.147661in}{1.202380in}}%
\pgfpathcurveto{\pgfqpoint{2.149503in}{1.202380in}}{\pgfqpoint{2.151270in}{1.203111in}}{\pgfqpoint{2.152572in}{1.204414in}}%
\pgfpathcurveto{\pgfqpoint{2.153874in}{1.205716in}}{\pgfqpoint{2.154606in}{1.207482in}}{\pgfqpoint{2.154606in}{1.209324in}}%
\pgfpathcurveto{\pgfqpoint{2.154606in}{1.211166in}}{\pgfqpoint{2.153874in}{1.212932in}}{\pgfqpoint{2.152572in}{1.214235in}}%
\pgfpathcurveto{\pgfqpoint{2.151270in}{1.215537in}}{\pgfqpoint{2.149503in}{1.216269in}}{\pgfqpoint{2.147661in}{1.216269in}}%
\pgfpathcurveto{\pgfqpoint{2.145820in}{1.216269in}}{\pgfqpoint{2.144053in}{1.215537in}}{\pgfqpoint{2.142751in}{1.214235in}}%
\pgfpathcurveto{\pgfqpoint{2.141449in}{1.212932in}}{\pgfqpoint{2.140717in}{1.211166in}}{\pgfqpoint{2.140717in}{1.209324in}}%
\pgfpathcurveto{\pgfqpoint{2.140717in}{1.207482in}}{\pgfqpoint{2.141449in}{1.205716in}}{\pgfqpoint{2.142751in}{1.204414in}}%
\pgfpathcurveto{\pgfqpoint{2.144053in}{1.203111in}}{\pgfqpoint{2.145820in}{1.202380in}}{\pgfqpoint{2.147661in}{1.202380in}}%
\pgfpathlineto{\pgfqpoint{2.147661in}{1.202380in}}%
\pgfpathclose%
\pgfusepath{stroke,fill}%
\end{pgfscope}%
\begin{pgfscope}%
\pgfpathrectangle{\pgfqpoint{0.661006in}{0.524170in}}{\pgfqpoint{4.194036in}{1.071446in}}%
\pgfusepath{clip}%
\pgfsetbuttcap%
\pgfsetroundjoin%
\definecolor{currentfill}{rgb}{0.527239,0.623087,0.781558}%
\pgfsetfillcolor{currentfill}%
\pgfsetfillopacity{0.700000}%
\pgfsetlinewidth{1.003750pt}%
\definecolor{currentstroke}{rgb}{0.527239,0.623087,0.781558}%
\pgfsetstrokecolor{currentstroke}%
\pgfsetstrokeopacity{0.700000}%
\pgfsetdash{}{0pt}%
\pgfpathmoveto{\pgfqpoint{2.116336in}{1.209683in}}%
\pgfpathcurveto{\pgfqpoint{2.118177in}{1.209683in}}{\pgfqpoint{2.119944in}{1.210415in}}{\pgfqpoint{2.121246in}{1.211717in}}%
\pgfpathcurveto{\pgfqpoint{2.122549in}{1.213019in}}{\pgfqpoint{2.123280in}{1.214786in}}{\pgfqpoint{2.123280in}{1.216628in}}%
\pgfpathcurveto{\pgfqpoint{2.123280in}{1.218469in}}{\pgfqpoint{2.122549in}{1.220236in}}{\pgfqpoint{2.121246in}{1.221538in}}%
\pgfpathcurveto{\pgfqpoint{2.119944in}{1.222840in}}{\pgfqpoint{2.118177in}{1.223572in}}{\pgfqpoint{2.116336in}{1.223572in}}%
\pgfpathcurveto{\pgfqpoint{2.114494in}{1.223572in}}{\pgfqpoint{2.112728in}{1.222840in}}{\pgfqpoint{2.111425in}{1.221538in}}%
\pgfpathcurveto{\pgfqpoint{2.110123in}{1.220236in}}{\pgfqpoint{2.109391in}{1.218469in}}{\pgfqpoint{2.109391in}{1.216628in}}%
\pgfpathcurveto{\pgfqpoint{2.109391in}{1.214786in}}{\pgfqpoint{2.110123in}{1.213019in}}{\pgfqpoint{2.111425in}{1.211717in}}%
\pgfpathcurveto{\pgfqpoint{2.112728in}{1.210415in}}{\pgfqpoint{2.114494in}{1.209683in}}{\pgfqpoint{2.116336in}{1.209683in}}%
\pgfpathlineto{\pgfqpoint{2.116336in}{1.209683in}}%
\pgfpathclose%
\pgfusepath{stroke,fill}%
\end{pgfscope}%
\begin{pgfscope}%
\pgfpathrectangle{\pgfqpoint{0.661006in}{0.524170in}}{\pgfqpoint{4.194036in}{1.071446in}}%
\pgfusepath{clip}%
\pgfsetbuttcap%
\pgfsetroundjoin%
\definecolor{currentfill}{rgb}{0.527239,0.623087,0.781558}%
\pgfsetfillcolor{currentfill}%
\pgfsetfillopacity{0.700000}%
\pgfsetlinewidth{1.003750pt}%
\definecolor{currentstroke}{rgb}{0.527239,0.623087,0.781558}%
\pgfsetstrokecolor{currentstroke}%
\pgfsetstrokeopacity{0.700000}%
\pgfsetdash{}{0pt}%
\pgfpathmoveto{\pgfqpoint{2.073902in}{1.220811in}}%
\pgfpathcurveto{\pgfqpoint{2.075744in}{1.220811in}}{\pgfqpoint{2.077510in}{1.221542in}}{\pgfqpoint{2.078813in}{1.222845in}}%
\pgfpathcurveto{\pgfqpoint{2.080115in}{1.224147in}}{\pgfqpoint{2.080847in}{1.225914in}}{\pgfqpoint{2.080847in}{1.227755in}}%
\pgfpathcurveto{\pgfqpoint{2.080847in}{1.229597in}}{\pgfqpoint{2.080115in}{1.231363in}}{\pgfqpoint{2.078813in}{1.232666in}}%
\pgfpathcurveto{\pgfqpoint{2.077510in}{1.233968in}}{\pgfqpoint{2.075744in}{1.234700in}}{\pgfqpoint{2.073902in}{1.234700in}}%
\pgfpathcurveto{\pgfqpoint{2.072060in}{1.234700in}}{\pgfqpoint{2.070294in}{1.233968in}}{\pgfqpoint{2.068992in}{1.232666in}}%
\pgfpathcurveto{\pgfqpoint{2.067689in}{1.231363in}}{\pgfqpoint{2.066958in}{1.229597in}}{\pgfqpoint{2.066958in}{1.227755in}}%
\pgfpathcurveto{\pgfqpoint{2.066958in}{1.225914in}}{\pgfqpoint{2.067689in}{1.224147in}}{\pgfqpoint{2.068992in}{1.222845in}}%
\pgfpathcurveto{\pgfqpoint{2.070294in}{1.221542in}}{\pgfqpoint{2.072060in}{1.220811in}}{\pgfqpoint{2.073902in}{1.220811in}}%
\pgfpathlineto{\pgfqpoint{2.073902in}{1.220811in}}%
\pgfpathclose%
\pgfusepath{stroke,fill}%
\end{pgfscope}%
\begin{pgfscope}%
\pgfpathrectangle{\pgfqpoint{0.661006in}{0.524170in}}{\pgfqpoint{4.194036in}{1.071446in}}%
\pgfusepath{clip}%
\pgfsetbuttcap%
\pgfsetroundjoin%
\definecolor{currentfill}{rgb}{0.523924,0.618652,0.778740}%
\pgfsetfillcolor{currentfill}%
\pgfsetfillopacity{0.700000}%
\pgfsetlinewidth{1.003750pt}%
\definecolor{currentstroke}{rgb}{0.523924,0.618652,0.778740}%
\pgfsetstrokecolor{currentstroke}%
\pgfsetstrokeopacity{0.700000}%
\pgfsetdash{}{0pt}%
\pgfpathmoveto{\pgfqpoint{2.023474in}{1.230577in}}%
\pgfpathcurveto{\pgfqpoint{2.025316in}{1.230577in}}{\pgfqpoint{2.027082in}{1.231309in}}{\pgfqpoint{2.028385in}{1.232611in}}%
\pgfpathcurveto{\pgfqpoint{2.029687in}{1.233913in}}{\pgfqpoint{2.030419in}{1.235680in}}{\pgfqpoint{2.030419in}{1.237521in}}%
\pgfpathcurveto{\pgfqpoint{2.030419in}{1.239363in}}{\pgfqpoint{2.029687in}{1.241129in}}{\pgfqpoint{2.028385in}{1.242432in}}%
\pgfpathcurveto{\pgfqpoint{2.027082in}{1.243734in}}{\pgfqpoint{2.025316in}{1.244466in}}{\pgfqpoint{2.023474in}{1.244466in}}%
\pgfpathcurveto{\pgfqpoint{2.021633in}{1.244466in}}{\pgfqpoint{2.019866in}{1.243734in}}{\pgfqpoint{2.018564in}{1.242432in}}%
\pgfpathcurveto{\pgfqpoint{2.017262in}{1.241129in}}{\pgfqpoint{2.016530in}{1.239363in}}{\pgfqpoint{2.016530in}{1.237521in}}%
\pgfpathcurveto{\pgfqpoint{2.016530in}{1.235680in}}{\pgfqpoint{2.017262in}{1.233913in}}{\pgfqpoint{2.018564in}{1.232611in}}%
\pgfpathcurveto{\pgfqpoint{2.019866in}{1.231309in}}{\pgfqpoint{2.021633in}{1.230577in}}{\pgfqpoint{2.023474in}{1.230577in}}%
\pgfpathlineto{\pgfqpoint{2.023474in}{1.230577in}}%
\pgfpathclose%
\pgfusepath{stroke,fill}%
\end{pgfscope}%
\begin{pgfscope}%
\pgfpathrectangle{\pgfqpoint{0.661006in}{0.524170in}}{\pgfqpoint{4.194036in}{1.071446in}}%
\pgfusepath{clip}%
\pgfsetbuttcap%
\pgfsetroundjoin%
\definecolor{currentfill}{rgb}{0.523924,0.618652,0.778740}%
\pgfsetfillcolor{currentfill}%
\pgfsetfillopacity{0.700000}%
\pgfsetlinewidth{1.003750pt}%
\definecolor{currentstroke}{rgb}{0.523924,0.618652,0.778740}%
\pgfsetstrokecolor{currentstroke}%
\pgfsetstrokeopacity{0.700000}%
\pgfsetdash{}{0pt}%
\pgfpathmoveto{\pgfqpoint{2.006231in}{1.236374in}}%
\pgfpathcurveto{\pgfqpoint{2.008073in}{1.236374in}}{\pgfqpoint{2.009839in}{1.237106in}}{\pgfqpoint{2.011142in}{1.238408in}}%
\pgfpathcurveto{\pgfqpoint{2.012444in}{1.239711in}}{\pgfqpoint{2.013176in}{1.241477in}}{\pgfqpoint{2.013176in}{1.243319in}}%
\pgfpathcurveto{\pgfqpoint{2.013176in}{1.245161in}}{\pgfqpoint{2.012444in}{1.246927in}}{\pgfqpoint{2.011142in}{1.248229in}}%
\pgfpathcurveto{\pgfqpoint{2.009839in}{1.249532in}}{\pgfqpoint{2.008073in}{1.250263in}}{\pgfqpoint{2.006231in}{1.250263in}}%
\pgfpathcurveto{\pgfqpoint{2.004390in}{1.250263in}}{\pgfqpoint{2.002623in}{1.249532in}}{\pgfqpoint{2.001321in}{1.248229in}}%
\pgfpathcurveto{\pgfqpoint{2.000018in}{1.246927in}}{\pgfqpoint{1.999287in}{1.245161in}}{\pgfqpoint{1.999287in}{1.243319in}}%
\pgfpathcurveto{\pgfqpoint{1.999287in}{1.241477in}}{\pgfqpoint{2.000018in}{1.239711in}}{\pgfqpoint{2.001321in}{1.238408in}}%
\pgfpathcurveto{\pgfqpoint{2.002623in}{1.237106in}}{\pgfqpoint{2.004390in}{1.236374in}}{\pgfqpoint{2.006231in}{1.236374in}}%
\pgfpathlineto{\pgfqpoint{2.006231in}{1.236374in}}%
\pgfpathclose%
\pgfusepath{stroke,fill}%
\end{pgfscope}%
\begin{pgfscope}%
\pgfpathrectangle{\pgfqpoint{0.661006in}{0.524170in}}{\pgfqpoint{4.194036in}{1.071446in}}%
\pgfusepath{clip}%
\pgfsetbuttcap%
\pgfsetroundjoin%
\definecolor{currentfill}{rgb}{0.523924,0.618652,0.778740}%
\pgfsetfillcolor{currentfill}%
\pgfsetfillopacity{0.700000}%
\pgfsetlinewidth{1.003750pt}%
\definecolor{currentstroke}{rgb}{0.523924,0.618652,0.778740}%
\pgfsetstrokecolor{currentstroke}%
\pgfsetstrokeopacity{0.700000}%
\pgfsetdash{}{0pt}%
\pgfpathmoveto{\pgfqpoint{1.990940in}{1.239461in}}%
\pgfpathcurveto{\pgfqpoint{1.992782in}{1.239461in}}{\pgfqpoint{1.994548in}{1.240192in}}{\pgfqpoint{1.995851in}{1.241495in}}%
\pgfpathcurveto{\pgfqpoint{1.997153in}{1.242797in}}{\pgfqpoint{1.997885in}{1.244563in}}{\pgfqpoint{1.997885in}{1.246405in}}%
\pgfpathcurveto{\pgfqpoint{1.997885in}{1.248247in}}{\pgfqpoint{1.997153in}{1.250013in}}{\pgfqpoint{1.995851in}{1.251316in}}%
\pgfpathcurveto{\pgfqpoint{1.994548in}{1.252618in}}{\pgfqpoint{1.992782in}{1.253350in}}{\pgfqpoint{1.990940in}{1.253350in}}%
\pgfpathcurveto{\pgfqpoint{1.989099in}{1.253350in}}{\pgfqpoint{1.987332in}{1.252618in}}{\pgfqpoint{1.986030in}{1.251316in}}%
\pgfpathcurveto{\pgfqpoint{1.984727in}{1.250013in}}{\pgfqpoint{1.983996in}{1.248247in}}{\pgfqpoint{1.983996in}{1.246405in}}%
\pgfpathcurveto{\pgfqpoint{1.983996in}{1.244563in}}{\pgfqpoint{1.984727in}{1.242797in}}{\pgfqpoint{1.986030in}{1.241495in}}%
\pgfpathcurveto{\pgfqpoint{1.987332in}{1.240192in}}{\pgfqpoint{1.989099in}{1.239461in}}{\pgfqpoint{1.990940in}{1.239461in}}%
\pgfpathlineto{\pgfqpoint{1.990940in}{1.239461in}}%
\pgfpathclose%
\pgfusepath{stroke,fill}%
\end{pgfscope}%
\begin{pgfscope}%
\pgfpathrectangle{\pgfqpoint{0.661006in}{0.524170in}}{\pgfqpoint{4.194036in}{1.071446in}}%
\pgfusepath{clip}%
\pgfsetbuttcap%
\pgfsetroundjoin%
\definecolor{currentfill}{rgb}{0.523924,0.618652,0.778740}%
\pgfsetfillcolor{currentfill}%
\pgfsetfillopacity{0.700000}%
\pgfsetlinewidth{1.003750pt}%
\definecolor{currentstroke}{rgb}{0.523924,0.618652,0.778740}%
\pgfsetstrokecolor{currentstroke}%
\pgfsetstrokeopacity{0.700000}%
\pgfsetdash{}{0pt}%
\pgfpathmoveto{\pgfqpoint{1.971420in}{1.243621in}}%
\pgfpathcurveto{\pgfqpoint{1.973261in}{1.243621in}}{\pgfqpoint{1.975028in}{1.244352in}}{\pgfqpoint{1.976330in}{1.245655in}}%
\pgfpathcurveto{\pgfqpoint{1.977633in}{1.246957in}}{\pgfqpoint{1.978364in}{1.248723in}}{\pgfqpoint{1.978364in}{1.250565in}}%
\pgfpathcurveto{\pgfqpoint{1.978364in}{1.252407in}}{\pgfqpoint{1.977633in}{1.254173in}}{\pgfqpoint{1.976330in}{1.255476in}}%
\pgfpathcurveto{\pgfqpoint{1.975028in}{1.256778in}}{\pgfqpoint{1.973261in}{1.257510in}}{\pgfqpoint{1.971420in}{1.257510in}}%
\pgfpathcurveto{\pgfqpoint{1.969578in}{1.257510in}}{\pgfqpoint{1.967812in}{1.256778in}}{\pgfqpoint{1.966509in}{1.255476in}}%
\pgfpathcurveto{\pgfqpoint{1.965207in}{1.254173in}}{\pgfqpoint{1.964475in}{1.252407in}}{\pgfqpoint{1.964475in}{1.250565in}}%
\pgfpathcurveto{\pgfqpoint{1.964475in}{1.248723in}}{\pgfqpoint{1.965207in}{1.246957in}}{\pgfqpoint{1.966509in}{1.245655in}}%
\pgfpathcurveto{\pgfqpoint{1.967812in}{1.244352in}}{\pgfqpoint{1.969578in}{1.243621in}}{\pgfqpoint{1.971420in}{1.243621in}}%
\pgfpathlineto{\pgfqpoint{1.971420in}{1.243621in}}%
\pgfpathclose%
\pgfusepath{stroke,fill}%
\end{pgfscope}%
\begin{pgfscope}%
\pgfpathrectangle{\pgfqpoint{0.661006in}{0.524170in}}{\pgfqpoint{4.194036in}{1.071446in}}%
\pgfusepath{clip}%
\pgfsetbuttcap%
\pgfsetroundjoin%
\definecolor{currentfill}{rgb}{0.520630,0.614210,0.775896}%
\pgfsetfillcolor{currentfill}%
\pgfsetfillopacity{0.700000}%
\pgfsetlinewidth{1.003750pt}%
\definecolor{currentstroke}{rgb}{0.520630,0.614210,0.775896}%
\pgfsetstrokecolor{currentstroke}%
\pgfsetstrokeopacity{0.700000}%
\pgfsetdash{}{0pt}%
\pgfpathmoveto{\pgfqpoint{1.952829in}{1.248563in}}%
\pgfpathcurveto{\pgfqpoint{1.954671in}{1.248563in}}{\pgfqpoint{1.956437in}{1.249295in}}{\pgfqpoint{1.957739in}{1.250597in}}%
\pgfpathcurveto{\pgfqpoint{1.959042in}{1.251899in}}{\pgfqpoint{1.959773in}{1.253666in}}{\pgfqpoint{1.959773in}{1.255508in}}%
\pgfpathcurveto{\pgfqpoint{1.959773in}{1.257349in}}{\pgfqpoint{1.959042in}{1.259116in}}{\pgfqpoint{1.957739in}{1.260418in}}%
\pgfpathcurveto{\pgfqpoint{1.956437in}{1.261720in}}{\pgfqpoint{1.954671in}{1.262452in}}{\pgfqpoint{1.952829in}{1.262452in}}%
\pgfpathcurveto{\pgfqpoint{1.950987in}{1.262452in}}{\pgfqpoint{1.949221in}{1.261720in}}{\pgfqpoint{1.947918in}{1.260418in}}%
\pgfpathcurveto{\pgfqpoint{1.946616in}{1.259116in}}{\pgfqpoint{1.945884in}{1.257349in}}{\pgfqpoint{1.945884in}{1.255508in}}%
\pgfpathcurveto{\pgfqpoint{1.945884in}{1.253666in}}{\pgfqpoint{1.946616in}{1.251899in}}{\pgfqpoint{1.947918in}{1.250597in}}%
\pgfpathcurveto{\pgfqpoint{1.949221in}{1.249295in}}{\pgfqpoint{1.950987in}{1.248563in}}{\pgfqpoint{1.952829in}{1.248563in}}%
\pgfpathlineto{\pgfqpoint{1.952829in}{1.248563in}}%
\pgfpathclose%
\pgfusepath{stroke,fill}%
\end{pgfscope}%
\begin{pgfscope}%
\pgfpathrectangle{\pgfqpoint{0.661006in}{0.524170in}}{\pgfqpoint{4.194036in}{1.071446in}}%
\pgfusepath{clip}%
\pgfsetbuttcap%
\pgfsetroundjoin%
\definecolor{currentfill}{rgb}{0.520630,0.614210,0.775896}%
\pgfsetfillcolor{currentfill}%
\pgfsetfillopacity{0.700000}%
\pgfsetlinewidth{1.003750pt}%
\definecolor{currentstroke}{rgb}{0.520630,0.614210,0.775896}%
\pgfsetstrokecolor{currentstroke}%
\pgfsetstrokeopacity{0.700000}%
\pgfsetdash{}{0pt}%
\pgfpathmoveto{\pgfqpoint{1.944231in}{1.247692in}}%
\pgfpathcurveto{\pgfqpoint{1.946072in}{1.247692in}}{\pgfqpoint{1.947839in}{1.248424in}}{\pgfqpoint{1.949141in}{1.249726in}}%
\pgfpathcurveto{\pgfqpoint{1.950443in}{1.251028in}}{\pgfqpoint{1.951175in}{1.252795in}}{\pgfqpoint{1.951175in}{1.254636in}}%
\pgfpathcurveto{\pgfqpoint{1.951175in}{1.256478in}}{\pgfqpoint{1.950443in}{1.258245in}}{\pgfqpoint{1.949141in}{1.259547in}}%
\pgfpathcurveto{\pgfqpoint{1.947839in}{1.260849in}}{\pgfqpoint{1.946072in}{1.261581in}}{\pgfqpoint{1.944231in}{1.261581in}}%
\pgfpathcurveto{\pgfqpoint{1.942389in}{1.261581in}}{\pgfqpoint{1.940622in}{1.260849in}}{\pgfqpoint{1.939320in}{1.259547in}}%
\pgfpathcurveto{\pgfqpoint{1.938018in}{1.258245in}}{\pgfqpoint{1.937286in}{1.256478in}}{\pgfqpoint{1.937286in}{1.254636in}}%
\pgfpathcurveto{\pgfqpoint{1.937286in}{1.252795in}}{\pgfqpoint{1.938018in}{1.251028in}}{\pgfqpoint{1.939320in}{1.249726in}}%
\pgfpathcurveto{\pgfqpoint{1.940622in}{1.248424in}}{\pgfqpoint{1.942389in}{1.247692in}}{\pgfqpoint{1.944231in}{1.247692in}}%
\pgfpathlineto{\pgfqpoint{1.944231in}{1.247692in}}%
\pgfpathclose%
\pgfusepath{stroke,fill}%
\end{pgfscope}%
\begin{pgfscope}%
\pgfpathrectangle{\pgfqpoint{0.661006in}{0.524170in}}{\pgfqpoint{4.194036in}{1.071446in}}%
\pgfusepath{clip}%
\pgfsetbuttcap%
\pgfsetroundjoin%
\definecolor{currentfill}{rgb}{0.520630,0.614210,0.775896}%
\pgfsetfillcolor{currentfill}%
\pgfsetfillopacity{0.700000}%
\pgfsetlinewidth{1.003750pt}%
\definecolor{currentstroke}{rgb}{0.520630,0.614210,0.775896}%
\pgfsetstrokecolor{currentstroke}%
\pgfsetstrokeopacity{0.700000}%
\pgfsetdash{}{0pt}%
\pgfpathmoveto{\pgfqpoint{1.987129in}{1.239179in}}%
\pgfpathcurveto{\pgfqpoint{1.988971in}{1.239179in}}{\pgfqpoint{1.990737in}{1.239911in}}{\pgfqpoint{1.992040in}{1.241213in}}%
\pgfpathcurveto{\pgfqpoint{1.993342in}{1.242516in}}{\pgfqpoint{1.994074in}{1.244282in}}{\pgfqpoint{1.994074in}{1.246124in}}%
\pgfpathcurveto{\pgfqpoint{1.994074in}{1.247965in}}{\pgfqpoint{1.993342in}{1.249732in}}{\pgfqpoint{1.992040in}{1.251034in}}%
\pgfpathcurveto{\pgfqpoint{1.990737in}{1.252337in}}{\pgfqpoint{1.988971in}{1.253068in}}{\pgfqpoint{1.987129in}{1.253068in}}%
\pgfpathcurveto{\pgfqpoint{1.985287in}{1.253068in}}{\pgfqpoint{1.983521in}{1.252337in}}{\pgfqpoint{1.982219in}{1.251034in}}%
\pgfpathcurveto{\pgfqpoint{1.980916in}{1.249732in}}{\pgfqpoint{1.980185in}{1.247965in}}{\pgfqpoint{1.980185in}{1.246124in}}%
\pgfpathcurveto{\pgfqpoint{1.980185in}{1.244282in}}{\pgfqpoint{1.980916in}{1.242516in}}{\pgfqpoint{1.982219in}{1.241213in}}%
\pgfpathcurveto{\pgfqpoint{1.983521in}{1.239911in}}{\pgfqpoint{1.985287in}{1.239179in}}{\pgfqpoint{1.987129in}{1.239179in}}%
\pgfpathlineto{\pgfqpoint{1.987129in}{1.239179in}}%
\pgfpathclose%
\pgfusepath{stroke,fill}%
\end{pgfscope}%
\begin{pgfscope}%
\pgfpathrectangle{\pgfqpoint{0.661006in}{0.524170in}}{\pgfqpoint{4.194036in}{1.071446in}}%
\pgfusepath{clip}%
\pgfsetbuttcap%
\pgfsetroundjoin%
\definecolor{currentfill}{rgb}{0.517356,0.609763,0.773028}%
\pgfsetfillcolor{currentfill}%
\pgfsetfillopacity{0.700000}%
\pgfsetlinewidth{1.003750pt}%
\definecolor{currentstroke}{rgb}{0.517356,0.609763,0.773028}%
\pgfsetstrokecolor{currentstroke}%
\pgfsetstrokeopacity{0.700000}%
\pgfsetdash{}{0pt}%
\pgfpathmoveto{\pgfqpoint{2.010182in}{1.232288in}}%
\pgfpathcurveto{\pgfqpoint{2.012023in}{1.232288in}}{\pgfqpoint{2.013790in}{1.233020in}}{\pgfqpoint{2.015092in}{1.234322in}}%
\pgfpathcurveto{\pgfqpoint{2.016395in}{1.235624in}}{\pgfqpoint{2.017126in}{1.237391in}}{\pgfqpoint{2.017126in}{1.239232in}}%
\pgfpathcurveto{\pgfqpoint{2.017126in}{1.241074in}}{\pgfqpoint{2.016395in}{1.242841in}}{\pgfqpoint{2.015092in}{1.244143in}}%
\pgfpathcurveto{\pgfqpoint{2.013790in}{1.245445in}}{\pgfqpoint{2.012023in}{1.246177in}}{\pgfqpoint{2.010182in}{1.246177in}}%
\pgfpathcurveto{\pgfqpoint{2.008340in}{1.246177in}}{\pgfqpoint{2.006574in}{1.245445in}}{\pgfqpoint{2.005271in}{1.244143in}}%
\pgfpathcurveto{\pgfqpoint{2.003969in}{1.242841in}}{\pgfqpoint{2.003237in}{1.241074in}}{\pgfqpoint{2.003237in}{1.239232in}}%
\pgfpathcurveto{\pgfqpoint{2.003237in}{1.237391in}}{\pgfqpoint{2.003969in}{1.235624in}}{\pgfqpoint{2.005271in}{1.234322in}}%
\pgfpathcurveto{\pgfqpoint{2.006574in}{1.233020in}}{\pgfqpoint{2.008340in}{1.232288in}}{\pgfqpoint{2.010182in}{1.232288in}}%
\pgfpathlineto{\pgfqpoint{2.010182in}{1.232288in}}%
\pgfpathclose%
\pgfusepath{stroke,fill}%
\end{pgfscope}%
\begin{pgfscope}%
\pgfpathrectangle{\pgfqpoint{0.661006in}{0.524170in}}{\pgfqpoint{4.194036in}{1.071446in}}%
\pgfusepath{clip}%
\pgfsetbuttcap%
\pgfsetroundjoin%
\definecolor{currentfill}{rgb}{0.517356,0.609763,0.773028}%
\pgfsetfillcolor{currentfill}%
\pgfsetfillopacity{0.700000}%
\pgfsetlinewidth{1.003750pt}%
\definecolor{currentstroke}{rgb}{0.517356,0.609763,0.773028}%
\pgfsetstrokecolor{currentstroke}%
\pgfsetstrokeopacity{0.700000}%
\pgfsetdash{}{0pt}%
\pgfpathmoveto{\pgfqpoint{2.013156in}{1.230405in}}%
\pgfpathcurveto{\pgfqpoint{2.014998in}{1.230405in}}{\pgfqpoint{2.016765in}{1.231136in}}{\pgfqpoint{2.018067in}{1.232439in}}%
\pgfpathcurveto{\pgfqpoint{2.019369in}{1.233741in}}{\pgfqpoint{2.020101in}{1.235507in}}{\pgfqpoint{2.020101in}{1.237349in}}%
\pgfpathcurveto{\pgfqpoint{2.020101in}{1.239191in}}{\pgfqpoint{2.019369in}{1.240957in}}{\pgfqpoint{2.018067in}{1.242260in}}%
\pgfpathcurveto{\pgfqpoint{2.016765in}{1.243562in}}{\pgfqpoint{2.014998in}{1.244293in}}{\pgfqpoint{2.013156in}{1.244293in}}%
\pgfpathcurveto{\pgfqpoint{2.011315in}{1.244293in}}{\pgfqpoint{2.009548in}{1.243562in}}{\pgfqpoint{2.008246in}{1.242260in}}%
\pgfpathcurveto{\pgfqpoint{2.006944in}{1.240957in}}{\pgfqpoint{2.006212in}{1.239191in}}{\pgfqpoint{2.006212in}{1.237349in}}%
\pgfpathcurveto{\pgfqpoint{2.006212in}{1.235507in}}{\pgfqpoint{2.006944in}{1.233741in}}{\pgfqpoint{2.008246in}{1.232439in}}%
\pgfpathcurveto{\pgfqpoint{2.009548in}{1.231136in}}{\pgfqpoint{2.011315in}{1.230405in}}{\pgfqpoint{2.013156in}{1.230405in}}%
\pgfpathlineto{\pgfqpoint{2.013156in}{1.230405in}}%
\pgfpathclose%
\pgfusepath{stroke,fill}%
\end{pgfscope}%
\begin{pgfscope}%
\pgfpathrectangle{\pgfqpoint{0.661006in}{0.524170in}}{\pgfqpoint{4.194036in}{1.071446in}}%
\pgfusepath{clip}%
\pgfsetbuttcap%
\pgfsetroundjoin%
\definecolor{currentfill}{rgb}{0.514101,0.605311,0.770133}%
\pgfsetfillcolor{currentfill}%
\pgfsetfillopacity{0.700000}%
\pgfsetlinewidth{1.003750pt}%
\definecolor{currentstroke}{rgb}{0.514101,0.605311,0.770133}%
\pgfsetstrokecolor{currentstroke}%
\pgfsetstrokeopacity{0.700000}%
\pgfsetdash{}{0pt}%
\pgfpathmoveto{\pgfqpoint{2.025287in}{1.229428in}}%
\pgfpathcurveto{\pgfqpoint{2.027129in}{1.229428in}}{\pgfqpoint{2.028895in}{1.230160in}}{\pgfqpoint{2.030197in}{1.231462in}}%
\pgfpathcurveto{\pgfqpoint{2.031500in}{1.232764in}}{\pgfqpoint{2.032231in}{1.234531in}}{\pgfqpoint{2.032231in}{1.236373in}}%
\pgfpathcurveto{\pgfqpoint{2.032231in}{1.238214in}}{\pgfqpoint{2.031500in}{1.239981in}}{\pgfqpoint{2.030197in}{1.241283in}}%
\pgfpathcurveto{\pgfqpoint{2.028895in}{1.242585in}}{\pgfqpoint{2.027129in}{1.243317in}}{\pgfqpoint{2.025287in}{1.243317in}}%
\pgfpathcurveto{\pgfqpoint{2.023445in}{1.243317in}}{\pgfqpoint{2.021679in}{1.242585in}}{\pgfqpoint{2.020376in}{1.241283in}}%
\pgfpathcurveto{\pgfqpoint{2.019074in}{1.239981in}}{\pgfqpoint{2.018342in}{1.238214in}}{\pgfqpoint{2.018342in}{1.236373in}}%
\pgfpathcurveto{\pgfqpoint{2.018342in}{1.234531in}}{\pgfqpoint{2.019074in}{1.232764in}}{\pgfqpoint{2.020376in}{1.231462in}}%
\pgfpathcurveto{\pgfqpoint{2.021679in}{1.230160in}}{\pgfqpoint{2.023445in}{1.229428in}}{\pgfqpoint{2.025287in}{1.229428in}}%
\pgfpathlineto{\pgfqpoint{2.025287in}{1.229428in}}%
\pgfpathclose%
\pgfusepath{stroke,fill}%
\end{pgfscope}%
\begin{pgfscope}%
\pgfpathrectangle{\pgfqpoint{0.661006in}{0.524170in}}{\pgfqpoint{4.194036in}{1.071446in}}%
\pgfusepath{clip}%
\pgfsetbuttcap%
\pgfsetroundjoin%
\definecolor{currentfill}{rgb}{0.514101,0.605311,0.770133}%
\pgfsetfillcolor{currentfill}%
\pgfsetfillopacity{0.700000}%
\pgfsetlinewidth{1.003750pt}%
\definecolor{currentstroke}{rgb}{0.514101,0.605311,0.770133}%
\pgfsetstrokecolor{currentstroke}%
\pgfsetstrokeopacity{0.700000}%
\pgfsetdash{}{0pt}%
\pgfpathmoveto{\pgfqpoint{2.044807in}{1.225237in}}%
\pgfpathcurveto{\pgfqpoint{2.046649in}{1.225237in}}{\pgfqpoint{2.048416in}{1.225969in}}{\pgfqpoint{2.049718in}{1.227271in}}%
\pgfpathcurveto{\pgfqpoint{2.051020in}{1.228573in}}{\pgfqpoint{2.051752in}{1.230340in}}{\pgfqpoint{2.051752in}{1.232181in}}%
\pgfpathcurveto{\pgfqpoint{2.051752in}{1.234023in}}{\pgfqpoint{2.051020in}{1.235790in}}{\pgfqpoint{2.049718in}{1.237092in}}%
\pgfpathcurveto{\pgfqpoint{2.048416in}{1.238394in}}{\pgfqpoint{2.046649in}{1.239126in}}{\pgfqpoint{2.044807in}{1.239126in}}%
\pgfpathcurveto{\pgfqpoint{2.042966in}{1.239126in}}{\pgfqpoint{2.041199in}{1.238394in}}{\pgfqpoint{2.039897in}{1.237092in}}%
\pgfpathcurveto{\pgfqpoint{2.038595in}{1.235790in}}{\pgfqpoint{2.037863in}{1.234023in}}{\pgfqpoint{2.037863in}{1.232181in}}%
\pgfpathcurveto{\pgfqpoint{2.037863in}{1.230340in}}{\pgfqpoint{2.038595in}{1.228573in}}{\pgfqpoint{2.039897in}{1.227271in}}%
\pgfpathcurveto{\pgfqpoint{2.041199in}{1.225969in}}{\pgfqpoint{2.042966in}{1.225237in}}{\pgfqpoint{2.044807in}{1.225237in}}%
\pgfpathlineto{\pgfqpoint{2.044807in}{1.225237in}}%
\pgfpathclose%
\pgfusepath{stroke,fill}%
\end{pgfscope}%
\begin{pgfscope}%
\pgfpathrectangle{\pgfqpoint{0.661006in}{0.524170in}}{\pgfqpoint{4.194036in}{1.071446in}}%
\pgfusepath{clip}%
\pgfsetbuttcap%
\pgfsetroundjoin%
\definecolor{currentfill}{rgb}{0.514101,0.605311,0.770133}%
\pgfsetfillcolor{currentfill}%
\pgfsetfillopacity{0.700000}%
\pgfsetlinewidth{1.003750pt}%
\definecolor{currentstroke}{rgb}{0.514101,0.605311,0.770133}%
\pgfsetstrokecolor{currentstroke}%
\pgfsetstrokeopacity{0.700000}%
\pgfsetdash{}{0pt}%
\pgfpathmoveto{\pgfqpoint{2.084359in}{1.216557in}}%
\pgfpathcurveto{\pgfqpoint{2.086201in}{1.216557in}}{\pgfqpoint{2.087968in}{1.217289in}}{\pgfqpoint{2.089270in}{1.218591in}}%
\pgfpathcurveto{\pgfqpoint{2.090572in}{1.219894in}}{\pgfqpoint{2.091304in}{1.221660in}}{\pgfqpoint{2.091304in}{1.223502in}}%
\pgfpathcurveto{\pgfqpoint{2.091304in}{1.225343in}}{\pgfqpoint{2.090572in}{1.227110in}}{\pgfqpoint{2.089270in}{1.228412in}}%
\pgfpathcurveto{\pgfqpoint{2.087968in}{1.229715in}}{\pgfqpoint{2.086201in}{1.230446in}}{\pgfqpoint{2.084359in}{1.230446in}}%
\pgfpathcurveto{\pgfqpoint{2.082518in}{1.230446in}}{\pgfqpoint{2.080751in}{1.229715in}}{\pgfqpoint{2.079449in}{1.228412in}}%
\pgfpathcurveto{\pgfqpoint{2.078147in}{1.227110in}}{\pgfqpoint{2.077415in}{1.225343in}}{\pgfqpoint{2.077415in}{1.223502in}}%
\pgfpathcurveto{\pgfqpoint{2.077415in}{1.221660in}}{\pgfqpoint{2.078147in}{1.219894in}}{\pgfqpoint{2.079449in}{1.218591in}}%
\pgfpathcurveto{\pgfqpoint{2.080751in}{1.217289in}}{\pgfqpoint{2.082518in}{1.216557in}}{\pgfqpoint{2.084359in}{1.216557in}}%
\pgfpathlineto{\pgfqpoint{2.084359in}{1.216557in}}%
\pgfpathclose%
\pgfusepath{stroke,fill}%
\end{pgfscope}%
\begin{pgfscope}%
\pgfpathrectangle{\pgfqpoint{0.661006in}{0.524170in}}{\pgfqpoint{4.194036in}{1.071446in}}%
\pgfusepath{clip}%
\pgfsetbuttcap%
\pgfsetroundjoin%
\definecolor{currentfill}{rgb}{0.514101,0.605311,0.770133}%
\pgfsetfillcolor{currentfill}%
\pgfsetfillopacity{0.700000}%
\pgfsetlinewidth{1.003750pt}%
\definecolor{currentstroke}{rgb}{0.514101,0.605311,0.770133}%
\pgfsetstrokecolor{currentstroke}%
\pgfsetstrokeopacity{0.700000}%
\pgfsetdash{}{0pt}%
\pgfpathmoveto{\pgfqpoint{2.122796in}{1.206472in}}%
\pgfpathcurveto{\pgfqpoint{2.124638in}{1.206472in}}{\pgfqpoint{2.126404in}{1.207203in}}{\pgfqpoint{2.127707in}{1.208506in}}%
\pgfpathcurveto{\pgfqpoint{2.129009in}{1.209808in}}{\pgfqpoint{2.129741in}{1.211574in}}{\pgfqpoint{2.129741in}{1.213416in}}%
\pgfpathcurveto{\pgfqpoint{2.129741in}{1.215258in}}{\pgfqpoint{2.129009in}{1.217024in}}{\pgfqpoint{2.127707in}{1.218326in}}%
\pgfpathcurveto{\pgfqpoint{2.126404in}{1.219629in}}{\pgfqpoint{2.124638in}{1.220360in}}{\pgfqpoint{2.122796in}{1.220360in}}%
\pgfpathcurveto{\pgfqpoint{2.120954in}{1.220360in}}{\pgfqpoint{2.119188in}{1.219629in}}{\pgfqpoint{2.117886in}{1.218326in}}%
\pgfpathcurveto{\pgfqpoint{2.116583in}{1.217024in}}{\pgfqpoint{2.115852in}{1.215258in}}{\pgfqpoint{2.115852in}{1.213416in}}%
\pgfpathcurveto{\pgfqpoint{2.115852in}{1.211574in}}{\pgfqpoint{2.116583in}{1.209808in}}{\pgfqpoint{2.117886in}{1.208506in}}%
\pgfpathcurveto{\pgfqpoint{2.119188in}{1.207203in}}{\pgfqpoint{2.120954in}{1.206472in}}{\pgfqpoint{2.122796in}{1.206472in}}%
\pgfpathlineto{\pgfqpoint{2.122796in}{1.206472in}}%
\pgfpathclose%
\pgfusepath{stroke,fill}%
\end{pgfscope}%
\begin{pgfscope}%
\pgfpathrectangle{\pgfqpoint{0.661006in}{0.524170in}}{\pgfqpoint{4.194036in}{1.071446in}}%
\pgfusepath{clip}%
\pgfsetbuttcap%
\pgfsetroundjoin%
\definecolor{currentfill}{rgb}{0.514101,0.605311,0.770133}%
\pgfsetfillcolor{currentfill}%
\pgfsetfillopacity{0.700000}%
\pgfsetlinewidth{1.003750pt}%
\definecolor{currentstroke}{rgb}{0.514101,0.605311,0.770133}%
\pgfsetstrokecolor{currentstroke}%
\pgfsetstrokeopacity{0.700000}%
\pgfsetdash{}{0pt}%
\pgfpathmoveto{\pgfqpoint{2.156399in}{1.198710in}}%
\pgfpathcurveto{\pgfqpoint{2.158241in}{1.198710in}}{\pgfqpoint{2.160007in}{1.199442in}}{\pgfqpoint{2.161310in}{1.200744in}}%
\pgfpathcurveto{\pgfqpoint{2.162612in}{1.202046in}}{\pgfqpoint{2.163344in}{1.203813in}}{\pgfqpoint{2.163344in}{1.205654in}}%
\pgfpathcurveto{\pgfqpoint{2.163344in}{1.207496in}}{\pgfqpoint{2.162612in}{1.209262in}}{\pgfqpoint{2.161310in}{1.210565in}}%
\pgfpathcurveto{\pgfqpoint{2.160007in}{1.211867in}}{\pgfqpoint{2.158241in}{1.212599in}}{\pgfqpoint{2.156399in}{1.212599in}}%
\pgfpathcurveto{\pgfqpoint{2.154557in}{1.212599in}}{\pgfqpoint{2.152791in}{1.211867in}}{\pgfqpoint{2.151489in}{1.210565in}}%
\pgfpathcurveto{\pgfqpoint{2.150186in}{1.209262in}}{\pgfqpoint{2.149455in}{1.207496in}}{\pgfqpoint{2.149455in}{1.205654in}}%
\pgfpathcurveto{\pgfqpoint{2.149455in}{1.203813in}}{\pgfqpoint{2.150186in}{1.202046in}}{\pgfqpoint{2.151489in}{1.200744in}}%
\pgfpathcurveto{\pgfqpoint{2.152791in}{1.199442in}}{\pgfqpoint{2.154557in}{1.198710in}}{\pgfqpoint{2.156399in}{1.198710in}}%
\pgfpathlineto{\pgfqpoint{2.156399in}{1.198710in}}%
\pgfpathclose%
\pgfusepath{stroke,fill}%
\end{pgfscope}%
\begin{pgfscope}%
\pgfpathrectangle{\pgfqpoint{0.661006in}{0.524170in}}{\pgfqpoint{4.194036in}{1.071446in}}%
\pgfusepath{clip}%
\pgfsetbuttcap%
\pgfsetroundjoin%
\definecolor{currentfill}{rgb}{0.510866,0.600854,0.767213}%
\pgfsetfillcolor{currentfill}%
\pgfsetfillopacity{0.700000}%
\pgfsetlinewidth{1.003750pt}%
\definecolor{currentstroke}{rgb}{0.510866,0.600854,0.767213}%
\pgfsetstrokecolor{currentstroke}%
\pgfsetstrokeopacity{0.700000}%
\pgfsetdash{}{0pt}%
\pgfpathmoveto{\pgfqpoint{2.165741in}{1.196265in}}%
\pgfpathcurveto{\pgfqpoint{2.167583in}{1.196265in}}{\pgfqpoint{2.169349in}{1.196996in}}{\pgfqpoint{2.170652in}{1.198299in}}%
\pgfpathcurveto{\pgfqpoint{2.171954in}{1.199601in}}{\pgfqpoint{2.172686in}{1.201367in}}{\pgfqpoint{2.172686in}{1.203209in}}%
\pgfpathcurveto{\pgfqpoint{2.172686in}{1.205051in}}{\pgfqpoint{2.171954in}{1.206817in}}{\pgfqpoint{2.170652in}{1.208119in}}%
\pgfpathcurveto{\pgfqpoint{2.169349in}{1.209422in}}{\pgfqpoint{2.167583in}{1.210153in}}{\pgfqpoint{2.165741in}{1.210153in}}%
\pgfpathcurveto{\pgfqpoint{2.163899in}{1.210153in}}{\pgfqpoint{2.162133in}{1.209422in}}{\pgfqpoint{2.160831in}{1.208119in}}%
\pgfpathcurveto{\pgfqpoint{2.159528in}{1.206817in}}{\pgfqpoint{2.158797in}{1.205051in}}{\pgfqpoint{2.158797in}{1.203209in}}%
\pgfpathcurveto{\pgfqpoint{2.158797in}{1.201367in}}{\pgfqpoint{2.159528in}{1.199601in}}{\pgfqpoint{2.160831in}{1.198299in}}%
\pgfpathcurveto{\pgfqpoint{2.162133in}{1.196996in}}{\pgfqpoint{2.163899in}{1.196265in}}{\pgfqpoint{2.165741in}{1.196265in}}%
\pgfpathlineto{\pgfqpoint{2.165741in}{1.196265in}}%
\pgfpathclose%
\pgfusepath{stroke,fill}%
\end{pgfscope}%
\begin{pgfscope}%
\pgfpathrectangle{\pgfqpoint{0.661006in}{0.524170in}}{\pgfqpoint{4.194036in}{1.071446in}}%
\pgfusepath{clip}%
\pgfsetbuttcap%
\pgfsetroundjoin%
\definecolor{currentfill}{rgb}{0.510866,0.600854,0.767213}%
\pgfsetfillcolor{currentfill}%
\pgfsetfillopacity{0.700000}%
\pgfsetlinewidth{1.003750pt}%
\definecolor{currentstroke}{rgb}{0.510866,0.600854,0.767213}%
\pgfsetstrokecolor{currentstroke}%
\pgfsetstrokeopacity{0.700000}%
\pgfsetdash{}{0pt}%
\pgfpathmoveto{\pgfqpoint{2.158955in}{1.197853in}}%
\pgfpathcurveto{\pgfqpoint{2.160797in}{1.197853in}}{\pgfqpoint{2.162564in}{1.198585in}}{\pgfqpoint{2.163866in}{1.199887in}}%
\pgfpathcurveto{\pgfqpoint{2.165168in}{1.201190in}}{\pgfqpoint{2.165900in}{1.202956in}}{\pgfqpoint{2.165900in}{1.204798in}}%
\pgfpathcurveto{\pgfqpoint{2.165900in}{1.206640in}}{\pgfqpoint{2.165168in}{1.208406in}}{\pgfqpoint{2.163866in}{1.209708in}}%
\pgfpathcurveto{\pgfqpoint{2.162564in}{1.211011in}}{\pgfqpoint{2.160797in}{1.211742in}}{\pgfqpoint{2.158955in}{1.211742in}}%
\pgfpathcurveto{\pgfqpoint{2.157114in}{1.211742in}}{\pgfqpoint{2.155347in}{1.211011in}}{\pgfqpoint{2.154045in}{1.209708in}}%
\pgfpathcurveto{\pgfqpoint{2.152743in}{1.208406in}}{\pgfqpoint{2.152011in}{1.206640in}}{\pgfqpoint{2.152011in}{1.204798in}}%
\pgfpathcurveto{\pgfqpoint{2.152011in}{1.202956in}}{\pgfqpoint{2.152743in}{1.201190in}}{\pgfqpoint{2.154045in}{1.199887in}}%
\pgfpathcurveto{\pgfqpoint{2.155347in}{1.198585in}}{\pgfqpoint{2.157114in}{1.197853in}}{\pgfqpoint{2.158955in}{1.197853in}}%
\pgfpathlineto{\pgfqpoint{2.158955in}{1.197853in}}%
\pgfpathclose%
\pgfusepath{stroke,fill}%
\end{pgfscope}%
\begin{pgfscope}%
\pgfpathrectangle{\pgfqpoint{0.661006in}{0.524170in}}{\pgfqpoint{4.194036in}{1.071446in}}%
\pgfusepath{clip}%
\pgfsetbuttcap%
\pgfsetroundjoin%
\definecolor{currentfill}{rgb}{0.507650,0.596392,0.764266}%
\pgfsetfillcolor{currentfill}%
\pgfsetfillopacity{0.700000}%
\pgfsetlinewidth{1.003750pt}%
\definecolor{currentstroke}{rgb}{0.507650,0.596392,0.764266}%
\pgfsetstrokecolor{currentstroke}%
\pgfsetstrokeopacity{0.700000}%
\pgfsetdash{}{0pt}%
\pgfpathmoveto{\pgfqpoint{2.136693in}{1.202311in}}%
\pgfpathcurveto{\pgfqpoint{2.138535in}{1.202311in}}{\pgfqpoint{2.140301in}{1.203042in}}{\pgfqpoint{2.141603in}{1.204345in}}%
\pgfpathcurveto{\pgfqpoint{2.142906in}{1.205647in}}{\pgfqpoint{2.143637in}{1.207414in}}{\pgfqpoint{2.143637in}{1.209255in}}%
\pgfpathcurveto{\pgfqpoint{2.143637in}{1.211097in}}{\pgfqpoint{2.142906in}{1.212863in}}{\pgfqpoint{2.141603in}{1.214166in}}%
\pgfpathcurveto{\pgfqpoint{2.140301in}{1.215468in}}{\pgfqpoint{2.138535in}{1.216200in}}{\pgfqpoint{2.136693in}{1.216200in}}%
\pgfpathcurveto{\pgfqpoint{2.134851in}{1.216200in}}{\pgfqpoint{2.133085in}{1.215468in}}{\pgfqpoint{2.131782in}{1.214166in}}%
\pgfpathcurveto{\pgfqpoint{2.130480in}{1.212863in}}{\pgfqpoint{2.129748in}{1.211097in}}{\pgfqpoint{2.129748in}{1.209255in}}%
\pgfpathcurveto{\pgfqpoint{2.129748in}{1.207414in}}{\pgfqpoint{2.130480in}{1.205647in}}{\pgfqpoint{2.131782in}{1.204345in}}%
\pgfpathcurveto{\pgfqpoint{2.133085in}{1.203042in}}{\pgfqpoint{2.134851in}{1.202311in}}{\pgfqpoint{2.136693in}{1.202311in}}%
\pgfpathlineto{\pgfqpoint{2.136693in}{1.202311in}}%
\pgfpathclose%
\pgfusepath{stroke,fill}%
\end{pgfscope}%
\begin{pgfscope}%
\pgfpathrectangle{\pgfqpoint{0.661006in}{0.524170in}}{\pgfqpoint{4.194036in}{1.071446in}}%
\pgfusepath{clip}%
\pgfsetbuttcap%
\pgfsetroundjoin%
\definecolor{currentfill}{rgb}{0.507650,0.596392,0.764266}%
\pgfsetfillcolor{currentfill}%
\pgfsetfillopacity{0.700000}%
\pgfsetlinewidth{1.003750pt}%
\definecolor{currentstroke}{rgb}{0.507650,0.596392,0.764266}%
\pgfsetstrokecolor{currentstroke}%
\pgfsetstrokeopacity{0.700000}%
\pgfsetdash{}{0pt}%
\pgfpathmoveto{\pgfqpoint{2.107459in}{1.209486in}}%
\pgfpathcurveto{\pgfqpoint{2.109300in}{1.209486in}}{\pgfqpoint{2.111067in}{1.210218in}}{\pgfqpoint{2.112369in}{1.211520in}}%
\pgfpathcurveto{\pgfqpoint{2.113671in}{1.212822in}}{\pgfqpoint{2.114403in}{1.214589in}}{\pgfqpoint{2.114403in}{1.216431in}}%
\pgfpathcurveto{\pgfqpoint{2.114403in}{1.218272in}}{\pgfqpoint{2.113671in}{1.220039in}}{\pgfqpoint{2.112369in}{1.221341in}}%
\pgfpathcurveto{\pgfqpoint{2.111067in}{1.222643in}}{\pgfqpoint{2.109300in}{1.223375in}}{\pgfqpoint{2.107459in}{1.223375in}}%
\pgfpathcurveto{\pgfqpoint{2.105617in}{1.223375in}}{\pgfqpoint{2.103850in}{1.222643in}}{\pgfqpoint{2.102548in}{1.221341in}}%
\pgfpathcurveto{\pgfqpoint{2.101246in}{1.220039in}}{\pgfqpoint{2.100514in}{1.218272in}}{\pgfqpoint{2.100514in}{1.216431in}}%
\pgfpathcurveto{\pgfqpoint{2.100514in}{1.214589in}}{\pgfqpoint{2.101246in}{1.212822in}}{\pgfqpoint{2.102548in}{1.211520in}}%
\pgfpathcurveto{\pgfqpoint{2.103850in}{1.210218in}}{\pgfqpoint{2.105617in}{1.209486in}}{\pgfqpoint{2.107459in}{1.209486in}}%
\pgfpathlineto{\pgfqpoint{2.107459in}{1.209486in}}%
\pgfpathclose%
\pgfusepath{stroke,fill}%
\end{pgfscope}%
\begin{pgfscope}%
\pgfpathrectangle{\pgfqpoint{0.661006in}{0.524170in}}{\pgfqpoint{4.194036in}{1.071446in}}%
\pgfusepath{clip}%
\pgfsetbuttcap%
\pgfsetroundjoin%
\definecolor{currentfill}{rgb}{0.507650,0.596392,0.764266}%
\pgfsetfillcolor{currentfill}%
\pgfsetfillopacity{0.700000}%
\pgfsetlinewidth{1.003750pt}%
\definecolor{currentstroke}{rgb}{0.507650,0.596392,0.764266}%
\pgfsetstrokecolor{currentstroke}%
\pgfsetstrokeopacity{0.700000}%
\pgfsetdash{}{0pt}%
\pgfpathmoveto{\pgfqpoint{2.074048in}{1.217716in}}%
\pgfpathcurveto{\pgfqpoint{2.075890in}{1.217716in}}{\pgfqpoint{2.077656in}{1.218447in}}{\pgfqpoint{2.078959in}{1.219750in}}%
\pgfpathcurveto{\pgfqpoint{2.080261in}{1.221052in}}{\pgfqpoint{2.080993in}{1.222818in}}{\pgfqpoint{2.080993in}{1.224660in}}%
\pgfpathcurveto{\pgfqpoint{2.080993in}{1.226502in}}{\pgfqpoint{2.080261in}{1.228268in}}{\pgfqpoint{2.078959in}{1.229571in}}%
\pgfpathcurveto{\pgfqpoint{2.077656in}{1.230873in}}{\pgfqpoint{2.075890in}{1.231604in}}{\pgfqpoint{2.074048in}{1.231604in}}%
\pgfpathcurveto{\pgfqpoint{2.072206in}{1.231604in}}{\pgfqpoint{2.070440in}{1.230873in}}{\pgfqpoint{2.069138in}{1.229571in}}%
\pgfpathcurveto{\pgfqpoint{2.067835in}{1.228268in}}{\pgfqpoint{2.067104in}{1.226502in}}{\pgfqpoint{2.067104in}{1.224660in}}%
\pgfpathcurveto{\pgfqpoint{2.067104in}{1.222818in}}{\pgfqpoint{2.067835in}{1.221052in}}{\pgfqpoint{2.069138in}{1.219750in}}%
\pgfpathcurveto{\pgfqpoint{2.070440in}{1.218447in}}{\pgfqpoint{2.072206in}{1.217716in}}{\pgfqpoint{2.074048in}{1.217716in}}%
\pgfpathlineto{\pgfqpoint{2.074048in}{1.217716in}}%
\pgfpathclose%
\pgfusepath{stroke,fill}%
\end{pgfscope}%
\begin{pgfscope}%
\pgfpathrectangle{\pgfqpoint{0.661006in}{0.524170in}}{\pgfqpoint{4.194036in}{1.071446in}}%
\pgfusepath{clip}%
\pgfsetbuttcap%
\pgfsetroundjoin%
\definecolor{currentfill}{rgb}{0.507650,0.596392,0.764266}%
\pgfsetfillcolor{currentfill}%
\pgfsetfillopacity{0.700000}%
\pgfsetlinewidth{1.003750pt}%
\definecolor{currentstroke}{rgb}{0.507650,0.596392,0.764266}%
\pgfsetstrokecolor{currentstroke}%
\pgfsetstrokeopacity{0.700000}%
\pgfsetdash{}{0pt}%
\pgfpathmoveto{\pgfqpoint{2.035465in}{1.225779in}}%
\pgfpathcurveto{\pgfqpoint{2.037307in}{1.225779in}}{\pgfqpoint{2.039074in}{1.226511in}}{\pgfqpoint{2.040376in}{1.227813in}}%
\pgfpathcurveto{\pgfqpoint{2.041678in}{1.229115in}}{\pgfqpoint{2.042410in}{1.230882in}}{\pgfqpoint{2.042410in}{1.232723in}}%
\pgfpathcurveto{\pgfqpoint{2.042410in}{1.234565in}}{\pgfqpoint{2.041678in}{1.236332in}}{\pgfqpoint{2.040376in}{1.237634in}}%
\pgfpathcurveto{\pgfqpoint{2.039074in}{1.238936in}}{\pgfqpoint{2.037307in}{1.239668in}}{\pgfqpoint{2.035465in}{1.239668in}}%
\pgfpathcurveto{\pgfqpoint{2.033624in}{1.239668in}}{\pgfqpoint{2.031857in}{1.238936in}}{\pgfqpoint{2.030555in}{1.237634in}}%
\pgfpathcurveto{\pgfqpoint{2.029253in}{1.236332in}}{\pgfqpoint{2.028521in}{1.234565in}}{\pgfqpoint{2.028521in}{1.232723in}}%
\pgfpathcurveto{\pgfqpoint{2.028521in}{1.230882in}}{\pgfqpoint{2.029253in}{1.229115in}}{\pgfqpoint{2.030555in}{1.227813in}}%
\pgfpathcurveto{\pgfqpoint{2.031857in}{1.226511in}}{\pgfqpoint{2.033624in}{1.225779in}}{\pgfqpoint{2.035465in}{1.225779in}}%
\pgfpathlineto{\pgfqpoint{2.035465in}{1.225779in}}%
\pgfpathclose%
\pgfusepath{stroke,fill}%
\end{pgfscope}%
\begin{pgfscope}%
\pgfpathrectangle{\pgfqpoint{0.661006in}{0.524170in}}{\pgfqpoint{4.194036in}{1.071446in}}%
\pgfusepath{clip}%
\pgfsetbuttcap%
\pgfsetroundjoin%
\definecolor{currentfill}{rgb}{0.507650,0.596392,0.764266}%
\pgfsetfillcolor{currentfill}%
\pgfsetfillopacity{0.700000}%
\pgfsetlinewidth{1.003750pt}%
\definecolor{currentstroke}{rgb}{0.507650,0.596392,0.764266}%
\pgfsetstrokecolor{currentstroke}%
\pgfsetstrokeopacity{0.700000}%
\pgfsetdash{}{0pt}%
\pgfpathmoveto{\pgfqpoint{2.011158in}{1.230231in}}%
\pgfpathcurveto{\pgfqpoint{2.012999in}{1.230231in}}{\pgfqpoint{2.014766in}{1.230963in}}{\pgfqpoint{2.016068in}{1.232265in}}%
\pgfpathcurveto{\pgfqpoint{2.017371in}{1.233568in}}{\pgfqpoint{2.018102in}{1.235334in}}{\pgfqpoint{2.018102in}{1.237176in}}%
\pgfpathcurveto{\pgfqpoint{2.018102in}{1.239018in}}{\pgfqpoint{2.017371in}{1.240784in}}{\pgfqpoint{2.016068in}{1.242086in}}%
\pgfpathcurveto{\pgfqpoint{2.014766in}{1.243389in}}{\pgfqpoint{2.012999in}{1.244120in}}{\pgfqpoint{2.011158in}{1.244120in}}%
\pgfpathcurveto{\pgfqpoint{2.009316in}{1.244120in}}{\pgfqpoint{2.007550in}{1.243389in}}{\pgfqpoint{2.006247in}{1.242086in}}%
\pgfpathcurveto{\pgfqpoint{2.004945in}{1.240784in}}{\pgfqpoint{2.004213in}{1.239018in}}{\pgfqpoint{2.004213in}{1.237176in}}%
\pgfpathcurveto{\pgfqpoint{2.004213in}{1.235334in}}{\pgfqpoint{2.004945in}{1.233568in}}{\pgfqpoint{2.006247in}{1.232265in}}%
\pgfpathcurveto{\pgfqpoint{2.007550in}{1.230963in}}{\pgfqpoint{2.009316in}{1.230231in}}{\pgfqpoint{2.011158in}{1.230231in}}%
\pgfpathlineto{\pgfqpoint{2.011158in}{1.230231in}}%
\pgfpathclose%
\pgfusepath{stroke,fill}%
\end{pgfscope}%
\begin{pgfscope}%
\pgfpathrectangle{\pgfqpoint{0.661006in}{0.524170in}}{\pgfqpoint{4.194036in}{1.071446in}}%
\pgfusepath{clip}%
\pgfsetbuttcap%
\pgfsetroundjoin%
\definecolor{currentfill}{rgb}{0.504453,0.591925,0.761293}%
\pgfsetfillcolor{currentfill}%
\pgfsetfillopacity{0.700000}%
\pgfsetlinewidth{1.003750pt}%
\definecolor{currentstroke}{rgb}{0.504453,0.591925,0.761293}%
\pgfsetstrokecolor{currentstroke}%
\pgfsetstrokeopacity{0.700000}%
\pgfsetdash{}{0pt}%
\pgfpathmoveto{\pgfqpoint{2.000561in}{1.231455in}}%
\pgfpathcurveto{\pgfqpoint{2.002403in}{1.231455in}}{\pgfqpoint{2.004169in}{1.232186in}}{\pgfqpoint{2.005471in}{1.233489in}}%
\pgfpathcurveto{\pgfqpoint{2.006774in}{1.234791in}}{\pgfqpoint{2.007505in}{1.236558in}}{\pgfqpoint{2.007505in}{1.238399in}}%
\pgfpathcurveto{\pgfqpoint{2.007505in}{1.240241in}}{\pgfqpoint{2.006774in}{1.242007in}}{\pgfqpoint{2.005471in}{1.243310in}}%
\pgfpathcurveto{\pgfqpoint{2.004169in}{1.244612in}}{\pgfqpoint{2.002403in}{1.245344in}}{\pgfqpoint{2.000561in}{1.245344in}}%
\pgfpathcurveto{\pgfqpoint{1.998719in}{1.245344in}}{\pgfqpoint{1.996953in}{1.244612in}}{\pgfqpoint{1.995651in}{1.243310in}}%
\pgfpathcurveto{\pgfqpoint{1.994348in}{1.242007in}}{\pgfqpoint{1.993617in}{1.240241in}}{\pgfqpoint{1.993617in}{1.238399in}}%
\pgfpathcurveto{\pgfqpoint{1.993617in}{1.236558in}}{\pgfqpoint{1.994348in}{1.234791in}}{\pgfqpoint{1.995651in}{1.233489in}}%
\pgfpathcurveto{\pgfqpoint{1.996953in}{1.232186in}}{\pgfqpoint{1.998719in}{1.231455in}}{\pgfqpoint{2.000561in}{1.231455in}}%
\pgfpathlineto{\pgfqpoint{2.000561in}{1.231455in}}%
\pgfpathclose%
\pgfusepath{stroke,fill}%
\end{pgfscope}%
\begin{pgfscope}%
\pgfpathrectangle{\pgfqpoint{0.661006in}{0.524170in}}{\pgfqpoint{4.194036in}{1.071446in}}%
\pgfusepath{clip}%
\pgfsetbuttcap%
\pgfsetroundjoin%
\definecolor{currentfill}{rgb}{0.504453,0.591925,0.761293}%
\pgfsetfillcolor{currentfill}%
\pgfsetfillopacity{0.700000}%
\pgfsetlinewidth{1.003750pt}%
\definecolor{currentstroke}{rgb}{0.504453,0.591925,0.761293}%
\pgfsetstrokecolor{currentstroke}%
\pgfsetstrokeopacity{0.700000}%
\pgfsetdash{}{0pt}%
\pgfpathmoveto{\pgfqpoint{2.006464in}{1.230744in}}%
\pgfpathcurveto{\pgfqpoint{2.008305in}{1.230744in}}{\pgfqpoint{2.010072in}{1.231476in}}{\pgfqpoint{2.011374in}{1.232778in}}%
\pgfpathcurveto{\pgfqpoint{2.012676in}{1.234081in}}{\pgfqpoint{2.013408in}{1.235847in}}{\pgfqpoint{2.013408in}{1.237689in}}%
\pgfpathcurveto{\pgfqpoint{2.013408in}{1.239530in}}{\pgfqpoint{2.012676in}{1.241297in}}{\pgfqpoint{2.011374in}{1.242599in}}%
\pgfpathcurveto{\pgfqpoint{2.010072in}{1.243901in}}{\pgfqpoint{2.008305in}{1.244633in}}{\pgfqpoint{2.006464in}{1.244633in}}%
\pgfpathcurveto{\pgfqpoint{2.004622in}{1.244633in}}{\pgfqpoint{2.002855in}{1.243901in}}{\pgfqpoint{2.001553in}{1.242599in}}%
\pgfpathcurveto{\pgfqpoint{2.000251in}{1.241297in}}{\pgfqpoint{1.999519in}{1.239530in}}{\pgfqpoint{1.999519in}{1.237689in}}%
\pgfpathcurveto{\pgfqpoint{1.999519in}{1.235847in}}{\pgfqpoint{2.000251in}{1.234081in}}{\pgfqpoint{2.001553in}{1.232778in}}%
\pgfpathcurveto{\pgfqpoint{2.002855in}{1.231476in}}{\pgfqpoint{2.004622in}{1.230744in}}{\pgfqpoint{2.006464in}{1.230744in}}%
\pgfpathlineto{\pgfqpoint{2.006464in}{1.230744in}}%
\pgfpathclose%
\pgfusepath{stroke,fill}%
\end{pgfscope}%
\begin{pgfscope}%
\pgfpathrectangle{\pgfqpoint{0.661006in}{0.524170in}}{\pgfqpoint{4.194036in}{1.071446in}}%
\pgfusepath{clip}%
\pgfsetbuttcap%
\pgfsetroundjoin%
\definecolor{currentfill}{rgb}{0.501274,0.587454,0.758294}%
\pgfsetfillcolor{currentfill}%
\pgfsetfillopacity{0.700000}%
\pgfsetlinewidth{1.003750pt}%
\definecolor{currentstroke}{rgb}{0.501274,0.587454,0.758294}%
\pgfsetstrokecolor{currentstroke}%
\pgfsetstrokeopacity{0.700000}%
\pgfsetdash{}{0pt}%
\pgfpathmoveto{\pgfqpoint{1.997447in}{1.232081in}}%
\pgfpathcurveto{\pgfqpoint{1.999289in}{1.232081in}}{\pgfqpoint{2.001055in}{1.232813in}}{\pgfqpoint{2.002357in}{1.234115in}}%
\pgfpathcurveto{\pgfqpoint{2.003660in}{1.235417in}}{\pgfqpoint{2.004391in}{1.237184in}}{\pgfqpoint{2.004391in}{1.239026in}}%
\pgfpathcurveto{\pgfqpoint{2.004391in}{1.240867in}}{\pgfqpoint{2.003660in}{1.242634in}}{\pgfqpoint{2.002357in}{1.243936in}}%
\pgfpathcurveto{\pgfqpoint{2.001055in}{1.245238in}}{\pgfqpoint{1.999289in}{1.245970in}}{\pgfqpoint{1.997447in}{1.245970in}}%
\pgfpathcurveto{\pgfqpoint{1.995605in}{1.245970in}}{\pgfqpoint{1.993839in}{1.245238in}}{\pgfqpoint{1.992537in}{1.243936in}}%
\pgfpathcurveto{\pgfqpoint{1.991234in}{1.242634in}}{\pgfqpoint{1.990503in}{1.240867in}}{\pgfqpoint{1.990503in}{1.239026in}}%
\pgfpathcurveto{\pgfqpoint{1.990503in}{1.237184in}}{\pgfqpoint{1.991234in}{1.235417in}}{\pgfqpoint{1.992537in}{1.234115in}}%
\pgfpathcurveto{\pgfqpoint{1.993839in}{1.232813in}}{\pgfqpoint{1.995605in}{1.232081in}}{\pgfqpoint{1.997447in}{1.232081in}}%
\pgfpathlineto{\pgfqpoint{1.997447in}{1.232081in}}%
\pgfpathclose%
\pgfusepath{stroke,fill}%
\end{pgfscope}%
\begin{pgfscope}%
\pgfpathrectangle{\pgfqpoint{0.661006in}{0.524170in}}{\pgfqpoint{4.194036in}{1.071446in}}%
\pgfusepath{clip}%
\pgfsetbuttcap%
\pgfsetroundjoin%
\definecolor{currentfill}{rgb}{0.501274,0.587454,0.758294}%
\pgfsetfillcolor{currentfill}%
\pgfsetfillopacity{0.700000}%
\pgfsetlinewidth{1.003750pt}%
\definecolor{currentstroke}{rgb}{0.501274,0.587454,0.758294}%
\pgfsetstrokecolor{currentstroke}%
\pgfsetstrokeopacity{0.700000}%
\pgfsetdash{}{0pt}%
\pgfpathmoveto{\pgfqpoint{1.984340in}{1.235987in}}%
\pgfpathcurveto{\pgfqpoint{1.986182in}{1.235987in}}{\pgfqpoint{1.987949in}{1.236719in}}{\pgfqpoint{1.989251in}{1.238021in}}%
\pgfpathcurveto{\pgfqpoint{1.990553in}{1.239324in}}{\pgfqpoint{1.991285in}{1.241090in}}{\pgfqpoint{1.991285in}{1.242932in}}%
\pgfpathcurveto{\pgfqpoint{1.991285in}{1.244773in}}{\pgfqpoint{1.990553in}{1.246540in}}{\pgfqpoint{1.989251in}{1.247842in}}%
\pgfpathcurveto{\pgfqpoint{1.987949in}{1.249145in}}{\pgfqpoint{1.986182in}{1.249876in}}{\pgfqpoint{1.984340in}{1.249876in}}%
\pgfpathcurveto{\pgfqpoint{1.982499in}{1.249876in}}{\pgfqpoint{1.980732in}{1.249145in}}{\pgfqpoint{1.979430in}{1.247842in}}%
\pgfpathcurveto{\pgfqpoint{1.978128in}{1.246540in}}{\pgfqpoint{1.977396in}{1.244773in}}{\pgfqpoint{1.977396in}{1.242932in}}%
\pgfpathcurveto{\pgfqpoint{1.977396in}{1.241090in}}{\pgfqpoint{1.978128in}{1.239324in}}{\pgfqpoint{1.979430in}{1.238021in}}%
\pgfpathcurveto{\pgfqpoint{1.980732in}{1.236719in}}{\pgfqpoint{1.982499in}{1.235987in}}{\pgfqpoint{1.984340in}{1.235987in}}%
\pgfpathlineto{\pgfqpoint{1.984340in}{1.235987in}}%
\pgfpathclose%
\pgfusepath{stroke,fill}%
\end{pgfscope}%
\begin{pgfscope}%
\pgfpathrectangle{\pgfqpoint{0.661006in}{0.524170in}}{\pgfqpoint{4.194036in}{1.071446in}}%
\pgfusepath{clip}%
\pgfsetbuttcap%
\pgfsetroundjoin%
\definecolor{currentfill}{rgb}{0.501274,0.587454,0.758294}%
\pgfsetfillcolor{currentfill}%
\pgfsetfillopacity{0.700000}%
\pgfsetlinewidth{1.003750pt}%
\definecolor{currentstroke}{rgb}{0.501274,0.587454,0.758294}%
\pgfsetstrokecolor{currentstroke}%
\pgfsetstrokeopacity{0.700000}%
\pgfsetdash{}{0pt}%
\pgfpathmoveto{\pgfqpoint{1.955432in}{1.240605in}}%
\pgfpathcurveto{\pgfqpoint{1.957273in}{1.240605in}}{\pgfqpoint{1.959040in}{1.241336in}}{\pgfqpoint{1.960342in}{1.242639in}}%
\pgfpathcurveto{\pgfqpoint{1.961644in}{1.243941in}}{\pgfqpoint{1.962376in}{1.245707in}}{\pgfqpoint{1.962376in}{1.247549in}}%
\pgfpathcurveto{\pgfqpoint{1.962376in}{1.249391in}}{\pgfqpoint{1.961644in}{1.251157in}}{\pgfqpoint{1.960342in}{1.252460in}}%
\pgfpathcurveto{\pgfqpoint{1.959040in}{1.253762in}}{\pgfqpoint{1.957273in}{1.254494in}}{\pgfqpoint{1.955432in}{1.254494in}}%
\pgfpathcurveto{\pgfqpoint{1.953590in}{1.254494in}}{\pgfqpoint{1.951823in}{1.253762in}}{\pgfqpoint{1.950521in}{1.252460in}}%
\pgfpathcurveto{\pgfqpoint{1.949219in}{1.251157in}}{\pgfqpoint{1.948487in}{1.249391in}}{\pgfqpoint{1.948487in}{1.247549in}}%
\pgfpathcurveto{\pgfqpoint{1.948487in}{1.245707in}}{\pgfqpoint{1.949219in}{1.243941in}}{\pgfqpoint{1.950521in}{1.242639in}}%
\pgfpathcurveto{\pgfqpoint{1.951823in}{1.241336in}}{\pgfqpoint{1.953590in}{1.240605in}}{\pgfqpoint{1.955432in}{1.240605in}}%
\pgfpathlineto{\pgfqpoint{1.955432in}{1.240605in}}%
\pgfpathclose%
\pgfusepath{stroke,fill}%
\end{pgfscope}%
\begin{pgfscope}%
\pgfpathrectangle{\pgfqpoint{0.661006in}{0.524170in}}{\pgfqpoint{4.194036in}{1.071446in}}%
\pgfusepath{clip}%
\pgfsetbuttcap%
\pgfsetroundjoin%
\definecolor{currentfill}{rgb}{0.501274,0.587454,0.758294}%
\pgfsetfillcolor{currentfill}%
\pgfsetfillopacity{0.700000}%
\pgfsetlinewidth{1.003750pt}%
\definecolor{currentstroke}{rgb}{0.501274,0.587454,0.758294}%
\pgfsetstrokecolor{currentstroke}%
\pgfsetstrokeopacity{0.700000}%
\pgfsetdash{}{0pt}%
\pgfpathmoveto{\pgfqpoint{1.938281in}{1.245104in}}%
\pgfpathcurveto{\pgfqpoint{1.940123in}{1.245104in}}{\pgfqpoint{1.941890in}{1.245836in}}{\pgfqpoint{1.943192in}{1.247138in}}%
\pgfpathcurveto{\pgfqpoint{1.944494in}{1.248440in}}{\pgfqpoint{1.945226in}{1.250207in}}{\pgfqpoint{1.945226in}{1.252049in}}%
\pgfpathcurveto{\pgfqpoint{1.945226in}{1.253890in}}{\pgfqpoint{1.944494in}{1.255657in}}{\pgfqpoint{1.943192in}{1.256959in}}%
\pgfpathcurveto{\pgfqpoint{1.941890in}{1.258261in}}{\pgfqpoint{1.940123in}{1.258993in}}{\pgfqpoint{1.938281in}{1.258993in}}%
\pgfpathcurveto{\pgfqpoint{1.936440in}{1.258993in}}{\pgfqpoint{1.934673in}{1.258261in}}{\pgfqpoint{1.933371in}{1.256959in}}%
\pgfpathcurveto{\pgfqpoint{1.932069in}{1.255657in}}{\pgfqpoint{1.931337in}{1.253890in}}{\pgfqpoint{1.931337in}{1.252049in}}%
\pgfpathcurveto{\pgfqpoint{1.931337in}{1.250207in}}{\pgfqpoint{1.932069in}{1.248440in}}{\pgfqpoint{1.933371in}{1.247138in}}%
\pgfpathcurveto{\pgfqpoint{1.934673in}{1.245836in}}{\pgfqpoint{1.936440in}{1.245104in}}{\pgfqpoint{1.938281in}{1.245104in}}%
\pgfpathlineto{\pgfqpoint{1.938281in}{1.245104in}}%
\pgfpathclose%
\pgfusepath{stroke,fill}%
\end{pgfscope}%
\begin{pgfscope}%
\pgfpathrectangle{\pgfqpoint{0.661006in}{0.524170in}}{\pgfqpoint{4.194036in}{1.071446in}}%
\pgfusepath{clip}%
\pgfsetbuttcap%
\pgfsetroundjoin%
\definecolor{currentfill}{rgb}{0.501274,0.587454,0.758294}%
\pgfsetfillcolor{currentfill}%
\pgfsetfillopacity{0.700000}%
\pgfsetlinewidth{1.003750pt}%
\definecolor{currentstroke}{rgb}{0.501274,0.587454,0.758294}%
\pgfsetstrokecolor{currentstroke}%
\pgfsetstrokeopacity{0.700000}%
\pgfsetdash{}{0pt}%
\pgfpathmoveto{\pgfqpoint{1.915740in}{1.249705in}}%
\pgfpathcurveto{\pgfqpoint{1.917582in}{1.249705in}}{\pgfqpoint{1.919348in}{1.250436in}}{\pgfqpoint{1.920650in}{1.251739in}}%
\pgfpathcurveto{\pgfqpoint{1.921953in}{1.253041in}}{\pgfqpoint{1.922684in}{1.254807in}}{\pgfqpoint{1.922684in}{1.256649in}}%
\pgfpathcurveto{\pgfqpoint{1.922684in}{1.258491in}}{\pgfqpoint{1.921953in}{1.260257in}}{\pgfqpoint{1.920650in}{1.261560in}}%
\pgfpathcurveto{\pgfqpoint{1.919348in}{1.262862in}}{\pgfqpoint{1.917582in}{1.263594in}}{\pgfqpoint{1.915740in}{1.263594in}}%
\pgfpathcurveto{\pgfqpoint{1.913898in}{1.263594in}}{\pgfqpoint{1.912132in}{1.262862in}}{\pgfqpoint{1.910830in}{1.261560in}}%
\pgfpathcurveto{\pgfqpoint{1.909527in}{1.260257in}}{\pgfqpoint{1.908796in}{1.258491in}}{\pgfqpoint{1.908796in}{1.256649in}}%
\pgfpathcurveto{\pgfqpoint{1.908796in}{1.254807in}}{\pgfqpoint{1.909527in}{1.253041in}}{\pgfqpoint{1.910830in}{1.251739in}}%
\pgfpathcurveto{\pgfqpoint{1.912132in}{1.250436in}}{\pgfqpoint{1.913898in}{1.249705in}}{\pgfqpoint{1.915740in}{1.249705in}}%
\pgfpathlineto{\pgfqpoint{1.915740in}{1.249705in}}%
\pgfpathclose%
\pgfusepath{stroke,fill}%
\end{pgfscope}%
\begin{pgfscope}%
\pgfpathrectangle{\pgfqpoint{0.661006in}{0.524170in}}{\pgfqpoint{4.194036in}{1.071446in}}%
\pgfusepath{clip}%
\pgfsetbuttcap%
\pgfsetroundjoin%
\definecolor{currentfill}{rgb}{0.498113,0.582979,0.755267}%
\pgfsetfillcolor{currentfill}%
\pgfsetfillopacity{0.700000}%
\pgfsetlinewidth{1.003750pt}%
\definecolor{currentstroke}{rgb}{0.498113,0.582979,0.755267}%
\pgfsetstrokecolor{currentstroke}%
\pgfsetstrokeopacity{0.700000}%
\pgfsetdash{}{0pt}%
\pgfpathmoveto{\pgfqpoint{1.897521in}{1.253622in}}%
\pgfpathcurveto{\pgfqpoint{1.899363in}{1.253622in}}{\pgfqpoint{1.901129in}{1.254353in}}{\pgfqpoint{1.902431in}{1.255656in}}%
\pgfpathcurveto{\pgfqpoint{1.903734in}{1.256958in}}{\pgfqpoint{1.904465in}{1.258725in}}{\pgfqpoint{1.904465in}{1.260566in}}%
\pgfpathcurveto{\pgfqpoint{1.904465in}{1.262408in}}{\pgfqpoint{1.903734in}{1.264174in}}{\pgfqpoint{1.902431in}{1.265477in}}%
\pgfpathcurveto{\pgfqpoint{1.901129in}{1.266779in}}{\pgfqpoint{1.899363in}{1.267511in}}{\pgfqpoint{1.897521in}{1.267511in}}%
\pgfpathcurveto{\pgfqpoint{1.895679in}{1.267511in}}{\pgfqpoint{1.893913in}{1.266779in}}{\pgfqpoint{1.892610in}{1.265477in}}%
\pgfpathcurveto{\pgfqpoint{1.891308in}{1.264174in}}{\pgfqpoint{1.890577in}{1.262408in}}{\pgfqpoint{1.890577in}{1.260566in}}%
\pgfpathcurveto{\pgfqpoint{1.890577in}{1.258725in}}{\pgfqpoint{1.891308in}{1.256958in}}{\pgfqpoint{1.892610in}{1.255656in}}%
\pgfpathcurveto{\pgfqpoint{1.893913in}{1.254353in}}{\pgfqpoint{1.895679in}{1.253622in}}{\pgfqpoint{1.897521in}{1.253622in}}%
\pgfpathlineto{\pgfqpoint{1.897521in}{1.253622in}}%
\pgfpathclose%
\pgfusepath{stroke,fill}%
\end{pgfscope}%
\begin{pgfscope}%
\pgfpathrectangle{\pgfqpoint{0.661006in}{0.524170in}}{\pgfqpoint{4.194036in}{1.071446in}}%
\pgfusepath{clip}%
\pgfsetbuttcap%
\pgfsetroundjoin%
\definecolor{currentfill}{rgb}{0.498113,0.582979,0.755267}%
\pgfsetfillcolor{currentfill}%
\pgfsetfillopacity{0.700000}%
\pgfsetlinewidth{1.003750pt}%
\definecolor{currentstroke}{rgb}{0.498113,0.582979,0.755267}%
\pgfsetstrokecolor{currentstroke}%
\pgfsetstrokeopacity{0.700000}%
\pgfsetdash{}{0pt}%
\pgfpathmoveto{\pgfqpoint{1.890131in}{1.254724in}}%
\pgfpathcurveto{\pgfqpoint{1.891973in}{1.254724in}}{\pgfqpoint{1.893739in}{1.255455in}}{\pgfqpoint{1.895042in}{1.256758in}}%
\pgfpathcurveto{\pgfqpoint{1.896344in}{1.258060in}}{\pgfqpoint{1.897076in}{1.259826in}}{\pgfqpoint{1.897076in}{1.261668in}}%
\pgfpathcurveto{\pgfqpoint{1.897076in}{1.263510in}}{\pgfqpoint{1.896344in}{1.265276in}}{\pgfqpoint{1.895042in}{1.266578in}}%
\pgfpathcurveto{\pgfqpoint{1.893739in}{1.267881in}}{\pgfqpoint{1.891973in}{1.268612in}}{\pgfqpoint{1.890131in}{1.268612in}}%
\pgfpathcurveto{\pgfqpoint{1.888289in}{1.268612in}}{\pgfqpoint{1.886523in}{1.267881in}}{\pgfqpoint{1.885221in}{1.266578in}}%
\pgfpathcurveto{\pgfqpoint{1.883918in}{1.265276in}}{\pgfqpoint{1.883187in}{1.263510in}}{\pgfqpoint{1.883187in}{1.261668in}}%
\pgfpathcurveto{\pgfqpoint{1.883187in}{1.259826in}}{\pgfqpoint{1.883918in}{1.258060in}}{\pgfqpoint{1.885221in}{1.256758in}}%
\pgfpathcurveto{\pgfqpoint{1.886523in}{1.255455in}}{\pgfqpoint{1.888289in}{1.254724in}}{\pgfqpoint{1.890131in}{1.254724in}}%
\pgfpathlineto{\pgfqpoint{1.890131in}{1.254724in}}%
\pgfpathclose%
\pgfusepath{stroke,fill}%
\end{pgfscope}%
\begin{pgfscope}%
\pgfpathrectangle{\pgfqpoint{0.661006in}{0.524170in}}{\pgfqpoint{4.194036in}{1.071446in}}%
\pgfusepath{clip}%
\pgfsetbuttcap%
\pgfsetroundjoin%
\definecolor{currentfill}{rgb}{0.494969,0.578499,0.752214}%
\pgfsetfillcolor{currentfill}%
\pgfsetfillopacity{0.700000}%
\pgfsetlinewidth{1.003750pt}%
\definecolor{currentstroke}{rgb}{0.494969,0.578499,0.752214}%
\pgfsetstrokecolor{currentstroke}%
\pgfsetstrokeopacity{0.700000}%
\pgfsetdash{}{0pt}%
\pgfpathmoveto{\pgfqpoint{1.894732in}{1.252813in}}%
\pgfpathcurveto{\pgfqpoint{1.896574in}{1.252813in}}{\pgfqpoint{1.898341in}{1.253545in}}{\pgfqpoint{1.899643in}{1.254847in}}%
\pgfpathcurveto{\pgfqpoint{1.900945in}{1.256149in}}{\pgfqpoint{1.901677in}{1.257916in}}{\pgfqpoint{1.901677in}{1.259757in}}%
\pgfpathcurveto{\pgfqpoint{1.901677in}{1.261599in}}{\pgfqpoint{1.900945in}{1.263366in}}{\pgfqpoint{1.899643in}{1.264668in}}%
\pgfpathcurveto{\pgfqpoint{1.898341in}{1.265970in}}{\pgfqpoint{1.896574in}{1.266702in}}{\pgfqpoint{1.894732in}{1.266702in}}%
\pgfpathcurveto{\pgfqpoint{1.892891in}{1.266702in}}{\pgfqpoint{1.891124in}{1.265970in}}{\pgfqpoint{1.889822in}{1.264668in}}%
\pgfpathcurveto{\pgfqpoint{1.888520in}{1.263366in}}{\pgfqpoint{1.887788in}{1.261599in}}{\pgfqpoint{1.887788in}{1.259757in}}%
\pgfpathcurveto{\pgfqpoint{1.887788in}{1.257916in}}{\pgfqpoint{1.888520in}{1.256149in}}{\pgfqpoint{1.889822in}{1.254847in}}%
\pgfpathcurveto{\pgfqpoint{1.891124in}{1.253545in}}{\pgfqpoint{1.892891in}{1.252813in}}{\pgfqpoint{1.894732in}{1.252813in}}%
\pgfpathlineto{\pgfqpoint{1.894732in}{1.252813in}}%
\pgfpathclose%
\pgfusepath{stroke,fill}%
\end{pgfscope}%
\begin{pgfscope}%
\pgfpathrectangle{\pgfqpoint{0.661006in}{0.524170in}}{\pgfqpoint{4.194036in}{1.071446in}}%
\pgfusepath{clip}%
\pgfsetbuttcap%
\pgfsetroundjoin%
\definecolor{currentfill}{rgb}{0.494969,0.578499,0.752214}%
\pgfsetfillcolor{currentfill}%
\pgfsetfillopacity{0.700000}%
\pgfsetlinewidth{1.003750pt}%
\definecolor{currentstroke}{rgb}{0.494969,0.578499,0.752214}%
\pgfsetstrokecolor{currentstroke}%
\pgfsetstrokeopacity{0.700000}%
\pgfsetdash{}{0pt}%
\pgfpathmoveto{\pgfqpoint{1.904051in}{1.250850in}}%
\pgfpathcurveto{\pgfqpoint{1.905893in}{1.250850in}}{\pgfqpoint{1.907659in}{1.251581in}}{\pgfqpoint{1.908961in}{1.252883in}}%
\pgfpathcurveto{\pgfqpoint{1.910264in}{1.254186in}}{\pgfqpoint{1.910995in}{1.255952in}}{\pgfqpoint{1.910995in}{1.257794in}}%
\pgfpathcurveto{\pgfqpoint{1.910995in}{1.259636in}}{\pgfqpoint{1.910264in}{1.261402in}}{\pgfqpoint{1.908961in}{1.262704in}}%
\pgfpathcurveto{\pgfqpoint{1.907659in}{1.264007in}}{\pgfqpoint{1.905893in}{1.264738in}}{\pgfqpoint{1.904051in}{1.264738in}}%
\pgfpathcurveto{\pgfqpoint{1.902209in}{1.264738in}}{\pgfqpoint{1.900443in}{1.264007in}}{\pgfqpoint{1.899141in}{1.262704in}}%
\pgfpathcurveto{\pgfqpoint{1.897838in}{1.261402in}}{\pgfqpoint{1.897107in}{1.259636in}}{\pgfqpoint{1.897107in}{1.257794in}}%
\pgfpathcurveto{\pgfqpoint{1.897107in}{1.255952in}}{\pgfqpoint{1.897838in}{1.254186in}}{\pgfqpoint{1.899141in}{1.252883in}}%
\pgfpathcurveto{\pgfqpoint{1.900443in}{1.251581in}}{\pgfqpoint{1.902209in}{1.250850in}}{\pgfqpoint{1.904051in}{1.250850in}}%
\pgfpathlineto{\pgfqpoint{1.904051in}{1.250850in}}%
\pgfpathclose%
\pgfusepath{stroke,fill}%
\end{pgfscope}%
\begin{pgfscope}%
\pgfpathrectangle{\pgfqpoint{0.661006in}{0.524170in}}{\pgfqpoint{4.194036in}{1.071446in}}%
\pgfusepath{clip}%
\pgfsetbuttcap%
\pgfsetroundjoin%
\definecolor{currentfill}{rgb}{0.494969,0.578499,0.752214}%
\pgfsetfillcolor{currentfill}%
\pgfsetfillopacity{0.700000}%
\pgfsetlinewidth{1.003750pt}%
\definecolor{currentstroke}{rgb}{0.494969,0.578499,0.752214}%
\pgfsetstrokecolor{currentstroke}%
\pgfsetstrokeopacity{0.700000}%
\pgfsetdash{}{0pt}%
\pgfpathmoveto{\pgfqpoint{1.915089in}{1.248066in}}%
\pgfpathcurveto{\pgfqpoint{1.916931in}{1.248066in}}{\pgfqpoint{1.918698in}{1.248798in}}{\pgfqpoint{1.920000in}{1.250100in}}%
\pgfpathcurveto{\pgfqpoint{1.921302in}{1.251403in}}{\pgfqpoint{1.922034in}{1.253169in}}{\pgfqpoint{1.922034in}{1.255011in}}%
\pgfpathcurveto{\pgfqpoint{1.922034in}{1.256852in}}{\pgfqpoint{1.921302in}{1.258619in}}{\pgfqpoint{1.920000in}{1.259921in}}%
\pgfpathcurveto{\pgfqpoint{1.918698in}{1.261224in}}{\pgfqpoint{1.916931in}{1.261955in}}{\pgfqpoint{1.915089in}{1.261955in}}%
\pgfpathcurveto{\pgfqpoint{1.913248in}{1.261955in}}{\pgfqpoint{1.911481in}{1.261224in}}{\pgfqpoint{1.910179in}{1.259921in}}%
\pgfpathcurveto{\pgfqpoint{1.908877in}{1.258619in}}{\pgfqpoint{1.908145in}{1.256852in}}{\pgfqpoint{1.908145in}{1.255011in}}%
\pgfpathcurveto{\pgfqpoint{1.908145in}{1.253169in}}{\pgfqpoint{1.908877in}{1.251403in}}{\pgfqpoint{1.910179in}{1.250100in}}%
\pgfpathcurveto{\pgfqpoint{1.911481in}{1.248798in}}{\pgfqpoint{1.913248in}{1.248066in}}{\pgfqpoint{1.915089in}{1.248066in}}%
\pgfpathlineto{\pgfqpoint{1.915089in}{1.248066in}}%
\pgfpathclose%
\pgfusepath{stroke,fill}%
\end{pgfscope}%
\begin{pgfscope}%
\pgfpathrectangle{\pgfqpoint{0.661006in}{0.524170in}}{\pgfqpoint{4.194036in}{1.071446in}}%
\pgfusepath{clip}%
\pgfsetbuttcap%
\pgfsetroundjoin%
\definecolor{currentfill}{rgb}{0.491843,0.574017,0.749133}%
\pgfsetfillcolor{currentfill}%
\pgfsetfillopacity{0.700000}%
\pgfsetlinewidth{1.003750pt}%
\definecolor{currentstroke}{rgb}{0.491843,0.574017,0.749133}%
\pgfsetstrokecolor{currentstroke}%
\pgfsetstrokeopacity{0.700000}%
\pgfsetdash{}{0pt}%
\pgfpathmoveto{\pgfqpoint{1.919226in}{1.245856in}}%
\pgfpathcurveto{\pgfqpoint{1.921068in}{1.245856in}}{\pgfqpoint{1.922834in}{1.246588in}}{\pgfqpoint{1.924136in}{1.247890in}}%
\pgfpathcurveto{\pgfqpoint{1.925439in}{1.249192in}}{\pgfqpoint{1.926170in}{1.250959in}}{\pgfqpoint{1.926170in}{1.252801in}}%
\pgfpathcurveto{\pgfqpoint{1.926170in}{1.254642in}}{\pgfqpoint{1.925439in}{1.256409in}}{\pgfqpoint{1.924136in}{1.257711in}}%
\pgfpathcurveto{\pgfqpoint{1.922834in}{1.259013in}}{\pgfqpoint{1.921068in}{1.259745in}}{\pgfqpoint{1.919226in}{1.259745in}}%
\pgfpathcurveto{\pgfqpoint{1.917384in}{1.259745in}}{\pgfqpoint{1.915618in}{1.259013in}}{\pgfqpoint{1.914315in}{1.257711in}}%
\pgfpathcurveto{\pgfqpoint{1.913013in}{1.256409in}}{\pgfqpoint{1.912281in}{1.254642in}}{\pgfqpoint{1.912281in}{1.252801in}}%
\pgfpathcurveto{\pgfqpoint{1.912281in}{1.250959in}}{\pgfqpoint{1.913013in}{1.249192in}}{\pgfqpoint{1.914315in}{1.247890in}}%
\pgfpathcurveto{\pgfqpoint{1.915618in}{1.246588in}}{\pgfqpoint{1.917384in}{1.245856in}}{\pgfqpoint{1.919226in}{1.245856in}}%
\pgfpathlineto{\pgfqpoint{1.919226in}{1.245856in}}%
\pgfpathclose%
\pgfusepath{stroke,fill}%
\end{pgfscope}%
\begin{pgfscope}%
\pgfpathrectangle{\pgfqpoint{0.661006in}{0.524170in}}{\pgfqpoint{4.194036in}{1.071446in}}%
\pgfusepath{clip}%
\pgfsetbuttcap%
\pgfsetroundjoin%
\definecolor{currentfill}{rgb}{0.491843,0.574017,0.749133}%
\pgfsetfillcolor{currentfill}%
\pgfsetfillopacity{0.700000}%
\pgfsetlinewidth{1.003750pt}%
\definecolor{currentstroke}{rgb}{0.491843,0.574017,0.749133}%
\pgfsetstrokecolor{currentstroke}%
\pgfsetstrokeopacity{0.700000}%
\pgfsetdash{}{0pt}%
\pgfpathmoveto{\pgfqpoint{1.938560in}{1.241222in}}%
\pgfpathcurveto{\pgfqpoint{1.940402in}{1.241222in}}{\pgfqpoint{1.942169in}{1.241953in}}{\pgfqpoint{1.943471in}{1.243256in}}%
\pgfpathcurveto{\pgfqpoint{1.944773in}{1.244558in}}{\pgfqpoint{1.945505in}{1.246324in}}{\pgfqpoint{1.945505in}{1.248166in}}%
\pgfpathcurveto{\pgfqpoint{1.945505in}{1.250008in}}{\pgfqpoint{1.944773in}{1.251774in}}{\pgfqpoint{1.943471in}{1.253076in}}%
\pgfpathcurveto{\pgfqpoint{1.942169in}{1.254379in}}{\pgfqpoint{1.940402in}{1.255110in}}{\pgfqpoint{1.938560in}{1.255110in}}%
\pgfpathcurveto{\pgfqpoint{1.936719in}{1.255110in}}{\pgfqpoint{1.934952in}{1.254379in}}{\pgfqpoint{1.933650in}{1.253076in}}%
\pgfpathcurveto{\pgfqpoint{1.932348in}{1.251774in}}{\pgfqpoint{1.931616in}{1.250008in}}{\pgfqpoint{1.931616in}{1.248166in}}%
\pgfpathcurveto{\pgfqpoint{1.931616in}{1.246324in}}{\pgfqpoint{1.932348in}{1.244558in}}{\pgfqpoint{1.933650in}{1.243256in}}%
\pgfpathcurveto{\pgfqpoint{1.934952in}{1.241953in}}{\pgfqpoint{1.936719in}{1.241222in}}{\pgfqpoint{1.938560in}{1.241222in}}%
\pgfpathlineto{\pgfqpoint{1.938560in}{1.241222in}}%
\pgfpathclose%
\pgfusepath{stroke,fill}%
\end{pgfscope}%
\begin{pgfscope}%
\pgfpathrectangle{\pgfqpoint{0.661006in}{0.524170in}}{\pgfqpoint{4.194036in}{1.071446in}}%
\pgfusepath{clip}%
\pgfsetbuttcap%
\pgfsetroundjoin%
\definecolor{currentfill}{rgb}{0.491843,0.574017,0.749133}%
\pgfsetfillcolor{currentfill}%
\pgfsetfillopacity{0.700000}%
\pgfsetlinewidth{1.003750pt}%
\definecolor{currentstroke}{rgb}{0.491843,0.574017,0.749133}%
\pgfsetstrokecolor{currentstroke}%
\pgfsetstrokeopacity{0.700000}%
\pgfsetdash{}{0pt}%
\pgfpathmoveto{\pgfqpoint{1.971606in}{1.235242in}}%
\pgfpathcurveto{\pgfqpoint{1.973447in}{1.235242in}}{\pgfqpoint{1.975214in}{1.235974in}}{\pgfqpoint{1.976516in}{1.237276in}}%
\pgfpathcurveto{\pgfqpoint{1.977818in}{1.238578in}}{\pgfqpoint{1.978550in}{1.240345in}}{\pgfqpoint{1.978550in}{1.242187in}}%
\pgfpathcurveto{\pgfqpoint{1.978550in}{1.244028in}}{\pgfqpoint{1.977818in}{1.245795in}}{\pgfqpoint{1.976516in}{1.247097in}}%
\pgfpathcurveto{\pgfqpoint{1.975214in}{1.248399in}}{\pgfqpoint{1.973447in}{1.249131in}}{\pgfqpoint{1.971606in}{1.249131in}}%
\pgfpathcurveto{\pgfqpoint{1.969764in}{1.249131in}}{\pgfqpoint{1.967997in}{1.248399in}}{\pgfqpoint{1.966695in}{1.247097in}}%
\pgfpathcurveto{\pgfqpoint{1.965393in}{1.245795in}}{\pgfqpoint{1.964661in}{1.244028in}}{\pgfqpoint{1.964661in}{1.242187in}}%
\pgfpathcurveto{\pgfqpoint{1.964661in}{1.240345in}}{\pgfqpoint{1.965393in}{1.238578in}}{\pgfqpoint{1.966695in}{1.237276in}}%
\pgfpathcurveto{\pgfqpoint{1.967997in}{1.235974in}}{\pgfqpoint{1.969764in}{1.235242in}}{\pgfqpoint{1.971606in}{1.235242in}}%
\pgfpathlineto{\pgfqpoint{1.971606in}{1.235242in}}%
\pgfpathclose%
\pgfusepath{stroke,fill}%
\end{pgfscope}%
\begin{pgfscope}%
\pgfpathrectangle{\pgfqpoint{0.661006in}{0.524170in}}{\pgfqpoint{4.194036in}{1.071446in}}%
\pgfusepath{clip}%
\pgfsetbuttcap%
\pgfsetroundjoin%
\definecolor{currentfill}{rgb}{0.491843,0.574017,0.749133}%
\pgfsetfillcolor{currentfill}%
\pgfsetfillopacity{0.700000}%
\pgfsetlinewidth{1.003750pt}%
\definecolor{currentstroke}{rgb}{0.491843,0.574017,0.749133}%
\pgfsetstrokecolor{currentstroke}%
\pgfsetstrokeopacity{0.700000}%
\pgfsetdash{}{0pt}%
\pgfpathmoveto{\pgfqpoint{1.981784in}{1.230898in}}%
\pgfpathcurveto{\pgfqpoint{1.983626in}{1.230898in}}{\pgfqpoint{1.985392in}{1.231630in}}{\pgfqpoint{1.986695in}{1.232932in}}%
\pgfpathcurveto{\pgfqpoint{1.987997in}{1.234235in}}{\pgfqpoint{1.988729in}{1.236001in}}{\pgfqpoint{1.988729in}{1.237843in}}%
\pgfpathcurveto{\pgfqpoint{1.988729in}{1.239684in}}{\pgfqpoint{1.987997in}{1.241451in}}{\pgfqpoint{1.986695in}{1.242753in}}%
\pgfpathcurveto{\pgfqpoint{1.985392in}{1.244055in}}{\pgfqpoint{1.983626in}{1.244787in}}{\pgfqpoint{1.981784in}{1.244787in}}%
\pgfpathcurveto{\pgfqpoint{1.979943in}{1.244787in}}{\pgfqpoint{1.978176in}{1.244055in}}{\pgfqpoint{1.976874in}{1.242753in}}%
\pgfpathcurveto{\pgfqpoint{1.975571in}{1.241451in}}{\pgfqpoint{1.974840in}{1.239684in}}{\pgfqpoint{1.974840in}{1.237843in}}%
\pgfpathcurveto{\pgfqpoint{1.974840in}{1.236001in}}{\pgfqpoint{1.975571in}{1.234235in}}{\pgfqpoint{1.976874in}{1.232932in}}%
\pgfpathcurveto{\pgfqpoint{1.978176in}{1.231630in}}{\pgfqpoint{1.979943in}{1.230898in}}{\pgfqpoint{1.981784in}{1.230898in}}%
\pgfpathlineto{\pgfqpoint{1.981784in}{1.230898in}}%
\pgfpathclose%
\pgfusepath{stroke,fill}%
\end{pgfscope}%
\begin{pgfscope}%
\pgfpathrectangle{\pgfqpoint{0.661006in}{0.524170in}}{\pgfqpoint{4.194036in}{1.071446in}}%
\pgfusepath{clip}%
\pgfsetbuttcap%
\pgfsetroundjoin%
\definecolor{currentfill}{rgb}{0.488734,0.569531,0.746024}%
\pgfsetfillcolor{currentfill}%
\pgfsetfillopacity{0.700000}%
\pgfsetlinewidth{1.003750pt}%
\definecolor{currentstroke}{rgb}{0.488734,0.569531,0.746024}%
\pgfsetstrokecolor{currentstroke}%
\pgfsetstrokeopacity{0.700000}%
\pgfsetdash{}{0pt}%
\pgfpathmoveto{\pgfqpoint{2.015434in}{1.224575in}}%
\pgfpathcurveto{\pgfqpoint{2.017275in}{1.224575in}}{\pgfqpoint{2.019042in}{1.225307in}}{\pgfqpoint{2.020344in}{1.226609in}}%
\pgfpathcurveto{\pgfqpoint{2.021646in}{1.227911in}}{\pgfqpoint{2.022378in}{1.229678in}}{\pgfqpoint{2.022378in}{1.231520in}}%
\pgfpathcurveto{\pgfqpoint{2.022378in}{1.233361in}}{\pgfqpoint{2.021646in}{1.235128in}}{\pgfqpoint{2.020344in}{1.236430in}}%
\pgfpathcurveto{\pgfqpoint{2.019042in}{1.237732in}}{\pgfqpoint{2.017275in}{1.238464in}}{\pgfqpoint{2.015434in}{1.238464in}}%
\pgfpathcurveto{\pgfqpoint{2.013592in}{1.238464in}}{\pgfqpoint{2.011826in}{1.237732in}}{\pgfqpoint{2.010523in}{1.236430in}}%
\pgfpathcurveto{\pgfqpoint{2.009221in}{1.235128in}}{\pgfqpoint{2.008489in}{1.233361in}}{\pgfqpoint{2.008489in}{1.231520in}}%
\pgfpathcurveto{\pgfqpoint{2.008489in}{1.229678in}}{\pgfqpoint{2.009221in}{1.227911in}}{\pgfqpoint{2.010523in}{1.226609in}}%
\pgfpathcurveto{\pgfqpoint{2.011826in}{1.225307in}}{\pgfqpoint{2.013592in}{1.224575in}}{\pgfqpoint{2.015434in}{1.224575in}}%
\pgfpathlineto{\pgfqpoint{2.015434in}{1.224575in}}%
\pgfpathclose%
\pgfusepath{stroke,fill}%
\end{pgfscope}%
\begin{pgfscope}%
\pgfpathrectangle{\pgfqpoint{0.661006in}{0.524170in}}{\pgfqpoint{4.194036in}{1.071446in}}%
\pgfusepath{clip}%
\pgfsetbuttcap%
\pgfsetroundjoin%
\definecolor{currentfill}{rgb}{0.488734,0.569531,0.746024}%
\pgfsetfillcolor{currentfill}%
\pgfsetfillopacity{0.700000}%
\pgfsetlinewidth{1.003750pt}%
\definecolor{currentstroke}{rgb}{0.488734,0.569531,0.746024}%
\pgfsetstrokecolor{currentstroke}%
\pgfsetstrokeopacity{0.700000}%
\pgfsetdash{}{0pt}%
\pgfpathmoveto{\pgfqpoint{2.024636in}{1.220851in}}%
\pgfpathcurveto{\pgfqpoint{2.026478in}{1.220851in}}{\pgfqpoint{2.028244in}{1.221582in}}{\pgfqpoint{2.029547in}{1.222885in}}%
\pgfpathcurveto{\pgfqpoint{2.030849in}{1.224187in}}{\pgfqpoint{2.031581in}{1.225953in}}{\pgfqpoint{2.031581in}{1.227795in}}%
\pgfpathcurveto{\pgfqpoint{2.031581in}{1.229637in}}{\pgfqpoint{2.030849in}{1.231403in}}{\pgfqpoint{2.029547in}{1.232705in}}%
\pgfpathcurveto{\pgfqpoint{2.028244in}{1.234008in}}{\pgfqpoint{2.026478in}{1.234739in}}{\pgfqpoint{2.024636in}{1.234739in}}%
\pgfpathcurveto{\pgfqpoint{2.022795in}{1.234739in}}{\pgfqpoint{2.021028in}{1.234008in}}{\pgfqpoint{2.019726in}{1.232705in}}%
\pgfpathcurveto{\pgfqpoint{2.018423in}{1.231403in}}{\pgfqpoint{2.017692in}{1.229637in}}{\pgfqpoint{2.017692in}{1.227795in}}%
\pgfpathcurveto{\pgfqpoint{2.017692in}{1.225953in}}{\pgfqpoint{2.018423in}{1.224187in}}{\pgfqpoint{2.019726in}{1.222885in}}%
\pgfpathcurveto{\pgfqpoint{2.021028in}{1.221582in}}{\pgfqpoint{2.022795in}{1.220851in}}{\pgfqpoint{2.024636in}{1.220851in}}%
\pgfpathlineto{\pgfqpoint{2.024636in}{1.220851in}}%
\pgfpathclose%
\pgfusepath{stroke,fill}%
\end{pgfscope}%
\begin{pgfscope}%
\pgfpathrectangle{\pgfqpoint{0.661006in}{0.524170in}}{\pgfqpoint{4.194036in}{1.071446in}}%
\pgfusepath{clip}%
\pgfsetbuttcap%
\pgfsetroundjoin%
\definecolor{currentfill}{rgb}{0.488734,0.569531,0.746024}%
\pgfsetfillcolor{currentfill}%
\pgfsetfillopacity{0.700000}%
\pgfsetlinewidth{1.003750pt}%
\definecolor{currentstroke}{rgb}{0.488734,0.569531,0.746024}%
\pgfsetstrokecolor{currentstroke}%
\pgfsetstrokeopacity{0.700000}%
\pgfsetdash{}{0pt}%
\pgfpathmoveto{\pgfqpoint{2.023381in}{1.222025in}}%
\pgfpathcurveto{\pgfqpoint{2.025223in}{1.222025in}}{\pgfqpoint{2.026990in}{1.222757in}}{\pgfqpoint{2.028292in}{1.224059in}}%
\pgfpathcurveto{\pgfqpoint{2.029594in}{1.225361in}}{\pgfqpoint{2.030326in}{1.227128in}}{\pgfqpoint{2.030326in}{1.228970in}}%
\pgfpathcurveto{\pgfqpoint{2.030326in}{1.230811in}}{\pgfqpoint{2.029594in}{1.232578in}}{\pgfqpoint{2.028292in}{1.233880in}}%
\pgfpathcurveto{\pgfqpoint{2.026990in}{1.235182in}}{\pgfqpoint{2.025223in}{1.235914in}}{\pgfqpoint{2.023381in}{1.235914in}}%
\pgfpathcurveto{\pgfqpoint{2.021540in}{1.235914in}}{\pgfqpoint{2.019773in}{1.235182in}}{\pgfqpoint{2.018471in}{1.233880in}}%
\pgfpathcurveto{\pgfqpoint{2.017169in}{1.232578in}}{\pgfqpoint{2.016437in}{1.230811in}}{\pgfqpoint{2.016437in}{1.228970in}}%
\pgfpathcurveto{\pgfqpoint{2.016437in}{1.227128in}}{\pgfqpoint{2.017169in}{1.225361in}}{\pgfqpoint{2.018471in}{1.224059in}}%
\pgfpathcurveto{\pgfqpoint{2.019773in}{1.222757in}}{\pgfqpoint{2.021540in}{1.222025in}}{\pgfqpoint{2.023381in}{1.222025in}}%
\pgfpathlineto{\pgfqpoint{2.023381in}{1.222025in}}%
\pgfpathclose%
\pgfusepath{stroke,fill}%
\end{pgfscope}%
\begin{pgfscope}%
\pgfpathrectangle{\pgfqpoint{0.661006in}{0.524170in}}{\pgfqpoint{4.194036in}{1.071446in}}%
\pgfusepath{clip}%
\pgfsetbuttcap%
\pgfsetroundjoin%
\definecolor{currentfill}{rgb}{0.485641,0.565041,0.742888}%
\pgfsetfillcolor{currentfill}%
\pgfsetfillopacity{0.700000}%
\pgfsetlinewidth{1.003750pt}%
\definecolor{currentstroke}{rgb}{0.485641,0.565041,0.742888}%
\pgfsetstrokecolor{currentstroke}%
\pgfsetstrokeopacity{0.700000}%
\pgfsetdash{}{0pt}%
\pgfpathmoveto{\pgfqpoint{2.013203in}{1.226553in}}%
\pgfpathcurveto{\pgfqpoint{2.015044in}{1.226553in}}{\pgfqpoint{2.016811in}{1.227285in}}{\pgfqpoint{2.018113in}{1.228587in}}%
\pgfpathcurveto{\pgfqpoint{2.019416in}{1.229889in}}{\pgfqpoint{2.020147in}{1.231656in}}{\pgfqpoint{2.020147in}{1.233498in}}%
\pgfpathcurveto{\pgfqpoint{2.020147in}{1.235339in}}{\pgfqpoint{2.019416in}{1.237106in}}{\pgfqpoint{2.018113in}{1.238408in}}%
\pgfpathcurveto{\pgfqpoint{2.016811in}{1.239710in}}{\pgfqpoint{2.015044in}{1.240442in}}{\pgfqpoint{2.013203in}{1.240442in}}%
\pgfpathcurveto{\pgfqpoint{2.011361in}{1.240442in}}{\pgfqpoint{2.009595in}{1.239710in}}{\pgfqpoint{2.008292in}{1.238408in}}%
\pgfpathcurveto{\pgfqpoint{2.006990in}{1.237106in}}{\pgfqpoint{2.006258in}{1.235339in}}{\pgfqpoint{2.006258in}{1.233498in}}%
\pgfpathcurveto{\pgfqpoint{2.006258in}{1.231656in}}{\pgfqpoint{2.006990in}{1.229889in}}{\pgfqpoint{2.008292in}{1.228587in}}%
\pgfpathcurveto{\pgfqpoint{2.009595in}{1.227285in}}{\pgfqpoint{2.011361in}{1.226553in}}{\pgfqpoint{2.013203in}{1.226553in}}%
\pgfpathlineto{\pgfqpoint{2.013203in}{1.226553in}}%
\pgfpathclose%
\pgfusepath{stroke,fill}%
\end{pgfscope}%
\begin{pgfscope}%
\pgfpathrectangle{\pgfqpoint{0.661006in}{0.524170in}}{\pgfqpoint{4.194036in}{1.071446in}}%
\pgfusepath{clip}%
\pgfsetbuttcap%
\pgfsetroundjoin%
\definecolor{currentfill}{rgb}{0.485641,0.565041,0.742888}%
\pgfsetfillcolor{currentfill}%
\pgfsetfillopacity{0.700000}%
\pgfsetlinewidth{1.003750pt}%
\definecolor{currentstroke}{rgb}{0.485641,0.565041,0.742888}%
\pgfsetstrokecolor{currentstroke}%
\pgfsetstrokeopacity{0.700000}%
\pgfsetdash{}{0pt}%
\pgfpathmoveto{\pgfqpoint{2.008509in}{1.227126in}}%
\pgfpathcurveto{\pgfqpoint{2.010350in}{1.227126in}}{\pgfqpoint{2.012117in}{1.227857in}}{\pgfqpoint{2.013419in}{1.229160in}}%
\pgfpathcurveto{\pgfqpoint{2.014721in}{1.230462in}}{\pgfqpoint{2.015453in}{1.232229in}}{\pgfqpoint{2.015453in}{1.234070in}}%
\pgfpathcurveto{\pgfqpoint{2.015453in}{1.235912in}}{\pgfqpoint{2.014721in}{1.237678in}}{\pgfqpoint{2.013419in}{1.238981in}}%
\pgfpathcurveto{\pgfqpoint{2.012117in}{1.240283in}}{\pgfqpoint{2.010350in}{1.241015in}}{\pgfqpoint{2.008509in}{1.241015in}}%
\pgfpathcurveto{\pgfqpoint{2.006667in}{1.241015in}}{\pgfqpoint{2.004900in}{1.240283in}}{\pgfqpoint{2.003598in}{1.238981in}}%
\pgfpathcurveto{\pgfqpoint{2.002296in}{1.237678in}}{\pgfqpoint{2.001564in}{1.235912in}}{\pgfqpoint{2.001564in}{1.234070in}}%
\pgfpathcurveto{\pgfqpoint{2.001564in}{1.232229in}}{\pgfqpoint{2.002296in}{1.230462in}}{\pgfqpoint{2.003598in}{1.229160in}}%
\pgfpathcurveto{\pgfqpoint{2.004900in}{1.227857in}}{\pgfqpoint{2.006667in}{1.227126in}}{\pgfqpoint{2.008509in}{1.227126in}}%
\pgfpathlineto{\pgfqpoint{2.008509in}{1.227126in}}%
\pgfpathclose%
\pgfusepath{stroke,fill}%
\end{pgfscope}%
\begin{pgfscope}%
\pgfpathrectangle{\pgfqpoint{0.661006in}{0.524170in}}{\pgfqpoint{4.194036in}{1.071446in}}%
\pgfusepath{clip}%
\pgfsetbuttcap%
\pgfsetroundjoin%
\definecolor{currentfill}{rgb}{0.485641,0.565041,0.742888}%
\pgfsetfillcolor{currentfill}%
\pgfsetfillopacity{0.700000}%
\pgfsetlinewidth{1.003750pt}%
\definecolor{currentstroke}{rgb}{0.485641,0.565041,0.742888}%
\pgfsetstrokecolor{currentstroke}%
\pgfsetstrokeopacity{0.700000}%
\pgfsetdash{}{0pt}%
\pgfpathmoveto{\pgfqpoint{2.023893in}{1.222685in}}%
\pgfpathcurveto{\pgfqpoint{2.025734in}{1.222685in}}{\pgfqpoint{2.027501in}{1.223417in}}{\pgfqpoint{2.028803in}{1.224719in}}%
\pgfpathcurveto{\pgfqpoint{2.030105in}{1.226021in}}{\pgfqpoint{2.030837in}{1.227788in}}{\pgfqpoint{2.030837in}{1.229629in}}%
\pgfpathcurveto{\pgfqpoint{2.030837in}{1.231471in}}{\pgfqpoint{2.030105in}{1.233238in}}{\pgfqpoint{2.028803in}{1.234540in}}%
\pgfpathcurveto{\pgfqpoint{2.027501in}{1.235842in}}{\pgfqpoint{2.025734in}{1.236574in}}{\pgfqpoint{2.023893in}{1.236574in}}%
\pgfpathcurveto{\pgfqpoint{2.022051in}{1.236574in}}{\pgfqpoint{2.020284in}{1.235842in}}{\pgfqpoint{2.018982in}{1.234540in}}%
\pgfpathcurveto{\pgfqpoint{2.017680in}{1.233238in}}{\pgfqpoint{2.016948in}{1.231471in}}{\pgfqpoint{2.016948in}{1.229629in}}%
\pgfpathcurveto{\pgfqpoint{2.016948in}{1.227788in}}{\pgfqpoint{2.017680in}{1.226021in}}{\pgfqpoint{2.018982in}{1.224719in}}%
\pgfpathcurveto{\pgfqpoint{2.020284in}{1.223417in}}{\pgfqpoint{2.022051in}{1.222685in}}{\pgfqpoint{2.023893in}{1.222685in}}%
\pgfpathlineto{\pgfqpoint{2.023893in}{1.222685in}}%
\pgfpathclose%
\pgfusepath{stroke,fill}%
\end{pgfscope}%
\begin{pgfscope}%
\pgfpathrectangle{\pgfqpoint{0.661006in}{0.524170in}}{\pgfqpoint{4.194036in}{1.071446in}}%
\pgfusepath{clip}%
\pgfsetbuttcap%
\pgfsetroundjoin%
\definecolor{currentfill}{rgb}{0.485641,0.565041,0.742888}%
\pgfsetfillcolor{currentfill}%
\pgfsetfillopacity{0.700000}%
\pgfsetlinewidth{1.003750pt}%
\definecolor{currentstroke}{rgb}{0.485641,0.565041,0.742888}%
\pgfsetstrokecolor{currentstroke}%
\pgfsetstrokeopacity{0.700000}%
\pgfsetdash{}{0pt}%
\pgfpathmoveto{\pgfqpoint{2.050059in}{1.217483in}}%
\pgfpathcurveto{\pgfqpoint{2.051901in}{1.217483in}}{\pgfqpoint{2.053667in}{1.218215in}}{\pgfqpoint{2.054970in}{1.219517in}}%
\pgfpathcurveto{\pgfqpoint{2.056272in}{1.220820in}}{\pgfqpoint{2.057004in}{1.222586in}}{\pgfqpoint{2.057004in}{1.224428in}}%
\pgfpathcurveto{\pgfqpoint{2.057004in}{1.226269in}}{\pgfqpoint{2.056272in}{1.228036in}}{\pgfqpoint{2.054970in}{1.229338in}}%
\pgfpathcurveto{\pgfqpoint{2.053667in}{1.230641in}}{\pgfqpoint{2.051901in}{1.231372in}}{\pgfqpoint{2.050059in}{1.231372in}}%
\pgfpathcurveto{\pgfqpoint{2.048218in}{1.231372in}}{\pgfqpoint{2.046451in}{1.230641in}}{\pgfqpoint{2.045149in}{1.229338in}}%
\pgfpathcurveto{\pgfqpoint{2.043847in}{1.228036in}}{\pgfqpoint{2.043115in}{1.226269in}}{\pgfqpoint{2.043115in}{1.224428in}}%
\pgfpathcurveto{\pgfqpoint{2.043115in}{1.222586in}}{\pgfqpoint{2.043847in}{1.220820in}}{\pgfqpoint{2.045149in}{1.219517in}}%
\pgfpathcurveto{\pgfqpoint{2.046451in}{1.218215in}}{\pgfqpoint{2.048218in}{1.217483in}}{\pgfqpoint{2.050059in}{1.217483in}}%
\pgfpathlineto{\pgfqpoint{2.050059in}{1.217483in}}%
\pgfpathclose%
\pgfusepath{stroke,fill}%
\end{pgfscope}%
\begin{pgfscope}%
\pgfpathrectangle{\pgfqpoint{0.661006in}{0.524170in}}{\pgfqpoint{4.194036in}{1.071446in}}%
\pgfusepath{clip}%
\pgfsetbuttcap%
\pgfsetroundjoin%
\definecolor{currentfill}{rgb}{0.482564,0.560549,0.739723}%
\pgfsetfillcolor{currentfill}%
\pgfsetfillopacity{0.700000}%
\pgfsetlinewidth{1.003750pt}%
\definecolor{currentstroke}{rgb}{0.482564,0.560549,0.739723}%
\pgfsetstrokecolor{currentstroke}%
\pgfsetstrokeopacity{0.700000}%
\pgfsetdash{}{0pt}%
\pgfpathmoveto{\pgfqpoint{2.059893in}{1.214203in}}%
\pgfpathcurveto{\pgfqpoint{2.061734in}{1.214203in}}{\pgfqpoint{2.063501in}{1.214935in}}{\pgfqpoint{2.064803in}{1.216237in}}%
\pgfpathcurveto{\pgfqpoint{2.066105in}{1.217539in}}{\pgfqpoint{2.066837in}{1.219306in}}{\pgfqpoint{2.066837in}{1.221147in}}%
\pgfpathcurveto{\pgfqpoint{2.066837in}{1.222989in}}{\pgfqpoint{2.066105in}{1.224756in}}{\pgfqpoint{2.064803in}{1.226058in}}%
\pgfpathcurveto{\pgfqpoint{2.063501in}{1.227360in}}{\pgfqpoint{2.061734in}{1.228092in}}{\pgfqpoint{2.059893in}{1.228092in}}%
\pgfpathcurveto{\pgfqpoint{2.058051in}{1.228092in}}{\pgfqpoint{2.056284in}{1.227360in}}{\pgfqpoint{2.054982in}{1.226058in}}%
\pgfpathcurveto{\pgfqpoint{2.053680in}{1.224756in}}{\pgfqpoint{2.052948in}{1.222989in}}{\pgfqpoint{2.052948in}{1.221147in}}%
\pgfpathcurveto{\pgfqpoint{2.052948in}{1.219306in}}{\pgfqpoint{2.053680in}{1.217539in}}{\pgfqpoint{2.054982in}{1.216237in}}%
\pgfpathcurveto{\pgfqpoint{2.056284in}{1.214935in}}{\pgfqpoint{2.058051in}{1.214203in}}{\pgfqpoint{2.059893in}{1.214203in}}%
\pgfpathlineto{\pgfqpoint{2.059893in}{1.214203in}}%
\pgfpathclose%
\pgfusepath{stroke,fill}%
\end{pgfscope}%
\begin{pgfscope}%
\pgfpathrectangle{\pgfqpoint{0.661006in}{0.524170in}}{\pgfqpoint{4.194036in}{1.071446in}}%
\pgfusepath{clip}%
\pgfsetbuttcap%
\pgfsetroundjoin%
\definecolor{currentfill}{rgb}{0.482564,0.560549,0.739723}%
\pgfsetfillcolor{currentfill}%
\pgfsetfillopacity{0.700000}%
\pgfsetlinewidth{1.003750pt}%
\definecolor{currentstroke}{rgb}{0.482564,0.560549,0.739723}%
\pgfsetstrokecolor{currentstroke}%
\pgfsetstrokeopacity{0.700000}%
\pgfsetdash{}{0pt}%
\pgfpathmoveto{\pgfqpoint{2.074832in}{1.212834in}}%
\pgfpathcurveto{\pgfqpoint{2.076673in}{1.212834in}}{\pgfqpoint{2.078440in}{1.213566in}}{\pgfqpoint{2.079742in}{1.214868in}}%
\pgfpathcurveto{\pgfqpoint{2.081044in}{1.216171in}}{\pgfqpoint{2.081776in}{1.217937in}}{\pgfqpoint{2.081776in}{1.219779in}}%
\pgfpathcurveto{\pgfqpoint{2.081776in}{1.221621in}}{\pgfqpoint{2.081044in}{1.223387in}}{\pgfqpoint{2.079742in}{1.224689in}}%
\pgfpathcurveto{\pgfqpoint{2.078440in}{1.225992in}}{\pgfqpoint{2.076673in}{1.226723in}}{\pgfqpoint{2.074832in}{1.226723in}}%
\pgfpathcurveto{\pgfqpoint{2.072990in}{1.226723in}}{\pgfqpoint{2.071223in}{1.225992in}}{\pgfqpoint{2.069921in}{1.224689in}}%
\pgfpathcurveto{\pgfqpoint{2.068619in}{1.223387in}}{\pgfqpoint{2.067887in}{1.221621in}}{\pgfqpoint{2.067887in}{1.219779in}}%
\pgfpathcurveto{\pgfqpoint{2.067887in}{1.217937in}}{\pgfqpoint{2.068619in}{1.216171in}}{\pgfqpoint{2.069921in}{1.214868in}}%
\pgfpathcurveto{\pgfqpoint{2.071223in}{1.213566in}}{\pgfqpoint{2.072990in}{1.212834in}}{\pgfqpoint{2.074832in}{1.212834in}}%
\pgfpathlineto{\pgfqpoint{2.074832in}{1.212834in}}%
\pgfpathclose%
\pgfusepath{stroke,fill}%
\end{pgfscope}%
\begin{pgfscope}%
\pgfpathrectangle{\pgfqpoint{0.661006in}{0.524170in}}{\pgfqpoint{4.194036in}{1.071446in}}%
\pgfusepath{clip}%
\pgfsetbuttcap%
\pgfsetroundjoin%
\definecolor{currentfill}{rgb}{0.482564,0.560549,0.739723}%
\pgfsetfillcolor{currentfill}%
\pgfsetfillopacity{0.700000}%
\pgfsetlinewidth{1.003750pt}%
\definecolor{currentstroke}{rgb}{0.482564,0.560549,0.739723}%
\pgfsetstrokecolor{currentstroke}%
\pgfsetstrokeopacity{0.700000}%
\pgfsetdash{}{0pt}%
\pgfpathmoveto{\pgfqpoint{2.076458in}{1.212686in}}%
\pgfpathcurveto{\pgfqpoint{2.078300in}{1.212686in}}{\pgfqpoint{2.080067in}{1.213418in}}{\pgfqpoint{2.081369in}{1.214720in}}%
\pgfpathcurveto{\pgfqpoint{2.082671in}{1.216022in}}{\pgfqpoint{2.083403in}{1.217789in}}{\pgfqpoint{2.083403in}{1.219631in}}%
\pgfpathcurveto{\pgfqpoint{2.083403in}{1.221472in}}{\pgfqpoint{2.082671in}{1.223239in}}{\pgfqpoint{2.081369in}{1.224541in}}%
\pgfpathcurveto{\pgfqpoint{2.080067in}{1.225843in}}{\pgfqpoint{2.078300in}{1.226575in}}{\pgfqpoint{2.076458in}{1.226575in}}%
\pgfpathcurveto{\pgfqpoint{2.074617in}{1.226575in}}{\pgfqpoint{2.072850in}{1.225843in}}{\pgfqpoint{2.071548in}{1.224541in}}%
\pgfpathcurveto{\pgfqpoint{2.070246in}{1.223239in}}{\pgfqpoint{2.069514in}{1.221472in}}{\pgfqpoint{2.069514in}{1.219631in}}%
\pgfpathcurveto{\pgfqpoint{2.069514in}{1.217789in}}{\pgfqpoint{2.070246in}{1.216022in}}{\pgfqpoint{2.071548in}{1.214720in}}%
\pgfpathcurveto{\pgfqpoint{2.072850in}{1.213418in}}{\pgfqpoint{2.074617in}{1.212686in}}{\pgfqpoint{2.076458in}{1.212686in}}%
\pgfpathlineto{\pgfqpoint{2.076458in}{1.212686in}}%
\pgfpathclose%
\pgfusepath{stroke,fill}%
\end{pgfscope}%
\begin{pgfscope}%
\pgfpathrectangle{\pgfqpoint{0.661006in}{0.524170in}}{\pgfqpoint{4.194036in}{1.071446in}}%
\pgfusepath{clip}%
\pgfsetbuttcap%
\pgfsetroundjoin%
\definecolor{currentfill}{rgb}{0.479502,0.556054,0.736531}%
\pgfsetfillcolor{currentfill}%
\pgfsetfillopacity{0.700000}%
\pgfsetlinewidth{1.003750pt}%
\definecolor{currentstroke}{rgb}{0.479502,0.556054,0.736531}%
\pgfsetstrokecolor{currentstroke}%
\pgfsetstrokeopacity{0.700000}%
\pgfsetdash{}{0pt}%
\pgfpathmoveto{\pgfqpoint{2.066884in}{1.215300in}}%
\pgfpathcurveto{\pgfqpoint{2.068726in}{1.215300in}}{\pgfqpoint{2.070492in}{1.216032in}}{\pgfqpoint{2.071794in}{1.217334in}}%
\pgfpathcurveto{\pgfqpoint{2.073097in}{1.218637in}}{\pgfqpoint{2.073828in}{1.220403in}}{\pgfqpoint{2.073828in}{1.222245in}}%
\pgfpathcurveto{\pgfqpoint{2.073828in}{1.224087in}}{\pgfqpoint{2.073097in}{1.225853in}}{\pgfqpoint{2.071794in}{1.227155in}}%
\pgfpathcurveto{\pgfqpoint{2.070492in}{1.228458in}}{\pgfqpoint{2.068726in}{1.229189in}}{\pgfqpoint{2.066884in}{1.229189in}}%
\pgfpathcurveto{\pgfqpoint{2.065042in}{1.229189in}}{\pgfqpoint{2.063276in}{1.228458in}}{\pgfqpoint{2.061974in}{1.227155in}}%
\pgfpathcurveto{\pgfqpoint{2.060671in}{1.225853in}}{\pgfqpoint{2.059940in}{1.224087in}}{\pgfqpoint{2.059940in}{1.222245in}}%
\pgfpathcurveto{\pgfqpoint{2.059940in}{1.220403in}}{\pgfqpoint{2.060671in}{1.218637in}}{\pgfqpoint{2.061974in}{1.217334in}}%
\pgfpathcurveto{\pgfqpoint{2.063276in}{1.216032in}}{\pgfqpoint{2.065042in}{1.215300in}}{\pgfqpoint{2.066884in}{1.215300in}}%
\pgfpathlineto{\pgfqpoint{2.066884in}{1.215300in}}%
\pgfpathclose%
\pgfusepath{stroke,fill}%
\end{pgfscope}%
\begin{pgfscope}%
\pgfpathrectangle{\pgfqpoint{0.661006in}{0.524170in}}{\pgfqpoint{4.194036in}{1.071446in}}%
\pgfusepath{clip}%
\pgfsetbuttcap%
\pgfsetroundjoin%
\definecolor{currentfill}{rgb}{0.479502,0.556054,0.736531}%
\pgfsetfillcolor{currentfill}%
\pgfsetfillopacity{0.700000}%
\pgfsetlinewidth{1.003750pt}%
\definecolor{currentstroke}{rgb}{0.479502,0.556054,0.736531}%
\pgfsetstrokecolor{currentstroke}%
\pgfsetstrokeopacity{0.700000}%
\pgfsetdash{}{0pt}%
\pgfpathmoveto{\pgfqpoint{2.051825in}{1.217426in}}%
\pgfpathcurveto{\pgfqpoint{2.053667in}{1.217426in}}{\pgfqpoint{2.055434in}{1.218157in}}{\pgfqpoint{2.056736in}{1.219460in}}%
\pgfpathcurveto{\pgfqpoint{2.058038in}{1.220762in}}{\pgfqpoint{2.058770in}{1.222529in}}{\pgfqpoint{2.058770in}{1.224370in}}%
\pgfpathcurveto{\pgfqpoint{2.058770in}{1.226212in}}{\pgfqpoint{2.058038in}{1.227978in}}{\pgfqpoint{2.056736in}{1.229281in}}%
\pgfpathcurveto{\pgfqpoint{2.055434in}{1.230583in}}{\pgfqpoint{2.053667in}{1.231315in}}{\pgfqpoint{2.051825in}{1.231315in}}%
\pgfpathcurveto{\pgfqpoint{2.049984in}{1.231315in}}{\pgfqpoint{2.048217in}{1.230583in}}{\pgfqpoint{2.046915in}{1.229281in}}%
\pgfpathcurveto{\pgfqpoint{2.045613in}{1.227978in}}{\pgfqpoint{2.044881in}{1.226212in}}{\pgfqpoint{2.044881in}{1.224370in}}%
\pgfpathcurveto{\pgfqpoint{2.044881in}{1.222529in}}{\pgfqpoint{2.045613in}{1.220762in}}{\pgfqpoint{2.046915in}{1.219460in}}%
\pgfpathcurveto{\pgfqpoint{2.048217in}{1.218157in}}{\pgfqpoint{2.049984in}{1.217426in}}{\pgfqpoint{2.051825in}{1.217426in}}%
\pgfpathlineto{\pgfqpoint{2.051825in}{1.217426in}}%
\pgfpathclose%
\pgfusepath{stroke,fill}%
\end{pgfscope}%
\begin{pgfscope}%
\pgfpathrectangle{\pgfqpoint{0.661006in}{0.524170in}}{\pgfqpoint{4.194036in}{1.071446in}}%
\pgfusepath{clip}%
\pgfsetbuttcap%
\pgfsetroundjoin%
\definecolor{currentfill}{rgb}{0.479502,0.556054,0.736531}%
\pgfsetfillcolor{currentfill}%
\pgfsetfillopacity{0.700000}%
\pgfsetlinewidth{1.003750pt}%
\definecolor{currentstroke}{rgb}{0.479502,0.556054,0.736531}%
\pgfsetstrokecolor{currentstroke}%
\pgfsetstrokeopacity{0.700000}%
\pgfsetdash{}{0pt}%
\pgfpathmoveto{\pgfqpoint{2.052894in}{1.216216in}}%
\pgfpathcurveto{\pgfqpoint{2.054736in}{1.216216in}}{\pgfqpoint{2.056503in}{1.216948in}}{\pgfqpoint{2.057805in}{1.218250in}}%
\pgfpathcurveto{\pgfqpoint{2.059107in}{1.219552in}}{\pgfqpoint{2.059839in}{1.221319in}}{\pgfqpoint{2.059839in}{1.223161in}}%
\pgfpathcurveto{\pgfqpoint{2.059839in}{1.225002in}}{\pgfqpoint{2.059107in}{1.226769in}}{\pgfqpoint{2.057805in}{1.228071in}}%
\pgfpathcurveto{\pgfqpoint{2.056503in}{1.229373in}}{\pgfqpoint{2.054736in}{1.230105in}}{\pgfqpoint{2.052894in}{1.230105in}}%
\pgfpathcurveto{\pgfqpoint{2.051053in}{1.230105in}}{\pgfqpoint{2.049286in}{1.229373in}}{\pgfqpoint{2.047984in}{1.228071in}}%
\pgfpathcurveto{\pgfqpoint{2.046682in}{1.226769in}}{\pgfqpoint{2.045950in}{1.225002in}}{\pgfqpoint{2.045950in}{1.223161in}}%
\pgfpathcurveto{\pgfqpoint{2.045950in}{1.221319in}}{\pgfqpoint{2.046682in}{1.219552in}}{\pgfqpoint{2.047984in}{1.218250in}}%
\pgfpathcurveto{\pgfqpoint{2.049286in}{1.216948in}}{\pgfqpoint{2.051053in}{1.216216in}}{\pgfqpoint{2.052894in}{1.216216in}}%
\pgfpathlineto{\pgfqpoint{2.052894in}{1.216216in}}%
\pgfpathclose%
\pgfusepath{stroke,fill}%
\end{pgfscope}%
\begin{pgfscope}%
\pgfpathrectangle{\pgfqpoint{0.661006in}{0.524170in}}{\pgfqpoint{4.194036in}{1.071446in}}%
\pgfusepath{clip}%
\pgfsetbuttcap%
\pgfsetroundjoin%
\definecolor{currentfill}{rgb}{0.479502,0.556054,0.736531}%
\pgfsetfillcolor{currentfill}%
\pgfsetfillopacity{0.700000}%
\pgfsetlinewidth{1.003750pt}%
\definecolor{currentstroke}{rgb}{0.479502,0.556054,0.736531}%
\pgfsetstrokecolor{currentstroke}%
\pgfsetstrokeopacity{0.700000}%
\pgfsetdash{}{0pt}%
\pgfpathmoveto{\pgfqpoint{2.071160in}{1.212292in}}%
\pgfpathcurveto{\pgfqpoint{2.073002in}{1.212292in}}{\pgfqpoint{2.074768in}{1.213024in}}{\pgfqpoint{2.076070in}{1.214326in}}%
\pgfpathcurveto{\pgfqpoint{2.077373in}{1.215628in}}{\pgfqpoint{2.078104in}{1.217395in}}{\pgfqpoint{2.078104in}{1.219236in}}%
\pgfpathcurveto{\pgfqpoint{2.078104in}{1.221078in}}{\pgfqpoint{2.077373in}{1.222845in}}{\pgfqpoint{2.076070in}{1.224147in}}%
\pgfpathcurveto{\pgfqpoint{2.074768in}{1.225449in}}{\pgfqpoint{2.073002in}{1.226181in}}{\pgfqpoint{2.071160in}{1.226181in}}%
\pgfpathcurveto{\pgfqpoint{2.069318in}{1.226181in}}{\pgfqpoint{2.067552in}{1.225449in}}{\pgfqpoint{2.066249in}{1.224147in}}%
\pgfpathcurveto{\pgfqpoint{2.064947in}{1.222845in}}{\pgfqpoint{2.064215in}{1.221078in}}{\pgfqpoint{2.064215in}{1.219236in}}%
\pgfpathcurveto{\pgfqpoint{2.064215in}{1.217395in}}{\pgfqpoint{2.064947in}{1.215628in}}{\pgfqpoint{2.066249in}{1.214326in}}%
\pgfpathcurveto{\pgfqpoint{2.067552in}{1.213024in}}{\pgfqpoint{2.069318in}{1.212292in}}{\pgfqpoint{2.071160in}{1.212292in}}%
\pgfpathlineto{\pgfqpoint{2.071160in}{1.212292in}}%
\pgfpathclose%
\pgfusepath{stroke,fill}%
\end{pgfscope}%
\begin{pgfscope}%
\pgfpathrectangle{\pgfqpoint{0.661006in}{0.524170in}}{\pgfqpoint{4.194036in}{1.071446in}}%
\pgfusepath{clip}%
\pgfsetbuttcap%
\pgfsetroundjoin%
\definecolor{currentfill}{rgb}{0.476456,0.551557,0.733310}%
\pgfsetfillcolor{currentfill}%
\pgfsetfillopacity{0.700000}%
\pgfsetlinewidth{1.003750pt}%
\definecolor{currentstroke}{rgb}{0.476456,0.551557,0.733310}%
\pgfsetstrokecolor{currentstroke}%
\pgfsetstrokeopacity{0.700000}%
\pgfsetdash{}{0pt}%
\pgfpathmoveto{\pgfqpoint{2.083523in}{1.206785in}}%
\pgfpathcurveto{\pgfqpoint{2.085365in}{1.206785in}}{\pgfqpoint{2.087131in}{1.207517in}}{\pgfqpoint{2.088433in}{1.208819in}}%
\pgfpathcurveto{\pgfqpoint{2.089736in}{1.210121in}}{\pgfqpoint{2.090467in}{1.211888in}}{\pgfqpoint{2.090467in}{1.213729in}}%
\pgfpathcurveto{\pgfqpoint{2.090467in}{1.215571in}}{\pgfqpoint{2.089736in}{1.217338in}}{\pgfqpoint{2.088433in}{1.218640in}}%
\pgfpathcurveto{\pgfqpoint{2.087131in}{1.219942in}}{\pgfqpoint{2.085365in}{1.220674in}}{\pgfqpoint{2.083523in}{1.220674in}}%
\pgfpathcurveto{\pgfqpoint{2.081681in}{1.220674in}}{\pgfqpoint{2.079915in}{1.219942in}}{\pgfqpoint{2.078612in}{1.218640in}}%
\pgfpathcurveto{\pgfqpoint{2.077310in}{1.217338in}}{\pgfqpoint{2.076578in}{1.215571in}}{\pgfqpoint{2.076578in}{1.213729in}}%
\pgfpathcurveto{\pgfqpoint{2.076578in}{1.211888in}}{\pgfqpoint{2.077310in}{1.210121in}}{\pgfqpoint{2.078612in}{1.208819in}}%
\pgfpathcurveto{\pgfqpoint{2.079915in}{1.207517in}}{\pgfqpoint{2.081681in}{1.206785in}}{\pgfqpoint{2.083523in}{1.206785in}}%
\pgfpathlineto{\pgfqpoint{2.083523in}{1.206785in}}%
\pgfpathclose%
\pgfusepath{stroke,fill}%
\end{pgfscope}%
\begin{pgfscope}%
\pgfpathrectangle{\pgfqpoint{0.661006in}{0.524170in}}{\pgfqpoint{4.194036in}{1.071446in}}%
\pgfusepath{clip}%
\pgfsetbuttcap%
\pgfsetroundjoin%
\definecolor{currentfill}{rgb}{0.476456,0.551557,0.733310}%
\pgfsetfillcolor{currentfill}%
\pgfsetfillopacity{0.700000}%
\pgfsetlinewidth{1.003750pt}%
\definecolor{currentstroke}{rgb}{0.476456,0.551557,0.733310}%
\pgfsetstrokecolor{currentstroke}%
\pgfsetstrokeopacity{0.700000}%
\pgfsetdash{}{0pt}%
\pgfpathmoveto{\pgfqpoint{2.102625in}{1.201558in}}%
\pgfpathcurveto{\pgfqpoint{2.104467in}{1.201558in}}{\pgfqpoint{2.106233in}{1.202289in}}{\pgfqpoint{2.107535in}{1.203592in}}%
\pgfpathcurveto{\pgfqpoint{2.108838in}{1.204894in}}{\pgfqpoint{2.109569in}{1.206660in}}{\pgfqpoint{2.109569in}{1.208502in}}%
\pgfpathcurveto{\pgfqpoint{2.109569in}{1.210344in}}{\pgfqpoint{2.108838in}{1.212110in}}{\pgfqpoint{2.107535in}{1.213413in}}%
\pgfpathcurveto{\pgfqpoint{2.106233in}{1.214715in}}{\pgfqpoint{2.104467in}{1.215447in}}{\pgfqpoint{2.102625in}{1.215447in}}%
\pgfpathcurveto{\pgfqpoint{2.100783in}{1.215447in}}{\pgfqpoint{2.099017in}{1.214715in}}{\pgfqpoint{2.097715in}{1.213413in}}%
\pgfpathcurveto{\pgfqpoint{2.096412in}{1.212110in}}{\pgfqpoint{2.095681in}{1.210344in}}{\pgfqpoint{2.095681in}{1.208502in}}%
\pgfpathcurveto{\pgfqpoint{2.095681in}{1.206660in}}{\pgfqpoint{2.096412in}{1.204894in}}{\pgfqpoint{2.097715in}{1.203592in}}%
\pgfpathcurveto{\pgfqpoint{2.099017in}{1.202289in}}{\pgfqpoint{2.100783in}{1.201558in}}{\pgfqpoint{2.102625in}{1.201558in}}%
\pgfpathlineto{\pgfqpoint{2.102625in}{1.201558in}}%
\pgfpathclose%
\pgfusepath{stroke,fill}%
\end{pgfscope}%
\begin{pgfscope}%
\pgfpathrectangle{\pgfqpoint{0.661006in}{0.524170in}}{\pgfqpoint{4.194036in}{1.071446in}}%
\pgfusepath{clip}%
\pgfsetbuttcap%
\pgfsetroundjoin%
\definecolor{currentfill}{rgb}{0.476456,0.551557,0.733310}%
\pgfsetfillcolor{currentfill}%
\pgfsetfillopacity{0.700000}%
\pgfsetlinewidth{1.003750pt}%
\definecolor{currentstroke}{rgb}{0.476456,0.551557,0.733310}%
\pgfsetstrokecolor{currentstroke}%
\pgfsetstrokeopacity{0.700000}%
\pgfsetdash{}{0pt}%
\pgfpathmoveto{\pgfqpoint{2.123726in}{1.196291in}}%
\pgfpathcurveto{\pgfqpoint{2.125567in}{1.196291in}}{\pgfqpoint{2.127334in}{1.197023in}}{\pgfqpoint{2.128636in}{1.198325in}}%
\pgfpathcurveto{\pgfqpoint{2.129938in}{1.199628in}}{\pgfqpoint{2.130670in}{1.201394in}}{\pgfqpoint{2.130670in}{1.203236in}}%
\pgfpathcurveto{\pgfqpoint{2.130670in}{1.205078in}}{\pgfqpoint{2.129938in}{1.206844in}}{\pgfqpoint{2.128636in}{1.208146in}}%
\pgfpathcurveto{\pgfqpoint{2.127334in}{1.209449in}}{\pgfqpoint{2.125567in}{1.210180in}}{\pgfqpoint{2.123726in}{1.210180in}}%
\pgfpathcurveto{\pgfqpoint{2.121884in}{1.210180in}}{\pgfqpoint{2.120117in}{1.209449in}}{\pgfqpoint{2.118815in}{1.208146in}}%
\pgfpathcurveto{\pgfqpoint{2.117513in}{1.206844in}}{\pgfqpoint{2.116781in}{1.205078in}}{\pgfqpoint{2.116781in}{1.203236in}}%
\pgfpathcurveto{\pgfqpoint{2.116781in}{1.201394in}}{\pgfqpoint{2.117513in}{1.199628in}}{\pgfqpoint{2.118815in}{1.198325in}}%
\pgfpathcurveto{\pgfqpoint{2.120117in}{1.197023in}}{\pgfqpoint{2.121884in}{1.196291in}}{\pgfqpoint{2.123726in}{1.196291in}}%
\pgfpathlineto{\pgfqpoint{2.123726in}{1.196291in}}%
\pgfpathclose%
\pgfusepath{stroke,fill}%
\end{pgfscope}%
\begin{pgfscope}%
\pgfpathrectangle{\pgfqpoint{0.661006in}{0.524170in}}{\pgfqpoint{4.194036in}{1.071446in}}%
\pgfusepath{clip}%
\pgfsetbuttcap%
\pgfsetroundjoin%
\definecolor{currentfill}{rgb}{0.473425,0.547057,0.730060}%
\pgfsetfillcolor{currentfill}%
\pgfsetfillopacity{0.700000}%
\pgfsetlinewidth{1.003750pt}%
\definecolor{currentstroke}{rgb}{0.473425,0.547057,0.730060}%
\pgfsetstrokecolor{currentstroke}%
\pgfsetstrokeopacity{0.700000}%
\pgfsetdash{}{0pt}%
\pgfpathmoveto{\pgfqpoint{2.141666in}{1.192990in}}%
\pgfpathcurveto{\pgfqpoint{2.143508in}{1.192990in}}{\pgfqpoint{2.145274in}{1.193722in}}{\pgfqpoint{2.146576in}{1.195024in}}%
\pgfpathcurveto{\pgfqpoint{2.147879in}{1.196326in}}{\pgfqpoint{2.148610in}{1.198093in}}{\pgfqpoint{2.148610in}{1.199934in}}%
\pgfpathcurveto{\pgfqpoint{2.148610in}{1.201776in}}{\pgfqpoint{2.147879in}{1.203543in}}{\pgfqpoint{2.146576in}{1.204845in}}%
\pgfpathcurveto{\pgfqpoint{2.145274in}{1.206147in}}{\pgfqpoint{2.143508in}{1.206879in}}{\pgfqpoint{2.141666in}{1.206879in}}%
\pgfpathcurveto{\pgfqpoint{2.139824in}{1.206879in}}{\pgfqpoint{2.138058in}{1.206147in}}{\pgfqpoint{2.136755in}{1.204845in}}%
\pgfpathcurveto{\pgfqpoint{2.135453in}{1.203543in}}{\pgfqpoint{2.134721in}{1.201776in}}{\pgfqpoint{2.134721in}{1.199934in}}%
\pgfpathcurveto{\pgfqpoint{2.134721in}{1.198093in}}{\pgfqpoint{2.135453in}{1.196326in}}{\pgfqpoint{2.136755in}{1.195024in}}%
\pgfpathcurveto{\pgfqpoint{2.138058in}{1.193722in}}{\pgfqpoint{2.139824in}{1.192990in}}{\pgfqpoint{2.141666in}{1.192990in}}%
\pgfpathlineto{\pgfqpoint{2.141666in}{1.192990in}}%
\pgfpathclose%
\pgfusepath{stroke,fill}%
\end{pgfscope}%
\begin{pgfscope}%
\pgfpathrectangle{\pgfqpoint{0.661006in}{0.524170in}}{\pgfqpoint{4.194036in}{1.071446in}}%
\pgfusepath{clip}%
\pgfsetbuttcap%
\pgfsetroundjoin%
\definecolor{currentfill}{rgb}{0.473425,0.547057,0.730060}%
\pgfsetfillcolor{currentfill}%
\pgfsetfillopacity{0.700000}%
\pgfsetlinewidth{1.003750pt}%
\definecolor{currentstroke}{rgb}{0.473425,0.547057,0.730060}%
\pgfsetstrokecolor{currentstroke}%
\pgfsetstrokeopacity{0.700000}%
\pgfsetdash{}{0pt}%
\pgfpathmoveto{\pgfqpoint{2.160443in}{1.189569in}}%
\pgfpathcurveto{\pgfqpoint{2.162284in}{1.189569in}}{\pgfqpoint{2.164051in}{1.190301in}}{\pgfqpoint{2.165353in}{1.191603in}}%
\pgfpathcurveto{\pgfqpoint{2.166655in}{1.192906in}}{\pgfqpoint{2.167387in}{1.194672in}}{\pgfqpoint{2.167387in}{1.196514in}}%
\pgfpathcurveto{\pgfqpoint{2.167387in}{1.198355in}}{\pgfqpoint{2.166655in}{1.200122in}}{\pgfqpoint{2.165353in}{1.201424in}}%
\pgfpathcurveto{\pgfqpoint{2.164051in}{1.202726in}}{\pgfqpoint{2.162284in}{1.203458in}}{\pgfqpoint{2.160443in}{1.203458in}}%
\pgfpathcurveto{\pgfqpoint{2.158601in}{1.203458in}}{\pgfqpoint{2.156834in}{1.202726in}}{\pgfqpoint{2.155532in}{1.201424in}}%
\pgfpathcurveto{\pgfqpoint{2.154230in}{1.200122in}}{\pgfqpoint{2.153498in}{1.198355in}}{\pgfqpoint{2.153498in}{1.196514in}}%
\pgfpathcurveto{\pgfqpoint{2.153498in}{1.194672in}}{\pgfqpoint{2.154230in}{1.192906in}}{\pgfqpoint{2.155532in}{1.191603in}}%
\pgfpathcurveto{\pgfqpoint{2.156834in}{1.190301in}}{\pgfqpoint{2.158601in}{1.189569in}}{\pgfqpoint{2.160443in}{1.189569in}}%
\pgfpathlineto{\pgfqpoint{2.160443in}{1.189569in}}%
\pgfpathclose%
\pgfusepath{stroke,fill}%
\end{pgfscope}%
\begin{pgfscope}%
\pgfpathrectangle{\pgfqpoint{0.661006in}{0.524170in}}{\pgfqpoint{4.194036in}{1.071446in}}%
\pgfusepath{clip}%
\pgfsetbuttcap%
\pgfsetroundjoin%
\definecolor{currentfill}{rgb}{0.470408,0.542556,0.726782}%
\pgfsetfillcolor{currentfill}%
\pgfsetfillopacity{0.700000}%
\pgfsetlinewidth{1.003750pt}%
\definecolor{currentstroke}{rgb}{0.470408,0.542556,0.726782}%
\pgfsetstrokecolor{currentstroke}%
\pgfsetstrokeopacity{0.700000}%
\pgfsetdash{}{0pt}%
\pgfpathmoveto{\pgfqpoint{2.166206in}{1.186580in}}%
\pgfpathcurveto{\pgfqpoint{2.168048in}{1.186580in}}{\pgfqpoint{2.169814in}{1.187312in}}{\pgfqpoint{2.171116in}{1.188614in}}%
\pgfpathcurveto{\pgfqpoint{2.172419in}{1.189916in}}{\pgfqpoint{2.173150in}{1.191683in}}{\pgfqpoint{2.173150in}{1.193524in}}%
\pgfpathcurveto{\pgfqpoint{2.173150in}{1.195366in}}{\pgfqpoint{2.172419in}{1.197133in}}{\pgfqpoint{2.171116in}{1.198435in}}%
\pgfpathcurveto{\pgfqpoint{2.169814in}{1.199737in}}{\pgfqpoint{2.168048in}{1.200469in}}{\pgfqpoint{2.166206in}{1.200469in}}%
\pgfpathcurveto{\pgfqpoint{2.164364in}{1.200469in}}{\pgfqpoint{2.162598in}{1.199737in}}{\pgfqpoint{2.161295in}{1.198435in}}%
\pgfpathcurveto{\pgfqpoint{2.159993in}{1.197133in}}{\pgfqpoint{2.159261in}{1.195366in}}{\pgfqpoint{2.159261in}{1.193524in}}%
\pgfpathcurveto{\pgfqpoint{2.159261in}{1.191683in}}{\pgfqpoint{2.159993in}{1.189916in}}{\pgfqpoint{2.161295in}{1.188614in}}%
\pgfpathcurveto{\pgfqpoint{2.162598in}{1.187312in}}{\pgfqpoint{2.164364in}{1.186580in}}{\pgfqpoint{2.166206in}{1.186580in}}%
\pgfpathlineto{\pgfqpoint{2.166206in}{1.186580in}}%
\pgfpathclose%
\pgfusepath{stroke,fill}%
\end{pgfscope}%
\begin{pgfscope}%
\pgfpathrectangle{\pgfqpoint{0.661006in}{0.524170in}}{\pgfqpoint{4.194036in}{1.071446in}}%
\pgfusepath{clip}%
\pgfsetbuttcap%
\pgfsetroundjoin%
\definecolor{currentfill}{rgb}{0.470408,0.542556,0.726782}%
\pgfsetfillcolor{currentfill}%
\pgfsetfillopacity{0.700000}%
\pgfsetlinewidth{1.003750pt}%
\definecolor{currentstroke}{rgb}{0.470408,0.542556,0.726782}%
\pgfsetstrokecolor{currentstroke}%
\pgfsetstrokeopacity{0.700000}%
\pgfsetdash{}{0pt}%
\pgfpathmoveto{\pgfqpoint{2.175209in}{1.183826in}}%
\pgfpathcurveto{\pgfqpoint{2.177051in}{1.183826in}}{\pgfqpoint{2.178817in}{1.184558in}}{\pgfqpoint{2.180120in}{1.185860in}}%
\pgfpathcurveto{\pgfqpoint{2.181422in}{1.187162in}}{\pgfqpoint{2.182154in}{1.188929in}}{\pgfqpoint{2.182154in}{1.190770in}}%
\pgfpathcurveto{\pgfqpoint{2.182154in}{1.192612in}}{\pgfqpoint{2.181422in}{1.194379in}}{\pgfqpoint{2.180120in}{1.195681in}}%
\pgfpathcurveto{\pgfqpoint{2.178817in}{1.196983in}}{\pgfqpoint{2.177051in}{1.197715in}}{\pgfqpoint{2.175209in}{1.197715in}}%
\pgfpathcurveto{\pgfqpoint{2.173367in}{1.197715in}}{\pgfqpoint{2.171601in}{1.196983in}}{\pgfqpoint{2.170299in}{1.195681in}}%
\pgfpathcurveto{\pgfqpoint{2.168996in}{1.194379in}}{\pgfqpoint{2.168265in}{1.192612in}}{\pgfqpoint{2.168265in}{1.190770in}}%
\pgfpathcurveto{\pgfqpoint{2.168265in}{1.188929in}}{\pgfqpoint{2.168996in}{1.187162in}}{\pgfqpoint{2.170299in}{1.185860in}}%
\pgfpathcurveto{\pgfqpoint{2.171601in}{1.184558in}}{\pgfqpoint{2.173367in}{1.183826in}}{\pgfqpoint{2.175209in}{1.183826in}}%
\pgfpathlineto{\pgfqpoint{2.175209in}{1.183826in}}%
\pgfpathclose%
\pgfusepath{stroke,fill}%
\end{pgfscope}%
\begin{pgfscope}%
\pgfpathrectangle{\pgfqpoint{0.661006in}{0.524170in}}{\pgfqpoint{4.194036in}{1.071446in}}%
\pgfusepath{clip}%
\pgfsetbuttcap%
\pgfsetroundjoin%
\definecolor{currentfill}{rgb}{0.470408,0.542556,0.726782}%
\pgfsetfillcolor{currentfill}%
\pgfsetfillopacity{0.700000}%
\pgfsetlinewidth{1.003750pt}%
\definecolor{currentstroke}{rgb}{0.470408,0.542556,0.726782}%
\pgfsetstrokecolor{currentstroke}%
\pgfsetstrokeopacity{0.700000}%
\pgfsetdash{}{0pt}%
\pgfpathmoveto{\pgfqpoint{2.161465in}{1.181329in}}%
\pgfpathcurveto{\pgfqpoint{2.163307in}{1.181329in}}{\pgfqpoint{2.165073in}{1.182061in}}{\pgfqpoint{2.166376in}{1.183363in}}%
\pgfpathcurveto{\pgfqpoint{2.167678in}{1.184666in}}{\pgfqpoint{2.168410in}{1.186432in}}{\pgfqpoint{2.168410in}{1.188274in}}%
\pgfpathcurveto{\pgfqpoint{2.168410in}{1.190115in}}{\pgfqpoint{2.167678in}{1.191882in}}{\pgfqpoint{2.166376in}{1.193184in}}%
\pgfpathcurveto{\pgfqpoint{2.165073in}{1.194486in}}{\pgfqpoint{2.163307in}{1.195218in}}{\pgfqpoint{2.161465in}{1.195218in}}%
\pgfpathcurveto{\pgfqpoint{2.159624in}{1.195218in}}{\pgfqpoint{2.157857in}{1.194486in}}{\pgfqpoint{2.156555in}{1.193184in}}%
\pgfpathcurveto{\pgfqpoint{2.155252in}{1.191882in}}{\pgfqpoint{2.154521in}{1.190115in}}{\pgfqpoint{2.154521in}{1.188274in}}%
\pgfpathcurveto{\pgfqpoint{2.154521in}{1.186432in}}{\pgfqpoint{2.155252in}{1.184666in}}{\pgfqpoint{2.156555in}{1.183363in}}%
\pgfpathcurveto{\pgfqpoint{2.157857in}{1.182061in}}{\pgfqpoint{2.159624in}{1.181329in}}{\pgfqpoint{2.161465in}{1.181329in}}%
\pgfpathlineto{\pgfqpoint{2.161465in}{1.181329in}}%
\pgfpathclose%
\pgfusepath{stroke,fill}%
\end{pgfscope}%
\begin{pgfscope}%
\pgfpathrectangle{\pgfqpoint{0.661006in}{0.524170in}}{\pgfqpoint{4.194036in}{1.071446in}}%
\pgfusepath{clip}%
\pgfsetbuttcap%
\pgfsetroundjoin%
\definecolor{currentfill}{rgb}{0.470408,0.542556,0.726782}%
\pgfsetfillcolor{currentfill}%
\pgfsetfillopacity{0.700000}%
\pgfsetlinewidth{1.003750pt}%
\definecolor{currentstroke}{rgb}{0.470408,0.542556,0.726782}%
\pgfsetstrokecolor{currentstroke}%
\pgfsetstrokeopacity{0.700000}%
\pgfsetdash{}{0pt}%
\pgfpathmoveto{\pgfqpoint{2.167321in}{1.183436in}}%
\pgfpathcurveto{\pgfqpoint{2.169163in}{1.183436in}}{\pgfqpoint{2.170930in}{1.184168in}}{\pgfqpoint{2.172232in}{1.185470in}}%
\pgfpathcurveto{\pgfqpoint{2.173534in}{1.186772in}}{\pgfqpoint{2.174266in}{1.188539in}}{\pgfqpoint{2.174266in}{1.190381in}}%
\pgfpathcurveto{\pgfqpoint{2.174266in}{1.192222in}}{\pgfqpoint{2.173534in}{1.193989in}}{\pgfqpoint{2.172232in}{1.195291in}}%
\pgfpathcurveto{\pgfqpoint{2.170930in}{1.196593in}}{\pgfqpoint{2.169163in}{1.197325in}}{\pgfqpoint{2.167321in}{1.197325in}}%
\pgfpathcurveto{\pgfqpoint{2.165480in}{1.197325in}}{\pgfqpoint{2.163713in}{1.196593in}}{\pgfqpoint{2.162411in}{1.195291in}}%
\pgfpathcurveto{\pgfqpoint{2.161109in}{1.193989in}}{\pgfqpoint{2.160377in}{1.192222in}}{\pgfqpoint{2.160377in}{1.190381in}}%
\pgfpathcurveto{\pgfqpoint{2.160377in}{1.188539in}}{\pgfqpoint{2.161109in}{1.186772in}}{\pgfqpoint{2.162411in}{1.185470in}}%
\pgfpathcurveto{\pgfqpoint{2.163713in}{1.184168in}}{\pgfqpoint{2.165480in}{1.183436in}}{\pgfqpoint{2.167321in}{1.183436in}}%
\pgfpathlineto{\pgfqpoint{2.167321in}{1.183436in}}%
\pgfpathclose%
\pgfusepath{stroke,fill}%
\end{pgfscope}%
\begin{pgfscope}%
\pgfpathrectangle{\pgfqpoint{0.661006in}{0.524170in}}{\pgfqpoint{4.194036in}{1.071446in}}%
\pgfusepath{clip}%
\pgfsetbuttcap%
\pgfsetroundjoin%
\definecolor{currentfill}{rgb}{0.470408,0.542556,0.726782}%
\pgfsetfillcolor{currentfill}%
\pgfsetfillopacity{0.700000}%
\pgfsetlinewidth{1.003750pt}%
\definecolor{currentstroke}{rgb}{0.470408,0.542556,0.726782}%
\pgfsetstrokecolor{currentstroke}%
\pgfsetstrokeopacity{0.700000}%
\pgfsetdash{}{0pt}%
\pgfpathmoveto{\pgfqpoint{2.184053in}{1.181446in}}%
\pgfpathcurveto{\pgfqpoint{2.185895in}{1.181446in}}{\pgfqpoint{2.187661in}{1.182178in}}{\pgfqpoint{2.188964in}{1.183480in}}%
\pgfpathcurveto{\pgfqpoint{2.190266in}{1.184782in}}{\pgfqpoint{2.190998in}{1.186549in}}{\pgfqpoint{2.190998in}{1.188391in}}%
\pgfpathcurveto{\pgfqpoint{2.190998in}{1.190232in}}{\pgfqpoint{2.190266in}{1.191999in}}{\pgfqpoint{2.188964in}{1.193301in}}%
\pgfpathcurveto{\pgfqpoint{2.187661in}{1.194603in}}{\pgfqpoint{2.185895in}{1.195335in}}{\pgfqpoint{2.184053in}{1.195335in}}%
\pgfpathcurveto{\pgfqpoint{2.182211in}{1.195335in}}{\pgfqpoint{2.180445in}{1.194603in}}{\pgfqpoint{2.179143in}{1.193301in}}%
\pgfpathcurveto{\pgfqpoint{2.177840in}{1.191999in}}{\pgfqpoint{2.177109in}{1.190232in}}{\pgfqpoint{2.177109in}{1.188391in}}%
\pgfpathcurveto{\pgfqpoint{2.177109in}{1.186549in}}{\pgfqpoint{2.177840in}{1.184782in}}{\pgfqpoint{2.179143in}{1.183480in}}%
\pgfpathcurveto{\pgfqpoint{2.180445in}{1.182178in}}{\pgfqpoint{2.182211in}{1.181446in}}{\pgfqpoint{2.184053in}{1.181446in}}%
\pgfpathlineto{\pgfqpoint{2.184053in}{1.181446in}}%
\pgfpathclose%
\pgfusepath{stroke,fill}%
\end{pgfscope}%
\begin{pgfscope}%
\pgfpathrectangle{\pgfqpoint{0.661006in}{0.524170in}}{\pgfqpoint{4.194036in}{1.071446in}}%
\pgfusepath{clip}%
\pgfsetbuttcap%
\pgfsetroundjoin%
\definecolor{currentfill}{rgb}{0.467406,0.538053,0.723474}%
\pgfsetfillcolor{currentfill}%
\pgfsetfillopacity{0.700000}%
\pgfsetlinewidth{1.003750pt}%
\definecolor{currentstroke}{rgb}{0.467406,0.538053,0.723474}%
\pgfsetstrokecolor{currentstroke}%
\pgfsetstrokeopacity{0.700000}%
\pgfsetdash{}{0pt}%
\pgfpathmoveto{\pgfqpoint{2.201854in}{1.177003in}}%
\pgfpathcurveto{\pgfqpoint{2.203696in}{1.177003in}}{\pgfqpoint{2.205462in}{1.177735in}}{\pgfqpoint{2.206764in}{1.179037in}}%
\pgfpathcurveto{\pgfqpoint{2.208067in}{1.180340in}}{\pgfqpoint{2.208798in}{1.182106in}}{\pgfqpoint{2.208798in}{1.183948in}}%
\pgfpathcurveto{\pgfqpoint{2.208798in}{1.185790in}}{\pgfqpoint{2.208067in}{1.187556in}}{\pgfqpoint{2.206764in}{1.188858in}}%
\pgfpathcurveto{\pgfqpoint{2.205462in}{1.190161in}}{\pgfqpoint{2.203696in}{1.190892in}}{\pgfqpoint{2.201854in}{1.190892in}}%
\pgfpathcurveto{\pgfqpoint{2.200012in}{1.190892in}}{\pgfqpoint{2.198246in}{1.190161in}}{\pgfqpoint{2.196943in}{1.188858in}}%
\pgfpathcurveto{\pgfqpoint{2.195641in}{1.187556in}}{\pgfqpoint{2.194909in}{1.185790in}}{\pgfqpoint{2.194909in}{1.183948in}}%
\pgfpathcurveto{\pgfqpoint{2.194909in}{1.182106in}}{\pgfqpoint{2.195641in}{1.180340in}}{\pgfqpoint{2.196943in}{1.179037in}}%
\pgfpathcurveto{\pgfqpoint{2.198246in}{1.177735in}}{\pgfqpoint{2.200012in}{1.177003in}}{\pgfqpoint{2.201854in}{1.177003in}}%
\pgfpathlineto{\pgfqpoint{2.201854in}{1.177003in}}%
\pgfpathclose%
\pgfusepath{stroke,fill}%
\end{pgfscope}%
\begin{pgfscope}%
\pgfpathrectangle{\pgfqpoint{0.661006in}{0.524170in}}{\pgfqpoint{4.194036in}{1.071446in}}%
\pgfusepath{clip}%
\pgfsetbuttcap%
\pgfsetroundjoin%
\definecolor{currentfill}{rgb}{0.467406,0.538053,0.723474}%
\pgfsetfillcolor{currentfill}%
\pgfsetfillopacity{0.700000}%
\pgfsetlinewidth{1.003750pt}%
\definecolor{currentstroke}{rgb}{0.467406,0.538053,0.723474}%
\pgfsetstrokecolor{currentstroke}%
\pgfsetstrokeopacity{0.700000}%
\pgfsetdash{}{0pt}%
\pgfpathmoveto{\pgfqpoint{2.212451in}{1.172868in}}%
\pgfpathcurveto{\pgfqpoint{2.214292in}{1.172868in}}{\pgfqpoint{2.216059in}{1.173600in}}{\pgfqpoint{2.217361in}{1.174902in}}%
\pgfpathcurveto{\pgfqpoint{2.218663in}{1.176204in}}{\pgfqpoint{2.219395in}{1.177971in}}{\pgfqpoint{2.219395in}{1.179812in}}%
\pgfpathcurveto{\pgfqpoint{2.219395in}{1.181654in}}{\pgfqpoint{2.218663in}{1.183421in}}{\pgfqpoint{2.217361in}{1.184723in}}%
\pgfpathcurveto{\pgfqpoint{2.216059in}{1.186025in}}{\pgfqpoint{2.214292in}{1.186757in}}{\pgfqpoint{2.212451in}{1.186757in}}%
\pgfpathcurveto{\pgfqpoint{2.210609in}{1.186757in}}{\pgfqpoint{2.208843in}{1.186025in}}{\pgfqpoint{2.207540in}{1.184723in}}%
\pgfpathcurveto{\pgfqpoint{2.206238in}{1.183421in}}{\pgfqpoint{2.205506in}{1.181654in}}{\pgfqpoint{2.205506in}{1.179812in}}%
\pgfpathcurveto{\pgfqpoint{2.205506in}{1.177971in}}{\pgfqpoint{2.206238in}{1.176204in}}{\pgfqpoint{2.207540in}{1.174902in}}%
\pgfpathcurveto{\pgfqpoint{2.208843in}{1.173600in}}{\pgfqpoint{2.210609in}{1.172868in}}{\pgfqpoint{2.212451in}{1.172868in}}%
\pgfpathlineto{\pgfqpoint{2.212451in}{1.172868in}}%
\pgfpathclose%
\pgfusepath{stroke,fill}%
\end{pgfscope}%
\begin{pgfscope}%
\pgfpathrectangle{\pgfqpoint{0.661006in}{0.524170in}}{\pgfqpoint{4.194036in}{1.071446in}}%
\pgfusepath{clip}%
\pgfsetbuttcap%
\pgfsetroundjoin%
\definecolor{currentfill}{rgb}{0.464417,0.533548,0.720137}%
\pgfsetfillcolor{currentfill}%
\pgfsetfillopacity{0.700000}%
\pgfsetlinewidth{1.003750pt}%
\definecolor{currentstroke}{rgb}{0.464417,0.533548,0.720137}%
\pgfsetstrokecolor{currentstroke}%
\pgfsetstrokeopacity{0.700000}%
\pgfsetdash{}{0pt}%
\pgfpathmoveto{\pgfqpoint{2.234853in}{1.168016in}}%
\pgfpathcurveto{\pgfqpoint{2.236694in}{1.168016in}}{\pgfqpoint{2.238461in}{1.168747in}}{\pgfqpoint{2.239763in}{1.170050in}}%
\pgfpathcurveto{\pgfqpoint{2.241065in}{1.171352in}}{\pgfqpoint{2.241797in}{1.173118in}}{\pgfqpoint{2.241797in}{1.174960in}}%
\pgfpathcurveto{\pgfqpoint{2.241797in}{1.176802in}}{\pgfqpoint{2.241065in}{1.178568in}}{\pgfqpoint{2.239763in}{1.179871in}}%
\pgfpathcurveto{\pgfqpoint{2.238461in}{1.181173in}}{\pgfqpoint{2.236694in}{1.181905in}}{\pgfqpoint{2.234853in}{1.181905in}}%
\pgfpathcurveto{\pgfqpoint{2.233011in}{1.181905in}}{\pgfqpoint{2.231245in}{1.181173in}}{\pgfqpoint{2.229942in}{1.179871in}}%
\pgfpathcurveto{\pgfqpoint{2.228640in}{1.178568in}}{\pgfqpoint{2.227908in}{1.176802in}}{\pgfqpoint{2.227908in}{1.174960in}}%
\pgfpathcurveto{\pgfqpoint{2.227908in}{1.173118in}}{\pgfqpoint{2.228640in}{1.171352in}}{\pgfqpoint{2.229942in}{1.170050in}}%
\pgfpathcurveto{\pgfqpoint{2.231245in}{1.168747in}}{\pgfqpoint{2.233011in}{1.168016in}}{\pgfqpoint{2.234853in}{1.168016in}}%
\pgfpathlineto{\pgfqpoint{2.234853in}{1.168016in}}%
\pgfpathclose%
\pgfusepath{stroke,fill}%
\end{pgfscope}%
\begin{pgfscope}%
\pgfpathrectangle{\pgfqpoint{0.661006in}{0.524170in}}{\pgfqpoint{4.194036in}{1.071446in}}%
\pgfusepath{clip}%
\pgfsetbuttcap%
\pgfsetroundjoin%
\definecolor{currentfill}{rgb}{0.464417,0.533548,0.720137}%
\pgfsetfillcolor{currentfill}%
\pgfsetfillopacity{0.700000}%
\pgfsetlinewidth{1.003750pt}%
\definecolor{currentstroke}{rgb}{0.464417,0.533548,0.720137}%
\pgfsetstrokecolor{currentstroke}%
\pgfsetstrokeopacity{0.700000}%
\pgfsetdash{}{0pt}%
\pgfpathmoveto{\pgfqpoint{2.227509in}{1.165709in}}%
\pgfpathcurveto{\pgfqpoint{2.229351in}{1.165709in}}{\pgfqpoint{2.231118in}{1.166441in}}{\pgfqpoint{2.232420in}{1.167743in}}%
\pgfpathcurveto{\pgfqpoint{2.233722in}{1.169045in}}{\pgfqpoint{2.234454in}{1.170812in}}{\pgfqpoint{2.234454in}{1.172653in}}%
\pgfpathcurveto{\pgfqpoint{2.234454in}{1.174495in}}{\pgfqpoint{2.233722in}{1.176262in}}{\pgfqpoint{2.232420in}{1.177564in}}%
\pgfpathcurveto{\pgfqpoint{2.231118in}{1.178866in}}{\pgfqpoint{2.229351in}{1.179598in}}{\pgfqpoint{2.227509in}{1.179598in}}%
\pgfpathcurveto{\pgfqpoint{2.225668in}{1.179598in}}{\pgfqpoint{2.223901in}{1.178866in}}{\pgfqpoint{2.222599in}{1.177564in}}%
\pgfpathcurveto{\pgfqpoint{2.221297in}{1.176262in}}{\pgfqpoint{2.220565in}{1.174495in}}{\pgfqpoint{2.220565in}{1.172653in}}%
\pgfpathcurveto{\pgfqpoint{2.220565in}{1.170812in}}{\pgfqpoint{2.221297in}{1.169045in}}{\pgfqpoint{2.222599in}{1.167743in}}%
\pgfpathcurveto{\pgfqpoint{2.223901in}{1.166441in}}{\pgfqpoint{2.225668in}{1.165709in}}{\pgfqpoint{2.227509in}{1.165709in}}%
\pgfpathlineto{\pgfqpoint{2.227509in}{1.165709in}}%
\pgfpathclose%
\pgfusepath{stroke,fill}%
\end{pgfscope}%
\begin{pgfscope}%
\pgfpathrectangle{\pgfqpoint{0.661006in}{0.524170in}}{\pgfqpoint{4.194036in}{1.071446in}}%
\pgfusepath{clip}%
\pgfsetbuttcap%
\pgfsetroundjoin%
\definecolor{currentfill}{rgb}{0.464417,0.533548,0.720137}%
\pgfsetfillcolor{currentfill}%
\pgfsetfillopacity{0.700000}%
\pgfsetlinewidth{1.003750pt}%
\definecolor{currentstroke}{rgb}{0.464417,0.533548,0.720137}%
\pgfsetstrokecolor{currentstroke}%
\pgfsetstrokeopacity{0.700000}%
\pgfsetdash{}{0pt}%
\pgfpathmoveto{\pgfqpoint{2.221281in}{1.169901in}}%
\pgfpathcurveto{\pgfqpoint{2.223123in}{1.169901in}}{\pgfqpoint{2.224890in}{1.170633in}}{\pgfqpoint{2.226192in}{1.171935in}}%
\pgfpathcurveto{\pgfqpoint{2.227494in}{1.173237in}}{\pgfqpoint{2.228226in}{1.175004in}}{\pgfqpoint{2.228226in}{1.176846in}}%
\pgfpathcurveto{\pgfqpoint{2.228226in}{1.178687in}}{\pgfqpoint{2.227494in}{1.180454in}}{\pgfqpoint{2.226192in}{1.181756in}}%
\pgfpathcurveto{\pgfqpoint{2.224890in}{1.183058in}}{\pgfqpoint{2.223123in}{1.183790in}}{\pgfqpoint{2.221281in}{1.183790in}}%
\pgfpathcurveto{\pgfqpoint{2.219440in}{1.183790in}}{\pgfqpoint{2.217673in}{1.183058in}}{\pgfqpoint{2.216371in}{1.181756in}}%
\pgfpathcurveto{\pgfqpoint{2.215069in}{1.180454in}}{\pgfqpoint{2.214337in}{1.178687in}}{\pgfqpoint{2.214337in}{1.176846in}}%
\pgfpathcurveto{\pgfqpoint{2.214337in}{1.175004in}}{\pgfqpoint{2.215069in}{1.173237in}}{\pgfqpoint{2.216371in}{1.171935in}}%
\pgfpathcurveto{\pgfqpoint{2.217673in}{1.170633in}}{\pgfqpoint{2.219440in}{1.169901in}}{\pgfqpoint{2.221281in}{1.169901in}}%
\pgfpathlineto{\pgfqpoint{2.221281in}{1.169901in}}%
\pgfpathclose%
\pgfusepath{stroke,fill}%
\end{pgfscope}%
\begin{pgfscope}%
\pgfpathrectangle{\pgfqpoint{0.661006in}{0.524170in}}{\pgfqpoint{4.194036in}{1.071446in}}%
\pgfusepath{clip}%
\pgfsetbuttcap%
\pgfsetroundjoin%
\definecolor{currentfill}{rgb}{0.464417,0.533548,0.720137}%
\pgfsetfillcolor{currentfill}%
\pgfsetfillopacity{0.700000}%
\pgfsetlinewidth{1.003750pt}%
\definecolor{currentstroke}{rgb}{0.464417,0.533548,0.720137}%
\pgfsetstrokecolor{currentstroke}%
\pgfsetstrokeopacity{0.700000}%
\pgfsetdash{}{0pt}%
\pgfpathmoveto{\pgfqpoint{2.215100in}{1.173171in}}%
\pgfpathcurveto{\pgfqpoint{2.216942in}{1.173171in}}{\pgfqpoint{2.218708in}{1.173903in}}{\pgfqpoint{2.220010in}{1.175205in}}%
\pgfpathcurveto{\pgfqpoint{2.221313in}{1.176507in}}{\pgfqpoint{2.222044in}{1.178274in}}{\pgfqpoint{2.222044in}{1.180116in}}%
\pgfpathcurveto{\pgfqpoint{2.222044in}{1.181957in}}{\pgfqpoint{2.221313in}{1.183724in}}{\pgfqpoint{2.220010in}{1.185026in}}%
\pgfpathcurveto{\pgfqpoint{2.218708in}{1.186328in}}{\pgfqpoint{2.216942in}{1.187060in}}{\pgfqpoint{2.215100in}{1.187060in}}%
\pgfpathcurveto{\pgfqpoint{2.213258in}{1.187060in}}{\pgfqpoint{2.211492in}{1.186328in}}{\pgfqpoint{2.210189in}{1.185026in}}%
\pgfpathcurveto{\pgfqpoint{2.208887in}{1.183724in}}{\pgfqpoint{2.208155in}{1.181957in}}{\pgfqpoint{2.208155in}{1.180116in}}%
\pgfpathcurveto{\pgfqpoint{2.208155in}{1.178274in}}{\pgfqpoint{2.208887in}{1.176507in}}{\pgfqpoint{2.210189in}{1.175205in}}%
\pgfpathcurveto{\pgfqpoint{2.211492in}{1.173903in}}{\pgfqpoint{2.213258in}{1.173171in}}{\pgfqpoint{2.215100in}{1.173171in}}%
\pgfpathlineto{\pgfqpoint{2.215100in}{1.173171in}}%
\pgfpathclose%
\pgfusepath{stroke,fill}%
\end{pgfscope}%
\begin{pgfscope}%
\pgfpathrectangle{\pgfqpoint{0.661006in}{0.524170in}}{\pgfqpoint{4.194036in}{1.071446in}}%
\pgfusepath{clip}%
\pgfsetbuttcap%
\pgfsetroundjoin%
\definecolor{currentfill}{rgb}{0.464417,0.533548,0.720137}%
\pgfsetfillcolor{currentfill}%
\pgfsetfillopacity{0.700000}%
\pgfsetlinewidth{1.003750pt}%
\definecolor{currentstroke}{rgb}{0.464417,0.533548,0.720137}%
\pgfsetstrokecolor{currentstroke}%
\pgfsetstrokeopacity{0.700000}%
\pgfsetdash{}{0pt}%
\pgfpathmoveto{\pgfqpoint{2.205526in}{1.172358in}}%
\pgfpathcurveto{\pgfqpoint{2.207367in}{1.172358in}}{\pgfqpoint{2.209134in}{1.173089in}}{\pgfqpoint{2.210436in}{1.174392in}}%
\pgfpathcurveto{\pgfqpoint{2.211738in}{1.175694in}}{\pgfqpoint{2.212470in}{1.177460in}}{\pgfqpoint{2.212470in}{1.179302in}}%
\pgfpathcurveto{\pgfqpoint{2.212470in}{1.181144in}}{\pgfqpoint{2.211738in}{1.182910in}}{\pgfqpoint{2.210436in}{1.184213in}}%
\pgfpathcurveto{\pgfqpoint{2.209134in}{1.185515in}}{\pgfqpoint{2.207367in}{1.186247in}}{\pgfqpoint{2.205526in}{1.186247in}}%
\pgfpathcurveto{\pgfqpoint{2.203684in}{1.186247in}}{\pgfqpoint{2.201917in}{1.185515in}}{\pgfqpoint{2.200615in}{1.184213in}}%
\pgfpathcurveto{\pgfqpoint{2.199313in}{1.182910in}}{\pgfqpoint{2.198581in}{1.181144in}}{\pgfqpoint{2.198581in}{1.179302in}}%
\pgfpathcurveto{\pgfqpoint{2.198581in}{1.177460in}}{\pgfqpoint{2.199313in}{1.175694in}}{\pgfqpoint{2.200615in}{1.174392in}}%
\pgfpathcurveto{\pgfqpoint{2.201917in}{1.173089in}}{\pgfqpoint{2.203684in}{1.172358in}}{\pgfqpoint{2.205526in}{1.172358in}}%
\pgfpathlineto{\pgfqpoint{2.205526in}{1.172358in}}%
\pgfpathclose%
\pgfusepath{stroke,fill}%
\end{pgfscope}%
\begin{pgfscope}%
\pgfpathrectangle{\pgfqpoint{0.661006in}{0.524170in}}{\pgfqpoint{4.194036in}{1.071446in}}%
\pgfusepath{clip}%
\pgfsetbuttcap%
\pgfsetroundjoin%
\definecolor{currentfill}{rgb}{0.461441,0.529042,0.716771}%
\pgfsetfillcolor{currentfill}%
\pgfsetfillopacity{0.700000}%
\pgfsetlinewidth{1.003750pt}%
\definecolor{currentstroke}{rgb}{0.461441,0.529042,0.716771}%
\pgfsetstrokecolor{currentstroke}%
\pgfsetstrokeopacity{0.700000}%
\pgfsetdash{}{0pt}%
\pgfpathmoveto{\pgfqpoint{2.214403in}{1.173458in}}%
\pgfpathcurveto{\pgfqpoint{2.216244in}{1.173458in}}{\pgfqpoint{2.218011in}{1.174190in}}{\pgfqpoint{2.219313in}{1.175492in}}%
\pgfpathcurveto{\pgfqpoint{2.220615in}{1.176794in}}{\pgfqpoint{2.221347in}{1.178561in}}{\pgfqpoint{2.221347in}{1.180402in}}%
\pgfpathcurveto{\pgfqpoint{2.221347in}{1.182244in}}{\pgfqpoint{2.220615in}{1.184011in}}{\pgfqpoint{2.219313in}{1.185313in}}%
\pgfpathcurveto{\pgfqpoint{2.218011in}{1.186615in}}{\pgfqpoint{2.216244in}{1.187347in}}{\pgfqpoint{2.214403in}{1.187347in}}%
\pgfpathcurveto{\pgfqpoint{2.212561in}{1.187347in}}{\pgfqpoint{2.210795in}{1.186615in}}{\pgfqpoint{2.209492in}{1.185313in}}%
\pgfpathcurveto{\pgfqpoint{2.208190in}{1.184011in}}{\pgfqpoint{2.207458in}{1.182244in}}{\pgfqpoint{2.207458in}{1.180402in}}%
\pgfpathcurveto{\pgfqpoint{2.207458in}{1.178561in}}{\pgfqpoint{2.208190in}{1.176794in}}{\pgfqpoint{2.209492in}{1.175492in}}%
\pgfpathcurveto{\pgfqpoint{2.210795in}{1.174190in}}{\pgfqpoint{2.212561in}{1.173458in}}{\pgfqpoint{2.214403in}{1.173458in}}%
\pgfpathlineto{\pgfqpoint{2.214403in}{1.173458in}}%
\pgfpathclose%
\pgfusepath{stroke,fill}%
\end{pgfscope}%
\begin{pgfscope}%
\pgfpathrectangle{\pgfqpoint{0.661006in}{0.524170in}}{\pgfqpoint{4.194036in}{1.071446in}}%
\pgfusepath{clip}%
\pgfsetbuttcap%
\pgfsetroundjoin%
\definecolor{currentfill}{rgb}{0.461441,0.529042,0.716771}%
\pgfsetfillcolor{currentfill}%
\pgfsetfillopacity{0.700000}%
\pgfsetlinewidth{1.003750pt}%
\definecolor{currentstroke}{rgb}{0.461441,0.529042,0.716771}%
\pgfsetstrokecolor{currentstroke}%
\pgfsetstrokeopacity{0.700000}%
\pgfsetdash{}{0pt}%
\pgfpathmoveto{\pgfqpoint{2.220398in}{1.172714in}}%
\pgfpathcurveto{\pgfqpoint{2.222240in}{1.172714in}}{\pgfqpoint{2.224007in}{1.173446in}}{\pgfqpoint{2.225309in}{1.174748in}}%
\pgfpathcurveto{\pgfqpoint{2.226611in}{1.176051in}}{\pgfqpoint{2.227343in}{1.177817in}}{\pgfqpoint{2.227343in}{1.179659in}}%
\pgfpathcurveto{\pgfqpoint{2.227343in}{1.181501in}}{\pgfqpoint{2.226611in}{1.183267in}}{\pgfqpoint{2.225309in}{1.184569in}}%
\pgfpathcurveto{\pgfqpoint{2.224007in}{1.185872in}}{\pgfqpoint{2.222240in}{1.186603in}}{\pgfqpoint{2.220398in}{1.186603in}}%
\pgfpathcurveto{\pgfqpoint{2.218557in}{1.186603in}}{\pgfqpoint{2.216790in}{1.185872in}}{\pgfqpoint{2.215488in}{1.184569in}}%
\pgfpathcurveto{\pgfqpoint{2.214186in}{1.183267in}}{\pgfqpoint{2.213454in}{1.181501in}}{\pgfqpoint{2.213454in}{1.179659in}}%
\pgfpathcurveto{\pgfqpoint{2.213454in}{1.177817in}}{\pgfqpoint{2.214186in}{1.176051in}}{\pgfqpoint{2.215488in}{1.174748in}}%
\pgfpathcurveto{\pgfqpoint{2.216790in}{1.173446in}}{\pgfqpoint{2.218557in}{1.172714in}}{\pgfqpoint{2.220398in}{1.172714in}}%
\pgfpathlineto{\pgfqpoint{2.220398in}{1.172714in}}%
\pgfpathclose%
\pgfusepath{stroke,fill}%
\end{pgfscope}%
\begin{pgfscope}%
\pgfpathrectangle{\pgfqpoint{0.661006in}{0.524170in}}{\pgfqpoint{4.194036in}{1.071446in}}%
\pgfusepath{clip}%
\pgfsetbuttcap%
\pgfsetroundjoin%
\definecolor{currentfill}{rgb}{0.458478,0.524535,0.713375}%
\pgfsetfillcolor{currentfill}%
\pgfsetfillopacity{0.700000}%
\pgfsetlinewidth{1.003750pt}%
\definecolor{currentstroke}{rgb}{0.458478,0.524535,0.713375}%
\pgfsetstrokecolor{currentstroke}%
\pgfsetstrokeopacity{0.700000}%
\pgfsetdash{}{0pt}%
\pgfpathmoveto{\pgfqpoint{2.230192in}{1.171135in}}%
\pgfpathcurveto{\pgfqpoint{2.232033in}{1.171135in}}{\pgfqpoint{2.233800in}{1.171867in}}{\pgfqpoint{2.235102in}{1.173169in}}%
\pgfpathcurveto{\pgfqpoint{2.236404in}{1.174471in}}{\pgfqpoint{2.237136in}{1.176238in}}{\pgfqpoint{2.237136in}{1.178079in}}%
\pgfpathcurveto{\pgfqpoint{2.237136in}{1.179921in}}{\pgfqpoint{2.236404in}{1.181687in}}{\pgfqpoint{2.235102in}{1.182990in}}%
\pgfpathcurveto{\pgfqpoint{2.233800in}{1.184292in}}{\pgfqpoint{2.232033in}{1.185024in}}{\pgfqpoint{2.230192in}{1.185024in}}%
\pgfpathcurveto{\pgfqpoint{2.228350in}{1.185024in}}{\pgfqpoint{2.226584in}{1.184292in}}{\pgfqpoint{2.225281in}{1.182990in}}%
\pgfpathcurveto{\pgfqpoint{2.223979in}{1.181687in}}{\pgfqpoint{2.223247in}{1.179921in}}{\pgfqpoint{2.223247in}{1.178079in}}%
\pgfpathcurveto{\pgfqpoint{2.223247in}{1.176238in}}{\pgfqpoint{2.223979in}{1.174471in}}{\pgfqpoint{2.225281in}{1.173169in}}%
\pgfpathcurveto{\pgfqpoint{2.226584in}{1.171867in}}{\pgfqpoint{2.228350in}{1.171135in}}{\pgfqpoint{2.230192in}{1.171135in}}%
\pgfpathlineto{\pgfqpoint{2.230192in}{1.171135in}}%
\pgfpathclose%
\pgfusepath{stroke,fill}%
\end{pgfscope}%
\begin{pgfscope}%
\pgfpathrectangle{\pgfqpoint{0.661006in}{0.524170in}}{\pgfqpoint{4.194036in}{1.071446in}}%
\pgfusepath{clip}%
\pgfsetbuttcap%
\pgfsetroundjoin%
\definecolor{currentfill}{rgb}{0.458478,0.524535,0.713375}%
\pgfsetfillcolor{currentfill}%
\pgfsetfillopacity{0.700000}%
\pgfsetlinewidth{1.003750pt}%
\definecolor{currentstroke}{rgb}{0.458478,0.524535,0.713375}%
\pgfsetstrokecolor{currentstroke}%
\pgfsetstrokeopacity{0.700000}%
\pgfsetdash{}{0pt}%
\pgfpathmoveto{\pgfqpoint{2.253444in}{1.165093in}}%
\pgfpathcurveto{\pgfqpoint{2.255285in}{1.165093in}}{\pgfqpoint{2.257052in}{1.165825in}}{\pgfqpoint{2.258354in}{1.167127in}}%
\pgfpathcurveto{\pgfqpoint{2.259656in}{1.168430in}}{\pgfqpoint{2.260388in}{1.170196in}}{\pgfqpoint{2.260388in}{1.172038in}}%
\pgfpathcurveto{\pgfqpoint{2.260388in}{1.173880in}}{\pgfqpoint{2.259656in}{1.175646in}}{\pgfqpoint{2.258354in}{1.176948in}}%
\pgfpathcurveto{\pgfqpoint{2.257052in}{1.178251in}}{\pgfqpoint{2.255285in}{1.178982in}}{\pgfqpoint{2.253444in}{1.178982in}}%
\pgfpathcurveto{\pgfqpoint{2.251602in}{1.178982in}}{\pgfqpoint{2.249835in}{1.178251in}}{\pgfqpoint{2.248533in}{1.176948in}}%
\pgfpathcurveto{\pgfqpoint{2.247231in}{1.175646in}}{\pgfqpoint{2.246499in}{1.173880in}}{\pgfqpoint{2.246499in}{1.172038in}}%
\pgfpathcurveto{\pgfqpoint{2.246499in}{1.170196in}}{\pgfqpoint{2.247231in}{1.168430in}}{\pgfqpoint{2.248533in}{1.167127in}}%
\pgfpathcurveto{\pgfqpoint{2.249835in}{1.165825in}}{\pgfqpoint{2.251602in}{1.165093in}}{\pgfqpoint{2.253444in}{1.165093in}}%
\pgfpathlineto{\pgfqpoint{2.253444in}{1.165093in}}%
\pgfpathclose%
\pgfusepath{stroke,fill}%
\end{pgfscope}%
\begin{pgfscope}%
\pgfpathrectangle{\pgfqpoint{0.661006in}{0.524170in}}{\pgfqpoint{4.194036in}{1.071446in}}%
\pgfusepath{clip}%
\pgfsetbuttcap%
\pgfsetroundjoin%
\definecolor{currentfill}{rgb}{0.455528,0.520027,0.709950}%
\pgfsetfillcolor{currentfill}%
\pgfsetfillopacity{0.700000}%
\pgfsetlinewidth{1.003750pt}%
\definecolor{currentstroke}{rgb}{0.455528,0.520027,0.709950}%
\pgfsetstrokecolor{currentstroke}%
\pgfsetstrokeopacity{0.700000}%
\pgfsetdash{}{0pt}%
\pgfpathmoveto{\pgfqpoint{2.278216in}{1.158048in}}%
\pgfpathcurveto{\pgfqpoint{2.280058in}{1.158048in}}{\pgfqpoint{2.281824in}{1.158780in}}{\pgfqpoint{2.283126in}{1.160082in}}%
\pgfpathcurveto{\pgfqpoint{2.284429in}{1.161385in}}{\pgfqpoint{2.285160in}{1.163151in}}{\pgfqpoint{2.285160in}{1.164993in}}%
\pgfpathcurveto{\pgfqpoint{2.285160in}{1.166834in}}{\pgfqpoint{2.284429in}{1.168601in}}{\pgfqpoint{2.283126in}{1.169903in}}%
\pgfpathcurveto{\pgfqpoint{2.281824in}{1.171206in}}{\pgfqpoint{2.280058in}{1.171937in}}{\pgfqpoint{2.278216in}{1.171937in}}%
\pgfpathcurveto{\pgfqpoint{2.276374in}{1.171937in}}{\pgfqpoint{2.274608in}{1.171206in}}{\pgfqpoint{2.273306in}{1.169903in}}%
\pgfpathcurveto{\pgfqpoint{2.272003in}{1.168601in}}{\pgfqpoint{2.271272in}{1.166834in}}{\pgfqpoint{2.271272in}{1.164993in}}%
\pgfpathcurveto{\pgfqpoint{2.271272in}{1.163151in}}{\pgfqpoint{2.272003in}{1.161385in}}{\pgfqpoint{2.273306in}{1.160082in}}%
\pgfpathcurveto{\pgfqpoint{2.274608in}{1.158780in}}{\pgfqpoint{2.276374in}{1.158048in}}{\pgfqpoint{2.278216in}{1.158048in}}%
\pgfpathlineto{\pgfqpoint{2.278216in}{1.158048in}}%
\pgfpathclose%
\pgfusepath{stroke,fill}%
\end{pgfscope}%
\begin{pgfscope}%
\pgfpathrectangle{\pgfqpoint{0.661006in}{0.524170in}}{\pgfqpoint{4.194036in}{1.071446in}}%
\pgfusepath{clip}%
\pgfsetbuttcap%
\pgfsetroundjoin%
\definecolor{currentfill}{rgb}{0.455528,0.520027,0.709950}%
\pgfsetfillcolor{currentfill}%
\pgfsetfillopacity{0.700000}%
\pgfsetlinewidth{1.003750pt}%
\definecolor{currentstroke}{rgb}{0.455528,0.520027,0.709950}%
\pgfsetstrokecolor{currentstroke}%
\pgfsetstrokeopacity{0.700000}%
\pgfsetdash{}{0pt}%
\pgfpathmoveto{\pgfqpoint{2.292856in}{1.155946in}}%
\pgfpathcurveto{\pgfqpoint{2.294698in}{1.155946in}}{\pgfqpoint{2.296465in}{1.156678in}}{\pgfqpoint{2.297767in}{1.157980in}}%
\pgfpathcurveto{\pgfqpoint{2.299069in}{1.159282in}}{\pgfqpoint{2.299801in}{1.161049in}}{\pgfqpoint{2.299801in}{1.162890in}}%
\pgfpathcurveto{\pgfqpoint{2.299801in}{1.164732in}}{\pgfqpoint{2.299069in}{1.166499in}}{\pgfqpoint{2.297767in}{1.167801in}}%
\pgfpathcurveto{\pgfqpoint{2.296465in}{1.169103in}}{\pgfqpoint{2.294698in}{1.169835in}}{\pgfqpoint{2.292856in}{1.169835in}}%
\pgfpathcurveto{\pgfqpoint{2.291015in}{1.169835in}}{\pgfqpoint{2.289248in}{1.169103in}}{\pgfqpoint{2.287946in}{1.167801in}}%
\pgfpathcurveto{\pgfqpoint{2.286644in}{1.166499in}}{\pgfqpoint{2.285912in}{1.164732in}}{\pgfqpoint{2.285912in}{1.162890in}}%
\pgfpathcurveto{\pgfqpoint{2.285912in}{1.161049in}}{\pgfqpoint{2.286644in}{1.159282in}}{\pgfqpoint{2.287946in}{1.157980in}}%
\pgfpathcurveto{\pgfqpoint{2.289248in}{1.156678in}}{\pgfqpoint{2.291015in}{1.155946in}}{\pgfqpoint{2.292856in}{1.155946in}}%
\pgfpathlineto{\pgfqpoint{2.292856in}{1.155946in}}%
\pgfpathclose%
\pgfusepath{stroke,fill}%
\end{pgfscope}%
\begin{pgfscope}%
\pgfpathrectangle{\pgfqpoint{0.661006in}{0.524170in}}{\pgfqpoint{4.194036in}{1.071446in}}%
\pgfusepath{clip}%
\pgfsetbuttcap%
\pgfsetroundjoin%
\definecolor{currentfill}{rgb}{0.455528,0.520027,0.709950}%
\pgfsetfillcolor{currentfill}%
\pgfsetfillopacity{0.700000}%
\pgfsetlinewidth{1.003750pt}%
\definecolor{currentstroke}{rgb}{0.455528,0.520027,0.709950}%
\pgfsetstrokecolor{currentstroke}%
\pgfsetstrokeopacity{0.700000}%
\pgfsetdash{}{0pt}%
\pgfpathmoveto{\pgfqpoint{2.274172in}{1.157441in}}%
\pgfpathcurveto{\pgfqpoint{2.276014in}{1.157441in}}{\pgfqpoint{2.277781in}{1.158172in}}{\pgfqpoint{2.279083in}{1.159475in}}%
\pgfpathcurveto{\pgfqpoint{2.280385in}{1.160777in}}{\pgfqpoint{2.281117in}{1.162544in}}{\pgfqpoint{2.281117in}{1.164385in}}%
\pgfpathcurveto{\pgfqpoint{2.281117in}{1.166227in}}{\pgfqpoint{2.280385in}{1.167993in}}{\pgfqpoint{2.279083in}{1.169296in}}%
\pgfpathcurveto{\pgfqpoint{2.277781in}{1.170598in}}{\pgfqpoint{2.276014in}{1.171330in}}{\pgfqpoint{2.274172in}{1.171330in}}%
\pgfpathcurveto{\pgfqpoint{2.272331in}{1.171330in}}{\pgfqpoint{2.270564in}{1.170598in}}{\pgfqpoint{2.269262in}{1.169296in}}%
\pgfpathcurveto{\pgfqpoint{2.267960in}{1.167993in}}{\pgfqpoint{2.267228in}{1.166227in}}{\pgfqpoint{2.267228in}{1.164385in}}%
\pgfpathcurveto{\pgfqpoint{2.267228in}{1.162544in}}{\pgfqpoint{2.267960in}{1.160777in}}{\pgfqpoint{2.269262in}{1.159475in}}%
\pgfpathcurveto{\pgfqpoint{2.270564in}{1.158172in}}{\pgfqpoint{2.272331in}{1.157441in}}{\pgfqpoint{2.274172in}{1.157441in}}%
\pgfpathlineto{\pgfqpoint{2.274172in}{1.157441in}}%
\pgfpathclose%
\pgfusepath{stroke,fill}%
\end{pgfscope}%
\begin{pgfscope}%
\pgfpathrectangle{\pgfqpoint{0.661006in}{0.524170in}}{\pgfqpoint{4.194036in}{1.071446in}}%
\pgfusepath{clip}%
\pgfsetbuttcap%
\pgfsetroundjoin%
\definecolor{currentfill}{rgb}{0.455528,0.520027,0.709950}%
\pgfsetfillcolor{currentfill}%
\pgfsetfillopacity{0.700000}%
\pgfsetlinewidth{1.003750pt}%
\definecolor{currentstroke}{rgb}{0.455528,0.520027,0.709950}%
\pgfsetstrokecolor{currentstroke}%
\pgfsetstrokeopacity{0.700000}%
\pgfsetdash{}{0pt}%
\pgfpathmoveto{\pgfqpoint{2.271616in}{1.158764in}}%
\pgfpathcurveto{\pgfqpoint{2.273458in}{1.158764in}}{\pgfqpoint{2.275224in}{1.159496in}}{\pgfqpoint{2.276527in}{1.160798in}}%
\pgfpathcurveto{\pgfqpoint{2.277829in}{1.162100in}}{\pgfqpoint{2.278561in}{1.163867in}}{\pgfqpoint{2.278561in}{1.165708in}}%
\pgfpathcurveto{\pgfqpoint{2.278561in}{1.167550in}}{\pgfqpoint{2.277829in}{1.169316in}}{\pgfqpoint{2.276527in}{1.170619in}}%
\pgfpathcurveto{\pgfqpoint{2.275224in}{1.171921in}}{\pgfqpoint{2.273458in}{1.172653in}}{\pgfqpoint{2.271616in}{1.172653in}}%
\pgfpathcurveto{\pgfqpoint{2.269775in}{1.172653in}}{\pgfqpoint{2.268008in}{1.171921in}}{\pgfqpoint{2.266706in}{1.170619in}}%
\pgfpathcurveto{\pgfqpoint{2.265404in}{1.169316in}}{\pgfqpoint{2.264672in}{1.167550in}}{\pgfqpoint{2.264672in}{1.165708in}}%
\pgfpathcurveto{\pgfqpoint{2.264672in}{1.163867in}}{\pgfqpoint{2.265404in}{1.162100in}}{\pgfqpoint{2.266706in}{1.160798in}}%
\pgfpathcurveto{\pgfqpoint{2.268008in}{1.159496in}}{\pgfqpoint{2.269775in}{1.158764in}}{\pgfqpoint{2.271616in}{1.158764in}}%
\pgfpathlineto{\pgfqpoint{2.271616in}{1.158764in}}%
\pgfpathclose%
\pgfusepath{stroke,fill}%
\end{pgfscope}%
\begin{pgfscope}%
\pgfpathrectangle{\pgfqpoint{0.661006in}{0.524170in}}{\pgfqpoint{4.194036in}{1.071446in}}%
\pgfusepath{clip}%
\pgfsetbuttcap%
\pgfsetroundjoin%
\definecolor{currentfill}{rgb}{0.455528,0.520027,0.709950}%
\pgfsetfillcolor{currentfill}%
\pgfsetfillopacity{0.700000}%
\pgfsetlinewidth{1.003750pt}%
\definecolor{currentstroke}{rgb}{0.455528,0.520027,0.709950}%
\pgfsetstrokecolor{currentstroke}%
\pgfsetstrokeopacity{0.700000}%
\pgfsetdash{}{0pt}%
\pgfpathmoveto{\pgfqpoint{2.286257in}{1.155559in}}%
\pgfpathcurveto{\pgfqpoint{2.288098in}{1.155559in}}{\pgfqpoint{2.289865in}{1.156290in}}{\pgfqpoint{2.291167in}{1.157593in}}%
\pgfpathcurveto{\pgfqpoint{2.292469in}{1.158895in}}{\pgfqpoint{2.293201in}{1.160661in}}{\pgfqpoint{2.293201in}{1.162503in}}%
\pgfpathcurveto{\pgfqpoint{2.293201in}{1.164345in}}{\pgfqpoint{2.292469in}{1.166111in}}{\pgfqpoint{2.291167in}{1.167414in}}%
\pgfpathcurveto{\pgfqpoint{2.289865in}{1.168716in}}{\pgfqpoint{2.288098in}{1.169448in}}{\pgfqpoint{2.286257in}{1.169448in}}%
\pgfpathcurveto{\pgfqpoint{2.284415in}{1.169448in}}{\pgfqpoint{2.282648in}{1.168716in}}{\pgfqpoint{2.281346in}{1.167414in}}%
\pgfpathcurveto{\pgfqpoint{2.280044in}{1.166111in}}{\pgfqpoint{2.279312in}{1.164345in}}{\pgfqpoint{2.279312in}{1.162503in}}%
\pgfpathcurveto{\pgfqpoint{2.279312in}{1.160661in}}{\pgfqpoint{2.280044in}{1.158895in}}{\pgfqpoint{2.281346in}{1.157593in}}%
\pgfpathcurveto{\pgfqpoint{2.282648in}{1.156290in}}{\pgfqpoint{2.284415in}{1.155559in}}{\pgfqpoint{2.286257in}{1.155559in}}%
\pgfpathlineto{\pgfqpoint{2.286257in}{1.155559in}}%
\pgfpathclose%
\pgfusepath{stroke,fill}%
\end{pgfscope}%
\begin{pgfscope}%
\pgfpathrectangle{\pgfqpoint{0.661006in}{0.524170in}}{\pgfqpoint{4.194036in}{1.071446in}}%
\pgfusepath{clip}%
\pgfsetbuttcap%
\pgfsetroundjoin%
\definecolor{currentfill}{rgb}{0.452589,0.515519,0.706494}%
\pgfsetfillcolor{currentfill}%
\pgfsetfillopacity{0.700000}%
\pgfsetlinewidth{1.003750pt}%
\definecolor{currentstroke}{rgb}{0.452589,0.515519,0.706494}%
\pgfsetstrokecolor{currentstroke}%
\pgfsetstrokeopacity{0.700000}%
\pgfsetdash{}{0pt}%
\pgfpathmoveto{\pgfqpoint{2.305498in}{1.149408in}}%
\pgfpathcurveto{\pgfqpoint{2.307340in}{1.149408in}}{\pgfqpoint{2.309106in}{1.150140in}}{\pgfqpoint{2.310409in}{1.151442in}}%
\pgfpathcurveto{\pgfqpoint{2.311711in}{1.152744in}}{\pgfqpoint{2.312443in}{1.154511in}}{\pgfqpoint{2.312443in}{1.156352in}}%
\pgfpathcurveto{\pgfqpoint{2.312443in}{1.158194in}}{\pgfqpoint{2.311711in}{1.159960in}}{\pgfqpoint{2.310409in}{1.161263in}}%
\pgfpathcurveto{\pgfqpoint{2.309106in}{1.162565in}}{\pgfqpoint{2.307340in}{1.163297in}}{\pgfqpoint{2.305498in}{1.163297in}}%
\pgfpathcurveto{\pgfqpoint{2.303656in}{1.163297in}}{\pgfqpoint{2.301890in}{1.162565in}}{\pgfqpoint{2.300588in}{1.161263in}}%
\pgfpathcurveto{\pgfqpoint{2.299285in}{1.159960in}}{\pgfqpoint{2.298554in}{1.158194in}}{\pgfqpoint{2.298554in}{1.156352in}}%
\pgfpathcurveto{\pgfqpoint{2.298554in}{1.154511in}}{\pgfqpoint{2.299285in}{1.152744in}}{\pgfqpoint{2.300588in}{1.151442in}}%
\pgfpathcurveto{\pgfqpoint{2.301890in}{1.150140in}}{\pgfqpoint{2.303656in}{1.149408in}}{\pgfqpoint{2.305498in}{1.149408in}}%
\pgfpathlineto{\pgfqpoint{2.305498in}{1.149408in}}%
\pgfpathclose%
\pgfusepath{stroke,fill}%
\end{pgfscope}%
\begin{pgfscope}%
\pgfpathrectangle{\pgfqpoint{0.661006in}{0.524170in}}{\pgfqpoint{4.194036in}{1.071446in}}%
\pgfusepath{clip}%
\pgfsetbuttcap%
\pgfsetroundjoin%
\definecolor{currentfill}{rgb}{0.452589,0.515519,0.706494}%
\pgfsetfillcolor{currentfill}%
\pgfsetfillopacity{0.700000}%
\pgfsetlinewidth{1.003750pt}%
\definecolor{currentstroke}{rgb}{0.452589,0.515519,0.706494}%
\pgfsetstrokecolor{currentstroke}%
\pgfsetstrokeopacity{0.700000}%
\pgfsetdash{}{0pt}%
\pgfpathmoveto{\pgfqpoint{2.326134in}{1.146826in}}%
\pgfpathcurveto{\pgfqpoint{2.327976in}{1.146826in}}{\pgfqpoint{2.329742in}{1.147558in}}{\pgfqpoint{2.331045in}{1.148860in}}%
\pgfpathcurveto{\pgfqpoint{2.332347in}{1.150162in}}{\pgfqpoint{2.333078in}{1.151929in}}{\pgfqpoint{2.333078in}{1.153771in}}%
\pgfpathcurveto{\pgfqpoint{2.333078in}{1.155612in}}{\pgfqpoint{2.332347in}{1.157379in}}{\pgfqpoint{2.331045in}{1.158681in}}%
\pgfpathcurveto{\pgfqpoint{2.329742in}{1.159983in}}{\pgfqpoint{2.327976in}{1.160715in}}{\pgfqpoint{2.326134in}{1.160715in}}%
\pgfpathcurveto{\pgfqpoint{2.324292in}{1.160715in}}{\pgfqpoint{2.322526in}{1.159983in}}{\pgfqpoint{2.321224in}{1.158681in}}%
\pgfpathcurveto{\pgfqpoint{2.319921in}{1.157379in}}{\pgfqpoint{2.319190in}{1.155612in}}{\pgfqpoint{2.319190in}{1.153771in}}%
\pgfpathcurveto{\pgfqpoint{2.319190in}{1.151929in}}{\pgfqpoint{2.319921in}{1.150162in}}{\pgfqpoint{2.321224in}{1.148860in}}%
\pgfpathcurveto{\pgfqpoint{2.322526in}{1.147558in}}{\pgfqpoint{2.324292in}{1.146826in}}{\pgfqpoint{2.326134in}{1.146826in}}%
\pgfpathlineto{\pgfqpoint{2.326134in}{1.146826in}}%
\pgfpathclose%
\pgfusepath{stroke,fill}%
\end{pgfscope}%
\begin{pgfscope}%
\pgfpathrectangle{\pgfqpoint{0.661006in}{0.524170in}}{\pgfqpoint{4.194036in}{1.071446in}}%
\pgfusepath{clip}%
\pgfsetbuttcap%
\pgfsetroundjoin%
\definecolor{currentfill}{rgb}{0.449662,0.511010,0.703009}%
\pgfsetfillcolor{currentfill}%
\pgfsetfillopacity{0.700000}%
\pgfsetlinewidth{1.003750pt}%
\definecolor{currentstroke}{rgb}{0.449662,0.511010,0.703009}%
\pgfsetstrokecolor{currentstroke}%
\pgfsetstrokeopacity{0.700000}%
\pgfsetdash{}{0pt}%
\pgfpathmoveto{\pgfqpoint{2.326738in}{1.146944in}}%
\pgfpathcurveto{\pgfqpoint{2.328580in}{1.146944in}}{\pgfqpoint{2.330346in}{1.147675in}}{\pgfqpoint{2.331649in}{1.148978in}}%
\pgfpathcurveto{\pgfqpoint{2.332951in}{1.150280in}}{\pgfqpoint{2.333683in}{1.152046in}}{\pgfqpoint{2.333683in}{1.153888in}}%
\pgfpathcurveto{\pgfqpoint{2.333683in}{1.155730in}}{\pgfqpoint{2.332951in}{1.157496in}}{\pgfqpoint{2.331649in}{1.158799in}}%
\pgfpathcurveto{\pgfqpoint{2.330346in}{1.160101in}}{\pgfqpoint{2.328580in}{1.160833in}}{\pgfqpoint{2.326738in}{1.160833in}}%
\pgfpathcurveto{\pgfqpoint{2.324897in}{1.160833in}}{\pgfqpoint{2.323130in}{1.160101in}}{\pgfqpoint{2.321828in}{1.158799in}}%
\pgfpathcurveto{\pgfqpoint{2.320526in}{1.157496in}}{\pgfqpoint{2.319794in}{1.155730in}}{\pgfqpoint{2.319794in}{1.153888in}}%
\pgfpathcurveto{\pgfqpoint{2.319794in}{1.152046in}}{\pgfqpoint{2.320526in}{1.150280in}}{\pgfqpoint{2.321828in}{1.148978in}}%
\pgfpathcurveto{\pgfqpoint{2.323130in}{1.147675in}}{\pgfqpoint{2.324897in}{1.146944in}}{\pgfqpoint{2.326738in}{1.146944in}}%
\pgfpathlineto{\pgfqpoint{2.326738in}{1.146944in}}%
\pgfpathclose%
\pgfusepath{stroke,fill}%
\end{pgfscope}%
\begin{pgfscope}%
\pgfpathrectangle{\pgfqpoint{0.661006in}{0.524170in}}{\pgfqpoint{4.194036in}{1.071446in}}%
\pgfusepath{clip}%
\pgfsetbuttcap%
\pgfsetroundjoin%
\definecolor{currentfill}{rgb}{0.449662,0.511010,0.703009}%
\pgfsetfillcolor{currentfill}%
\pgfsetfillopacity{0.700000}%
\pgfsetlinewidth{1.003750pt}%
\definecolor{currentstroke}{rgb}{0.449662,0.511010,0.703009}%
\pgfsetstrokecolor{currentstroke}%
\pgfsetstrokeopacity{0.700000}%
\pgfsetdash{}{0pt}%
\pgfpathmoveto{\pgfqpoint{2.338683in}{1.147317in}}%
\pgfpathcurveto{\pgfqpoint{2.340525in}{1.147317in}}{\pgfqpoint{2.342291in}{1.148049in}}{\pgfqpoint{2.343593in}{1.149351in}}%
\pgfpathcurveto{\pgfqpoint{2.344896in}{1.150654in}}{\pgfqpoint{2.345627in}{1.152420in}}{\pgfqpoint{2.345627in}{1.154262in}}%
\pgfpathcurveto{\pgfqpoint{2.345627in}{1.156104in}}{\pgfqpoint{2.344896in}{1.157870in}}{\pgfqpoint{2.343593in}{1.159172in}}%
\pgfpathcurveto{\pgfqpoint{2.342291in}{1.160475in}}{\pgfqpoint{2.340525in}{1.161206in}}{\pgfqpoint{2.338683in}{1.161206in}}%
\pgfpathcurveto{\pgfqpoint{2.336841in}{1.161206in}}{\pgfqpoint{2.335075in}{1.160475in}}{\pgfqpoint{2.333772in}{1.159172in}}%
\pgfpathcurveto{\pgfqpoint{2.332470in}{1.157870in}}{\pgfqpoint{2.331738in}{1.156104in}}{\pgfqpoint{2.331738in}{1.154262in}}%
\pgfpathcurveto{\pgfqpoint{2.331738in}{1.152420in}}{\pgfqpoint{2.332470in}{1.150654in}}{\pgfqpoint{2.333772in}{1.149351in}}%
\pgfpathcurveto{\pgfqpoint{2.335075in}{1.148049in}}{\pgfqpoint{2.336841in}{1.147317in}}{\pgfqpoint{2.338683in}{1.147317in}}%
\pgfpathlineto{\pgfqpoint{2.338683in}{1.147317in}}%
\pgfpathclose%
\pgfusepath{stroke,fill}%
\end{pgfscope}%
\begin{pgfscope}%
\pgfpathrectangle{\pgfqpoint{0.661006in}{0.524170in}}{\pgfqpoint{4.194036in}{1.071446in}}%
\pgfusepath{clip}%
\pgfsetbuttcap%
\pgfsetroundjoin%
\definecolor{currentfill}{rgb}{0.449662,0.511010,0.703009}%
\pgfsetfillcolor{currentfill}%
\pgfsetfillopacity{0.700000}%
\pgfsetlinewidth{1.003750pt}%
\definecolor{currentstroke}{rgb}{0.449662,0.511010,0.703009}%
\pgfsetstrokecolor{currentstroke}%
\pgfsetstrokeopacity{0.700000}%
\pgfsetdash{}{0pt}%
\pgfpathmoveto{\pgfqpoint{2.327528in}{1.145640in}}%
\pgfpathcurveto{\pgfqpoint{2.329370in}{1.145640in}}{\pgfqpoint{2.331137in}{1.146372in}}{\pgfqpoint{2.332439in}{1.147674in}}%
\pgfpathcurveto{\pgfqpoint{2.333741in}{1.148976in}}{\pgfqpoint{2.334473in}{1.150743in}}{\pgfqpoint{2.334473in}{1.152584in}}%
\pgfpathcurveto{\pgfqpoint{2.334473in}{1.154426in}}{\pgfqpoint{2.333741in}{1.156192in}}{\pgfqpoint{2.332439in}{1.157495in}}%
\pgfpathcurveto{\pgfqpoint{2.331137in}{1.158797in}}{\pgfqpoint{2.329370in}{1.159529in}}{\pgfqpoint{2.327528in}{1.159529in}}%
\pgfpathcurveto{\pgfqpoint{2.325687in}{1.159529in}}{\pgfqpoint{2.323920in}{1.158797in}}{\pgfqpoint{2.322618in}{1.157495in}}%
\pgfpathcurveto{\pgfqpoint{2.321316in}{1.156192in}}{\pgfqpoint{2.320584in}{1.154426in}}{\pgfqpoint{2.320584in}{1.152584in}}%
\pgfpathcurveto{\pgfqpoint{2.320584in}{1.150743in}}{\pgfqpoint{2.321316in}{1.148976in}}{\pgfqpoint{2.322618in}{1.147674in}}%
\pgfpathcurveto{\pgfqpoint{2.323920in}{1.146372in}}{\pgfqpoint{2.325687in}{1.145640in}}{\pgfqpoint{2.327528in}{1.145640in}}%
\pgfpathlineto{\pgfqpoint{2.327528in}{1.145640in}}%
\pgfpathclose%
\pgfusepath{stroke,fill}%
\end{pgfscope}%
\begin{pgfscope}%
\pgfpathrectangle{\pgfqpoint{0.661006in}{0.524170in}}{\pgfqpoint{4.194036in}{1.071446in}}%
\pgfusepath{clip}%
\pgfsetbuttcap%
\pgfsetroundjoin%
\definecolor{currentfill}{rgb}{0.449662,0.511010,0.703009}%
\pgfsetfillcolor{currentfill}%
\pgfsetfillopacity{0.700000}%
\pgfsetlinewidth{1.003750pt}%
\definecolor{currentstroke}{rgb}{0.449662,0.511010,0.703009}%
\pgfsetstrokecolor{currentstroke}%
\pgfsetstrokeopacity{0.700000}%
\pgfsetdash{}{0pt}%
\pgfpathmoveto{\pgfqpoint{2.352022in}{1.143265in}}%
\pgfpathcurveto{\pgfqpoint{2.353864in}{1.143265in}}{\pgfqpoint{2.355630in}{1.143996in}}{\pgfqpoint{2.356932in}{1.145299in}}%
\pgfpathcurveto{\pgfqpoint{2.358235in}{1.146601in}}{\pgfqpoint{2.358966in}{1.148367in}}{\pgfqpoint{2.358966in}{1.150209in}}%
\pgfpathcurveto{\pgfqpoint{2.358966in}{1.152051in}}{\pgfqpoint{2.358235in}{1.153817in}}{\pgfqpoint{2.356932in}{1.155120in}}%
\pgfpathcurveto{\pgfqpoint{2.355630in}{1.156422in}}{\pgfqpoint{2.353864in}{1.157154in}}{\pgfqpoint{2.352022in}{1.157154in}}%
\pgfpathcurveto{\pgfqpoint{2.350180in}{1.157154in}}{\pgfqpoint{2.348414in}{1.156422in}}{\pgfqpoint{2.347111in}{1.155120in}}%
\pgfpathcurveto{\pgfqpoint{2.345809in}{1.153817in}}{\pgfqpoint{2.345077in}{1.152051in}}{\pgfqpoint{2.345077in}{1.150209in}}%
\pgfpathcurveto{\pgfqpoint{2.345077in}{1.148367in}}{\pgfqpoint{2.345809in}{1.146601in}}{\pgfqpoint{2.347111in}{1.145299in}}%
\pgfpathcurveto{\pgfqpoint{2.348414in}{1.143996in}}{\pgfqpoint{2.350180in}{1.143265in}}{\pgfqpoint{2.352022in}{1.143265in}}%
\pgfpathlineto{\pgfqpoint{2.352022in}{1.143265in}}%
\pgfpathclose%
\pgfusepath{stroke,fill}%
\end{pgfscope}%
\begin{pgfscope}%
\pgfpathrectangle{\pgfqpoint{0.661006in}{0.524170in}}{\pgfqpoint{4.194036in}{1.071446in}}%
\pgfusepath{clip}%
\pgfsetbuttcap%
\pgfsetroundjoin%
\definecolor{currentfill}{rgb}{0.449662,0.511010,0.703009}%
\pgfsetfillcolor{currentfill}%
\pgfsetfillopacity{0.700000}%
\pgfsetlinewidth{1.003750pt}%
\definecolor{currentstroke}{rgb}{0.449662,0.511010,0.703009}%
\pgfsetstrokecolor{currentstroke}%
\pgfsetstrokeopacity{0.700000}%
\pgfsetdash{}{0pt}%
\pgfpathmoveto{\pgfqpoint{2.351371in}{1.142436in}}%
\pgfpathcurveto{\pgfqpoint{2.353213in}{1.142436in}}{\pgfqpoint{2.354979in}{1.143168in}}{\pgfqpoint{2.356282in}{1.144470in}}%
\pgfpathcurveto{\pgfqpoint{2.357584in}{1.145773in}}{\pgfqpoint{2.358316in}{1.147539in}}{\pgfqpoint{2.358316in}{1.149381in}}%
\pgfpathcurveto{\pgfqpoint{2.358316in}{1.151223in}}{\pgfqpoint{2.357584in}{1.152989in}}{\pgfqpoint{2.356282in}{1.154291in}}%
\pgfpathcurveto{\pgfqpoint{2.354979in}{1.155594in}}{\pgfqpoint{2.353213in}{1.156325in}}{\pgfqpoint{2.351371in}{1.156325in}}%
\pgfpathcurveto{\pgfqpoint{2.349529in}{1.156325in}}{\pgfqpoint{2.347763in}{1.155594in}}{\pgfqpoint{2.346461in}{1.154291in}}%
\pgfpathcurveto{\pgfqpoint{2.345158in}{1.152989in}}{\pgfqpoint{2.344427in}{1.151223in}}{\pgfqpoint{2.344427in}{1.149381in}}%
\pgfpathcurveto{\pgfqpoint{2.344427in}{1.147539in}}{\pgfqpoint{2.345158in}{1.145773in}}{\pgfqpoint{2.346461in}{1.144470in}}%
\pgfpathcurveto{\pgfqpoint{2.347763in}{1.143168in}}{\pgfqpoint{2.349529in}{1.142436in}}{\pgfqpoint{2.351371in}{1.142436in}}%
\pgfpathlineto{\pgfqpoint{2.351371in}{1.142436in}}%
\pgfpathclose%
\pgfusepath{stroke,fill}%
\end{pgfscope}%
\begin{pgfscope}%
\pgfpathrectangle{\pgfqpoint{0.661006in}{0.524170in}}{\pgfqpoint{4.194036in}{1.071446in}}%
\pgfusepath{clip}%
\pgfsetbuttcap%
\pgfsetroundjoin%
\definecolor{currentfill}{rgb}{0.446747,0.506501,0.699493}%
\pgfsetfillcolor{currentfill}%
\pgfsetfillopacity{0.700000}%
\pgfsetlinewidth{1.003750pt}%
\definecolor{currentstroke}{rgb}{0.446747,0.506501,0.699493}%
\pgfsetstrokecolor{currentstroke}%
\pgfsetstrokeopacity{0.700000}%
\pgfsetdash{}{0pt}%
\pgfpathmoveto{\pgfqpoint{2.362858in}{1.140646in}}%
\pgfpathcurveto{\pgfqpoint{2.364699in}{1.140646in}}{\pgfqpoint{2.366466in}{1.141378in}}{\pgfqpoint{2.367768in}{1.142680in}}%
\pgfpathcurveto{\pgfqpoint{2.369070in}{1.143982in}}{\pgfqpoint{2.369802in}{1.145749in}}{\pgfqpoint{2.369802in}{1.147590in}}%
\pgfpathcurveto{\pgfqpoint{2.369802in}{1.149432in}}{\pgfqpoint{2.369070in}{1.151198in}}{\pgfqpoint{2.367768in}{1.152501in}}%
\pgfpathcurveto{\pgfqpoint{2.366466in}{1.153803in}}{\pgfqpoint{2.364699in}{1.154535in}}{\pgfqpoint{2.362858in}{1.154535in}}%
\pgfpathcurveto{\pgfqpoint{2.361016in}{1.154535in}}{\pgfqpoint{2.359250in}{1.153803in}}{\pgfqpoint{2.357947in}{1.152501in}}%
\pgfpathcurveto{\pgfqpoint{2.356645in}{1.151198in}}{\pgfqpoint{2.355913in}{1.149432in}}{\pgfqpoint{2.355913in}{1.147590in}}%
\pgfpathcurveto{\pgfqpoint{2.355913in}{1.145749in}}{\pgfqpoint{2.356645in}{1.143982in}}{\pgfqpoint{2.357947in}{1.142680in}}%
\pgfpathcurveto{\pgfqpoint{2.359250in}{1.141378in}}{\pgfqpoint{2.361016in}{1.140646in}}{\pgfqpoint{2.362858in}{1.140646in}}%
\pgfpathlineto{\pgfqpoint{2.362858in}{1.140646in}}%
\pgfpathclose%
\pgfusepath{stroke,fill}%
\end{pgfscope}%
\begin{pgfscope}%
\pgfpathrectangle{\pgfqpoint{0.661006in}{0.524170in}}{\pgfqpoint{4.194036in}{1.071446in}}%
\pgfusepath{clip}%
\pgfsetbuttcap%
\pgfsetroundjoin%
\definecolor{currentfill}{rgb}{0.446747,0.506501,0.699493}%
\pgfsetfillcolor{currentfill}%
\pgfsetfillopacity{0.700000}%
\pgfsetlinewidth{1.003750pt}%
\definecolor{currentstroke}{rgb}{0.446747,0.506501,0.699493}%
\pgfsetstrokecolor{currentstroke}%
\pgfsetstrokeopacity{0.700000}%
\pgfsetdash{}{0pt}%
\pgfpathmoveto{\pgfqpoint{2.357832in}{1.141138in}}%
\pgfpathcurveto{\pgfqpoint{2.359673in}{1.141138in}}{\pgfqpoint{2.361440in}{1.141870in}}{\pgfqpoint{2.362742in}{1.143172in}}%
\pgfpathcurveto{\pgfqpoint{2.364044in}{1.144474in}}{\pgfqpoint{2.364776in}{1.146241in}}{\pgfqpoint{2.364776in}{1.148083in}}%
\pgfpathcurveto{\pgfqpoint{2.364776in}{1.149924in}}{\pgfqpoint{2.364044in}{1.151691in}}{\pgfqpoint{2.362742in}{1.152993in}}%
\pgfpathcurveto{\pgfqpoint{2.361440in}{1.154295in}}{\pgfqpoint{2.359673in}{1.155027in}}{\pgfqpoint{2.357832in}{1.155027in}}%
\pgfpathcurveto{\pgfqpoint{2.355990in}{1.155027in}}{\pgfqpoint{2.354223in}{1.154295in}}{\pgfqpoint{2.352921in}{1.152993in}}%
\pgfpathcurveto{\pgfqpoint{2.351619in}{1.151691in}}{\pgfqpoint{2.350887in}{1.149924in}}{\pgfqpoint{2.350887in}{1.148083in}}%
\pgfpathcurveto{\pgfqpoint{2.350887in}{1.146241in}}{\pgfqpoint{2.351619in}{1.144474in}}{\pgfqpoint{2.352921in}{1.143172in}}%
\pgfpathcurveto{\pgfqpoint{2.354223in}{1.141870in}}{\pgfqpoint{2.355990in}{1.141138in}}{\pgfqpoint{2.357832in}{1.141138in}}%
\pgfpathlineto{\pgfqpoint{2.357832in}{1.141138in}}%
\pgfpathclose%
\pgfusepath{stroke,fill}%
\end{pgfscope}%
\begin{pgfscope}%
\pgfpathrectangle{\pgfqpoint{0.661006in}{0.524170in}}{\pgfqpoint{4.194036in}{1.071446in}}%
\pgfusepath{clip}%
\pgfsetbuttcap%
\pgfsetroundjoin%
\definecolor{currentfill}{rgb}{0.443842,0.501993,0.695947}%
\pgfsetfillcolor{currentfill}%
\pgfsetfillopacity{0.700000}%
\pgfsetlinewidth{1.003750pt}%
\definecolor{currentstroke}{rgb}{0.443842,0.501993,0.695947}%
\pgfsetstrokecolor{currentstroke}%
\pgfsetstrokeopacity{0.700000}%
\pgfsetdash{}{0pt}%
\pgfpathmoveto{\pgfqpoint{2.356019in}{1.140835in}}%
\pgfpathcurveto{\pgfqpoint{2.357861in}{1.140835in}}{\pgfqpoint{2.359627in}{1.141567in}}{\pgfqpoint{2.360929in}{1.142869in}}%
\pgfpathcurveto{\pgfqpoint{2.362232in}{1.144171in}}{\pgfqpoint{2.362963in}{1.145938in}}{\pgfqpoint{2.362963in}{1.147779in}}%
\pgfpathcurveto{\pgfqpoint{2.362963in}{1.149621in}}{\pgfqpoint{2.362232in}{1.151387in}}{\pgfqpoint{2.360929in}{1.152690in}}%
\pgfpathcurveto{\pgfqpoint{2.359627in}{1.153992in}}{\pgfqpoint{2.357861in}{1.154724in}}{\pgfqpoint{2.356019in}{1.154724in}}%
\pgfpathcurveto{\pgfqpoint{2.354177in}{1.154724in}}{\pgfqpoint{2.352411in}{1.153992in}}{\pgfqpoint{2.351108in}{1.152690in}}%
\pgfpathcurveto{\pgfqpoint{2.349806in}{1.151387in}}{\pgfqpoint{2.349074in}{1.149621in}}{\pgfqpoint{2.349074in}{1.147779in}}%
\pgfpathcurveto{\pgfqpoint{2.349074in}{1.145938in}}{\pgfqpoint{2.349806in}{1.144171in}}{\pgfqpoint{2.351108in}{1.142869in}}%
\pgfpathcurveto{\pgfqpoint{2.352411in}{1.141567in}}{\pgfqpoint{2.354177in}{1.140835in}}{\pgfqpoint{2.356019in}{1.140835in}}%
\pgfpathlineto{\pgfqpoint{2.356019in}{1.140835in}}%
\pgfpathclose%
\pgfusepath{stroke,fill}%
\end{pgfscope}%
\begin{pgfscope}%
\pgfpathrectangle{\pgfqpoint{0.661006in}{0.524170in}}{\pgfqpoint{4.194036in}{1.071446in}}%
\pgfusepath{clip}%
\pgfsetbuttcap%
\pgfsetroundjoin%
\definecolor{currentfill}{rgb}{0.443842,0.501993,0.695947}%
\pgfsetfillcolor{currentfill}%
\pgfsetfillopacity{0.700000}%
\pgfsetlinewidth{1.003750pt}%
\definecolor{currentstroke}{rgb}{0.443842,0.501993,0.695947}%
\pgfsetstrokecolor{currentstroke}%
\pgfsetstrokeopacity{0.700000}%
\pgfsetdash{}{0pt}%
\pgfpathmoveto{\pgfqpoint{2.363874in}{1.138753in}}%
\pgfpathcurveto{\pgfqpoint{2.365715in}{1.138753in}}{\pgfqpoint{2.367482in}{1.139484in}}{\pgfqpoint{2.368784in}{1.140787in}}%
\pgfpathcurveto{\pgfqpoint{2.370086in}{1.142089in}}{\pgfqpoint{2.370818in}{1.143856in}}{\pgfqpoint{2.370818in}{1.145697in}}%
\pgfpathcurveto{\pgfqpoint{2.370818in}{1.147539in}}{\pgfqpoint{2.370086in}{1.149305in}}{\pgfqpoint{2.368784in}{1.150608in}}%
\pgfpathcurveto{\pgfqpoint{2.367482in}{1.151910in}}{\pgfqpoint{2.365715in}{1.152642in}}{\pgfqpoint{2.363874in}{1.152642in}}%
\pgfpathcurveto{\pgfqpoint{2.362032in}{1.152642in}}{\pgfqpoint{2.360265in}{1.151910in}}{\pgfqpoint{2.358963in}{1.150608in}}%
\pgfpathcurveto{\pgfqpoint{2.357661in}{1.149305in}}{\pgfqpoint{2.356929in}{1.147539in}}{\pgfqpoint{2.356929in}{1.145697in}}%
\pgfpathcurveto{\pgfqpoint{2.356929in}{1.143856in}}{\pgfqpoint{2.357661in}{1.142089in}}{\pgfqpoint{2.358963in}{1.140787in}}%
\pgfpathcurveto{\pgfqpoint{2.360265in}{1.139484in}}{\pgfqpoint{2.362032in}{1.138753in}}{\pgfqpoint{2.363874in}{1.138753in}}%
\pgfpathlineto{\pgfqpoint{2.363874in}{1.138753in}}%
\pgfpathclose%
\pgfusepath{stroke,fill}%
\end{pgfscope}%
\begin{pgfscope}%
\pgfpathrectangle{\pgfqpoint{0.661006in}{0.524170in}}{\pgfqpoint{4.194036in}{1.071446in}}%
\pgfusepath{clip}%
\pgfsetbuttcap%
\pgfsetroundjoin%
\definecolor{currentfill}{rgb}{0.443842,0.501993,0.695947}%
\pgfsetfillcolor{currentfill}%
\pgfsetfillopacity{0.700000}%
\pgfsetlinewidth{1.003750pt}%
\definecolor{currentstroke}{rgb}{0.443842,0.501993,0.695947}%
\pgfsetstrokecolor{currentstroke}%
\pgfsetstrokeopacity{0.700000}%
\pgfsetdash{}{0pt}%
\pgfpathmoveto{\pgfqpoint{2.389994in}{1.134617in}}%
\pgfpathcurveto{\pgfqpoint{2.391835in}{1.134617in}}{\pgfqpoint{2.393602in}{1.135349in}}{\pgfqpoint{2.394904in}{1.136651in}}%
\pgfpathcurveto{\pgfqpoint{2.396207in}{1.137954in}}{\pgfqpoint{2.396938in}{1.139720in}}{\pgfqpoint{2.396938in}{1.141562in}}%
\pgfpathcurveto{\pgfqpoint{2.396938in}{1.143404in}}{\pgfqpoint{2.396207in}{1.145170in}}{\pgfqpoint{2.394904in}{1.146472in}}%
\pgfpathcurveto{\pgfqpoint{2.393602in}{1.147775in}}{\pgfqpoint{2.391835in}{1.148506in}}{\pgfqpoint{2.389994in}{1.148506in}}%
\pgfpathcurveto{\pgfqpoint{2.388152in}{1.148506in}}{\pgfqpoint{2.386386in}{1.147775in}}{\pgfqpoint{2.385083in}{1.146472in}}%
\pgfpathcurveto{\pgfqpoint{2.383781in}{1.145170in}}{\pgfqpoint{2.383049in}{1.143404in}}{\pgfqpoint{2.383049in}{1.141562in}}%
\pgfpathcurveto{\pgfqpoint{2.383049in}{1.139720in}}{\pgfqpoint{2.383781in}{1.137954in}}{\pgfqpoint{2.385083in}{1.136651in}}%
\pgfpathcurveto{\pgfqpoint{2.386386in}{1.135349in}}{\pgfqpoint{2.388152in}{1.134617in}}{\pgfqpoint{2.389994in}{1.134617in}}%
\pgfpathlineto{\pgfqpoint{2.389994in}{1.134617in}}%
\pgfpathclose%
\pgfusepath{stroke,fill}%
\end{pgfscope}%
\begin{pgfscope}%
\pgfpathrectangle{\pgfqpoint{0.661006in}{0.524170in}}{\pgfqpoint{4.194036in}{1.071446in}}%
\pgfusepath{clip}%
\pgfsetbuttcap%
\pgfsetroundjoin%
\definecolor{currentfill}{rgb}{0.443842,0.501993,0.695947}%
\pgfsetfillcolor{currentfill}%
\pgfsetfillopacity{0.700000}%
\pgfsetlinewidth{1.003750pt}%
\definecolor{currentstroke}{rgb}{0.443842,0.501993,0.695947}%
\pgfsetstrokecolor{currentstroke}%
\pgfsetstrokeopacity{0.700000}%
\pgfsetdash{}{0pt}%
\pgfpathmoveto{\pgfqpoint{2.397570in}{1.131795in}}%
\pgfpathcurveto{\pgfqpoint{2.399411in}{1.131795in}}{\pgfqpoint{2.401178in}{1.132527in}}{\pgfqpoint{2.402480in}{1.133829in}}%
\pgfpathcurveto{\pgfqpoint{2.403782in}{1.135132in}}{\pgfqpoint{2.404514in}{1.136898in}}{\pgfqpoint{2.404514in}{1.138740in}}%
\pgfpathcurveto{\pgfqpoint{2.404514in}{1.140581in}}{\pgfqpoint{2.403782in}{1.142348in}}{\pgfqpoint{2.402480in}{1.143650in}}%
\pgfpathcurveto{\pgfqpoint{2.401178in}{1.144952in}}{\pgfqpoint{2.399411in}{1.145684in}}{\pgfqpoint{2.397570in}{1.145684in}}%
\pgfpathcurveto{\pgfqpoint{2.395728in}{1.145684in}}{\pgfqpoint{2.393961in}{1.144952in}}{\pgfqpoint{2.392659in}{1.143650in}}%
\pgfpathcurveto{\pgfqpoint{2.391357in}{1.142348in}}{\pgfqpoint{2.390625in}{1.140581in}}{\pgfqpoint{2.390625in}{1.138740in}}%
\pgfpathcurveto{\pgfqpoint{2.390625in}{1.136898in}}{\pgfqpoint{2.391357in}{1.135132in}}{\pgfqpoint{2.392659in}{1.133829in}}%
\pgfpathcurveto{\pgfqpoint{2.393961in}{1.132527in}}{\pgfqpoint{2.395728in}{1.131795in}}{\pgfqpoint{2.397570in}{1.131795in}}%
\pgfpathlineto{\pgfqpoint{2.397570in}{1.131795in}}%
\pgfpathclose%
\pgfusepath{stroke,fill}%
\end{pgfscope}%
\begin{pgfscope}%
\pgfpathrectangle{\pgfqpoint{0.661006in}{0.524170in}}{\pgfqpoint{4.194036in}{1.071446in}}%
\pgfusepath{clip}%
\pgfsetbuttcap%
\pgfsetroundjoin%
\definecolor{currentfill}{rgb}{0.443842,0.501993,0.695947}%
\pgfsetfillcolor{currentfill}%
\pgfsetfillopacity{0.700000}%
\pgfsetlinewidth{1.003750pt}%
\definecolor{currentstroke}{rgb}{0.443842,0.501993,0.695947}%
\pgfsetstrokecolor{currentstroke}%
\pgfsetstrokeopacity{0.700000}%
\pgfsetdash{}{0pt}%
\pgfpathmoveto{\pgfqpoint{2.405471in}{1.129871in}}%
\pgfpathcurveto{\pgfqpoint{2.407312in}{1.129871in}}{\pgfqpoint{2.409079in}{1.130603in}}{\pgfqpoint{2.410381in}{1.131905in}}%
\pgfpathcurveto{\pgfqpoint{2.411683in}{1.133207in}}{\pgfqpoint{2.412415in}{1.134974in}}{\pgfqpoint{2.412415in}{1.136815in}}%
\pgfpathcurveto{\pgfqpoint{2.412415in}{1.138657in}}{\pgfqpoint{2.411683in}{1.140424in}}{\pgfqpoint{2.410381in}{1.141726in}}%
\pgfpathcurveto{\pgfqpoint{2.409079in}{1.143028in}}{\pgfqpoint{2.407312in}{1.143760in}}{\pgfqpoint{2.405471in}{1.143760in}}%
\pgfpathcurveto{\pgfqpoint{2.403629in}{1.143760in}}{\pgfqpoint{2.401862in}{1.143028in}}{\pgfqpoint{2.400560in}{1.141726in}}%
\pgfpathcurveto{\pgfqpoint{2.399258in}{1.140424in}}{\pgfqpoint{2.398526in}{1.138657in}}{\pgfqpoint{2.398526in}{1.136815in}}%
\pgfpathcurveto{\pgfqpoint{2.398526in}{1.134974in}}{\pgfqpoint{2.399258in}{1.133207in}}{\pgfqpoint{2.400560in}{1.131905in}}%
\pgfpathcurveto{\pgfqpoint{2.401862in}{1.130603in}}{\pgfqpoint{2.403629in}{1.129871in}}{\pgfqpoint{2.405471in}{1.129871in}}%
\pgfpathlineto{\pgfqpoint{2.405471in}{1.129871in}}%
\pgfpathclose%
\pgfusepath{stroke,fill}%
\end{pgfscope}%
\begin{pgfscope}%
\pgfpathrectangle{\pgfqpoint{0.661006in}{0.524170in}}{\pgfqpoint{4.194036in}{1.071446in}}%
\pgfusepath{clip}%
\pgfsetbuttcap%
\pgfsetroundjoin%
\definecolor{currentfill}{rgb}{0.440947,0.497484,0.692371}%
\pgfsetfillcolor{currentfill}%
\pgfsetfillopacity{0.700000}%
\pgfsetlinewidth{1.003750pt}%
\definecolor{currentstroke}{rgb}{0.440947,0.497484,0.692371}%
\pgfsetstrokecolor{currentstroke}%
\pgfsetstrokeopacity{0.700000}%
\pgfsetdash{}{0pt}%
\pgfpathmoveto{\pgfqpoint{2.406168in}{1.129677in}}%
\pgfpathcurveto{\pgfqpoint{2.408010in}{1.129677in}}{\pgfqpoint{2.409776in}{1.130408in}}{\pgfqpoint{2.411078in}{1.131711in}}%
\pgfpathcurveto{\pgfqpoint{2.412381in}{1.133013in}}{\pgfqpoint{2.413112in}{1.134779in}}{\pgfqpoint{2.413112in}{1.136621in}}%
\pgfpathcurveto{\pgfqpoint{2.413112in}{1.138463in}}{\pgfqpoint{2.412381in}{1.140229in}}{\pgfqpoint{2.411078in}{1.141531in}}%
\pgfpathcurveto{\pgfqpoint{2.409776in}{1.142834in}}{\pgfqpoint{2.408010in}{1.143565in}}{\pgfqpoint{2.406168in}{1.143565in}}%
\pgfpathcurveto{\pgfqpoint{2.404326in}{1.143565in}}{\pgfqpoint{2.402560in}{1.142834in}}{\pgfqpoint{2.401257in}{1.141531in}}%
\pgfpathcurveto{\pgfqpoint{2.399955in}{1.140229in}}{\pgfqpoint{2.399223in}{1.138463in}}{\pgfqpoint{2.399223in}{1.136621in}}%
\pgfpathcurveto{\pgfqpoint{2.399223in}{1.134779in}}{\pgfqpoint{2.399955in}{1.133013in}}{\pgfqpoint{2.401257in}{1.131711in}}%
\pgfpathcurveto{\pgfqpoint{2.402560in}{1.130408in}}{\pgfqpoint{2.404326in}{1.129677in}}{\pgfqpoint{2.406168in}{1.129677in}}%
\pgfpathlineto{\pgfqpoint{2.406168in}{1.129677in}}%
\pgfpathclose%
\pgfusepath{stroke,fill}%
\end{pgfscope}%
\begin{pgfscope}%
\pgfpathrectangle{\pgfqpoint{0.661006in}{0.524170in}}{\pgfqpoint{4.194036in}{1.071446in}}%
\pgfusepath{clip}%
\pgfsetbuttcap%
\pgfsetroundjoin%
\definecolor{currentfill}{rgb}{0.440947,0.497484,0.692371}%
\pgfsetfillcolor{currentfill}%
\pgfsetfillopacity{0.700000}%
\pgfsetlinewidth{1.003750pt}%
\definecolor{currentstroke}{rgb}{0.440947,0.497484,0.692371}%
\pgfsetstrokecolor{currentstroke}%
\pgfsetstrokeopacity{0.700000}%
\pgfsetdash{}{0pt}%
\pgfpathmoveto{\pgfqpoint{2.401567in}{1.129878in}}%
\pgfpathcurveto{\pgfqpoint{2.403408in}{1.129878in}}{\pgfqpoint{2.405175in}{1.130610in}}{\pgfqpoint{2.406477in}{1.131912in}}%
\pgfpathcurveto{\pgfqpoint{2.407779in}{1.133214in}}{\pgfqpoint{2.408511in}{1.134981in}}{\pgfqpoint{2.408511in}{1.136823in}}%
\pgfpathcurveto{\pgfqpoint{2.408511in}{1.138664in}}{\pgfqpoint{2.407779in}{1.140431in}}{\pgfqpoint{2.406477in}{1.141733in}}%
\pgfpathcurveto{\pgfqpoint{2.405175in}{1.143035in}}{\pgfqpoint{2.403408in}{1.143767in}}{\pgfqpoint{2.401567in}{1.143767in}}%
\pgfpathcurveto{\pgfqpoint{2.399725in}{1.143767in}}{\pgfqpoint{2.397958in}{1.143035in}}{\pgfqpoint{2.396656in}{1.141733in}}%
\pgfpathcurveto{\pgfqpoint{2.395354in}{1.140431in}}{\pgfqpoint{2.394622in}{1.138664in}}{\pgfqpoint{2.394622in}{1.136823in}}%
\pgfpathcurveto{\pgfqpoint{2.394622in}{1.134981in}}{\pgfqpoint{2.395354in}{1.133214in}}{\pgfqpoint{2.396656in}{1.131912in}}%
\pgfpathcurveto{\pgfqpoint{2.397958in}{1.130610in}}{\pgfqpoint{2.399725in}{1.129878in}}{\pgfqpoint{2.401567in}{1.129878in}}%
\pgfpathlineto{\pgfqpoint{2.401567in}{1.129878in}}%
\pgfpathclose%
\pgfusepath{stroke,fill}%
\end{pgfscope}%
\begin{pgfscope}%
\pgfpathrectangle{\pgfqpoint{0.661006in}{0.524170in}}{\pgfqpoint{4.194036in}{1.071446in}}%
\pgfusepath{clip}%
\pgfsetbuttcap%
\pgfsetroundjoin%
\definecolor{currentfill}{rgb}{0.438063,0.492977,0.688764}%
\pgfsetfillcolor{currentfill}%
\pgfsetfillopacity{0.700000}%
\pgfsetlinewidth{1.003750pt}%
\definecolor{currentstroke}{rgb}{0.438063,0.492977,0.688764}%
\pgfsetstrokecolor{currentstroke}%
\pgfsetstrokeopacity{0.700000}%
\pgfsetdash{}{0pt}%
\pgfpathmoveto{\pgfqpoint{2.415417in}{1.127710in}}%
\pgfpathcurveto{\pgfqpoint{2.417259in}{1.127710in}}{\pgfqpoint{2.419025in}{1.128442in}}{\pgfqpoint{2.420327in}{1.129744in}}%
\pgfpathcurveto{\pgfqpoint{2.421630in}{1.131047in}}{\pgfqpoint{2.422361in}{1.132813in}}{\pgfqpoint{2.422361in}{1.134655in}}%
\pgfpathcurveto{\pgfqpoint{2.422361in}{1.136497in}}{\pgfqpoint{2.421630in}{1.138263in}}{\pgfqpoint{2.420327in}{1.139565in}}%
\pgfpathcurveto{\pgfqpoint{2.419025in}{1.140868in}}{\pgfqpoint{2.417259in}{1.141599in}}{\pgfqpoint{2.415417in}{1.141599in}}%
\pgfpathcurveto{\pgfqpoint{2.413575in}{1.141599in}}{\pgfqpoint{2.411809in}{1.140868in}}{\pgfqpoint{2.410506in}{1.139565in}}%
\pgfpathcurveto{\pgfqpoint{2.409204in}{1.138263in}}{\pgfqpoint{2.408472in}{1.136497in}}{\pgfqpoint{2.408472in}{1.134655in}}%
\pgfpathcurveto{\pgfqpoint{2.408472in}{1.132813in}}{\pgfqpoint{2.409204in}{1.131047in}}{\pgfqpoint{2.410506in}{1.129744in}}%
\pgfpathcurveto{\pgfqpoint{2.411809in}{1.128442in}}{\pgfqpoint{2.413575in}{1.127710in}}{\pgfqpoint{2.415417in}{1.127710in}}%
\pgfpathlineto{\pgfqpoint{2.415417in}{1.127710in}}%
\pgfpathclose%
\pgfusepath{stroke,fill}%
\end{pgfscope}%
\begin{pgfscope}%
\pgfpathrectangle{\pgfqpoint{0.661006in}{0.524170in}}{\pgfqpoint{4.194036in}{1.071446in}}%
\pgfusepath{clip}%
\pgfsetbuttcap%
\pgfsetroundjoin%
\definecolor{currentfill}{rgb}{0.438063,0.492977,0.688764}%
\pgfsetfillcolor{currentfill}%
\pgfsetfillopacity{0.700000}%
\pgfsetlinewidth{1.003750pt}%
\definecolor{currentstroke}{rgb}{0.438063,0.492977,0.688764}%
\pgfsetstrokecolor{currentstroke}%
\pgfsetstrokeopacity{0.700000}%
\pgfsetdash{}{0pt}%
\pgfpathmoveto{\pgfqpoint{2.420111in}{1.125697in}}%
\pgfpathcurveto{\pgfqpoint{2.421953in}{1.125697in}}{\pgfqpoint{2.423719in}{1.126429in}}{\pgfqpoint{2.425021in}{1.127731in}}%
\pgfpathcurveto{\pgfqpoint{2.426324in}{1.129033in}}{\pgfqpoint{2.427055in}{1.130800in}}{\pgfqpoint{2.427055in}{1.132641in}}%
\pgfpathcurveto{\pgfqpoint{2.427055in}{1.134483in}}{\pgfqpoint{2.426324in}{1.136249in}}{\pgfqpoint{2.425021in}{1.137552in}}%
\pgfpathcurveto{\pgfqpoint{2.423719in}{1.138854in}}{\pgfqpoint{2.421953in}{1.139586in}}{\pgfqpoint{2.420111in}{1.139586in}}%
\pgfpathcurveto{\pgfqpoint{2.418269in}{1.139586in}}{\pgfqpoint{2.416503in}{1.138854in}}{\pgfqpoint{2.415201in}{1.137552in}}%
\pgfpathcurveto{\pgfqpoint{2.413898in}{1.136249in}}{\pgfqpoint{2.413167in}{1.134483in}}{\pgfqpoint{2.413167in}{1.132641in}}%
\pgfpathcurveto{\pgfqpoint{2.413167in}{1.130800in}}{\pgfqpoint{2.413898in}{1.129033in}}{\pgfqpoint{2.415201in}{1.127731in}}%
\pgfpathcurveto{\pgfqpoint{2.416503in}{1.126429in}}{\pgfqpoint{2.418269in}{1.125697in}}{\pgfqpoint{2.420111in}{1.125697in}}%
\pgfpathlineto{\pgfqpoint{2.420111in}{1.125697in}}%
\pgfpathclose%
\pgfusepath{stroke,fill}%
\end{pgfscope}%
\begin{pgfscope}%
\pgfpathrectangle{\pgfqpoint{0.661006in}{0.524170in}}{\pgfqpoint{4.194036in}{1.071446in}}%
\pgfusepath{clip}%
\pgfsetbuttcap%
\pgfsetroundjoin%
\definecolor{currentfill}{rgb}{0.438063,0.492977,0.688764}%
\pgfsetfillcolor{currentfill}%
\pgfsetfillopacity{0.700000}%
\pgfsetlinewidth{1.003750pt}%
\definecolor{currentstroke}{rgb}{0.438063,0.492977,0.688764}%
\pgfsetstrokecolor{currentstroke}%
\pgfsetstrokeopacity{0.700000}%
\pgfsetdash{}{0pt}%
\pgfpathmoveto{\pgfqpoint{2.418996in}{1.125473in}}%
\pgfpathcurveto{\pgfqpoint{2.420837in}{1.125473in}}{\pgfqpoint{2.422604in}{1.126205in}}{\pgfqpoint{2.423906in}{1.127507in}}%
\pgfpathcurveto{\pgfqpoint{2.425208in}{1.128810in}}{\pgfqpoint{2.425940in}{1.130576in}}{\pgfqpoint{2.425940in}{1.132418in}}%
\pgfpathcurveto{\pgfqpoint{2.425940in}{1.134259in}}{\pgfqpoint{2.425208in}{1.136026in}}{\pgfqpoint{2.423906in}{1.137328in}}%
\pgfpathcurveto{\pgfqpoint{2.422604in}{1.138630in}}{\pgfqpoint{2.420837in}{1.139362in}}{\pgfqpoint{2.418996in}{1.139362in}}%
\pgfpathcurveto{\pgfqpoint{2.417154in}{1.139362in}}{\pgfqpoint{2.415387in}{1.138630in}}{\pgfqpoint{2.414085in}{1.137328in}}%
\pgfpathcurveto{\pgfqpoint{2.412783in}{1.136026in}}{\pgfqpoint{2.412051in}{1.134259in}}{\pgfqpoint{2.412051in}{1.132418in}}%
\pgfpathcurveto{\pgfqpoint{2.412051in}{1.130576in}}{\pgfqpoint{2.412783in}{1.128810in}}{\pgfqpoint{2.414085in}{1.127507in}}%
\pgfpathcurveto{\pgfqpoint{2.415387in}{1.126205in}}{\pgfqpoint{2.417154in}{1.125473in}}{\pgfqpoint{2.418996in}{1.125473in}}%
\pgfpathlineto{\pgfqpoint{2.418996in}{1.125473in}}%
\pgfpathclose%
\pgfusepath{stroke,fill}%
\end{pgfscope}%
\begin{pgfscope}%
\pgfpathrectangle{\pgfqpoint{0.661006in}{0.524170in}}{\pgfqpoint{4.194036in}{1.071446in}}%
\pgfusepath{clip}%
\pgfsetbuttcap%
\pgfsetroundjoin%
\definecolor{currentfill}{rgb}{0.435188,0.488470,0.685126}%
\pgfsetfillcolor{currentfill}%
\pgfsetfillopacity{0.700000}%
\pgfsetlinewidth{1.003750pt}%
\definecolor{currentstroke}{rgb}{0.435188,0.488470,0.685126}%
\pgfsetstrokecolor{currentstroke}%
\pgfsetstrokeopacity{0.700000}%
\pgfsetdash{}{0pt}%
\pgfpathmoveto{\pgfqpoint{2.436703in}{1.121975in}}%
\pgfpathcurveto{\pgfqpoint{2.438545in}{1.121975in}}{\pgfqpoint{2.440312in}{1.122706in}}{\pgfqpoint{2.441614in}{1.124009in}}%
\pgfpathcurveto{\pgfqpoint{2.442916in}{1.125311in}}{\pgfqpoint{2.443648in}{1.127077in}}{\pgfqpoint{2.443648in}{1.128919in}}%
\pgfpathcurveto{\pgfqpoint{2.443648in}{1.130761in}}{\pgfqpoint{2.442916in}{1.132527in}}{\pgfqpoint{2.441614in}{1.133829in}}%
\pgfpathcurveto{\pgfqpoint{2.440312in}{1.135132in}}{\pgfqpoint{2.438545in}{1.135863in}}{\pgfqpoint{2.436703in}{1.135863in}}%
\pgfpathcurveto{\pgfqpoint{2.434862in}{1.135863in}}{\pgfqpoint{2.433095in}{1.135132in}}{\pgfqpoint{2.431793in}{1.133829in}}%
\pgfpathcurveto{\pgfqpoint{2.430491in}{1.132527in}}{\pgfqpoint{2.429759in}{1.130761in}}{\pgfqpoint{2.429759in}{1.128919in}}%
\pgfpathcurveto{\pgfqpoint{2.429759in}{1.127077in}}{\pgfqpoint{2.430491in}{1.125311in}}{\pgfqpoint{2.431793in}{1.124009in}}%
\pgfpathcurveto{\pgfqpoint{2.433095in}{1.122706in}}{\pgfqpoint{2.434862in}{1.121975in}}{\pgfqpoint{2.436703in}{1.121975in}}%
\pgfpathlineto{\pgfqpoint{2.436703in}{1.121975in}}%
\pgfpathclose%
\pgfusepath{stroke,fill}%
\end{pgfscope}%
\begin{pgfscope}%
\pgfpathrectangle{\pgfqpoint{0.661006in}{0.524170in}}{\pgfqpoint{4.194036in}{1.071446in}}%
\pgfusepath{clip}%
\pgfsetbuttcap%
\pgfsetroundjoin%
\definecolor{currentfill}{rgb}{0.435188,0.488470,0.685126}%
\pgfsetfillcolor{currentfill}%
\pgfsetfillopacity{0.700000}%
\pgfsetlinewidth{1.003750pt}%
\definecolor{currentstroke}{rgb}{0.435188,0.488470,0.685126}%
\pgfsetstrokecolor{currentstroke}%
\pgfsetstrokeopacity{0.700000}%
\pgfsetdash{}{0pt}%
\pgfpathmoveto{\pgfqpoint{2.452625in}{1.117472in}}%
\pgfpathcurveto{\pgfqpoint{2.454467in}{1.117472in}}{\pgfqpoint{2.456233in}{1.118204in}}{\pgfqpoint{2.457536in}{1.119506in}}%
\pgfpathcurveto{\pgfqpoint{2.458838in}{1.120808in}}{\pgfqpoint{2.459570in}{1.122575in}}{\pgfqpoint{2.459570in}{1.124416in}}%
\pgfpathcurveto{\pgfqpoint{2.459570in}{1.126258in}}{\pgfqpoint{2.458838in}{1.128025in}}{\pgfqpoint{2.457536in}{1.129327in}}%
\pgfpathcurveto{\pgfqpoint{2.456233in}{1.130629in}}{\pgfqpoint{2.454467in}{1.131361in}}{\pgfqpoint{2.452625in}{1.131361in}}%
\pgfpathcurveto{\pgfqpoint{2.450783in}{1.131361in}}{\pgfqpoint{2.449017in}{1.130629in}}{\pgfqpoint{2.447715in}{1.129327in}}%
\pgfpathcurveto{\pgfqpoint{2.446412in}{1.128025in}}{\pgfqpoint{2.445681in}{1.126258in}}{\pgfqpoint{2.445681in}{1.124416in}}%
\pgfpathcurveto{\pgfqpoint{2.445681in}{1.122575in}}{\pgfqpoint{2.446412in}{1.120808in}}{\pgfqpoint{2.447715in}{1.119506in}}%
\pgfpathcurveto{\pgfqpoint{2.449017in}{1.118204in}}{\pgfqpoint{2.450783in}{1.117472in}}{\pgfqpoint{2.452625in}{1.117472in}}%
\pgfpathlineto{\pgfqpoint{2.452625in}{1.117472in}}%
\pgfpathclose%
\pgfusepath{stroke,fill}%
\end{pgfscope}%
\begin{pgfscope}%
\pgfpathrectangle{\pgfqpoint{0.661006in}{0.524170in}}{\pgfqpoint{4.194036in}{1.071446in}}%
\pgfusepath{clip}%
\pgfsetbuttcap%
\pgfsetroundjoin%
\definecolor{currentfill}{rgb}{0.435188,0.488470,0.685126}%
\pgfsetfillcolor{currentfill}%
\pgfsetfillopacity{0.700000}%
\pgfsetlinewidth{1.003750pt}%
\definecolor{currentstroke}{rgb}{0.435188,0.488470,0.685126}%
\pgfsetstrokecolor{currentstroke}%
\pgfsetstrokeopacity{0.700000}%
\pgfsetdash{}{0pt}%
\pgfpathmoveto{\pgfqpoint{2.482019in}{1.110204in}}%
\pgfpathcurveto{\pgfqpoint{2.483860in}{1.110204in}}{\pgfqpoint{2.485627in}{1.110936in}}{\pgfqpoint{2.486929in}{1.112238in}}%
\pgfpathcurveto{\pgfqpoint{2.488231in}{1.113541in}}{\pgfqpoint{2.488963in}{1.115307in}}{\pgfqpoint{2.488963in}{1.117149in}}%
\pgfpathcurveto{\pgfqpoint{2.488963in}{1.118991in}}{\pgfqpoint{2.488231in}{1.120757in}}{\pgfqpoint{2.486929in}{1.122059in}}%
\pgfpathcurveto{\pgfqpoint{2.485627in}{1.123362in}}{\pgfqpoint{2.483860in}{1.124093in}}{\pgfqpoint{2.482019in}{1.124093in}}%
\pgfpathcurveto{\pgfqpoint{2.480177in}{1.124093in}}{\pgfqpoint{2.478411in}{1.123362in}}{\pgfqpoint{2.477108in}{1.122059in}}%
\pgfpathcurveto{\pgfqpoint{2.475806in}{1.120757in}}{\pgfqpoint{2.475074in}{1.118991in}}{\pgfqpoint{2.475074in}{1.117149in}}%
\pgfpathcurveto{\pgfqpoint{2.475074in}{1.115307in}}{\pgfqpoint{2.475806in}{1.113541in}}{\pgfqpoint{2.477108in}{1.112238in}}%
\pgfpathcurveto{\pgfqpoint{2.478411in}{1.110936in}}{\pgfqpoint{2.480177in}{1.110204in}}{\pgfqpoint{2.482019in}{1.110204in}}%
\pgfpathlineto{\pgfqpoint{2.482019in}{1.110204in}}%
\pgfpathclose%
\pgfusepath{stroke,fill}%
\end{pgfscope}%
\begin{pgfscope}%
\pgfpathrectangle{\pgfqpoint{0.661006in}{0.524170in}}{\pgfqpoint{4.194036in}{1.071446in}}%
\pgfusepath{clip}%
\pgfsetbuttcap%
\pgfsetroundjoin%
\definecolor{currentfill}{rgb}{0.435188,0.488470,0.685126}%
\pgfsetfillcolor{currentfill}%
\pgfsetfillopacity{0.700000}%
\pgfsetlinewidth{1.003750pt}%
\definecolor{currentstroke}{rgb}{0.435188,0.488470,0.685126}%
\pgfsetstrokecolor{currentstroke}%
\pgfsetstrokeopacity{0.700000}%
\pgfsetdash{}{0pt}%
\pgfpathmoveto{\pgfqpoint{2.527752in}{1.101102in}}%
\pgfpathcurveto{\pgfqpoint{2.529594in}{1.101102in}}{\pgfqpoint{2.531360in}{1.101834in}}{\pgfqpoint{2.532663in}{1.103136in}}%
\pgfpathcurveto{\pgfqpoint{2.533965in}{1.104438in}}{\pgfqpoint{2.534697in}{1.106205in}}{\pgfqpoint{2.534697in}{1.108046in}}%
\pgfpathcurveto{\pgfqpoint{2.534697in}{1.109888in}}{\pgfqpoint{2.533965in}{1.111655in}}{\pgfqpoint{2.532663in}{1.112957in}}%
\pgfpathcurveto{\pgfqpoint{2.531360in}{1.114259in}}{\pgfqpoint{2.529594in}{1.114991in}}{\pgfqpoint{2.527752in}{1.114991in}}%
\pgfpathcurveto{\pgfqpoint{2.525911in}{1.114991in}}{\pgfqpoint{2.524144in}{1.114259in}}{\pgfqpoint{2.522842in}{1.112957in}}%
\pgfpathcurveto{\pgfqpoint{2.521540in}{1.111655in}}{\pgfqpoint{2.520808in}{1.109888in}}{\pgfqpoint{2.520808in}{1.108046in}}%
\pgfpathcurveto{\pgfqpoint{2.520808in}{1.106205in}}{\pgfqpoint{2.521540in}{1.104438in}}{\pgfqpoint{2.522842in}{1.103136in}}%
\pgfpathcurveto{\pgfqpoint{2.524144in}{1.101834in}}{\pgfqpoint{2.525911in}{1.101102in}}{\pgfqpoint{2.527752in}{1.101102in}}%
\pgfpathlineto{\pgfqpoint{2.527752in}{1.101102in}}%
\pgfpathclose%
\pgfusepath{stroke,fill}%
\end{pgfscope}%
\begin{pgfscope}%
\pgfpathrectangle{\pgfqpoint{0.661006in}{0.524170in}}{\pgfqpoint{4.194036in}{1.071446in}}%
\pgfusepath{clip}%
\pgfsetbuttcap%
\pgfsetroundjoin%
\definecolor{currentfill}{rgb}{0.435188,0.488470,0.685126}%
\pgfsetfillcolor{currentfill}%
\pgfsetfillopacity{0.700000}%
\pgfsetlinewidth{1.003750pt}%
\definecolor{currentstroke}{rgb}{0.435188,0.488470,0.685126}%
\pgfsetstrokecolor{currentstroke}%
\pgfsetstrokeopacity{0.700000}%
\pgfsetdash{}{0pt}%
\pgfpathmoveto{\pgfqpoint{2.556847in}{1.092345in}}%
\pgfpathcurveto{\pgfqpoint{2.558689in}{1.092345in}}{\pgfqpoint{2.560455in}{1.093076in}}{\pgfqpoint{2.561758in}{1.094379in}}%
\pgfpathcurveto{\pgfqpoint{2.563060in}{1.095681in}}{\pgfqpoint{2.563791in}{1.097447in}}{\pgfqpoint{2.563791in}{1.099289in}}%
\pgfpathcurveto{\pgfqpoint{2.563791in}{1.101131in}}{\pgfqpoint{2.563060in}{1.102897in}}{\pgfqpoint{2.561758in}{1.104199in}}%
\pgfpathcurveto{\pgfqpoint{2.560455in}{1.105502in}}{\pgfqpoint{2.558689in}{1.106233in}}{\pgfqpoint{2.556847in}{1.106233in}}%
\pgfpathcurveto{\pgfqpoint{2.555005in}{1.106233in}}{\pgfqpoint{2.553239in}{1.105502in}}{\pgfqpoint{2.551937in}{1.104199in}}%
\pgfpathcurveto{\pgfqpoint{2.550634in}{1.102897in}}{\pgfqpoint{2.549903in}{1.101131in}}{\pgfqpoint{2.549903in}{1.099289in}}%
\pgfpathcurveto{\pgfqpoint{2.549903in}{1.097447in}}{\pgfqpoint{2.550634in}{1.095681in}}{\pgfqpoint{2.551937in}{1.094379in}}%
\pgfpathcurveto{\pgfqpoint{2.553239in}{1.093076in}}{\pgfqpoint{2.555005in}{1.092345in}}{\pgfqpoint{2.556847in}{1.092345in}}%
\pgfpathlineto{\pgfqpoint{2.556847in}{1.092345in}}%
\pgfpathclose%
\pgfusepath{stroke,fill}%
\end{pgfscope}%
\begin{pgfscope}%
\pgfpathrectangle{\pgfqpoint{0.661006in}{0.524170in}}{\pgfqpoint{4.194036in}{1.071446in}}%
\pgfusepath{clip}%
\pgfsetbuttcap%
\pgfsetroundjoin%
\definecolor{currentfill}{rgb}{0.432322,0.483965,0.681458}%
\pgfsetfillcolor{currentfill}%
\pgfsetfillopacity{0.700000}%
\pgfsetlinewidth{1.003750pt}%
\definecolor{currentstroke}{rgb}{0.432322,0.483965,0.681458}%
\pgfsetstrokecolor{currentstroke}%
\pgfsetstrokeopacity{0.700000}%
\pgfsetdash{}{0pt}%
\pgfpathmoveto{\pgfqpoint{2.593471in}{1.082833in}}%
\pgfpathcurveto{\pgfqpoint{2.595313in}{1.082833in}}{\pgfqpoint{2.597079in}{1.083564in}}{\pgfqpoint{2.598382in}{1.084866in}}%
\pgfpathcurveto{\pgfqpoint{2.599684in}{1.086169in}}{\pgfqpoint{2.600416in}{1.087935in}}{\pgfqpoint{2.600416in}{1.089777in}}%
\pgfpathcurveto{\pgfqpoint{2.600416in}{1.091619in}}{\pgfqpoint{2.599684in}{1.093385in}}{\pgfqpoint{2.598382in}{1.094687in}}%
\pgfpathcurveto{\pgfqpoint{2.597079in}{1.095990in}}{\pgfqpoint{2.595313in}{1.096721in}}{\pgfqpoint{2.593471in}{1.096721in}}%
\pgfpathcurveto{\pgfqpoint{2.591629in}{1.096721in}}{\pgfqpoint{2.589863in}{1.095990in}}{\pgfqpoint{2.588561in}{1.094687in}}%
\pgfpathcurveto{\pgfqpoint{2.587258in}{1.093385in}}{\pgfqpoint{2.586527in}{1.091619in}}{\pgfqpoint{2.586527in}{1.089777in}}%
\pgfpathcurveto{\pgfqpoint{2.586527in}{1.087935in}}{\pgfqpoint{2.587258in}{1.086169in}}{\pgfqpoint{2.588561in}{1.084866in}}%
\pgfpathcurveto{\pgfqpoint{2.589863in}{1.083564in}}{\pgfqpoint{2.591629in}{1.082833in}}{\pgfqpoint{2.593471in}{1.082833in}}%
\pgfpathlineto{\pgfqpoint{2.593471in}{1.082833in}}%
\pgfpathclose%
\pgfusepath{stroke,fill}%
\end{pgfscope}%
\begin{pgfscope}%
\pgfpathrectangle{\pgfqpoint{0.661006in}{0.524170in}}{\pgfqpoint{4.194036in}{1.071446in}}%
\pgfusepath{clip}%
\pgfsetbuttcap%
\pgfsetroundjoin%
\definecolor{currentfill}{rgb}{0.432322,0.483965,0.681458}%
\pgfsetfillcolor{currentfill}%
\pgfsetfillopacity{0.700000}%
\pgfsetlinewidth{1.003750pt}%
\definecolor{currentstroke}{rgb}{0.432322,0.483965,0.681458}%
\pgfsetstrokecolor{currentstroke}%
\pgfsetstrokeopacity{0.700000}%
\pgfsetdash{}{0pt}%
\pgfpathmoveto{\pgfqpoint{2.652032in}{1.069664in}}%
\pgfpathcurveto{\pgfqpoint{2.653874in}{1.069664in}}{\pgfqpoint{2.655641in}{1.070395in}}{\pgfqpoint{2.656943in}{1.071698in}}%
\pgfpathcurveto{\pgfqpoint{2.658245in}{1.073000in}}{\pgfqpoint{2.658977in}{1.074766in}}{\pgfqpoint{2.658977in}{1.076608in}}%
\pgfpathcurveto{\pgfqpoint{2.658977in}{1.078450in}}{\pgfqpoint{2.658245in}{1.080216in}}{\pgfqpoint{2.656943in}{1.081518in}}%
\pgfpathcurveto{\pgfqpoint{2.655641in}{1.082821in}}{\pgfqpoint{2.653874in}{1.083552in}}{\pgfqpoint{2.652032in}{1.083552in}}%
\pgfpathcurveto{\pgfqpoint{2.650191in}{1.083552in}}{\pgfqpoint{2.648424in}{1.082821in}}{\pgfqpoint{2.647122in}{1.081518in}}%
\pgfpathcurveto{\pgfqpoint{2.645820in}{1.080216in}}{\pgfqpoint{2.645088in}{1.078450in}}{\pgfqpoint{2.645088in}{1.076608in}}%
\pgfpathcurveto{\pgfqpoint{2.645088in}{1.074766in}}{\pgfqpoint{2.645820in}{1.073000in}}{\pgfqpoint{2.647122in}{1.071698in}}%
\pgfpathcurveto{\pgfqpoint{2.648424in}{1.070395in}}{\pgfqpoint{2.650191in}{1.069664in}}{\pgfqpoint{2.652032in}{1.069664in}}%
\pgfpathlineto{\pgfqpoint{2.652032in}{1.069664in}}%
\pgfpathclose%
\pgfusepath{stroke,fill}%
\end{pgfscope}%
\begin{pgfscope}%
\pgfpathrectangle{\pgfqpoint{0.661006in}{0.524170in}}{\pgfqpoint{4.194036in}{1.071446in}}%
\pgfusepath{clip}%
\pgfsetbuttcap%
\pgfsetroundjoin%
\definecolor{currentfill}{rgb}{0.429465,0.479461,0.677758}%
\pgfsetfillcolor{currentfill}%
\pgfsetfillopacity{0.700000}%
\pgfsetlinewidth{1.003750pt}%
\definecolor{currentstroke}{rgb}{0.429465,0.479461,0.677758}%
\pgfsetstrokecolor{currentstroke}%
\pgfsetstrokeopacity{0.700000}%
\pgfsetdash{}{0pt}%
\pgfpathmoveto{\pgfqpoint{2.702739in}{1.059229in}}%
\pgfpathcurveto{\pgfqpoint{2.704581in}{1.059229in}}{\pgfqpoint{2.706347in}{1.059960in}}{\pgfqpoint{2.707650in}{1.061263in}}%
\pgfpathcurveto{\pgfqpoint{2.708952in}{1.062565in}}{\pgfqpoint{2.709684in}{1.064331in}}{\pgfqpoint{2.709684in}{1.066173in}}%
\pgfpathcurveto{\pgfqpoint{2.709684in}{1.068015in}}{\pgfqpoint{2.708952in}{1.069781in}}{\pgfqpoint{2.707650in}{1.071084in}}%
\pgfpathcurveto{\pgfqpoint{2.706347in}{1.072386in}}{\pgfqpoint{2.704581in}{1.073118in}}{\pgfqpoint{2.702739in}{1.073118in}}%
\pgfpathcurveto{\pgfqpoint{2.700897in}{1.073118in}}{\pgfqpoint{2.699131in}{1.072386in}}{\pgfqpoint{2.697829in}{1.071084in}}%
\pgfpathcurveto{\pgfqpoint{2.696526in}{1.069781in}}{\pgfqpoint{2.695795in}{1.068015in}}{\pgfqpoint{2.695795in}{1.066173in}}%
\pgfpathcurveto{\pgfqpoint{2.695795in}{1.064331in}}{\pgfqpoint{2.696526in}{1.062565in}}{\pgfqpoint{2.697829in}{1.061263in}}%
\pgfpathcurveto{\pgfqpoint{2.699131in}{1.059960in}}{\pgfqpoint{2.700897in}{1.059229in}}{\pgfqpoint{2.702739in}{1.059229in}}%
\pgfpathlineto{\pgfqpoint{2.702739in}{1.059229in}}%
\pgfpathclose%
\pgfusepath{stroke,fill}%
\end{pgfscope}%
\begin{pgfscope}%
\pgfpathrectangle{\pgfqpoint{0.661006in}{0.524170in}}{\pgfqpoint{4.194036in}{1.071446in}}%
\pgfusepath{clip}%
\pgfsetbuttcap%
\pgfsetroundjoin%
\definecolor{currentfill}{rgb}{0.429465,0.479461,0.677758}%
\pgfsetfillcolor{currentfill}%
\pgfsetfillopacity{0.700000}%
\pgfsetlinewidth{1.003750pt}%
\definecolor{currentstroke}{rgb}{0.429465,0.479461,0.677758}%
\pgfsetstrokecolor{currentstroke}%
\pgfsetstrokeopacity{0.700000}%
\pgfsetdash{}{0pt}%
\pgfpathmoveto{\pgfqpoint{2.744847in}{1.049436in}}%
\pgfpathcurveto{\pgfqpoint{2.746689in}{1.049436in}}{\pgfqpoint{2.748456in}{1.050168in}}{\pgfqpoint{2.749758in}{1.051470in}}%
\pgfpathcurveto{\pgfqpoint{2.751060in}{1.052772in}}{\pgfqpoint{2.751792in}{1.054539in}}{\pgfqpoint{2.751792in}{1.056380in}}%
\pgfpathcurveto{\pgfqpoint{2.751792in}{1.058222in}}{\pgfqpoint{2.751060in}{1.059989in}}{\pgfqpoint{2.749758in}{1.061291in}}%
\pgfpathcurveto{\pgfqpoint{2.748456in}{1.062593in}}{\pgfqpoint{2.746689in}{1.063325in}}{\pgfqpoint{2.744847in}{1.063325in}}%
\pgfpathcurveto{\pgfqpoint{2.743006in}{1.063325in}}{\pgfqpoint{2.741239in}{1.062593in}}{\pgfqpoint{2.739937in}{1.061291in}}%
\pgfpathcurveto{\pgfqpoint{2.738635in}{1.059989in}}{\pgfqpoint{2.737903in}{1.058222in}}{\pgfqpoint{2.737903in}{1.056380in}}%
\pgfpathcurveto{\pgfqpoint{2.737903in}{1.054539in}}{\pgfqpoint{2.738635in}{1.052772in}}{\pgfqpoint{2.739937in}{1.051470in}}%
\pgfpathcurveto{\pgfqpoint{2.741239in}{1.050168in}}{\pgfqpoint{2.743006in}{1.049436in}}{\pgfqpoint{2.744847in}{1.049436in}}%
\pgfpathlineto{\pgfqpoint{2.744847in}{1.049436in}}%
\pgfpathclose%
\pgfusepath{stroke,fill}%
\end{pgfscope}%
\begin{pgfscope}%
\pgfpathrectangle{\pgfqpoint{0.661006in}{0.524170in}}{\pgfqpoint{4.194036in}{1.071446in}}%
\pgfusepath{clip}%
\pgfsetbuttcap%
\pgfsetroundjoin%
\definecolor{currentfill}{rgb}{0.429465,0.479461,0.677758}%
\pgfsetfillcolor{currentfill}%
\pgfsetfillopacity{0.700000}%
\pgfsetlinewidth{1.003750pt}%
\definecolor{currentstroke}{rgb}{0.429465,0.479461,0.677758}%
\pgfsetstrokecolor{currentstroke}%
\pgfsetstrokeopacity{0.700000}%
\pgfsetdash{}{0pt}%
\pgfpathmoveto{\pgfqpoint{2.799598in}{1.037971in}}%
\pgfpathcurveto{\pgfqpoint{2.801439in}{1.037971in}}{\pgfqpoint{2.803206in}{1.038702in}}{\pgfqpoint{2.804508in}{1.040005in}}%
\pgfpathcurveto{\pgfqpoint{2.805810in}{1.041307in}}{\pgfqpoint{2.806542in}{1.043073in}}{\pgfqpoint{2.806542in}{1.044915in}}%
\pgfpathcurveto{\pgfqpoint{2.806542in}{1.046757in}}{\pgfqpoint{2.805810in}{1.048523in}}{\pgfqpoint{2.804508in}{1.049825in}}%
\pgfpathcurveto{\pgfqpoint{2.803206in}{1.051128in}}{\pgfqpoint{2.801439in}{1.051859in}}{\pgfqpoint{2.799598in}{1.051859in}}%
\pgfpathcurveto{\pgfqpoint{2.797756in}{1.051859in}}{\pgfqpoint{2.795989in}{1.051128in}}{\pgfqpoint{2.794687in}{1.049825in}}%
\pgfpathcurveto{\pgfqpoint{2.793385in}{1.048523in}}{\pgfqpoint{2.792653in}{1.046757in}}{\pgfqpoint{2.792653in}{1.044915in}}%
\pgfpathcurveto{\pgfqpoint{2.792653in}{1.043073in}}{\pgfqpoint{2.793385in}{1.041307in}}{\pgfqpoint{2.794687in}{1.040005in}}%
\pgfpathcurveto{\pgfqpoint{2.795989in}{1.038702in}}{\pgfqpoint{2.797756in}{1.037971in}}{\pgfqpoint{2.799598in}{1.037971in}}%
\pgfpathlineto{\pgfqpoint{2.799598in}{1.037971in}}%
\pgfpathclose%
\pgfusepath{stroke,fill}%
\end{pgfscope}%
\begin{pgfscope}%
\pgfpathrectangle{\pgfqpoint{0.661006in}{0.524170in}}{\pgfqpoint{4.194036in}{1.071446in}}%
\pgfusepath{clip}%
\pgfsetbuttcap%
\pgfsetroundjoin%
\definecolor{currentfill}{rgb}{0.429465,0.479461,0.677758}%
\pgfsetfillcolor{currentfill}%
\pgfsetfillopacity{0.700000}%
\pgfsetlinewidth{1.003750pt}%
\definecolor{currentstroke}{rgb}{0.429465,0.479461,0.677758}%
\pgfsetstrokecolor{currentstroke}%
\pgfsetstrokeopacity{0.700000}%
\pgfsetdash{}{0pt}%
\pgfpathmoveto{\pgfqpoint{2.845796in}{1.027679in}}%
\pgfpathcurveto{\pgfqpoint{2.847638in}{1.027679in}}{\pgfqpoint{2.849404in}{1.028411in}}{\pgfqpoint{2.850707in}{1.029713in}}%
\pgfpathcurveto{\pgfqpoint{2.852009in}{1.031015in}}{\pgfqpoint{2.852740in}{1.032782in}}{\pgfqpoint{2.852740in}{1.034624in}}%
\pgfpathcurveto{\pgfqpoint{2.852740in}{1.036465in}}{\pgfqpoint{2.852009in}{1.038232in}}{\pgfqpoint{2.850707in}{1.039534in}}%
\pgfpathcurveto{\pgfqpoint{2.849404in}{1.040836in}}{\pgfqpoint{2.847638in}{1.041568in}}{\pgfqpoint{2.845796in}{1.041568in}}%
\pgfpathcurveto{\pgfqpoint{2.843954in}{1.041568in}}{\pgfqpoint{2.842188in}{1.040836in}}{\pgfqpoint{2.840886in}{1.039534in}}%
\pgfpathcurveto{\pgfqpoint{2.839583in}{1.038232in}}{\pgfqpoint{2.838852in}{1.036465in}}{\pgfqpoint{2.838852in}{1.034624in}}%
\pgfpathcurveto{\pgfqpoint{2.838852in}{1.032782in}}{\pgfqpoint{2.839583in}{1.031015in}}{\pgfqpoint{2.840886in}{1.029713in}}%
\pgfpathcurveto{\pgfqpoint{2.842188in}{1.028411in}}{\pgfqpoint{2.843954in}{1.027679in}}{\pgfqpoint{2.845796in}{1.027679in}}%
\pgfpathlineto{\pgfqpoint{2.845796in}{1.027679in}}%
\pgfpathclose%
\pgfusepath{stroke,fill}%
\end{pgfscope}%
\begin{pgfscope}%
\pgfpathrectangle{\pgfqpoint{0.661006in}{0.524170in}}{\pgfqpoint{4.194036in}{1.071446in}}%
\pgfusepath{clip}%
\pgfsetbuttcap%
\pgfsetroundjoin%
\definecolor{currentfill}{rgb}{0.429465,0.479461,0.677758}%
\pgfsetfillcolor{currentfill}%
\pgfsetfillopacity{0.700000}%
\pgfsetlinewidth{1.003750pt}%
\definecolor{currentstroke}{rgb}{0.429465,0.479461,0.677758}%
\pgfsetstrokecolor{currentstroke}%
\pgfsetstrokeopacity{0.700000}%
\pgfsetdash{}{0pt}%
\pgfpathmoveto{\pgfqpoint{2.908726in}{1.012803in}}%
\pgfpathcurveto{\pgfqpoint{2.910568in}{1.012803in}}{\pgfqpoint{2.912334in}{1.013535in}}{\pgfqpoint{2.913637in}{1.014837in}}%
\pgfpathcurveto{\pgfqpoint{2.914939in}{1.016140in}}{\pgfqpoint{2.915671in}{1.017906in}}{\pgfqpoint{2.915671in}{1.019748in}}%
\pgfpathcurveto{\pgfqpoint{2.915671in}{1.021590in}}{\pgfqpoint{2.914939in}{1.023356in}}{\pgfqpoint{2.913637in}{1.024658in}}%
\pgfpathcurveto{\pgfqpoint{2.912334in}{1.025961in}}{\pgfqpoint{2.910568in}{1.026692in}}{\pgfqpoint{2.908726in}{1.026692in}}%
\pgfpathcurveto{\pgfqpoint{2.906885in}{1.026692in}}{\pgfqpoint{2.905118in}{1.025961in}}{\pgfqpoint{2.903816in}{1.024658in}}%
\pgfpathcurveto{\pgfqpoint{2.902513in}{1.023356in}}{\pgfqpoint{2.901782in}{1.021590in}}{\pgfqpoint{2.901782in}{1.019748in}}%
\pgfpathcurveto{\pgfqpoint{2.901782in}{1.017906in}}{\pgfqpoint{2.902513in}{1.016140in}}{\pgfqpoint{2.903816in}{1.014837in}}%
\pgfpathcurveto{\pgfqpoint{2.905118in}{1.013535in}}{\pgfqpoint{2.906885in}{1.012803in}}{\pgfqpoint{2.908726in}{1.012803in}}%
\pgfpathlineto{\pgfqpoint{2.908726in}{1.012803in}}%
\pgfpathclose%
\pgfusepath{stroke,fill}%
\end{pgfscope}%
\begin{pgfscope}%
\pgfpathrectangle{\pgfqpoint{0.661006in}{0.524170in}}{\pgfqpoint{4.194036in}{1.071446in}}%
\pgfusepath{clip}%
\pgfsetbuttcap%
\pgfsetroundjoin%
\definecolor{currentfill}{rgb}{0.426616,0.474958,0.674027}%
\pgfsetfillcolor{currentfill}%
\pgfsetfillopacity{0.700000}%
\pgfsetlinewidth{1.003750pt}%
\definecolor{currentstroke}{rgb}{0.426616,0.474958,0.674027}%
\pgfsetstrokecolor{currentstroke}%
\pgfsetstrokeopacity{0.700000}%
\pgfsetdash{}{0pt}%
\pgfpathmoveto{\pgfqpoint{2.965243in}{1.001652in}}%
\pgfpathcurveto{\pgfqpoint{2.967084in}{1.001652in}}{\pgfqpoint{2.968851in}{1.002383in}}{\pgfqpoint{2.970153in}{1.003686in}}%
\pgfpathcurveto{\pgfqpoint{2.971455in}{1.004988in}}{\pgfqpoint{2.972187in}{1.006755in}}{\pgfqpoint{2.972187in}{1.008596in}}%
\pgfpathcurveto{\pgfqpoint{2.972187in}{1.010438in}}{\pgfqpoint{2.971455in}{1.012204in}}{\pgfqpoint{2.970153in}{1.013507in}}%
\pgfpathcurveto{\pgfqpoint{2.968851in}{1.014809in}}{\pgfqpoint{2.967084in}{1.015541in}}{\pgfqpoint{2.965243in}{1.015541in}}%
\pgfpathcurveto{\pgfqpoint{2.963401in}{1.015541in}}{\pgfqpoint{2.961634in}{1.014809in}}{\pgfqpoint{2.960332in}{1.013507in}}%
\pgfpathcurveto{\pgfqpoint{2.959030in}{1.012204in}}{\pgfqpoint{2.958298in}{1.010438in}}{\pgfqpoint{2.958298in}{1.008596in}}%
\pgfpathcurveto{\pgfqpoint{2.958298in}{1.006755in}}{\pgfqpoint{2.959030in}{1.004988in}}{\pgfqpoint{2.960332in}{1.003686in}}%
\pgfpathcurveto{\pgfqpoint{2.961634in}{1.002383in}}{\pgfqpoint{2.963401in}{1.001652in}}{\pgfqpoint{2.965243in}{1.001652in}}%
\pgfpathlineto{\pgfqpoint{2.965243in}{1.001652in}}%
\pgfpathclose%
\pgfusepath{stroke,fill}%
\end{pgfscope}%
\begin{pgfscope}%
\pgfpathrectangle{\pgfqpoint{0.661006in}{0.524170in}}{\pgfqpoint{4.194036in}{1.071446in}}%
\pgfusepath{clip}%
\pgfsetbuttcap%
\pgfsetroundjoin%
\definecolor{currentfill}{rgb}{0.426616,0.474958,0.674027}%
\pgfsetfillcolor{currentfill}%
\pgfsetfillopacity{0.700000}%
\pgfsetlinewidth{1.003750pt}%
\definecolor{currentstroke}{rgb}{0.426616,0.474958,0.674027}%
\pgfsetstrokecolor{currentstroke}%
\pgfsetstrokeopacity{0.700000}%
\pgfsetdash{}{0pt}%
\pgfpathmoveto{\pgfqpoint{2.999818in}{0.994846in}}%
\pgfpathcurveto{\pgfqpoint{3.001660in}{0.994846in}}{\pgfqpoint{3.003426in}{0.995578in}}{\pgfqpoint{3.004729in}{0.996880in}}%
\pgfpathcurveto{\pgfqpoint{3.006031in}{0.998182in}}{\pgfqpoint{3.006763in}{0.999949in}}{\pgfqpoint{3.006763in}{1.001790in}}%
\pgfpathcurveto{\pgfqpoint{3.006763in}{1.003632in}}{\pgfqpoint{3.006031in}{1.005399in}}{\pgfqpoint{3.004729in}{1.006701in}}%
\pgfpathcurveto{\pgfqpoint{3.003426in}{1.008003in}}{\pgfqpoint{3.001660in}{1.008735in}}{\pgfqpoint{2.999818in}{1.008735in}}%
\pgfpathcurveto{\pgfqpoint{2.997977in}{1.008735in}}{\pgfqpoint{2.996210in}{1.008003in}}{\pgfqpoint{2.994908in}{1.006701in}}%
\pgfpathcurveto{\pgfqpoint{2.993606in}{1.005399in}}{\pgfqpoint{2.992874in}{1.003632in}}{\pgfqpoint{2.992874in}{1.001790in}}%
\pgfpathcurveto{\pgfqpoint{2.992874in}{0.999949in}}{\pgfqpoint{2.993606in}{0.998182in}}{\pgfqpoint{2.994908in}{0.996880in}}%
\pgfpathcurveto{\pgfqpoint{2.996210in}{0.995578in}}{\pgfqpoint{2.997977in}{0.994846in}}{\pgfqpoint{2.999818in}{0.994846in}}%
\pgfpathlineto{\pgfqpoint{2.999818in}{0.994846in}}%
\pgfpathclose%
\pgfusepath{stroke,fill}%
\end{pgfscope}%
\begin{pgfscope}%
\pgfpathrectangle{\pgfqpoint{0.661006in}{0.524170in}}{\pgfqpoint{4.194036in}{1.071446in}}%
\pgfusepath{clip}%
\pgfsetbuttcap%
\pgfsetroundjoin%
\definecolor{currentfill}{rgb}{0.423775,0.470457,0.670265}%
\pgfsetfillcolor{currentfill}%
\pgfsetfillopacity{0.700000}%
\pgfsetlinewidth{1.003750pt}%
\definecolor{currentstroke}{rgb}{0.423775,0.470457,0.670265}%
\pgfsetstrokecolor{currentstroke}%
\pgfsetstrokeopacity{0.700000}%
\pgfsetdash{}{0pt}%
\pgfpathmoveto{\pgfqpoint{3.028777in}{0.986575in}}%
\pgfpathcurveto{\pgfqpoint{3.030619in}{0.986575in}}{\pgfqpoint{3.032385in}{0.987307in}}{\pgfqpoint{3.033687in}{0.988609in}}%
\pgfpathcurveto{\pgfqpoint{3.034990in}{0.989912in}}{\pgfqpoint{3.035721in}{0.991678in}}{\pgfqpoint{3.035721in}{0.993520in}}%
\pgfpathcurveto{\pgfqpoint{3.035721in}{0.995362in}}{\pgfqpoint{3.034990in}{0.997128in}}{\pgfqpoint{3.033687in}{0.998430in}}%
\pgfpathcurveto{\pgfqpoint{3.032385in}{0.999733in}}{\pgfqpoint{3.030619in}{1.000464in}}{\pgfqpoint{3.028777in}{1.000464in}}%
\pgfpathcurveto{\pgfqpoint{3.026935in}{1.000464in}}{\pgfqpoint{3.025169in}{0.999733in}}{\pgfqpoint{3.023866in}{0.998430in}}%
\pgfpathcurveto{\pgfqpoint{3.022564in}{0.997128in}}{\pgfqpoint{3.021832in}{0.995362in}}{\pgfqpoint{3.021832in}{0.993520in}}%
\pgfpathcurveto{\pgfqpoint{3.021832in}{0.991678in}}{\pgfqpoint{3.022564in}{0.989912in}}{\pgfqpoint{3.023866in}{0.988609in}}%
\pgfpathcurveto{\pgfqpoint{3.025169in}{0.987307in}}{\pgfqpoint{3.026935in}{0.986575in}}{\pgfqpoint{3.028777in}{0.986575in}}%
\pgfpathlineto{\pgfqpoint{3.028777in}{0.986575in}}%
\pgfpathclose%
\pgfusepath{stroke,fill}%
\end{pgfscope}%
\begin{pgfscope}%
\pgfpathrectangle{\pgfqpoint{0.661006in}{0.524170in}}{\pgfqpoint{4.194036in}{1.071446in}}%
\pgfusepath{clip}%
\pgfsetbuttcap%
\pgfsetroundjoin%
\definecolor{currentfill}{rgb}{0.423775,0.470457,0.670265}%
\pgfsetfillcolor{currentfill}%
\pgfsetfillopacity{0.700000}%
\pgfsetlinewidth{1.003750pt}%
\definecolor{currentstroke}{rgb}{0.423775,0.470457,0.670265}%
\pgfsetstrokecolor{currentstroke}%
\pgfsetstrokeopacity{0.700000}%
\pgfsetdash{}{0pt}%
\pgfpathmoveto{\pgfqpoint{3.059777in}{0.981448in}}%
\pgfpathcurveto{\pgfqpoint{3.061619in}{0.981448in}}{\pgfqpoint{3.063385in}{0.982179in}}{\pgfqpoint{3.064688in}{0.983482in}}%
\pgfpathcurveto{\pgfqpoint{3.065990in}{0.984784in}}{\pgfqpoint{3.066722in}{0.986550in}}{\pgfqpoint{3.066722in}{0.988392in}}%
\pgfpathcurveto{\pgfqpoint{3.066722in}{0.990234in}}{\pgfqpoint{3.065990in}{0.992000in}}{\pgfqpoint{3.064688in}{0.993303in}}%
\pgfpathcurveto{\pgfqpoint{3.063385in}{0.994605in}}{\pgfqpoint{3.061619in}{0.995337in}}{\pgfqpoint{3.059777in}{0.995337in}}%
\pgfpathcurveto{\pgfqpoint{3.057936in}{0.995337in}}{\pgfqpoint{3.056169in}{0.994605in}}{\pgfqpoint{3.054867in}{0.993303in}}%
\pgfpathcurveto{\pgfqpoint{3.053565in}{0.992000in}}{\pgfqpoint{3.052833in}{0.990234in}}{\pgfqpoint{3.052833in}{0.988392in}}%
\pgfpathcurveto{\pgfqpoint{3.052833in}{0.986550in}}{\pgfqpoint{3.053565in}{0.984784in}}{\pgfqpoint{3.054867in}{0.983482in}}%
\pgfpathcurveto{\pgfqpoint{3.056169in}{0.982179in}}{\pgfqpoint{3.057936in}{0.981448in}}{\pgfqpoint{3.059777in}{0.981448in}}%
\pgfpathlineto{\pgfqpoint{3.059777in}{0.981448in}}%
\pgfpathclose%
\pgfusepath{stroke,fill}%
\end{pgfscope}%
\begin{pgfscope}%
\pgfpathrectangle{\pgfqpoint{0.661006in}{0.524170in}}{\pgfqpoint{4.194036in}{1.071446in}}%
\pgfusepath{clip}%
\pgfsetbuttcap%
\pgfsetroundjoin%
\definecolor{currentfill}{rgb}{0.423775,0.470457,0.670265}%
\pgfsetfillcolor{currentfill}%
\pgfsetfillopacity{0.700000}%
\pgfsetlinewidth{1.003750pt}%
\definecolor{currentstroke}{rgb}{0.423775,0.470457,0.670265}%
\pgfsetstrokecolor{currentstroke}%
\pgfsetstrokeopacity{0.700000}%
\pgfsetdash{}{0pt}%
\pgfpathmoveto{\pgfqpoint{3.087849in}{0.976105in}}%
\pgfpathcurveto{\pgfqpoint{3.089691in}{0.976105in}}{\pgfqpoint{3.091458in}{0.976837in}}{\pgfqpoint{3.092760in}{0.978139in}}%
\pgfpathcurveto{\pgfqpoint{3.094062in}{0.979441in}}{\pgfqpoint{3.094794in}{0.981208in}}{\pgfqpoint{3.094794in}{0.983049in}}%
\pgfpathcurveto{\pgfqpoint{3.094794in}{0.984891in}}{\pgfqpoint{3.094062in}{0.986658in}}{\pgfqpoint{3.092760in}{0.987960in}}%
\pgfpathcurveto{\pgfqpoint{3.091458in}{0.989262in}}{\pgfqpoint{3.089691in}{0.989994in}}{\pgfqpoint{3.087849in}{0.989994in}}%
\pgfpathcurveto{\pgfqpoint{3.086008in}{0.989994in}}{\pgfqpoint{3.084241in}{0.989262in}}{\pgfqpoint{3.082939in}{0.987960in}}%
\pgfpathcurveto{\pgfqpoint{3.081637in}{0.986658in}}{\pgfqpoint{3.080905in}{0.984891in}}{\pgfqpoint{3.080905in}{0.983049in}}%
\pgfpathcurveto{\pgfqpoint{3.080905in}{0.981208in}}{\pgfqpoint{3.081637in}{0.979441in}}{\pgfqpoint{3.082939in}{0.978139in}}%
\pgfpathcurveto{\pgfqpoint{3.084241in}{0.976837in}}{\pgfqpoint{3.086008in}{0.976105in}}{\pgfqpoint{3.087849in}{0.976105in}}%
\pgfpathlineto{\pgfqpoint{3.087849in}{0.976105in}}%
\pgfpathclose%
\pgfusepath{stroke,fill}%
\end{pgfscope}%
\begin{pgfscope}%
\pgfpathrectangle{\pgfqpoint{0.661006in}{0.524170in}}{\pgfqpoint{4.194036in}{1.071446in}}%
\pgfusepath{clip}%
\pgfsetbuttcap%
\pgfsetroundjoin%
\definecolor{currentfill}{rgb}{0.423775,0.470457,0.670265}%
\pgfsetfillcolor{currentfill}%
\pgfsetfillopacity{0.700000}%
\pgfsetlinewidth{1.003750pt}%
\definecolor{currentstroke}{rgb}{0.423775,0.470457,0.670265}%
\pgfsetstrokecolor{currentstroke}%
\pgfsetstrokeopacity{0.700000}%
\pgfsetdash{}{0pt}%
\pgfpathmoveto{\pgfqpoint{3.113784in}{0.967898in}}%
\pgfpathcurveto{\pgfqpoint{3.115625in}{0.967898in}}{\pgfqpoint{3.117392in}{0.968630in}}{\pgfqpoint{3.118694in}{0.969932in}}%
\pgfpathcurveto{\pgfqpoint{3.119997in}{0.971234in}}{\pgfqpoint{3.120728in}{0.973001in}}{\pgfqpoint{3.120728in}{0.974842in}}%
\pgfpathcurveto{\pgfqpoint{3.120728in}{0.976684in}}{\pgfqpoint{3.119997in}{0.978451in}}{\pgfqpoint{3.118694in}{0.979753in}}%
\pgfpathcurveto{\pgfqpoint{3.117392in}{0.981055in}}{\pgfqpoint{3.115625in}{0.981787in}}{\pgfqpoint{3.113784in}{0.981787in}}%
\pgfpathcurveto{\pgfqpoint{3.111942in}{0.981787in}}{\pgfqpoint{3.110176in}{0.981055in}}{\pgfqpoint{3.108873in}{0.979753in}}%
\pgfpathcurveto{\pgfqpoint{3.107571in}{0.978451in}}{\pgfqpoint{3.106839in}{0.976684in}}{\pgfqpoint{3.106839in}{0.974842in}}%
\pgfpathcurveto{\pgfqpoint{3.106839in}{0.973001in}}{\pgfqpoint{3.107571in}{0.971234in}}{\pgfqpoint{3.108873in}{0.969932in}}%
\pgfpathcurveto{\pgfqpoint{3.110176in}{0.968630in}}{\pgfqpoint{3.111942in}{0.967898in}}{\pgfqpoint{3.113784in}{0.967898in}}%
\pgfpathlineto{\pgfqpoint{3.113784in}{0.967898in}}%
\pgfpathclose%
\pgfusepath{stroke,fill}%
\end{pgfscope}%
\begin{pgfscope}%
\pgfpathrectangle{\pgfqpoint{0.661006in}{0.524170in}}{\pgfqpoint{4.194036in}{1.071446in}}%
\pgfusepath{clip}%
\pgfsetbuttcap%
\pgfsetroundjoin%
\definecolor{currentfill}{rgb}{0.423775,0.470457,0.670265}%
\pgfsetfillcolor{currentfill}%
\pgfsetfillopacity{0.700000}%
\pgfsetlinewidth{1.003750pt}%
\definecolor{currentstroke}{rgb}{0.423775,0.470457,0.670265}%
\pgfsetstrokecolor{currentstroke}%
\pgfsetstrokeopacity{0.700000}%
\pgfsetdash{}{0pt}%
\pgfpathmoveto{\pgfqpoint{3.135907in}{0.965607in}}%
\pgfpathcurveto{\pgfqpoint{3.137749in}{0.965607in}}{\pgfqpoint{3.139515in}{0.966339in}}{\pgfqpoint{3.140817in}{0.967641in}}%
\pgfpathcurveto{\pgfqpoint{3.142120in}{0.968944in}}{\pgfqpoint{3.142851in}{0.970710in}}{\pgfqpoint{3.142851in}{0.972552in}}%
\pgfpathcurveto{\pgfqpoint{3.142851in}{0.974394in}}{\pgfqpoint{3.142120in}{0.976160in}}{\pgfqpoint{3.140817in}{0.977462in}}%
\pgfpathcurveto{\pgfqpoint{3.139515in}{0.978765in}}{\pgfqpoint{3.137749in}{0.979496in}}{\pgfqpoint{3.135907in}{0.979496in}}%
\pgfpathcurveto{\pgfqpoint{3.134065in}{0.979496in}}{\pgfqpoint{3.132299in}{0.978765in}}{\pgfqpoint{3.130996in}{0.977462in}}%
\pgfpathcurveto{\pgfqpoint{3.129694in}{0.976160in}}{\pgfqpoint{3.128963in}{0.974394in}}{\pgfqpoint{3.128963in}{0.972552in}}%
\pgfpathcurveto{\pgfqpoint{3.128963in}{0.970710in}}{\pgfqpoint{3.129694in}{0.968944in}}{\pgfqpoint{3.130996in}{0.967641in}}%
\pgfpathcurveto{\pgfqpoint{3.132299in}{0.966339in}}{\pgfqpoint{3.134065in}{0.965607in}}{\pgfqpoint{3.135907in}{0.965607in}}%
\pgfpathlineto{\pgfqpoint{3.135907in}{0.965607in}}%
\pgfpathclose%
\pgfusepath{stroke,fill}%
\end{pgfscope}%
\begin{pgfscope}%
\pgfpathrectangle{\pgfqpoint{0.661006in}{0.524170in}}{\pgfqpoint{4.194036in}{1.071446in}}%
\pgfusepath{clip}%
\pgfsetbuttcap%
\pgfsetroundjoin%
\definecolor{currentfill}{rgb}{0.420941,0.465959,0.666472}%
\pgfsetfillcolor{currentfill}%
\pgfsetfillopacity{0.700000}%
\pgfsetlinewidth{1.003750pt}%
\definecolor{currentstroke}{rgb}{0.420941,0.465959,0.666472}%
\pgfsetstrokecolor{currentstroke}%
\pgfsetstrokeopacity{0.700000}%
\pgfsetdash{}{0pt}%
\pgfpathmoveto{\pgfqpoint{3.113087in}{0.970427in}}%
\pgfpathcurveto{\pgfqpoint{3.114928in}{0.970427in}}{\pgfqpoint{3.116695in}{0.971159in}}{\pgfqpoint{3.117997in}{0.972461in}}%
\pgfpathcurveto{\pgfqpoint{3.119299in}{0.973763in}}{\pgfqpoint{3.120031in}{0.975530in}}{\pgfqpoint{3.120031in}{0.977371in}}%
\pgfpathcurveto{\pgfqpoint{3.120031in}{0.979213in}}{\pgfqpoint{3.119299in}{0.980980in}}{\pgfqpoint{3.117997in}{0.982282in}}%
\pgfpathcurveto{\pgfqpoint{3.116695in}{0.983584in}}{\pgfqpoint{3.114928in}{0.984316in}}{\pgfqpoint{3.113087in}{0.984316in}}%
\pgfpathcurveto{\pgfqpoint{3.111245in}{0.984316in}}{\pgfqpoint{3.109478in}{0.983584in}}{\pgfqpoint{3.108176in}{0.982282in}}%
\pgfpathcurveto{\pgfqpoint{3.106874in}{0.980980in}}{\pgfqpoint{3.106142in}{0.979213in}}{\pgfqpoint{3.106142in}{0.977371in}}%
\pgfpathcurveto{\pgfqpoint{3.106142in}{0.975530in}}{\pgfqpoint{3.106874in}{0.973763in}}{\pgfqpoint{3.108176in}{0.972461in}}%
\pgfpathcurveto{\pgfqpoint{3.109478in}{0.971159in}}{\pgfqpoint{3.111245in}{0.970427in}}{\pgfqpoint{3.113087in}{0.970427in}}%
\pgfpathlineto{\pgfqpoint{3.113087in}{0.970427in}}%
\pgfpathclose%
\pgfusepath{stroke,fill}%
\end{pgfscope}%
\begin{pgfscope}%
\pgfpathrectangle{\pgfqpoint{0.661006in}{0.524170in}}{\pgfqpoint{4.194036in}{1.071446in}}%
\pgfusepath{clip}%
\pgfsetbuttcap%
\pgfsetroundjoin%
\definecolor{currentfill}{rgb}{0.420941,0.465959,0.666472}%
\pgfsetfillcolor{currentfill}%
\pgfsetfillopacity{0.700000}%
\pgfsetlinewidth{1.003750pt}%
\definecolor{currentstroke}{rgb}{0.420941,0.465959,0.666472}%
\pgfsetstrokecolor{currentstroke}%
\pgfsetstrokeopacity{0.700000}%
\pgfsetdash{}{0pt}%
\pgfpathmoveto{\pgfqpoint{3.101700in}{0.973864in}}%
\pgfpathcurveto{\pgfqpoint{3.103541in}{0.973864in}}{\pgfqpoint{3.105308in}{0.974596in}}{\pgfqpoint{3.106610in}{0.975898in}}%
\pgfpathcurveto{\pgfqpoint{3.107912in}{0.977201in}}{\pgfqpoint{3.108644in}{0.978967in}}{\pgfqpoint{3.108644in}{0.980809in}}%
\pgfpathcurveto{\pgfqpoint{3.108644in}{0.982650in}}{\pgfqpoint{3.107912in}{0.984417in}}{\pgfqpoint{3.106610in}{0.985719in}}%
\pgfpathcurveto{\pgfqpoint{3.105308in}{0.987021in}}{\pgfqpoint{3.103541in}{0.987753in}}{\pgfqpoint{3.101700in}{0.987753in}}%
\pgfpathcurveto{\pgfqpoint{3.099858in}{0.987753in}}{\pgfqpoint{3.098092in}{0.987021in}}{\pgfqpoint{3.096789in}{0.985719in}}%
\pgfpathcurveto{\pgfqpoint{3.095487in}{0.984417in}}{\pgfqpoint{3.094755in}{0.982650in}}{\pgfqpoint{3.094755in}{0.980809in}}%
\pgfpathcurveto{\pgfqpoint{3.094755in}{0.978967in}}{\pgfqpoint{3.095487in}{0.977201in}}{\pgfqpoint{3.096789in}{0.975898in}}%
\pgfpathcurveto{\pgfqpoint{3.098092in}{0.974596in}}{\pgfqpoint{3.099858in}{0.973864in}}{\pgfqpoint{3.101700in}{0.973864in}}%
\pgfpathlineto{\pgfqpoint{3.101700in}{0.973864in}}%
\pgfpathclose%
\pgfusepath{stroke,fill}%
\end{pgfscope}%
\begin{pgfscope}%
\pgfpathrectangle{\pgfqpoint{0.661006in}{0.524170in}}{\pgfqpoint{4.194036in}{1.071446in}}%
\pgfusepath{clip}%
\pgfsetbuttcap%
\pgfsetroundjoin%
\definecolor{currentfill}{rgb}{0.418114,0.461462,0.662648}%
\pgfsetfillcolor{currentfill}%
\pgfsetfillopacity{0.700000}%
\pgfsetlinewidth{1.003750pt}%
\definecolor{currentstroke}{rgb}{0.418114,0.461462,0.662648}%
\pgfsetstrokecolor{currentstroke}%
\pgfsetstrokeopacity{0.700000}%
\pgfsetdash{}{0pt}%
\pgfpathmoveto{\pgfqpoint{3.124009in}{0.967551in}}%
\pgfpathcurveto{\pgfqpoint{3.125850in}{0.967551in}}{\pgfqpoint{3.127617in}{0.968283in}}{\pgfqpoint{3.128919in}{0.969585in}}%
\pgfpathcurveto{\pgfqpoint{3.130222in}{0.970888in}}{\pgfqpoint{3.130953in}{0.972654in}}{\pgfqpoint{3.130953in}{0.974496in}}%
\pgfpathcurveto{\pgfqpoint{3.130953in}{0.976337in}}{\pgfqpoint{3.130222in}{0.978104in}}{\pgfqpoint{3.128919in}{0.979406in}}%
\pgfpathcurveto{\pgfqpoint{3.127617in}{0.980708in}}{\pgfqpoint{3.125850in}{0.981440in}}{\pgfqpoint{3.124009in}{0.981440in}}%
\pgfpathcurveto{\pgfqpoint{3.122167in}{0.981440in}}{\pgfqpoint{3.120401in}{0.980708in}}{\pgfqpoint{3.119098in}{0.979406in}}%
\pgfpathcurveto{\pgfqpoint{3.117796in}{0.978104in}}{\pgfqpoint{3.117064in}{0.976337in}}{\pgfqpoint{3.117064in}{0.974496in}}%
\pgfpathcurveto{\pgfqpoint{3.117064in}{0.972654in}}{\pgfqpoint{3.117796in}{0.970888in}}{\pgfqpoint{3.119098in}{0.969585in}}%
\pgfpathcurveto{\pgfqpoint{3.120401in}{0.968283in}}{\pgfqpoint{3.122167in}{0.967551in}}{\pgfqpoint{3.124009in}{0.967551in}}%
\pgfpathlineto{\pgfqpoint{3.124009in}{0.967551in}}%
\pgfpathclose%
\pgfusepath{stroke,fill}%
\end{pgfscope}%
\begin{pgfscope}%
\pgfpathrectangle{\pgfqpoint{0.661006in}{0.524170in}}{\pgfqpoint{4.194036in}{1.071446in}}%
\pgfusepath{clip}%
\pgfsetbuttcap%
\pgfsetroundjoin%
\definecolor{currentfill}{rgb}{0.418114,0.461462,0.662648}%
\pgfsetfillcolor{currentfill}%
\pgfsetfillopacity{0.700000}%
\pgfsetlinewidth{1.003750pt}%
\definecolor{currentstroke}{rgb}{0.418114,0.461462,0.662648}%
\pgfsetstrokecolor{currentstroke}%
\pgfsetstrokeopacity{0.700000}%
\pgfsetdash{}{0pt}%
\pgfpathmoveto{\pgfqpoint{3.154684in}{0.962116in}}%
\pgfpathcurveto{\pgfqpoint{3.156525in}{0.962116in}}{\pgfqpoint{3.158292in}{0.962848in}}{\pgfqpoint{3.159594in}{0.964150in}}%
\pgfpathcurveto{\pgfqpoint{3.160896in}{0.965453in}}{\pgfqpoint{3.161628in}{0.967219in}}{\pgfqpoint{3.161628in}{0.969061in}}%
\pgfpathcurveto{\pgfqpoint{3.161628in}{0.970902in}}{\pgfqpoint{3.160896in}{0.972669in}}{\pgfqpoint{3.159594in}{0.973971in}}%
\pgfpathcurveto{\pgfqpoint{3.158292in}{0.975274in}}{\pgfqpoint{3.156525in}{0.976005in}}{\pgfqpoint{3.154684in}{0.976005in}}%
\pgfpathcurveto{\pgfqpoint{3.152842in}{0.976005in}}{\pgfqpoint{3.151076in}{0.975274in}}{\pgfqpoint{3.149773in}{0.973971in}}%
\pgfpathcurveto{\pgfqpoint{3.148471in}{0.972669in}}{\pgfqpoint{3.147739in}{0.970902in}}{\pgfqpoint{3.147739in}{0.969061in}}%
\pgfpathcurveto{\pgfqpoint{3.147739in}{0.967219in}}{\pgfqpoint{3.148471in}{0.965453in}}{\pgfqpoint{3.149773in}{0.964150in}}%
\pgfpathcurveto{\pgfqpoint{3.151076in}{0.962848in}}{\pgfqpoint{3.152842in}{0.962116in}}{\pgfqpoint{3.154684in}{0.962116in}}%
\pgfpathlineto{\pgfqpoint{3.154684in}{0.962116in}}%
\pgfpathclose%
\pgfusepath{stroke,fill}%
\end{pgfscope}%
\begin{pgfscope}%
\pgfpathrectangle{\pgfqpoint{0.661006in}{0.524170in}}{\pgfqpoint{4.194036in}{1.071446in}}%
\pgfusepath{clip}%
\pgfsetbuttcap%
\pgfsetroundjoin%
\definecolor{currentfill}{rgb}{0.415294,0.456969,0.658792}%
\pgfsetfillcolor{currentfill}%
\pgfsetfillopacity{0.700000}%
\pgfsetlinewidth{1.003750pt}%
\definecolor{currentstroke}{rgb}{0.415294,0.456969,0.658792}%
\pgfsetstrokecolor{currentstroke}%
\pgfsetstrokeopacity{0.700000}%
\pgfsetdash{}{0pt}%
\pgfpathmoveto{\pgfqpoint{3.179828in}{0.954104in}}%
\pgfpathcurveto{\pgfqpoint{3.181670in}{0.954104in}}{\pgfqpoint{3.183436in}{0.954835in}}{\pgfqpoint{3.184738in}{0.956138in}}%
\pgfpathcurveto{\pgfqpoint{3.186041in}{0.957440in}}{\pgfqpoint{3.186772in}{0.959206in}}{\pgfqpoint{3.186772in}{0.961048in}}%
\pgfpathcurveto{\pgfqpoint{3.186772in}{0.962890in}}{\pgfqpoint{3.186041in}{0.964656in}}{\pgfqpoint{3.184738in}{0.965959in}}%
\pgfpathcurveto{\pgfqpoint{3.183436in}{0.967261in}}{\pgfqpoint{3.181670in}{0.967992in}}{\pgfqpoint{3.179828in}{0.967992in}}%
\pgfpathcurveto{\pgfqpoint{3.177986in}{0.967992in}}{\pgfqpoint{3.176220in}{0.967261in}}{\pgfqpoint{3.174917in}{0.965959in}}%
\pgfpathcurveto{\pgfqpoint{3.173615in}{0.964656in}}{\pgfqpoint{3.172884in}{0.962890in}}{\pgfqpoint{3.172884in}{0.961048in}}%
\pgfpathcurveto{\pgfqpoint{3.172884in}{0.959206in}}{\pgfqpoint{3.173615in}{0.957440in}}{\pgfqpoint{3.174917in}{0.956138in}}%
\pgfpathcurveto{\pgfqpoint{3.176220in}{0.954835in}}{\pgfqpoint{3.177986in}{0.954104in}}{\pgfqpoint{3.179828in}{0.954104in}}%
\pgfpathlineto{\pgfqpoint{3.179828in}{0.954104in}}%
\pgfpathclose%
\pgfusepath{stroke,fill}%
\end{pgfscope}%
\begin{pgfscope}%
\pgfpathrectangle{\pgfqpoint{0.661006in}{0.524170in}}{\pgfqpoint{4.194036in}{1.071446in}}%
\pgfusepath{clip}%
\pgfsetbuttcap%
\pgfsetroundjoin%
\definecolor{currentfill}{rgb}{0.415294,0.456969,0.658792}%
\pgfsetfillcolor{currentfill}%
\pgfsetfillopacity{0.700000}%
\pgfsetlinewidth{1.003750pt}%
\definecolor{currentstroke}{rgb}{0.415294,0.456969,0.658792}%
\pgfsetstrokecolor{currentstroke}%
\pgfsetstrokeopacity{0.700000}%
\pgfsetdash{}{0pt}%
\pgfpathmoveto{\pgfqpoint{3.212408in}{0.949466in}}%
\pgfpathcurveto{\pgfqpoint{3.214250in}{0.949466in}}{\pgfqpoint{3.216017in}{0.950198in}}{\pgfqpoint{3.217319in}{0.951500in}}%
\pgfpathcurveto{\pgfqpoint{3.218621in}{0.952803in}}{\pgfqpoint{3.219353in}{0.954569in}}{\pgfqpoint{3.219353in}{0.956411in}}%
\pgfpathcurveto{\pgfqpoint{3.219353in}{0.958253in}}{\pgfqpoint{3.218621in}{0.960019in}}{\pgfqpoint{3.217319in}{0.961321in}}%
\pgfpathcurveto{\pgfqpoint{3.216017in}{0.962624in}}{\pgfqpoint{3.214250in}{0.963355in}}{\pgfqpoint{3.212408in}{0.963355in}}%
\pgfpathcurveto{\pgfqpoint{3.210567in}{0.963355in}}{\pgfqpoint{3.208800in}{0.962624in}}{\pgfqpoint{3.207498in}{0.961321in}}%
\pgfpathcurveto{\pgfqpoint{3.206196in}{0.960019in}}{\pgfqpoint{3.205464in}{0.958253in}}{\pgfqpoint{3.205464in}{0.956411in}}%
\pgfpathcurveto{\pgfqpoint{3.205464in}{0.954569in}}{\pgfqpoint{3.206196in}{0.952803in}}{\pgfqpoint{3.207498in}{0.951500in}}%
\pgfpathcurveto{\pgfqpoint{3.208800in}{0.950198in}}{\pgfqpoint{3.210567in}{0.949466in}}{\pgfqpoint{3.212408in}{0.949466in}}%
\pgfpathlineto{\pgfqpoint{3.212408in}{0.949466in}}%
\pgfpathclose%
\pgfusepath{stroke,fill}%
\end{pgfscope}%
\begin{pgfscope}%
\pgfpathrectangle{\pgfqpoint{0.661006in}{0.524170in}}{\pgfqpoint{4.194036in}{1.071446in}}%
\pgfusepath{clip}%
\pgfsetbuttcap%
\pgfsetroundjoin%
\definecolor{currentfill}{rgb}{0.415294,0.456969,0.658792}%
\pgfsetfillcolor{currentfill}%
\pgfsetfillopacity{0.700000}%
\pgfsetlinewidth{1.003750pt}%
\definecolor{currentstroke}{rgb}{0.415294,0.456969,0.658792}%
\pgfsetstrokecolor{currentstroke}%
\pgfsetstrokeopacity{0.700000}%
\pgfsetdash{}{0pt}%
\pgfpathmoveto{\pgfqpoint{3.216173in}{0.948239in}}%
\pgfpathcurveto{\pgfqpoint{3.218015in}{0.948239in}}{\pgfqpoint{3.219781in}{0.948971in}}{\pgfqpoint{3.221084in}{0.950273in}}%
\pgfpathcurveto{\pgfqpoint{3.222386in}{0.951576in}}{\pgfqpoint{3.223118in}{0.953342in}}{\pgfqpoint{3.223118in}{0.955184in}}%
\pgfpathcurveto{\pgfqpoint{3.223118in}{0.957026in}}{\pgfqpoint{3.222386in}{0.958792in}}{\pgfqpoint{3.221084in}{0.960094in}}%
\pgfpathcurveto{\pgfqpoint{3.219781in}{0.961397in}}{\pgfqpoint{3.218015in}{0.962128in}}{\pgfqpoint{3.216173in}{0.962128in}}%
\pgfpathcurveto{\pgfqpoint{3.214331in}{0.962128in}}{\pgfqpoint{3.212565in}{0.961397in}}{\pgfqpoint{3.211263in}{0.960094in}}%
\pgfpathcurveto{\pgfqpoint{3.209960in}{0.958792in}}{\pgfqpoint{3.209229in}{0.957026in}}{\pgfqpoint{3.209229in}{0.955184in}}%
\pgfpathcurveto{\pgfqpoint{3.209229in}{0.953342in}}{\pgfqpoint{3.209960in}{0.951576in}}{\pgfqpoint{3.211263in}{0.950273in}}%
\pgfpathcurveto{\pgfqpoint{3.212565in}{0.948971in}}{\pgfqpoint{3.214331in}{0.948239in}}{\pgfqpoint{3.216173in}{0.948239in}}%
\pgfpathlineto{\pgfqpoint{3.216173in}{0.948239in}}%
\pgfpathclose%
\pgfusepath{stroke,fill}%
\end{pgfscope}%
\begin{pgfscope}%
\pgfpathrectangle{\pgfqpoint{0.661006in}{0.524170in}}{\pgfqpoint{4.194036in}{1.071446in}}%
\pgfusepath{clip}%
\pgfsetbuttcap%
\pgfsetroundjoin%
\definecolor{currentfill}{rgb}{0.415294,0.456969,0.658792}%
\pgfsetfillcolor{currentfill}%
\pgfsetfillopacity{0.700000}%
\pgfsetlinewidth{1.003750pt}%
\definecolor{currentstroke}{rgb}{0.415294,0.456969,0.658792}%
\pgfsetstrokecolor{currentstroke}%
\pgfsetstrokeopacity{0.700000}%
\pgfsetdash{}{0pt}%
\pgfpathmoveto{\pgfqpoint{3.197005in}{0.952749in}}%
\pgfpathcurveto{\pgfqpoint{3.198846in}{0.952749in}}{\pgfqpoint{3.200613in}{0.953481in}}{\pgfqpoint{3.201915in}{0.954783in}}%
\pgfpathcurveto{\pgfqpoint{3.203217in}{0.956085in}}{\pgfqpoint{3.203949in}{0.957852in}}{\pgfqpoint{3.203949in}{0.959693in}}%
\pgfpathcurveto{\pgfqpoint{3.203949in}{0.961535in}}{\pgfqpoint{3.203217in}{0.963302in}}{\pgfqpoint{3.201915in}{0.964604in}}%
\pgfpathcurveto{\pgfqpoint{3.200613in}{0.965906in}}{\pgfqpoint{3.198846in}{0.966638in}}{\pgfqpoint{3.197005in}{0.966638in}}%
\pgfpathcurveto{\pgfqpoint{3.195163in}{0.966638in}}{\pgfqpoint{3.193396in}{0.965906in}}{\pgfqpoint{3.192094in}{0.964604in}}%
\pgfpathcurveto{\pgfqpoint{3.190792in}{0.963302in}}{\pgfqpoint{3.190060in}{0.961535in}}{\pgfqpoint{3.190060in}{0.959693in}}%
\pgfpathcurveto{\pgfqpoint{3.190060in}{0.957852in}}{\pgfqpoint{3.190792in}{0.956085in}}{\pgfqpoint{3.192094in}{0.954783in}}%
\pgfpathcurveto{\pgfqpoint{3.193396in}{0.953481in}}{\pgfqpoint{3.195163in}{0.952749in}}{\pgfqpoint{3.197005in}{0.952749in}}%
\pgfpathlineto{\pgfqpoint{3.197005in}{0.952749in}}%
\pgfpathclose%
\pgfusepath{stroke,fill}%
\end{pgfscope}%
\begin{pgfscope}%
\pgfpathrectangle{\pgfqpoint{0.661006in}{0.524170in}}{\pgfqpoint{4.194036in}{1.071446in}}%
\pgfusepath{clip}%
\pgfsetbuttcap%
\pgfsetroundjoin%
\definecolor{currentfill}{rgb}{0.415294,0.456969,0.658792}%
\pgfsetfillcolor{currentfill}%
\pgfsetfillopacity{0.700000}%
\pgfsetlinewidth{1.003750pt}%
\definecolor{currentstroke}{rgb}{0.415294,0.456969,0.658792}%
\pgfsetstrokecolor{currentstroke}%
\pgfsetstrokeopacity{0.700000}%
\pgfsetdash{}{0pt}%
\pgfpathmoveto{\pgfqpoint{3.192191in}{0.953305in}}%
\pgfpathcurveto{\pgfqpoint{3.194033in}{0.953305in}}{\pgfqpoint{3.195799in}{0.954037in}}{\pgfqpoint{3.197101in}{0.955339in}}%
\pgfpathcurveto{\pgfqpoint{3.198404in}{0.956641in}}{\pgfqpoint{3.199135in}{0.958408in}}{\pgfqpoint{3.199135in}{0.960249in}}%
\pgfpathcurveto{\pgfqpoint{3.199135in}{0.962091in}}{\pgfqpoint{3.198404in}{0.963858in}}{\pgfqpoint{3.197101in}{0.965160in}}%
\pgfpathcurveto{\pgfqpoint{3.195799in}{0.966462in}}{\pgfqpoint{3.194033in}{0.967194in}}{\pgfqpoint{3.192191in}{0.967194in}}%
\pgfpathcurveto{\pgfqpoint{3.190349in}{0.967194in}}{\pgfqpoint{3.188583in}{0.966462in}}{\pgfqpoint{3.187280in}{0.965160in}}%
\pgfpathcurveto{\pgfqpoint{3.185978in}{0.963858in}}{\pgfqpoint{3.185246in}{0.962091in}}{\pgfqpoint{3.185246in}{0.960249in}}%
\pgfpathcurveto{\pgfqpoint{3.185246in}{0.958408in}}{\pgfqpoint{3.185978in}{0.956641in}}{\pgfqpoint{3.187280in}{0.955339in}}%
\pgfpathcurveto{\pgfqpoint{3.188583in}{0.954037in}}{\pgfqpoint{3.190349in}{0.953305in}}{\pgfqpoint{3.192191in}{0.953305in}}%
\pgfpathlineto{\pgfqpoint{3.192191in}{0.953305in}}%
\pgfpathclose%
\pgfusepath{stroke,fill}%
\end{pgfscope}%
\begin{pgfscope}%
\pgfpathrectangle{\pgfqpoint{0.661006in}{0.524170in}}{\pgfqpoint{4.194036in}{1.071446in}}%
\pgfusepath{clip}%
\pgfsetbuttcap%
\pgfsetroundjoin%
\definecolor{currentfill}{rgb}{0.412480,0.452478,0.654904}%
\pgfsetfillcolor{currentfill}%
\pgfsetfillopacity{0.700000}%
\pgfsetlinewidth{1.003750pt}%
\definecolor{currentstroke}{rgb}{0.412480,0.452478,0.654904}%
\pgfsetstrokecolor{currentstroke}%
\pgfsetstrokeopacity{0.700000}%
\pgfsetdash{}{0pt}%
\pgfpathmoveto{\pgfqpoint{3.186428in}{0.953046in}}%
\pgfpathcurveto{\pgfqpoint{3.188269in}{0.953046in}}{\pgfqpoint{3.190036in}{0.953777in}}{\pgfqpoint{3.191338in}{0.955080in}}%
\pgfpathcurveto{\pgfqpoint{3.192640in}{0.956382in}}{\pgfqpoint{3.193372in}{0.958149in}}{\pgfqpoint{3.193372in}{0.959990in}}%
\pgfpathcurveto{\pgfqpoint{3.193372in}{0.961832in}}{\pgfqpoint{3.192640in}{0.963598in}}{\pgfqpoint{3.191338in}{0.964901in}}%
\pgfpathcurveto{\pgfqpoint{3.190036in}{0.966203in}}{\pgfqpoint{3.188269in}{0.966935in}}{\pgfqpoint{3.186428in}{0.966935in}}%
\pgfpathcurveto{\pgfqpoint{3.184586in}{0.966935in}}{\pgfqpoint{3.182820in}{0.966203in}}{\pgfqpoint{3.181517in}{0.964901in}}%
\pgfpathcurveto{\pgfqpoint{3.180215in}{0.963598in}}{\pgfqpoint{3.179483in}{0.961832in}}{\pgfqpoint{3.179483in}{0.959990in}}%
\pgfpathcurveto{\pgfqpoint{3.179483in}{0.958149in}}{\pgfqpoint{3.180215in}{0.956382in}}{\pgfqpoint{3.181517in}{0.955080in}}%
\pgfpathcurveto{\pgfqpoint{3.182820in}{0.953777in}}{\pgfqpoint{3.184586in}{0.953046in}}{\pgfqpoint{3.186428in}{0.953046in}}%
\pgfpathlineto{\pgfqpoint{3.186428in}{0.953046in}}%
\pgfpathclose%
\pgfusepath{stroke,fill}%
\end{pgfscope}%
\begin{pgfscope}%
\pgfpathrectangle{\pgfqpoint{0.661006in}{0.524170in}}{\pgfqpoint{4.194036in}{1.071446in}}%
\pgfusepath{clip}%
\pgfsetbuttcap%
\pgfsetroundjoin%
\definecolor{currentfill}{rgb}{0.412480,0.452478,0.654904}%
\pgfsetfillcolor{currentfill}%
\pgfsetfillopacity{0.700000}%
\pgfsetlinewidth{1.003750pt}%
\definecolor{currentstroke}{rgb}{0.412480,0.452478,0.654904}%
\pgfsetstrokecolor{currentstroke}%
\pgfsetstrokeopacity{0.700000}%
\pgfsetdash{}{0pt}%
\pgfpathmoveto{\pgfqpoint{3.210642in}{0.950576in}}%
\pgfpathcurveto{\pgfqpoint{3.212484in}{0.950576in}}{\pgfqpoint{3.214251in}{0.951308in}}{\pgfqpoint{3.215553in}{0.952610in}}%
\pgfpathcurveto{\pgfqpoint{3.216855in}{0.953912in}}{\pgfqpoint{3.217587in}{0.955679in}}{\pgfqpoint{3.217587in}{0.957520in}}%
\pgfpathcurveto{\pgfqpoint{3.217587in}{0.959362in}}{\pgfqpoint{3.216855in}{0.961129in}}{\pgfqpoint{3.215553in}{0.962431in}}%
\pgfpathcurveto{\pgfqpoint{3.214251in}{0.963733in}}{\pgfqpoint{3.212484in}{0.964465in}}{\pgfqpoint{3.210642in}{0.964465in}}%
\pgfpathcurveto{\pgfqpoint{3.208801in}{0.964465in}}{\pgfqpoint{3.207034in}{0.963733in}}{\pgfqpoint{3.205732in}{0.962431in}}%
\pgfpathcurveto{\pgfqpoint{3.204430in}{0.961129in}}{\pgfqpoint{3.203698in}{0.959362in}}{\pgfqpoint{3.203698in}{0.957520in}}%
\pgfpathcurveto{\pgfqpoint{3.203698in}{0.955679in}}{\pgfqpoint{3.204430in}{0.953912in}}{\pgfqpoint{3.205732in}{0.952610in}}%
\pgfpathcurveto{\pgfqpoint{3.207034in}{0.951308in}}{\pgfqpoint{3.208801in}{0.950576in}}{\pgfqpoint{3.210642in}{0.950576in}}%
\pgfpathlineto{\pgfqpoint{3.210642in}{0.950576in}}%
\pgfpathclose%
\pgfusepath{stroke,fill}%
\end{pgfscope}%
\begin{pgfscope}%
\pgfpathrectangle{\pgfqpoint{0.661006in}{0.524170in}}{\pgfqpoint{4.194036in}{1.071446in}}%
\pgfusepath{clip}%
\pgfsetbuttcap%
\pgfsetroundjoin%
\definecolor{currentfill}{rgb}{0.409671,0.447990,0.650985}%
\pgfsetfillcolor{currentfill}%
\pgfsetfillopacity{0.700000}%
\pgfsetlinewidth{1.003750pt}%
\definecolor{currentstroke}{rgb}{0.409671,0.447990,0.650985}%
\pgfsetstrokecolor{currentstroke}%
\pgfsetstrokeopacity{0.700000}%
\pgfsetdash{}{0pt}%
\pgfpathmoveto{\pgfqpoint{3.210968in}{0.948860in}}%
\pgfpathcurveto{\pgfqpoint{3.212809in}{0.948860in}}{\pgfqpoint{3.214576in}{0.949592in}}{\pgfqpoint{3.215878in}{0.950894in}}%
\pgfpathcurveto{\pgfqpoint{3.217180in}{0.952196in}}{\pgfqpoint{3.217912in}{0.953963in}}{\pgfqpoint{3.217912in}{0.955804in}}%
\pgfpathcurveto{\pgfqpoint{3.217912in}{0.957646in}}{\pgfqpoint{3.217180in}{0.959413in}}{\pgfqpoint{3.215878in}{0.960715in}}%
\pgfpathcurveto{\pgfqpoint{3.214576in}{0.962017in}}{\pgfqpoint{3.212809in}{0.962749in}}{\pgfqpoint{3.210968in}{0.962749in}}%
\pgfpathcurveto{\pgfqpoint{3.209126in}{0.962749in}}{\pgfqpoint{3.207360in}{0.962017in}}{\pgfqpoint{3.206057in}{0.960715in}}%
\pgfpathcurveto{\pgfqpoint{3.204755in}{0.959413in}}{\pgfqpoint{3.204023in}{0.957646in}}{\pgfqpoint{3.204023in}{0.955804in}}%
\pgfpathcurveto{\pgfqpoint{3.204023in}{0.953963in}}{\pgfqpoint{3.204755in}{0.952196in}}{\pgfqpoint{3.206057in}{0.950894in}}%
\pgfpathcurveto{\pgfqpoint{3.207360in}{0.949592in}}{\pgfqpoint{3.209126in}{0.948860in}}{\pgfqpoint{3.210968in}{0.948860in}}%
\pgfpathlineto{\pgfqpoint{3.210968in}{0.948860in}}%
\pgfpathclose%
\pgfusepath{stroke,fill}%
\end{pgfscope}%
\begin{pgfscope}%
\pgfpathrectangle{\pgfqpoint{0.661006in}{0.524170in}}{\pgfqpoint{4.194036in}{1.071446in}}%
\pgfusepath{clip}%
\pgfsetbuttcap%
\pgfsetroundjoin%
\definecolor{currentfill}{rgb}{0.409671,0.447990,0.650985}%
\pgfsetfillcolor{currentfill}%
\pgfsetfillopacity{0.700000}%
\pgfsetlinewidth{1.003750pt}%
\definecolor{currentstroke}{rgb}{0.409671,0.447990,0.650985}%
\pgfsetstrokecolor{currentstroke}%
\pgfsetstrokeopacity{0.700000}%
\pgfsetdash{}{0pt}%
\pgfpathmoveto{\pgfqpoint{3.210317in}{0.949183in}}%
\pgfpathcurveto{\pgfqpoint{3.212159in}{0.949183in}}{\pgfqpoint{3.213925in}{0.949914in}}{\pgfqpoint{3.215227in}{0.951217in}}%
\pgfpathcurveto{\pgfqpoint{3.216530in}{0.952519in}}{\pgfqpoint{3.217261in}{0.954285in}}{\pgfqpoint{3.217261in}{0.956127in}}%
\pgfpathcurveto{\pgfqpoint{3.217261in}{0.957969in}}{\pgfqpoint{3.216530in}{0.959735in}}{\pgfqpoint{3.215227in}{0.961038in}}%
\pgfpathcurveto{\pgfqpoint{3.213925in}{0.962340in}}{\pgfqpoint{3.212159in}{0.963072in}}{\pgfqpoint{3.210317in}{0.963072in}}%
\pgfpathcurveto{\pgfqpoint{3.208475in}{0.963072in}}{\pgfqpoint{3.206709in}{0.962340in}}{\pgfqpoint{3.205407in}{0.961038in}}%
\pgfpathcurveto{\pgfqpoint{3.204104in}{0.959735in}}{\pgfqpoint{3.203373in}{0.957969in}}{\pgfqpoint{3.203373in}{0.956127in}}%
\pgfpathcurveto{\pgfqpoint{3.203373in}{0.954285in}}{\pgfqpoint{3.204104in}{0.952519in}}{\pgfqpoint{3.205407in}{0.951217in}}%
\pgfpathcurveto{\pgfqpoint{3.206709in}{0.949914in}}{\pgfqpoint{3.208475in}{0.949183in}}{\pgfqpoint{3.210317in}{0.949183in}}%
\pgfpathlineto{\pgfqpoint{3.210317in}{0.949183in}}%
\pgfpathclose%
\pgfusepath{stroke,fill}%
\end{pgfscope}%
\begin{pgfscope}%
\pgfpathrectangle{\pgfqpoint{0.661006in}{0.524170in}}{\pgfqpoint{4.194036in}{1.071446in}}%
\pgfusepath{clip}%
\pgfsetbuttcap%
\pgfsetroundjoin%
\definecolor{currentfill}{rgb}{0.409671,0.447990,0.650985}%
\pgfsetfillcolor{currentfill}%
\pgfsetfillopacity{0.700000}%
\pgfsetlinewidth{1.003750pt}%
\definecolor{currentstroke}{rgb}{0.409671,0.447990,0.650985}%
\pgfsetstrokecolor{currentstroke}%
\pgfsetstrokeopacity{0.700000}%
\pgfsetdash{}{0pt}%
\pgfpathmoveto{\pgfqpoint{3.200789in}{0.951802in}}%
\pgfpathcurveto{\pgfqpoint{3.202631in}{0.951802in}}{\pgfqpoint{3.204397in}{0.952534in}}{\pgfqpoint{3.205700in}{0.953836in}}%
\pgfpathcurveto{\pgfqpoint{3.207002in}{0.955138in}}{\pgfqpoint{3.207734in}{0.956905in}}{\pgfqpoint{3.207734in}{0.958746in}}%
\pgfpathcurveto{\pgfqpoint{3.207734in}{0.960588in}}{\pgfqpoint{3.207002in}{0.962354in}}{\pgfqpoint{3.205700in}{0.963657in}}%
\pgfpathcurveto{\pgfqpoint{3.204397in}{0.964959in}}{\pgfqpoint{3.202631in}{0.965691in}}{\pgfqpoint{3.200789in}{0.965691in}}%
\pgfpathcurveto{\pgfqpoint{3.198947in}{0.965691in}}{\pgfqpoint{3.197181in}{0.964959in}}{\pgfqpoint{3.195879in}{0.963657in}}%
\pgfpathcurveto{\pgfqpoint{3.194576in}{0.962354in}}{\pgfqpoint{3.193845in}{0.960588in}}{\pgfqpoint{3.193845in}{0.958746in}}%
\pgfpathcurveto{\pgfqpoint{3.193845in}{0.956905in}}{\pgfqpoint{3.194576in}{0.955138in}}{\pgfqpoint{3.195879in}{0.953836in}}%
\pgfpathcurveto{\pgfqpoint{3.197181in}{0.952534in}}{\pgfqpoint{3.198947in}{0.951802in}}{\pgfqpoint{3.200789in}{0.951802in}}%
\pgfpathlineto{\pgfqpoint{3.200789in}{0.951802in}}%
\pgfpathclose%
\pgfusepath{stroke,fill}%
\end{pgfscope}%
\begin{pgfscope}%
\pgfpathrectangle{\pgfqpoint{0.661006in}{0.524170in}}{\pgfqpoint{4.194036in}{1.071446in}}%
\pgfusepath{clip}%
\pgfsetbuttcap%
\pgfsetroundjoin%
\definecolor{currentfill}{rgb}{0.409671,0.447990,0.650985}%
\pgfsetfillcolor{currentfill}%
\pgfsetfillopacity{0.700000}%
\pgfsetlinewidth{1.003750pt}%
\definecolor{currentstroke}{rgb}{0.409671,0.447990,0.650985}%
\pgfsetstrokecolor{currentstroke}%
\pgfsetstrokeopacity{0.700000}%
\pgfsetdash{}{0pt}%
\pgfpathmoveto{\pgfqpoint{3.191773in}{0.953955in}}%
\pgfpathcurveto{\pgfqpoint{3.193614in}{0.953955in}}{\pgfqpoint{3.195381in}{0.954687in}}{\pgfqpoint{3.196683in}{0.955989in}}%
\pgfpathcurveto{\pgfqpoint{3.197985in}{0.957292in}}{\pgfqpoint{3.198717in}{0.959058in}}{\pgfqpoint{3.198717in}{0.960900in}}%
\pgfpathcurveto{\pgfqpoint{3.198717in}{0.962742in}}{\pgfqpoint{3.197985in}{0.964508in}}{\pgfqpoint{3.196683in}{0.965810in}}%
\pgfpathcurveto{\pgfqpoint{3.195381in}{0.967113in}}{\pgfqpoint{3.193614in}{0.967844in}}{\pgfqpoint{3.191773in}{0.967844in}}%
\pgfpathcurveto{\pgfqpoint{3.189931in}{0.967844in}}{\pgfqpoint{3.188164in}{0.967113in}}{\pgfqpoint{3.186862in}{0.965810in}}%
\pgfpathcurveto{\pgfqpoint{3.185560in}{0.964508in}}{\pgfqpoint{3.184828in}{0.962742in}}{\pgfqpoint{3.184828in}{0.960900in}}%
\pgfpathcurveto{\pgfqpoint{3.184828in}{0.959058in}}{\pgfqpoint{3.185560in}{0.957292in}}{\pgfqpoint{3.186862in}{0.955989in}}%
\pgfpathcurveto{\pgfqpoint{3.188164in}{0.954687in}}{\pgfqpoint{3.189931in}{0.953955in}}{\pgfqpoint{3.191773in}{0.953955in}}%
\pgfpathlineto{\pgfqpoint{3.191773in}{0.953955in}}%
\pgfpathclose%
\pgfusepath{stroke,fill}%
\end{pgfscope}%
\begin{pgfscope}%
\pgfpathrectangle{\pgfqpoint{0.661006in}{0.524170in}}{\pgfqpoint{4.194036in}{1.071446in}}%
\pgfusepath{clip}%
\pgfsetbuttcap%
\pgfsetroundjoin%
\definecolor{currentfill}{rgb}{0.409671,0.447990,0.650985}%
\pgfsetfillcolor{currentfill}%
\pgfsetfillopacity{0.700000}%
\pgfsetlinewidth{1.003750pt}%
\definecolor{currentstroke}{rgb}{0.409671,0.447990,0.650985}%
\pgfsetstrokecolor{currentstroke}%
\pgfsetstrokeopacity{0.700000}%
\pgfsetdash{}{0pt}%
\pgfpathmoveto{\pgfqpoint{3.182617in}{0.955121in}}%
\pgfpathcurveto{\pgfqpoint{3.184458in}{0.955121in}}{\pgfqpoint{3.186225in}{0.955853in}}{\pgfqpoint{3.187527in}{0.957155in}}%
\pgfpathcurveto{\pgfqpoint{3.188829in}{0.958458in}}{\pgfqpoint{3.189561in}{0.960224in}}{\pgfqpoint{3.189561in}{0.962066in}}%
\pgfpathcurveto{\pgfqpoint{3.189561in}{0.963908in}}{\pgfqpoint{3.188829in}{0.965674in}}{\pgfqpoint{3.187527in}{0.966976in}}%
\pgfpathcurveto{\pgfqpoint{3.186225in}{0.968279in}}{\pgfqpoint{3.184458in}{0.969010in}}{\pgfqpoint{3.182617in}{0.969010in}}%
\pgfpathcurveto{\pgfqpoint{3.180775in}{0.969010in}}{\pgfqpoint{3.179008in}{0.968279in}}{\pgfqpoint{3.177706in}{0.966976in}}%
\pgfpathcurveto{\pgfqpoint{3.176404in}{0.965674in}}{\pgfqpoint{3.175672in}{0.963908in}}{\pgfqpoint{3.175672in}{0.962066in}}%
\pgfpathcurveto{\pgfqpoint{3.175672in}{0.960224in}}{\pgfqpoint{3.176404in}{0.958458in}}{\pgfqpoint{3.177706in}{0.957155in}}%
\pgfpathcurveto{\pgfqpoint{3.179008in}{0.955853in}}{\pgfqpoint{3.180775in}{0.955121in}}{\pgfqpoint{3.182617in}{0.955121in}}%
\pgfpathlineto{\pgfqpoint{3.182617in}{0.955121in}}%
\pgfpathclose%
\pgfusepath{stroke,fill}%
\end{pgfscope}%
\begin{pgfscope}%
\pgfpathrectangle{\pgfqpoint{0.661006in}{0.524170in}}{\pgfqpoint{4.194036in}{1.071446in}}%
\pgfusepath{clip}%
\pgfsetbuttcap%
\pgfsetroundjoin%
\definecolor{currentfill}{rgb}{0.406868,0.443505,0.647034}%
\pgfsetfillcolor{currentfill}%
\pgfsetfillopacity{0.700000}%
\pgfsetlinewidth{1.003750pt}%
\definecolor{currentstroke}{rgb}{0.406868,0.443505,0.647034}%
\pgfsetstrokecolor{currentstroke}%
\pgfsetstrokeopacity{0.700000}%
\pgfsetdash{}{0pt}%
\pgfpathmoveto{\pgfqpoint{3.171648in}{0.957589in}}%
\pgfpathcurveto{\pgfqpoint{3.173490in}{0.957589in}}{\pgfqpoint{3.175256in}{0.958320in}}{\pgfqpoint{3.176558in}{0.959622in}}%
\pgfpathcurveto{\pgfqpoint{3.177861in}{0.960925in}}{\pgfqpoint{3.178592in}{0.962691in}}{\pgfqpoint{3.178592in}{0.964533in}}%
\pgfpathcurveto{\pgfqpoint{3.178592in}{0.966375in}}{\pgfqpoint{3.177861in}{0.968141in}}{\pgfqpoint{3.176558in}{0.969443in}}%
\pgfpathcurveto{\pgfqpoint{3.175256in}{0.970746in}}{\pgfqpoint{3.173490in}{0.971477in}}{\pgfqpoint{3.171648in}{0.971477in}}%
\pgfpathcurveto{\pgfqpoint{3.169806in}{0.971477in}}{\pgfqpoint{3.168040in}{0.970746in}}{\pgfqpoint{3.166737in}{0.969443in}}%
\pgfpathcurveto{\pgfqpoint{3.165435in}{0.968141in}}{\pgfqpoint{3.164704in}{0.966375in}}{\pgfqpoint{3.164704in}{0.964533in}}%
\pgfpathcurveto{\pgfqpoint{3.164704in}{0.962691in}}{\pgfqpoint{3.165435in}{0.960925in}}{\pgfqpoint{3.166737in}{0.959622in}}%
\pgfpathcurveto{\pgfqpoint{3.168040in}{0.958320in}}{\pgfqpoint{3.169806in}{0.957589in}}{\pgfqpoint{3.171648in}{0.957589in}}%
\pgfpathlineto{\pgfqpoint{3.171648in}{0.957589in}}%
\pgfpathclose%
\pgfusepath{stroke,fill}%
\end{pgfscope}%
\begin{pgfscope}%
\pgfpathrectangle{\pgfqpoint{0.661006in}{0.524170in}}{\pgfqpoint{4.194036in}{1.071446in}}%
\pgfusepath{clip}%
\pgfsetbuttcap%
\pgfsetroundjoin%
\definecolor{currentfill}{rgb}{0.406868,0.443505,0.647034}%
\pgfsetfillcolor{currentfill}%
\pgfsetfillopacity{0.700000}%
\pgfsetlinewidth{1.003750pt}%
\definecolor{currentstroke}{rgb}{0.406868,0.443505,0.647034}%
\pgfsetstrokecolor{currentstroke}%
\pgfsetstrokeopacity{0.700000}%
\pgfsetdash{}{0pt}%
\pgfpathmoveto{\pgfqpoint{3.156310in}{0.961871in}}%
\pgfpathcurveto{\pgfqpoint{3.158152in}{0.961871in}}{\pgfqpoint{3.159919in}{0.962603in}}{\pgfqpoint{3.161221in}{0.963905in}}%
\pgfpathcurveto{\pgfqpoint{3.162523in}{0.965208in}}{\pgfqpoint{3.163255in}{0.966974in}}{\pgfqpoint{3.163255in}{0.968816in}}%
\pgfpathcurveto{\pgfqpoint{3.163255in}{0.970657in}}{\pgfqpoint{3.162523in}{0.972424in}}{\pgfqpoint{3.161221in}{0.973726in}}%
\pgfpathcurveto{\pgfqpoint{3.159919in}{0.975028in}}{\pgfqpoint{3.158152in}{0.975760in}}{\pgfqpoint{3.156310in}{0.975760in}}%
\pgfpathcurveto{\pgfqpoint{3.154469in}{0.975760in}}{\pgfqpoint{3.152702in}{0.975028in}}{\pgfqpoint{3.151400in}{0.973726in}}%
\pgfpathcurveto{\pgfqpoint{3.150098in}{0.972424in}}{\pgfqpoint{3.149366in}{0.970657in}}{\pgfqpoint{3.149366in}{0.968816in}}%
\pgfpathcurveto{\pgfqpoint{3.149366in}{0.966974in}}{\pgfqpoint{3.150098in}{0.965208in}}{\pgfqpoint{3.151400in}{0.963905in}}%
\pgfpathcurveto{\pgfqpoint{3.152702in}{0.962603in}}{\pgfqpoint{3.154469in}{0.961871in}}{\pgfqpoint{3.156310in}{0.961871in}}%
\pgfpathlineto{\pgfqpoint{3.156310in}{0.961871in}}%
\pgfpathclose%
\pgfusepath{stroke,fill}%
\end{pgfscope}%
\begin{pgfscope}%
\pgfpathrectangle{\pgfqpoint{0.661006in}{0.524170in}}{\pgfqpoint{4.194036in}{1.071446in}}%
\pgfusepath{clip}%
\pgfsetbuttcap%
\pgfsetroundjoin%
\definecolor{currentfill}{rgb}{0.404069,0.439024,0.643051}%
\pgfsetfillcolor{currentfill}%
\pgfsetfillopacity{0.700000}%
\pgfsetlinewidth{1.003750pt}%
\definecolor{currentstroke}{rgb}{0.404069,0.439024,0.643051}%
\pgfsetstrokecolor{currentstroke}%
\pgfsetstrokeopacity{0.700000}%
\pgfsetdash{}{0pt}%
\pgfpathmoveto{\pgfqpoint{3.134373in}{0.965284in}}%
\pgfpathcurveto{\pgfqpoint{3.136215in}{0.965284in}}{\pgfqpoint{3.137981in}{0.966015in}}{\pgfqpoint{3.139284in}{0.967318in}}%
\pgfpathcurveto{\pgfqpoint{3.140586in}{0.968620in}}{\pgfqpoint{3.141318in}{0.970386in}}{\pgfqpoint{3.141318in}{0.972228in}}%
\pgfpathcurveto{\pgfqpoint{3.141318in}{0.974070in}}{\pgfqpoint{3.140586in}{0.975836in}}{\pgfqpoint{3.139284in}{0.977138in}}%
\pgfpathcurveto{\pgfqpoint{3.137981in}{0.978441in}}{\pgfqpoint{3.136215in}{0.979172in}}{\pgfqpoint{3.134373in}{0.979172in}}%
\pgfpathcurveto{\pgfqpoint{3.132532in}{0.979172in}}{\pgfqpoint{3.130765in}{0.978441in}}{\pgfqpoint{3.129463in}{0.977138in}}%
\pgfpathcurveto{\pgfqpoint{3.128160in}{0.975836in}}{\pgfqpoint{3.127429in}{0.974070in}}{\pgfqpoint{3.127429in}{0.972228in}}%
\pgfpathcurveto{\pgfqpoint{3.127429in}{0.970386in}}{\pgfqpoint{3.128160in}{0.968620in}}{\pgfqpoint{3.129463in}{0.967318in}}%
\pgfpathcurveto{\pgfqpoint{3.130765in}{0.966015in}}{\pgfqpoint{3.132532in}{0.965284in}}{\pgfqpoint{3.134373in}{0.965284in}}%
\pgfpathlineto{\pgfqpoint{3.134373in}{0.965284in}}%
\pgfpathclose%
\pgfusepath{stroke,fill}%
\end{pgfscope}%
\begin{pgfscope}%
\pgfpathrectangle{\pgfqpoint{0.661006in}{0.524170in}}{\pgfqpoint{4.194036in}{1.071446in}}%
\pgfusepath{clip}%
\pgfsetbuttcap%
\pgfsetroundjoin%
\definecolor{currentfill}{rgb}{0.404069,0.439024,0.643051}%
\pgfsetfillcolor{currentfill}%
\pgfsetfillopacity{0.700000}%
\pgfsetlinewidth{1.003750pt}%
\definecolor{currentstroke}{rgb}{0.404069,0.439024,0.643051}%
\pgfsetstrokecolor{currentstroke}%
\pgfsetstrokeopacity{0.700000}%
\pgfsetdash{}{0pt}%
\pgfpathmoveto{\pgfqpoint{3.123962in}{0.965857in}}%
\pgfpathcurveto{\pgfqpoint{3.125804in}{0.965857in}}{\pgfqpoint{3.127571in}{0.966589in}}{\pgfqpoint{3.128873in}{0.967891in}}%
\pgfpathcurveto{\pgfqpoint{3.130175in}{0.969193in}}{\pgfqpoint{3.130907in}{0.970960in}}{\pgfqpoint{3.130907in}{0.972801in}}%
\pgfpathcurveto{\pgfqpoint{3.130907in}{0.974643in}}{\pgfqpoint{3.130175in}{0.976409in}}{\pgfqpoint{3.128873in}{0.977712in}}%
\pgfpathcurveto{\pgfqpoint{3.127571in}{0.979014in}}{\pgfqpoint{3.125804in}{0.979746in}}{\pgfqpoint{3.123962in}{0.979746in}}%
\pgfpathcurveto{\pgfqpoint{3.122121in}{0.979746in}}{\pgfqpoint{3.120354in}{0.979014in}}{\pgfqpoint{3.119052in}{0.977712in}}%
\pgfpathcurveto{\pgfqpoint{3.117750in}{0.976409in}}{\pgfqpoint{3.117018in}{0.974643in}}{\pgfqpoint{3.117018in}{0.972801in}}%
\pgfpathcurveto{\pgfqpoint{3.117018in}{0.970960in}}{\pgfqpoint{3.117750in}{0.969193in}}{\pgfqpoint{3.119052in}{0.967891in}}%
\pgfpathcurveto{\pgfqpoint{3.120354in}{0.966589in}}{\pgfqpoint{3.122121in}{0.965857in}}{\pgfqpoint{3.123962in}{0.965857in}}%
\pgfpathlineto{\pgfqpoint{3.123962in}{0.965857in}}%
\pgfpathclose%
\pgfusepath{stroke,fill}%
\end{pgfscope}%
\begin{pgfscope}%
\pgfpathrectangle{\pgfqpoint{0.661006in}{0.524170in}}{\pgfqpoint{4.194036in}{1.071446in}}%
\pgfusepath{clip}%
\pgfsetbuttcap%
\pgfsetroundjoin%
\definecolor{currentfill}{rgb}{0.404069,0.439024,0.643051}%
\pgfsetfillcolor{currentfill}%
\pgfsetfillopacity{0.700000}%
\pgfsetlinewidth{1.003750pt}%
\definecolor{currentstroke}{rgb}{0.404069,0.439024,0.643051}%
\pgfsetstrokecolor{currentstroke}%
\pgfsetstrokeopacity{0.700000}%
\pgfsetdash{}{0pt}%
\pgfpathmoveto{\pgfqpoint{3.137145in}{0.964454in}}%
\pgfpathcurveto{\pgfqpoint{3.138987in}{0.964454in}}{\pgfqpoint{3.140753in}{0.965185in}}{\pgfqpoint{3.142056in}{0.966488in}}%
\pgfpathcurveto{\pgfqpoint{3.143358in}{0.967790in}}{\pgfqpoint{3.144090in}{0.969556in}}{\pgfqpoint{3.144090in}{0.971398in}}%
\pgfpathcurveto{\pgfqpoint{3.144090in}{0.973240in}}{\pgfqpoint{3.143358in}{0.975006in}}{\pgfqpoint{3.142056in}{0.976309in}}%
\pgfpathcurveto{\pgfqpoint{3.140753in}{0.977611in}}{\pgfqpoint{3.138987in}{0.978343in}}{\pgfqpoint{3.137145in}{0.978343in}}%
\pgfpathcurveto{\pgfqpoint{3.135304in}{0.978343in}}{\pgfqpoint{3.133537in}{0.977611in}}{\pgfqpoint{3.132235in}{0.976309in}}%
\pgfpathcurveto{\pgfqpoint{3.130933in}{0.975006in}}{\pgfqpoint{3.130201in}{0.973240in}}{\pgfqpoint{3.130201in}{0.971398in}}%
\pgfpathcurveto{\pgfqpoint{3.130201in}{0.969556in}}{\pgfqpoint{3.130933in}{0.967790in}}{\pgfqpoint{3.132235in}{0.966488in}}%
\pgfpathcurveto{\pgfqpoint{3.133537in}{0.965185in}}{\pgfqpoint{3.135304in}{0.964454in}}{\pgfqpoint{3.137145in}{0.964454in}}%
\pgfpathlineto{\pgfqpoint{3.137145in}{0.964454in}}%
\pgfpathclose%
\pgfusepath{stroke,fill}%
\end{pgfscope}%
\begin{pgfscope}%
\pgfpathrectangle{\pgfqpoint{0.661006in}{0.524170in}}{\pgfqpoint{4.194036in}{1.071446in}}%
\pgfusepath{clip}%
\pgfsetbuttcap%
\pgfsetroundjoin%
\definecolor{currentfill}{rgb}{0.401275,0.434546,0.639036}%
\pgfsetfillcolor{currentfill}%
\pgfsetfillopacity{0.700000}%
\pgfsetlinewidth{1.003750pt}%
\definecolor{currentstroke}{rgb}{0.401275,0.434546,0.639036}%
\pgfsetstrokecolor{currentstroke}%
\pgfsetstrokeopacity{0.700000}%
\pgfsetdash{}{0pt}%
\pgfpathmoveto{\pgfqpoint{3.146411in}{0.961335in}}%
\pgfpathcurveto{\pgfqpoint{3.148253in}{0.961335in}}{\pgfqpoint{3.150019in}{0.962066in}}{\pgfqpoint{3.151321in}{0.963369in}}%
\pgfpathcurveto{\pgfqpoint{3.152624in}{0.964671in}}{\pgfqpoint{3.153355in}{0.966438in}}{\pgfqpoint{3.153355in}{0.968279in}}%
\pgfpathcurveto{\pgfqpoint{3.153355in}{0.970121in}}{\pgfqpoint{3.152624in}{0.971887in}}{\pgfqpoint{3.151321in}{0.973190in}}%
\pgfpathcurveto{\pgfqpoint{3.150019in}{0.974492in}}{\pgfqpoint{3.148253in}{0.975224in}}{\pgfqpoint{3.146411in}{0.975224in}}%
\pgfpathcurveto{\pgfqpoint{3.144569in}{0.975224in}}{\pgfqpoint{3.142803in}{0.974492in}}{\pgfqpoint{3.141500in}{0.973190in}}%
\pgfpathcurveto{\pgfqpoint{3.140198in}{0.971887in}}{\pgfqpoint{3.139466in}{0.970121in}}{\pgfqpoint{3.139466in}{0.968279in}}%
\pgfpathcurveto{\pgfqpoint{3.139466in}{0.966438in}}{\pgfqpoint{3.140198in}{0.964671in}}{\pgfqpoint{3.141500in}{0.963369in}}%
\pgfpathcurveto{\pgfqpoint{3.142803in}{0.962066in}}{\pgfqpoint{3.144569in}{0.961335in}}{\pgfqpoint{3.146411in}{0.961335in}}%
\pgfpathlineto{\pgfqpoint{3.146411in}{0.961335in}}%
\pgfpathclose%
\pgfusepath{stroke,fill}%
\end{pgfscope}%
\begin{pgfscope}%
\pgfpathrectangle{\pgfqpoint{0.661006in}{0.524170in}}{\pgfqpoint{4.194036in}{1.071446in}}%
\pgfusepath{clip}%
\pgfsetbuttcap%
\pgfsetroundjoin%
\definecolor{currentfill}{rgb}{0.401275,0.434546,0.639036}%
\pgfsetfillcolor{currentfill}%
\pgfsetfillopacity{0.700000}%
\pgfsetlinewidth{1.003750pt}%
\definecolor{currentstroke}{rgb}{0.401275,0.434546,0.639036}%
\pgfsetstrokecolor{currentstroke}%
\pgfsetstrokeopacity{0.700000}%
\pgfsetdash{}{0pt}%
\pgfpathmoveto{\pgfqpoint{3.158681in}{0.957031in}}%
\pgfpathcurveto{\pgfqpoint{3.160522in}{0.957031in}}{\pgfqpoint{3.162289in}{0.957763in}}{\pgfqpoint{3.163591in}{0.959065in}}%
\pgfpathcurveto{\pgfqpoint{3.164894in}{0.960367in}}{\pgfqpoint{3.165625in}{0.962134in}}{\pgfqpoint{3.165625in}{0.963975in}}%
\pgfpathcurveto{\pgfqpoint{3.165625in}{0.965817in}}{\pgfqpoint{3.164894in}{0.967584in}}{\pgfqpoint{3.163591in}{0.968886in}}%
\pgfpathcurveto{\pgfqpoint{3.162289in}{0.970188in}}{\pgfqpoint{3.160522in}{0.970920in}}{\pgfqpoint{3.158681in}{0.970920in}}%
\pgfpathcurveto{\pgfqpoint{3.156839in}{0.970920in}}{\pgfqpoint{3.155073in}{0.970188in}}{\pgfqpoint{3.153770in}{0.968886in}}%
\pgfpathcurveto{\pgfqpoint{3.152468in}{0.967584in}}{\pgfqpoint{3.151736in}{0.965817in}}{\pgfqpoint{3.151736in}{0.963975in}}%
\pgfpathcurveto{\pgfqpoint{3.151736in}{0.962134in}}{\pgfqpoint{3.152468in}{0.960367in}}{\pgfqpoint{3.153770in}{0.959065in}}%
\pgfpathcurveto{\pgfqpoint{3.155073in}{0.957763in}}{\pgfqpoint{3.156839in}{0.957031in}}{\pgfqpoint{3.158681in}{0.957031in}}%
\pgfpathlineto{\pgfqpoint{3.158681in}{0.957031in}}%
\pgfpathclose%
\pgfusepath{stroke,fill}%
\end{pgfscope}%
\begin{pgfscope}%
\pgfpathrectangle{\pgfqpoint{0.661006in}{0.524170in}}{\pgfqpoint{4.194036in}{1.071446in}}%
\pgfusepath{clip}%
\pgfsetbuttcap%
\pgfsetroundjoin%
\definecolor{currentfill}{rgb}{0.401275,0.434546,0.639036}%
\pgfsetfillcolor{currentfill}%
\pgfsetfillopacity{0.700000}%
\pgfsetlinewidth{1.003750pt}%
\definecolor{currentstroke}{rgb}{0.401275,0.434546,0.639036}%
\pgfsetstrokecolor{currentstroke}%
\pgfsetstrokeopacity{0.700000}%
\pgfsetdash{}{0pt}%
\pgfpathmoveto{\pgfqpoint{3.183407in}{0.951591in}}%
\pgfpathcurveto{\pgfqpoint{3.185248in}{0.951591in}}{\pgfqpoint{3.187015in}{0.952323in}}{\pgfqpoint{3.188317in}{0.953625in}}%
\pgfpathcurveto{\pgfqpoint{3.189619in}{0.954928in}}{\pgfqpoint{3.190351in}{0.956694in}}{\pgfqpoint{3.190351in}{0.958536in}}%
\pgfpathcurveto{\pgfqpoint{3.190351in}{0.960377in}}{\pgfqpoint{3.189619in}{0.962144in}}{\pgfqpoint{3.188317in}{0.963446in}}%
\pgfpathcurveto{\pgfqpoint{3.187015in}{0.964749in}}{\pgfqpoint{3.185248in}{0.965480in}}{\pgfqpoint{3.183407in}{0.965480in}}%
\pgfpathcurveto{\pgfqpoint{3.181565in}{0.965480in}}{\pgfqpoint{3.179799in}{0.964749in}}{\pgfqpoint{3.178496in}{0.963446in}}%
\pgfpathcurveto{\pgfqpoint{3.177194in}{0.962144in}}{\pgfqpoint{3.176462in}{0.960377in}}{\pgfqpoint{3.176462in}{0.958536in}}%
\pgfpathcurveto{\pgfqpoint{3.176462in}{0.956694in}}{\pgfqpoint{3.177194in}{0.954928in}}{\pgfqpoint{3.178496in}{0.953625in}}%
\pgfpathcurveto{\pgfqpoint{3.179799in}{0.952323in}}{\pgfqpoint{3.181565in}{0.951591in}}{\pgfqpoint{3.183407in}{0.951591in}}%
\pgfpathlineto{\pgfqpoint{3.183407in}{0.951591in}}%
\pgfpathclose%
\pgfusepath{stroke,fill}%
\end{pgfscope}%
\begin{pgfscope}%
\pgfpathrectangle{\pgfqpoint{0.661006in}{0.524170in}}{\pgfqpoint{4.194036in}{1.071446in}}%
\pgfusepath{clip}%
\pgfsetbuttcap%
\pgfsetroundjoin%
\definecolor{currentfill}{rgb}{0.401275,0.434546,0.639036}%
\pgfsetfillcolor{currentfill}%
\pgfsetfillopacity{0.700000}%
\pgfsetlinewidth{1.003750pt}%
\definecolor{currentstroke}{rgb}{0.401275,0.434546,0.639036}%
\pgfsetstrokecolor{currentstroke}%
\pgfsetstrokeopacity{0.700000}%
\pgfsetdash{}{0pt}%
\pgfpathmoveto{\pgfqpoint{3.205669in}{0.947983in}}%
\pgfpathcurveto{\pgfqpoint{3.207511in}{0.947983in}}{\pgfqpoint{3.209277in}{0.948714in}}{\pgfqpoint{3.210580in}{0.950017in}}%
\pgfpathcurveto{\pgfqpoint{3.211882in}{0.951319in}}{\pgfqpoint{3.212614in}{0.953085in}}{\pgfqpoint{3.212614in}{0.954927in}}%
\pgfpathcurveto{\pgfqpoint{3.212614in}{0.956769in}}{\pgfqpoint{3.211882in}{0.958535in}}{\pgfqpoint{3.210580in}{0.959838in}}%
\pgfpathcurveto{\pgfqpoint{3.209277in}{0.961140in}}{\pgfqpoint{3.207511in}{0.961872in}}{\pgfqpoint{3.205669in}{0.961872in}}%
\pgfpathcurveto{\pgfqpoint{3.203828in}{0.961872in}}{\pgfqpoint{3.202061in}{0.961140in}}{\pgfqpoint{3.200759in}{0.959838in}}%
\pgfpathcurveto{\pgfqpoint{3.199457in}{0.958535in}}{\pgfqpoint{3.198725in}{0.956769in}}{\pgfqpoint{3.198725in}{0.954927in}}%
\pgfpathcurveto{\pgfqpoint{3.198725in}{0.953085in}}{\pgfqpoint{3.199457in}{0.951319in}}{\pgfqpoint{3.200759in}{0.950017in}}%
\pgfpathcurveto{\pgfqpoint{3.202061in}{0.948714in}}{\pgfqpoint{3.203828in}{0.947983in}}{\pgfqpoint{3.205669in}{0.947983in}}%
\pgfpathlineto{\pgfqpoint{3.205669in}{0.947983in}}%
\pgfpathclose%
\pgfusepath{stroke,fill}%
\end{pgfscope}%
\begin{pgfscope}%
\pgfpathrectangle{\pgfqpoint{0.661006in}{0.524170in}}{\pgfqpoint{4.194036in}{1.071446in}}%
\pgfusepath{clip}%
\pgfsetbuttcap%
\pgfsetroundjoin%
\definecolor{currentfill}{rgb}{0.398485,0.430073,0.634990}%
\pgfsetfillcolor{currentfill}%
\pgfsetfillopacity{0.700000}%
\pgfsetlinewidth{1.003750pt}%
\definecolor{currentstroke}{rgb}{0.398485,0.430073,0.634990}%
\pgfsetstrokecolor{currentstroke}%
\pgfsetstrokeopacity{0.700000}%
\pgfsetdash{}{0pt}%
\pgfpathmoveto{\pgfqpoint{3.201068in}{0.948404in}}%
\pgfpathcurveto{\pgfqpoint{3.202910in}{0.948404in}}{\pgfqpoint{3.204676in}{0.949136in}}{\pgfqpoint{3.205979in}{0.950438in}}%
\pgfpathcurveto{\pgfqpoint{3.207281in}{0.951740in}}{\pgfqpoint{3.208012in}{0.953507in}}{\pgfqpoint{3.208012in}{0.955348in}}%
\pgfpathcurveto{\pgfqpoint{3.208012in}{0.957190in}}{\pgfqpoint{3.207281in}{0.958957in}}{\pgfqpoint{3.205979in}{0.960259in}}%
\pgfpathcurveto{\pgfqpoint{3.204676in}{0.961561in}}{\pgfqpoint{3.202910in}{0.962293in}}{\pgfqpoint{3.201068in}{0.962293in}}%
\pgfpathcurveto{\pgfqpoint{3.199226in}{0.962293in}}{\pgfqpoint{3.197460in}{0.961561in}}{\pgfqpoint{3.196158in}{0.960259in}}%
\pgfpathcurveto{\pgfqpoint{3.194855in}{0.958957in}}{\pgfqpoint{3.194124in}{0.957190in}}{\pgfqpoint{3.194124in}{0.955348in}}%
\pgfpathcurveto{\pgfqpoint{3.194124in}{0.953507in}}{\pgfqpoint{3.194855in}{0.951740in}}{\pgfqpoint{3.196158in}{0.950438in}}%
\pgfpathcurveto{\pgfqpoint{3.197460in}{0.949136in}}{\pgfqpoint{3.199226in}{0.948404in}}{\pgfqpoint{3.201068in}{0.948404in}}%
\pgfpathlineto{\pgfqpoint{3.201068in}{0.948404in}}%
\pgfpathclose%
\pgfusepath{stroke,fill}%
\end{pgfscope}%
\begin{pgfscope}%
\pgfpathrectangle{\pgfqpoint{0.661006in}{0.524170in}}{\pgfqpoint{4.194036in}{1.071446in}}%
\pgfusepath{clip}%
\pgfsetbuttcap%
\pgfsetroundjoin%
\definecolor{currentfill}{rgb}{0.398485,0.430073,0.634990}%
\pgfsetfillcolor{currentfill}%
\pgfsetfillopacity{0.700000}%
\pgfsetlinewidth{1.003750pt}%
\definecolor{currentstroke}{rgb}{0.398485,0.430073,0.634990}%
\pgfsetstrokecolor{currentstroke}%
\pgfsetstrokeopacity{0.700000}%
\pgfsetdash{}{0pt}%
\pgfpathmoveto{\pgfqpoint{3.194468in}{0.948530in}}%
\pgfpathcurveto{\pgfqpoint{3.196310in}{0.948530in}}{\pgfqpoint{3.198076in}{0.949262in}}{\pgfqpoint{3.199379in}{0.950564in}}%
\pgfpathcurveto{\pgfqpoint{3.200681in}{0.951866in}}{\pgfqpoint{3.201413in}{0.953633in}}{\pgfqpoint{3.201413in}{0.955474in}}%
\pgfpathcurveto{\pgfqpoint{3.201413in}{0.957316in}}{\pgfqpoint{3.200681in}{0.959083in}}{\pgfqpoint{3.199379in}{0.960385in}}%
\pgfpathcurveto{\pgfqpoint{3.198076in}{0.961687in}}{\pgfqpoint{3.196310in}{0.962419in}}{\pgfqpoint{3.194468in}{0.962419in}}%
\pgfpathcurveto{\pgfqpoint{3.192627in}{0.962419in}}{\pgfqpoint{3.190860in}{0.961687in}}{\pgfqpoint{3.189558in}{0.960385in}}%
\pgfpathcurveto{\pgfqpoint{3.188256in}{0.959083in}}{\pgfqpoint{3.187524in}{0.957316in}}{\pgfqpoint{3.187524in}{0.955474in}}%
\pgfpathcurveto{\pgfqpoint{3.187524in}{0.953633in}}{\pgfqpoint{3.188256in}{0.951866in}}{\pgfqpoint{3.189558in}{0.950564in}}%
\pgfpathcurveto{\pgfqpoint{3.190860in}{0.949262in}}{\pgfqpoint{3.192627in}{0.948530in}}{\pgfqpoint{3.194468in}{0.948530in}}%
\pgfpathlineto{\pgfqpoint{3.194468in}{0.948530in}}%
\pgfpathclose%
\pgfusepath{stroke,fill}%
\end{pgfscope}%
\begin{pgfscope}%
\pgfpathrectangle{\pgfqpoint{0.661006in}{0.524170in}}{\pgfqpoint{4.194036in}{1.071446in}}%
\pgfusepath{clip}%
\pgfsetbuttcap%
\pgfsetroundjoin%
\definecolor{currentfill}{rgb}{0.398485,0.430073,0.634990}%
\pgfsetfillcolor{currentfill}%
\pgfsetfillopacity{0.700000}%
\pgfsetlinewidth{1.003750pt}%
\definecolor{currentstroke}{rgb}{0.398485,0.430073,0.634990}%
\pgfsetstrokecolor{currentstroke}%
\pgfsetstrokeopacity{0.700000}%
\pgfsetdash{}{0pt}%
\pgfpathmoveto{\pgfqpoint{3.218079in}{0.943173in}}%
\pgfpathcurveto{\pgfqpoint{3.219920in}{0.943173in}}{\pgfqpoint{3.221687in}{0.943905in}}{\pgfqpoint{3.222989in}{0.945207in}}%
\pgfpathcurveto{\pgfqpoint{3.224291in}{0.946509in}}{\pgfqpoint{3.225023in}{0.948276in}}{\pgfqpoint{3.225023in}{0.950118in}}%
\pgfpathcurveto{\pgfqpoint{3.225023in}{0.951959in}}{\pgfqpoint{3.224291in}{0.953726in}}{\pgfqpoint{3.222989in}{0.955028in}}%
\pgfpathcurveto{\pgfqpoint{3.221687in}{0.956330in}}{\pgfqpoint{3.219920in}{0.957062in}}{\pgfqpoint{3.218079in}{0.957062in}}%
\pgfpathcurveto{\pgfqpoint{3.216237in}{0.957062in}}{\pgfqpoint{3.214471in}{0.956330in}}{\pgfqpoint{3.213168in}{0.955028in}}%
\pgfpathcurveto{\pgfqpoint{3.211866in}{0.953726in}}{\pgfqpoint{3.211134in}{0.951959in}}{\pgfqpoint{3.211134in}{0.950118in}}%
\pgfpathcurveto{\pgfqpoint{3.211134in}{0.948276in}}{\pgfqpoint{3.211866in}{0.946509in}}{\pgfqpoint{3.213168in}{0.945207in}}%
\pgfpathcurveto{\pgfqpoint{3.214471in}{0.943905in}}{\pgfqpoint{3.216237in}{0.943173in}}{\pgfqpoint{3.218079in}{0.943173in}}%
\pgfpathlineto{\pgfqpoint{3.218079in}{0.943173in}}%
\pgfpathclose%
\pgfusepath{stroke,fill}%
\end{pgfscope}%
\begin{pgfscope}%
\pgfpathrectangle{\pgfqpoint{0.661006in}{0.524170in}}{\pgfqpoint{4.194036in}{1.071446in}}%
\pgfusepath{clip}%
\pgfsetbuttcap%
\pgfsetroundjoin%
\definecolor{currentfill}{rgb}{0.395698,0.425603,0.630912}%
\pgfsetfillcolor{currentfill}%
\pgfsetfillopacity{0.700000}%
\pgfsetlinewidth{1.003750pt}%
\definecolor{currentstroke}{rgb}{0.395698,0.425603,0.630912}%
\pgfsetstrokecolor{currentstroke}%
\pgfsetstrokeopacity{0.700000}%
\pgfsetdash{}{0pt}%
\pgfpathmoveto{\pgfqpoint{3.244896in}{0.937248in}}%
\pgfpathcurveto{\pgfqpoint{3.246738in}{0.937248in}}{\pgfqpoint{3.248504in}{0.937980in}}{\pgfqpoint{3.249807in}{0.939282in}}%
\pgfpathcurveto{\pgfqpoint{3.251109in}{0.940584in}}{\pgfqpoint{3.251841in}{0.942351in}}{\pgfqpoint{3.251841in}{0.944193in}}%
\pgfpathcurveto{\pgfqpoint{3.251841in}{0.946034in}}{\pgfqpoint{3.251109in}{0.947801in}}{\pgfqpoint{3.249807in}{0.949103in}}%
\pgfpathcurveto{\pgfqpoint{3.248504in}{0.950405in}}{\pgfqpoint{3.246738in}{0.951137in}}{\pgfqpoint{3.244896in}{0.951137in}}%
\pgfpathcurveto{\pgfqpoint{3.243054in}{0.951137in}}{\pgfqpoint{3.241288in}{0.950405in}}{\pgfqpoint{3.239986in}{0.949103in}}%
\pgfpathcurveto{\pgfqpoint{3.238683in}{0.947801in}}{\pgfqpoint{3.237952in}{0.946034in}}{\pgfqpoint{3.237952in}{0.944193in}}%
\pgfpathcurveto{\pgfqpoint{3.237952in}{0.942351in}}{\pgfqpoint{3.238683in}{0.940584in}}{\pgfqpoint{3.239986in}{0.939282in}}%
\pgfpathcurveto{\pgfqpoint{3.241288in}{0.937980in}}{\pgfqpoint{3.243054in}{0.937248in}}{\pgfqpoint{3.244896in}{0.937248in}}%
\pgfpathlineto{\pgfqpoint{3.244896in}{0.937248in}}%
\pgfpathclose%
\pgfusepath{stroke,fill}%
\end{pgfscope}%
\begin{pgfscope}%
\pgfpathrectangle{\pgfqpoint{0.661006in}{0.524170in}}{\pgfqpoint{4.194036in}{1.071446in}}%
\pgfusepath{clip}%
\pgfsetbuttcap%
\pgfsetroundjoin%
\definecolor{currentfill}{rgb}{0.395698,0.425603,0.630912}%
\pgfsetfillcolor{currentfill}%
\pgfsetfillopacity{0.700000}%
\pgfsetlinewidth{1.003750pt}%
\definecolor{currentstroke}{rgb}{0.395698,0.425603,0.630912}%
\pgfsetstrokecolor{currentstroke}%
\pgfsetstrokeopacity{0.700000}%
\pgfsetdash{}{0pt}%
\pgfpathmoveto{\pgfqpoint{3.270040in}{0.932605in}}%
\pgfpathcurveto{\pgfqpoint{3.271882in}{0.932605in}}{\pgfqpoint{3.273648in}{0.933337in}}{\pgfqpoint{3.274951in}{0.934639in}}%
\pgfpathcurveto{\pgfqpoint{3.276253in}{0.935941in}}{\pgfqpoint{3.276985in}{0.937708in}}{\pgfqpoint{3.276985in}{0.939549in}}%
\pgfpathcurveto{\pgfqpoint{3.276985in}{0.941391in}}{\pgfqpoint{3.276253in}{0.943158in}}{\pgfqpoint{3.274951in}{0.944460in}}%
\pgfpathcurveto{\pgfqpoint{3.273648in}{0.945762in}}{\pgfqpoint{3.271882in}{0.946494in}}{\pgfqpoint{3.270040in}{0.946494in}}%
\pgfpathcurveto{\pgfqpoint{3.268199in}{0.946494in}}{\pgfqpoint{3.266432in}{0.945762in}}{\pgfqpoint{3.265130in}{0.944460in}}%
\pgfpathcurveto{\pgfqpoint{3.263828in}{0.943158in}}{\pgfqpoint{3.263096in}{0.941391in}}{\pgfqpoint{3.263096in}{0.939549in}}%
\pgfpathcurveto{\pgfqpoint{3.263096in}{0.937708in}}{\pgfqpoint{3.263828in}{0.935941in}}{\pgfqpoint{3.265130in}{0.934639in}}%
\pgfpathcurveto{\pgfqpoint{3.266432in}{0.933337in}}{\pgfqpoint{3.268199in}{0.932605in}}{\pgfqpoint{3.270040in}{0.932605in}}%
\pgfpathlineto{\pgfqpoint{3.270040in}{0.932605in}}%
\pgfpathclose%
\pgfusepath{stroke,fill}%
\end{pgfscope}%
\begin{pgfscope}%
\pgfpathrectangle{\pgfqpoint{0.661006in}{0.524170in}}{\pgfqpoint{4.194036in}{1.071446in}}%
\pgfusepath{clip}%
\pgfsetbuttcap%
\pgfsetroundjoin%
\definecolor{currentfill}{rgb}{0.395698,0.425603,0.630912}%
\pgfsetfillcolor{currentfill}%
\pgfsetfillopacity{0.700000}%
\pgfsetlinewidth{1.003750pt}%
\definecolor{currentstroke}{rgb}{0.395698,0.425603,0.630912}%
\pgfsetstrokecolor{currentstroke}%
\pgfsetstrokeopacity{0.700000}%
\pgfsetdash{}{0pt}%
\pgfpathmoveto{\pgfqpoint{3.270970in}{0.930100in}}%
\pgfpathcurveto{\pgfqpoint{3.272811in}{0.930100in}}{\pgfqpoint{3.274578in}{0.930831in}}{\pgfqpoint{3.275880in}{0.932134in}}%
\pgfpathcurveto{\pgfqpoint{3.277183in}{0.933436in}}{\pgfqpoint{3.277914in}{0.935202in}}{\pgfqpoint{3.277914in}{0.937044in}}%
\pgfpathcurveto{\pgfqpoint{3.277914in}{0.938886in}}{\pgfqpoint{3.277183in}{0.940652in}}{\pgfqpoint{3.275880in}{0.941954in}}%
\pgfpathcurveto{\pgfqpoint{3.274578in}{0.943257in}}{\pgfqpoint{3.272811in}{0.943988in}}{\pgfqpoint{3.270970in}{0.943988in}}%
\pgfpathcurveto{\pgfqpoint{3.269128in}{0.943988in}}{\pgfqpoint{3.267362in}{0.943257in}}{\pgfqpoint{3.266059in}{0.941954in}}%
\pgfpathcurveto{\pgfqpoint{3.264757in}{0.940652in}}{\pgfqpoint{3.264025in}{0.938886in}}{\pgfqpoint{3.264025in}{0.937044in}}%
\pgfpathcurveto{\pgfqpoint{3.264025in}{0.935202in}}{\pgfqpoint{3.264757in}{0.933436in}}{\pgfqpoint{3.266059in}{0.932134in}}%
\pgfpathcurveto{\pgfqpoint{3.267362in}{0.930831in}}{\pgfqpoint{3.269128in}{0.930100in}}{\pgfqpoint{3.270970in}{0.930100in}}%
\pgfpathlineto{\pgfqpoint{3.270970in}{0.930100in}}%
\pgfpathclose%
\pgfusepath{stroke,fill}%
\end{pgfscope}%
\begin{pgfscope}%
\pgfpathrectangle{\pgfqpoint{0.661006in}{0.524170in}}{\pgfqpoint{4.194036in}{1.071446in}}%
\pgfusepath{clip}%
\pgfsetbuttcap%
\pgfsetroundjoin%
\definecolor{currentfill}{rgb}{0.395698,0.425603,0.630912}%
\pgfsetfillcolor{currentfill}%
\pgfsetfillopacity{0.700000}%
\pgfsetlinewidth{1.003750pt}%
\definecolor{currentstroke}{rgb}{0.395698,0.425603,0.630912}%
\pgfsetstrokecolor{currentstroke}%
\pgfsetstrokeopacity{0.700000}%
\pgfsetdash{}{0pt}%
\pgfpathmoveto{\pgfqpoint{3.279940in}{0.929016in}}%
\pgfpathcurveto{\pgfqpoint{3.281782in}{0.929016in}}{\pgfqpoint{3.283548in}{0.929748in}}{\pgfqpoint{3.284850in}{0.931050in}}%
\pgfpathcurveto{\pgfqpoint{3.286153in}{0.932352in}}{\pgfqpoint{3.286884in}{0.934119in}}{\pgfqpoint{3.286884in}{0.935960in}}%
\pgfpathcurveto{\pgfqpoint{3.286884in}{0.937802in}}{\pgfqpoint{3.286153in}{0.939569in}}{\pgfqpoint{3.284850in}{0.940871in}}%
\pgfpathcurveto{\pgfqpoint{3.283548in}{0.942173in}}{\pgfqpoint{3.281782in}{0.942905in}}{\pgfqpoint{3.279940in}{0.942905in}}%
\pgfpathcurveto{\pgfqpoint{3.278098in}{0.942905in}}{\pgfqpoint{3.276332in}{0.942173in}}{\pgfqpoint{3.275029in}{0.940871in}}%
\pgfpathcurveto{\pgfqpoint{3.273727in}{0.939569in}}{\pgfqpoint{3.272995in}{0.937802in}}{\pgfqpoint{3.272995in}{0.935960in}}%
\pgfpathcurveto{\pgfqpoint{3.272995in}{0.934119in}}{\pgfqpoint{3.273727in}{0.932352in}}{\pgfqpoint{3.275029in}{0.931050in}}%
\pgfpathcurveto{\pgfqpoint{3.276332in}{0.929748in}}{\pgfqpoint{3.278098in}{0.929016in}}{\pgfqpoint{3.279940in}{0.929016in}}%
\pgfpathlineto{\pgfqpoint{3.279940in}{0.929016in}}%
\pgfpathclose%
\pgfusepath{stroke,fill}%
\end{pgfscope}%
\begin{pgfscope}%
\pgfpathrectangle{\pgfqpoint{0.661006in}{0.524170in}}{\pgfqpoint{4.194036in}{1.071446in}}%
\pgfusepath{clip}%
\pgfsetbuttcap%
\pgfsetroundjoin%
\definecolor{currentfill}{rgb}{0.392914,0.421139,0.626801}%
\pgfsetfillcolor{currentfill}%
\pgfsetfillopacity{0.700000}%
\pgfsetlinewidth{1.003750pt}%
\definecolor{currentstroke}{rgb}{0.392914,0.421139,0.626801}%
\pgfsetstrokecolor{currentstroke}%
\pgfsetstrokeopacity{0.700000}%
\pgfsetdash{}{0pt}%
\pgfpathmoveto{\pgfqpoint{3.277623in}{0.928380in}}%
\pgfpathcurveto{\pgfqpoint{3.279464in}{0.928380in}}{\pgfqpoint{3.281231in}{0.929112in}}{\pgfqpoint{3.282533in}{0.930414in}}%
\pgfpathcurveto{\pgfqpoint{3.283835in}{0.931717in}}{\pgfqpoint{3.284567in}{0.933483in}}{\pgfqpoint{3.284567in}{0.935325in}}%
\pgfpathcurveto{\pgfqpoint{3.284567in}{0.937167in}}{\pgfqpoint{3.283835in}{0.938933in}}{\pgfqpoint{3.282533in}{0.940235in}}%
\pgfpathcurveto{\pgfqpoint{3.281231in}{0.941538in}}{\pgfqpoint{3.279464in}{0.942269in}}{\pgfqpoint{3.277623in}{0.942269in}}%
\pgfpathcurveto{\pgfqpoint{3.275781in}{0.942269in}}{\pgfqpoint{3.274015in}{0.941538in}}{\pgfqpoint{3.272712in}{0.940235in}}%
\pgfpathcurveto{\pgfqpoint{3.271410in}{0.938933in}}{\pgfqpoint{3.270678in}{0.937167in}}{\pgfqpoint{3.270678in}{0.935325in}}%
\pgfpathcurveto{\pgfqpoint{3.270678in}{0.933483in}}{\pgfqpoint{3.271410in}{0.931717in}}{\pgfqpoint{3.272712in}{0.930414in}}%
\pgfpathcurveto{\pgfqpoint{3.274015in}{0.929112in}}{\pgfqpoint{3.275781in}{0.928380in}}{\pgfqpoint{3.277623in}{0.928380in}}%
\pgfpathlineto{\pgfqpoint{3.277623in}{0.928380in}}%
\pgfpathclose%
\pgfusepath{stroke,fill}%
\end{pgfscope}%
\begin{pgfscope}%
\pgfpathrectangle{\pgfqpoint{0.661006in}{0.524170in}}{\pgfqpoint{4.194036in}{1.071446in}}%
\pgfusepath{clip}%
\pgfsetbuttcap%
\pgfsetroundjoin%
\definecolor{currentfill}{rgb}{0.392914,0.421139,0.626801}%
\pgfsetfillcolor{currentfill}%
\pgfsetfillopacity{0.700000}%
\pgfsetlinewidth{1.003750pt}%
\definecolor{currentstroke}{rgb}{0.392914,0.421139,0.626801}%
\pgfsetstrokecolor{currentstroke}%
\pgfsetstrokeopacity{0.700000}%
\pgfsetdash{}{0pt}%
\pgfpathmoveto{\pgfqpoint{3.291978in}{0.925525in}}%
\pgfpathcurveto{\pgfqpoint{3.293819in}{0.925525in}}{\pgfqpoint{3.295586in}{0.926257in}}{\pgfqpoint{3.296888in}{0.927559in}}%
\pgfpathcurveto{\pgfqpoint{3.298190in}{0.928862in}}{\pgfqpoint{3.298922in}{0.930628in}}{\pgfqpoint{3.298922in}{0.932470in}}%
\pgfpathcurveto{\pgfqpoint{3.298922in}{0.934312in}}{\pgfqpoint{3.298190in}{0.936078in}}{\pgfqpoint{3.296888in}{0.937380in}}%
\pgfpathcurveto{\pgfqpoint{3.295586in}{0.938683in}}{\pgfqpoint{3.293819in}{0.939414in}}{\pgfqpoint{3.291978in}{0.939414in}}%
\pgfpathcurveto{\pgfqpoint{3.290136in}{0.939414in}}{\pgfqpoint{3.288369in}{0.938683in}}{\pgfqpoint{3.287067in}{0.937380in}}%
\pgfpathcurveto{\pgfqpoint{3.285765in}{0.936078in}}{\pgfqpoint{3.285033in}{0.934312in}}{\pgfqpoint{3.285033in}{0.932470in}}%
\pgfpathcurveto{\pgfqpoint{3.285033in}{0.930628in}}{\pgfqpoint{3.285765in}{0.928862in}}{\pgfqpoint{3.287067in}{0.927559in}}%
\pgfpathcurveto{\pgfqpoint{3.288369in}{0.926257in}}{\pgfqpoint{3.290136in}{0.925525in}}{\pgfqpoint{3.291978in}{0.925525in}}%
\pgfpathlineto{\pgfqpoint{3.291978in}{0.925525in}}%
\pgfpathclose%
\pgfusepath{stroke,fill}%
\end{pgfscope}%
\begin{pgfscope}%
\pgfpathrectangle{\pgfqpoint{0.661006in}{0.524170in}}{\pgfqpoint{4.194036in}{1.071446in}}%
\pgfusepath{clip}%
\pgfsetbuttcap%
\pgfsetroundjoin%
\definecolor{currentfill}{rgb}{0.392914,0.421139,0.626801}%
\pgfsetfillcolor{currentfill}%
\pgfsetfillopacity{0.700000}%
\pgfsetlinewidth{1.003750pt}%
\definecolor{currentstroke}{rgb}{0.392914,0.421139,0.626801}%
\pgfsetstrokecolor{currentstroke}%
\pgfsetstrokeopacity{0.700000}%
\pgfsetdash{}{0pt}%
\pgfpathmoveto{\pgfqpoint{3.306618in}{0.920857in}}%
\pgfpathcurveto{\pgfqpoint{3.308460in}{0.920857in}}{\pgfqpoint{3.310226in}{0.921589in}}{\pgfqpoint{3.311528in}{0.922891in}}%
\pgfpathcurveto{\pgfqpoint{3.312831in}{0.924193in}}{\pgfqpoint{3.313562in}{0.925960in}}{\pgfqpoint{3.313562in}{0.927802in}}%
\pgfpathcurveto{\pgfqpoint{3.313562in}{0.929643in}}{\pgfqpoint{3.312831in}{0.931410in}}{\pgfqpoint{3.311528in}{0.932712in}}%
\pgfpathcurveto{\pgfqpoint{3.310226in}{0.934014in}}{\pgfqpoint{3.308460in}{0.934746in}}{\pgfqpoint{3.306618in}{0.934746in}}%
\pgfpathcurveto{\pgfqpoint{3.304776in}{0.934746in}}{\pgfqpoint{3.303010in}{0.934014in}}{\pgfqpoint{3.301707in}{0.932712in}}%
\pgfpathcurveto{\pgfqpoint{3.300405in}{0.931410in}}{\pgfqpoint{3.299673in}{0.929643in}}{\pgfqpoint{3.299673in}{0.927802in}}%
\pgfpathcurveto{\pgfqpoint{3.299673in}{0.925960in}}{\pgfqpoint{3.300405in}{0.924193in}}{\pgfqpoint{3.301707in}{0.922891in}}%
\pgfpathcurveto{\pgfqpoint{3.303010in}{0.921589in}}{\pgfqpoint{3.304776in}{0.920857in}}{\pgfqpoint{3.306618in}{0.920857in}}%
\pgfpathlineto{\pgfqpoint{3.306618in}{0.920857in}}%
\pgfpathclose%
\pgfusepath{stroke,fill}%
\end{pgfscope}%
\begin{pgfscope}%
\pgfpathrectangle{\pgfqpoint{0.661006in}{0.524170in}}{\pgfqpoint{4.194036in}{1.071446in}}%
\pgfusepath{clip}%
\pgfsetbuttcap%
\pgfsetroundjoin%
\definecolor{currentfill}{rgb}{0.390133,0.416678,0.622659}%
\pgfsetfillcolor{currentfill}%
\pgfsetfillopacity{0.700000}%
\pgfsetlinewidth{1.003750pt}%
\definecolor{currentstroke}{rgb}{0.390133,0.416678,0.622659}%
\pgfsetstrokecolor{currentstroke}%
\pgfsetstrokeopacity{0.700000}%
\pgfsetdash{}{0pt}%
\pgfpathmoveto{\pgfqpoint{3.337618in}{0.915194in}}%
\pgfpathcurveto{\pgfqpoint{3.339460in}{0.915194in}}{\pgfqpoint{3.341226in}{0.915925in}}{\pgfqpoint{3.342529in}{0.917228in}}%
\pgfpathcurveto{\pgfqpoint{3.343831in}{0.918530in}}{\pgfqpoint{3.344563in}{0.920297in}}{\pgfqpoint{3.344563in}{0.922138in}}%
\pgfpathcurveto{\pgfqpoint{3.344563in}{0.923980in}}{\pgfqpoint{3.343831in}{0.925746in}}{\pgfqpoint{3.342529in}{0.927049in}}%
\pgfpathcurveto{\pgfqpoint{3.341226in}{0.928351in}}{\pgfqpoint{3.339460in}{0.929083in}}{\pgfqpoint{3.337618in}{0.929083in}}%
\pgfpathcurveto{\pgfqpoint{3.335776in}{0.929083in}}{\pgfqpoint{3.334010in}{0.928351in}}{\pgfqpoint{3.332708in}{0.927049in}}%
\pgfpathcurveto{\pgfqpoint{3.331405in}{0.925746in}}{\pgfqpoint{3.330674in}{0.923980in}}{\pgfqpoint{3.330674in}{0.922138in}}%
\pgfpathcurveto{\pgfqpoint{3.330674in}{0.920297in}}{\pgfqpoint{3.331405in}{0.918530in}}{\pgfqpoint{3.332708in}{0.917228in}}%
\pgfpathcurveto{\pgfqpoint{3.334010in}{0.915925in}}{\pgfqpoint{3.335776in}{0.915194in}}{\pgfqpoint{3.337618in}{0.915194in}}%
\pgfpathlineto{\pgfqpoint{3.337618in}{0.915194in}}%
\pgfpathclose%
\pgfusepath{stroke,fill}%
\end{pgfscope}%
\begin{pgfscope}%
\pgfpathrectangle{\pgfqpoint{0.661006in}{0.524170in}}{\pgfqpoint{4.194036in}{1.071446in}}%
\pgfusepath{clip}%
\pgfsetbuttcap%
\pgfsetroundjoin%
\definecolor{currentfill}{rgb}{0.390133,0.416678,0.622659}%
\pgfsetfillcolor{currentfill}%
\pgfsetfillopacity{0.700000}%
\pgfsetlinewidth{1.003750pt}%
\definecolor{currentstroke}{rgb}{0.390133,0.416678,0.622659}%
\pgfsetstrokecolor{currentstroke}%
\pgfsetstrokeopacity{0.700000}%
\pgfsetdash{}{0pt}%
\pgfpathmoveto{\pgfqpoint{3.367178in}{0.908173in}}%
\pgfpathcurveto{\pgfqpoint{3.369019in}{0.908173in}}{\pgfqpoint{3.370786in}{0.908904in}}{\pgfqpoint{3.372088in}{0.910207in}}%
\pgfpathcurveto{\pgfqpoint{3.373390in}{0.911509in}}{\pgfqpoint{3.374122in}{0.913275in}}{\pgfqpoint{3.374122in}{0.915117in}}%
\pgfpathcurveto{\pgfqpoint{3.374122in}{0.916959in}}{\pgfqpoint{3.373390in}{0.918725in}}{\pgfqpoint{3.372088in}{0.920028in}}%
\pgfpathcurveto{\pgfqpoint{3.370786in}{0.921330in}}{\pgfqpoint{3.369019in}{0.922062in}}{\pgfqpoint{3.367178in}{0.922062in}}%
\pgfpathcurveto{\pgfqpoint{3.365336in}{0.922062in}}{\pgfqpoint{3.363570in}{0.921330in}}{\pgfqpoint{3.362267in}{0.920028in}}%
\pgfpathcurveto{\pgfqpoint{3.360965in}{0.918725in}}{\pgfqpoint{3.360233in}{0.916959in}}{\pgfqpoint{3.360233in}{0.915117in}}%
\pgfpathcurveto{\pgfqpoint{3.360233in}{0.913275in}}{\pgfqpoint{3.360965in}{0.911509in}}{\pgfqpoint{3.362267in}{0.910207in}}%
\pgfpathcurveto{\pgfqpoint{3.363570in}{0.908904in}}{\pgfqpoint{3.365336in}{0.908173in}}{\pgfqpoint{3.367178in}{0.908173in}}%
\pgfpathlineto{\pgfqpoint{3.367178in}{0.908173in}}%
\pgfpathclose%
\pgfusepath{stroke,fill}%
\end{pgfscope}%
\begin{pgfscope}%
\pgfpathrectangle{\pgfqpoint{0.661006in}{0.524170in}}{\pgfqpoint{4.194036in}{1.071446in}}%
\pgfusepath{clip}%
\pgfsetbuttcap%
\pgfsetroundjoin%
\definecolor{currentfill}{rgb}{0.390133,0.416678,0.622659}%
\pgfsetfillcolor{currentfill}%
\pgfsetfillopacity{0.700000}%
\pgfsetlinewidth{1.003750pt}%
\definecolor{currentstroke}{rgb}{0.390133,0.416678,0.622659}%
\pgfsetstrokecolor{currentstroke}%
\pgfsetstrokeopacity{0.700000}%
\pgfsetdash{}{0pt}%
\pgfpathmoveto{\pgfqpoint{3.388139in}{0.900960in}}%
\pgfpathcurveto{\pgfqpoint{3.389981in}{0.900960in}}{\pgfqpoint{3.391747in}{0.901692in}}{\pgfqpoint{3.393049in}{0.902994in}}%
\pgfpathcurveto{\pgfqpoint{3.394352in}{0.904296in}}{\pgfqpoint{3.395083in}{0.906063in}}{\pgfqpoint{3.395083in}{0.907904in}}%
\pgfpathcurveto{\pgfqpoint{3.395083in}{0.909746in}}{\pgfqpoint{3.394352in}{0.911512in}}{\pgfqpoint{3.393049in}{0.912815in}}%
\pgfpathcurveto{\pgfqpoint{3.391747in}{0.914117in}}{\pgfqpoint{3.389981in}{0.914849in}}{\pgfqpoint{3.388139in}{0.914849in}}%
\pgfpathcurveto{\pgfqpoint{3.386297in}{0.914849in}}{\pgfqpoint{3.384531in}{0.914117in}}{\pgfqpoint{3.383228in}{0.912815in}}%
\pgfpathcurveto{\pgfqpoint{3.381926in}{0.911512in}}{\pgfqpoint{3.381194in}{0.909746in}}{\pgfqpoint{3.381194in}{0.907904in}}%
\pgfpathcurveto{\pgfqpoint{3.381194in}{0.906063in}}{\pgfqpoint{3.381926in}{0.904296in}}{\pgfqpoint{3.383228in}{0.902994in}}%
\pgfpathcurveto{\pgfqpoint{3.384531in}{0.901692in}}{\pgfqpoint{3.386297in}{0.900960in}}{\pgfqpoint{3.388139in}{0.900960in}}%
\pgfpathlineto{\pgfqpoint{3.388139in}{0.900960in}}%
\pgfpathclose%
\pgfusepath{stroke,fill}%
\end{pgfscope}%
\begin{pgfscope}%
\pgfpathrectangle{\pgfqpoint{0.661006in}{0.524170in}}{\pgfqpoint{4.194036in}{1.071446in}}%
\pgfusepath{clip}%
\pgfsetbuttcap%
\pgfsetroundjoin%
\definecolor{currentfill}{rgb}{0.390133,0.416678,0.622659}%
\pgfsetfillcolor{currentfill}%
\pgfsetfillopacity{0.700000}%
\pgfsetlinewidth{1.003750pt}%
\definecolor{currentstroke}{rgb}{0.390133,0.416678,0.622659}%
\pgfsetstrokecolor{currentstroke}%
\pgfsetstrokeopacity{0.700000}%
\pgfsetdash{}{0pt}%
\pgfpathmoveto{\pgfqpoint{3.423229in}{0.894194in}}%
\pgfpathcurveto{\pgfqpoint{3.425071in}{0.894194in}}{\pgfqpoint{3.426837in}{0.894926in}}{\pgfqpoint{3.428140in}{0.896228in}}%
\pgfpathcurveto{\pgfqpoint{3.429442in}{0.897531in}}{\pgfqpoint{3.430174in}{0.899297in}}{\pgfqpoint{3.430174in}{0.901139in}}%
\pgfpathcurveto{\pgfqpoint{3.430174in}{0.902980in}}{\pgfqpoint{3.429442in}{0.904747in}}{\pgfqpoint{3.428140in}{0.906049in}}%
\pgfpathcurveto{\pgfqpoint{3.426837in}{0.907351in}}{\pgfqpoint{3.425071in}{0.908083in}}{\pgfqpoint{3.423229in}{0.908083in}}%
\pgfpathcurveto{\pgfqpoint{3.421388in}{0.908083in}}{\pgfqpoint{3.419621in}{0.907351in}}{\pgfqpoint{3.418319in}{0.906049in}}%
\pgfpathcurveto{\pgfqpoint{3.417017in}{0.904747in}}{\pgfqpoint{3.416285in}{0.902980in}}{\pgfqpoint{3.416285in}{0.901139in}}%
\pgfpathcurveto{\pgfqpoint{3.416285in}{0.899297in}}{\pgfqpoint{3.417017in}{0.897531in}}{\pgfqpoint{3.418319in}{0.896228in}}%
\pgfpathcurveto{\pgfqpoint{3.419621in}{0.894926in}}{\pgfqpoint{3.421388in}{0.894194in}}{\pgfqpoint{3.423229in}{0.894194in}}%
\pgfpathlineto{\pgfqpoint{3.423229in}{0.894194in}}%
\pgfpathclose%
\pgfusepath{stroke,fill}%
\end{pgfscope}%
\begin{pgfscope}%
\pgfpathrectangle{\pgfqpoint{0.661006in}{0.524170in}}{\pgfqpoint{4.194036in}{1.071446in}}%
\pgfusepath{clip}%
\pgfsetbuttcap%
\pgfsetroundjoin%
\definecolor{currentfill}{rgb}{0.387354,0.412223,0.618484}%
\pgfsetfillcolor{currentfill}%
\pgfsetfillopacity{0.700000}%
\pgfsetlinewidth{1.003750pt}%
\definecolor{currentstroke}{rgb}{0.387354,0.412223,0.618484}%
\pgfsetstrokecolor{currentstroke}%
\pgfsetstrokeopacity{0.700000}%
\pgfsetdash{}{0pt}%
\pgfpathmoveto{\pgfqpoint{3.448838in}{0.889189in}}%
\pgfpathcurveto{\pgfqpoint{3.450680in}{0.889189in}}{\pgfqpoint{3.452446in}{0.889920in}}{\pgfqpoint{3.453749in}{0.891223in}}%
\pgfpathcurveto{\pgfqpoint{3.455051in}{0.892525in}}{\pgfqpoint{3.455783in}{0.894292in}}{\pgfqpoint{3.455783in}{0.896133in}}%
\pgfpathcurveto{\pgfqpoint{3.455783in}{0.897975in}}{\pgfqpoint{3.455051in}{0.899741in}}{\pgfqpoint{3.453749in}{0.901044in}}%
\pgfpathcurveto{\pgfqpoint{3.452446in}{0.902346in}}{\pgfqpoint{3.450680in}{0.903078in}}{\pgfqpoint{3.448838in}{0.903078in}}%
\pgfpathcurveto{\pgfqpoint{3.446997in}{0.903078in}}{\pgfqpoint{3.445230in}{0.902346in}}{\pgfqpoint{3.443928in}{0.901044in}}%
\pgfpathcurveto{\pgfqpoint{3.442625in}{0.899741in}}{\pgfqpoint{3.441894in}{0.897975in}}{\pgfqpoint{3.441894in}{0.896133in}}%
\pgfpathcurveto{\pgfqpoint{3.441894in}{0.894292in}}{\pgfqpoint{3.442625in}{0.892525in}}{\pgfqpoint{3.443928in}{0.891223in}}%
\pgfpathcurveto{\pgfqpoint{3.445230in}{0.889920in}}{\pgfqpoint{3.446997in}{0.889189in}}{\pgfqpoint{3.448838in}{0.889189in}}%
\pgfpathlineto{\pgfqpoint{3.448838in}{0.889189in}}%
\pgfpathclose%
\pgfusepath{stroke,fill}%
\end{pgfscope}%
\begin{pgfscope}%
\pgfpathrectangle{\pgfqpoint{0.661006in}{0.524170in}}{\pgfqpoint{4.194036in}{1.071446in}}%
\pgfusepath{clip}%
\pgfsetbuttcap%
\pgfsetroundjoin%
\definecolor{currentfill}{rgb}{0.387354,0.412223,0.618484}%
\pgfsetfillcolor{currentfill}%
\pgfsetfillopacity{0.700000}%
\pgfsetlinewidth{1.003750pt}%
\definecolor{currentstroke}{rgb}{0.387354,0.412223,0.618484}%
\pgfsetstrokecolor{currentstroke}%
\pgfsetstrokeopacity{0.700000}%
\pgfsetdash{}{0pt}%
\pgfpathmoveto{\pgfqpoint{3.469335in}{0.885147in}}%
\pgfpathcurveto{\pgfqpoint{3.471176in}{0.885147in}}{\pgfqpoint{3.472943in}{0.885879in}}{\pgfqpoint{3.474245in}{0.887181in}}%
\pgfpathcurveto{\pgfqpoint{3.475547in}{0.888483in}}{\pgfqpoint{3.476279in}{0.890250in}}{\pgfqpoint{3.476279in}{0.892091in}}%
\pgfpathcurveto{\pgfqpoint{3.476279in}{0.893933in}}{\pgfqpoint{3.475547in}{0.895700in}}{\pgfqpoint{3.474245in}{0.897002in}}%
\pgfpathcurveto{\pgfqpoint{3.472943in}{0.898304in}}{\pgfqpoint{3.471176in}{0.899036in}}{\pgfqpoint{3.469335in}{0.899036in}}%
\pgfpathcurveto{\pgfqpoint{3.467493in}{0.899036in}}{\pgfqpoint{3.465726in}{0.898304in}}{\pgfqpoint{3.464424in}{0.897002in}}%
\pgfpathcurveto{\pgfqpoint{3.463122in}{0.895700in}}{\pgfqpoint{3.462390in}{0.893933in}}{\pgfqpoint{3.462390in}{0.892091in}}%
\pgfpathcurveto{\pgfqpoint{3.462390in}{0.890250in}}{\pgfqpoint{3.463122in}{0.888483in}}{\pgfqpoint{3.464424in}{0.887181in}}%
\pgfpathcurveto{\pgfqpoint{3.465726in}{0.885879in}}{\pgfqpoint{3.467493in}{0.885147in}}{\pgfqpoint{3.469335in}{0.885147in}}%
\pgfpathlineto{\pgfqpoint{3.469335in}{0.885147in}}%
\pgfpathclose%
\pgfusepath{stroke,fill}%
\end{pgfscope}%
\begin{pgfscope}%
\pgfpathrectangle{\pgfqpoint{0.661006in}{0.524170in}}{\pgfqpoint{4.194036in}{1.071446in}}%
\pgfusepath{clip}%
\pgfsetbuttcap%
\pgfsetroundjoin%
\definecolor{currentfill}{rgb}{0.387354,0.412223,0.618484}%
\pgfsetfillcolor{currentfill}%
\pgfsetfillopacity{0.700000}%
\pgfsetlinewidth{1.003750pt}%
\definecolor{currentstroke}{rgb}{0.387354,0.412223,0.618484}%
\pgfsetstrokecolor{currentstroke}%
\pgfsetstrokeopacity{0.700000}%
\pgfsetdash{}{0pt}%
\pgfpathmoveto{\pgfqpoint{3.476306in}{0.882482in}}%
\pgfpathcurveto{\pgfqpoint{3.478148in}{0.882482in}}{\pgfqpoint{3.479914in}{0.883214in}}{\pgfqpoint{3.481217in}{0.884516in}}%
\pgfpathcurveto{\pgfqpoint{3.482519in}{0.885818in}}{\pgfqpoint{3.483251in}{0.887585in}}{\pgfqpoint{3.483251in}{0.889426in}}%
\pgfpathcurveto{\pgfqpoint{3.483251in}{0.891268in}}{\pgfqpoint{3.482519in}{0.893035in}}{\pgfqpoint{3.481217in}{0.894337in}}%
\pgfpathcurveto{\pgfqpoint{3.479914in}{0.895639in}}{\pgfqpoint{3.478148in}{0.896371in}}{\pgfqpoint{3.476306in}{0.896371in}}%
\pgfpathcurveto{\pgfqpoint{3.474465in}{0.896371in}}{\pgfqpoint{3.472698in}{0.895639in}}{\pgfqpoint{3.471396in}{0.894337in}}%
\pgfpathcurveto{\pgfqpoint{3.470094in}{0.893035in}}{\pgfqpoint{3.469362in}{0.891268in}}{\pgfqpoint{3.469362in}{0.889426in}}%
\pgfpathcurveto{\pgfqpoint{3.469362in}{0.887585in}}{\pgfqpoint{3.470094in}{0.885818in}}{\pgfqpoint{3.471396in}{0.884516in}}%
\pgfpathcurveto{\pgfqpoint{3.472698in}{0.883214in}}{\pgfqpoint{3.474465in}{0.882482in}}{\pgfqpoint{3.476306in}{0.882482in}}%
\pgfpathlineto{\pgfqpoint{3.476306in}{0.882482in}}%
\pgfpathclose%
\pgfusepath{stroke,fill}%
\end{pgfscope}%
\begin{pgfscope}%
\pgfpathrectangle{\pgfqpoint{0.661006in}{0.524170in}}{\pgfqpoint{4.194036in}{1.071446in}}%
\pgfusepath{clip}%
\pgfsetbuttcap%
\pgfsetroundjoin%
\definecolor{currentfill}{rgb}{0.384576,0.407773,0.614278}%
\pgfsetfillcolor{currentfill}%
\pgfsetfillopacity{0.700000}%
\pgfsetlinewidth{1.003750pt}%
\definecolor{currentstroke}{rgb}{0.384576,0.407773,0.614278}%
\pgfsetstrokecolor{currentstroke}%
\pgfsetstrokeopacity{0.700000}%
\pgfsetdash{}{0pt}%
\pgfpathmoveto{\pgfqpoint{3.482767in}{0.882818in}}%
\pgfpathcurveto{\pgfqpoint{3.484608in}{0.882818in}}{\pgfqpoint{3.486375in}{0.883550in}}{\pgfqpoint{3.487677in}{0.884852in}}%
\pgfpathcurveto{\pgfqpoint{3.488979in}{0.886154in}}{\pgfqpoint{3.489711in}{0.887921in}}{\pgfqpoint{3.489711in}{0.889763in}}%
\pgfpathcurveto{\pgfqpoint{3.489711in}{0.891604in}}{\pgfqpoint{3.488979in}{0.893371in}}{\pgfqpoint{3.487677in}{0.894673in}}%
\pgfpathcurveto{\pgfqpoint{3.486375in}{0.895975in}}{\pgfqpoint{3.484608in}{0.896707in}}{\pgfqpoint{3.482767in}{0.896707in}}%
\pgfpathcurveto{\pgfqpoint{3.480925in}{0.896707in}}{\pgfqpoint{3.479158in}{0.895975in}}{\pgfqpoint{3.477856in}{0.894673in}}%
\pgfpathcurveto{\pgfqpoint{3.476554in}{0.893371in}}{\pgfqpoint{3.475822in}{0.891604in}}{\pgfqpoint{3.475822in}{0.889763in}}%
\pgfpathcurveto{\pgfqpoint{3.475822in}{0.887921in}}{\pgfqpoint{3.476554in}{0.886154in}}{\pgfqpoint{3.477856in}{0.884852in}}%
\pgfpathcurveto{\pgfqpoint{3.479158in}{0.883550in}}{\pgfqpoint{3.480925in}{0.882818in}}{\pgfqpoint{3.482767in}{0.882818in}}%
\pgfpathlineto{\pgfqpoint{3.482767in}{0.882818in}}%
\pgfpathclose%
\pgfusepath{stroke,fill}%
\end{pgfscope}%
\begin{pgfscope}%
\pgfpathrectangle{\pgfqpoint{0.661006in}{0.524170in}}{\pgfqpoint{4.194036in}{1.071446in}}%
\pgfusepath{clip}%
\pgfsetbuttcap%
\pgfsetroundjoin%
\definecolor{currentfill}{rgb}{0.384576,0.407773,0.614278}%
\pgfsetfillcolor{currentfill}%
\pgfsetfillopacity{0.700000}%
\pgfsetlinewidth{1.003750pt}%
\definecolor{currentstroke}{rgb}{0.384576,0.407773,0.614278}%
\pgfsetstrokecolor{currentstroke}%
\pgfsetstrokeopacity{0.700000}%
\pgfsetdash{}{0pt}%
\pgfpathmoveto{\pgfqpoint{3.468173in}{0.885062in}}%
\pgfpathcurveto{\pgfqpoint{3.470014in}{0.885062in}}{\pgfqpoint{3.471781in}{0.885794in}}{\pgfqpoint{3.473083in}{0.887096in}}%
\pgfpathcurveto{\pgfqpoint{3.474385in}{0.888398in}}{\pgfqpoint{3.475117in}{0.890165in}}{\pgfqpoint{3.475117in}{0.892006in}}%
\pgfpathcurveto{\pgfqpoint{3.475117in}{0.893848in}}{\pgfqpoint{3.474385in}{0.895615in}}{\pgfqpoint{3.473083in}{0.896917in}}%
\pgfpathcurveto{\pgfqpoint{3.471781in}{0.898219in}}{\pgfqpoint{3.470014in}{0.898951in}}{\pgfqpoint{3.468173in}{0.898951in}}%
\pgfpathcurveto{\pgfqpoint{3.466331in}{0.898951in}}{\pgfqpoint{3.464565in}{0.898219in}}{\pgfqpoint{3.463262in}{0.896917in}}%
\pgfpathcurveto{\pgfqpoint{3.461960in}{0.895615in}}{\pgfqpoint{3.461228in}{0.893848in}}{\pgfqpoint{3.461228in}{0.892006in}}%
\pgfpathcurveto{\pgfqpoint{3.461228in}{0.890165in}}{\pgfqpoint{3.461960in}{0.888398in}}{\pgfqpoint{3.463262in}{0.887096in}}%
\pgfpathcurveto{\pgfqpoint{3.464565in}{0.885794in}}{\pgfqpoint{3.466331in}{0.885062in}}{\pgfqpoint{3.468173in}{0.885062in}}%
\pgfpathlineto{\pgfqpoint{3.468173in}{0.885062in}}%
\pgfpathclose%
\pgfusepath{stroke,fill}%
\end{pgfscope}%
\begin{pgfscope}%
\pgfpathrectangle{\pgfqpoint{0.661006in}{0.524170in}}{\pgfqpoint{4.194036in}{1.071446in}}%
\pgfusepath{clip}%
\pgfsetbuttcap%
\pgfsetroundjoin%
\definecolor{currentfill}{rgb}{0.381800,0.403328,0.610039}%
\pgfsetfillcolor{currentfill}%
\pgfsetfillopacity{0.700000}%
\pgfsetlinewidth{1.003750pt}%
\definecolor{currentstroke}{rgb}{0.381800,0.403328,0.610039}%
\pgfsetstrokecolor{currentstroke}%
\pgfsetstrokeopacity{0.700000}%
\pgfsetdash{}{0pt}%
\pgfpathmoveto{\pgfqpoint{3.475888in}{0.882357in}}%
\pgfpathcurveto{\pgfqpoint{3.477730in}{0.882357in}}{\pgfqpoint{3.479496in}{0.883089in}}{\pgfqpoint{3.480798in}{0.884391in}}%
\pgfpathcurveto{\pgfqpoint{3.482101in}{0.885694in}}{\pgfqpoint{3.482832in}{0.887460in}}{\pgfqpoint{3.482832in}{0.889302in}}%
\pgfpathcurveto{\pgfqpoint{3.482832in}{0.891144in}}{\pgfqpoint{3.482101in}{0.892910in}}{\pgfqpoint{3.480798in}{0.894212in}}%
\pgfpathcurveto{\pgfqpoint{3.479496in}{0.895515in}}{\pgfqpoint{3.477730in}{0.896246in}}{\pgfqpoint{3.475888in}{0.896246in}}%
\pgfpathcurveto{\pgfqpoint{3.474046in}{0.896246in}}{\pgfqpoint{3.472280in}{0.895515in}}{\pgfqpoint{3.470977in}{0.894212in}}%
\pgfpathcurveto{\pgfqpoint{3.469675in}{0.892910in}}{\pgfqpoint{3.468944in}{0.891144in}}{\pgfqpoint{3.468944in}{0.889302in}}%
\pgfpathcurveto{\pgfqpoint{3.468944in}{0.887460in}}{\pgfqpoint{3.469675in}{0.885694in}}{\pgfqpoint{3.470977in}{0.884391in}}%
\pgfpathcurveto{\pgfqpoint{3.472280in}{0.883089in}}{\pgfqpoint{3.474046in}{0.882357in}}{\pgfqpoint{3.475888in}{0.882357in}}%
\pgfpathlineto{\pgfqpoint{3.475888in}{0.882357in}}%
\pgfpathclose%
\pgfusepath{stroke,fill}%
\end{pgfscope}%
\begin{pgfscope}%
\pgfpathrectangle{\pgfqpoint{0.661006in}{0.524170in}}{\pgfqpoint{4.194036in}{1.071446in}}%
\pgfusepath{clip}%
\pgfsetbuttcap%
\pgfsetroundjoin%
\definecolor{currentfill}{rgb}{0.381800,0.403328,0.610039}%
\pgfsetfillcolor{currentfill}%
\pgfsetfillopacity{0.700000}%
\pgfsetlinewidth{1.003750pt}%
\definecolor{currentstroke}{rgb}{0.381800,0.403328,0.610039}%
\pgfsetstrokecolor{currentstroke}%
\pgfsetstrokeopacity{0.700000}%
\pgfsetdash{}{0pt}%
\pgfpathmoveto{\pgfqpoint{3.491252in}{0.879407in}}%
\pgfpathcurveto{\pgfqpoint{3.493094in}{0.879407in}}{\pgfqpoint{3.494860in}{0.880139in}}{\pgfqpoint{3.496162in}{0.881441in}}%
\pgfpathcurveto{\pgfqpoint{3.497465in}{0.882743in}}{\pgfqpoint{3.498196in}{0.884510in}}{\pgfqpoint{3.498196in}{0.886352in}}%
\pgfpathcurveto{\pgfqpoint{3.498196in}{0.888193in}}{\pgfqpoint{3.497465in}{0.889960in}}{\pgfqpoint{3.496162in}{0.891262in}}%
\pgfpathcurveto{\pgfqpoint{3.494860in}{0.892564in}}{\pgfqpoint{3.493094in}{0.893296in}}{\pgfqpoint{3.491252in}{0.893296in}}%
\pgfpathcurveto{\pgfqpoint{3.489410in}{0.893296in}}{\pgfqpoint{3.487644in}{0.892564in}}{\pgfqpoint{3.486342in}{0.891262in}}%
\pgfpathcurveto{\pgfqpoint{3.485039in}{0.889960in}}{\pgfqpoint{3.484308in}{0.888193in}}{\pgfqpoint{3.484308in}{0.886352in}}%
\pgfpathcurveto{\pgfqpoint{3.484308in}{0.884510in}}{\pgfqpoint{3.485039in}{0.882743in}}{\pgfqpoint{3.486342in}{0.881441in}}%
\pgfpathcurveto{\pgfqpoint{3.487644in}{0.880139in}}{\pgfqpoint{3.489410in}{0.879407in}}{\pgfqpoint{3.491252in}{0.879407in}}%
\pgfpathlineto{\pgfqpoint{3.491252in}{0.879407in}}%
\pgfpathclose%
\pgfusepath{stroke,fill}%
\end{pgfscope}%
\begin{pgfscope}%
\pgfpathrectangle{\pgfqpoint{0.661006in}{0.524170in}}{\pgfqpoint{4.194036in}{1.071446in}}%
\pgfusepath{clip}%
\pgfsetbuttcap%
\pgfsetroundjoin%
\definecolor{currentfill}{rgb}{0.381800,0.403328,0.610039}%
\pgfsetfillcolor{currentfill}%
\pgfsetfillopacity{0.700000}%
\pgfsetlinewidth{1.003750pt}%
\definecolor{currentstroke}{rgb}{0.381800,0.403328,0.610039}%
\pgfsetstrokecolor{currentstroke}%
\pgfsetstrokeopacity{0.700000}%
\pgfsetdash{}{0pt}%
\pgfpathmoveto{\pgfqpoint{3.505540in}{0.875344in}}%
\pgfpathcurveto{\pgfqpoint{3.507382in}{0.875344in}}{\pgfqpoint{3.509149in}{0.876075in}}{\pgfqpoint{3.510451in}{0.877378in}}%
\pgfpathcurveto{\pgfqpoint{3.511753in}{0.878680in}}{\pgfqpoint{3.512485in}{0.880446in}}{\pgfqpoint{3.512485in}{0.882288in}}%
\pgfpathcurveto{\pgfqpoint{3.512485in}{0.884130in}}{\pgfqpoint{3.511753in}{0.885896in}}{\pgfqpoint{3.510451in}{0.887199in}}%
\pgfpathcurveto{\pgfqpoint{3.509149in}{0.888501in}}{\pgfqpoint{3.507382in}{0.889233in}}{\pgfqpoint{3.505540in}{0.889233in}}%
\pgfpathcurveto{\pgfqpoint{3.503699in}{0.889233in}}{\pgfqpoint{3.501932in}{0.888501in}}{\pgfqpoint{3.500630in}{0.887199in}}%
\pgfpathcurveto{\pgfqpoint{3.499328in}{0.885896in}}{\pgfqpoint{3.498596in}{0.884130in}}{\pgfqpoint{3.498596in}{0.882288in}}%
\pgfpathcurveto{\pgfqpoint{3.498596in}{0.880446in}}{\pgfqpoint{3.499328in}{0.878680in}}{\pgfqpoint{3.500630in}{0.877378in}}%
\pgfpathcurveto{\pgfqpoint{3.501932in}{0.876075in}}{\pgfqpoint{3.503699in}{0.875344in}}{\pgfqpoint{3.505540in}{0.875344in}}%
\pgfpathlineto{\pgfqpoint{3.505540in}{0.875344in}}%
\pgfpathclose%
\pgfusepath{stroke,fill}%
\end{pgfscope}%
\begin{pgfscope}%
\pgfpathrectangle{\pgfqpoint{0.661006in}{0.524170in}}{\pgfqpoint{4.194036in}{1.071446in}}%
\pgfusepath{clip}%
\pgfsetbuttcap%
\pgfsetroundjoin%
\definecolor{currentfill}{rgb}{0.381800,0.403328,0.610039}%
\pgfsetfillcolor{currentfill}%
\pgfsetfillopacity{0.700000}%
\pgfsetlinewidth{1.003750pt}%
\definecolor{currentstroke}{rgb}{0.381800,0.403328,0.610039}%
\pgfsetstrokecolor{currentstroke}%
\pgfsetstrokeopacity{0.700000}%
\pgfsetdash{}{0pt}%
\pgfpathmoveto{\pgfqpoint{3.517485in}{0.871968in}}%
\pgfpathcurveto{\pgfqpoint{3.519327in}{0.871968in}}{\pgfqpoint{3.521093in}{0.872700in}}{\pgfqpoint{3.522396in}{0.874002in}}%
\pgfpathcurveto{\pgfqpoint{3.523698in}{0.875304in}}{\pgfqpoint{3.524430in}{0.877071in}}{\pgfqpoint{3.524430in}{0.878913in}}%
\pgfpathcurveto{\pgfqpoint{3.524430in}{0.880754in}}{\pgfqpoint{3.523698in}{0.882521in}}{\pgfqpoint{3.522396in}{0.883823in}}%
\pgfpathcurveto{\pgfqpoint{3.521093in}{0.885125in}}{\pgfqpoint{3.519327in}{0.885857in}}{\pgfqpoint{3.517485in}{0.885857in}}%
\pgfpathcurveto{\pgfqpoint{3.515643in}{0.885857in}}{\pgfqpoint{3.513877in}{0.885125in}}{\pgfqpoint{3.512575in}{0.883823in}}%
\pgfpathcurveto{\pgfqpoint{3.511272in}{0.882521in}}{\pgfqpoint{3.510541in}{0.880754in}}{\pgfqpoint{3.510541in}{0.878913in}}%
\pgfpathcurveto{\pgfqpoint{3.510541in}{0.877071in}}{\pgfqpoint{3.511272in}{0.875304in}}{\pgfqpoint{3.512575in}{0.874002in}}%
\pgfpathcurveto{\pgfqpoint{3.513877in}{0.872700in}}{\pgfqpoint{3.515643in}{0.871968in}}{\pgfqpoint{3.517485in}{0.871968in}}%
\pgfpathlineto{\pgfqpoint{3.517485in}{0.871968in}}%
\pgfpathclose%
\pgfusepath{stroke,fill}%
\end{pgfscope}%
\begin{pgfscope}%
\pgfpathrectangle{\pgfqpoint{0.661006in}{0.524170in}}{\pgfqpoint{4.194036in}{1.071446in}}%
\pgfusepath{clip}%
\pgfsetbuttcap%
\pgfsetroundjoin%
\definecolor{currentfill}{rgb}{0.381800,0.403328,0.610039}%
\pgfsetfillcolor{currentfill}%
\pgfsetfillopacity{0.700000}%
\pgfsetlinewidth{1.003750pt}%
\definecolor{currentstroke}{rgb}{0.381800,0.403328,0.610039}%
\pgfsetstrokecolor{currentstroke}%
\pgfsetstrokeopacity{0.700000}%
\pgfsetdash{}{0pt}%
\pgfpathmoveto{\pgfqpoint{3.532358in}{0.868179in}}%
\pgfpathcurveto{\pgfqpoint{3.534199in}{0.868179in}}{\pgfqpoint{3.535966in}{0.868911in}}{\pgfqpoint{3.537268in}{0.870213in}}%
\pgfpathcurveto{\pgfqpoint{3.538571in}{0.871515in}}{\pgfqpoint{3.539302in}{0.873282in}}{\pgfqpoint{3.539302in}{0.875123in}}%
\pgfpathcurveto{\pgfqpoint{3.539302in}{0.876965in}}{\pgfqpoint{3.538571in}{0.878732in}}{\pgfqpoint{3.537268in}{0.880034in}}%
\pgfpathcurveto{\pgfqpoint{3.535966in}{0.881336in}}{\pgfqpoint{3.534199in}{0.882068in}}{\pgfqpoint{3.532358in}{0.882068in}}%
\pgfpathcurveto{\pgfqpoint{3.530516in}{0.882068in}}{\pgfqpoint{3.528750in}{0.881336in}}{\pgfqpoint{3.527447in}{0.880034in}}%
\pgfpathcurveto{\pgfqpoint{3.526145in}{0.878732in}}{\pgfqpoint{3.525413in}{0.876965in}}{\pgfqpoint{3.525413in}{0.875123in}}%
\pgfpathcurveto{\pgfqpoint{3.525413in}{0.873282in}}{\pgfqpoint{3.526145in}{0.871515in}}{\pgfqpoint{3.527447in}{0.870213in}}%
\pgfpathcurveto{\pgfqpoint{3.528750in}{0.868911in}}{\pgfqpoint{3.530516in}{0.868179in}}{\pgfqpoint{3.532358in}{0.868179in}}%
\pgfpathlineto{\pgfqpoint{3.532358in}{0.868179in}}%
\pgfpathclose%
\pgfusepath{stroke,fill}%
\end{pgfscope}%
\begin{pgfscope}%
\pgfpathrectangle{\pgfqpoint{0.661006in}{0.524170in}}{\pgfqpoint{4.194036in}{1.071446in}}%
\pgfusepath{clip}%
\pgfsetbuttcap%
\pgfsetroundjoin%
\definecolor{currentfill}{rgb}{0.379025,0.398889,0.605769}%
\pgfsetfillcolor{currentfill}%
\pgfsetfillopacity{0.700000}%
\pgfsetlinewidth{1.003750pt}%
\definecolor{currentstroke}{rgb}{0.379025,0.398889,0.605769}%
\pgfsetstrokecolor{currentstroke}%
\pgfsetstrokeopacity{0.700000}%
\pgfsetdash{}{0pt}%
\pgfpathmoveto{\pgfqpoint{3.546208in}{0.865701in}}%
\pgfpathcurveto{\pgfqpoint{3.548050in}{0.865701in}}{\pgfqpoint{3.549816in}{0.866432in}}{\pgfqpoint{3.551118in}{0.867735in}}%
\pgfpathcurveto{\pgfqpoint{3.552421in}{0.869037in}}{\pgfqpoint{3.553152in}{0.870803in}}{\pgfqpoint{3.553152in}{0.872645in}}%
\pgfpathcurveto{\pgfqpoint{3.553152in}{0.874487in}}{\pgfqpoint{3.552421in}{0.876253in}}{\pgfqpoint{3.551118in}{0.877556in}}%
\pgfpathcurveto{\pgfqpoint{3.549816in}{0.878858in}}{\pgfqpoint{3.548050in}{0.879590in}}{\pgfqpoint{3.546208in}{0.879590in}}%
\pgfpathcurveto{\pgfqpoint{3.544366in}{0.879590in}}{\pgfqpoint{3.542600in}{0.878858in}}{\pgfqpoint{3.541298in}{0.877556in}}%
\pgfpathcurveto{\pgfqpoint{3.539995in}{0.876253in}}{\pgfqpoint{3.539264in}{0.874487in}}{\pgfqpoint{3.539264in}{0.872645in}}%
\pgfpathcurveto{\pgfqpoint{3.539264in}{0.870803in}}{\pgfqpoint{3.539995in}{0.869037in}}{\pgfqpoint{3.541298in}{0.867735in}}%
\pgfpathcurveto{\pgfqpoint{3.542600in}{0.866432in}}{\pgfqpoint{3.544366in}{0.865701in}}{\pgfqpoint{3.546208in}{0.865701in}}%
\pgfpathlineto{\pgfqpoint{3.546208in}{0.865701in}}%
\pgfpathclose%
\pgfusepath{stroke,fill}%
\end{pgfscope}%
\begin{pgfscope}%
\pgfpathrectangle{\pgfqpoint{0.661006in}{0.524170in}}{\pgfqpoint{4.194036in}{1.071446in}}%
\pgfusepath{clip}%
\pgfsetbuttcap%
\pgfsetroundjoin%
\definecolor{currentfill}{rgb}{0.379025,0.398889,0.605769}%
\pgfsetfillcolor{currentfill}%
\pgfsetfillopacity{0.700000}%
\pgfsetlinewidth{1.003750pt}%
\definecolor{currentstroke}{rgb}{0.379025,0.398889,0.605769}%
\pgfsetstrokecolor{currentstroke}%
\pgfsetstrokeopacity{0.700000}%
\pgfsetdash{}{0pt}%
\pgfpathmoveto{\pgfqpoint{3.554295in}{0.863005in}}%
\pgfpathcurveto{\pgfqpoint{3.556137in}{0.863005in}}{\pgfqpoint{3.557903in}{0.863737in}}{\pgfqpoint{3.559206in}{0.865039in}}%
\pgfpathcurveto{\pgfqpoint{3.560508in}{0.866341in}}{\pgfqpoint{3.561239in}{0.868108in}}{\pgfqpoint{3.561239in}{0.869949in}}%
\pgfpathcurveto{\pgfqpoint{3.561239in}{0.871791in}}{\pgfqpoint{3.560508in}{0.873557in}}{\pgfqpoint{3.559206in}{0.874860in}}%
\pgfpathcurveto{\pgfqpoint{3.557903in}{0.876162in}}{\pgfqpoint{3.556137in}{0.876894in}}{\pgfqpoint{3.554295in}{0.876894in}}%
\pgfpathcurveto{\pgfqpoint{3.552453in}{0.876894in}}{\pgfqpoint{3.550687in}{0.876162in}}{\pgfqpoint{3.549385in}{0.874860in}}%
\pgfpathcurveto{\pgfqpoint{3.548082in}{0.873557in}}{\pgfqpoint{3.547351in}{0.871791in}}{\pgfqpoint{3.547351in}{0.869949in}}%
\pgfpathcurveto{\pgfqpoint{3.547351in}{0.868108in}}{\pgfqpoint{3.548082in}{0.866341in}}{\pgfqpoint{3.549385in}{0.865039in}}%
\pgfpathcurveto{\pgfqpoint{3.550687in}{0.863737in}}{\pgfqpoint{3.552453in}{0.863005in}}{\pgfqpoint{3.554295in}{0.863005in}}%
\pgfpathlineto{\pgfqpoint{3.554295in}{0.863005in}}%
\pgfpathclose%
\pgfusepath{stroke,fill}%
\end{pgfscope}%
\begin{pgfscope}%
\pgfpathrectangle{\pgfqpoint{0.661006in}{0.524170in}}{\pgfqpoint{4.194036in}{1.071446in}}%
\pgfusepath{clip}%
\pgfsetbuttcap%
\pgfsetroundjoin%
\definecolor{currentfill}{rgb}{0.376250,0.394455,0.601466}%
\pgfsetfillcolor{currentfill}%
\pgfsetfillopacity{0.700000}%
\pgfsetlinewidth{1.003750pt}%
\definecolor{currentstroke}{rgb}{0.376250,0.394455,0.601466}%
\pgfsetstrokecolor{currentstroke}%
\pgfsetstrokeopacity{0.700000}%
\pgfsetdash{}{0pt}%
\pgfpathmoveto{\pgfqpoint{3.567309in}{0.859328in}}%
\pgfpathcurveto{\pgfqpoint{3.569150in}{0.859328in}}{\pgfqpoint{3.570917in}{0.860060in}}{\pgfqpoint{3.572219in}{0.861362in}}%
\pgfpathcurveto{\pgfqpoint{3.573521in}{0.862665in}}{\pgfqpoint{3.574253in}{0.864431in}}{\pgfqpoint{3.574253in}{0.866273in}}%
\pgfpathcurveto{\pgfqpoint{3.574253in}{0.868114in}}{\pgfqpoint{3.573521in}{0.869881in}}{\pgfqpoint{3.572219in}{0.871183in}}%
\pgfpathcurveto{\pgfqpoint{3.570917in}{0.872485in}}{\pgfqpoint{3.569150in}{0.873217in}}{\pgfqpoint{3.567309in}{0.873217in}}%
\pgfpathcurveto{\pgfqpoint{3.565467in}{0.873217in}}{\pgfqpoint{3.563700in}{0.872485in}}{\pgfqpoint{3.562398in}{0.871183in}}%
\pgfpathcurveto{\pgfqpoint{3.561096in}{0.869881in}}{\pgfqpoint{3.560364in}{0.868114in}}{\pgfqpoint{3.560364in}{0.866273in}}%
\pgfpathcurveto{\pgfqpoint{3.560364in}{0.864431in}}{\pgfqpoint{3.561096in}{0.862665in}}{\pgfqpoint{3.562398in}{0.861362in}}%
\pgfpathcurveto{\pgfqpoint{3.563700in}{0.860060in}}{\pgfqpoint{3.565467in}{0.859328in}}{\pgfqpoint{3.567309in}{0.859328in}}%
\pgfpathlineto{\pgfqpoint{3.567309in}{0.859328in}}%
\pgfpathclose%
\pgfusepath{stroke,fill}%
\end{pgfscope}%
\begin{pgfscope}%
\pgfpathrectangle{\pgfqpoint{0.661006in}{0.524170in}}{\pgfqpoint{4.194036in}{1.071446in}}%
\pgfusepath{clip}%
\pgfsetbuttcap%
\pgfsetroundjoin%
\definecolor{currentfill}{rgb}{0.376250,0.394455,0.601466}%
\pgfsetfillcolor{currentfill}%
\pgfsetfillopacity{0.700000}%
\pgfsetlinewidth{1.003750pt}%
\definecolor{currentstroke}{rgb}{0.376250,0.394455,0.601466}%
\pgfsetstrokecolor{currentstroke}%
\pgfsetstrokeopacity{0.700000}%
\pgfsetdash{}{0pt}%
\pgfpathmoveto{\pgfqpoint{3.580880in}{0.857638in}}%
\pgfpathcurveto{\pgfqpoint{3.582722in}{0.857638in}}{\pgfqpoint{3.584488in}{0.858370in}}{\pgfqpoint{3.585790in}{0.859672in}}%
\pgfpathcurveto{\pgfqpoint{3.587093in}{0.860974in}}{\pgfqpoint{3.587824in}{0.862741in}}{\pgfqpoint{3.587824in}{0.864583in}}%
\pgfpathcurveto{\pgfqpoint{3.587824in}{0.866424in}}{\pgfqpoint{3.587093in}{0.868191in}}{\pgfqpoint{3.585790in}{0.869493in}}%
\pgfpathcurveto{\pgfqpoint{3.584488in}{0.870795in}}{\pgfqpoint{3.582722in}{0.871527in}}{\pgfqpoint{3.580880in}{0.871527in}}%
\pgfpathcurveto{\pgfqpoint{3.579038in}{0.871527in}}{\pgfqpoint{3.577272in}{0.870795in}}{\pgfqpoint{3.575970in}{0.869493in}}%
\pgfpathcurveto{\pgfqpoint{3.574667in}{0.868191in}}{\pgfqpoint{3.573936in}{0.866424in}}{\pgfqpoint{3.573936in}{0.864583in}}%
\pgfpathcurveto{\pgfqpoint{3.573936in}{0.862741in}}{\pgfqpoint{3.574667in}{0.860974in}}{\pgfqpoint{3.575970in}{0.859672in}}%
\pgfpathcurveto{\pgfqpoint{3.577272in}{0.858370in}}{\pgfqpoint{3.579038in}{0.857638in}}{\pgfqpoint{3.580880in}{0.857638in}}%
\pgfpathlineto{\pgfqpoint{3.580880in}{0.857638in}}%
\pgfpathclose%
\pgfusepath{stroke,fill}%
\end{pgfscope}%
\begin{pgfscope}%
\pgfpathrectangle{\pgfqpoint{0.661006in}{0.524170in}}{\pgfqpoint{4.194036in}{1.071446in}}%
\pgfusepath{clip}%
\pgfsetbuttcap%
\pgfsetroundjoin%
\definecolor{currentfill}{rgb}{0.376250,0.394455,0.601466}%
\pgfsetfillcolor{currentfill}%
\pgfsetfillopacity{0.700000}%
\pgfsetlinewidth{1.003750pt}%
\definecolor{currentstroke}{rgb}{0.376250,0.394455,0.601466}%
\pgfsetstrokecolor{currentstroke}%
\pgfsetstrokeopacity{0.700000}%
\pgfsetdash{}{0pt}%
\pgfpathmoveto{\pgfqpoint{3.593429in}{0.855070in}}%
\pgfpathcurveto{\pgfqpoint{3.595271in}{0.855070in}}{\pgfqpoint{3.597037in}{0.855802in}}{\pgfqpoint{3.598339in}{0.857104in}}%
\pgfpathcurveto{\pgfqpoint{3.599642in}{0.858407in}}{\pgfqpoint{3.600373in}{0.860173in}}{\pgfqpoint{3.600373in}{0.862015in}}%
\pgfpathcurveto{\pgfqpoint{3.600373in}{0.863857in}}{\pgfqpoint{3.599642in}{0.865623in}}{\pgfqpoint{3.598339in}{0.866925in}}%
\pgfpathcurveto{\pgfqpoint{3.597037in}{0.868228in}}{\pgfqpoint{3.595271in}{0.868959in}}{\pgfqpoint{3.593429in}{0.868959in}}%
\pgfpathcurveto{\pgfqpoint{3.591587in}{0.868959in}}{\pgfqpoint{3.589821in}{0.868228in}}{\pgfqpoint{3.588518in}{0.866925in}}%
\pgfpathcurveto{\pgfqpoint{3.587216in}{0.865623in}}{\pgfqpoint{3.586484in}{0.863857in}}{\pgfqpoint{3.586484in}{0.862015in}}%
\pgfpathcurveto{\pgfqpoint{3.586484in}{0.860173in}}{\pgfqpoint{3.587216in}{0.858407in}}{\pgfqpoint{3.588518in}{0.857104in}}%
\pgfpathcurveto{\pgfqpoint{3.589821in}{0.855802in}}{\pgfqpoint{3.591587in}{0.855070in}}{\pgfqpoint{3.593429in}{0.855070in}}%
\pgfpathlineto{\pgfqpoint{3.593429in}{0.855070in}}%
\pgfpathclose%
\pgfusepath{stroke,fill}%
\end{pgfscope}%
\begin{pgfscope}%
\pgfpathrectangle{\pgfqpoint{0.661006in}{0.524170in}}{\pgfqpoint{4.194036in}{1.071446in}}%
\pgfusepath{clip}%
\pgfsetbuttcap%
\pgfsetroundjoin%
\definecolor{currentfill}{rgb}{0.376250,0.394455,0.601466}%
\pgfsetfillcolor{currentfill}%
\pgfsetfillopacity{0.700000}%
\pgfsetlinewidth{1.003750pt}%
\definecolor{currentstroke}{rgb}{0.376250,0.394455,0.601466}%
\pgfsetstrokecolor{currentstroke}%
\pgfsetstrokeopacity{0.700000}%
\pgfsetdash{}{0pt}%
\pgfpathmoveto{\pgfqpoint{3.595474in}{0.855652in}}%
\pgfpathcurveto{\pgfqpoint{3.597316in}{0.855652in}}{\pgfqpoint{3.599082in}{0.856384in}}{\pgfqpoint{3.600384in}{0.857686in}}%
\pgfpathcurveto{\pgfqpoint{3.601687in}{0.858988in}}{\pgfqpoint{3.602418in}{0.860755in}}{\pgfqpoint{3.602418in}{0.862596in}}%
\pgfpathcurveto{\pgfqpoint{3.602418in}{0.864438in}}{\pgfqpoint{3.601687in}{0.866205in}}{\pgfqpoint{3.600384in}{0.867507in}}%
\pgfpathcurveto{\pgfqpoint{3.599082in}{0.868809in}}{\pgfqpoint{3.597316in}{0.869541in}}{\pgfqpoint{3.595474in}{0.869541in}}%
\pgfpathcurveto{\pgfqpoint{3.593632in}{0.869541in}}{\pgfqpoint{3.591866in}{0.868809in}}{\pgfqpoint{3.590563in}{0.867507in}}%
\pgfpathcurveto{\pgfqpoint{3.589261in}{0.866205in}}{\pgfqpoint{3.588529in}{0.864438in}}{\pgfqpoint{3.588529in}{0.862596in}}%
\pgfpathcurveto{\pgfqpoint{3.588529in}{0.860755in}}{\pgfqpoint{3.589261in}{0.858988in}}{\pgfqpoint{3.590563in}{0.857686in}}%
\pgfpathcurveto{\pgfqpoint{3.591866in}{0.856384in}}{\pgfqpoint{3.593632in}{0.855652in}}{\pgfqpoint{3.595474in}{0.855652in}}%
\pgfpathlineto{\pgfqpoint{3.595474in}{0.855652in}}%
\pgfpathclose%
\pgfusepath{stroke,fill}%
\end{pgfscope}%
\begin{pgfscope}%
\pgfpathrectangle{\pgfqpoint{0.661006in}{0.524170in}}{\pgfqpoint{4.194036in}{1.071446in}}%
\pgfusepath{clip}%
\pgfsetbuttcap%
\pgfsetroundjoin%
\definecolor{currentfill}{rgb}{0.376250,0.394455,0.601466}%
\pgfsetfillcolor{currentfill}%
\pgfsetfillopacity{0.700000}%
\pgfsetlinewidth{1.003750pt}%
\definecolor{currentstroke}{rgb}{0.376250,0.394455,0.601466}%
\pgfsetstrokecolor{currentstroke}%
\pgfsetstrokeopacity{0.700000}%
\pgfsetdash{}{0pt}%
\pgfpathmoveto{\pgfqpoint{3.593847in}{0.854490in}}%
\pgfpathcurveto{\pgfqpoint{3.595689in}{0.854490in}}{\pgfqpoint{3.597455in}{0.855221in}}{\pgfqpoint{3.598758in}{0.856524in}}%
\pgfpathcurveto{\pgfqpoint{3.600060in}{0.857826in}}{\pgfqpoint{3.600792in}{0.859592in}}{\pgfqpoint{3.600792in}{0.861434in}}%
\pgfpathcurveto{\pgfqpoint{3.600792in}{0.863276in}}{\pgfqpoint{3.600060in}{0.865042in}}{\pgfqpoint{3.598758in}{0.866344in}}%
\pgfpathcurveto{\pgfqpoint{3.597455in}{0.867647in}}{\pgfqpoint{3.595689in}{0.868378in}}{\pgfqpoint{3.593847in}{0.868378in}}%
\pgfpathcurveto{\pgfqpoint{3.592005in}{0.868378in}}{\pgfqpoint{3.590239in}{0.867647in}}{\pgfqpoint{3.588937in}{0.866344in}}%
\pgfpathcurveto{\pgfqpoint{3.587634in}{0.865042in}}{\pgfqpoint{3.586903in}{0.863276in}}{\pgfqpoint{3.586903in}{0.861434in}}%
\pgfpathcurveto{\pgfqpoint{3.586903in}{0.859592in}}{\pgfqpoint{3.587634in}{0.857826in}}{\pgfqpoint{3.588937in}{0.856524in}}%
\pgfpathcurveto{\pgfqpoint{3.590239in}{0.855221in}}{\pgfqpoint{3.592005in}{0.854490in}}{\pgfqpoint{3.593847in}{0.854490in}}%
\pgfpathlineto{\pgfqpoint{3.593847in}{0.854490in}}%
\pgfpathclose%
\pgfusepath{stroke,fill}%
\end{pgfscope}%
\begin{pgfscope}%
\pgfpathrectangle{\pgfqpoint{0.661006in}{0.524170in}}{\pgfqpoint{4.194036in}{1.071446in}}%
\pgfusepath{clip}%
\pgfsetbuttcap%
\pgfsetroundjoin%
\definecolor{currentfill}{rgb}{0.373475,0.390028,0.597131}%
\pgfsetfillcolor{currentfill}%
\pgfsetfillopacity{0.700000}%
\pgfsetlinewidth{1.003750pt}%
\definecolor{currentstroke}{rgb}{0.373475,0.390028,0.597131}%
\pgfsetstrokecolor{currentstroke}%
\pgfsetstrokeopacity{0.700000}%
\pgfsetdash{}{0pt}%
\pgfpathmoveto{\pgfqpoint{3.600493in}{0.852335in}}%
\pgfpathcurveto{\pgfqpoint{3.602335in}{0.852335in}}{\pgfqpoint{3.604102in}{0.853067in}}{\pgfqpoint{3.605404in}{0.854369in}}%
\pgfpathcurveto{\pgfqpoint{3.606706in}{0.855672in}}{\pgfqpoint{3.607438in}{0.857438in}}{\pgfqpoint{3.607438in}{0.859280in}}%
\pgfpathcurveto{\pgfqpoint{3.607438in}{0.861122in}}{\pgfqpoint{3.606706in}{0.862888in}}{\pgfqpoint{3.605404in}{0.864190in}}%
\pgfpathcurveto{\pgfqpoint{3.604102in}{0.865493in}}{\pgfqpoint{3.602335in}{0.866224in}}{\pgfqpoint{3.600493in}{0.866224in}}%
\pgfpathcurveto{\pgfqpoint{3.598652in}{0.866224in}}{\pgfqpoint{3.596885in}{0.865493in}}{\pgfqpoint{3.595583in}{0.864190in}}%
\pgfpathcurveto{\pgfqpoint{3.594281in}{0.862888in}}{\pgfqpoint{3.593549in}{0.861122in}}{\pgfqpoint{3.593549in}{0.859280in}}%
\pgfpathcurveto{\pgfqpoint{3.593549in}{0.857438in}}{\pgfqpoint{3.594281in}{0.855672in}}{\pgfqpoint{3.595583in}{0.854369in}}%
\pgfpathcurveto{\pgfqpoint{3.596885in}{0.853067in}}{\pgfqpoint{3.598652in}{0.852335in}}{\pgfqpoint{3.600493in}{0.852335in}}%
\pgfpathlineto{\pgfqpoint{3.600493in}{0.852335in}}%
\pgfpathclose%
\pgfusepath{stroke,fill}%
\end{pgfscope}%
\begin{pgfscope}%
\pgfpathrectangle{\pgfqpoint{0.661006in}{0.524170in}}{\pgfqpoint{4.194036in}{1.071446in}}%
\pgfusepath{clip}%
\pgfsetbuttcap%
\pgfsetroundjoin%
\definecolor{currentfill}{rgb}{0.373475,0.390028,0.597131}%
\pgfsetfillcolor{currentfill}%
\pgfsetfillopacity{0.700000}%
\pgfsetlinewidth{1.003750pt}%
\definecolor{currentstroke}{rgb}{0.373475,0.390028,0.597131}%
\pgfsetstrokecolor{currentstroke}%
\pgfsetstrokeopacity{0.700000}%
\pgfsetdash{}{0pt}%
\pgfpathmoveto{\pgfqpoint{3.612763in}{0.849002in}}%
\pgfpathcurveto{\pgfqpoint{3.614605in}{0.849002in}}{\pgfqpoint{3.616372in}{0.849734in}}{\pgfqpoint{3.617674in}{0.851036in}}%
\pgfpathcurveto{\pgfqpoint{3.618976in}{0.852339in}}{\pgfqpoint{3.619708in}{0.854105in}}{\pgfqpoint{3.619708in}{0.855947in}}%
\pgfpathcurveto{\pgfqpoint{3.619708in}{0.857788in}}{\pgfqpoint{3.618976in}{0.859555in}}{\pgfqpoint{3.617674in}{0.860857in}}%
\pgfpathcurveto{\pgfqpoint{3.616372in}{0.862159in}}{\pgfqpoint{3.614605in}{0.862891in}}{\pgfqpoint{3.612763in}{0.862891in}}%
\pgfpathcurveto{\pgfqpoint{3.610922in}{0.862891in}}{\pgfqpoint{3.609155in}{0.862159in}}{\pgfqpoint{3.607853in}{0.860857in}}%
\pgfpathcurveto{\pgfqpoint{3.606551in}{0.859555in}}{\pgfqpoint{3.605819in}{0.857788in}}{\pgfqpoint{3.605819in}{0.855947in}}%
\pgfpathcurveto{\pgfqpoint{3.605819in}{0.854105in}}{\pgfqpoint{3.606551in}{0.852339in}}{\pgfqpoint{3.607853in}{0.851036in}}%
\pgfpathcurveto{\pgfqpoint{3.609155in}{0.849734in}}{\pgfqpoint{3.610922in}{0.849002in}}{\pgfqpoint{3.612763in}{0.849002in}}%
\pgfpathlineto{\pgfqpoint{3.612763in}{0.849002in}}%
\pgfpathclose%
\pgfusepath{stroke,fill}%
\end{pgfscope}%
\begin{pgfscope}%
\pgfpathrectangle{\pgfqpoint{0.661006in}{0.524170in}}{\pgfqpoint{4.194036in}{1.071446in}}%
\pgfusepath{clip}%
\pgfsetbuttcap%
\pgfsetroundjoin%
\definecolor{currentfill}{rgb}{0.370699,0.385607,0.592764}%
\pgfsetfillcolor{currentfill}%
\pgfsetfillopacity{0.700000}%
\pgfsetlinewidth{1.003750pt}%
\definecolor{currentstroke}{rgb}{0.370699,0.385607,0.592764}%
\pgfsetstrokecolor{currentstroke}%
\pgfsetstrokeopacity{0.700000}%
\pgfsetdash{}{0pt}%
\pgfpathmoveto{\pgfqpoint{3.624652in}{0.846925in}}%
\pgfpathcurveto{\pgfqpoint{3.626493in}{0.846925in}}{\pgfqpoint{3.628260in}{0.847657in}}{\pgfqpoint{3.629562in}{0.848959in}}%
\pgfpathcurveto{\pgfqpoint{3.630864in}{0.850262in}}{\pgfqpoint{3.631596in}{0.852028in}}{\pgfqpoint{3.631596in}{0.853870in}}%
\pgfpathcurveto{\pgfqpoint{3.631596in}{0.855711in}}{\pgfqpoint{3.630864in}{0.857478in}}{\pgfqpoint{3.629562in}{0.858780in}}%
\pgfpathcurveto{\pgfqpoint{3.628260in}{0.860083in}}{\pgfqpoint{3.626493in}{0.860814in}}{\pgfqpoint{3.624652in}{0.860814in}}%
\pgfpathcurveto{\pgfqpoint{3.622810in}{0.860814in}}{\pgfqpoint{3.621043in}{0.860083in}}{\pgfqpoint{3.619741in}{0.858780in}}%
\pgfpathcurveto{\pgfqpoint{3.618439in}{0.857478in}}{\pgfqpoint{3.617707in}{0.855711in}}{\pgfqpoint{3.617707in}{0.853870in}}%
\pgfpathcurveto{\pgfqpoint{3.617707in}{0.852028in}}{\pgfqpoint{3.618439in}{0.850262in}}{\pgfqpoint{3.619741in}{0.848959in}}%
\pgfpathcurveto{\pgfqpoint{3.621043in}{0.847657in}}{\pgfqpoint{3.622810in}{0.846925in}}{\pgfqpoint{3.624652in}{0.846925in}}%
\pgfpathlineto{\pgfqpoint{3.624652in}{0.846925in}}%
\pgfpathclose%
\pgfusepath{stroke,fill}%
\end{pgfscope}%
\begin{pgfscope}%
\pgfpathrectangle{\pgfqpoint{0.661006in}{0.524170in}}{\pgfqpoint{4.194036in}{1.071446in}}%
\pgfusepath{clip}%
\pgfsetbuttcap%
\pgfsetroundjoin%
\definecolor{currentfill}{rgb}{0.370699,0.385607,0.592764}%
\pgfsetfillcolor{currentfill}%
\pgfsetfillopacity{0.700000}%
\pgfsetlinewidth{1.003750pt}%
\definecolor{currentstroke}{rgb}{0.370699,0.385607,0.592764}%
\pgfsetstrokecolor{currentstroke}%
\pgfsetstrokeopacity{0.700000}%
\pgfsetdash{}{0pt}%
\pgfpathmoveto{\pgfqpoint{3.621919in}{0.847834in}}%
\pgfpathcurveto{\pgfqpoint{3.623761in}{0.847834in}}{\pgfqpoint{3.625528in}{0.848566in}}{\pgfqpoint{3.626830in}{0.849868in}}%
\pgfpathcurveto{\pgfqpoint{3.628132in}{0.851170in}}{\pgfqpoint{3.628864in}{0.852937in}}{\pgfqpoint{3.628864in}{0.854778in}}%
\pgfpathcurveto{\pgfqpoint{3.628864in}{0.856620in}}{\pgfqpoint{3.628132in}{0.858386in}}{\pgfqpoint{3.626830in}{0.859689in}}%
\pgfpathcurveto{\pgfqpoint{3.625528in}{0.860991in}}{\pgfqpoint{3.623761in}{0.861723in}}{\pgfqpoint{3.621919in}{0.861723in}}%
\pgfpathcurveto{\pgfqpoint{3.620078in}{0.861723in}}{\pgfqpoint{3.618311in}{0.860991in}}{\pgfqpoint{3.617009in}{0.859689in}}%
\pgfpathcurveto{\pgfqpoint{3.615707in}{0.858386in}}{\pgfqpoint{3.614975in}{0.856620in}}{\pgfqpoint{3.614975in}{0.854778in}}%
\pgfpathcurveto{\pgfqpoint{3.614975in}{0.852937in}}{\pgfqpoint{3.615707in}{0.851170in}}{\pgfqpoint{3.617009in}{0.849868in}}%
\pgfpathcurveto{\pgfqpoint{3.618311in}{0.848566in}}{\pgfqpoint{3.620078in}{0.847834in}}{\pgfqpoint{3.621919in}{0.847834in}}%
\pgfpathlineto{\pgfqpoint{3.621919in}{0.847834in}}%
\pgfpathclose%
\pgfusepath{stroke,fill}%
\end{pgfscope}%
\begin{pgfscope}%
\pgfpathrectangle{\pgfqpoint{0.661006in}{0.524170in}}{\pgfqpoint{4.194036in}{1.071446in}}%
\pgfusepath{clip}%
\pgfsetbuttcap%
\pgfsetroundjoin%
\definecolor{currentfill}{rgb}{0.370699,0.385607,0.592764}%
\pgfsetfillcolor{currentfill}%
\pgfsetfillopacity{0.700000}%
\pgfsetlinewidth{1.003750pt}%
\definecolor{currentstroke}{rgb}{0.370699,0.385607,0.592764}%
\pgfsetstrokecolor{currentstroke}%
\pgfsetstrokeopacity{0.700000}%
\pgfsetdash{}{0pt}%
\pgfpathmoveto{\pgfqpoint{3.616342in}{0.845976in}}%
\pgfpathcurveto{\pgfqpoint{3.618184in}{0.845976in}}{\pgfqpoint{3.619950in}{0.846708in}}{\pgfqpoint{3.621253in}{0.848010in}}%
\pgfpathcurveto{\pgfqpoint{3.622555in}{0.849312in}}{\pgfqpoint{3.623287in}{0.851079in}}{\pgfqpoint{3.623287in}{0.852921in}}%
\pgfpathcurveto{\pgfqpoint{3.623287in}{0.854762in}}{\pgfqpoint{3.622555in}{0.856529in}}{\pgfqpoint{3.621253in}{0.857831in}}%
\pgfpathcurveto{\pgfqpoint{3.619950in}{0.859133in}}{\pgfqpoint{3.618184in}{0.859865in}}{\pgfqpoint{3.616342in}{0.859865in}}%
\pgfpathcurveto{\pgfqpoint{3.614500in}{0.859865in}}{\pgfqpoint{3.612734in}{0.859133in}}{\pgfqpoint{3.611432in}{0.857831in}}%
\pgfpathcurveto{\pgfqpoint{3.610129in}{0.856529in}}{\pgfqpoint{3.609398in}{0.854762in}}{\pgfqpoint{3.609398in}{0.852921in}}%
\pgfpathcurveto{\pgfqpoint{3.609398in}{0.851079in}}{\pgfqpoint{3.610129in}{0.849312in}}{\pgfqpoint{3.611432in}{0.848010in}}%
\pgfpathcurveto{\pgfqpoint{3.612734in}{0.846708in}}{\pgfqpoint{3.614500in}{0.845976in}}{\pgfqpoint{3.616342in}{0.845976in}}%
\pgfpathlineto{\pgfqpoint{3.616342in}{0.845976in}}%
\pgfpathclose%
\pgfusepath{stroke,fill}%
\end{pgfscope}%
\begin{pgfscope}%
\pgfpathrectangle{\pgfqpoint{0.661006in}{0.524170in}}{\pgfqpoint{4.194036in}{1.071446in}}%
\pgfusepath{clip}%
\pgfsetbuttcap%
\pgfsetroundjoin%
\definecolor{currentfill}{rgb}{0.370699,0.385607,0.592764}%
\pgfsetfillcolor{currentfill}%
\pgfsetfillopacity{0.700000}%
\pgfsetlinewidth{1.003750pt}%
\definecolor{currentstroke}{rgb}{0.370699,0.385607,0.592764}%
\pgfsetstrokecolor{currentstroke}%
\pgfsetstrokeopacity{0.700000}%
\pgfsetdash{}{0pt}%
\pgfpathmoveto{\pgfqpoint{3.617551in}{0.846467in}}%
\pgfpathcurveto{\pgfqpoint{3.619392in}{0.846467in}}{\pgfqpoint{3.621159in}{0.847199in}}{\pgfqpoint{3.622461in}{0.848501in}}%
\pgfpathcurveto{\pgfqpoint{3.623763in}{0.849803in}}{\pgfqpoint{3.624495in}{0.851570in}}{\pgfqpoint{3.624495in}{0.853411in}}%
\pgfpathcurveto{\pgfqpoint{3.624495in}{0.855253in}}{\pgfqpoint{3.623763in}{0.857020in}}{\pgfqpoint{3.622461in}{0.858322in}}%
\pgfpathcurveto{\pgfqpoint{3.621159in}{0.859624in}}{\pgfqpoint{3.619392in}{0.860356in}}{\pgfqpoint{3.617551in}{0.860356in}}%
\pgfpathcurveto{\pgfqpoint{3.615709in}{0.860356in}}{\pgfqpoint{3.613942in}{0.859624in}}{\pgfqpoint{3.612640in}{0.858322in}}%
\pgfpathcurveto{\pgfqpoint{3.611338in}{0.857020in}}{\pgfqpoint{3.610606in}{0.855253in}}{\pgfqpoint{3.610606in}{0.853411in}}%
\pgfpathcurveto{\pgfqpoint{3.610606in}{0.851570in}}{\pgfqpoint{3.611338in}{0.849803in}}{\pgfqpoint{3.612640in}{0.848501in}}%
\pgfpathcurveto{\pgfqpoint{3.613942in}{0.847199in}}{\pgfqpoint{3.615709in}{0.846467in}}{\pgfqpoint{3.617551in}{0.846467in}}%
\pgfpathlineto{\pgfqpoint{3.617551in}{0.846467in}}%
\pgfpathclose%
\pgfusepath{stroke,fill}%
\end{pgfscope}%
\begin{pgfscope}%
\pgfpathrectangle{\pgfqpoint{0.661006in}{0.524170in}}{\pgfqpoint{4.194036in}{1.071446in}}%
\pgfusepath{clip}%
\pgfsetbuttcap%
\pgfsetroundjoin%
\definecolor{currentfill}{rgb}{0.370699,0.385607,0.592764}%
\pgfsetfillcolor{currentfill}%
\pgfsetfillopacity{0.700000}%
\pgfsetlinewidth{1.003750pt}%
\definecolor{currentstroke}{rgb}{0.370699,0.385607,0.592764}%
\pgfsetstrokecolor{currentstroke}%
\pgfsetstrokeopacity{0.700000}%
\pgfsetdash{}{0pt}%
\pgfpathmoveto{\pgfqpoint{3.627218in}{0.845501in}}%
\pgfpathcurveto{\pgfqpoint{3.629060in}{0.845501in}}{\pgfqpoint{3.630826in}{0.846233in}}{\pgfqpoint{3.632128in}{0.847535in}}%
\pgfpathcurveto{\pgfqpoint{3.633431in}{0.848838in}}{\pgfqpoint{3.634162in}{0.850604in}}{\pgfqpoint{3.634162in}{0.852446in}}%
\pgfpathcurveto{\pgfqpoint{3.634162in}{0.854288in}}{\pgfqpoint{3.633431in}{0.856054in}}{\pgfqpoint{3.632128in}{0.857356in}}%
\pgfpathcurveto{\pgfqpoint{3.630826in}{0.858659in}}{\pgfqpoint{3.629060in}{0.859390in}}{\pgfqpoint{3.627218in}{0.859390in}}%
\pgfpathcurveto{\pgfqpoint{3.625376in}{0.859390in}}{\pgfqpoint{3.623610in}{0.858659in}}{\pgfqpoint{3.622307in}{0.857356in}}%
\pgfpathcurveto{\pgfqpoint{3.621005in}{0.856054in}}{\pgfqpoint{3.620273in}{0.854288in}}{\pgfqpoint{3.620273in}{0.852446in}}%
\pgfpathcurveto{\pgfqpoint{3.620273in}{0.850604in}}{\pgfqpoint{3.621005in}{0.848838in}}{\pgfqpoint{3.622307in}{0.847535in}}%
\pgfpathcurveto{\pgfqpoint{3.623610in}{0.846233in}}{\pgfqpoint{3.625376in}{0.845501in}}{\pgfqpoint{3.627218in}{0.845501in}}%
\pgfpathlineto{\pgfqpoint{3.627218in}{0.845501in}}%
\pgfpathclose%
\pgfusepath{stroke,fill}%
\end{pgfscope}%
\begin{pgfscope}%
\pgfpathrectangle{\pgfqpoint{0.661006in}{0.524170in}}{\pgfqpoint{4.194036in}{1.071446in}}%
\pgfusepath{clip}%
\pgfsetbuttcap%
\pgfsetroundjoin%
\definecolor{currentfill}{rgb}{0.367922,0.381192,0.588365}%
\pgfsetfillcolor{currentfill}%
\pgfsetfillopacity{0.700000}%
\pgfsetlinewidth{1.003750pt}%
\definecolor{currentstroke}{rgb}{0.367922,0.381192,0.588365}%
\pgfsetstrokecolor{currentstroke}%
\pgfsetstrokeopacity{0.700000}%
\pgfsetdash{}{0pt}%
\pgfpathmoveto{\pgfqpoint{3.628519in}{0.845340in}}%
\pgfpathcurveto{\pgfqpoint{3.630361in}{0.845340in}}{\pgfqpoint{3.632127in}{0.846072in}}{\pgfqpoint{3.633430in}{0.847374in}}%
\pgfpathcurveto{\pgfqpoint{3.634732in}{0.848676in}}{\pgfqpoint{3.635464in}{0.850443in}}{\pgfqpoint{3.635464in}{0.852284in}}%
\pgfpathcurveto{\pgfqpoint{3.635464in}{0.854126in}}{\pgfqpoint{3.634732in}{0.855892in}}{\pgfqpoint{3.633430in}{0.857195in}}%
\pgfpathcurveto{\pgfqpoint{3.632127in}{0.858497in}}{\pgfqpoint{3.630361in}{0.859229in}}{\pgfqpoint{3.628519in}{0.859229in}}%
\pgfpathcurveto{\pgfqpoint{3.626678in}{0.859229in}}{\pgfqpoint{3.624911in}{0.858497in}}{\pgfqpoint{3.623609in}{0.857195in}}%
\pgfpathcurveto{\pgfqpoint{3.622306in}{0.855892in}}{\pgfqpoint{3.621575in}{0.854126in}}{\pgfqpoint{3.621575in}{0.852284in}}%
\pgfpathcurveto{\pgfqpoint{3.621575in}{0.850443in}}{\pgfqpoint{3.622306in}{0.848676in}}{\pgfqpoint{3.623609in}{0.847374in}}%
\pgfpathcurveto{\pgfqpoint{3.624911in}{0.846072in}}{\pgfqpoint{3.626678in}{0.845340in}}{\pgfqpoint{3.628519in}{0.845340in}}%
\pgfpathlineto{\pgfqpoint{3.628519in}{0.845340in}}%
\pgfpathclose%
\pgfusepath{stroke,fill}%
\end{pgfscope}%
\begin{pgfscope}%
\pgfpathrectangle{\pgfqpoint{0.661006in}{0.524170in}}{\pgfqpoint{4.194036in}{1.071446in}}%
\pgfusepath{clip}%
\pgfsetbuttcap%
\pgfsetroundjoin%
\definecolor{currentfill}{rgb}{0.367922,0.381192,0.588365}%
\pgfsetfillcolor{currentfill}%
\pgfsetfillopacity{0.700000}%
\pgfsetlinewidth{1.003750pt}%
\definecolor{currentstroke}{rgb}{0.367922,0.381192,0.588365}%
\pgfsetstrokecolor{currentstroke}%
\pgfsetstrokeopacity{0.700000}%
\pgfsetdash{}{0pt}%
\pgfpathmoveto{\pgfqpoint{3.623314in}{0.844700in}}%
\pgfpathcurveto{\pgfqpoint{3.625155in}{0.844700in}}{\pgfqpoint{3.626922in}{0.845432in}}{\pgfqpoint{3.628224in}{0.846734in}}%
\pgfpathcurveto{\pgfqpoint{3.629526in}{0.848036in}}{\pgfqpoint{3.630258in}{0.849803in}}{\pgfqpoint{3.630258in}{0.851644in}}%
\pgfpathcurveto{\pgfqpoint{3.630258in}{0.853486in}}{\pgfqpoint{3.629526in}{0.855253in}}{\pgfqpoint{3.628224in}{0.856555in}}%
\pgfpathcurveto{\pgfqpoint{3.626922in}{0.857857in}}{\pgfqpoint{3.625155in}{0.858589in}}{\pgfqpoint{3.623314in}{0.858589in}}%
\pgfpathcurveto{\pgfqpoint{3.621472in}{0.858589in}}{\pgfqpoint{3.619706in}{0.857857in}}{\pgfqpoint{3.618403in}{0.856555in}}%
\pgfpathcurveto{\pgfqpoint{3.617101in}{0.855253in}}{\pgfqpoint{3.616369in}{0.853486in}}{\pgfqpoint{3.616369in}{0.851644in}}%
\pgfpathcurveto{\pgfqpoint{3.616369in}{0.849803in}}{\pgfqpoint{3.617101in}{0.848036in}}{\pgfqpoint{3.618403in}{0.846734in}}%
\pgfpathcurveto{\pgfqpoint{3.619706in}{0.845432in}}{\pgfqpoint{3.621472in}{0.844700in}}{\pgfqpoint{3.623314in}{0.844700in}}%
\pgfpathlineto{\pgfqpoint{3.623314in}{0.844700in}}%
\pgfpathclose%
\pgfusepath{stroke,fill}%
\end{pgfscope}%
\begin{pgfscope}%
\pgfpathrectangle{\pgfqpoint{0.661006in}{0.524170in}}{\pgfqpoint{4.194036in}{1.071446in}}%
\pgfusepath{clip}%
\pgfsetbuttcap%
\pgfsetroundjoin%
\definecolor{currentfill}{rgb}{0.365144,0.376784,0.583933}%
\pgfsetfillcolor{currentfill}%
\pgfsetfillopacity{0.700000}%
\pgfsetlinewidth{1.003750pt}%
\definecolor{currentstroke}{rgb}{0.365144,0.376784,0.583933}%
\pgfsetstrokecolor{currentstroke}%
\pgfsetstrokeopacity{0.700000}%
\pgfsetdash{}{0pt}%
\pgfpathmoveto{\pgfqpoint{3.629635in}{0.842438in}}%
\pgfpathcurveto{\pgfqpoint{3.631476in}{0.842438in}}{\pgfqpoint{3.633243in}{0.843169in}}{\pgfqpoint{3.634545in}{0.844472in}}%
\pgfpathcurveto{\pgfqpoint{3.635847in}{0.845774in}}{\pgfqpoint{3.636579in}{0.847541in}}{\pgfqpoint{3.636579in}{0.849382in}}%
\pgfpathcurveto{\pgfqpoint{3.636579in}{0.851224in}}{\pgfqpoint{3.635847in}{0.852990in}}{\pgfqpoint{3.634545in}{0.854293in}}%
\pgfpathcurveto{\pgfqpoint{3.633243in}{0.855595in}}{\pgfqpoint{3.631476in}{0.856327in}}{\pgfqpoint{3.629635in}{0.856327in}}%
\pgfpathcurveto{\pgfqpoint{3.627793in}{0.856327in}}{\pgfqpoint{3.626026in}{0.855595in}}{\pgfqpoint{3.624724in}{0.854293in}}%
\pgfpathcurveto{\pgfqpoint{3.623422in}{0.852990in}}{\pgfqpoint{3.622690in}{0.851224in}}{\pgfqpoint{3.622690in}{0.849382in}}%
\pgfpathcurveto{\pgfqpoint{3.622690in}{0.847541in}}{\pgfqpoint{3.623422in}{0.845774in}}{\pgfqpoint{3.624724in}{0.844472in}}%
\pgfpathcurveto{\pgfqpoint{3.626026in}{0.843169in}}{\pgfqpoint{3.627793in}{0.842438in}}{\pgfqpoint{3.629635in}{0.842438in}}%
\pgfpathlineto{\pgfqpoint{3.629635in}{0.842438in}}%
\pgfpathclose%
\pgfusepath{stroke,fill}%
\end{pgfscope}%
\begin{pgfscope}%
\pgfpathrectangle{\pgfqpoint{0.661006in}{0.524170in}}{\pgfqpoint{4.194036in}{1.071446in}}%
\pgfusepath{clip}%
\pgfsetbuttcap%
\pgfsetroundjoin%
\definecolor{currentfill}{rgb}{0.365144,0.376784,0.583933}%
\pgfsetfillcolor{currentfill}%
\pgfsetfillopacity{0.700000}%
\pgfsetlinewidth{1.003750pt}%
\definecolor{currentstroke}{rgb}{0.365144,0.376784,0.583933}%
\pgfsetstrokecolor{currentstroke}%
\pgfsetstrokeopacity{0.700000}%
\pgfsetdash{}{0pt}%
\pgfpathmoveto{\pgfqpoint{3.627869in}{0.841831in}}%
\pgfpathcurveto{\pgfqpoint{3.629710in}{0.841831in}}{\pgfqpoint{3.631477in}{0.842562in}}{\pgfqpoint{3.632779in}{0.843865in}}%
\pgfpathcurveto{\pgfqpoint{3.634081in}{0.845167in}}{\pgfqpoint{3.634813in}{0.846933in}}{\pgfqpoint{3.634813in}{0.848775in}}%
\pgfpathcurveto{\pgfqpoint{3.634813in}{0.850617in}}{\pgfqpoint{3.634081in}{0.852383in}}{\pgfqpoint{3.632779in}{0.853686in}}%
\pgfpathcurveto{\pgfqpoint{3.631477in}{0.854988in}}{\pgfqpoint{3.629710in}{0.855720in}}{\pgfqpoint{3.627869in}{0.855720in}}%
\pgfpathcurveto{\pgfqpoint{3.626027in}{0.855720in}}{\pgfqpoint{3.624260in}{0.854988in}}{\pgfqpoint{3.622958in}{0.853686in}}%
\pgfpathcurveto{\pgfqpoint{3.621656in}{0.852383in}}{\pgfqpoint{3.620924in}{0.850617in}}{\pgfqpoint{3.620924in}{0.848775in}}%
\pgfpathcurveto{\pgfqpoint{3.620924in}{0.846933in}}{\pgfqpoint{3.621656in}{0.845167in}}{\pgfqpoint{3.622958in}{0.843865in}}%
\pgfpathcurveto{\pgfqpoint{3.624260in}{0.842562in}}{\pgfqpoint{3.626027in}{0.841831in}}{\pgfqpoint{3.627869in}{0.841831in}}%
\pgfpathlineto{\pgfqpoint{3.627869in}{0.841831in}}%
\pgfpathclose%
\pgfusepath{stroke,fill}%
\end{pgfscope}%
\begin{pgfscope}%
\pgfpathrectangle{\pgfqpoint{0.661006in}{0.524170in}}{\pgfqpoint{4.194036in}{1.071446in}}%
\pgfusepath{clip}%
\pgfsetbuttcap%
\pgfsetroundjoin%
\definecolor{currentfill}{rgb}{0.365144,0.376784,0.583933}%
\pgfsetfillcolor{currentfill}%
\pgfsetfillopacity{0.700000}%
\pgfsetlinewidth{1.003750pt}%
\definecolor{currentstroke}{rgb}{0.365144,0.376784,0.583933}%
\pgfsetstrokecolor{currentstroke}%
\pgfsetstrokeopacity{0.700000}%
\pgfsetdash{}{0pt}%
\pgfpathmoveto{\pgfqpoint{3.643113in}{0.838070in}}%
\pgfpathcurveto{\pgfqpoint{3.644955in}{0.838070in}}{\pgfqpoint{3.646721in}{0.838802in}}{\pgfqpoint{3.648024in}{0.840104in}}%
\pgfpathcurveto{\pgfqpoint{3.649326in}{0.841406in}}{\pgfqpoint{3.650057in}{0.843173in}}{\pgfqpoint{3.650057in}{0.845015in}}%
\pgfpathcurveto{\pgfqpoint{3.650057in}{0.846856in}}{\pgfqpoint{3.649326in}{0.848623in}}{\pgfqpoint{3.648024in}{0.849925in}}%
\pgfpathcurveto{\pgfqpoint{3.646721in}{0.851227in}}{\pgfqpoint{3.644955in}{0.851959in}}{\pgfqpoint{3.643113in}{0.851959in}}%
\pgfpathcurveto{\pgfqpoint{3.641271in}{0.851959in}}{\pgfqpoint{3.639505in}{0.851227in}}{\pgfqpoint{3.638203in}{0.849925in}}%
\pgfpathcurveto{\pgfqpoint{3.636900in}{0.848623in}}{\pgfqpoint{3.636169in}{0.846856in}}{\pgfqpoint{3.636169in}{0.845015in}}%
\pgfpathcurveto{\pgfqpoint{3.636169in}{0.843173in}}{\pgfqpoint{3.636900in}{0.841406in}}{\pgfqpoint{3.638203in}{0.840104in}}%
\pgfpathcurveto{\pgfqpoint{3.639505in}{0.838802in}}{\pgfqpoint{3.641271in}{0.838070in}}{\pgfqpoint{3.643113in}{0.838070in}}%
\pgfpathlineto{\pgfqpoint{3.643113in}{0.838070in}}%
\pgfpathclose%
\pgfusepath{stroke,fill}%
\end{pgfscope}%
\begin{pgfscope}%
\pgfpathrectangle{\pgfqpoint{0.661006in}{0.524170in}}{\pgfqpoint{4.194036in}{1.071446in}}%
\pgfusepath{clip}%
\pgfsetbuttcap%
\pgfsetroundjoin%
\definecolor{currentfill}{rgb}{0.362364,0.372383,0.579470}%
\pgfsetfillcolor{currentfill}%
\pgfsetfillopacity{0.700000}%
\pgfsetlinewidth{1.003750pt}%
\definecolor{currentstroke}{rgb}{0.362364,0.372383,0.579470}%
\pgfsetstrokecolor{currentstroke}%
\pgfsetstrokeopacity{0.700000}%
\pgfsetdash{}{0pt}%
\pgfpathmoveto{\pgfqpoint{3.657800in}{0.835606in}}%
\pgfpathcurveto{\pgfqpoint{3.659642in}{0.835606in}}{\pgfqpoint{3.661408in}{0.836337in}}{\pgfqpoint{3.662710in}{0.837640in}}%
\pgfpathcurveto{\pgfqpoint{3.664013in}{0.838942in}}{\pgfqpoint{3.664744in}{0.840708in}}{\pgfqpoint{3.664744in}{0.842550in}}%
\pgfpathcurveto{\pgfqpoint{3.664744in}{0.844392in}}{\pgfqpoint{3.664013in}{0.846158in}}{\pgfqpoint{3.662710in}{0.847461in}}%
\pgfpathcurveto{\pgfqpoint{3.661408in}{0.848763in}}{\pgfqpoint{3.659642in}{0.849494in}}{\pgfqpoint{3.657800in}{0.849494in}}%
\pgfpathcurveto{\pgfqpoint{3.655958in}{0.849494in}}{\pgfqpoint{3.654192in}{0.848763in}}{\pgfqpoint{3.652889in}{0.847461in}}%
\pgfpathcurveto{\pgfqpoint{3.651587in}{0.846158in}}{\pgfqpoint{3.650855in}{0.844392in}}{\pgfqpoint{3.650855in}{0.842550in}}%
\pgfpathcurveto{\pgfqpoint{3.650855in}{0.840708in}}{\pgfqpoint{3.651587in}{0.838942in}}{\pgfqpoint{3.652889in}{0.837640in}}%
\pgfpathcurveto{\pgfqpoint{3.654192in}{0.836337in}}{\pgfqpoint{3.655958in}{0.835606in}}{\pgfqpoint{3.657800in}{0.835606in}}%
\pgfpathlineto{\pgfqpoint{3.657800in}{0.835606in}}%
\pgfpathclose%
\pgfusepath{stroke,fill}%
\end{pgfscope}%
\begin{pgfscope}%
\pgfpathrectangle{\pgfqpoint{0.661006in}{0.524170in}}{\pgfqpoint{4.194036in}{1.071446in}}%
\pgfusepath{clip}%
\pgfsetbuttcap%
\pgfsetroundjoin%
\definecolor{currentfill}{rgb}{0.362364,0.372383,0.579470}%
\pgfsetfillcolor{currentfill}%
\pgfsetfillopacity{0.700000}%
\pgfsetlinewidth{1.003750pt}%
\definecolor{currentstroke}{rgb}{0.362364,0.372383,0.579470}%
\pgfsetstrokecolor{currentstroke}%
\pgfsetstrokeopacity{0.700000}%
\pgfsetdash{}{0pt}%
\pgfpathmoveto{\pgfqpoint{3.664911in}{0.832514in}}%
\pgfpathcurveto{\pgfqpoint{3.666753in}{0.832514in}}{\pgfqpoint{3.668519in}{0.833245in}}{\pgfqpoint{3.669821in}{0.834548in}}%
\pgfpathcurveto{\pgfqpoint{3.671124in}{0.835850in}}{\pgfqpoint{3.671855in}{0.837617in}}{\pgfqpoint{3.671855in}{0.839458in}}%
\pgfpathcurveto{\pgfqpoint{3.671855in}{0.841300in}}{\pgfqpoint{3.671124in}{0.843066in}}{\pgfqpoint{3.669821in}{0.844369in}}%
\pgfpathcurveto{\pgfqpoint{3.668519in}{0.845671in}}{\pgfqpoint{3.666753in}{0.846403in}}{\pgfqpoint{3.664911in}{0.846403in}}%
\pgfpathcurveto{\pgfqpoint{3.663069in}{0.846403in}}{\pgfqpoint{3.661303in}{0.845671in}}{\pgfqpoint{3.660000in}{0.844369in}}%
\pgfpathcurveto{\pgfqpoint{3.658698in}{0.843066in}}{\pgfqpoint{3.657966in}{0.841300in}}{\pgfqpoint{3.657966in}{0.839458in}}%
\pgfpathcurveto{\pgfqpoint{3.657966in}{0.837617in}}{\pgfqpoint{3.658698in}{0.835850in}}{\pgfqpoint{3.660000in}{0.834548in}}%
\pgfpathcurveto{\pgfqpoint{3.661303in}{0.833245in}}{\pgfqpoint{3.663069in}{0.832514in}}{\pgfqpoint{3.664911in}{0.832514in}}%
\pgfpathlineto{\pgfqpoint{3.664911in}{0.832514in}}%
\pgfpathclose%
\pgfusepath{stroke,fill}%
\end{pgfscope}%
\begin{pgfscope}%
\pgfpathrectangle{\pgfqpoint{0.661006in}{0.524170in}}{\pgfqpoint{4.194036in}{1.071446in}}%
\pgfusepath{clip}%
\pgfsetbuttcap%
\pgfsetroundjoin%
\definecolor{currentfill}{rgb}{0.362364,0.372383,0.579470}%
\pgfsetfillcolor{currentfill}%
\pgfsetfillopacity{0.700000}%
\pgfsetlinewidth{1.003750pt}%
\definecolor{currentstroke}{rgb}{0.362364,0.372383,0.579470}%
\pgfsetstrokecolor{currentstroke}%
\pgfsetstrokeopacity{0.700000}%
\pgfsetdash{}{0pt}%
\pgfpathmoveto{\pgfqpoint{3.662726in}{0.831256in}}%
\pgfpathcurveto{\pgfqpoint{3.664568in}{0.831256in}}{\pgfqpoint{3.666335in}{0.831988in}}{\pgfqpoint{3.667637in}{0.833290in}}%
\pgfpathcurveto{\pgfqpoint{3.668939in}{0.834593in}}{\pgfqpoint{3.669671in}{0.836359in}}{\pgfqpoint{3.669671in}{0.838201in}}%
\pgfpathcurveto{\pgfqpoint{3.669671in}{0.840042in}}{\pgfqpoint{3.668939in}{0.841809in}}{\pgfqpoint{3.667637in}{0.843111in}}%
\pgfpathcurveto{\pgfqpoint{3.666335in}{0.844414in}}{\pgfqpoint{3.664568in}{0.845145in}}{\pgfqpoint{3.662726in}{0.845145in}}%
\pgfpathcurveto{\pgfqpoint{3.660885in}{0.845145in}}{\pgfqpoint{3.659118in}{0.844414in}}{\pgfqpoint{3.657816in}{0.843111in}}%
\pgfpathcurveto{\pgfqpoint{3.656514in}{0.841809in}}{\pgfqpoint{3.655782in}{0.840042in}}{\pgfqpoint{3.655782in}{0.838201in}}%
\pgfpathcurveto{\pgfqpoint{3.655782in}{0.836359in}}{\pgfqpoint{3.656514in}{0.834593in}}{\pgfqpoint{3.657816in}{0.833290in}}%
\pgfpathcurveto{\pgfqpoint{3.659118in}{0.831988in}}{\pgfqpoint{3.660885in}{0.831256in}}{\pgfqpoint{3.662726in}{0.831256in}}%
\pgfpathlineto{\pgfqpoint{3.662726in}{0.831256in}}%
\pgfpathclose%
\pgfusepath{stroke,fill}%
\end{pgfscope}%
\begin{pgfscope}%
\pgfpathrectangle{\pgfqpoint{0.661006in}{0.524170in}}{\pgfqpoint{4.194036in}{1.071446in}}%
\pgfusepath{clip}%
\pgfsetbuttcap%
\pgfsetroundjoin%
\definecolor{currentfill}{rgb}{0.362364,0.372383,0.579470}%
\pgfsetfillcolor{currentfill}%
\pgfsetfillopacity{0.700000}%
\pgfsetlinewidth{1.003750pt}%
\definecolor{currentstroke}{rgb}{0.362364,0.372383,0.579470}%
\pgfsetstrokecolor{currentstroke}%
\pgfsetstrokeopacity{0.700000}%
\pgfsetdash{}{0pt}%
\pgfpathmoveto{\pgfqpoint{3.677519in}{0.827685in}}%
\pgfpathcurveto{\pgfqpoint{3.679361in}{0.827685in}}{\pgfqpoint{3.681128in}{0.828417in}}{\pgfqpoint{3.682430in}{0.829719in}}%
\pgfpathcurveto{\pgfqpoint{3.683732in}{0.831021in}}{\pgfqpoint{3.684464in}{0.832788in}}{\pgfqpoint{3.684464in}{0.834630in}}%
\pgfpathcurveto{\pgfqpoint{3.684464in}{0.836471in}}{\pgfqpoint{3.683732in}{0.838238in}}{\pgfqpoint{3.682430in}{0.839540in}}%
\pgfpathcurveto{\pgfqpoint{3.681128in}{0.840842in}}{\pgfqpoint{3.679361in}{0.841574in}}{\pgfqpoint{3.677519in}{0.841574in}}%
\pgfpathcurveto{\pgfqpoint{3.675678in}{0.841574in}}{\pgfqpoint{3.673911in}{0.840842in}}{\pgfqpoint{3.672609in}{0.839540in}}%
\pgfpathcurveto{\pgfqpoint{3.671307in}{0.838238in}}{\pgfqpoint{3.670575in}{0.836471in}}{\pgfqpoint{3.670575in}{0.834630in}}%
\pgfpathcurveto{\pgfqpoint{3.670575in}{0.832788in}}{\pgfqpoint{3.671307in}{0.831021in}}{\pgfqpoint{3.672609in}{0.829719in}}%
\pgfpathcurveto{\pgfqpoint{3.673911in}{0.828417in}}{\pgfqpoint{3.675678in}{0.827685in}}{\pgfqpoint{3.677519in}{0.827685in}}%
\pgfpathlineto{\pgfqpoint{3.677519in}{0.827685in}}%
\pgfpathclose%
\pgfusepath{stroke,fill}%
\end{pgfscope}%
\begin{pgfscope}%
\pgfpathrectangle{\pgfqpoint{0.661006in}{0.524170in}}{\pgfqpoint{4.194036in}{1.071446in}}%
\pgfusepath{clip}%
\pgfsetbuttcap%
\pgfsetroundjoin%
\definecolor{currentfill}{rgb}{0.359581,0.367990,0.574975}%
\pgfsetfillcolor{currentfill}%
\pgfsetfillopacity{0.700000}%
\pgfsetlinewidth{1.003750pt}%
\definecolor{currentstroke}{rgb}{0.359581,0.367990,0.574975}%
\pgfsetstrokecolor{currentstroke}%
\pgfsetstrokeopacity{0.700000}%
\pgfsetdash{}{0pt}%
\pgfpathmoveto{\pgfqpoint{3.686058in}{0.824857in}}%
\pgfpathcurveto{\pgfqpoint{3.687900in}{0.824857in}}{\pgfqpoint{3.689666in}{0.825588in}}{\pgfqpoint{3.690968in}{0.826891in}}%
\pgfpathcurveto{\pgfqpoint{3.692271in}{0.828193in}}{\pgfqpoint{3.693002in}{0.829959in}}{\pgfqpoint{3.693002in}{0.831801in}}%
\pgfpathcurveto{\pgfqpoint{3.693002in}{0.833643in}}{\pgfqpoint{3.692271in}{0.835409in}}{\pgfqpoint{3.690968in}{0.836711in}}%
\pgfpathcurveto{\pgfqpoint{3.689666in}{0.838014in}}{\pgfqpoint{3.687900in}{0.838745in}}{\pgfqpoint{3.686058in}{0.838745in}}%
\pgfpathcurveto{\pgfqpoint{3.684216in}{0.838745in}}{\pgfqpoint{3.682450in}{0.838014in}}{\pgfqpoint{3.681148in}{0.836711in}}%
\pgfpathcurveto{\pgfqpoint{3.679845in}{0.835409in}}{\pgfqpoint{3.679114in}{0.833643in}}{\pgfqpoint{3.679114in}{0.831801in}}%
\pgfpathcurveto{\pgfqpoint{3.679114in}{0.829959in}}{\pgfqpoint{3.679845in}{0.828193in}}{\pgfqpoint{3.681148in}{0.826891in}}%
\pgfpathcurveto{\pgfqpoint{3.682450in}{0.825588in}}{\pgfqpoint{3.684216in}{0.824857in}}{\pgfqpoint{3.686058in}{0.824857in}}%
\pgfpathlineto{\pgfqpoint{3.686058in}{0.824857in}}%
\pgfpathclose%
\pgfusepath{stroke,fill}%
\end{pgfscope}%
\begin{pgfscope}%
\pgfpathrectangle{\pgfqpoint{0.661006in}{0.524170in}}{\pgfqpoint{4.194036in}{1.071446in}}%
\pgfusepath{clip}%
\pgfsetbuttcap%
\pgfsetroundjoin%
\definecolor{currentfill}{rgb}{0.359581,0.367990,0.574975}%
\pgfsetfillcolor{currentfill}%
\pgfsetfillopacity{0.700000}%
\pgfsetlinewidth{1.003750pt}%
\definecolor{currentstroke}{rgb}{0.359581,0.367990,0.574975}%
\pgfsetstrokecolor{currentstroke}%
\pgfsetstrokeopacity{0.700000}%
\pgfsetdash{}{0pt}%
\pgfpathmoveto{\pgfqpoint{3.692890in}{0.822565in}}%
\pgfpathcurveto{\pgfqpoint{3.694732in}{0.822565in}}{\pgfqpoint{3.696498in}{0.823296in}}{\pgfqpoint{3.697801in}{0.824599in}}%
\pgfpathcurveto{\pgfqpoint{3.699103in}{0.825901in}}{\pgfqpoint{3.699835in}{0.827667in}}{\pgfqpoint{3.699835in}{0.829509in}}%
\pgfpathcurveto{\pgfqpoint{3.699835in}{0.831351in}}{\pgfqpoint{3.699103in}{0.833117in}}{\pgfqpoint{3.697801in}{0.834420in}}%
\pgfpathcurveto{\pgfqpoint{3.696498in}{0.835722in}}{\pgfqpoint{3.694732in}{0.836454in}}{\pgfqpoint{3.692890in}{0.836454in}}%
\pgfpathcurveto{\pgfqpoint{3.691048in}{0.836454in}}{\pgfqpoint{3.689282in}{0.835722in}}{\pgfqpoint{3.687980in}{0.834420in}}%
\pgfpathcurveto{\pgfqpoint{3.686677in}{0.833117in}}{\pgfqpoint{3.685946in}{0.831351in}}{\pgfqpoint{3.685946in}{0.829509in}}%
\pgfpathcurveto{\pgfqpoint{3.685946in}{0.827667in}}{\pgfqpoint{3.686677in}{0.825901in}}{\pgfqpoint{3.687980in}{0.824599in}}%
\pgfpathcurveto{\pgfqpoint{3.689282in}{0.823296in}}{\pgfqpoint{3.691048in}{0.822565in}}{\pgfqpoint{3.692890in}{0.822565in}}%
\pgfpathlineto{\pgfqpoint{3.692890in}{0.822565in}}%
\pgfpathclose%
\pgfusepath{stroke,fill}%
\end{pgfscope}%
\begin{pgfscope}%
\pgfpathrectangle{\pgfqpoint{0.661006in}{0.524170in}}{\pgfqpoint{4.194036in}{1.071446in}}%
\pgfusepath{clip}%
\pgfsetbuttcap%
\pgfsetroundjoin%
\definecolor{currentfill}{rgb}{0.359581,0.367990,0.574975}%
\pgfsetfillcolor{currentfill}%
\pgfsetfillopacity{0.700000}%
\pgfsetlinewidth{1.003750pt}%
\definecolor{currentstroke}{rgb}{0.359581,0.367990,0.574975}%
\pgfsetstrokecolor{currentstroke}%
\pgfsetstrokeopacity{0.700000}%
\pgfsetdash{}{0pt}%
\pgfpathmoveto{\pgfqpoint{3.684292in}{0.823973in}}%
\pgfpathcurveto{\pgfqpoint{3.686134in}{0.823973in}}{\pgfqpoint{3.687900in}{0.824705in}}{\pgfqpoint{3.689202in}{0.826007in}}%
\pgfpathcurveto{\pgfqpoint{3.690505in}{0.827309in}}{\pgfqpoint{3.691236in}{0.829076in}}{\pgfqpoint{3.691236in}{0.830918in}}%
\pgfpathcurveto{\pgfqpoint{3.691236in}{0.832759in}}{\pgfqpoint{3.690505in}{0.834526in}}{\pgfqpoint{3.689202in}{0.835828in}}%
\pgfpathcurveto{\pgfqpoint{3.687900in}{0.837130in}}{\pgfqpoint{3.686134in}{0.837862in}}{\pgfqpoint{3.684292in}{0.837862in}}%
\pgfpathcurveto{\pgfqpoint{3.682450in}{0.837862in}}{\pgfqpoint{3.680684in}{0.837130in}}{\pgfqpoint{3.679381in}{0.835828in}}%
\pgfpathcurveto{\pgfqpoint{3.678079in}{0.834526in}}{\pgfqpoint{3.677347in}{0.832759in}}{\pgfqpoint{3.677347in}{0.830918in}}%
\pgfpathcurveto{\pgfqpoint{3.677347in}{0.829076in}}{\pgfqpoint{3.678079in}{0.827309in}}{\pgfqpoint{3.679381in}{0.826007in}}%
\pgfpathcurveto{\pgfqpoint{3.680684in}{0.824705in}}{\pgfqpoint{3.682450in}{0.823973in}}{\pgfqpoint{3.684292in}{0.823973in}}%
\pgfpathlineto{\pgfqpoint{3.684292in}{0.823973in}}%
\pgfpathclose%
\pgfusepath{stroke,fill}%
\end{pgfscope}%
\begin{pgfscope}%
\pgfpathrectangle{\pgfqpoint{0.661006in}{0.524170in}}{\pgfqpoint{4.194036in}{1.071446in}}%
\pgfusepath{clip}%
\pgfsetbuttcap%
\pgfsetroundjoin%
\definecolor{currentfill}{rgb}{0.356796,0.363603,0.570448}%
\pgfsetfillcolor{currentfill}%
\pgfsetfillopacity{0.700000}%
\pgfsetlinewidth{1.003750pt}%
\definecolor{currentstroke}{rgb}{0.356796,0.363603,0.570448}%
\pgfsetstrokecolor{currentstroke}%
\pgfsetstrokeopacity{0.700000}%
\pgfsetdash{}{0pt}%
\pgfpathmoveto{\pgfqpoint{3.669930in}{0.826719in}}%
\pgfpathcurveto{\pgfqpoint{3.671772in}{0.826719in}}{\pgfqpoint{3.673539in}{0.827451in}}{\pgfqpoint{3.674841in}{0.828753in}}%
\pgfpathcurveto{\pgfqpoint{3.676143in}{0.830055in}}{\pgfqpoint{3.676875in}{0.831822in}}{\pgfqpoint{3.676875in}{0.833663in}}%
\pgfpathcurveto{\pgfqpoint{3.676875in}{0.835505in}}{\pgfqpoint{3.676143in}{0.837272in}}{\pgfqpoint{3.674841in}{0.838574in}}%
\pgfpathcurveto{\pgfqpoint{3.673539in}{0.839876in}}{\pgfqpoint{3.671772in}{0.840608in}}{\pgfqpoint{3.669930in}{0.840608in}}%
\pgfpathcurveto{\pgfqpoint{3.668089in}{0.840608in}}{\pgfqpoint{3.666322in}{0.839876in}}{\pgfqpoint{3.665020in}{0.838574in}}%
\pgfpathcurveto{\pgfqpoint{3.663718in}{0.837272in}}{\pgfqpoint{3.662986in}{0.835505in}}{\pgfqpoint{3.662986in}{0.833663in}}%
\pgfpathcurveto{\pgfqpoint{3.662986in}{0.831822in}}{\pgfqpoint{3.663718in}{0.830055in}}{\pgfqpoint{3.665020in}{0.828753in}}%
\pgfpathcurveto{\pgfqpoint{3.666322in}{0.827451in}}{\pgfqpoint{3.668089in}{0.826719in}}{\pgfqpoint{3.669930in}{0.826719in}}%
\pgfpathlineto{\pgfqpoint{3.669930in}{0.826719in}}%
\pgfpathclose%
\pgfusepath{stroke,fill}%
\end{pgfscope}%
\begin{pgfscope}%
\pgfpathrectangle{\pgfqpoint{0.661006in}{0.524170in}}{\pgfqpoint{4.194036in}{1.071446in}}%
\pgfusepath{clip}%
\pgfsetbuttcap%
\pgfsetroundjoin%
\definecolor{currentfill}{rgb}{0.356796,0.363603,0.570448}%
\pgfsetfillcolor{currentfill}%
\pgfsetfillopacity{0.700000}%
\pgfsetlinewidth{1.003750pt}%
\definecolor{currentstroke}{rgb}{0.356796,0.363603,0.570448}%
\pgfsetstrokecolor{currentstroke}%
\pgfsetstrokeopacity{0.700000}%
\pgfsetdash{}{0pt}%
\pgfpathmoveto{\pgfqpoint{3.659055in}{0.828185in}}%
\pgfpathcurveto{\pgfqpoint{3.660896in}{0.828185in}}{\pgfqpoint{3.662663in}{0.828917in}}{\pgfqpoint{3.663965in}{0.830219in}}%
\pgfpathcurveto{\pgfqpoint{3.665267in}{0.831522in}}{\pgfqpoint{3.665999in}{0.833288in}}{\pgfqpoint{3.665999in}{0.835130in}}%
\pgfpathcurveto{\pgfqpoint{3.665999in}{0.836971in}}{\pgfqpoint{3.665267in}{0.838738in}}{\pgfqpoint{3.663965in}{0.840040in}}%
\pgfpathcurveto{\pgfqpoint{3.662663in}{0.841342in}}{\pgfqpoint{3.660896in}{0.842074in}}{\pgfqpoint{3.659055in}{0.842074in}}%
\pgfpathcurveto{\pgfqpoint{3.657213in}{0.842074in}}{\pgfqpoint{3.655447in}{0.841342in}}{\pgfqpoint{3.654144in}{0.840040in}}%
\pgfpathcurveto{\pgfqpoint{3.652842in}{0.838738in}}{\pgfqpoint{3.652110in}{0.836971in}}{\pgfqpoint{3.652110in}{0.835130in}}%
\pgfpathcurveto{\pgfqpoint{3.652110in}{0.833288in}}{\pgfqpoint{3.652842in}{0.831522in}}{\pgfqpoint{3.654144in}{0.830219in}}%
\pgfpathcurveto{\pgfqpoint{3.655447in}{0.828917in}}{\pgfqpoint{3.657213in}{0.828185in}}{\pgfqpoint{3.659055in}{0.828185in}}%
\pgfpathlineto{\pgfqpoint{3.659055in}{0.828185in}}%
\pgfpathclose%
\pgfusepath{stroke,fill}%
\end{pgfscope}%
\begin{pgfscope}%
\pgfpathrectangle{\pgfqpoint{0.661006in}{0.524170in}}{\pgfqpoint{4.194036in}{1.071446in}}%
\pgfusepath{clip}%
\pgfsetbuttcap%
\pgfsetroundjoin%
\definecolor{currentfill}{rgb}{0.356796,0.363603,0.570448}%
\pgfsetfillcolor{currentfill}%
\pgfsetfillopacity{0.700000}%
\pgfsetlinewidth{1.003750pt}%
\definecolor{currentstroke}{rgb}{0.356796,0.363603,0.570448}%
\pgfsetstrokecolor{currentstroke}%
\pgfsetstrokeopacity{0.700000}%
\pgfsetdash{}{0pt}%
\pgfpathmoveto{\pgfqpoint{3.666026in}{0.827823in}}%
\pgfpathcurveto{\pgfqpoint{3.667868in}{0.827823in}}{\pgfqpoint{3.669635in}{0.828555in}}{\pgfqpoint{3.670937in}{0.829857in}}%
\pgfpathcurveto{\pgfqpoint{3.672239in}{0.831160in}}{\pgfqpoint{3.672971in}{0.832926in}}{\pgfqpoint{3.672971in}{0.834768in}}%
\pgfpathcurveto{\pgfqpoint{3.672971in}{0.836609in}}{\pgfqpoint{3.672239in}{0.838376in}}{\pgfqpoint{3.670937in}{0.839678in}}%
\pgfpathcurveto{\pgfqpoint{3.669635in}{0.840981in}}{\pgfqpoint{3.667868in}{0.841712in}}{\pgfqpoint{3.666026in}{0.841712in}}%
\pgfpathcurveto{\pgfqpoint{3.664185in}{0.841712in}}{\pgfqpoint{3.662418in}{0.840981in}}{\pgfqpoint{3.661116in}{0.839678in}}%
\pgfpathcurveto{\pgfqpoint{3.659814in}{0.838376in}}{\pgfqpoint{3.659082in}{0.836609in}}{\pgfqpoint{3.659082in}{0.834768in}}%
\pgfpathcurveto{\pgfqpoint{3.659082in}{0.832926in}}{\pgfqpoint{3.659814in}{0.831160in}}{\pgfqpoint{3.661116in}{0.829857in}}%
\pgfpathcurveto{\pgfqpoint{3.662418in}{0.828555in}}{\pgfqpoint{3.664185in}{0.827823in}}{\pgfqpoint{3.666026in}{0.827823in}}%
\pgfpathlineto{\pgfqpoint{3.666026in}{0.827823in}}%
\pgfpathclose%
\pgfusepath{stroke,fill}%
\end{pgfscope}%
\begin{pgfscope}%
\pgfpathrectangle{\pgfqpoint{0.661006in}{0.524170in}}{\pgfqpoint{4.194036in}{1.071446in}}%
\pgfusepath{clip}%
\pgfsetbuttcap%
\pgfsetroundjoin%
\definecolor{currentfill}{rgb}{0.356796,0.363603,0.570448}%
\pgfsetfillcolor{currentfill}%
\pgfsetfillopacity{0.700000}%
\pgfsetlinewidth{1.003750pt}%
\definecolor{currentstroke}{rgb}{0.356796,0.363603,0.570448}%
\pgfsetstrokecolor{currentstroke}%
\pgfsetstrokeopacity{0.700000}%
\pgfsetdash{}{0pt}%
\pgfpathmoveto{\pgfqpoint{3.661797in}{0.827220in}}%
\pgfpathcurveto{\pgfqpoint{3.663639in}{0.827220in}}{\pgfqpoint{3.665405in}{0.827952in}}{\pgfqpoint{3.666707in}{0.829254in}}%
\pgfpathcurveto{\pgfqpoint{3.668010in}{0.830556in}}{\pgfqpoint{3.668741in}{0.832323in}}{\pgfqpoint{3.668741in}{0.834165in}}%
\pgfpathcurveto{\pgfqpoint{3.668741in}{0.836006in}}{\pgfqpoint{3.668010in}{0.837773in}}{\pgfqpoint{3.666707in}{0.839075in}}%
\pgfpathcurveto{\pgfqpoint{3.665405in}{0.840377in}}{\pgfqpoint{3.663639in}{0.841109in}}{\pgfqpoint{3.661797in}{0.841109in}}%
\pgfpathcurveto{\pgfqpoint{3.659955in}{0.841109in}}{\pgfqpoint{3.658189in}{0.840377in}}{\pgfqpoint{3.656886in}{0.839075in}}%
\pgfpathcurveto{\pgfqpoint{3.655584in}{0.837773in}}{\pgfqpoint{3.654852in}{0.836006in}}{\pgfqpoint{3.654852in}{0.834165in}}%
\pgfpathcurveto{\pgfqpoint{3.654852in}{0.832323in}}{\pgfqpoint{3.655584in}{0.830556in}}{\pgfqpoint{3.656886in}{0.829254in}}%
\pgfpathcurveto{\pgfqpoint{3.658189in}{0.827952in}}{\pgfqpoint{3.659955in}{0.827220in}}{\pgfqpoint{3.661797in}{0.827220in}}%
\pgfpathlineto{\pgfqpoint{3.661797in}{0.827220in}}%
\pgfpathclose%
\pgfusepath{stroke,fill}%
\end{pgfscope}%
\begin{pgfscope}%
\pgfpathrectangle{\pgfqpoint{0.661006in}{0.524170in}}{\pgfqpoint{4.194036in}{1.071446in}}%
\pgfusepath{clip}%
\pgfsetbuttcap%
\pgfsetroundjoin%
\definecolor{currentfill}{rgb}{0.354007,0.359224,0.565888}%
\pgfsetfillcolor{currentfill}%
\pgfsetfillopacity{0.700000}%
\pgfsetlinewidth{1.003750pt}%
\definecolor{currentstroke}{rgb}{0.354007,0.359224,0.565888}%
\pgfsetstrokecolor{currentstroke}%
\pgfsetstrokeopacity{0.700000}%
\pgfsetdash{}{0pt}%
\pgfpathmoveto{\pgfqpoint{3.654175in}{0.827851in}}%
\pgfpathcurveto{\pgfqpoint{3.656016in}{0.827851in}}{\pgfqpoint{3.657783in}{0.828583in}}{\pgfqpoint{3.659085in}{0.829885in}}%
\pgfpathcurveto{\pgfqpoint{3.660387in}{0.831188in}}{\pgfqpoint{3.661119in}{0.832954in}}{\pgfqpoint{3.661119in}{0.834796in}}%
\pgfpathcurveto{\pgfqpoint{3.661119in}{0.836637in}}{\pgfqpoint{3.660387in}{0.838404in}}{\pgfqpoint{3.659085in}{0.839706in}}%
\pgfpathcurveto{\pgfqpoint{3.657783in}{0.841008in}}{\pgfqpoint{3.656016in}{0.841740in}}{\pgfqpoint{3.654175in}{0.841740in}}%
\pgfpathcurveto{\pgfqpoint{3.652333in}{0.841740in}}{\pgfqpoint{3.650566in}{0.841008in}}{\pgfqpoint{3.649264in}{0.839706in}}%
\pgfpathcurveto{\pgfqpoint{3.647962in}{0.838404in}}{\pgfqpoint{3.647230in}{0.836637in}}{\pgfqpoint{3.647230in}{0.834796in}}%
\pgfpathcurveto{\pgfqpoint{3.647230in}{0.832954in}}{\pgfqpoint{3.647962in}{0.831188in}}{\pgfqpoint{3.649264in}{0.829885in}}%
\pgfpathcurveto{\pgfqpoint{3.650566in}{0.828583in}}{\pgfqpoint{3.652333in}{0.827851in}}{\pgfqpoint{3.654175in}{0.827851in}}%
\pgfpathlineto{\pgfqpoint{3.654175in}{0.827851in}}%
\pgfpathclose%
\pgfusepath{stroke,fill}%
\end{pgfscope}%
\begin{pgfscope}%
\pgfpathrectangle{\pgfqpoint{0.661006in}{0.524170in}}{\pgfqpoint{4.194036in}{1.071446in}}%
\pgfusepath{clip}%
\pgfsetbuttcap%
\pgfsetroundjoin%
\definecolor{currentfill}{rgb}{0.354007,0.359224,0.565888}%
\pgfsetfillcolor{currentfill}%
\pgfsetfillopacity{0.700000}%
\pgfsetlinewidth{1.003750pt}%
\definecolor{currentstroke}{rgb}{0.354007,0.359224,0.565888}%
\pgfsetstrokecolor{currentstroke}%
\pgfsetstrokeopacity{0.700000}%
\pgfsetdash{}{0pt}%
\pgfpathmoveto{\pgfqpoint{3.650782in}{0.827331in}}%
\pgfpathcurveto{\pgfqpoint{3.652623in}{0.827331in}}{\pgfqpoint{3.654390in}{0.828062in}}{\pgfqpoint{3.655692in}{0.829365in}}%
\pgfpathcurveto{\pgfqpoint{3.656995in}{0.830667in}}{\pgfqpoint{3.657726in}{0.832434in}}{\pgfqpoint{3.657726in}{0.834275in}}%
\pgfpathcurveto{\pgfqpoint{3.657726in}{0.836117in}}{\pgfqpoint{3.656995in}{0.837883in}}{\pgfqpoint{3.655692in}{0.839186in}}%
\pgfpathcurveto{\pgfqpoint{3.654390in}{0.840488in}}{\pgfqpoint{3.652623in}{0.841220in}}{\pgfqpoint{3.650782in}{0.841220in}}%
\pgfpathcurveto{\pgfqpoint{3.648940in}{0.841220in}}{\pgfqpoint{3.647174in}{0.840488in}}{\pgfqpoint{3.645871in}{0.839186in}}%
\pgfpathcurveto{\pgfqpoint{3.644569in}{0.837883in}}{\pgfqpoint{3.643837in}{0.836117in}}{\pgfqpoint{3.643837in}{0.834275in}}%
\pgfpathcurveto{\pgfqpoint{3.643837in}{0.832434in}}{\pgfqpoint{3.644569in}{0.830667in}}{\pgfqpoint{3.645871in}{0.829365in}}%
\pgfpathcurveto{\pgfqpoint{3.647174in}{0.828062in}}{\pgfqpoint{3.648940in}{0.827331in}}{\pgfqpoint{3.650782in}{0.827331in}}%
\pgfpathlineto{\pgfqpoint{3.650782in}{0.827331in}}%
\pgfpathclose%
\pgfusepath{stroke,fill}%
\end{pgfscope}%
\begin{pgfscope}%
\pgfpathrectangle{\pgfqpoint{0.661006in}{0.524170in}}{\pgfqpoint{4.194036in}{1.071446in}}%
\pgfusepath{clip}%
\pgfsetbuttcap%
\pgfsetroundjoin%
\definecolor{currentfill}{rgb}{0.354007,0.359224,0.565888}%
\pgfsetfillcolor{currentfill}%
\pgfsetfillopacity{0.700000}%
\pgfsetlinewidth{1.003750pt}%
\definecolor{currentstroke}{rgb}{0.354007,0.359224,0.565888}%
\pgfsetstrokecolor{currentstroke}%
\pgfsetstrokeopacity{0.700000}%
\pgfsetdash{}{0pt}%
\pgfpathmoveto{\pgfqpoint{3.661007in}{0.826112in}}%
\pgfpathcurveto{\pgfqpoint{3.662848in}{0.826112in}}{\pgfqpoint{3.664615in}{0.826844in}}{\pgfqpoint{3.665917in}{0.828146in}}%
\pgfpathcurveto{\pgfqpoint{3.667220in}{0.829448in}}{\pgfqpoint{3.667951in}{0.831215in}}{\pgfqpoint{3.667951in}{0.833057in}}%
\pgfpathcurveto{\pgfqpoint{3.667951in}{0.834898in}}{\pgfqpoint{3.667220in}{0.836665in}}{\pgfqpoint{3.665917in}{0.837967in}}%
\pgfpathcurveto{\pgfqpoint{3.664615in}{0.839269in}}{\pgfqpoint{3.662848in}{0.840001in}}{\pgfqpoint{3.661007in}{0.840001in}}%
\pgfpathcurveto{\pgfqpoint{3.659165in}{0.840001in}}{\pgfqpoint{3.657399in}{0.839269in}}{\pgfqpoint{3.656096in}{0.837967in}}%
\pgfpathcurveto{\pgfqpoint{3.654794in}{0.836665in}}{\pgfqpoint{3.654062in}{0.834898in}}{\pgfqpoint{3.654062in}{0.833057in}}%
\pgfpathcurveto{\pgfqpoint{3.654062in}{0.831215in}}{\pgfqpoint{3.654794in}{0.829448in}}{\pgfqpoint{3.656096in}{0.828146in}}%
\pgfpathcurveto{\pgfqpoint{3.657399in}{0.826844in}}{\pgfqpoint{3.659165in}{0.826112in}}{\pgfqpoint{3.661007in}{0.826112in}}%
\pgfpathlineto{\pgfqpoint{3.661007in}{0.826112in}}%
\pgfpathclose%
\pgfusepath{stroke,fill}%
\end{pgfscope}%
\begin{pgfscope}%
\pgfpathrectangle{\pgfqpoint{0.661006in}{0.524170in}}{\pgfqpoint{4.194036in}{1.071446in}}%
\pgfusepath{clip}%
\pgfsetbuttcap%
\pgfsetroundjoin%
\definecolor{currentfill}{rgb}{0.351215,0.354854,0.561297}%
\pgfsetfillcolor{currentfill}%
\pgfsetfillopacity{0.700000}%
\pgfsetlinewidth{1.003750pt}%
\definecolor{currentstroke}{rgb}{0.351215,0.354854,0.561297}%
\pgfsetstrokecolor{currentstroke}%
\pgfsetstrokeopacity{0.700000}%
\pgfsetdash{}{0pt}%
\pgfpathmoveto{\pgfqpoint{3.639674in}{0.825651in}}%
\pgfpathcurveto{\pgfqpoint{3.641515in}{0.825651in}}{\pgfqpoint{3.643282in}{0.826383in}}{\pgfqpoint{3.644584in}{0.827685in}}%
\pgfpathcurveto{\pgfqpoint{3.645886in}{0.828988in}}{\pgfqpoint{3.646618in}{0.830754in}}{\pgfqpoint{3.646618in}{0.832596in}}%
\pgfpathcurveto{\pgfqpoint{3.646618in}{0.834438in}}{\pgfqpoint{3.645886in}{0.836204in}}{\pgfqpoint{3.644584in}{0.837506in}}%
\pgfpathcurveto{\pgfqpoint{3.643282in}{0.838809in}}{\pgfqpoint{3.641515in}{0.839540in}}{\pgfqpoint{3.639674in}{0.839540in}}%
\pgfpathcurveto{\pgfqpoint{3.637832in}{0.839540in}}{\pgfqpoint{3.636066in}{0.838809in}}{\pgfqpoint{3.634763in}{0.837506in}}%
\pgfpathcurveto{\pgfqpoint{3.633461in}{0.836204in}}{\pgfqpoint{3.632729in}{0.834438in}}{\pgfqpoint{3.632729in}{0.832596in}}%
\pgfpathcurveto{\pgfqpoint{3.632729in}{0.830754in}}{\pgfqpoint{3.633461in}{0.828988in}}{\pgfqpoint{3.634763in}{0.827685in}}%
\pgfpathcurveto{\pgfqpoint{3.636066in}{0.826383in}}{\pgfqpoint{3.637832in}{0.825651in}}{\pgfqpoint{3.639674in}{0.825651in}}%
\pgfpathlineto{\pgfqpoint{3.639674in}{0.825651in}}%
\pgfpathclose%
\pgfusepath{stroke,fill}%
\end{pgfscope}%
\begin{pgfscope}%
\pgfpathrectangle{\pgfqpoint{0.661006in}{0.524170in}}{\pgfqpoint{4.194036in}{1.071446in}}%
\pgfusepath{clip}%
\pgfsetbuttcap%
\pgfsetroundjoin%
\definecolor{currentfill}{rgb}{0.351215,0.354854,0.561297}%
\pgfsetfillcolor{currentfill}%
\pgfsetfillopacity{0.700000}%
\pgfsetlinewidth{1.003750pt}%
\definecolor{currentstroke}{rgb}{0.351215,0.354854,0.561297}%
\pgfsetstrokecolor{currentstroke}%
\pgfsetstrokeopacity{0.700000}%
\pgfsetdash{}{0pt}%
\pgfpathmoveto{\pgfqpoint{3.646692in}{0.828242in}}%
\pgfpathcurveto{\pgfqpoint{3.648533in}{0.828242in}}{\pgfqpoint{3.650300in}{0.828974in}}{\pgfqpoint{3.651602in}{0.830276in}}%
\pgfpathcurveto{\pgfqpoint{3.652905in}{0.831578in}}{\pgfqpoint{3.653636in}{0.833345in}}{\pgfqpoint{3.653636in}{0.835187in}}%
\pgfpathcurveto{\pgfqpoint{3.653636in}{0.837028in}}{\pgfqpoint{3.652905in}{0.838795in}}{\pgfqpoint{3.651602in}{0.840097in}}%
\pgfpathcurveto{\pgfqpoint{3.650300in}{0.841399in}}{\pgfqpoint{3.648533in}{0.842131in}}{\pgfqpoint{3.646692in}{0.842131in}}%
\pgfpathcurveto{\pgfqpoint{3.644850in}{0.842131in}}{\pgfqpoint{3.643084in}{0.841399in}}{\pgfqpoint{3.641781in}{0.840097in}}%
\pgfpathcurveto{\pgfqpoint{3.640479in}{0.838795in}}{\pgfqpoint{3.639747in}{0.837028in}}{\pgfqpoint{3.639747in}{0.835187in}}%
\pgfpathcurveto{\pgfqpoint{3.639747in}{0.833345in}}{\pgfqpoint{3.640479in}{0.831578in}}{\pgfqpoint{3.641781in}{0.830276in}}%
\pgfpathcurveto{\pgfqpoint{3.643084in}{0.828974in}}{\pgfqpoint{3.644850in}{0.828242in}}{\pgfqpoint{3.646692in}{0.828242in}}%
\pgfpathlineto{\pgfqpoint{3.646692in}{0.828242in}}%
\pgfpathclose%
\pgfusepath{stroke,fill}%
\end{pgfscope}%
\begin{pgfscope}%
\pgfpathrectangle{\pgfqpoint{0.661006in}{0.524170in}}{\pgfqpoint{4.194036in}{1.071446in}}%
\pgfusepath{clip}%
\pgfsetbuttcap%
\pgfsetroundjoin%
\definecolor{currentfill}{rgb}{0.351215,0.354854,0.561297}%
\pgfsetfillcolor{currentfill}%
\pgfsetfillopacity{0.700000}%
\pgfsetlinewidth{1.003750pt}%
\definecolor{currentstroke}{rgb}{0.351215,0.354854,0.561297}%
\pgfsetstrokecolor{currentstroke}%
\pgfsetstrokeopacity{0.700000}%
\pgfsetdash{}{0pt}%
\pgfpathmoveto{\pgfqpoint{3.615459in}{0.832223in}}%
\pgfpathcurveto{\pgfqpoint{3.617301in}{0.832223in}}{\pgfqpoint{3.619067in}{0.832954in}}{\pgfqpoint{3.620370in}{0.834257in}}%
\pgfpathcurveto{\pgfqpoint{3.621672in}{0.835559in}}{\pgfqpoint{3.622404in}{0.837325in}}{\pgfqpoint{3.622404in}{0.839167in}}%
\pgfpathcurveto{\pgfqpoint{3.622404in}{0.841009in}}{\pgfqpoint{3.621672in}{0.842775in}}{\pgfqpoint{3.620370in}{0.844077in}}%
\pgfpathcurveto{\pgfqpoint{3.619067in}{0.845380in}}{\pgfqpoint{3.617301in}{0.846111in}}{\pgfqpoint{3.615459in}{0.846111in}}%
\pgfpathcurveto{\pgfqpoint{3.613617in}{0.846111in}}{\pgfqpoint{3.611851in}{0.845380in}}{\pgfqpoint{3.610549in}{0.844077in}}%
\pgfpathcurveto{\pgfqpoint{3.609246in}{0.842775in}}{\pgfqpoint{3.608515in}{0.841009in}}{\pgfqpoint{3.608515in}{0.839167in}}%
\pgfpathcurveto{\pgfqpoint{3.608515in}{0.837325in}}{\pgfqpoint{3.609246in}{0.835559in}}{\pgfqpoint{3.610549in}{0.834257in}}%
\pgfpathcurveto{\pgfqpoint{3.611851in}{0.832954in}}{\pgfqpoint{3.613617in}{0.832223in}}{\pgfqpoint{3.615459in}{0.832223in}}%
\pgfpathlineto{\pgfqpoint{3.615459in}{0.832223in}}%
\pgfpathclose%
\pgfusepath{stroke,fill}%
\end{pgfscope}%
\begin{pgfscope}%
\pgfpathrectangle{\pgfqpoint{0.661006in}{0.524170in}}{\pgfqpoint{4.194036in}{1.071446in}}%
\pgfusepath{clip}%
\pgfsetbuttcap%
\pgfsetroundjoin%
\definecolor{currentfill}{rgb}{0.351215,0.354854,0.561297}%
\pgfsetfillcolor{currentfill}%
\pgfsetfillopacity{0.700000}%
\pgfsetlinewidth{1.003750pt}%
\definecolor{currentstroke}{rgb}{0.351215,0.354854,0.561297}%
\pgfsetstrokecolor{currentstroke}%
\pgfsetstrokeopacity{0.700000}%
\pgfsetdash{}{0pt}%
\pgfpathmoveto{\pgfqpoint{3.604713in}{0.835661in}}%
\pgfpathcurveto{\pgfqpoint{3.606555in}{0.835661in}}{\pgfqpoint{3.608321in}{0.836392in}}{\pgfqpoint{3.609623in}{0.837695in}}%
\pgfpathcurveto{\pgfqpoint{3.610926in}{0.838997in}}{\pgfqpoint{3.611657in}{0.840763in}}{\pgfqpoint{3.611657in}{0.842605in}}%
\pgfpathcurveto{\pgfqpoint{3.611657in}{0.844447in}}{\pgfqpoint{3.610926in}{0.846213in}}{\pgfqpoint{3.609623in}{0.847515in}}%
\pgfpathcurveto{\pgfqpoint{3.608321in}{0.848818in}}{\pgfqpoint{3.606555in}{0.849549in}}{\pgfqpoint{3.604713in}{0.849549in}}%
\pgfpathcurveto{\pgfqpoint{3.602871in}{0.849549in}}{\pgfqpoint{3.601105in}{0.848818in}}{\pgfqpoint{3.599802in}{0.847515in}}%
\pgfpathcurveto{\pgfqpoint{3.598500in}{0.846213in}}{\pgfqpoint{3.597768in}{0.844447in}}{\pgfqpoint{3.597768in}{0.842605in}}%
\pgfpathcurveto{\pgfqpoint{3.597768in}{0.840763in}}{\pgfqpoint{3.598500in}{0.838997in}}{\pgfqpoint{3.599802in}{0.837695in}}%
\pgfpathcurveto{\pgfqpoint{3.601105in}{0.836392in}}{\pgfqpoint{3.602871in}{0.835661in}}{\pgfqpoint{3.604713in}{0.835661in}}%
\pgfpathlineto{\pgfqpoint{3.604713in}{0.835661in}}%
\pgfpathclose%
\pgfusepath{stroke,fill}%
\end{pgfscope}%
\begin{pgfscope}%
\pgfpathrectangle{\pgfqpoint{0.661006in}{0.524170in}}{\pgfqpoint{4.194036in}{1.071446in}}%
\pgfusepath{clip}%
\pgfsetbuttcap%
\pgfsetroundjoin%
\definecolor{currentfill}{rgb}{0.348418,0.350491,0.556674}%
\pgfsetfillcolor{currentfill}%
\pgfsetfillopacity{0.700000}%
\pgfsetlinewidth{1.003750pt}%
\definecolor{currentstroke}{rgb}{0.348418,0.350491,0.556674}%
\pgfsetstrokecolor{currentstroke}%
\pgfsetstrokeopacity{0.700000}%
\pgfsetdash{}{0pt}%
\pgfpathmoveto{\pgfqpoint{3.588038in}{0.838515in}}%
\pgfpathcurveto{\pgfqpoint{3.589879in}{0.838515in}}{\pgfqpoint{3.591646in}{0.839247in}}{\pgfqpoint{3.592948in}{0.840549in}}%
\pgfpathcurveto{\pgfqpoint{3.594250in}{0.841852in}}{\pgfqpoint{3.594982in}{0.843618in}}{\pgfqpoint{3.594982in}{0.845460in}}%
\pgfpathcurveto{\pgfqpoint{3.594982in}{0.847302in}}{\pgfqpoint{3.594250in}{0.849068in}}{\pgfqpoint{3.592948in}{0.850370in}}%
\pgfpathcurveto{\pgfqpoint{3.591646in}{0.851673in}}{\pgfqpoint{3.589879in}{0.852404in}}{\pgfqpoint{3.588038in}{0.852404in}}%
\pgfpathcurveto{\pgfqpoint{3.586196in}{0.852404in}}{\pgfqpoint{3.584429in}{0.851673in}}{\pgfqpoint{3.583127in}{0.850370in}}%
\pgfpathcurveto{\pgfqpoint{3.581825in}{0.849068in}}{\pgfqpoint{3.581093in}{0.847302in}}{\pgfqpoint{3.581093in}{0.845460in}}%
\pgfpathcurveto{\pgfqpoint{3.581093in}{0.843618in}}{\pgfqpoint{3.581825in}{0.841852in}}{\pgfqpoint{3.583127in}{0.840549in}}%
\pgfpathcurveto{\pgfqpoint{3.584429in}{0.839247in}}{\pgfqpoint{3.586196in}{0.838515in}}{\pgfqpoint{3.588038in}{0.838515in}}%
\pgfpathlineto{\pgfqpoint{3.588038in}{0.838515in}}%
\pgfpathclose%
\pgfusepath{stroke,fill}%
\end{pgfscope}%
\begin{pgfscope}%
\pgfpathrectangle{\pgfqpoint{0.661006in}{0.524170in}}{\pgfqpoint{4.194036in}{1.071446in}}%
\pgfusepath{clip}%
\pgfsetbuttcap%
\pgfsetroundjoin%
\definecolor{currentfill}{rgb}{0.348418,0.350491,0.556674}%
\pgfsetfillcolor{currentfill}%
\pgfsetfillopacity{0.700000}%
\pgfsetlinewidth{1.003750pt}%
\definecolor{currentstroke}{rgb}{0.348418,0.350491,0.556674}%
\pgfsetstrokecolor{currentstroke}%
\pgfsetstrokeopacity{0.700000}%
\pgfsetdash{}{0pt}%
\pgfpathmoveto{\pgfqpoint{3.564799in}{0.840455in}}%
\pgfpathcurveto{\pgfqpoint{3.566641in}{0.840455in}}{\pgfqpoint{3.568407in}{0.841187in}}{\pgfqpoint{3.569709in}{0.842489in}}%
\pgfpathcurveto{\pgfqpoint{3.571012in}{0.843791in}}{\pgfqpoint{3.571743in}{0.845558in}}{\pgfqpoint{3.571743in}{0.847399in}}%
\pgfpathcurveto{\pgfqpoint{3.571743in}{0.849241in}}{\pgfqpoint{3.571012in}{0.851008in}}{\pgfqpoint{3.569709in}{0.852310in}}%
\pgfpathcurveto{\pgfqpoint{3.568407in}{0.853612in}}{\pgfqpoint{3.566641in}{0.854344in}}{\pgfqpoint{3.564799in}{0.854344in}}%
\pgfpathcurveto{\pgfqpoint{3.562957in}{0.854344in}}{\pgfqpoint{3.561191in}{0.853612in}}{\pgfqpoint{3.559888in}{0.852310in}}%
\pgfpathcurveto{\pgfqpoint{3.558586in}{0.851008in}}{\pgfqpoint{3.557854in}{0.849241in}}{\pgfqpoint{3.557854in}{0.847399in}}%
\pgfpathcurveto{\pgfqpoint{3.557854in}{0.845558in}}{\pgfqpoint{3.558586in}{0.843791in}}{\pgfqpoint{3.559888in}{0.842489in}}%
\pgfpathcurveto{\pgfqpoint{3.561191in}{0.841187in}}{\pgfqpoint{3.562957in}{0.840455in}}{\pgfqpoint{3.564799in}{0.840455in}}%
\pgfpathlineto{\pgfqpoint{3.564799in}{0.840455in}}%
\pgfpathclose%
\pgfusepath{stroke,fill}%
\end{pgfscope}%
\begin{pgfscope}%
\pgfpathrectangle{\pgfqpoint{0.661006in}{0.524170in}}{\pgfqpoint{4.194036in}{1.071446in}}%
\pgfusepath{clip}%
\pgfsetbuttcap%
\pgfsetroundjoin%
\definecolor{currentfill}{rgb}{0.348418,0.350491,0.556674}%
\pgfsetfillcolor{currentfill}%
\pgfsetfillopacity{0.700000}%
\pgfsetlinewidth{1.003750pt}%
\definecolor{currentstroke}{rgb}{0.348418,0.350491,0.556674}%
\pgfsetstrokecolor{currentstroke}%
\pgfsetstrokeopacity{0.700000}%
\pgfsetdash{}{0pt}%
\pgfpathmoveto{\pgfqpoint{3.579811in}{0.836020in}}%
\pgfpathcurveto{\pgfqpoint{3.581653in}{0.836020in}}{\pgfqpoint{3.583419in}{0.836752in}}{\pgfqpoint{3.584722in}{0.838054in}}%
\pgfpathcurveto{\pgfqpoint{3.586024in}{0.839356in}}{\pgfqpoint{3.586756in}{0.841123in}}{\pgfqpoint{3.586756in}{0.842964in}}%
\pgfpathcurveto{\pgfqpoint{3.586756in}{0.844806in}}{\pgfqpoint{3.586024in}{0.846573in}}{\pgfqpoint{3.584722in}{0.847875in}}%
\pgfpathcurveto{\pgfqpoint{3.583419in}{0.849177in}}{\pgfqpoint{3.581653in}{0.849909in}}{\pgfqpoint{3.579811in}{0.849909in}}%
\pgfpathcurveto{\pgfqpoint{3.577969in}{0.849909in}}{\pgfqpoint{3.576203in}{0.849177in}}{\pgfqpoint{3.574901in}{0.847875in}}%
\pgfpathcurveto{\pgfqpoint{3.573598in}{0.846573in}}{\pgfqpoint{3.572867in}{0.844806in}}{\pgfqpoint{3.572867in}{0.842964in}}%
\pgfpathcurveto{\pgfqpoint{3.572867in}{0.841123in}}{\pgfqpoint{3.573598in}{0.839356in}}{\pgfqpoint{3.574901in}{0.838054in}}%
\pgfpathcurveto{\pgfqpoint{3.576203in}{0.836752in}}{\pgfqpoint{3.577969in}{0.836020in}}{\pgfqpoint{3.579811in}{0.836020in}}%
\pgfpathlineto{\pgfqpoint{3.579811in}{0.836020in}}%
\pgfpathclose%
\pgfusepath{stroke,fill}%
\end{pgfscope}%
\begin{pgfscope}%
\pgfpathrectangle{\pgfqpoint{0.661006in}{0.524170in}}{\pgfqpoint{4.194036in}{1.071446in}}%
\pgfusepath{clip}%
\pgfsetbuttcap%
\pgfsetroundjoin%
\definecolor{currentfill}{rgb}{0.345617,0.346136,0.552019}%
\pgfsetfillcolor{currentfill}%
\pgfsetfillopacity{0.700000}%
\pgfsetlinewidth{1.003750pt}%
\definecolor{currentstroke}{rgb}{0.345617,0.346136,0.552019}%
\pgfsetstrokecolor{currentstroke}%
\pgfsetstrokeopacity{0.700000}%
\pgfsetdash{}{0pt}%
\pgfpathmoveto{\pgfqpoint{3.605095in}{0.829201in}}%
\pgfpathcurveto{\pgfqpoint{3.606936in}{0.829201in}}{\pgfqpoint{3.608703in}{0.829932in}}{\pgfqpoint{3.610005in}{0.831235in}}%
\pgfpathcurveto{\pgfqpoint{3.611307in}{0.832537in}}{\pgfqpoint{3.612039in}{0.834304in}}{\pgfqpoint{3.612039in}{0.836145in}}%
\pgfpathcurveto{\pgfqpoint{3.612039in}{0.837987in}}{\pgfqpoint{3.611307in}{0.839753in}}{\pgfqpoint{3.610005in}{0.841056in}}%
\pgfpathcurveto{\pgfqpoint{3.608703in}{0.842358in}}{\pgfqpoint{3.606936in}{0.843090in}}{\pgfqpoint{3.605095in}{0.843090in}}%
\pgfpathcurveto{\pgfqpoint{3.603253in}{0.843090in}}{\pgfqpoint{3.601486in}{0.842358in}}{\pgfqpoint{3.600184in}{0.841056in}}%
\pgfpathcurveto{\pgfqpoint{3.598882in}{0.839753in}}{\pgfqpoint{3.598150in}{0.837987in}}{\pgfqpoint{3.598150in}{0.836145in}}%
\pgfpathcurveto{\pgfqpoint{3.598150in}{0.834304in}}{\pgfqpoint{3.598882in}{0.832537in}}{\pgfqpoint{3.600184in}{0.831235in}}%
\pgfpathcurveto{\pgfqpoint{3.601486in}{0.829932in}}{\pgfqpoint{3.603253in}{0.829201in}}{\pgfqpoint{3.605095in}{0.829201in}}%
\pgfpathlineto{\pgfqpoint{3.605095in}{0.829201in}}%
\pgfpathclose%
\pgfusepath{stroke,fill}%
\end{pgfscope}%
\begin{pgfscope}%
\pgfpathrectangle{\pgfqpoint{0.661006in}{0.524170in}}{\pgfqpoint{4.194036in}{1.071446in}}%
\pgfusepath{clip}%
\pgfsetbuttcap%
\pgfsetroundjoin%
\definecolor{currentfill}{rgb}{0.345617,0.346136,0.552019}%
\pgfsetfillcolor{currentfill}%
\pgfsetfillopacity{0.700000}%
\pgfsetlinewidth{1.003750pt}%
\definecolor{currentstroke}{rgb}{0.345617,0.346136,0.552019}%
\pgfsetstrokecolor{currentstroke}%
\pgfsetstrokeopacity{0.700000}%
\pgfsetdash{}{0pt}%
\pgfpathmoveto{\pgfqpoint{3.637257in}{0.821428in}}%
\pgfpathcurveto{\pgfqpoint{3.639099in}{0.821428in}}{\pgfqpoint{3.640865in}{0.822160in}}{\pgfqpoint{3.642167in}{0.823462in}}%
\pgfpathcurveto{\pgfqpoint{3.643470in}{0.824764in}}{\pgfqpoint{3.644201in}{0.826531in}}{\pgfqpoint{3.644201in}{0.828372in}}%
\pgfpathcurveto{\pgfqpoint{3.644201in}{0.830214in}}{\pgfqpoint{3.643470in}{0.831981in}}{\pgfqpoint{3.642167in}{0.833283in}}%
\pgfpathcurveto{\pgfqpoint{3.640865in}{0.834585in}}{\pgfqpoint{3.639099in}{0.835317in}}{\pgfqpoint{3.637257in}{0.835317in}}%
\pgfpathcurveto{\pgfqpoint{3.635415in}{0.835317in}}{\pgfqpoint{3.633649in}{0.834585in}}{\pgfqpoint{3.632346in}{0.833283in}}%
\pgfpathcurveto{\pgfqpoint{3.631044in}{0.831981in}}{\pgfqpoint{3.630312in}{0.830214in}}{\pgfqpoint{3.630312in}{0.828372in}}%
\pgfpathcurveto{\pgfqpoint{3.630312in}{0.826531in}}{\pgfqpoint{3.631044in}{0.824764in}}{\pgfqpoint{3.632346in}{0.823462in}}%
\pgfpathcurveto{\pgfqpoint{3.633649in}{0.822160in}}{\pgfqpoint{3.635415in}{0.821428in}}{\pgfqpoint{3.637257in}{0.821428in}}%
\pgfpathlineto{\pgfqpoint{3.637257in}{0.821428in}}%
\pgfpathclose%
\pgfusepath{stroke,fill}%
\end{pgfscope}%
\begin{pgfscope}%
\pgfpathrectangle{\pgfqpoint{0.661006in}{0.524170in}}{\pgfqpoint{4.194036in}{1.071446in}}%
\pgfusepath{clip}%
\pgfsetbuttcap%
\pgfsetroundjoin%
\definecolor{currentfill}{rgb}{0.342811,0.341790,0.547332}%
\pgfsetfillcolor{currentfill}%
\pgfsetfillopacity{0.700000}%
\pgfsetlinewidth{1.003750pt}%
\definecolor{currentstroke}{rgb}{0.342811,0.341790,0.547332}%
\pgfsetstrokecolor{currentstroke}%
\pgfsetstrokeopacity{0.700000}%
\pgfsetdash{}{0pt}%
\pgfpathmoveto{\pgfqpoint{3.676437in}{0.811866in}}%
\pgfpathcurveto{\pgfqpoint{3.678279in}{0.811866in}}{\pgfqpoint{3.680045in}{0.812598in}}{\pgfqpoint{3.681348in}{0.813900in}}%
\pgfpathcurveto{\pgfqpoint{3.682650in}{0.815202in}}{\pgfqpoint{3.683382in}{0.816969in}}{\pgfqpoint{3.683382in}{0.818810in}}%
\pgfpathcurveto{\pgfqpoint{3.683382in}{0.820652in}}{\pgfqpoint{3.682650in}{0.822419in}}{\pgfqpoint{3.681348in}{0.823721in}}%
\pgfpathcurveto{\pgfqpoint{3.680045in}{0.825023in}}{\pgfqpoint{3.678279in}{0.825755in}}{\pgfqpoint{3.676437in}{0.825755in}}%
\pgfpathcurveto{\pgfqpoint{3.674596in}{0.825755in}}{\pgfqpoint{3.672829in}{0.825023in}}{\pgfqpoint{3.671527in}{0.823721in}}%
\pgfpathcurveto{\pgfqpoint{3.670224in}{0.822419in}}{\pgfqpoint{3.669493in}{0.820652in}}{\pgfqpoint{3.669493in}{0.818810in}}%
\pgfpathcurveto{\pgfqpoint{3.669493in}{0.816969in}}{\pgfqpoint{3.670224in}{0.815202in}}{\pgfqpoint{3.671527in}{0.813900in}}%
\pgfpathcurveto{\pgfqpoint{3.672829in}{0.812598in}}{\pgfqpoint{3.674596in}{0.811866in}}{\pgfqpoint{3.676437in}{0.811866in}}%
\pgfpathlineto{\pgfqpoint{3.676437in}{0.811866in}}%
\pgfpathclose%
\pgfusepath{stroke,fill}%
\end{pgfscope}%
\begin{pgfscope}%
\pgfpathrectangle{\pgfqpoint{0.661006in}{0.524170in}}{\pgfqpoint{4.194036in}{1.071446in}}%
\pgfusepath{clip}%
\pgfsetbuttcap%
\pgfsetroundjoin%
\definecolor{currentfill}{rgb}{0.342811,0.341790,0.547332}%
\pgfsetfillcolor{currentfill}%
\pgfsetfillopacity{0.700000}%
\pgfsetlinewidth{1.003750pt}%
\definecolor{currentstroke}{rgb}{0.342811,0.341790,0.547332}%
\pgfsetstrokecolor{currentstroke}%
\pgfsetstrokeopacity{0.700000}%
\pgfsetdash{}{0pt}%
\pgfpathmoveto{\pgfqpoint{3.679784in}{0.808576in}}%
\pgfpathcurveto{\pgfqpoint{3.681625in}{0.808576in}}{\pgfqpoint{3.683392in}{0.809308in}}{\pgfqpoint{3.684694in}{0.810610in}}%
\pgfpathcurveto{\pgfqpoint{3.685996in}{0.811912in}}{\pgfqpoint{3.686728in}{0.813679in}}{\pgfqpoint{3.686728in}{0.815521in}}%
\pgfpathcurveto{\pgfqpoint{3.686728in}{0.817362in}}{\pgfqpoint{3.685996in}{0.819129in}}{\pgfqpoint{3.684694in}{0.820431in}}%
\pgfpathcurveto{\pgfqpoint{3.683392in}{0.821733in}}{\pgfqpoint{3.681625in}{0.822465in}}{\pgfqpoint{3.679784in}{0.822465in}}%
\pgfpathcurveto{\pgfqpoint{3.677942in}{0.822465in}}{\pgfqpoint{3.676175in}{0.821733in}}{\pgfqpoint{3.674873in}{0.820431in}}%
\pgfpathcurveto{\pgfqpoint{3.673571in}{0.819129in}}{\pgfqpoint{3.672839in}{0.817362in}}{\pgfqpoint{3.672839in}{0.815521in}}%
\pgfpathcurveto{\pgfqpoint{3.672839in}{0.813679in}}{\pgfqpoint{3.673571in}{0.811912in}}{\pgfqpoint{3.674873in}{0.810610in}}%
\pgfpathcurveto{\pgfqpoint{3.676175in}{0.809308in}}{\pgfqpoint{3.677942in}{0.808576in}}{\pgfqpoint{3.679784in}{0.808576in}}%
\pgfpathlineto{\pgfqpoint{3.679784in}{0.808576in}}%
\pgfpathclose%
\pgfusepath{stroke,fill}%
\end{pgfscope}%
\begin{pgfscope}%
\pgfpathrectangle{\pgfqpoint{0.661006in}{0.524170in}}{\pgfqpoint{4.194036in}{1.071446in}}%
\pgfusepath{clip}%
\pgfsetbuttcap%
\pgfsetroundjoin%
\definecolor{currentfill}{rgb}{0.342811,0.341790,0.547332}%
\pgfsetfillcolor{currentfill}%
\pgfsetfillopacity{0.700000}%
\pgfsetlinewidth{1.003750pt}%
\definecolor{currentstroke}{rgb}{0.342811,0.341790,0.547332}%
\pgfsetstrokecolor{currentstroke}%
\pgfsetstrokeopacity{0.700000}%
\pgfsetdash{}{0pt}%
\pgfpathmoveto{\pgfqpoint{3.692890in}{0.804790in}}%
\pgfpathcurveto{\pgfqpoint{3.694732in}{0.804790in}}{\pgfqpoint{3.696498in}{0.805522in}}{\pgfqpoint{3.697801in}{0.806824in}}%
\pgfpathcurveto{\pgfqpoint{3.699103in}{0.808126in}}{\pgfqpoint{3.699835in}{0.809893in}}{\pgfqpoint{3.699835in}{0.811734in}}%
\pgfpathcurveto{\pgfqpoint{3.699835in}{0.813576in}}{\pgfqpoint{3.699103in}{0.815343in}}{\pgfqpoint{3.697801in}{0.816645in}}%
\pgfpathcurveto{\pgfqpoint{3.696498in}{0.817947in}}{\pgfqpoint{3.694732in}{0.818679in}}{\pgfqpoint{3.692890in}{0.818679in}}%
\pgfpathcurveto{\pgfqpoint{3.691048in}{0.818679in}}{\pgfqpoint{3.689282in}{0.817947in}}{\pgfqpoint{3.687980in}{0.816645in}}%
\pgfpathcurveto{\pgfqpoint{3.686677in}{0.815343in}}{\pgfqpoint{3.685946in}{0.813576in}}{\pgfqpoint{3.685946in}{0.811734in}}%
\pgfpathcurveto{\pgfqpoint{3.685946in}{0.809893in}}{\pgfqpoint{3.686677in}{0.808126in}}{\pgfqpoint{3.687980in}{0.806824in}}%
\pgfpathcurveto{\pgfqpoint{3.689282in}{0.805522in}}{\pgfqpoint{3.691048in}{0.804790in}}{\pgfqpoint{3.692890in}{0.804790in}}%
\pgfpathlineto{\pgfqpoint{3.692890in}{0.804790in}}%
\pgfpathclose%
\pgfusepath{stroke,fill}%
\end{pgfscope}%
\begin{pgfscope}%
\pgfpathrectangle{\pgfqpoint{0.661006in}{0.524170in}}{\pgfqpoint{4.194036in}{1.071446in}}%
\pgfusepath{clip}%
\pgfsetbuttcap%
\pgfsetroundjoin%
\definecolor{currentfill}{rgb}{0.342811,0.341790,0.547332}%
\pgfsetfillcolor{currentfill}%
\pgfsetfillopacity{0.700000}%
\pgfsetlinewidth{1.003750pt}%
\definecolor{currentstroke}{rgb}{0.342811,0.341790,0.547332}%
\pgfsetstrokecolor{currentstroke}%
\pgfsetstrokeopacity{0.700000}%
\pgfsetdash{}{0pt}%
\pgfpathmoveto{\pgfqpoint{3.715710in}{0.799472in}}%
\pgfpathcurveto{\pgfqpoint{3.717552in}{0.799472in}}{\pgfqpoint{3.719319in}{0.800204in}}{\pgfqpoint{3.720621in}{0.801506in}}%
\pgfpathcurveto{\pgfqpoint{3.721923in}{0.802809in}}{\pgfqpoint{3.722655in}{0.804575in}}{\pgfqpoint{3.722655in}{0.806417in}}%
\pgfpathcurveto{\pgfqpoint{3.722655in}{0.808258in}}{\pgfqpoint{3.721923in}{0.810025in}}{\pgfqpoint{3.720621in}{0.811327in}}%
\pgfpathcurveto{\pgfqpoint{3.719319in}{0.812630in}}{\pgfqpoint{3.717552in}{0.813361in}}{\pgfqpoint{3.715710in}{0.813361in}}%
\pgfpathcurveto{\pgfqpoint{3.713869in}{0.813361in}}{\pgfqpoint{3.712102in}{0.812630in}}{\pgfqpoint{3.710800in}{0.811327in}}%
\pgfpathcurveto{\pgfqpoint{3.709498in}{0.810025in}}{\pgfqpoint{3.708766in}{0.808258in}}{\pgfqpoint{3.708766in}{0.806417in}}%
\pgfpathcurveto{\pgfqpoint{3.708766in}{0.804575in}}{\pgfqpoint{3.709498in}{0.802809in}}{\pgfqpoint{3.710800in}{0.801506in}}%
\pgfpathcurveto{\pgfqpoint{3.712102in}{0.800204in}}{\pgfqpoint{3.713869in}{0.799472in}}{\pgfqpoint{3.715710in}{0.799472in}}%
\pgfpathlineto{\pgfqpoint{3.715710in}{0.799472in}}%
\pgfpathclose%
\pgfusepath{stroke,fill}%
\end{pgfscope}%
\begin{pgfscope}%
\pgfpathrectangle{\pgfqpoint{0.661006in}{0.524170in}}{\pgfqpoint{4.194036in}{1.071446in}}%
\pgfusepath{clip}%
\pgfsetbuttcap%
\pgfsetroundjoin%
\definecolor{currentfill}{rgb}{0.342811,0.341790,0.547332}%
\pgfsetfillcolor{currentfill}%
\pgfsetfillopacity{0.700000}%
\pgfsetlinewidth{1.003750pt}%
\definecolor{currentstroke}{rgb}{0.342811,0.341790,0.547332}%
\pgfsetstrokecolor{currentstroke}%
\pgfsetstrokeopacity{0.700000}%
\pgfsetdash{}{0pt}%
\pgfpathmoveto{\pgfqpoint{3.728631in}{0.795722in}}%
\pgfpathcurveto{\pgfqpoint{3.730473in}{0.795722in}}{\pgfqpoint{3.732239in}{0.796454in}}{\pgfqpoint{3.733542in}{0.797756in}}%
\pgfpathcurveto{\pgfqpoint{3.734844in}{0.799058in}}{\pgfqpoint{3.735576in}{0.800825in}}{\pgfqpoint{3.735576in}{0.802666in}}%
\pgfpathcurveto{\pgfqpoint{3.735576in}{0.804508in}}{\pgfqpoint{3.734844in}{0.806275in}}{\pgfqpoint{3.733542in}{0.807577in}}%
\pgfpathcurveto{\pgfqpoint{3.732239in}{0.808879in}}{\pgfqpoint{3.730473in}{0.809611in}}{\pgfqpoint{3.728631in}{0.809611in}}%
\pgfpathcurveto{\pgfqpoint{3.726789in}{0.809611in}}{\pgfqpoint{3.725023in}{0.808879in}}{\pgfqpoint{3.723721in}{0.807577in}}%
\pgfpathcurveto{\pgfqpoint{3.722418in}{0.806275in}}{\pgfqpoint{3.721687in}{0.804508in}}{\pgfqpoint{3.721687in}{0.802666in}}%
\pgfpathcurveto{\pgfqpoint{3.721687in}{0.800825in}}{\pgfqpoint{3.722418in}{0.799058in}}{\pgfqpoint{3.723721in}{0.797756in}}%
\pgfpathcurveto{\pgfqpoint{3.725023in}{0.796454in}}{\pgfqpoint{3.726789in}{0.795722in}}{\pgfqpoint{3.728631in}{0.795722in}}%
\pgfpathlineto{\pgfqpoint{3.728631in}{0.795722in}}%
\pgfpathclose%
\pgfusepath{stroke,fill}%
\end{pgfscope}%
\begin{pgfscope}%
\pgfpathrectangle{\pgfqpoint{0.661006in}{0.524170in}}{\pgfqpoint{4.194036in}{1.071446in}}%
\pgfusepath{clip}%
\pgfsetbuttcap%
\pgfsetroundjoin%
\definecolor{currentfill}{rgb}{0.339999,0.337453,0.542614}%
\pgfsetfillcolor{currentfill}%
\pgfsetfillopacity{0.700000}%
\pgfsetlinewidth{1.003750pt}%
\definecolor{currentstroke}{rgb}{0.339999,0.337453,0.542614}%
\pgfsetstrokecolor{currentstroke}%
\pgfsetstrokeopacity{0.700000}%
\pgfsetdash{}{0pt}%
\pgfpathmoveto{\pgfqpoint{3.719010in}{0.799141in}}%
\pgfpathcurveto{\pgfqpoint{3.720852in}{0.799141in}}{\pgfqpoint{3.722619in}{0.799873in}}{\pgfqpoint{3.723921in}{0.801175in}}%
\pgfpathcurveto{\pgfqpoint{3.725223in}{0.802478in}}{\pgfqpoint{3.725955in}{0.804244in}}{\pgfqpoint{3.725955in}{0.806086in}}%
\pgfpathcurveto{\pgfqpoint{3.725955in}{0.807927in}}{\pgfqpoint{3.725223in}{0.809694in}}{\pgfqpoint{3.723921in}{0.810996in}}%
\pgfpathcurveto{\pgfqpoint{3.722619in}{0.812299in}}{\pgfqpoint{3.720852in}{0.813030in}}{\pgfqpoint{3.719010in}{0.813030in}}%
\pgfpathcurveto{\pgfqpoint{3.717169in}{0.813030in}}{\pgfqpoint{3.715402in}{0.812299in}}{\pgfqpoint{3.714100in}{0.810996in}}%
\pgfpathcurveto{\pgfqpoint{3.712798in}{0.809694in}}{\pgfqpoint{3.712066in}{0.807927in}}{\pgfqpoint{3.712066in}{0.806086in}}%
\pgfpathcurveto{\pgfqpoint{3.712066in}{0.804244in}}{\pgfqpoint{3.712798in}{0.802478in}}{\pgfqpoint{3.714100in}{0.801175in}}%
\pgfpathcurveto{\pgfqpoint{3.715402in}{0.799873in}}{\pgfqpoint{3.717169in}{0.799141in}}{\pgfqpoint{3.719010in}{0.799141in}}%
\pgfpathlineto{\pgfqpoint{3.719010in}{0.799141in}}%
\pgfpathclose%
\pgfusepath{stroke,fill}%
\end{pgfscope}%
\begin{pgfscope}%
\pgfpathrectangle{\pgfqpoint{0.661006in}{0.524170in}}{\pgfqpoint{4.194036in}{1.071446in}}%
\pgfusepath{clip}%
\pgfsetbuttcap%
\pgfsetroundjoin%
\definecolor{currentfill}{rgb}{0.339999,0.337453,0.542614}%
\pgfsetfillcolor{currentfill}%
\pgfsetfillopacity{0.700000}%
\pgfsetlinewidth{1.003750pt}%
\definecolor{currentstroke}{rgb}{0.339999,0.337453,0.542614}%
\pgfsetstrokecolor{currentstroke}%
\pgfsetstrokeopacity{0.700000}%
\pgfsetdash{}{0pt}%
\pgfpathmoveto{\pgfqpoint{3.707112in}{0.800608in}}%
\pgfpathcurveto{\pgfqpoint{3.708954in}{0.800608in}}{\pgfqpoint{3.710720in}{0.801340in}}{\pgfqpoint{3.712023in}{0.802642in}}%
\pgfpathcurveto{\pgfqpoint{3.713325in}{0.803944in}}{\pgfqpoint{3.714057in}{0.805711in}}{\pgfqpoint{3.714057in}{0.807552in}}%
\pgfpathcurveto{\pgfqpoint{3.714057in}{0.809394in}}{\pgfqpoint{3.713325in}{0.811161in}}{\pgfqpoint{3.712023in}{0.812463in}}%
\pgfpathcurveto{\pgfqpoint{3.710720in}{0.813765in}}{\pgfqpoint{3.708954in}{0.814497in}}{\pgfqpoint{3.707112in}{0.814497in}}%
\pgfpathcurveto{\pgfqpoint{3.705271in}{0.814497in}}{\pgfqpoint{3.703504in}{0.813765in}}{\pgfqpoint{3.702202in}{0.812463in}}%
\pgfpathcurveto{\pgfqpoint{3.700899in}{0.811161in}}{\pgfqpoint{3.700168in}{0.809394in}}{\pgfqpoint{3.700168in}{0.807552in}}%
\pgfpathcurveto{\pgfqpoint{3.700168in}{0.805711in}}{\pgfqpoint{3.700899in}{0.803944in}}{\pgfqpoint{3.702202in}{0.802642in}}%
\pgfpathcurveto{\pgfqpoint{3.703504in}{0.801340in}}{\pgfqpoint{3.705271in}{0.800608in}}{\pgfqpoint{3.707112in}{0.800608in}}%
\pgfpathlineto{\pgfqpoint{3.707112in}{0.800608in}}%
\pgfpathclose%
\pgfusepath{stroke,fill}%
\end{pgfscope}%
\begin{pgfscope}%
\pgfpathrectangle{\pgfqpoint{0.661006in}{0.524170in}}{\pgfqpoint{4.194036in}{1.071446in}}%
\pgfusepath{clip}%
\pgfsetbuttcap%
\pgfsetroundjoin%
\definecolor{currentfill}{rgb}{0.337181,0.333124,0.537864}%
\pgfsetfillcolor{currentfill}%
\pgfsetfillopacity{0.700000}%
\pgfsetlinewidth{1.003750pt}%
\definecolor{currentstroke}{rgb}{0.337181,0.333124,0.537864}%
\pgfsetstrokecolor{currentstroke}%
\pgfsetstrokeopacity{0.700000}%
\pgfsetdash{}{0pt}%
\pgfpathmoveto{\pgfqpoint{3.701558in}{0.801971in}}%
\pgfpathcurveto{\pgfqpoint{3.703400in}{0.801971in}}{\pgfqpoint{3.705166in}{0.802703in}}{\pgfqpoint{3.706469in}{0.804005in}}%
\pgfpathcurveto{\pgfqpoint{3.707771in}{0.805307in}}{\pgfqpoint{3.708503in}{0.807074in}}{\pgfqpoint{3.708503in}{0.808915in}}%
\pgfpathcurveto{\pgfqpoint{3.708503in}{0.810757in}}{\pgfqpoint{3.707771in}{0.812524in}}{\pgfqpoint{3.706469in}{0.813826in}}%
\pgfpathcurveto{\pgfqpoint{3.705166in}{0.815128in}}{\pgfqpoint{3.703400in}{0.815860in}}{\pgfqpoint{3.701558in}{0.815860in}}%
\pgfpathcurveto{\pgfqpoint{3.699716in}{0.815860in}}{\pgfqpoint{3.697950in}{0.815128in}}{\pgfqpoint{3.696648in}{0.813826in}}%
\pgfpathcurveto{\pgfqpoint{3.695345in}{0.812524in}}{\pgfqpoint{3.694614in}{0.810757in}}{\pgfqpoint{3.694614in}{0.808915in}}%
\pgfpathcurveto{\pgfqpoint{3.694614in}{0.807074in}}{\pgfqpoint{3.695345in}{0.805307in}}{\pgfqpoint{3.696648in}{0.804005in}}%
\pgfpathcurveto{\pgfqpoint{3.697950in}{0.802703in}}{\pgfqpoint{3.699716in}{0.801971in}}{\pgfqpoint{3.701558in}{0.801971in}}%
\pgfpathlineto{\pgfqpoint{3.701558in}{0.801971in}}%
\pgfpathclose%
\pgfusepath{stroke,fill}%
\end{pgfscope}%
\begin{pgfscope}%
\pgfpathrectangle{\pgfqpoint{0.661006in}{0.524170in}}{\pgfqpoint{4.194036in}{1.071446in}}%
\pgfusepath{clip}%
\pgfsetbuttcap%
\pgfsetroundjoin%
\definecolor{currentfill}{rgb}{0.337181,0.333124,0.537864}%
\pgfsetfillcolor{currentfill}%
\pgfsetfillopacity{0.700000}%
\pgfsetlinewidth{1.003750pt}%
\definecolor{currentstroke}{rgb}{0.337181,0.333124,0.537864}%
\pgfsetstrokecolor{currentstroke}%
\pgfsetstrokeopacity{0.700000}%
\pgfsetdash{}{0pt}%
\pgfpathmoveto{\pgfqpoint{3.684524in}{0.805004in}}%
\pgfpathcurveto{\pgfqpoint{3.686366in}{0.805004in}}{\pgfqpoint{3.688132in}{0.805735in}}{\pgfqpoint{3.689435in}{0.807038in}}%
\pgfpathcurveto{\pgfqpoint{3.690737in}{0.808340in}}{\pgfqpoint{3.691469in}{0.810106in}}{\pgfqpoint{3.691469in}{0.811948in}}%
\pgfpathcurveto{\pgfqpoint{3.691469in}{0.813790in}}{\pgfqpoint{3.690737in}{0.815556in}}{\pgfqpoint{3.689435in}{0.816859in}}%
\pgfpathcurveto{\pgfqpoint{3.688132in}{0.818161in}}{\pgfqpoint{3.686366in}{0.818893in}}{\pgfqpoint{3.684524in}{0.818893in}}%
\pgfpathcurveto{\pgfqpoint{3.682683in}{0.818893in}}{\pgfqpoint{3.680916in}{0.818161in}}{\pgfqpoint{3.679614in}{0.816859in}}%
\pgfpathcurveto{\pgfqpoint{3.678312in}{0.815556in}}{\pgfqpoint{3.677580in}{0.813790in}}{\pgfqpoint{3.677580in}{0.811948in}}%
\pgfpathcurveto{\pgfqpoint{3.677580in}{0.810106in}}{\pgfqpoint{3.678312in}{0.808340in}}{\pgfqpoint{3.679614in}{0.807038in}}%
\pgfpathcurveto{\pgfqpoint{3.680916in}{0.805735in}}{\pgfqpoint{3.682683in}{0.805004in}}{\pgfqpoint{3.684524in}{0.805004in}}%
\pgfpathlineto{\pgfqpoint{3.684524in}{0.805004in}}%
\pgfpathclose%
\pgfusepath{stroke,fill}%
\end{pgfscope}%
\begin{pgfscope}%
\pgfpathrectangle{\pgfqpoint{0.661006in}{0.524170in}}{\pgfqpoint{4.194036in}{1.071446in}}%
\pgfusepath{clip}%
\pgfsetbuttcap%
\pgfsetroundjoin%
\definecolor{currentfill}{rgb}{0.337181,0.333124,0.537864}%
\pgfsetfillcolor{currentfill}%
\pgfsetfillopacity{0.700000}%
\pgfsetlinewidth{1.003750pt}%
\definecolor{currentstroke}{rgb}{0.337181,0.333124,0.537864}%
\pgfsetstrokecolor{currentstroke}%
\pgfsetstrokeopacity{0.700000}%
\pgfsetdash{}{0pt}%
\pgfpathmoveto{\pgfqpoint{3.677832in}{0.807119in}}%
\pgfpathcurveto{\pgfqpoint{3.679673in}{0.807119in}}{\pgfqpoint{3.681440in}{0.807851in}}{\pgfqpoint{3.682742in}{0.809153in}}%
\pgfpathcurveto{\pgfqpoint{3.684044in}{0.810456in}}{\pgfqpoint{3.684776in}{0.812222in}}{\pgfqpoint{3.684776in}{0.814064in}}%
\pgfpathcurveto{\pgfqpoint{3.684776in}{0.815905in}}{\pgfqpoint{3.684044in}{0.817672in}}{\pgfqpoint{3.682742in}{0.818974in}}%
\pgfpathcurveto{\pgfqpoint{3.681440in}{0.820276in}}{\pgfqpoint{3.679673in}{0.821008in}}{\pgfqpoint{3.677832in}{0.821008in}}%
\pgfpathcurveto{\pgfqpoint{3.675990in}{0.821008in}}{\pgfqpoint{3.674223in}{0.820276in}}{\pgfqpoint{3.672921in}{0.818974in}}%
\pgfpathcurveto{\pgfqpoint{3.671619in}{0.817672in}}{\pgfqpoint{3.670887in}{0.815905in}}{\pgfqpoint{3.670887in}{0.814064in}}%
\pgfpathcurveto{\pgfqpoint{3.670887in}{0.812222in}}{\pgfqpoint{3.671619in}{0.810456in}}{\pgfqpoint{3.672921in}{0.809153in}}%
\pgfpathcurveto{\pgfqpoint{3.674223in}{0.807851in}}{\pgfqpoint{3.675990in}{0.807119in}}{\pgfqpoint{3.677832in}{0.807119in}}%
\pgfpathlineto{\pgfqpoint{3.677832in}{0.807119in}}%
\pgfpathclose%
\pgfusepath{stroke,fill}%
\end{pgfscope}%
\begin{pgfscope}%
\pgfpathrectangle{\pgfqpoint{0.661006in}{0.524170in}}{\pgfqpoint{4.194036in}{1.071446in}}%
\pgfusepath{clip}%
\pgfsetbuttcap%
\pgfsetroundjoin%
\definecolor{currentfill}{rgb}{0.337181,0.333124,0.537864}%
\pgfsetfillcolor{currentfill}%
\pgfsetfillopacity{0.700000}%
\pgfsetlinewidth{1.003750pt}%
\definecolor{currentstroke}{rgb}{0.337181,0.333124,0.537864}%
\pgfsetstrokecolor{currentstroke}%
\pgfsetstrokeopacity{0.700000}%
\pgfsetdash{}{0pt}%
\pgfpathmoveto{\pgfqpoint{3.668071in}{0.808004in}}%
\pgfpathcurveto{\pgfqpoint{3.669913in}{0.808004in}}{\pgfqpoint{3.671680in}{0.808736in}}{\pgfqpoint{3.672982in}{0.810038in}}%
\pgfpathcurveto{\pgfqpoint{3.674284in}{0.811340in}}{\pgfqpoint{3.675016in}{0.813107in}}{\pgfqpoint{3.675016in}{0.814949in}}%
\pgfpathcurveto{\pgfqpoint{3.675016in}{0.816790in}}{\pgfqpoint{3.674284in}{0.818557in}}{\pgfqpoint{3.672982in}{0.819859in}}%
\pgfpathcurveto{\pgfqpoint{3.671680in}{0.821161in}}{\pgfqpoint{3.669913in}{0.821893in}}{\pgfqpoint{3.668071in}{0.821893in}}%
\pgfpathcurveto{\pgfqpoint{3.666230in}{0.821893in}}{\pgfqpoint{3.664463in}{0.821161in}}{\pgfqpoint{3.663161in}{0.819859in}}%
\pgfpathcurveto{\pgfqpoint{3.661859in}{0.818557in}}{\pgfqpoint{3.661127in}{0.816790in}}{\pgfqpoint{3.661127in}{0.814949in}}%
\pgfpathcurveto{\pgfqpoint{3.661127in}{0.813107in}}{\pgfqpoint{3.661859in}{0.811340in}}{\pgfqpoint{3.663161in}{0.810038in}}%
\pgfpathcurveto{\pgfqpoint{3.664463in}{0.808736in}}{\pgfqpoint{3.666230in}{0.808004in}}{\pgfqpoint{3.668071in}{0.808004in}}%
\pgfpathlineto{\pgfqpoint{3.668071in}{0.808004in}}%
\pgfpathclose%
\pgfusepath{stroke,fill}%
\end{pgfscope}%
\begin{pgfscope}%
\pgfpathrectangle{\pgfqpoint{0.661006in}{0.524170in}}{\pgfqpoint{4.194036in}{1.071446in}}%
\pgfusepath{clip}%
\pgfsetbuttcap%
\pgfsetroundjoin%
\definecolor{currentfill}{rgb}{0.337181,0.333124,0.537864}%
\pgfsetfillcolor{currentfill}%
\pgfsetfillopacity{0.700000}%
\pgfsetlinewidth{1.003750pt}%
\definecolor{currentstroke}{rgb}{0.337181,0.333124,0.537864}%
\pgfsetstrokecolor{currentstroke}%
\pgfsetstrokeopacity{0.700000}%
\pgfsetdash{}{0pt}%
\pgfpathmoveto{\pgfqpoint{3.655848in}{0.810946in}}%
\pgfpathcurveto{\pgfqpoint{3.657690in}{0.810946in}}{\pgfqpoint{3.659456in}{0.811677in}}{\pgfqpoint{3.660758in}{0.812980in}}%
\pgfpathcurveto{\pgfqpoint{3.662061in}{0.814282in}}{\pgfqpoint{3.662792in}{0.816048in}}{\pgfqpoint{3.662792in}{0.817890in}}%
\pgfpathcurveto{\pgfqpoint{3.662792in}{0.819732in}}{\pgfqpoint{3.662061in}{0.821498in}}{\pgfqpoint{3.660758in}{0.822801in}}%
\pgfpathcurveto{\pgfqpoint{3.659456in}{0.824103in}}{\pgfqpoint{3.657690in}{0.824835in}}{\pgfqpoint{3.655848in}{0.824835in}}%
\pgfpathcurveto{\pgfqpoint{3.654006in}{0.824835in}}{\pgfqpoint{3.652240in}{0.824103in}}{\pgfqpoint{3.650937in}{0.822801in}}%
\pgfpathcurveto{\pgfqpoint{3.649635in}{0.821498in}}{\pgfqpoint{3.648903in}{0.819732in}}{\pgfqpoint{3.648903in}{0.817890in}}%
\pgfpathcurveto{\pgfqpoint{3.648903in}{0.816048in}}{\pgfqpoint{3.649635in}{0.814282in}}{\pgfqpoint{3.650937in}{0.812980in}}%
\pgfpathcurveto{\pgfqpoint{3.652240in}{0.811677in}}{\pgfqpoint{3.654006in}{0.810946in}}{\pgfqpoint{3.655848in}{0.810946in}}%
\pgfpathlineto{\pgfqpoint{3.655848in}{0.810946in}}%
\pgfpathclose%
\pgfusepath{stroke,fill}%
\end{pgfscope}%
\begin{pgfscope}%
\pgfpathrectangle{\pgfqpoint{0.661006in}{0.524170in}}{\pgfqpoint{4.194036in}{1.071446in}}%
\pgfusepath{clip}%
\pgfsetbuttcap%
\pgfsetroundjoin%
\definecolor{currentfill}{rgb}{0.334357,0.328805,0.533083}%
\pgfsetfillcolor{currentfill}%
\pgfsetfillopacity{0.700000}%
\pgfsetlinewidth{1.003750pt}%
\definecolor{currentstroke}{rgb}{0.334357,0.328805,0.533083}%
\pgfsetstrokecolor{currentstroke}%
\pgfsetstrokeopacity{0.700000}%
\pgfsetdash{}{0pt}%
\pgfpathmoveto{\pgfqpoint{3.643671in}{0.813665in}}%
\pgfpathcurveto{\pgfqpoint{3.645512in}{0.813665in}}{\pgfqpoint{3.647279in}{0.814397in}}{\pgfqpoint{3.648581in}{0.815699in}}%
\pgfpathcurveto{\pgfqpoint{3.649884in}{0.817002in}}{\pgfqpoint{3.650615in}{0.818768in}}{\pgfqpoint{3.650615in}{0.820610in}}%
\pgfpathcurveto{\pgfqpoint{3.650615in}{0.822451in}}{\pgfqpoint{3.649884in}{0.824218in}}{\pgfqpoint{3.648581in}{0.825520in}}%
\pgfpathcurveto{\pgfqpoint{3.647279in}{0.826822in}}{\pgfqpoint{3.645512in}{0.827554in}}{\pgfqpoint{3.643671in}{0.827554in}}%
\pgfpathcurveto{\pgfqpoint{3.641829in}{0.827554in}}{\pgfqpoint{3.640063in}{0.826822in}}{\pgfqpoint{3.638760in}{0.825520in}}%
\pgfpathcurveto{\pgfqpoint{3.637458in}{0.824218in}}{\pgfqpoint{3.636726in}{0.822451in}}{\pgfqpoint{3.636726in}{0.820610in}}%
\pgfpathcurveto{\pgfqpoint{3.636726in}{0.818768in}}{\pgfqpoint{3.637458in}{0.817002in}}{\pgfqpoint{3.638760in}{0.815699in}}%
\pgfpathcurveto{\pgfqpoint{3.640063in}{0.814397in}}{\pgfqpoint{3.641829in}{0.813665in}}{\pgfqpoint{3.643671in}{0.813665in}}%
\pgfpathlineto{\pgfqpoint{3.643671in}{0.813665in}}%
\pgfpathclose%
\pgfusepath{stroke,fill}%
\end{pgfscope}%
\begin{pgfscope}%
\pgfpathrectangle{\pgfqpoint{0.661006in}{0.524170in}}{\pgfqpoint{4.194036in}{1.071446in}}%
\pgfusepath{clip}%
\pgfsetbuttcap%
\pgfsetroundjoin%
\definecolor{currentfill}{rgb}{0.334357,0.328805,0.533083}%
\pgfsetfillcolor{currentfill}%
\pgfsetfillopacity{0.700000}%
\pgfsetlinewidth{1.003750pt}%
\definecolor{currentstroke}{rgb}{0.334357,0.328805,0.533083}%
\pgfsetstrokecolor{currentstroke}%
\pgfsetstrokeopacity{0.700000}%
\pgfsetdash{}{0pt}%
\pgfpathmoveto{\pgfqpoint{3.653803in}{0.810340in}}%
\pgfpathcurveto{\pgfqpoint{3.655645in}{0.810340in}}{\pgfqpoint{3.657411in}{0.811072in}}{\pgfqpoint{3.658713in}{0.812374in}}%
\pgfpathcurveto{\pgfqpoint{3.660016in}{0.813676in}}{\pgfqpoint{3.660747in}{0.815443in}}{\pgfqpoint{3.660747in}{0.817284in}}%
\pgfpathcurveto{\pgfqpoint{3.660747in}{0.819126in}}{\pgfqpoint{3.660016in}{0.820892in}}{\pgfqpoint{3.658713in}{0.822195in}}%
\pgfpathcurveto{\pgfqpoint{3.657411in}{0.823497in}}{\pgfqpoint{3.655645in}{0.824229in}}{\pgfqpoint{3.653803in}{0.824229in}}%
\pgfpathcurveto{\pgfqpoint{3.651961in}{0.824229in}}{\pgfqpoint{3.650195in}{0.823497in}}{\pgfqpoint{3.648892in}{0.822195in}}%
\pgfpathcurveto{\pgfqpoint{3.647590in}{0.820892in}}{\pgfqpoint{3.646858in}{0.819126in}}{\pgfqpoint{3.646858in}{0.817284in}}%
\pgfpathcurveto{\pgfqpoint{3.646858in}{0.815443in}}{\pgfqpoint{3.647590in}{0.813676in}}{\pgfqpoint{3.648892in}{0.812374in}}%
\pgfpathcurveto{\pgfqpoint{3.650195in}{0.811072in}}{\pgfqpoint{3.651961in}{0.810340in}}{\pgfqpoint{3.653803in}{0.810340in}}%
\pgfpathlineto{\pgfqpoint{3.653803in}{0.810340in}}%
\pgfpathclose%
\pgfusepath{stroke,fill}%
\end{pgfscope}%
\begin{pgfscope}%
\pgfpathrectangle{\pgfqpoint{0.661006in}{0.524170in}}{\pgfqpoint{4.194036in}{1.071446in}}%
\pgfusepath{clip}%
\pgfsetbuttcap%
\pgfsetroundjoin%
\definecolor{currentfill}{rgb}{0.331526,0.324495,0.528270}%
\pgfsetfillcolor{currentfill}%
\pgfsetfillopacity{0.700000}%
\pgfsetlinewidth{1.003750pt}%
\definecolor{currentstroke}{rgb}{0.331526,0.324495,0.528270}%
\pgfsetstrokecolor{currentstroke}%
\pgfsetstrokeopacity{0.700000}%
\pgfsetdash{}{0pt}%
\pgfpathmoveto{\pgfqpoint{3.685128in}{0.802795in}}%
\pgfpathcurveto{\pgfqpoint{3.686970in}{0.802795in}}{\pgfqpoint{3.688737in}{0.803527in}}{\pgfqpoint{3.690039in}{0.804829in}}%
\pgfpathcurveto{\pgfqpoint{3.691341in}{0.806131in}}{\pgfqpoint{3.692073in}{0.807898in}}{\pgfqpoint{3.692073in}{0.809740in}}%
\pgfpathcurveto{\pgfqpoint{3.692073in}{0.811581in}}{\pgfqpoint{3.691341in}{0.813348in}}{\pgfqpoint{3.690039in}{0.814650in}}%
\pgfpathcurveto{\pgfqpoint{3.688737in}{0.815952in}}{\pgfqpoint{3.686970in}{0.816684in}}{\pgfqpoint{3.685128in}{0.816684in}}%
\pgfpathcurveto{\pgfqpoint{3.683287in}{0.816684in}}{\pgfqpoint{3.681520in}{0.815952in}}{\pgfqpoint{3.680218in}{0.814650in}}%
\pgfpathcurveto{\pgfqpoint{3.678916in}{0.813348in}}{\pgfqpoint{3.678184in}{0.811581in}}{\pgfqpoint{3.678184in}{0.809740in}}%
\pgfpathcurveto{\pgfqpoint{3.678184in}{0.807898in}}{\pgfqpoint{3.678916in}{0.806131in}}{\pgfqpoint{3.680218in}{0.804829in}}%
\pgfpathcurveto{\pgfqpoint{3.681520in}{0.803527in}}{\pgfqpoint{3.683287in}{0.802795in}}{\pgfqpoint{3.685128in}{0.802795in}}%
\pgfpathlineto{\pgfqpoint{3.685128in}{0.802795in}}%
\pgfpathclose%
\pgfusepath{stroke,fill}%
\end{pgfscope}%
\begin{pgfscope}%
\pgfpathrectangle{\pgfqpoint{0.661006in}{0.524170in}}{\pgfqpoint{4.194036in}{1.071446in}}%
\pgfusepath{clip}%
\pgfsetbuttcap%
\pgfsetroundjoin%
\definecolor{currentfill}{rgb}{0.331526,0.324495,0.528270}%
\pgfsetfillcolor{currentfill}%
\pgfsetfillopacity{0.700000}%
\pgfsetlinewidth{1.003750pt}%
\definecolor{currentstroke}{rgb}{0.331526,0.324495,0.528270}%
\pgfsetstrokecolor{currentstroke}%
\pgfsetstrokeopacity{0.700000}%
\pgfsetdash{}{0pt}%
\pgfpathmoveto{\pgfqpoint{3.717709in}{0.795743in}}%
\pgfpathcurveto{\pgfqpoint{3.719551in}{0.795743in}}{\pgfqpoint{3.721317in}{0.796475in}}{\pgfqpoint{3.722619in}{0.797777in}}%
\pgfpathcurveto{\pgfqpoint{3.723922in}{0.799079in}}{\pgfqpoint{3.724653in}{0.800846in}}{\pgfqpoint{3.724653in}{0.802687in}}%
\pgfpathcurveto{\pgfqpoint{3.724653in}{0.804529in}}{\pgfqpoint{3.723922in}{0.806296in}}{\pgfqpoint{3.722619in}{0.807598in}}%
\pgfpathcurveto{\pgfqpoint{3.721317in}{0.808900in}}{\pgfqpoint{3.719551in}{0.809632in}}{\pgfqpoint{3.717709in}{0.809632in}}%
\pgfpathcurveto{\pgfqpoint{3.715867in}{0.809632in}}{\pgfqpoint{3.714101in}{0.808900in}}{\pgfqpoint{3.712799in}{0.807598in}}%
\pgfpathcurveto{\pgfqpoint{3.711496in}{0.806296in}}{\pgfqpoint{3.710765in}{0.804529in}}{\pgfqpoint{3.710765in}{0.802687in}}%
\pgfpathcurveto{\pgfqpoint{3.710765in}{0.800846in}}{\pgfqpoint{3.711496in}{0.799079in}}{\pgfqpoint{3.712799in}{0.797777in}}%
\pgfpathcurveto{\pgfqpoint{3.714101in}{0.796475in}}{\pgfqpoint{3.715867in}{0.795743in}}{\pgfqpoint{3.717709in}{0.795743in}}%
\pgfpathlineto{\pgfqpoint{3.717709in}{0.795743in}}%
\pgfpathclose%
\pgfusepath{stroke,fill}%
\end{pgfscope}%
\begin{pgfscope}%
\pgfpathrectangle{\pgfqpoint{0.661006in}{0.524170in}}{\pgfqpoint{4.194036in}{1.071446in}}%
\pgfusepath{clip}%
\pgfsetbuttcap%
\pgfsetroundjoin%
\definecolor{currentfill}{rgb}{0.328687,0.320195,0.523426}%
\pgfsetfillcolor{currentfill}%
\pgfsetfillopacity{0.700000}%
\pgfsetlinewidth{1.003750pt}%
\definecolor{currentstroke}{rgb}{0.328687,0.320195,0.523426}%
\pgfsetstrokecolor{currentstroke}%
\pgfsetstrokeopacity{0.700000}%
\pgfsetdash{}{0pt}%
\pgfpathmoveto{\pgfqpoint{3.731141in}{0.793019in}}%
\pgfpathcurveto{\pgfqpoint{3.732983in}{0.793019in}}{\pgfqpoint{3.734749in}{0.793751in}}{\pgfqpoint{3.736051in}{0.795053in}}%
\pgfpathcurveto{\pgfqpoint{3.737354in}{0.796355in}}{\pgfqpoint{3.738085in}{0.798122in}}{\pgfqpoint{3.738085in}{0.799963in}}%
\pgfpathcurveto{\pgfqpoint{3.738085in}{0.801805in}}{\pgfqpoint{3.737354in}{0.803572in}}{\pgfqpoint{3.736051in}{0.804874in}}%
\pgfpathcurveto{\pgfqpoint{3.734749in}{0.806176in}}{\pgfqpoint{3.732983in}{0.806908in}}{\pgfqpoint{3.731141in}{0.806908in}}%
\pgfpathcurveto{\pgfqpoint{3.729299in}{0.806908in}}{\pgfqpoint{3.727533in}{0.806176in}}{\pgfqpoint{3.726230in}{0.804874in}}%
\pgfpathcurveto{\pgfqpoint{3.724928in}{0.803572in}}{\pgfqpoint{3.724196in}{0.801805in}}{\pgfqpoint{3.724196in}{0.799963in}}%
\pgfpathcurveto{\pgfqpoint{3.724196in}{0.798122in}}{\pgfqpoint{3.724928in}{0.796355in}}{\pgfqpoint{3.726230in}{0.795053in}}%
\pgfpathcurveto{\pgfqpoint{3.727533in}{0.793751in}}{\pgfqpoint{3.729299in}{0.793019in}}{\pgfqpoint{3.731141in}{0.793019in}}%
\pgfpathlineto{\pgfqpoint{3.731141in}{0.793019in}}%
\pgfpathclose%
\pgfusepath{stroke,fill}%
\end{pgfscope}%
\begin{pgfscope}%
\pgfpathrectangle{\pgfqpoint{0.661006in}{0.524170in}}{\pgfqpoint{4.194036in}{1.071446in}}%
\pgfusepath{clip}%
\pgfsetbuttcap%
\pgfsetroundjoin%
\definecolor{currentfill}{rgb}{0.328687,0.320195,0.523426}%
\pgfsetfillcolor{currentfill}%
\pgfsetfillopacity{0.700000}%
\pgfsetlinewidth{1.003750pt}%
\definecolor{currentstroke}{rgb}{0.328687,0.320195,0.523426}%
\pgfsetstrokecolor{currentstroke}%
\pgfsetstrokeopacity{0.700000}%
\pgfsetdash{}{0pt}%
\pgfpathmoveto{\pgfqpoint{3.715060in}{0.797476in}}%
\pgfpathcurveto{\pgfqpoint{3.716902in}{0.797476in}}{\pgfqpoint{3.718668in}{0.798207in}}{\pgfqpoint{3.719970in}{0.799510in}}%
\pgfpathcurveto{\pgfqpoint{3.721273in}{0.800812in}}{\pgfqpoint{3.722004in}{0.802578in}}{\pgfqpoint{3.722004in}{0.804420in}}%
\pgfpathcurveto{\pgfqpoint{3.722004in}{0.806262in}}{\pgfqpoint{3.721273in}{0.808028in}}{\pgfqpoint{3.719970in}{0.809331in}}%
\pgfpathcurveto{\pgfqpoint{3.718668in}{0.810633in}}{\pgfqpoint{3.716902in}{0.811365in}}{\pgfqpoint{3.715060in}{0.811365in}}%
\pgfpathcurveto{\pgfqpoint{3.713218in}{0.811365in}}{\pgfqpoint{3.711452in}{0.810633in}}{\pgfqpoint{3.710149in}{0.809331in}}%
\pgfpathcurveto{\pgfqpoint{3.708847in}{0.808028in}}{\pgfqpoint{3.708115in}{0.806262in}}{\pgfqpoint{3.708115in}{0.804420in}}%
\pgfpathcurveto{\pgfqpoint{3.708115in}{0.802578in}}{\pgfqpoint{3.708847in}{0.800812in}}{\pgfqpoint{3.710149in}{0.799510in}}%
\pgfpathcurveto{\pgfqpoint{3.711452in}{0.798207in}}{\pgfqpoint{3.713218in}{0.797476in}}{\pgfqpoint{3.715060in}{0.797476in}}%
\pgfpathlineto{\pgfqpoint{3.715060in}{0.797476in}}%
\pgfpathclose%
\pgfusepath{stroke,fill}%
\end{pgfscope}%
\begin{pgfscope}%
\pgfpathrectangle{\pgfqpoint{0.661006in}{0.524170in}}{\pgfqpoint{4.194036in}{1.071446in}}%
\pgfusepath{clip}%
\pgfsetbuttcap%
\pgfsetroundjoin%
\definecolor{currentfill}{rgb}{0.328687,0.320195,0.523426}%
\pgfsetfillcolor{currentfill}%
\pgfsetfillopacity{0.700000}%
\pgfsetlinewidth{1.003750pt}%
\definecolor{currentstroke}{rgb}{0.328687,0.320195,0.523426}%
\pgfsetstrokecolor{currentstroke}%
\pgfsetstrokeopacity{0.700000}%
\pgfsetdash{}{0pt}%
\pgfpathmoveto{\pgfqpoint{3.684757in}{0.803597in}}%
\pgfpathcurveto{\pgfqpoint{3.686598in}{0.803597in}}{\pgfqpoint{3.688365in}{0.804328in}}{\pgfqpoint{3.689667in}{0.805631in}}%
\pgfpathcurveto{\pgfqpoint{3.690969in}{0.806933in}}{\pgfqpoint{3.691701in}{0.808699in}}{\pgfqpoint{3.691701in}{0.810541in}}%
\pgfpathcurveto{\pgfqpoint{3.691701in}{0.812383in}}{\pgfqpoint{3.690969in}{0.814149in}}{\pgfqpoint{3.689667in}{0.815452in}}%
\pgfpathcurveto{\pgfqpoint{3.688365in}{0.816754in}}{\pgfqpoint{3.686598in}{0.817486in}}{\pgfqpoint{3.684757in}{0.817486in}}%
\pgfpathcurveto{\pgfqpoint{3.682915in}{0.817486in}}{\pgfqpoint{3.681148in}{0.816754in}}{\pgfqpoint{3.679846in}{0.815452in}}%
\pgfpathcurveto{\pgfqpoint{3.678544in}{0.814149in}}{\pgfqpoint{3.677812in}{0.812383in}}{\pgfqpoint{3.677812in}{0.810541in}}%
\pgfpathcurveto{\pgfqpoint{3.677812in}{0.808699in}}{\pgfqpoint{3.678544in}{0.806933in}}{\pgfqpoint{3.679846in}{0.805631in}}%
\pgfpathcurveto{\pgfqpoint{3.681148in}{0.804328in}}{\pgfqpoint{3.682915in}{0.803597in}}{\pgfqpoint{3.684757in}{0.803597in}}%
\pgfpathlineto{\pgfqpoint{3.684757in}{0.803597in}}%
\pgfpathclose%
\pgfusepath{stroke,fill}%
\end{pgfscope}%
\begin{pgfscope}%
\pgfpathrectangle{\pgfqpoint{0.661006in}{0.524170in}}{\pgfqpoint{4.194036in}{1.071446in}}%
\pgfusepath{clip}%
\pgfsetbuttcap%
\pgfsetroundjoin%
\definecolor{currentfill}{rgb}{0.328687,0.320195,0.523426}%
\pgfsetfillcolor{currentfill}%
\pgfsetfillopacity{0.700000}%
\pgfsetlinewidth{1.003750pt}%
\definecolor{currentstroke}{rgb}{0.328687,0.320195,0.523426}%
\pgfsetstrokecolor{currentstroke}%
\pgfsetstrokeopacity{0.700000}%
\pgfsetdash{}{0pt}%
\pgfpathmoveto{\pgfqpoint{3.667700in}{0.806938in}}%
\pgfpathcurveto{\pgfqpoint{3.669541in}{0.806938in}}{\pgfqpoint{3.671308in}{0.807669in}}{\pgfqpoint{3.672610in}{0.808972in}}%
\pgfpathcurveto{\pgfqpoint{3.673912in}{0.810274in}}{\pgfqpoint{3.674644in}{0.812040in}}{\pgfqpoint{3.674644in}{0.813882in}}%
\pgfpathcurveto{\pgfqpoint{3.674644in}{0.815724in}}{\pgfqpoint{3.673912in}{0.817490in}}{\pgfqpoint{3.672610in}{0.818793in}}%
\pgfpathcurveto{\pgfqpoint{3.671308in}{0.820095in}}{\pgfqpoint{3.669541in}{0.820827in}}{\pgfqpoint{3.667700in}{0.820827in}}%
\pgfpathcurveto{\pgfqpoint{3.665858in}{0.820827in}}{\pgfqpoint{3.664091in}{0.820095in}}{\pgfqpoint{3.662789in}{0.818793in}}%
\pgfpathcurveto{\pgfqpoint{3.661487in}{0.817490in}}{\pgfqpoint{3.660755in}{0.815724in}}{\pgfqpoint{3.660755in}{0.813882in}}%
\pgfpathcurveto{\pgfqpoint{3.660755in}{0.812040in}}{\pgfqpoint{3.661487in}{0.810274in}}{\pgfqpoint{3.662789in}{0.808972in}}%
\pgfpathcurveto{\pgfqpoint{3.664091in}{0.807669in}}{\pgfqpoint{3.665858in}{0.806938in}}{\pgfqpoint{3.667700in}{0.806938in}}%
\pgfpathlineto{\pgfqpoint{3.667700in}{0.806938in}}%
\pgfpathclose%
\pgfusepath{stroke,fill}%
\end{pgfscope}%
\begin{pgfscope}%
\pgfpathrectangle{\pgfqpoint{0.661006in}{0.524170in}}{\pgfqpoint{4.194036in}{1.071446in}}%
\pgfusepath{clip}%
\pgfsetbuttcap%
\pgfsetroundjoin%
\definecolor{currentfill}{rgb}{0.328687,0.320195,0.523426}%
\pgfsetfillcolor{currentfill}%
\pgfsetfillopacity{0.700000}%
\pgfsetlinewidth{1.003750pt}%
\definecolor{currentstroke}{rgb}{0.328687,0.320195,0.523426}%
\pgfsetstrokecolor{currentstroke}%
\pgfsetstrokeopacity{0.700000}%
\pgfsetdash{}{0pt}%
\pgfpathmoveto{\pgfqpoint{3.670830in}{0.807241in}}%
\pgfpathcurveto{\pgfqpoint{3.672672in}{0.807241in}}{\pgfqpoint{3.674438in}{0.807972in}}{\pgfqpoint{3.675741in}{0.809275in}}%
\pgfpathcurveto{\pgfqpoint{3.677043in}{0.810577in}}{\pgfqpoint{3.677775in}{0.812343in}}{\pgfqpoint{3.677775in}{0.814185in}}%
\pgfpathcurveto{\pgfqpoint{3.677775in}{0.816027in}}{\pgfqpoint{3.677043in}{0.817793in}}{\pgfqpoint{3.675741in}{0.819096in}}%
\pgfpathcurveto{\pgfqpoint{3.674438in}{0.820398in}}{\pgfqpoint{3.672672in}{0.821130in}}{\pgfqpoint{3.670830in}{0.821130in}}%
\pgfpathcurveto{\pgfqpoint{3.668988in}{0.821130in}}{\pgfqpoint{3.667222in}{0.820398in}}{\pgfqpoint{3.665920in}{0.819096in}}%
\pgfpathcurveto{\pgfqpoint{3.664617in}{0.817793in}}{\pgfqpoint{3.663886in}{0.816027in}}{\pgfqpoint{3.663886in}{0.814185in}}%
\pgfpathcurveto{\pgfqpoint{3.663886in}{0.812343in}}{\pgfqpoint{3.664617in}{0.810577in}}{\pgfqpoint{3.665920in}{0.809275in}}%
\pgfpathcurveto{\pgfqpoint{3.667222in}{0.807972in}}{\pgfqpoint{3.668988in}{0.807241in}}{\pgfqpoint{3.670830in}{0.807241in}}%
\pgfpathlineto{\pgfqpoint{3.670830in}{0.807241in}}%
\pgfpathclose%
\pgfusepath{stroke,fill}%
\end{pgfscope}%
\begin{pgfscope}%
\pgfpathrectangle{\pgfqpoint{0.661006in}{0.524170in}}{\pgfqpoint{4.194036in}{1.071446in}}%
\pgfusepath{clip}%
\pgfsetbuttcap%
\pgfsetroundjoin%
\definecolor{currentfill}{rgb}{0.325841,0.315905,0.518550}%
\pgfsetfillcolor{currentfill}%
\pgfsetfillopacity{0.700000}%
\pgfsetlinewidth{1.003750pt}%
\definecolor{currentstroke}{rgb}{0.325841,0.315905,0.518550}%
\pgfsetstrokecolor{currentstroke}%
\pgfsetstrokeopacity{0.700000}%
\pgfsetdash{}{0pt}%
\pgfpathmoveto{\pgfqpoint{3.693494in}{0.803053in}}%
\pgfpathcurveto{\pgfqpoint{3.695336in}{0.803053in}}{\pgfqpoint{3.697103in}{0.803784in}}{\pgfqpoint{3.698405in}{0.805087in}}%
\pgfpathcurveto{\pgfqpoint{3.699707in}{0.806389in}}{\pgfqpoint{3.700439in}{0.808155in}}{\pgfqpoint{3.700439in}{0.809997in}}%
\pgfpathcurveto{\pgfqpoint{3.700439in}{0.811839in}}{\pgfqpoint{3.699707in}{0.813605in}}{\pgfqpoint{3.698405in}{0.814908in}}%
\pgfpathcurveto{\pgfqpoint{3.697103in}{0.816210in}}{\pgfqpoint{3.695336in}{0.816942in}}{\pgfqpoint{3.693494in}{0.816942in}}%
\pgfpathcurveto{\pgfqpoint{3.691653in}{0.816942in}}{\pgfqpoint{3.689886in}{0.816210in}}{\pgfqpoint{3.688584in}{0.814908in}}%
\pgfpathcurveto{\pgfqpoint{3.687282in}{0.813605in}}{\pgfqpoint{3.686550in}{0.811839in}}{\pgfqpoint{3.686550in}{0.809997in}}%
\pgfpathcurveto{\pgfqpoint{3.686550in}{0.808155in}}{\pgfqpoint{3.687282in}{0.806389in}}{\pgfqpoint{3.688584in}{0.805087in}}%
\pgfpathcurveto{\pgfqpoint{3.689886in}{0.803784in}}{\pgfqpoint{3.691653in}{0.803053in}}{\pgfqpoint{3.693494in}{0.803053in}}%
\pgfpathlineto{\pgfqpoint{3.693494in}{0.803053in}}%
\pgfpathclose%
\pgfusepath{stroke,fill}%
\end{pgfscope}%
\begin{pgfscope}%
\pgfpathrectangle{\pgfqpoint{0.661006in}{0.524170in}}{\pgfqpoint{4.194036in}{1.071446in}}%
\pgfusepath{clip}%
\pgfsetbuttcap%
\pgfsetroundjoin%
\definecolor{currentfill}{rgb}{0.325841,0.315905,0.518550}%
\pgfsetfillcolor{currentfill}%
\pgfsetfillopacity{0.700000}%
\pgfsetlinewidth{1.003750pt}%
\definecolor{currentstroke}{rgb}{0.325841,0.315905,0.518550}%
\pgfsetstrokecolor{currentstroke}%
\pgfsetstrokeopacity{0.700000}%
\pgfsetdash{}{0pt}%
\pgfpathmoveto{\pgfqpoint{3.708274in}{0.797278in}}%
\pgfpathcurveto{\pgfqpoint{3.710116in}{0.797278in}}{\pgfqpoint{3.711882in}{0.798010in}}{\pgfqpoint{3.713185in}{0.799312in}}%
\pgfpathcurveto{\pgfqpoint{3.714487in}{0.800614in}}{\pgfqpoint{3.715219in}{0.802381in}}{\pgfqpoint{3.715219in}{0.804223in}}%
\pgfpathcurveto{\pgfqpoint{3.715219in}{0.806064in}}{\pgfqpoint{3.714487in}{0.807831in}}{\pgfqpoint{3.713185in}{0.809133in}}%
\pgfpathcurveto{\pgfqpoint{3.711882in}{0.810435in}}{\pgfqpoint{3.710116in}{0.811167in}}{\pgfqpoint{3.708274in}{0.811167in}}%
\pgfpathcurveto{\pgfqpoint{3.706432in}{0.811167in}}{\pgfqpoint{3.704666in}{0.810435in}}{\pgfqpoint{3.703364in}{0.809133in}}%
\pgfpathcurveto{\pgfqpoint{3.702061in}{0.807831in}}{\pgfqpoint{3.701330in}{0.806064in}}{\pgfqpoint{3.701330in}{0.804223in}}%
\pgfpathcurveto{\pgfqpoint{3.701330in}{0.802381in}}{\pgfqpoint{3.702061in}{0.800614in}}{\pgfqpoint{3.703364in}{0.799312in}}%
\pgfpathcurveto{\pgfqpoint{3.704666in}{0.798010in}}{\pgfqpoint{3.706432in}{0.797278in}}{\pgfqpoint{3.708274in}{0.797278in}}%
\pgfpathlineto{\pgfqpoint{3.708274in}{0.797278in}}%
\pgfpathclose%
\pgfusepath{stroke,fill}%
\end{pgfscope}%
\begin{pgfscope}%
\pgfpathrectangle{\pgfqpoint{0.661006in}{0.524170in}}{\pgfqpoint{4.194036in}{1.071446in}}%
\pgfusepath{clip}%
\pgfsetbuttcap%
\pgfsetroundjoin%
\definecolor{currentfill}{rgb}{0.322987,0.311624,0.513643}%
\pgfsetfillcolor{currentfill}%
\pgfsetfillopacity{0.700000}%
\pgfsetlinewidth{1.003750pt}%
\definecolor{currentstroke}{rgb}{0.322987,0.311624,0.513643}%
\pgfsetstrokecolor{currentstroke}%
\pgfsetstrokeopacity{0.700000}%
\pgfsetdash{}{0pt}%
\pgfpathmoveto{\pgfqpoint{3.723983in}{0.794377in}}%
\pgfpathcurveto{\pgfqpoint{3.725825in}{0.794377in}}{\pgfqpoint{3.727592in}{0.795109in}}{\pgfqpoint{3.728894in}{0.796411in}}%
\pgfpathcurveto{\pgfqpoint{3.730196in}{0.797714in}}{\pgfqpoint{3.730928in}{0.799480in}}{\pgfqpoint{3.730928in}{0.801322in}}%
\pgfpathcurveto{\pgfqpoint{3.730928in}{0.803163in}}{\pgfqpoint{3.730196in}{0.804930in}}{\pgfqpoint{3.728894in}{0.806232in}}%
\pgfpathcurveto{\pgfqpoint{3.727592in}{0.807534in}}{\pgfqpoint{3.725825in}{0.808266in}}{\pgfqpoint{3.723983in}{0.808266in}}%
\pgfpathcurveto{\pgfqpoint{3.722142in}{0.808266in}}{\pgfqpoint{3.720375in}{0.807534in}}{\pgfqpoint{3.719073in}{0.806232in}}%
\pgfpathcurveto{\pgfqpoint{3.717771in}{0.804930in}}{\pgfqpoint{3.717039in}{0.803163in}}{\pgfqpoint{3.717039in}{0.801322in}}%
\pgfpathcurveto{\pgfqpoint{3.717039in}{0.799480in}}{\pgfqpoint{3.717771in}{0.797714in}}{\pgfqpoint{3.719073in}{0.796411in}}%
\pgfpathcurveto{\pgfqpoint{3.720375in}{0.795109in}}{\pgfqpoint{3.722142in}{0.794377in}}{\pgfqpoint{3.723983in}{0.794377in}}%
\pgfpathlineto{\pgfqpoint{3.723983in}{0.794377in}}%
\pgfpathclose%
\pgfusepath{stroke,fill}%
\end{pgfscope}%
\begin{pgfscope}%
\pgfpathrectangle{\pgfqpoint{0.661006in}{0.524170in}}{\pgfqpoint{4.194036in}{1.071446in}}%
\pgfusepath{clip}%
\pgfsetbuttcap%
\pgfsetroundjoin%
\definecolor{currentfill}{rgb}{0.322987,0.311624,0.513643}%
\pgfsetfillcolor{currentfill}%
\pgfsetfillopacity{0.700000}%
\pgfsetlinewidth{1.003750pt}%
\definecolor{currentstroke}{rgb}{0.322987,0.311624,0.513643}%
\pgfsetstrokecolor{currentstroke}%
\pgfsetstrokeopacity{0.700000}%
\pgfsetdash{}{0pt}%
\pgfpathmoveto{\pgfqpoint{3.728352in}{0.792729in}}%
\pgfpathcurveto{\pgfqpoint{3.730194in}{0.792729in}}{\pgfqpoint{3.731961in}{0.793461in}}{\pgfqpoint{3.733263in}{0.794763in}}%
\pgfpathcurveto{\pgfqpoint{3.734565in}{0.796065in}}{\pgfqpoint{3.735297in}{0.797832in}}{\pgfqpoint{3.735297in}{0.799673in}}%
\pgfpathcurveto{\pgfqpoint{3.735297in}{0.801515in}}{\pgfqpoint{3.734565in}{0.803282in}}{\pgfqpoint{3.733263in}{0.804584in}}%
\pgfpathcurveto{\pgfqpoint{3.731961in}{0.805886in}}{\pgfqpoint{3.730194in}{0.806618in}}{\pgfqpoint{3.728352in}{0.806618in}}%
\pgfpathcurveto{\pgfqpoint{3.726511in}{0.806618in}}{\pgfqpoint{3.724744in}{0.805886in}}{\pgfqpoint{3.723442in}{0.804584in}}%
\pgfpathcurveto{\pgfqpoint{3.722140in}{0.803282in}}{\pgfqpoint{3.721408in}{0.801515in}}{\pgfqpoint{3.721408in}{0.799673in}}%
\pgfpathcurveto{\pgfqpoint{3.721408in}{0.797832in}}{\pgfqpoint{3.722140in}{0.796065in}}{\pgfqpoint{3.723442in}{0.794763in}}%
\pgfpathcurveto{\pgfqpoint{3.724744in}{0.793461in}}{\pgfqpoint{3.726511in}{0.792729in}}{\pgfqpoint{3.728352in}{0.792729in}}%
\pgfpathlineto{\pgfqpoint{3.728352in}{0.792729in}}%
\pgfpathclose%
\pgfusepath{stroke,fill}%
\end{pgfscope}%
\begin{pgfscope}%
\pgfpathrectangle{\pgfqpoint{0.661006in}{0.524170in}}{\pgfqpoint{4.194036in}{1.071446in}}%
\pgfusepath{clip}%
\pgfsetbuttcap%
\pgfsetroundjoin%
\definecolor{currentfill}{rgb}{0.322987,0.311624,0.513643}%
\pgfsetfillcolor{currentfill}%
\pgfsetfillopacity{0.700000}%
\pgfsetlinewidth{1.003750pt}%
\definecolor{currentstroke}{rgb}{0.322987,0.311624,0.513643}%
\pgfsetstrokecolor{currentstroke}%
\pgfsetstrokeopacity{0.700000}%
\pgfsetdash{}{0pt}%
\pgfpathmoveto{\pgfqpoint{3.746897in}{0.790584in}}%
\pgfpathcurveto{\pgfqpoint{3.748738in}{0.790584in}}{\pgfqpoint{3.750505in}{0.791316in}}{\pgfqpoint{3.751807in}{0.792618in}}%
\pgfpathcurveto{\pgfqpoint{3.753109in}{0.793920in}}{\pgfqpoint{3.753841in}{0.795687in}}{\pgfqpoint{3.753841in}{0.797528in}}%
\pgfpathcurveto{\pgfqpoint{3.753841in}{0.799370in}}{\pgfqpoint{3.753109in}{0.801137in}}{\pgfqpoint{3.751807in}{0.802439in}}%
\pgfpathcurveto{\pgfqpoint{3.750505in}{0.803741in}}{\pgfqpoint{3.748738in}{0.804473in}}{\pgfqpoint{3.746897in}{0.804473in}}%
\pgfpathcurveto{\pgfqpoint{3.745055in}{0.804473in}}{\pgfqpoint{3.743289in}{0.803741in}}{\pgfqpoint{3.741986in}{0.802439in}}%
\pgfpathcurveto{\pgfqpoint{3.740684in}{0.801137in}}{\pgfqpoint{3.739952in}{0.799370in}}{\pgfqpoint{3.739952in}{0.797528in}}%
\pgfpathcurveto{\pgfqpoint{3.739952in}{0.795687in}}{\pgfqpoint{3.740684in}{0.793920in}}{\pgfqpoint{3.741986in}{0.792618in}}%
\pgfpathcurveto{\pgfqpoint{3.743289in}{0.791316in}}{\pgfqpoint{3.745055in}{0.790584in}}{\pgfqpoint{3.746897in}{0.790584in}}%
\pgfpathlineto{\pgfqpoint{3.746897in}{0.790584in}}%
\pgfpathclose%
\pgfusepath{stroke,fill}%
\end{pgfscope}%
\begin{pgfscope}%
\pgfpathrectangle{\pgfqpoint{0.661006in}{0.524170in}}{\pgfqpoint{4.194036in}{1.071446in}}%
\pgfusepath{clip}%
\pgfsetbuttcap%
\pgfsetroundjoin%
\definecolor{currentfill}{rgb}{0.322987,0.311624,0.513643}%
\pgfsetfillcolor{currentfill}%
\pgfsetfillopacity{0.700000}%
\pgfsetlinewidth{1.003750pt}%
\definecolor{currentstroke}{rgb}{0.322987,0.311624,0.513643}%
\pgfsetstrokecolor{currentstroke}%
\pgfsetstrokeopacity{0.700000}%
\pgfsetdash{}{0pt}%
\pgfpathmoveto{\pgfqpoint{3.743504in}{0.790093in}}%
\pgfpathcurveto{\pgfqpoint{3.745346in}{0.790093in}}{\pgfqpoint{3.747112in}{0.790825in}}{\pgfqpoint{3.748414in}{0.792127in}}%
\pgfpathcurveto{\pgfqpoint{3.749717in}{0.793429in}}{\pgfqpoint{3.750448in}{0.795196in}}{\pgfqpoint{3.750448in}{0.797037in}}%
\pgfpathcurveto{\pgfqpoint{3.750448in}{0.798879in}}{\pgfqpoint{3.749717in}{0.800646in}}{\pgfqpoint{3.748414in}{0.801948in}}%
\pgfpathcurveto{\pgfqpoint{3.747112in}{0.803250in}}{\pgfqpoint{3.745346in}{0.803982in}}{\pgfqpoint{3.743504in}{0.803982in}}%
\pgfpathcurveto{\pgfqpoint{3.741662in}{0.803982in}}{\pgfqpoint{3.739896in}{0.803250in}}{\pgfqpoint{3.738593in}{0.801948in}}%
\pgfpathcurveto{\pgfqpoint{3.737291in}{0.800646in}}{\pgfqpoint{3.736559in}{0.798879in}}{\pgfqpoint{3.736559in}{0.797037in}}%
\pgfpathcurveto{\pgfqpoint{3.736559in}{0.795196in}}{\pgfqpoint{3.737291in}{0.793429in}}{\pgfqpoint{3.738593in}{0.792127in}}%
\pgfpathcurveto{\pgfqpoint{3.739896in}{0.790825in}}{\pgfqpoint{3.741662in}{0.790093in}}{\pgfqpoint{3.743504in}{0.790093in}}%
\pgfpathlineto{\pgfqpoint{3.743504in}{0.790093in}}%
\pgfpathclose%
\pgfusepath{stroke,fill}%
\end{pgfscope}%
\begin{pgfscope}%
\pgfpathrectangle{\pgfqpoint{0.661006in}{0.524170in}}{\pgfqpoint{4.194036in}{1.071446in}}%
\pgfusepath{clip}%
\pgfsetbuttcap%
\pgfsetroundjoin%
\definecolor{currentfill}{rgb}{0.322987,0.311624,0.513643}%
\pgfsetfillcolor{currentfill}%
\pgfsetfillopacity{0.700000}%
\pgfsetlinewidth{1.003750pt}%
\definecolor{currentstroke}{rgb}{0.322987,0.311624,0.513643}%
\pgfsetstrokecolor{currentstroke}%
\pgfsetstrokeopacity{0.700000}%
\pgfsetdash{}{0pt}%
\pgfpathmoveto{\pgfqpoint{3.761862in}{0.787135in}}%
\pgfpathcurveto{\pgfqpoint{3.763704in}{0.787135in}}{\pgfqpoint{3.765471in}{0.787867in}}{\pgfqpoint{3.766773in}{0.789169in}}%
\pgfpathcurveto{\pgfqpoint{3.768075in}{0.790471in}}{\pgfqpoint{3.768807in}{0.792238in}}{\pgfqpoint{3.768807in}{0.794079in}}%
\pgfpathcurveto{\pgfqpoint{3.768807in}{0.795921in}}{\pgfqpoint{3.768075in}{0.797687in}}{\pgfqpoint{3.766773in}{0.798990in}}%
\pgfpathcurveto{\pgfqpoint{3.765471in}{0.800292in}}{\pgfqpoint{3.763704in}{0.801024in}}{\pgfqpoint{3.761862in}{0.801024in}}%
\pgfpathcurveto{\pgfqpoint{3.760021in}{0.801024in}}{\pgfqpoint{3.758254in}{0.800292in}}{\pgfqpoint{3.756952in}{0.798990in}}%
\pgfpathcurveto{\pgfqpoint{3.755650in}{0.797687in}}{\pgfqpoint{3.754918in}{0.795921in}}{\pgfqpoint{3.754918in}{0.794079in}}%
\pgfpathcurveto{\pgfqpoint{3.754918in}{0.792238in}}{\pgfqpoint{3.755650in}{0.790471in}}{\pgfqpoint{3.756952in}{0.789169in}}%
\pgfpathcurveto{\pgfqpoint{3.758254in}{0.787867in}}{\pgfqpoint{3.760021in}{0.787135in}}{\pgfqpoint{3.761862in}{0.787135in}}%
\pgfpathlineto{\pgfqpoint{3.761862in}{0.787135in}}%
\pgfpathclose%
\pgfusepath{stroke,fill}%
\end{pgfscope}%
\begin{pgfscope}%
\pgfpathrectangle{\pgfqpoint{0.661006in}{0.524170in}}{\pgfqpoint{4.194036in}{1.071446in}}%
\pgfusepath{clip}%
\pgfsetbuttcap%
\pgfsetroundjoin%
\definecolor{currentfill}{rgb}{0.320124,0.307354,0.508706}%
\pgfsetfillcolor{currentfill}%
\pgfsetfillopacity{0.700000}%
\pgfsetlinewidth{1.003750pt}%
\definecolor{currentstroke}{rgb}{0.320124,0.307354,0.508706}%
\pgfsetstrokecolor{currentstroke}%
\pgfsetstrokeopacity{0.700000}%
\pgfsetdash{}{0pt}%
\pgfpathmoveto{\pgfqpoint{3.778780in}{0.783280in}}%
\pgfpathcurveto{\pgfqpoint{3.780622in}{0.783280in}}{\pgfqpoint{3.782388in}{0.784012in}}{\pgfqpoint{3.783691in}{0.785314in}}%
\pgfpathcurveto{\pgfqpoint{3.784993in}{0.786616in}}{\pgfqpoint{3.785725in}{0.788383in}}{\pgfqpoint{3.785725in}{0.790224in}}%
\pgfpathcurveto{\pgfqpoint{3.785725in}{0.792066in}}{\pgfqpoint{3.784993in}{0.793832in}}{\pgfqpoint{3.783691in}{0.795135in}}%
\pgfpathcurveto{\pgfqpoint{3.782388in}{0.796437in}}{\pgfqpoint{3.780622in}{0.797169in}}{\pgfqpoint{3.778780in}{0.797169in}}%
\pgfpathcurveto{\pgfqpoint{3.776938in}{0.797169in}}{\pgfqpoint{3.775172in}{0.796437in}}{\pgfqpoint{3.773870in}{0.795135in}}%
\pgfpathcurveto{\pgfqpoint{3.772567in}{0.793832in}}{\pgfqpoint{3.771836in}{0.792066in}}{\pgfqpoint{3.771836in}{0.790224in}}%
\pgfpathcurveto{\pgfqpoint{3.771836in}{0.788383in}}{\pgfqpoint{3.772567in}{0.786616in}}{\pgfqpoint{3.773870in}{0.785314in}}%
\pgfpathcurveto{\pgfqpoint{3.775172in}{0.784012in}}{\pgfqpoint{3.776938in}{0.783280in}}{\pgfqpoint{3.778780in}{0.783280in}}%
\pgfpathlineto{\pgfqpoint{3.778780in}{0.783280in}}%
\pgfpathclose%
\pgfusepath{stroke,fill}%
\end{pgfscope}%
\begin{pgfscope}%
\pgfpathrectangle{\pgfqpoint{0.661006in}{0.524170in}}{\pgfqpoint{4.194036in}{1.071446in}}%
\pgfusepath{clip}%
\pgfsetbuttcap%
\pgfsetroundjoin%
\definecolor{currentfill}{rgb}{0.320124,0.307354,0.508706}%
\pgfsetfillcolor{currentfill}%
\pgfsetfillopacity{0.700000}%
\pgfsetlinewidth{1.003750pt}%
\definecolor{currentstroke}{rgb}{0.320124,0.307354,0.508706}%
\pgfsetstrokecolor{currentstroke}%
\pgfsetstrokeopacity{0.700000}%
\pgfsetdash{}{0pt}%
\pgfpathmoveto{\pgfqpoint{3.796070in}{0.779246in}}%
\pgfpathcurveto{\pgfqpoint{3.797911in}{0.779246in}}{\pgfqpoint{3.799678in}{0.779978in}}{\pgfqpoint{3.800980in}{0.781280in}}%
\pgfpathcurveto{\pgfqpoint{3.802282in}{0.782582in}}{\pgfqpoint{3.803014in}{0.784349in}}{\pgfqpoint{3.803014in}{0.786190in}}%
\pgfpathcurveto{\pgfqpoint{3.803014in}{0.788032in}}{\pgfqpoint{3.802282in}{0.789799in}}{\pgfqpoint{3.800980in}{0.791101in}}%
\pgfpathcurveto{\pgfqpoint{3.799678in}{0.792403in}}{\pgfqpoint{3.797911in}{0.793135in}}{\pgfqpoint{3.796070in}{0.793135in}}%
\pgfpathcurveto{\pgfqpoint{3.794228in}{0.793135in}}{\pgfqpoint{3.792461in}{0.792403in}}{\pgfqpoint{3.791159in}{0.791101in}}%
\pgfpathcurveto{\pgfqpoint{3.789857in}{0.789799in}}{\pgfqpoint{3.789125in}{0.788032in}}{\pgfqpoint{3.789125in}{0.786190in}}%
\pgfpathcurveto{\pgfqpoint{3.789125in}{0.784349in}}{\pgfqpoint{3.789857in}{0.782582in}}{\pgfqpoint{3.791159in}{0.781280in}}%
\pgfpathcurveto{\pgfqpoint{3.792461in}{0.779978in}}{\pgfqpoint{3.794228in}{0.779246in}}{\pgfqpoint{3.796070in}{0.779246in}}%
\pgfpathlineto{\pgfqpoint{3.796070in}{0.779246in}}%
\pgfpathclose%
\pgfusepath{stroke,fill}%
\end{pgfscope}%
\begin{pgfscope}%
\pgfpathrectangle{\pgfqpoint{0.661006in}{0.524170in}}{\pgfqpoint{4.194036in}{1.071446in}}%
\pgfusepath{clip}%
\pgfsetbuttcap%
\pgfsetroundjoin%
\definecolor{currentfill}{rgb}{0.317251,0.303094,0.503737}%
\pgfsetfillcolor{currentfill}%
\pgfsetfillopacity{0.700000}%
\pgfsetlinewidth{1.003750pt}%
\definecolor{currentstroke}{rgb}{0.317251,0.303094,0.503737}%
\pgfsetstrokecolor{currentstroke}%
\pgfsetstrokeopacity{0.700000}%
\pgfsetdash{}{0pt}%
\pgfpathmoveto{\pgfqpoint{3.802158in}{0.777131in}}%
\pgfpathcurveto{\pgfqpoint{3.804000in}{0.777131in}}{\pgfqpoint{3.805766in}{0.777862in}}{\pgfqpoint{3.807069in}{0.779165in}}%
\pgfpathcurveto{\pgfqpoint{3.808371in}{0.780467in}}{\pgfqpoint{3.809103in}{0.782233in}}{\pgfqpoint{3.809103in}{0.784075in}}%
\pgfpathcurveto{\pgfqpoint{3.809103in}{0.785917in}}{\pgfqpoint{3.808371in}{0.787683in}}{\pgfqpoint{3.807069in}{0.788986in}}%
\pgfpathcurveto{\pgfqpoint{3.805766in}{0.790288in}}{\pgfqpoint{3.804000in}{0.791020in}}{\pgfqpoint{3.802158in}{0.791020in}}%
\pgfpathcurveto{\pgfqpoint{3.800316in}{0.791020in}}{\pgfqpoint{3.798550in}{0.790288in}}{\pgfqpoint{3.797248in}{0.788986in}}%
\pgfpathcurveto{\pgfqpoint{3.795945in}{0.787683in}}{\pgfqpoint{3.795214in}{0.785917in}}{\pgfqpoint{3.795214in}{0.784075in}}%
\pgfpathcurveto{\pgfqpoint{3.795214in}{0.782233in}}{\pgfqpoint{3.795945in}{0.780467in}}{\pgfqpoint{3.797248in}{0.779165in}}%
\pgfpathcurveto{\pgfqpoint{3.798550in}{0.777862in}}{\pgfqpoint{3.800316in}{0.777131in}}{\pgfqpoint{3.802158in}{0.777131in}}%
\pgfpathlineto{\pgfqpoint{3.802158in}{0.777131in}}%
\pgfpathclose%
\pgfusepath{stroke,fill}%
\end{pgfscope}%
\begin{pgfscope}%
\pgfpathrectangle{\pgfqpoint{0.661006in}{0.524170in}}{\pgfqpoint{4.194036in}{1.071446in}}%
\pgfusepath{clip}%
\pgfsetbuttcap%
\pgfsetroundjoin%
\definecolor{currentfill}{rgb}{0.317251,0.303094,0.503737}%
\pgfsetfillcolor{currentfill}%
\pgfsetfillopacity{0.700000}%
\pgfsetlinewidth{1.003750pt}%
\definecolor{currentstroke}{rgb}{0.317251,0.303094,0.503737}%
\pgfsetstrokecolor{currentstroke}%
\pgfsetstrokeopacity{0.700000}%
\pgfsetdash{}{0pt}%
\pgfpathmoveto{\pgfqpoint{3.812801in}{0.773100in}}%
\pgfpathcurveto{\pgfqpoint{3.814643in}{0.773100in}}{\pgfqpoint{3.816410in}{0.773831in}}{\pgfqpoint{3.817712in}{0.775134in}}%
\pgfpathcurveto{\pgfqpoint{3.819014in}{0.776436in}}{\pgfqpoint{3.819746in}{0.778202in}}{\pgfqpoint{3.819746in}{0.780044in}}%
\pgfpathcurveto{\pgfqpoint{3.819746in}{0.781886in}}{\pgfqpoint{3.819014in}{0.783652in}}{\pgfqpoint{3.817712in}{0.784954in}}%
\pgfpathcurveto{\pgfqpoint{3.816410in}{0.786257in}}{\pgfqpoint{3.814643in}{0.786988in}}{\pgfqpoint{3.812801in}{0.786988in}}%
\pgfpathcurveto{\pgfqpoint{3.810960in}{0.786988in}}{\pgfqpoint{3.809193in}{0.786257in}}{\pgfqpoint{3.807891in}{0.784954in}}%
\pgfpathcurveto{\pgfqpoint{3.806589in}{0.783652in}}{\pgfqpoint{3.805857in}{0.781886in}}{\pgfqpoint{3.805857in}{0.780044in}}%
\pgfpathcurveto{\pgfqpoint{3.805857in}{0.778202in}}{\pgfqpoint{3.806589in}{0.776436in}}{\pgfqpoint{3.807891in}{0.775134in}}%
\pgfpathcurveto{\pgfqpoint{3.809193in}{0.773831in}}{\pgfqpoint{3.810960in}{0.773100in}}{\pgfqpoint{3.812801in}{0.773100in}}%
\pgfpathlineto{\pgfqpoint{3.812801in}{0.773100in}}%
\pgfpathclose%
\pgfusepath{stroke,fill}%
\end{pgfscope}%
\begin{pgfscope}%
\pgfpathrectangle{\pgfqpoint{0.661006in}{0.524170in}}{\pgfqpoint{4.194036in}{1.071446in}}%
\pgfusepath{clip}%
\pgfsetbuttcap%
\pgfsetroundjoin%
\definecolor{currentfill}{rgb}{0.317251,0.303094,0.503737}%
\pgfsetfillcolor{currentfill}%
\pgfsetfillopacity{0.700000}%
\pgfsetlinewidth{1.003750pt}%
\definecolor{currentstroke}{rgb}{0.317251,0.303094,0.503737}%
\pgfsetstrokecolor{currentstroke}%
\pgfsetstrokeopacity{0.700000}%
\pgfsetdash{}{0pt}%
\pgfpathmoveto{\pgfqpoint{3.845568in}{0.767826in}}%
\pgfpathcurveto{\pgfqpoint{3.847410in}{0.767826in}}{\pgfqpoint{3.849176in}{0.768558in}}{\pgfqpoint{3.850478in}{0.769860in}}%
\pgfpathcurveto{\pgfqpoint{3.851781in}{0.771162in}}{\pgfqpoint{3.852512in}{0.772929in}}{\pgfqpoint{3.852512in}{0.774771in}}%
\pgfpathcurveto{\pgfqpoint{3.852512in}{0.776612in}}{\pgfqpoint{3.851781in}{0.778379in}}{\pgfqpoint{3.850478in}{0.779681in}}%
\pgfpathcurveto{\pgfqpoint{3.849176in}{0.780983in}}{\pgfqpoint{3.847410in}{0.781715in}}{\pgfqpoint{3.845568in}{0.781715in}}%
\pgfpathcurveto{\pgfqpoint{3.843726in}{0.781715in}}{\pgfqpoint{3.841960in}{0.780983in}}{\pgfqpoint{3.840657in}{0.779681in}}%
\pgfpathcurveto{\pgfqpoint{3.839355in}{0.778379in}}{\pgfqpoint{3.838623in}{0.776612in}}{\pgfqpoint{3.838623in}{0.774771in}}%
\pgfpathcurveto{\pgfqpoint{3.838623in}{0.772929in}}{\pgfqpoint{3.839355in}{0.771162in}}{\pgfqpoint{3.840657in}{0.769860in}}%
\pgfpathcurveto{\pgfqpoint{3.841960in}{0.768558in}}{\pgfqpoint{3.843726in}{0.767826in}}{\pgfqpoint{3.845568in}{0.767826in}}%
\pgfpathlineto{\pgfqpoint{3.845568in}{0.767826in}}%
\pgfpathclose%
\pgfusepath{stroke,fill}%
\end{pgfscope}%
\begin{pgfscope}%
\pgfpathrectangle{\pgfqpoint{0.661006in}{0.524170in}}{\pgfqpoint{4.194036in}{1.071446in}}%
\pgfusepath{clip}%
\pgfsetbuttcap%
\pgfsetroundjoin%
\definecolor{currentfill}{rgb}{0.317251,0.303094,0.503737}%
\pgfsetfillcolor{currentfill}%
\pgfsetfillopacity{0.700000}%
\pgfsetlinewidth{1.003750pt}%
\definecolor{currentstroke}{rgb}{0.317251,0.303094,0.503737}%
\pgfsetstrokecolor{currentstroke}%
\pgfsetstrokeopacity{0.700000}%
\pgfsetdash{}{0pt}%
\pgfpathmoveto{\pgfqpoint{3.858694in}{0.765176in}}%
\pgfpathcurveto{\pgfqpoint{3.860536in}{0.765176in}}{\pgfqpoint{3.862303in}{0.765907in}}{\pgfqpoint{3.863605in}{0.767210in}}%
\pgfpathcurveto{\pgfqpoint{3.864907in}{0.768512in}}{\pgfqpoint{3.865639in}{0.770278in}}{\pgfqpoint{3.865639in}{0.772120in}}%
\pgfpathcurveto{\pgfqpoint{3.865639in}{0.773962in}}{\pgfqpoint{3.864907in}{0.775728in}}{\pgfqpoint{3.863605in}{0.777030in}}%
\pgfpathcurveto{\pgfqpoint{3.862303in}{0.778333in}}{\pgfqpoint{3.860536in}{0.779064in}}{\pgfqpoint{3.858694in}{0.779064in}}%
\pgfpathcurveto{\pgfqpoint{3.856853in}{0.779064in}}{\pgfqpoint{3.855086in}{0.778333in}}{\pgfqpoint{3.853784in}{0.777030in}}%
\pgfpathcurveto{\pgfqpoint{3.852482in}{0.775728in}}{\pgfqpoint{3.851750in}{0.773962in}}{\pgfqpoint{3.851750in}{0.772120in}}%
\pgfpathcurveto{\pgfqpoint{3.851750in}{0.770278in}}{\pgfqpoint{3.852482in}{0.768512in}}{\pgfqpoint{3.853784in}{0.767210in}}%
\pgfpathcurveto{\pgfqpoint{3.855086in}{0.765907in}}{\pgfqpoint{3.856853in}{0.765176in}}{\pgfqpoint{3.858694in}{0.765176in}}%
\pgfpathlineto{\pgfqpoint{3.858694in}{0.765176in}}%
\pgfpathclose%
\pgfusepath{stroke,fill}%
\end{pgfscope}%
\begin{pgfscope}%
\pgfpathrectangle{\pgfqpoint{0.661006in}{0.524170in}}{\pgfqpoint{4.194036in}{1.071446in}}%
\pgfusepath{clip}%
\pgfsetbuttcap%
\pgfsetroundjoin%
\definecolor{currentfill}{rgb}{0.317251,0.303094,0.503737}%
\pgfsetfillcolor{currentfill}%
\pgfsetfillopacity{0.700000}%
\pgfsetlinewidth{1.003750pt}%
\definecolor{currentstroke}{rgb}{0.317251,0.303094,0.503737}%
\pgfsetstrokecolor{currentstroke}%
\pgfsetstrokeopacity{0.700000}%
\pgfsetdash{}{0pt}%
\pgfpathmoveto{\pgfqpoint{3.878892in}{0.759944in}}%
\pgfpathcurveto{\pgfqpoint{3.880734in}{0.759944in}}{\pgfqpoint{3.882500in}{0.760676in}}{\pgfqpoint{3.883803in}{0.761978in}}%
\pgfpathcurveto{\pgfqpoint{3.885105in}{0.763280in}}{\pgfqpoint{3.885837in}{0.765047in}}{\pgfqpoint{3.885837in}{0.766888in}}%
\pgfpathcurveto{\pgfqpoint{3.885837in}{0.768730in}}{\pgfqpoint{3.885105in}{0.770497in}}{\pgfqpoint{3.883803in}{0.771799in}}%
\pgfpathcurveto{\pgfqpoint{3.882500in}{0.773101in}}{\pgfqpoint{3.880734in}{0.773833in}}{\pgfqpoint{3.878892in}{0.773833in}}%
\pgfpathcurveto{\pgfqpoint{3.877050in}{0.773833in}}{\pgfqpoint{3.875284in}{0.773101in}}{\pgfqpoint{3.873982in}{0.771799in}}%
\pgfpathcurveto{\pgfqpoint{3.872679in}{0.770497in}}{\pgfqpoint{3.871948in}{0.768730in}}{\pgfqpoint{3.871948in}{0.766888in}}%
\pgfpathcurveto{\pgfqpoint{3.871948in}{0.765047in}}{\pgfqpoint{3.872679in}{0.763280in}}{\pgfqpoint{3.873982in}{0.761978in}}%
\pgfpathcurveto{\pgfqpoint{3.875284in}{0.760676in}}{\pgfqpoint{3.877050in}{0.759944in}}{\pgfqpoint{3.878892in}{0.759944in}}%
\pgfpathlineto{\pgfqpoint{3.878892in}{0.759944in}}%
\pgfpathclose%
\pgfusepath{stroke,fill}%
\end{pgfscope}%
\begin{pgfscope}%
\pgfpathrectangle{\pgfqpoint{0.661006in}{0.524170in}}{\pgfqpoint{4.194036in}{1.071446in}}%
\pgfusepath{clip}%
\pgfsetbuttcap%
\pgfsetroundjoin%
\definecolor{currentfill}{rgb}{0.314369,0.298845,0.498738}%
\pgfsetfillcolor{currentfill}%
\pgfsetfillopacity{0.700000}%
\pgfsetlinewidth{1.003750pt}%
\definecolor{currentstroke}{rgb}{0.314369,0.298845,0.498738}%
\pgfsetstrokecolor{currentstroke}%
\pgfsetstrokeopacity{0.700000}%
\pgfsetdash{}{0pt}%
\pgfpathmoveto{\pgfqpoint{3.893718in}{0.757100in}}%
\pgfpathcurveto{\pgfqpoint{3.895560in}{0.757100in}}{\pgfqpoint{3.897327in}{0.757832in}}{\pgfqpoint{3.898629in}{0.759134in}}%
\pgfpathcurveto{\pgfqpoint{3.899931in}{0.760436in}}{\pgfqpoint{3.900663in}{0.762203in}}{\pgfqpoint{3.900663in}{0.764045in}}%
\pgfpathcurveto{\pgfqpoint{3.900663in}{0.765886in}}{\pgfqpoint{3.899931in}{0.767653in}}{\pgfqpoint{3.898629in}{0.768955in}}%
\pgfpathcurveto{\pgfqpoint{3.897327in}{0.770257in}}{\pgfqpoint{3.895560in}{0.770989in}}{\pgfqpoint{3.893718in}{0.770989in}}%
\pgfpathcurveto{\pgfqpoint{3.891877in}{0.770989in}}{\pgfqpoint{3.890110in}{0.770257in}}{\pgfqpoint{3.888808in}{0.768955in}}%
\pgfpathcurveto{\pgfqpoint{3.887506in}{0.767653in}}{\pgfqpoint{3.886774in}{0.765886in}}{\pgfqpoint{3.886774in}{0.764045in}}%
\pgfpathcurveto{\pgfqpoint{3.886774in}{0.762203in}}{\pgfqpoint{3.887506in}{0.760436in}}{\pgfqpoint{3.888808in}{0.759134in}}%
\pgfpathcurveto{\pgfqpoint{3.890110in}{0.757832in}}{\pgfqpoint{3.891877in}{0.757100in}}{\pgfqpoint{3.893718in}{0.757100in}}%
\pgfpathlineto{\pgfqpoint{3.893718in}{0.757100in}}%
\pgfpathclose%
\pgfusepath{stroke,fill}%
\end{pgfscope}%
\begin{pgfscope}%
\pgfpathrectangle{\pgfqpoint{0.661006in}{0.524170in}}{\pgfqpoint{4.194036in}{1.071446in}}%
\pgfusepath{clip}%
\pgfsetbuttcap%
\pgfsetroundjoin%
\definecolor{currentfill}{rgb}{0.314369,0.298845,0.498738}%
\pgfsetfillcolor{currentfill}%
\pgfsetfillopacity{0.700000}%
\pgfsetlinewidth{1.003750pt}%
\definecolor{currentstroke}{rgb}{0.314369,0.298845,0.498738}%
\pgfsetstrokecolor{currentstroke}%
\pgfsetstrokeopacity{0.700000}%
\pgfsetdash{}{0pt}%
\pgfpathmoveto{\pgfqpoint{3.911240in}{0.754689in}}%
\pgfpathcurveto{\pgfqpoint{3.913082in}{0.754689in}}{\pgfqpoint{3.914848in}{0.755421in}}{\pgfqpoint{3.916151in}{0.756723in}}%
\pgfpathcurveto{\pgfqpoint{3.917453in}{0.758025in}}{\pgfqpoint{3.918185in}{0.759792in}}{\pgfqpoint{3.918185in}{0.761633in}}%
\pgfpathcurveto{\pgfqpoint{3.918185in}{0.763475in}}{\pgfqpoint{3.917453in}{0.765242in}}{\pgfqpoint{3.916151in}{0.766544in}}%
\pgfpathcurveto{\pgfqpoint{3.914848in}{0.767846in}}{\pgfqpoint{3.913082in}{0.768578in}}{\pgfqpoint{3.911240in}{0.768578in}}%
\pgfpathcurveto{\pgfqpoint{3.909399in}{0.768578in}}{\pgfqpoint{3.907632in}{0.767846in}}{\pgfqpoint{3.906330in}{0.766544in}}%
\pgfpathcurveto{\pgfqpoint{3.905028in}{0.765242in}}{\pgfqpoint{3.904296in}{0.763475in}}{\pgfqpoint{3.904296in}{0.761633in}}%
\pgfpathcurveto{\pgfqpoint{3.904296in}{0.759792in}}{\pgfqpoint{3.905028in}{0.758025in}}{\pgfqpoint{3.906330in}{0.756723in}}%
\pgfpathcurveto{\pgfqpoint{3.907632in}{0.755421in}}{\pgfqpoint{3.909399in}{0.754689in}}{\pgfqpoint{3.911240in}{0.754689in}}%
\pgfpathlineto{\pgfqpoint{3.911240in}{0.754689in}}%
\pgfpathclose%
\pgfusepath{stroke,fill}%
\end{pgfscope}%
\begin{pgfscope}%
\pgfpathrectangle{\pgfqpoint{0.661006in}{0.524170in}}{\pgfqpoint{4.194036in}{1.071446in}}%
\pgfusepath{clip}%
\pgfsetbuttcap%
\pgfsetroundjoin%
\definecolor{currentfill}{rgb}{0.311477,0.294606,0.493708}%
\pgfsetfillcolor{currentfill}%
\pgfsetfillopacity{0.700000}%
\pgfsetlinewidth{1.003750pt}%
\definecolor{currentstroke}{rgb}{0.311477,0.294606,0.493708}%
\pgfsetstrokecolor{currentstroke}%
\pgfsetstrokeopacity{0.700000}%
\pgfsetdash{}{0pt}%
\pgfpathmoveto{\pgfqpoint{3.910078in}{0.754095in}}%
\pgfpathcurveto{\pgfqpoint{3.911920in}{0.754095in}}{\pgfqpoint{3.913687in}{0.754827in}}{\pgfqpoint{3.914989in}{0.756129in}}%
\pgfpathcurveto{\pgfqpoint{3.916291in}{0.757432in}}{\pgfqpoint{3.917023in}{0.759198in}}{\pgfqpoint{3.917023in}{0.761040in}}%
\pgfpathcurveto{\pgfqpoint{3.917023in}{0.762881in}}{\pgfqpoint{3.916291in}{0.764648in}}{\pgfqpoint{3.914989in}{0.765950in}}%
\pgfpathcurveto{\pgfqpoint{3.913687in}{0.767252in}}{\pgfqpoint{3.911920in}{0.767984in}}{\pgfqpoint{3.910078in}{0.767984in}}%
\pgfpathcurveto{\pgfqpoint{3.908237in}{0.767984in}}{\pgfqpoint{3.906470in}{0.767252in}}{\pgfqpoint{3.905168in}{0.765950in}}%
\pgfpathcurveto{\pgfqpoint{3.903866in}{0.764648in}}{\pgfqpoint{3.903134in}{0.762881in}}{\pgfqpoint{3.903134in}{0.761040in}}%
\pgfpathcurveto{\pgfqpoint{3.903134in}{0.759198in}}{\pgfqpoint{3.903866in}{0.757432in}}{\pgfqpoint{3.905168in}{0.756129in}}%
\pgfpathcurveto{\pgfqpoint{3.906470in}{0.754827in}}{\pgfqpoint{3.908237in}{0.754095in}}{\pgfqpoint{3.910078in}{0.754095in}}%
\pgfpathlineto{\pgfqpoint{3.910078in}{0.754095in}}%
\pgfpathclose%
\pgfusepath{stroke,fill}%
\end{pgfscope}%
\begin{pgfscope}%
\pgfpathrectangle{\pgfqpoint{0.661006in}{0.524170in}}{\pgfqpoint{4.194036in}{1.071446in}}%
\pgfusepath{clip}%
\pgfsetbuttcap%
\pgfsetroundjoin%
\definecolor{currentfill}{rgb}{0.311477,0.294606,0.493708}%
\pgfsetfillcolor{currentfill}%
\pgfsetfillopacity{0.700000}%
\pgfsetlinewidth{1.003750pt}%
\definecolor{currentstroke}{rgb}{0.311477,0.294606,0.493708}%
\pgfsetstrokecolor{currentstroke}%
\pgfsetstrokeopacity{0.700000}%
\pgfsetdash{}{0pt}%
\pgfpathmoveto{\pgfqpoint{3.921372in}{0.752677in}}%
\pgfpathcurveto{\pgfqpoint{3.923214in}{0.752677in}}{\pgfqpoint{3.924980in}{0.753409in}}{\pgfqpoint{3.926283in}{0.754711in}}%
\pgfpathcurveto{\pgfqpoint{3.927585in}{0.756013in}}{\pgfqpoint{3.928317in}{0.757780in}}{\pgfqpoint{3.928317in}{0.759622in}}%
\pgfpathcurveto{\pgfqpoint{3.928317in}{0.761463in}}{\pgfqpoint{3.927585in}{0.763230in}}{\pgfqpoint{3.926283in}{0.764532in}}%
\pgfpathcurveto{\pgfqpoint{3.924980in}{0.765834in}}{\pgfqpoint{3.923214in}{0.766566in}}{\pgfqpoint{3.921372in}{0.766566in}}%
\pgfpathcurveto{\pgfqpoint{3.919531in}{0.766566in}}{\pgfqpoint{3.917764in}{0.765834in}}{\pgfqpoint{3.916462in}{0.764532in}}%
\pgfpathcurveto{\pgfqpoint{3.915160in}{0.763230in}}{\pgfqpoint{3.914428in}{0.761463in}}{\pgfqpoint{3.914428in}{0.759622in}}%
\pgfpathcurveto{\pgfqpoint{3.914428in}{0.757780in}}{\pgfqpoint{3.915160in}{0.756013in}}{\pgfqpoint{3.916462in}{0.754711in}}%
\pgfpathcurveto{\pgfqpoint{3.917764in}{0.753409in}}{\pgfqpoint{3.919531in}{0.752677in}}{\pgfqpoint{3.921372in}{0.752677in}}%
\pgfpathlineto{\pgfqpoint{3.921372in}{0.752677in}}%
\pgfpathclose%
\pgfusepath{stroke,fill}%
\end{pgfscope}%
\begin{pgfscope}%
\pgfpathrectangle{\pgfqpoint{0.661006in}{0.524170in}}{\pgfqpoint{4.194036in}{1.071446in}}%
\pgfusepath{clip}%
\pgfsetbuttcap%
\pgfsetroundjoin%
\definecolor{currentfill}{rgb}{0.311477,0.294606,0.493708}%
\pgfsetfillcolor{currentfill}%
\pgfsetfillopacity{0.700000}%
\pgfsetlinewidth{1.003750pt}%
\definecolor{currentstroke}{rgb}{0.311477,0.294606,0.493708}%
\pgfsetstrokecolor{currentstroke}%
\pgfsetstrokeopacity{0.700000}%
\pgfsetdash{}{0pt}%
\pgfpathmoveto{\pgfqpoint{3.927740in}{0.750217in}}%
\pgfpathcurveto{\pgfqpoint{3.929581in}{0.750217in}}{\pgfqpoint{3.931348in}{0.750949in}}{\pgfqpoint{3.932650in}{0.752251in}}%
\pgfpathcurveto{\pgfqpoint{3.933952in}{0.753553in}}{\pgfqpoint{3.934684in}{0.755320in}}{\pgfqpoint{3.934684in}{0.757161in}}%
\pgfpathcurveto{\pgfqpoint{3.934684in}{0.759003in}}{\pgfqpoint{3.933952in}{0.760769in}}{\pgfqpoint{3.932650in}{0.762072in}}%
\pgfpathcurveto{\pgfqpoint{3.931348in}{0.763374in}}{\pgfqpoint{3.929581in}{0.764106in}}{\pgfqpoint{3.927740in}{0.764106in}}%
\pgfpathcurveto{\pgfqpoint{3.925898in}{0.764106in}}{\pgfqpoint{3.924131in}{0.763374in}}{\pgfqpoint{3.922829in}{0.762072in}}%
\pgfpathcurveto{\pgfqpoint{3.921527in}{0.760769in}}{\pgfqpoint{3.920795in}{0.759003in}}{\pgfqpoint{3.920795in}{0.757161in}}%
\pgfpathcurveto{\pgfqpoint{3.920795in}{0.755320in}}{\pgfqpoint{3.921527in}{0.753553in}}{\pgfqpoint{3.922829in}{0.752251in}}%
\pgfpathcurveto{\pgfqpoint{3.924131in}{0.750949in}}{\pgfqpoint{3.925898in}{0.750217in}}{\pgfqpoint{3.927740in}{0.750217in}}%
\pgfpathlineto{\pgfqpoint{3.927740in}{0.750217in}}%
\pgfpathclose%
\pgfusepath{stroke,fill}%
\end{pgfscope}%
\begin{pgfscope}%
\pgfpathrectangle{\pgfqpoint{0.661006in}{0.524170in}}{\pgfqpoint{4.194036in}{1.071446in}}%
\pgfusepath{clip}%
\pgfsetbuttcap%
\pgfsetroundjoin%
\definecolor{currentfill}{rgb}{0.308575,0.290379,0.488647}%
\pgfsetfillcolor{currentfill}%
\pgfsetfillopacity{0.700000}%
\pgfsetlinewidth{1.003750pt}%
\definecolor{currentstroke}{rgb}{0.308575,0.290379,0.488647}%
\pgfsetstrokecolor{currentstroke}%
\pgfsetstrokeopacity{0.700000}%
\pgfsetdash{}{0pt}%
\pgfpathmoveto{\pgfqpoint{3.943728in}{0.747998in}}%
\pgfpathcurveto{\pgfqpoint{3.945570in}{0.747998in}}{\pgfqpoint{3.947336in}{0.748730in}}{\pgfqpoint{3.948638in}{0.750032in}}%
\pgfpathcurveto{\pgfqpoint{3.949941in}{0.751334in}}{\pgfqpoint{3.950672in}{0.753101in}}{\pgfqpoint{3.950672in}{0.754943in}}%
\pgfpathcurveto{\pgfqpoint{3.950672in}{0.756784in}}{\pgfqpoint{3.949941in}{0.758551in}}{\pgfqpoint{3.948638in}{0.759853in}}%
\pgfpathcurveto{\pgfqpoint{3.947336in}{0.761155in}}{\pgfqpoint{3.945570in}{0.761887in}}{\pgfqpoint{3.943728in}{0.761887in}}%
\pgfpathcurveto{\pgfqpoint{3.941886in}{0.761887in}}{\pgfqpoint{3.940120in}{0.761155in}}{\pgfqpoint{3.938817in}{0.759853in}}%
\pgfpathcurveto{\pgfqpoint{3.937515in}{0.758551in}}{\pgfqpoint{3.936783in}{0.756784in}}{\pgfqpoint{3.936783in}{0.754943in}}%
\pgfpathcurveto{\pgfqpoint{3.936783in}{0.753101in}}{\pgfqpoint{3.937515in}{0.751334in}}{\pgfqpoint{3.938817in}{0.750032in}}%
\pgfpathcurveto{\pgfqpoint{3.940120in}{0.748730in}}{\pgfqpoint{3.941886in}{0.747998in}}{\pgfqpoint{3.943728in}{0.747998in}}%
\pgfpathlineto{\pgfqpoint{3.943728in}{0.747998in}}%
\pgfpathclose%
\pgfusepath{stroke,fill}%
\end{pgfscope}%
\begin{pgfscope}%
\pgfpathrectangle{\pgfqpoint{0.661006in}{0.524170in}}{\pgfqpoint{4.194036in}{1.071446in}}%
\pgfusepath{clip}%
\pgfsetbuttcap%
\pgfsetroundjoin%
\definecolor{currentfill}{rgb}{0.308575,0.290379,0.488647}%
\pgfsetfillcolor{currentfill}%
\pgfsetfillopacity{0.700000}%
\pgfsetlinewidth{1.003750pt}%
\definecolor{currentstroke}{rgb}{0.308575,0.290379,0.488647}%
\pgfsetstrokecolor{currentstroke}%
\pgfsetstrokeopacity{0.700000}%
\pgfsetdash{}{0pt}%
\pgfpathmoveto{\pgfqpoint{3.946145in}{0.746278in}}%
\pgfpathcurveto{\pgfqpoint{3.947986in}{0.746278in}}{\pgfqpoint{3.949753in}{0.747010in}}{\pgfqpoint{3.951055in}{0.748312in}}%
\pgfpathcurveto{\pgfqpoint{3.952357in}{0.749614in}}{\pgfqpoint{3.953089in}{0.751381in}}{\pgfqpoint{3.953089in}{0.753222in}}%
\pgfpathcurveto{\pgfqpoint{3.953089in}{0.755064in}}{\pgfqpoint{3.952357in}{0.756831in}}{\pgfqpoint{3.951055in}{0.758133in}}%
\pgfpathcurveto{\pgfqpoint{3.949753in}{0.759435in}}{\pgfqpoint{3.947986in}{0.760167in}}{\pgfqpoint{3.946145in}{0.760167in}}%
\pgfpathcurveto{\pgfqpoint{3.944303in}{0.760167in}}{\pgfqpoint{3.942536in}{0.759435in}}{\pgfqpoint{3.941234in}{0.758133in}}%
\pgfpathcurveto{\pgfqpoint{3.939932in}{0.756831in}}{\pgfqpoint{3.939200in}{0.755064in}}{\pgfqpoint{3.939200in}{0.753222in}}%
\pgfpathcurveto{\pgfqpoint{3.939200in}{0.751381in}}{\pgfqpoint{3.939932in}{0.749614in}}{\pgfqpoint{3.941234in}{0.748312in}}%
\pgfpathcurveto{\pgfqpoint{3.942536in}{0.747010in}}{\pgfqpoint{3.944303in}{0.746278in}}{\pgfqpoint{3.946145in}{0.746278in}}%
\pgfpathlineto{\pgfqpoint{3.946145in}{0.746278in}}%
\pgfpathclose%
\pgfusepath{stroke,fill}%
\end{pgfscope}%
\begin{pgfscope}%
\pgfpathrectangle{\pgfqpoint{0.661006in}{0.524170in}}{\pgfqpoint{4.194036in}{1.071446in}}%
\pgfusepath{clip}%
\pgfsetbuttcap%
\pgfsetroundjoin%
\definecolor{currentfill}{rgb}{0.308575,0.290379,0.488647}%
\pgfsetfillcolor{currentfill}%
\pgfsetfillopacity{0.700000}%
\pgfsetlinewidth{1.003750pt}%
\definecolor{currentstroke}{rgb}{0.308575,0.290379,0.488647}%
\pgfsetstrokecolor{currentstroke}%
\pgfsetstrokeopacity{0.700000}%
\pgfsetdash{}{0pt}%
\pgfpathmoveto{\pgfqpoint{3.946191in}{0.744969in}}%
\pgfpathcurveto{\pgfqpoint{3.948033in}{0.744969in}}{\pgfqpoint{3.949799in}{0.745701in}}{\pgfqpoint{3.951102in}{0.747003in}}%
\pgfpathcurveto{\pgfqpoint{3.952404in}{0.748305in}}{\pgfqpoint{3.953136in}{0.750072in}}{\pgfqpoint{3.953136in}{0.751913in}}%
\pgfpathcurveto{\pgfqpoint{3.953136in}{0.753755in}}{\pgfqpoint{3.952404in}{0.755522in}}{\pgfqpoint{3.951102in}{0.756824in}}%
\pgfpathcurveto{\pgfqpoint{3.949799in}{0.758126in}}{\pgfqpoint{3.948033in}{0.758858in}}{\pgfqpoint{3.946191in}{0.758858in}}%
\pgfpathcurveto{\pgfqpoint{3.944349in}{0.758858in}}{\pgfqpoint{3.942583in}{0.758126in}}{\pgfqpoint{3.941281in}{0.756824in}}%
\pgfpathcurveto{\pgfqpoint{3.939978in}{0.755522in}}{\pgfqpoint{3.939247in}{0.753755in}}{\pgfqpoint{3.939247in}{0.751913in}}%
\pgfpathcurveto{\pgfqpoint{3.939247in}{0.750072in}}{\pgfqpoint{3.939978in}{0.748305in}}{\pgfqpoint{3.941281in}{0.747003in}}%
\pgfpathcurveto{\pgfqpoint{3.942583in}{0.745701in}}{\pgfqpoint{3.944349in}{0.744969in}}{\pgfqpoint{3.946191in}{0.744969in}}%
\pgfpathlineto{\pgfqpoint{3.946191in}{0.744969in}}%
\pgfpathclose%
\pgfusepath{stroke,fill}%
\end{pgfscope}%
\begin{pgfscope}%
\pgfpathrectangle{\pgfqpoint{0.661006in}{0.524170in}}{\pgfqpoint{4.194036in}{1.071446in}}%
\pgfusepath{clip}%
\pgfsetbuttcap%
\pgfsetroundjoin%
\definecolor{currentfill}{rgb}{0.308575,0.290379,0.488647}%
\pgfsetfillcolor{currentfill}%
\pgfsetfillopacity{0.700000}%
\pgfsetlinewidth{1.003750pt}%
\definecolor{currentstroke}{rgb}{0.308575,0.290379,0.488647}%
\pgfsetstrokecolor{currentstroke}%
\pgfsetstrokeopacity{0.700000}%
\pgfsetdash{}{0pt}%
\pgfpathmoveto{\pgfqpoint{3.945819in}{0.748219in}}%
\pgfpathcurveto{\pgfqpoint{3.947661in}{0.748219in}}{\pgfqpoint{3.949427in}{0.748950in}}{\pgfqpoint{3.950730in}{0.750253in}}%
\pgfpathcurveto{\pgfqpoint{3.952032in}{0.751555in}}{\pgfqpoint{3.952764in}{0.753321in}}{\pgfqpoint{3.952764in}{0.755163in}}%
\pgfpathcurveto{\pgfqpoint{3.952764in}{0.757005in}}{\pgfqpoint{3.952032in}{0.758771in}}{\pgfqpoint{3.950730in}{0.760074in}}%
\pgfpathcurveto{\pgfqpoint{3.949427in}{0.761376in}}{\pgfqpoint{3.947661in}{0.762108in}}{\pgfqpoint{3.945819in}{0.762108in}}%
\pgfpathcurveto{\pgfqpoint{3.943978in}{0.762108in}}{\pgfqpoint{3.942211in}{0.761376in}}{\pgfqpoint{3.940909in}{0.760074in}}%
\pgfpathcurveto{\pgfqpoint{3.939607in}{0.758771in}}{\pgfqpoint{3.938875in}{0.757005in}}{\pgfqpoint{3.938875in}{0.755163in}}%
\pgfpathcurveto{\pgfqpoint{3.938875in}{0.753321in}}{\pgfqpoint{3.939607in}{0.751555in}}{\pgfqpoint{3.940909in}{0.750253in}}%
\pgfpathcurveto{\pgfqpoint{3.942211in}{0.748950in}}{\pgfqpoint{3.943978in}{0.748219in}}{\pgfqpoint{3.945819in}{0.748219in}}%
\pgfpathlineto{\pgfqpoint{3.945819in}{0.748219in}}%
\pgfpathclose%
\pgfusepath{stroke,fill}%
\end{pgfscope}%
\begin{pgfscope}%
\pgfpathrectangle{\pgfqpoint{0.661006in}{0.524170in}}{\pgfqpoint{4.194036in}{1.071446in}}%
\pgfusepath{clip}%
\pgfsetbuttcap%
\pgfsetroundjoin%
\definecolor{currentfill}{rgb}{0.308575,0.290379,0.488647}%
\pgfsetfillcolor{currentfill}%
\pgfsetfillopacity{0.700000}%
\pgfsetlinewidth{1.003750pt}%
\definecolor{currentstroke}{rgb}{0.308575,0.290379,0.488647}%
\pgfsetstrokecolor{currentstroke}%
\pgfsetstrokeopacity{0.700000}%
\pgfsetdash{}{0pt}%
\pgfpathmoveto{\pgfqpoint{3.939359in}{0.748222in}}%
\pgfpathcurveto{\pgfqpoint{3.941201in}{0.748222in}}{\pgfqpoint{3.942967in}{0.748953in}}{\pgfqpoint{3.944269in}{0.750256in}}%
\pgfpathcurveto{\pgfqpoint{3.945572in}{0.751558in}}{\pgfqpoint{3.946303in}{0.753324in}}{\pgfqpoint{3.946303in}{0.755166in}}%
\pgfpathcurveto{\pgfqpoint{3.946303in}{0.757008in}}{\pgfqpoint{3.945572in}{0.758774in}}{\pgfqpoint{3.944269in}{0.760077in}}%
\pgfpathcurveto{\pgfqpoint{3.942967in}{0.761379in}}{\pgfqpoint{3.941201in}{0.762111in}}{\pgfqpoint{3.939359in}{0.762111in}}%
\pgfpathcurveto{\pgfqpoint{3.937517in}{0.762111in}}{\pgfqpoint{3.935751in}{0.761379in}}{\pgfqpoint{3.934449in}{0.760077in}}%
\pgfpathcurveto{\pgfqpoint{3.933146in}{0.758774in}}{\pgfqpoint{3.932415in}{0.757008in}}{\pgfqpoint{3.932415in}{0.755166in}}%
\pgfpathcurveto{\pgfqpoint{3.932415in}{0.753324in}}{\pgfqpoint{3.933146in}{0.751558in}}{\pgfqpoint{3.934449in}{0.750256in}}%
\pgfpathcurveto{\pgfqpoint{3.935751in}{0.748953in}}{\pgfqpoint{3.937517in}{0.748222in}}{\pgfqpoint{3.939359in}{0.748222in}}%
\pgfpathlineto{\pgfqpoint{3.939359in}{0.748222in}}%
\pgfpathclose%
\pgfusepath{stroke,fill}%
\end{pgfscope}%
\begin{pgfscope}%
\pgfpathrectangle{\pgfqpoint{0.661006in}{0.524170in}}{\pgfqpoint{4.194036in}{1.071446in}}%
\pgfusepath{clip}%
\pgfsetbuttcap%
\pgfsetroundjoin%
\definecolor{currentfill}{rgb}{0.305661,0.286163,0.483556}%
\pgfsetfillcolor{currentfill}%
\pgfsetfillopacity{0.700000}%
\pgfsetlinewidth{1.003750pt}%
\definecolor{currentstroke}{rgb}{0.305661,0.286163,0.483556}%
\pgfsetstrokecolor{currentstroke}%
\pgfsetstrokeopacity{0.700000}%
\pgfsetdash{}{0pt}%
\pgfpathmoveto{\pgfqpoint{3.941822in}{0.748193in}}%
\pgfpathcurveto{\pgfqpoint{3.943664in}{0.748193in}}{\pgfqpoint{3.945430in}{0.748925in}}{\pgfqpoint{3.946733in}{0.750227in}}%
\pgfpathcurveto{\pgfqpoint{3.948035in}{0.751529in}}{\pgfqpoint{3.948767in}{0.753296in}}{\pgfqpoint{3.948767in}{0.755138in}}%
\pgfpathcurveto{\pgfqpoint{3.948767in}{0.756979in}}{\pgfqpoint{3.948035in}{0.758746in}}{\pgfqpoint{3.946733in}{0.760048in}}%
\pgfpathcurveto{\pgfqpoint{3.945430in}{0.761350in}}{\pgfqpoint{3.943664in}{0.762082in}}{\pgfqpoint{3.941822in}{0.762082in}}%
\pgfpathcurveto{\pgfqpoint{3.939981in}{0.762082in}}{\pgfqpoint{3.938214in}{0.761350in}}{\pgfqpoint{3.936912in}{0.760048in}}%
\pgfpathcurveto{\pgfqpoint{3.935610in}{0.758746in}}{\pgfqpoint{3.934878in}{0.756979in}}{\pgfqpoint{3.934878in}{0.755138in}}%
\pgfpathcurveto{\pgfqpoint{3.934878in}{0.753296in}}{\pgfqpoint{3.935610in}{0.751529in}}{\pgfqpoint{3.936912in}{0.750227in}}%
\pgfpathcurveto{\pgfqpoint{3.938214in}{0.748925in}}{\pgfqpoint{3.939981in}{0.748193in}}{\pgfqpoint{3.941822in}{0.748193in}}%
\pgfpathlineto{\pgfqpoint{3.941822in}{0.748193in}}%
\pgfpathclose%
\pgfusepath{stroke,fill}%
\end{pgfscope}%
\begin{pgfscope}%
\pgfpathrectangle{\pgfqpoint{0.661006in}{0.524170in}}{\pgfqpoint{4.194036in}{1.071446in}}%
\pgfusepath{clip}%
\pgfsetbuttcap%
\pgfsetroundjoin%
\definecolor{currentfill}{rgb}{0.305661,0.286163,0.483556}%
\pgfsetfillcolor{currentfill}%
\pgfsetfillopacity{0.700000}%
\pgfsetlinewidth{1.003750pt}%
\definecolor{currentstroke}{rgb}{0.305661,0.286163,0.483556}%
\pgfsetstrokecolor{currentstroke}%
\pgfsetstrokeopacity{0.700000}%
\pgfsetdash{}{0pt}%
\pgfpathmoveto{\pgfqpoint{3.950318in}{0.744411in}}%
\pgfpathcurveto{\pgfqpoint{3.952159in}{0.744411in}}{\pgfqpoint{3.953926in}{0.745143in}}{\pgfqpoint{3.955228in}{0.746445in}}%
\pgfpathcurveto{\pgfqpoint{3.956530in}{0.747748in}}{\pgfqpoint{3.957262in}{0.749514in}}{\pgfqpoint{3.957262in}{0.751356in}}%
\pgfpathcurveto{\pgfqpoint{3.957262in}{0.753197in}}{\pgfqpoint{3.956530in}{0.754964in}}{\pgfqpoint{3.955228in}{0.756266in}}%
\pgfpathcurveto{\pgfqpoint{3.953926in}{0.757568in}}{\pgfqpoint{3.952159in}{0.758300in}}{\pgfqpoint{3.950318in}{0.758300in}}%
\pgfpathcurveto{\pgfqpoint{3.948476in}{0.758300in}}{\pgfqpoint{3.946709in}{0.757568in}}{\pgfqpoint{3.945407in}{0.756266in}}%
\pgfpathcurveto{\pgfqpoint{3.944105in}{0.754964in}}{\pgfqpoint{3.943373in}{0.753197in}}{\pgfqpoint{3.943373in}{0.751356in}}%
\pgfpathcurveto{\pgfqpoint{3.943373in}{0.749514in}}{\pgfqpoint{3.944105in}{0.747748in}}{\pgfqpoint{3.945407in}{0.746445in}}%
\pgfpathcurveto{\pgfqpoint{3.946709in}{0.745143in}}{\pgfqpoint{3.948476in}{0.744411in}}{\pgfqpoint{3.950318in}{0.744411in}}%
\pgfpathlineto{\pgfqpoint{3.950318in}{0.744411in}}%
\pgfpathclose%
\pgfusepath{stroke,fill}%
\end{pgfscope}%
\begin{pgfscope}%
\pgfpathrectangle{\pgfqpoint{0.661006in}{0.524170in}}{\pgfqpoint{4.194036in}{1.071446in}}%
\pgfusepath{clip}%
\pgfsetbuttcap%
\pgfsetroundjoin%
\definecolor{currentfill}{rgb}{0.302736,0.281958,0.478435}%
\pgfsetfillcolor{currentfill}%
\pgfsetfillopacity{0.700000}%
\pgfsetlinewidth{1.003750pt}%
\definecolor{currentstroke}{rgb}{0.302736,0.281958,0.478435}%
\pgfsetstrokecolor{currentstroke}%
\pgfsetstrokeopacity{0.700000}%
\pgfsetdash{}{0pt}%
\pgfpathmoveto{\pgfqpoint{3.978493in}{0.738853in}}%
\pgfpathcurveto{\pgfqpoint{3.980334in}{0.738853in}}{\pgfqpoint{3.982101in}{0.739584in}}{\pgfqpoint{3.983403in}{0.740887in}}%
\pgfpathcurveto{\pgfqpoint{3.984706in}{0.742189in}}{\pgfqpoint{3.985437in}{0.743955in}}{\pgfqpoint{3.985437in}{0.745797in}}%
\pgfpathcurveto{\pgfqpoint{3.985437in}{0.747639in}}{\pgfqpoint{3.984706in}{0.749405in}}{\pgfqpoint{3.983403in}{0.750708in}}%
\pgfpathcurveto{\pgfqpoint{3.982101in}{0.752010in}}{\pgfqpoint{3.980334in}{0.752741in}}{\pgfqpoint{3.978493in}{0.752741in}}%
\pgfpathcurveto{\pgfqpoint{3.976651in}{0.752741in}}{\pgfqpoint{3.974885in}{0.752010in}}{\pgfqpoint{3.973582in}{0.750708in}}%
\pgfpathcurveto{\pgfqpoint{3.972280in}{0.749405in}}{\pgfqpoint{3.971548in}{0.747639in}}{\pgfqpoint{3.971548in}{0.745797in}}%
\pgfpathcurveto{\pgfqpoint{3.971548in}{0.743955in}}{\pgfqpoint{3.972280in}{0.742189in}}{\pgfqpoint{3.973582in}{0.740887in}}%
\pgfpathcurveto{\pgfqpoint{3.974885in}{0.739584in}}{\pgfqpoint{3.976651in}{0.738853in}}{\pgfqpoint{3.978493in}{0.738853in}}%
\pgfpathlineto{\pgfqpoint{3.978493in}{0.738853in}}%
\pgfpathclose%
\pgfusepath{stroke,fill}%
\end{pgfscope}%
\begin{pgfscope}%
\pgfpathrectangle{\pgfqpoint{0.661006in}{0.524170in}}{\pgfqpoint{4.194036in}{1.071446in}}%
\pgfusepath{clip}%
\pgfsetbuttcap%
\pgfsetroundjoin%
\definecolor{currentfill}{rgb}{0.302736,0.281958,0.478435}%
\pgfsetfillcolor{currentfill}%
\pgfsetfillopacity{0.700000}%
\pgfsetlinewidth{1.003750pt}%
\definecolor{currentstroke}{rgb}{0.302736,0.281958,0.478435}%
\pgfsetstrokecolor{currentstroke}%
\pgfsetstrokeopacity{0.700000}%
\pgfsetdash{}{0pt}%
\pgfpathmoveto{\pgfqpoint{4.027340in}{0.729373in}}%
\pgfpathcurveto{\pgfqpoint{4.029182in}{0.729373in}}{\pgfqpoint{4.030949in}{0.730104in}}{\pgfqpoint{4.032251in}{0.731407in}}%
\pgfpathcurveto{\pgfqpoint{4.033553in}{0.732709in}}{\pgfqpoint{4.034285in}{0.734475in}}{\pgfqpoint{4.034285in}{0.736317in}}%
\pgfpathcurveto{\pgfqpoint{4.034285in}{0.738159in}}{\pgfqpoint{4.033553in}{0.739925in}}{\pgfqpoint{4.032251in}{0.741228in}}%
\pgfpathcurveto{\pgfqpoint{4.030949in}{0.742530in}}{\pgfqpoint{4.029182in}{0.743262in}}{\pgfqpoint{4.027340in}{0.743262in}}%
\pgfpathcurveto{\pgfqpoint{4.025499in}{0.743262in}}{\pgfqpoint{4.023732in}{0.742530in}}{\pgfqpoint{4.022430in}{0.741228in}}%
\pgfpathcurveto{\pgfqpoint{4.021128in}{0.739925in}}{\pgfqpoint{4.020396in}{0.738159in}}{\pgfqpoint{4.020396in}{0.736317in}}%
\pgfpathcurveto{\pgfqpoint{4.020396in}{0.734475in}}{\pgfqpoint{4.021128in}{0.732709in}}{\pgfqpoint{4.022430in}{0.731407in}}%
\pgfpathcurveto{\pgfqpoint{4.023732in}{0.730104in}}{\pgfqpoint{4.025499in}{0.729373in}}{\pgfqpoint{4.027340in}{0.729373in}}%
\pgfpathlineto{\pgfqpoint{4.027340in}{0.729373in}}%
\pgfpathclose%
\pgfusepath{stroke,fill}%
\end{pgfscope}%
\begin{pgfscope}%
\pgfpathrectangle{\pgfqpoint{0.661006in}{0.524170in}}{\pgfqpoint{4.194036in}{1.071446in}}%
\pgfusepath{clip}%
\pgfsetbuttcap%
\pgfsetroundjoin%
\definecolor{currentfill}{rgb}{0.302736,0.281958,0.478435}%
\pgfsetfillcolor{currentfill}%
\pgfsetfillopacity{0.700000}%
\pgfsetlinewidth{1.003750pt}%
\definecolor{currentstroke}{rgb}{0.302736,0.281958,0.478435}%
\pgfsetstrokecolor{currentstroke}%
\pgfsetstrokeopacity{0.700000}%
\pgfsetdash{}{0pt}%
\pgfpathmoveto{\pgfqpoint{4.051323in}{0.722948in}}%
\pgfpathcurveto{\pgfqpoint{4.053164in}{0.722948in}}{\pgfqpoint{4.054931in}{0.723680in}}{\pgfqpoint{4.056233in}{0.724982in}}%
\pgfpathcurveto{\pgfqpoint{4.057535in}{0.726285in}}{\pgfqpoint{4.058267in}{0.728051in}}{\pgfqpoint{4.058267in}{0.729893in}}%
\pgfpathcurveto{\pgfqpoint{4.058267in}{0.731734in}}{\pgfqpoint{4.057535in}{0.733501in}}{\pgfqpoint{4.056233in}{0.734803in}}%
\pgfpathcurveto{\pgfqpoint{4.054931in}{0.736105in}}{\pgfqpoint{4.053164in}{0.736837in}}{\pgfqpoint{4.051323in}{0.736837in}}%
\pgfpathcurveto{\pgfqpoint{4.049481in}{0.736837in}}{\pgfqpoint{4.047714in}{0.736105in}}{\pgfqpoint{4.046412in}{0.734803in}}%
\pgfpathcurveto{\pgfqpoint{4.045110in}{0.733501in}}{\pgfqpoint{4.044378in}{0.731734in}}{\pgfqpoint{4.044378in}{0.729893in}}%
\pgfpathcurveto{\pgfqpoint{4.044378in}{0.728051in}}{\pgfqpoint{4.045110in}{0.726285in}}{\pgfqpoint{4.046412in}{0.724982in}}%
\pgfpathcurveto{\pgfqpoint{4.047714in}{0.723680in}}{\pgfqpoint{4.049481in}{0.722948in}}{\pgfqpoint{4.051323in}{0.722948in}}%
\pgfpathlineto{\pgfqpoint{4.051323in}{0.722948in}}%
\pgfpathclose%
\pgfusepath{stroke,fill}%
\end{pgfscope}%
\begin{pgfscope}%
\pgfpathrectangle{\pgfqpoint{0.661006in}{0.524170in}}{\pgfqpoint{4.194036in}{1.071446in}}%
\pgfusepath{clip}%
\pgfsetbuttcap%
\pgfsetroundjoin%
\definecolor{currentfill}{rgb}{0.302736,0.281958,0.478435}%
\pgfsetfillcolor{currentfill}%
\pgfsetfillopacity{0.700000}%
\pgfsetlinewidth{1.003750pt}%
\definecolor{currentstroke}{rgb}{0.302736,0.281958,0.478435}%
\pgfsetstrokecolor{currentstroke}%
\pgfsetstrokeopacity{0.700000}%
\pgfsetdash{}{0pt}%
\pgfpathmoveto{\pgfqpoint{4.076513in}{0.718010in}}%
\pgfpathcurveto{\pgfqpoint{4.078355in}{0.718010in}}{\pgfqpoint{4.080121in}{0.718742in}}{\pgfqpoint{4.081424in}{0.720044in}}%
\pgfpathcurveto{\pgfqpoint{4.082726in}{0.721347in}}{\pgfqpoint{4.083458in}{0.723113in}}{\pgfqpoint{4.083458in}{0.724955in}}%
\pgfpathcurveto{\pgfqpoint{4.083458in}{0.726796in}}{\pgfqpoint{4.082726in}{0.728563in}}{\pgfqpoint{4.081424in}{0.729865in}}%
\pgfpathcurveto{\pgfqpoint{4.080121in}{0.731167in}}{\pgfqpoint{4.078355in}{0.731899in}}{\pgfqpoint{4.076513in}{0.731899in}}%
\pgfpathcurveto{\pgfqpoint{4.074672in}{0.731899in}}{\pgfqpoint{4.072905in}{0.731167in}}{\pgfqpoint{4.071603in}{0.729865in}}%
\pgfpathcurveto{\pgfqpoint{4.070301in}{0.728563in}}{\pgfqpoint{4.069569in}{0.726796in}}{\pgfqpoint{4.069569in}{0.724955in}}%
\pgfpathcurveto{\pgfqpoint{4.069569in}{0.723113in}}{\pgfqpoint{4.070301in}{0.721347in}}{\pgfqpoint{4.071603in}{0.720044in}}%
\pgfpathcurveto{\pgfqpoint{4.072905in}{0.718742in}}{\pgfqpoint{4.074672in}{0.718010in}}{\pgfqpoint{4.076513in}{0.718010in}}%
\pgfpathlineto{\pgfqpoint{4.076513in}{0.718010in}}%
\pgfpathclose%
\pgfusepath{stroke,fill}%
\end{pgfscope}%
\begin{pgfscope}%
\pgfpathrectangle{\pgfqpoint{0.661006in}{0.524170in}}{\pgfqpoint{4.194036in}{1.071446in}}%
\pgfusepath{clip}%
\pgfsetbuttcap%
\pgfsetroundjoin%
\definecolor{currentfill}{rgb}{0.302736,0.281958,0.478435}%
\pgfsetfillcolor{currentfill}%
\pgfsetfillopacity{0.700000}%
\pgfsetlinewidth{1.003750pt}%
\definecolor{currentstroke}{rgb}{0.302736,0.281958,0.478435}%
\pgfsetstrokecolor{currentstroke}%
\pgfsetstrokeopacity{0.700000}%
\pgfsetdash{}{0pt}%
\pgfpathmoveto{\pgfqpoint{4.081300in}{0.714642in}}%
\pgfpathcurveto{\pgfqpoint{4.083142in}{0.714642in}}{\pgfqpoint{4.084909in}{0.715374in}}{\pgfqpoint{4.086211in}{0.716676in}}%
\pgfpathcurveto{\pgfqpoint{4.087513in}{0.717978in}}{\pgfqpoint{4.088245in}{0.719745in}}{\pgfqpoint{4.088245in}{0.721586in}}%
\pgfpathcurveto{\pgfqpoint{4.088245in}{0.723428in}}{\pgfqpoint{4.087513in}{0.725195in}}{\pgfqpoint{4.086211in}{0.726497in}}%
\pgfpathcurveto{\pgfqpoint{4.084909in}{0.727799in}}{\pgfqpoint{4.083142in}{0.728531in}}{\pgfqpoint{4.081300in}{0.728531in}}%
\pgfpathcurveto{\pgfqpoint{4.079459in}{0.728531in}}{\pgfqpoint{4.077692in}{0.727799in}}{\pgfqpoint{4.076390in}{0.726497in}}%
\pgfpathcurveto{\pgfqpoint{4.075088in}{0.725195in}}{\pgfqpoint{4.074356in}{0.723428in}}{\pgfqpoint{4.074356in}{0.721586in}}%
\pgfpathcurveto{\pgfqpoint{4.074356in}{0.719745in}}{\pgfqpoint{4.075088in}{0.717978in}}{\pgfqpoint{4.076390in}{0.716676in}}%
\pgfpathcurveto{\pgfqpoint{4.077692in}{0.715374in}}{\pgfqpoint{4.079459in}{0.714642in}}{\pgfqpoint{4.081300in}{0.714642in}}%
\pgfpathlineto{\pgfqpoint{4.081300in}{0.714642in}}%
\pgfpathclose%
\pgfusepath{stroke,fill}%
\end{pgfscope}%
\begin{pgfscope}%
\pgfpathrectangle{\pgfqpoint{0.661006in}{0.524170in}}{\pgfqpoint{4.194036in}{1.071446in}}%
\pgfusepath{clip}%
\pgfsetbuttcap%
\pgfsetroundjoin%
\definecolor{currentfill}{rgb}{0.299800,0.277765,0.473283}%
\pgfsetfillcolor{currentfill}%
\pgfsetfillopacity{0.700000}%
\pgfsetlinewidth{1.003750pt}%
\definecolor{currentstroke}{rgb}{0.299800,0.277765,0.473283}%
\pgfsetstrokecolor{currentstroke}%
\pgfsetstrokeopacity{0.700000}%
\pgfsetdash{}{0pt}%
\pgfpathmoveto{\pgfqpoint{4.094175in}{0.715850in}}%
\pgfpathcurveto{\pgfqpoint{4.096016in}{0.715850in}}{\pgfqpoint{4.097783in}{0.716582in}}{\pgfqpoint{4.099085in}{0.717884in}}%
\pgfpathcurveto{\pgfqpoint{4.100387in}{0.719187in}}{\pgfqpoint{4.101119in}{0.720953in}}{\pgfqpoint{4.101119in}{0.722795in}}%
\pgfpathcurveto{\pgfqpoint{4.101119in}{0.724637in}}{\pgfqpoint{4.100387in}{0.726403in}}{\pgfqpoint{4.099085in}{0.727705in}}%
\pgfpathcurveto{\pgfqpoint{4.097783in}{0.729008in}}{\pgfqpoint{4.096016in}{0.729739in}}{\pgfqpoint{4.094175in}{0.729739in}}%
\pgfpathcurveto{\pgfqpoint{4.092333in}{0.729739in}}{\pgfqpoint{4.090566in}{0.729008in}}{\pgfqpoint{4.089264in}{0.727705in}}%
\pgfpathcurveto{\pgfqpoint{4.087962in}{0.726403in}}{\pgfqpoint{4.087230in}{0.724637in}}{\pgfqpoint{4.087230in}{0.722795in}}%
\pgfpathcurveto{\pgfqpoint{4.087230in}{0.720953in}}{\pgfqpoint{4.087962in}{0.719187in}}{\pgfqpoint{4.089264in}{0.717884in}}%
\pgfpathcurveto{\pgfqpoint{4.090566in}{0.716582in}}{\pgfqpoint{4.092333in}{0.715850in}}{\pgfqpoint{4.094175in}{0.715850in}}%
\pgfpathlineto{\pgfqpoint{4.094175in}{0.715850in}}%
\pgfpathclose%
\pgfusepath{stroke,fill}%
\end{pgfscope}%
\begin{pgfscope}%
\pgfpathrectangle{\pgfqpoint{0.661006in}{0.524170in}}{\pgfqpoint{4.194036in}{1.071446in}}%
\pgfusepath{clip}%
\pgfsetbuttcap%
\pgfsetroundjoin%
\definecolor{currentfill}{rgb}{0.299800,0.277765,0.473283}%
\pgfsetfillcolor{currentfill}%
\pgfsetfillopacity{0.700000}%
\pgfsetlinewidth{1.003750pt}%
\definecolor{currentstroke}{rgb}{0.299800,0.277765,0.473283}%
\pgfsetstrokecolor{currentstroke}%
\pgfsetstrokeopacity{0.700000}%
\pgfsetdash{}{0pt}%
\pgfpathmoveto{\pgfqpoint{4.099798in}{0.713045in}}%
\pgfpathcurveto{\pgfqpoint{4.101640in}{0.713045in}}{\pgfqpoint{4.103407in}{0.713777in}}{\pgfqpoint{4.104709in}{0.715079in}}%
\pgfpathcurveto{\pgfqpoint{4.106011in}{0.716381in}}{\pgfqpoint{4.106743in}{0.718148in}}{\pgfqpoint{4.106743in}{0.719989in}}%
\pgfpathcurveto{\pgfqpoint{4.106743in}{0.721831in}}{\pgfqpoint{4.106011in}{0.723597in}}{\pgfqpoint{4.104709in}{0.724900in}}%
\pgfpathcurveto{\pgfqpoint{4.103407in}{0.726202in}}{\pgfqpoint{4.101640in}{0.726934in}}{\pgfqpoint{4.099798in}{0.726934in}}%
\pgfpathcurveto{\pgfqpoint{4.097957in}{0.726934in}}{\pgfqpoint{4.096190in}{0.726202in}}{\pgfqpoint{4.094888in}{0.724900in}}%
\pgfpathcurveto{\pgfqpoint{4.093586in}{0.723597in}}{\pgfqpoint{4.092854in}{0.721831in}}{\pgfqpoint{4.092854in}{0.719989in}}%
\pgfpathcurveto{\pgfqpoint{4.092854in}{0.718148in}}{\pgfqpoint{4.093586in}{0.716381in}}{\pgfqpoint{4.094888in}{0.715079in}}%
\pgfpathcurveto{\pgfqpoint{4.096190in}{0.713777in}}{\pgfqpoint{4.097957in}{0.713045in}}{\pgfqpoint{4.099798in}{0.713045in}}%
\pgfpathlineto{\pgfqpoint{4.099798in}{0.713045in}}%
\pgfpathclose%
\pgfusepath{stroke,fill}%
\end{pgfscope}%
\begin{pgfscope}%
\pgfpathrectangle{\pgfqpoint{0.661006in}{0.524170in}}{\pgfqpoint{4.194036in}{1.071446in}}%
\pgfusepath{clip}%
\pgfsetbuttcap%
\pgfsetroundjoin%
\definecolor{currentfill}{rgb}{0.296851,0.273584,0.468102}%
\pgfsetfillcolor{currentfill}%
\pgfsetfillopacity{0.700000}%
\pgfsetlinewidth{1.003750pt}%
\definecolor{currentstroke}{rgb}{0.296851,0.273584,0.468102}%
\pgfsetstrokecolor{currentstroke}%
\pgfsetstrokeopacity{0.700000}%
\pgfsetdash{}{0pt}%
\pgfpathmoveto{\pgfqpoint{4.110953in}{0.710701in}}%
\pgfpathcurveto{\pgfqpoint{4.112795in}{0.710701in}}{\pgfqpoint{4.114561in}{0.711433in}}{\pgfqpoint{4.115863in}{0.712735in}}%
\pgfpathcurveto{\pgfqpoint{4.117166in}{0.714038in}}{\pgfqpoint{4.117897in}{0.715804in}}{\pgfqpoint{4.117897in}{0.717646in}}%
\pgfpathcurveto{\pgfqpoint{4.117897in}{0.719487in}}{\pgfqpoint{4.117166in}{0.721254in}}{\pgfqpoint{4.115863in}{0.722556in}}%
\pgfpathcurveto{\pgfqpoint{4.114561in}{0.723859in}}{\pgfqpoint{4.112795in}{0.724590in}}{\pgfqpoint{4.110953in}{0.724590in}}%
\pgfpathcurveto{\pgfqpoint{4.109111in}{0.724590in}}{\pgfqpoint{4.107345in}{0.723859in}}{\pgfqpoint{4.106042in}{0.722556in}}%
\pgfpathcurveto{\pgfqpoint{4.104740in}{0.721254in}}{\pgfqpoint{4.104008in}{0.719487in}}{\pgfqpoint{4.104008in}{0.717646in}}%
\pgfpathcurveto{\pgfqpoint{4.104008in}{0.715804in}}{\pgfqpoint{4.104740in}{0.714038in}}{\pgfqpoint{4.106042in}{0.712735in}}%
\pgfpathcurveto{\pgfqpoint{4.107345in}{0.711433in}}{\pgfqpoint{4.109111in}{0.710701in}}{\pgfqpoint{4.110953in}{0.710701in}}%
\pgfpathlineto{\pgfqpoint{4.110953in}{0.710701in}}%
\pgfpathclose%
\pgfusepath{stroke,fill}%
\end{pgfscope}%
\begin{pgfscope}%
\pgfpathrectangle{\pgfqpoint{0.661006in}{0.524170in}}{\pgfqpoint{4.194036in}{1.071446in}}%
\pgfusepath{clip}%
\pgfsetbuttcap%
\pgfsetroundjoin%
\definecolor{currentfill}{rgb}{0.296851,0.273584,0.468102}%
\pgfsetfillcolor{currentfill}%
\pgfsetfillopacity{0.700000}%
\pgfsetlinewidth{1.003750pt}%
\definecolor{currentstroke}{rgb}{0.296851,0.273584,0.468102}%
\pgfsetstrokecolor{currentstroke}%
\pgfsetstrokeopacity{0.700000}%
\pgfsetdash{}{0pt}%
\pgfpathmoveto{\pgfqpoint{4.119923in}{0.708001in}}%
\pgfpathcurveto{\pgfqpoint{4.121765in}{0.708001in}}{\pgfqpoint{4.123531in}{0.708733in}}{\pgfqpoint{4.124833in}{0.710035in}}%
\pgfpathcurveto{\pgfqpoint{4.126136in}{0.711338in}}{\pgfqpoint{4.126867in}{0.713104in}}{\pgfqpoint{4.126867in}{0.714946in}}%
\pgfpathcurveto{\pgfqpoint{4.126867in}{0.716788in}}{\pgfqpoint{4.126136in}{0.718554in}}{\pgfqpoint{4.124833in}{0.719856in}}%
\pgfpathcurveto{\pgfqpoint{4.123531in}{0.721159in}}{\pgfqpoint{4.121765in}{0.721890in}}{\pgfqpoint{4.119923in}{0.721890in}}%
\pgfpathcurveto{\pgfqpoint{4.118081in}{0.721890in}}{\pgfqpoint{4.116315in}{0.721159in}}{\pgfqpoint{4.115013in}{0.719856in}}%
\pgfpathcurveto{\pgfqpoint{4.113710in}{0.718554in}}{\pgfqpoint{4.112979in}{0.716788in}}{\pgfqpoint{4.112979in}{0.714946in}}%
\pgfpathcurveto{\pgfqpoint{4.112979in}{0.713104in}}{\pgfqpoint{4.113710in}{0.711338in}}{\pgfqpoint{4.115013in}{0.710035in}}%
\pgfpathcurveto{\pgfqpoint{4.116315in}{0.708733in}}{\pgfqpoint{4.118081in}{0.708001in}}{\pgfqpoint{4.119923in}{0.708001in}}%
\pgfpathlineto{\pgfqpoint{4.119923in}{0.708001in}}%
\pgfpathclose%
\pgfusepath{stroke,fill}%
\end{pgfscope}%
\begin{pgfscope}%
\pgfpathrectangle{\pgfqpoint{0.661006in}{0.524170in}}{\pgfqpoint{4.194036in}{1.071446in}}%
\pgfusepath{clip}%
\pgfsetbuttcap%
\pgfsetroundjoin%
\definecolor{currentfill}{rgb}{0.296851,0.273584,0.468102}%
\pgfsetfillcolor{currentfill}%
\pgfsetfillopacity{0.700000}%
\pgfsetlinewidth{1.003750pt}%
\definecolor{currentstroke}{rgb}{0.296851,0.273584,0.468102}%
\pgfsetstrokecolor{currentstroke}%
\pgfsetstrokeopacity{0.700000}%
\pgfsetdash{}{0pt}%
\pgfpathmoveto{\pgfqpoint{4.134610in}{0.705672in}}%
\pgfpathcurveto{\pgfqpoint{4.136452in}{0.705672in}}{\pgfqpoint{4.138218in}{0.706404in}}{\pgfqpoint{4.139520in}{0.707706in}}%
\pgfpathcurveto{\pgfqpoint{4.140823in}{0.709009in}}{\pgfqpoint{4.141554in}{0.710775in}}{\pgfqpoint{4.141554in}{0.712617in}}%
\pgfpathcurveto{\pgfqpoint{4.141554in}{0.714459in}}{\pgfqpoint{4.140823in}{0.716225in}}{\pgfqpoint{4.139520in}{0.717527in}}%
\pgfpathcurveto{\pgfqpoint{4.138218in}{0.718830in}}{\pgfqpoint{4.136452in}{0.719561in}}{\pgfqpoint{4.134610in}{0.719561in}}%
\pgfpathcurveto{\pgfqpoint{4.132768in}{0.719561in}}{\pgfqpoint{4.131002in}{0.718830in}}{\pgfqpoint{4.129699in}{0.717527in}}%
\pgfpathcurveto{\pgfqpoint{4.128397in}{0.716225in}}{\pgfqpoint{4.127665in}{0.714459in}}{\pgfqpoint{4.127665in}{0.712617in}}%
\pgfpathcurveto{\pgfqpoint{4.127665in}{0.710775in}}{\pgfqpoint{4.128397in}{0.709009in}}{\pgfqpoint{4.129699in}{0.707706in}}%
\pgfpathcurveto{\pgfqpoint{4.131002in}{0.706404in}}{\pgfqpoint{4.132768in}{0.705672in}}{\pgfqpoint{4.134610in}{0.705672in}}%
\pgfpathlineto{\pgfqpoint{4.134610in}{0.705672in}}%
\pgfpathclose%
\pgfusepath{stroke,fill}%
\end{pgfscope}%
\begin{pgfscope}%
\pgfpathrectangle{\pgfqpoint{0.661006in}{0.524170in}}{\pgfqpoint{4.194036in}{1.071446in}}%
\pgfusepath{clip}%
\pgfsetbuttcap%
\pgfsetroundjoin%
\definecolor{currentfill}{rgb}{0.296851,0.273584,0.468102}%
\pgfsetfillcolor{currentfill}%
\pgfsetfillopacity{0.700000}%
\pgfsetlinewidth{1.003750pt}%
\definecolor{currentstroke}{rgb}{0.296851,0.273584,0.468102}%
\pgfsetstrokecolor{currentstroke}%
\pgfsetstrokeopacity{0.700000}%
\pgfsetdash{}{0pt}%
\pgfpathmoveto{\pgfqpoint{4.146369in}{0.703849in}}%
\pgfpathcurveto{\pgfqpoint{4.148210in}{0.703849in}}{\pgfqpoint{4.149977in}{0.704581in}}{\pgfqpoint{4.151279in}{0.705883in}}%
\pgfpathcurveto{\pgfqpoint{4.152581in}{0.707185in}}{\pgfqpoint{4.153313in}{0.708952in}}{\pgfqpoint{4.153313in}{0.710794in}}%
\pgfpathcurveto{\pgfqpoint{4.153313in}{0.712635in}}{\pgfqpoint{4.152581in}{0.714402in}}{\pgfqpoint{4.151279in}{0.715704in}}%
\pgfpathcurveto{\pgfqpoint{4.149977in}{0.717006in}}{\pgfqpoint{4.148210in}{0.717738in}}{\pgfqpoint{4.146369in}{0.717738in}}%
\pgfpathcurveto{\pgfqpoint{4.144527in}{0.717738in}}{\pgfqpoint{4.142760in}{0.717006in}}{\pgfqpoint{4.141458in}{0.715704in}}%
\pgfpathcurveto{\pgfqpoint{4.140156in}{0.714402in}}{\pgfqpoint{4.139424in}{0.712635in}}{\pgfqpoint{4.139424in}{0.710794in}}%
\pgfpathcurveto{\pgfqpoint{4.139424in}{0.708952in}}{\pgfqpoint{4.140156in}{0.707185in}}{\pgfqpoint{4.141458in}{0.705883in}}%
\pgfpathcurveto{\pgfqpoint{4.142760in}{0.704581in}}{\pgfqpoint{4.144527in}{0.703849in}}{\pgfqpoint{4.146369in}{0.703849in}}%
\pgfpathlineto{\pgfqpoint{4.146369in}{0.703849in}}%
\pgfpathclose%
\pgfusepath{stroke,fill}%
\end{pgfscope}%
\begin{pgfscope}%
\pgfpathrectangle{\pgfqpoint{0.661006in}{0.524170in}}{\pgfqpoint{4.194036in}{1.071446in}}%
\pgfusepath{clip}%
\pgfsetbuttcap%
\pgfsetroundjoin%
\definecolor{currentfill}{rgb}{0.296851,0.273584,0.468102}%
\pgfsetfillcolor{currentfill}%
\pgfsetfillopacity{0.700000}%
\pgfsetlinewidth{1.003750pt}%
\definecolor{currentstroke}{rgb}{0.296851,0.273584,0.468102}%
\pgfsetstrokecolor{currentstroke}%
\pgfsetstrokeopacity{0.700000}%
\pgfsetdash{}{0pt}%
\pgfpathmoveto{\pgfqpoint{4.148785in}{0.702514in}}%
\pgfpathcurveto{\pgfqpoint{4.150627in}{0.702514in}}{\pgfqpoint{4.152394in}{0.703246in}}{\pgfqpoint{4.153696in}{0.704548in}}%
\pgfpathcurveto{\pgfqpoint{4.154998in}{0.705850in}}{\pgfqpoint{4.155730in}{0.707617in}}{\pgfqpoint{4.155730in}{0.709458in}}%
\pgfpathcurveto{\pgfqpoint{4.155730in}{0.711300in}}{\pgfqpoint{4.154998in}{0.713067in}}{\pgfqpoint{4.153696in}{0.714369in}}%
\pgfpathcurveto{\pgfqpoint{4.152394in}{0.715671in}}{\pgfqpoint{4.150627in}{0.716403in}}{\pgfqpoint{4.148785in}{0.716403in}}%
\pgfpathcurveto{\pgfqpoint{4.146944in}{0.716403in}}{\pgfqpoint{4.145177in}{0.715671in}}{\pgfqpoint{4.143875in}{0.714369in}}%
\pgfpathcurveto{\pgfqpoint{4.142573in}{0.713067in}}{\pgfqpoint{4.141841in}{0.711300in}}{\pgfqpoint{4.141841in}{0.709458in}}%
\pgfpathcurveto{\pgfqpoint{4.141841in}{0.707617in}}{\pgfqpoint{4.142573in}{0.705850in}}{\pgfqpoint{4.143875in}{0.704548in}}%
\pgfpathcurveto{\pgfqpoint{4.145177in}{0.703246in}}{\pgfqpoint{4.146944in}{0.702514in}}{\pgfqpoint{4.148785in}{0.702514in}}%
\pgfpathlineto{\pgfqpoint{4.148785in}{0.702514in}}%
\pgfpathclose%
\pgfusepath{stroke,fill}%
\end{pgfscope}%
\begin{pgfscope}%
\pgfpathrectangle{\pgfqpoint{0.661006in}{0.524170in}}{\pgfqpoint{4.194036in}{1.071446in}}%
\pgfusepath{clip}%
\pgfsetbuttcap%
\pgfsetroundjoin%
\definecolor{currentfill}{rgb}{0.293889,0.269415,0.462891}%
\pgfsetfillcolor{currentfill}%
\pgfsetfillopacity{0.700000}%
\pgfsetlinewidth{1.003750pt}%
\definecolor{currentstroke}{rgb}{0.293889,0.269415,0.462891}%
\pgfsetstrokecolor{currentstroke}%
\pgfsetstrokeopacity{0.700000}%
\pgfsetdash{}{0pt}%
\pgfpathmoveto{\pgfqpoint{4.152597in}{0.701675in}}%
\pgfpathcurveto{\pgfqpoint{4.154438in}{0.701675in}}{\pgfqpoint{4.156205in}{0.702407in}}{\pgfqpoint{4.157507in}{0.703709in}}%
\pgfpathcurveto{\pgfqpoint{4.158809in}{0.705012in}}{\pgfqpoint{4.159541in}{0.706778in}}{\pgfqpoint{4.159541in}{0.708620in}}%
\pgfpathcurveto{\pgfqpoint{4.159541in}{0.710461in}}{\pgfqpoint{4.158809in}{0.712228in}}{\pgfqpoint{4.157507in}{0.713530in}}%
\pgfpathcurveto{\pgfqpoint{4.156205in}{0.714833in}}{\pgfqpoint{4.154438in}{0.715564in}}{\pgfqpoint{4.152597in}{0.715564in}}%
\pgfpathcurveto{\pgfqpoint{4.150755in}{0.715564in}}{\pgfqpoint{4.148988in}{0.714833in}}{\pgfqpoint{4.147686in}{0.713530in}}%
\pgfpathcurveto{\pgfqpoint{4.146384in}{0.712228in}}{\pgfqpoint{4.145652in}{0.710461in}}{\pgfqpoint{4.145652in}{0.708620in}}%
\pgfpathcurveto{\pgfqpoint{4.145652in}{0.706778in}}{\pgfqpoint{4.146384in}{0.705012in}}{\pgfqpoint{4.147686in}{0.703709in}}%
\pgfpathcurveto{\pgfqpoint{4.148988in}{0.702407in}}{\pgfqpoint{4.150755in}{0.701675in}}{\pgfqpoint{4.152597in}{0.701675in}}%
\pgfpathlineto{\pgfqpoint{4.152597in}{0.701675in}}%
\pgfpathclose%
\pgfusepath{stroke,fill}%
\end{pgfscope}%
\begin{pgfscope}%
\pgfpathrectangle{\pgfqpoint{0.661006in}{0.524170in}}{\pgfqpoint{4.194036in}{1.071446in}}%
\pgfusepath{clip}%
\pgfsetbuttcap%
\pgfsetroundjoin%
\definecolor{currentfill}{rgb}{0.293889,0.269415,0.462891}%
\pgfsetfillcolor{currentfill}%
\pgfsetfillopacity{0.700000}%
\pgfsetlinewidth{1.003750pt}%
\definecolor{currentstroke}{rgb}{0.293889,0.269415,0.462891}%
\pgfsetstrokecolor{currentstroke}%
\pgfsetstrokeopacity{0.700000}%
\pgfsetdash{}{0pt}%
\pgfpathmoveto{\pgfqpoint{4.157384in}{0.700263in}}%
\pgfpathcurveto{\pgfqpoint{4.159225in}{0.700263in}}{\pgfqpoint{4.160992in}{0.700994in}}{\pgfqpoint{4.162294in}{0.702297in}}%
\pgfpathcurveto{\pgfqpoint{4.163596in}{0.703599in}}{\pgfqpoint{4.164328in}{0.705365in}}{\pgfqpoint{4.164328in}{0.707207in}}%
\pgfpathcurveto{\pgfqpoint{4.164328in}{0.709049in}}{\pgfqpoint{4.163596in}{0.710815in}}{\pgfqpoint{4.162294in}{0.712118in}}%
\pgfpathcurveto{\pgfqpoint{4.160992in}{0.713420in}}{\pgfqpoint{4.159225in}{0.714152in}}{\pgfqpoint{4.157384in}{0.714152in}}%
\pgfpathcurveto{\pgfqpoint{4.155542in}{0.714152in}}{\pgfqpoint{4.153775in}{0.713420in}}{\pgfqpoint{4.152473in}{0.712118in}}%
\pgfpathcurveto{\pgfqpoint{4.151171in}{0.710815in}}{\pgfqpoint{4.150439in}{0.709049in}}{\pgfqpoint{4.150439in}{0.707207in}}%
\pgfpathcurveto{\pgfqpoint{4.150439in}{0.705365in}}{\pgfqpoint{4.151171in}{0.703599in}}{\pgfqpoint{4.152473in}{0.702297in}}%
\pgfpathcurveto{\pgfqpoint{4.153775in}{0.700994in}}{\pgfqpoint{4.155542in}{0.700263in}}{\pgfqpoint{4.157384in}{0.700263in}}%
\pgfpathlineto{\pgfqpoint{4.157384in}{0.700263in}}%
\pgfpathclose%
\pgfusepath{stroke,fill}%
\end{pgfscope}%
\begin{pgfscope}%
\pgfpathrectangle{\pgfqpoint{0.661006in}{0.524170in}}{\pgfqpoint{4.194036in}{1.071446in}}%
\pgfusepath{clip}%
\pgfsetbuttcap%
\pgfsetroundjoin%
\definecolor{currentfill}{rgb}{0.290914,0.265258,0.457650}%
\pgfsetfillcolor{currentfill}%
\pgfsetfillopacity{0.700000}%
\pgfsetlinewidth{1.003750pt}%
\definecolor{currentstroke}{rgb}{0.290914,0.265258,0.457650}%
\pgfsetstrokecolor{currentstroke}%
\pgfsetstrokeopacity{0.700000}%
\pgfsetdash{}{0pt}%
\pgfpathmoveto{\pgfqpoint{4.158778in}{0.702327in}}%
\pgfpathcurveto{\pgfqpoint{4.160620in}{0.702327in}}{\pgfqpoint{4.162386in}{0.703058in}}{\pgfqpoint{4.163688in}{0.704361in}}%
\pgfpathcurveto{\pgfqpoint{4.164991in}{0.705663in}}{\pgfqpoint{4.165722in}{0.707429in}}{\pgfqpoint{4.165722in}{0.709271in}}%
\pgfpathcurveto{\pgfqpoint{4.165722in}{0.711113in}}{\pgfqpoint{4.164991in}{0.712879in}}{\pgfqpoint{4.163688in}{0.714181in}}%
\pgfpathcurveto{\pgfqpoint{4.162386in}{0.715484in}}{\pgfqpoint{4.160620in}{0.716215in}}{\pgfqpoint{4.158778in}{0.716215in}}%
\pgfpathcurveto{\pgfqpoint{4.156936in}{0.716215in}}{\pgfqpoint{4.155170in}{0.715484in}}{\pgfqpoint{4.153868in}{0.714181in}}%
\pgfpathcurveto{\pgfqpoint{4.152565in}{0.712879in}}{\pgfqpoint{4.151834in}{0.711113in}}{\pgfqpoint{4.151834in}{0.709271in}}%
\pgfpathcurveto{\pgfqpoint{4.151834in}{0.707429in}}{\pgfqpoint{4.152565in}{0.705663in}}{\pgfqpoint{4.153868in}{0.704361in}}%
\pgfpathcurveto{\pgfqpoint{4.155170in}{0.703058in}}{\pgfqpoint{4.156936in}{0.702327in}}{\pgfqpoint{4.158778in}{0.702327in}}%
\pgfpathlineto{\pgfqpoint{4.158778in}{0.702327in}}%
\pgfpathclose%
\pgfusepath{stroke,fill}%
\end{pgfscope}%
\begin{pgfscope}%
\pgfpathrectangle{\pgfqpoint{0.661006in}{0.524170in}}{\pgfqpoint{4.194036in}{1.071446in}}%
\pgfusepath{clip}%
\pgfsetbuttcap%
\pgfsetroundjoin%
\definecolor{currentfill}{rgb}{0.290914,0.265258,0.457650}%
\pgfsetfillcolor{currentfill}%
\pgfsetfillopacity{0.700000}%
\pgfsetlinewidth{1.003750pt}%
\definecolor{currentstroke}{rgb}{0.290914,0.265258,0.457650}%
\pgfsetstrokecolor{currentstroke}%
\pgfsetstrokeopacity{0.700000}%
\pgfsetdash{}{0pt}%
\pgfpathmoveto{\pgfqpoint{4.143859in}{0.704807in}}%
\pgfpathcurveto{\pgfqpoint{4.145701in}{0.704807in}}{\pgfqpoint{4.147467in}{0.705538in}}{\pgfqpoint{4.148769in}{0.706841in}}%
\pgfpathcurveto{\pgfqpoint{4.150072in}{0.708143in}}{\pgfqpoint{4.150803in}{0.709909in}}{\pgfqpoint{4.150803in}{0.711751in}}%
\pgfpathcurveto{\pgfqpoint{4.150803in}{0.713593in}}{\pgfqpoint{4.150072in}{0.715359in}}{\pgfqpoint{4.148769in}{0.716662in}}%
\pgfpathcurveto{\pgfqpoint{4.147467in}{0.717964in}}{\pgfqpoint{4.145701in}{0.718696in}}{\pgfqpoint{4.143859in}{0.718696in}}%
\pgfpathcurveto{\pgfqpoint{4.142017in}{0.718696in}}{\pgfqpoint{4.140251in}{0.717964in}}{\pgfqpoint{4.138948in}{0.716662in}}%
\pgfpathcurveto{\pgfqpoint{4.137646in}{0.715359in}}{\pgfqpoint{4.136914in}{0.713593in}}{\pgfqpoint{4.136914in}{0.711751in}}%
\pgfpathcurveto{\pgfqpoint{4.136914in}{0.709909in}}{\pgfqpoint{4.137646in}{0.708143in}}{\pgfqpoint{4.138948in}{0.706841in}}%
\pgfpathcurveto{\pgfqpoint{4.140251in}{0.705538in}}{\pgfqpoint{4.142017in}{0.704807in}}{\pgfqpoint{4.143859in}{0.704807in}}%
\pgfpathlineto{\pgfqpoint{4.143859in}{0.704807in}}%
\pgfpathclose%
\pgfusepath{stroke,fill}%
\end{pgfscope}%
\begin{pgfscope}%
\pgfpathrectangle{\pgfqpoint{0.661006in}{0.524170in}}{\pgfqpoint{4.194036in}{1.071446in}}%
\pgfusepath{clip}%
\pgfsetbuttcap%
\pgfsetroundjoin%
\definecolor{currentfill}{rgb}{0.287925,0.261113,0.452379}%
\pgfsetfillcolor{currentfill}%
\pgfsetfillopacity{0.700000}%
\pgfsetlinewidth{1.003750pt}%
\definecolor{currentstroke}{rgb}{0.287925,0.261113,0.452379}%
\pgfsetstrokecolor{currentstroke}%
\pgfsetstrokeopacity{0.700000}%
\pgfsetdash{}{0pt}%
\pgfpathmoveto{\pgfqpoint{4.132286in}{0.706146in}}%
\pgfpathcurveto{\pgfqpoint{4.134128in}{0.706146in}}{\pgfqpoint{4.135894in}{0.706877in}}{\pgfqpoint{4.137196in}{0.708180in}}%
\pgfpathcurveto{\pgfqpoint{4.138499in}{0.709482in}}{\pgfqpoint{4.139230in}{0.711248in}}{\pgfqpoint{4.139230in}{0.713090in}}%
\pgfpathcurveto{\pgfqpoint{4.139230in}{0.714932in}}{\pgfqpoint{4.138499in}{0.716698in}}{\pgfqpoint{4.137196in}{0.718001in}}%
\pgfpathcurveto{\pgfqpoint{4.135894in}{0.719303in}}{\pgfqpoint{4.134128in}{0.720035in}}{\pgfqpoint{4.132286in}{0.720035in}}%
\pgfpathcurveto{\pgfqpoint{4.130444in}{0.720035in}}{\pgfqpoint{4.128678in}{0.719303in}}{\pgfqpoint{4.127376in}{0.718001in}}%
\pgfpathcurveto{\pgfqpoint{4.126073in}{0.716698in}}{\pgfqpoint{4.125342in}{0.714932in}}{\pgfqpoint{4.125342in}{0.713090in}}%
\pgfpathcurveto{\pgfqpoint{4.125342in}{0.711248in}}{\pgfqpoint{4.126073in}{0.709482in}}{\pgfqpoint{4.127376in}{0.708180in}}%
\pgfpathcurveto{\pgfqpoint{4.128678in}{0.706877in}}{\pgfqpoint{4.130444in}{0.706146in}}{\pgfqpoint{4.132286in}{0.706146in}}%
\pgfpathlineto{\pgfqpoint{4.132286in}{0.706146in}}%
\pgfpathclose%
\pgfusepath{stroke,fill}%
\end{pgfscope}%
\begin{pgfscope}%
\pgfpathrectangle{\pgfqpoint{0.661006in}{0.524170in}}{\pgfqpoint{4.194036in}{1.071446in}}%
\pgfusepath{clip}%
\pgfsetbuttcap%
\pgfsetroundjoin%
\definecolor{currentfill}{rgb}{0.287925,0.261113,0.452379}%
\pgfsetfillcolor{currentfill}%
\pgfsetfillopacity{0.700000}%
\pgfsetlinewidth{1.003750pt}%
\definecolor{currentstroke}{rgb}{0.287925,0.261113,0.452379}%
\pgfsetstrokecolor{currentstroke}%
\pgfsetstrokeopacity{0.700000}%
\pgfsetdash{}{0pt}%
\pgfpathmoveto{\pgfqpoint{4.141953in}{0.703651in}}%
\pgfpathcurveto{\pgfqpoint{4.143795in}{0.703651in}}{\pgfqpoint{4.145561in}{0.704383in}}{\pgfqpoint{4.146864in}{0.705685in}}%
\pgfpathcurveto{\pgfqpoint{4.148166in}{0.706988in}}{\pgfqpoint{4.148898in}{0.708754in}}{\pgfqpoint{4.148898in}{0.710596in}}%
\pgfpathcurveto{\pgfqpoint{4.148898in}{0.712438in}}{\pgfqpoint{4.148166in}{0.714204in}}{\pgfqpoint{4.146864in}{0.715506in}}%
\pgfpathcurveto{\pgfqpoint{4.145561in}{0.716809in}}{\pgfqpoint{4.143795in}{0.717540in}}{\pgfqpoint{4.141953in}{0.717540in}}%
\pgfpathcurveto{\pgfqpoint{4.140112in}{0.717540in}}{\pgfqpoint{4.138345in}{0.716809in}}{\pgfqpoint{4.137043in}{0.715506in}}%
\pgfpathcurveto{\pgfqpoint{4.135741in}{0.714204in}}{\pgfqpoint{4.135009in}{0.712438in}}{\pgfqpoint{4.135009in}{0.710596in}}%
\pgfpathcurveto{\pgfqpoint{4.135009in}{0.708754in}}{\pgfqpoint{4.135741in}{0.706988in}}{\pgfqpoint{4.137043in}{0.705685in}}%
\pgfpathcurveto{\pgfqpoint{4.138345in}{0.704383in}}{\pgfqpoint{4.140112in}{0.703651in}}{\pgfqpoint{4.141953in}{0.703651in}}%
\pgfpathlineto{\pgfqpoint{4.141953in}{0.703651in}}%
\pgfpathclose%
\pgfusepath{stroke,fill}%
\end{pgfscope}%
\begin{pgfscope}%
\pgfpathrectangle{\pgfqpoint{0.661006in}{0.524170in}}{\pgfqpoint{4.194036in}{1.071446in}}%
\pgfusepath{clip}%
\pgfsetbuttcap%
\pgfsetroundjoin%
\definecolor{currentfill}{rgb}{0.287925,0.261113,0.452379}%
\pgfsetfillcolor{currentfill}%
\pgfsetfillopacity{0.700000}%
\pgfsetlinewidth{1.003750pt}%
\definecolor{currentstroke}{rgb}{0.287925,0.261113,0.452379}%
\pgfsetstrokecolor{currentstroke}%
\pgfsetstrokeopacity{0.700000}%
\pgfsetdash{}{0pt}%
\pgfpathmoveto{\pgfqpoint{4.137491in}{0.703046in}}%
\pgfpathcurveto{\pgfqpoint{4.139333in}{0.703046in}}{\pgfqpoint{4.141100in}{0.703778in}}{\pgfqpoint{4.142402in}{0.705080in}}%
\pgfpathcurveto{\pgfqpoint{4.143704in}{0.706383in}}{\pgfqpoint{4.144436in}{0.708149in}}{\pgfqpoint{4.144436in}{0.709991in}}%
\pgfpathcurveto{\pgfqpoint{4.144436in}{0.711833in}}{\pgfqpoint{4.143704in}{0.713599in}}{\pgfqpoint{4.142402in}{0.714901in}}%
\pgfpathcurveto{\pgfqpoint{4.141100in}{0.716204in}}{\pgfqpoint{4.139333in}{0.716935in}}{\pgfqpoint{4.137491in}{0.716935in}}%
\pgfpathcurveto{\pgfqpoint{4.135650in}{0.716935in}}{\pgfqpoint{4.133883in}{0.716204in}}{\pgfqpoint{4.132581in}{0.714901in}}%
\pgfpathcurveto{\pgfqpoint{4.131279in}{0.713599in}}{\pgfqpoint{4.130547in}{0.711833in}}{\pgfqpoint{4.130547in}{0.709991in}}%
\pgfpathcurveto{\pgfqpoint{4.130547in}{0.708149in}}{\pgfqpoint{4.131279in}{0.706383in}}{\pgfqpoint{4.132581in}{0.705080in}}%
\pgfpathcurveto{\pgfqpoint{4.133883in}{0.703778in}}{\pgfqpoint{4.135650in}{0.703046in}}{\pgfqpoint{4.137491in}{0.703046in}}%
\pgfpathlineto{\pgfqpoint{4.137491in}{0.703046in}}%
\pgfpathclose%
\pgfusepath{stroke,fill}%
\end{pgfscope}%
\begin{pgfscope}%
\pgfpathrectangle{\pgfqpoint{0.661006in}{0.524170in}}{\pgfqpoint{4.194036in}{1.071446in}}%
\pgfusepath{clip}%
\pgfsetbuttcap%
\pgfsetroundjoin%
\definecolor{currentfill}{rgb}{0.287925,0.261113,0.452379}%
\pgfsetfillcolor{currentfill}%
\pgfsetfillopacity{0.700000}%
\pgfsetlinewidth{1.003750pt}%
\definecolor{currentstroke}{rgb}{0.287925,0.261113,0.452379}%
\pgfsetstrokecolor{currentstroke}%
\pgfsetstrokeopacity{0.700000}%
\pgfsetdash{}{0pt}%
\pgfpathmoveto{\pgfqpoint{4.152271in}{0.701517in}}%
\pgfpathcurveto{\pgfqpoint{4.154113in}{0.701517in}}{\pgfqpoint{4.155879in}{0.702249in}}{\pgfqpoint{4.157182in}{0.703551in}}%
\pgfpathcurveto{\pgfqpoint{4.158484in}{0.704854in}}{\pgfqpoint{4.159216in}{0.706620in}}{\pgfqpoint{4.159216in}{0.708462in}}%
\pgfpathcurveto{\pgfqpoint{4.159216in}{0.710304in}}{\pgfqpoint{4.158484in}{0.712070in}}{\pgfqpoint{4.157182in}{0.713372in}}%
\pgfpathcurveto{\pgfqpoint{4.155879in}{0.714675in}}{\pgfqpoint{4.154113in}{0.715406in}}{\pgfqpoint{4.152271in}{0.715406in}}%
\pgfpathcurveto{\pgfqpoint{4.150430in}{0.715406in}}{\pgfqpoint{4.148663in}{0.714675in}}{\pgfqpoint{4.147361in}{0.713372in}}%
\pgfpathcurveto{\pgfqpoint{4.146058in}{0.712070in}}{\pgfqpoint{4.145327in}{0.710304in}}{\pgfqpoint{4.145327in}{0.708462in}}%
\pgfpathcurveto{\pgfqpoint{4.145327in}{0.706620in}}{\pgfqpoint{4.146058in}{0.704854in}}{\pgfqpoint{4.147361in}{0.703551in}}%
\pgfpathcurveto{\pgfqpoint{4.148663in}{0.702249in}}{\pgfqpoint{4.150430in}{0.701517in}}{\pgfqpoint{4.152271in}{0.701517in}}%
\pgfpathlineto{\pgfqpoint{4.152271in}{0.701517in}}%
\pgfpathclose%
\pgfusepath{stroke,fill}%
\end{pgfscope}%
\begin{pgfscope}%
\pgfpathrectangle{\pgfqpoint{0.661006in}{0.524170in}}{\pgfqpoint{4.194036in}{1.071446in}}%
\pgfusepath{clip}%
\pgfsetbuttcap%
\pgfsetroundjoin%
\definecolor{currentfill}{rgb}{0.287925,0.261113,0.452379}%
\pgfsetfillcolor{currentfill}%
\pgfsetfillopacity{0.700000}%
\pgfsetlinewidth{1.003750pt}%
\definecolor{currentstroke}{rgb}{0.287925,0.261113,0.452379}%
\pgfsetstrokecolor{currentstroke}%
\pgfsetstrokeopacity{0.700000}%
\pgfsetdash{}{0pt}%
\pgfpathmoveto{\pgfqpoint{4.155989in}{0.699002in}}%
\pgfpathcurveto{\pgfqpoint{4.157831in}{0.699002in}}{\pgfqpoint{4.159598in}{0.699734in}}{\pgfqpoint{4.160900in}{0.701036in}}%
\pgfpathcurveto{\pgfqpoint{4.162202in}{0.702338in}}{\pgfqpoint{4.162934in}{0.704105in}}{\pgfqpoint{4.162934in}{0.705947in}}%
\pgfpathcurveto{\pgfqpoint{4.162934in}{0.707788in}}{\pgfqpoint{4.162202in}{0.709555in}}{\pgfqpoint{4.160900in}{0.710857in}}%
\pgfpathcurveto{\pgfqpoint{4.159598in}{0.712159in}}{\pgfqpoint{4.157831in}{0.712891in}}{\pgfqpoint{4.155989in}{0.712891in}}%
\pgfpathcurveto{\pgfqpoint{4.154148in}{0.712891in}}{\pgfqpoint{4.152381in}{0.712159in}}{\pgfqpoint{4.151079in}{0.710857in}}%
\pgfpathcurveto{\pgfqpoint{4.149777in}{0.709555in}}{\pgfqpoint{4.149045in}{0.707788in}}{\pgfqpoint{4.149045in}{0.705947in}}%
\pgfpathcurveto{\pgfqpoint{4.149045in}{0.704105in}}{\pgfqpoint{4.149777in}{0.702338in}}{\pgfqpoint{4.151079in}{0.701036in}}%
\pgfpathcurveto{\pgfqpoint{4.152381in}{0.699734in}}{\pgfqpoint{4.154148in}{0.699002in}}{\pgfqpoint{4.155989in}{0.699002in}}%
\pgfpathlineto{\pgfqpoint{4.155989in}{0.699002in}}%
\pgfpathclose%
\pgfusepath{stroke,fill}%
\end{pgfscope}%
\begin{pgfscope}%
\pgfpathrectangle{\pgfqpoint{0.661006in}{0.524170in}}{\pgfqpoint{4.194036in}{1.071446in}}%
\pgfusepath{clip}%
\pgfsetbuttcap%
\pgfsetroundjoin%
\definecolor{currentfill}{rgb}{0.284923,0.256981,0.447080}%
\pgfsetfillcolor{currentfill}%
\pgfsetfillopacity{0.700000}%
\pgfsetlinewidth{1.003750pt}%
\definecolor{currentstroke}{rgb}{0.284923,0.256981,0.447080}%
\pgfsetstrokecolor{currentstroke}%
\pgfsetstrokeopacity{0.700000}%
\pgfsetdash{}{0pt}%
\pgfpathmoveto{\pgfqpoint{4.170444in}{0.697141in}}%
\pgfpathcurveto{\pgfqpoint{4.172285in}{0.697141in}}{\pgfqpoint{4.174052in}{0.697873in}}{\pgfqpoint{4.175354in}{0.699175in}}%
\pgfpathcurveto{\pgfqpoint{4.176657in}{0.700478in}}{\pgfqpoint{4.177388in}{0.702244in}}{\pgfqpoint{4.177388in}{0.704086in}}%
\pgfpathcurveto{\pgfqpoint{4.177388in}{0.705928in}}{\pgfqpoint{4.176657in}{0.707694in}}{\pgfqpoint{4.175354in}{0.708996in}}%
\pgfpathcurveto{\pgfqpoint{4.174052in}{0.710299in}}{\pgfqpoint{4.172285in}{0.711030in}}{\pgfqpoint{4.170444in}{0.711030in}}%
\pgfpathcurveto{\pgfqpoint{4.168602in}{0.711030in}}{\pgfqpoint{4.166836in}{0.710299in}}{\pgfqpoint{4.165533in}{0.708996in}}%
\pgfpathcurveto{\pgfqpoint{4.164231in}{0.707694in}}{\pgfqpoint{4.163499in}{0.705928in}}{\pgfqpoint{4.163499in}{0.704086in}}%
\pgfpathcurveto{\pgfqpoint{4.163499in}{0.702244in}}{\pgfqpoint{4.164231in}{0.700478in}}{\pgfqpoint{4.165533in}{0.699175in}}%
\pgfpathcurveto{\pgfqpoint{4.166836in}{0.697873in}}{\pgfqpoint{4.168602in}{0.697141in}}{\pgfqpoint{4.170444in}{0.697141in}}%
\pgfpathlineto{\pgfqpoint{4.170444in}{0.697141in}}%
\pgfpathclose%
\pgfusepath{stroke,fill}%
\end{pgfscope}%
\begin{pgfscope}%
\pgfpathrectangle{\pgfqpoint{0.661006in}{0.524170in}}{\pgfqpoint{4.194036in}{1.071446in}}%
\pgfusepath{clip}%
\pgfsetbuttcap%
\pgfsetroundjoin%
\definecolor{currentfill}{rgb}{0.284923,0.256981,0.447080}%
\pgfsetfillcolor{currentfill}%
\pgfsetfillopacity{0.700000}%
\pgfsetlinewidth{1.003750pt}%
\definecolor{currentstroke}{rgb}{0.284923,0.256981,0.447080}%
\pgfsetstrokecolor{currentstroke}%
\pgfsetstrokeopacity{0.700000}%
\pgfsetdash{}{0pt}%
\pgfpathmoveto{\pgfqpoint{4.170444in}{0.698938in}}%
\pgfpathcurveto{\pgfqpoint{4.172285in}{0.698938in}}{\pgfqpoint{4.174052in}{0.699669in}}{\pgfqpoint{4.175354in}{0.700972in}}%
\pgfpathcurveto{\pgfqpoint{4.176657in}{0.702274in}}{\pgfqpoint{4.177388in}{0.704040in}}{\pgfqpoint{4.177388in}{0.705882in}}%
\pgfpathcurveto{\pgfqpoint{4.177388in}{0.707724in}}{\pgfqpoint{4.176657in}{0.709490in}}{\pgfqpoint{4.175354in}{0.710793in}}%
\pgfpathcurveto{\pgfqpoint{4.174052in}{0.712095in}}{\pgfqpoint{4.172285in}{0.712827in}}{\pgfqpoint{4.170444in}{0.712827in}}%
\pgfpathcurveto{\pgfqpoint{4.168602in}{0.712827in}}{\pgfqpoint{4.166836in}{0.712095in}}{\pgfqpoint{4.165533in}{0.710793in}}%
\pgfpathcurveto{\pgfqpoint{4.164231in}{0.709490in}}{\pgfqpoint{4.163499in}{0.707724in}}{\pgfqpoint{4.163499in}{0.705882in}}%
\pgfpathcurveto{\pgfqpoint{4.163499in}{0.704040in}}{\pgfqpoint{4.164231in}{0.702274in}}{\pgfqpoint{4.165533in}{0.700972in}}%
\pgfpathcurveto{\pgfqpoint{4.166836in}{0.699669in}}{\pgfqpoint{4.168602in}{0.698938in}}{\pgfqpoint{4.170444in}{0.698938in}}%
\pgfpathlineto{\pgfqpoint{4.170444in}{0.698938in}}%
\pgfpathclose%
\pgfusepath{stroke,fill}%
\end{pgfscope}%
\begin{pgfscope}%
\pgfpathrectangle{\pgfqpoint{0.661006in}{0.524170in}}{\pgfqpoint{4.194036in}{1.071446in}}%
\pgfusepath{clip}%
\pgfsetbuttcap%
\pgfsetroundjoin%
\definecolor{currentfill}{rgb}{0.281905,0.252862,0.441751}%
\pgfsetfillcolor{currentfill}%
\pgfsetfillopacity{0.700000}%
\pgfsetlinewidth{1.003750pt}%
\definecolor{currentstroke}{rgb}{0.281905,0.252862,0.441751}%
\pgfsetstrokecolor{currentstroke}%
\pgfsetstrokeopacity{0.700000}%
\pgfsetdash{}{0pt}%
\pgfpathmoveto{\pgfqpoint{4.158313in}{0.701329in}}%
\pgfpathcurveto{\pgfqpoint{4.160155in}{0.701329in}}{\pgfqpoint{4.161921in}{0.702060in}}{\pgfqpoint{4.163224in}{0.703363in}}%
\pgfpathcurveto{\pgfqpoint{4.164526in}{0.704665in}}{\pgfqpoint{4.165258in}{0.706432in}}{\pgfqpoint{4.165258in}{0.708273in}}%
\pgfpathcurveto{\pgfqpoint{4.165258in}{0.710115in}}{\pgfqpoint{4.164526in}{0.711881in}}{\pgfqpoint{4.163224in}{0.713184in}}%
\pgfpathcurveto{\pgfqpoint{4.161921in}{0.714486in}}{\pgfqpoint{4.160155in}{0.715218in}}{\pgfqpoint{4.158313in}{0.715218in}}%
\pgfpathcurveto{\pgfqpoint{4.156472in}{0.715218in}}{\pgfqpoint{4.154705in}{0.714486in}}{\pgfqpoint{4.153403in}{0.713184in}}%
\pgfpathcurveto{\pgfqpoint{4.152101in}{0.711881in}}{\pgfqpoint{4.151369in}{0.710115in}}{\pgfqpoint{4.151369in}{0.708273in}}%
\pgfpathcurveto{\pgfqpoint{4.151369in}{0.706432in}}{\pgfqpoint{4.152101in}{0.704665in}}{\pgfqpoint{4.153403in}{0.703363in}}%
\pgfpathcurveto{\pgfqpoint{4.154705in}{0.702060in}}{\pgfqpoint{4.156472in}{0.701329in}}{\pgfqpoint{4.158313in}{0.701329in}}%
\pgfpathlineto{\pgfqpoint{4.158313in}{0.701329in}}%
\pgfpathclose%
\pgfusepath{stroke,fill}%
\end{pgfscope}%
\begin{pgfscope}%
\pgfpathrectangle{\pgfqpoint{0.661006in}{0.524170in}}{\pgfqpoint{4.194036in}{1.071446in}}%
\pgfusepath{clip}%
\pgfsetbuttcap%
\pgfsetroundjoin%
\definecolor{currentfill}{rgb}{0.281905,0.252862,0.441751}%
\pgfsetfillcolor{currentfill}%
\pgfsetfillopacity{0.700000}%
\pgfsetlinewidth{1.003750pt}%
\definecolor{currentstroke}{rgb}{0.281905,0.252862,0.441751}%
\pgfsetstrokecolor{currentstroke}%
\pgfsetstrokeopacity{0.700000}%
\pgfsetdash{}{0pt}%
\pgfpathmoveto{\pgfqpoint{4.156268in}{0.700579in}}%
\pgfpathcurveto{\pgfqpoint{4.158110in}{0.700579in}}{\pgfqpoint{4.159876in}{0.701311in}}{\pgfqpoint{4.161179in}{0.702613in}}%
\pgfpathcurveto{\pgfqpoint{4.162481in}{0.703915in}}{\pgfqpoint{4.163213in}{0.705682in}}{\pgfqpoint{4.163213in}{0.707524in}}%
\pgfpathcurveto{\pgfqpoint{4.163213in}{0.709365in}}{\pgfqpoint{4.162481in}{0.711132in}}{\pgfqpoint{4.161179in}{0.712434in}}%
\pgfpathcurveto{\pgfqpoint{4.159876in}{0.713736in}}{\pgfqpoint{4.158110in}{0.714468in}}{\pgfqpoint{4.156268in}{0.714468in}}%
\pgfpathcurveto{\pgfqpoint{4.154427in}{0.714468in}}{\pgfqpoint{4.152660in}{0.713736in}}{\pgfqpoint{4.151358in}{0.712434in}}%
\pgfpathcurveto{\pgfqpoint{4.150056in}{0.711132in}}{\pgfqpoint{4.149324in}{0.709365in}}{\pgfqpoint{4.149324in}{0.707524in}}%
\pgfpathcurveto{\pgfqpoint{4.149324in}{0.705682in}}{\pgfqpoint{4.150056in}{0.703915in}}{\pgfqpoint{4.151358in}{0.702613in}}%
\pgfpathcurveto{\pgfqpoint{4.152660in}{0.701311in}}{\pgfqpoint{4.154427in}{0.700579in}}{\pgfqpoint{4.156268in}{0.700579in}}%
\pgfpathlineto{\pgfqpoint{4.156268in}{0.700579in}}%
\pgfpathclose%
\pgfusepath{stroke,fill}%
\end{pgfscope}%
\begin{pgfscope}%
\pgfpathrectangle{\pgfqpoint{0.661006in}{0.524170in}}{\pgfqpoint{4.194036in}{1.071446in}}%
\pgfusepath{clip}%
\pgfsetbuttcap%
\pgfsetroundjoin%
\definecolor{currentfill}{rgb}{0.281905,0.252862,0.441751}%
\pgfsetfillcolor{currentfill}%
\pgfsetfillopacity{0.700000}%
\pgfsetlinewidth{1.003750pt}%
\definecolor{currentstroke}{rgb}{0.281905,0.252862,0.441751}%
\pgfsetstrokecolor{currentstroke}%
\pgfsetstrokeopacity{0.700000}%
\pgfsetdash{}{0pt}%
\pgfpathmoveto{\pgfqpoint{4.141203in}{0.704743in}}%
\pgfpathcurveto{\pgfqpoint{4.143045in}{0.704743in}}{\pgfqpoint{4.144811in}{0.705475in}}{\pgfqpoint{4.146113in}{0.706777in}}%
\pgfpathcurveto{\pgfqpoint{4.147416in}{0.708080in}}{\pgfqpoint{4.148147in}{0.709846in}}{\pgfqpoint{4.148147in}{0.711688in}}%
\pgfpathcurveto{\pgfqpoint{4.148147in}{0.713529in}}{\pgfqpoint{4.147416in}{0.715296in}}{\pgfqpoint{4.146113in}{0.716598in}}%
\pgfpathcurveto{\pgfqpoint{4.144811in}{0.717900in}}{\pgfqpoint{4.143045in}{0.718632in}}{\pgfqpoint{4.141203in}{0.718632in}}%
\pgfpathcurveto{\pgfqpoint{4.139361in}{0.718632in}}{\pgfqpoint{4.137595in}{0.717900in}}{\pgfqpoint{4.136293in}{0.716598in}}%
\pgfpathcurveto{\pgfqpoint{4.134990in}{0.715296in}}{\pgfqpoint{4.134259in}{0.713529in}}{\pgfqpoint{4.134259in}{0.711688in}}%
\pgfpathcurveto{\pgfqpoint{4.134259in}{0.709846in}}{\pgfqpoint{4.134990in}{0.708080in}}{\pgfqpoint{4.136293in}{0.706777in}}%
\pgfpathcurveto{\pgfqpoint{4.137595in}{0.705475in}}{\pgfqpoint{4.139361in}{0.704743in}}{\pgfqpoint{4.141203in}{0.704743in}}%
\pgfpathlineto{\pgfqpoint{4.141203in}{0.704743in}}%
\pgfpathclose%
\pgfusepath{stroke,fill}%
\end{pgfscope}%
\begin{pgfscope}%
\pgfpathrectangle{\pgfqpoint{0.661006in}{0.524170in}}{\pgfqpoint{4.194036in}{1.071446in}}%
\pgfusepath{clip}%
\pgfsetbuttcap%
\pgfsetroundjoin%
\definecolor{currentfill}{rgb}{0.281905,0.252862,0.441751}%
\pgfsetfillcolor{currentfill}%
\pgfsetfillopacity{0.700000}%
\pgfsetlinewidth{1.003750pt}%
\definecolor{currentstroke}{rgb}{0.281905,0.252862,0.441751}%
\pgfsetstrokecolor{currentstroke}%
\pgfsetstrokeopacity{0.700000}%
\pgfsetdash{}{0pt}%
\pgfpathmoveto{\pgfqpoint{4.111697in}{0.712322in}}%
\pgfpathcurveto{\pgfqpoint{4.113538in}{0.712322in}}{\pgfqpoint{4.115305in}{0.713054in}}{\pgfqpoint{4.116607in}{0.714356in}}%
\pgfpathcurveto{\pgfqpoint{4.117909in}{0.715658in}}{\pgfqpoint{4.118641in}{0.717425in}}{\pgfqpoint{4.118641in}{0.719267in}}%
\pgfpathcurveto{\pgfqpoint{4.118641in}{0.721108in}}{\pgfqpoint{4.117909in}{0.722875in}}{\pgfqpoint{4.116607in}{0.724177in}}%
\pgfpathcurveto{\pgfqpoint{4.115305in}{0.725479in}}{\pgfqpoint{4.113538in}{0.726211in}}{\pgfqpoint{4.111697in}{0.726211in}}%
\pgfpathcurveto{\pgfqpoint{4.109855in}{0.726211in}}{\pgfqpoint{4.108088in}{0.725479in}}{\pgfqpoint{4.106786in}{0.724177in}}%
\pgfpathcurveto{\pgfqpoint{4.105484in}{0.722875in}}{\pgfqpoint{4.104752in}{0.721108in}}{\pgfqpoint{4.104752in}{0.719267in}}%
\pgfpathcurveto{\pgfqpoint{4.104752in}{0.717425in}}{\pgfqpoint{4.105484in}{0.715658in}}{\pgfqpoint{4.106786in}{0.714356in}}%
\pgfpathcurveto{\pgfqpoint{4.108088in}{0.713054in}}{\pgfqpoint{4.109855in}{0.712322in}}{\pgfqpoint{4.111697in}{0.712322in}}%
\pgfpathlineto{\pgfqpoint{4.111697in}{0.712322in}}%
\pgfpathclose%
\pgfusepath{stroke,fill}%
\end{pgfscope}%
\begin{pgfscope}%
\pgfpathrectangle{\pgfqpoint{0.661006in}{0.524170in}}{\pgfqpoint{4.194036in}{1.071446in}}%
\pgfusepath{clip}%
\pgfsetbuttcap%
\pgfsetroundjoin%
\definecolor{currentfill}{rgb}{0.281905,0.252862,0.441751}%
\pgfsetfillcolor{currentfill}%
\pgfsetfillopacity{0.700000}%
\pgfsetlinewidth{1.003750pt}%
\definecolor{currentstroke}{rgb}{0.281905,0.252862,0.441751}%
\pgfsetstrokecolor{currentstroke}%
\pgfsetstrokeopacity{0.700000}%
\pgfsetdash{}{0pt}%
\pgfpathmoveto{\pgfqpoint{4.085762in}{0.716912in}}%
\pgfpathcurveto{\pgfqpoint{4.087604in}{0.716912in}}{\pgfqpoint{4.089370in}{0.717644in}}{\pgfqpoint{4.090673in}{0.718946in}}%
\pgfpathcurveto{\pgfqpoint{4.091975in}{0.720248in}}{\pgfqpoint{4.092707in}{0.722015in}}{\pgfqpoint{4.092707in}{0.723857in}}%
\pgfpathcurveto{\pgfqpoint{4.092707in}{0.725698in}}{\pgfqpoint{4.091975in}{0.727465in}}{\pgfqpoint{4.090673in}{0.728767in}}%
\pgfpathcurveto{\pgfqpoint{4.089370in}{0.730069in}}{\pgfqpoint{4.087604in}{0.730801in}}{\pgfqpoint{4.085762in}{0.730801in}}%
\pgfpathcurveto{\pgfqpoint{4.083921in}{0.730801in}}{\pgfqpoint{4.082154in}{0.730069in}}{\pgfqpoint{4.080852in}{0.728767in}}%
\pgfpathcurveto{\pgfqpoint{4.079550in}{0.727465in}}{\pgfqpoint{4.078818in}{0.725698in}}{\pgfqpoint{4.078818in}{0.723857in}}%
\pgfpathcurveto{\pgfqpoint{4.078818in}{0.722015in}}{\pgfqpoint{4.079550in}{0.720248in}}{\pgfqpoint{4.080852in}{0.718946in}}%
\pgfpathcurveto{\pgfqpoint{4.082154in}{0.717644in}}{\pgfqpoint{4.083921in}{0.716912in}}{\pgfqpoint{4.085762in}{0.716912in}}%
\pgfpathlineto{\pgfqpoint{4.085762in}{0.716912in}}%
\pgfpathclose%
\pgfusepath{stroke,fill}%
\end{pgfscope}%
\begin{pgfscope}%
\pgfpathrectangle{\pgfqpoint{0.661006in}{0.524170in}}{\pgfqpoint{4.194036in}{1.071446in}}%
\pgfusepath{clip}%
\pgfsetbuttcap%
\pgfsetroundjoin%
\definecolor{currentfill}{rgb}{0.278873,0.248756,0.436393}%
\pgfsetfillcolor{currentfill}%
\pgfsetfillopacity{0.700000}%
\pgfsetlinewidth{1.003750pt}%
\definecolor{currentstroke}{rgb}{0.278873,0.248756,0.436393}%
\pgfsetstrokecolor{currentstroke}%
\pgfsetstrokeopacity{0.700000}%
\pgfsetdash{}{0pt}%
\pgfpathmoveto{\pgfqpoint{4.087389in}{0.717544in}}%
\pgfpathcurveto{\pgfqpoint{4.089231in}{0.717544in}}{\pgfqpoint{4.090997in}{0.718275in}}{\pgfqpoint{4.092299in}{0.719578in}}%
\pgfpathcurveto{\pgfqpoint{4.093602in}{0.720880in}}{\pgfqpoint{4.094333in}{0.722646in}}{\pgfqpoint{4.094333in}{0.724488in}}%
\pgfpathcurveto{\pgfqpoint{4.094333in}{0.726330in}}{\pgfqpoint{4.093602in}{0.728096in}}{\pgfqpoint{4.092299in}{0.729399in}}%
\pgfpathcurveto{\pgfqpoint{4.090997in}{0.730701in}}{\pgfqpoint{4.089231in}{0.731432in}}{\pgfqpoint{4.087389in}{0.731432in}}%
\pgfpathcurveto{\pgfqpoint{4.085547in}{0.731432in}}{\pgfqpoint{4.083781in}{0.730701in}}{\pgfqpoint{4.082479in}{0.729399in}}%
\pgfpathcurveto{\pgfqpoint{4.081176in}{0.728096in}}{\pgfqpoint{4.080445in}{0.726330in}}{\pgfqpoint{4.080445in}{0.724488in}}%
\pgfpathcurveto{\pgfqpoint{4.080445in}{0.722646in}}{\pgfqpoint{4.081176in}{0.720880in}}{\pgfqpoint{4.082479in}{0.719578in}}%
\pgfpathcurveto{\pgfqpoint{4.083781in}{0.718275in}}{\pgfqpoint{4.085547in}{0.717544in}}{\pgfqpoint{4.087389in}{0.717544in}}%
\pgfpathlineto{\pgfqpoint{4.087389in}{0.717544in}}%
\pgfpathclose%
\pgfusepath{stroke,fill}%
\end{pgfscope}%
\begin{pgfscope}%
\pgfpathrectangle{\pgfqpoint{0.661006in}{0.524170in}}{\pgfqpoint{4.194036in}{1.071446in}}%
\pgfusepath{clip}%
\pgfsetbuttcap%
\pgfsetroundjoin%
\definecolor{currentfill}{rgb}{0.278873,0.248756,0.436393}%
\pgfsetfillcolor{currentfill}%
\pgfsetfillopacity{0.700000}%
\pgfsetlinewidth{1.003750pt}%
\definecolor{currentstroke}{rgb}{0.278873,0.248756,0.436393}%
\pgfsetstrokecolor{currentstroke}%
\pgfsetstrokeopacity{0.700000}%
\pgfsetdash{}{0pt}%
\pgfpathmoveto{\pgfqpoint{4.098497in}{0.716389in}}%
\pgfpathcurveto{\pgfqpoint{4.100339in}{0.716389in}}{\pgfqpoint{4.102105in}{0.717120in}}{\pgfqpoint{4.103407in}{0.718422in}}%
\pgfpathcurveto{\pgfqpoint{4.104710in}{0.719725in}}{\pgfqpoint{4.105441in}{0.721491in}}{\pgfqpoint{4.105441in}{0.723333in}}%
\pgfpathcurveto{\pgfqpoint{4.105441in}{0.725175in}}{\pgfqpoint{4.104710in}{0.726941in}}{\pgfqpoint{4.103407in}{0.728243in}}%
\pgfpathcurveto{\pgfqpoint{4.102105in}{0.729546in}}{\pgfqpoint{4.100339in}{0.730277in}}{\pgfqpoint{4.098497in}{0.730277in}}%
\pgfpathcurveto{\pgfqpoint{4.096655in}{0.730277in}}{\pgfqpoint{4.094889in}{0.729546in}}{\pgfqpoint{4.093587in}{0.728243in}}%
\pgfpathcurveto{\pgfqpoint{4.092284in}{0.726941in}}{\pgfqpoint{4.091553in}{0.725175in}}{\pgfqpoint{4.091553in}{0.723333in}}%
\pgfpathcurveto{\pgfqpoint{4.091553in}{0.721491in}}{\pgfqpoint{4.092284in}{0.719725in}}{\pgfqpoint{4.093587in}{0.718422in}}%
\pgfpathcurveto{\pgfqpoint{4.094889in}{0.717120in}}{\pgfqpoint{4.096655in}{0.716389in}}{\pgfqpoint{4.098497in}{0.716389in}}%
\pgfpathlineto{\pgfqpoint{4.098497in}{0.716389in}}%
\pgfpathclose%
\pgfusepath{stroke,fill}%
\end{pgfscope}%
\begin{pgfscope}%
\pgfpathrectangle{\pgfqpoint{0.661006in}{0.524170in}}{\pgfqpoint{4.194036in}{1.071446in}}%
\pgfusepath{clip}%
\pgfsetbuttcap%
\pgfsetroundjoin%
\definecolor{currentfill}{rgb}{0.275826,0.244663,0.431007}%
\pgfsetfillcolor{currentfill}%
\pgfsetfillopacity{0.700000}%
\pgfsetlinewidth{1.003750pt}%
\definecolor{currentstroke}{rgb}{0.275826,0.244663,0.431007}%
\pgfsetstrokecolor{currentstroke}%
\pgfsetstrokeopacity{0.700000}%
\pgfsetdash{}{0pt}%
\pgfpathmoveto{\pgfqpoint{4.105562in}{0.713497in}}%
\pgfpathcurveto{\pgfqpoint{4.107403in}{0.713497in}}{\pgfqpoint{4.109170in}{0.714229in}}{\pgfqpoint{4.110472in}{0.715531in}}%
\pgfpathcurveto{\pgfqpoint{4.111774in}{0.716833in}}{\pgfqpoint{4.112506in}{0.718600in}}{\pgfqpoint{4.112506in}{0.720441in}}%
\pgfpathcurveto{\pgfqpoint{4.112506in}{0.722283in}}{\pgfqpoint{4.111774in}{0.724049in}}{\pgfqpoint{4.110472in}{0.725352in}}%
\pgfpathcurveto{\pgfqpoint{4.109170in}{0.726654in}}{\pgfqpoint{4.107403in}{0.727386in}}{\pgfqpoint{4.105562in}{0.727386in}}%
\pgfpathcurveto{\pgfqpoint{4.103720in}{0.727386in}}{\pgfqpoint{4.101953in}{0.726654in}}{\pgfqpoint{4.100651in}{0.725352in}}%
\pgfpathcurveto{\pgfqpoint{4.099349in}{0.724049in}}{\pgfqpoint{4.098617in}{0.722283in}}{\pgfqpoint{4.098617in}{0.720441in}}%
\pgfpathcurveto{\pgfqpoint{4.098617in}{0.718600in}}{\pgfqpoint{4.099349in}{0.716833in}}{\pgfqpoint{4.100651in}{0.715531in}}%
\pgfpathcurveto{\pgfqpoint{4.101953in}{0.714229in}}{\pgfqpoint{4.103720in}{0.713497in}}{\pgfqpoint{4.105562in}{0.713497in}}%
\pgfpathlineto{\pgfqpoint{4.105562in}{0.713497in}}%
\pgfpathclose%
\pgfusepath{stroke,fill}%
\end{pgfscope}%
\begin{pgfscope}%
\pgfpathrectangle{\pgfqpoint{0.661006in}{0.524170in}}{\pgfqpoint{4.194036in}{1.071446in}}%
\pgfusepath{clip}%
\pgfsetbuttcap%
\pgfsetroundjoin%
\definecolor{currentfill}{rgb}{0.275826,0.244663,0.431007}%
\pgfsetfillcolor{currentfill}%
\pgfsetfillopacity{0.700000}%
\pgfsetlinewidth{1.003750pt}%
\definecolor{currentstroke}{rgb}{0.275826,0.244663,0.431007}%
\pgfsetstrokecolor{currentstroke}%
\pgfsetstrokeopacity{0.700000}%
\pgfsetdash{}{0pt}%
\pgfpathmoveto{\pgfqpoint{4.123037in}{0.708802in}}%
\pgfpathcurveto{\pgfqpoint{4.124879in}{0.708802in}}{\pgfqpoint{4.126645in}{0.709533in}}{\pgfqpoint{4.127947in}{0.710836in}}%
\pgfpathcurveto{\pgfqpoint{4.129250in}{0.712138in}}{\pgfqpoint{4.129981in}{0.713904in}}{\pgfqpoint{4.129981in}{0.715746in}}%
\pgfpathcurveto{\pgfqpoint{4.129981in}{0.717588in}}{\pgfqpoint{4.129250in}{0.719354in}}{\pgfqpoint{4.127947in}{0.720657in}}%
\pgfpathcurveto{\pgfqpoint{4.126645in}{0.721959in}}{\pgfqpoint{4.124879in}{0.722691in}}{\pgfqpoint{4.123037in}{0.722691in}}%
\pgfpathcurveto{\pgfqpoint{4.121195in}{0.722691in}}{\pgfqpoint{4.119429in}{0.721959in}}{\pgfqpoint{4.118127in}{0.720657in}}%
\pgfpathcurveto{\pgfqpoint{4.116824in}{0.719354in}}{\pgfqpoint{4.116093in}{0.717588in}}{\pgfqpoint{4.116093in}{0.715746in}}%
\pgfpathcurveto{\pgfqpoint{4.116093in}{0.713904in}}{\pgfqpoint{4.116824in}{0.712138in}}{\pgfqpoint{4.118127in}{0.710836in}}%
\pgfpathcurveto{\pgfqpoint{4.119429in}{0.709533in}}{\pgfqpoint{4.121195in}{0.708802in}}{\pgfqpoint{4.123037in}{0.708802in}}%
\pgfpathlineto{\pgfqpoint{4.123037in}{0.708802in}}%
\pgfpathclose%
\pgfusepath{stroke,fill}%
\end{pgfscope}%
\begin{pgfscope}%
\pgfpathrectangle{\pgfqpoint{0.661006in}{0.524170in}}{\pgfqpoint{4.194036in}{1.071446in}}%
\pgfusepath{clip}%
\pgfsetbuttcap%
\pgfsetroundjoin%
\definecolor{currentfill}{rgb}{0.275826,0.244663,0.431007}%
\pgfsetfillcolor{currentfill}%
\pgfsetfillopacity{0.700000}%
\pgfsetlinewidth{1.003750pt}%
\definecolor{currentstroke}{rgb}{0.275826,0.244663,0.431007}%
\pgfsetstrokecolor{currentstroke}%
\pgfsetstrokeopacity{0.700000}%
\pgfsetdash{}{0pt}%
\pgfpathmoveto{\pgfqpoint{4.135493in}{0.705205in}}%
\pgfpathcurveto{\pgfqpoint{4.137335in}{0.705205in}}{\pgfqpoint{4.139101in}{0.705937in}}{\pgfqpoint{4.140403in}{0.707239in}}%
\pgfpathcurveto{\pgfqpoint{4.141706in}{0.708541in}}{\pgfqpoint{4.142437in}{0.710308in}}{\pgfqpoint{4.142437in}{0.712150in}}%
\pgfpathcurveto{\pgfqpoint{4.142437in}{0.713991in}}{\pgfqpoint{4.141706in}{0.715758in}}{\pgfqpoint{4.140403in}{0.717060in}}%
\pgfpathcurveto{\pgfqpoint{4.139101in}{0.718362in}}{\pgfqpoint{4.137335in}{0.719094in}}{\pgfqpoint{4.135493in}{0.719094in}}%
\pgfpathcurveto{\pgfqpoint{4.133651in}{0.719094in}}{\pgfqpoint{4.131885in}{0.718362in}}{\pgfqpoint{4.130582in}{0.717060in}}%
\pgfpathcurveto{\pgfqpoint{4.129280in}{0.715758in}}{\pgfqpoint{4.128548in}{0.713991in}}{\pgfqpoint{4.128548in}{0.712150in}}%
\pgfpathcurveto{\pgfqpoint{4.128548in}{0.710308in}}{\pgfqpoint{4.129280in}{0.708541in}}{\pgfqpoint{4.130582in}{0.707239in}}%
\pgfpathcurveto{\pgfqpoint{4.131885in}{0.705937in}}{\pgfqpoint{4.133651in}{0.705205in}}{\pgfqpoint{4.135493in}{0.705205in}}%
\pgfpathlineto{\pgfqpoint{4.135493in}{0.705205in}}%
\pgfpathclose%
\pgfusepath{stroke,fill}%
\end{pgfscope}%
\begin{pgfscope}%
\pgfpathrectangle{\pgfqpoint{0.661006in}{0.524170in}}{\pgfqpoint{4.194036in}{1.071446in}}%
\pgfusepath{clip}%
\pgfsetbuttcap%
\pgfsetroundjoin%
\definecolor{currentfill}{rgb}{0.275826,0.244663,0.431007}%
\pgfsetfillcolor{currentfill}%
\pgfsetfillopacity{0.700000}%
\pgfsetlinewidth{1.003750pt}%
\definecolor{currentstroke}{rgb}{0.275826,0.244663,0.431007}%
\pgfsetstrokecolor{currentstroke}%
\pgfsetstrokeopacity{0.700000}%
\pgfsetdash{}{0pt}%
\pgfpathmoveto{\pgfqpoint{4.152411in}{0.703958in}}%
\pgfpathcurveto{\pgfqpoint{4.154252in}{0.703958in}}{\pgfqpoint{4.156019in}{0.704690in}}{\pgfqpoint{4.157321in}{0.705992in}}%
\pgfpathcurveto{\pgfqpoint{4.158623in}{0.707295in}}{\pgfqpoint{4.159355in}{0.709061in}}{\pgfqpoint{4.159355in}{0.710903in}}%
\pgfpathcurveto{\pgfqpoint{4.159355in}{0.712744in}}{\pgfqpoint{4.158623in}{0.714511in}}{\pgfqpoint{4.157321in}{0.715813in}}%
\pgfpathcurveto{\pgfqpoint{4.156019in}{0.717116in}}{\pgfqpoint{4.154252in}{0.717847in}}{\pgfqpoint{4.152411in}{0.717847in}}%
\pgfpathcurveto{\pgfqpoint{4.150569in}{0.717847in}}{\pgfqpoint{4.148802in}{0.717116in}}{\pgfqpoint{4.147500in}{0.715813in}}%
\pgfpathcurveto{\pgfqpoint{4.146198in}{0.714511in}}{\pgfqpoint{4.145466in}{0.712744in}}{\pgfqpoint{4.145466in}{0.710903in}}%
\pgfpathcurveto{\pgfqpoint{4.145466in}{0.709061in}}{\pgfqpoint{4.146198in}{0.707295in}}{\pgfqpoint{4.147500in}{0.705992in}}%
\pgfpathcurveto{\pgfqpoint{4.148802in}{0.704690in}}{\pgfqpoint{4.150569in}{0.703958in}}{\pgfqpoint{4.152411in}{0.703958in}}%
\pgfpathlineto{\pgfqpoint{4.152411in}{0.703958in}}%
\pgfpathclose%
\pgfusepath{stroke,fill}%
\end{pgfscope}%
\begin{pgfscope}%
\pgfpathrectangle{\pgfqpoint{0.661006in}{0.524170in}}{\pgfqpoint{4.194036in}{1.071446in}}%
\pgfusepath{clip}%
\pgfsetbuttcap%
\pgfsetroundjoin%
\definecolor{currentfill}{rgb}{0.275826,0.244663,0.431007}%
\pgfsetfillcolor{currentfill}%
\pgfsetfillopacity{0.700000}%
\pgfsetlinewidth{1.003750pt}%
\definecolor{currentstroke}{rgb}{0.275826,0.244663,0.431007}%
\pgfsetstrokecolor{currentstroke}%
\pgfsetstrokeopacity{0.700000}%
\pgfsetdash{}{0pt}%
\pgfpathmoveto{\pgfqpoint{4.158964in}{0.702049in}}%
\pgfpathcurveto{\pgfqpoint{4.160806in}{0.702049in}}{\pgfqpoint{4.162572in}{0.702781in}}{\pgfqpoint{4.163874in}{0.704083in}}%
\pgfpathcurveto{\pgfqpoint{4.165177in}{0.705386in}}{\pgfqpoint{4.165908in}{0.707152in}}{\pgfqpoint{4.165908in}{0.708994in}}%
\pgfpathcurveto{\pgfqpoint{4.165908in}{0.710835in}}{\pgfqpoint{4.165177in}{0.712602in}}{\pgfqpoint{4.163874in}{0.713904in}}%
\pgfpathcurveto{\pgfqpoint{4.162572in}{0.715206in}}{\pgfqpoint{4.160806in}{0.715938in}}{\pgfqpoint{4.158964in}{0.715938in}}%
\pgfpathcurveto{\pgfqpoint{4.157122in}{0.715938in}}{\pgfqpoint{4.155356in}{0.715206in}}{\pgfqpoint{4.154053in}{0.713904in}}%
\pgfpathcurveto{\pgfqpoint{4.152751in}{0.712602in}}{\pgfqpoint{4.152019in}{0.710835in}}{\pgfqpoint{4.152019in}{0.708994in}}%
\pgfpathcurveto{\pgfqpoint{4.152019in}{0.707152in}}{\pgfqpoint{4.152751in}{0.705386in}}{\pgfqpoint{4.154053in}{0.704083in}}%
\pgfpathcurveto{\pgfqpoint{4.155356in}{0.702781in}}{\pgfqpoint{4.157122in}{0.702049in}}{\pgfqpoint{4.158964in}{0.702049in}}%
\pgfpathlineto{\pgfqpoint{4.158964in}{0.702049in}}%
\pgfpathclose%
\pgfusepath{stroke,fill}%
\end{pgfscope}%
\begin{pgfscope}%
\pgfpathrectangle{\pgfqpoint{0.661006in}{0.524170in}}{\pgfqpoint{4.194036in}{1.071446in}}%
\pgfusepath{clip}%
\pgfsetbuttcap%
\pgfsetroundjoin%
\definecolor{currentfill}{rgb}{0.272762,0.240584,0.425592}%
\pgfsetfillcolor{currentfill}%
\pgfsetfillopacity{0.700000}%
\pgfsetlinewidth{1.003750pt}%
\definecolor{currentstroke}{rgb}{0.272762,0.240584,0.425592}%
\pgfsetstrokecolor{currentstroke}%
\pgfsetstrokeopacity{0.700000}%
\pgfsetdash{}{0pt}%
\pgfpathmoveto{\pgfqpoint{4.166214in}{0.700153in}}%
\pgfpathcurveto{\pgfqpoint{4.168056in}{0.700153in}}{\pgfqpoint{4.169823in}{0.700884in}}{\pgfqpoint{4.171125in}{0.702187in}}%
\pgfpathcurveto{\pgfqpoint{4.172427in}{0.703489in}}{\pgfqpoint{4.173159in}{0.705255in}}{\pgfqpoint{4.173159in}{0.707097in}}%
\pgfpathcurveto{\pgfqpoint{4.173159in}{0.708939in}}{\pgfqpoint{4.172427in}{0.710705in}}{\pgfqpoint{4.171125in}{0.712008in}}%
\pgfpathcurveto{\pgfqpoint{4.169823in}{0.713310in}}{\pgfqpoint{4.168056in}{0.714042in}}{\pgfqpoint{4.166214in}{0.714042in}}%
\pgfpathcurveto{\pgfqpoint{4.164373in}{0.714042in}}{\pgfqpoint{4.162606in}{0.713310in}}{\pgfqpoint{4.161304in}{0.712008in}}%
\pgfpathcurveto{\pgfqpoint{4.160002in}{0.710705in}}{\pgfqpoint{4.159270in}{0.708939in}}{\pgfqpoint{4.159270in}{0.707097in}}%
\pgfpathcurveto{\pgfqpoint{4.159270in}{0.705255in}}{\pgfqpoint{4.160002in}{0.703489in}}{\pgfqpoint{4.161304in}{0.702187in}}%
\pgfpathcurveto{\pgfqpoint{4.162606in}{0.700884in}}{\pgfqpoint{4.164373in}{0.700153in}}{\pgfqpoint{4.166214in}{0.700153in}}%
\pgfpathlineto{\pgfqpoint{4.166214in}{0.700153in}}%
\pgfpathclose%
\pgfusepath{stroke,fill}%
\end{pgfscope}%
\begin{pgfscope}%
\pgfpathrectangle{\pgfqpoint{0.661006in}{0.524170in}}{\pgfqpoint{4.194036in}{1.071446in}}%
\pgfusepath{clip}%
\pgfsetbuttcap%
\pgfsetroundjoin%
\definecolor{currentfill}{rgb}{0.272762,0.240584,0.425592}%
\pgfsetfillcolor{currentfill}%
\pgfsetfillopacity{0.700000}%
\pgfsetlinewidth{1.003750pt}%
\definecolor{currentstroke}{rgb}{0.272762,0.240584,0.425592}%
\pgfsetstrokecolor{currentstroke}%
\pgfsetstrokeopacity{0.700000}%
\pgfsetdash{}{0pt}%
\pgfpathmoveto{\pgfqpoint{4.173000in}{0.697773in}}%
\pgfpathcurveto{\pgfqpoint{4.174842in}{0.697773in}}{\pgfqpoint{4.176608in}{0.698505in}}{\pgfqpoint{4.177911in}{0.699807in}}%
\pgfpathcurveto{\pgfqpoint{4.179213in}{0.701110in}}{\pgfqpoint{4.179944in}{0.702876in}}{\pgfqpoint{4.179944in}{0.704718in}}%
\pgfpathcurveto{\pgfqpoint{4.179944in}{0.706560in}}{\pgfqpoint{4.179213in}{0.708326in}}{\pgfqpoint{4.177911in}{0.709628in}}%
\pgfpathcurveto{\pgfqpoint{4.176608in}{0.710931in}}{\pgfqpoint{4.174842in}{0.711662in}}{\pgfqpoint{4.173000in}{0.711662in}}%
\pgfpathcurveto{\pgfqpoint{4.171158in}{0.711662in}}{\pgfqpoint{4.169392in}{0.710931in}}{\pgfqpoint{4.168090in}{0.709628in}}%
\pgfpathcurveto{\pgfqpoint{4.166787in}{0.708326in}}{\pgfqpoint{4.166056in}{0.706560in}}{\pgfqpoint{4.166056in}{0.704718in}}%
\pgfpathcurveto{\pgfqpoint{4.166056in}{0.702876in}}{\pgfqpoint{4.166787in}{0.701110in}}{\pgfqpoint{4.168090in}{0.699807in}}%
\pgfpathcurveto{\pgfqpoint{4.169392in}{0.698505in}}{\pgfqpoint{4.171158in}{0.697773in}}{\pgfqpoint{4.173000in}{0.697773in}}%
\pgfpathlineto{\pgfqpoint{4.173000in}{0.697773in}}%
\pgfpathclose%
\pgfusepath{stroke,fill}%
\end{pgfscope}%
\begin{pgfscope}%
\pgfpathrectangle{\pgfqpoint{0.661006in}{0.524170in}}{\pgfqpoint{4.194036in}{1.071446in}}%
\pgfusepath{clip}%
\pgfsetbuttcap%
\pgfsetroundjoin%
\definecolor{currentfill}{rgb}{0.269682,0.236518,0.420149}%
\pgfsetfillcolor{currentfill}%
\pgfsetfillopacity{0.700000}%
\pgfsetlinewidth{1.003750pt}%
\definecolor{currentstroke}{rgb}{0.269682,0.236518,0.420149}%
\pgfsetstrokecolor{currentstroke}%
\pgfsetstrokeopacity{0.700000}%
\pgfsetdash{}{0pt}%
\pgfpathmoveto{\pgfqpoint{4.179925in}{0.697173in}}%
\pgfpathcurveto{\pgfqpoint{4.181767in}{0.697173in}}{\pgfqpoint{4.183533in}{0.697904in}}{\pgfqpoint{4.184836in}{0.699207in}}%
\pgfpathcurveto{\pgfqpoint{4.186138in}{0.700509in}}{\pgfqpoint{4.186870in}{0.702275in}}{\pgfqpoint{4.186870in}{0.704117in}}%
\pgfpathcurveto{\pgfqpoint{4.186870in}{0.705959in}}{\pgfqpoint{4.186138in}{0.707725in}}{\pgfqpoint{4.184836in}{0.709028in}}%
\pgfpathcurveto{\pgfqpoint{4.183533in}{0.710330in}}{\pgfqpoint{4.181767in}{0.711062in}}{\pgfqpoint{4.179925in}{0.711062in}}%
\pgfpathcurveto{\pgfqpoint{4.178083in}{0.711062in}}{\pgfqpoint{4.176317in}{0.710330in}}{\pgfqpoint{4.175015in}{0.709028in}}%
\pgfpathcurveto{\pgfqpoint{4.173712in}{0.707725in}}{\pgfqpoint{4.172981in}{0.705959in}}{\pgfqpoint{4.172981in}{0.704117in}}%
\pgfpathcurveto{\pgfqpoint{4.172981in}{0.702275in}}{\pgfqpoint{4.173712in}{0.700509in}}{\pgfqpoint{4.175015in}{0.699207in}}%
\pgfpathcurveto{\pgfqpoint{4.176317in}{0.697904in}}{\pgfqpoint{4.178083in}{0.697173in}}{\pgfqpoint{4.179925in}{0.697173in}}%
\pgfpathlineto{\pgfqpoint{4.179925in}{0.697173in}}%
\pgfpathclose%
\pgfusepath{stroke,fill}%
\end{pgfscope}%
\begin{pgfscope}%
\pgfpathrectangle{\pgfqpoint{0.661006in}{0.524170in}}{\pgfqpoint{4.194036in}{1.071446in}}%
\pgfusepath{clip}%
\pgfsetbuttcap%
\pgfsetroundjoin%
\definecolor{currentfill}{rgb}{0.269682,0.236518,0.420149}%
\pgfsetfillcolor{currentfill}%
\pgfsetfillopacity{0.700000}%
\pgfsetlinewidth{1.003750pt}%
\definecolor{currentstroke}{rgb}{0.269682,0.236518,0.420149}%
\pgfsetstrokecolor{currentstroke}%
\pgfsetstrokeopacity{0.700000}%
\pgfsetdash{}{0pt}%
\pgfpathmoveto{\pgfqpoint{4.164448in}{0.700567in}}%
\pgfpathcurveto{\pgfqpoint{4.166290in}{0.700567in}}{\pgfqpoint{4.168056in}{0.701298in}}{\pgfqpoint{4.169359in}{0.702601in}}%
\pgfpathcurveto{\pgfqpoint{4.170661in}{0.703903in}}{\pgfqpoint{4.171393in}{0.705669in}}{\pgfqpoint{4.171393in}{0.707511in}}%
\pgfpathcurveto{\pgfqpoint{4.171393in}{0.709353in}}{\pgfqpoint{4.170661in}{0.711119in}}{\pgfqpoint{4.169359in}{0.712422in}}%
\pgfpathcurveto{\pgfqpoint{4.168056in}{0.713724in}}{\pgfqpoint{4.166290in}{0.714456in}}{\pgfqpoint{4.164448in}{0.714456in}}%
\pgfpathcurveto{\pgfqpoint{4.162607in}{0.714456in}}{\pgfqpoint{4.160840in}{0.713724in}}{\pgfqpoint{4.159538in}{0.712422in}}%
\pgfpathcurveto{\pgfqpoint{4.158235in}{0.711119in}}{\pgfqpoint{4.157504in}{0.709353in}}{\pgfqpoint{4.157504in}{0.707511in}}%
\pgfpathcurveto{\pgfqpoint{4.157504in}{0.705669in}}{\pgfqpoint{4.158235in}{0.703903in}}{\pgfqpoint{4.159538in}{0.702601in}}%
\pgfpathcurveto{\pgfqpoint{4.160840in}{0.701298in}}{\pgfqpoint{4.162607in}{0.700567in}}{\pgfqpoint{4.164448in}{0.700567in}}%
\pgfpathlineto{\pgfqpoint{4.164448in}{0.700567in}}%
\pgfpathclose%
\pgfusepath{stroke,fill}%
\end{pgfscope}%
\begin{pgfscope}%
\pgfpathrectangle{\pgfqpoint{0.661006in}{0.524170in}}{\pgfqpoint{4.194036in}{1.071446in}}%
\pgfusepath{clip}%
\pgfsetbuttcap%
\pgfsetroundjoin%
\definecolor{currentfill}{rgb}{0.269682,0.236518,0.420149}%
\pgfsetfillcolor{currentfill}%
\pgfsetfillopacity{0.700000}%
\pgfsetlinewidth{1.003750pt}%
\definecolor{currentstroke}{rgb}{0.269682,0.236518,0.420149}%
\pgfsetstrokecolor{currentstroke}%
\pgfsetstrokeopacity{0.700000}%
\pgfsetdash{}{0pt}%
\pgfpathmoveto{\pgfqpoint{4.148583in}{0.704174in}}%
\pgfpathcurveto{\pgfqpoint{4.150425in}{0.704174in}}{\pgfqpoint{4.152191in}{0.704905in}}{\pgfqpoint{4.153493in}{0.706208in}}%
\pgfpathcurveto{\pgfqpoint{4.154796in}{0.707510in}}{\pgfqpoint{4.155527in}{0.709277in}}{\pgfqpoint{4.155527in}{0.711118in}}%
\pgfpathcurveto{\pgfqpoint{4.155527in}{0.712960in}}{\pgfqpoint{4.154796in}{0.714726in}}{\pgfqpoint{4.153493in}{0.716029in}}%
\pgfpathcurveto{\pgfqpoint{4.152191in}{0.717331in}}{\pgfqpoint{4.150425in}{0.718063in}}{\pgfqpoint{4.148583in}{0.718063in}}%
\pgfpathcurveto{\pgfqpoint{4.146741in}{0.718063in}}{\pgfqpoint{4.144975in}{0.717331in}}{\pgfqpoint{4.143672in}{0.716029in}}%
\pgfpathcurveto{\pgfqpoint{4.142370in}{0.714726in}}{\pgfqpoint{4.141638in}{0.712960in}}{\pgfqpoint{4.141638in}{0.711118in}}%
\pgfpathcurveto{\pgfqpoint{4.141638in}{0.709277in}}{\pgfqpoint{4.142370in}{0.707510in}}{\pgfqpoint{4.143672in}{0.706208in}}%
\pgfpathcurveto{\pgfqpoint{4.144975in}{0.704905in}}{\pgfqpoint{4.146741in}{0.704174in}}{\pgfqpoint{4.148583in}{0.704174in}}%
\pgfpathlineto{\pgfqpoint{4.148583in}{0.704174in}}%
\pgfpathclose%
\pgfusepath{stroke,fill}%
\end{pgfscope}%
\begin{pgfscope}%
\pgfpathrectangle{\pgfqpoint{0.661006in}{0.524170in}}{\pgfqpoint{4.194036in}{1.071446in}}%
\pgfusepath{clip}%
\pgfsetbuttcap%
\pgfsetroundjoin%
\definecolor{currentfill}{rgb}{0.266586,0.232466,0.414678}%
\pgfsetfillcolor{currentfill}%
\pgfsetfillopacity{0.700000}%
\pgfsetlinewidth{1.003750pt}%
\definecolor{currentstroke}{rgb}{0.266586,0.232466,0.414678}%
\pgfsetstrokecolor{currentstroke}%
\pgfsetstrokeopacity{0.700000}%
\pgfsetdash{}{0pt}%
\pgfpathmoveto{\pgfqpoint{4.134517in}{0.705232in}}%
\pgfpathcurveto{\pgfqpoint{4.136359in}{0.705232in}}{\pgfqpoint{4.138125in}{0.705963in}}{\pgfqpoint{4.139427in}{0.707266in}}%
\pgfpathcurveto{\pgfqpoint{4.140730in}{0.708568in}}{\pgfqpoint{4.141461in}{0.710334in}}{\pgfqpoint{4.141461in}{0.712176in}}%
\pgfpathcurveto{\pgfqpoint{4.141461in}{0.714018in}}{\pgfqpoint{4.140730in}{0.715784in}}{\pgfqpoint{4.139427in}{0.717087in}}%
\pgfpathcurveto{\pgfqpoint{4.138125in}{0.718389in}}{\pgfqpoint{4.136359in}{0.719121in}}{\pgfqpoint{4.134517in}{0.719121in}}%
\pgfpathcurveto{\pgfqpoint{4.132675in}{0.719121in}}{\pgfqpoint{4.130909in}{0.718389in}}{\pgfqpoint{4.129606in}{0.717087in}}%
\pgfpathcurveto{\pgfqpoint{4.128304in}{0.715784in}}{\pgfqpoint{4.127572in}{0.714018in}}{\pgfqpoint{4.127572in}{0.712176in}}%
\pgfpathcurveto{\pgfqpoint{4.127572in}{0.710334in}}{\pgfqpoint{4.128304in}{0.708568in}}{\pgfqpoint{4.129606in}{0.707266in}}%
\pgfpathcurveto{\pgfqpoint{4.130909in}{0.705963in}}{\pgfqpoint{4.132675in}{0.705232in}}{\pgfqpoint{4.134517in}{0.705232in}}%
\pgfpathlineto{\pgfqpoint{4.134517in}{0.705232in}}%
\pgfpathclose%
\pgfusepath{stroke,fill}%
\end{pgfscope}%
\begin{pgfscope}%
\pgfpathrectangle{\pgfqpoint{0.661006in}{0.524170in}}{\pgfqpoint{4.194036in}{1.071446in}}%
\pgfusepath{clip}%
\pgfsetbuttcap%
\pgfsetroundjoin%
\definecolor{currentfill}{rgb}{0.266586,0.232466,0.414678}%
\pgfsetfillcolor{currentfill}%
\pgfsetfillopacity{0.700000}%
\pgfsetlinewidth{1.003750pt}%
\definecolor{currentstroke}{rgb}{0.266586,0.232466,0.414678}%
\pgfsetstrokecolor{currentstroke}%
\pgfsetstrokeopacity{0.700000}%
\pgfsetdash{}{0pt}%
\pgfpathmoveto{\pgfqpoint{4.147205in}{0.705493in}}%
\pgfpathcurveto{\pgfqpoint{4.149047in}{0.705493in}}{\pgfqpoint{4.150813in}{0.706224in}}{\pgfqpoint{4.152116in}{0.707527in}}%
\pgfpathcurveto{\pgfqpoint{4.153418in}{0.708829in}}{\pgfqpoint{4.154150in}{0.710595in}}{\pgfqpoint{4.154150in}{0.712437in}}%
\pgfpathcurveto{\pgfqpoint{4.154150in}{0.714279in}}{\pgfqpoint{4.153418in}{0.716045in}}{\pgfqpoint{4.152116in}{0.717348in}}%
\pgfpathcurveto{\pgfqpoint{4.150813in}{0.718650in}}{\pgfqpoint{4.149047in}{0.719382in}}{\pgfqpoint{4.147205in}{0.719382in}}%
\pgfpathcurveto{\pgfqpoint{4.145363in}{0.719382in}}{\pgfqpoint{4.143597in}{0.718650in}}{\pgfqpoint{4.142295in}{0.717348in}}%
\pgfpathcurveto{\pgfqpoint{4.140992in}{0.716045in}}{\pgfqpoint{4.140261in}{0.714279in}}{\pgfqpoint{4.140261in}{0.712437in}}%
\pgfpathcurveto{\pgfqpoint{4.140261in}{0.710595in}}{\pgfqpoint{4.140992in}{0.708829in}}{\pgfqpoint{4.142295in}{0.707527in}}%
\pgfpathcurveto{\pgfqpoint{4.143597in}{0.706224in}}{\pgfqpoint{4.145363in}{0.705493in}}{\pgfqpoint{4.147205in}{0.705493in}}%
\pgfpathlineto{\pgfqpoint{4.147205in}{0.705493in}}%
\pgfpathclose%
\pgfusepath{stroke,fill}%
\end{pgfscope}%
\begin{pgfscope}%
\pgfpathrectangle{\pgfqpoint{0.661006in}{0.524170in}}{\pgfqpoint{4.194036in}{1.071446in}}%
\pgfusepath{clip}%
\pgfsetbuttcap%
\pgfsetroundjoin%
\definecolor{currentfill}{rgb}{0.266586,0.232466,0.414678}%
\pgfsetfillcolor{currentfill}%
\pgfsetfillopacity{0.700000}%
\pgfsetlinewidth{1.003750pt}%
\definecolor{currentstroke}{rgb}{0.266586,0.232466,0.414678}%
\pgfsetstrokecolor{currentstroke}%
\pgfsetstrokeopacity{0.700000}%
\pgfsetdash{}{0pt}%
\pgfpathmoveto{\pgfqpoint{4.163286in}{0.699734in}}%
\pgfpathcurveto{\pgfqpoint{4.165128in}{0.699734in}}{\pgfqpoint{4.166894in}{0.700466in}}{\pgfqpoint{4.168197in}{0.701768in}}%
\pgfpathcurveto{\pgfqpoint{4.169499in}{0.703070in}}{\pgfqpoint{4.170231in}{0.704837in}}{\pgfqpoint{4.170231in}{0.706679in}}%
\pgfpathcurveto{\pgfqpoint{4.170231in}{0.708520in}}{\pgfqpoint{4.169499in}{0.710287in}}{\pgfqpoint{4.168197in}{0.711589in}}%
\pgfpathcurveto{\pgfqpoint{4.166894in}{0.712891in}}{\pgfqpoint{4.165128in}{0.713623in}}{\pgfqpoint{4.163286in}{0.713623in}}%
\pgfpathcurveto{\pgfqpoint{4.161445in}{0.713623in}}{\pgfqpoint{4.159678in}{0.712891in}}{\pgfqpoint{4.158376in}{0.711589in}}%
\pgfpathcurveto{\pgfqpoint{4.157074in}{0.710287in}}{\pgfqpoint{4.156342in}{0.708520in}}{\pgfqpoint{4.156342in}{0.706679in}}%
\pgfpathcurveto{\pgfqpoint{4.156342in}{0.704837in}}{\pgfqpoint{4.157074in}{0.703070in}}{\pgfqpoint{4.158376in}{0.701768in}}%
\pgfpathcurveto{\pgfqpoint{4.159678in}{0.700466in}}{\pgfqpoint{4.161445in}{0.699734in}}{\pgfqpoint{4.163286in}{0.699734in}}%
\pgfpathlineto{\pgfqpoint{4.163286in}{0.699734in}}%
\pgfpathclose%
\pgfusepath{stroke,fill}%
\end{pgfscope}%
\begin{pgfscope}%
\pgfpathrectangle{\pgfqpoint{0.661006in}{0.524170in}}{\pgfqpoint{4.194036in}{1.071446in}}%
\pgfusepath{clip}%
\pgfsetbuttcap%
\pgfsetroundjoin%
\definecolor{currentfill}{rgb}{0.266586,0.232466,0.414678}%
\pgfsetfillcolor{currentfill}%
\pgfsetfillopacity{0.700000}%
\pgfsetlinewidth{1.003750pt}%
\definecolor{currentstroke}{rgb}{0.266586,0.232466,0.414678}%
\pgfsetstrokecolor{currentstroke}%
\pgfsetstrokeopacity{0.700000}%
\pgfsetdash{}{0pt}%
\pgfpathmoveto{\pgfqpoint{4.203071in}{0.689865in}}%
\pgfpathcurveto{\pgfqpoint{4.204913in}{0.689865in}}{\pgfqpoint{4.206679in}{0.690597in}}{\pgfqpoint{4.207981in}{0.691899in}}%
\pgfpathcurveto{\pgfqpoint{4.209284in}{0.693202in}}{\pgfqpoint{4.210015in}{0.694968in}}{\pgfqpoint{4.210015in}{0.696810in}}%
\pgfpathcurveto{\pgfqpoint{4.210015in}{0.698651in}}{\pgfqpoint{4.209284in}{0.700418in}}{\pgfqpoint{4.207981in}{0.701720in}}%
\pgfpathcurveto{\pgfqpoint{4.206679in}{0.703022in}}{\pgfqpoint{4.204913in}{0.703754in}}{\pgfqpoint{4.203071in}{0.703754in}}%
\pgfpathcurveto{\pgfqpoint{4.201229in}{0.703754in}}{\pgfqpoint{4.199463in}{0.703022in}}{\pgfqpoint{4.198160in}{0.701720in}}%
\pgfpathcurveto{\pgfqpoint{4.196858in}{0.700418in}}{\pgfqpoint{4.196126in}{0.698651in}}{\pgfqpoint{4.196126in}{0.696810in}}%
\pgfpathcurveto{\pgfqpoint{4.196126in}{0.694968in}}{\pgfqpoint{4.196858in}{0.693202in}}{\pgfqpoint{4.198160in}{0.691899in}}%
\pgfpathcurveto{\pgfqpoint{4.199463in}{0.690597in}}{\pgfqpoint{4.201229in}{0.689865in}}{\pgfqpoint{4.203071in}{0.689865in}}%
\pgfpathlineto{\pgfqpoint{4.203071in}{0.689865in}}%
\pgfpathclose%
\pgfusepath{stroke,fill}%
\end{pgfscope}%
\begin{pgfscope}%
\pgfpathrectangle{\pgfqpoint{0.661006in}{0.524170in}}{\pgfqpoint{4.194036in}{1.071446in}}%
\pgfusepath{clip}%
\pgfsetbuttcap%
\pgfsetroundjoin%
\definecolor{currentfill}{rgb}{0.263472,0.228428,0.409179}%
\pgfsetfillcolor{currentfill}%
\pgfsetfillopacity{0.700000}%
\pgfsetlinewidth{1.003750pt}%
\definecolor{currentstroke}{rgb}{0.263472,0.228428,0.409179}%
\pgfsetstrokecolor{currentstroke}%
\pgfsetstrokeopacity{0.700000}%
\pgfsetdash{}{0pt}%
\pgfpathmoveto{\pgfqpoint{4.239091in}{0.681432in}}%
\pgfpathcurveto{\pgfqpoint{4.240932in}{0.681432in}}{\pgfqpoint{4.242699in}{0.682163in}}{\pgfqpoint{4.244001in}{0.683466in}}%
\pgfpathcurveto{\pgfqpoint{4.245303in}{0.684768in}}{\pgfqpoint{4.246035in}{0.686534in}}{\pgfqpoint{4.246035in}{0.688376in}}%
\pgfpathcurveto{\pgfqpoint{4.246035in}{0.690218in}}{\pgfqpoint{4.245303in}{0.691984in}}{\pgfqpoint{4.244001in}{0.693287in}}%
\pgfpathcurveto{\pgfqpoint{4.242699in}{0.694589in}}{\pgfqpoint{4.240932in}{0.695320in}}{\pgfqpoint{4.239091in}{0.695320in}}%
\pgfpathcurveto{\pgfqpoint{4.237249in}{0.695320in}}{\pgfqpoint{4.235482in}{0.694589in}}{\pgfqpoint{4.234180in}{0.693287in}}%
\pgfpathcurveto{\pgfqpoint{4.232878in}{0.691984in}}{\pgfqpoint{4.232146in}{0.690218in}}{\pgfqpoint{4.232146in}{0.688376in}}%
\pgfpathcurveto{\pgfqpoint{4.232146in}{0.686534in}}{\pgfqpoint{4.232878in}{0.684768in}}{\pgfqpoint{4.234180in}{0.683466in}}%
\pgfpathcurveto{\pgfqpoint{4.235482in}{0.682163in}}{\pgfqpoint{4.237249in}{0.681432in}}{\pgfqpoint{4.239091in}{0.681432in}}%
\pgfpathlineto{\pgfqpoint{4.239091in}{0.681432in}}%
\pgfpathclose%
\pgfusepath{stroke,fill}%
\end{pgfscope}%
\begin{pgfscope}%
\pgfpathrectangle{\pgfqpoint{0.661006in}{0.524170in}}{\pgfqpoint{4.194036in}{1.071446in}}%
\pgfusepath{clip}%
\pgfsetbuttcap%
\pgfsetroundjoin%
\definecolor{currentfill}{rgb}{0.263472,0.228428,0.409179}%
\pgfsetfillcolor{currentfill}%
\pgfsetfillopacity{0.700000}%
\pgfsetlinewidth{1.003750pt}%
\definecolor{currentstroke}{rgb}{0.263472,0.228428,0.409179}%
\pgfsetstrokecolor{currentstroke}%
\pgfsetstrokeopacity{0.700000}%
\pgfsetdash{}{0pt}%
\pgfpathmoveto{\pgfqpoint{4.278782in}{0.675621in}}%
\pgfpathcurveto{\pgfqpoint{4.280624in}{0.675621in}}{\pgfqpoint{4.282390in}{0.676353in}}{\pgfqpoint{4.283693in}{0.677655in}}%
\pgfpathcurveto{\pgfqpoint{4.284995in}{0.678958in}}{\pgfqpoint{4.285727in}{0.680724in}}{\pgfqpoint{4.285727in}{0.682566in}}%
\pgfpathcurveto{\pgfqpoint{4.285727in}{0.684408in}}{\pgfqpoint{4.284995in}{0.686174in}}{\pgfqpoint{4.283693in}{0.687476in}}%
\pgfpathcurveto{\pgfqpoint{4.282390in}{0.688779in}}{\pgfqpoint{4.280624in}{0.689510in}}{\pgfqpoint{4.278782in}{0.689510in}}%
\pgfpathcurveto{\pgfqpoint{4.276941in}{0.689510in}}{\pgfqpoint{4.275174in}{0.688779in}}{\pgfqpoint{4.273872in}{0.687476in}}%
\pgfpathcurveto{\pgfqpoint{4.272570in}{0.686174in}}{\pgfqpoint{4.271838in}{0.684408in}}{\pgfqpoint{4.271838in}{0.682566in}}%
\pgfpathcurveto{\pgfqpoint{4.271838in}{0.680724in}}{\pgfqpoint{4.272570in}{0.678958in}}{\pgfqpoint{4.273872in}{0.677655in}}%
\pgfpathcurveto{\pgfqpoint{4.275174in}{0.676353in}}{\pgfqpoint{4.276941in}{0.675621in}}{\pgfqpoint{4.278782in}{0.675621in}}%
\pgfpathlineto{\pgfqpoint{4.278782in}{0.675621in}}%
\pgfpathclose%
\pgfusepath{stroke,fill}%
\end{pgfscope}%
\begin{pgfscope}%
\pgfpathrectangle{\pgfqpoint{0.661006in}{0.524170in}}{\pgfqpoint{4.194036in}{1.071446in}}%
\pgfusepath{clip}%
\pgfsetbuttcap%
\pgfsetroundjoin%
\definecolor{currentfill}{rgb}{0.263472,0.228428,0.409179}%
\pgfsetfillcolor{currentfill}%
\pgfsetfillopacity{0.700000}%
\pgfsetlinewidth{1.003750pt}%
\definecolor{currentstroke}{rgb}{0.263472,0.228428,0.409179}%
\pgfsetstrokecolor{currentstroke}%
\pgfsetstrokeopacity{0.700000}%
\pgfsetdash{}{0pt}%
\pgfpathmoveto{\pgfqpoint{4.279991in}{0.673162in}}%
\pgfpathcurveto{\pgfqpoint{4.281832in}{0.673162in}}{\pgfqpoint{4.283599in}{0.673894in}}{\pgfqpoint{4.284901in}{0.675196in}}%
\pgfpathcurveto{\pgfqpoint{4.286203in}{0.676498in}}{\pgfqpoint{4.286935in}{0.678265in}}{\pgfqpoint{4.286935in}{0.680106in}}%
\pgfpathcurveto{\pgfqpoint{4.286935in}{0.681948in}}{\pgfqpoint{4.286203in}{0.683715in}}{\pgfqpoint{4.284901in}{0.685017in}}%
\pgfpathcurveto{\pgfqpoint{4.283599in}{0.686319in}}{\pgfqpoint{4.281832in}{0.687051in}}{\pgfqpoint{4.279991in}{0.687051in}}%
\pgfpathcurveto{\pgfqpoint{4.278149in}{0.687051in}}{\pgfqpoint{4.276382in}{0.686319in}}{\pgfqpoint{4.275080in}{0.685017in}}%
\pgfpathcurveto{\pgfqpoint{4.273778in}{0.683715in}}{\pgfqpoint{4.273046in}{0.681948in}}{\pgfqpoint{4.273046in}{0.680106in}}%
\pgfpathcurveto{\pgfqpoint{4.273046in}{0.678265in}}{\pgfqpoint{4.273778in}{0.676498in}}{\pgfqpoint{4.275080in}{0.675196in}}%
\pgfpathcurveto{\pgfqpoint{4.276382in}{0.673894in}}{\pgfqpoint{4.278149in}{0.673162in}}{\pgfqpoint{4.279991in}{0.673162in}}%
\pgfpathlineto{\pgfqpoint{4.279991in}{0.673162in}}%
\pgfpathclose%
\pgfusepath{stroke,fill}%
\end{pgfscope}%
\begin{pgfscope}%
\pgfpathrectangle{\pgfqpoint{0.661006in}{0.524170in}}{\pgfqpoint{4.194036in}{1.071446in}}%
\pgfusepath{clip}%
\pgfsetbuttcap%
\pgfsetroundjoin%
\definecolor{currentfill}{rgb}{0.260341,0.224403,0.403652}%
\pgfsetfillcolor{currentfill}%
\pgfsetfillopacity{0.700000}%
\pgfsetlinewidth{1.003750pt}%
\definecolor{currentstroke}{rgb}{0.260341,0.224403,0.403652}%
\pgfsetstrokecolor{currentstroke}%
\pgfsetstrokeopacity{0.700000}%
\pgfsetdash{}{0pt}%
\pgfpathmoveto{\pgfqpoint{4.292911in}{0.670947in}}%
\pgfpathcurveto{\pgfqpoint{4.294753in}{0.670947in}}{\pgfqpoint{4.296520in}{0.671678in}}{\pgfqpoint{4.297822in}{0.672981in}}%
\pgfpathcurveto{\pgfqpoint{4.299124in}{0.674283in}}{\pgfqpoint{4.299856in}{0.676050in}}{\pgfqpoint{4.299856in}{0.677891in}}%
\pgfpathcurveto{\pgfqpoint{4.299856in}{0.679733in}}{\pgfqpoint{4.299124in}{0.681499in}}{\pgfqpoint{4.297822in}{0.682802in}}%
\pgfpathcurveto{\pgfqpoint{4.296520in}{0.684104in}}{\pgfqpoint{4.294753in}{0.684836in}}{\pgfqpoint{4.292911in}{0.684836in}}%
\pgfpathcurveto{\pgfqpoint{4.291070in}{0.684836in}}{\pgfqpoint{4.289303in}{0.684104in}}{\pgfqpoint{4.288001in}{0.682802in}}%
\pgfpathcurveto{\pgfqpoint{4.286699in}{0.681499in}}{\pgfqpoint{4.285967in}{0.679733in}}{\pgfqpoint{4.285967in}{0.677891in}}%
\pgfpathcurveto{\pgfqpoint{4.285967in}{0.676050in}}{\pgfqpoint{4.286699in}{0.674283in}}{\pgfqpoint{4.288001in}{0.672981in}}%
\pgfpathcurveto{\pgfqpoint{4.289303in}{0.671678in}}{\pgfqpoint{4.291070in}{0.670947in}}{\pgfqpoint{4.292911in}{0.670947in}}%
\pgfpathlineto{\pgfqpoint{4.292911in}{0.670947in}}%
\pgfpathclose%
\pgfusepath{stroke,fill}%
\end{pgfscope}%
\begin{pgfscope}%
\pgfpathrectangle{\pgfqpoint{0.661006in}{0.524170in}}{\pgfqpoint{4.194036in}{1.071446in}}%
\pgfusepath{clip}%
\pgfsetbuttcap%
\pgfsetroundjoin%
\definecolor{currentfill}{rgb}{0.260341,0.224403,0.403652}%
\pgfsetfillcolor{currentfill}%
\pgfsetfillopacity{0.700000}%
\pgfsetlinewidth{1.003750pt}%
\definecolor{currentstroke}{rgb}{0.260341,0.224403,0.403652}%
\pgfsetstrokecolor{currentstroke}%
\pgfsetstrokeopacity{0.700000}%
\pgfsetdash{}{0pt}%
\pgfpathmoveto{\pgfqpoint{4.317358in}{0.665596in}}%
\pgfpathcurveto{\pgfqpoint{4.319200in}{0.665596in}}{\pgfqpoint{4.320967in}{0.666328in}}{\pgfqpoint{4.322269in}{0.667630in}}%
\pgfpathcurveto{\pgfqpoint{4.323571in}{0.668933in}}{\pgfqpoint{4.324303in}{0.670699in}}{\pgfqpoint{4.324303in}{0.672541in}}%
\pgfpathcurveto{\pgfqpoint{4.324303in}{0.674383in}}{\pgfqpoint{4.323571in}{0.676149in}}{\pgfqpoint{4.322269in}{0.677451in}}%
\pgfpathcurveto{\pgfqpoint{4.320967in}{0.678754in}}{\pgfqpoint{4.319200in}{0.679485in}}{\pgfqpoint{4.317358in}{0.679485in}}%
\pgfpathcurveto{\pgfqpoint{4.315517in}{0.679485in}}{\pgfqpoint{4.313750in}{0.678754in}}{\pgfqpoint{4.312448in}{0.677451in}}%
\pgfpathcurveto{\pgfqpoint{4.311146in}{0.676149in}}{\pgfqpoint{4.310414in}{0.674383in}}{\pgfqpoint{4.310414in}{0.672541in}}%
\pgfpathcurveto{\pgfqpoint{4.310414in}{0.670699in}}{\pgfqpoint{4.311146in}{0.668933in}}{\pgfqpoint{4.312448in}{0.667630in}}%
\pgfpathcurveto{\pgfqpoint{4.313750in}{0.666328in}}{\pgfqpoint{4.315517in}{0.665596in}}{\pgfqpoint{4.317358in}{0.665596in}}%
\pgfpathlineto{\pgfqpoint{4.317358in}{0.665596in}}%
\pgfpathclose%
\pgfusepath{stroke,fill}%
\end{pgfscope}%
\begin{pgfscope}%
\pgfpathrectangle{\pgfqpoint{0.661006in}{0.524170in}}{\pgfqpoint{4.194036in}{1.071446in}}%
\pgfusepath{clip}%
\pgfsetbuttcap%
\pgfsetroundjoin%
\definecolor{currentfill}{rgb}{0.260341,0.224403,0.403652}%
\pgfsetfillcolor{currentfill}%
\pgfsetfillopacity{0.700000}%
\pgfsetlinewidth{1.003750pt}%
\definecolor{currentstroke}{rgb}{0.260341,0.224403,0.403652}%
\pgfsetstrokecolor{currentstroke}%
\pgfsetstrokeopacity{0.700000}%
\pgfsetdash{}{0pt}%
\pgfpathmoveto{\pgfqpoint{4.335624in}{0.661915in}}%
\pgfpathcurveto{\pgfqpoint{4.337466in}{0.661915in}}{\pgfqpoint{4.339232in}{0.662647in}}{\pgfqpoint{4.340534in}{0.663949in}}%
\pgfpathcurveto{\pgfqpoint{4.341837in}{0.665251in}}{\pgfqpoint{4.342568in}{0.667018in}}{\pgfqpoint{4.342568in}{0.668860in}}%
\pgfpathcurveto{\pgfqpoint{4.342568in}{0.670701in}}{\pgfqpoint{4.341837in}{0.672468in}}{\pgfqpoint{4.340534in}{0.673770in}}%
\pgfpathcurveto{\pgfqpoint{4.339232in}{0.675072in}}{\pgfqpoint{4.337466in}{0.675804in}}{\pgfqpoint{4.335624in}{0.675804in}}%
\pgfpathcurveto{\pgfqpoint{4.333782in}{0.675804in}}{\pgfqpoint{4.332016in}{0.675072in}}{\pgfqpoint{4.330713in}{0.673770in}}%
\pgfpathcurveto{\pgfqpoint{4.329411in}{0.672468in}}{\pgfqpoint{4.328679in}{0.670701in}}{\pgfqpoint{4.328679in}{0.668860in}}%
\pgfpathcurveto{\pgfqpoint{4.328679in}{0.667018in}}{\pgfqpoint{4.329411in}{0.665251in}}{\pgfqpoint{4.330713in}{0.663949in}}%
\pgfpathcurveto{\pgfqpoint{4.332016in}{0.662647in}}{\pgfqpoint{4.333782in}{0.661915in}}{\pgfqpoint{4.335624in}{0.661915in}}%
\pgfpathlineto{\pgfqpoint{4.335624in}{0.661915in}}%
\pgfpathclose%
\pgfusepath{stroke,fill}%
\end{pgfscope}%
\begin{pgfscope}%
\pgfpathrectangle{\pgfqpoint{0.661006in}{0.524170in}}{\pgfqpoint{4.194036in}{1.071446in}}%
\pgfusepath{clip}%
\pgfsetbuttcap%
\pgfsetroundjoin%
\definecolor{currentfill}{rgb}{0.260341,0.224403,0.403652}%
\pgfsetfillcolor{currentfill}%
\pgfsetfillopacity{0.700000}%
\pgfsetlinewidth{1.003750pt}%
\definecolor{currentstroke}{rgb}{0.260341,0.224403,0.403652}%
\pgfsetstrokecolor{currentstroke}%
\pgfsetstrokeopacity{0.700000}%
\pgfsetdash{}{0pt}%
\pgfpathmoveto{\pgfqpoint{4.332231in}{0.662096in}}%
\pgfpathcurveto{\pgfqpoint{4.334073in}{0.662096in}}{\pgfqpoint{4.335839in}{0.662828in}}{\pgfqpoint{4.337142in}{0.664130in}}%
\pgfpathcurveto{\pgfqpoint{4.338444in}{0.665433in}}{\pgfqpoint{4.339176in}{0.667199in}}{\pgfqpoint{4.339176in}{0.669041in}}%
\pgfpathcurveto{\pgfqpoint{4.339176in}{0.670882in}}{\pgfqpoint{4.338444in}{0.672649in}}{\pgfqpoint{4.337142in}{0.673951in}}%
\pgfpathcurveto{\pgfqpoint{4.335839in}{0.675253in}}{\pgfqpoint{4.334073in}{0.675985in}}{\pgfqpoint{4.332231in}{0.675985in}}%
\pgfpathcurveto{\pgfqpoint{4.330389in}{0.675985in}}{\pgfqpoint{4.328623in}{0.675253in}}{\pgfqpoint{4.327321in}{0.673951in}}%
\pgfpathcurveto{\pgfqpoint{4.326018in}{0.672649in}}{\pgfqpoint{4.325287in}{0.670882in}}{\pgfqpoint{4.325287in}{0.669041in}}%
\pgfpathcurveto{\pgfqpoint{4.325287in}{0.667199in}}{\pgfqpoint{4.326018in}{0.665433in}}{\pgfqpoint{4.327321in}{0.664130in}}%
\pgfpathcurveto{\pgfqpoint{4.328623in}{0.662828in}}{\pgfqpoint{4.330389in}{0.662096in}}{\pgfqpoint{4.332231in}{0.662096in}}%
\pgfpathlineto{\pgfqpoint{4.332231in}{0.662096in}}%
\pgfpathclose%
\pgfusepath{stroke,fill}%
\end{pgfscope}%
\begin{pgfscope}%
\pgfpathrectangle{\pgfqpoint{0.661006in}{0.524170in}}{\pgfqpoint{4.194036in}{1.071446in}}%
\pgfusepath{clip}%
\pgfsetbuttcap%
\pgfsetroundjoin%
\definecolor{currentfill}{rgb}{0.257192,0.220393,0.398097}%
\pgfsetfillcolor{currentfill}%
\pgfsetfillopacity{0.700000}%
\pgfsetlinewidth{1.003750pt}%
\definecolor{currentstroke}{rgb}{0.257192,0.220393,0.398097}%
\pgfsetstrokecolor{currentstroke}%
\pgfsetstrokeopacity{0.700000}%
\pgfsetdash{}{0pt}%
\pgfpathmoveto{\pgfqpoint{4.319310in}{0.666096in}}%
\pgfpathcurveto{\pgfqpoint{4.321152in}{0.666096in}}{\pgfqpoint{4.322919in}{0.666827in}}{\pgfqpoint{4.324221in}{0.668130in}}%
\pgfpathcurveto{\pgfqpoint{4.325523in}{0.669432in}}{\pgfqpoint{4.326255in}{0.671198in}}{\pgfqpoint{4.326255in}{0.673040in}}%
\pgfpathcurveto{\pgfqpoint{4.326255in}{0.674882in}}{\pgfqpoint{4.325523in}{0.676648in}}{\pgfqpoint{4.324221in}{0.677950in}}%
\pgfpathcurveto{\pgfqpoint{4.322919in}{0.679253in}}{\pgfqpoint{4.321152in}{0.679984in}}{\pgfqpoint{4.319310in}{0.679984in}}%
\pgfpathcurveto{\pgfqpoint{4.317469in}{0.679984in}}{\pgfqpoint{4.315702in}{0.679253in}}{\pgfqpoint{4.314400in}{0.677950in}}%
\pgfpathcurveto{\pgfqpoint{4.313098in}{0.676648in}}{\pgfqpoint{4.312366in}{0.674882in}}{\pgfqpoint{4.312366in}{0.673040in}}%
\pgfpathcurveto{\pgfqpoint{4.312366in}{0.671198in}}{\pgfqpoint{4.313098in}{0.669432in}}{\pgfqpoint{4.314400in}{0.668130in}}%
\pgfpathcurveto{\pgfqpoint{4.315702in}{0.666827in}}{\pgfqpoint{4.317469in}{0.666096in}}{\pgfqpoint{4.319310in}{0.666096in}}%
\pgfpathlineto{\pgfqpoint{4.319310in}{0.666096in}}%
\pgfpathclose%
\pgfusepath{stroke,fill}%
\end{pgfscope}%
\begin{pgfscope}%
\pgfpathrectangle{\pgfqpoint{0.661006in}{0.524170in}}{\pgfqpoint{4.194036in}{1.071446in}}%
\pgfusepath{clip}%
\pgfsetbuttcap%
\pgfsetroundjoin%
\definecolor{currentfill}{rgb}{0.257192,0.220393,0.398097}%
\pgfsetfillcolor{currentfill}%
\pgfsetfillopacity{0.700000}%
\pgfsetlinewidth{1.003750pt}%
\definecolor{currentstroke}{rgb}{0.257192,0.220393,0.398097}%
\pgfsetstrokecolor{currentstroke}%
\pgfsetstrokeopacity{0.700000}%
\pgfsetdash{}{0pt}%
\pgfpathmoveto{\pgfqpoint{4.310251in}{0.667338in}}%
\pgfpathcurveto{\pgfqpoint{4.312092in}{0.667338in}}{\pgfqpoint{4.313859in}{0.668070in}}{\pgfqpoint{4.315161in}{0.669372in}}%
\pgfpathcurveto{\pgfqpoint{4.316463in}{0.670674in}}{\pgfqpoint{4.317195in}{0.672441in}}{\pgfqpoint{4.317195in}{0.674283in}}%
\pgfpathcurveto{\pgfqpoint{4.317195in}{0.676124in}}{\pgfqpoint{4.316463in}{0.677891in}}{\pgfqpoint{4.315161in}{0.679193in}}%
\pgfpathcurveto{\pgfqpoint{4.313859in}{0.680495in}}{\pgfqpoint{4.312092in}{0.681227in}}{\pgfqpoint{4.310251in}{0.681227in}}%
\pgfpathcurveto{\pgfqpoint{4.308409in}{0.681227in}}{\pgfqpoint{4.306642in}{0.680495in}}{\pgfqpoint{4.305340in}{0.679193in}}%
\pgfpathcurveto{\pgfqpoint{4.304038in}{0.677891in}}{\pgfqpoint{4.303306in}{0.676124in}}{\pgfqpoint{4.303306in}{0.674283in}}%
\pgfpathcurveto{\pgfqpoint{4.303306in}{0.672441in}}{\pgfqpoint{4.304038in}{0.670674in}}{\pgfqpoint{4.305340in}{0.669372in}}%
\pgfpathcurveto{\pgfqpoint{4.306642in}{0.668070in}}{\pgfqpoint{4.308409in}{0.667338in}}{\pgfqpoint{4.310251in}{0.667338in}}%
\pgfpathlineto{\pgfqpoint{4.310251in}{0.667338in}}%
\pgfpathclose%
\pgfusepath{stroke,fill}%
\end{pgfscope}%
\begin{pgfscope}%
\pgfpathrectangle{\pgfqpoint{0.661006in}{0.524170in}}{\pgfqpoint{4.194036in}{1.071446in}}%
\pgfusepath{clip}%
\pgfsetbuttcap%
\pgfsetroundjoin%
\definecolor{currentfill}{rgb}{0.257192,0.220393,0.398097}%
\pgfsetfillcolor{currentfill}%
\pgfsetfillopacity{0.700000}%
\pgfsetlinewidth{1.003750pt}%
\definecolor{currentstroke}{rgb}{0.257192,0.220393,0.398097}%
\pgfsetstrokecolor{currentstroke}%
\pgfsetstrokeopacity{0.700000}%
\pgfsetdash{}{0pt}%
\pgfpathmoveto{\pgfqpoint{4.319822in}{0.666337in}}%
\pgfpathcurveto{\pgfqpoint{4.321663in}{0.666337in}}{\pgfqpoint{4.323430in}{0.667069in}}{\pgfqpoint{4.324732in}{0.668371in}}%
\pgfpathcurveto{\pgfqpoint{4.326034in}{0.669673in}}{\pgfqpoint{4.326766in}{0.671440in}}{\pgfqpoint{4.326766in}{0.673281in}}%
\pgfpathcurveto{\pgfqpoint{4.326766in}{0.675123in}}{\pgfqpoint{4.326034in}{0.676889in}}{\pgfqpoint{4.324732in}{0.678192in}}%
\pgfpathcurveto{\pgfqpoint{4.323430in}{0.679494in}}{\pgfqpoint{4.321663in}{0.680226in}}{\pgfqpoint{4.319822in}{0.680226in}}%
\pgfpathcurveto{\pgfqpoint{4.317980in}{0.680226in}}{\pgfqpoint{4.316213in}{0.679494in}}{\pgfqpoint{4.314911in}{0.678192in}}%
\pgfpathcurveto{\pgfqpoint{4.313609in}{0.676889in}}{\pgfqpoint{4.312877in}{0.675123in}}{\pgfqpoint{4.312877in}{0.673281in}}%
\pgfpathcurveto{\pgfqpoint{4.312877in}{0.671440in}}{\pgfqpoint{4.313609in}{0.669673in}}{\pgfqpoint{4.314911in}{0.668371in}}%
\pgfpathcurveto{\pgfqpoint{4.316213in}{0.667069in}}{\pgfqpoint{4.317980in}{0.666337in}}{\pgfqpoint{4.319822in}{0.666337in}}%
\pgfpathlineto{\pgfqpoint{4.319822in}{0.666337in}}%
\pgfpathclose%
\pgfusepath{stroke,fill}%
\end{pgfscope}%
\begin{pgfscope}%
\pgfpathrectangle{\pgfqpoint{0.661006in}{0.524170in}}{\pgfqpoint{4.194036in}{1.071446in}}%
\pgfusepath{clip}%
\pgfsetbuttcap%
\pgfsetroundjoin%
\definecolor{currentfill}{rgb}{0.254024,0.216398,0.392516}%
\pgfsetfillcolor{currentfill}%
\pgfsetfillopacity{0.700000}%
\pgfsetlinewidth{1.003750pt}%
\definecolor{currentstroke}{rgb}{0.254024,0.216398,0.392516}%
\pgfsetstrokecolor{currentstroke}%
\pgfsetstrokeopacity{0.700000}%
\pgfsetdash{}{0pt}%
\pgfpathmoveto{\pgfqpoint{4.322099in}{0.665196in}}%
\pgfpathcurveto{\pgfqpoint{4.323941in}{0.665196in}}{\pgfqpoint{4.325707in}{0.665928in}}{\pgfqpoint{4.327009in}{0.667230in}}%
\pgfpathcurveto{\pgfqpoint{4.328312in}{0.668533in}}{\pgfqpoint{4.329043in}{0.670299in}}{\pgfqpoint{4.329043in}{0.672141in}}%
\pgfpathcurveto{\pgfqpoint{4.329043in}{0.673982in}}{\pgfqpoint{4.328312in}{0.675749in}}{\pgfqpoint{4.327009in}{0.677051in}}%
\pgfpathcurveto{\pgfqpoint{4.325707in}{0.678354in}}{\pgfqpoint{4.323941in}{0.679085in}}{\pgfqpoint{4.322099in}{0.679085in}}%
\pgfpathcurveto{\pgfqpoint{4.320257in}{0.679085in}}{\pgfqpoint{4.318491in}{0.678354in}}{\pgfqpoint{4.317189in}{0.677051in}}%
\pgfpathcurveto{\pgfqpoint{4.315886in}{0.675749in}}{\pgfqpoint{4.315155in}{0.673982in}}{\pgfqpoint{4.315155in}{0.672141in}}%
\pgfpathcurveto{\pgfqpoint{4.315155in}{0.670299in}}{\pgfqpoint{4.315886in}{0.668533in}}{\pgfqpoint{4.317189in}{0.667230in}}%
\pgfpathcurveto{\pgfqpoint{4.318491in}{0.665928in}}{\pgfqpoint{4.320257in}{0.665196in}}{\pgfqpoint{4.322099in}{0.665196in}}%
\pgfpathlineto{\pgfqpoint{4.322099in}{0.665196in}}%
\pgfpathclose%
\pgfusepath{stroke,fill}%
\end{pgfscope}%
\begin{pgfscope}%
\pgfpathrectangle{\pgfqpoint{0.661006in}{0.524170in}}{\pgfqpoint{4.194036in}{1.071446in}}%
\pgfusepath{clip}%
\pgfsetbuttcap%
\pgfsetroundjoin%
\definecolor{currentfill}{rgb}{0.254024,0.216398,0.392516}%
\pgfsetfillcolor{currentfill}%
\pgfsetfillopacity{0.700000}%
\pgfsetlinewidth{1.003750pt}%
\definecolor{currentstroke}{rgb}{0.254024,0.216398,0.392516}%
\pgfsetstrokecolor{currentstroke}%
\pgfsetstrokeopacity{0.700000}%
\pgfsetdash{}{0pt}%
\pgfpathmoveto{\pgfqpoint{4.320984in}{0.664516in}}%
\pgfpathcurveto{\pgfqpoint{4.322825in}{0.664516in}}{\pgfqpoint{4.324592in}{0.665247in}}{\pgfqpoint{4.325894in}{0.666550in}}%
\pgfpathcurveto{\pgfqpoint{4.327196in}{0.667852in}}{\pgfqpoint{4.327928in}{0.669618in}}{\pgfqpoint{4.327928in}{0.671460in}}%
\pgfpathcurveto{\pgfqpoint{4.327928in}{0.673302in}}{\pgfqpoint{4.327196in}{0.675068in}}{\pgfqpoint{4.325894in}{0.676371in}}%
\pgfpathcurveto{\pgfqpoint{4.324592in}{0.677673in}}{\pgfqpoint{4.322825in}{0.678405in}}{\pgfqpoint{4.320984in}{0.678405in}}%
\pgfpathcurveto{\pgfqpoint{4.319142in}{0.678405in}}{\pgfqpoint{4.317375in}{0.677673in}}{\pgfqpoint{4.316073in}{0.676371in}}%
\pgfpathcurveto{\pgfqpoint{4.314771in}{0.675068in}}{\pgfqpoint{4.314039in}{0.673302in}}{\pgfqpoint{4.314039in}{0.671460in}}%
\pgfpathcurveto{\pgfqpoint{4.314039in}{0.669618in}}{\pgfqpoint{4.314771in}{0.667852in}}{\pgfqpoint{4.316073in}{0.666550in}}%
\pgfpathcurveto{\pgfqpoint{4.317375in}{0.665247in}}{\pgfqpoint{4.319142in}{0.664516in}}{\pgfqpoint{4.320984in}{0.664516in}}%
\pgfpathlineto{\pgfqpoint{4.320984in}{0.664516in}}%
\pgfpathclose%
\pgfusepath{stroke,fill}%
\end{pgfscope}%
\begin{pgfscope}%
\pgfpathrectangle{\pgfqpoint{0.661006in}{0.524170in}}{\pgfqpoint{4.194036in}{1.071446in}}%
\pgfusepath{clip}%
\pgfsetbuttcap%
\pgfsetroundjoin%
\definecolor{currentfill}{rgb}{0.250838,0.212417,0.386908}%
\pgfsetfillcolor{currentfill}%
\pgfsetfillopacity{0.700000}%
\pgfsetlinewidth{1.003750pt}%
\definecolor{currentstroke}{rgb}{0.250838,0.212417,0.386908}%
\pgfsetstrokecolor{currentstroke}%
\pgfsetstrokeopacity{0.700000}%
\pgfsetdash{}{0pt}%
\pgfpathmoveto{\pgfqpoint{4.306436in}{0.667712in}}%
\pgfpathcurveto{\pgfqpoint{4.308278in}{0.667712in}}{\pgfqpoint{4.310044in}{0.668444in}}{\pgfqpoint{4.311347in}{0.669746in}}%
\pgfpathcurveto{\pgfqpoint{4.312649in}{0.671048in}}{\pgfqpoint{4.313381in}{0.672815in}}{\pgfqpoint{4.313381in}{0.674656in}}%
\pgfpathcurveto{\pgfqpoint{4.313381in}{0.676498in}}{\pgfqpoint{4.312649in}{0.678265in}}{\pgfqpoint{4.311347in}{0.679567in}}%
\pgfpathcurveto{\pgfqpoint{4.310044in}{0.680869in}}{\pgfqpoint{4.308278in}{0.681601in}}{\pgfqpoint{4.306436in}{0.681601in}}%
\pgfpathcurveto{\pgfqpoint{4.304595in}{0.681601in}}{\pgfqpoint{4.302828in}{0.680869in}}{\pgfqpoint{4.301526in}{0.679567in}}%
\pgfpathcurveto{\pgfqpoint{4.300223in}{0.678265in}}{\pgfqpoint{4.299492in}{0.676498in}}{\pgfqpoint{4.299492in}{0.674656in}}%
\pgfpathcurveto{\pgfqpoint{4.299492in}{0.672815in}}{\pgfqpoint{4.300223in}{0.671048in}}{\pgfqpoint{4.301526in}{0.669746in}}%
\pgfpathcurveto{\pgfqpoint{4.302828in}{0.668444in}}{\pgfqpoint{4.304595in}{0.667712in}}{\pgfqpoint{4.306436in}{0.667712in}}%
\pgfpathlineto{\pgfqpoint{4.306436in}{0.667712in}}%
\pgfpathclose%
\pgfusepath{stroke,fill}%
\end{pgfscope}%
\begin{pgfscope}%
\pgfpathrectangle{\pgfqpoint{0.661006in}{0.524170in}}{\pgfqpoint{4.194036in}{1.071446in}}%
\pgfusepath{clip}%
\pgfsetbuttcap%
\pgfsetroundjoin%
\definecolor{currentfill}{rgb}{0.250838,0.212417,0.386908}%
\pgfsetfillcolor{currentfill}%
\pgfsetfillopacity{0.700000}%
\pgfsetlinewidth{1.003750pt}%
\definecolor{currentstroke}{rgb}{0.250838,0.212417,0.386908}%
\pgfsetstrokecolor{currentstroke}%
\pgfsetstrokeopacity{0.700000}%
\pgfsetdash{}{0pt}%
\pgfpathmoveto{\pgfqpoint{4.279340in}{0.675355in}}%
\pgfpathcurveto{\pgfqpoint{4.281182in}{0.675355in}}{\pgfqpoint{4.282948in}{0.676087in}}{\pgfqpoint{4.284250in}{0.677389in}}%
\pgfpathcurveto{\pgfqpoint{4.285553in}{0.678691in}}{\pgfqpoint{4.286284in}{0.680458in}}{\pgfqpoint{4.286284in}{0.682299in}}%
\pgfpathcurveto{\pgfqpoint{4.286284in}{0.684141in}}{\pgfqpoint{4.285553in}{0.685908in}}{\pgfqpoint{4.284250in}{0.687210in}}%
\pgfpathcurveto{\pgfqpoint{4.282948in}{0.688512in}}{\pgfqpoint{4.281182in}{0.689244in}}{\pgfqpoint{4.279340in}{0.689244in}}%
\pgfpathcurveto{\pgfqpoint{4.277498in}{0.689244in}}{\pgfqpoint{4.275732in}{0.688512in}}{\pgfqpoint{4.274429in}{0.687210in}}%
\pgfpathcurveto{\pgfqpoint{4.273127in}{0.685908in}}{\pgfqpoint{4.272396in}{0.684141in}}{\pgfqpoint{4.272396in}{0.682299in}}%
\pgfpathcurveto{\pgfqpoint{4.272396in}{0.680458in}}{\pgfqpoint{4.273127in}{0.678691in}}{\pgfqpoint{4.274429in}{0.677389in}}%
\pgfpathcurveto{\pgfqpoint{4.275732in}{0.676087in}}{\pgfqpoint{4.277498in}{0.675355in}}{\pgfqpoint{4.279340in}{0.675355in}}%
\pgfpathlineto{\pgfqpoint{4.279340in}{0.675355in}}%
\pgfpathclose%
\pgfusepath{stroke,fill}%
\end{pgfscope}%
\begin{pgfscope}%
\pgfpathrectangle{\pgfqpoint{0.661006in}{0.524170in}}{\pgfqpoint{4.194036in}{1.071446in}}%
\pgfusepath{clip}%
\pgfsetbuttcap%
\pgfsetroundjoin%
\definecolor{currentfill}{rgb}{0.250838,0.212417,0.386908}%
\pgfsetfillcolor{currentfill}%
\pgfsetfillopacity{0.700000}%
\pgfsetlinewidth{1.003750pt}%
\definecolor{currentstroke}{rgb}{0.250838,0.212417,0.386908}%
\pgfsetstrokecolor{currentstroke}%
\pgfsetstrokeopacity{0.700000}%
\pgfsetdash{}{0pt}%
\pgfpathmoveto{\pgfqpoint{4.231840in}{0.684684in}}%
\pgfpathcurveto{\pgfqpoint{4.233682in}{0.684684in}}{\pgfqpoint{4.235448in}{0.685415in}}{\pgfqpoint{4.236751in}{0.686718in}}%
\pgfpathcurveto{\pgfqpoint{4.238053in}{0.688020in}}{\pgfqpoint{4.238785in}{0.689786in}}{\pgfqpoint{4.238785in}{0.691628in}}%
\pgfpathcurveto{\pgfqpoint{4.238785in}{0.693470in}}{\pgfqpoint{4.238053in}{0.695236in}}{\pgfqpoint{4.236751in}{0.696539in}}%
\pgfpathcurveto{\pgfqpoint{4.235448in}{0.697841in}}{\pgfqpoint{4.233682in}{0.698573in}}{\pgfqpoint{4.231840in}{0.698573in}}%
\pgfpathcurveto{\pgfqpoint{4.229999in}{0.698573in}}{\pgfqpoint{4.228232in}{0.697841in}}{\pgfqpoint{4.226930in}{0.696539in}}%
\pgfpathcurveto{\pgfqpoint{4.225627in}{0.695236in}}{\pgfqpoint{4.224896in}{0.693470in}}{\pgfqpoint{4.224896in}{0.691628in}}%
\pgfpathcurveto{\pgfqpoint{4.224896in}{0.689786in}}{\pgfqpoint{4.225627in}{0.688020in}}{\pgfqpoint{4.226930in}{0.686718in}}%
\pgfpathcurveto{\pgfqpoint{4.228232in}{0.685415in}}{\pgfqpoint{4.229999in}{0.684684in}}{\pgfqpoint{4.231840in}{0.684684in}}%
\pgfpathlineto{\pgfqpoint{4.231840in}{0.684684in}}%
\pgfpathclose%
\pgfusepath{stroke,fill}%
\end{pgfscope}%
\begin{pgfscope}%
\pgfpathrectangle{\pgfqpoint{0.661006in}{0.524170in}}{\pgfqpoint{4.194036in}{1.071446in}}%
\pgfusepath{clip}%
\pgfsetbuttcap%
\pgfsetroundjoin%
\definecolor{currentfill}{rgb}{0.250838,0.212417,0.386908}%
\pgfsetfillcolor{currentfill}%
\pgfsetfillopacity{0.700000}%
\pgfsetlinewidth{1.003750pt}%
\definecolor{currentstroke}{rgb}{0.250838,0.212417,0.386908}%
\pgfsetstrokecolor{currentstroke}%
\pgfsetstrokeopacity{0.700000}%
\pgfsetdash{}{0pt}%
\pgfpathmoveto{\pgfqpoint{4.198237in}{0.690180in}}%
\pgfpathcurveto{\pgfqpoint{4.200079in}{0.690180in}}{\pgfqpoint{4.201845in}{0.690912in}}{\pgfqpoint{4.203148in}{0.692214in}}%
\pgfpathcurveto{\pgfqpoint{4.204450in}{0.693517in}}{\pgfqpoint{4.205182in}{0.695283in}}{\pgfqpoint{4.205182in}{0.697125in}}%
\pgfpathcurveto{\pgfqpoint{4.205182in}{0.698966in}}{\pgfqpoint{4.204450in}{0.700733in}}{\pgfqpoint{4.203148in}{0.702035in}}%
\pgfpathcurveto{\pgfqpoint{4.201845in}{0.703338in}}{\pgfqpoint{4.200079in}{0.704069in}}{\pgfqpoint{4.198237in}{0.704069in}}%
\pgfpathcurveto{\pgfqpoint{4.196395in}{0.704069in}}{\pgfqpoint{4.194629in}{0.703338in}}{\pgfqpoint{4.193327in}{0.702035in}}%
\pgfpathcurveto{\pgfqpoint{4.192024in}{0.700733in}}{\pgfqpoint{4.191293in}{0.698966in}}{\pgfqpoint{4.191293in}{0.697125in}}%
\pgfpathcurveto{\pgfqpoint{4.191293in}{0.695283in}}{\pgfqpoint{4.192024in}{0.693517in}}{\pgfqpoint{4.193327in}{0.692214in}}%
\pgfpathcurveto{\pgfqpoint{4.194629in}{0.690912in}}{\pgfqpoint{4.196395in}{0.690180in}}{\pgfqpoint{4.198237in}{0.690180in}}%
\pgfpathlineto{\pgfqpoint{4.198237in}{0.690180in}}%
\pgfpathclose%
\pgfusepath{stroke,fill}%
\end{pgfscope}%
\begin{pgfscope}%
\pgfpathrectangle{\pgfqpoint{0.661006in}{0.524170in}}{\pgfqpoint{4.194036in}{1.071446in}}%
\pgfusepath{clip}%
\pgfsetbuttcap%
\pgfsetroundjoin%
\definecolor{currentfill}{rgb}{0.250838,0.212417,0.386908}%
\pgfsetfillcolor{currentfill}%
\pgfsetfillopacity{0.700000}%
\pgfsetlinewidth{1.003750pt}%
\definecolor{currentstroke}{rgb}{0.250838,0.212417,0.386908}%
\pgfsetstrokecolor{currentstroke}%
\pgfsetstrokeopacity{0.700000}%
\pgfsetdash{}{0pt}%
\pgfpathmoveto{\pgfqpoint{4.190708in}{0.693049in}}%
\pgfpathcurveto{\pgfqpoint{4.192550in}{0.693049in}}{\pgfqpoint{4.194316in}{0.693781in}}{\pgfqpoint{4.195618in}{0.695083in}}%
\pgfpathcurveto{\pgfqpoint{4.196921in}{0.696385in}}{\pgfqpoint{4.197652in}{0.698152in}}{\pgfqpoint{4.197652in}{0.699994in}}%
\pgfpathcurveto{\pgfqpoint{4.197652in}{0.701835in}}{\pgfqpoint{4.196921in}{0.703602in}}{\pgfqpoint{4.195618in}{0.704904in}}%
\pgfpathcurveto{\pgfqpoint{4.194316in}{0.706206in}}{\pgfqpoint{4.192550in}{0.706938in}}{\pgfqpoint{4.190708in}{0.706938in}}%
\pgfpathcurveto{\pgfqpoint{4.188866in}{0.706938in}}{\pgfqpoint{4.187100in}{0.706206in}}{\pgfqpoint{4.185797in}{0.704904in}}%
\pgfpathcurveto{\pgfqpoint{4.184495in}{0.703602in}}{\pgfqpoint{4.183763in}{0.701835in}}{\pgfqpoint{4.183763in}{0.699994in}}%
\pgfpathcurveto{\pgfqpoint{4.183763in}{0.698152in}}{\pgfqpoint{4.184495in}{0.696385in}}{\pgfqpoint{4.185797in}{0.695083in}}%
\pgfpathcurveto{\pgfqpoint{4.187100in}{0.693781in}}{\pgfqpoint{4.188866in}{0.693049in}}{\pgfqpoint{4.190708in}{0.693049in}}%
\pgfpathlineto{\pgfqpoint{4.190708in}{0.693049in}}%
\pgfpathclose%
\pgfusepath{stroke,fill}%
\end{pgfscope}%
\begin{pgfscope}%
\pgfpathrectangle{\pgfqpoint{0.661006in}{0.524170in}}{\pgfqpoint{4.194036in}{1.071446in}}%
\pgfusepath{clip}%
\pgfsetbuttcap%
\pgfsetroundjoin%
\definecolor{currentfill}{rgb}{0.247632,0.208451,0.381272}%
\pgfsetfillcolor{currentfill}%
\pgfsetfillopacity{0.700000}%
\pgfsetlinewidth{1.003750pt}%
\definecolor{currentstroke}{rgb}{0.247632,0.208451,0.381272}%
\pgfsetstrokecolor{currentstroke}%
\pgfsetstrokeopacity{0.700000}%
\pgfsetdash{}{0pt}%
\pgfpathmoveto{\pgfqpoint{4.162403in}{0.696551in}}%
\pgfpathcurveto{\pgfqpoint{4.164245in}{0.696551in}}{\pgfqpoint{4.166011in}{0.697283in}}{\pgfqpoint{4.167314in}{0.698585in}}%
\pgfpathcurveto{\pgfqpoint{4.168616in}{0.699887in}}{\pgfqpoint{4.169348in}{0.701654in}}{\pgfqpoint{4.169348in}{0.703495in}}%
\pgfpathcurveto{\pgfqpoint{4.169348in}{0.705337in}}{\pgfqpoint{4.168616in}{0.707104in}}{\pgfqpoint{4.167314in}{0.708406in}}%
\pgfpathcurveto{\pgfqpoint{4.166011in}{0.709708in}}{\pgfqpoint{4.164245in}{0.710440in}}{\pgfqpoint{4.162403in}{0.710440in}}%
\pgfpathcurveto{\pgfqpoint{4.160562in}{0.710440in}}{\pgfqpoint{4.158795in}{0.709708in}}{\pgfqpoint{4.157493in}{0.708406in}}%
\pgfpathcurveto{\pgfqpoint{4.156190in}{0.707104in}}{\pgfqpoint{4.155459in}{0.705337in}}{\pgfqpoint{4.155459in}{0.703495in}}%
\pgfpathcurveto{\pgfqpoint{4.155459in}{0.701654in}}{\pgfqpoint{4.156190in}{0.699887in}}{\pgfqpoint{4.157493in}{0.698585in}}%
\pgfpathcurveto{\pgfqpoint{4.158795in}{0.697283in}}{\pgfqpoint{4.160562in}{0.696551in}}{\pgfqpoint{4.162403in}{0.696551in}}%
\pgfpathlineto{\pgfqpoint{4.162403in}{0.696551in}}%
\pgfpathclose%
\pgfusepath{stroke,fill}%
\end{pgfscope}%
\begin{pgfscope}%
\pgfpathrectangle{\pgfqpoint{0.661006in}{0.524170in}}{\pgfqpoint{4.194036in}{1.071446in}}%
\pgfusepath{clip}%
\pgfsetbuttcap%
\pgfsetroundjoin%
\definecolor{currentfill}{rgb}{0.247632,0.208451,0.381272}%
\pgfsetfillcolor{currentfill}%
\pgfsetfillopacity{0.700000}%
\pgfsetlinewidth{1.003750pt}%
\definecolor{currentstroke}{rgb}{0.247632,0.208451,0.381272}%
\pgfsetstrokecolor{currentstroke}%
\pgfsetstrokeopacity{0.700000}%
\pgfsetdash{}{0pt}%
\pgfpathmoveto{\pgfqpoint{4.156826in}{0.698286in}}%
\pgfpathcurveto{\pgfqpoint{4.158668in}{0.698286in}}{\pgfqpoint{4.160434in}{0.699018in}}{\pgfqpoint{4.161736in}{0.700320in}}%
\pgfpathcurveto{\pgfqpoint{4.163039in}{0.701622in}}{\pgfqpoint{4.163770in}{0.703389in}}{\pgfqpoint{4.163770in}{0.705231in}}%
\pgfpathcurveto{\pgfqpoint{4.163770in}{0.707072in}}{\pgfqpoint{4.163039in}{0.708839in}}{\pgfqpoint{4.161736in}{0.710141in}}%
\pgfpathcurveto{\pgfqpoint{4.160434in}{0.711443in}}{\pgfqpoint{4.158668in}{0.712175in}}{\pgfqpoint{4.156826in}{0.712175in}}%
\pgfpathcurveto{\pgfqpoint{4.154984in}{0.712175in}}{\pgfqpoint{4.153218in}{0.711443in}}{\pgfqpoint{4.151915in}{0.710141in}}%
\pgfpathcurveto{\pgfqpoint{4.150613in}{0.708839in}}{\pgfqpoint{4.149882in}{0.707072in}}{\pgfqpoint{4.149882in}{0.705231in}}%
\pgfpathcurveto{\pgfqpoint{4.149882in}{0.703389in}}{\pgfqpoint{4.150613in}{0.701622in}}{\pgfqpoint{4.151915in}{0.700320in}}%
\pgfpathcurveto{\pgfqpoint{4.153218in}{0.699018in}}{\pgfqpoint{4.154984in}{0.698286in}}{\pgfqpoint{4.156826in}{0.698286in}}%
\pgfpathlineto{\pgfqpoint{4.156826in}{0.698286in}}%
\pgfpathclose%
\pgfusepath{stroke,fill}%
\end{pgfscope}%
\begin{pgfscope}%
\pgfpathrectangle{\pgfqpoint{0.661006in}{0.524170in}}{\pgfqpoint{4.194036in}{1.071446in}}%
\pgfusepath{clip}%
\pgfsetbuttcap%
\pgfsetroundjoin%
\definecolor{currentfill}{rgb}{0.244407,0.204500,0.375611}%
\pgfsetfillcolor{currentfill}%
\pgfsetfillopacity{0.700000}%
\pgfsetlinewidth{1.003750pt}%
\definecolor{currentstroke}{rgb}{0.244407,0.204500,0.375611}%
\pgfsetstrokecolor{currentstroke}%
\pgfsetstrokeopacity{0.700000}%
\pgfsetdash{}{0pt}%
\pgfpathmoveto{\pgfqpoint{4.164169in}{0.697910in}}%
\pgfpathcurveto{\pgfqpoint{4.166011in}{0.697910in}}{\pgfqpoint{4.167778in}{0.698642in}}{\pgfqpoint{4.169080in}{0.699944in}}%
\pgfpathcurveto{\pgfqpoint{4.170382in}{0.701246in}}{\pgfqpoint{4.171114in}{0.703013in}}{\pgfqpoint{4.171114in}{0.704854in}}%
\pgfpathcurveto{\pgfqpoint{4.171114in}{0.706696in}}{\pgfqpoint{4.170382in}{0.708462in}}{\pgfqpoint{4.169080in}{0.709765in}}%
\pgfpathcurveto{\pgfqpoint{4.167778in}{0.711067in}}{\pgfqpoint{4.166011in}{0.711799in}}{\pgfqpoint{4.164169in}{0.711799in}}%
\pgfpathcurveto{\pgfqpoint{4.162328in}{0.711799in}}{\pgfqpoint{4.160561in}{0.711067in}}{\pgfqpoint{4.159259in}{0.709765in}}%
\pgfpathcurveto{\pgfqpoint{4.157957in}{0.708462in}}{\pgfqpoint{4.157225in}{0.706696in}}{\pgfqpoint{4.157225in}{0.704854in}}%
\pgfpathcurveto{\pgfqpoint{4.157225in}{0.703013in}}{\pgfqpoint{4.157957in}{0.701246in}}{\pgfqpoint{4.159259in}{0.699944in}}%
\pgfpathcurveto{\pgfqpoint{4.160561in}{0.698642in}}{\pgfqpoint{4.162328in}{0.697910in}}{\pgfqpoint{4.164169in}{0.697910in}}%
\pgfpathlineto{\pgfqpoint{4.164169in}{0.697910in}}%
\pgfpathclose%
\pgfusepath{stroke,fill}%
\end{pgfscope}%
\begin{pgfscope}%
\pgfpathrectangle{\pgfqpoint{0.661006in}{0.524170in}}{\pgfqpoint{4.194036in}{1.071446in}}%
\pgfusepath{clip}%
\pgfsetbuttcap%
\pgfsetroundjoin%
\definecolor{currentfill}{rgb}{0.244407,0.204500,0.375611}%
\pgfsetfillcolor{currentfill}%
\pgfsetfillopacity{0.700000}%
\pgfsetlinewidth{1.003750pt}%
\definecolor{currentstroke}{rgb}{0.244407,0.204500,0.375611}%
\pgfsetstrokecolor{currentstroke}%
\pgfsetstrokeopacity{0.700000}%
\pgfsetdash{}{0pt}%
\pgfpathmoveto{\pgfqpoint{4.174115in}{0.694685in}}%
\pgfpathcurveto{\pgfqpoint{4.175957in}{0.694685in}}{\pgfqpoint{4.177724in}{0.695417in}}{\pgfqpoint{4.179026in}{0.696719in}}%
\pgfpathcurveto{\pgfqpoint{4.180328in}{0.698022in}}{\pgfqpoint{4.181060in}{0.699788in}}{\pgfqpoint{4.181060in}{0.701630in}}%
\pgfpathcurveto{\pgfqpoint{4.181060in}{0.703472in}}{\pgfqpoint{4.180328in}{0.705238in}}{\pgfqpoint{4.179026in}{0.706540in}}%
\pgfpathcurveto{\pgfqpoint{4.177724in}{0.707843in}}{\pgfqpoint{4.175957in}{0.708574in}}{\pgfqpoint{4.174115in}{0.708574in}}%
\pgfpathcurveto{\pgfqpoint{4.172274in}{0.708574in}}{\pgfqpoint{4.170507in}{0.707843in}}{\pgfqpoint{4.169205in}{0.706540in}}%
\pgfpathcurveto{\pgfqpoint{4.167903in}{0.705238in}}{\pgfqpoint{4.167171in}{0.703472in}}{\pgfqpoint{4.167171in}{0.701630in}}%
\pgfpathcurveto{\pgfqpoint{4.167171in}{0.699788in}}{\pgfqpoint{4.167903in}{0.698022in}}{\pgfqpoint{4.169205in}{0.696719in}}%
\pgfpathcurveto{\pgfqpoint{4.170507in}{0.695417in}}{\pgfqpoint{4.172274in}{0.694685in}}{\pgfqpoint{4.174115in}{0.694685in}}%
\pgfpathlineto{\pgfqpoint{4.174115in}{0.694685in}}%
\pgfpathclose%
\pgfusepath{stroke,fill}%
\end{pgfscope}%
\begin{pgfscope}%
\pgfpathrectangle{\pgfqpoint{0.661006in}{0.524170in}}{\pgfqpoint{4.194036in}{1.071446in}}%
\pgfusepath{clip}%
\pgfsetbuttcap%
\pgfsetroundjoin%
\definecolor{currentfill}{rgb}{0.244407,0.204500,0.375611}%
\pgfsetfillcolor{currentfill}%
\pgfsetfillopacity{0.700000}%
\pgfsetlinewidth{1.003750pt}%
\definecolor{currentstroke}{rgb}{0.244407,0.204500,0.375611}%
\pgfsetstrokecolor{currentstroke}%
\pgfsetstrokeopacity{0.700000}%
\pgfsetdash{}{0pt}%
\pgfpathmoveto{\pgfqpoint{4.191827in}{0.690267in}}%
\pgfpathcurveto{\pgfqpoint{4.193668in}{0.690267in}}{\pgfqpoint{4.195435in}{0.690999in}}{\pgfqpoint{4.196737in}{0.692301in}}%
\pgfpathcurveto{\pgfqpoint{4.198039in}{0.693604in}}{\pgfqpoint{4.198771in}{0.695370in}}{\pgfqpoint{4.198771in}{0.697212in}}%
\pgfpathcurveto{\pgfqpoint{4.198771in}{0.699053in}}{\pgfqpoint{4.198039in}{0.700820in}}{\pgfqpoint{4.196737in}{0.702122in}}%
\pgfpathcurveto{\pgfqpoint{4.195435in}{0.703425in}}{\pgfqpoint{4.193668in}{0.704156in}}{\pgfqpoint{4.191827in}{0.704156in}}%
\pgfpathcurveto{\pgfqpoint{4.189985in}{0.704156in}}{\pgfqpoint{4.188218in}{0.703425in}}{\pgfqpoint{4.186916in}{0.702122in}}%
\pgfpathcurveto{\pgfqpoint{4.185614in}{0.700820in}}{\pgfqpoint{4.184882in}{0.699053in}}{\pgfqpoint{4.184882in}{0.697212in}}%
\pgfpathcurveto{\pgfqpoint{4.184882in}{0.695370in}}{\pgfqpoint{4.185614in}{0.693604in}}{\pgfqpoint{4.186916in}{0.692301in}}%
\pgfpathcurveto{\pgfqpoint{4.188218in}{0.690999in}}{\pgfqpoint{4.189985in}{0.690267in}}{\pgfqpoint{4.191827in}{0.690267in}}%
\pgfpathlineto{\pgfqpoint{4.191827in}{0.690267in}}%
\pgfpathclose%
\pgfusepath{stroke,fill}%
\end{pgfscope}%
\begin{pgfscope}%
\pgfpathrectangle{\pgfqpoint{0.661006in}{0.524170in}}{\pgfqpoint{4.194036in}{1.071446in}}%
\pgfusepath{clip}%
\pgfsetbuttcap%
\pgfsetroundjoin%
\definecolor{currentfill}{rgb}{0.244407,0.204500,0.375611}%
\pgfsetfillcolor{currentfill}%
\pgfsetfillopacity{0.700000}%
\pgfsetlinewidth{1.003750pt}%
\definecolor{currentstroke}{rgb}{0.244407,0.204500,0.375611}%
\pgfsetstrokecolor{currentstroke}%
\pgfsetstrokeopacity{0.700000}%
\pgfsetdash{}{0pt}%
\pgfpathmoveto{\pgfqpoint{4.209810in}{0.686161in}}%
\pgfpathcurveto{\pgfqpoint{4.211652in}{0.686161in}}{\pgfqpoint{4.213418in}{0.686893in}}{\pgfqpoint{4.214720in}{0.688195in}}%
\pgfpathcurveto{\pgfqpoint{4.216023in}{0.689498in}}{\pgfqpoint{4.216754in}{0.691264in}}{\pgfqpoint{4.216754in}{0.693106in}}%
\pgfpathcurveto{\pgfqpoint{4.216754in}{0.694948in}}{\pgfqpoint{4.216023in}{0.696714in}}{\pgfqpoint{4.214720in}{0.698016in}}%
\pgfpathcurveto{\pgfqpoint{4.213418in}{0.699319in}}{\pgfqpoint{4.211652in}{0.700050in}}{\pgfqpoint{4.209810in}{0.700050in}}%
\pgfpathcurveto{\pgfqpoint{4.207968in}{0.700050in}}{\pgfqpoint{4.206202in}{0.699319in}}{\pgfqpoint{4.204900in}{0.698016in}}%
\pgfpathcurveto{\pgfqpoint{4.203597in}{0.696714in}}{\pgfqpoint{4.202866in}{0.694948in}}{\pgfqpoint{4.202866in}{0.693106in}}%
\pgfpathcurveto{\pgfqpoint{4.202866in}{0.691264in}}{\pgfqpoint{4.203597in}{0.689498in}}{\pgfqpoint{4.204900in}{0.688195in}}%
\pgfpathcurveto{\pgfqpoint{4.206202in}{0.686893in}}{\pgfqpoint{4.207968in}{0.686161in}}{\pgfqpoint{4.209810in}{0.686161in}}%
\pgfpathlineto{\pgfqpoint{4.209810in}{0.686161in}}%
\pgfpathclose%
\pgfusepath{stroke,fill}%
\end{pgfscope}%
\begin{pgfscope}%
\pgfpathrectangle{\pgfqpoint{0.661006in}{0.524170in}}{\pgfqpoint{4.194036in}{1.071446in}}%
\pgfusepath{clip}%
\pgfsetbuttcap%
\pgfsetroundjoin%
\definecolor{currentfill}{rgb}{0.244407,0.204500,0.375611}%
\pgfsetfillcolor{currentfill}%
\pgfsetfillopacity{0.700000}%
\pgfsetlinewidth{1.003750pt}%
\definecolor{currentstroke}{rgb}{0.244407,0.204500,0.375611}%
\pgfsetstrokecolor{currentstroke}%
\pgfsetstrokeopacity{0.700000}%
\pgfsetdash{}{0pt}%
\pgfpathmoveto{\pgfqpoint{4.235884in}{0.682028in}}%
\pgfpathcurveto{\pgfqpoint{4.237725in}{0.682028in}}{\pgfqpoint{4.239492in}{0.682760in}}{\pgfqpoint{4.240794in}{0.684062in}}%
\pgfpathcurveto{\pgfqpoint{4.242096in}{0.685364in}}{\pgfqpoint{4.242828in}{0.687131in}}{\pgfqpoint{4.242828in}{0.688972in}}%
\pgfpathcurveto{\pgfqpoint{4.242828in}{0.690814in}}{\pgfqpoint{4.242096in}{0.692580in}}{\pgfqpoint{4.240794in}{0.693883in}}%
\pgfpathcurveto{\pgfqpoint{4.239492in}{0.695185in}}{\pgfqpoint{4.237725in}{0.695917in}}{\pgfqpoint{4.235884in}{0.695917in}}%
\pgfpathcurveto{\pgfqpoint{4.234042in}{0.695917in}}{\pgfqpoint{4.232276in}{0.695185in}}{\pgfqpoint{4.230973in}{0.693883in}}%
\pgfpathcurveto{\pgfqpoint{4.229671in}{0.692580in}}{\pgfqpoint{4.228939in}{0.690814in}}{\pgfqpoint{4.228939in}{0.688972in}}%
\pgfpathcurveto{\pgfqpoint{4.228939in}{0.687131in}}{\pgfqpoint{4.229671in}{0.685364in}}{\pgfqpoint{4.230973in}{0.684062in}}%
\pgfpathcurveto{\pgfqpoint{4.232276in}{0.682760in}}{\pgfqpoint{4.234042in}{0.682028in}}{\pgfqpoint{4.235884in}{0.682028in}}%
\pgfpathlineto{\pgfqpoint{4.235884in}{0.682028in}}%
\pgfpathclose%
\pgfusepath{stroke,fill}%
\end{pgfscope}%
\begin{pgfscope}%
\pgfpathrectangle{\pgfqpoint{0.661006in}{0.524170in}}{\pgfqpoint{4.194036in}{1.071446in}}%
\pgfusepath{clip}%
\pgfsetbuttcap%
\pgfsetroundjoin%
\definecolor{currentfill}{rgb}{0.241161,0.200564,0.369923}%
\pgfsetfillcolor{currentfill}%
\pgfsetfillopacity{0.700000}%
\pgfsetlinewidth{1.003750pt}%
\definecolor{currentstroke}{rgb}{0.241161,0.200564,0.369923}%
\pgfsetstrokecolor{currentstroke}%
\pgfsetstrokeopacity{0.700000}%
\pgfsetdash{}{0pt}%
\pgfpathmoveto{\pgfqpoint{4.256055in}{0.676442in}}%
\pgfpathcurveto{\pgfqpoint{4.257897in}{0.676442in}}{\pgfqpoint{4.259663in}{0.677174in}}{\pgfqpoint{4.260965in}{0.678476in}}%
\pgfpathcurveto{\pgfqpoint{4.262268in}{0.679778in}}{\pgfqpoint{4.262999in}{0.681545in}}{\pgfqpoint{4.262999in}{0.683387in}}%
\pgfpathcurveto{\pgfqpoint{4.262999in}{0.685228in}}{\pgfqpoint{4.262268in}{0.686995in}}{\pgfqpoint{4.260965in}{0.688297in}}%
\pgfpathcurveto{\pgfqpoint{4.259663in}{0.689599in}}{\pgfqpoint{4.257897in}{0.690331in}}{\pgfqpoint{4.256055in}{0.690331in}}%
\pgfpathcurveto{\pgfqpoint{4.254213in}{0.690331in}}{\pgfqpoint{4.252447in}{0.689599in}}{\pgfqpoint{4.251144in}{0.688297in}}%
\pgfpathcurveto{\pgfqpoint{4.249842in}{0.686995in}}{\pgfqpoint{4.249110in}{0.685228in}}{\pgfqpoint{4.249110in}{0.683387in}}%
\pgfpathcurveto{\pgfqpoint{4.249110in}{0.681545in}}{\pgfqpoint{4.249842in}{0.679778in}}{\pgfqpoint{4.251144in}{0.678476in}}%
\pgfpathcurveto{\pgfqpoint{4.252447in}{0.677174in}}{\pgfqpoint{4.254213in}{0.676442in}}{\pgfqpoint{4.256055in}{0.676442in}}%
\pgfpathlineto{\pgfqpoint{4.256055in}{0.676442in}}%
\pgfpathclose%
\pgfusepath{stroke,fill}%
\end{pgfscope}%
\begin{pgfscope}%
\pgfpathrectangle{\pgfqpoint{0.661006in}{0.524170in}}{\pgfqpoint{4.194036in}{1.071446in}}%
\pgfusepath{clip}%
\pgfsetbuttcap%
\pgfsetroundjoin%
\definecolor{currentfill}{rgb}{0.241161,0.200564,0.369923}%
\pgfsetfillcolor{currentfill}%
\pgfsetfillopacity{0.700000}%
\pgfsetlinewidth{1.003750pt}%
\definecolor{currentstroke}{rgb}{0.241161,0.200564,0.369923}%
\pgfsetstrokecolor{currentstroke}%
\pgfsetstrokeopacity{0.700000}%
\pgfsetdash{}{0pt}%
\pgfpathmoveto{\pgfqpoint{4.264886in}{0.672881in}}%
\pgfpathcurveto{\pgfqpoint{4.266727in}{0.672881in}}{\pgfqpoint{4.268494in}{0.673612in}}{\pgfqpoint{4.269796in}{0.674915in}}%
\pgfpathcurveto{\pgfqpoint{4.271098in}{0.676217in}}{\pgfqpoint{4.271830in}{0.677984in}}{\pgfqpoint{4.271830in}{0.679825in}}%
\pgfpathcurveto{\pgfqpoint{4.271830in}{0.681667in}}{\pgfqpoint{4.271098in}{0.683433in}}{\pgfqpoint{4.269796in}{0.684736in}}%
\pgfpathcurveto{\pgfqpoint{4.268494in}{0.686038in}}{\pgfqpoint{4.266727in}{0.686770in}}{\pgfqpoint{4.264886in}{0.686770in}}%
\pgfpathcurveto{\pgfqpoint{4.263044in}{0.686770in}}{\pgfqpoint{4.261277in}{0.686038in}}{\pgfqpoint{4.259975in}{0.684736in}}%
\pgfpathcurveto{\pgfqpoint{4.258673in}{0.683433in}}{\pgfqpoint{4.257941in}{0.681667in}}{\pgfqpoint{4.257941in}{0.679825in}}%
\pgfpathcurveto{\pgfqpoint{4.257941in}{0.677984in}}{\pgfqpoint{4.258673in}{0.676217in}}{\pgfqpoint{4.259975in}{0.674915in}}%
\pgfpathcurveto{\pgfqpoint{4.261277in}{0.673612in}}{\pgfqpoint{4.263044in}{0.672881in}}{\pgfqpoint{4.264886in}{0.672881in}}%
\pgfpathlineto{\pgfqpoint{4.264886in}{0.672881in}}%
\pgfpathclose%
\pgfusepath{stroke,fill}%
\end{pgfscope}%
\begin{pgfscope}%
\pgfpathrectangle{\pgfqpoint{0.661006in}{0.524170in}}{\pgfqpoint{4.194036in}{1.071446in}}%
\pgfusepath{clip}%
\pgfsetbuttcap%
\pgfsetroundjoin%
\definecolor{currentfill}{rgb}{0.237895,0.196644,0.364209}%
\pgfsetfillcolor{currentfill}%
\pgfsetfillopacity{0.700000}%
\pgfsetlinewidth{1.003750pt}%
\definecolor{currentstroke}{rgb}{0.237895,0.196644,0.364209}%
\pgfsetstrokecolor{currentstroke}%
\pgfsetstrokeopacity{0.700000}%
\pgfsetdash{}{0pt}%
\pgfpathmoveto{\pgfqpoint{4.299465in}{0.667566in}}%
\pgfpathcurveto{\pgfqpoint{4.301306in}{0.667566in}}{\pgfqpoint{4.303073in}{0.668298in}}{\pgfqpoint{4.304375in}{0.669600in}}%
\pgfpathcurveto{\pgfqpoint{4.305677in}{0.670902in}}{\pgfqpoint{4.306409in}{0.672669in}}{\pgfqpoint{4.306409in}{0.674510in}}%
\pgfpathcurveto{\pgfqpoint{4.306409in}{0.676352in}}{\pgfqpoint{4.305677in}{0.678119in}}{\pgfqpoint{4.304375in}{0.679421in}}%
\pgfpathcurveto{\pgfqpoint{4.303073in}{0.680723in}}{\pgfqpoint{4.301306in}{0.681455in}}{\pgfqpoint{4.299465in}{0.681455in}}%
\pgfpathcurveto{\pgfqpoint{4.297623in}{0.681455in}}{\pgfqpoint{4.295856in}{0.680723in}}{\pgfqpoint{4.294554in}{0.679421in}}%
\pgfpathcurveto{\pgfqpoint{4.293252in}{0.678119in}}{\pgfqpoint{4.292520in}{0.676352in}}{\pgfqpoint{4.292520in}{0.674510in}}%
\pgfpathcurveto{\pgfqpoint{4.292520in}{0.672669in}}{\pgfqpoint{4.293252in}{0.670902in}}{\pgfqpoint{4.294554in}{0.669600in}}%
\pgfpathcurveto{\pgfqpoint{4.295856in}{0.668298in}}{\pgfqpoint{4.297623in}{0.667566in}}{\pgfqpoint{4.299465in}{0.667566in}}%
\pgfpathlineto{\pgfqpoint{4.299465in}{0.667566in}}%
\pgfpathclose%
\pgfusepath{stroke,fill}%
\end{pgfscope}%
\begin{pgfscope}%
\pgfpathrectangle{\pgfqpoint{0.661006in}{0.524170in}}{\pgfqpoint{4.194036in}{1.071446in}}%
\pgfusepath{clip}%
\pgfsetbuttcap%
\pgfsetroundjoin%
\definecolor{currentfill}{rgb}{0.237895,0.196644,0.364209}%
\pgfsetfillcolor{currentfill}%
\pgfsetfillopacity{0.700000}%
\pgfsetlinewidth{1.003750pt}%
\definecolor{currentstroke}{rgb}{0.237895,0.196644,0.364209}%
\pgfsetstrokecolor{currentstroke}%
\pgfsetstrokeopacity{0.700000}%
\pgfsetdash{}{0pt}%
\pgfpathmoveto{\pgfqpoint{4.289472in}{0.667537in}}%
\pgfpathcurveto{\pgfqpoint{4.291314in}{0.667537in}}{\pgfqpoint{4.293080in}{0.668269in}}{\pgfqpoint{4.294382in}{0.669571in}}%
\pgfpathcurveto{\pgfqpoint{4.295685in}{0.670873in}}{\pgfqpoint{4.296416in}{0.672640in}}{\pgfqpoint{4.296416in}{0.674482in}}%
\pgfpathcurveto{\pgfqpoint{4.296416in}{0.676323in}}{\pgfqpoint{4.295685in}{0.678090in}}{\pgfqpoint{4.294382in}{0.679392in}}%
\pgfpathcurveto{\pgfqpoint{4.293080in}{0.680694in}}{\pgfqpoint{4.291314in}{0.681426in}}{\pgfqpoint{4.289472in}{0.681426in}}%
\pgfpathcurveto{\pgfqpoint{4.287630in}{0.681426in}}{\pgfqpoint{4.285864in}{0.680694in}}{\pgfqpoint{4.284562in}{0.679392in}}%
\pgfpathcurveto{\pgfqpoint{4.283259in}{0.678090in}}{\pgfqpoint{4.282528in}{0.676323in}}{\pgfqpoint{4.282528in}{0.674482in}}%
\pgfpathcurveto{\pgfqpoint{4.282528in}{0.672640in}}{\pgfqpoint{4.283259in}{0.670873in}}{\pgfqpoint{4.284562in}{0.669571in}}%
\pgfpathcurveto{\pgfqpoint{4.285864in}{0.668269in}}{\pgfqpoint{4.287630in}{0.667537in}}{\pgfqpoint{4.289472in}{0.667537in}}%
\pgfpathlineto{\pgfqpoint{4.289472in}{0.667537in}}%
\pgfpathclose%
\pgfusepath{stroke,fill}%
\end{pgfscope}%
\begin{pgfscope}%
\pgfpathrectangle{\pgfqpoint{0.661006in}{0.524170in}}{\pgfqpoint{4.194036in}{1.071446in}}%
\pgfusepath{clip}%
\pgfsetbuttcap%
\pgfsetroundjoin%
\definecolor{currentfill}{rgb}{0.237895,0.196644,0.364209}%
\pgfsetfillcolor{currentfill}%
\pgfsetfillopacity{0.700000}%
\pgfsetlinewidth{1.003750pt}%
\definecolor{currentstroke}{rgb}{0.237895,0.196644,0.364209}%
\pgfsetstrokecolor{currentstroke}%
\pgfsetstrokeopacity{0.700000}%
\pgfsetdash{}{0pt}%
\pgfpathmoveto{\pgfqpoint{4.291378in}{0.668674in}}%
\pgfpathcurveto{\pgfqpoint{4.293219in}{0.668674in}}{\pgfqpoint{4.294986in}{0.669405in}}{\pgfqpoint{4.296288in}{0.670708in}}%
\pgfpathcurveto{\pgfqpoint{4.297590in}{0.672010in}}{\pgfqpoint{4.298322in}{0.673776in}}{\pgfqpoint{4.298322in}{0.675618in}}%
\pgfpathcurveto{\pgfqpoint{4.298322in}{0.677460in}}{\pgfqpoint{4.297590in}{0.679226in}}{\pgfqpoint{4.296288in}{0.680529in}}%
\pgfpathcurveto{\pgfqpoint{4.294986in}{0.681831in}}{\pgfqpoint{4.293219in}{0.682563in}}{\pgfqpoint{4.291378in}{0.682563in}}%
\pgfpathcurveto{\pgfqpoint{4.289536in}{0.682563in}}{\pgfqpoint{4.287769in}{0.681831in}}{\pgfqpoint{4.286467in}{0.680529in}}%
\pgfpathcurveto{\pgfqpoint{4.285165in}{0.679226in}}{\pgfqpoint{4.284433in}{0.677460in}}{\pgfqpoint{4.284433in}{0.675618in}}%
\pgfpathcurveto{\pgfqpoint{4.284433in}{0.673776in}}{\pgfqpoint{4.285165in}{0.672010in}}{\pgfqpoint{4.286467in}{0.670708in}}%
\pgfpathcurveto{\pgfqpoint{4.287769in}{0.669405in}}{\pgfqpoint{4.289536in}{0.668674in}}{\pgfqpoint{4.291378in}{0.668674in}}%
\pgfpathlineto{\pgfqpoint{4.291378in}{0.668674in}}%
\pgfpathclose%
\pgfusepath{stroke,fill}%
\end{pgfscope}%
\begin{pgfscope}%
\pgfpathrectangle{\pgfqpoint{0.661006in}{0.524170in}}{\pgfqpoint{4.194036in}{1.071446in}}%
\pgfusepath{clip}%
\pgfsetbuttcap%
\pgfsetroundjoin%
\definecolor{currentfill}{rgb}{0.237895,0.196644,0.364209}%
\pgfsetfillcolor{currentfill}%
\pgfsetfillopacity{0.700000}%
\pgfsetlinewidth{1.003750pt}%
\definecolor{currentstroke}{rgb}{0.237895,0.196644,0.364209}%
\pgfsetstrokecolor{currentstroke}%
\pgfsetstrokeopacity{0.700000}%
\pgfsetdash{}{0pt}%
\pgfpathmoveto{\pgfqpoint{4.295282in}{0.667433in}}%
\pgfpathcurveto{\pgfqpoint{4.297123in}{0.667433in}}{\pgfqpoint{4.298890in}{0.668164in}}{\pgfqpoint{4.300192in}{0.669467in}}%
\pgfpathcurveto{\pgfqpoint{4.301494in}{0.670769in}}{\pgfqpoint{4.302226in}{0.672535in}}{\pgfqpoint{4.302226in}{0.674377in}}%
\pgfpathcurveto{\pgfqpoint{4.302226in}{0.676219in}}{\pgfqpoint{4.301494in}{0.677985in}}{\pgfqpoint{4.300192in}{0.679288in}}%
\pgfpathcurveto{\pgfqpoint{4.298890in}{0.680590in}}{\pgfqpoint{4.297123in}{0.681322in}}{\pgfqpoint{4.295282in}{0.681322in}}%
\pgfpathcurveto{\pgfqpoint{4.293440in}{0.681322in}}{\pgfqpoint{4.291673in}{0.680590in}}{\pgfqpoint{4.290371in}{0.679288in}}%
\pgfpathcurveto{\pgfqpoint{4.289069in}{0.677985in}}{\pgfqpoint{4.288337in}{0.676219in}}{\pgfqpoint{4.288337in}{0.674377in}}%
\pgfpathcurveto{\pgfqpoint{4.288337in}{0.672535in}}{\pgfqpoint{4.289069in}{0.670769in}}{\pgfqpoint{4.290371in}{0.669467in}}%
\pgfpathcurveto{\pgfqpoint{4.291673in}{0.668164in}}{\pgfqpoint{4.293440in}{0.667433in}}{\pgfqpoint{4.295282in}{0.667433in}}%
\pgfpathlineto{\pgfqpoint{4.295282in}{0.667433in}}%
\pgfpathclose%
\pgfusepath{stroke,fill}%
\end{pgfscope}%
\begin{pgfscope}%
\pgfpathrectangle{\pgfqpoint{0.661006in}{0.524170in}}{\pgfqpoint{4.194036in}{1.071446in}}%
\pgfusepath{clip}%
\pgfsetbuttcap%
\pgfsetroundjoin%
\definecolor{currentfill}{rgb}{0.237895,0.196644,0.364209}%
\pgfsetfillcolor{currentfill}%
\pgfsetfillopacity{0.700000}%
\pgfsetlinewidth{1.003750pt}%
\definecolor{currentstroke}{rgb}{0.237895,0.196644,0.364209}%
\pgfsetstrokecolor{currentstroke}%
\pgfsetstrokeopacity{0.700000}%
\pgfsetdash{}{0pt}%
\pgfpathmoveto{\pgfqpoint{4.308714in}{0.662755in}}%
\pgfpathcurveto{\pgfqpoint{4.310555in}{0.662755in}}{\pgfqpoint{4.312322in}{0.663487in}}{\pgfqpoint{4.313624in}{0.664789in}}%
\pgfpathcurveto{\pgfqpoint{4.314926in}{0.666092in}}{\pgfqpoint{4.315658in}{0.667858in}}{\pgfqpoint{4.315658in}{0.669700in}}%
\pgfpathcurveto{\pgfqpoint{4.315658in}{0.671541in}}{\pgfqpoint{4.314926in}{0.673308in}}{\pgfqpoint{4.313624in}{0.674610in}}%
\pgfpathcurveto{\pgfqpoint{4.312322in}{0.675913in}}{\pgfqpoint{4.310555in}{0.676644in}}{\pgfqpoint{4.308714in}{0.676644in}}%
\pgfpathcurveto{\pgfqpoint{4.306872in}{0.676644in}}{\pgfqpoint{4.305105in}{0.675913in}}{\pgfqpoint{4.303803in}{0.674610in}}%
\pgfpathcurveto{\pgfqpoint{4.302501in}{0.673308in}}{\pgfqpoint{4.301769in}{0.671541in}}{\pgfqpoint{4.301769in}{0.669700in}}%
\pgfpathcurveto{\pgfqpoint{4.301769in}{0.667858in}}{\pgfqpoint{4.302501in}{0.666092in}}{\pgfqpoint{4.303803in}{0.664789in}}%
\pgfpathcurveto{\pgfqpoint{4.305105in}{0.663487in}}{\pgfqpoint{4.306872in}{0.662755in}}{\pgfqpoint{4.308714in}{0.662755in}}%
\pgfpathlineto{\pgfqpoint{4.308714in}{0.662755in}}%
\pgfpathclose%
\pgfusepath{stroke,fill}%
\end{pgfscope}%
\begin{pgfscope}%
\pgfpathrectangle{\pgfqpoint{0.661006in}{0.524170in}}{\pgfqpoint{4.194036in}{1.071446in}}%
\pgfusepath{clip}%
\pgfsetbuttcap%
\pgfsetroundjoin%
\definecolor{currentfill}{rgb}{0.234607,0.192738,0.358470}%
\pgfsetfillcolor{currentfill}%
\pgfsetfillopacity{0.700000}%
\pgfsetlinewidth{1.003750pt}%
\definecolor{currentstroke}{rgb}{0.234607,0.192738,0.358470}%
\pgfsetstrokecolor{currentstroke}%
\pgfsetstrokeopacity{0.700000}%
\pgfsetdash{}{0pt}%
\pgfpathmoveto{\pgfqpoint{4.319729in}{0.659602in}}%
\pgfpathcurveto{\pgfqpoint{4.321570in}{0.659602in}}{\pgfqpoint{4.323337in}{0.660333in}}{\pgfqpoint{4.324639in}{0.661636in}}%
\pgfpathcurveto{\pgfqpoint{4.325941in}{0.662938in}}{\pgfqpoint{4.326673in}{0.664705in}}{\pgfqpoint{4.326673in}{0.666546in}}%
\pgfpathcurveto{\pgfqpoint{4.326673in}{0.668388in}}{\pgfqpoint{4.325941in}{0.670154in}}{\pgfqpoint{4.324639in}{0.671457in}}%
\pgfpathcurveto{\pgfqpoint{4.323337in}{0.672759in}}{\pgfqpoint{4.321570in}{0.673491in}}{\pgfqpoint{4.319729in}{0.673491in}}%
\pgfpathcurveto{\pgfqpoint{4.317887in}{0.673491in}}{\pgfqpoint{4.316120in}{0.672759in}}{\pgfqpoint{4.314818in}{0.671457in}}%
\pgfpathcurveto{\pgfqpoint{4.313516in}{0.670154in}}{\pgfqpoint{4.312784in}{0.668388in}}{\pgfqpoint{4.312784in}{0.666546in}}%
\pgfpathcurveto{\pgfqpoint{4.312784in}{0.664705in}}{\pgfqpoint{4.313516in}{0.662938in}}{\pgfqpoint{4.314818in}{0.661636in}}%
\pgfpathcurveto{\pgfqpoint{4.316120in}{0.660333in}}{\pgfqpoint{4.317887in}{0.659602in}}{\pgfqpoint{4.319729in}{0.659602in}}%
\pgfpathlineto{\pgfqpoint{4.319729in}{0.659602in}}%
\pgfpathclose%
\pgfusepath{stroke,fill}%
\end{pgfscope}%
\begin{pgfscope}%
\pgfpathrectangle{\pgfqpoint{0.661006in}{0.524170in}}{\pgfqpoint{4.194036in}{1.071446in}}%
\pgfusepath{clip}%
\pgfsetbuttcap%
\pgfsetroundjoin%
\definecolor{currentfill}{rgb}{0.234607,0.192738,0.358470}%
\pgfsetfillcolor{currentfill}%
\pgfsetfillopacity{0.700000}%
\pgfsetlinewidth{1.003750pt}%
\definecolor{currentstroke}{rgb}{0.234607,0.192738,0.358470}%
\pgfsetstrokecolor{currentstroke}%
\pgfsetstrokeopacity{0.700000}%
\pgfsetdash{}{0pt}%
\pgfpathmoveto{\pgfqpoint{4.330418in}{0.657693in}}%
\pgfpathcurveto{\pgfqpoint{4.332260in}{0.657693in}}{\pgfqpoint{4.334027in}{0.658425in}}{\pgfqpoint{4.335329in}{0.659727in}}%
\pgfpathcurveto{\pgfqpoint{4.336631in}{0.661030in}}{\pgfqpoint{4.337363in}{0.662796in}}{\pgfqpoint{4.337363in}{0.664638in}}%
\pgfpathcurveto{\pgfqpoint{4.337363in}{0.666479in}}{\pgfqpoint{4.336631in}{0.668246in}}{\pgfqpoint{4.335329in}{0.669548in}}%
\pgfpathcurveto{\pgfqpoint{4.334027in}{0.670850in}}{\pgfqpoint{4.332260in}{0.671582in}}{\pgfqpoint{4.330418in}{0.671582in}}%
\pgfpathcurveto{\pgfqpoint{4.328577in}{0.671582in}}{\pgfqpoint{4.326810in}{0.670850in}}{\pgfqpoint{4.325508in}{0.669548in}}%
\pgfpathcurveto{\pgfqpoint{4.324206in}{0.668246in}}{\pgfqpoint{4.323474in}{0.666479in}}{\pgfqpoint{4.323474in}{0.664638in}}%
\pgfpathcurveto{\pgfqpoint{4.323474in}{0.662796in}}{\pgfqpoint{4.324206in}{0.661030in}}{\pgfqpoint{4.325508in}{0.659727in}}%
\pgfpathcurveto{\pgfqpoint{4.326810in}{0.658425in}}{\pgfqpoint{4.328577in}{0.657693in}}{\pgfqpoint{4.330418in}{0.657693in}}%
\pgfpathlineto{\pgfqpoint{4.330418in}{0.657693in}}%
\pgfpathclose%
\pgfusepath{stroke,fill}%
\end{pgfscope}%
\begin{pgfscope}%
\pgfpathrectangle{\pgfqpoint{0.661006in}{0.524170in}}{\pgfqpoint{4.194036in}{1.071446in}}%
\pgfusepath{clip}%
\pgfsetbuttcap%
\pgfsetroundjoin%
\definecolor{currentfill}{rgb}{0.231299,0.188849,0.352705}%
\pgfsetfillcolor{currentfill}%
\pgfsetfillopacity{0.700000}%
\pgfsetlinewidth{1.003750pt}%
\definecolor{currentstroke}{rgb}{0.231299,0.188849,0.352705}%
\pgfsetstrokecolor{currentstroke}%
\pgfsetstrokeopacity{0.700000}%
\pgfsetdash{}{0pt}%
\pgfpathmoveto{\pgfqpoint{4.337529in}{0.653246in}}%
\pgfpathcurveto{\pgfqpoint{4.339371in}{0.653246in}}{\pgfqpoint{4.341138in}{0.653978in}}{\pgfqpoint{4.342440in}{0.655280in}}%
\pgfpathcurveto{\pgfqpoint{4.343742in}{0.656582in}}{\pgfqpoint{4.344474in}{0.658349in}}{\pgfqpoint{4.344474in}{0.660190in}}%
\pgfpathcurveto{\pgfqpoint{4.344474in}{0.662032in}}{\pgfqpoint{4.343742in}{0.663799in}}{\pgfqpoint{4.342440in}{0.665101in}}%
\pgfpathcurveto{\pgfqpoint{4.341138in}{0.666403in}}{\pgfqpoint{4.339371in}{0.667135in}}{\pgfqpoint{4.337529in}{0.667135in}}%
\pgfpathcurveto{\pgfqpoint{4.335688in}{0.667135in}}{\pgfqpoint{4.333921in}{0.666403in}}{\pgfqpoint{4.332619in}{0.665101in}}%
\pgfpathcurveto{\pgfqpoint{4.331317in}{0.663799in}}{\pgfqpoint{4.330585in}{0.662032in}}{\pgfqpoint{4.330585in}{0.660190in}}%
\pgfpathcurveto{\pgfqpoint{4.330585in}{0.658349in}}{\pgfqpoint{4.331317in}{0.656582in}}{\pgfqpoint{4.332619in}{0.655280in}}%
\pgfpathcurveto{\pgfqpoint{4.333921in}{0.653978in}}{\pgfqpoint{4.335688in}{0.653246in}}{\pgfqpoint{4.337529in}{0.653246in}}%
\pgfpathlineto{\pgfqpoint{4.337529in}{0.653246in}}%
\pgfpathclose%
\pgfusepath{stroke,fill}%
\end{pgfscope}%
\begin{pgfscope}%
\pgfpathrectangle{\pgfqpoint{0.661006in}{0.524170in}}{\pgfqpoint{4.194036in}{1.071446in}}%
\pgfusepath{clip}%
\pgfsetbuttcap%
\pgfsetroundjoin%
\definecolor{currentfill}{rgb}{0.231299,0.188849,0.352705}%
\pgfsetfillcolor{currentfill}%
\pgfsetfillopacity{0.700000}%
\pgfsetlinewidth{1.003750pt}%
\definecolor{currentstroke}{rgb}{0.231299,0.188849,0.352705}%
\pgfsetstrokecolor{currentstroke}%
\pgfsetstrokeopacity{0.700000}%
\pgfsetdash{}{0pt}%
\pgfpathmoveto{\pgfqpoint{4.357983in}{0.649010in}}%
\pgfpathcurveto{\pgfqpoint{4.359824in}{0.649010in}}{\pgfqpoint{4.361591in}{0.649742in}}{\pgfqpoint{4.362893in}{0.651044in}}%
\pgfpathcurveto{\pgfqpoint{4.364196in}{0.652346in}}{\pgfqpoint{4.364927in}{0.654113in}}{\pgfqpoint{4.364927in}{0.655955in}}%
\pgfpathcurveto{\pgfqpoint{4.364927in}{0.657796in}}{\pgfqpoint{4.364196in}{0.659563in}}{\pgfqpoint{4.362893in}{0.660865in}}%
\pgfpathcurveto{\pgfqpoint{4.361591in}{0.662167in}}{\pgfqpoint{4.359824in}{0.662899in}}{\pgfqpoint{4.357983in}{0.662899in}}%
\pgfpathcurveto{\pgfqpoint{4.356141in}{0.662899in}}{\pgfqpoint{4.354375in}{0.662167in}}{\pgfqpoint{4.353072in}{0.660865in}}%
\pgfpathcurveto{\pgfqpoint{4.351770in}{0.659563in}}{\pgfqpoint{4.351038in}{0.657796in}}{\pgfqpoint{4.351038in}{0.655955in}}%
\pgfpathcurveto{\pgfqpoint{4.351038in}{0.654113in}}{\pgfqpoint{4.351770in}{0.652346in}}{\pgfqpoint{4.353072in}{0.651044in}}%
\pgfpathcurveto{\pgfqpoint{4.354375in}{0.649742in}}{\pgfqpoint{4.356141in}{0.649010in}}{\pgfqpoint{4.357983in}{0.649010in}}%
\pgfpathlineto{\pgfqpoint{4.357983in}{0.649010in}}%
\pgfpathclose%
\pgfusepath{stroke,fill}%
\end{pgfscope}%
\begin{pgfscope}%
\pgfpathrectangle{\pgfqpoint{0.661006in}{0.524170in}}{\pgfqpoint{4.194036in}{1.071446in}}%
\pgfusepath{clip}%
\pgfsetbuttcap%
\pgfsetroundjoin%
\definecolor{currentfill}{rgb}{0.231299,0.188849,0.352705}%
\pgfsetfillcolor{currentfill}%
\pgfsetfillopacity{0.700000}%
\pgfsetlinewidth{1.003750pt}%
\definecolor{currentstroke}{rgb}{0.231299,0.188849,0.352705}%
\pgfsetstrokecolor{currentstroke}%
\pgfsetstrokeopacity{0.700000}%
\pgfsetdash{}{0pt}%
\pgfpathmoveto{\pgfqpoint{4.369459in}{0.647745in}}%
\pgfpathcurveto{\pgfqpoint{4.371301in}{0.647745in}}{\pgfqpoint{4.373068in}{0.648477in}}{\pgfqpoint{4.374370in}{0.649779in}}%
\pgfpathcurveto{\pgfqpoint{4.375672in}{0.651082in}}{\pgfqpoint{4.376404in}{0.652848in}}{\pgfqpoint{4.376404in}{0.654690in}}%
\pgfpathcurveto{\pgfqpoint{4.376404in}{0.656532in}}{\pgfqpoint{4.375672in}{0.658298in}}{\pgfqpoint{4.374370in}{0.659600in}}%
\pgfpathcurveto{\pgfqpoint{4.373068in}{0.660903in}}{\pgfqpoint{4.371301in}{0.661634in}}{\pgfqpoint{4.369459in}{0.661634in}}%
\pgfpathcurveto{\pgfqpoint{4.367618in}{0.661634in}}{\pgfqpoint{4.365851in}{0.660903in}}{\pgfqpoint{4.364549in}{0.659600in}}%
\pgfpathcurveto{\pgfqpoint{4.363247in}{0.658298in}}{\pgfqpoint{4.362515in}{0.656532in}}{\pgfqpoint{4.362515in}{0.654690in}}%
\pgfpathcurveto{\pgfqpoint{4.362515in}{0.652848in}}{\pgfqpoint{4.363247in}{0.651082in}}{\pgfqpoint{4.364549in}{0.649779in}}%
\pgfpathcurveto{\pgfqpoint{4.365851in}{0.648477in}}{\pgfqpoint{4.367618in}{0.647745in}}{\pgfqpoint{4.369459in}{0.647745in}}%
\pgfpathlineto{\pgfqpoint{4.369459in}{0.647745in}}%
\pgfpathclose%
\pgfusepath{stroke,fill}%
\end{pgfscope}%
\begin{pgfscope}%
\pgfpathrectangle{\pgfqpoint{0.661006in}{0.524170in}}{\pgfqpoint{4.194036in}{1.071446in}}%
\pgfusepath{clip}%
\pgfsetbuttcap%
\pgfsetroundjoin%
\definecolor{currentfill}{rgb}{0.227968,0.184975,0.346915}%
\pgfsetfillcolor{currentfill}%
\pgfsetfillopacity{0.700000}%
\pgfsetlinewidth{1.003750pt}%
\definecolor{currentstroke}{rgb}{0.227968,0.184975,0.346915}%
\pgfsetstrokecolor{currentstroke}%
\pgfsetstrokeopacity{0.700000}%
\pgfsetdash{}{0pt}%
\pgfpathmoveto{\pgfqpoint{4.361140in}{0.648634in}}%
\pgfpathcurveto{\pgfqpoint{4.362982in}{0.648634in}}{\pgfqpoint{4.364748in}{0.649365in}}{\pgfqpoint{4.366050in}{0.650668in}}%
\pgfpathcurveto{\pgfqpoint{4.367353in}{0.651970in}}{\pgfqpoint{4.368084in}{0.653736in}}{\pgfqpoint{4.368084in}{0.655578in}}%
\pgfpathcurveto{\pgfqpoint{4.368084in}{0.657420in}}{\pgfqpoint{4.367353in}{0.659186in}}{\pgfqpoint{4.366050in}{0.660488in}}%
\pgfpathcurveto{\pgfqpoint{4.364748in}{0.661791in}}{\pgfqpoint{4.362982in}{0.662522in}}{\pgfqpoint{4.361140in}{0.662522in}}%
\pgfpathcurveto{\pgfqpoint{4.359298in}{0.662522in}}{\pgfqpoint{4.357532in}{0.661791in}}{\pgfqpoint{4.356229in}{0.660488in}}%
\pgfpathcurveto{\pgfqpoint{4.354927in}{0.659186in}}{\pgfqpoint{4.354195in}{0.657420in}}{\pgfqpoint{4.354195in}{0.655578in}}%
\pgfpathcurveto{\pgfqpoint{4.354195in}{0.653736in}}{\pgfqpoint{4.354927in}{0.651970in}}{\pgfqpoint{4.356229in}{0.650668in}}%
\pgfpathcurveto{\pgfqpoint{4.357532in}{0.649365in}}{\pgfqpoint{4.359298in}{0.648634in}}{\pgfqpoint{4.361140in}{0.648634in}}%
\pgfpathlineto{\pgfqpoint{4.361140in}{0.648634in}}%
\pgfpathclose%
\pgfusepath{stroke,fill}%
\end{pgfscope}%
\begin{pgfscope}%
\pgfpathrectangle{\pgfqpoint{0.661006in}{0.524170in}}{\pgfqpoint{4.194036in}{1.071446in}}%
\pgfusepath{clip}%
\pgfsetbuttcap%
\pgfsetroundjoin%
\definecolor{currentfill}{rgb}{0.227968,0.184975,0.346915}%
\pgfsetfillcolor{currentfill}%
\pgfsetfillopacity{0.700000}%
\pgfsetlinewidth{1.003750pt}%
\definecolor{currentstroke}{rgb}{0.227968,0.184975,0.346915}%
\pgfsetstrokecolor{currentstroke}%
\pgfsetstrokeopacity{0.700000}%
\pgfsetdash{}{0pt}%
\pgfpathmoveto{\pgfqpoint{4.354075in}{0.648596in}}%
\pgfpathcurveto{\pgfqpoint{4.355917in}{0.648596in}}{\pgfqpoint{4.357684in}{0.649327in}}{\pgfqpoint{4.358986in}{0.650630in}}%
\pgfpathcurveto{\pgfqpoint{4.360288in}{0.651932in}}{\pgfqpoint{4.361020in}{0.653698in}}{\pgfqpoint{4.361020in}{0.655540in}}%
\pgfpathcurveto{\pgfqpoint{4.361020in}{0.657382in}}{\pgfqpoint{4.360288in}{0.659148in}}{\pgfqpoint{4.358986in}{0.660450in}}%
\pgfpathcurveto{\pgfqpoint{4.357684in}{0.661753in}}{\pgfqpoint{4.355917in}{0.662484in}}{\pgfqpoint{4.354075in}{0.662484in}}%
\pgfpathcurveto{\pgfqpoint{4.352234in}{0.662484in}}{\pgfqpoint{4.350467in}{0.661753in}}{\pgfqpoint{4.349165in}{0.660450in}}%
\pgfpathcurveto{\pgfqpoint{4.347863in}{0.659148in}}{\pgfqpoint{4.347131in}{0.657382in}}{\pgfqpoint{4.347131in}{0.655540in}}%
\pgfpathcurveto{\pgfqpoint{4.347131in}{0.653698in}}{\pgfqpoint{4.347863in}{0.651932in}}{\pgfqpoint{4.349165in}{0.650630in}}%
\pgfpathcurveto{\pgfqpoint{4.350467in}{0.649327in}}{\pgfqpoint{4.352234in}{0.648596in}}{\pgfqpoint{4.354075in}{0.648596in}}%
\pgfpathlineto{\pgfqpoint{4.354075in}{0.648596in}}%
\pgfpathclose%
\pgfusepath{stroke,fill}%
\end{pgfscope}%
\begin{pgfscope}%
\pgfpathrectangle{\pgfqpoint{0.661006in}{0.524170in}}{\pgfqpoint{4.194036in}{1.071446in}}%
\pgfusepath{clip}%
\pgfsetbuttcap%
\pgfsetroundjoin%
\definecolor{currentfill}{rgb}{0.227968,0.184975,0.346915}%
\pgfsetfillcolor{currentfill}%
\pgfsetfillopacity{0.700000}%
\pgfsetlinewidth{1.003750pt}%
\definecolor{currentstroke}{rgb}{0.227968,0.184975,0.346915}%
\pgfsetstrokecolor{currentstroke}%
\pgfsetstrokeopacity{0.700000}%
\pgfsetdash{}{0pt}%
\pgfpathmoveto{\pgfqpoint{4.360629in}{0.645866in}}%
\pgfpathcurveto{\pgfqpoint{4.362470in}{0.645866in}}{\pgfqpoint{4.364237in}{0.646598in}}{\pgfqpoint{4.365539in}{0.647900in}}%
\pgfpathcurveto{\pgfqpoint{4.366841in}{0.649203in}}{\pgfqpoint{4.367573in}{0.650969in}}{\pgfqpoint{4.367573in}{0.652811in}}%
\pgfpathcurveto{\pgfqpoint{4.367573in}{0.654653in}}{\pgfqpoint{4.366841in}{0.656419in}}{\pgfqpoint{4.365539in}{0.657721in}}%
\pgfpathcurveto{\pgfqpoint{4.364237in}{0.659024in}}{\pgfqpoint{4.362470in}{0.659755in}}{\pgfqpoint{4.360629in}{0.659755in}}%
\pgfpathcurveto{\pgfqpoint{4.358787in}{0.659755in}}{\pgfqpoint{4.357020in}{0.659024in}}{\pgfqpoint{4.355718in}{0.657721in}}%
\pgfpathcurveto{\pgfqpoint{4.354416in}{0.656419in}}{\pgfqpoint{4.353684in}{0.654653in}}{\pgfqpoint{4.353684in}{0.652811in}}%
\pgfpathcurveto{\pgfqpoint{4.353684in}{0.650969in}}{\pgfqpoint{4.354416in}{0.649203in}}{\pgfqpoint{4.355718in}{0.647900in}}%
\pgfpathcurveto{\pgfqpoint{4.357020in}{0.646598in}}{\pgfqpoint{4.358787in}{0.645866in}}{\pgfqpoint{4.360629in}{0.645866in}}%
\pgfpathlineto{\pgfqpoint{4.360629in}{0.645866in}}%
\pgfpathclose%
\pgfusepath{stroke,fill}%
\end{pgfscope}%
\begin{pgfscope}%
\pgfpathrectangle{\pgfqpoint{0.661006in}{0.524170in}}{\pgfqpoint{4.194036in}{1.071446in}}%
\pgfusepath{clip}%
\pgfsetbuttcap%
\pgfsetroundjoin%
\definecolor{currentfill}{rgb}{0.227968,0.184975,0.346915}%
\pgfsetfillcolor{currentfill}%
\pgfsetfillopacity{0.700000}%
\pgfsetlinewidth{1.003750pt}%
\definecolor{currentstroke}{rgb}{0.227968,0.184975,0.346915}%
\pgfsetstrokecolor{currentstroke}%
\pgfsetstrokeopacity{0.700000}%
\pgfsetdash{}{0pt}%
\pgfpathmoveto{\pgfqpoint{4.379777in}{0.642309in}}%
\pgfpathcurveto{\pgfqpoint{4.381619in}{0.642309in}}{\pgfqpoint{4.383385in}{0.643041in}}{\pgfqpoint{4.384688in}{0.644343in}}%
\pgfpathcurveto{\pgfqpoint{4.385990in}{0.645646in}}{\pgfqpoint{4.386722in}{0.647412in}}{\pgfqpoint{4.386722in}{0.649254in}}%
\pgfpathcurveto{\pgfqpoint{4.386722in}{0.651095in}}{\pgfqpoint{4.385990in}{0.652862in}}{\pgfqpoint{4.384688in}{0.654164in}}%
\pgfpathcurveto{\pgfqpoint{4.383385in}{0.655466in}}{\pgfqpoint{4.381619in}{0.656198in}}{\pgfqpoint{4.379777in}{0.656198in}}%
\pgfpathcurveto{\pgfqpoint{4.377936in}{0.656198in}}{\pgfqpoint{4.376169in}{0.655466in}}{\pgfqpoint{4.374867in}{0.654164in}}%
\pgfpathcurveto{\pgfqpoint{4.373565in}{0.652862in}}{\pgfqpoint{4.372833in}{0.651095in}}{\pgfqpoint{4.372833in}{0.649254in}}%
\pgfpathcurveto{\pgfqpoint{4.372833in}{0.647412in}}{\pgfqpoint{4.373565in}{0.645646in}}{\pgfqpoint{4.374867in}{0.644343in}}%
\pgfpathcurveto{\pgfqpoint{4.376169in}{0.643041in}}{\pgfqpoint{4.377936in}{0.642309in}}{\pgfqpoint{4.379777in}{0.642309in}}%
\pgfpathlineto{\pgfqpoint{4.379777in}{0.642309in}}%
\pgfpathclose%
\pgfusepath{stroke,fill}%
\end{pgfscope}%
\begin{pgfscope}%
\pgfpathrectangle{\pgfqpoint{0.661006in}{0.524170in}}{\pgfqpoint{4.194036in}{1.071446in}}%
\pgfusepath{clip}%
\pgfsetbuttcap%
\pgfsetroundjoin%
\definecolor{currentfill}{rgb}{0.224615,0.181117,0.341100}%
\pgfsetfillcolor{currentfill}%
\pgfsetfillopacity{0.700000}%
\pgfsetlinewidth{1.003750pt}%
\definecolor{currentstroke}{rgb}{0.224615,0.181117,0.341100}%
\pgfsetstrokecolor{currentstroke}%
\pgfsetstrokeopacity{0.700000}%
\pgfsetdash{}{0pt}%
\pgfpathmoveto{\pgfqpoint{4.406734in}{0.637187in}}%
\pgfpathcurveto{\pgfqpoint{4.408576in}{0.637187in}}{\pgfqpoint{4.410342in}{0.637918in}}{\pgfqpoint{4.411645in}{0.639221in}}%
\pgfpathcurveto{\pgfqpoint{4.412947in}{0.640523in}}{\pgfqpoint{4.413679in}{0.642289in}}{\pgfqpoint{4.413679in}{0.644131in}}%
\pgfpathcurveto{\pgfqpoint{4.413679in}{0.645973in}}{\pgfqpoint{4.412947in}{0.647739in}}{\pgfqpoint{4.411645in}{0.649042in}}%
\pgfpathcurveto{\pgfqpoint{4.410342in}{0.650344in}}{\pgfqpoint{4.408576in}{0.651076in}}{\pgfqpoint{4.406734in}{0.651076in}}%
\pgfpathcurveto{\pgfqpoint{4.404892in}{0.651076in}}{\pgfqpoint{4.403126in}{0.650344in}}{\pgfqpoint{4.401824in}{0.649042in}}%
\pgfpathcurveto{\pgfqpoint{4.400521in}{0.647739in}}{\pgfqpoint{4.399790in}{0.645973in}}{\pgfqpoint{4.399790in}{0.644131in}}%
\pgfpathcurveto{\pgfqpoint{4.399790in}{0.642289in}}{\pgfqpoint{4.400521in}{0.640523in}}{\pgfqpoint{4.401824in}{0.639221in}}%
\pgfpathcurveto{\pgfqpoint{4.403126in}{0.637918in}}{\pgfqpoint{4.404892in}{0.637187in}}{\pgfqpoint{4.406734in}{0.637187in}}%
\pgfpathlineto{\pgfqpoint{4.406734in}{0.637187in}}%
\pgfpathclose%
\pgfusepath{stroke,fill}%
\end{pgfscope}%
\begin{pgfscope}%
\pgfpathrectangle{\pgfqpoint{0.661006in}{0.524170in}}{\pgfqpoint{4.194036in}{1.071446in}}%
\pgfusepath{clip}%
\pgfsetbuttcap%
\pgfsetroundjoin%
\definecolor{currentfill}{rgb}{0.224615,0.181117,0.341100}%
\pgfsetfillcolor{currentfill}%
\pgfsetfillopacity{0.700000}%
\pgfsetlinewidth{1.003750pt}%
\definecolor{currentstroke}{rgb}{0.224615,0.181117,0.341100}%
\pgfsetstrokecolor{currentstroke}%
\pgfsetstrokeopacity{0.700000}%
\pgfsetdash{}{0pt}%
\pgfpathmoveto{\pgfqpoint{4.400506in}{0.636737in}}%
\pgfpathcurveto{\pgfqpoint{4.402348in}{0.636737in}}{\pgfqpoint{4.404114in}{0.637469in}}{\pgfqpoint{4.405417in}{0.638771in}}%
\pgfpathcurveto{\pgfqpoint{4.406719in}{0.640073in}}{\pgfqpoint{4.407451in}{0.641840in}}{\pgfqpoint{4.407451in}{0.643681in}}%
\pgfpathcurveto{\pgfqpoint{4.407451in}{0.645523in}}{\pgfqpoint{4.406719in}{0.647290in}}{\pgfqpoint{4.405417in}{0.648592in}}%
\pgfpathcurveto{\pgfqpoint{4.404114in}{0.649894in}}{\pgfqpoint{4.402348in}{0.650626in}}{\pgfqpoint{4.400506in}{0.650626in}}%
\pgfpathcurveto{\pgfqpoint{4.398664in}{0.650626in}}{\pgfqpoint{4.396898in}{0.649894in}}{\pgfqpoint{4.395596in}{0.648592in}}%
\pgfpathcurveto{\pgfqpoint{4.394293in}{0.647290in}}{\pgfqpoint{4.393562in}{0.645523in}}{\pgfqpoint{4.393562in}{0.643681in}}%
\pgfpathcurveto{\pgfqpoint{4.393562in}{0.641840in}}{\pgfqpoint{4.394293in}{0.640073in}}{\pgfqpoint{4.395596in}{0.638771in}}%
\pgfpathcurveto{\pgfqpoint{4.396898in}{0.637469in}}{\pgfqpoint{4.398664in}{0.636737in}}{\pgfqpoint{4.400506in}{0.636737in}}%
\pgfpathlineto{\pgfqpoint{4.400506in}{0.636737in}}%
\pgfpathclose%
\pgfusepath{stroke,fill}%
\end{pgfscope}%
\begin{pgfscope}%
\pgfpathrectangle{\pgfqpoint{0.661006in}{0.524170in}}{\pgfqpoint{4.194036in}{1.071446in}}%
\pgfusepath{clip}%
\pgfsetbuttcap%
\pgfsetroundjoin%
\definecolor{currentfill}{rgb}{0.224615,0.181117,0.341100}%
\pgfsetfillcolor{currentfill}%
\pgfsetfillopacity{0.700000}%
\pgfsetlinewidth{1.003750pt}%
\definecolor{currentstroke}{rgb}{0.224615,0.181117,0.341100}%
\pgfsetstrokecolor{currentstroke}%
\pgfsetstrokeopacity{0.700000}%
\pgfsetdash{}{0pt}%
\pgfpathmoveto{\pgfqpoint{4.382984in}{0.638977in}}%
\pgfpathcurveto{\pgfqpoint{4.384826in}{0.638977in}}{\pgfqpoint{4.386592in}{0.639709in}}{\pgfqpoint{4.387895in}{0.641011in}}%
\pgfpathcurveto{\pgfqpoint{4.389197in}{0.642313in}}{\pgfqpoint{4.389929in}{0.644080in}}{\pgfqpoint{4.389929in}{0.645921in}}%
\pgfpathcurveto{\pgfqpoint{4.389929in}{0.647763in}}{\pgfqpoint{4.389197in}{0.649530in}}{\pgfqpoint{4.387895in}{0.650832in}}%
\pgfpathcurveto{\pgfqpoint{4.386592in}{0.652134in}}{\pgfqpoint{4.384826in}{0.652866in}}{\pgfqpoint{4.382984in}{0.652866in}}%
\pgfpathcurveto{\pgfqpoint{4.381143in}{0.652866in}}{\pgfqpoint{4.379376in}{0.652134in}}{\pgfqpoint{4.378074in}{0.650832in}}%
\pgfpathcurveto{\pgfqpoint{4.376771in}{0.649530in}}{\pgfqpoint{4.376040in}{0.647763in}}{\pgfqpoint{4.376040in}{0.645921in}}%
\pgfpathcurveto{\pgfqpoint{4.376040in}{0.644080in}}{\pgfqpoint{4.376771in}{0.642313in}}{\pgfqpoint{4.378074in}{0.641011in}}%
\pgfpathcurveto{\pgfqpoint{4.379376in}{0.639709in}}{\pgfqpoint{4.381143in}{0.638977in}}{\pgfqpoint{4.382984in}{0.638977in}}%
\pgfpathlineto{\pgfqpoint{4.382984in}{0.638977in}}%
\pgfpathclose%
\pgfusepath{stroke,fill}%
\end{pgfscope}%
\begin{pgfscope}%
\pgfpathrectangle{\pgfqpoint{0.661006in}{0.524170in}}{\pgfqpoint{4.194036in}{1.071446in}}%
\pgfusepath{clip}%
\pgfsetbuttcap%
\pgfsetroundjoin%
\definecolor{currentfill}{rgb}{0.221240,0.177276,0.335260}%
\pgfsetfillcolor{currentfill}%
\pgfsetfillopacity{0.700000}%
\pgfsetlinewidth{1.003750pt}%
\definecolor{currentstroke}{rgb}{0.221240,0.177276,0.335260}%
\pgfsetstrokecolor{currentstroke}%
\pgfsetstrokeopacity{0.700000}%
\pgfsetdash{}{0pt}%
\pgfpathmoveto{\pgfqpoint{4.391954in}{0.637219in}}%
\pgfpathcurveto{\pgfqpoint{4.393796in}{0.637219in}}{\pgfqpoint{4.395563in}{0.637951in}}{\pgfqpoint{4.396865in}{0.639253in}}%
\pgfpathcurveto{\pgfqpoint{4.398167in}{0.640556in}}{\pgfqpoint{4.398899in}{0.642322in}}{\pgfqpoint{4.398899in}{0.644164in}}%
\pgfpathcurveto{\pgfqpoint{4.398899in}{0.646006in}}{\pgfqpoint{4.398167in}{0.647772in}}{\pgfqpoint{4.396865in}{0.649074in}}%
\pgfpathcurveto{\pgfqpoint{4.395563in}{0.650377in}}{\pgfqpoint{4.393796in}{0.651108in}}{\pgfqpoint{4.391954in}{0.651108in}}%
\pgfpathcurveto{\pgfqpoint{4.390113in}{0.651108in}}{\pgfqpoint{4.388346in}{0.650377in}}{\pgfqpoint{4.387044in}{0.649074in}}%
\pgfpathcurveto{\pgfqpoint{4.385742in}{0.647772in}}{\pgfqpoint{4.385010in}{0.646006in}}{\pgfqpoint{4.385010in}{0.644164in}}%
\pgfpathcurveto{\pgfqpoint{4.385010in}{0.642322in}}{\pgfqpoint{4.385742in}{0.640556in}}{\pgfqpoint{4.387044in}{0.639253in}}%
\pgfpathcurveto{\pgfqpoint{4.388346in}{0.637951in}}{\pgfqpoint{4.390113in}{0.637219in}}{\pgfqpoint{4.391954in}{0.637219in}}%
\pgfpathlineto{\pgfqpoint{4.391954in}{0.637219in}}%
\pgfpathclose%
\pgfusepath{stroke,fill}%
\end{pgfscope}%
\begin{pgfscope}%
\pgfpathrectangle{\pgfqpoint{0.661006in}{0.524170in}}{\pgfqpoint{4.194036in}{1.071446in}}%
\pgfusepath{clip}%
\pgfsetbuttcap%
\pgfsetroundjoin%
\definecolor{currentfill}{rgb}{0.221240,0.177276,0.335260}%
\pgfsetfillcolor{currentfill}%
\pgfsetfillopacity{0.700000}%
\pgfsetlinewidth{1.003750pt}%
\definecolor{currentstroke}{rgb}{0.221240,0.177276,0.335260}%
\pgfsetstrokecolor{currentstroke}%
\pgfsetstrokeopacity{0.700000}%
\pgfsetdash{}{0pt}%
\pgfpathmoveto{\pgfqpoint{4.400460in}{0.634404in}}%
\pgfpathcurveto{\pgfqpoint{4.402301in}{0.634404in}}{\pgfqpoint{4.404068in}{0.635136in}}{\pgfqpoint{4.405370in}{0.636438in}}%
\pgfpathcurveto{\pgfqpoint{4.406672in}{0.637741in}}{\pgfqpoint{4.407404in}{0.639507in}}{\pgfqpoint{4.407404in}{0.641349in}}%
\pgfpathcurveto{\pgfqpoint{4.407404in}{0.643190in}}{\pgfqpoint{4.406672in}{0.644957in}}{\pgfqpoint{4.405370in}{0.646259in}}%
\pgfpathcurveto{\pgfqpoint{4.404068in}{0.647561in}}{\pgfqpoint{4.402301in}{0.648293in}}{\pgfqpoint{4.400460in}{0.648293in}}%
\pgfpathcurveto{\pgfqpoint{4.398618in}{0.648293in}}{\pgfqpoint{4.396851in}{0.647561in}}{\pgfqpoint{4.395549in}{0.646259in}}%
\pgfpathcurveto{\pgfqpoint{4.394247in}{0.644957in}}{\pgfqpoint{4.393515in}{0.643190in}}{\pgfqpoint{4.393515in}{0.641349in}}%
\pgfpathcurveto{\pgfqpoint{4.393515in}{0.639507in}}{\pgfqpoint{4.394247in}{0.637741in}}{\pgfqpoint{4.395549in}{0.636438in}}%
\pgfpathcurveto{\pgfqpoint{4.396851in}{0.635136in}}{\pgfqpoint{4.398618in}{0.634404in}}{\pgfqpoint{4.400460in}{0.634404in}}%
\pgfpathlineto{\pgfqpoint{4.400460in}{0.634404in}}%
\pgfpathclose%
\pgfusepath{stroke,fill}%
\end{pgfscope}%
\begin{pgfscope}%
\pgfpathrectangle{\pgfqpoint{0.661006in}{0.524170in}}{\pgfqpoint{4.194036in}{1.071446in}}%
\pgfusepath{clip}%
\pgfsetbuttcap%
\pgfsetroundjoin%
\definecolor{currentfill}{rgb}{0.221240,0.177276,0.335260}%
\pgfsetfillcolor{currentfill}%
\pgfsetfillopacity{0.700000}%
\pgfsetlinewidth{1.003750pt}%
\definecolor{currentstroke}{rgb}{0.221240,0.177276,0.335260}%
\pgfsetstrokecolor{currentstroke}%
\pgfsetstrokeopacity{0.700000}%
\pgfsetdash{}{0pt}%
\pgfpathmoveto{\pgfqpoint{4.413892in}{0.632407in}}%
\pgfpathcurveto{\pgfqpoint{4.415733in}{0.632407in}}{\pgfqpoint{4.417500in}{0.633139in}}{\pgfqpoint{4.418802in}{0.634441in}}%
\pgfpathcurveto{\pgfqpoint{4.420104in}{0.635743in}}{\pgfqpoint{4.420836in}{0.637510in}}{\pgfqpoint{4.420836in}{0.639352in}}%
\pgfpathcurveto{\pgfqpoint{4.420836in}{0.641193in}}{\pgfqpoint{4.420104in}{0.642960in}}{\pgfqpoint{4.418802in}{0.644262in}}%
\pgfpathcurveto{\pgfqpoint{4.417500in}{0.645564in}}{\pgfqpoint{4.415733in}{0.646296in}}{\pgfqpoint{4.413892in}{0.646296in}}%
\pgfpathcurveto{\pgfqpoint{4.412050in}{0.646296in}}{\pgfqpoint{4.410283in}{0.645564in}}{\pgfqpoint{4.408981in}{0.644262in}}%
\pgfpathcurveto{\pgfqpoint{4.407679in}{0.642960in}}{\pgfqpoint{4.406947in}{0.641193in}}{\pgfqpoint{4.406947in}{0.639352in}}%
\pgfpathcurveto{\pgfqpoint{4.406947in}{0.637510in}}{\pgfqpoint{4.407679in}{0.635743in}}{\pgfqpoint{4.408981in}{0.634441in}}%
\pgfpathcurveto{\pgfqpoint{4.410283in}{0.633139in}}{\pgfqpoint{4.412050in}{0.632407in}}{\pgfqpoint{4.413892in}{0.632407in}}%
\pgfpathlineto{\pgfqpoint{4.413892in}{0.632407in}}%
\pgfpathclose%
\pgfusepath{stroke,fill}%
\end{pgfscope}%
\begin{pgfscope}%
\pgfpathrectangle{\pgfqpoint{0.661006in}{0.524170in}}{\pgfqpoint{4.194036in}{1.071446in}}%
\pgfusepath{clip}%
\pgfsetbuttcap%
\pgfsetroundjoin%
\definecolor{currentfill}{rgb}{0.221240,0.177276,0.335260}%
\pgfsetfillcolor{currentfill}%
\pgfsetfillopacity{0.700000}%
\pgfsetlinewidth{1.003750pt}%
\definecolor{currentstroke}{rgb}{0.221240,0.177276,0.335260}%
\pgfsetstrokecolor{currentstroke}%
\pgfsetstrokeopacity{0.700000}%
\pgfsetdash{}{0pt}%
\pgfpathmoveto{\pgfqpoint{4.413055in}{0.632030in}}%
\pgfpathcurveto{\pgfqpoint{4.414897in}{0.632030in}}{\pgfqpoint{4.416663in}{0.632761in}}{\pgfqpoint{4.417965in}{0.634064in}}%
\pgfpathcurveto{\pgfqpoint{4.419268in}{0.635366in}}{\pgfqpoint{4.419999in}{0.637132in}}{\pgfqpoint{4.419999in}{0.638974in}}%
\pgfpathcurveto{\pgfqpoint{4.419999in}{0.640816in}}{\pgfqpoint{4.419268in}{0.642582in}}{\pgfqpoint{4.417965in}{0.643885in}}%
\pgfpathcurveto{\pgfqpoint{4.416663in}{0.645187in}}{\pgfqpoint{4.414897in}{0.645919in}}{\pgfqpoint{4.413055in}{0.645919in}}%
\pgfpathcurveto{\pgfqpoint{4.411213in}{0.645919in}}{\pgfqpoint{4.409447in}{0.645187in}}{\pgfqpoint{4.408145in}{0.643885in}}%
\pgfpathcurveto{\pgfqpoint{4.406842in}{0.642582in}}{\pgfqpoint{4.406111in}{0.640816in}}{\pgfqpoint{4.406111in}{0.638974in}}%
\pgfpathcurveto{\pgfqpoint{4.406111in}{0.637132in}}{\pgfqpoint{4.406842in}{0.635366in}}{\pgfqpoint{4.408145in}{0.634064in}}%
\pgfpathcurveto{\pgfqpoint{4.409447in}{0.632761in}}{\pgfqpoint{4.411213in}{0.632030in}}{\pgfqpoint{4.413055in}{0.632030in}}%
\pgfpathlineto{\pgfqpoint{4.413055in}{0.632030in}}%
\pgfpathclose%
\pgfusepath{stroke,fill}%
\end{pgfscope}%
\begin{pgfscope}%
\pgfpathrectangle{\pgfqpoint{0.661006in}{0.524170in}}{\pgfqpoint{4.194036in}{1.071446in}}%
\pgfusepath{clip}%
\pgfsetbuttcap%
\pgfsetroundjoin%
\definecolor{currentfill}{rgb}{0.217841,0.173450,0.329397}%
\pgfsetfillcolor{currentfill}%
\pgfsetfillopacity{0.700000}%
\pgfsetlinewidth{1.003750pt}%
\definecolor{currentstroke}{rgb}{0.217841,0.173450,0.329397}%
\pgfsetstrokecolor{currentstroke}%
\pgfsetstrokeopacity{0.700000}%
\pgfsetdash{}{0pt}%
\pgfpathmoveto{\pgfqpoint{4.413872in}{0.629918in}}%
\pgfpathcurveto{\pgfqpoint{4.415713in}{0.629918in}}{\pgfqpoint{4.417480in}{0.630650in}}{\pgfqpoint{4.418782in}{0.631952in}}%
\pgfpathcurveto{\pgfqpoint{4.420084in}{0.633255in}}{\pgfqpoint{4.420816in}{0.635021in}}{\pgfqpoint{4.420816in}{0.636863in}}%
\pgfpathcurveto{\pgfqpoint{4.420816in}{0.638704in}}{\pgfqpoint{4.420084in}{0.640471in}}{\pgfqpoint{4.418782in}{0.641773in}}%
\pgfpathcurveto{\pgfqpoint{4.417480in}{0.643076in}}{\pgfqpoint{4.415713in}{0.643807in}}{\pgfqpoint{4.413872in}{0.643807in}}%
\pgfpathcurveto{\pgfqpoint{4.412030in}{0.643807in}}{\pgfqpoint{4.410263in}{0.643076in}}{\pgfqpoint{4.408961in}{0.641773in}}%
\pgfpathcurveto{\pgfqpoint{4.407659in}{0.640471in}}{\pgfqpoint{4.406927in}{0.638704in}}{\pgfqpoint{4.406927in}{0.636863in}}%
\pgfpathcurveto{\pgfqpoint{4.406927in}{0.635021in}}{\pgfqpoint{4.407659in}{0.633255in}}{\pgfqpoint{4.408961in}{0.631952in}}%
\pgfpathcurveto{\pgfqpoint{4.410263in}{0.630650in}}{\pgfqpoint{4.412030in}{0.629918in}}{\pgfqpoint{4.413872in}{0.629918in}}%
\pgfpathlineto{\pgfqpoint{4.413872in}{0.629918in}}%
\pgfpathclose%
\pgfusepath{stroke,fill}%
\end{pgfscope}%
\begin{pgfscope}%
\pgfpathrectangle{\pgfqpoint{0.661006in}{0.524170in}}{\pgfqpoint{4.194036in}{1.071446in}}%
\pgfusepath{clip}%
\pgfsetbuttcap%
\pgfsetroundjoin%
\definecolor{currentfill}{rgb}{0.217841,0.173450,0.329397}%
\pgfsetfillcolor{currentfill}%
\pgfsetfillopacity{0.700000}%
\pgfsetlinewidth{1.003750pt}%
\definecolor{currentstroke}{rgb}{0.217841,0.173450,0.329397}%
\pgfsetstrokecolor{currentstroke}%
\pgfsetstrokeopacity{0.700000}%
\pgfsetdash{}{0pt}%
\pgfpathmoveto{\pgfqpoint{4.438106in}{0.625057in}}%
\pgfpathcurveto{\pgfqpoint{4.439948in}{0.625057in}}{\pgfqpoint{4.441714in}{0.625788in}}{\pgfqpoint{4.443017in}{0.627091in}}%
\pgfpathcurveto{\pgfqpoint{4.444319in}{0.628393in}}{\pgfqpoint{4.445051in}{0.630159in}}{\pgfqpoint{4.445051in}{0.632001in}}%
\pgfpathcurveto{\pgfqpoint{4.445051in}{0.633843in}}{\pgfqpoint{4.444319in}{0.635609in}}{\pgfqpoint{4.443017in}{0.636911in}}%
\pgfpathcurveto{\pgfqpoint{4.441714in}{0.638214in}}{\pgfqpoint{4.439948in}{0.638945in}}{\pgfqpoint{4.438106in}{0.638945in}}%
\pgfpathcurveto{\pgfqpoint{4.436265in}{0.638945in}}{\pgfqpoint{4.434498in}{0.638214in}}{\pgfqpoint{4.433196in}{0.636911in}}%
\pgfpathcurveto{\pgfqpoint{4.431893in}{0.635609in}}{\pgfqpoint{4.431162in}{0.633843in}}{\pgfqpoint{4.431162in}{0.632001in}}%
\pgfpathcurveto{\pgfqpoint{4.431162in}{0.630159in}}{\pgfqpoint{4.431893in}{0.628393in}}{\pgfqpoint{4.433196in}{0.627091in}}%
\pgfpathcurveto{\pgfqpoint{4.434498in}{0.625788in}}{\pgfqpoint{4.436265in}{0.625057in}}{\pgfqpoint{4.438106in}{0.625057in}}%
\pgfpathlineto{\pgfqpoint{4.438106in}{0.625057in}}%
\pgfpathclose%
\pgfusepath{stroke,fill}%
\end{pgfscope}%
\begin{pgfscope}%
\pgfpathrectangle{\pgfqpoint{0.661006in}{0.524170in}}{\pgfqpoint{4.194036in}{1.071446in}}%
\pgfusepath{clip}%
\pgfsetbuttcap%
\pgfsetroundjoin%
\definecolor{currentfill}{rgb}{0.217841,0.173450,0.329397}%
\pgfsetfillcolor{currentfill}%
\pgfsetfillopacity{0.700000}%
\pgfsetlinewidth{1.003750pt}%
\definecolor{currentstroke}{rgb}{0.217841,0.173450,0.329397}%
\pgfsetstrokecolor{currentstroke}%
\pgfsetstrokeopacity{0.700000}%
\pgfsetdash{}{0pt}%
\pgfpathmoveto{\pgfqpoint{4.436944in}{0.623928in}}%
\pgfpathcurveto{\pgfqpoint{4.438786in}{0.623928in}}{\pgfqpoint{4.440552in}{0.624660in}}{\pgfqpoint{4.441855in}{0.625962in}}%
\pgfpathcurveto{\pgfqpoint{4.443157in}{0.627264in}}{\pgfqpoint{4.443889in}{0.629031in}}{\pgfqpoint{4.443889in}{0.630873in}}%
\pgfpathcurveto{\pgfqpoint{4.443889in}{0.632714in}}{\pgfqpoint{4.443157in}{0.634481in}}{\pgfqpoint{4.441855in}{0.635783in}}%
\pgfpathcurveto{\pgfqpoint{4.440552in}{0.637085in}}{\pgfqpoint{4.438786in}{0.637817in}}{\pgfqpoint{4.436944in}{0.637817in}}%
\pgfpathcurveto{\pgfqpoint{4.435103in}{0.637817in}}{\pgfqpoint{4.433336in}{0.637085in}}{\pgfqpoint{4.432034in}{0.635783in}}%
\pgfpathcurveto{\pgfqpoint{4.430732in}{0.634481in}}{\pgfqpoint{4.430000in}{0.632714in}}{\pgfqpoint{4.430000in}{0.630873in}}%
\pgfpathcurveto{\pgfqpoint{4.430000in}{0.629031in}}{\pgfqpoint{4.430732in}{0.627264in}}{\pgfqpoint{4.432034in}{0.625962in}}%
\pgfpathcurveto{\pgfqpoint{4.433336in}{0.624660in}}{\pgfqpoint{4.435103in}{0.623928in}}{\pgfqpoint{4.436944in}{0.623928in}}%
\pgfpathlineto{\pgfqpoint{4.436944in}{0.623928in}}%
\pgfpathclose%
\pgfusepath{stroke,fill}%
\end{pgfscope}%
\begin{pgfscope}%
\pgfpathrectangle{\pgfqpoint{0.661006in}{0.524170in}}{\pgfqpoint{4.194036in}{1.071446in}}%
\pgfusepath{clip}%
\pgfsetbuttcap%
\pgfsetroundjoin%
\definecolor{currentfill}{rgb}{0.214419,0.169641,0.323509}%
\pgfsetfillcolor{currentfill}%
\pgfsetfillopacity{0.700000}%
\pgfsetlinewidth{1.003750pt}%
\definecolor{currentstroke}{rgb}{0.214419,0.169641,0.323509}%
\pgfsetstrokecolor{currentstroke}%
\pgfsetstrokeopacity{0.700000}%
\pgfsetdash{}{0pt}%
\pgfpathmoveto{\pgfqpoint{4.450237in}{0.621564in}}%
\pgfpathcurveto{\pgfqpoint{4.452078in}{0.621564in}}{\pgfqpoint{4.453845in}{0.622296in}}{\pgfqpoint{4.455147in}{0.623598in}}%
\pgfpathcurveto{\pgfqpoint{4.456449in}{0.624900in}}{\pgfqpoint{4.457181in}{0.626667in}}{\pgfqpoint{4.457181in}{0.628508in}}%
\pgfpathcurveto{\pgfqpoint{4.457181in}{0.630350in}}{\pgfqpoint{4.456449in}{0.632116in}}{\pgfqpoint{4.455147in}{0.633419in}}%
\pgfpathcurveto{\pgfqpoint{4.453845in}{0.634721in}}{\pgfqpoint{4.452078in}{0.635453in}}{\pgfqpoint{4.450237in}{0.635453in}}%
\pgfpathcurveto{\pgfqpoint{4.448395in}{0.635453in}}{\pgfqpoint{4.446629in}{0.634721in}}{\pgfqpoint{4.445326in}{0.633419in}}%
\pgfpathcurveto{\pgfqpoint{4.444024in}{0.632116in}}{\pgfqpoint{4.443292in}{0.630350in}}{\pgfqpoint{4.443292in}{0.628508in}}%
\pgfpathcurveto{\pgfqpoint{4.443292in}{0.626667in}}{\pgfqpoint{4.444024in}{0.624900in}}{\pgfqpoint{4.445326in}{0.623598in}}%
\pgfpathcurveto{\pgfqpoint{4.446629in}{0.622296in}}{\pgfqpoint{4.448395in}{0.621564in}}{\pgfqpoint{4.450237in}{0.621564in}}%
\pgfpathlineto{\pgfqpoint{4.450237in}{0.621564in}}%
\pgfpathclose%
\pgfusepath{stroke,fill}%
\end{pgfscope}%
\begin{pgfscope}%
\pgfpathrectangle{\pgfqpoint{0.661006in}{0.524170in}}{\pgfqpoint{4.194036in}{1.071446in}}%
\pgfusepath{clip}%
\pgfsetbuttcap%
\pgfsetroundjoin%
\definecolor{currentfill}{rgb}{0.214419,0.169641,0.323509}%
\pgfsetfillcolor{currentfill}%
\pgfsetfillopacity{0.700000}%
\pgfsetlinewidth{1.003750pt}%
\definecolor{currentstroke}{rgb}{0.214419,0.169641,0.323509}%
\pgfsetstrokecolor{currentstroke}%
\pgfsetstrokeopacity{0.700000}%
\pgfsetdash{}{0pt}%
\pgfpathmoveto{\pgfqpoint{4.459021in}{0.619446in}}%
\pgfpathcurveto{\pgfqpoint{4.460863in}{0.619446in}}{\pgfqpoint{4.462629in}{0.620178in}}{\pgfqpoint{4.463931in}{0.621480in}}%
\pgfpathcurveto{\pgfqpoint{4.465234in}{0.622783in}}{\pgfqpoint{4.465965in}{0.624549in}}{\pgfqpoint{4.465965in}{0.626391in}}%
\pgfpathcurveto{\pgfqpoint{4.465965in}{0.628232in}}{\pgfqpoint{4.465234in}{0.629999in}}{\pgfqpoint{4.463931in}{0.631301in}}%
\pgfpathcurveto{\pgfqpoint{4.462629in}{0.632603in}}{\pgfqpoint{4.460863in}{0.633335in}}{\pgfqpoint{4.459021in}{0.633335in}}%
\pgfpathcurveto{\pgfqpoint{4.457179in}{0.633335in}}{\pgfqpoint{4.455413in}{0.632603in}}{\pgfqpoint{4.454110in}{0.631301in}}%
\pgfpathcurveto{\pgfqpoint{4.452808in}{0.629999in}}{\pgfqpoint{4.452077in}{0.628232in}}{\pgfqpoint{4.452077in}{0.626391in}}%
\pgfpathcurveto{\pgfqpoint{4.452077in}{0.624549in}}{\pgfqpoint{4.452808in}{0.622783in}}{\pgfqpoint{4.454110in}{0.621480in}}%
\pgfpathcurveto{\pgfqpoint{4.455413in}{0.620178in}}{\pgfqpoint{4.457179in}{0.619446in}}{\pgfqpoint{4.459021in}{0.619446in}}%
\pgfpathlineto{\pgfqpoint{4.459021in}{0.619446in}}%
\pgfpathclose%
\pgfusepath{stroke,fill}%
\end{pgfscope}%
\begin{pgfscope}%
\pgfpathrectangle{\pgfqpoint{0.661006in}{0.524170in}}{\pgfqpoint{4.194036in}{1.071446in}}%
\pgfusepath{clip}%
\pgfsetbuttcap%
\pgfsetroundjoin%
\definecolor{currentfill}{rgb}{0.214419,0.169641,0.323509}%
\pgfsetfillcolor{currentfill}%
\pgfsetfillopacity{0.700000}%
\pgfsetlinewidth{1.003750pt}%
\definecolor{currentstroke}{rgb}{0.214419,0.169641,0.323509}%
\pgfsetstrokecolor{currentstroke}%
\pgfsetstrokeopacity{0.700000}%
\pgfsetdash{}{0pt}%
\pgfpathmoveto{\pgfqpoint{4.466504in}{0.615446in}}%
\pgfpathcurveto{\pgfqpoint{4.468345in}{0.615446in}}{\pgfqpoint{4.470112in}{0.616178in}}{\pgfqpoint{4.471414in}{0.617480in}}%
\pgfpathcurveto{\pgfqpoint{4.472717in}{0.618782in}}{\pgfqpoint{4.473448in}{0.620549in}}{\pgfqpoint{4.473448in}{0.622390in}}%
\pgfpathcurveto{\pgfqpoint{4.473448in}{0.624232in}}{\pgfqpoint{4.472717in}{0.625998in}}{\pgfqpoint{4.471414in}{0.627301in}}%
\pgfpathcurveto{\pgfqpoint{4.470112in}{0.628603in}}{\pgfqpoint{4.468345in}{0.629335in}}{\pgfqpoint{4.466504in}{0.629335in}}%
\pgfpathcurveto{\pgfqpoint{4.464662in}{0.629335in}}{\pgfqpoint{4.462896in}{0.628603in}}{\pgfqpoint{4.461593in}{0.627301in}}%
\pgfpathcurveto{\pgfqpoint{4.460291in}{0.625998in}}{\pgfqpoint{4.459559in}{0.624232in}}{\pgfqpoint{4.459559in}{0.622390in}}%
\pgfpathcurveto{\pgfqpoint{4.459559in}{0.620549in}}{\pgfqpoint{4.460291in}{0.618782in}}{\pgfqpoint{4.461593in}{0.617480in}}%
\pgfpathcurveto{\pgfqpoint{4.462896in}{0.616178in}}{\pgfqpoint{4.464662in}{0.615446in}}{\pgfqpoint{4.466504in}{0.615446in}}%
\pgfpathlineto{\pgfqpoint{4.466504in}{0.615446in}}%
\pgfpathclose%
\pgfusepath{stroke,fill}%
\end{pgfscope}%
\begin{pgfscope}%
\pgfpathrectangle{\pgfqpoint{0.661006in}{0.524170in}}{\pgfqpoint{4.194036in}{1.071446in}}%
\pgfusepath{clip}%
\pgfsetbuttcap%
\pgfsetroundjoin%
\definecolor{currentfill}{rgb}{0.214419,0.169641,0.323509}%
\pgfsetfillcolor{currentfill}%
\pgfsetfillopacity{0.700000}%
\pgfsetlinewidth{1.003750pt}%
\definecolor{currentstroke}{rgb}{0.214419,0.169641,0.323509}%
\pgfsetstrokecolor{currentstroke}%
\pgfsetstrokeopacity{0.700000}%
\pgfsetdash{}{0pt}%
\pgfpathmoveto{\pgfqpoint{4.482492in}{0.613692in}}%
\pgfpathcurveto{\pgfqpoint{4.484334in}{0.613692in}}{\pgfqpoint{4.486100in}{0.614423in}}{\pgfqpoint{4.487402in}{0.615726in}}%
\pgfpathcurveto{\pgfqpoint{4.488705in}{0.617028in}}{\pgfqpoint{4.489436in}{0.618794in}}{\pgfqpoint{4.489436in}{0.620636in}}%
\pgfpathcurveto{\pgfqpoint{4.489436in}{0.622478in}}{\pgfqpoint{4.488705in}{0.624244in}}{\pgfqpoint{4.487402in}{0.625547in}}%
\pgfpathcurveto{\pgfqpoint{4.486100in}{0.626849in}}{\pgfqpoint{4.484334in}{0.627581in}}{\pgfqpoint{4.482492in}{0.627581in}}%
\pgfpathcurveto{\pgfqpoint{4.480650in}{0.627581in}}{\pgfqpoint{4.478884in}{0.626849in}}{\pgfqpoint{4.477582in}{0.625547in}}%
\pgfpathcurveto{\pgfqpoint{4.476279in}{0.624244in}}{\pgfqpoint{4.475548in}{0.622478in}}{\pgfqpoint{4.475548in}{0.620636in}}%
\pgfpathcurveto{\pgfqpoint{4.475548in}{0.618794in}}{\pgfqpoint{4.476279in}{0.617028in}}{\pgfqpoint{4.477582in}{0.615726in}}%
\pgfpathcurveto{\pgfqpoint{4.478884in}{0.614423in}}{\pgfqpoint{4.480650in}{0.613692in}}{\pgfqpoint{4.482492in}{0.613692in}}%
\pgfpathlineto{\pgfqpoint{4.482492in}{0.613692in}}%
\pgfpathclose%
\pgfusepath{stroke,fill}%
\end{pgfscope}%
\begin{pgfscope}%
\pgfpathrectangle{\pgfqpoint{0.661006in}{0.524170in}}{\pgfqpoint{4.194036in}{1.071446in}}%
\pgfusepath{clip}%
\pgfsetbuttcap%
\pgfsetroundjoin%
\definecolor{currentfill}{rgb}{0.210973,0.165849,0.317598}%
\pgfsetfillcolor{currentfill}%
\pgfsetfillopacity{0.700000}%
\pgfsetlinewidth{1.003750pt}%
\definecolor{currentstroke}{rgb}{0.210973,0.165849,0.317598}%
\pgfsetstrokecolor{currentstroke}%
\pgfsetstrokeopacity{0.700000}%
\pgfsetdash{}{0pt}%
\pgfpathmoveto{\pgfqpoint{4.485002in}{0.610650in}}%
\pgfpathcurveto{\pgfqpoint{4.486843in}{0.610650in}}{\pgfqpoint{4.488610in}{0.611382in}}{\pgfqpoint{4.489912in}{0.612684in}}%
\pgfpathcurveto{\pgfqpoint{4.491214in}{0.613986in}}{\pgfqpoint{4.491946in}{0.615753in}}{\pgfqpoint{4.491946in}{0.617594in}}%
\pgfpathcurveto{\pgfqpoint{4.491946in}{0.619436in}}{\pgfqpoint{4.491214in}{0.621203in}}{\pgfqpoint{4.489912in}{0.622505in}}%
\pgfpathcurveto{\pgfqpoint{4.488610in}{0.623807in}}{\pgfqpoint{4.486843in}{0.624539in}}{\pgfqpoint{4.485002in}{0.624539in}}%
\pgfpathcurveto{\pgfqpoint{4.483160in}{0.624539in}}{\pgfqpoint{4.481394in}{0.623807in}}{\pgfqpoint{4.480091in}{0.622505in}}%
\pgfpathcurveto{\pgfqpoint{4.478789in}{0.621203in}}{\pgfqpoint{4.478057in}{0.619436in}}{\pgfqpoint{4.478057in}{0.617594in}}%
\pgfpathcurveto{\pgfqpoint{4.478057in}{0.615753in}}{\pgfqpoint{4.478789in}{0.613986in}}{\pgfqpoint{4.480091in}{0.612684in}}%
\pgfpathcurveto{\pgfqpoint{4.481394in}{0.611382in}}{\pgfqpoint{4.483160in}{0.610650in}}{\pgfqpoint{4.485002in}{0.610650in}}%
\pgfpathlineto{\pgfqpoint{4.485002in}{0.610650in}}%
\pgfpathclose%
\pgfusepath{stroke,fill}%
\end{pgfscope}%
\begin{pgfscope}%
\pgfpathrectangle{\pgfqpoint{0.661006in}{0.524170in}}{\pgfqpoint{4.194036in}{1.071446in}}%
\pgfusepath{clip}%
\pgfsetbuttcap%
\pgfsetroundjoin%
\definecolor{currentfill}{rgb}{0.210973,0.165849,0.317598}%
\pgfsetfillcolor{currentfill}%
\pgfsetfillopacity{0.700000}%
\pgfsetlinewidth{1.003750pt}%
\definecolor{currentstroke}{rgb}{0.210973,0.165849,0.317598}%
\pgfsetstrokecolor{currentstroke}%
\pgfsetstrokeopacity{0.700000}%
\pgfsetdash{}{0pt}%
\pgfpathmoveto{\pgfqpoint{4.510332in}{0.605401in}}%
\pgfpathcurveto{\pgfqpoint{4.512174in}{0.605401in}}{\pgfqpoint{4.513940in}{0.606133in}}{\pgfqpoint{4.515242in}{0.607435in}}%
\pgfpathcurveto{\pgfqpoint{4.516545in}{0.608738in}}{\pgfqpoint{4.517276in}{0.610504in}}{\pgfqpoint{4.517276in}{0.612346in}}%
\pgfpathcurveto{\pgfqpoint{4.517276in}{0.614187in}}{\pgfqpoint{4.516545in}{0.615954in}}{\pgfqpoint{4.515242in}{0.617256in}}%
\pgfpathcurveto{\pgfqpoint{4.513940in}{0.618558in}}{\pgfqpoint{4.512174in}{0.619290in}}{\pgfqpoint{4.510332in}{0.619290in}}%
\pgfpathcurveto{\pgfqpoint{4.508490in}{0.619290in}}{\pgfqpoint{4.506724in}{0.618558in}}{\pgfqpoint{4.505421in}{0.617256in}}%
\pgfpathcurveto{\pgfqpoint{4.504119in}{0.615954in}}{\pgfqpoint{4.503387in}{0.614187in}}{\pgfqpoint{4.503387in}{0.612346in}}%
\pgfpathcurveto{\pgfqpoint{4.503387in}{0.610504in}}{\pgfqpoint{4.504119in}{0.608738in}}{\pgfqpoint{4.505421in}{0.607435in}}%
\pgfpathcurveto{\pgfqpoint{4.506724in}{0.606133in}}{\pgfqpoint{4.508490in}{0.605401in}}{\pgfqpoint{4.510332in}{0.605401in}}%
\pgfpathlineto{\pgfqpoint{4.510332in}{0.605401in}}%
\pgfpathclose%
\pgfusepath{stroke,fill}%
\end{pgfscope}%
\begin{pgfscope}%
\pgfpathrectangle{\pgfqpoint{0.661006in}{0.524170in}}{\pgfqpoint{4.194036in}{1.071446in}}%
\pgfusepath{clip}%
\pgfsetbuttcap%
\pgfsetroundjoin%
\definecolor{currentfill}{rgb}{0.210973,0.165849,0.317598}%
\pgfsetfillcolor{currentfill}%
\pgfsetfillopacity{0.700000}%
\pgfsetlinewidth{1.003750pt}%
\definecolor{currentstroke}{rgb}{0.210973,0.165849,0.317598}%
\pgfsetstrokecolor{currentstroke}%
\pgfsetstrokeopacity{0.700000}%
\pgfsetdash{}{0pt}%
\pgfpathmoveto{\pgfqpoint{4.542076in}{0.598961in}}%
\pgfpathcurveto{\pgfqpoint{4.543917in}{0.598961in}}{\pgfqpoint{4.545684in}{0.599693in}}{\pgfqpoint{4.546986in}{0.600995in}}%
\pgfpathcurveto{\pgfqpoint{4.548289in}{0.602297in}}{\pgfqpoint{4.549020in}{0.604064in}}{\pgfqpoint{4.549020in}{0.605905in}}%
\pgfpathcurveto{\pgfqpoint{4.549020in}{0.607747in}}{\pgfqpoint{4.548289in}{0.609513in}}{\pgfqpoint{4.546986in}{0.610816in}}%
\pgfpathcurveto{\pgfqpoint{4.545684in}{0.612118in}}{\pgfqpoint{4.543917in}{0.612850in}}{\pgfqpoint{4.542076in}{0.612850in}}%
\pgfpathcurveto{\pgfqpoint{4.540234in}{0.612850in}}{\pgfqpoint{4.538468in}{0.612118in}}{\pgfqpoint{4.537165in}{0.610816in}}%
\pgfpathcurveto{\pgfqpoint{4.535863in}{0.609513in}}{\pgfqpoint{4.535131in}{0.607747in}}{\pgfqpoint{4.535131in}{0.605905in}}%
\pgfpathcurveto{\pgfqpoint{4.535131in}{0.604064in}}{\pgfqpoint{4.535863in}{0.602297in}}{\pgfqpoint{4.537165in}{0.600995in}}%
\pgfpathcurveto{\pgfqpoint{4.538468in}{0.599693in}}{\pgfqpoint{4.540234in}{0.598961in}}{\pgfqpoint{4.542076in}{0.598961in}}%
\pgfpathlineto{\pgfqpoint{4.542076in}{0.598961in}}%
\pgfpathclose%
\pgfusepath{stroke,fill}%
\end{pgfscope}%
\begin{pgfscope}%
\pgfpathrectangle{\pgfqpoint{0.661006in}{0.524170in}}{\pgfqpoint{4.194036in}{1.071446in}}%
\pgfusepath{clip}%
\pgfsetbuttcap%
\pgfsetroundjoin%
\definecolor{currentfill}{rgb}{0.207503,0.162073,0.311663}%
\pgfsetfillcolor{currentfill}%
\pgfsetfillopacity{0.700000}%
\pgfsetlinewidth{1.003750pt}%
\definecolor{currentstroke}{rgb}{0.207503,0.162073,0.311663}%
\pgfsetstrokecolor{currentstroke}%
\pgfsetstrokeopacity{0.700000}%
\pgfsetdash{}{0pt}%
\pgfpathmoveto{\pgfqpoint{4.564943in}{0.593708in}}%
\pgfpathcurveto{\pgfqpoint{4.566784in}{0.593708in}}{\pgfqpoint{4.568551in}{0.594439in}}{\pgfqpoint{4.569853in}{0.595742in}}%
\pgfpathcurveto{\pgfqpoint{4.571155in}{0.597044in}}{\pgfqpoint{4.571887in}{0.598811in}}{\pgfqpoint{4.571887in}{0.600652in}}%
\pgfpathcurveto{\pgfqpoint{4.571887in}{0.602494in}}{\pgfqpoint{4.571155in}{0.604260in}}{\pgfqpoint{4.569853in}{0.605563in}}%
\pgfpathcurveto{\pgfqpoint{4.568551in}{0.606865in}}{\pgfqpoint{4.566784in}{0.607597in}}{\pgfqpoint{4.564943in}{0.607597in}}%
\pgfpathcurveto{\pgfqpoint{4.563101in}{0.607597in}}{\pgfqpoint{4.561334in}{0.606865in}}{\pgfqpoint{4.560032in}{0.605563in}}%
\pgfpathcurveto{\pgfqpoint{4.558730in}{0.604260in}}{\pgfqpoint{4.557998in}{0.602494in}}{\pgfqpoint{4.557998in}{0.600652in}}%
\pgfpathcurveto{\pgfqpoint{4.557998in}{0.598811in}}{\pgfqpoint{4.558730in}{0.597044in}}{\pgfqpoint{4.560032in}{0.595742in}}%
\pgfpathcurveto{\pgfqpoint{4.561334in}{0.594439in}}{\pgfqpoint{4.563101in}{0.593708in}}{\pgfqpoint{4.564943in}{0.593708in}}%
\pgfpathlineto{\pgfqpoint{4.564943in}{0.593708in}}%
\pgfpathclose%
\pgfusepath{stroke,fill}%
\end{pgfscope}%
\begin{pgfscope}%
\pgfpathrectangle{\pgfqpoint{0.661006in}{0.524170in}}{\pgfqpoint{4.194036in}{1.071446in}}%
\pgfusepath{clip}%
\pgfsetbuttcap%
\pgfsetroundjoin%
\definecolor{currentfill}{rgb}{0.207503,0.162073,0.311663}%
\pgfsetfillcolor{currentfill}%
\pgfsetfillopacity{0.700000}%
\pgfsetlinewidth{1.003750pt}%
\definecolor{currentstroke}{rgb}{0.207503,0.162073,0.311663}%
\pgfsetstrokecolor{currentstroke}%
\pgfsetstrokeopacity{0.700000}%
\pgfsetdash{}{0pt}%
\pgfpathmoveto{\pgfqpoint{4.590180in}{0.587349in}}%
\pgfpathcurveto{\pgfqpoint{4.592021in}{0.587349in}}{\pgfqpoint{4.593788in}{0.588081in}}{\pgfqpoint{4.595090in}{0.589383in}}%
\pgfpathcurveto{\pgfqpoint{4.596392in}{0.590685in}}{\pgfqpoint{4.597124in}{0.592452in}}{\pgfqpoint{4.597124in}{0.594293in}}%
\pgfpathcurveto{\pgfqpoint{4.597124in}{0.596135in}}{\pgfqpoint{4.596392in}{0.597902in}}{\pgfqpoint{4.595090in}{0.599204in}}%
\pgfpathcurveto{\pgfqpoint{4.593788in}{0.600506in}}{\pgfqpoint{4.592021in}{0.601238in}}{\pgfqpoint{4.590180in}{0.601238in}}%
\pgfpathcurveto{\pgfqpoint{4.588338in}{0.601238in}}{\pgfqpoint{4.586572in}{0.600506in}}{\pgfqpoint{4.585269in}{0.599204in}}%
\pgfpathcurveto{\pgfqpoint{4.583967in}{0.597902in}}{\pgfqpoint{4.583235in}{0.596135in}}{\pgfqpoint{4.583235in}{0.594293in}}%
\pgfpathcurveto{\pgfqpoint{4.583235in}{0.592452in}}{\pgfqpoint{4.583967in}{0.590685in}}{\pgfqpoint{4.585269in}{0.589383in}}%
\pgfpathcurveto{\pgfqpoint{4.586572in}{0.588081in}}{\pgfqpoint{4.588338in}{0.587349in}}{\pgfqpoint{4.590180in}{0.587349in}}%
\pgfpathlineto{\pgfqpoint{4.590180in}{0.587349in}}%
\pgfpathclose%
\pgfusepath{stroke,fill}%
\end{pgfscope}%
\begin{pgfscope}%
\pgfpathrectangle{\pgfqpoint{0.661006in}{0.524170in}}{\pgfqpoint{4.194036in}{1.071446in}}%
\pgfusepath{clip}%
\pgfsetbuttcap%
\pgfsetroundjoin%
\definecolor{currentfill}{rgb}{0.204008,0.158314,0.305705}%
\pgfsetfillcolor{currentfill}%
\pgfsetfillopacity{0.700000}%
\pgfsetlinewidth{1.003750pt}%
\definecolor{currentstroke}{rgb}{0.204008,0.158314,0.305705}%
\pgfsetstrokecolor{currentstroke}%
\pgfsetstrokeopacity{0.700000}%
\pgfsetdash{}{0pt}%
\pgfpathmoveto{\pgfqpoint{4.596919in}{0.584600in}}%
\pgfpathcurveto{\pgfqpoint{4.598761in}{0.584600in}}{\pgfqpoint{4.600527in}{0.585332in}}{\pgfqpoint{4.601829in}{0.586634in}}%
\pgfpathcurveto{\pgfqpoint{4.603132in}{0.587937in}}{\pgfqpoint{4.603863in}{0.589703in}}{\pgfqpoint{4.603863in}{0.591545in}}%
\pgfpathcurveto{\pgfqpoint{4.603863in}{0.593386in}}{\pgfqpoint{4.603132in}{0.595153in}}{\pgfqpoint{4.601829in}{0.596455in}}%
\pgfpathcurveto{\pgfqpoint{4.600527in}{0.597758in}}{\pgfqpoint{4.598761in}{0.598489in}}{\pgfqpoint{4.596919in}{0.598489in}}%
\pgfpathcurveto{\pgfqpoint{4.595077in}{0.598489in}}{\pgfqpoint{4.593311in}{0.597758in}}{\pgfqpoint{4.592008in}{0.596455in}}%
\pgfpathcurveto{\pgfqpoint{4.590706in}{0.595153in}}{\pgfqpoint{4.589974in}{0.593386in}}{\pgfqpoint{4.589974in}{0.591545in}}%
\pgfpathcurveto{\pgfqpoint{4.589974in}{0.589703in}}{\pgfqpoint{4.590706in}{0.587937in}}{\pgfqpoint{4.592008in}{0.586634in}}%
\pgfpathcurveto{\pgfqpoint{4.593311in}{0.585332in}}{\pgfqpoint{4.595077in}{0.584600in}}{\pgfqpoint{4.596919in}{0.584600in}}%
\pgfpathlineto{\pgfqpoint{4.596919in}{0.584600in}}%
\pgfpathclose%
\pgfusepath{stroke,fill}%
\end{pgfscope}%
\begin{pgfscope}%
\pgfpathrectangle{\pgfqpoint{0.661006in}{0.524170in}}{\pgfqpoint{4.194036in}{1.071446in}}%
\pgfusepath{clip}%
\pgfsetbuttcap%
\pgfsetroundjoin%
\definecolor{currentfill}{rgb}{0.204008,0.158314,0.305705}%
\pgfsetfillcolor{currentfill}%
\pgfsetfillopacity{0.700000}%
\pgfsetlinewidth{1.003750pt}%
\definecolor{currentstroke}{rgb}{0.204008,0.158314,0.305705}%
\pgfsetstrokecolor{currentstroke}%
\pgfsetstrokeopacity{0.700000}%
\pgfsetdash{}{0pt}%
\pgfpathmoveto{\pgfqpoint{4.591464in}{0.585920in}}%
\pgfpathcurveto{\pgfqpoint{4.593306in}{0.585920in}}{\pgfqpoint{4.595073in}{0.586652in}}{\pgfqpoint{4.596375in}{0.587954in}}%
\pgfpathcurveto{\pgfqpoint{4.597677in}{0.589256in}}{\pgfqpoint{4.598409in}{0.591023in}}{\pgfqpoint{4.598409in}{0.592864in}}%
\pgfpathcurveto{\pgfqpoint{4.598409in}{0.594706in}}{\pgfqpoint{4.597677in}{0.596473in}}{\pgfqpoint{4.596375in}{0.597775in}}%
\pgfpathcurveto{\pgfqpoint{4.595073in}{0.599077in}}{\pgfqpoint{4.593306in}{0.599809in}}{\pgfqpoint{4.591464in}{0.599809in}}%
\pgfpathcurveto{\pgfqpoint{4.589623in}{0.599809in}}{\pgfqpoint{4.587856in}{0.599077in}}{\pgfqpoint{4.586554in}{0.597775in}}%
\pgfpathcurveto{\pgfqpoint{4.585252in}{0.596473in}}{\pgfqpoint{4.584520in}{0.594706in}}{\pgfqpoint{4.584520in}{0.592864in}}%
\pgfpathcurveto{\pgfqpoint{4.584520in}{0.591023in}}{\pgfqpoint{4.585252in}{0.589256in}}{\pgfqpoint{4.586554in}{0.587954in}}%
\pgfpathcurveto{\pgfqpoint{4.587856in}{0.586652in}}{\pgfqpoint{4.589623in}{0.585920in}}{\pgfqpoint{4.591464in}{0.585920in}}%
\pgfpathlineto{\pgfqpoint{4.591464in}{0.585920in}}%
\pgfpathclose%
\pgfusepath{stroke,fill}%
\end{pgfscope}%
\begin{pgfscope}%
\pgfpathrectangle{\pgfqpoint{0.661006in}{0.524170in}}{\pgfqpoint{4.194036in}{1.071446in}}%
\pgfusepath{clip}%
\pgfsetbuttcap%
\pgfsetroundjoin%
\definecolor{currentfill}{rgb}{0.204008,0.158314,0.305705}%
\pgfsetfillcolor{currentfill}%
\pgfsetfillopacity{0.700000}%
\pgfsetlinewidth{1.003750pt}%
\definecolor{currentstroke}{rgb}{0.204008,0.158314,0.305705}%
\pgfsetstrokecolor{currentstroke}%
\pgfsetstrokeopacity{0.700000}%
\pgfsetdash{}{0pt}%
\pgfpathmoveto{\pgfqpoint{4.598778in}{0.586119in}}%
\pgfpathcurveto{\pgfqpoint{4.600620in}{0.586119in}}{\pgfqpoint{4.602386in}{0.586851in}}{\pgfqpoint{4.603688in}{0.588153in}}%
\pgfpathcurveto{\pgfqpoint{4.604991in}{0.589455in}}{\pgfqpoint{4.605722in}{0.591222in}}{\pgfqpoint{4.605722in}{0.593063in}}%
\pgfpathcurveto{\pgfqpoint{4.605722in}{0.594905in}}{\pgfqpoint{4.604991in}{0.596672in}}{\pgfqpoint{4.603688in}{0.597974in}}%
\pgfpathcurveto{\pgfqpoint{4.602386in}{0.599276in}}{\pgfqpoint{4.600620in}{0.600008in}}{\pgfqpoint{4.598778in}{0.600008in}}%
\pgfpathcurveto{\pgfqpoint{4.596936in}{0.600008in}}{\pgfqpoint{4.595170in}{0.599276in}}{\pgfqpoint{4.593868in}{0.597974in}}%
\pgfpathcurveto{\pgfqpoint{4.592565in}{0.596672in}}{\pgfqpoint{4.591834in}{0.594905in}}{\pgfqpoint{4.591834in}{0.593063in}}%
\pgfpathcurveto{\pgfqpoint{4.591834in}{0.591222in}}{\pgfqpoint{4.592565in}{0.589455in}}{\pgfqpoint{4.593868in}{0.588153in}}%
\pgfpathcurveto{\pgfqpoint{4.595170in}{0.586851in}}{\pgfqpoint{4.596936in}{0.586119in}}{\pgfqpoint{4.598778in}{0.586119in}}%
\pgfpathlineto{\pgfqpoint{4.598778in}{0.586119in}}%
\pgfpathclose%
\pgfusepath{stroke,fill}%
\end{pgfscope}%
\begin{pgfscope}%
\pgfpathrectangle{\pgfqpoint{0.661006in}{0.524170in}}{\pgfqpoint{4.194036in}{1.071446in}}%
\pgfusepath{clip}%
\pgfsetbuttcap%
\pgfsetroundjoin%
\definecolor{currentfill}{rgb}{0.204008,0.158314,0.305705}%
\pgfsetfillcolor{currentfill}%
\pgfsetfillopacity{0.700000}%
\pgfsetlinewidth{1.003750pt}%
\definecolor{currentstroke}{rgb}{0.204008,0.158314,0.305705}%
\pgfsetstrokecolor{currentstroke}%
\pgfsetstrokeopacity{0.700000}%
\pgfsetdash{}{0pt}%
\pgfpathmoveto{\pgfqpoint{4.609003in}{0.583749in}}%
\pgfpathcurveto{\pgfqpoint{4.610845in}{0.583749in}}{\pgfqpoint{4.612611in}{0.584480in}}{\pgfqpoint{4.613913in}{0.585783in}}%
\pgfpathcurveto{\pgfqpoint{4.615216in}{0.587085in}}{\pgfqpoint{4.615947in}{0.588851in}}{\pgfqpoint{4.615947in}{0.590693in}}%
\pgfpathcurveto{\pgfqpoint{4.615947in}{0.592535in}}{\pgfqpoint{4.615216in}{0.594301in}}{\pgfqpoint{4.613913in}{0.595603in}}%
\pgfpathcurveto{\pgfqpoint{4.612611in}{0.596906in}}{\pgfqpoint{4.610845in}{0.597637in}}{\pgfqpoint{4.609003in}{0.597637in}}%
\pgfpathcurveto{\pgfqpoint{4.607161in}{0.597637in}}{\pgfqpoint{4.605395in}{0.596906in}}{\pgfqpoint{4.604093in}{0.595603in}}%
\pgfpathcurveto{\pgfqpoint{4.602790in}{0.594301in}}{\pgfqpoint{4.602059in}{0.592535in}}{\pgfqpoint{4.602059in}{0.590693in}}%
\pgfpathcurveto{\pgfqpoint{4.602059in}{0.588851in}}{\pgfqpoint{4.602790in}{0.587085in}}{\pgfqpoint{4.604093in}{0.585783in}}%
\pgfpathcurveto{\pgfqpoint{4.605395in}{0.584480in}}{\pgfqpoint{4.607161in}{0.583749in}}{\pgfqpoint{4.609003in}{0.583749in}}%
\pgfpathlineto{\pgfqpoint{4.609003in}{0.583749in}}%
\pgfpathclose%
\pgfusepath{stroke,fill}%
\end{pgfscope}%
\begin{pgfscope}%
\pgfpathrectangle{\pgfqpoint{0.661006in}{0.524170in}}{\pgfqpoint{4.194036in}{1.071446in}}%
\pgfusepath{clip}%
\pgfsetbuttcap%
\pgfsetroundjoin%
\definecolor{currentfill}{rgb}{0.204008,0.158314,0.305705}%
\pgfsetfillcolor{currentfill}%
\pgfsetfillopacity{0.700000}%
\pgfsetlinewidth{1.003750pt}%
\definecolor{currentstroke}{rgb}{0.204008,0.158314,0.305705}%
\pgfsetstrokecolor{currentstroke}%
\pgfsetstrokeopacity{0.700000}%
\pgfsetdash{}{0pt}%
\pgfpathmoveto{\pgfqpoint{4.617648in}{0.580155in}}%
\pgfpathcurveto{\pgfqpoint{4.619489in}{0.580155in}}{\pgfqpoint{4.621256in}{0.580887in}}{\pgfqpoint{4.622558in}{0.582189in}}%
\pgfpathcurveto{\pgfqpoint{4.623861in}{0.583492in}}{\pgfqpoint{4.624592in}{0.585258in}}{\pgfqpoint{4.624592in}{0.587100in}}%
\pgfpathcurveto{\pgfqpoint{4.624592in}{0.588942in}}{\pgfqpoint{4.623861in}{0.590708in}}{\pgfqpoint{4.622558in}{0.592010in}}%
\pgfpathcurveto{\pgfqpoint{4.621256in}{0.593313in}}{\pgfqpoint{4.619489in}{0.594044in}}{\pgfqpoint{4.617648in}{0.594044in}}%
\pgfpathcurveto{\pgfqpoint{4.615806in}{0.594044in}}{\pgfqpoint{4.614040in}{0.593313in}}{\pgfqpoint{4.612737in}{0.592010in}}%
\pgfpathcurveto{\pgfqpoint{4.611435in}{0.590708in}}{\pgfqpoint{4.610703in}{0.588942in}}{\pgfqpoint{4.610703in}{0.587100in}}%
\pgfpathcurveto{\pgfqpoint{4.610703in}{0.585258in}}{\pgfqpoint{4.611435in}{0.583492in}}{\pgfqpoint{4.612737in}{0.582189in}}%
\pgfpathcurveto{\pgfqpoint{4.614040in}{0.580887in}}{\pgfqpoint{4.615806in}{0.580155in}}{\pgfqpoint{4.617648in}{0.580155in}}%
\pgfpathlineto{\pgfqpoint{4.617648in}{0.580155in}}%
\pgfpathclose%
\pgfusepath{stroke,fill}%
\end{pgfscope}%
\begin{pgfscope}%
\pgfpathrectangle{\pgfqpoint{0.661006in}{0.524170in}}{\pgfqpoint{4.194036in}{1.071446in}}%
\pgfusepath{clip}%
\pgfsetbuttcap%
\pgfsetroundjoin%
\definecolor{currentfill}{rgb}{0.200488,0.154573,0.299724}%
\pgfsetfillcolor{currentfill}%
\pgfsetfillopacity{0.700000}%
\pgfsetlinewidth{1.003750pt}%
\definecolor{currentstroke}{rgb}{0.200488,0.154573,0.299724}%
\pgfsetstrokecolor{currentstroke}%
\pgfsetstrokeopacity{0.700000}%
\pgfsetdash{}{0pt}%
\pgfpathmoveto{\pgfqpoint{4.621645in}{0.579913in}}%
\pgfpathcurveto{\pgfqpoint{4.623487in}{0.579913in}}{\pgfqpoint{4.625253in}{0.580645in}}{\pgfqpoint{4.626555in}{0.581947in}}%
\pgfpathcurveto{\pgfqpoint{4.627858in}{0.583249in}}{\pgfqpoint{4.628589in}{0.585016in}}{\pgfqpoint{4.628589in}{0.586857in}}%
\pgfpathcurveto{\pgfqpoint{4.628589in}{0.588699in}}{\pgfqpoint{4.627858in}{0.590466in}}{\pgfqpoint{4.626555in}{0.591768in}}%
\pgfpathcurveto{\pgfqpoint{4.625253in}{0.593070in}}{\pgfqpoint{4.623487in}{0.593802in}}{\pgfqpoint{4.621645in}{0.593802in}}%
\pgfpathcurveto{\pgfqpoint{4.619803in}{0.593802in}}{\pgfqpoint{4.618037in}{0.593070in}}{\pgfqpoint{4.616734in}{0.591768in}}%
\pgfpathcurveto{\pgfqpoint{4.615432in}{0.590466in}}{\pgfqpoint{4.614700in}{0.588699in}}{\pgfqpoint{4.614700in}{0.586857in}}%
\pgfpathcurveto{\pgfqpoint{4.614700in}{0.585016in}}{\pgfqpoint{4.615432in}{0.583249in}}{\pgfqpoint{4.616734in}{0.581947in}}%
\pgfpathcurveto{\pgfqpoint{4.618037in}{0.580645in}}{\pgfqpoint{4.619803in}{0.579913in}}{\pgfqpoint{4.621645in}{0.579913in}}%
\pgfpathlineto{\pgfqpoint{4.621645in}{0.579913in}}%
\pgfpathclose%
\pgfusepath{stroke,fill}%
\end{pgfscope}%
\begin{pgfscope}%
\pgfpathrectangle{\pgfqpoint{0.661006in}{0.524170in}}{\pgfqpoint{4.194036in}{1.071446in}}%
\pgfusepath{clip}%
\pgfsetbuttcap%
\pgfsetroundjoin%
\definecolor{currentfill}{rgb}{0.200488,0.154573,0.299724}%
\pgfsetfillcolor{currentfill}%
\pgfsetfillopacity{0.700000}%
\pgfsetlinewidth{1.003750pt}%
\definecolor{currentstroke}{rgb}{0.200488,0.154573,0.299724}%
\pgfsetstrokecolor{currentstroke}%
\pgfsetstrokeopacity{0.700000}%
\pgfsetdash{}{0pt}%
\pgfpathmoveto{\pgfqpoint{4.614301in}{0.581790in}}%
\pgfpathcurveto{\pgfqpoint{4.616143in}{0.581790in}}{\pgfqpoint{4.617910in}{0.582522in}}{\pgfqpoint{4.619212in}{0.583824in}}%
\pgfpathcurveto{\pgfqpoint{4.620514in}{0.585127in}}{\pgfqpoint{4.621246in}{0.586893in}}{\pgfqpoint{4.621246in}{0.588735in}}%
\pgfpathcurveto{\pgfqpoint{4.621246in}{0.590577in}}{\pgfqpoint{4.620514in}{0.592343in}}{\pgfqpoint{4.619212in}{0.593645in}}%
\pgfpathcurveto{\pgfqpoint{4.617910in}{0.594948in}}{\pgfqpoint{4.616143in}{0.595679in}}{\pgfqpoint{4.614301in}{0.595679in}}%
\pgfpathcurveto{\pgfqpoint{4.612460in}{0.595679in}}{\pgfqpoint{4.610693in}{0.594948in}}{\pgfqpoint{4.609391in}{0.593645in}}%
\pgfpathcurveto{\pgfqpoint{4.608089in}{0.592343in}}{\pgfqpoint{4.607357in}{0.590577in}}{\pgfqpoint{4.607357in}{0.588735in}}%
\pgfpathcurveto{\pgfqpoint{4.607357in}{0.586893in}}{\pgfqpoint{4.608089in}{0.585127in}}{\pgfqpoint{4.609391in}{0.583824in}}%
\pgfpathcurveto{\pgfqpoint{4.610693in}{0.582522in}}{\pgfqpoint{4.612460in}{0.581790in}}{\pgfqpoint{4.614301in}{0.581790in}}%
\pgfpathlineto{\pgfqpoint{4.614301in}{0.581790in}}%
\pgfpathclose%
\pgfusepath{stroke,fill}%
\end{pgfscope}%
\begin{pgfscope}%
\pgfpathrectangle{\pgfqpoint{0.661006in}{0.524170in}}{\pgfqpoint{4.194036in}{1.071446in}}%
\pgfusepath{clip}%
\pgfsetbuttcap%
\pgfsetroundjoin%
\definecolor{currentfill}{rgb}{0.196942,0.150848,0.293722}%
\pgfsetfillcolor{currentfill}%
\pgfsetfillopacity{0.700000}%
\pgfsetlinewidth{1.003750pt}%
\definecolor{currentstroke}{rgb}{0.196942,0.150848,0.293722}%
\pgfsetstrokecolor{currentstroke}%
\pgfsetstrokeopacity{0.700000}%
\pgfsetdash{}{0pt}%
\pgfpathmoveto{\pgfqpoint{4.613511in}{0.581696in}}%
\pgfpathcurveto{\pgfqpoint{4.615353in}{0.581696in}}{\pgfqpoint{4.617119in}{0.582428in}}{\pgfqpoint{4.618422in}{0.583730in}}%
\pgfpathcurveto{\pgfqpoint{4.619724in}{0.585032in}}{\pgfqpoint{4.620456in}{0.586799in}}{\pgfqpoint{4.620456in}{0.588640in}}%
\pgfpathcurveto{\pgfqpoint{4.620456in}{0.590482in}}{\pgfqpoint{4.619724in}{0.592249in}}{\pgfqpoint{4.618422in}{0.593551in}}%
\pgfpathcurveto{\pgfqpoint{4.617119in}{0.594853in}}{\pgfqpoint{4.615353in}{0.595585in}}{\pgfqpoint{4.613511in}{0.595585in}}%
\pgfpathcurveto{\pgfqpoint{4.611670in}{0.595585in}}{\pgfqpoint{4.609903in}{0.594853in}}{\pgfqpoint{4.608601in}{0.593551in}}%
\pgfpathcurveto{\pgfqpoint{4.607299in}{0.592249in}}{\pgfqpoint{4.606567in}{0.590482in}}{\pgfqpoint{4.606567in}{0.588640in}}%
\pgfpathcurveto{\pgfqpoint{4.606567in}{0.586799in}}{\pgfqpoint{4.607299in}{0.585032in}}{\pgfqpoint{4.608601in}{0.583730in}}%
\pgfpathcurveto{\pgfqpoint{4.609903in}{0.582428in}}{\pgfqpoint{4.611670in}{0.581696in}}{\pgfqpoint{4.613511in}{0.581696in}}%
\pgfpathlineto{\pgfqpoint{4.613511in}{0.581696in}}%
\pgfpathclose%
\pgfusepath{stroke,fill}%
\end{pgfscope}%
\begin{pgfscope}%
\pgfpathrectangle{\pgfqpoint{0.661006in}{0.524170in}}{\pgfqpoint{4.194036in}{1.071446in}}%
\pgfusepath{clip}%
\pgfsetbuttcap%
\pgfsetroundjoin%
\definecolor{currentfill}{rgb}{0.196942,0.150848,0.293722}%
\pgfsetfillcolor{currentfill}%
\pgfsetfillopacity{0.700000}%
\pgfsetlinewidth{1.003750pt}%
\definecolor{currentstroke}{rgb}{0.196942,0.150848,0.293722}%
\pgfsetstrokecolor{currentstroke}%
\pgfsetstrokeopacity{0.700000}%
\pgfsetdash{}{0pt}%
\pgfpathmoveto{\pgfqpoint{4.597941in}{0.585222in}}%
\pgfpathcurveto{\pgfqpoint{4.599783in}{0.585222in}}{\pgfqpoint{4.601550in}{0.585954in}}{\pgfqpoint{4.602852in}{0.587256in}}%
\pgfpathcurveto{\pgfqpoint{4.604154in}{0.588558in}}{\pgfqpoint{4.604886in}{0.590325in}}{\pgfqpoint{4.604886in}{0.592166in}}%
\pgfpathcurveto{\pgfqpoint{4.604886in}{0.594008in}}{\pgfqpoint{4.604154in}{0.595774in}}{\pgfqpoint{4.602852in}{0.597077in}}%
\pgfpathcurveto{\pgfqpoint{4.601550in}{0.598379in}}{\pgfqpoint{4.599783in}{0.599111in}}{\pgfqpoint{4.597941in}{0.599111in}}%
\pgfpathcurveto{\pgfqpoint{4.596100in}{0.599111in}}{\pgfqpoint{4.594333in}{0.598379in}}{\pgfqpoint{4.593031in}{0.597077in}}%
\pgfpathcurveto{\pgfqpoint{4.591729in}{0.595774in}}{\pgfqpoint{4.590997in}{0.594008in}}{\pgfqpoint{4.590997in}{0.592166in}}%
\pgfpathcurveto{\pgfqpoint{4.590997in}{0.590325in}}{\pgfqpoint{4.591729in}{0.588558in}}{\pgfqpoint{4.593031in}{0.587256in}}%
\pgfpathcurveto{\pgfqpoint{4.594333in}{0.585954in}}{\pgfqpoint{4.596100in}{0.585222in}}{\pgfqpoint{4.597941in}{0.585222in}}%
\pgfpathlineto{\pgfqpoint{4.597941in}{0.585222in}}%
\pgfpathclose%
\pgfusepath{stroke,fill}%
\end{pgfscope}%
\begin{pgfscope}%
\pgfpathrectangle{\pgfqpoint{0.661006in}{0.524170in}}{\pgfqpoint{4.194036in}{1.071446in}}%
\pgfusepath{clip}%
\pgfsetbuttcap%
\pgfsetroundjoin%
\definecolor{currentfill}{rgb}{0.196942,0.150848,0.293722}%
\pgfsetfillcolor{currentfill}%
\pgfsetfillopacity{0.700000}%
\pgfsetlinewidth{1.003750pt}%
\definecolor{currentstroke}{rgb}{0.196942,0.150848,0.293722}%
\pgfsetstrokecolor{currentstroke}%
\pgfsetstrokeopacity{0.700000}%
\pgfsetdash{}{0pt}%
\pgfpathmoveto{\pgfqpoint{4.576144in}{0.589825in}}%
\pgfpathcurveto{\pgfqpoint{4.577985in}{0.589825in}}{\pgfqpoint{4.579752in}{0.590557in}}{\pgfqpoint{4.581054in}{0.591859in}}%
\pgfpathcurveto{\pgfqpoint{4.582356in}{0.593162in}}{\pgfqpoint{4.583088in}{0.594928in}}{\pgfqpoint{4.583088in}{0.596770in}}%
\pgfpathcurveto{\pgfqpoint{4.583088in}{0.598611in}}{\pgfqpoint{4.582356in}{0.600378in}}{\pgfqpoint{4.581054in}{0.601680in}}%
\pgfpathcurveto{\pgfqpoint{4.579752in}{0.602983in}}{\pgfqpoint{4.577985in}{0.603714in}}{\pgfqpoint{4.576144in}{0.603714in}}%
\pgfpathcurveto{\pgfqpoint{4.574302in}{0.603714in}}{\pgfqpoint{4.572535in}{0.602983in}}{\pgfqpoint{4.571233in}{0.601680in}}%
\pgfpathcurveto{\pgfqpoint{4.569931in}{0.600378in}}{\pgfqpoint{4.569199in}{0.598611in}}{\pgfqpoint{4.569199in}{0.596770in}}%
\pgfpathcurveto{\pgfqpoint{4.569199in}{0.594928in}}{\pgfqpoint{4.569931in}{0.593162in}}{\pgfqpoint{4.571233in}{0.591859in}}%
\pgfpathcurveto{\pgfqpoint{4.572535in}{0.590557in}}{\pgfqpoint{4.574302in}{0.589825in}}{\pgfqpoint{4.576144in}{0.589825in}}%
\pgfpathlineto{\pgfqpoint{4.576144in}{0.589825in}}%
\pgfpathclose%
\pgfusepath{stroke,fill}%
\end{pgfscope}%
\begin{pgfscope}%
\pgfpathrectangle{\pgfqpoint{0.661006in}{0.524170in}}{\pgfqpoint{4.194036in}{1.071446in}}%
\pgfusepath{clip}%
\pgfsetbuttcap%
\pgfsetroundjoin%
\definecolor{currentfill}{rgb}{0.196942,0.150848,0.293722}%
\pgfsetfillcolor{currentfill}%
\pgfsetfillopacity{0.700000}%
\pgfsetlinewidth{1.003750pt}%
\definecolor{currentstroke}{rgb}{0.196942,0.150848,0.293722}%
\pgfsetstrokecolor{currentstroke}%
\pgfsetstrokeopacity{0.700000}%
\pgfsetdash{}{0pt}%
\pgfpathmoveto{\pgfqpoint{4.566616in}{0.592548in}}%
\pgfpathcurveto{\pgfqpoint{4.568457in}{0.592548in}}{\pgfqpoint{4.570224in}{0.593280in}}{\pgfqpoint{4.571526in}{0.594582in}}%
\pgfpathcurveto{\pgfqpoint{4.572829in}{0.595885in}}{\pgfqpoint{4.573560in}{0.597651in}}{\pgfqpoint{4.573560in}{0.599493in}}%
\pgfpathcurveto{\pgfqpoint{4.573560in}{0.601334in}}{\pgfqpoint{4.572829in}{0.603101in}}{\pgfqpoint{4.571526in}{0.604403in}}%
\pgfpathcurveto{\pgfqpoint{4.570224in}{0.605705in}}{\pgfqpoint{4.568457in}{0.606437in}}{\pgfqpoint{4.566616in}{0.606437in}}%
\pgfpathcurveto{\pgfqpoint{4.564774in}{0.606437in}}{\pgfqpoint{4.563008in}{0.605705in}}{\pgfqpoint{4.561705in}{0.604403in}}%
\pgfpathcurveto{\pgfqpoint{4.560403in}{0.603101in}}{\pgfqpoint{4.559671in}{0.601334in}}{\pgfqpoint{4.559671in}{0.599493in}}%
\pgfpathcurveto{\pgfqpoint{4.559671in}{0.597651in}}{\pgfqpoint{4.560403in}{0.595885in}}{\pgfqpoint{4.561705in}{0.594582in}}%
\pgfpathcurveto{\pgfqpoint{4.563008in}{0.593280in}}{\pgfqpoint{4.564774in}{0.592548in}}{\pgfqpoint{4.566616in}{0.592548in}}%
\pgfpathlineto{\pgfqpoint{4.566616in}{0.592548in}}%
\pgfpathclose%
\pgfusepath{stroke,fill}%
\end{pgfscope}%
\begin{pgfscope}%
\pgfpathrectangle{\pgfqpoint{0.661006in}{0.524170in}}{\pgfqpoint{4.194036in}{1.071446in}}%
\pgfusepath{clip}%
\pgfsetbuttcap%
\pgfsetroundjoin%
\definecolor{currentfill}{rgb}{0.196942,0.150848,0.293722}%
\pgfsetfillcolor{currentfill}%
\pgfsetfillopacity{0.700000}%
\pgfsetlinewidth{1.003750pt}%
\definecolor{currentstroke}{rgb}{0.196942,0.150848,0.293722}%
\pgfsetstrokecolor{currentstroke}%
\pgfsetstrokeopacity{0.700000}%
\pgfsetdash{}{0pt}%
\pgfpathmoveto{\pgfqpoint{4.565593in}{0.591437in}}%
\pgfpathcurveto{\pgfqpoint{4.567435in}{0.591437in}}{\pgfqpoint{4.569201in}{0.592168in}}{\pgfqpoint{4.570504in}{0.593471in}}%
\pgfpathcurveto{\pgfqpoint{4.571806in}{0.594773in}}{\pgfqpoint{4.572538in}{0.596539in}}{\pgfqpoint{4.572538in}{0.598381in}}%
\pgfpathcurveto{\pgfqpoint{4.572538in}{0.600223in}}{\pgfqpoint{4.571806in}{0.601989in}}{\pgfqpoint{4.570504in}{0.603292in}}%
\pgfpathcurveto{\pgfqpoint{4.569201in}{0.604594in}}{\pgfqpoint{4.567435in}{0.605326in}}{\pgfqpoint{4.565593in}{0.605326in}}%
\pgfpathcurveto{\pgfqpoint{4.563752in}{0.605326in}}{\pgfqpoint{4.561985in}{0.604594in}}{\pgfqpoint{4.560683in}{0.603292in}}%
\pgfpathcurveto{\pgfqpoint{4.559381in}{0.601989in}}{\pgfqpoint{4.558649in}{0.600223in}}{\pgfqpoint{4.558649in}{0.598381in}}%
\pgfpathcurveto{\pgfqpoint{4.558649in}{0.596539in}}{\pgfqpoint{4.559381in}{0.594773in}}{\pgfqpoint{4.560683in}{0.593471in}}%
\pgfpathcurveto{\pgfqpoint{4.561985in}{0.592168in}}{\pgfqpoint{4.563752in}{0.591437in}}{\pgfqpoint{4.565593in}{0.591437in}}%
\pgfpathlineto{\pgfqpoint{4.565593in}{0.591437in}}%
\pgfpathclose%
\pgfusepath{stroke,fill}%
\end{pgfscope}%
\begin{pgfscope}%
\pgfpathrectangle{\pgfqpoint{0.661006in}{0.524170in}}{\pgfqpoint{4.194036in}{1.071446in}}%
\pgfusepath{clip}%
\pgfsetbuttcap%
\pgfsetroundjoin%
\definecolor{currentfill}{rgb}{0.193371,0.147140,0.287697}%
\pgfsetfillcolor{currentfill}%
\pgfsetfillopacity{0.700000}%
\pgfsetlinewidth{1.003750pt}%
\definecolor{currentstroke}{rgb}{0.193371,0.147140,0.287697}%
\pgfsetstrokecolor{currentstroke}%
\pgfsetstrokeopacity{0.700000}%
\pgfsetdash{}{0pt}%
\pgfpathmoveto{\pgfqpoint{4.568428in}{0.590547in}}%
\pgfpathcurveto{\pgfqpoint{4.570270in}{0.590547in}}{\pgfqpoint{4.572037in}{0.591278in}}{\pgfqpoint{4.573339in}{0.592581in}}%
\pgfpathcurveto{\pgfqpoint{4.574641in}{0.593883in}}{\pgfqpoint{4.575373in}{0.595649in}}{\pgfqpoint{4.575373in}{0.597491in}}%
\pgfpathcurveto{\pgfqpoint{4.575373in}{0.599333in}}{\pgfqpoint{4.574641in}{0.601099in}}{\pgfqpoint{4.573339in}{0.602402in}}%
\pgfpathcurveto{\pgfqpoint{4.572037in}{0.603704in}}{\pgfqpoint{4.570270in}{0.604436in}}{\pgfqpoint{4.568428in}{0.604436in}}%
\pgfpathcurveto{\pgfqpoint{4.566587in}{0.604436in}}{\pgfqpoint{4.564820in}{0.603704in}}{\pgfqpoint{4.563518in}{0.602402in}}%
\pgfpathcurveto{\pgfqpoint{4.562216in}{0.601099in}}{\pgfqpoint{4.561484in}{0.599333in}}{\pgfqpoint{4.561484in}{0.597491in}}%
\pgfpathcurveto{\pgfqpoint{4.561484in}{0.595649in}}{\pgfqpoint{4.562216in}{0.593883in}}{\pgfqpoint{4.563518in}{0.592581in}}%
\pgfpathcurveto{\pgfqpoint{4.564820in}{0.591278in}}{\pgfqpoint{4.566587in}{0.590547in}}{\pgfqpoint{4.568428in}{0.590547in}}%
\pgfpathlineto{\pgfqpoint{4.568428in}{0.590547in}}%
\pgfpathclose%
\pgfusepath{stroke,fill}%
\end{pgfscope}%
\begin{pgfscope}%
\pgfpathrectangle{\pgfqpoint{0.661006in}{0.524170in}}{\pgfqpoint{4.194036in}{1.071446in}}%
\pgfusepath{clip}%
\pgfsetbuttcap%
\pgfsetroundjoin%
\definecolor{currentfill}{rgb}{0.193371,0.147140,0.287697}%
\pgfsetfillcolor{currentfill}%
\pgfsetfillopacity{0.700000}%
\pgfsetlinewidth{1.003750pt}%
\definecolor{currentstroke}{rgb}{0.193371,0.147140,0.287697}%
\pgfsetstrokecolor{currentstroke}%
\pgfsetstrokeopacity{0.700000}%
\pgfsetdash{}{0pt}%
\pgfpathmoveto{\pgfqpoint{4.563283in}{0.592695in}}%
\pgfpathcurveto{\pgfqpoint{4.565124in}{0.592695in}}{\pgfqpoint{4.566891in}{0.593427in}}{\pgfqpoint{4.568193in}{0.594729in}}%
\pgfpathcurveto{\pgfqpoint{4.569495in}{0.596031in}}{\pgfqpoint{4.570227in}{0.597798in}}{\pgfqpoint{4.570227in}{0.599639in}}%
\pgfpathcurveto{\pgfqpoint{4.570227in}{0.601481in}}{\pgfqpoint{4.569495in}{0.603248in}}{\pgfqpoint{4.568193in}{0.604550in}}%
\pgfpathcurveto{\pgfqpoint{4.566891in}{0.605852in}}{\pgfqpoint{4.565124in}{0.606584in}}{\pgfqpoint{4.563283in}{0.606584in}}%
\pgfpathcurveto{\pgfqpoint{4.561441in}{0.606584in}}{\pgfqpoint{4.559674in}{0.605852in}}{\pgfqpoint{4.558372in}{0.604550in}}%
\pgfpathcurveto{\pgfqpoint{4.557070in}{0.603248in}}{\pgfqpoint{4.556338in}{0.601481in}}{\pgfqpoint{4.556338in}{0.599639in}}%
\pgfpathcurveto{\pgfqpoint{4.556338in}{0.597798in}}{\pgfqpoint{4.557070in}{0.596031in}}{\pgfqpoint{4.558372in}{0.594729in}}%
\pgfpathcurveto{\pgfqpoint{4.559674in}{0.593427in}}{\pgfqpoint{4.561441in}{0.592695in}}{\pgfqpoint{4.563283in}{0.592695in}}%
\pgfpathlineto{\pgfqpoint{4.563283in}{0.592695in}}%
\pgfpathclose%
\pgfusepath{stroke,fill}%
\end{pgfscope}%
\begin{pgfscope}%
\pgfpathrectangle{\pgfqpoint{0.661006in}{0.524170in}}{\pgfqpoint{4.194036in}{1.071446in}}%
\pgfusepath{clip}%
\pgfsetbuttcap%
\pgfsetroundjoin%
\definecolor{currentfill}{rgb}{0.189773,0.143450,0.281650}%
\pgfsetfillcolor{currentfill}%
\pgfsetfillopacity{0.700000}%
\pgfsetlinewidth{1.003750pt}%
\definecolor{currentstroke}{rgb}{0.189773,0.143450,0.281650}%
\pgfsetstrokecolor{currentstroke}%
\pgfsetstrokeopacity{0.700000}%
\pgfsetdash{}{0pt}%
\pgfpathmoveto{\pgfqpoint{4.550535in}{0.595757in}}%
\pgfpathcurveto{\pgfqpoint{4.552376in}{0.595757in}}{\pgfqpoint{4.554143in}{0.596489in}}{\pgfqpoint{4.555445in}{0.597791in}}%
\pgfpathcurveto{\pgfqpoint{4.556747in}{0.599094in}}{\pgfqpoint{4.557479in}{0.600860in}}{\pgfqpoint{4.557479in}{0.602702in}}%
\pgfpathcurveto{\pgfqpoint{4.557479in}{0.604543in}}{\pgfqpoint{4.556747in}{0.606310in}}{\pgfqpoint{4.555445in}{0.607612in}}%
\pgfpathcurveto{\pgfqpoint{4.554143in}{0.608915in}}{\pgfqpoint{4.552376in}{0.609646in}}{\pgfqpoint{4.550535in}{0.609646in}}%
\pgfpathcurveto{\pgfqpoint{4.548693in}{0.609646in}}{\pgfqpoint{4.546926in}{0.608915in}}{\pgfqpoint{4.545624in}{0.607612in}}%
\pgfpathcurveto{\pgfqpoint{4.544322in}{0.606310in}}{\pgfqpoint{4.543590in}{0.604543in}}{\pgfqpoint{4.543590in}{0.602702in}}%
\pgfpathcurveto{\pgfqpoint{4.543590in}{0.600860in}}{\pgfqpoint{4.544322in}{0.599094in}}{\pgfqpoint{4.545624in}{0.597791in}}%
\pgfpathcurveto{\pgfqpoint{4.546926in}{0.596489in}}{\pgfqpoint{4.548693in}{0.595757in}}{\pgfqpoint{4.550535in}{0.595757in}}%
\pgfpathlineto{\pgfqpoint{4.550535in}{0.595757in}}%
\pgfpathclose%
\pgfusepath{stroke,fill}%
\end{pgfscope}%
\begin{pgfscope}%
\pgfpathrectangle{\pgfqpoint{0.661006in}{0.524170in}}{\pgfqpoint{4.194036in}{1.071446in}}%
\pgfusepath{clip}%
\pgfsetbuttcap%
\pgfsetroundjoin%
\definecolor{currentfill}{rgb}{0.189773,0.143450,0.281650}%
\pgfsetfillcolor{currentfill}%
\pgfsetfillopacity{0.700000}%
\pgfsetlinewidth{1.003750pt}%
\definecolor{currentstroke}{rgb}{0.189773,0.143450,0.281650}%
\pgfsetstrokecolor{currentstroke}%
\pgfsetstrokeopacity{0.700000}%
\pgfsetdash{}{0pt}%
\pgfpathmoveto{\pgfqpoint{4.561782in}{0.592304in}}%
\pgfpathcurveto{\pgfqpoint{4.563624in}{0.592304in}}{\pgfqpoint{4.565390in}{0.593036in}}{\pgfqpoint{4.566693in}{0.594338in}}%
\pgfpathcurveto{\pgfqpoint{4.567995in}{0.595640in}}{\pgfqpoint{4.568727in}{0.597407in}}{\pgfqpoint{4.568727in}{0.599249in}}%
\pgfpathcurveto{\pgfqpoint{4.568727in}{0.601090in}}{\pgfqpoint{4.567995in}{0.602857in}}{\pgfqpoint{4.566693in}{0.604159in}}%
\pgfpathcurveto{\pgfqpoint{4.565390in}{0.605461in}}{\pgfqpoint{4.563624in}{0.606193in}}{\pgfqpoint{4.561782in}{0.606193in}}%
\pgfpathcurveto{\pgfqpoint{4.559940in}{0.606193in}}{\pgfqpoint{4.558174in}{0.605461in}}{\pgfqpoint{4.556872in}{0.604159in}}%
\pgfpathcurveto{\pgfqpoint{4.555569in}{0.602857in}}{\pgfqpoint{4.554838in}{0.601090in}}{\pgfqpoint{4.554838in}{0.599249in}}%
\pgfpathcurveto{\pgfqpoint{4.554838in}{0.597407in}}{\pgfqpoint{4.555569in}{0.595640in}}{\pgfqpoint{4.556872in}{0.594338in}}%
\pgfpathcurveto{\pgfqpoint{4.558174in}{0.593036in}}{\pgfqpoint{4.559940in}{0.592304in}}{\pgfqpoint{4.561782in}{0.592304in}}%
\pgfpathlineto{\pgfqpoint{4.561782in}{0.592304in}}%
\pgfpathclose%
\pgfusepath{stroke,fill}%
\end{pgfscope}%
\begin{pgfscope}%
\pgfpathrectangle{\pgfqpoint{0.661006in}{0.524170in}}{\pgfqpoint{4.194036in}{1.071446in}}%
\pgfusepath{clip}%
\pgfsetbuttcap%
\pgfsetroundjoin%
\definecolor{currentfill}{rgb}{0.189773,0.143450,0.281650}%
\pgfsetfillcolor{currentfill}%
\pgfsetfillopacity{0.700000}%
\pgfsetlinewidth{1.003750pt}%
\definecolor{currentstroke}{rgb}{0.189773,0.143450,0.281650}%
\pgfsetstrokecolor{currentstroke}%
\pgfsetstrokeopacity{0.700000}%
\pgfsetdash{}{0pt}%
\pgfpathmoveto{\pgfqpoint{4.561410in}{0.591434in}}%
\pgfpathcurveto{\pgfqpoint{4.563252in}{0.591434in}}{\pgfqpoint{4.565019in}{0.592165in}}{\pgfqpoint{4.566321in}{0.593468in}}%
\pgfpathcurveto{\pgfqpoint{4.567623in}{0.594770in}}{\pgfqpoint{4.568355in}{0.596537in}}{\pgfqpoint{4.568355in}{0.598378in}}%
\pgfpathcurveto{\pgfqpoint{4.568355in}{0.600220in}}{\pgfqpoint{4.567623in}{0.601986in}}{\pgfqpoint{4.566321in}{0.603289in}}%
\pgfpathcurveto{\pgfqpoint{4.565019in}{0.604591in}}{\pgfqpoint{4.563252in}{0.605323in}}{\pgfqpoint{4.561410in}{0.605323in}}%
\pgfpathcurveto{\pgfqpoint{4.559569in}{0.605323in}}{\pgfqpoint{4.557802in}{0.604591in}}{\pgfqpoint{4.556500in}{0.603289in}}%
\pgfpathcurveto{\pgfqpoint{4.555198in}{0.601986in}}{\pgfqpoint{4.554466in}{0.600220in}}{\pgfqpoint{4.554466in}{0.598378in}}%
\pgfpathcurveto{\pgfqpoint{4.554466in}{0.596537in}}{\pgfqpoint{4.555198in}{0.594770in}}{\pgfqpoint{4.556500in}{0.593468in}}%
\pgfpathcurveto{\pgfqpoint{4.557802in}{0.592165in}}{\pgfqpoint{4.559569in}{0.591434in}}{\pgfqpoint{4.561410in}{0.591434in}}%
\pgfpathlineto{\pgfqpoint{4.561410in}{0.591434in}}%
\pgfpathclose%
\pgfusepath{stroke,fill}%
\end{pgfscope}%
\begin{pgfscope}%
\pgfpathrectangle{\pgfqpoint{0.661006in}{0.524170in}}{\pgfqpoint{4.194036in}{1.071446in}}%
\pgfusepath{clip}%
\pgfsetbuttcap%
\pgfsetroundjoin%
\definecolor{currentfill}{rgb}{0.189773,0.143450,0.281650}%
\pgfsetfillcolor{currentfill}%
\pgfsetfillopacity{0.700000}%
\pgfsetlinewidth{1.003750pt}%
\definecolor{currentstroke}{rgb}{0.189773,0.143450,0.281650}%
\pgfsetstrokecolor{currentstroke}%
\pgfsetstrokeopacity{0.700000}%
\pgfsetdash{}{0pt}%
\pgfpathmoveto{\pgfqpoint{4.563269in}{0.590718in}}%
\pgfpathcurveto{\pgfqpoint{4.565111in}{0.590718in}}{\pgfqpoint{4.566878in}{0.591450in}}{\pgfqpoint{4.568180in}{0.592752in}}%
\pgfpathcurveto{\pgfqpoint{4.569482in}{0.594054in}}{\pgfqpoint{4.570214in}{0.595821in}}{\pgfqpoint{4.570214in}{0.597662in}}%
\pgfpathcurveto{\pgfqpoint{4.570214in}{0.599504in}}{\pgfqpoint{4.569482in}{0.601271in}}{\pgfqpoint{4.568180in}{0.602573in}}%
\pgfpathcurveto{\pgfqpoint{4.566878in}{0.603875in}}{\pgfqpoint{4.565111in}{0.604607in}}{\pgfqpoint{4.563269in}{0.604607in}}%
\pgfpathcurveto{\pgfqpoint{4.561428in}{0.604607in}}{\pgfqpoint{4.559661in}{0.603875in}}{\pgfqpoint{4.558359in}{0.602573in}}%
\pgfpathcurveto{\pgfqpoint{4.557057in}{0.601271in}}{\pgfqpoint{4.556325in}{0.599504in}}{\pgfqpoint{4.556325in}{0.597662in}}%
\pgfpathcurveto{\pgfqpoint{4.556325in}{0.595821in}}{\pgfqpoint{4.557057in}{0.594054in}}{\pgfqpoint{4.558359in}{0.592752in}}%
\pgfpathcurveto{\pgfqpoint{4.559661in}{0.591450in}}{\pgfqpoint{4.561428in}{0.590718in}}{\pgfqpoint{4.563269in}{0.590718in}}%
\pgfpathlineto{\pgfqpoint{4.563269in}{0.590718in}}%
\pgfpathclose%
\pgfusepath{stroke,fill}%
\end{pgfscope}%
\begin{pgfscope}%
\pgfpathrectangle{\pgfqpoint{0.661006in}{0.524170in}}{\pgfqpoint{4.194036in}{1.071446in}}%
\pgfusepath{clip}%
\pgfsetbuttcap%
\pgfsetroundjoin%
\definecolor{currentfill}{rgb}{0.189773,0.143450,0.281650}%
\pgfsetfillcolor{currentfill}%
\pgfsetfillopacity{0.700000}%
\pgfsetlinewidth{1.003750pt}%
\definecolor{currentstroke}{rgb}{0.189773,0.143450,0.281650}%
\pgfsetstrokecolor{currentstroke}%
\pgfsetstrokeopacity{0.700000}%
\pgfsetdash{}{0pt}%
\pgfpathmoveto{\pgfqpoint{4.576701in}{0.588603in}}%
\pgfpathcurveto{\pgfqpoint{4.578543in}{0.588603in}}{\pgfqpoint{4.580310in}{0.589335in}}{\pgfqpoint{4.581612in}{0.590637in}}%
\pgfpathcurveto{\pgfqpoint{4.582914in}{0.591939in}}{\pgfqpoint{4.583646in}{0.593706in}}{\pgfqpoint{4.583646in}{0.595547in}}%
\pgfpathcurveto{\pgfqpoint{4.583646in}{0.597389in}}{\pgfqpoint{4.582914in}{0.599156in}}{\pgfqpoint{4.581612in}{0.600458in}}%
\pgfpathcurveto{\pgfqpoint{4.580310in}{0.601760in}}{\pgfqpoint{4.578543in}{0.602492in}}{\pgfqpoint{4.576701in}{0.602492in}}%
\pgfpathcurveto{\pgfqpoint{4.574860in}{0.602492in}}{\pgfqpoint{4.573093in}{0.601760in}}{\pgfqpoint{4.571791in}{0.600458in}}%
\pgfpathcurveto{\pgfqpoint{4.570489in}{0.599156in}}{\pgfqpoint{4.569757in}{0.597389in}}{\pgfqpoint{4.569757in}{0.595547in}}%
\pgfpathcurveto{\pgfqpoint{4.569757in}{0.593706in}}{\pgfqpoint{4.570489in}{0.591939in}}{\pgfqpoint{4.571791in}{0.590637in}}%
\pgfpathcurveto{\pgfqpoint{4.573093in}{0.589335in}}{\pgfqpoint{4.574860in}{0.588603in}}{\pgfqpoint{4.576701in}{0.588603in}}%
\pgfpathlineto{\pgfqpoint{4.576701in}{0.588603in}}%
\pgfpathclose%
\pgfusepath{stroke,fill}%
\end{pgfscope}%
\begin{pgfscope}%
\pgfpathrectangle{\pgfqpoint{0.661006in}{0.524170in}}{\pgfqpoint{4.194036in}{1.071446in}}%
\pgfusepath{clip}%
\pgfsetbuttcap%
\pgfsetroundjoin%
\definecolor{currentfill}{rgb}{0.186148,0.139778,0.275582}%
\pgfsetfillcolor{currentfill}%
\pgfsetfillopacity{0.700000}%
\pgfsetlinewidth{1.003750pt}%
\definecolor{currentstroke}{rgb}{0.186148,0.139778,0.275582}%
\pgfsetstrokecolor{currentstroke}%
\pgfsetstrokeopacity{0.700000}%
\pgfsetdash{}{0pt}%
\pgfpathmoveto{\pgfqpoint{4.590691in}{0.585626in}}%
\pgfpathcurveto{\pgfqpoint{4.592533in}{0.585626in}}{\pgfqpoint{4.594299in}{0.586358in}}{\pgfqpoint{4.595601in}{0.587660in}}%
\pgfpathcurveto{\pgfqpoint{4.596904in}{0.588963in}}{\pgfqpoint{4.597635in}{0.590729in}}{\pgfqpoint{4.597635in}{0.592571in}}%
\pgfpathcurveto{\pgfqpoint{4.597635in}{0.594412in}}{\pgfqpoint{4.596904in}{0.596179in}}{\pgfqpoint{4.595601in}{0.597481in}}%
\pgfpathcurveto{\pgfqpoint{4.594299in}{0.598783in}}{\pgfqpoint{4.592533in}{0.599515in}}{\pgfqpoint{4.590691in}{0.599515in}}%
\pgfpathcurveto{\pgfqpoint{4.588849in}{0.599515in}}{\pgfqpoint{4.587083in}{0.598783in}}{\pgfqpoint{4.585781in}{0.597481in}}%
\pgfpathcurveto{\pgfqpoint{4.584478in}{0.596179in}}{\pgfqpoint{4.583747in}{0.594412in}}{\pgfqpoint{4.583747in}{0.592571in}}%
\pgfpathcurveto{\pgfqpoint{4.583747in}{0.590729in}}{\pgfqpoint{4.584478in}{0.588963in}}{\pgfqpoint{4.585781in}{0.587660in}}%
\pgfpathcurveto{\pgfqpoint{4.587083in}{0.586358in}}{\pgfqpoint{4.588849in}{0.585626in}}{\pgfqpoint{4.590691in}{0.585626in}}%
\pgfpathlineto{\pgfqpoint{4.590691in}{0.585626in}}%
\pgfpathclose%
\pgfusepath{stroke,fill}%
\end{pgfscope}%
\begin{pgfscope}%
\pgfpathrectangle{\pgfqpoint{0.661006in}{0.524170in}}{\pgfqpoint{4.194036in}{1.071446in}}%
\pgfusepath{clip}%
\pgfsetbuttcap%
\pgfsetroundjoin%
\definecolor{currentfill}{rgb}{0.186148,0.139778,0.275582}%
\pgfsetfillcolor{currentfill}%
\pgfsetfillopacity{0.700000}%
\pgfsetlinewidth{1.003750pt}%
\definecolor{currentstroke}{rgb}{0.186148,0.139778,0.275582}%
\pgfsetstrokecolor{currentstroke}%
\pgfsetstrokeopacity{0.700000}%
\pgfsetdash{}{0pt}%
\pgfpathmoveto{\pgfqpoint{4.610630in}{0.580007in}}%
\pgfpathcurveto{\pgfqpoint{4.612471in}{0.580007in}}{\pgfqpoint{4.614238in}{0.580738in}}{\pgfqpoint{4.615540in}{0.582041in}}%
\pgfpathcurveto{\pgfqpoint{4.616842in}{0.583343in}}{\pgfqpoint{4.617574in}{0.585109in}}{\pgfqpoint{4.617574in}{0.586951in}}%
\pgfpathcurveto{\pgfqpoint{4.617574in}{0.588793in}}{\pgfqpoint{4.616842in}{0.590559in}}{\pgfqpoint{4.615540in}{0.591861in}}%
\pgfpathcurveto{\pgfqpoint{4.614238in}{0.593164in}}{\pgfqpoint{4.612471in}{0.593895in}}{\pgfqpoint{4.610630in}{0.593895in}}%
\pgfpathcurveto{\pgfqpoint{4.608788in}{0.593895in}}{\pgfqpoint{4.607022in}{0.593164in}}{\pgfqpoint{4.605719in}{0.591861in}}%
\pgfpathcurveto{\pgfqpoint{4.604417in}{0.590559in}}{\pgfqpoint{4.603685in}{0.588793in}}{\pgfqpoint{4.603685in}{0.586951in}}%
\pgfpathcurveto{\pgfqpoint{4.603685in}{0.585109in}}{\pgfqpoint{4.604417in}{0.583343in}}{\pgfqpoint{4.605719in}{0.582041in}}%
\pgfpathcurveto{\pgfqpoint{4.607022in}{0.580738in}}{\pgfqpoint{4.608788in}{0.580007in}}{\pgfqpoint{4.610630in}{0.580007in}}%
\pgfpathlineto{\pgfqpoint{4.610630in}{0.580007in}}%
\pgfpathclose%
\pgfusepath{stroke,fill}%
\end{pgfscope}%
\begin{pgfscope}%
\pgfpathrectangle{\pgfqpoint{0.661006in}{0.524170in}}{\pgfqpoint{4.194036in}{1.071446in}}%
\pgfusepath{clip}%
\pgfsetbuttcap%
\pgfsetroundjoin%
\definecolor{currentfill}{rgb}{0.182497,0.136123,0.269492}%
\pgfsetfillcolor{currentfill}%
\pgfsetfillopacity{0.700000}%
\pgfsetlinewidth{1.003750pt}%
\definecolor{currentstroke}{rgb}{0.182497,0.136123,0.269492}%
\pgfsetstrokecolor{currentstroke}%
\pgfsetstrokeopacity{0.700000}%
\pgfsetdash{}{0pt}%
\pgfpathmoveto{\pgfqpoint{4.639864in}{0.574173in}}%
\pgfpathcurveto{\pgfqpoint{4.641706in}{0.574173in}}{\pgfqpoint{4.643472in}{0.574904in}}{\pgfqpoint{4.644774in}{0.576207in}}%
\pgfpathcurveto{\pgfqpoint{4.646077in}{0.577509in}}{\pgfqpoint{4.646808in}{0.579275in}}{\pgfqpoint{4.646808in}{0.581117in}}%
\pgfpathcurveto{\pgfqpoint{4.646808in}{0.582959in}}{\pgfqpoint{4.646077in}{0.584725in}}{\pgfqpoint{4.644774in}{0.586028in}}%
\pgfpathcurveto{\pgfqpoint{4.643472in}{0.587330in}}{\pgfqpoint{4.641706in}{0.588062in}}{\pgfqpoint{4.639864in}{0.588062in}}%
\pgfpathcurveto{\pgfqpoint{4.638022in}{0.588062in}}{\pgfqpoint{4.636256in}{0.587330in}}{\pgfqpoint{4.634953in}{0.586028in}}%
\pgfpathcurveto{\pgfqpoint{4.633651in}{0.584725in}}{\pgfqpoint{4.632919in}{0.582959in}}{\pgfqpoint{4.632919in}{0.581117in}}%
\pgfpathcurveto{\pgfqpoint{4.632919in}{0.579275in}}{\pgfqpoint{4.633651in}{0.577509in}}{\pgfqpoint{4.634953in}{0.576207in}}%
\pgfpathcurveto{\pgfqpoint{4.636256in}{0.574904in}}{\pgfqpoint{4.638022in}{0.574173in}}{\pgfqpoint{4.639864in}{0.574173in}}%
\pgfpathlineto{\pgfqpoint{4.639864in}{0.574173in}}%
\pgfpathclose%
\pgfusepath{stroke,fill}%
\end{pgfscope}%
\begin{pgfscope}%
\pgfpathrectangle{\pgfqpoint{0.661006in}{0.524170in}}{\pgfqpoint{4.194036in}{1.071446in}}%
\pgfusepath{clip}%
\pgfsetbuttcap%
\pgfsetroundjoin%
\definecolor{currentfill}{rgb}{0.182497,0.136123,0.269492}%
\pgfsetfillcolor{currentfill}%
\pgfsetfillopacity{0.700000}%
\pgfsetlinewidth{1.003750pt}%
\definecolor{currentstroke}{rgb}{0.182497,0.136123,0.269492}%
\pgfsetstrokecolor{currentstroke}%
\pgfsetstrokeopacity{0.700000}%
\pgfsetdash{}{0pt}%
\pgfpathmoveto{\pgfqpoint{4.649531in}{0.569972in}}%
\pgfpathcurveto{\pgfqpoint{4.651373in}{0.569972in}}{\pgfqpoint{4.653139in}{0.570703in}}{\pgfqpoint{4.654442in}{0.572006in}}%
\pgfpathcurveto{\pgfqpoint{4.655744in}{0.573308in}}{\pgfqpoint{4.656476in}{0.575074in}}{\pgfqpoint{4.656476in}{0.576916in}}%
\pgfpathcurveto{\pgfqpoint{4.656476in}{0.578758in}}{\pgfqpoint{4.655744in}{0.580524in}}{\pgfqpoint{4.654442in}{0.581826in}}%
\pgfpathcurveto{\pgfqpoint{4.653139in}{0.583129in}}{\pgfqpoint{4.651373in}{0.583860in}}{\pgfqpoint{4.649531in}{0.583860in}}%
\pgfpathcurveto{\pgfqpoint{4.647689in}{0.583860in}}{\pgfqpoint{4.645923in}{0.583129in}}{\pgfqpoint{4.644621in}{0.581826in}}%
\pgfpathcurveto{\pgfqpoint{4.643318in}{0.580524in}}{\pgfqpoint{4.642587in}{0.578758in}}{\pgfqpoint{4.642587in}{0.576916in}}%
\pgfpathcurveto{\pgfqpoint{4.642587in}{0.575074in}}{\pgfqpoint{4.643318in}{0.573308in}}{\pgfqpoint{4.644621in}{0.572006in}}%
\pgfpathcurveto{\pgfqpoint{4.645923in}{0.570703in}}{\pgfqpoint{4.647689in}{0.569972in}}{\pgfqpoint{4.649531in}{0.569972in}}%
\pgfpathlineto{\pgfqpoint{4.649531in}{0.569972in}}%
\pgfpathclose%
\pgfusepath{stroke,fill}%
\end{pgfscope}%
\begin{pgfscope}%
\pgfpathrectangle{\pgfqpoint{0.661006in}{0.524170in}}{\pgfqpoint{4.194036in}{1.071446in}}%
\pgfusepath{clip}%
\pgfsetbuttcap%
\pgfsetroundjoin%
\definecolor{currentfill}{rgb}{0.178818,0.132486,0.263383}%
\pgfsetfillcolor{currentfill}%
\pgfsetfillopacity{0.700000}%
\pgfsetlinewidth{1.003750pt}%
\definecolor{currentstroke}{rgb}{0.178818,0.132486,0.263383}%
\pgfsetstrokecolor{currentstroke}%
\pgfsetstrokeopacity{0.700000}%
\pgfsetdash{}{0pt}%
\pgfpathmoveto{\pgfqpoint{4.652738in}{0.570470in}}%
\pgfpathcurveto{\pgfqpoint{4.654580in}{0.570470in}}{\pgfqpoint{4.656346in}{0.571201in}}{\pgfqpoint{4.657649in}{0.572504in}}%
\pgfpathcurveto{\pgfqpoint{4.658951in}{0.573806in}}{\pgfqpoint{4.659683in}{0.575572in}}{\pgfqpoint{4.659683in}{0.577414in}}%
\pgfpathcurveto{\pgfqpoint{4.659683in}{0.579256in}}{\pgfqpoint{4.658951in}{0.581022in}}{\pgfqpoint{4.657649in}{0.582325in}}%
\pgfpathcurveto{\pgfqpoint{4.656346in}{0.583627in}}{\pgfqpoint{4.654580in}{0.584359in}}{\pgfqpoint{4.652738in}{0.584359in}}%
\pgfpathcurveto{\pgfqpoint{4.650896in}{0.584359in}}{\pgfqpoint{4.649130in}{0.583627in}}{\pgfqpoint{4.647828in}{0.582325in}}%
\pgfpathcurveto{\pgfqpoint{4.646525in}{0.581022in}}{\pgfqpoint{4.645794in}{0.579256in}}{\pgfqpoint{4.645794in}{0.577414in}}%
\pgfpathcurveto{\pgfqpoint{4.645794in}{0.575572in}}{\pgfqpoint{4.646525in}{0.573806in}}{\pgfqpoint{4.647828in}{0.572504in}}%
\pgfpathcurveto{\pgfqpoint{4.649130in}{0.571201in}}{\pgfqpoint{4.650896in}{0.570470in}}{\pgfqpoint{4.652738in}{0.570470in}}%
\pgfpathlineto{\pgfqpoint{4.652738in}{0.570470in}}%
\pgfpathclose%
\pgfusepath{stroke,fill}%
\end{pgfscope}%
\begin{pgfscope}%
\pgfpathrectangle{\pgfqpoint{0.661006in}{0.524170in}}{\pgfqpoint{4.194036in}{1.071446in}}%
\pgfusepath{clip}%
\pgfsetbuttcap%
\pgfsetroundjoin%
\definecolor{currentfill}{rgb}{0.178818,0.132486,0.263383}%
\pgfsetfillcolor{currentfill}%
\pgfsetfillopacity{0.700000}%
\pgfsetlinewidth{1.003750pt}%
\definecolor{currentstroke}{rgb}{0.178818,0.132486,0.263383}%
\pgfsetstrokecolor{currentstroke}%
\pgfsetstrokeopacity{0.700000}%
\pgfsetdash{}{0pt}%
\pgfpathmoveto{\pgfqpoint{4.644790in}{0.571933in}}%
\pgfpathcurveto{\pgfqpoint{4.646632in}{0.571933in}}{\pgfqpoint{4.648399in}{0.572665in}}{\pgfqpoint{4.649701in}{0.573967in}}%
\pgfpathcurveto{\pgfqpoint{4.651003in}{0.575270in}}{\pgfqpoint{4.651735in}{0.577036in}}{\pgfqpoint{4.651735in}{0.578878in}}%
\pgfpathcurveto{\pgfqpoint{4.651735in}{0.580720in}}{\pgfqpoint{4.651003in}{0.582486in}}{\pgfqpoint{4.649701in}{0.583788in}}%
\pgfpathcurveto{\pgfqpoint{4.648399in}{0.585091in}}{\pgfqpoint{4.646632in}{0.585822in}}{\pgfqpoint{4.644790in}{0.585822in}}%
\pgfpathcurveto{\pgfqpoint{4.642949in}{0.585822in}}{\pgfqpoint{4.641182in}{0.585091in}}{\pgfqpoint{4.639880in}{0.583788in}}%
\pgfpathcurveto{\pgfqpoint{4.638578in}{0.582486in}}{\pgfqpoint{4.637846in}{0.580720in}}{\pgfqpoint{4.637846in}{0.578878in}}%
\pgfpathcurveto{\pgfqpoint{4.637846in}{0.577036in}}{\pgfqpoint{4.638578in}{0.575270in}}{\pgfqpoint{4.639880in}{0.573967in}}%
\pgfpathcurveto{\pgfqpoint{4.641182in}{0.572665in}}{\pgfqpoint{4.642949in}{0.571933in}}{\pgfqpoint{4.644790in}{0.571933in}}%
\pgfpathlineto{\pgfqpoint{4.644790in}{0.571933in}}%
\pgfpathclose%
\pgfusepath{stroke,fill}%
\end{pgfscope}%
\begin{pgfscope}%
\pgfpathrectangle{\pgfqpoint{0.661006in}{0.524170in}}{\pgfqpoint{4.194036in}{1.071446in}}%
\pgfusepath{clip}%
\pgfsetbuttcap%
\pgfsetroundjoin%
\definecolor{currentfill}{rgb}{0.178818,0.132486,0.263383}%
\pgfsetfillcolor{currentfill}%
\pgfsetfillopacity{0.700000}%
\pgfsetlinewidth{1.003750pt}%
\definecolor{currentstroke}{rgb}{0.178818,0.132486,0.263383}%
\pgfsetstrokecolor{currentstroke}%
\pgfsetstrokeopacity{0.700000}%
\pgfsetdash{}{0pt}%
\pgfpathmoveto{\pgfqpoint{4.622993in}{0.576906in}}%
\pgfpathcurveto{\pgfqpoint{4.624834in}{0.576906in}}{\pgfqpoint{4.626601in}{0.577638in}}{\pgfqpoint{4.627903in}{0.578940in}}%
\pgfpathcurveto{\pgfqpoint{4.629205in}{0.580242in}}{\pgfqpoint{4.629937in}{0.582009in}}{\pgfqpoint{4.629937in}{0.583850in}}%
\pgfpathcurveto{\pgfqpoint{4.629937in}{0.585692in}}{\pgfqpoint{4.629205in}{0.587459in}}{\pgfqpoint{4.627903in}{0.588761in}}%
\pgfpathcurveto{\pgfqpoint{4.626601in}{0.590063in}}{\pgfqpoint{4.624834in}{0.590795in}}{\pgfqpoint{4.622993in}{0.590795in}}%
\pgfpathcurveto{\pgfqpoint{4.621151in}{0.590795in}}{\pgfqpoint{4.619384in}{0.590063in}}{\pgfqpoint{4.618082in}{0.588761in}}%
\pgfpathcurveto{\pgfqpoint{4.616780in}{0.587459in}}{\pgfqpoint{4.616048in}{0.585692in}}{\pgfqpoint{4.616048in}{0.583850in}}%
\pgfpathcurveto{\pgfqpoint{4.616048in}{0.582009in}}{\pgfqpoint{4.616780in}{0.580242in}}{\pgfqpoint{4.618082in}{0.578940in}}%
\pgfpathcurveto{\pgfqpoint{4.619384in}{0.577638in}}{\pgfqpoint{4.621151in}{0.576906in}}{\pgfqpoint{4.622993in}{0.576906in}}%
\pgfpathlineto{\pgfqpoint{4.622993in}{0.576906in}}%
\pgfpathclose%
\pgfusepath{stroke,fill}%
\end{pgfscope}%
\begin{pgfscope}%
\pgfpathrectangle{\pgfqpoint{0.661006in}{0.524170in}}{\pgfqpoint{4.194036in}{1.071446in}}%
\pgfusepath{clip}%
\pgfsetbuttcap%
\pgfsetroundjoin%
\definecolor{currentfill}{rgb}{0.178818,0.132486,0.263383}%
\pgfsetfillcolor{currentfill}%
\pgfsetfillopacity{0.700000}%
\pgfsetlinewidth{1.003750pt}%
\definecolor{currentstroke}{rgb}{0.178818,0.132486,0.263383}%
\pgfsetstrokecolor{currentstroke}%
\pgfsetstrokeopacity{0.700000}%
\pgfsetdash{}{0pt}%
\pgfpathmoveto{\pgfqpoint{4.596408in}{0.581234in}}%
\pgfpathcurveto{\pgfqpoint{4.598249in}{0.581234in}}{\pgfqpoint{4.600016in}{0.581966in}}{\pgfqpoint{4.601318in}{0.583268in}}%
\pgfpathcurveto{\pgfqpoint{4.602620in}{0.584570in}}{\pgfqpoint{4.603352in}{0.586337in}}{\pgfqpoint{4.603352in}{0.588179in}}%
\pgfpathcurveto{\pgfqpoint{4.603352in}{0.590020in}}{\pgfqpoint{4.602620in}{0.591787in}}{\pgfqpoint{4.601318in}{0.593089in}}%
\pgfpathcurveto{\pgfqpoint{4.600016in}{0.594391in}}{\pgfqpoint{4.598249in}{0.595123in}}{\pgfqpoint{4.596408in}{0.595123in}}%
\pgfpathcurveto{\pgfqpoint{4.594566in}{0.595123in}}{\pgfqpoint{4.592799in}{0.594391in}}{\pgfqpoint{4.591497in}{0.593089in}}%
\pgfpathcurveto{\pgfqpoint{4.590195in}{0.591787in}}{\pgfqpoint{4.589463in}{0.590020in}}{\pgfqpoint{4.589463in}{0.588179in}}%
\pgfpathcurveto{\pgfqpoint{4.589463in}{0.586337in}}{\pgfqpoint{4.590195in}{0.584570in}}{\pgfqpoint{4.591497in}{0.583268in}}%
\pgfpathcurveto{\pgfqpoint{4.592799in}{0.581966in}}{\pgfqpoint{4.594566in}{0.581234in}}{\pgfqpoint{4.596408in}{0.581234in}}%
\pgfpathlineto{\pgfqpoint{4.596408in}{0.581234in}}%
\pgfpathclose%
\pgfusepath{stroke,fill}%
\end{pgfscope}%
\begin{pgfscope}%
\pgfpathrectangle{\pgfqpoint{0.661006in}{0.524170in}}{\pgfqpoint{4.194036in}{1.071446in}}%
\pgfusepath{clip}%
\pgfsetbuttcap%
\pgfsetroundjoin%
\definecolor{currentfill}{rgb}{0.178818,0.132486,0.263383}%
\pgfsetfillcolor{currentfill}%
\pgfsetfillopacity{0.700000}%
\pgfsetlinewidth{1.003750pt}%
\definecolor{currentstroke}{rgb}{0.178818,0.132486,0.263383}%
\pgfsetstrokecolor{currentstroke}%
\pgfsetstrokeopacity{0.700000}%
\pgfsetdash{}{0pt}%
\pgfpathmoveto{\pgfqpoint{4.603290in}{0.579732in}}%
\pgfpathcurveto{\pgfqpoint{4.605131in}{0.579732in}}{\pgfqpoint{4.606898in}{0.580463in}}{\pgfqpoint{4.608200in}{0.581766in}}%
\pgfpathcurveto{\pgfqpoint{4.609502in}{0.583068in}}{\pgfqpoint{4.610234in}{0.584834in}}{\pgfqpoint{4.610234in}{0.586676in}}%
\pgfpathcurveto{\pgfqpoint{4.610234in}{0.588518in}}{\pgfqpoint{4.609502in}{0.590284in}}{\pgfqpoint{4.608200in}{0.591587in}}%
\pgfpathcurveto{\pgfqpoint{4.606898in}{0.592889in}}{\pgfqpoint{4.605131in}{0.593621in}}{\pgfqpoint{4.603290in}{0.593621in}}%
\pgfpathcurveto{\pgfqpoint{4.601448in}{0.593621in}}{\pgfqpoint{4.599681in}{0.592889in}}{\pgfqpoint{4.598379in}{0.591587in}}%
\pgfpathcurveto{\pgfqpoint{4.597077in}{0.590284in}}{\pgfqpoint{4.596345in}{0.588518in}}{\pgfqpoint{4.596345in}{0.586676in}}%
\pgfpathcurveto{\pgfqpoint{4.596345in}{0.584834in}}{\pgfqpoint{4.597077in}{0.583068in}}{\pgfqpoint{4.598379in}{0.581766in}}%
\pgfpathcurveto{\pgfqpoint{4.599681in}{0.580463in}}{\pgfqpoint{4.601448in}{0.579732in}}{\pgfqpoint{4.603290in}{0.579732in}}%
\pgfpathlineto{\pgfqpoint{4.603290in}{0.579732in}}%
\pgfpathclose%
\pgfusepath{stroke,fill}%
\end{pgfscope}%
\begin{pgfscope}%
\pgfpathrectangle{\pgfqpoint{0.661006in}{0.524170in}}{\pgfqpoint{4.194036in}{1.071446in}}%
\pgfusepath{clip}%
\pgfsetbuttcap%
\pgfsetroundjoin%
\definecolor{currentfill}{rgb}{0.175111,0.128867,0.257252}%
\pgfsetfillcolor{currentfill}%
\pgfsetfillopacity{0.700000}%
\pgfsetlinewidth{1.003750pt}%
\definecolor{currentstroke}{rgb}{0.175111,0.128867,0.257252}%
\pgfsetstrokecolor{currentstroke}%
\pgfsetstrokeopacity{0.700000}%
\pgfsetdash{}{0pt}%
\pgfpathmoveto{\pgfqpoint{4.624619in}{0.574029in}}%
\pgfpathcurveto{\pgfqpoint{4.626461in}{0.574029in}}{\pgfqpoint{4.628228in}{0.574761in}}{\pgfqpoint{4.629530in}{0.576063in}}%
\pgfpathcurveto{\pgfqpoint{4.630832in}{0.577366in}}{\pgfqpoint{4.631564in}{0.579132in}}{\pgfqpoint{4.631564in}{0.580974in}}%
\pgfpathcurveto{\pgfqpoint{4.631564in}{0.582816in}}{\pgfqpoint{4.630832in}{0.584582in}}{\pgfqpoint{4.629530in}{0.585884in}}%
\pgfpathcurveto{\pgfqpoint{4.628228in}{0.587187in}}{\pgfqpoint{4.626461in}{0.587918in}}{\pgfqpoint{4.624619in}{0.587918in}}%
\pgfpathcurveto{\pgfqpoint{4.622778in}{0.587918in}}{\pgfqpoint{4.621011in}{0.587187in}}{\pgfqpoint{4.619709in}{0.585884in}}%
\pgfpathcurveto{\pgfqpoint{4.618407in}{0.584582in}}{\pgfqpoint{4.617675in}{0.582816in}}{\pgfqpoint{4.617675in}{0.580974in}}%
\pgfpathcurveto{\pgfqpoint{4.617675in}{0.579132in}}{\pgfqpoint{4.618407in}{0.577366in}}{\pgfqpoint{4.619709in}{0.576063in}}%
\pgfpathcurveto{\pgfqpoint{4.621011in}{0.574761in}}{\pgfqpoint{4.622778in}{0.574029in}}{\pgfqpoint{4.624619in}{0.574029in}}%
\pgfpathlineto{\pgfqpoint{4.624619in}{0.574029in}}%
\pgfpathclose%
\pgfusepath{stroke,fill}%
\end{pgfscope}%
\begin{pgfscope}%
\pgfpathrectangle{\pgfqpoint{0.661006in}{0.524170in}}{\pgfqpoint{4.194036in}{1.071446in}}%
\pgfusepath{clip}%
\pgfsetbuttcap%
\pgfsetroundjoin%
\definecolor{currentfill}{rgb}{0.175111,0.128867,0.257252}%
\pgfsetfillcolor{currentfill}%
\pgfsetfillopacity{0.700000}%
\pgfsetlinewidth{1.003750pt}%
\definecolor{currentstroke}{rgb}{0.175111,0.128867,0.257252}%
\pgfsetstrokecolor{currentstroke}%
\pgfsetstrokeopacity{0.700000}%
\pgfsetdash{}{0pt}%
\pgfpathmoveto{\pgfqpoint{4.650461in}{0.569319in}}%
\pgfpathcurveto{\pgfqpoint{4.652302in}{0.569319in}}{\pgfqpoint{4.654069in}{0.570051in}}{\pgfqpoint{4.655371in}{0.571353in}}%
\pgfpathcurveto{\pgfqpoint{4.656673in}{0.572656in}}{\pgfqpoint{4.657405in}{0.574422in}}{\pgfqpoint{4.657405in}{0.576264in}}%
\pgfpathcurveto{\pgfqpoint{4.657405in}{0.578106in}}{\pgfqpoint{4.656673in}{0.579872in}}{\pgfqpoint{4.655371in}{0.581174in}}%
\pgfpathcurveto{\pgfqpoint{4.654069in}{0.582477in}}{\pgfqpoint{4.652302in}{0.583208in}}{\pgfqpoint{4.650461in}{0.583208in}}%
\pgfpathcurveto{\pgfqpoint{4.648619in}{0.583208in}}{\pgfqpoint{4.646853in}{0.582477in}}{\pgfqpoint{4.645550in}{0.581174in}}%
\pgfpathcurveto{\pgfqpoint{4.644248in}{0.579872in}}{\pgfqpoint{4.643516in}{0.578106in}}{\pgfqpoint{4.643516in}{0.576264in}}%
\pgfpathcurveto{\pgfqpoint{4.643516in}{0.574422in}}{\pgfqpoint{4.644248in}{0.572656in}}{\pgfqpoint{4.645550in}{0.571353in}}%
\pgfpathcurveto{\pgfqpoint{4.646853in}{0.570051in}}{\pgfqpoint{4.648619in}{0.569319in}}{\pgfqpoint{4.650461in}{0.569319in}}%
\pgfpathlineto{\pgfqpoint{4.650461in}{0.569319in}}%
\pgfpathclose%
\pgfusepath{stroke,fill}%
\end{pgfscope}%
\begin{pgfscope}%
\pgfpathrectangle{\pgfqpoint{0.661006in}{0.524170in}}{\pgfqpoint{4.194036in}{1.071446in}}%
\pgfusepath{clip}%
\pgfsetbuttcap%
\pgfsetroundjoin%
\definecolor{currentfill}{rgb}{0.171376,0.125266,0.251102}%
\pgfsetfillcolor{currentfill}%
\pgfsetfillopacity{0.700000}%
\pgfsetlinewidth{1.003750pt}%
\definecolor{currentstroke}{rgb}{0.171376,0.125266,0.251102}%
\pgfsetstrokecolor{currentstroke}%
\pgfsetstrokeopacity{0.700000}%
\pgfsetdash{}{0pt}%
\pgfpathmoveto{\pgfqpoint{4.658780in}{0.567501in}}%
\pgfpathcurveto{\pgfqpoint{4.660622in}{0.567501in}}{\pgfqpoint{4.662388in}{0.568233in}}{\pgfqpoint{4.663691in}{0.569535in}}%
\pgfpathcurveto{\pgfqpoint{4.664993in}{0.570837in}}{\pgfqpoint{4.665725in}{0.572604in}}{\pgfqpoint{4.665725in}{0.574446in}}%
\pgfpathcurveto{\pgfqpoint{4.665725in}{0.576287in}}{\pgfqpoint{4.664993in}{0.578054in}}{\pgfqpoint{4.663691in}{0.579356in}}%
\pgfpathcurveto{\pgfqpoint{4.662388in}{0.580658in}}{\pgfqpoint{4.660622in}{0.581390in}}{\pgfqpoint{4.658780in}{0.581390in}}%
\pgfpathcurveto{\pgfqpoint{4.656938in}{0.581390in}}{\pgfqpoint{4.655172in}{0.580658in}}{\pgfqpoint{4.653870in}{0.579356in}}%
\pgfpathcurveto{\pgfqpoint{4.652567in}{0.578054in}}{\pgfqpoint{4.651836in}{0.576287in}}{\pgfqpoint{4.651836in}{0.574446in}}%
\pgfpathcurveto{\pgfqpoint{4.651836in}{0.572604in}}{\pgfqpoint{4.652567in}{0.570837in}}{\pgfqpoint{4.653870in}{0.569535in}}%
\pgfpathcurveto{\pgfqpoint{4.655172in}{0.568233in}}{\pgfqpoint{4.656938in}{0.567501in}}{\pgfqpoint{4.658780in}{0.567501in}}%
\pgfpathlineto{\pgfqpoint{4.658780in}{0.567501in}}%
\pgfpathclose%
\pgfusepath{stroke,fill}%
\end{pgfscope}%
\begin{pgfscope}%
\pgfpathrectangle{\pgfqpoint{0.661006in}{0.524170in}}{\pgfqpoint{4.194036in}{1.071446in}}%
\pgfusepath{clip}%
\pgfsetbuttcap%
\pgfsetroundjoin%
\definecolor{currentfill}{rgb}{0.171376,0.125266,0.251102}%
\pgfsetfillcolor{currentfill}%
\pgfsetfillopacity{0.700000}%
\pgfsetlinewidth{1.003750pt}%
\definecolor{currentstroke}{rgb}{0.171376,0.125266,0.251102}%
\pgfsetstrokecolor{currentstroke}%
\pgfsetstrokeopacity{0.700000}%
\pgfsetdash{}{0pt}%
\pgfpathmoveto{\pgfqpoint{4.657851in}{0.566538in}}%
\pgfpathcurveto{\pgfqpoint{4.659692in}{0.566538in}}{\pgfqpoint{4.661459in}{0.567270in}}{\pgfqpoint{4.662761in}{0.568572in}}%
\pgfpathcurveto{\pgfqpoint{4.664063in}{0.569874in}}{\pgfqpoint{4.664795in}{0.571641in}}{\pgfqpoint{4.664795in}{0.573482in}}%
\pgfpathcurveto{\pgfqpoint{4.664795in}{0.575324in}}{\pgfqpoint{4.664063in}{0.577091in}}{\pgfqpoint{4.662761in}{0.578393in}}%
\pgfpathcurveto{\pgfqpoint{4.661459in}{0.579695in}}{\pgfqpoint{4.659692in}{0.580427in}}{\pgfqpoint{4.657851in}{0.580427in}}%
\pgfpathcurveto{\pgfqpoint{4.656009in}{0.580427in}}{\pgfqpoint{4.654242in}{0.579695in}}{\pgfqpoint{4.652940in}{0.578393in}}%
\pgfpathcurveto{\pgfqpoint{4.651638in}{0.577091in}}{\pgfqpoint{4.650906in}{0.575324in}}{\pgfqpoint{4.650906in}{0.573482in}}%
\pgfpathcurveto{\pgfqpoint{4.650906in}{0.571641in}}{\pgfqpoint{4.651638in}{0.569874in}}{\pgfqpoint{4.652940in}{0.568572in}}%
\pgfpathcurveto{\pgfqpoint{4.654242in}{0.567270in}}{\pgfqpoint{4.656009in}{0.566538in}}{\pgfqpoint{4.657851in}{0.566538in}}%
\pgfpathlineto{\pgfqpoint{4.657851in}{0.566538in}}%
\pgfpathclose%
\pgfusepath{stroke,fill}%
\end{pgfscope}%
\begin{pgfscope}%
\pgfpathrectangle{\pgfqpoint{0.661006in}{0.524170in}}{\pgfqpoint{4.194036in}{1.071446in}}%
\pgfusepath{clip}%
\pgfsetbuttcap%
\pgfsetroundjoin%
\definecolor{currentfill}{rgb}{0.171376,0.125266,0.251102}%
\pgfsetfillcolor{currentfill}%
\pgfsetfillopacity{0.700000}%
\pgfsetlinewidth{1.003750pt}%
\definecolor{currentstroke}{rgb}{0.171376,0.125266,0.251102}%
\pgfsetstrokecolor{currentstroke}%
\pgfsetstrokeopacity{0.700000}%
\pgfsetdash{}{0pt}%
\pgfpathmoveto{\pgfqpoint{4.664404in}{0.565927in}}%
\pgfpathcurveto{\pgfqpoint{4.666246in}{0.565927in}}{\pgfqpoint{4.668012in}{0.566659in}}{\pgfqpoint{4.669314in}{0.567961in}}%
\pgfpathcurveto{\pgfqpoint{4.670617in}{0.569264in}}{\pgfqpoint{4.671348in}{0.571030in}}{\pgfqpoint{4.671348in}{0.572872in}}%
\pgfpathcurveto{\pgfqpoint{4.671348in}{0.574714in}}{\pgfqpoint{4.670617in}{0.576480in}}{\pgfqpoint{4.669314in}{0.577782in}}%
\pgfpathcurveto{\pgfqpoint{4.668012in}{0.579085in}}{\pgfqpoint{4.666246in}{0.579816in}}{\pgfqpoint{4.664404in}{0.579816in}}%
\pgfpathcurveto{\pgfqpoint{4.662562in}{0.579816in}}{\pgfqpoint{4.660796in}{0.579085in}}{\pgfqpoint{4.659493in}{0.577782in}}%
\pgfpathcurveto{\pgfqpoint{4.658191in}{0.576480in}}{\pgfqpoint{4.657459in}{0.574714in}}{\pgfqpoint{4.657459in}{0.572872in}}%
\pgfpathcurveto{\pgfqpoint{4.657459in}{0.571030in}}{\pgfqpoint{4.658191in}{0.569264in}}{\pgfqpoint{4.659493in}{0.567961in}}%
\pgfpathcurveto{\pgfqpoint{4.660796in}{0.566659in}}{\pgfqpoint{4.662562in}{0.565927in}}{\pgfqpoint{4.664404in}{0.565927in}}%
\pgfpathlineto{\pgfqpoint{4.664404in}{0.565927in}}%
\pgfpathclose%
\pgfusepath{stroke,fill}%
\end{pgfscope}%
\begin{pgfscope}%
\pgfpathrectangle{\pgfqpoint{0.661006in}{0.524170in}}{\pgfqpoint{4.194036in}{1.071446in}}%
\pgfusepath{clip}%
\pgfsetbuttcap%
\pgfsetroundjoin%
\definecolor{currentfill}{rgb}{0.171376,0.125266,0.251102}%
\pgfsetfillcolor{currentfill}%
\pgfsetfillopacity{0.700000}%
\pgfsetlinewidth{1.003750pt}%
\definecolor{currentstroke}{rgb}{0.171376,0.125266,0.251102}%
\pgfsetstrokecolor{currentstroke}%
\pgfsetstrokeopacity{0.700000}%
\pgfsetdash{}{0pt}%
\pgfpathmoveto{\pgfqpoint{4.653854in}{0.567058in}}%
\pgfpathcurveto{\pgfqpoint{4.655695in}{0.567058in}}{\pgfqpoint{4.657462in}{0.567790in}}{\pgfqpoint{4.658764in}{0.569092in}}%
\pgfpathcurveto{\pgfqpoint{4.660066in}{0.570394in}}{\pgfqpoint{4.660798in}{0.572161in}}{\pgfqpoint{4.660798in}{0.574003in}}%
\pgfpathcurveto{\pgfqpoint{4.660798in}{0.575844in}}{\pgfqpoint{4.660066in}{0.577611in}}{\pgfqpoint{4.658764in}{0.578913in}}%
\pgfpathcurveto{\pgfqpoint{4.657462in}{0.580215in}}{\pgfqpoint{4.655695in}{0.580947in}}{\pgfqpoint{4.653854in}{0.580947in}}%
\pgfpathcurveto{\pgfqpoint{4.652012in}{0.580947in}}{\pgfqpoint{4.650245in}{0.580215in}}{\pgfqpoint{4.648943in}{0.578913in}}%
\pgfpathcurveto{\pgfqpoint{4.647641in}{0.577611in}}{\pgfqpoint{4.646909in}{0.575844in}}{\pgfqpoint{4.646909in}{0.574003in}}%
\pgfpathcurveto{\pgfqpoint{4.646909in}{0.572161in}}{\pgfqpoint{4.647641in}{0.570394in}}{\pgfqpoint{4.648943in}{0.569092in}}%
\pgfpathcurveto{\pgfqpoint{4.650245in}{0.567790in}}{\pgfqpoint{4.652012in}{0.567058in}}{\pgfqpoint{4.653854in}{0.567058in}}%
\pgfpathlineto{\pgfqpoint{4.653854in}{0.567058in}}%
\pgfpathclose%
\pgfusepath{stroke,fill}%
\end{pgfscope}%
\begin{pgfscope}%
\pgfpathrectangle{\pgfqpoint{0.661006in}{0.524170in}}{\pgfqpoint{4.194036in}{1.071446in}}%
\pgfusepath{clip}%
\pgfsetbuttcap%
\pgfsetroundjoin%
\definecolor{currentfill}{rgb}{0.171376,0.125266,0.251102}%
\pgfsetfillcolor{currentfill}%
\pgfsetfillopacity{0.700000}%
\pgfsetlinewidth{1.003750pt}%
\definecolor{currentstroke}{rgb}{0.171376,0.125266,0.251102}%
\pgfsetstrokecolor{currentstroke}%
\pgfsetstrokeopacity{0.700000}%
\pgfsetdash{}{0pt}%
\pgfpathmoveto{\pgfqpoint{4.641677in}{0.569375in}}%
\pgfpathcurveto{\pgfqpoint{4.643518in}{0.569375in}}{\pgfqpoint{4.645285in}{0.570106in}}{\pgfqpoint{4.646587in}{0.571409in}}%
\pgfpathcurveto{\pgfqpoint{4.647889in}{0.572711in}}{\pgfqpoint{4.648621in}{0.574477in}}{\pgfqpoint{4.648621in}{0.576319in}}%
\pgfpathcurveto{\pgfqpoint{4.648621in}{0.578161in}}{\pgfqpoint{4.647889in}{0.579927in}}{\pgfqpoint{4.646587in}{0.581230in}}%
\pgfpathcurveto{\pgfqpoint{4.645285in}{0.582532in}}{\pgfqpoint{4.643518in}{0.583264in}}{\pgfqpoint{4.641677in}{0.583264in}}%
\pgfpathcurveto{\pgfqpoint{4.639835in}{0.583264in}}{\pgfqpoint{4.638068in}{0.582532in}}{\pgfqpoint{4.636766in}{0.581230in}}%
\pgfpathcurveto{\pgfqpoint{4.635464in}{0.579927in}}{\pgfqpoint{4.634732in}{0.578161in}}{\pgfqpoint{4.634732in}{0.576319in}}%
\pgfpathcurveto{\pgfqpoint{4.634732in}{0.574477in}}{\pgfqpoint{4.635464in}{0.572711in}}{\pgfqpoint{4.636766in}{0.571409in}}%
\pgfpathcurveto{\pgfqpoint{4.638068in}{0.570106in}}{\pgfqpoint{4.639835in}{0.569375in}}{\pgfqpoint{4.641677in}{0.569375in}}%
\pgfpathlineto{\pgfqpoint{4.641677in}{0.569375in}}%
\pgfpathclose%
\pgfusepath{stroke,fill}%
\end{pgfscope}%
\begin{pgfscope}%
\pgfpathrectangle{\pgfqpoint{0.661006in}{0.524170in}}{\pgfqpoint{4.194036in}{1.071446in}}%
\pgfusepath{clip}%
\pgfsetbuttcap%
\pgfsetroundjoin%
\definecolor{currentfill}{rgb}{0.167612,0.121684,0.244932}%
\pgfsetfillcolor{currentfill}%
\pgfsetfillopacity{0.700000}%
\pgfsetlinewidth{1.003750pt}%
\definecolor{currentstroke}{rgb}{0.167612,0.121684,0.244932}%
\pgfsetstrokecolor{currentstroke}%
\pgfsetstrokeopacity{0.700000}%
\pgfsetdash{}{0pt}%
\pgfpathmoveto{\pgfqpoint{4.629825in}{0.571754in}}%
\pgfpathcurveto{\pgfqpoint{4.631666in}{0.571754in}}{\pgfqpoint{4.633433in}{0.572486in}}{\pgfqpoint{4.634735in}{0.573788in}}%
\pgfpathcurveto{\pgfqpoint{4.636038in}{0.575090in}}{\pgfqpoint{4.636769in}{0.576857in}}{\pgfqpoint{4.636769in}{0.578698in}}%
\pgfpathcurveto{\pgfqpoint{4.636769in}{0.580540in}}{\pgfqpoint{4.636038in}{0.582307in}}{\pgfqpoint{4.634735in}{0.583609in}}%
\pgfpathcurveto{\pgfqpoint{4.633433in}{0.584911in}}{\pgfqpoint{4.631666in}{0.585643in}}{\pgfqpoint{4.629825in}{0.585643in}}%
\pgfpathcurveto{\pgfqpoint{4.627983in}{0.585643in}}{\pgfqpoint{4.626217in}{0.584911in}}{\pgfqpoint{4.624914in}{0.583609in}}%
\pgfpathcurveto{\pgfqpoint{4.623612in}{0.582307in}}{\pgfqpoint{4.622880in}{0.580540in}}{\pgfqpoint{4.622880in}{0.578698in}}%
\pgfpathcurveto{\pgfqpoint{4.622880in}{0.576857in}}{\pgfqpoint{4.623612in}{0.575090in}}{\pgfqpoint{4.624914in}{0.573788in}}%
\pgfpathcurveto{\pgfqpoint{4.626217in}{0.572486in}}{\pgfqpoint{4.627983in}{0.571754in}}{\pgfqpoint{4.629825in}{0.571754in}}%
\pgfpathlineto{\pgfqpoint{4.629825in}{0.571754in}}%
\pgfpathclose%
\pgfusepath{stroke,fill}%
\end{pgfscope}%
\begin{pgfscope}%
\pgfpathrectangle{\pgfqpoint{0.661006in}{0.524170in}}{\pgfqpoint{4.194036in}{1.071446in}}%
\pgfusepath{clip}%
\pgfsetbuttcap%
\pgfsetroundjoin%
\definecolor{currentfill}{rgb}{0.167612,0.121684,0.244932}%
\pgfsetfillcolor{currentfill}%
\pgfsetfillopacity{0.700000}%
\pgfsetlinewidth{1.003750pt}%
\definecolor{currentstroke}{rgb}{0.167612,0.121684,0.244932}%
\pgfsetstrokecolor{currentstroke}%
\pgfsetstrokeopacity{0.700000}%
\pgfsetdash{}{0pt}%
\pgfpathmoveto{\pgfqpoint{4.612442in}{0.576211in}}%
\pgfpathcurveto{\pgfqpoint{4.614284in}{0.576211in}}{\pgfqpoint{4.616051in}{0.576943in}}{\pgfqpoint{4.617353in}{0.578245in}}%
\pgfpathcurveto{\pgfqpoint{4.618655in}{0.579547in}}{\pgfqpoint{4.619387in}{0.581314in}}{\pgfqpoint{4.619387in}{0.583155in}}%
\pgfpathcurveto{\pgfqpoint{4.619387in}{0.584997in}}{\pgfqpoint{4.618655in}{0.586764in}}{\pgfqpoint{4.617353in}{0.588066in}}%
\pgfpathcurveto{\pgfqpoint{4.616051in}{0.589368in}}{\pgfqpoint{4.614284in}{0.590100in}}{\pgfqpoint{4.612442in}{0.590100in}}%
\pgfpathcurveto{\pgfqpoint{4.610601in}{0.590100in}}{\pgfqpoint{4.608834in}{0.589368in}}{\pgfqpoint{4.607532in}{0.588066in}}%
\pgfpathcurveto{\pgfqpoint{4.606230in}{0.586764in}}{\pgfqpoint{4.605498in}{0.584997in}}{\pgfqpoint{4.605498in}{0.583155in}}%
\pgfpathcurveto{\pgfqpoint{4.605498in}{0.581314in}}{\pgfqpoint{4.606230in}{0.579547in}}{\pgfqpoint{4.607532in}{0.578245in}}%
\pgfpathcurveto{\pgfqpoint{4.608834in}{0.576943in}}{\pgfqpoint{4.610601in}{0.576211in}}{\pgfqpoint{4.612442in}{0.576211in}}%
\pgfpathlineto{\pgfqpoint{4.612442in}{0.576211in}}%
\pgfpathclose%
\pgfusepath{stroke,fill}%
\end{pgfscope}%
\begin{pgfscope}%
\pgfpathrectangle{\pgfqpoint{0.661006in}{0.524170in}}{\pgfqpoint{4.194036in}{1.071446in}}%
\pgfusepath{clip}%
\pgfsetbuttcap%
\pgfsetroundjoin%
\definecolor{currentfill}{rgb}{0.167612,0.121684,0.244932}%
\pgfsetfillcolor{currentfill}%
\pgfsetfillopacity{0.700000}%
\pgfsetlinewidth{1.003750pt}%
\definecolor{currentstroke}{rgb}{0.167612,0.121684,0.244932}%
\pgfsetstrokecolor{currentstroke}%
\pgfsetstrokeopacity{0.700000}%
\pgfsetdash{}{0pt}%
\pgfpathmoveto{\pgfqpoint{4.597477in}{0.579164in}}%
\pgfpathcurveto{\pgfqpoint{4.599318in}{0.579164in}}{\pgfqpoint{4.601085in}{0.579895in}}{\pgfqpoint{4.602387in}{0.581198in}}%
\pgfpathcurveto{\pgfqpoint{4.603689in}{0.582500in}}{\pgfqpoint{4.604421in}{0.584266in}}{\pgfqpoint{4.604421in}{0.586108in}}%
\pgfpathcurveto{\pgfqpoint{4.604421in}{0.587950in}}{\pgfqpoint{4.603689in}{0.589716in}}{\pgfqpoint{4.602387in}{0.591018in}}%
\pgfpathcurveto{\pgfqpoint{4.601085in}{0.592321in}}{\pgfqpoint{4.599318in}{0.593052in}}{\pgfqpoint{4.597477in}{0.593052in}}%
\pgfpathcurveto{\pgfqpoint{4.595635in}{0.593052in}}{\pgfqpoint{4.593868in}{0.592321in}}{\pgfqpoint{4.592566in}{0.591018in}}%
\pgfpathcurveto{\pgfqpoint{4.591264in}{0.589716in}}{\pgfqpoint{4.590532in}{0.587950in}}{\pgfqpoint{4.590532in}{0.586108in}}%
\pgfpathcurveto{\pgfqpoint{4.590532in}{0.584266in}}{\pgfqpoint{4.591264in}{0.582500in}}{\pgfqpoint{4.592566in}{0.581198in}}%
\pgfpathcurveto{\pgfqpoint{4.593868in}{0.579895in}}{\pgfqpoint{4.595635in}{0.579164in}}{\pgfqpoint{4.597477in}{0.579164in}}%
\pgfpathlineto{\pgfqpoint{4.597477in}{0.579164in}}%
\pgfpathclose%
\pgfusepath{stroke,fill}%
\end{pgfscope}%
\begin{pgfscope}%
\pgfpathrectangle{\pgfqpoint{0.661006in}{0.524170in}}{\pgfqpoint{4.194036in}{1.071446in}}%
\pgfusepath{clip}%
\pgfsetrectcap%
\pgfsetroundjoin%
\pgfsetlinewidth{0.803000pt}%
\definecolor{currentstroke}{rgb}{0.450000,0.450000,0.450000}%
\pgfsetstrokecolor{currentstroke}%
\pgfsetdash{}{0pt}%
\pgfpathmoveto{\pgfqpoint{0.792013in}{0.524170in}}%
\pgfpathlineto{\pgfqpoint{0.792013in}{1.595616in}}%
\pgfusepath{stroke}%
\end{pgfscope}%
\begin{pgfscope}%
\pgfsetbuttcap%
\pgfsetroundjoin%
\definecolor{currentfill}{rgb}{0.000000,0.000000,0.000000}%
\pgfsetfillcolor{currentfill}%
\pgfsetlinewidth{0.803000pt}%
\definecolor{currentstroke}{rgb}{0.000000,0.000000,0.000000}%
\pgfsetstrokecolor{currentstroke}%
\pgfsetdash{}{0pt}%
\pgfsys@defobject{currentmarker}{\pgfqpoint{0.000000in}{-0.048611in}}{\pgfqpoint{0.000000in}{0.000000in}}{%
\pgfpathmoveto{\pgfqpoint{0.000000in}{0.000000in}}%
\pgfpathlineto{\pgfqpoint{0.000000in}{-0.048611in}}%
\pgfusepath{stroke,fill}%
}%
\begin{pgfscope}%
\pgfsys@transformshift{0.792013in}{0.524170in}%
\pgfsys@useobject{currentmarker}{}%
\end{pgfscope}%
\end{pgfscope}%
\begin{pgfscope}%
\definecolor{textcolor}{rgb}{0.000000,0.000000,0.000000}%
\pgfsetstrokecolor{textcolor}%
\pgfsetfillcolor{textcolor}%
\pgftext[x=0.792013in,y=0.426948in,,top]{\color{textcolor}\rmfamily\fontsize{8.000000}{9.600000}\selectfont \(\displaystyle {995}\)}%
\end{pgfscope}%
\begin{pgfscope}%
\pgfpathrectangle{\pgfqpoint{0.661006in}{0.524170in}}{\pgfqpoint{4.194036in}{1.071446in}}%
\pgfusepath{clip}%
\pgfsetrectcap%
\pgfsetroundjoin%
\pgfsetlinewidth{0.803000pt}%
\definecolor{currentstroke}{rgb}{0.450000,0.450000,0.450000}%
\pgfsetstrokecolor{currentstroke}%
\pgfsetdash{}{0pt}%
\pgfpathmoveto{\pgfqpoint{1.396217in}{0.524170in}}%
\pgfpathlineto{\pgfqpoint{1.396217in}{1.595616in}}%
\pgfusepath{stroke}%
\end{pgfscope}%
\begin{pgfscope}%
\pgfsetbuttcap%
\pgfsetroundjoin%
\definecolor{currentfill}{rgb}{0.000000,0.000000,0.000000}%
\pgfsetfillcolor{currentfill}%
\pgfsetlinewidth{0.803000pt}%
\definecolor{currentstroke}{rgb}{0.000000,0.000000,0.000000}%
\pgfsetstrokecolor{currentstroke}%
\pgfsetdash{}{0pt}%
\pgfsys@defobject{currentmarker}{\pgfqpoint{0.000000in}{-0.048611in}}{\pgfqpoint{0.000000in}{0.000000in}}{%
\pgfpathmoveto{\pgfqpoint{0.000000in}{0.000000in}}%
\pgfpathlineto{\pgfqpoint{0.000000in}{-0.048611in}}%
\pgfusepath{stroke,fill}%
}%
\begin{pgfscope}%
\pgfsys@transformshift{1.396217in}{0.524170in}%
\pgfsys@useobject{currentmarker}{}%
\end{pgfscope}%
\end{pgfscope}%
\begin{pgfscope}%
\definecolor{textcolor}{rgb}{0.000000,0.000000,0.000000}%
\pgfsetstrokecolor{textcolor}%
\pgfsetfillcolor{textcolor}%
\pgftext[x=1.396217in,y=0.426948in,,top]{\color{textcolor}\rmfamily\fontsize{8.000000}{9.600000}\selectfont \(\displaystyle {996}\)}%
\end{pgfscope}%
\begin{pgfscope}%
\pgfpathrectangle{\pgfqpoint{0.661006in}{0.524170in}}{\pgfqpoint{4.194036in}{1.071446in}}%
\pgfusepath{clip}%
\pgfsetrectcap%
\pgfsetroundjoin%
\pgfsetlinewidth{0.803000pt}%
\definecolor{currentstroke}{rgb}{0.450000,0.450000,0.450000}%
\pgfsetstrokecolor{currentstroke}%
\pgfsetdash{}{0pt}%
\pgfpathmoveto{\pgfqpoint{2.000422in}{0.524170in}}%
\pgfpathlineto{\pgfqpoint{2.000422in}{1.595616in}}%
\pgfusepath{stroke}%
\end{pgfscope}%
\begin{pgfscope}%
\pgfsetbuttcap%
\pgfsetroundjoin%
\definecolor{currentfill}{rgb}{0.000000,0.000000,0.000000}%
\pgfsetfillcolor{currentfill}%
\pgfsetlinewidth{0.803000pt}%
\definecolor{currentstroke}{rgb}{0.000000,0.000000,0.000000}%
\pgfsetstrokecolor{currentstroke}%
\pgfsetdash{}{0pt}%
\pgfsys@defobject{currentmarker}{\pgfqpoint{0.000000in}{-0.048611in}}{\pgfqpoint{0.000000in}{0.000000in}}{%
\pgfpathmoveto{\pgfqpoint{0.000000in}{0.000000in}}%
\pgfpathlineto{\pgfqpoint{0.000000in}{-0.048611in}}%
\pgfusepath{stroke,fill}%
}%
\begin{pgfscope}%
\pgfsys@transformshift{2.000422in}{0.524170in}%
\pgfsys@useobject{currentmarker}{}%
\end{pgfscope}%
\end{pgfscope}%
\begin{pgfscope}%
\definecolor{textcolor}{rgb}{0.000000,0.000000,0.000000}%
\pgfsetstrokecolor{textcolor}%
\pgfsetfillcolor{textcolor}%
\pgftext[x=2.000422in,y=0.426948in,,top]{\color{textcolor}\rmfamily\fontsize{8.000000}{9.600000}\selectfont \(\displaystyle {997}\)}%
\end{pgfscope}%
\begin{pgfscope}%
\pgfpathrectangle{\pgfqpoint{0.661006in}{0.524170in}}{\pgfqpoint{4.194036in}{1.071446in}}%
\pgfusepath{clip}%
\pgfsetrectcap%
\pgfsetroundjoin%
\pgfsetlinewidth{0.803000pt}%
\definecolor{currentstroke}{rgb}{0.450000,0.450000,0.450000}%
\pgfsetstrokecolor{currentstroke}%
\pgfsetdash{}{0pt}%
\pgfpathmoveto{\pgfqpoint{2.604626in}{0.524170in}}%
\pgfpathlineto{\pgfqpoint{2.604626in}{1.595616in}}%
\pgfusepath{stroke}%
\end{pgfscope}%
\begin{pgfscope}%
\pgfsetbuttcap%
\pgfsetroundjoin%
\definecolor{currentfill}{rgb}{0.000000,0.000000,0.000000}%
\pgfsetfillcolor{currentfill}%
\pgfsetlinewidth{0.803000pt}%
\definecolor{currentstroke}{rgb}{0.000000,0.000000,0.000000}%
\pgfsetstrokecolor{currentstroke}%
\pgfsetdash{}{0pt}%
\pgfsys@defobject{currentmarker}{\pgfqpoint{0.000000in}{-0.048611in}}{\pgfqpoint{0.000000in}{0.000000in}}{%
\pgfpathmoveto{\pgfqpoint{0.000000in}{0.000000in}}%
\pgfpathlineto{\pgfqpoint{0.000000in}{-0.048611in}}%
\pgfusepath{stroke,fill}%
}%
\begin{pgfscope}%
\pgfsys@transformshift{2.604626in}{0.524170in}%
\pgfsys@useobject{currentmarker}{}%
\end{pgfscope}%
\end{pgfscope}%
\begin{pgfscope}%
\definecolor{textcolor}{rgb}{0.000000,0.000000,0.000000}%
\pgfsetstrokecolor{textcolor}%
\pgfsetfillcolor{textcolor}%
\pgftext[x=2.604626in,y=0.426948in,,top]{\color{textcolor}\rmfamily\fontsize{8.000000}{9.600000}\selectfont \(\displaystyle {998}\)}%
\end{pgfscope}%
\begin{pgfscope}%
\pgfpathrectangle{\pgfqpoint{0.661006in}{0.524170in}}{\pgfqpoint{4.194036in}{1.071446in}}%
\pgfusepath{clip}%
\pgfsetrectcap%
\pgfsetroundjoin%
\pgfsetlinewidth{0.803000pt}%
\definecolor{currentstroke}{rgb}{0.450000,0.450000,0.450000}%
\pgfsetstrokecolor{currentstroke}%
\pgfsetdash{}{0pt}%
\pgfpathmoveto{\pgfqpoint{3.208830in}{0.524170in}}%
\pgfpathlineto{\pgfqpoint{3.208830in}{1.595616in}}%
\pgfusepath{stroke}%
\end{pgfscope}%
\begin{pgfscope}%
\pgfsetbuttcap%
\pgfsetroundjoin%
\definecolor{currentfill}{rgb}{0.000000,0.000000,0.000000}%
\pgfsetfillcolor{currentfill}%
\pgfsetlinewidth{0.803000pt}%
\definecolor{currentstroke}{rgb}{0.000000,0.000000,0.000000}%
\pgfsetstrokecolor{currentstroke}%
\pgfsetdash{}{0pt}%
\pgfsys@defobject{currentmarker}{\pgfqpoint{0.000000in}{-0.048611in}}{\pgfqpoint{0.000000in}{0.000000in}}{%
\pgfpathmoveto{\pgfqpoint{0.000000in}{0.000000in}}%
\pgfpathlineto{\pgfqpoint{0.000000in}{-0.048611in}}%
\pgfusepath{stroke,fill}%
}%
\begin{pgfscope}%
\pgfsys@transformshift{3.208830in}{0.524170in}%
\pgfsys@useobject{currentmarker}{}%
\end{pgfscope}%
\end{pgfscope}%
\begin{pgfscope}%
\definecolor{textcolor}{rgb}{0.000000,0.000000,0.000000}%
\pgfsetstrokecolor{textcolor}%
\pgfsetfillcolor{textcolor}%
\pgftext[x=3.208830in,y=0.426948in,,top]{\color{textcolor}\rmfamily\fontsize{8.000000}{9.600000}\selectfont \(\displaystyle {999}\)}%
\end{pgfscope}%
\begin{pgfscope}%
\pgfpathrectangle{\pgfqpoint{0.661006in}{0.524170in}}{\pgfqpoint{4.194036in}{1.071446in}}%
\pgfusepath{clip}%
\pgfsetrectcap%
\pgfsetroundjoin%
\pgfsetlinewidth{0.803000pt}%
\definecolor{currentstroke}{rgb}{0.450000,0.450000,0.450000}%
\pgfsetstrokecolor{currentstroke}%
\pgfsetdash{}{0pt}%
\pgfpathmoveto{\pgfqpoint{3.813034in}{0.524170in}}%
\pgfpathlineto{\pgfqpoint{3.813034in}{1.595616in}}%
\pgfusepath{stroke}%
\end{pgfscope}%
\begin{pgfscope}%
\pgfsetbuttcap%
\pgfsetroundjoin%
\definecolor{currentfill}{rgb}{0.000000,0.000000,0.000000}%
\pgfsetfillcolor{currentfill}%
\pgfsetlinewidth{0.803000pt}%
\definecolor{currentstroke}{rgb}{0.000000,0.000000,0.000000}%
\pgfsetstrokecolor{currentstroke}%
\pgfsetdash{}{0pt}%
\pgfsys@defobject{currentmarker}{\pgfqpoint{0.000000in}{-0.048611in}}{\pgfqpoint{0.000000in}{0.000000in}}{%
\pgfpathmoveto{\pgfqpoint{0.000000in}{0.000000in}}%
\pgfpathlineto{\pgfqpoint{0.000000in}{-0.048611in}}%
\pgfusepath{stroke,fill}%
}%
\begin{pgfscope}%
\pgfsys@transformshift{3.813034in}{0.524170in}%
\pgfsys@useobject{currentmarker}{}%
\end{pgfscope}%
\end{pgfscope}%
\begin{pgfscope}%
\definecolor{textcolor}{rgb}{0.000000,0.000000,0.000000}%
\pgfsetstrokecolor{textcolor}%
\pgfsetfillcolor{textcolor}%
\pgftext[x=3.813034in,y=0.426948in,,top]{\color{textcolor}\rmfamily\fontsize{8.000000}{9.600000}\selectfont \(\displaystyle {1000}\)}%
\end{pgfscope}%
\begin{pgfscope}%
\pgfpathrectangle{\pgfqpoint{0.661006in}{0.524170in}}{\pgfqpoint{4.194036in}{1.071446in}}%
\pgfusepath{clip}%
\pgfsetrectcap%
\pgfsetroundjoin%
\pgfsetlinewidth{0.803000pt}%
\definecolor{currentstroke}{rgb}{0.450000,0.450000,0.450000}%
\pgfsetstrokecolor{currentstroke}%
\pgfsetdash{}{0pt}%
\pgfpathmoveto{\pgfqpoint{4.417238in}{0.524170in}}%
\pgfpathlineto{\pgfqpoint{4.417238in}{1.595616in}}%
\pgfusepath{stroke}%
\end{pgfscope}%
\begin{pgfscope}%
\pgfsetbuttcap%
\pgfsetroundjoin%
\definecolor{currentfill}{rgb}{0.000000,0.000000,0.000000}%
\pgfsetfillcolor{currentfill}%
\pgfsetlinewidth{0.803000pt}%
\definecolor{currentstroke}{rgb}{0.000000,0.000000,0.000000}%
\pgfsetstrokecolor{currentstroke}%
\pgfsetdash{}{0pt}%
\pgfsys@defobject{currentmarker}{\pgfqpoint{0.000000in}{-0.048611in}}{\pgfqpoint{0.000000in}{0.000000in}}{%
\pgfpathmoveto{\pgfqpoint{0.000000in}{0.000000in}}%
\pgfpathlineto{\pgfqpoint{0.000000in}{-0.048611in}}%
\pgfusepath{stroke,fill}%
}%
\begin{pgfscope}%
\pgfsys@transformshift{4.417238in}{0.524170in}%
\pgfsys@useobject{currentmarker}{}%
\end{pgfscope}%
\end{pgfscope}%
\begin{pgfscope}%
\definecolor{textcolor}{rgb}{0.000000,0.000000,0.000000}%
\pgfsetstrokecolor{textcolor}%
\pgfsetfillcolor{textcolor}%
\pgftext[x=4.417238in,y=0.426948in,,top]{\color{textcolor}\rmfamily\fontsize{8.000000}{9.600000}\selectfont \(\displaystyle {1001}\)}%
\end{pgfscope}%
\begin{pgfscope}%
\definecolor{textcolor}{rgb}{0.000000,0.000000,0.000000}%
\pgfsetstrokecolor{textcolor}%
\pgfsetfillcolor{textcolor}%
\pgftext[x=2.758024in,y=0.272725in,,top]{\color{textcolor}\rmfamily\fontsize{10.000000}{12.000000}\selectfont Pressure in \unit{\hecto\pascal}}%
\end{pgfscope}%
\begin{pgfscope}%
\pgfpathrectangle{\pgfqpoint{0.661006in}{0.524170in}}{\pgfqpoint{4.194036in}{1.071446in}}%
\pgfusepath{clip}%
\pgfsetrectcap%
\pgfsetroundjoin%
\pgfsetlinewidth{0.803000pt}%
\definecolor{currentstroke}{rgb}{0.450000,0.450000,0.450000}%
\pgfsetstrokecolor{currentstroke}%
\pgfsetdash{}{0pt}%
\pgfpathmoveto{\pgfqpoint{0.661006in}{0.608024in}}%
\pgfpathlineto{\pgfqpoint{4.855042in}{0.608024in}}%
\pgfusepath{stroke}%
\end{pgfscope}%
\begin{pgfscope}%
\pgfsetbuttcap%
\pgfsetroundjoin%
\definecolor{currentfill}{rgb}{0.000000,0.000000,0.000000}%
\pgfsetfillcolor{currentfill}%
\pgfsetlinewidth{0.803000pt}%
\definecolor{currentstroke}{rgb}{0.000000,0.000000,0.000000}%
\pgfsetstrokecolor{currentstroke}%
\pgfsetdash{}{0pt}%
\pgfsys@defobject{currentmarker}{\pgfqpoint{-0.048611in}{0.000000in}}{\pgfqpoint{-0.000000in}{0.000000in}}{%
\pgfpathmoveto{\pgfqpoint{-0.000000in}{0.000000in}}%
\pgfpathlineto{\pgfqpoint{-0.048611in}{0.000000in}}%
\pgfusepath{stroke,fill}%
}%
\begin{pgfscope}%
\pgfsys@transformshift{0.661006in}{0.608024in}%
\pgfsys@useobject{currentmarker}{}%
\end{pgfscope}%
\end{pgfscope}%
\begin{pgfscope}%
\definecolor{textcolor}{rgb}{0.000000,0.000000,0.000000}%
\pgfsetstrokecolor{textcolor}%
\pgfsetfillcolor{textcolor}%
\pgftext[x=0.412932in, y=0.569469in, left, base]{\color{textcolor}\rmfamily\fontsize{8.000000}{9.600000}\selectfont \(\displaystyle {1.0}\)}%
\end{pgfscope}%
\begin{pgfscope}%
\pgfpathrectangle{\pgfqpoint{0.661006in}{0.524170in}}{\pgfqpoint{4.194036in}{1.071446in}}%
\pgfusepath{clip}%
\pgfsetrectcap%
\pgfsetroundjoin%
\pgfsetlinewidth{0.803000pt}%
\definecolor{currentstroke}{rgb}{0.450000,0.450000,0.450000}%
\pgfsetstrokecolor{currentstroke}%
\pgfsetdash{}{0pt}%
\pgfpathmoveto{\pgfqpoint{0.661006in}{0.913250in}}%
\pgfpathlineto{\pgfqpoint{4.855042in}{0.913250in}}%
\pgfusepath{stroke}%
\end{pgfscope}%
\begin{pgfscope}%
\pgfsetbuttcap%
\pgfsetroundjoin%
\definecolor{currentfill}{rgb}{0.000000,0.000000,0.000000}%
\pgfsetfillcolor{currentfill}%
\pgfsetlinewidth{0.803000pt}%
\definecolor{currentstroke}{rgb}{0.000000,0.000000,0.000000}%
\pgfsetstrokecolor{currentstroke}%
\pgfsetdash{}{0pt}%
\pgfsys@defobject{currentmarker}{\pgfqpoint{-0.048611in}{0.000000in}}{\pgfqpoint{-0.000000in}{0.000000in}}{%
\pgfpathmoveto{\pgfqpoint{-0.000000in}{0.000000in}}%
\pgfpathlineto{\pgfqpoint{-0.048611in}{0.000000in}}%
\pgfusepath{stroke,fill}%
}%
\begin{pgfscope}%
\pgfsys@transformshift{0.661006in}{0.913250in}%
\pgfsys@useobject{currentmarker}{}%
\end{pgfscope}%
\end{pgfscope}%
\begin{pgfscope}%
\definecolor{textcolor}{rgb}{0.000000,0.000000,0.000000}%
\pgfsetstrokecolor{textcolor}%
\pgfsetfillcolor{textcolor}%
\pgftext[x=0.412932in, y=0.874695in, left, base]{\color{textcolor}\rmfamily\fontsize{8.000000}{9.600000}\selectfont \(\displaystyle {1.5}\)}%
\end{pgfscope}%
\begin{pgfscope}%
\pgfpathrectangle{\pgfqpoint{0.661006in}{0.524170in}}{\pgfqpoint{4.194036in}{1.071446in}}%
\pgfusepath{clip}%
\pgfsetrectcap%
\pgfsetroundjoin%
\pgfsetlinewidth{0.803000pt}%
\definecolor{currentstroke}{rgb}{0.450000,0.450000,0.450000}%
\pgfsetstrokecolor{currentstroke}%
\pgfsetdash{}{0pt}%
\pgfpathmoveto{\pgfqpoint{0.661006in}{1.218476in}}%
\pgfpathlineto{\pgfqpoint{4.855042in}{1.218476in}}%
\pgfusepath{stroke}%
\end{pgfscope}%
\begin{pgfscope}%
\pgfsetbuttcap%
\pgfsetroundjoin%
\definecolor{currentfill}{rgb}{0.000000,0.000000,0.000000}%
\pgfsetfillcolor{currentfill}%
\pgfsetlinewidth{0.803000pt}%
\definecolor{currentstroke}{rgb}{0.000000,0.000000,0.000000}%
\pgfsetstrokecolor{currentstroke}%
\pgfsetdash{}{0pt}%
\pgfsys@defobject{currentmarker}{\pgfqpoint{-0.048611in}{0.000000in}}{\pgfqpoint{-0.000000in}{0.000000in}}{%
\pgfpathmoveto{\pgfqpoint{-0.000000in}{0.000000in}}%
\pgfpathlineto{\pgfqpoint{-0.048611in}{0.000000in}}%
\pgfusepath{stroke,fill}%
}%
\begin{pgfscope}%
\pgfsys@transformshift{0.661006in}{1.218476in}%
\pgfsys@useobject{currentmarker}{}%
\end{pgfscope}%
\end{pgfscope}%
\begin{pgfscope}%
\definecolor{textcolor}{rgb}{0.000000,0.000000,0.000000}%
\pgfsetstrokecolor{textcolor}%
\pgfsetfillcolor{textcolor}%
\pgftext[x=0.412932in, y=1.179920in, left, base]{\color{textcolor}\rmfamily\fontsize{8.000000}{9.600000}\selectfont \(\displaystyle {2.0}\)}%
\end{pgfscope}%
\begin{pgfscope}%
\pgfpathrectangle{\pgfqpoint{0.661006in}{0.524170in}}{\pgfqpoint{4.194036in}{1.071446in}}%
\pgfusepath{clip}%
\pgfsetrectcap%
\pgfsetroundjoin%
\pgfsetlinewidth{0.803000pt}%
\definecolor{currentstroke}{rgb}{0.450000,0.450000,0.450000}%
\pgfsetstrokecolor{currentstroke}%
\pgfsetdash{}{0pt}%
\pgfpathmoveto{\pgfqpoint{0.661006in}{1.523702in}}%
\pgfpathlineto{\pgfqpoint{4.855042in}{1.523702in}}%
\pgfusepath{stroke}%
\end{pgfscope}%
\begin{pgfscope}%
\pgfsetbuttcap%
\pgfsetroundjoin%
\definecolor{currentfill}{rgb}{0.000000,0.000000,0.000000}%
\pgfsetfillcolor{currentfill}%
\pgfsetlinewidth{0.803000pt}%
\definecolor{currentstroke}{rgb}{0.000000,0.000000,0.000000}%
\pgfsetstrokecolor{currentstroke}%
\pgfsetdash{}{0pt}%
\pgfsys@defobject{currentmarker}{\pgfqpoint{-0.048611in}{0.000000in}}{\pgfqpoint{-0.000000in}{0.000000in}}{%
\pgfpathmoveto{\pgfqpoint{-0.000000in}{0.000000in}}%
\pgfpathlineto{\pgfqpoint{-0.048611in}{0.000000in}}%
\pgfusepath{stroke,fill}%
}%
\begin{pgfscope}%
\pgfsys@transformshift{0.661006in}{1.523702in}%
\pgfsys@useobject{currentmarker}{}%
\end{pgfscope}%
\end{pgfscope}%
\begin{pgfscope}%
\definecolor{textcolor}{rgb}{0.000000,0.000000,0.000000}%
\pgfsetstrokecolor{textcolor}%
\pgfsetfillcolor{textcolor}%
\pgftext[x=0.412932in, y=1.485146in, left, base]{\color{textcolor}\rmfamily\fontsize{8.000000}{9.600000}\selectfont \(\displaystyle {2.5}\)}%
\end{pgfscope}%
\begin{pgfscope}%
\definecolor{textcolor}{rgb}{0.000000,0.000000,0.000000}%
\pgfsetstrokecolor{textcolor}%
\pgfsetfillcolor{textcolor}%
\pgftext[x=0.357377in,y=1.059893in,,bottom,rotate=90.000000]{\color{textcolor}\rmfamily\fontsize{10.000000}{12.000000}\selectfont Voltage in \unit{\V}}%
\end{pgfscope}%
\begin{pgfscope}%
\pgfpathrectangle{\pgfqpoint{0.661006in}{0.524170in}}{\pgfqpoint{4.194036in}{1.071446in}}%
\pgfusepath{clip}%
\pgfsetrectcap%
\pgfsetroundjoin%
\pgfsetlinewidth{1.505625pt}%
\definecolor{currentstroke}{rgb}{0.835294,0.368627,0.000000}%
\pgfsetstrokecolor{currentstroke}%
\pgfsetstrokeopacity{0.700000}%
\pgfsetdash{}{0pt}%
\pgfpathmoveto{\pgfqpoint{0.891939in}{1.525132in}}%
\pgfpathlineto{\pgfqpoint{0.851644in}{1.535251in}}%
\pgfpathlineto{\pgfqpoint{4.664404in}{0.577777in}}%
\pgfpathlineto{\pgfqpoint{4.597477in}{0.594584in}}%
\pgfusepath{stroke}%
\end{pgfscope}%
\begin{pgfscope}%
\pgfsetrectcap%
\pgfsetmiterjoin%
\pgfsetlinewidth{0.803000pt}%
\definecolor{currentstroke}{rgb}{0.000000,0.000000,0.000000}%
\pgfsetstrokecolor{currentstroke}%
\pgfsetdash{}{0pt}%
\pgfpathmoveto{\pgfqpoint{0.661006in}{0.524170in}}%
\pgfpathlineto{\pgfqpoint{0.661006in}{1.595616in}}%
\pgfusepath{stroke}%
\end{pgfscope}%
\begin{pgfscope}%
\pgfsetrectcap%
\pgfsetmiterjoin%
\pgfsetlinewidth{0.803000pt}%
\definecolor{currentstroke}{rgb}{0.000000,0.000000,0.000000}%
\pgfsetstrokecolor{currentstroke}%
\pgfsetdash{}{0pt}%
\pgfpathmoveto{\pgfqpoint{4.855042in}{0.524170in}}%
\pgfpathlineto{\pgfqpoint{4.855042in}{1.595616in}}%
\pgfusepath{stroke}%
\end{pgfscope}%
\begin{pgfscope}%
\pgfsetrectcap%
\pgfsetmiterjoin%
\pgfsetlinewidth{0.803000pt}%
\definecolor{currentstroke}{rgb}{0.000000,0.000000,0.000000}%
\pgfsetstrokecolor{currentstroke}%
\pgfsetdash{}{0pt}%
\pgfpathmoveto{\pgfqpoint{0.661006in}{0.524170in}}%
\pgfpathlineto{\pgfqpoint{4.855042in}{0.524170in}}%
\pgfusepath{stroke}%
\end{pgfscope}%
\begin{pgfscope}%
\pgfsetrectcap%
\pgfsetmiterjoin%
\pgfsetlinewidth{0.803000pt}%
\definecolor{currentstroke}{rgb}{0.000000,0.000000,0.000000}%
\pgfsetstrokecolor{currentstroke}%
\pgfsetdash{}{0pt}%
\pgfpathmoveto{\pgfqpoint{0.661006in}{1.595616in}}%
\pgfpathlineto{\pgfqpoint{4.855042in}{1.595616in}}%
\pgfusepath{stroke}%
\end{pgfscope}%
\begin{pgfscope}%
\pgfsetbuttcap%
\pgfsetmiterjoin%
\definecolor{currentfill}{rgb}{1.000000,1.000000,1.000000}%
\pgfsetfillcolor{currentfill}%
\pgfsetfillopacity{0.800000}%
\pgfsetlinewidth{1.003750pt}%
\definecolor{currentstroke}{rgb}{0.800000,0.800000,0.800000}%
\pgfsetstrokecolor{currentstroke}%
\pgfsetstrokeopacity{0.800000}%
\pgfsetdash{}{0pt}%
\pgfpathmoveto{\pgfqpoint{1.100048in}{0.554926in}}%
\pgfpathlineto{\pgfqpoint{2.898281in}{0.554926in}}%
\pgfpathquadraticcurveto{\pgfqpoint{2.939948in}{0.554926in}}{\pgfqpoint{2.939948in}{0.596592in}}%
\pgfpathlineto{\pgfqpoint{2.939948in}{0.756021in}}%
\pgfpathquadraticcurveto{\pgfqpoint{2.939948in}{0.797688in}}{\pgfqpoint{2.898281in}{0.797688in}}%
\pgfpathlineto{\pgfqpoint{1.100048in}{0.797688in}}%
\pgfpathquadraticcurveto{\pgfqpoint{1.058381in}{0.797688in}}{\pgfqpoint{1.058381in}{0.756021in}}%
\pgfpathlineto{\pgfqpoint{1.058381in}{0.596592in}}%
\pgfpathquadraticcurveto{\pgfqpoint{1.058381in}{0.554926in}}{\pgfqpoint{1.100048in}{0.554926in}}%
\pgfpathlineto{\pgfqpoint{1.100048in}{0.554926in}}%
\pgfpathclose%
\pgfusepath{stroke,fill}%
\end{pgfscope}%
\begin{pgfscope}%
\definecolor{textcolor}{rgb}{0.000000,0.000000,0.000000}%
\pgfsetstrokecolor{textcolor}%
\pgfsetfillcolor{textcolor}%
\pgftext[x=2.898281in,y=0.631314in,right,base]{\color{textcolor}\rmfamily\fontsize{10.000000}{12.000000}\selectfont (-2.486e-01 ± 9e-04) \unit{\V \per \hecto\pascal}}%
\end{pgfscope}%
\begin{pgfscope}%
\pgfsetbuttcap%
\pgfsetmiterjoin%
\definecolor{currentfill}{rgb}{1.000000,1.000000,1.000000}%
\pgfsetfillcolor{currentfill}%
\pgfsetfillopacity{0.800000}%
\pgfsetlinewidth{1.003750pt}%
\definecolor{currentstroke}{rgb}{0.800000,0.800000,0.800000}%
\pgfsetstrokecolor{currentstroke}%
\pgfsetstrokeopacity{0.800000}%
\pgfsetdash{}{0pt}%
\pgfpathmoveto{\pgfqpoint{3.738264in}{1.194505in}}%
\pgfpathlineto{\pgfqpoint{4.777264in}{1.194505in}}%
\pgfpathquadraticcurveto{\pgfqpoint{4.799486in}{1.194505in}}{\pgfqpoint{4.799486in}{1.216727in}}%
\pgfpathlineto{\pgfqpoint{4.799486in}{1.517838in}}%
\pgfpathquadraticcurveto{\pgfqpoint{4.799486in}{1.540060in}}{\pgfqpoint{4.777264in}{1.540060in}}%
\pgfpathlineto{\pgfqpoint{3.738264in}{1.540060in}}%
\pgfpathquadraticcurveto{\pgfqpoint{3.716042in}{1.540060in}}{\pgfqpoint{3.716042in}{1.517838in}}%
\pgfpathlineto{\pgfqpoint{3.716042in}{1.216727in}}%
\pgfpathquadraticcurveto{\pgfqpoint{3.716042in}{1.194505in}}{\pgfqpoint{3.738264in}{1.194505in}}%
\pgfpathlineto{\pgfqpoint{3.738264in}{1.194505in}}%
\pgfpathclose%
\pgfusepath{stroke,fill}%
\end{pgfscope}%
\begin{pgfscope}%
\pgfsetbuttcap%
\pgfsetroundjoin%
\definecolor{currentfill}{rgb}{0.780108,0.874195,0.919757}%
\pgfsetfillcolor{currentfill}%
\pgfsetfillopacity{0.700000}%
\pgfsetlinewidth{1.003750pt}%
\definecolor{currentstroke}{rgb}{0.780108,0.874195,0.919757}%
\pgfsetstrokecolor{currentstroke}%
\pgfsetstrokeopacity{0.700000}%
\pgfsetdash{}{0pt}%
\pgfpathmoveto{\pgfqpoint{3.871597in}{1.440060in}}%
\pgfpathcurveto{\pgfqpoint{3.873439in}{1.440060in}}{\pgfqpoint{3.875205in}{1.440792in}}{\pgfqpoint{3.876508in}{1.442094in}}%
\pgfpathcurveto{\pgfqpoint{3.877810in}{1.443396in}}{\pgfqpoint{3.878542in}{1.445163in}}{\pgfqpoint{3.878542in}{1.447005in}}%
\pgfpathcurveto{\pgfqpoint{3.878542in}{1.448846in}}{\pgfqpoint{3.877810in}{1.450613in}}{\pgfqpoint{3.876508in}{1.451915in}}%
\pgfpathcurveto{\pgfqpoint{3.875205in}{1.453217in}}{\pgfqpoint{3.873439in}{1.453949in}}{\pgfqpoint{3.871597in}{1.453949in}}%
\pgfpathcurveto{\pgfqpoint{3.869756in}{1.453949in}}{\pgfqpoint{3.867989in}{1.453217in}}{\pgfqpoint{3.866687in}{1.451915in}}%
\pgfpathcurveto{\pgfqpoint{3.865385in}{1.450613in}}{\pgfqpoint{3.864653in}{1.448846in}}{\pgfqpoint{3.864653in}{1.447005in}}%
\pgfpathcurveto{\pgfqpoint{3.864653in}{1.445163in}}{\pgfqpoint{3.865385in}{1.443396in}}{\pgfqpoint{3.866687in}{1.442094in}}%
\pgfpathcurveto{\pgfqpoint{3.867989in}{1.440792in}}{\pgfqpoint{3.869756in}{1.440060in}}{\pgfqpoint{3.871597in}{1.440060in}}%
\pgfpathlineto{\pgfqpoint{3.871597in}{1.440060in}}%
\pgfpathclose%
\pgfusepath{stroke,fill}%
\end{pgfscope}%
\begin{pgfscope}%
\definecolor{textcolor}{rgb}{0.000000,0.000000,0.000000}%
\pgfsetstrokecolor{textcolor}%
\pgfsetfillcolor{textcolor}%
\pgftext[x=4.071597in,y=1.417838in,left,base]{\color{textcolor}\rmfamily\fontsize{8.000000}{9.600000}\selectfont Piezo voltage}%
\end{pgfscope}%
\begin{pgfscope}%
\pgfsetrectcap%
\pgfsetroundjoin%
\pgfsetlinewidth{1.505625pt}%
\definecolor{currentstroke}{rgb}{0.835294,0.368627,0.000000}%
\pgfsetstrokecolor{currentstroke}%
\pgfsetstrokeopacity{0.700000}%
\pgfsetdash{}{0pt}%
\pgfpathmoveto{\pgfqpoint{3.760486in}{1.300616in}}%
\pgfpathlineto{\pgfqpoint{3.871597in}{1.300616in}}%
\pgfpathlineto{\pgfqpoint{3.982708in}{1.300616in}}%
\pgfusepath{stroke}%
\end{pgfscope}%
\begin{pgfscope}%
\definecolor{textcolor}{rgb}{0.000000,0.000000,0.000000}%
\pgfsetstrokecolor{textcolor}%
\pgfsetfillcolor{textcolor}%
\pgftext[x=4.071597in,y=1.261727in,left,base]{\color{textcolor}\rmfamily\fontsize{8.000000}{9.600000}\selectfont Regression}%
\end{pgfscope}%
\end{pgfpicture}%
\makeatother%
\endgroup%

    \caption{A \qty{780}{\nm} laser locked to the rubidium \ce{^{85}Rb} $D_2$ line. The piezo mounted inside the resonator moves the back reflecting mirror to follow the air pressure and to keep the laser frequency locked to the rubidium transition. The piezo voltage closely follows the changes in air pressure.}
    \label{fig:atomics_master_barometer}
\end{figure}

An o-ring seal, like demonstrated by \cite{ecdl_silicone_housing}, can improve the stability of the diode lasers and ensure the performance of the system becomes insensitive to weather conditions. In addition, the ample use of epoxy resin inside the resonator causes a sensitivity to humidity as the epoxy resin swells when exposed to water \cite{epoxy_humidity}. \citeauthor{thesis_tilman} \cite{thesis_tilman} already presented major improvements by presenting a mechanical mount for the collimation lens, but both the back reflecting mirror and the lenses forming the \textit{cat-eye} \cite{ecdl_paris} are fastened using epoxy resin. The currently employed technique of using silica gel to maintain a constant humidity should be superseded by a more permanent solution. A sealed resonator can therefore add an import part to the day-to-day stability of the experiments.
